\makeatletter
\@ifundefined{全三巻@一括ビルド}{%
{\huge
\ruby{天}{そら} う つ %空白有り
\ruby{浪}{なみ}}  {\normalsize 第一}
\vspace*{3zw}

\Entry{其一}
}
\makeatother

% メモ 校正終了 2024-03-28

\原本頁{1-3}%
\ruby{秋}{あき}は
\ruby{海樓}{かい|ろう}の
\ruby{垂簾}{すだ|れ}に
\ruby{動}{うご}きて、
%
ばつと
\ruby{吹}{ふ}き
\ruby{來}{く}る
\ruby{沖}{おき}の
\ruby{風}{かぜ}は、
%
\ruby{夕日}{ゆふ|ひ}の
\ruby{餘光}{よ|くわう}
\ruby{美}{うる}はしきが
\ruby{中}{なか}に、
%
\ruby{無限}{む|げん}の
\ruby{爽凉}{さう|りやう}の
\ruby{氣}{き}を
\ruby{齎}{もた}らせば、
%
\ruby{白帆}{しら|ほ}
\ruby{明}{あか}るき
\ruby{{\換字{遠}}方}{とほ|く}の
\ruby{{\換字{船}}}{ふね}の
\ruby{數々}{かず|〳〵}も、
%
\ruby{{\換字{鉛}}色}{なまり|いろ}なして
\ruby{漫々}{まん|〳〵}たる
\ruby{潮}{うしほ}の
\ruby{果}{はて}に
\ruby{却}{かへ}つて
\ruby{物淋}{もの|さび}しう
\ruby{見}{み}え
\ruby{渡}{わた}りつ、
%
\ruby{竹芝}{たけ|しば}の
\ruby{浦}{うら}の
\ruby{浪}{なみ}
\ruby{靜}{しづ}かに、
%
\ruby{增上寺}{ぞう|じやう|じ}の
\ruby{鐘聲}{か|ね}に
\ruby{暮}{く}れ
\ruby{行}{ゆ}かんとす。

\原本頁{1-8}%
\ruby{此}{こ}の
\ruby[<h|]{夕}{ゆふべ}
\ruby{此}{こ}の
\ruby{時}{とき}、
%
『
\ruby{見}{み}はらし
』の
\ruby{樓上}{ろう|じよう}の
\ruby{一室}{いつ|しつ}に、
%
\ruby{貸}{か}し
\ruby{{\換字{浴}}衣}{ゆ|かた}の
\ruby{胸元}{むな|もと}
ゆたかに
くつろげて、
%
\ruby{醉}{ゑひ}に% 「醉」は原本通り「ゑ」で調整
\ruby{嘯}{うそぶ}く
\ruby{大胡坐}{おほ|あぐ|ら}、
%
たゞ% 原本は「ヾ」片仮名繰{\換字{返}}し記号(濁点)を使用してる
\ruby{秋}{あき}の
\ruby{飮酒}{さ|け}に
\ruby{宜}{よろ}しきを
\原本頁{2-1}%
\ruby{知}{し}つて
\ruby{其}{そ}の
\ruby{他}{た}を
\ruby{知}{し}らぬ
\ruby{面構}{つら|がま}へ
きび〳〵と、
%
あはれも
\ruby{絲瓜}{へち|ま}も
あるものか、
%
\ruby{鴫}{しぎ}が
\ruby{飛}{と}んだら
\ruby{撃}{う}つて
\ruby[g]{下物}{さかな}、
%
と
\ruby{云}{い}はぬ
ばかりの
\ruby{顏}{かほ}つきして、
%
いづれも
\ruby{勇}{いさ}みを
\ruby{含}{ふく}む
\ruby{酒盃}{さか|づき}の
\ruby{{\換字{遣}}}{や}り
\ruby{取}{と}り、
%
\ruby{火}{ひ}の
\ruby{珠}{たま}も
\ruby{挾}{はさ}んで
\ruby{食}{く}ふべき
\ruby{年齡}{とし|ばへ}の
\ruby{勢}{いきほ}ひに、
%
\ruby[g]{此方}{こなた}の
\ruby{壯語}{さう|ご}、
%
\ruby[g]{彼方}{かなた}の
\ruby{傲語}{がう|ご}、
%
\ruby{或}{あるひ}は
\ruby{彼}{かれ}
\ruby{此}{これ}
\ruby{哄然}{ど|つ}と
\ruby{一齊}{いち|ど}の
\ruby{天狗}{てん|ぐ}
\ruby{笑}{わら}ひの
\ruby{響}{どよみ}の
\ruby{中}{うち}に、
%
\ruby{間{\換字{近}}}{ま|ぢか}く
\ruby{{\換字{通}}}{とほ}る
\ruby{滊車}{き|しや}の
\ruby{音}{おと}をも
\ruby{埋}{うづ}めて
\ruby{仕舞}{し|ま}ふまで、
%
\ruby{無邪氣}{む|じや|き}に
\ruby{睦}{むつ}み
\ruby{語}{かた}らへる
\ruby{四人}{よ|にん}
\ruby{{\換字{連}}}{づれ}
あり。

\原本頁{2-7}%
\ruby{陽氣}{やう|き}の
\ruby{歡笑}{くわん|せう}は
\ruby{一}{ひ}トしきり
\ruby{濟}{す}みて、
%
\ruby{今}{いま}しも
\ruby{談話}{はな|し}は
\ruby{少}{すこ}し
\ruby{沈}{しづ}みぬ。

\原本頁{2-8}%
\ruby{手}{て}さき
\ruby{頸筋}{くび|すぢ}に
\ruby{洋服}{やう|ふく}の
\ruby{痕}{あと}
\ruby{{\換字{判}}然}{はつ|きり}と
\ruby{知}{し}れて、
%
\ruby{誰}{た}が
\ruby{眼}{め}にも
\ruby{{\換字{船}}人}{ふな|のり}と
\ruby{映}{うつ}る
\ruby{赭顏}{あから|がほ}の
\ruby{日}{ひ}に
\ruby{焦}{や}けきつたる
\ruby[g]{羽{\換字{勝}}}{はがち}
\ruby{千{\換字{造}}}{せん|ざう}は、
%
\ruby{酒盃}{さか|づき}を
\ruby{擧}{あ}げて
\ruby{一}{ひ}ト
\ruby{口}{くち}
\ruby{飮}{の}みしが、
%
\ruby{不興氣}{ふ|きよう|げ}に
\ruby{復下}{また|した}に
\ruby{置}{お}きて、

\原本頁{2-11}%
『フーム』

\原本頁{3-1}%
とばかり
\ruby{力無}{ちから|な}く
\ruby{答}{こた}へつ、
%
\ruby{{\換字{猶}}}{なほ}
\ruby{其}{そ}の
\ruby{對手}{あひ|て}の
\ruby{何事}{なに|ごと}をか
\ruby{語}{かた}り
\ruby{添}{そ}ふるを
\ruby{待}{ま}つが
\ruby{如}{ごと}き
\ruby{意}{こゝろ}を
\ruby{其}{そ}の
\ruby{語氣}{ご|き}に
\ruby{現}{あらは}したり。

\原本頁{3-3}%
\ruby[g]{羽{\換字{勝}}}{はがち}に
\ruby{對}{むか}ひて
\ruby{坐}{ざ}せる
\ruby{小男}{こ|をとこ}の、
%
\ruby[h|]{面}{おもて}
\ruby{淸}{きよ}らにして
\ruby{桃花}{とう|くわ}の
\ruby{如}{ごと}き
\ruby[g]{山瀬}{やませ}
\ruby{荒吉}{あら|きち}は
\ruby{其意}{その|い}を
\ruby{悟}{さと}つて、
%
\ruby{果}{はた}して
\ruby{直}{たゞち}に
\ruby{言葉}{こと|ば}を
\ruby{足}{た}しぬ。

\原本頁{3-5}%
『ト
\ruby{云}{い}ふ
\ruby{次第}{し|だい}なので
\ruby[g]{水野}{みづの}
\ruby{君}{くん}は
\ruby{來}{こ}んのさ。
%
\ruby{今}{いま}
\ruby{話}{はな}した
\ruby{内{\換字{情}}}{ない|じやう}も
\ruby{解}{わか}つて
\ruby{居}{ゐ}たので、
%
\ruby{今日}{け|ふ}の
\ruby{會合}{くわい|がふ}の
\ruby{發起人}{ほつ|き|にん}の
\ruby{僕}{ぼく}は、
%
\ruby{十{\換字{分}}}{じう|ぶん}に
\ruby{{\換字{情}}理}{じやう|り}を
\ruby{盡}{つく}した
\ruby{手紙}{て|がみ}を
\ruby{與}{や}つて、
%
\ruby{是非}{ぜ|ひ}
\ruby{出}{で}て
\ruby{來}{く}るやうにと
\ruby{勸}{すゝ}めたんだが、
%
たゞ% 原本は「ヾ」片仮名繰{\換字{返}}し記号(濁点)を使用してる
\ruby{差支}{さし|つかへ}があつて
\ruby{行}{ゆ}かれないといふ
\ruby{冷淡}{れい|たん}
\ruby{極}{きは}まる
\ruby{{\換字{返}}事}{へん|じ}なんで、
%
\ruby{仕方}{し|かた}が
\ruby{無}{な}いと
\ruby{斷念}{あき|ら}めて
\ruby{仕舞}{し|ま}つた。
%
\ruby{實}{じつ}に
\ruby[g]{水野}{みづの}
\ruby{君}{くん}にも
\ruby{似合}{に|あ}はない、
%
\ruby[g]{全然}{まるで}
\ruby[g]{無茶}{むちや}
\ruby[g]{苦茶}{くちや}になつて
\ruby{居}{ゐ}られるのだからね。
』

\原本頁{3-11}%
\ruby{見}{み}る〳〵
\ruby[g]{羽{\換字{勝}}}{はがち}が
\ruby{面}{おもて}には
\ruby[h|]{憂色}{いう|しよく}
\ruby[||->]{現}{あらは}れ、
%
その
\ruby{眼}{め}は
\ruby{沈思}{ちん|し}に
\ruby{凝然}{じ|つ}と
\ruby{動}{うご}かずなりたり。

\原本頁{4-2}%
\ruby[g]{羽{\換字{勝}}}{はがち}が
\ruby{左方}{ひだ|り}に
\ruby{坐}{ざ}して
\ruby{默々}{もく|〳〵}と
\ruby{飮}{の}み
\ruby{居}{ゐ}し
\ruby{骨太}{ほね|ぶと}
\ruby{岩疊}{がん|でふ}づくりの
\ruby[g]{日方}{ひかた}
\ruby{八郞}{はち|らう}は、
%
\ruby{突然}{とつ|ぜん}として
\ruby{牛}{うし}の
\ruby{吼}{ほ}ゆるが
\ruby{如}{ごと}くに
\ruby{呌}{さけ}び
\ruby{出}{だ}し、

\原本頁{4-4}%
『
\ruby[g]{山瀬}{やませ}、
%
\ruby{貴樣}{き|さま}も
\ruby{今}{いま}は
\ruby{堂々}{だう|〴〵}たる
\ruby{新聞記者}{しん|ぶん|き|しや}だ。
%
\ruby{往時}{むか|し}のやうに
\ruby{想像談}{さう|〴〵|だん}や
\ruby{法螺話}{ほ|ら|ばなし}は
\ruby{語}{かた}るまいな。
』

\原本頁{4-6}%
と、
%
\ruby{詰}{なじ}り
\ruby{氣味}{ぎ|み}に
\ruby{問}{と}ひ
\ruby{糺}{たゞ}せば、
%
\ruby[g]{山瀬}{やませ}は
\ruby{聊}{いさゝ}か
\ruby{怫然}{む|つ}として、

\原本頁{4-7}%
『
\ruby[g]{日方}{ひかた}
\ruby{陸軍少尉}{りく|ぐん|せう|ゐ}
\ruby{殿}{どの}に
\ruby{伺}{うかゞ}ひます。
%
\ruby{報告}{はう|こく}は
\ruby{無責任}{む|せき|にん}を
\ruby{以}{もつ}て
\ruby{作爲}{さく|ゐ}すべきもので
ござりまする
\ruby{歟}{か}。
%
は〻は〻は〻。% TODO 原本の最後の「〻」には見えないが、当該グリフがみつからない
% TODO 踊り字で表現するなら二の次点(ゆすり点)でなく「はゝはゝはゝ」では?
』

\原本頁{4-9}%
と
\ruby{{\換字{遣}}}{や}り
\ruby{{\換字{返}}}{かへ}して
\ruby{笑}{わら}ふ。

\原本頁{4-10}%
\ruby[g]{日方}{ひかた}は
\ruby[g]{山瀬}{やませ}の
\ruby{戱言}{たは|むれ}には
\ruby{頓着無}{とん|ぢやく|な}く、
%
\ruby{怒}{いか}れるが
\ruby{如}{ごと}く
\ruby{眞面目}{ま|じ|め}になりて、

\原本頁{4-11}%
『ムヽ、
%
して
\ruby{見}{み}れば
\ruby{全}{まつた}く
\ruby{事實}{じ|じつ}と
\ruby{見}{み}える。
%
イヤ
\ruby{怪}{け}しからん、
%
\ruby{實}{じつ}に
\ruby{怪}{け}しからん。
%
\ruby{何}{なん}だ!。
%
\ruby{愚劣}{ぐ|れつ}
\ruby{極}{きは}まる!。
%
\ruby{馬鹿}{ば|か}
\g詰めruby{々々}{〳〵}しい。% 本来なら \g詰めruby{々々}{〴〵}しい。
%
ナニ?。
%
\ruby{戀愛}{れん|あい}に
\ruby{陷}{おちい}つて
\ruby{苦悶}{く|もん}しちよる、
%
それで
\ruby{朋友}{ほう|いう}の
\ruby{集會}{しふ|くわい}にも
\ruby{出席}{しゆつ|せき}しないと?。
%
たツ
\ruby{白痴野郎}{たは|け|や|らう}め、
%
\ruby{何}{なん}といふ
\ruby{事}{こつ}た。
%
そんな
\ruby{愚}{ぐ}な
\ruby{奴}{やつ}では
\ruby{無}{な}かつたが、
%
\ruby{{\換字{魔}}}{ま}にでも
\ruby{憑}{つ}かれ
\ruby{居}{を}つたか、
%
\ruby{下}{くだ}らない。
%
\ruby[g]{山瀬}{やませ}、
%
\ruby{貴樣}{き|さま}も
\ruby{幹事}{かん|じ}
\ruby{甲{\換字{斐}}}{が|ひ}がない。
%
\ruby{其樣}{そ|ん}な
\ruby{生溫}{なま|ぬる}つこい
\ruby{事}{こと}を
\ruby{云}{い}はす
\ruby{法}{はふ}が
\ruby{有}{あ}るかい!。
%
\ruby{領上}{えり|がみ}に
\ruby{手}{て}を
\ruby{掛}{か}けて
\ruby{引摺}{ひき|ず}つて
\ruby{來}{く}りやあ、
%
\ruby{一同}{みん|な}で
\ruby{引擲}{ひつ|ぱた}いて
\ruby{正氣}{しやう|き}に
\ruby{仕}{し}て
\ruby{{\換字{遣}}}{や}るのに。
%
ゑゝ、
%
\ruby{理由}{わ|け}を
\ruby{聞}{き}かぬ
\ruby{間}{うち}は
\ruby{知}{し}らぬが
\ruby{佛}{ほとけ}で
\ruby{腹}{はら}も
\ruby{立}{た}たなかつたが、
%
\ruby{聞}{き}いて
\ruby{見}{み}りやあ
\ruby{馬鹿}{ば|か}
\g詰めruby{々々}{〳〵}しくつて
\ruby{腹}{はら}が
\ruby{立}{た}つ。
%
\ruby[g]{山瀬}{やませ}!。
%
\ruby{一體}{いつ|たい}
\ruby{貴樣}{き|さま}が
\ruby{薄}{うす}つぺらで
\ruby{眞底}{しん|そこ}からの
\ruby{信實氣}{しつ|じつ|ぎ}が
\ruby{足}{た}らん。
%
\ruby{本來}{ほん|らい}
\ruby{我々}{われ|〳〵}
\ruby{七人}{しち|にん}は
\ruby{何樣}{ど|う}いふ
\ruby{{\換字{交}}{\換字{情}}}{な|か}だ。
%
みんな
\ruby{野州}{や|しう}の
\ruby{田舎}{ゐな|か}
\ruby{漢}{もの}、
%
\ruby{碌}{ろく}な
\ruby{親}{おや}を
\ruby{持}{も}つたものは
\ruby{一人}{ひと|り}も
\ruby{無}{な}くつて、
%
\ruby{役塲}{やく|ば}の% 原文通り「塲」
\ruby{書記}{しよ|き}や
\ruby{小學}{せう|がく}
\ruby{敎師}{けう|し}、
%
\ruby{乃公}{お|ら}あ
\ruby{人力車}{く|る|ま}も
\原本頁{6-1}%
\ruby{曳}{ひつ}ぱつた
\ruby{{\換字{貧}}}{ひん}
\ruby{書生}{しよ|せい}だが、
%
\ruby{自己}{う|ぬ}が
\ruby{腕臑}{うで|すね}で
\ruby{食}{く}ふ
\ruby{{\換字{貧}}乏}{びん|ばふ}
\ruby{同士}{どう|し}、
%
\ruby{何時}{い|つ}と
\ruby{無}{な}く
\ruby{知}{し}り
\ruby{合}{あ}ひになつた
\ruby{七人}{しち|にん}が、
%
\ruby[g]{男兒}{をとこ}と
\ruby{生}{うま}れて
\ruby[g]{此狀}{これ}ぢやあ
\ruby{死}{し}ねぬ、
%
\ruby{志}{こゝろざ}す
ところは
\ruby{異}{ちが}つても
\ruby{互}{たがひ}に
\ruby{助}{たす}け
\ruby{幇}{たす}け
\ruby{合}{あ}つて、
%
\ruby{或時}{ある|とき}は
\ruby{兄}{あに}となつて
\ruby{學資}{がく|し}も
\ruby{貢}{みつ}ぎ、
%
\ruby{或時}{ある|とき}は
\ruby{弟}{おとゝ}となつて
\ruby{恩}{おん}を
\ruby{報}{はう}じ、
%
\ruby{勵}{はげ}み
\ruby{合}{あ}ひ
\ruby{擁護}{か|ば}ひ
\ruby{合}{あ}つて
\ruby{{\換字{進}}}{すゝ}んで
\ruby{行}{い}つたら、
%
\ruby{世}{よ}に
\ruby{立}{た}つて
\ruby{生}{い}き
\ruby{甲{\換字{斐}}}{が|ひ}のある
\ruby{身}{み}
ともなれやうと、
%
\ruby{七人}{しち|にん}
\ruby{集}{あつ}まつた
\ruby{宇都宮}{う|つの|みや}の
\ruby{二荒山神社}{ふた|あら|やま|じん|じや}の
\ruby{廣{\換字{前}}}{ひろ|まへ}で、
%
\ruby{此}{こ}の
\ruby[h|]{願}{ねがひ}
\ruby{此}{こ}の
\ruby[h|]{心}{こゝろ}
\ruby{渝}{かは}るまじ、
%
\ruby{必}{かなら}ず
\ruby{信義}{しん|ぎ}を
\ruby{盡}{つく}し
\ruby{合}{あ}はんと、
%
\ruby{神}{かみ}に
\ruby{誓}{ちか}つた
\ruby[g]{{\換字{交}}{\換字{情}}}{なか}では
\ruby{無}{な}いか。
%
\ruby{指折}{ゆび|を}り
\ruby{數}{かぞ}ふれば
\ruby{{\換字{速}}}{はや}いもので
\ruby{既}{はや}
\ruby{七年}{しち|ねん}の
\ruby{往時}{むか|し}になるが、
%
\ruby{其時}{そ|れ}からといふものは
\ruby{段々}{だん|〴〵}と、
%
\ruby{苦}{くる}しい
\ruby{同士}{どう|し}で
\ruby{無理}{む|り}
\ruby{才覺}{さい|かく}、
%
\ruby{三人}{さん|にん}の
\ruby{財布}{さい|ふ}を
\ruby{揮}{ふる}つては
\ruby{一人}{いち|にん}の
\ruby{{\換字{遊}}學}{いう|がく}の
\ruby{支度}{し|たく}を
\ruby{拵}{こしら}へ、
%
\ruby{五人}{ご|にん}の
\ruby{着物}{き|もの}を
\ruby{賣}{う}つては
\ruby{一人}{いち|にん}の
\ruby{身}{み}の
\ruby{立}{た}つ
\ruby{本錢}{もと|で}とするといふ
\ruby{始末}{し|まつ}で、
%
ボツリ〳〵と
\ruby{皆}{みな}
\ruby{東京}{とう|きやう}へ、
%
\ruby{漸}{やうや}く
\原本頁{7-1}%
\ruby{這}{は}ひ
\ruby{出}{だ}して
それ〴〵に、
%
\ruby{志}{こゝろざ}す
\ruby{{\換字{道}}}{みち}へと
\ruby{身}{み}を
\ruby{入}{い}れた、
%
\ruby{如是}{かう|いふ}
\ruby{{\換字{交}}{\換字{情}}}{な|か}だのに
\ruby{何}{なん}の
\ruby{事}{こつ}た!。
%
\ruby{胸糞}{むな|くそ}の
\ruby{惡}{わる}い
\ruby{戀愛}{れん|あい}なんぞに
\ruby[g]{水野}{みづの}が
\ruby{{\換字{迷}}}{まよ}つてるなら
\ruby{何故}{な|ぜ}
\ruby{打棄}{うつ|ちや}つて
\ruby{置}{お}く?。
%
{\換字{志}}かも
\ruby[g]{羽{\換字{勝}}}{はがち}が
\ruby{始}{はじ}めて
\ruby{首尾}{しゆ|び}よく
\ruby{{\換字{遠}}洋漁業}{ゑん|やう|ぎよ|げふ}の
\ruby{長}{なが}い
\ruby{航海}{かう|かい}を、
%
\ruby{{\換字{終}}}{をは}つて
\ruby{來}{き}た
\ruby{今日}{け|ふ}の
\ruby{欣喜}{よろ|こび}の
\ruby{集會}{あつ|まり}に、
%
\ruby{自己}{お|の}が
\ruby{{\換字{勝}}手}{かつ|て}の
\ruby{女沙汰}{をんな|ざ|た}のために
\ruby{不參}{ふ|さん}とは、
%
\ruby{我々}{われ|〳〵}を
\ruby{踏}{ふ}み
\ruby{付}{つ}けた
\ruby{憎}{にく}い
\ruby{我儘}{わが|まゝ}。
%
\ruby[g]{山瀬}{やませ}
\ruby{汝}{きさま}は
\ruby{何故}{な|ぜ}
\ruby{打棄}{うつ|ちや}つて
\ruby{置}{お}く?。
%
\ruby{汝}{きさま}が
\ruby{新聞記者}{しん|ぶん|き|しや}になつた
\ruby{時}{とき}は、
%
\ruby{我々}{われ|〳〵}
\ruby{七人}{しち|にん}
\ruby{皆}{みな}
\ruby{揃}{そろ}つた。
%
\ruby{乃公}{お|れ}が
\ruby{士官候補生}{し|くわん|こう|ほ|せい}になつた
\ruby{時}{とき}にも
\ruby{皆}{みな}
\ruby{集}{あつ}まつて
\ruby{悅}{よろこ}んで
\ruby{吳}{く}れた。
%
\ruby[g]{羽{\換字{勝}}}{はがち}
\ruby{君}{くん}の
\ruby{今日}{け|ふ}の
\ruby{祝賀}{よろ|こび}の
\ruby{會}{くわい}には、
%
\ruby{楢井}{なら|い}は
\ruby{北海{\換字{道}}}{ほく|かい|だう}に
\ruby{行}{い}つて
\ruby{居}{を}り、
%
\ruby[g]{名倉}{なぐら}は
\ruby{病氣}{びやう|き}、
%
\ruby{二人}{ふた|り}
\ruby{缺}{か}けて
\ruby{居}{ゐ}るさへ
\ruby{殘念}{ざん|ねん}なに、
%
\ruby[g]{水野}{みづの}まで
\ruby{來}{こ}ぬので
\ruby[h|]{只}{たつた}% 後進入させないようにした
\ruby{四人}{よ|にん}、
%
\ruby{第一}{だい|いち}
\ruby[g]{羽{\換字{勝}}}{はがち}
\ruby{君}{くん}にも
\ruby{氣}{き}の
\ruby{毒}{どく}
\ruby{千萬}{せん|ばん}だ。
%
\ruby{戀愛}{れん|あい}も
\ruby{糞}{くそ}もあるものか、
%
\ruby{世間}{せ|けん}
\ruby{一統}{いつ|とう}の
\ruby{愚物}{ぐ|ぶつ}は
\ruby{知}{し}らず、
%
\ruby{何時}{い|つ}でも
\ruby{現在}{げん|ざい}に
\ruby{滿足}{まん|ぞく}せいで、
%
\ruby{永久}{えい|きう}に
\原本頁{8-1}%
\ruby{{\換字{進}}}{すゝ}んで
\ruby{{\換字{飽}}}{あ}くこと% TODO 異体字、関連字
\ruby{知}{し}らぬを
\ruby{理想}{り|さう}と
\ruby{定}{さだ}めた
\ruby{我我}{われ|〳〵}% 原本通り踊り字をつかwず「我我」
\ruby{七人}{しち|にん}、
%
\ruby{戀愛}{れん|あい}
なんぞといふ
アタ
\ruby{{\換字{嫌}}}{いや}らしい
\ruby{濕氣}{しつ|け}の
\ruby{蠹}{むし}に、% ここのみ「蠹」
%
\ruby{魂魄}{たま|しひ}を
\ruby{蝕}{くは}せて
\ruby{居}{ゐ}る
\ruby{間}{ま}は
\ruby{無}{な}い
\ruby{筈}{はず}。
%
\ruby{一體}{いつ|たい}
\ruby{全體}{ぜん|たい}
\ruby{癪}{しやく}に
\ruby{觸}{さは}る!。
%
\ruby{何}{なに}を
\ruby{讀}{よ}んでも
\ruby{何處}{ど|こ}へ
\ruby{行}{い}つても、
%
\ruby{此頃}{この|ごろ}は
\ruby{戀愛}{れん|あい}といふ
\ruby{奴}{やつ}ばかり
\ruby{轉}{ころ}げて
\ruby{居}{ゐ}をるが、
%
\ruby{戀愛}{れん|あい}たあ
\ruby{何}{なん}だ?、
%
\ruby{何}{なん}だ
\ruby{正體}{しやう|たい}は?。
%
\ruby{自己}{う|ぬ}から
\ruby{見}{み}りやあ
\ruby{貴}{い}いか
\ruby{知}{し}らぬが、
%
\ruby{他}{ひと}から
\ruby{見}{み}りやあ
\ruby{石决明}{あは|びつ|かひ}を
\ruby{當}{あ}てがつて
\ruby{{\換字{遣}}}{や}る
\ruby{價値}{ね|うち}も
\ruby{無}{な}い
\ruby{馬糞}{ば|ふん}に
\ruby{劣}{おと}つた
\ruby{貨物}{しろ|もの}で、
%
\ruby{高}{たか}が
\ruby{女}{をんな}に
びりつく
\ruby{事}{こと}だ!。
%
\ruby[g]{水野}{みづの}は
\ruby{釅}{きぶ}い
\ruby{醋}{す}のやうな
\ruby{恐}{おそ}ろしい
ところのある
\ruby{奴}{やつ}ぢやつたが、
%
\ruby{{\換字{浮}}世}{うき|よ}に
\ruby{感染}{か|ぶ}れたのは
\ruby{氣}{き}が
\ruby{{\換字{緩}}}{ゆる}んだ
\ruby{歟}{か}。
%
\ruby{打棄}{うつ|ちや}つて
\ruby{置}{おい}ては
\ruby{利益}{た|め}にならん。
%
\ruby{直}{すぐ}
これから
\ruby{行}{い}つて
\ruby{引摺}{ひき|ず}つて
\ruby{來}{こ}やう。
%
さあ
\ruby[g]{山瀬}{やませ}!
\ruby{一緖}{いつ|しよ}に
\ruby{行}{ゆ}け、
%
\ruby{立}{た}たぬかやい。
%
\ruby[g]{水野}{みづの}めを
\ruby{引張}{ひつ|ぱ}つて
\ruby{來}{き}て
\ruby{此處}{こ|ゝ}で
\ruby{諫}{いさ}めて、
%
\ruby{諫}{いさ}めて
\ruby{聽}{き}かずば
\ruby{擲}{たゝ}き
\ruby{撲}{なぐ}つて、
%
\ruby{正氣}{しやう|き}に
\ruby{{\換字{返}}}{かへ}らせて
\ruby{吳}{く}れにやならぬ、
%
さあ
\ruby{立}{た}て
\ruby[g]{山瀬}{やませ}!。
』

\原本頁{9-2}%
と
\ruby{云}{い}ひざまに、
%
\ruby{五{\換字{分}}}{ご|ぶ}の
\ruby{慷慨}{かう|がい}、
%
\ruby{五{\換字{分}}}{ご|ぶ}の
\ruby{醉}{ゑひ}、% 「醉」は原本通り「ゑ」で調整
%
\ruby[g]{山瀬}{やませ}が
\ruby{肩頭}{かた|さき}を
\ruby{引攫}{ひつ|つか}んで
\ruby{氣勢}{いき|ほひ}
\ruby{猛}{もう}に
\ruby{立上}{たち|あが}つたり。
