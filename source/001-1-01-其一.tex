\makeatletter
\@ifundefined{全三巻@一括ビルド}{%
{\huge
\ruby{天}{そら} う つ % 空白有り
\ruby{浪}{なみ}}  {\normalsize 第一}
\vspace*{3zw}

\Entry{其一}
}
\makeatother

% メモ 校正終了 2024-03-28 2024-05-22 2024-06-15
\原本頁{1-3}%
\ruby{秋}{あき}は
\ruby[g]{海樓}{かいろう}の
\ruby[g]{垂簾}{すだれ }に
\ruby{動}{うご}きて、
%
ばつと
\ruby{吹}{ふ}き
\ruby{來}{く}る
\ruby{沖}{おき}の
\ruby{風}{かぜ}は、
%
\ruby[g]{夕日}{ゆふひ }の
\ruby{餘}{よ}
\ruby[||j>]{光}{くわう}% 行末行頭の境界付近なので特例処置を施す
\ruby[||j>]{美}{ うる}はしきが
\ruby{中}{なか}に、
%
\ruby[g]{無限}{む げん}の
\ruby[||j>]{爽}{さう}
\ruby[||j>]{凉}{りやう}の
% \ruby{爽凉}{さう|りやう}の
\ruby{氣}{き}を
\ruby{齎}{もた}らせば、
%
\ruby[g]{白帆}{しらほ }
\ruby{明}{あか}るき
\ruby[g]{{\換字{遠}}方}{とほく }の
\ruby{{\換字{船}}}{ふね}の
\ruby[g]{數々}{かず〳〵}も、
%
\ruby[||j>]{{\換字{鉛}}}{なまり}
\ruby[||j>]{色}{ いろ}なして
% \ruby{{\換字{鉛}}色}{なまり|いろ}なして
\ruby[g]{漫々}{まん〳〵}たる
\ruby{潮}{うしほ}の
\ruby{果}{はて}に
\ruby{却}{かへ}つて
\ruby[g]{物淋}{ものさび}しう
\ruby{見}{み}え
\ruby{渡}{わた}りつ、
%
\ruby[g]{竹芝}{たけしば}の
\ruby{浦}{うら}の
\ruby{浪}{なみ}
\ruby{靜}{しづ}かに、
%
\ruby{增上寺}{ぞう|じやう|じ}の
\ruby[g]{鐘聲}{か ね }に
\ruby{暮}{く}れ
\ruby{行}{ゆ}かんとす。

\原本頁{1-8}%
\ruby{此}{こ}の
\ruby[||j>]{夕}{ゆふべ}
\ruby[||j>]{此}{ こ}の% 原本に合わせて調整
\ruby{時}{とき}、
%
『
\ruby{見}{み}はらし
』の
\ruby[||j>]{樓}{ろう}
\ruby[||j>]{上}{じよう}の
% \ruby{樓上}{ろう|じよう}の
\ruby[g]{一室}{いつしつ}に、
%
\ruby{貸}{か}し
\ruby[g]{{\換字{浴}}衣}{ゆ かた}の
\ruby[g]{胸元}{むなもと}% 「、『見はらし』」の部分の影響か改行制御できず
ゆたかに
くつろげて、
%
\ruby{醉}{ゑひ}に% 「醉」は原本通り「ゑ」で調整
\ruby{嘯}{うそぶ}く
\ruby{大胡坐}{おほ|あぐ|ら}、
%
たゞ% 原本は「ヾ」片仮名繰{\換字{返}}し記号(濁点)を使用してる
\ruby{秋}{あき}の
\ruby[g]{飮酒}{さ け }に
\ruby{宜}{よろ}しきを
\原本頁{2-1}\改行%
\ruby{知}{し}つて
\ruby{其}{そ}の
\ruby{他}{た}を
\ruby{知}{し}らぬ
\ruby[g]{面構}{つらがま}へ
きび〳〵と、
%
あはれも
\ruby[g]{絲瓜}{へちま }も
あるものか、
%
\ruby{鴫}{しぎ}が
\ruby{飛}{と}んだら
\ruby{撃}{う}つて
\ruby[g]{下物}{さかな }、
%
と
\ruby{云}{い}はぬ
ばかりの
\ruby{顏}{かほ}つきして、
%
いづれも
\ruby{勇}{いさ}みを
\ruby{含}{ふく}む
\ruby[g]{酒盃}{さかづき}の
\ruby{{\換字{遣}}}{や}り
\ruby{取}{と}り、
%
\ruby{火}{ひ}の
\ruby{珠}{たま}も
\ruby{挾}{はさ}んで
\ruby{食}{く}ふべき
\ruby[g]{年齡}{としばへ}の
\ruby{勢}{いきほ}ひに、
%
\ruby[g]{此方}{こなた }の% ルビ調整(原本通り)
\ruby[g]{壯語}{さうご }、
%
\ruby[g]{彼方}{かなた }の
\ruby[g]{傲語}{がうご }、
%
\ruby{或}{あるひ}は
\ruby{彼}{かれ}
\ruby{此}{これ}
\ruby[g]{哄然}{ど つ }と
\ruby[g]{一齊}{いちど }の
\ruby[g]{天狗}{てんぐ }
\ruby{笑}{わら}ひの
\ruby{響}{どよみ}の
\ruby{中}{うち}に、
%
\ruby[g]{間{\換字{近}}}{ま ぢか}く
\ruby{{\換字{通}}}{とほ}る
\ruby[g]{滊車}{き しや}の
\ruby{音}{おと}をも
\ruby{埋}{うづ}めて
\ruby[g]{仕舞}{し ま }ふまで、
%
\ruby{無邪氣}{む|じや|き}に
\ruby{睦}{むつ}み
\ruby{語}{かた}らへる
\ruby[g]{四人}{よ にん}
\ruby{{\換字{連}}}{づれ}
あり。

\原本頁{2-7}%
\ruby[g]{陽氣}{やうき }の
\ruby[||j>]{歡}{くわん}
\ruby[||j>]{笑}{ せう}は
% \ruby{歡笑}{くわん|せう}は
\ruby{一}{ひ}トしきり
\ruby{濟}{す}みて、
%
\ruby{今}{いま}しも
\ruby[g]{談話}{はなし }は
\ruby{少}{すこ}し
\ruby{沈}{しづ}みぬ。

\原本頁{2-8}%
\ruby{手}{て}さき
\ruby[g]{頸筋}{くびすぢ}に
\ruby[g]{洋服}{やうふく}の
\ruby{痕}{あと}
\ruby[g]{{\換字{判}}然}{はつきり}と
\ruby{知}{し}れて、
%
\ruby{誰}{た}が
\ruby{眼}{め}にも
\ruby[g]{{\換字{船}}人}{ふなのり}と
\ruby{映}{うつ}る
\ruby[<j||]{赭}{あから}% 行末行頭の境界付近なので特例処置を施す
\ruby[||j>]{顏}{がほ}の
% \ruby{赭顏}{あから|がほ}の
\ruby{日}{ひ}に
\ruby{焦}{や}けきつたる
\ruby[g]{羽{\換字{勝}}}{は がち}
\ruby[g]{千{\換字{造}}}{せんざう}は、
%
\ruby[g]{酒盃}{さかづき}を
\ruby{擧}{あ}げて
\ruby{一}{ひ}ト
\ruby{口}{くち}
\ruby{飮}{の}みしが、
%
\ruby{不興氣}{ふ|きよう|げ}に
\ruby[g]{復下}{またした}に
\ruby{置}{お}きて、

\原本頁{2-11}%
『
フーム
』

\原本頁{3-1}%
とばかり
\ruby[g]{力無}{ちからな}く
\ruby{答}{こた}へつ、
%
\ruby{{\換字{猶}}}{なほ}
\ruby{其}{そ}の
\ruby[g]{對手}{あひて }の
\ruby[g]{何事}{なにごと}をか
\ruby{語}{かた}り
\ruby{添}{そ}ふるを
\ruby{待}{ま}つが
\ruby{如}{ごと}き
\ruby{意}{こゝろ}を
\ruby{其}{そ}の
\ruby[g]{語氣}{ご き }に
\ruby{現}{あらは}したり。

\原本頁{3-3}%
\ruby[g]{羽{\換字{勝}}}{は がち}に
\ruby{對}{むか}ひて
\ruby{坐}{ざ}せる
\ruby[g]{小男}{こをとこ}の、
%
\ruby[||j>]{面}{おもて}% 原本に合わせて調整
\ruby[||j>]{淸}{ きよ}らにして
\ruby[g]{桃花}{とうくわ}の
\ruby{如}{ごと}き
\ruby[g]{山瀬}{やませ }
\ruby[g]{荒吉}{あらきち}は
\ruby{其}{その}
\ruby{意}{い}を
\ruby{悟}{さと}つて、
%
\ruby{果}{はた}して
\ruby{直}{たゞち}に
\ruby[g]{言葉}{ことば }を
\ruby{足}{た}しぬ。

\原本頁{3-5}%
『
ト
\ruby{云}{い}ふ
\ruby[g]{次第}{し だい}なので
\ruby[g]{水野}{みづの }
\ruby{君}{くん}は
\ruby{來}{こ}んのさ。
%
\ruby{今}{いま}
\ruby{話}{はな}した
\ruby[||j>]{内}{ない}
\ruby[||j>]{{\換字{情}}}{じやう}も
% \ruby{内{\換字{情}}}{ない|じやう}も
\ruby{解}{わか}つて
\ruby{居}{ゐ}たので、
%
\ruby[g]{今日}{け ふ }の
\ruby[||j>]{會}{くわい}
\ruby[||j>]{合}{ がふ}の
% \ruby{會合}{くわい|がふ}の
\ruby{發起人}{ほつ|き|にん}の
\ruby{僕}{ぼく}は、
%
\ruby[g]{十{\換字{分}}}{じふぶん}に
\ruby[g]{{\換字{情}}理}{じやうり}を
\ruby{盡}{つく}した
\ruby[g]{手紙}{て がみ}を
\ruby{與}{や}つて、
%
\ruby[g]{是非}{ぜ ひ }
\ruby{出}{で}て
\ruby{來}{く}るやうにと
\ruby{勸}{すゝ}めたんだが、
%
たゞ% 原本は「ヾ」片仮名繰{\換字{返}}し記号(濁点)を使用してる
\ruby[||j>]{差}{さし}
\ruby[||j>]{支}{つかへ}があつて
% \ruby{差支}{さし|つかへ}があつて
\ruby{行}{ゆ}かれないといふ
\ruby[g]{冷淡}{れいたん}
\ruby{極}{きは}まる
\ruby[g]{{\換字{返}}事}{へんじ }なんで、
%
\ruby[g]{仕方}{し かた}が
\ruby{無}{な}いと
\ruby[g]{斷念}{あきら }めて
\ruby[g]{仕舞}{し ま }つた。
%
\ruby{實}{じつ}に
\ruby[g]{水野}{みづの }
\ruby{君}{くん}にも
\ruby[g]{似合}{に あ }はない、
%
\ruby[g]{全然}{まるで }
\ruby[g]{無茶}{む ちや}
\ruby[g]{苦茶}{く ちや}になつて
\ruby{居}{ゐ}られるのだからね。
』

\原本頁{3-11}%
\ruby{見}{み}る〳〵
\ruby[g]{羽{\換字{勝}}}{は がち}が
\ruby{面}{おもて}には
\ruby[||j>]{憂}{いう}
\ruby[||j>]{色}{しよく}
% \ruby[||j>]{憂色}{いう|しよく}
\ruby[||j>]{現}{ あら}はれ、% 原本に合わせて調整
%
その
\ruby{眼}{め}は
\ruby[g]{沈思}{ちんし }に
\ruby[g]{凝然}{じ つ }と
\ruby{動}{うご}かずなりたり。

\原本頁{4-2}%
\ruby[g]{羽{\換字{勝}}}{は がち}が
\ruby[g]{左方}{ひだり }に
\ruby{坐}{ざ}して
\ruby[g]{默々}{もく〳〵}と
\ruby{飮}{の}み
\ruby{居}{ゐ}し
\ruby[g]{骨太}{ほねぶと}
\ruby[g]{岩疊}{がんでふ}づくりの
\ruby[g]{日方}{ひ かた}
\ruby[g]{八郞}{はちらう}は、
%
\ruby[g]{突然}{とつぜん}として
\ruby{牛}{うし}の
\ruby{吼}{ほ}ゆるが
\ruby{如}{ごと}くに
\ruby{呌}{さけ}び
\ruby{出}{だ}し、

\原本頁{4-4}%
『
\ruby[g]{山瀬}{やませ }、
%
\ruby[g]{貴樣}{き さま}も
\ruby{今}{いま}は
\ruby[g]{堂々}{だう〴〵}たる
\ruby{新聞記者}{しん|ぶん|き|しや}だ。
%
\ruby[g]{往時}{むかし }のやうに
\ruby{想像談}{さう|〴〵|だん}や
\ruby[g]{法螺}{ほ ら }
\ruby[<j||]{話}{ばなし}は
\ruby{語}{かた}るまいな。
』

\原本頁{4-6}%
と、
%
\ruby{詰}{なじ}り
\ruby[g]{氣味}{ぎ み }に
\ruby{問}{と}ひ
\ruby{糺}{たゞ}せば、
%
\ruby[g]{山瀬}{やませ }は
\ruby{聊}{いさゝ}か
\ruby[g]{怫然}{む つ }として、

\原本頁{4-7}%
『
\ruby[g]{日方}{ひ かた}
\ruby{陸軍少尉}{りく|ぐん|せう|ゐ}
\ruby{殿}{どの}に
\ruby{伺}{うかゞ}ひます。
%
\ruby[g]{報告}{はうこく}は
\ruby{無責任}{む|せき|にん}を
\ruby{以}{もつ}て
\ruby[g]{作爲}{さくゐ }すべきもので
ござりまする
\ruby{歟}{か}。
%
はゝはゝはゝ。% 踊り字調整「〻(二の字点、揺すり点)に見えるが(ゝ)」% 原本の最後の「〻」には見えない
』

\原本頁{4-9}%
と
\ruby{{\換字{遣}}}{や}り
\ruby{{\換字{返}}}{かへ}して
\ruby{笑}{わら}ふ。

\原本頁{4-10}%
\ruby[g]{日方}{ひ かた}は
\ruby[g]{山瀬}{やませ }の
\ruby[g]{戱言}{たはむれ}には
\ruby{頓着無}{とん|ぢやく|な}く、
%
\ruby{怒}{いか}れるが
\ruby{如}{ごと}く
\ruby{眞面目}{ま|じ|め}になりて
\改行% 校正作業の簡略化のため
、

\原本頁{4-11}%
『
ムゝ、% ルビ調整(原本通り)平仮名繰返し記号踊り字。本来は「ヽ]
%
して
\ruby{見}{み}れば
\ruby{全}{まつた}く
\ruby[g]{事實}{じ じつ}と
\ruby{見}{み}える。
%
イヤ
\ruby{怪}{け}しからん、
%
\ruby{實}{じつ}に
\ruby{怪}{け}しからん。
%
\ruby{何}{なん}だ!。
%
\ruby[g]{愚劣}{ぐ れつ}
\ruby{極}{きは}まる!。
%
\ruby[g]{馬鹿}{ば か }
\ruby[g]{々々}{ 〳〵 }しい。% 本来なら \ruby[g]{々々}{ 〴〵 }しい。
%
ナニ?。
%
\ruby[g]{戀愛}{れんあい}に
\ruby{陷}{おちい}つて
\ruby[g]{苦悶}{く もん}しちよる、
%
それで
\ruby[g]{朋友}{ほういう}の
\ruby[||j>]{集}{しふ}
\ruby[||j>]{會}{くわい}にも
% \ruby{集會}{しふ|くわい}にも
\ruby[||j>]{出}{しゆつ}
\ruby[||j>]{席}{ せき}しないと?。
% \ruby{出席}{しゆつ|せき}しないと?。
%
たツ
\ruby{白痴野郎}{た|はけ|や|らう}め、
%
\ruby{何}{なん}といふ
\ruby{事}{こつ}た。
%
そんな
\ruby{愚}{ぐ}な
\ruby{奴}{やつ}では
\ruby{無}{な}かつたが、
%
\ruby{{\換字{魔}}}{ま}にでも
\ruby{憑}{つ}かれ
\ruby{居}{を}つたか、
%
\ruby{下}{くだ}らない。
%
\ruby[g]{山瀬}{やませ }、
%
\ruby[g]{貴樣}{き さま}も
\ruby[g]{幹事}{かんじ }
\ruby[g]{甲{\換字{斐}}}{が ひ }がない。
%
\ruby[g]{其樣}{そ ん }な
\ruby[g]{生溫}{なまぬる}つこい
\ruby{事}{こと}を
\ruby{云}{い}はす
\ruby{法}{はふ}が
\ruby{有}{あ}るかい!。
%
\ruby[g]{領上}{えりがみ}に
\ruby{手}{て}を
\ruby{掛}{か}けて
\ruby[g]{引摺}{ひきず }つて
\ruby{來}{く}りやあ、
%
\ruby[g]{一同}{みんな }で
\ruby[g]{引擲}{ひつぱた}いて
\ruby[g]{正氣}{しやうき}に
\ruby{仕}{し}て
\ruby{{\換字{遣}}}{や}るのに。
%
ゑゝ、
%
\ruby[g]{理由}{わ け }を
\ruby{聞}{き}かぬ
\ruby{間}{うち}は
\ruby{知}{し}らぬが
\ruby{佛}{ほとけ}で
\ruby{腹}{はら}も
\ruby{立}{た}たなかつたが、
%
\ruby{聞}{き}いて
\ruby{見}{み}りやあ
\ruby[g]{馬鹿}{ば か }
\ruby[g]{々々}{ 〳〵 }しくつて% 本来なら \ruby[g]{々々}{ 〴〵 }しくつて
\ruby{腹}{はら}が
\ruby{立}{た}つ。
%
\ruby[g]{山瀬}{やませ }!
\改行% 校正作業の簡略化のため
。
%
\原本頁{5-9}\改行%
\ruby[g]{一體}{いつたい}
\ruby[g]{貴樣}{き さま}が
\ruby{薄}{うす}つぺらで
\ruby[g]{眞底}{しんそこ}からの
\ruby{信實氣}{しつ|じつ|ぎ}が
\ruby{足}{た}らん。
%
\ruby[g]{本來}{ほんらい}
\ruby[g]{我々}{われ〳〵}
\ruby[g]{七人}{しちにん}は% 原本には漢数字「七」のルビ無し
\ruby[g]{何樣}{ど う }いふ
\ruby[g]{{\換字{交}}{\換字{情}}}{な か }だ。
%
みんな
\ruby[g]{野州}{や しう}の
\ruby[g]{田舎}{ゐなか }
\ruby{{\換字{漢}}}{もの}、
%
\ruby{碌}{ろく}な
\ruby{親}{おや}を
\ruby{持}{も}つたものは
\ruby[g]{一人}{ひとり }も
\ruby{無}{な}くつて、
%
\ruby[g]{役塲}{やくば }の% 原文通り「塲」
\ruby[g]{書記}{しよき }や
\ruby[g]{小學}{せうがく}
\ruby[g]{敎師}{けうし }、
%
\ruby[g]{乃公}{お ら }あ
\ruby{人力車}{く|る|ま}も
\原本頁{6-1}%
\ruby{曳}{ひつ}ぱつた
\ruby{{\換字{貧}}}{ひん}
\ruby[g]{書生}{しよせい}だが、
%
\ruby[g]{自己}{う ぬ }が
\ruby[g]{腕臑}{うですね}で
\ruby{食}{く}ふ
\ruby[g]{{\換字{貧}}乏}{びんばふ}
\ruby[g]{同士}{どうし }、
%
\ruby[g]{何時}{い つ }と
\ruby{無}{な}く
\原本頁{6-2}\改行%
\ruby{知}{し}り
\ruby{合}{あ}ひになつた
\ruby[g]{七人}{しちにん}が、% 原本には漢数字「七」のルビ無し
%
\ruby[g]{男兒}{をとこ }と
\ruby{生}{うま}れて
\ruby[g]{此狀}{こ れ }ぢやあ
\ruby{死}{し}ねぬ
%
\makeatletter
\@ifundefined{デバッグ@ビルド}{%
  、%
  \ruby[<j>]{志}{こゝろざ}す% ルビ調整
  }{%
  \ruby[<g>]{、志}{こゝろざ}す% ルビ調整(行末の特殊処理)「、」部分にルビを押し込む
}%
\makeatother
ところは
\ruby{異}{ちが}つても
\ruby{互}{たがひ}に
\ruby{助}{たす}け
\ruby{幇}{たす}け
\ruby{合}{あ}つて、
%
\ruby[g]{或時}{あるとき}は
\ruby{兄}{あに}となつて
\ruby[g]{學資}{がくし }も
\ruby{貢}{みつ}ぎ、
%
\ruby[g]{或時}{あるとき}は
\ruby{弟}{おとゝ}となつて
\ruby{恩}{おん}を
\ruby{報}{はう}じ、
%
\ruby{勵}{はげ}み
\ruby{合}{あ}ひ
\ruby[g]{擁護}{か ば }ひ
\ruby{合}{あ}つて
\原本頁{6-5}\改行%
\ruby{{\換字{進}}}{すゝ}んで
\ruby{行}{い}つたら、
%
\ruby{世}{よ}に
\ruby{立}{た}つて
\ruby{生}{い}き
\ruby[g]{甲{\換字{斐}}}{が ひ }のある
\ruby{身}{み}
ともなれやうと
\改行% 校正作業の簡略化のため
、
%
\原本頁{6-6}\改行%
\ruby[g]{七人}{しちにん}% 原本には漢数字「七」のルビ無し
\ruby{集}{あつ}まつた
\ruby{宇都宮}{う|つの|みや}の
\ruby{二荒山神社}{ふた|あら|やま|じん|じや}の
\ruby[g]{廣{\換字{前}}}{ひろまへ}で、
%
\ruby{此}{こ}の
\ruby[||j>]{願}{ねがひ}
\ruby[||j>]{此}{ こ}の% 原本に合わせて調整
\ruby[||j>]{心}{こゝろ}
\ruby[||j>]{渝}{ かは}るまじ、% 原本に合わせて調整
%
\ruby{必}{かなら}ず
\ruby[g]{信義}{しんぎ }を
\ruby{盡}{つく}し
\ruby{合}{あ}はんと、
%
\ruby{神}{かみ}に
\ruby{誓}{ちか}つた
\ruby[g]{{\換字{交}}{\換字{情}}}{な か }では
\ruby{無}{な}いか。
%
\原本頁{6-8}\改行%
\ruby[g]{指折}{ゆびを }り
\ruby{數}{かぞ}ふれば
\ruby{{\換字{速}}}{はや}いもので
\ruby{既}{はや}
\ruby[g]{七年}{しちねん}の% 原本には漢数字「七」のルビ無し
\ruby[g]{往時}{むかし }になるが、
%
\ruby[g]{其時}{そ れ }からといふものは
\ruby[g]{段々}{だん〳〵}と、% ルビ調整(原本通り)踊り字表記
%
\ruby{苦}{くる}しい
\ruby[g]{同士}{どうし }で
\ruby[g]{無理}{む り }
\ruby[g]{才覺}{さいかく}、
%
\ruby[g]{三人}{さんにん}の
\ruby[g]{財布}{さいふ }を
\ruby{揮}{ふる}つては
\ruby[g]{一人}{いちにん}の
\ruby[g]{{\換字{遊}}學}{いうがく}の
\ruby[g]{支度}{し たく}を
\ruby{拵}{こしら}へ、
%
\ruby[g]{五人}{ご にん}の
\ruby[g]{着物}{き もの}を
\ruby{賣}{う}つては
\ruby[g]{一人}{いちにん}の
\ruby{身}{み}の
\ruby{立}{た}つ
\ruby[g]{本錢}{もとで }とするといふ
\ruby[g]{始末}{し まつ}で、
%
ボツリ〳〵と
\ruby{皆}{みな}
\ruby[||j>]{東}{とう}
\ruby[||j>]{京}{きやう}へ、
% \ruby{東京}{とう|きやう}へ、
%
\ruby{漸}{やうや}く
\原本頁{7-1}\改行%
\ruby{這}{は}ひ
\ruby{出}{だ}して
それ〴〵に、
%
\ruby[<j>]{志}{こゝろざ}す
\ruby{{\換字{道}}}{みち}へと
\ruby{身}{み}を
\ruby{入}{い}れた、
%
\ruby[g]{如是}{かういふ}
\ruby[g]{{\換字{交}}{\換字{情}}}{な か }だのに
\ruby{何}{なん}の
\ruby{事}{こつ}た!。
%
\ruby[g]{胸糞}{むなくそ}の
\ruby{惡}{わる}い
\ruby[g]{戀愛}{れんあい}なんぞに
\ruby[g]{水野}{みづの }が
\ruby{{\換字{迷}}}{まよ}つてるなら
\ruby[g]{何故}{な ぜ }
\ruby[g]{打棄}{うつちや}つて
\ruby{置}{お}く?。
%
{\換字{志}}かも
\ruby[g]{羽{\換字{勝}}}{は がち}が
\ruby{始}{はじ}めて
\ruby[g]{首尾}{しゆび }よく
\ruby{{\換字{遠}}洋漁業}{ゑん|やう|ぎよ|げふ}の
\ruby{長}{なが}
\原本頁{7-4}\改行%
い
\ruby[g]{航海}{かうかい}を、
%
\ruby{{\換字{終}}}{をは}つて
\ruby{來}{き}た
\ruby[g]{今日}{け ふ }の
\ruby[g]{欣喜}{よろこび}の
\ruby[g]{集會}{あつまり}に、
%
\ruby[g]{自己}{お の }が
\ruby[g]{{\換字{勝}}手}{かつて }の
\ruby[||j>]{女}{をんな}
\ruby[||j>]{沙}{ ざ}
\原本頁{7-5}\改行%
\ruby[||j>]{汰}{た}のために
\ruby[g]{不參}{ふ さん}とは、
%
\ruby[g]{我々}{われ〳〵}を
\ruby{踏}{ふ}み
\ruby{付}{つ}けた
\ruby{憎}{にく}い
\ruby[g]{我儘}{わがまゝ}。
%
\ruby[||j>]{山瀬}{やま|せ}
\ruby[<j||]{汝}{きさま}は% ルビ調整(原本通り)
\ruby{何}{な}
\原本頁{7-6}\改行%
\ruby{故}{ぜ}
\ruby[g]{打棄}{うつちや}つて
\ruby{置}{お}く?。
%
\ruby{汝}{きさま}が
\ruby{新聞記者}{しん|ぶん|き|しや}になつた
\ruby{時}{とき}は、
%
\ruby[g]{我々}{われ〳〵}
\ruby[g]{七人}{しちにん}% 原本には漢数字「七」のルビ無し
\ruby{皆}{みな}
\ruby{揃}{そろ}
\原本頁{7-7}\改行%
つた。
%
\ruby[g]{乃公}{お れ }が
\ruby{士官候補生}{し|くわん|こう|ほ|せい}になつた
\ruby{時}{とき}にも
\ruby{皆}{みな}
\ruby{集}{あつ}まつて
\ruby{悅}{よろこ}んで
\ruby{吳}{く}れた。
%
\ruby[g]{羽{\換字{勝}}}{は がち}
\ruby{君}{くん}の
\ruby[g]{今日}{け ふ }の
\ruby[g]{祝賀}{よろこび}の
\ruby{會}{くわい}には、
%
\ruby[g]{楢井}{ならい }は
\ruby{北海{\換字{道}}}{ほく|かい|だう}に
\ruby{行}{い}つて
\ruby{居}{を}り
\改行% 校正作業の簡略化のため
、
%
\原本頁{7-9}\改行%
\ruby[g]{名倉}{な ぐら}は
\ruby[g]{病氣}{びやうき}、
%
\ruby[g]{二人}{ふたり }
\ruby{缺}{か}けて
\ruby{居}{ゐ}るさへ
\ruby[g]{殘念}{ざんねん}なに、
%
\ruby[g]{水野}{みづの }まで
\ruby{來}{こ}ぬので
\原本頁{7-10}\改行%
\ruby[||j>]{只}{たつた}% 後突き出させないようにした
\ruby[||j>]{四人}{ よ|にん}、
%
\ruby[g]{第一}{だいいち}
\ruby[g]{羽{\換字{勝}}}{は がち}
\ruby{君}{くん}にも
\ruby{氣}{き}の
\ruby{毒}{どく}
\ruby[g]{千萬}{せんばん}だ。
%
\ruby[g]{戀愛}{れんあい}も
\ruby{糞}{くそ}もあるものか
\改行% 校正作業の簡略化のため
、
%
\原本頁{7-11}\改行%
\ruby[g]{世間}{せ けん}
\ruby[g]{一統}{いつとう}の
\ruby[g]{愚物}{ぐ ぶつ}は
\ruby{知}{し}らず、
%
\ruby[g]{何時}{い つ }でも
\ruby[g]{現在}{げんざい}に
\ruby[g]{滿足}{まんぞく}せいで、
%
\ruby[g]{永久}{えいきう}に
\原本頁{8-1}\改行%
\ruby{{\換字{進}}}{すゝ}んで
\ruby{{\換字{飽}}}{あ}くこと
\ruby{知}{し}らぬを
\ruby[g]{理想}{り さう}と
\ruby{定}{さだ}めた
\ruby[g]{我我}{われ〳〵}% 原本通り非踊り字表記「我我」
\ruby[g]{七人}{しちにん}、% 原本には漢数字「七」のルビ無し
%
\ruby[g]{戀愛}{れんあい}
なんぞといふ
アタ
\ruby{{\換字{嫌}}}{いや}らしい
\ruby[g]{濕氣}{しつけ }の
\ruby{蠹}{むし}に、% ここのみ「蠹」
%
\ruby[g]{魂魄}{たましひ}を
\ruby{蝕}{くは}せて
\ruby{居}{ゐ}る
\ruby{間}{ま}は
\ruby{無}{な}い
\ruby{筈}{はず}。
%
\原本頁{8-1}\改行%
\ruby[g]{一體}{いつたい}
\ruby[g]{全體}{ぜんたい}
\ruby{癪}{しやく}に
\ruby{觸}{さは}る!。
%
\ruby{何}{なに}を
\ruby{讀}{よ}んでも
\ruby[g]{何處}{ど こ }へ
\ruby{行}{い}つても、
%
\ruby[g]{此頃}{このごろ}は
\ruby[g]{戀愛}{れんあい}といふ
\ruby{奴}{やつ}ばかり
\ruby{轉}{ころ}げて
\ruby{居}{ゐ}をるが、% 原本通りで(を)がある(国会図書館 コマ番号 8/134 p-8 l-4)
%
\ruby[g]{戀愛}{れんあい}たあ
\ruby{何}{なん}だ?、
%
\ruby{何}{なん}だ
\ruby[<j||]{正}{しやう}% 行末行頭の境界付近なので特例処置を施す
\ruby[<j||]{體}{たい}は?。
% \ruby{正體}{しやう|たい}は?。
%
\ruby[g]{自己}{う ぬ }から
\ruby{見}{み}りやあ
\ruby{貴}{い}いか
\ruby{知}{し}らぬが、
%
\ruby{他}{ひと}から
\ruby{見}{み}りやあ
\ruby{石决明}{あは|びつ|かひ}を
\ruby{當}{あ}てがつて
\ruby{{\換字{遣}}}{や}る
\ruby[g]{價値}{ね うち}も
\ruby{無}{な}い
\ruby[g]{馬糞}{ば ふん}に
\ruby{劣}{おと}つた
\ruby[g]{貨物}{しろもの}で、
%
\ruby{高}{たか}が
\ruby{女}{をんな}に
びりつく
\ruby{事}{こと}だ!。
%
\ruby[g]{水野}{みづの }は
\ruby{釅}{きぶ}い
\ruby{醋}{す}のやうな
\ruby{恐}{おそ}ろしい
ところのある
\原本頁{8-8}\改行%
\ruby{奴}{やつ}ぢやつたが、
%
\ruby[g]{{\換字{浮}}世}{うきよ }に
\ruby[g]{{\換字{感}}染}{か ぶ }れたのは
\ruby{氣}{き}が
\ruby{{\換字{緩}}}{ゆる}んだ
\ruby{歟}{か}。
%
\ruby[g]{打棄}{うつちや}つて
\ruby{置}{おい}ては
\ruby[g]{利益}{た め }にならん。
%
\ruby{直}{すぐ}
これから
\ruby{行}{い}つて
\ruby[g]{引摺}{ひきず }つて
\ruby{來}{こ}やう。
%
さあ
\ruby[g]{山瀬}{やませ }!\inhibitglue{}%
\ruby[g]{一緖}{いつしよ}に
\ruby{行}{ゆ}け、
%
\ruby{立}{た}たぬかやい。
%
\ruby[g]{水野}{みづの }めを
\ruby[g]{引張}{ひつぱ }つて
\ruby{來}{き}て
\ruby[g]{此處}{こ ゝ }で
\原本頁{8-11}\改行%
\ruby{諫}{いさ}めて、
%
\ruby{諫}{いさ}めて
\ruby{聽}{き}かずば
\ruby{擲}{たゝ}き
\ruby{撲}{なぐ}つて、
%
\ruby[g]{正氣}{しやうき}に
\ruby{{\換字{返}}}{かへ}らせて
\ruby{吳}{く}れにやならぬ、
%
さあ
\ruby{立}{た}て
\ruby[g]{山瀬}{やませ }!。
』

\原本頁{9-2}%
と
\ruby{云}{い}ひざまに、
%
\ruby[g]{五{\換字{分}}}{ご ぶ }の
\ruby[g]{慷慨}{かうがい}、
%
\ruby[g]{五{\換字{分}}}{ご ぶ }の
\ruby{醉}{ゑひ}、% 「醉」は原本通り「ゑ」で調整
%
\ruby[g]{山瀬}{やませ }が
\ruby[g]{肩頭}{かたさき}を
\ruby[g]{引攫}{ひつつか}んで
\ruby[g]{氣勢}{いきほひ}
\ruby{猛}{もう}に
\ruby[g]{立上}{たちあが}つたり。
