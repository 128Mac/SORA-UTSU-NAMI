\Entry{其二}

% メモ 校正終了 2024-03-28 2024-05-22 2024-06-15
\原本頁{9-5}%
\ruby[g]{薄墨}{うすずみ}の
\ruby{夕}{ゆふべ}の
\ruby{色}{いろ}は
\ruby[g]{物蔭}{ものかげ}より
\ruby{擴}{ひろ}まりて、
%
\ruby[g]{廓然}{くわらり}と
\ruby{晴}{は}れやかなりし
\ruby{樓}{ろう}の
\ruby{上}{うへ}も、
%
\ruby[g]{手許}{て もと}
やうやく
\ruby{暗}{くら}くなり、
%
いづくに
\ruby{歸}{かへ}る
\ruby{鵜}{う}の
\ruby{鳥}{とり}の、
%
\ruby{浪}{なみ}を
\ruby{{\換字{摩}}}{す}つて
\ruby{飛}{と}ぶ
\ruby[g]{羽音}{は おと}も
\ruby{寂}{さ}びたり。
%
\ruby{右}{みぎ}の
\ruby{方}{かた}は
\ruby[g]{高輪}{たかなわ}
\ruby{八ツ山}{や||やま}% 地名なので一つにした
\ruby[g]{品川}{しながは}の
\ruby{一}{ひ}トつゞき、
%
\ruby{森}{もり}も
\ruby[g]{人家}{じんか }も
たゞ
\ruby{一}{ひ}ト
\ruby{筆}{ふで}の
なすり
\ruby{書}{がき}と
\ruby{黑}{くろ}み、
%
\ruby{左}{ひだり}に
\ruby{低}{ひく}き
\ruby[g]{築地}{つきぢ }
\ruby[g]{月島}{つきしま}、
%
\ruby[g]{洲崎}{す さき}は
\ruby{微}{かすか}にして
\ruby{{\換字{消}}}{き}えん
とする
\ruby{時}{とき}、
%
\ruby[g]{其處}{そ こ }に
\ruby[g]{電燈}{でんとう}の
\ruby[g]{白々}{しろ〴〵}と
\ruby{輝}{かゞや}き
\ruby{出}{い}づれば、
%
\ruby[g]{燈火}{ともしび}
\ruby{華}{はな}やかに
\ruby[g]{此家}{こ こ }にも
\ruby{點}{つ}きて、
%
\ruby{室}{へや}の
\ruby{内}{うち}
ぱつと
\ruby{明}{あか}るくなり、
%
\原本頁{10-1}%
\ruby{外}{そと}は
\ruby{全}{まつた}く
\ruby{海}{うみ}
\ruby{玄}{くろ}く
\ruby{風}{かぜ}
\ruby{睡}{ねむ}れる
\ruby{穩}{おだ}やかなる
\ruby{夜}{よ}となり
\ruby{畢}{をは}んぬ。

\原本頁{10-2}%
\ruby[g]{日方}{ひ かた}が
\ruby{急}{せ}き
\ruby{{\換字{込}}}{こ}み
\ruby[g]{調子}{てうし }に
\ruby[g]{物言}{ものい }ひても、
%
\ruby[g]{特{\換字{更}}}{ことさら}に
\ruby[g]{沈着}{おちつき}を
\ruby{爲}{つく}れる
\ruby[g]{山瀬}{やませ }
\ruby[g]{荒吉}{あらきち}は、
%
\ruby{言}{い}ひ
\ruby{爭}{あらそ}はん
ともせで
\ruby{良}{やゝ}
\ruby[g]{少時}{しばし }、
%
\ruby[g]{何事}{なにごと}をか
\ruby{思}{おも}ひ
\ruby{{\換字{廻}}}{めぐ}らし
\ruby{居}{ゐ}けるが、
%
\ruby{今}{いま}しも
\ruby[g]{燈火}{ともしび}の
\ruby{光}{ひかり}を
\ruby{得}{え}て、
%
\ruby{心}{こゝろ}の
\ruby{中}{うち}に
\ruby{索}{たづ}ね
\ruby{得}{え}し
\ruby[g]{言葉}{ことば }の
\ruby[<j>]{緖}{いとぐち}をや
\ruby{求}{もと}め
\ruby{得}{え}けん、
%
\ruby{逸}{はや}り
きつたる
\ruby[g]{日方}{ひ かた}の
\ruby{面}{おもて}の、
%
いさゝか
\ruby{怒}{いかり}をさへ
\ruby{帶}{お}びたるを、
%
\ruby{愛}{あい}するが
\ruby{如}{ごと}く
\ruby[g]{打見}{うちみ }やりて、

\原本頁{10-7}%
『
マア
\ruby{坐}{すわ}つて
\ruby{吳}{く}れ、
%
\ruby[g]{日方}{ひ かた}!。
%
\ruby[g]{成程}{なるほど}
\ruby[g]{打棄}{うつちや}つて
\ruby{置}{お}いては
\ruby[g]{水野}{みづの }の
\ruby{不利益}{ふ|た|め}になるから、
%
\ruby{君}{きみ}と
\ruby[g]{一緖}{いつしよ}に
\ruby{{\換字{尋}}}{たづ}ねて
\ruby{行}{い}つて、
%
\ruby[g]{隨{\換字{分}}}{ずゐぶん}
\ruby[||j>]{忠}{ちゆう}% 原本通り(ちゆう)(国会図書館 コマ番号 9/134 p10 l8)
\ruby[||j>]{告}{ こく}
\footnote{%
原本では「\ruby{忠}{ちゆう}」ないし「\ruby{忠}{ちう}」の二通りのルビが使われている。
その使用例『
第一巻「\ruby{忠告}{ちゆう|こく}」
      「\ruby[g]{忠義}{ちゆうぎ}」
第二巻「\ruby[g]{忠告}{ちうこく}」
第三巻「\ruby[g]{忠義}{ちうぎ }」
』
(国会図書館 コマ番号 9/134 p-10 l-08)
}%
も
% \ruby{忠告}{ちゆう|こく}も
\ruby{試}{こゝろ}みやう
\改行% 校正作業の簡略化のため
。
%
\原本頁{10-9}\改行%
\ruby{併}{しか}し
\ruby[g]{水野}{みづの }のところは
\ruby[g]{大{\換字{分}}}{だいぶん}
\ruby{{\換字{遠}}}{とほ}い。
%
\ruby{{\換字{連}}}{つ}れて
\ruby{來}{く}るにしても
\ruby[g]{時間}{と き }が
かゝる。
%
もう
\ruby{此}{こ}の
\ruby{{\換字{通}}}{とほ}り
\ruby{夜}{よ}にも
\ruby{入}{い}つて
\ruby{居}{ゐ}る。
%
\ruby{{\換字{連}}}{つ}れて
\ruby{來}{き}たにしたところで
\ruby{話}{はな}す
\ruby{間}{ま}も
\ruby{無}{な}い。
%
\ruby[g]{第一}{だいいち}
\ruby[g]{左樣}{さ う }で
\ruby{無}{な}くつてさへ、
%
\ruby[g]{七人}{しちにん}の% 原本には漢数字「七」のルビ無し
\ruby{中}{うち}が
\ruby[g]{三人}{さんにん}
\ruby{缺}{か}けて、
%
\原本頁{11-1}%
\ruby[g]{四人}{よ にん}しか
\ruby{居}{を}らぬ
\ruby{此}{こ}の
\ruby{席}{せき}を、
%
\ruby{君}{きみ}と
\ruby{僕}{ぼく}と
\ruby[g]{二人}{ふたり }
\ruby{脫}{ぬ}けて
\ruby[g]{仕舞}{し ま }へば
\原本頁{11-2}\改行%
\ruby{後}{あと}は
\ruby[g]{何樣}{ど う }だ。
%
\ruby[g]{羽{\換字{勝}}}{は がち}
\ruby{君}{くん}と
\ruby[g]{島木}{しまき }
\ruby{君}{くん}とたつた
\ruby[g]{二人}{ふたり }だ。
%
\ruby[g]{今日}{け ふ }の
\ruby{客}{きやく}たる
\ruby[g]{羽{\換字{勝}}}{は がち}
\ruby{君}{くん}を、
%
\ruby[g]{島木}{しまき }
\ruby{君}{くん}と
\ruby[||j>]{只}{たつた}
\ruby[||j>]{二人}{ ふた|り}に% ルビ調整(原本通り)
\ruby{仕}{し}て
\ruby[g]{仕舞}{し ま }つて、
%
\ruby[g]{僕等}{ぼくら }が
\ruby{出}{で}て
\ruby{行}{い}くといふのは
\ruby[g]{{\換字{勝}}手}{かつて }
\ruby{{\換字{過}}}{す}ぎる。
%
それでは
\ruby{餘}{あんま}り
\ruby[g]{無禮}{ぶ れい}になる。
%
こゝを
\ruby[g]{無理}{む り }に
\ruby{君}{きみ}と
\ruby[g]{二人}{ふたり }で
\ruby{出}{で}て
\ruby{行}{い}つたら、
%
\ruby[g]{水野}{みづの }には
\ruby[g]{成程}{なるほど}
\ruby[g]{親切}{しんせつ}にも
ならう、
%
\ruby{併}{しか}し
\ruby[g]{羽{\換字{勝}}}{は がち}
\ruby{君}{くん}には
\ruby[g]{失敬}{しつけい}に
\ruby{當}{あた}らう。
%
もと〳〵
\ruby{君}{きみ}が
\ruby{怒}{おこ}り
\ruby{立}{た}つたのも、
%
つまりは
\ruby[g]{水野}{みづの }が
\ruby[g]{羽{\換字{勝}}}{は がち}
\ruby{君}{くん}に
\ruby{對}{たい}する
\ruby[g]{仕方}{し かた}が
\ruby[g]{冷淡}{れいたん}
だといふのにあらう。
%
\ruby[g]{羽{\換字{勝}}}{は がち}
\ruby{君}{くん}に
\ruby[g]{滿足}{まんぞく}を
\ruby{{\換字{感}}}{かん}ぜしめぬ
\ruby[g]{其事}{そ れ }が
\ruby{惡}{にく}むべき
\ruby[g]{我儘}{わがまゝ}
だといふのだ。
%
それだのに
\ruby{今}{いま}
\ruby[g]{僕等}{ぼくら }が
\ruby[g]{此席}{こ ゝ }を
\ruby{去}{さ}つては、
%
たゞ
\ruby{淋}{さび}しさを
\ruby{增}{ま}すばかりで、
%
\原本頁{11-10}\改行%
\ruby[g]{羽{\換字{勝}}}{は がち}
\ruby{君}{くん}は
いよ〳〵
おもしろく
\ruby{無}{な}く
\ruby{{\換字{感}}}{かん}じやう。
%
\ruby[g]{今日}{け ふ }は
\ruby{既}{もう}
\ruby[g]{十{\換字{分}}}{じふぶん}に
\ruby[g]{談笑}{だんせう}も% ルビ調整(原本通り)「だんせ(う)」
\ruby{仕}{し}て、
%
\ruby[g]{大{\換字{分}}}{だいぶ }
\ruby{醉}{ゑひ}さえも% 「醉」は原本通り「ゑ」で調整
\ruby{{\換字{廻}}}{まは}つて
\ruby{居}{ゐ}る。
%
\ruby[g]{談話}{はなし }の
\ruby{序}{つひで}から
\ruby[g]{不圖}{ふ と }
\ruby[g]{水野}{みづの }の
\原本頁{12-1}\改行%
\ruby{事}{こと}が
\ruby{出}{で}て、
%
\ruby{始}{はじ}めて
\ruby{君}{きみ}は
\ruby{其}{それ}を
\ruby{聞}{き}いた
ところから、
%
\ruby{大}{おほき}に
\ruby{忌}{いま}はしくも
\原本頁{12-2}\改行%
\ruby{{\換字{感}}}{かん}じたらうが、
%
\ruby{何}{なに}も
\ruby{今}{いま}が
\ruby{今}{いま}で
\ruby{無}{な}くちやならぬといふ
\ruby{事}{こと}では
\ruby{無}{な}いから、
%
\ruby{彼}{かれ}を
\ruby{訪}{と}ふのは
\ruby[g]{明日}{あ す }でも
\ruby{明後日}{あさ|つ|て}でもの
\ruby{事}{こと}として、
%
\ruby{其}{その}
\ruby{時}{とき}
\ruby[g]{戀愛}{れんあい}
\ruby{{\換字{嫌}}}{ぎら}ひの
\ruby{君}{きみ}の
\ruby[g]{存{\換字{分}}}{ぞんぶん}に、
%
\ruby{諫}{いさ}めるとも
\ruby{擲}{なぐ}る
ともするが
\ruby{宜}{よ}からう。
%
\ruby[g]{今日}{け ふ }は
\ruby{先}{ま}づ
\ruby[g]{堪{\換字{忍}}}{かんにん}して% 原文通り「堪忍」
\ruby[g]{一同}{みんな }と
\ruby{共}{とも}に、
%
\ruby{飮}{の}んで
\ruby{居}{ゐ}て
\ruby{吳}{く}れたつて
\ruby{可}{よ}いでは
\ruby{無}{な}いか。
』

\原本頁{12-7}%
と、
%
\ruby{他}{ひと}の
\ruby{言}{い}ふ
ところは
\ruby{斜}{なゝめ}に
\ruby{外}{そ}らせて、
%
\ruby{我}{わ}が
\ruby{言}{い}ふ
ところは
\ruby{斜}{なゝめ}に
\ruby{徹}{とほ}す
\ruby[g]{才子}{さいし }の
\ruby{面}{おもて}は
\ruby{笑}{ゑみ}を
\ruby{湛}{たゝ}へて、
%
\ruby{巧}{たくみ}に
\ruby[g]{粗獷}{ぶ こつ}なる
\ruby[g]{相手}{あひて }を
\ruby{制}{せい}すれば、
%
\ruby[<j||]{正}{しやう}
\ruby[<j||]{直}{ぢき}
% \ruby{正直}{しやう|ぢき}
\ruby{一三昧}{いつ|さん|まい}の
% いっ‐さんまい【一三昧】 の解説
% 1 仏語。雑念を去り一心に修行に専念すること。
% 2 ほかのことに構わず、一つのことだけに心を用いること
\ruby[g]{日方}{ひ かた}は、
%
\ruby{脆}{もろ}くも、
%
\ruby[g]{羽{\換字{勝}}}{は がち}を
\ruby{重}{おも}んずる
\ruby{{\換字{情}}}{こゝろ}より、

\原本頁{12-10}%
『
ムー、
%
\ruby{此}{こ}の
\ruby{席}{せき}が
\ruby{淋}{さび}しくなる?。
%
ア、
%
\ruby[g]{其處}{そ こ }へは
\ruby{些}{ちつと}も
\ruby{氣}{き}が
つかなかつた。
%
\ruby[g]{成程}{なるほど}
\ruby{今}{いま}
\ruby{直}{すぐ}
\ruby[g]{引張}{ひつぱ }つて
\ruby{來}{こ}やうと
\ruby{云}{い}つたのは、
%
\ruby[g]{乃公}{お れ }が
\ruby{惡}{わる}かつた。
%
\原本頁{13-1}%
こいつは
\ruby[g]{一番}{いちばん}
\ruby[g]{山瀬}{やませ }に
やられた。
%
ハヽヽ。
%
どうも
\ruby[g]{山瀬}{やませ }は
\ruby[g]{乃公}{お れ }より
\ruby[g]{怜悧}{り こう}だ。% ルビ調整(原本通り)(りこう)
%
ハヽハヽ。
』

\原本頁{13-3}%
と、
%
\ruby{露}{つゆ}ばかりの
\ruby[g]{我執}{が しふ}も
\ruby{無}{な}く
\ruby{笑}{わら}つて
\ruby[g]{仕舞}{し ま }つて、
%
\ruby[g]{霽々}{はれ〴〵}したる
\ruby[g]{顏色}{かほつき}にも
\ruby{著}{しる}き
\ruby{胸}{むね}に
\ruby{何}{なに}も
\ruby{{\換字{遺}}}{のこ}さぬ
\ruby[g]{有樣}{ありさま}は、
%
\ruby{譬}{たと}へば
\ruby{風}{かぜ}
\ruby{{\換字{過}}}{す}ぎて
\ruby{林}{はやし}
おのづから
\ruby[<j||]{靜}{しづか}に、
%
\ruby{雲}{くも}
\ruby{去}{さ}つて
\ruby{山}{やま}
\ruby{{\換字{更}}}{さら}に
\ruby{靑}{あを}きが
\ruby{如}{ごと}くなりしが、
%
\ruby{例}{れい}の
\ruby{癖}{くせ}とて
\ruby[g]{突然}{とつぜん}と、

\原本頁{13-6}%
『
ヤ、
%
\ruby{時}{とき}に
\ruby[g]{羽{\換字{勝}}}{は がち}
\ruby{君}{くん}
\ruby[g]{一盃}{いつぱい}
\ruby{吳}{く}れたまへ。
』

\原本頁{13-7}%
と
\ruby{云}{い}ひ
\ruby{出}{いだ}したり。
%
\ruby[g]{羽{\換字{勝}}}{は がち}は
\ruby[g]{機{\換字{嫌}}}{き げん}
\ruby{好}{よ}く
\ruby[<j>]{盃}{さかづき}を
さして、

\原本頁{13-8}%
『
\ruby[g]{相變}{あひかは}らず
\ruby{君}{きみ}は
\ruby{君}{きみ}の
\ruby[g]{氣風}{き ふう}で
\ruby[g]{押{\換字{通}}}{おしとほ}すナ。
%
どうだ
\ruby[g]{軍{\換字{隊}}}{ぐんたい}の
\ruby[||j>]{生}{せい}
\ruby[||j>]{活}{くわつ}は
% \ruby{生活}{せい|くわつ}は
\ruby[g]{{\換字{愉}}快}{ゆくわい}かネ。
』

\原本頁{13-10}%
と
\ruby{懷}{なつ}かし
\ruby{氣}{げ}に
\ruby{問}{と}へば、

\原本頁{13-11}%
『
ムヽ。
%
\ruby[g]{左樣}{さ う }さ、
%
\ruby[<j||]{快}{くわい}
\ruby[||j>]{活}{くわつ}な
\ruby{事}{こと}ばかりといふ
\ruby{譯}{わけ}にも
\ruby{行}{ゆ}かん。
%
\ruby[g]{僕等}{ぼくら }の
\原本頁{14-1}%
\ruby[g]{身{\換字{分}}}{み ぶん}では
\ruby[g]{隨{\換字{分}}}{ずゐぶん}
\ruby[g]{箱詰}{はこづめ}に
なるのを
\ruby{甘}{あま}んじ
なけりやあならん
\ruby{事}{こと}もあるが、
%
\ruby{其}{それ}が
\ruby{{\換字{即}}}{すなは}ち
\ruby[g]{紀律}{き りつ}で、
%
\ruby[g]{紀律}{き りつ}が
\ruby{{\換字{即}}}{すなは}ち
\ruby[g]{精神}{せいしん}である、
%
といふやうに
\ruby[<j||]{考}{かんが}へて% 行末行頭の境界付近なので特例処置を施す
\ruby{居}{ゐ}りやあ、
%
\ruby{別}{べつ}に
\ruby[g]{窮屈}{きうくつ}にも
\ruby{{\換字{感}}}{かん}じない。
%
ホワイトシヤツを
\ruby[g]{着慣}{き な }
\改行% 校正作業の簡略化のため
れ
\原本頁{14-4}\改行% 原本では一行が29文字になっているため
て
\ruby{見}{み}ると、
%
\ruby{彼}{あ}の
\ruby{硬}{こは}いものを
\ruby{身}{み}につけるのが、
%
\ruby{却}{かへ}つて
\ruby{好}{い}い
\makeatletter
\@ifundefined{デバッグ@ビルド}{%
  \ruby[<j||]{心}{こゝろ}
  \ruby[||j>]{持}{もち}に
  }{%
  \ruby[||j>]{心}{こゝろ}
  \ruby[||j>]{持}{ もち}に
  }%
\makeatother
% \ruby{心持}{こゝろ|もち}に
% \原本頁{14-5}\改行%
\ruby{思}{おも}へて
\ruby{來}{く}る。
%
\ruby[g]{丁度}{ちやうど}
それと
\ruby{同}{おな}じ
\ruby{事}{こと}で、
%
\ruby{慣}{な}れて
みると
\ruby[||j>]{嚴}{げん}
\ruby[||j>]{肅}{しゆく}な
% \ruby{嚴肅}{げん|しゆく}な
\ruby{中}{うち}には
\原本頁{14-6}\改行%
\ruby[g]{{\換字{愉}}快}{ゆくわい}が
あるから、
%
\ruby{僕}{ぼく}は
まあ
\ruby{不}{ふ}
\ruby[g]{{\換字{愉}}快}{ゆくわい}には
\ruby{日}{ひ}を
\ruby{{\換字{送}}}{おく}らん。
』

\原本頁{14-7}%
と
\ruby{答}{こた}へて
\ruby{其}{そ}の
\ruby[<j>]{盃}{さかづき}を
\ruby{乾}{ほ}して
\ruby{洗}{あら}ふ。

\原本頁{14-8}%
『
\ruby[g]{左樣}{さ う }だ。
%
\ruby[g]{紀律}{き りつ}を
\ruby[||j>]{{\換字{尊}}}{そん}
\ruby[||j>]{重}{ちやう}する
% \ruby{{\換字{尊}}重}{そん|ちやう}する
\ruby{中}{うち}には
\ruby[g]{{\換字{愉}}快}{ゆくわい}が
ある。
%
そして
\ruby{何}{なん}の
\ruby[g]{方面}{はうめん}の
\ruby{事}{こと}でも
\ruby[g]{紀律}{き りつ}は
\ruby[g]{大切}{たいせつ}だ。
%
\ruby{{\換字{船}}}{ふね}の
\ruby{中}{うち}などは
\ruby{特}{こと}に
\ruby[g]{然樣}{さ う }だ。
%
そればかりぢやあ
\ruby{無}{な}い、
%
\ruby{僕}{ぼく}が
\ruby{私}{ひそか}に
\ruby{思}{おも}ふには、
%
\ruby[g]{身體}{からだ }を
\ruby{扱}{あつか}ふのに
\ruby[g]{紀律}{き りつ}が
\ruby{無}{な}いと
\ruby[g]{身體}{からだ }が
\ruby{衰}{おとろ}へる、
%
\ruby{心}{こゝろ}を
\ruby{扱}{あつか}ふにも
\ruby[g]{紀律}{き りつ}が
\ruby{無}{な}いと
\ruby{心}{こゝろ}が
\ruby{歪}{ゆが}んで、
%
そこで
\原本頁{15-1}%
\ruby[g]{戀愛}{れんあい}
などゝいふものに
\ruby{取}{と}り
\ruby{憑}{つ}かれるのだ。
』

\原本頁{15-2}%
と
\ruby{云}{い}ひながら
\ruby{徐}{しづか}に
\ruby[g]{酒盃}{さかづき}を
\ruby{受}{う}くれば、
%
\ruby[g]{日方}{ひ かた}は

\原本頁{15-3}%
『
\ruby[g]{確論}{かくろん}、
%
\ruby[g]{確論}{かくろん}。
』

\原本頁{15-4}%
と
\ruby{悅}{よろこ}び
\ruby{叫}{さけ}んで、
%
\ruby{自}{みづか}ら
\ruby{{\換字{酌}}}{しやく}を
\ruby{仕}{し}て
\ruby{{\換字{遣}}}{や}らんと
\ruby[g]{徳利}{とくり }を
\ruby{擧}{あ}ぐれば、
%
\ruby{既}{はや}
\ruby{飮}{の}み
\ruby{盡}{つく}して
\ruby{二三滴}{に|さん|てき}のみ。
%
\ruby[g]{山瀬}{やませ }は
\ruby{急}{いそ}ぎ
\ruby{手}{て}を
\ruby{拍}{たゝ}き
\ruby{立}{た}つ。

\原本頁{15-6}%
\ruby[g]{此時}{このとき}まで
にや〳〵と
\ruby{笑}{わら}ひながら、
%
\ruby[g]{人々}{ひと〴〵}の
\ruby{談}{はなし}を
のみ
\ruby{聞}{き}き
\ruby{居}{ゐ}たりし
\ruby{布袋肥胖}{ほ|てい|ぶ|と}りに
\ruby{肥}{ふと}つたる、
%
\ruby[g]{丸顏}{まるがほ}の
\ruby[g]{眼下}{め さが}りなる
\ruby[g]{島木}{しまき }は
\ruby{笑}{わら}つて、

\原本頁{15-8}%
『
ハヽヽ、
%
\ruby[g]{談話}{はなし }が
\ruby{惡}{わる}つ
\ruby{固}{かた}いから
\ruby{堪}{たま}りやあ
\ruby[g]{仕無}{し な }い。
%
\ruby{婢}{をんな}だつて
\ruby{何}{なん}だつて
\ruby{{\換字{逃}}}{に}げたつきりだ。
%
\ruby[g]{徳利}{とつくり}の
\ruby[g]{番兵}{ばんぺい}は
\ruby[g]{野暮}{や ぼ }ぢやあ
\ruby{使}{つか}へ
\ruby{無}{ね}えからな
\改行% 校正作業の簡略化のため
。
%
\原本頁{15-10}\改行%
ハヽヽ。
%
\ruby{何}{なん}だい?。
%
\ruby[g]{紀律}{き りつ}が
\ruby{無}{な}いと
いけ
\ruby{無}{な}いつて?。
%
\ruby[||j>]{戱}{じやう}
\ruby[||j>]{談}{ だん}
% \ruby{戱談}{じやう|だん}
\ruby{言}{ い}つちやあ% 原本に合わせて調整
いけない、
%
\ruby[g]{舞臺}{ぶ たい}に
\ruby{障}{さは}るぜ。
%
\ruby{不紀律}{ふ|き|りつ}の
\ruby[||j>]{大}{たい}
\ruby[||j>]{將}{しやう}、
% \ruby{大將}{たい|しやう}、
%
\ruby{實業家}{じつ|げふ|か}
\ruby{{\換字{兼}}}{けん}
\makeatletter
\@ifundefined{デバッグ@ビルド}{%
  \ruby{虛業家}{きよ|げふ|か}、
}{%
  \ruby{虛業家}{きよ|げふ|か}
  \改行% 校正作業の簡略化のため
  、
}%
\makeatother
%
\原本頁{16-1}\改行%
\ruby{相場師}{さう|ば|し}に% 原文通り「場」
なつたつて
%
\ruby[g]{一同}{みんな }に
\ruby{怒}{おこ}られた、
%
\ruby{御利益}{ご|り|やく}は
\ruby{未}{ま}だ
\ruby{蒙}{かうむ}ら
\ruby{無}{な}いが
% \原本頁{16-2}\改行%
\ruby{拜金宗}{はい|きん|しう}の
\ruby[g]{信徒}{しんと }の、
%
\ruby[g]{島木}{しまき }
\ruby{萬五郎}{まん|ご|らう}
\ruby{樣}{さま}が
\ruby[g]{此處}{こ ゝ }に
\ruby[g]{御坐}{お いで}なさるぜ。
%
\ruby{憚}{はゞか}りながら% 「憚 は(ゞ)か」
\ruby[g]{乃公}{お れ }が
\ruby[g]{何時}{い つ }
\ruby[g]{戀愛}{れんあい}に
\ruby{取}{と}り
\ruby{憑}{つ}かれた。
%
ハヽヽ。
%
\ruby{其}{そ}りやあ
\ruby[g]{左樣}{さ う }と
\原本頁{16-4}\改行%
\ruby[g]{水野}{みづの }の
\ruby{談}{はなし}は
\ruby{譯}{わけ}
\ruby{有}{あ}つて
\ruby[g]{一番}{いちばん}
\ruby[g]{乃公}{お れ }が
\ruby{知}{し}つてる。
%
どうも
\ruby[g]{一同}{みんな }が
\ruby{氣}{き}に
\ruby{仕}{し}て
\ruby{居}{ゐ}る。
%
\ruby[g]{羽{\換字{勝}}}{は がち}の
\ruby{腹}{はら}の
\ruby{中}{なか}では
\ruby{取}{と}り
\ruby{{\換字{分}}}{わ}け
\ruby{深}{ふか}く
\ruby[g]{心配}{しんぱい}して
\ruby{居}{ゐ}る
やうすだから
\ruby{話}{はな}して
\ruby{聞}{き}かさうか。
』

\原本頁{16-7}%
と、
%
\ruby{始}{はじめ}は
\ruby{戱}{たはむ}れ、
%
\ruby{{\換字{終}}}{をはり}は
\ruby{眞面目}{ま|じ|め}に
\ruby{云}{い}ひ
\ruby{出}{い}づれば、
%
\ruby[||j>]{謹}{きん}
\ruby[||j>]{聽}{ちやう}の
% \ruby{謹聽}{きん|ちやう}の
\ruby{聲}{こゑ}は
\ruby[g]{異口}{い く }
\ruby[g]{一齊}{いつせい}に
\ruby{出}{い}でぬ。
