\Entry{其二}

% メモ 校正終わり 2024-3-28
\原本頁{9-5}
\ruby{薄墨}{うす|ずみ}の
\ruby{夕}{ゆふべ}の
\ruby{色}{いろ}は
\ruby{物蔭}{もの|かげ}より
\ruby{擴}{ひろ}まりて、
%
\ruby{廓然}{くわ|らり}と
\ruby{晴}{は}れやかなりし
\ruby{樓}{ろう}の
\ruby{上}{うへ}も、
%
\ruby{手許}{て|もと}
やうやく
\ruby{暗}{くら}くなり、
%
いづくに
\ruby{歸}{かへ}る
\ruby{鵜}{う}の
\ruby{鳥}{とり}の、
%
\ruby{浪}{なみ}を
\ruby{{\換字{摩}}}{す}つて
\ruby{飛}{と}ぶ
\ruby{羽音}{は|おと}も
\ruby{寂}{さ}びたり。
%
\ruby{右}{みぎ}の
\ruby{方}{かた}は
\ruby[g]{高輪}{たかなわ}
\ruby[g]{八ツ山}{やつやま}% 地名なので一つにした
\ruby[g]{品川}{しながは}の
\ruby{一}{ひ}トつゞき、
%
\ruby{森}{もり}も
\ruby{人家}{じん|か}も
たゞ
\ruby{一}{ひ}ト
\ruby{筆}{ふで}の
なすり
\ruby{書}{がき}と
\ruby{黑}{くろ}み、
%
\ruby{左}{ひだり}に
\ruby{低}{ひく}き
\ruby[g]{築地}{つきぢ}
\ruby[g]{月島}{つきしま}、
%
\ruby[g]{洲崎}{すさき}は
\ruby{微}{かすか}にして
\ruby{{\換字{消}}}{き}えん
とする
\ruby{時}{とき}、
%
\ruby{其處}{そ|こ}に
\ruby{電燈}{でん|とう}の
\ruby{白々}{しろ|〴〵}と
\ruby{輝}{かゞや}き
\ruby{出}{い}づれば、
%
\ruby{燈火}{とも|しび}
\ruby{華}{はな}やかに
\ruby{此家}{こ|こ}にも
\ruby{點}{つ}きて、
%
\ruby{室}{へや}の
\ruby{内}{うち}
ぱつと
\ruby{明}{あか}るくなり、
%
\原本頁{10-1}
\ruby{外}{そと}は
\ruby{全}{まつた}く
\ruby{海}{うみ}
\ruby{玄}{くろ}く
\ruby{風}{かぜ}
\ruby{睡}{ねむ}れる
\ruby{穩}{おだ}やかなる
\ruby{夜}{よ}となり
\ruby{畢}{をは}んぬ。

\原本頁{10-2}
\ruby{日方}{ひ|かた}が
\ruby{急}{せ}き
\ruby{{\換字{込}}}{こ}み
\ruby{調子}{てう|し}に
\ruby{物言}{もの|い}ひても、
%
\ruby{特{\換字{更}}}{こと|さら}に
\ruby{沈着}{おち|つき}を
\ruby{爲}{つく}れる
\ruby{山瀬}{やま|せ}
\ruby{荒吉}{あら|きち}は、
%
\ruby{言}{い}ひ
\ruby{爭}{あらそ}はん
ともせで
\ruby{良}{やゝ}
\ruby{少時}{しば|し}、
%
\ruby{何事}{なに|ごと}をか
\ruby{思}{おも}ひ
\ruby{{\換字{廻}}}{めぐ}らし
\ruby{居}{ゐ}けるが、
%
\ruby{今}{いま}しも
\ruby{燈火}{とも|しび}の
\ruby{光}{ひかり}を
\ruby{得}{え}て、
%
\ruby{心}{こゝろ}の
\ruby{中}{うち}に
\ruby{索}{たづ}ね
\ruby{得}{え}し
\ruby{言葉}{こと|ば}の
\ruby{緖}{いとぐち}をや
\ruby{求}{もと}め
\ruby{得}{え}けん、
%
\ruby{逸}{はや}り
きつたる
\ruby{日方}{ひ|かた}の
\ruby{面}{おもて}の、
%
いさゝか
\ruby{怒}{いかり}をさへ
\ruby{帶}{お}びたるを、
%
\ruby{愛}{あい}するが
\ruby{如}{ごと}く
\ruby{打見}{うち|み}やりて、

\原本頁{10-7}
『マア
\ruby{坐}{すわ}つて
\ruby{吳}{く}れ、
%
\ruby{日方}{ひ|かた}!。
%
\ruby{成程}{なる|ほど}
\ruby{打棄}{うつ|ちや}つて
\ruby{置}{お}いては
\ruby{水野}{みづ|の}の
\ruby{不利益}{ふ|た|め}になるから、
%
\ruby{君}{きみ}と
\ruby{一緖}{いつ|しよ}に
\ruby{{\換字{尋}}}{たづ}ねて
\ruby{行}{い}つて、
%
\ruby{隨{\換字{分}}}{ずゐ|ぶん}
\ruby{忠告}{ちゆう|こく}も
\ruby{試}{こゝろ}みやう。
%
\ruby{併}{しか}し
\ruby{水野}{みづ|の}のところは
\ruby{大{\換字{分}}}{だい|ぶん}
\ruby{{\換字{遠}}}{とほ}い。
%
\ruby{{\換字{連}}}{つ}れて
\ruby{來}{く}るにしても
\ruby{時間}{と|き}が
かゝる。
%
もう
\ruby{此}{こ}の
\ruby{{\換字{通}}}{とほ}り
\ruby{夜}{よ}にも
\ruby{入}{い}つて
\ruby{居}{ゐ}る。
%
\ruby{{\換字{連}}}{つ}れて
\ruby{來}{き}たにしたところで
\ruby{話}{はな}す
\ruby{間}{ま}も
\ruby{無}{な}い。
%
\ruby{第一}{だい|いち}
\ruby{左樣}{さ|う}で
\ruby{無}{な}くつてさへ、
%
\ruby{七人}{しち|にん}の
\ruby{中}{うち}が
\ruby{三人}{さん|にん}
\ruby{缺}{か}けて、
%
\原本頁{11-1}
\ruby{四人}{よ|にん}しか
\ruby{居}{を}らぬ
\ruby{此}{こ}の
\ruby{席}{せき}を、
%
\ruby{君}{きみ}と
\ruby{僕}{ぼく}と
\ruby{二人}{ふた|り}
\ruby{脫}{ぬ}けて
\ruby{仕舞}{し|ま}へば
\ruby{後}{あと}は
\ruby{何樣}{ど|う}だ。
%
\ruby{羽{\換字{勝}}}{は|がち}
\ruby{君}{くん}と
\ruby{島木}{しま|き}
\ruby{君}{くん}とたつた
\ruby{二人}{ふた|り}だ。
%
\ruby{今日}{け|ふ}の
\ruby{客}{きやく}たる
\ruby{羽{\換字{勝}}}{は|がち}
\ruby{君}{くん}を、
%
\ruby{島木}{しま|き}
\ruby{君}{くん}と
\ruby[<h||]{只}{たつた}
\ruby{二人}{ふた|り}に
\ruby{仕}{し}て
\ruby{仕舞}{し|ま}つて、
%
\ruby{僕等}{ぼく|ら}が
\ruby{出}{で}て
\ruby{行}{い}くといふのは
\ruby{{\換字{勝}}手}{かつ|て}
\ruby{{\換字{過}}}{す}ぎる。
%
それでは
\ruby{餘}{あんま}り
\ruby{無禮}{ぶ|れい}になる。
%
こゝを
\ruby{無理}{む|り}に
\ruby{君}{きみ}と
\ruby{二人}{ふた|り}で
\ruby{出}{で}て
\ruby{行}{い}つたら、
%
\ruby{水野}{みづ|の}には
\ruby{成程}{なる|ほど}
\ruby{親切}{しん|せつ}にも
ならう、
%
\ruby{併}{しか}し
\ruby{羽{\換字{勝}}}{は|がち}
\ruby{君}{くん}には
\ruby{失敬}{しつ|けい}に
\ruby{當}{あた}らう。
%
もと〳〵
\ruby{君}{きみ}が
\ruby{怒}{おこ}り
\ruby{立}{た}つたのも、
%
つまりは
\ruby{水野}{みづ|の}が
\ruby{羽{\換字{勝}}}{は|がち}
\ruby{君}{くん}に
\ruby{對}{たい}する
\ruby{仕方}{し|かた}が
\ruby{冷淡}{れい|たん}
だといふのにあらう。
%
\ruby{羽{\換字{勝}}}{は|がち}
\ruby{君}{くん}に
\ruby{滿足}{まん|ぞく}を
\ruby{感}{かん}ぜしめぬ
\ruby{其事}{そ|れ}が
\ruby{惡}{にく}むべき
\ruby{我儘}{わが|まゝ}
だといふのだ。
%
それだのに
\ruby{今}{いま}
\ruby{僕等}{ぼく|ら}が
\ruby{此席}{こ|ゝ}を
\ruby{去}{さ}つては、
%
たゞ
\ruby{淋}{さび}しさを
\ruby{增}{ま}すばかりで、
%
\ruby{羽{\換字{勝}}}{は|がち}
\ruby{君}{くん}は
いよ〳〵
おもしろく
\ruby{無}{な}く
\ruby{感}{かん}じやう。
%
\ruby{今日}{け|ふ}は
\ruby{既}{もう}
\ruby{十{\換字{分}}}{じう|ぶん}に
\ruby{談笑}{だん|せう}も
\ruby{仕}{し}て、
%
\ruby{大{\換字{分}}}{だい|ぶ}
\ruby{醉}{ゑひ}さえも% 「醉」は原本通り「ゑ」で調整
\ruby{{\換字{廻}}}{まは}つて
\ruby{居}{ゐ}る。
%
\ruby{談話}{はな|し}の
\ruby{序}{つひで}から
\ruby{不圖}{ふ|と}
\ruby{水野}{みづ|の}の
\ruby{事}{こと}が
\ruby{出}{で}て、
%
\原本頁{12-1}
\ruby{始}{はじ}めて
\ruby{君}{きみ}は
\ruby{其}{それ}を
\ruby{聞}{き}いた
ところから、
%
\ruby{大}{おほき}に
\ruby{忌}{いま}はしくも
\ruby{感}{かん}じたらうが、
%
\ruby{何}{なに}も
\ruby{今}{いま}が
\ruby{今}{いま}で
\ruby{無}{な}くちやならぬといふ
\ruby{事}{こと}では
\ruby{無}{な}いから、
%
\ruby{彼}{かれ}を
\ruby{訪}{と}ふのは
\ruby{明日}{あ|す}でも
\ruby{明後日}{あさ|つ|て}でもの
\ruby{事}{こと}として、
%
\ruby{其時}{その|とき}
\ruby{戀愛}{れん|あい}
\ruby{{\換字{嫌}}}{ぎら}ひの
\ruby{君}{きみ}の
\ruby{存{\換字{分}}}{ぞん|ぶん}に、
%
\ruby{諫}{いさ}めるとも
\ruby{擲}{なぐ}る
ともするが
\ruby{宜}{よ}からう。
%
\ruby{今日}{け|ふ}は
\ruby{先}{ま}づ
\ruby{堪{\換字{忍}}}{かん|にん}して% 原文通り「堪忍」
\ruby{一同}{みん|な}と
\ruby{共}{とも}に、
%
\ruby{飮}{の}んで
\ruby{居}{ゐ}て
\ruby{吳}{く}れたつて
\ruby{可}{よ}いでは
\ruby{無}{な}いか。
』

\原本頁{12-7}
と、
%
\ruby{他}{ひと}の
\ruby{言}{い}ふ
ところは
\ruby{斜}{なゝめ}に
\ruby{外}{そ}らせて、
%
\ruby{我}{わ}が
\ruby{言}{い}ふ
ところは
\ruby{斜}{なゝめ}に
\ruby{徹}{とほ}す
\ruby{才子}{さい|し}の
\ruby{面}{おもて}は
\ruby{笑}{ゑみ}を
\ruby{湛}{たゝ}へて、
%
\ruby{巧}{たくみ}に
\ruby{粗獷}{ぶ|こつ}なる
\ruby{相手}{あひ|て}を
\ruby{制}{せい}すれば、
%
% 正直三昧となっていたが原本は「正直」「一三」「味」のように「一」が入っている
% 「一三」は「13(十三)」かあるいは「小林一三(いちぞう)明治6年(1873)1月3日生」なのかもしれない
% いずれにしても「一」がはっきり植字されているが意味がわからないので「一三」にはルビを振らない
\ruby[<g||]{正直}{しやうぢき}
\ruby{一三昧}{||まい}の
\ruby{日方}{ひ|かた}は、
%
\ruby{脆}{もろ}くも、
%
\ruby{羽{\換字{勝}}}{は|がち}を
\ruby{重}{おも}んずる
\ruby{{\換字{情}}}{こゝろ}より、

\原本頁{12-10}
『ムー、
%
\ruby{此}{こ}の
\ruby{席}{せき}が
\ruby{淋}{さび}しくなる?。
%
ア、
%
\ruby{其處}{そ|こ}へは
\ruby{些}{ちつと}も
\ruby{氣}{き}が
つかなかつた。
%
\ruby{成程}{なる|ほど}
\ruby{今}{いま}
\ruby{直}{すぐ}
\ruby{引張}{ひつ|ぱ}つて
\ruby{來}{こ}やうと
\ruby{云}{い}つたのは、
%
\ruby{乃公}{お|れ}が
\ruby{惡}{わる}かつた。
%
\原本頁{13-1}
こいつは
\ruby{一番}{いち|ばん}
\ruby{山瀬}{やま|せ}に
やられた。
%
ハヽヽ。
%
どうも
\ruby{山瀬}{やま|せ}は
\ruby{乃公}{お|れ}より
\ruby{怜悧}{り|こう}だ。
%
ハヽハヽ。
』

\原本頁{13-3}
と、
%
\ruby{露}{つゆ}ばかりの
\ruby{我執}{が|しふ}も
\ruby{無}{な}く
\ruby{笑}{わら}つて
\ruby{仕舞}{し|ま}つて、
%
\ruby{霽々}{はれ|〴〵}したる
\ruby{顏色}{かほ|つき}にも
\ruby{著}{しる}き
\ruby{胸}{むね}に
\ruby{何}{なに}も
\ruby{{\換字{遺}}}{のこ}さぬ
\ruby{有樣}{あり|さま}は、
%
\ruby{譬}{たと}へば
\ruby{風}{かぜ}
\ruby{{\換字{過}}}{す}ぎて
\ruby{林}{はやし}
おのづから
\ruby{靜}{しづか}に、
%
\ruby{雲}{くも}
\ruby{去}{さ}つて
\ruby{山}{やま}
\ruby{{\換字{更}}}{さら}に
\ruby{靑}{あを}きが
\ruby{如}{ごと}くなりしが、
%
\ruby{例}{れい}の
\ruby{癖}{くせ}とて
\ruby{突然}{とつ|ぜん}と、

\原本頁{13-6}
『ヤ、
%
\ruby{時}{とき}に
\ruby{羽{\換字{勝}}}{は|がち}
\ruby{君}{くん}
\ruby{一盃}{いつ|ぱい}
\ruby{吳}{く}れたまへ。
』

\原本頁{13-7}
と
\ruby{云}{い}ひ
\ruby{出}{いだ}したり。
%
\ruby{羽{\換字{勝}}}{は|がち}は
\ruby{機{\換字{嫌}}}{き|げん}
\ruby{好}{よ}く
\ruby{盃}{さかづき}を
さして、

\原本頁{13-8}
『
\ruby{相變}{あひ|かは}らず
\ruby{君}{きみ}は
\ruby{君}{きみ}の
\ruby{氣風}{き|ふう}で
\ruby{押{\換字{通}}}{おし|とほ}すナ。
%
どうだ
\ruby{軍{\換字{隊}}}{ぐん|たい}の
\ruby{生活}{せい|くわつ}は
\ruby{{\換字{愉}}快}{ゆ|くわい}かネ。
』

\原本頁{13-10}
と
\ruby{懷}{なつ}かし
\ruby{氣}{げ}に
\ruby{問}{と}へば、

\原本頁{13-11}
『ムヽ。
%
\ruby{左樣}{さ|う}さ、
%
\ruby{快活}{くわい|くわつ}な
\ruby{事}{こと}ばかりといふ
\ruby{譯}{わけ}にも
\ruby{行}{ゆ}かん。
%
\ruby{僕等}{ぼく|ら}の
\原本頁{14-1}
\ruby{身{\換字{分}}}{み|ぶん}では
\ruby{隨{\換字{分}}}{ずゐ|ぶん}
\ruby{箱詰}{はこ|づめ}に
なるのを
\ruby{甘}{あま}んじ
なけりやならん
\ruby{事}{こと}もあるが、
%
\ruby{其}{それ}が
\ruby{{\換字{即}}}{すなは}ち
\ruby{紀律}{き|りつ}で、
%
\ruby{紀律}{き|りつ}が
\ruby{{\換字{即}}}{すなは}ち
\ruby{精神}{せい|しん}である、
%
といふやうに
\ruby{考}{かんが}へて
\ruby{居}{ゐ}りやあ、
%
\ruby{別}{べつ}に
\ruby{窮屈}{きう|くつ}にも
\ruby{感}{かん}じない。
%
ホワイトシヤツを
\ruby{着慣}{き|な}れて
\ruby{見}{み}ると、
%
\ruby{彼}{あ}の
\ruby{硬}{こは}いものを
\ruby{身}{み}につけるのが、
%
\ruby{却}{かへ}つて
\ruby{好}{い}い
\ruby{心持}{こゝろ|もち}に
\ruby{思}{おも}へて
\ruby{來}{く}る。
%
\ruby{丁度}{ちやう|ど}
それと
\ruby{同}{おな}じ
\ruby{事}{こと}で、
%
\ruby{慣}{な}れて
みると
\ruby{嚴肅}{げん|しゆく}な
\ruby{中}{うち}には
\ruby{{\換字{愉}}快}{ゆ|くわい}が
あるから、
%
\ruby{僕}{ぼく}は
まあ
\ruby{不{\換字{愉}}快}{ふ|ゆ|くわい}には
\ruby{日}{ひ}を
\ruby{{\換字{送}}}{おく}らん。
』

\原本頁{14-7}
と
\ruby{答}{こた}へて
\ruby{其}{そ}の
\ruby{盃}{さかづき}を
\ruby{乾}{ほ}して
\ruby{洗}{あら}ふ。

\原本頁{14-8}
『
\ruby{左樣}{さ|う}だ。
%
\ruby{紀律}{き|りつ}を
\ruby{{\換字{尊}}重}{そん|ちやう}する
\ruby{中}{うち}には
\ruby{{\換字{愉}}快}{ゆ|くわい}が
ある。
%
そして
\ruby{何}{なん}の
\ruby{方面}{はう|めん}の
\ruby{事}{こと}でも
\ruby{紀律}{き|りつ}は
\ruby{大切}{たい|せつ}だ。
%
\ruby{{\換字{船}}}{ふね}の
\ruby{中}{うち}などは
\ruby{特}{こと}に
\ruby{然樣}{さ|う}だ。
%
そればかりぢやあ
\ruby{無}{な}い、
%
\ruby{僕}{ぼく}が
\ruby{私}{ひそか}に
\ruby{思}{おも}ふには、
%
\ruby{身體}{から|だ}を
\ruby{扱}{あつか}ふのに
\ruby{紀律}{き|りつ}が
\ruby{無}{な}いと
\ruby{身體}{から|だ}が
\ruby{衰}{おとろ}へる、
%
\ruby{心}{こゝろ}を
\ruby{扱}{あつか}ふにも
\ruby{紀律}{き|りつ}が
\ruby{無}{な}いと
\ruby{心}{こゝろ}が
\ruby{歪}{ゆが}んで、
%
そこで
\原本頁{15-1}
\ruby{戀愛}{れん|あい}
などゝいふものに
\ruby{取}{と}り
\ruby{憑}{つ}かれるのだ。
』

\原本頁{15-2}
と
\ruby{云}{い}ひながら
\ruby{徐}{しづか}に
\ruby{酒盃}{さか|づき}を
\ruby{受}{う}くれば、
%
\ruby{日方}{ひ|かた}は

\原本頁{15-3}
『
\ruby{確論}{かく|ろん}、
%
\ruby{確論}{かく|ろん}。
』

\原本頁{15-4}
と
\ruby{悅}{よろこ}び
\ruby{叫}{さけ}んで、
%
\ruby{自}{みづか}ら
\ruby{{\換字{酌}}}{しやく}を
\ruby{仕}{し}て
\ruby{{\換字{遣}}}{や}らんと
\ruby{徳利}{とく|り}を
\ruby{擧}{あ}ぐれば、
%
\ruby{既}{はや}
\ruby{飮}{の}み
\ruby{盡}{つく}して
\ruby{二三滴}{に|さん|てき}のみ。
%
\ruby{山瀬}{やま|せ}は
\ruby{急}{いそ}ぎ
\ruby{手}{て}を
\ruby{拍}{たゝ}き
\ruby{立}{た}つ。

\原本頁{15-6}
\ruby{此時}{この|とき}まで
にや〳〵と
\ruby{笑}{わら}ひながら、
%
\ruby{人々}{ひと|〴〵}の
\ruby{談}{はなし}を
のみ
\ruby{聞}{き}き
\ruby{居}{ゐ}たりし
\ruby{布袋肥胖}{ほ|てい|ぶ|と}りに
\ruby{肥}{ふと}つたる、
%
\ruby{丸顏}{まる|がほ}の
\ruby{眼下}{め|さが}りなる
\ruby{島木}{しま|き}は
\ruby{笑}{わら}つて、

\原本頁{15-8}
『ハヽヽ、
%
\ruby{談話}{はな|し}が
\ruby{惡}{わる}つ
\ruby{固}{かた}いから
\ruby{堪}{たま}りやあ
\ruby{仕無}{し|な}い。
%
\ruby{婢}{をんな}だつて
\ruby{何}{なん}だつて
\ruby{{\換字{逃}}}{に}げたつきりだ。
%
\ruby{徳利}{とつ|くり}の
\ruby{番兵}{ばん|ぺい}は
\ruby{野暮}{や|ぼ}ぢやあ
\ruby{使}{つか}へ
\ruby{無}{ね}えからな。
%
ハヽヽ。
%
\ruby{何}{なん}だい?。
%
\ruby{紀律}{き|りつ}が
\ruby{無}{な}いと
いけ
\ruby{無}{な}いつて?。
%
\ruby{戱談}{じやう|だん}
\ruby{言}{い}つちやあ
いけない、
%
\ruby{舞臺}{ぶ|たい}に
\ruby{障}{さは}るぜ。
%
\ruby{不紀律}{ふ|き|りつ}の
\ruby{大將}{たい|しやう}、
%
\ruby{實業家}{じつ|げふ|か}
\ruby{{\換字{兼}}}{けん}
\ruby{虛業家}{きよ|げふ|か}、
%
\原本頁{16-1}
\ruby{相場師}{さう|ば|し}に% 原文通り「場」
なつたつて、
%
\ruby{一同}{みん|な}に
\ruby{怒}{おこ}られた、
%
\ruby{御利益}{ご|り|やく}は
\ruby{未}{ま}だ
\ruby{蒙}{かうむ}ら
\ruby{無}{な}いが
\ruby{拜金宗}{はい|きん|しう}の
\ruby{信徒}{しん|と}の、
%
\ruby{島木}{しま|き}
\ruby{萬五郎}{まん|ご|らう}
\ruby{樣}{さま}が
\ruby{此處}{こ|ゝ}に
\ruby{御坐}{お|いで}なさるぜ。
%
\ruby{憚}{はゞか}りながら% 「憚 は(ゞ)か」
\ruby{乃公}{お|れ}が
\ruby{何時}{い|つ}
\ruby{戀愛}{れん|あい}に
\ruby{取}{と}り
\ruby{憑}{つ}かれた。
%
ハヽヽ。
%
\ruby{其}{そ}りやあ
\ruby{左樣}{さ|う}と
\ruby{水野}{みづ|の}の
\ruby{談}{はなし}は
\ruby{譯}{わけ}
\ruby{有}{あ}つて
\ruby{一番}{いち|ばん}
\ruby{乃公}{お|れ}が
\ruby{知}{し}つている。
%
どうも
\ruby{一同}{みん|な}が
\ruby{氣}{き}に
\ruby{仕}{し}て
\ruby{居}{ゐ}る。
%
\ruby{羽{\換字{勝}}}{は|がち}の
\ruby{腹}{はら}の
\ruby{中}{なか}では
\ruby{取}{と}り
\ruby{{\換字{分}}}{わ}け
\ruby{深}{ふか}く
\ruby{心配}{しん|ぱい}して
\ruby{居}{ゐ}る
やうすだから
\ruby{話}{はな}して
\ruby{聞}{き}かさうか。
』

\原本頁{16-7}
と、
%
\ruby{始}{はじめ}は
\ruby{戱}{たはむ}れ、
%
\ruby{{\換字{終}}}{をはり}は
\ruby{眞面目}{ま|じ|め}に
\ruby{云}{い}ひ
\ruby{出}{い}づれば、
%
\ruby{謹聽}{きん|ちやう}の
\ruby{聲}{こゑ}は
\ruby{異口}{い|く}
\ruby{一齊}{いつ|せい}に
\ruby{出}{い}でぬ。
