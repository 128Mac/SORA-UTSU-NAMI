\Entry{其二}

\ruby{薄墨}{うす|ずみ}の
\ruby{夕}{ゆふべ}の
\ruby{色}{いろ}は
\ruby{物蔭}{もの|かげ}より
\ruby{擴}{ひろ}まりて、
\ruby{廓然}{くわ|らり}と
\ruby{{\換字{晴}}}{は}れやかなりし
\ruby{樓}{ろう}の
\ruby{上}{うへ}も、
\ruby{手許}{て|もと}やうやく
\ruby{暗}{くら}くなり、いづくに
\ruby{歸}{かへ}る
\ruby{鵜}{う}の
\ruby{鳥}{とり}の、
\ruby{浪}{なみ}を
\ruby{{\換字{摩}}}{す}つて
\ruby{飛}{と}ぶ
\ruby{{\換字{羽}}音}{は|おと}も
\ruby{寂}{さ}びたり。
\ruby{右}{みぎ}の
\ruby{方}{かた}は
\ruby{高輪八}{たか|なわ|や}ツ
\ruby{山}{やま}
\ruby{品川}{しな|がは}の
\ruby{一}{ひ}トつゞき、
\ruby{森}{もり}も
\ruby{人家}{じん|か}もたゞ
\ruby{一}{ひ}ト
\ruby{筆}{ふで}のなすり
\ruby{書}{がき}と
\ruby{黑}{くろ}み、
\ruby{左}{ひだり}に
\ruby{低}{ひく}き
\ruby{築地月島}{つき|ぢ|つき|しま}、
\ruby{洲崎}{す|さき}は
\ruby{微}{かすか}にして
\ruby{{\換字{消}}}{き}えんとする
\ruby{時}{とき}、
\ruby{其處}{そ|こ}に
\ruby{電燈}{でん|とう}の
\ruby{白々}{しろ|〴〵}と
\ruby{輝}{かゞや}き
\ruby{出}{い}づれば、
\ruby{燈火}{とも|しび}
\ruby{華}{はな}やかに
\ruby{此家}{こ|こ}にも
\ruby{點}{つ}きて、
\ruby{室}{へや}の
\ruby{内}{うち}ぱつと
\ruby{明}{あか}るくなり、
\ruby{外}{そと}は
\ruby{全}{まつた}く
\ruby{海}{うみ}
\ruby{玄}{くろ}く
\ruby{風}{かぜ}
\ruby{睡}{ねむ}れる
\ruby{{\換字{穏}}}{おだ}やかなる
\ruby{夜}{よ}となり
\ruby{畢}{をは}んぬ。

\ruby[g]{日方}{ひかた}が
\ruby{急}{せ}き
\ruby{{\換字{込}}}{こ}み
\ruby{調子}{てう|し}に
\ruby{物言}{もの|い}ひても、
\ruby{特更}{こと|さら}に
\ruby{沈着}{おち|つき}を
\ruby{爲}{つく}れる
\ruby{山瀬荒吉}{やま|せ|あら|きち}は、
\ruby{言}{い}ひ
\ruby{爭}{あらそ}はんともせで
\ruby{良}{やゝ}
\ruby[g]{少時}{しばし}、
\ruby{何事}{なに|ごと}をか
\ruby{思}{おも}ひ
\ruby{廻}{めぐ}らし
\ruby{居}{ゐ}けるが、
\ruby{今}{いま}しも
\ruby{燈火}{とも|しび}の
\ruby{光}{ひかり}を
\ruby{得}{え}て、
\ruby{心}{こゝろ}の
\ruby{中}{うち}に
\ruby{索}{たづ}ね
\ruby{得}{え}し
\ruby[g]{言葉}{ことば}の
\ruby{緒}{いとぐち}をや
\ruby{求}{もと}め
\ruby{得}{え}けん、
\ruby{逸}{はや}りきつたる
\ruby[g]{日方}{ひかた}の
\ruby{面}{おもて}の、いさゝか
\ruby{怒}{いかり}をさへ
\ruby{{\換字{帯}}}{お}びたるを、
\ruby{愛}{あい}するが
\ruby{如}{ごと}く
\ruby{打見}{うち|み}やりて、

『マア
\ruby{坐}{すわ}つて
\ruby{{\換字{呉}}}{く}れ、
\ruby[g]{日方}{ひかた}!。
\ruby{成程}{なろ|ほど}
\ruby{打棄}{うつ|ちや}つて
\ruby{置}{おい}ては
\ruby[g]{水野}{みづの}の
\ruby{不利{\換字{益}}}{ふ|た|め}になるから、
\ruby{君}{きみ}と一
\ruby{緒}{しよ}に
\ruby{尋}{たづ}ねて
\ruby{行}{い}つて、
\ruby{隨分}{ずゐ|ぶん }
\ruby{忠告}{ちゆう|こく}も
\ruby{試}{こヽろ}みやう。
\ruby{倂}{しか}し
\ruby[g]{水野}{みづの}のところは
\ruby{大分}{だい|ぶん}
\ruby{{\換字{遠}}}{とほ}い。
\ruby{{\換字{連}}}{つ}れて
\ruby{來}{く}るにしても
\ruby{時間}{と|き}がかゝる。
もう
\ruby{此}{こ}の
\ruby{{\換字{通}}}{とほ}り
\ruby{夜}{よ}にも
\ruby{入}{い}つて
\ruby{居}{ゐ}る。
\ruby{{\換字{連}}}{つ}れて
\ruby{來}{き}たにしたところで
\ruby{話}{はな}す
\ruby{間}{ま}も
\ruby{無}{な}い。
\ruby{第一}{だい|いち}
\ruby{左様}{さ|う}で
\ruby{無}{な}くつてさへ、
\ruby{七人}{しち|にん}の
\ruby{中}{うち}が
\ruby{三人}{さん|にん}
\ruby{缺}{か}けて、
\ruby{四人}{よ|にん}しか
\ruby{居}{を}らぬ
\ruby{此}{こ}の
\ruby{席}{せき}を、
\ruby{君}{きみ}と
\ruby{僕}{ぼく}と
\ruby{二人}{ふた|り}
\ruby{脱}{ぬ}けて
\ruby{仕舞}{し|ま}へば
\ruby{後}{あと}は
\ruby{何様}{ど|う}だ。
\ruby[g]{{\換字{羽}\換字{勝}}君}{はがちくん}と
\ruby{島木君}{しま|き|くん}とたつた
\ruby{二人}{ふた|り}だ。
\ruby{今日}{け|ふ}の
\ruby{客}{きやく}たる
\ruby[g]{{\換字{羽}\換字{勝}}君}{はがちくん}を、
\ruby{島木君}{しま|き|くん}と
\ruby{只二人}{たつた|ふた|り}に
\ruby{仕}{し}て
\ruby{仕舞}{し|ま}つて、
\ruby{僕等}{ぼく|ら}が
\ruby{出}{で}て
\ruby{行}{い}くといふのは
\ruby{勝手}{かつ|て}
\ruby{{\換字{過}}}{す}ぎる。
それでは
\ruby{餘}{あんま}り
\ruby{無禮}{ぶ|れい}になる。
こゝを
\ruby{無理}{む|り}に
\ruby{君}{きみ}と
\ruby{二人}{ふた|り}で
\ruby{出}{で}て
\ruby{行}{い}つたら、
\ruby[g]{水野}{みづの}には
\ruby{成程親切}{なる|ほど|しん|せつ}にもならう。
\ruby{倂}{しか}し
\ruby[g]{{\換字{羽}\換字{勝}}君}{はがちくん}には
\ruby{失敬}{しつ|けい}に
\ruby{當}{あた}らう。
もと〳〵
\ruby{君}{きみ}が
\ruby{怒}{おこ}り
\ruby{立}{た}つたのも、つまりは
\ruby[g]{水野}{みづの}が
\ruby[g]{{\換字{羽}\換字{勝}}君}{はがちくん}に
\ruby{對}{たい}する
\ruby{仕方}{し|かた}が
\ruby{冷淡}{れい|たん}だといふのにあらう。
\ruby[g]{{\換字{羽}\換字{勝}}君}{はがちくん}に
\ruby{滿足}{まん|ぞく}を
\ruby{感}{かん}ぜしめぬ
\ruby{其事}{そ|れ}が
\ruby{惡}{にく}むべき
\ruby{我儘}{わが|まま}だといふのだ。
それだのに
\ruby{今}{いま}
\ruby{僕等}{ぼく|ら}が
\ruby{此席}{こ|こ}を
\ruby{去}{さ}つては、たゞ
\ruby{淋}{さび}しさを
\ruby{{\換字{増}}}{ま}すばかりで、
\ruby[g]{{\換字{羽}\換字{勝}}君}{はがちくん}はいよ〳〵おもしろく
\ruby{無}{な}く
\ruby{感}{かん}じやう。
\ruby{今日}{け|ふ}は
\ruby{既}{もう}十
\ruby{分}{ぶん}に
\ruby{談笑}{だん|せう}も
\ruby{仕}{し}て、
\ruby{大分}{だい|ぶ}
\ruby{醉}{よひ}さえも
\ruby{{\換字{廻}}}{まは}って
\ruby{居}{ゐ}る。
\ruby[g]{談話}{はなし}の
\ruby{序}{つひで}から
\ruby{不圖}{ふ|と}
\ruby[g]{水野}{みづの}の
\ruby{事}{こと}が
\ruby{出}{で}て、
\ruby{始}{はじ}めて
\ruby{君}{きみ}は
\ruby{其}{それ}を
\ruby{聞}{き}いたところから、
\ruby{大}{おほき}に
\ruby{忌}{いま}はわしくも
\ruby{感}{かん}じたらうが、
\ruby{何}{なに}も
\ruby{今}{いま}が
\ruby{今}{いま}でなくちやならぬといふ
\ruby{事}{こと}では
\ruby{無}{な}いから、
\ruby{彼}{かれ}を
\ruby{訪}{と}ふのは
\ruby{明日}{あ|す}でも
\ruby[g]{明後日}{あさつて}でもの
\ruby{事}{こと}として、
\ruby{其時}{その|とき}
\ruby{戀愛{\換字{嫌}}}{れん|あい|ぎら}ひの
\ruby{君}{きみ}の
\ruby{存分}{ぞん|ぶん}に、
\ruby{諫}{いさ}めるとも
\ruby{擲}{なぐ}るともするが
\ruby{宜}{よ}からう。
\ruby{今日}{け|ふ}は
\ruby{先}{ま}づ
\ruby{堪{\換字{忍}}}{かん|にん}して
\ruby[g]{一同}{みんな}と
\ruby{共}{とも}に、
\ruby{{\換字{飲}}}{の}んで
\ruby{居}{ゐ}て
\ruby{{\換字{呉}}}{く}れたつて
\ruby{可}{よ}いでは
\ruby{無}{な}いか。
』

と、
\ruby{他}{ひと}の
\ruby{言}{い}ふところは
\ruby{斜}{なヽめ}に
\ruby{外}{そ}らせて、
\ruby{我}{わ}が
\ruby{言}{い}ふところは
\ruby{斜}{なヽめ}に
\ruby{徹}{とほ}す
\ruby{才士}{さい|し}の
\ruby{面}{おもて}は
\ruby{笑}{ゑみ}を
\ruby{湛}{たヽ}へて、
\ruby{巧}{たくみ}に
\ruby{粗獷}{ぶ|こつ}なる
\ruby{相手}{あひ|て}を
\ruby{制}{せい}すれば、
\ruby[g]{正直三昧}{しやうぢきざんまい}の
\ruby[g]{日方}{ひかた}は、
\ruby{脆}{もろ}くも、
\ruby[g]{{\換字{羽}\換字{勝}}}{はがち}を
\ruby{重}{おも}んずる
\ruby{{\換字{情}}}{こヽろ}より、

『ムー、
\ruby{此}{こ}の
\ruby{席}{せき}が
\ruby{淋}{さび}しくなる?。
ア、
\ruby{其處}{そ|こ}へは
\ruby{些}{ちつと}も
\ruby{氣}{き}がつかなかつた。
\ruby{成程}{なる|ほど}
\ruby{今直}{いま|すぐ}
\ruby{引張}{ひつ|ぱ}つて
\ruby{來}{こ}やうと
\ruby{云}{い}ったのは、
\ruby{乃公}{お|れ}が
\ruby{惡}{わる}かつた。
こいつは
\ruby{一番}{いち|ばん}
\ruby{山瀬}{やま|せ}にやられた。
ハヽヽ。
どうも
\ruby{山瀬}{やま|せ}は
\ruby{乃公}{お|れ}より
\ruby{怜悧}{り|こう}だ。
ハヽヽ。
』

と、
\ruby{露}{つゆ}ばかりの
\ruby{我執}{が|しふ}も
\ruby{無}{な}く
\ruby{笑}{わら}つて
\ruby{仕舞}{し|ま}つて、
\ruby{霽々}{はれ|〴〵}したる
\ruby{顏色}{かほ|つき}にも
\ruby{著}{しる}き
\ruby{胸}{むね}に
\ruby{何}{なに}も
\ruby{{\換字{遺}}}{のこ}さぬ
\ruby{有様}{あり|さま}は、
\ruby{譬}{たと}へば
\ruby{風{\換字{過}}}{かぜ|す}ぎて
\ruby{林}{はやし}おのづから
\ruby{靜}{しづか}に、
\ruby{雲}{くも}
\ruby{去}{さ}つて
\ruby{山}{やま}
\ruby{更}{さら}に
\ruby{靑}{あお}きが
\ruby{如}{ごと}くなりしが、
\ruby{例}{れい}の
\ruby{癖}{くせ}とて
\ruby{突然}{とつ|ぜん}と、

『ヤ、時に
\ruby[g]{{\換字{羽}\換字{勝}}君}{はがちくん}
\ruby{一盃}{いつ|ぱい}
\ruby{吳}{く}れたまへ。
』

と
\ruby{云}{い}ひ
\ruby{出}{いだ}したり。
\ruby[g]{{\換字{羽}\換字{勝}}}{はがち}は
\ruby{機{\換字{嫌}}}{き|げん}
\ruby{良}{よ}く
\ruby{盃}{さかづき}をさして、

『
\ruby{相變}{あひ|かは}らず
\ruby{君}{きみ}は
\ruby{君}{きみ}の
\ruby{氣風}{き|ふう}で
\ruby{押{\換字{通}}}{おし|とほ}すナ。
どうだ
\ruby{軍{\換字{隊}}}{ぐん|たい}の
\ruby{生活}{せい|くわつ}は
\ruby{{\換字{愉}}快}{ゆ|くわい}かネ。
』

と
\ruby{{\換字{懐}}}{なつ}かし
\ruby{氣}{げ}に
\ruby{問}{と}へば、

『ムヽ。
\ruby{左様}{さ|う}さ。
\ruby{快活}{くわい|くわつ}な
\ruby{事}{こと}ばかりといふ
\ruby{譯}{わけ}にも行かん。
\ruby{僕等}{ぼく|ら}の
\ruby{身分}{み|ぶん}では
\ruby{隨分}{ずゐ|ぶん}
\ruby{箱詰}{はこ|づめ}になるのを
\ruby{甘}{あま}んじなけりやならん
\ruby{事}{こと}もあるが、
\ruby{其}{それ}が
\ruby{{\換字{即}}}{すなは}ち
\ruby{規律}{き|りつ}で、
\ruby{規律}{き|りつ}が
\ruby{{\換字{即}}}{すなは}ち
\ruby{精神}{せい|しん}である、といふやうに
\ruby{考}{かんが}へて
\ruby{居}{ゐ}りやあ、
\ruby{別}{べつ}に
\ruby{窮屈}{きう|くつ}にも
\ruby{感}{かん}じない。
ホワイトシヤツを
\ruby{着慣}{き|な}れて
\ruby{見}{み}ると、
\ruby{彼}{あ}の
\ruby{硬}{こは}いものを
\ruby{身}{み}につけるのが、
\ruby{却}{かへ}つて
\ruby{好}{い}い
\ruby{心持}{こヽろ|もち}に
\ruby{思}{おも}へて
\ruby{來}{く}る。
\ruby{丁度}{ちやう|ど}それと
\ruby{同}{おな}じ
\ruby{事}{こと}で、
\ruby{慣}{な}れてみると
\ruby{嚴肅}{げん|しゆく}な
\ruby{中}{うち}には
\ruby{{\換字{愉}}快}{ゆ|くわい}があるから、
\ruby{僕}{ぼく}はまあ
\ruby{不{\換字{愉}}快}{ふ|ゆ|くわい}には
\ruby{日}{ひ}を
\ruby{{\換字{送}}}{おく}らん。
』

と
\ruby{答}{こた}へて
\ruby{其}{そ}の
\ruby{盃}{さかづき}を
\ruby{乾}{ほ}して
\ruby{洗}{あら}ふ。

『
\ruby{左様}{さ|う}だ。
\ruby{規律}{き|りつ}を
\ruby{{\換字{尊}}重}{そん|ちやう}する
\ruby{中}{うち}には
\ruby{{\換字{愉}}快}{ゆ|くわい}がある。
そして
\ruby{何}{なん}の
\ruby{方面}{はう|めん}の
\ruby{事}{こと}でも
\ruby{規律}{き|りつ}は
\ruby{大切}{たい|せつ}だ。
\ruby{{\換字{船}}}{ふね}の
\ruby{中}{うち}などは
\ruby{特}{こと}に
\ruby{然様}{さ|う}だ。
そればかりぢやあ
\ruby{無}{な}い、
\ruby{僕}{ぼく}が
\ruby{私}{ひそか}に
\ruby{思}{おも}ふには、
\ruby[g]{身體}{からだ}を
\ruby{扱}{あつか}ふのに
\ruby{規律}{き|りつ}が
\ruby{無}{な}いと
\ruby[g]{身體}{からだ}が
\ruby{衰}{おとろ}へる、
\ruby{心}{こヽろ}を
\ruby{扱}{あつか}ふにも
\ruby{規律}{き|りつ}が
\ruby{無}{な}いと
\ruby{心}{こヽろ}が
\ruby{歪}{ゆが}んで、そこで
\ruby{戀愛}{れん|あい}などゝいふものに
\ruby{取}{と}り
\ruby{憑}{つ}かれるのだ。
』

と
\ruby{云}{い}ひながら
\ruby{徐}{しづか}に
\ruby{酒盃}{さか|づき}を
\ruby{受}{う}くれば、
\ruby[g]{日方}{ひかた}は

『
\ruby{確論}{かく|ろん}、
\ruby{確論}{かく|ろん}。
』

と
\ruby{{\換字{悅}}}{よろこ}び
\ruby{叫}{さけ}んで、
\ruby{自}{みづか}ら
\ruby{{\換字{酌}}}{しやく}を
\ruby{仕}{し}て
\ruby{{\換字{遣}}}{や}らんと
\ruby{徳利}{とく|り}を
\ruby{擧}{あ}ぐれば、
\ruby{既}{はや}
\ruby{{\換字{飲}}}{の}み
\ruby{盡}{つく}して
\ruby{二三滴}{に|さん|てき}のみ。
\ruby{山瀬}{やま|せ}は
\ruby{急}{いそ}ぎ
\ruby{手}{て}を
\ruby{拍}{たヽ}き
\ruby{立}{た}つ。

\ruby{此時}{この|とき}までにや〳〵と
\ruby{笑}{わら}ひながら、
\ruby{人々}{ひと|〴〵}の
\ruby{談}{はなし}をのみ
\ruby{聞}{き}き
\ruby{居}{ゐ}たりし
\ruby{布袋肥胖}{ほ|てい|ぶ|と}りに
\ruby{肥}{ふと}つたる、
\ruby{丸{\換字{顔}}}{まる|がほ}の
\ruby{眼下}{め|さが}りなる
\ruby{島木}{しま|き}は
\ruby{笑}{わら}つて、

『ハヽヽ、
\ruby{談話}{はな|し}が
\ruby{惡}{わる}つ
\ruby{固}{かた}いから
\ruby{堪}{たま}りやあ
\ruby{仕無}{し|な}い。
\ruby{婢}{をんな}だつて
\ruby{何}{なん}だつて
\ruby{逃}{に}げたつきりだ。
\ruby{徳利}{とつ|くり}の
\ruby{番兵}{ばん|ぺい}は
\ruby{野暮}{や|ぼ}ぢやあ
\ruby{使}{つか}へ
\ruby{無}{ね}えからな。
ハヽヽ、
\ruby{何}{なん}だい。
\ruby{規律}{き|りつ}が
\ruby{無}{な}いといけ
\ruby{無}{な}いつて?
\ruby{戯談}{じやう|だん}
\ruby{言}{い}つちやあいけない、
\ruby{舞臺}{ぶ|たい}に
\ruby{障}{さは}るぜ。
\ruby{不規律}{ふ|き|りつ}の
\ruby{大將}{たい|しやう}、
\ruby[g]{實業家{\換字{兼}}{\換字{虛}}業家}{じつげふかけんきよげふか}、
\ruby{相場師}{さう|ば|し}になつたつて、
\ruby[g]{一同}{みんな}に
\ruby{怒}{おこ}られた、
\ruby{御利{\換字{益}}}{ご|り|やく}は
\ruby{未}{ま}だ
\ruby{蒙}{かうむ}ら
\ruby{無}{な}いが
\ruby{拝金宗}{はい|きん|しう}の
\ruby{信徒}{しん|と}の、
\ruby{島木萬五郎様}{しま|き|まん|ご|らう|さま}が
\ruby{此處}{こ|こ}に
\ruby{御坐}{お|いで}なさるぜ。
\ruby{憚}{はゞか}りながら
\ruby{乃公}{お|れ}が
\ruby{何時}{い|つ}
\ruby{戀愛}{れん|あい}に
\ruby{取}{と}り
\ruby{憑}{つ}かれた。
ハヽヽ、
\ruby{其}{そ}りやあ
\ruby{左様}{さ|う}と
\ruby[g]{水野}{みづの}の
\ruby{談}{はなし}は
\ruby{譯}{わけ}
\ruby{有}{あ}つて
\ruby{一番}{いち|ばん}
\ruby{乃公}{お|れ}が
\ruby{知}{し}つている。
どうも
\ruby[g]{一同}{みんな}が
\ruby{氣}{き}に
\ruby{仕}{し}て
\ruby{居}{ゐ}る。
\ruby[g]{{\換字{羽}\換字{勝}}}{はがち}の
\ruby{腹}{はら}の
\ruby{中}{なか}では
\ruby{取}{と}り
\ruby{分}{わ}け
\ruby{深}{ふか}く
\ruby{心配}{しん|ぱい}して
\ruby{居}{ゐ}るやうすだから
\ruby{話}{はな}して
\ruby{聞}{き}かさうか。
』

と、
\ruby{始}{はじめ}は
\ruby{戯}{たはむ}れ、
\ruby{終}{をわり}は
\ruby{眞面目}{ま|じ|め}に
\ruby{云}{い}ひ
\ruby{出}{い}づれば、
\ruby{謹聽}{きん|ちやう}の
\ruby{聲}{こゑ}は
\ruby{異口一齊}{い|く|いつ|せい}に
\ruby{出}{い}でぬ。

