\Entry{其三}

\ruby{島木}{しま|き}は
\ruby{驕}{おご}れるにもあらず
\ruby{慢}{あなど}れるにもあらず、たゞ\GWI{u1b048}たゝかなる
\ruby[g]{放肆兒}{だゞつこ}の、
\ruby{一家}{いつ|か}の
\ruby{長者}{ちやう|じや}をもはゞからずして、
\ruby{自己}{お|の}の
\ruby{勝手}{かつ|て}に
\ruby{泣}{な}きも
\ruby{笑}{わら}ひもするやうに、\GWI{u1b048}かも
\ruby{其}{そ}の
\ruby[g]{小兒}{こども}らしき
\ruby{{\GWI{u984f-j}}}{かほ}に
\ruby{微笑}{ゑ|み}をうかめて、

『ハヽヽ、
\ruby[g]{日方}{ひかた}までが
\ruby{謹聽}{きん|ちやう}と
\ruby{吐}{ぬ}かし
\ruby{居}{を}つたな!。
\ruby{一體}{いつ|たい }
\ruby{汝}{きさま}は
\ruby{人}{ひと}は
\ruby{好}{よ}いが、
\ruby{我}{が}ばかり
\ruby{{\換字{強}}}{つよ}くつて
\ruby{思}{おも}ひ
\ruby{{\GWI{u9063-k}}}{や}りが
\ruby{足}{た}らない。
\ruby{此}{こ}の
\ruby{思}{おも}ひ
\ruby{{\GWI{u9063-k}}}{や}りの
\ruby{足}{た}らない
\ruby{手合}{て|あひ}が、
\ruby{他人}{た|にん}の
\ruby{戀愛}{れん|あい}の
\ruby{談}{はなし}などには、
\ruby{兎角}{と|かく}に
\ruby{點頭}{がつ|てん}しかねるものだ。
\ruby{線}{せん}の
\ruby{無}{な}い
\ruby{家}{うち}にやあ
\ruby{電話}{でん|わ}は
\ruby{{\GWI{u901a-k}}}{つう}じない、
\ruby{思}{おも}ひ
\ruby{{\GWI{u9063-k}}}{や}りの
\ruby{足}{た}らない
\ruby{奴等}{やつ|ら}にやあ
\ruby{戀愛}{れん|あい}は
\ruby{解}{げ}せない。
そこへ
\ruby{行}{い}つちやあ
\ruby{乃公}{お|れ}なんぞは、
\ruby{身}{み}に
\ruby[g]{經驗}{おぼえ}があつて
\ruby{同{\換字{情}}}{おもひ|やり}が
\ruby{{\換字{強}}}{つよ}いから、ツーと
\ruby{云}{い}やあカーと
\ruby{合點}{が|てん}がいくので、
\ruby{初心}{う|ぶ}な
\ruby[g]{水野}{みづの}の
\ruby{譚}{はなし}なんざあ、
\ruby{何程}{いく|ら}
\ruby{彼}{あれ}が
\ruby{心}{こゝろ}の
\ruby{奥}{おく}に
\ruby{秘}{かく}して
\ruby{居}{お}つても、
\ruby{深}{ふか}い
\ruby{井}{いど}の
\ruby{床}{そこ}を
\ruby{鏡}{かゞみ}で
\ruby{照}{て}らして、
\ruby{見}{み}て
\ruby{取}{と}るやうに
\ruby{譯}{わけ}も
\ruby{無}{な}く
\ruby{見抜}{み|ぬ}く。
\ruby{本來}{ほん|らい}
\ruby{戀}{こひ}といふ
\ruby{事}{こと}が
\ruby{罪惡}{つ|み}ぢやあ
\ruby{有}{あ}るまいし、
\ruby[g]{日方}{ひかた}のやうな
\ruby{暴論}{ばう|ろん}の
\ruby{愚論}{ぐ|ろん}・・・』

と
\ruby{云}{い}ひかくる
\ruby{時}{とき}
\ruby[g]{日方}{ひかた}は
\ruby{堪}{こら}へず、

『
\ruby{何}{なん}だ、
\ruby{暴論}{ばう|ろん}だと!。
こりやあ
\ruby{怪}{け}しからん。
\ruby{汝}{きさま}も
\ruby{戀愛}{れん|あい}の
\ruby{奴隷}{ど|れい}
\ruby{臭}{くさ}いぞ。
\ruby{身}{み}に
\ruby[g]{經驗}{おぼえ}があつてとは
\ruby{何}{なん}たる
\ruby{囈語}{ね|ごと}だ。
\ruby{聞}{き}きぐるしいことを
\ruby{吐}{ぬか}さずとも、さつさと
\ruby[g]{水野}{みづの}のことを
\ruby{話}{はな}すが
\ruby{可}{よ}い。
』

と
\ruby{怒鳴}{ど|な}りつくれば、
\ruby[g]{此方}{こなた}はいよ〳〵
\ruby{笑}{わら}い
\ruby{傾}{かたむ}き、

『
\ruby{安心}{あん|しん}しろ
\ruby[g]{日方}{ひかた}!。
\ruby{乃公}{お|ら}あ
\ruby{女}{をんな}に
\ruby{惚}{ほ}れて
\ruby{戀}{こひ}はおぼえねえ。
ヘン
\ruby{惚}{ほ}れられて
\ruby{惚}{ほ}れられて
\ruby{戀}{こひ}といふものは
\ruby{此様}{こ|ん}なものかと
\ruby{知}{し}つたんだからナ。
アハヽヽヽ、
\ruby{何様}{ど|う}だい
\ruby{奴}{やつこ}さん、
\ruby{如何}{い|かゞ}でござる!。
そこで
\ruby{惚}{ほ}れられて
\ruby{惚}{ほ}れられて
\ruby{悟}{さと}つて
\ruby{見}{み}ると、
\ruby[g]{水野}{みづの}を
\ruby{辯護}{べん|ご}するといふ
\ruby{譯}{わけ}ぢやあ
\ruby{無}{な}いが、
\ruby{戀}{こひ}は
\ruby{人間}{ひ|と}の
\ruby{{\換字{情}}}{じやう}の
\ruby{自然}{し|ぜん}の
\ruby[g]{發動}{うごき}で、
\ruby{何}{なに}も
\ruby{咎}{とが}め
\ruby{立}{だ}てをすることは
\ruby{有}{あ}りやしない。
\ruby[g]{日方}{ひかた}にやあ
\ruby[g]{日方}{ひかた}だけの
\ruby{愚論}{ぐ|ろん}もあらうが、
\ruby{乃公}{お|ら}あ
\ruby{戀}{こひ}に
\ruby{{\GWI{u8ff7-k}}}{まよ}つた
\ruby{彼}{あ}の
\ruby[g]{水野}{みづの}を、
\ruby[g]{憫然}{かはいさう}だたあ
\ruby{思}{おも}ふが
\ruby{惡}{にく}かあ
\ruby{思}{おも}はねえ。
』

と
\ruby{云}{い}はせも
\ruby{果}{は}てず
\ruby[g]{日方}{ひかた}は
\ruby{目}{め}を
\ruby{剥}{む}き、

『
\ruby{馬鹿野郎}{ば|か|や|らう}ッ。
』

と
\ruby{烈}{はげ}しく
\ruby{罵}{のゝ}しつたる
\ruby{裂帛}{れつ|ぱく}の
\ruby{一聲}{いつ|せい}に
\ruby{氣合}{き|あひ}
\ruby{籠}{こも}つて、
\ruby{人}{ひと}の
\ruby{肺腑}{はい|ふ}に
\ruby{響}{ひゞ}き
\ruby{徹}{てつ}したり。

『マァ
\ruby{待}{ま}ち
\ruby{玉}{たま}へ。
』

『
\ruby{爭}{あらそ}つちやいかん。
』

と、
\ruby{口}{くち}を
\ruby{衝}{つ}いて
\ruby{出}{い}でたる
\ruby{山瀬}{やま|せ}
\ruby[g]{{\GWI{u7fbd-k}\換字{勝}}}{はがち}の
\ruby{二人}{に|にん}の
\ruby{言葉}{こと|ば}は
\ruby{一句}{いつ|く}と
\ruby{一句}{いつ|く}と
\ruby{斷}{き}るゝ
\ruby{間}{ひま}
\ruby{無}{な}く
\ruby{巧}{たくみ}に
\ruby{續}{つゞ}きて、
\ruby[g]{突差}{とつさ}に
\ruby{緊}{きび}しく
\ruby{制}{せい}し
\ruby{止}{と}むれば、
\ruby[g]{流石}{さすが}に
\ruby[g]{日方}{ひかた}も
\ruby[g]{{\GWI{u7fbd-k}\換字{勝}}}{はがち}を
\ruby{憚}{はゞか}りて、
\ruby{言}{ものい}はんとして
\ruby{言}{い}はず
\ruby{已}{や}みけるが、
\ruby{眼}{め}には
\ruby{{\GWI{u7336-k}}}{なほ}
\ruby{稜角}{か|ど}を
\ruby{立}{た}てゝ
\ruby{島木}{しま|き}を
\ruby{睨}{にら}み、
\ruby{此}{こ}の
\ruby{時}{とき}
\ruby{遲}{おそ}く
\ruby{彼}{か}の
\ruby{時速}{とき|はや}く、

『そら
\ruby{{\換字{叉}}}{また}
\ruby{馬鹿野郎}{ば|か|や|らう}が
\ruby{御來臨}{お|い|で}なすつた。
ハヽヽ、
\ruby[g]{何程}{いくら }
\ruby{罵}{のヽし}られても
\ruby{相手}{あひ|て}にはならねえ。
\ruby{汝}{きさま}は
\ruby{乃公}{お|れ}に
\ruby{楯}{たて}をついても、
\ruby{乃公}{お|ら}あ
\ruby{汝}{きさま}を
\ruby{生呑}{まる|のみ}に
\ruby{吞}{の}んでゝ、そして
\ruby{腹}{はら}にも
\ruby{障}{さは}らねえから。
』

と、
\ruby{島木}{しま|き}の
\ruby{冷}{ひや}やかに
\ruby{一矢}{いつ|し}
\ruby{酬}{むく}ゆるに、

『
\ruby{何}{なん}だ、
\ruby{吞}{の}んで
\ruby{居}{ゐ}る。
\ruby{可矣}{よ|し}ツ、
\ruby{吞}{の}まれたつて
\ruby{鐵釘}{かな|くぎ}が
\ruby{何}{なん}となる!
\ruby{曲}{まが}りも
\ruby{仕無}{し|な}いは!。
\ruby{丸}{まる}くもならんは!。
』

と、
\ruby[g]{日方}{ひかた}は
\ruby{{\換字{叉}}}{また }
\ruby{直}{ただち}に
\ruby{熱}{ねつ}して
\ruby{答}{こた}ふ。

\ruby{悠然}{いう|ぜん}と
\ruby{笑}{ゑみ}を
\ruby{含}{ふく}める
\ruby[g]{{\GWI{u7fbd-k}\換字{勝}}}{はがち}は
\ruby{靜}{しづ}かに、

『
\ruby{可}{い}いさ、
\ruby[g]{二人}{ふたり}とも、もう
\ruby{可}{い}いさ。
ハヽヽ、
\ruby{互}{たがひ}に
\ruby{其}{そ}の
\ruby{位}{くらゐ}
\ruby{威張}{ ゐ|ば}つたら
\ruby{可}{い}いぢあ
\ruby{無}{な}いか。
\ruby{島木}{しま|き}は
\ruby[g]{日方}{ひかた}に
\ruby{關}{かま}はないで
\ruby{僕}{ぼく}に
\ruby{話}{はな}すつもりで
\ruby{話}{はな}して
\ruby{{\換字{呉}}}{く}れ
\ruby{玉}{たま}へ。
\ruby[g]{日方}{ひかた}はまた
\ruby{島木}{しま|き}に
\ruby{關}{かま}はないで
\ruby{僕}{ぼく}に
\ruby[g]{交際}{つきあ}つて
\ruby{聞}{きい}て
\ruby{居}{い}て
\ruby{{\換字{呉}}}{く}れ
\ruby{玉}{たま}へな。
つまりお
\ruby{互}{たがひ}に
\ruby[g]{水野}{みづの}の
\ruby{上}{うへ}が
\ruby{知}{し}りたいのだからネ。
』

と、
\ruby{優}{やさ}しく
\ruby{制}{せい}すれば、

『ャ、
\ruby{濟}{す}まなかつた、
\ruby{僕}{ぼく}が
\ruby{惡}{わる}かつた。
』

『ァ、
\ruby{左様}{さ|う}
\ruby{云}{い}はれりやあ
\ruby{乃公}{お|れ}も
\ruby{下}{くだ}らなかつた。
』

と
\ruby[g]{日方}{ひかた}も
\ruby{島木}{しま|き}も
\ruby{爭}{あらそ}ひ
\ruby{止}{や}みて、
\ruby{誰}{たれ}
\ruby{勸}{すヽ}めねど
\ruby{同}{おな}じ
\ruby{思}{おも}ひの、
\ruby{双方一時}{さう|はう|いち|じ}に
\ruby[g]{酒盃}{さかづき}を
\ruby{交}{かは}して、
\ruby{笑}{わら}つて
\ruby{仕舞}{し|ま}つて
\ruby{痕跡}{あと|かた}もなし。

\ruby{島木}{しま|き}は
\ruby{此度}{こ|たび}はやゝ
\ruby{眞面目}{ま|じ|め}に、
\ruby[g]{{\GWI{u7fbd-k}\換字{勝}}}{はがち}の
\ruby{方}{かた}に
\ruby{向}{むか}つて
\ruby{語}{かた}り
\ruby{出}{だ}したり。

『
\ruby[g]{一同}{みんな}も
\ruby{知}{し}っている
\ruby{{\GWI{u901a-k}}}{とほ}り
\ruby{彼}{あ}の
\ruby[g]{水野}{みづの}は、
\ruby{我等}{おれ|たち}の
\ruby{中}{なか}では
\ruby{一番年下}{いち|ばん|とし|した}、
\ruby{乃公}{お|れ}が
\ruby{今年}{こ|とし}は
\ruby{二十七}{に|じう|しち}だから、
\ruby{七}{しち}、
\ruby{六}{ろく}、
\ruby{五}{ご}、
\ruby{四}{し}と
\ruby{四}{よつ}つ
\ruby{目}{め}で
\ruby{丁度}{ちやう|ど}
\ruby{二十四}{に|じう|し}だ。
\ruby[g]{宇都宮}{みや}から
\ruby{東京}{とう|きやう}へ
\ruby{上}{のぼ}る
\ruby{時}{とき}にも、
\ruby{一番先}{いち|ばん|さき}へ
\ruby{出}{で}たのは
\ruby[g]{{\GWI{u7fbd-k}\換字{勝}}}{はがち}だつたが、
\ruby{一番後}{いち|ばん|あと}へ
\ruby{殘}{のこ}つたのは
\ruby[g]{水野}{みづの}だつた。
\ruby{若}{わか}いに
\ruby{似合}{に|あ}はず
\ruby{能}{よ}く
\ruby{出來}{で|き}たから、
\ruby{君}{きみ}は
\ruby{若}{わか}いけれども
\ruby{學業}{わ|ざ}が
\ruby{出來}{で|き}る、
\ruby{早}{はや}く
\ruby{東京}{とう|きやう}へ
\ruby{出}{で}て
\ruby{身}{み}を
\ruby{立}{た}てるが
\ruby{可}{い}いと、
\ruby{勸}{すゝ}めたのは
\ruby{乃公}{お|れ}
\ruby{一人}{ひと|り}で
\ruby{無}{な}かつたが、いや
\ruby{小生}{わた|くし}の
\ruby{志}{こヽろざ}すところは
\ruby{些}{ちと}
\ruby{{\GWI{u9055-k}}}{ちが}ふから、
\ruby{左様}{さ|う}
\ruby{急}{いそ}がないでも
\ruby{可}{い}い
\ruby{事}{こと}だ、
\ruby{他}{ほか}の
\ruby{人}{ひと}は
\ruby{一日遲}{いち|にち|おそ}ければ
\ruby{一日損}{いち|にち|そん}、
\ruby{少}{すこ}しも
\ruby{疾}{はや}く
\ruby{上京}{じやう|きやう}するが
\ruby{可}{い}い、と
\ruby{妙}{めう}に
\ruby{片意地}{かた|い|ぢ}に
\ruby{謙遜}{けん|そん}して
\ruby{出}{で}ず。
\ruby{二番}{に|ばん}に
\ruby{出}{で}たが
\ruby[g]{日方}{ひかた}
\ruby{山瀬}{やま|せ}、それから
\ruby{名倉}{な|ぐら}、それから
\ruby{楢井}{なら|い}、それから
\ruby{乃公}{お|れ}で、
\ruby{其後}{その|あと}から
\ruby{漸}{やつ}と
\ruby{上京}{じやう|きやう}した。
\ruby{其}{そ}の
\ruby{位}{くらゐ}\ %空白有り
\ruby{異}{おつ}に
\ruby{固}{かた}いところのある
\ruby{男}{をとこ}で、
\ruby{東京}{とう|きやう}へ
\ruby{出}{で}てからも
\ruby[g]{一同}{みんな}は
\ruby{誰}{たれ}しも、
\ruby{身}{み}を
\ruby{立}{た}てる
\ruby{{\GWI{u9053-k}}}{みち}に
\ruby{汲々}{きふ|〳〵}として、
\ruby{隨分}{ずゐ|ぶん}
\ruby{骨}{ほね}を
\ruby{折}{を}つてそれ〴〵に、
\ruby{辛}{から}く
\ruby{出世}{しゆ|つせ}も
\ruby{仕}{し}て
\ruby{來}{き}たに、
\ruby{彼}{あ}の
\ruby{男}{をとこ}ばかりは
\ruby{澄}{す}ましかへつて、
\ruby{今}{いま}でも
\ruby{小學{\換字{教}}師}{せう|がく|けう|し}で
\ruby{甘}{あま}んじて
\ruby{居}{お}る。
それで
\ruby{惰}{なま}けて
\ruby{居}{を}るのかと
\ruby{思}{おも}へば、
\ruby{一寸}{いつ|すん}の
\ruby{暇}{ひま}も
\ruby{惜}{をし}んで
\ruby{勉強}{べん|きやう}して、あらゆる
\ruby{方面}{はう|めん}に
\ruby{行}{ゆ}き
\ruby{渡}{わた}つて
\ruby{居}{ゐ}る。
\ruby{僕}{ぼく}は
\ruby{一生}{いつ|しやう}をかけて
\ruby{此}{こ}の
\ruby{世}{よ}の
\ruby{中}{なか}に、たゞ
\ruby{一篇}{いつ|ぺん}の
\ruby{詩}{し}を
\ruby{留}{とゞ}めれば
\ruby{可}{い}いのだ。
\ruby{今}{いま}は
\ruby{其}{そ}の
\ruby{準備}{よう|い}に
\ruby{勤}{つと}めて
\ruby{居}{ゐ}るので、
\ruby{他}{ほか}に
\ruby{慾}{よく}も
\ruby{無}{な}ければ
\ruby{望}{のぞみ}も
\ruby{無}{な}い、
\ruby{{\換字{半}}熟}{なま|にえ}なものを
\ruby{世}{よ}に
\ruby{出}{だ}して、
\ruby{今}{いま}っから
\ruby{文人顏}{ぶん|じん|がほ}するのも
\ruby{羞}{はづ}かしいから、もう
\ruby{十年}{じう|ねん}ばかりは
\ruby{小學讀本}{と|く|ほ|ん}いぢりで、たゞ〳〵
\ruby{勉{\換字{強}}}{べん|きやう}をするつもりだ、と
\ruby{隱君子氣質}{いん|くん|し|かた|ぎ}で
\ruby{日}{ひ}を
\ruby{經}{へ}て
\ruby{居}{ゐ}たのは、
\ruby[g]{{\GWI{u7fbd-k}\換字{勝}}}{はがち}はじめ
\ruby[g]{一同}{みんな}も
\ruby{知}{し}つて
\ruby{居}{ゐ}やう。
ところで
\ruby{此}{こ}の
\ruby{乃公}{お|れ}は
\ruby{金}{かね}まうけ
\ruby[g]{主義}{しゆぎ}、
\ruby{卑}{いや}しいと
\ruby{云}{い}つて
\ruby[g]{一同}{みんな}に
\ruby{罵}{のゝし}られた
\ruby{位}{くらゐ}だから、
\ruby{守}{まも}るところのある
\ruby{浪人肌}{らう|にん|はだ}の、
\ruby[g]{水野}{みづの}と
\ruby{氣}{き}の
\ruby{合}{あ}ふ
\ruby{譯}{わけ}は
\ruby{毫}{ちつと}も
\ruby{無}{な}いが、
\ruby{他}{ほか}の
\ruby{五人}{ご|にん}は
\ruby{上京}{じやう|きやう}して、
\ruby[g]{二人}{ふたり}だけ
\ruby{宮}{みや}に
\ruby{殘}{のこ}つた
\ruby{時}{とき}、
\ruby{彼}{あれ}が
\ruby{熱}{ねつ}を
\ruby{病}{や}んだのを
\ruby{介抱}{かい|はう}して、
\ruby{長}{なが}い
\ruby[g]{看護}{みとり}を
\ruby{爲}{し}て
\ruby{{\GWI{u9063-k}}}{や}つた、
\ruby{其事}{そ|れ}が
\ruby{{\GWI{u9396-k}}}{くさり}になつて
\ruby[g]{此地}{こつち}へ
\ruby{來}{き}ても、
\ruby{取}{と}り
\ruby{分}{わ}け
\ruby[g]{二人}{ふたり}は
\ruby{親}{した}しく
\ruby{仕}{し}て
\ruby{居}{ゐ}た。
\GWI{u1b048}かし
\ruby{乃公}{お|ら}あ
\ruby{俗物}{ぞく|ぶつ}、
\ruby[g]{水野}{みづの}は
\ruby{仙骨}{せん|こつ}、
\ruby[g]{此方}{こつち}は
\ruby{飛}{と}んだり
\ruby{跳}{はね}たりして
\ruby{悶躁}{も|が}いて
\ruby{居}{ゐ}るので、
\ruby{中々}{なか|〳〵}
\ruby{往來}{ゆき|き}することも
\ruby{多}{おほ}くは
\ruby{無}{な}かつた。
さあ
\ruby{此處}{こ|こ}で
\ruby{白狀}{はく|じやう}
\ruby{仕}{し}しなけりやならないが、
\ruby{丁度}{ちやう|ど}
\ruby[g]{一昨年}{をとヽし}の
\ruby{暮}{くれ}だつた。
\ruby{實}{じつ}は
\ruby{此}{こ}の
\ruby{乃公}{お|れ}が
\ruby{山氣}{やま|ぎ}に
\ruby{{\GWI{u9038-k}}}{はや}つて、
\ruby{危}{あぶ}ない
\ruby{橋}{はし}を
\ruby{渡}{わた}る
\ruby{輕業}{かる|わざ}をやつたところ、
\ruby{{\GWI{u904b-k}}}{うん}が
\ruby{惡}{わる}くつて
\ruby{可厭}{い|や}な
\ruby{目}{め}が
\ruby{出}{で}て、
\ruby{甘}{うま}く
\ruby{行}{い}きあ
\ruby{論}{ろん}はないことが
\ruby{打壞}{ぶつ|こわ}れたんで、たつた
\ruby{五十}{ご|じう}
\ruby{兩}{りやう}ばかりの
\ruby{有無}{ある|なし}で
\ruby{何様}{ど|う}にも
\ruby{仕切}{し|き}れない
\ruby{機會}{は|め}へ
\ruby{臨}{のぞ}んだ。
そも〳〵
\ruby{投機}{や|ま}を
\ruby{始}{はじ}めた
\ruby{其}{そ}の
\ruby{時}{とき}から、
\ruby{乃公}{お|ら}あ
\ruby{危}{あぶな}い
\ruby{事}{こと}をする
\ruby{代}{かは}りにやあ、
\ruby{乃公}{お|れ}が
\ruby{一六}{いち|ろく}
\ruby{沙汰}{ざ|た}を
\ruby{廢}{や}めぬ
\ruby{内}{うち}は、
\ruby{金錢}{きん|せん}に
\ruby{關}{かヽ}つた
\ruby{事}{こと}では
\ruby{決}{けつ}して
\ruby[g]{一同}{みんな}に、
\ruby{苦勞}{く|らう}は
\ruby{掛}{か}けぬと
\ruby{誓言}{ちか|ひ}を
\ruby{立}{た}つた
\ruby{表}{おもて}があるから
\ruby{誰}{だれ}にも
\ruby{云}{い}へず
\ruby{思案}{し|あん}に
\ruby{餘}{あま}つて
\ruby{獨語}{ひとり|ごと}のやうに、
\ruby{其譯}{その|わけ}を
\ruby[g]{水野}{みづの}に
\ruby{話}{はな}して
\ruby{見}{み}ると、
\ruby{手箱}{て|ばこ}の
\ruby{底}{そこ}から
\ruby{書}{か}いたものを
\ruby{出}{だ}して、
\ruby{此}{これ}を
\ruby{山瀬君}{やま|せ|くん}に
\ruby{頼}{たの}んで
\ruby{賣}{う}つて
\ruby{貰}{もら}つたら、
\ruby{其位}{その|くらゐ}の
\ruby{金}{かね}は
\ruby{出來}{で|き}るか
\ruby{知}{し}れぬ、
\ruby{出來}{で|き}たら
\ruby{使}{つか}ひ
\ruby{玉}{たま}へといふ
\ruby{話}{はなし}。
\ruby{當}{あて}にはならないと
\ruby{思}{おも}つたが、
\ruby{山瀬}{やま|せ}に
\ruby{頼}{たの}むと
\ruby{其事}{そ|れ}が
\ruby{出來}{で|き}て、そこで
\ruby{大}{おほき}に
\ruby{助}{たす}かつた。
\ruby{其}{そ}の
\ruby{味}{あじ}を
\ruby{占}{し}めたといふのでは
\ruby{無}{な}いが、
\ruby{其}{そ}の
\ruby{後}{のち}も
\ruby{種子}{た|ね}を
\ruby{耗}{す}つた
\ruby{其時}{その|とき}は、三
\ruby{度}{ど}といふもの
\ruby{助}{たす}けて
\ruby{貰}{もら}つて、
\ruby{矢種}{や|だね}をつぎ〳〵
\ruby{戦}{たヽか}つた
\ruby{末}{すゑ}、どうやら
\ruby{{\GWI{u9063-k}}}{や}つて
\ruby{行}{い}かれる
\ruby[g]{身體}{からだ}になつた。
そこで
\ruby[g]{水野}{みづの}に
\ruby{對}{むか}つて
\ruby{乃公}{お|れ}がいふには、
\ruby{貰}{もら}つたものを
\ruby{{\GWI{u8fd4-k}}}{かへ}さうとは
\ruby{云}{い}はないが、
\ruby{金}{かね}が
\ruby{要}{い}る
\ruby{時}{とき}は
\ruby{何時}{い|つ}でも
\ruby{云}{い}ひたまへ、
\ruby{乃公}{お|れ}が
\ruby{懷中}{ふと|ころ}だけなら
\ruby{洗}{さら}け
\ruby{出}{だ}すから、と
\ruby{此}{こ}の
\ruby{春}{はる}
\ruby{{\GWI{u9047-k}}}{あ}つた
\ruby{時}{とき}
\ruby{云}{い}つて
\ruby{置}{お}いた。
ところが
\ruby{金}{かね}を
\ruby{使}{つか}ふ
\ruby[g]{水野}{みづの}では
\ruby{無}{な}し、たゞ
\ruby{其限}{それ|ぎり}で
\ruby{濟}{す}んで
\ruby{居}{ゐ}たが、
\ruby{此}{こ}の
\ruby{夏}{なつ}になつて
\ruby{{\GWI{u9063-k}}}{や}つて
\ruby{來}{き}て、
\ruby[g]{眞赤}{まつか}な
\ruby{{\GWI{u984f-j}}}{かほ}をしてきまり
\ruby{惡}{わる}さうに、三十
\ruby{兩}{りやう}ばかり
\ruby{貸}{か}して
\ruby{{\換字{呉}}}{く}れろ、と
\ruby{云}{い}つたのが
\ruby{最初}{はじ|まり}で
\ruby{其後}{その|のち}も、ぼつり〳〵と
\ruby{持}{も}つて
\ruby{行}{ゆ}く。
\ruby{其事}{そ|れ}が
\ruby{乃公}{お|れ}が
\ruby{勘}{かん}を
\ruby{付}{つ}けたはじまりだつた。

