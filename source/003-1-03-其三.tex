\Entry{其三}

% メモ 校正終了 2024-03-28 2024-06-15
\原本頁{16-10}%
\ruby[g]{島木}{しまき }は
\ruby{驕}{おご}れるにもあらず
\ruby{慢}{あなど}れるにもあらず、
%
たゞ
\換字{志}たゝかなる
\原本頁{17-1}\改行%
\ruby{放肆兒}{だゞ|つ|こ}の、
%
\ruby[g]{一家}{いつか }の
\ruby[||j>]{長}{ちやう}
\ruby[||j>]{者}{ じや}をも
% \ruby{長者}{ちやう|じや}をも
はゞからずして、
%
\ruby[g]{自己}{お の }が
\ruby[g]{{\換字{勝}}手}{かつて }に
\ruby{泣}{な}きも
\ruby{笑}{わら}ひもするやうに、
%
\換字{志}かも
\ruby{其}{そ}の
\ruby[g]{小兒}{こ ども}らしき
\ruby{顏}{かほ}に
\ruby[g]{微笑}{ゑ み }を
うかめ
\改行% 校正作業の簡略化のため
て、

\原本頁{17-4}%
『
ハヽヽ、
%
\ruby[g]{日方}{ひ かた}までが
\ruby[||j>]{謹}{きん}
\ruby[||j>]{聽}{ちやう}と
% \ruby{謹聽}{きん|ちやう}と
\ruby{吐}{ぬ}かし
\ruby{居}{を}つたな!。
%
\ruby[g]{一體}{いつたい}
\ruby{汝}{きさま}は
\ruby{人}{ひと}は
\ruby{好}{い}いが、
%
\ruby{我}{が}ばかり
\ruby{{\換字{強}}}{つよ}くつて
\ruby{思}{おも}ひ
\ruby{{\換字{遣}}}{や}りが
\ruby{足}{た}らない。
%
\ruby{此}{こ}の
\ruby{思}{おも}ひ
\ruby{{\換字{遣}}}{や}りの
\ruby{足}{た}らない
\ruby[g]{手合}{て あひ}が、
%
\ruby[g]{他人}{た にん}の
\ruby[g]{戀愛}{れんあい}の
\ruby{談}{はなし}などには、
%
\ruby[g]{兎角}{と かく}に
\ruby[g]{點頭}{がつてん}
しかねるものだ。
%
\ruby{線}{せん}の
\ruby{無}{な}い
\ruby{家}{うち}にやあ
\ruby[g]{電話}{でんわ }は
\ruby{{\換字{通}}}{つう}じない、
%
\ruby{思}{おも}ひ
\ruby{{\換字{遣}}}{や}りの
\ruby{足}{た}らない
\ruby[g]{奴等}{やつら }にやあ
\ruby[g]{戀愛}{れんあい}は
\ruby{解}{げ}せない。
%
そこへ
\ruby{行}{い}つちやあ
\ruby[g]{乃公}{お れ }
なんぞは、
%
\ruby{身}{み}に
\ruby[g]{經驗}{おぼえ }が
あつて
\ruby[||j>]{同}{おも}
\ruby[||j>]{{\換字{情}}}{ひやり}が
% \ruby{同{\換字{情}}}{おも|ひやり}が
\ruby{{\換字{強}}}{つよ}いから、
%
ツーと
\ruby{云}{い}やあ
カーと
\ruby[g]{合點}{が てん}が
いくので、
%
\ruby[g]{初心}{う ぶ }な
\ruby[g]{水野}{みづの }の
\ruby{譚}{はなし}なんざあ、
%
\ruby[g]{何程}{いくら }
\ruby{彼}{あれ}が
\ruby{心}{こゝろ}の
\ruby{奧}{おく}に
\ruby{秘}{かく}して
\ruby{居}{を}つても、
%
\ruby{深}{ふか}い
\ruby{井}{ゐど}の
\ruby{床}{そこ}を
\ruby{鏡}{かゞみ}で
\ruby{照}{て}らして、
%
\ruby{見}{み}て
\ruby{取}{と}るやうに
\原本頁{18-1}%
\ruby{譯}{わけ}も
\ruby{無}{な}く
\ruby[g]{見拔}{み ぬ }く。
%
\ruby[g]{本來}{ほんらい}
\ruby{戀}{こひ}といふ
\ruby{事}{こと}が
\ruby[g]{罪惡}{つ み }ぢやあ
\ruby{有}{あ}るまいし、
%
\ruby[g]{日方}{ひ かた}のやうな
\ruby[g]{暴論}{ばうろん}の
\ruby[g]{愚論}{ぐ ろん}‥‥』

\原本頁{18-3}%
と
\ruby{云}{い}ひかくる
\ruby{時}{とき}
\ruby[g]{日方}{ひ かた}は
\ruby{堪}{こら}へず、

\原本頁{18-4}%
『
\ruby{何}{なん}だ、
%
\ruby[g]{暴論}{ばうろん}だと!。
%
こりやあ
\ruby{怪}{け}しからん。
%
\ruby{汝}{きさま}も
\ruby[g]{戀愛}{れんあい}の
\ruby[g]{奴隷}{ど れい}
\ruby{臭}{くさ}いぞ。
%
\ruby{身}{み}に
\ruby[g]{經驗}{おぼえ }が
あつてとは
\ruby{何}{なん}たる
\ruby[g]{囈語}{ね ごと}だ。
%
\ruby{聞}{き}き
ぐるしいことを
\ruby{吐}{ぬか}さずとも、
%
さつさと
\ruby[g]{水野}{みづの }の
ことを
\ruby{話}{はな}すが
\ruby{可}{よ}い。
』

\原本頁{18-7}%
と
\ruby[g]{怒鳴}{ど な }り
つくれば、
%
\ruby[g]{此方}{こなた }は% ルビ調整(原本通り)
いよ〳〵
\ruby{笑}{わら}い
\ruby{傾}{かたむ}き、

\原本頁{18-8}%
『
\ruby[g]{安心}{あんしん}しろ
\ruby[g]{日方}{ひ かた}!。
%
\ruby[g]{乃公}{お ら }あ
\ruby[g]{戀愛}{れんあい}の
\ruby[g]{奴隷}{ど れい}にやあ
ならねえ。
%
\ruby[g]{乃公}{お ら }あ
\ruby{女}{をんな}に
\ruby{惚}{ほ}れて
\ruby{戀}{こひ}は
おぼえねえ。
%
ヘン
\ruby{惚}{ほ}れられて
\ruby{惚}{ほ}れられて
\ruby{戀}{こひ}といふものは
\ruby[g]{此樣}{こ ん }なものかと
\ruby{知}{し}つたんだからナ。
%
アハヽヽヽ、
%
\ruby[g]{何樣}{ど う }だい
\ruby{奴}{やつこ}さん、
%
\ruby[g]{如何}{い かゞ}でござる!。
%
そこで
\ruby{惚}{ほ}れられて
\ruby{惚}{ほ}れられて
\ruby{悟}{さと}つて
\ruby{見}{み}ると、
%
\原本頁{19-1}%
\ruby[g]{水野}{みづの }を
\ruby[g]{辯護}{べんご }するといふ% 弁 瓣 辦 辧 辨 辩 (辯)
\ruby{譯}{わけ}ぢやあ
\ruby{無}{な}いが、
%
\ruby{戀}{こひ}は
\ruby[g]{人間}{ひ と }
の
\ruby{{\換字{情}}}{じやう}の
\ruby[g]{自然}{し ぜん}の
\ruby[g]{發動}{うごき }で、
%
\ruby{何}{なに}も
\ruby{咎}{とが}め
\ruby{立}{だて}を
することは
\ruby{有}{あ}りやしない。
%
\原本頁{19-3}\改行%
\ruby[g]{日方}{ひ かた}にやあ
\ruby[g]{日方}{ひ かた}だけの
\ruby[g]{愚論}{ぐ ろん}もあらうが、
%
\ruby[g]{乃公}{お ら }あ
\ruby{戀}{こひ}に
\ruby{{\換字{迷}}}{まよ}つた
\ruby{彼}{あ}の
\ruby[g]{水野}{みづの }を、
%
\ruby[||j>]{憫}{かは}
\ruby[||j>]{然}{いさう}だたあ% 「憫然 か(は)いさう」
% \ruby{憫然}{かは|いさう}だたあ% 「憫然 か(は)いさう」
\ruby{思}{おも}ふが
\ruby{惡}{にく}かあ
\ruby{思}{おも}はねえ。
』

\原本頁{19-5}%
と
\ruby{云}{い}はせも
\ruby{果}{は}てず
\ruby[g]{日方}{ひ かた}は
\ruby{目}{め}を
\ruby{剝}{む}き、

\原本頁{19-6}%
『
\ruby{馬鹿野郎}{ば|か|や|らう}ツ。
』

\原本頁{19-7}%
と
\ruby{烈}{はげ}しく
\ruby{罵}{のゝ}しつたる
\ruby[g]{裂帛}{れつぱく}の
\ruby[g]{一聲}{いつせい}に
\ruby[g]{氣合}{き あひ}
\ruby{籠}{こも}つて、
%
\ruby{人}{ひと}の
\ruby[g]{肺腑}{はいふ }に
\ruby{響}{ひゞ}き
\ruby{徹}{てつ}したり。

\原本頁{19-9}%
『
マア
\ruby{待}{ま}ち
\ruby{玉}{たま}へ。
』

\原本頁{19-10}%
『
\ruby{爭}{あらそ}つちや
いかん。
』

\原本頁{19-11}%
と、
%
\ruby{口}{くち}を
\ruby{衝}{つ}いて
\ruby{出}{い}でたる
\ruby[g]{山瀬}{やませ }
\ruby[g]{羽{\換字{勝}}}{は がち}の
\ruby[g]{二人}{に にん}の
\ruby[g]{言葉}{ことば }は
\ruby[g]{一句}{いつく }と
\ruby[g]{一句}{いつく }と
\原本頁{20-1}%
\ruby{斷}{き}るゝ
\ruby{間}{ひま}
\ruby{無}{な}く
\ruby{巧}{たくみ}に
\ruby{續}{つゞ}きて、
%
\ruby[g]{突差}{とつさ }に
\ruby{緊}{きび}しく
\ruby{制}{せい}し
\ruby{止}{と}むれば、
%
\ruby[g]{流石}{さすが }に
\ruby[g]{日方}{ひ かた}も
\ruby[g]{羽{\換字{勝}}}{は がち}を
\ruby{憚}{はゞか}りて、% 「憚 は(ゞ)か」
%
\ruby{言}{ものい}はんとして
\ruby{言}{い}はず
\ruby{已}{や}みけるが、
%
\ruby{眼}{め}には
\ruby{{\換字{猶}}}{なほ}
\ruby[g]{稜角}{か ど }を
\ruby{立}{た}てゝ
\ruby[g]{島木}{しまき }を
\ruby{睨}{にら}み、
%
\ruby{此}{こ}の
\ruby{時}{とき}
\ruby{遲}{おそ}く
\ruby{彼}{か}の
\ruby{時}{とき}
\ruby{{\換字{速}}}{はや}く、

\原本頁{20-4}%
『
そら
\ruby{{\換字{又}}}{また}
\ruby{馬鹿野郎}{ば|か|や|らう}が
\ruby{御來臨}{お|い|で}なすつた。
%
ハヽヽ、
%
\ruby[g]{何程}{いくら }
\ruby[<j||]{罵}{のゝし}られても% ルビ調整(原本通り)
\ruby[g]{相手}{あひて }には
ならねえ。
%
\ruby{汝}{きさま}は
\ruby[g]{乃公}{お れ }に
\ruby{楯}{たて}をついても、
%
\ruby[g]{乃公}{お ら }あ
\ruby{汝}{きさま}を
\ruby[g]{生呑}{まるのみ}に
\ruby{吞}{の}んでゝ、
%
そして
\ruby{腹}{はら}にも
\ruby{障}{さは}らねえから。
』

\原本頁{20-7}%
と、
%
\ruby[g]{島木}{しまき }の
\ruby{冷}{ひや}やかに
\ruby[g]{一矢}{いつし }
\ruby{酬}{むく}ゆるに、

\原本頁{20-8}%
『
\ruby{何}{なん}だ、
%
\ruby{吞}{の}んで
\ruby{居}{ゐ}る。
%
\ruby[g]{可矣}{よ し }ツ、
%
\ruby{吞}{の}まれたつて
\ruby[g]{鐵釘}{かなくぎ}が
\ruby{何}{なん}となる!。
%
\ruby{曲}{まが}りも
\ruby[g]{仕無}{し な }いは!、
%
\ruby{丸}{まる}くも
ならんは!。
』

\原本頁{20-10}%
と、
%
\ruby[g]{日方}{ひ かた}は
\ruby{{\換字{又}}}{また}
\ruby[||j>]{直}{たゞち}に
\ruby{熱}{ねつ}して
\ruby{答}{こた}ふ。

\原本頁{20-11}%
\ruby[g]{悠然}{いうぜん}と
\ruby{笑}{ゑみ}を
\ruby{含}{ふく}める
\ruby[g]{羽{\換字{勝}}}{は がち}は
\ruby{靜}{しづ}かに、

\原本頁{21-1}%
『
\ruby{可}{い}いさ、
%
\ruby[g]{二人}{ふたり }とも、
%
もう
\ruby{可}{い}いさ。
%
ハヽヽ、
%
\ruby{互}{たがひ}に
\ruby{其}{そ}の
\ruby[||j>]{位}{くらゐ}
\ruby[g]{威張}{ゐ ば }つたら% ルビ調整(原本通り)
\ruby{可}{い}いぢあ
\ruby{無}{な}いか。
%
\ruby[g]{島木}{しまき }は
\ruby[g]{日方}{ひ かた}に
\ruby{關}{かま}はないで
\ruby{僕}{ぼく}に
\ruby{話}{はな}すつもりで
\ruby{話}{はな}して
\ruby{吳}{く}れ
\ruby{玉}{たま}へ。
%
\ruby[g]{日方}{ひ かた}は
また
\ruby[g]{島木}{しまき }に
\ruby{關}{かま}はないで
\ruby{僕}{ぼく}に
\ruby[g]{{\換字{交}}際}{つきあ }つて
\ruby{聞}{きい}て
\ruby{居}{ゐ}て
\ruby{吳}{く}れ
\ruby{玉}{たま}へな。
%
つまり
お
\ruby{互}{たがひ}に
\ruby[g]{水野}{みづの }の
\ruby{上}{うへ}が
\ruby{知}{し}りたいのだからネ。
』

\原本頁{21-6}%
と、
%
\ruby{優}{やさ}しく
\ruby{制}{せい}すれば、

\原本頁{21-7}%
『
ヤ、% 小書きにすると右寄りになるので
%
\ruby{濟}{す}まなかつた、
%
\ruby{僕}{ぼく}が
\ruby{惡}{わる}かつた。
』

\原本頁{21-8}%
『
ア、% 小書きにすると右寄りになるので
%
\ruby[g]{左樣}{さ う }
\ruby{云}{い}はれりやあ
\ruby[g]{乃公}{お れ }も
\ruby{下}{くだ}らなかつた。
』

\原本頁{21-9}%
と、
\ruby[g]{日方}{ひ かた}も
\ruby[g]{島木}{しまき }も
\ruby{爭}{あらそ}ひ
\ruby{止}{や}みて、
%
\ruby{誰}{たれ}
\ruby{勸}{すゝ}めねど
\ruby{同}{おな}じ
\ruby{思}{おも}ひの、
%
\ruby[g]{双方}{さうはう}
\ruby[g]{一時}{いちじ }に
\ruby[g]{酒盃}{さかづき}を
\ruby{{\換字{交}}}{かは}して、
%
\ruby{笑}{わら}つて
\ruby[g]{仕舞}{し ま }つて
\ruby[g]{痕跡}{あとかた}もなし。

\原本頁{21-11}%
\ruby[g]{島木}{しまき }は
\ruby[g]{此度}{こ たび}は
やゝ
\ruby{眞面目}{ま|じ|め}に、
%
\ruby[g]{羽{\換字{勝}}}{は がち}の
\ruby{方}{かた}に
\ruby{向}{むか}つて
\ruby{語}{かた}り
\ruby{出}{だ}したり。

\原本頁{22-1}%
『% この島木の語りは其五の途中まで続く
\ruby[g]{一同}{みんな }も
\ruby{知}{し}つてる
\ruby{{\換字{通}}}{とほ}り
\ruby{彼}{あ}の
\ruby[g]{水野}{みづの }は、
%
\ruby[g]{我等}{おれたち}の
\ruby{中}{なか}では
\ruby[g]{一番}{いちばん}
\ruby[g]{年下}{としした}、
%
\ruby[g]{乃公}{お れ }が
\ruby[g]{今年}{こ とし}は
\ruby{二十七}{に|じふ|しち}だから、% 原本には漢数字「七」のルビ無し
%
\ruby{七}{しち}、% 原本には漢数字「七」のルビ無し
%
\ruby{六}{ろく}、
%
\ruby{五}{ご}、
%
\ruby{四}{よん}と
\ruby{四}{よ}つ
\ruby{目}{め}で
\ruby[g]{丁度}{ちやうど}
\ruby{二十四}{に|じふ|し}だ。% 国会図書館 コマ番号 15/134 p22 l3
%
\ruby{宇都宮}{み| |や}から
\ruby[||j>]{東}{とう}
\ruby[||j>]{京}{きやう}へ
% \ruby{東京}{とう|きやう}へ
\ruby{上}{のぼ}る
\ruby{時}{とき}にも、
%
\ruby[g]{一番}{いちばん}
\ruby{先}{さき}へ
\ruby{出}{で}たのは
\ruby[g]{羽{\換字{勝}}}{は がち}だつたが、
%
\ruby[g]{一番}{いちばん}
\ruby{後}{あと}へ
\ruby{殘}{のこ}つたのは
\ruby[g]{水野}{みづの }だつた。
%
\ruby{{\換字{若}}}{わか}いに
\ruby[g]{似合}{に あ }はず
\ruby{能}{よ}く
\ruby[g]{出來}{で き }たから、
%
\ruby{君}{きみ}は
\ruby{{\換字{若}}}{わか}いけれども
\ruby[g]{學業}{わ ざ }が
\ruby[g]{出來}{で き }る、
%
\ruby{早}{はや}く
\ruby[||j>]{東}{とう}
\ruby[||j>]{京}{きやう}へ
% \ruby{東京}{とう|きやう}へ
\ruby{出}{で}て
\ruby{身}{み}を
\ruby{立}{た}てるが
\ruby{可}{い}いと、
%
\ruby{勸}{すゝ}めたのは
\ruby[g]{乃公}{お れ }
\ruby[g]{一人}{ひとり }で
\ruby{無}{な}かつたが、
%
いや
\ruby[g]{小生}{わたくし}の
\ruby[<j>]{志}{こゝろざ}す
ところは
\ruby{些}{ちと}
\ruby{{\換字{違}}}{ちが}ふから、
%
\ruby[g]{左樣}{さ う }
\ruby{急}{いそ}がないでも
\ruby{可}{い}い
\ruby{事}{こと}だ、
%
\原本頁{22-8}\改行%
\ruby{他}{ほか}の
\ruby{人}{ひと}は
\ruby[g]{一日}{いちにち}
\ruby{遲}{おそ}ければ
\ruby[g]{一日}{いちにち}
\ruby{損}{そん}、
%
\ruby{少}{すこ}しも
\ruby{疾}{はや}く
\ruby[g]{上京}{じやうきやう}
するが
\ruby{可}{い}い、
%
と
\原本頁{22-9}\改行%
\ruby{妙}{めう}に
\ruby{片意地}{かた|い|ぢ}に
\ruby[g]{{\換字{謙}}遜}{けんそん}して
\ruby{出}{で}ず。
%
\ruby[g]{二番}{に ばん}に
\ruby{出}{で}たが
\ruby[g]{日方}{ひ かた}
\ruby[g]{山瀬}{やませ }、
%
それから
\原本頁{22-10}\改行%
\ruby[g]{名倉}{な ぐら}、
%
それから
\ruby[g]{楢井}{ならい }、
%
それから
\ruby[g]{乃公}{お れ }で、
%
\ruby{其}{その}
\ruby{後}{あと}から
\ruby{漸}{やつ}と
\ruby[g]{上京}{じやうきやう}
した
\改行% 校正作業の簡略化のため
。
%
\原本頁{22-11}\改行%
\ruby{其}{そ}の
\ruby[||j>]{位}{くらゐ}
\ruby{異}{ おつ}に
\ruby{固}{かた}いところのある
\ruby{男}{をとこ}で、
%
\ruby[||j>]{東}{とう}
\ruby[||j>]{京}{きやう}へ
% \ruby{東京}{とう|きやう}へ
\ruby{出}{で}てからも
\ruby[g]{一同}{みんな }は
\ruby{誰}{たれ}
\原本頁{23-1}\改行%
しも、
%
\ruby{身}{み}を
\ruby{立}{た}てる
\ruby{{\換字{道}}}{みち}に
\ruby[g]{汲々}{きふ〳〵}として、
%
\ruby[g]{隨{\換字{分}}}{ずゐぶん}
\ruby{骨}{ほね}を
\ruby{折}{を}つて
それ〴〵に
\改行% 校正作業の簡略化のため
、
%
\原本頁{23-2}\改行%
\ruby{辛}{から}く
\ruby[g]{出世}{しゆつせ}も
\ruby{仕}{し}て
\ruby{來}{き}たに、
%
\ruby{彼}{あ}の
\ruby{男}{をとこ}ばかりは
\ruby{澄}{す}ましかへつて、
%
\ruby{今}{いま}でも
\ruby[g]{小學}{せうがく}
\ruby[g]{敎師}{けうし }で
\ruby{甘}{あま}んじて
\ruby{居}{を}る。
%
それで
\ruby{惰}{なま}けて
\ruby{居}{を}るのかと
\ruby{思}{おも}へば、
%
\原本頁{23-4}\改行%
\ruby[g]{一寸}{いつすん}の
\ruby{暇}{ひま}も
\ruby{惜}{をし}んで
\ruby[||j>]{勉}{べん}
\ruby[||j>]{{\換字{強}}}{きやう}
% \ruby{勉{\換字{強}}}{べん|きやう}
して、
%
あらゆる
\ruby[g]{方面}{はうめん}に
\ruby{行}{ゆ}き
\ruby{渡}{わた}つて
\ruby{居}{ゐ}る。
%
\原本頁{23-5}\改行%
\ruby{僕}{ぼく}は
\ruby[||j>]{一}{いつ}
\ruby[||j>]{生}{しやう}を
% \ruby{一生}{いつ|しやう}を
かけて
\ruby{此}{こ}の
\ruby{世}{よ}の
\ruby{中}{なか}に、
%
たゞ
\ruby[g]{一篇}{いつぺん}の
\ruby{詩}{し}を
\ruby{{\換字{留}}}{とゞ}めれば
\ruby{可}{い}いのだ。
%
\ruby{今}{いま}は
\ruby{其}{そ}の
\ruby[g]{準備}{ようい }に
\ruby{{\換字{勤}}}{つと}めて
\ruby{居}{ゐ}るので、
%
\ruby{他}{ほか}に
\ruby{慾}{よく}も
\ruby{無}{な}ければ
\ruby{望}{のぞみ}も
\原本頁{23-8}\改行%
\ruby{無}{な}い、
%
\ruby[g]{{\換字{半}}熟}{なまにえ}なものを
\ruby{世}{よ}に
\ruby{出}{だ}して、
%
\ruby{今}{いま}つから
\ruby[g]{{\換字{文}}人}{ぶんじん}
\ruby{顏}{がほ}するのも
\ruby{羞}{はづ}か
しいから、
%
もう
\ruby[g]{十年}{じふねん}ばかりは
\ruby{小學讀本}{と|く|ほ|ん}
いぢりで、
%
たゞ〳〵
\makeatletter
\@ifundefined{デバッグ@ビルド}{%
  \ruby[||j>]{勉}{べん }
  \ruby[||j>]{{\換字{強}}}{きやう}
}{%
  \ruby[<j||]{勉}{べん }
  \ruby[<j||]{{\換字{強}}}{きやう}
}%
\makeatother
% \ruby{勉{\換字{強}}}{べん|きやう}
\原本頁{23-9}\改行%
を
するつもりだ、
%
と
\ruby{隱君子}{いん|くん|し}
\ruby[g]{氣質}{かたぎ }で
\ruby{日}{ひ}を
\ruby{經}{へ}て
\ruby{居}{ゐ}たのは、
%
\ruby[g]{羽{\換字{勝}}}{は がち}はじめ
\ruby[g]{一同}{みんな }も
\ruby{知}{し}つて
\ruby{居}{ゐ}やう。
%
ところで
\ruby{此}{こ}の
\ruby[g]{乃公}{お れ }は
\ruby{金}{かね}まうけ
\ruby[g]{主義}{しゆぎ }、
%
\ruby{卑}{いや}しいと
\ruby{云}{い}つて
\ruby[g]{一同}{みんな }に
\ruby{罵}{のゝし}られた
\ruby{位}{くらゐ}だから、
%
\ruby{守}{まも}るところのある
\ruby[g]{浪人}{らうにん}
\原本頁{24-1}\改行%
\ruby{肌}{はだ}の、
%
\ruby[g]{水野}{みづの }と
\ruby{氣}{き}の
\ruby{合}{あ}ふ
\ruby{譯}{わけ}は
\ruby{毫}{ちつと}も
\ruby{無}{な}いが、
%
\ruby{他}{ほか}の
\ruby[g]{五人}{ご にん}は
\ruby[g]{上京}{じやうきやう}
して、
%
\原本頁{24-2}\改行%
\ruby[g]{二人}{ふたり }だけ
\ruby{宮}{みや}に
\ruby{殘}{のこ}つた
\ruby{時}{とき}、
%
\ruby{彼}{あれ}が
\ruby{熱}{ねつ}を
\ruby{病}{や}んだのを
\ruby[g]{介抱}{かいはう}して、
%
\ruby{長}{なが}い
\ruby[g]{看護}{み とり}を
\ruby{爲}{し}て
\ruby{{\換字{遣}}}{や}つた、
%
\ruby[g]{其事}{そ れ }が
\ruby{{\換字{鎖}}}{くさり}になつて
\ruby[g]{此地}{こつち }へ
\ruby{來}{き}ても、
%
\ruby{取}{と}り
\ruby{{\換字{分}}}{わ}け
\ruby[g]{二人}{ふたり }は
\ruby{親}{した}しく
\ruby{仕}{し}て
\ruby{居}{ゐ}た。
%
\換字{志}かし
\ruby[g]{乃公}{お ら }あ
\ruby[g]{俗物}{ぞくぶつ}、
%
\ruby[g]{水野}{みづの }は
\ruby[g]{仙骨}{せんこつ}、
%
\ruby[g]{此方}{こつち }は% ルビ調整(原本通り)
\原本頁{24-5}\改行%
\ruby{飛}{と}んだり
\ruby{跳}{はね}たりして
\ruby[g]{悶躁}{も が }いて
\ruby{居}{ゐ}るので、
%
\ruby[g]{中々}{なか〳〵}
\ruby[g]{往來}{ゆきき }することも
\ruby{多}{おほ}くは
\ruby{無}{な}かつた。
%
さあ
\ruby[g]{此處}{こ ゝ }で
\ruby[||j>]{白}{はく}
\ruby[||j>]{狀}{じやう}
% \ruby{白狀}{はく|じやう}
\ruby[||j>]{仕}{ し}なけりや
ならないが、
%
\ruby[g]{丁度}{ちやうど}
\makeatletter
\@ifundefined{デバッグ@ビルド}{%
  \ruby[|g|]{一昨年}{をとゝし}の
}{%
  \ruby{一昨年}{を|とゝ|し}の
}%
\makeatother
\ruby{暮}{くれ}だつた。
%
\ruby{實}{じつ}は
\ruby{此}{こ}の
\ruby[g]{乃公}{お れ }が
\ruby[g]{山氣}{やまぎ }に
\ruby{逸}{はや}つて、
%
\ruby{危}{あぶな}い
\ruby{橋}{はし}を
\ruby{渡}{わた}る
\原本頁{24-8}\改行%
\ruby[g]{輕業}{かるわざ}をやつたところ、
%
\ruby{{\換字{運}}}{うん}が
\ruby{惡}{わる}くつて
\ruby[g]{可厭}{い や }な
\ruby{目}{め}が
\ruby{出}{で}て、
%
\ruby{甘}{うま}く
\ruby{行}{い}きあ
\ruby{論}{ろん}は
\ruby{無}{な}いことが
\ruby[g]{打壞}{ぶつこわ}れたんで、
%
たつた
\ruby[g]{五十}{ご じふ}
\ruby{兩}{りやう}ばかりの
\ruby[g]{有無}{あるなし}で
\原本頁{24-10}\改行%
\ruby[g]{何樣}{ど う }にも
\ruby[g]{仕切}{し き }れない
\ruby[g]{機會}{は め }へ
\ruby{臨}{のぞ}んだ。
%
そも〳〵
\ruby[g]{投機}{や ま }を
\ruby{始}{はじ}めた
\ruby{其}{そ}の
\原本頁{24-11}\改行%
\ruby{時}{とき}から、
%
\ruby[g]{乃公}{お ら }あ
\ruby{危}{あぶな}い
\ruby{事}{こと}をする
\ruby{代}{かは}りにやあ、
%
\ruby[g]{乃公}{お れ }が
\ruby[g]{一六}{いちろく}% ばくち・双六 (すごろく) で二つの賽 (さい) を振って、その目に一と六とが同時に出ること
\ruby[g]{沙汰}{ざ た }を
\ruby{廢}{や}めぬ
\ruby{内}{うち}は、
%
\原本頁{25-1}%
\ruby[g]{金錢}{きんせん}に
\ruby{關}{かゝ}つた
\ruby{事}{こと}では
\ruby{決}{けつ}して
\ruby[g]{一同}{みんな }に、
%
\ruby[g]{苦勞}{く らう}は
\ruby{掛}{か}けぬと
\原本頁{25-2}\改行%
\ruby[g]{誓言}{ちかひ }を
\ruby{立}{た}つた
\ruby{表}{おもて}が
あるから
\ruby{誰}{たれ}にも
\ruby{云}{い}へず、
%
\ruby[g]{思案}{し あん}に
\ruby{餘}{あま}つて
\ruby[||j>]{獨}{ひとり}
\ruby[||j>]{語}{ ごと}のやうに、
%
\ruby{其}{その}
\ruby{譯}{わけ}を
\ruby[g]{水野}{みづの }に
\ruby{話}{はな}して
\ruby{見}{み}ると、
%
\ruby[g]{手箱}{て ばこ}の
\ruby{底}{そこ}から
\ruby{書}{か}いたものを
\ruby{出}{だ}して、
%
\ruby{此}{これ}を
\ruby[g]{山瀬}{やませ }
\ruby{君}{くん}に
\ruby{頼}{たの}んで
\ruby{賣}{う}つて
\ruby{貰}{もら}つたら、
%
\ruby[||j>]{其}{その}
\ruby[||j>]{位}{くらゐ}の
% \ruby{其位}{その|くらゐ}の
\ruby{金}{かね}は
\ruby[g]{出來}{で き }るか
\ruby{知}{し}れぬ、
%
\ruby[g]{出來}{で き }たら
\ruby{使}{つか}ひ
\ruby{玉}{たま}へ
といふ
\ruby{話}{はなし}。
%
\ruby{當}{あて}には
ならないと
\原本頁{25-6}\改行%
\ruby{思}{おも}つたが、
%
\ruby[g]{山瀬}{やませ }に
\ruby{頼}{たの}むと
\ruby[g]{其事}{そ れ }が
\ruby[g]{出來}{で き }て、
%
そこで
\ruby{大}{おほき}に
\ruby{助}{たす}かつた。
%
\原本頁{25-7}\改行%
\ruby{其}{そ}の
\ruby{味}{あぢ}を
\ruby{占}{し}めた
といふのでは
\ruby{無}{な}いが、
%
\ruby{其}{そ}の
\ruby{後}{のち}も
\ruby[g]{種子}{た ね }を
\ruby{耗}{す}つた
\ruby{其}{その}
\原本頁{25-8}\改行%
\ruby{時}{とき}は、
%
\ruby[g]{三度}{さんど }といふもの
\ruby{助}{たす}けて
\ruby{貰}{もら}つて、
%
\ruby[g]{矢種}{や だね}を
つぎ〳〵
\ruby{戰}{たゝか}つた
\ruby{末}{すゑ}
\改行% 校正作業の簡略化のため
、
%
\原本頁{25-9}\改行%
どうやら
\ruby{{\換字{遣}}}{や}つて
\ruby{行}{い}かれる
\ruby[g]{身體}{からだ }になつた。
%
そこで
\ruby[g]{水野}{みづの }に
\ruby{對}{むか}つて
\ruby[g]{乃公}{お れ }が
いふには、
%
\ruby{貰}{もら}つたものを
\ruby{{\換字{返}}}{かへ}さうとは
\ruby{云}{い}はないが、
%
\ruby{金}{かね}が
\ruby{要}{い}る
\原本頁{25-11}\改行%
\ruby{時}{とき}は
\ruby[g]{何時}{い つ }でも
\ruby{云}{い}ひたまへ、
%
\ruby[g]{乃公}{お れ }が
\ruby[g]{懷中}{ふところ}だけなら
\ruby{洗}{さら}け
\ruby{出}{だ}すから、
%
\原本頁{26-1}\改行%
と
\ruby{此}{こ}の
\ruby{春}{はる}
\ruby{{\換字{遇}}}{あ}つた
\ruby{時}{とき}
\ruby{云}{い}つて
\ruby{置}{お}いた。
%
ところが
\ruby{金}{かね}を
\ruby{使}{つか}ふ
\ruby[g]{水野}{みづの }では
\ruby{無}{な}し、
%
たゞ
\ruby[g]{其限}{それぎり}で
\ruby{濟}{す}んで
\ruby{居}{ゐ}たが、
%
\ruby{此}{こ}の
\ruby{夏}{なつ}になつて
\ruby{{\換字{遣}}}{や}つて
\ruby{來}{き}て、
%
\ruby[g]{眞赤}{まつか }な
\ruby{顏}{かほ}をして
きまり
\ruby{惡}{わる}さうに、
%
\ruby[g]{三十}{さんじふ}
\ruby{兩}{りやう}
ばかり
\ruby{貸}{か}して
\ruby{吳}{く}れろ、
%
と
\原本頁{26-4}\改行%
\ruby{云}{い}つたのが
\ruby[g]{最初}{はじまり}で
\ruby{其}{その}
\ruby{後}{のち}も、% ルビ調整(原本通り)非踊り字表記
%
ぼつり〳〵と
\ruby{持}{も}つて
\ruby{行}{ゆ}く。
%
\ruby[g]{其事}{そ れ }が
\ruby[g]{乃公}{お れ }が
\ruby{勘}{かん}を
\ruby{付}{つ}けた
はじまりだつた。
