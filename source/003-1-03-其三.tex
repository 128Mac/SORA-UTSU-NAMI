\Entry{其三}

% メモ 校正終了 2024-03-28
\原本頁{16-10}%
\ruby{島木}{しま|き}は
\ruby{驕}{おご}れるにもあらず
\ruby{慢}{あなど}れるにもあらず、
%
たゞ
\換字{志}たゝかなる
\原本頁{17-1}\改行%
\ruby{放肆兒}{だゞ|つ|こ}の、
%
\ruby{一家}{いつ|か}の
\ruby[||j>]{長}{ちやう}
\ruby[||j>]{者}{ じや}をも
% \ruby{長者}{ちやう|じや}をも
はゞからずして、
%
\ruby{自己}{お|の}が
\ruby{{\換字{勝}}手}{かつ|て}に
\ruby{泣}{な}きも
\ruby{笑}{わら}ひもするやうに、
%
\換字{志}かも
\ruby{其}{そ}の
\ruby{小兒}{こ|ども}らしき
\ruby{顏}{かほ}に
\ruby{微笑}{ゑ|み}を
うかめて、

\原本頁{17-4}%
『ハヽヽ、
%
\ruby{日方}{ひ|かた}までが
\ruby[||j>]{謹}{きん}
\ruby[||j>]{聽}{ちやう}と
% \ruby{謹聽}{きん|ちやう}と
\ruby{吐}{ぬ}かし
\ruby{居}{を}つたな!。
%
\ruby{一體}{いつ|たい}
\ruby{汝}{きさま}は
\ruby{人}{ひと}は
\ruby{好}{い}いが、
%
\ruby{我}{が}ばかり
\ruby{{\換字{強}}}{つよ}くつて
\ruby{思}{おも}ひ
\ruby{{\換字{遣}}}{や}りが
\ruby{足}{た}らない。
%
\ruby{此}{こ}の
\ruby{思}{おも}ひ
\ruby{{\換字{遣}}}{や}りの
\ruby{足}{た}らない
\ruby{手合}{て|あひ}が、
%
\ruby{他人}{た|にん}の
\ruby{戀愛}{れん|あい}の
\ruby{談}{はなし}などには、
%
\ruby{兎角}{と|かく}に
\ruby{點頭}{がつ|てん}
しかねるものだ。
%
\ruby{線}{せん}の
\ruby{無}{な}い
\ruby{家}{うち}にやあ
\ruby{電話}{でん|わ}は
\ruby{{\換字{通}}}{つう}じない、
%
\ruby{思}{おも}ひ
\ruby{{\換字{遣}}}{や}りの
\ruby{足}{た}らない
\ruby{奴等}{やつ|ら}にやあ
\ruby{戀愛}{れん|あい}は
\ruby{解}{げ}せない。
%
そこへ
\ruby{行}{い}つちやあ
\ruby{乃公}{お|れ}
なんぞは、
%
\ruby{身}{み}に
\ruby{經驗}{おぼ|え}が
あつて
\ruby[||j>]{同}{おも}
\ruby[||j>]{{\換字{情}}}{ひやり}が
% \ruby{同{\換字{情}}}{おも|ひやり}が
\ruby{{\換字{強}}}{つよ}いから、
%
ツーと
\ruby{云}{い}やあ
カーと
\ruby{合點}{が|てん}が
いくので、
%
\ruby{初心}{う|ぶ}な
\ruby{水野}{みづ|の}の
\ruby{譚}{はなし}なんざあ、
%
\ruby{何程}{いく|ら}
\ruby{彼}{あれ}が
\ruby{心}{こゝろ}の
\ruby{奧}{おく}に
\ruby{秘}{かく}して
\ruby{居}{を}つても、
%
\ruby{深}{ふか}い
\ruby{井}{ゐど}の
\ruby{床}{そこ}を
\ruby{鏡}{かゞみ}で
\ruby{照}{て}らして、
%
\ruby{見}{み}て
\ruby{取}{と}るやうに
\原本頁{18-1}%
\ruby{譯}{わけ}も
\ruby{無}{な}く
\ruby{見拔}{み|ぬ}く。
%
\ruby{本來}{ほん|らい}
\ruby{戀}{こひ}といふ
\ruby{事}{こと}が
\ruby{罪惡}{つ|み}ぢやあ
\ruby{有}{あ}るまいし、
%
\ruby{日方}{ひ|かた}のやうな
\ruby{暴論}{ばう|ろん}の
\ruby{愚論}{ぐ|ろん}‥‥』

\原本頁{18-3}%
と
\ruby{云}{い}ひかくる
\ruby{時}{とき}
\ruby{日方}{ひ|かた}は
\ruby{堪}{こら}へず、

\原本頁{18-4}%
『
\ruby{何}{なん}だ、
%
\ruby{暴論}{ばう|ろん}だと!。
%
こりやあ
\ruby{怪}{け}しからん。
%
\ruby{汝}{きさま}も
\ruby{戀愛}{れん|あい}の
\ruby{奴隷}{ど|れい}
\ruby{臭}{くさ}いぞ。
%
\ruby{身}{み}に
\ruby{經驗}{おぼ|え}が
あつてとは
\ruby{何}{なん}たる
\ruby{囈語}{ね|ごと}だ。
%
\ruby{聞}{き}き
ぐるしいことを
\ruby{吐}{ぬか}さずとも、
%
さつさと
\ruby{水野}{みづ|の}の
ことを
\ruby{話}{はな}すが
\ruby{可}{よ}い。
』

\原本頁{18-7}%
と
\ruby{怒鳴}{ど|な}り
つくれば、
%
\ruby{此方}{こな|た}は
いよ〳〵
\ruby{笑}{わら}い
\ruby{傾}{かたむ}き、

\原本頁{18-8}%
『
\ruby{安心}{あん|しん}しろ
\ruby{日方}{ひ|かた}!。
%
\ruby{乃公}{お|ら}あ
\ruby{戀愛}{れん|あい}の
\ruby{奴隷}{ど|れい}にやあ
ならねえ。
%
\ruby{乃公}{お|ら}あ
\ruby{女}{をんな}に
\ruby{惚}{ほ}れて
\ruby{戀}{こひ}は
おぼえねえ。
%
ヘン
\ruby{惚}{ほ}れられて
\ruby{惚}{ほ}れられて
\ruby{戀}{こひ}といふものは
\ruby{此樣}{こ|ん}なものかと
\ruby{知}{し}つたんだからナ。
%
アハヽヽヽ、
%
\ruby{何樣}{ど|う}だい
\ruby{奴}{やつこ}さん、
%
\ruby{如何}{い|かゞ}でござる!。
%
そこで
\ruby{惚}{ほ}れられて
\ruby{惚}{ほ}れられて
\ruby{悟}{さと}つて
\ruby{見}{み}ると、
%
\原本頁{19-1}%
\ruby{水野}{みづ|の}を
\ruby{辯護}{べん|ご}するといふ% 弁 瓣 辦 辧 辨 辩 (辯)
\ruby{譯}{わけ}ぢやあ
\ruby{無}{な}いが、
%
\ruby{戀}{こひ}は
\ruby{人間}{ひ|と}の
\ruby{{\換字{情}}}{じやう}の
\ruby{自然}{し|ぜん}の
\ruby{發動}{うご|き}で、
%
\ruby{何}{なに}も
\ruby{咎}{とが}め
\ruby{立}{だ}てを
することは
\ruby{有}{あ}りやしない。
%
\原本頁{19-3}\改行%
\ruby{日方}{ひ|かた}にやあ
\ruby{日方}{ひ|かた}だけの
\ruby{愚論}{ぐ|ろん}もあらうが、
%
\ruby{乃公}{お|ら}あ
\ruby{戀}{こひ}に
\ruby{{\換字{迷}}}{まよ}つた
\ruby{彼}{あ}の
\ruby{水野}{みづ|の}を、
%
\ruby[||j>]{憫}{かは}
\ruby[||j>]{然}{いさう}だたあ% 「憫然 か(は)いさう」
% \ruby{憫然}{かは|いさう}だたあ% 「憫然 か(は)いさう」
\ruby{思}{おも}ふが
\ruby{惡}{にく}かあ
\ruby{思}{おも}はねえ。
』

\原本頁{19-5}%
と
\ruby{云}{い}はせも
\ruby{果}{は}てず
\ruby{日方}{ひ|かた}は
\ruby{目}{め}を
\ruby{剝}{む}き、

\原本頁{19-6}%
『
\ruby{馬鹿野郎}{ば|か|や|らう}ツ。
』

\原本頁{19-7}%
と
\ruby{烈}{はげ}しく
\ruby{罵}{のゝ}しつたる
\ruby{裂帛}{れつ|ぱく}の
\ruby{一聲}{いつ|せい}に
\ruby{氣合}{き|あひ}
\ruby{籠}{こも}つて、
%
\ruby{人}{ひと}の
\ruby{肺腑}{はい|ふ}に
\ruby{響}{ひゞ}き
\ruby{徹}{てつ}したり。

\原本頁{19-9}%
『マア
\ruby{待}{ま}ち
\ruby{玉}{たま}へ。
』

\原本頁{19-10}%
『
\ruby{爭}{あらそ}つちや
いかん。
』

\原本頁{19-11}%
と、
%
\ruby{口}{くち}を
\ruby{衝}{つ}いて
\ruby{出}{い}でたる
\ruby{山瀬}{やま|せ}
\ruby{羽{\換字{勝}}}{は|がち}の
\ruby{二人}{に|にん}の
\ruby{言葉}{こと|ば}は
\ruby{一句}{いつ|く}と
\ruby{一句}{いつ|く}と
\原本頁{20-1}%
\ruby{斷}{き}るゝ
\ruby{間}{ひま}
\ruby{無}{な}く
\ruby{巧}{たくみ}に
\ruby{續}{つゞ}きて、
%
\ruby{突差}{とつ|さ}に
\ruby{緊}{きび}しく
\ruby{制}{せい}し
\ruby{止}{と}むれば、
%
\ruby{流石}{さす|が}に
\ruby{日方}{ひ|かた}も
\ruby{羽{\換字{勝}}}{は|がち}を
\ruby{憚}{はゞか}りて、% 「憚 は(ゞ)か」
%
\ruby{言}{ものい}はんとして
\ruby{言}{い}はず
\ruby{已}{や}みけるが、
%
\ruby{眼}{め}には
\ruby{{\換字{猶}}}{なほ}
\ruby{稜角}{か|ど}を
\ruby{立}{た}てゝ
\ruby{島木}{しま|き}を
\ruby{睨}{にら}み、
%
\ruby{此}{こ}の
\ruby{時}{とき}
\ruby{遲}{おそ}く
\ruby{彼}{か}の
\ruby{時}{とき}
\ruby{{\換字{速}}}{はや}く、

\原本頁{20-4}%
『そら
\ruby{{\換字{又}}}{また}
\ruby{馬鹿野郎}{ば|か|や|らう}が
\ruby{御來臨}{お|い|で}なすつた。
%
ハヽヽ、
%
\ruby{何程}{いく|ら}
\ruby{罵}{のゝし}られても
\ruby{相手}{あひ|て}には
ならねえ。
%
\ruby{汝}{きさま}は
\ruby{乃公}{お|れ}に
\ruby{楯}{たて}をついても、
%
\ruby{乃公}{お|ら}あ
\ruby{汝}{きさま}を
\ruby{生呑}{まる|のみ}に
\ruby{吞}{の}んでゝ、
%
そして
\ruby{腹}{はら}にも
\ruby{障}{さは}らねえから。
』

\原本頁{20-7}%
と、
%
\ruby{島木}{しま|き}の
\ruby{冷}{ひや}やかに
\ruby{一矢}{いつ|し}
\ruby{酬}{むく}ゆるに、

\原本頁{20-8}%
『
\ruby{何}{なん}だ、
%
\ruby{吞}{の}んで
\ruby{居}{ゐ}る。
%
\ruby{可矣}{よ|し}ツ、
%
\ruby{吞}{の}まれたつて
\ruby{鐵釘}{かな|くぎ}が
\ruby{何}{なん}となる!。
%
\ruby{曲}{まが}りも
\ruby{仕無}{し|な}いは!、
%
\ruby{丸}{まる}くも
ならんは!。
』

\原本頁{20-10}%
と、
%
\ruby{日方}{ひ|かた}は
\ruby{{\換字{又}}}{また}
\ruby[||j>]{直}{たゞち}に
\ruby{熱}{ねつ}して
\ruby{答}{こた}ふ。

\原本頁{20-11}%
\ruby{悠然}{いう|ぜん}と
\ruby{笑}{ゑみ}を
\ruby{含}{ふく}める
\ruby{羽{\換字{勝}}}{は|がち}は
\ruby{靜}{しづ}かに、

\原本頁{21-1}%
『
\ruby{可}{い}いさ、
%
\ruby{二人}{ふた|り}とも、
%
もう
\ruby{可}{い}いさ。
%
ハヽヽ、
%
\ruby{互}{たがひ}に
\ruby{其}{そ}の
\ruby[<j||]{位}{くらゐ}
\ruby{威張}{ゐ|ば}つたら
\ruby{可}{い}いぢあ
\ruby{無}{な}いか。
%
\ruby{島木}{しま|き}は
\ruby{日方}{ひ|かた}に
\ruby{關}{かま}はないで
\ruby{僕}{ぼく}に
\ruby{話}{はな}すつもりで
\ruby{話}{はな}して
\ruby{吳}{く}れ
\ruby{玉}{たま}へ。
%
\ruby{日方}{ひ|かた}は
また
\ruby{島木}{しま|き}に
\ruby{關}{かま}はないで
\ruby{僕}{ぼく}に
\ruby{{\換字{交}}際}{つき|あ}つて
\ruby{聞}{きい}て
\ruby{居}{ゐ}て
\ruby{吳}{く}れ
\ruby{玉}{たま}へな。
%
つまり
お
\ruby{互}{たがひ}に
\ruby{水野}{みづ|の}の
\ruby{上}{うへ}が
\ruby{知}{し}りたいのだからネ。
』

\原本頁{21-6}%
と、
%
\ruby{優}{やさ}しく
\ruby{制}{せい}すれば、

\原本頁{21-7}%
『ヤ、% 小書きにすると右寄りになるので
%
\ruby{濟}{す}まなかつた、
%
\ruby{僕}{ぼく}が
\ruby{惡}{わる}かつた。
』

\原本頁{21-8}%
『ア、% 小書きにすると右寄りになるので
%
\ruby{左樣}{さ|う}
\ruby{云}{い}はれりやあ
\ruby{乃公}{お|れ}も
\ruby{下}{くだ}らなかつた。
』

\原本頁{21-9}%
と、
\ruby{日方}{ひ|かた}も
\ruby{島木}{しま|き}も
\ruby{爭}{あらそ}ひ
\ruby{止}{や}みて、
%
\ruby{誰}{たれ}
\ruby{勸}{すゝ}めねど
\ruby{同}{おな}じ
\ruby{思}{おも}ひの、
%
\ruby{双方}{さう|はう}
\ruby{一時}{いち|じ}に
\ruby{酒盃}{さか|づき}を
\ruby{{\換字{交}}}{かは}して、
%
\ruby{笑}{わら}つて
\ruby{仕舞}{し|ま}つて
\ruby{痕跡}{あと|かた}もなし。

\原本頁{21-11}%
\ruby{島木}{しま|き}は
\ruby{此度}{こ|たび}は
やゝ
\ruby{眞面目}{ま|じ|め}に、
%
\ruby{羽{\換字{勝}}}{は|がち}の
\ruby{方}{かた}に
\ruby{向}{むか}つて
\ruby{語}{かた}り
\ruby{出}{だ}したり。

\原本頁{22-1}%
『% この島木の語りは其五の途中まで続く
\ruby{一同}{みん|な}も
\ruby{知}{し}つてる
\ruby{{\換字{通}}}{とほ}り
\ruby{彼}{あ}の
\ruby{水野}{みづ|の}は、
%
\ruby{我等}{おれ|たち}の
\ruby{中}{なか}では
\ruby{一番}{いち|ばん}
\ruby{年下}{とし|した}、
%
\ruby{乃公}{お|れ}が
\ruby{今年}{こ|とし}は
\ruby{二十七}{に|じふ|なな}だから、
%
\ruby{七}{なな}、
%
\ruby{六}{ろく}、
%
\ruby{五}{ご}、
%
\ruby{四}{よん}と
\ruby{四}{よ}つ
\ruby{目}{め}で
\ruby{丁度}{ちやう|ど}
\ruby{二十四}{に|じふ|よん}だ。
%
\ruby{宇都宮}{み| |や}から
\ruby[||j>]{東}{とう}
\ruby[||j>]{京}{きやう}へ
% \ruby{東京}{とう|きやう}へ
\ruby{上}{のぼ}る
\ruby{時}{とき}にも、
%
\ruby{一番}{いち|ばん}
\ruby{先}{さき}へ
\ruby{出}{で}たのは
\ruby{羽{\換字{勝}}}{は|がち}だつたが、
%
\ruby{一番}{いち|ばん}
\ruby{後}{あと}へ
\ruby{殘}{のこ}つたのは
\ruby{水野}{みづ|の}だつた。
%
\ruby{{\換字{若}}}{わか}いに
\ruby{似合}{に|あ}はず
\ruby{能}{よ}く
\ruby{出來}{で|き}たから、
%
\ruby{君}{きみ}は
\ruby{{\換字{若}}}{わか}いけれども
\ruby{學業}{わ|ざ}が
\ruby{出來}{で|き}る、
%
\ruby{早}{はや}く
\ruby[||j>]{東}{とう}
\ruby[||j>]{京}{きやう}へ
% \ruby{東京}{とう|きやう}へ
\ruby{出}{で}て
\ruby{身}{み}を
\ruby{立}{た}てるが
\ruby{可}{い}いと、
%
\ruby{勸}{すゝ}めたのは
\ruby{乃公}{お|れ}
\ruby{一人}{ひと|り}で
\ruby{無}{な}かつたが、
%
いや
\ruby{小生}{わた|くし}の
\ruby[<j>]{志}{こゝろざ}す
ところは
\ruby{些}{ちと}
\ruby{{\換字{違}}}{ちが}ふから、
%
\ruby{左樣}{さ|う}
\ruby{急}{いそ}がないでも
\ruby{可}{い}い
\ruby{事}{こと}だ、
%
\原本頁{22-8}\改行%
\ruby{他}{ほか}の
\ruby{人}{ひと}は
\ruby{一日}{いち|にち}
\ruby{遲}{おそ}ければ
\ruby{一日}{いち|にち}
\ruby{損}{そん}、
%
\ruby{少}{すこ}しも
\ruby{疾}{はや}く
\ruby[<j||]{上}{じやう}
\ruby[||j>]{京}{きやう}
するが
\ruby{可}{い}い、
%
と
\原本頁{22-9}\改行%
\ruby{妙}{めう}に
\ruby{片意地}{かた|い|ぢ}に
\ruby{{\換字{謙}}遜}{けん|そん}して
\ruby{出}{で}ず。
%
\ruby{二番}{に|ばん}に
\ruby{出}{で}たが
\ruby{日方}{ひ|かた}
\ruby{山瀬}{やま|せ}、
%
それから
\原本頁{22-10}\改行%
\ruby{名倉}{な|ぐら}、
%
それから
\ruby{楢井}{なら|い}、
%
それから
\ruby{乃公}{お|れ}で、
%
\ruby{其}{その}
\ruby{後}{あと}から
\ruby{漸}{やつ}と
\ruby[<j||]{上}{じやう}
\ruby[||j>]{京}{きやう}
した。
%
\原本頁{22-11}\改行%
\ruby{其}{そ}の
\ruby[||j>]{位}{くらゐ}
\ruby{異}{ おつ}に
\ruby{固}{かた}いところのある
\ruby{男}{をとこ}で、
%
\ruby[||j>]{東}{とう}
\ruby[||j>]{京}{きやう}へ
% \ruby{東京}{とう|きやう}へ
\ruby{出}{で}てからも
\ruby{一同}{みん|な}は
\ruby{誰}{たれ}
\原本頁{23-1}\改行%
しも、
%
\ruby{身}{み}を
\ruby{立}{た}てる
\ruby{{\換字{道}}}{みち}に
\ruby{汲々}{きふ|〳〵}として、
%
\ruby{隨{\換字{分}}}{ずゐ|ぶん}
\ruby{骨}{ほね}を
\ruby{折}{を}つて
それ〴〵に、%
%
\ruby{辛}{から}く
\ruby{出世}{しゆつ|せ}も
\ruby{仕}{し}て
\ruby{來}{き}たに、
%
\ruby{彼}{あ}の
\ruby{男}{をとこ}ばかりは
\ruby{澄}{す}ましかへつて、
%
\ruby{今}{いま}でも
\ruby{小學}{せう|がく}
\ruby{敎師}{けう|し}で
\ruby{甘}{あま}んじて
\ruby{居}{を}る。
%
それで
\ruby{惰}{なま}けて
\ruby{居}{を}るのかと
\ruby{思}{おも}へば、
%
\原本頁{23-4}\改行%
\ruby{一寸}{いつ|すん}の
\ruby{暇}{ひま}も
\ruby{惜}{をし}んで
\ruby[<j||]{勉}{べん }% 行末行頭の境界付近なので特例処置を施す
\ruby[<j||]{{\換字{強}}}{きやう}して、
% \ruby{勉{\換字{強}}}{べん|きやう}して、
%
あらゆる
\ruby{方面}{はう|めん}に
\ruby{行}{ゆ}き
\ruby{渡}{わた}つて
\ruby{居}{ゐ}る。
%
\原本頁{23-5}\改行%
\ruby{僕}{ぼく}は
\ruby[||j>]{一}{いつ}
\ruby[||j>]{生}{しやう}を
% \ruby{一生}{いつ|しやう}を
かけて
\ruby{此}{こ}の
\ruby{世}{よ}の
\ruby{中}{なか}に、
%
たゞ
\ruby{一篇}{いつ|ぺん}の
\ruby{詩}{し}を
\ruby{{\換字{留}}}{とゞ}めれば
\ruby{可}{い}いのだ。
%
\ruby{今}{いま}は
\ruby{其}{そ}の
\ruby{準備}{よう|い}に
\ruby{{\換字{勤}}}{つと}めて
\ruby{居}{ゐ}るので、
%
\ruby{他}{ほか}に
\ruby{慾}{よく}も
\ruby{無}{な}ければ
\ruby{望}{のぞみ}も
\原本頁{23-8}\改行%
\ruby{無}{な}い、
%
\ruby{{\換字{半}}熟}{なま|にえ}なものを
\ruby{世}{よ}に
\ruby{出}{だ}して、
%
\ruby{今}{いま}つから
\ruby{{\換字{文}}人}{ぶん|じん}
\ruby{顏}{がほ}するのも
\ruby{羞}{はづ}か
しいから、
%
もう
\ruby{十年}{じふ|ねん}ばかりは
\ruby{小學讀本}{と|く|ほ|ん}
いぢりで、
%
たゞ〳〵
\ruby[||j>]{勉}{べん}
\ruby[||j>]{{\換字{強}}}{きやう}
% \ruby{勉{\換字{強}}}{べん|きやう}
\原本頁{23-9}\改行%
を
するつもりだ、
%
と
\ruby{隱君子}{いん|くん|し}
\ruby{氣質}{かた|ぎ}で
\ruby{日}{ひ}を
\ruby{經}{へ}て
\ruby{居}{ゐ}たのは、
%
\ruby{羽{\換字{勝}}}{は|がち}はじめ
\ruby{一同}{みん|な}も
\ruby{知}{し}つて
\ruby{居}{ゐ}やう。
%
ところで
\ruby{此}{こ}の
\ruby{乃公}{お|れ}は
\ruby{金}{かね}まうけ
\ruby{主義}{しゆ|ぎ}、
%
\ruby{卑}{いや}しいと
\ruby{云}{い}つて
\ruby{一同}{みん|な}に
\ruby{罵}{のゝし}られた
\ruby{位}{くらゐ}だから、
%
\ruby{守}{まも}るところのある
\ruby{浪人}{らう|にん}
\原本頁{24-1}\改行%
\ruby{肌}{はだ}の、
%
\ruby{水野}{みづ|の}と
\ruby{氣}{き}の
\ruby{合}{あ}ふ
\ruby{譯}{わけ}は
\ruby{毫}{ちつと}も
\ruby{無}{な}いが、
%
\ruby{他}{ほか}の
\ruby{五人}{ご|にん}は
\ruby[<j||]{上}{じやう}
\ruby[||j>]{京}{きやう}
して、
%
\原本頁{24-2}\改行%
\ruby{二人}{ふた|り}だけ
\ruby{宮}{みや}に
\ruby{殘}{のこ}つた
\ruby{時}{とき}、
%
\ruby{彼}{あれ}が
\ruby{熱}{ねつ}を
\ruby{病}{や}んだのを
\ruby{介抱}{かい|はう}して、
%
\ruby{長}{なが}い
\ruby{看護}{み|とり}を
\ruby{爲}{し}て
\ruby{{\換字{遣}}}{や}つた、
%
\ruby{其事}{そ|れ}が
\ruby{{\換字{鎖}}}{くさり}になつて
\ruby{此地}{こつ|ち}へ
\ruby{來}{き}ても、
%
\ruby{取}{と}り
\ruby{{\換字{分}}}{わ}け
\ruby{二人}{ふた|り}は
\ruby{親}{した}しく
\ruby{仕}{し}て
\ruby{居}{ゐ}た。
%
\換字{志}かし
\ruby{乃公}{お|ら}あ
\ruby{俗物}{ぞく|ぶつ}、
%
\ruby{水野}{みづ|の}は
\ruby{仙骨}{せん|こつ}、
%
\ruby{此方}{こつ|ち}は
\原本頁{24-5}\改行%
\ruby{飛}{と}んだり
\ruby{跳}{はね}たりして
\ruby{悶躁}{も|が}いて
\ruby{居}{ゐ}るので、
%
\ruby{中々}{なか|〳〵}
\ruby{往來}{ゆき|き}することも
\ruby{多}{おほ}くは
\ruby{無}{な}かつた。
%
さあ
\ruby{此處}{こ|ゝ}で
\ruby[||j>]{白}{はく}
\ruby[||j>]{狀}{じやう}
% \ruby{白狀}{はく|じやう}
\ruby[||j>]{仕}{ し}なけりや
ならないが、
%
\ruby{丁度}{ちやう|ど}
\ruby{一昨年}{を|とゝ|し}の
\ruby{暮}{くれ}だつた。
%
\ruby{實}{じつ}は
\ruby{此}{こ}の
\ruby{乃公}{お|れ}が
\ruby{山氣}{やま|ぎ}に
\ruby{逸}{はや}つて、
%
\ruby{危}{あぶな}い
\ruby{橋}{はし}を
\ruby{渡}{わた}る
\原本頁{24-8}\改行%
\ruby{輕業}{かる|わざ}をやつたところ、
%
\ruby{{\換字{運}}}{うん}が
\ruby{惡}{わる}くつて
\ruby{可厭}{い|や}な
\ruby{目}{め}が
\ruby{出}{で}て、
%
\ruby{甘}{うま}く
\ruby{行}{い}きあ
\ruby{論}{ろん}は
\ruby{無}{な}いことが
\ruby{打壞}{ぶつ|こわ}れたんで、
%
たつた
\ruby{五十}{ご|じふ}
\ruby{兩}{りやう}ばかりの
\ruby{有無}{ある|なし}で
\原本頁{24-10}\改行%
\ruby{何樣}{ど|う}にも
\ruby{仕切}{し|き}れない
\ruby{機會}{は|め}へ
\ruby{臨}{のぞ}んだ。
%
そも〳〵
\ruby{投機}{や|ま}を
\ruby{始}{はじ}めた
\ruby{其}{そ}の
\原本頁{24-11}\改行%
\ruby{時}{とき}から、
%
\ruby{乃公}{お|ら}あ
\ruby{危}{あぶな}い
\ruby{事}{こと}をする
\ruby{代}{かは}りにやあ、
%
\ruby{乃公}{お|れ}が
\ruby{一六}{いち|ろく}% ばくち・双六 (すごろく) で二つの賽 (さい) を振って、その目に一と六とが同時に出ること
\ruby{沙汰}{ざ|た}を
\ruby{廢}{や}めぬ
\ruby{内}{うち}は、
%
\原本頁{25-1}%
\ruby{金錢}{きん|せん}に
\ruby{關}{かゝ}つた
\ruby{事}{こと}では
\ruby{決}{けつ}して
\ruby{一同}{みん|な}に、
%
\ruby{苦勞}{く|らう}は
\ruby{掛}{か}けぬと
\原本頁{25-2}\改行%
\ruby{誓言}{ちか|ひ}を
\ruby{立}{た}つた
\ruby{表}{おもて}が
あるから
\ruby{誰}{たれ}にも
\ruby{云}{い}へず、
%
\ruby{思案}{し|あん}に
\ruby{餘}{あま}つて
\ruby[<j||]{獨}{ひとり}
\ruby{語}{ごと}のやうに、
%
\ruby{其}{その}
\ruby{譯}{わけ}を
\ruby{水野}{みづ|の}に
\ruby{話}{はな}して
\ruby{見}{み}ると、
%
\ruby{手箱}{て|ばこ}の
\ruby{底}{そこ}から
\ruby{書}{か}いたものを
\ruby{出}{だ}して、
%
\ruby{此}{これ}を
\ruby{山瀬}{やま|せ}
\ruby{君}{くん}に
\ruby{頼}{たの}んで
\ruby{賣}{う}つて
\ruby{貰}{もら}つたら、
%
\ruby[||j>]{其}{その}
\ruby[||j>]{位}{くらゐ}の
% \ruby{其位}{その|くらゐ}の
\ruby{金}{かね}は
\ruby{出來}{で|き}るか
\ruby{知}{し}れぬ、
%
\ruby{出來}{で|き}たら
\ruby{使}{つか}ひ
\ruby{玉}{たま}へ
といふ
\ruby{話}{はなし}。
%
\ruby{當}{あて}には
ならないと
\原本頁{25-6}\改行%
\ruby{思}{おも}つたが、
%
\ruby{山瀬}{やま|せ}に
\ruby{頼}{たの}むと
\ruby{其事}{そ|れ}が
\ruby{出來}{で|き}て、
%
そこで
\ruby{大}{おほき}に
\ruby{助}{たす}かつた。
%
\原本頁{25-7}\改行%
\ruby{其}{そ}の
\ruby{味}{あぢ}を
\ruby{占}{し}めた
といふのでは
\ruby{無}{な}いが、
%
\ruby{其}{そ}の
\ruby{後}{のち}も
\ruby{種子}{た|ね}を
\ruby{耗}{す}つた
\ruby{其}{その}
\原本頁{25-8}\改行%
\ruby{時}{とき}は、
%
\ruby{三度}{さん|ど}といふもの
\ruby{助}{たす}けて
\ruby{貰}{もら}つて、
%
\ruby{矢種}{や|だね}を
つぎ〳〵
\ruby{戰}{たゝか}つた
\ruby{末}{すゑ}、
%
\原本頁{25-9}\改行%
どうやら
\ruby{{\換字{遣}}}{や}つて
\ruby{行}{い}かれる
\ruby{身體}{から|だ}になつた。
%
そこで
\ruby{水野}{みづ|の}に
\ruby{對}{むか}つて
\ruby{乃公}{お|れ}が
いふには、
%
\ruby{貰}{もら}つたものを
\ruby{{\換字{返}}}{かへ}さうとは
\ruby{云}{い}はないが、
%
\ruby{金}{かね}が
\ruby{要}{い}る
\原本頁{25-11}\改行%
\ruby{時}{とき}は
\ruby{何時}{い|つ}でも
\ruby{云}{い}ひたまへ、
%
\ruby{乃公}{お|れ}が
\ruby{懷中}{ふと|ころ}だけなら
\ruby{洗}{さら}け
\ruby{出}{だ}すから、
%
\原本頁{26-1}\改行%
と
\ruby{此}{こ}の
\ruby{春}{はる}
\ruby{{\換字{遇}}}{あ}つた
\ruby{時}{とき}
\ruby{云}{い}つて
\ruby{置}{お}いた。
%
ところが
\ruby{金}{かね}を
\ruby{使}{つか}ふ
\ruby{水野}{みづ|の}では
\ruby{無}{な}し、
%
たゞ
\ruby{其限}{それ|ぎり}で
\ruby{濟}{す}んで
\ruby{居}{ゐ}たが、
%
\ruby{此}{こ}の
\ruby{夏}{なつ}になつて
\ruby{{\換字{遣}}}{や}つて
\ruby{來}{き}て、
%
\ruby{眞赤}{まつ|か}な
\ruby{顏}{かほ}をして
きまり
\ruby{惡}{わる}さうに、
%
\ruby{三十}{さん|じふ}
\ruby{兩}{りやう}
ばかり
\ruby{貸}{か}して
\ruby{吳}{く}れろ、
%
と
\原本頁{26-4}\改行%
\ruby{云}{い}つたのが
\ruby{最初}{はじ|まり}で
\ruby{其}{その}
\ruby{後}{のち}も、% 原本通り非踊り字表現
%
ぼつり〳〵と
\ruby{持}{も}つて
\ruby{行}{ゆ}く。
%
\ruby{其事}{そ|れ}が
\ruby{乃公}{お|れ}が
\ruby{勘}{かん}を
\ruby{付}{つ}けた
はじまりだつた。
