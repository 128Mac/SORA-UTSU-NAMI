\Entry{其四}

% メモ 校正終了 2024-03-30 2024-05-22 2024-06-15
\原本頁{26-7}%
\ruby[||j>]{考}{かんが}へて
\ruby{見}{み}りやあ
\ruby[g]{合點}{が てん}が
いかない。
%
\ruby[g]{多{\換字{分}}}{たんと }では
\ruby{無}{な}いが
\ruby[g]{給料}{きふれう}も
\ruby{取}{と}るし
\改行% 校正作業の簡略化のため
、
%
\原本頁{26-8}\改行%
\ruby{別}{べつ}に
\ruby[g]{蕩樂}{だうらく}の
\ruby{無}{な}い
\ruby{男}{をとこ}だから
\ruby[g]{其金}{そ れ }で
\ruby{一人身}{ひと|り|み}の
\ruby[g]{今日}{こんにち}を
\ruby{濟}{す}ませて、
%
\ruby[g]{剩餘}{あまり }で
\ruby[g]{書物}{しよもつ}を
\ruby{買}{か}つて
\ruby{讀}{よ}む
\ruby{位}{くらゐ}の
\ruby{事}{こと}。
%
その
\ruby[g]{書物}{しよもつ}を
\ruby{買}{か}ふにも
たゞは
\ruby{買}{か}はないで、
%
\ruby[g]{何時}{い つ }でも
\ruby{讀}{よ}んで
\ruby{了}{しま}つたのを
\ruby{下}{した}に
\ruby{{\換字{遣}}}{や}つて、
%
まだ
\ruby{讀}{よ}まぬもの
\原本頁{27-1}\改行%
と
\ruby[g]{取換}{とりか }へる。
%
それを
\ruby[g]{自{\換字{分}}}{じ ぶん}でも
\ruby[g]{可笑}{をかし }がつて、
%
\ruby{何}{なん}の
\ruby{事}{こと}は
\ruby{無}{な}い
\ruby{僕}{ぼく}の
\ruby{爲}{す}ることは
\ruby[g]{書肆}{ほんや }のために、
%
\ruby[g]{一枚}{いちまい}
\ruby[g]{一枚}{いちまい}
\ruby[g]{蟲拂}{むしはら}ひを
\ruby[g]{叮嚀}{ていねい}に
\ruby{仕}{し}て
\ruby{{\換字{遣}}}{や}る
やうなものだと
\ruby{云}{い}つて
\ruby{居}{ゐ}た
\ruby{程}{ほど}。
%
\ruby{併}{しか}し
\ruby[g]{左樣}{さ う }いふ
\ruby{{\換字{遣}}}{や}り
\ruby{方}{かた}を
して
\ruby{少}{すくな}い
\ruby{錢}{ぜに}で
\原本頁{27-4}\改行%
\ruby{多}{おほ}く
\ruby{讀}{よ}む、
%
それだけ
\ruby[g]{始末}{し まつ}の
\ruby{好}{い}い
\ruby{賢}{かしこ}い
\ruby[g]{水野}{みづの }が、
%
\ruby{何}{なん}の
\ruby{彼}{か}のと
\ruby{云}{い}つては
\ruby{金}{かね}を
\ruby{持}{も}つて
\ruby{行}{ゆ}く。
%
ハテ
\ruby{是}{これ}にやあ
\ruby{何}{なん}ぞ
\ruby[g]{仔細}{し さい}があらう、
%
\ruby{譯}{わけ}が
\ruby{無}{な}くちやあ
\ruby{要}{い}らない
\ruby{金}{かね}だ。
%
いくら
\ruby[g]{表面}{うはべ }は
\ruby[g]{物柔}{ものやは}らかな
\ruby{君子風}{くん|し|ふう}で、
%
\ruby{腹}{はら}の
\原本頁{27-7}\改行%
\ruby{底}{そこ}の
\ruby{底}{そこ}にやあ
\ruby{恐}{おそろ}ろしい% ルビ調整(原本通り)「おそろ」
\ruby[g]{高慢}{かうまん}、
%
\ruby{世界中}{せ|かい|ぢゆう}の
\ruby{奴}{やつ}を
\ruby[g]{相手}{あひて }にしても、
%
\ruby{鼻}{はな}の
\原本頁{27-8}\改行%
\ruby{頭}{さき}で
\ruby{笑}{わら}つて
\ruby{居}{ゐ}やうといふ
\ruby{沈毅{\換字{漢}}}{しつ|かり|もの}の、
%
\ruby{彼}{あ}の
\ruby[g]{水野}{みづの }でも、
%
\ruby[g]{年齡}{と し }は
\ruby[g]{年齡}{と し }だ。
%
\ruby{桃}{もゝ}の
\ruby{{\換字{速}}}{はや}いのも
\ruby{柹}{かき}の
\ruby{遲}{おそ}いのも、
%
いづれ
\ruby{時}{とき}が
\ruby{來}{く}りやあ
\ruby{花}{はな}は
\ruby{{\換字{咲}}}{さ}き
\原本頁{27-10}\改行%
\ruby{出}{だ}す。
%
\ruby{才}{さい}
はじけたも
\ruby{謹}{つゝ}しまやか
\footnote{%
検討対象の「\ruby{謹}{つゝ}しまやか」とルビを振っている件について、
「謹」の読みの一つに「つつしむ」があるので原本通りとする
(国会図書館 コマ番号17/134 p-27 l-10)}%
なも、
%
\ruby[g]{時{\換字{節}}}{じ せつ}
\ruby[g]{因緣}{いんねん}で
\ruby{{\換字{情}}}{こゝろ}が
\ruby{萌}{も}える。
%
\原本頁{27-11}\改行%
\ruby[g]{乃公}{お れ }のやうな
\ruby[g]{早熟}{はやなり}やあ
\ruby{十七八}{じふ|しち|はち}から、% 原本には漢数字「七」のルビ無し
%
\ruby[g]{白{\換字{粉}}}{おしろい}や
\ruby{油}{あぶら}の
\ruby{香}{にほひ}に
\ruby{鼻}{はな}も
ぴこつかせたが、
%
%ひこつく  ... ひくひく動く。主に鼻についていう。
%びこつかす ... 小刻みに動かす。ちょっちょっと動かす。
%びこつく  ... りきむ。虚勢を張る。ぴこつく。
\原本頁{28-1}%
\ruby{其}{その}
\ruby{代}{かは}り
\ruby[g]{{\換字{浮}}氣}{うはき }の
\ruby{掛}{か}け
\ruby{流}{なが}しで、
%
\ruby{笑}{わら}ふのも
\ruby{泣}{な}くのも
\ruby[g]{二日}{ふつか }か
\原本頁{28-2}\改行%
\ruby[g]{三日}{みつか }
\ruby{限}{き}り、
%
\ruby{思}{おも}ふも
\ruby{思}{おも}はれるも
\ruby{實}{じつ}は
\ruby{無}{な}くつて、
%
のほゝんで
\ruby[g]{今日}{け ふ }まで
\ruby[g]{無事}{ぶ じ }に
\ruby{來}{き}たが、
%
\ruby[g]{水野}{みづの }のやうな
\ruby[g]{彼樣}{あ ん }な
\ruby{男}{をとこ}が、
%
\ruby{惡}{わる}くすると
\ruby{唯}{たゞ}
\ruby[g]{一{\換字{途}}}{いちづ }に
\ruby[||j>]{純}{いつ}
\ruby[||j>]{粹}{ぽんぎ}の、
% \ruby{純粹}{いつ|ぽんぎ}の、
%
\ruby{眞正直}{まつ|しやう|ぢき}な
\ruby{戀}{こひ}に
\ruby{落}{お}ちて、
%
\ruby{人}{ひと}にも
\ruby{知}{し}らさず
\ruby{獨}{ひと}り
\ruby{苦}{くる}しみ、
%
\原本頁{28-5}\改行%
\ruby{思}{おも}ひ
\ruby{詰}{つ}め
\ruby{思}{おも}ひ
\ruby{詰}{つ}めて
\ruby{忘}{わす}れる
\ruby{間}{ま}も
\ruby{無}{な}く、
%
\ruby{胸}{むね}に
\ruby{解}{と}けかねる
\ruby[g]{凝塊}{し こり}を
\ruby{出}{で}かして、
%
\ruby{長}{なが}く
〳〵
\ruby{悶}{もだ}へて
\ruby{惱}{なや}むとも
あるもの。
%
\ruby{{\換字{若}}}{もし}や
\ruby[g]{其樣}{そ ん }な
\ruby{事}{こと}でゞ
\原本頁{28-7}\改行%
もあるならば、
%
\ruby[g]{朋友}{ともだち}の
よしみ、
%
\ruby[g]{年上}{としうへ}の
\ruby[g]{甲{\換字{斐}}}{か ひ }、
%
\ruby{特}{こと}には
\ruby{誰}{たれ}にも
\ruby{知}{し}らさず
\ruby[g]{内々}{ない〳〵}で、
%
\ruby{恩}{おん}を
\ruby{受}{う}けて
\ruby{居}{ゐ}る
\ruby[g]{譯合}{わけあひ}もあり、
%
\ruby{一}{ひ}ト
\ruby[g]{心配}{しんぱい}
\ruby[g]{仕無}{し な }けりやあ
ならぬと
\ruby{意}{こゝろ}を
\ruby{定}{さだ}めて、
%
さて
\ruby[g]{其時}{そ れ }から
\ruby[g]{水野}{みづの }の
\ruby[g]{樣子}{やうす }を
\ruby{見}{み}ると
\makeatletter
\@ifundefined{デバッグ@ビルド}{%
  \ruby[||j>]{推}{すゐ }
  \ruby[||j>]{量}{りやう}
}{%
  \ruby[<j||]{推}{すゐ }
  \ruby[<j||]{量}{りやう}% 行末行頭の境界付近なので特例処置を施す
}%
\makeatother
% \ruby{推量}{すゐ|りやう}
の
\ruby{{\換字{通}}}{とほ}り。
%
\ruby{何}{なん}と
\ruby{無}{な}く
\ruby{人}{ひと}に
\ruby[<j||]{隔}{へだて}
\ruby[||j>]{心}{ごゝろ}
がある。
%
\ruby{何}{なん}と
\ruby{無}{な}く
そは〳〵としたところがある。
%
\ruby[g]{此方}{こつち }から% ルビ調整(原本通り)
\ruby{話}{はな}す
\ruby{談}{はなし}には
\ruby{身}{み}を
\ruby{入}{い}れて
\ruby{聞}{き}かぬ。
%
\ruby{彼}{あれ}が
\ruby{話}{はな}す
\原本頁{29-1}\改行%
\ruby{談}{はなし}には
\ruby[g]{氣焰}{いきほひ}が
\ruby{足}{た}らぬ。
%
\ruby{人}{ひと}と
\ruby{對}{むか}ひあつて
\ruby{坐}{すわ}つて
\ruby{居}{ゐ}ながら、
%
\ruby[g]{談話}{はなし }が
\原本頁{29-2}\改行%
\ruby[g]{一寸}{ちよつと}
\ruby{斷}{た}えれば
\ruby{胸}{むね}の
\ruby{中}{なか}では、
%
\ruby{既}{もう}
\ruby[g]{他方}{よ そ }の
\ruby{事}{こと}を
\ruby{思}{おも}つて
\ruby{居}{ゐ}る
\ruby[g]{樣子}{やうす }。
%
\ruby[g]{將來}{ゆくすゑ}の
\ruby[g]{希望}{の ぞみ}は
\ruby{餘}{あま}り
\ruby{言}{い}はずに、
%
やゝもすると
\ruby{{\換字{過}}}{す}ぎた
\ruby{事}{こと}を
\ruby{云}{い}ひ
\ruby{出}{だ}しては
\改行% 校正作業の簡略化のため
、
%
\原本頁{29-4}\改行%
\ruby{無邪氣}{む|じや|き}だつた
\ruby[g]{往時}{むかし }を
なつかしがる。
%
\ruby{試}{こゝろ}みに
\ruby{{\換字{浮}}世話}{うき|よ|ばなし}を
\ruby[g]{三種}{み いろ}
\ruby[g]{四種}{よ いろ}
\ruby{爲}{し}て、
%
\ruby{何}{ど}の
\ruby{話}{はなし}が
\ruby{彼}{あれ}の
\ruby{胸}{むね}の
\ruby{中}{うち}と
\ruby{響}{ひゞ}き
\ruby{合}{あ}ふかと、
%
\ruby{探}{さぐ}つて
\ruby{見}{み}れば
\ruby[g]{全然}{すつかり}
\ruby{{\換字{分}}}{わか}つて、
%
\ruby{此}{こ}の
\ruby{絃}{いと}に
\ruby{和}{あ}つて
\ruby{鳴}{な}るのは
\ruby{其}{そ}の
\ruby{絃}{いと}と、
%
\ruby[g]{{\換字{判}}然}{ちやん }と
\ruby[||j>]{正}{しやう}
\ruby[||j>]{體}{ たい}の
% \ruby{正體}{しやう|たい}の
\ruby[g]{合點}{が てん}が
\原本頁{29-7}\改行%
いつた。
%
さあ
\ruby[g]{打棄}{うつちや}つて
\ruby{置}{お}く
\ruby{譯}{わけ}にやあ
\ruby{行}{い}かない。
%
\ruby[g]{相手}{あひて }さへ
\ruby{好}{よ}けりやあ
\ruby[g]{仔細}{し さい}は
\ruby{無}{な}いこと。
%
\ruby[g]{南方}{みなみ }へ
\ruby{枝}{えだ}が
さして
\ruby{花}{はな}が
\ruby{{\換字{咲}}}{さ}くに
\ruby{何}{なん}の
\ruby{罪}{つみ}!。
%
\原本頁{29-9}\改行%
\ruby[g]{人{\換字{情}}}{じやう }の%ここは「にんじやう」でなく原本通り「じやう」
\ruby[||j>]{溫}{あつた}
\ruby[||j>]{{\換字{暖}}}{ かみ}を
% \ruby{溫{\換字{暖}}}{あつた|かみ}を
\ruby{得}{え}やうと
おもつて、
%
\ruby{{\換字{若}}}{わか}い
\ruby{心}{こゝろ}の
\ruby{動}{うご}き
\ruby{出}{だ}すのが
\ruby{何}{なに}
\ruby[g]{無理}{む り }だらう!。
%
\ruby[g]{年齡}{と し }が
\ruby[g]{年齡}{と し }だもの、
%
\ruby{有}{あ}り
\ruby{内}{うち}の
\ruby{事}{こと}だ。
%
\ruby{然}{しか}し
\ruby{緣}{えん}は
\ruby{異}{い}なもの
\ruby{危}{あぶな}いもの、
%
よもやとは
\ruby{思}{おも}ふけれど、
%
\ruby{萬}{まん}が
\ruby{一}{いち}にも、
%
\ruby[g]{素性}{すじやう}や
\ruby{筋}{すぢ}の
\原本頁{30-1}\改行%
\ruby{惡}{わる}い
\ruby{女}{をんな}が
\ruby[g]{相手}{あひて }だつた
\ruby{日}{ひ}には
\ruby[g]{水野}{みづの }の
\ruby[g]{不幸}{ふ かう}、
%
\ruby{止}{と}め
\ruby{立}{だて}も
\ruby{爭}{あらそ}ひ
\ruby{立}{だて}も
\ruby[g]{仕無}{し な }けりや
ならぬ。
%
\ruby{金}{かね}の
\ruby{要}{い}るだけに
\ruby{氣}{き}がゝりな
ところがある。
%
と
\ruby{思}{おも}つたので
\ruby[g]{乃公}{お れ }の
\ruby[g]{身體}{からだ }にやあ
\ruby{暇}{ひま}も
\ruby{無}{な}かつたが、
%
\ruby[g]{或日}{あるひ }
\ruby[g]{水野}{みづの }の
\ruby[g]{不在}{る す }を
\原本頁{30-4}\改行%
\ruby{覗}{ねら}つて、
%
\ruby[g]{水野}{みづの }を
\ruby{置}{お}いて
\ruby[g]{世話}{せ わ }をして
\ruby{居}{ゐ}る
\ruby[g]{山路}{やまぢ }の
\ruby[g]{老夫}{おやぢ }を
\ruby{捕}{つかま}へて
\ruby{糺}{たゞ}しかけると、
%
\ruby{彼}{あ}の
\ruby[g]{老夫}{おやぢ }も
\ruby[g]{中々}{なか〳〵}の
\ruby{親切者}{しん|せつ|もの}で、
%
\ruby{特}{こと}さら
\ruby[g]{水野}{みづの }の
\ruby[g]{{\換字{平}}生}{ひ ごろ}の% ルビ調整(原本通り)
\ruby[g]{品行}{み もち}に
\ruby{惚}{ほ}れて
\ruby{居}{ゐ}るので、
%
\ruby{實}{じつ}は
\ruby[g]{水野}{みづの }
\ruby{樣}{さん}の
\ruby{御利益}{お|た|め}を
\ruby{思}{おも}つて、
%
\ruby[g]{貴下}{あなた }でも
\原本頁{30-7}\改行%
\ruby{御來臨}{お|い|で}になつたら
\ruby{申}{まを}し
\ruby{上}{あ}げたいと、
%
\ruby[g]{内々}{ない〳〵}
\ruby{願}{ねが}つて
\ruby{居}{ゐ}た
ところでござりました、
%
といふので
\ruby[g]{一切}{いつさい}の
\ruby[g]{事{\換字{情}}}{じじやう}は
\ruby[g]{老夫}{おやぢ }の
\ruby{口}{くち}から
\ruby{知}{し}れた。
