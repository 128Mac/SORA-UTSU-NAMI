\Entry{其四}

% メモ 校正終了 2024-03-30
\原本頁{27-7}%
\ruby[||j>]{考}{かんが}へて
\ruby{見}{み}りやあ
\ruby{合點}{が|てん}が
いかない。
%
\ruby{多{\換字{分}}}{たん|と}では
\ruby{無}{な}いが
\ruby{給料}{きふ|れう}も
\ruby{取}{と}るし、
%
\ruby{別}{べつ}に
\ruby{蕩樂}{だう|らく}の
\ruby{無}{な}い
\ruby{男}{をとこ}だから
\ruby{其金}{そ|れ}で
\ruby{一人身}{ひと|り|み}の
\ruby{今日}{こん|にち}を
\ruby{濟}{す}ませて、
%
\ruby{剩餘}{あま|り}で
\ruby{書物}{しよ|もつ}を
\ruby{買}{か}つて
\ruby{讀}{よ}む
\ruby{位}{くらゐ}の
\ruby{事}{こと}。
%
その
\ruby{書物}{しよ|もつ}を
\ruby{買}{か}ふにも
たゞは
\ruby{買}{か}はないで、
%
\ruby{何時}{い|つ}でも
\ruby{讀}{よ}んで
\ruby{了}{しま}つたのを
\ruby{下}{した}に
\ruby{{\換字{遣}}}{や}つて、
%
まだ
\ruby{讀}{よ}まぬものと
\ruby{取換}{とり|か}へる。
%
\原本頁{28-1}%
それを
\ruby{自{\換字{分}}}{じ|ぶん}でも
\ruby{可笑}{を|かし}がつて、
%
\ruby{何}{なん}の
\ruby{事}{こと}は
\ruby{無}{な}い
\ruby{僕}{ぼく}の
\ruby{爲}{す}ることは
\ruby{書肆}{ほん|や}のために、
%
\ruby{一枚}{いち|まい}
\ruby{一枚}{いち|まい}
\ruby{蟲拂}{むし|はら}ひを
\ruby{叮嚀}{てい|ねい}に
\ruby{仕}{し}て
\ruby{{\換字{遣}}}{や}る
やうなものだと
\ruby{云}{い}つて
\ruby{居}{ゐ}た
\ruby{程}{ほど}。
%
\ruby{併}{しか}し
\ruby{左樣}{さ|う}いふ
\ruby{{\換字{遣}}}{や}り
\ruby{方}{かた}を
して
\ruby{少}{すくな}い
\ruby{錢}{ぜに}で
\ruby{多}{おほ}く
\ruby{讀}{よ}む、
%
それだけ
\ruby{始末}{し|まつ}の
\ruby{好}{い}い
\ruby{賢}{かしこ}い
\ruby{水野}{みづ|の}が、
%
\ruby{何}{なん}の
\ruby{彼}{か}のと
\ruby{云}{い}つては
\ruby{金}{かね}を
\ruby{持}{も}つて
\ruby{行}{ゆ}く。
%
ハテ
\ruby{是}{これ}にやあ
\ruby{何}{なん}ぞ
\ruby{仔細}{し|さい}があらう、
%
\ruby{譯}{わけ}が
\ruby{無}{な}くちやあ
\ruby{要}{い}らない
\ruby{金}{かね}だ。
%
いくら
\ruby{表面}{うは|べ}は
\ruby{物柔}{もの|やは}らかな
\ruby{君子風}{くん|し|ふう}で、
%
\ruby{腹}{はら}の
\ruby{底}{そこ}の
\ruby{底}{そこ}にやあ
\ruby{恐}{おそろ}ろしい% ルビは原本通り「おそろ」
\ruby{高慢}{かう|まん}、
%
\ruby{世界中}{せ|かい|ぢゆう}の
\ruby{奴}{やつ}を
\ruby{相手}{あひ|て}にしても、
%
\ruby{鼻}{はな}の
\ruby{頭}{さき}で
\ruby{笑}{わら}つて
\ruby{居}{ゐ}やうといふ
\ruby{沈毅{\換字{漢}}}{しつ|かり|もの}の、
%
\ruby{彼}{あ}の
\ruby{水野}{みづ|の}でも、
%
\ruby{年齡}{と|し}は
\ruby{年齡}{と|し}だ。
%
\ruby{桃}{もゝ}の
\ruby{{\換字{速}}}{はや}いのも
\ruby{柹}{かき}の
\ruby{遲}{おそ}いのも、
%
いづれ
\ruby{時}{とき}が
\ruby{來}{く}りやあ
\ruby{花}{はな}は
\ruby{{\換字{咲}}}{さ}き
\ruby{出}{だ}す。
%
\ruby{才}{さい}
はじけたも
\ruby{謹}{つゝ}しまやかなも、
%
\ruby{時{\換字{節}}}{じ|せつ}
\ruby{因緣}{いん|ねん}で
\ruby{{\換字{情}}}{こゝろ}が
\ruby{萌}{も}える。
%
\ruby{乃公}{お|れ}のやうな
\ruby{早熟}{はや|なり}やあ
\ruby{十七八}{じふ|しち|はち}から、
%
\ruby{白{\換字{粉}}}{おし|ろい}や
\ruby{油}{あぶら}の
\ruby{香}{にほひ}に
\ruby{鼻}{はな}も
ぴこつかせたが、
%
%ひこつく  ... ひくひく動く。主に鼻についていう。
%びこつかす ... 小刻みに動かす。ちょっちょっと動かす。
%びこつく  ... りきむ。虚勢を張る。ぴこつく。
\原本頁{28-1}%
\ruby{其}{その}
\ruby{代}{かは}り
\ruby{{\換字{浮}}氣}{うは|き}の
\ruby{掛}{か}け
\ruby{流}{なが}しで、
%
\ruby{笑}{わら}ふのも
\ruby{泣}{な}くのも
\ruby{二日}{ふつ|か}か
\ruby{三日}{みつ|か}
\ruby{限}{き}り、
%
\ruby{思}{おも}ふも
\ruby{思}{おも}はれるも
\ruby{實}{じつ}は
\ruby{無}{な}くつて、
%
のほゝんで
\ruby{今日}{け|ふ}まで
\ruby{無事}{ぶ|じ}に
\ruby{來}{き}たが、
%
\ruby{水野}{みづ|の}のやうな
\ruby{彼樣}{あ|ん}な
\ruby{男}{をとこ}が、
%
\ruby{惡}{わる}くすると
\ruby{唯}{たゞ}
\ruby{一{\換字{途}}}{いち|づ}に
\ruby[||j>]{純}{いつ}
\ruby[||j>]{粹}{ぽんぎ}の、
% \ruby{純粹}{いつ|ぽんぎ}の、
%
\ruby{眞正直}{まつ|しやう|ぢき}な
\ruby{戀}{こひ}に
\ruby{落}{お}ちて、
%
\ruby{人}{ひと}にも
\ruby{知}{し}らさず
\ruby{獨}{ひと}り
\ruby{苦}{くる}しみ、
%
\ruby{思}{おも}ひ
\ruby{詰}{つ}め
\ruby{思}{おも}ひ
\ruby{詰}{つ}めて
\ruby{忘}{わす}れる
\ruby{間}{ま}も
\ruby{無}{な}く、
%
\ruby{胸}{むね}に
\ruby{解}{と}けかねる
\ruby{凝塊}{し|こり}を
\ruby{出}{で}かして、
%
\ruby{長}{なが}く
〳〵
\ruby{悶}{もだ}へて
\ruby{惱}{なや}むとも
あるもの。
%
\ruby{{\換字{若}}}{もし}や
\ruby{其樣}{そ|ん}な
\ruby{事}{こと}で
ゞもあるならば、
%
\ruby{朋友}{とも|だち}の
よしみ、
%
\ruby{年上}{とし|うへ}の
\ruby{甲{\換字{斐}}}{か|ひ}、
%
\ruby{特}{こと}には
\ruby{誰}{たれ}にも
\ruby{知}{し}らさず
\ruby{内々}{ない|〳〵}で、
%
\ruby{恩}{おん}を
\ruby{受}{う}けて
\ruby{居}{ゐ}る
\ruby{譯合}{わけ|あひ}もあり、
%
\ruby{一}{ひ}ト
\ruby{心配}{しん|ぱい}
\ruby{仕無}{し|な}けりあ
ならぬと
\ruby{意}{こゝろ}を
\ruby{定}{さだ}めて、
%
さて
\ruby{其時}{そ|れ}から
\ruby{水野}{みづ|の}の
\ruby{樣子}{やう|す}を
\ruby{見}{み}ると
\ruby[||j>]{推}{すゐ}
\ruby[||j>]{量}{りやう}の
% \ruby{推量}{すゐ|りやう}の
\ruby{{\換字{通}}}{とほ}り。
%
\ruby{何}{なん}と
\ruby{無}{な}く
\ruby{人}{ひと}に
\ruby{隔心}{へだて|ごゝろ}がある。
%
\ruby{何}{なん}と
\ruby{無}{な}く
そは〳〵としたところがある。
%
\ruby{此方}{こつ|ち}から
\ruby{話}{はな}す
\ruby{談}{はなし}には
\ruby{身}{み}を
\ruby{入}{い}れて
\ruby{聞}{き}かぬ。
%
\ruby{彼}{あれ}が
\ruby{話}{はな}す
\ruby{談}{はなし}には
\ruby{氣焰}{いき|ほひ}が
\ruby{足}{た}らぬ。
%
\原本頁{29-1}%
\ruby{人}{ひと}と
\ruby{對}{むか}ひあつて
\ruby{坐}{すわ}つて
\ruby{居}{ゐ}ながら、
%
\ruby{談話}{はな|し}が
\ruby{一寸}{ちよ|つと}
\ruby{斷}{た}えれば
\ruby{胸}{むね}の
\ruby{中}{なか}では、
%
\ruby{既}{もう}
\ruby{他方}{よ|そ}の
\ruby{事}{こと}を
\ruby{思}{おも}つて
\ruby{居}{ゐ}る
\ruby{樣子}{やう|す}。
%
\ruby{將來}{ゆく|すゑ}の
\ruby{希望}{の|ぞみ}は
\ruby{餘}{あま}り
\ruby{言}{い}はずに、
%
やゝもすると
\ruby{{\換字{過}}}{す}ぎた
\ruby{事}{こと}を
\ruby{云}{い}ひ
\ruby{出}{だ}しては、
%
\ruby{無邪氣}{む|じや|き}だつた
\ruby{往時}{むか|し}を
なつかしがる。
%
\ruby{試}{こゝろ}みに
\ruby{{\換字{浮}}世話}{うき|よ|ばなし}を
\ruby{三種}{みい|ろ}
\ruby{四種}{よ|いろ}
\ruby{爲}{し}て、
%
\ruby{何}{ど}の
\ruby{話}{はなし}が
\ruby{彼}{あれ}の
\ruby{胸}{むね}の
\ruby{中}{うち}と
\ruby{響}{ひゞ}き
\ruby{合}{あ}ふかと、
%
\ruby{探}{さぐ}つて
\ruby{見}{み}れば
\ruby{全然}{すつ|かり}
\ruby{{\換字{分}}}{わか}つて、
%
\ruby{此}{こ}の
\ruby{絃}{いと}に
\ruby{和}{あ}つて
\ruby{鳴}{な}るのは
\ruby{其}{そ}の
\ruby{絃}{いと}と、
%
\ruby{{\換字{判}}然}{ちや|ん}と
\ruby[||j>]{正}{しやう}
\ruby[||j>]{體}{ たい}の
% \ruby{正體}{しやう|たい}の
\ruby{合點}{が|てん}がいつた。
%
さあ
\ruby{打棄}{うつ|ちや}つて
\ruby{置}{お}く
\ruby{譯}{わけ}にやあ
\ruby{行}{い}かない。
%
\ruby{相手}{あひ|て}さへ
\ruby{好}{よ}けりやあ
\ruby{仔細}{し|さい}は
\ruby{無}{な}いこと。
%
\ruby{南方}{みな|み}へ
\ruby{枝}{えだ}が
さして
\ruby{花}{はな}が
\ruby{{\換字{咲}}}{さ}くに
\ruby{何}{なん}の
\ruby{罪}{つみ}!。
%
\ruby{人{\換字{情}}}{じや|う}の%ここは「にんじやう」でなく原本通り「じやう」
\ruby[||j>]{溫}{あつた}
\ruby[||j>]{{\換字{暖}}}{ かみ}を
% \ruby{溫{\換字{暖}}}{あつた|かみ}を
\ruby{得}{え}やうと
おもつて、
%
\ruby{{\換字{若}}}{わか}い
\ruby{心}{こゝろ}の
\ruby{動}{うご}き
\ruby{出}{だ}すのが
\ruby{何}{なに}
\ruby{無理}{む|り}だらう!。
%
\ruby{年齡}{と|し}が
\ruby{年齡}{と|し}だもの、
%
\ruby{有}{あ}り
\ruby{内}{うち}の
\ruby{事}{こと}だ。
%
\ruby{然}{しか}し
\ruby{緣}{えん}は
\ruby{異}{い}なもの
\ruby{危}{あぶな}いもの、
%
よもやとは
\ruby{思}{おも}ふけれど、
%
\ruby{萬}{まん}が
\ruby{一}{いち}にも、
%
\ruby{素性}{す|じやう}や
\ruby{筋}{すぢ}の
\原本頁{30-1}%
\ruby{惡}{わる}い
\ruby{女}{をんな}が
\ruby{相手}{あひ|て}だつた
\ruby{日}{ひ}には
\ruby{水野}{みづ|の}の
\ruby{不幸}{ふ|かう}、
%
\ruby{止}{と}め
\ruby{立}{だて}も
\ruby{爭}{あらそ}ひ
\ruby{立}{だて}も
\ruby{仕無}{し|な}けりや
ならぬ。
%
\ruby{金}{かね}の
\ruby{要}{い}るだけに
\ruby{氣}{き}がゝりな
ところがある。
%
と
\ruby{思}{おも}つたので
\ruby{乃公}{お|れ}の
\ruby{身體}{から|だ}にやあ
\ruby{暇}{ひま}も
\ruby{無}{な}かつたが、
%
\ruby{或日}{ある|ひ}
\ruby{水野}{みづ|の}の
\ruby{不在}{る|す}を
\ruby{覗}{ねら}つて、
%
\ruby{水野}{みづ|の}を
\ruby{置}{お}いて
\ruby{世話}{せ|わ}をして
\ruby{居}{ゐ}る
\ruby{山路}{やま|ぢ}の
\ruby{老夫}{おや|ぢ}を
\ruby{捕}{つかま}へて
\ruby{糺}{たゞ}しかけると、
%
\ruby{彼}{あ}の
\ruby{老夫}{おや|ぢ}も
\ruby{中々}{なか|〳〵}の
\ruby{親切者}{しん|せつ|もの}で、
%
\ruby{特}{こと}さら
\ruby{水野}{みづ|の}の
\ruby{{\換字{平}}生}{ひご|ろ}の
\ruby{品行}{み|もち}に
\ruby{惚}{ほ}れて
\ruby{居}{ゐ}るので、
%
\ruby{實}{じつ}は
\ruby{水野}{みづ|の}
\ruby{樣}{さん}の
\ruby{御利益}{お|た|め}を
\ruby{思}{おも}つて、
%
\ruby{貴下}{あな|た}でも
\ruby{御來臨}{お|い|で}になつたら
\ruby{申}{まを}し
\ruby{上}{あ}げたいと、
%
\ruby{内々}{ない|〳〵}
\ruby{願}{ねが}つて
\ruby{居}{ゐ}た
ところでござりました、
%
といふので
\ruby{一切}{いつ|さい}の
\ruby{事{\換字{情}}}{じ|じやう}は
\ruby{老夫}{おや|ぢ}の
\ruby{口}{くち}から
\ruby{知}{し}れた。
