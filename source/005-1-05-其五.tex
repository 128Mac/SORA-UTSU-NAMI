\Entry{其五}

% メモ 校正終了 2024-03-16 2024-05-22 2024-06-15
\原本頁{30-10}%
\ruby[g]{老夫}{おやぢ }の
\ruby[g]{談話}{はなし }を
\ruby{聞}{き}いて
\ruby{見}{み}りやあ
\ruby[g]{水野}{みづの }は
\ruby{實}{じつ}に
\ruby[||j>]{憫}{かは}
\ruby[||j>]{然}{いさう}だ。% 「憫然 か(は)いさう」
% \ruby{憫然}{かは|いさう}だ。% 「憫然 か(は)いさう」
%
\ruby[g]{勿論}{もちろん}
\ruby{其}{そ}の
\ruby[g]{老夫}{おやぢ }の
\ruby{云}{い}つたことが
\ruby{一}{いち}から
\ruby{十}{じふ}まで
\ruby[g]{眞實}{ほんと }とも
\ruby{限}{かぎ}るまいが、
%
\ruby[g]{岡目}{をかめ }の
\ruby[<j||]{{\換字{評}}}{ひやう}% 行末行頭の境界付近なので特例処置を施す
\ruby[||j>]{{\換字{判}}}{ばん}なり
% \ruby{{\換字{評}}{\換字{判}}}{ひやう|ばん}なり
\ruby[g]{老夫}{としより}の
\ruby[g]{言葉}{ことば }なり、
%
\ruby[g]{大體}{だいたい}は
\ruby{{\換字{違}}}{ちが}ふ
\ruby[g]{氣{\換字{遣}}}{きづかひ}は
あるまい。
%
そも〳〵は
\原本頁{31-3}\改行%
\ruby[g]{今年}{こ とし}の
\ruby{春}{はる}の
\ruby{始}{はじめ}、
%
\ruby[g]{水野}{みづの }の
\ruby{出}{で}て
\ruby{居}{ゐ}る
\ruby[g]{學校}{がくかう}の
\ruby{女敎師}{ぢよ|けう|し}が
\ruby[g]{一人}{ひとり }
\ruby[g]{故鄕}{く に }へ
\ruby{歸}{かへ}つたので
\ruby[g]{闕員}{けつゐん}が
\ruby[g]{出來}{で き }た、
%
\ruby{其}{そ}の
\ruby[g]{補闕}{ほ けつ}として
\ruby{新}{あらた}に
\ruby{來}{き}たのが、
%
まだ
\ruby[g]{敎員}{けうゐん}
\原本頁{31-5}\改行%
になりたての、
%
\ruby{年}{とし}の
\ruby{{\換字{若}}}{わか}い
\ruby[g]{岩崎}{いはさき}% 原本のこの部分は「いわさき」
\ruby{五十子}{い|そ|こ}といふ
\ruby{女}{をんな}だつた。
%
\ruby[g]{老夫}{おやぢ }も
\ruby[g]{度々}{たびたび}% ルビ調整(原本通り)非踊り字表記(行末行頭の境界付近)
\ruby{見}{み}て
\ruby{知}{し}つてるさうだが、
%
\ruby{極}{ごく}
\ruby[g]{可愛}{か あい}らしい
\ruby{惚}{ほ}れ〴〵すると
いふやうな
\ruby[g]{顏立}{かほだち}では
\ruby{無}{な}いけれど、
%
\ruby{眼}{め}の
\ruby{淸}{すゞ}しい
\ruby{鼻}{はな}の
\ruby{高}{たか}い
\ruby[g]{端然}{しやん }とした
\ruby{女}{をんな}で
\改行% 校正作業の簡略化のため
、
%
\原本頁{31-8}\改行%
まあ
\ruby[g]{當世}{たうせい}の
\ruby[g]{下司}{げ す }
\ruby[||j>]{根}{こん}
\ruby[||j>]{性}{じやう}から
% \ruby{根性}{こん|じやう}から
\ruby{云}{い}へば、
%
あれだけの
\ruby[g]{容貌}{きりやう}をもつて
\ruby{居}{ゐ}ながら、
%
\ruby{何}{なん}だつて
\ruby[g]{敎師}{けうし }
なんぞになつて
\ruby{居}{ゐ}るだらう、
%
と
\ruby[g]{蔭口}{かげぐち}も
\ruby{云}{い}はれ
\ruby{{\換字{兼}}}{かね}ない
\ruby{女}{をんな}ぶり
ださうさ。
%
\換字{志}かも
%
\ruby[g]{容貌}{きりやう}の
\ruby{佳}{い}い
\ruby{奴}{やつ}は
\ruby[g]{十人}{じふにん}が
\ruby[g]{八人}{はちにん}まで、
%
\ruby[g]{兎角}{と かく}
\ruby[g]{他人}{た にん}に
\ruby{甘}{あま}つたれるやうな
\ruby[g]{調子}{てうし }が
あつて、
%
\ruby[g]{學問}{がくもん}なぞは
\ruby{得}{え}て
\原本頁{32-1}%
\ruby[g]{出來}{で き }ないが、
%
\ruby[g]{中々}{なか〳〵}
\ruby[||j>]{其}{その}
\ruby[||j>]{女}{をんな}は
% \ruby{其女}{その|をんな}は
\ruby{能}{よ}く
\ruby[g]{出來}{で き }る
\ruby{上}{うへ}、
%
それこそ
\ruby[g]{日方}{ひ かた}の
\ruby{云}{い}ひ
\原本頁{32-2}\改行%
\ruby{草}{ぐさ}ぢやあ
\ruby{無}{な}いが、
%
いつでも
\ruby[g]{現在}{げんざい}に
\ruby[g]{滿足}{まんぞく}しないで、
%
\ruby[g]{永久}{えいきう}に
\ruby{{\換字{進}}}{すゝ}んで
\原本頁{32-3}\改行%
\ruby{{\換字{飽}}}{あ}くことを
\ruby{知}{し}らぬ
\ruby{歟}{か}、
%
\ruby[g]{{\換字{感}}心}{かんしん}に
\ruby[g]{自{\換字{分}}}{じ ぶん}は
\ruby[g]{自{\換字{分}}}{じ ぶん}の
\ruby[||j>]{勉}{べん}
\ruby[||j>]{{\換字{強}}}{きやう}を
% \ruby{勉{\換字{強}}}{べん|きやう}を
\ruby{仕}{し}て
\ruby{居}{ゐ}るさうだ。
%
\換字{志}て
\ruby{見}{み}りやあ
\ruby[g]{容貌}{きりやう}も
\ruby{佳}{よ}いし、
%
\ruby[||j>]{心}{こゝろ}
\ruby[||j>]{掛}{ がけ}も
% \ruby{心掛}{こゝろ|がけ}も
\ruby{可}{よ}いし、
%
\ruby{別}{べつ}に
\ruby{{\換字{難}}}{なん}は
\ruby{無}{な}い
\原本頁{32-5}\改行%
\ruby{女}{をんな}なんだ。
%
\ruby[g]{左樣}{さ う }いふ
\ruby{女}{をんな}が
\ruby{現}{あらは}れたので、
%
\ruby[g]{學校}{がくかう}の
\ruby{内}{うち}でも
\ruby{外}{そと}でも
\ruby{珍}{めづ}らしがつて、
%
\ruby{何}{なん}とか
\ruby{彼}{か}とか
\ruby[||j>]{{\換字{評}}}{ひやう}
\ruby[||j>]{{\換字{判}}}{ ばん}が
% \ruby{{\換字{評}}{\換字{判}}}{ひやう|ばん}が
\ruby{立}{た}つて
\ruby{居}{ゐ}たが、
%
\ruby{其}{その}
\ruby{内}{うち}に
\ruby[g]{水野}{みづの }が
\ruby{{\換字{迷}}}{まよ}ひ
\ruby{出}{だ}した。
%
\ruby[g]{何樣}{ど う }いふ
\ruby[g]{機會}{は め }から
\ruby[g]{水野}{みづの }の
\ruby{心}{こゝろ}が
\ruby[||j>]{其}{その}
\ruby[||j>]{女}{をんな}に
% \ruby{其女}{その|をんな}に
\ruby{傾}{かたむ}いたかは
\ruby{解}{わか}らないが、
%
\ruby[g]{乃公}{お れ }が
\ruby{思}{おも}ふにやあ
\ruby{別}{べつ}な
\ruby{事}{こと}はない。
%
\ruby{淨瑠璃}{じやう|る|り}の
\ruby[g]{{\換字{文}}句}{もんく }にある
\原本頁{32-9}\改行%
\ruby{{\換字{通}}}{とほ}り、
%
\ruby[g]{琥珀}{こ はく}の
\ruby{塵}{ちり}や
\ruby[g]{磁石}{じしやく}の
\ruby{針}{はり}で、
% 「琥珀の塵」... 琥珀には軽い塵(ちり)を吸いつける性質がある からかしら?
% 「神霊矢口渡(しんれいやぐちのわたし)」https://www.aozora.gr.jp/cards/001224/files/46628_29247.html
% お船のクドキのなかにある『右よ左と附廻つけまわす、琥珀の塵や磁石の針』という台詞から
\ruby{眼}{め}に
\ruby{見}{み}えて
\ruby[g]{何處}{ど こ }が
\ruby[g]{何樣}{ど う }と
いふ
\ruby{事}{こと}は
\ruby{無}{な}いが、
%
たゞ
\ruby{譯}{わけ}も
\ruby{無}{な}く
\ruby{引}{ひ}き
\ruby{寄}{よ}せられて、
%
\ruby[g]{心が}{こゝろ }
\ruby[g]{其處}{そ こ }へ
\ruby{行}{ゆ}くのが
\原本頁{32-11}\改行%
\ruby{戀}{こひ}の
\ruby{{\換字{習}}}{なら}ひだ。
%
こりあ
\ruby[g]{俗物}{ぞくぶつ}でも
\ruby[g]{仙骨}{せんこつ}でも
\ruby{同}{おな}じ
\ruby{事}{こと}、
%
いくら
\ruby[g]{水野}{みづの }が
\makeatletter
\@ifundefined{デバッグ@ビルド}{%
  \ruby[||j>]{俊}{すぐれ}
  \ruby[||j>]{才}{ もの}だつて、
}{%
  \ruby[<j||]{俊}{すぐれ}% 行末行頭の境界付近なので特例処置を施す
  \原本頁{33-1}\改行%
  \ruby[||j>]{才}{もの}だつて、
}%
\makeatother
% \ruby{俊才}{すぐれ|もの}だつて、
%
\原本頁{33-1}%
\ruby[g]{生血}{なまち }を
\ruby{包}{つゝ}んだ
\ruby[g]{五尺}{ごしやく}の
\ruby[g]{身體}{からだ }を、
%
\ruby{抱}{かゝ}へて
\ruby{居}{ゐ}るのだもの
\ruby[g]{無理}{む り }も
\ruby{無}{な}い、
%
\ruby[g]{矢張}{やつぱ }り
\ruby[g]{年齡}{と し }が
\ruby[g]{年齡}{と し }だから
\ruby{{\換字{迷}}}{まよ}つたんだらう。
%
\換字{志}かし
\ruby[g]{相手}{あひて }も
\ruby[||j>]{商}{しやう}
\ruby[||j>]{賣}{ ばい}
\ruby[||j>]{人}{ にん}ぢやあ
\ruby{無}{な}し、
%
\ruby[g]{水野}{みづの }も
\ruby[g]{獨身}{ひとりみ}で
\ruby{居}{ゐ}なけりやあ
ならぬ
といふので
\ruby{無}{な}いから、
%
\ruby{全}{まつた}く
\ruby{深}{ふか}く
\ruby{思}{おも}ひ
\ruby{{\換字{込}}}{こ}んだものならば、
%
\ruby{緣}{えん}を
\ruby{纏}{まと}めりやあ
\ruby{其}{それ}で
\ruby{可}{い}いのだが、
%
さあ、
%
\ruby[g]{水野}{みづの }の
\ruby{不仕合}{ふ|し|あはせ}といふのは
\ruby[g]{其處}{そ こ }の
\ruby{事}{こと}
\原本頁{33-6}\改行%
で、
%
\ruby{俗}{ぞく}にいふ
\ruby{蟲}{むし}が
\ruby{{\換字{嫌}}}{きら}ふと
いふものでゞもあらうか、
%
\ruby[||j>]{其}{その}
\ruby[||j>]{女}{をんな}が
% \ruby{其女}{その|をんな}が
\ruby[g]{水野}{みづの }
\原本頁{33-7}\改行%
の
\ruby[g]{眞心}{まごゝろ}を
\ruby{受}{う}け
\ruby{納}{い}れぬので、
%
それで
\ruby[g]{水野}{みづの }は
\ruby[g]{懊惱}{あうなう}して
\ruby{居}{ゐ}ると
いふのだ。
%
もつとも
\ruby[g]{水野}{みづの }が
\ruby{明}{あか}らさまに、
%
\ruby[||j>]{其}{その}
\ruby[||j>]{女}{をんな}に
% \ruby{其女}{その|をんな}に
\ruby[g]{何事}{なにごと}を
\ruby{云}{い}つたでも
あるまいから、
\ruby[||j>]{其}{その}
\ruby[||j>]{女}{をんな}も
% \ruby{其女}{その|をんな}も
\ruby[g]{水野}{みづの }に
\ruby{明}{あか}らさまに
\ruby[g]{何事}{なにごと}を
\ruby{云}{い}つた
でもあるまい
\原本頁{33-10}\改行%
が、
%
これは
\ruby[g]{世間}{せ けん}に
\ruby{老}{お}いた
\ruby[g]{山路}{やまぢ }の
\ruby[g]{老夫}{ぢゞい }が、
%
\ruby[g]{水野}{みづの }の
\ruby[g]{樣子}{やうす }を
\ruby{見}{み}て
\ruby{察}{さつ}しての
\ruby{話}{はなし}だ。
%
さて
\ruby{其}{それ}にしたところで
\ruby[g]{其限}{それぎ }りの
\ruby{事}{こと}なら、
%
\ruby[g]{芥火}{あくたび}の
% 芥火(あくたび)とは? 意味や使い方 https://kotobank.jp › word › 芥火-195168
% 〘 名詞 〙 ごみや、ちりを焼く火。特に、漁夫が、流れついた藻芥(もあくた)を集めてたく火。
\ruby{燃}{も}えるやうに
ぶすりぶすりと、% 原本通り非踊り字表記
%
\原本頁{34-1}%
\ruby[g]{水野}{みづの }が
\ruby{物}{もの}を
\ruby{思}{おも}つて
\ruby{居}{ゐ}るだけで
\ruby{濟}{す}むのだが、
%
こゝに
\ruby{其}{そ}の
\ruby{五十子}{い|そ|こ}の
\ruby{親}{おや}に
お
\ruby{關}{せき}といふ、
%
\ruby[g]{可憎}{い や }な
\ruby[g]{{\換字{強}}欲}{がうよく}な
\ruby[g]{惡婆}{あくば }
\原本頁{34-3}\改行%
がある。
%
\ruby[g]{勿論}{もちろん}
\ruby{生}{うみ}の
\ruby{母}{はゝ}では
\ruby{無}{な}くつて、
%
\ruby{五十子}{い|そ|こ}とは
\ruby[g]{別々}{べつ〳〵}に
\ruby{住}{す}んで
\ruby{居}{ゐ}るほど、
%
\ruby[g]{氣性}{きしやう}も
\ruby{合}{あ}はねば
\ruby{仲}{なか}も
\ruby{惡}{わる}いのだが、
%
\ruby[g]{時々}{とき〴〵}
\ruby{五十子}{い|そ|こ}の
ところ
\原本頁{34-5}\改行%
へ
\ruby{來}{き}ては
\ruby[g]{無理}{む り }を
\ruby{云}{い}つて、
%
\ruby{無}{な}け
\ruby{無}{な}しの
\ruby{金}{かね}を
\ruby{絞}{しぼ}つて
\ruby{行}{ゆ}く。
%
\ruby[g]{其奴}{そ いつ}が
\ruby[g]{水野}{みづの }の
\ruby{腹}{はら}を
\ruby{見}{み}て
\ruby{取}{と}つて、
%
\ruby{其}{そ}の
\ruby[g]{初心}{う ぶ }な
ところに
\ruby{付}{つ}け
\ruby{{\換字{込}}}{こ}んで、
%
いろいろ
さまざまな
\ruby{事}{こと}を
\ruby{云}{い}ひ
\ruby{散}{ち}らしちやあ、
%
つまり
\ruby[g]{幾干}{いくら }かづゝ
\ruby{捲}{ま}き
\ruby{上}{あ}げるさうだ。
%
\ruby{金}{かね}は
\ruby[g]{些少}{わづか }の
\ruby{事}{こと}だから
\ruby[g]{仔細}{し さい}は
\ruby{無}{な}いが、
%
\ruby{金}{かね}を
\ruby{取}{と}らう
\ruby{爲}{ため}
\原本頁{34-9}\改行%
ばつかりに
\ruby{其}{その}
\ruby[||j>]{婆}{ばゞあ}めが、
%
\ruby{好}{い}い
\ruby[g]{加減}{か げん}な
\ruby{事}{こと}を
\ruby{云}{い}つて
\ruby{煽}{あふ}り
\ruby{立}{た}つて
\ruby{燃}{も}え
\ruby{立}{た}たする。
%
ところが
\ruby[g]{一方}{いつぱう}ぢやあ
\ruby{{\換字{又}}}{また}、
%
\ruby[g]{肝心}{かんじん}の
\ruby{人}{ひと}に
よそ〳〵しく
\ruby{冷}{ひや}つこく
\ruby[g]{待{\換字{遇}}}{あしら }はれる。
%
\ruby{火}{ひ}に
あひ
\ruby{水}{みづ}に
あふのだから
\ruby{敵}{かな}はない、
%
\ruby[g]{水野}{みづの }の
\原本頁{35-1}\改行%
\ruby{心}{こゝろ}の
\ruby[g]{靜穩}{しづか }なことは、
%
\ruby{今}{いま}は
\ruby[g]{一時}{いつとき}でも
\ruby{有}{あ}りさうも
\ruby{無}{な}い
\ruby{譯}{わけ}。
%
そこで
\ruby{今}{いま}までの
\ruby[g]{行狀}{みもち }とは
\ruby{打}{う}つて
\ruby{變}{かは}つて、
%
\ruby{家}{うち}に
\ruby{居}{ゐ}る
\ruby{時}{とき}は
\ruby[g]{鬱々}{うつ〳〵}として、
%
たゞ
\原本頁{35-3}\改行%
\ruby{沈}{しづ}みきつて
\ruby{物}{もの}も
\ruby{言}{い}はず、
%
\ruby{机}{つくゑ}に
\ruby{對}{むか}つても
\ruby{書}{ほん}は
\ruby{讀}{よ}まずに、
%
\ruby{長太息}{ため|い|き}を
\原本頁{35-4}\改行%
\ruby{吐}{つ}く
\ruby{時}{とき}のみ
\ruby{多}{おほ}く、
%
\ruby{{\換字{朝}}}{あさ}は
\ruby{心}{こゝろ}よく
\ruby{起}{お}きる
\ruby{日}{ひ}も
\ruby{無}{な}く、
%
\ruby{夜}{よ}も
\ruby[g]{寐苦}{ね ぐる}しく
\ruby{{\換字{過}}}{すご}すさうだ。
%
これは
\ruby[g]{乃公}{お れ }が
\ruby[g]{老夫}{おやぢ }から
\ruby{聞}{き}いたゞけで、
%
\ruby[g]{無論}{む ろん}
\ruby[g]{山路}{やまぢ }の
\ruby[g]{老夫}{おやぢ }の
つもりでは、
%
\ruby[g]{乃公}{お ら }に
\ruby[g]{意見}{い けん}して
\ruby{{\換字{遣}}}{や}れと
いふのだつた。
%
\換字{志}かし
\原本頁{35-7}\改行%
\ruby[g]{乃公}{お れ }は
\ruby[g]{乃公}{お れ }の
\ruby[<j>]{考}{かんがへ}で、
%
\ruby[g]{水野}{みづの }のためには
\ruby[g]{幾干}{いくら }でも、
%
\ruby[g]{盡力}{つ く }したいと
\ruby{思}{おも}つて
\ruby{居}{ゐ}ることは
\ruby{思}{おも}つて
\ruby{居}{ゐ}るが、
%
\ruby[g]{意見}{い けん}を
\ruby{仕}{し}て
\ruby[g]{利益}{た め }に
なりさうな
\ruby{筋}{すぢ}
\原本頁{35-9}\改行%
では
\ruby{無}{な}いと、
%
\ruby[g]{見切}{み き }つて
つい
\ruby{其}{その}
\ruby{儘}{まゝ}に
\ruby{{\換字{過}}}{す}ごして
\ruby{來}{き}たのだ。
』% 其三での島木の最後の語りが終えたところ

\原本頁{35-10}%
\ruby{辛}{から}くも
\ruby[g]{此時}{こ ゝ }まで
\ruby{堪}{こら}へたりし
\ruby[g]{日方}{ひ かた}は
\ruby{再}{ふたゝ}び
\ruby{叫}{さけ}び
\ruby{出}{いだ}しぬ。

\原本頁{35-11}%
『
\ruby[g]{何故}{な ぜ }
\ruby[g]{意見}{い けん}を
\ruby{仕}{し}ても
\ruby[g]{利益}{た め }にならん?。
%
\ruby[g]{意見}{い けん}を
\ruby[g]{仕無}{し な }いで
\ruby{何}{なん}と
\ruby{爲}{す}るんだ?。
%
\原本頁{36-1}%
\ruby[g]{何樣}{ど う }して
\ruby[g]{水野}{みづの }の
\ruby{爲}{ため}に
\ruby[g]{盡力}{つ く }す?。
』

\原本頁{36-2}%
『
\ruby[g]{乃公}{お ら }あ
\ruby[g]{出來}{で き }る
\ruby{事}{こと}なら
\ruby[g]{水野}{みづの }の
\ruby{思}{おも}ひの、
%
\ruby{徹}{とほ}るやうに
\ruby{爲}{し}て
\ruby{{\換字{遣}}}{や}らうと
\ruby{思}{おも}つて
\ruby{居}{ゐ}るのだ。
』

\原本頁{36-4}%
『
\ruby{何}{なん}だと、
%
\ruby{馬鹿野郎}{ば|か|や|らう}ツ!、
%
\ruby{愚}{ぐ}にもつかん!。
%
そんな
\ruby{下}{くだ}らんことがあるものか、
%
\ruby[g]{貴樣}{き さま}は
\ruby[g]{一體}{いつたい}
\ruby[g]{腐敗}{ふ はい}して
\ruby{居}{ゐ}る!。
』

\原本頁{36-6}%
『
また
\ruby[g]{馬鹿}{ば か }
\ruby{呼}{よば}はりを
するナ!。
%
\ruby{汝}{きさま}こそ
\ruby[g]{馬鹿}{ば か }だ。
%
\ruby[g]{意見}{い けん}して
\ruby{役}{やく}に
\ruby{立}{た}つ
\ruby{位}{くらゐ}なら
\ruby[g]{乃公}{お れ }が
\ruby{爲}{す}るは。
%
\ruby{人}{ひと}は
\ruby[g]{銘々}{めい〳〵}に
\ruby[g]{{\換字{所}}考}{かんがへ}が
ある。
%
\ruby[g]{乃公}{お れ }は
\ruby[g]{乃公}{お れ }、
%
\ruby{汝}{きさま}は
\ruby{汝}{きさま}で
\ruby[g]{可矣}{い ゝ }ぢやあ
\ruby{無}{ね}えか。
%
\ruby[g]{意見}{い けん}が
\ruby{仕}{し}たけりやあ
\ruby[||j>]{汝}{きさま}
\ruby[||j>]{爲}{ し}ろ。
』

\原本頁{36-9}%
『
\ruby[g]{勿論}{もちろん}だ。
%
\ruby{諫}{いさ}めて
\ruby{{\換字{遣}}}{や}らないで
\ruby[g]{何樣}{ど う }するものか。
%
\ruby{女}{をんな}が
\ruby{美}{よ}くつても
\ruby{惡}{わる}くつても、
%
\ruby{何}{なん}だ!、
%
\ruby{女}{をんな}が!。
%
\ruby{苟}{いやし}くも
\ruby{大{\換字{丈}}夫}{だい|ぢやう|ぶ}たるものが
\ruby{高}{たか}が
\ruby{一}{いつ}
\ruby[g]{{\換字{婦}}人}{ぷ じん}に、
%
\ruby[<j>]{志}{こゝろざし}を% 原本では、「志」前後にアキを入れている
\ruby{喪}{うしな}ふとは
\ruby{何}{なん}たる
\ruby{事}{こつ}た。
%
\ruby{實}{じつ}に
\ruby{怪}{け}しからん、
%
はがゆい
\ruby{奴}{やつ}だ。
%
\原本頁{37-1}%
\ruby[g]{是非}{ぜ ひ }
\ruby{{\換字{尋}}}{たづ}ねて
\ruby{行}{い}つて
\ruby{大}{おほい}に
\ruby{諫}{いさ}める。
』

\原本頁{37-2}%
\ruby[g]{二人}{ふたり }の
\ruby[g]{問答}{もんだふ}は
こゝに
\ruby{已}{や}んで、
%
\ruby[g]{山瀬}{やませ }は
\ruby{爽}{さわ}やかに
\ruby{口}{くち}を
\ruby{開}{ひら}きぬ。

\原本頁{37-3}%
『
\ruby{僕}{ぼく}は
\ruby[g]{他人}{ひ と }の
\ruby[g]{意志}{い し }
\ruby[||j>]{{\換字{感}}}{かん}
\ruby[||j>]{{\換字{情}}}{じやう}の
% \ruby{{\換字{感}}{\換字{情}}}{かん|じやう}の
\ruby[g]{自由}{じ いう}を
\ruby[||j>]{{\換字{尊}}}{そん}
\ruby[||j>]{重}{ちよう}するから、
% \ruby{{\換字{尊}}重}{そん|ちよう}するから、
%
\ruby[g]{立入}{たちい }つては
\ruby{敢}{あへ}て
\ruby[g]{兎角}{と かく}を
\ruby{言}{い}はぬ。
%
\換字{志}かし
これは
\ruby[g]{水野}{みづの }
\ruby{君}{くん}のために
\ruby{不利益}{ふ|り|えき}と
\ruby{思}{おも}ふから
\改行% 校正作業の簡略化のため
、
%
\原本頁{37-5}\改行%
\ruby[g]{一應}{いちおう}は
\ruby[||j>]{忠}{ちゆう}% 原本通り(ちゆう)(国会図書館 コマ番号 22/134 p37 l5)
\ruby[||j>]{告}{ こく}を
% \ruby{忠告}{ちゆう|こく}を
\ruby{試}{こゝろ}みるつもりだ。
』

\原本頁{37-6}%
\ruby{人}{ひと}
\ruby{皆}{みな}
\ruby{語}{かた}れども
\ruby[g]{羽{\換字{勝}}}{は がち}は
\ruby{語}{かた}らず、
%
たゞ
\ruby{僅}{わづか}に
\ruby[g]{吁然}{ほ つ }と
\ruby{息}{いき}つけば、
%
\ruby{手}{て}にせし
\ruby{卷{\換字{煙}}草}{た|ば|こ}の
\ruby{{\換字{灰}}}{はい}の% ルビ調整(原本遠り)(はひ)
\ruby{長}{なが}く
\ruby{續}{つゞ}けるが、
%
ぼたりと
\ruby{膝}{ひざ}の
\ruby{上}{うへ}に
\ruby{落}{お}ちて
\ruby{脆}{もろ}く
\ruby{散}{ち}つたり。

\原本頁{37-9}%
\ruby[g]{夜色}{やしよく}は
\ruby[||j>]{樓}{ろう}
\ruby[||j>]{外}{ぐわい}に
% \ruby{樓外}{ろう|ぐわい}に
\ruby[g]{沈々}{ちん〳〵}として、
%
\ruby{澄}{す}みわたりたる
\ruby{天}{そら}に
かゝれる
\ruby[g]{星斗}{ほ し }は
\ruby[g]{爛然}{らんぜん}と
\ruby{明}{あき}らかに、
%
\ruby[g]{明日}{あ す }は
\ruby{風}{かぜ}にや
\ruby{其}{そ}の
\ruby{大}{おほき}なるは、
%
いづれも
\ruby[g]{煌々}{ひか〳〵}と
\ruby[g]{瞬目}{めはじき}して、
%
\ruby[g]{光の}{ひかり }
\ruby{芒}{のぎ}は
\ruby{搖}{ゆら}ぎに
\ruby{搖}{ゆら}げり。
