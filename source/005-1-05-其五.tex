\Entry{其五}
% メモ 校正終了 2024-03-16

\原本頁{30-10}%
\ruby[g]{老夫}{おやぢ}の
\ruby{談話}{はな|し}を
\ruby{聞}{き}いて
\ruby{見}{み}りやあ
\ruby[g]{水野}{みづの}は
\ruby{實}{じつ}に
\ruby{憫然}{かは|いさう}だ。% 「憫然 か(は)いさう」
%
\ruby{勿論}{もち|ろん}
\ruby{其}{そ}の
\ruby[g]{老夫}{おやぢ}の
\ruby{云}{い}つたことが
\ruby{一}{いち}から
\ruby{十}{じう}まで
\ruby[g]{眞實}{ほんと}とも
\ruby{限}{かぎ}るまいが、
\原本頁{31-1}%
%
\ruby{岡目}{をか|め}の
\ruby{{\換字{評}}{\換字{判}}}{ひやう|ばん}なり
\ruby{老夫}{とし|より}の
\ruby{言葉}{こと|ば}なり、
%
\ruby{大體}{だい|たい}は
\ruby{{\換字{違}}}{ちが}ふ
\ruby{氣{\換字{遣}}}{きづ|かひ}は
あるまい。
%
そも〳〵は
\ruby{今年}{こ|とし}の
\ruby{春}{はる}の
\ruby{始}{はじめ}、
%
\ruby[g]{水野}{みづの}の
\ruby{出}{で}て
\ruby{居}{ゐ}る
\ruby{學校}{がく|かう}の
\ruby{女敎師}{ぢよ|けう|し}が
\ruby{一人}{ひと|り}
\ruby{故鄕}{く|に}へ
\ruby{歸}{かへ}つたので
\ruby{闕員}{けつ|いん}が
\ruby{出來}{で|き}た、
%
\ruby{其}{そ}の
\ruby{補闕}{ほ|けつ}として
\ruby{新}{あらた}に
\ruby{來}{き}たのが、
%
まだ
\ruby{敎員}{けう|ゐん}になりたての、
%
\ruby{年}{とし}の
\ruby{{\換字{若}}}{わか}い
\ruby{岩崎}{いは|さき}% 原本のこの部分は「いわさき」
\ruby[g]{五十子}{いそこ}といふ
\ruby{女}{をんな}だつた。
%
\ruby[g]{老夫}{おやぢ}も
\ruby{度々}{たび|〳〵}
\ruby{見}{み}て
\ruby{知}{し}つているさうだが、
%
\ruby{極}{ごく}
\ruby{可愛}{か|あい}らしい
\ruby{惚}{ほ}れ〴〵すると
いふやうな
\ruby{顏立}{かほ|だち}では
\ruby{無}{な}いけれど、
%
\ruby{眼}{め}の
\ruby{淸}{すゞ}しい
\ruby{鼻}{はな}の
\ruby{高}{たか}い
\ruby{端然}{しや|ん}とした
\ruby{女}{をんな}で、
%
まあ
\ruby{當世}{たう|せい}の
\ruby{下司}{げ|す}
\ruby{根性}{こん|じやう}から
\ruby{云}{い}へば、
%
あれだけの
\ruby{容貌}{きり|やう}をもつて
\ruby{居}{ゐ}ながら、
%
\ruby{何}{なん}だつて
\ruby{敎師}{けう|し}
なんぞになつて
\ruby{居}{ゐ}るだらう、
%
と
\ruby{蔭口}{かげ|ぐち}も
\ruby{云}{い}はれ
\ruby{{\換字{兼}}}{かね}ない
\ruby{女}{をんな}ぶり
ださうさ。
%
\換字{志}かも、
%
\ruby{容貌}{きり|やう}の
\ruby{佳}{い}い
\ruby{奴}{やつ}は
\ruby{十人}{じう|にん}が
\ruby{八人}{はち|にん}まで、
%
\ruby{兎角}{と|かく}
\ruby{他人}{た|にん}に
\ruby{甘}{あま}つたれるやうな
\ruby{調子}{てう|し}が
あつて、
%
\ruby{學問}{がく|もん}なぞは
\ruby{得}{え}て
\原本頁{32-1}%
\ruby{出來}{で|き}ないが、
%
\ruby{中々}{なか|〳〵}
\ruby{其女}{その|をんな}は
\ruby{能}{よ}く
\ruby{出來}{で|き}る
\ruby{上}{うへ}、
%
それこそ
\ruby[g]{日方}{ひかた}の
\ruby{云}{い}ひ
\ruby{草}{ぐさ}ぢやあ
\ruby{無}{な}いが、
%
いつでも
\ruby{現在}{げん|ざい}に
\ruby{滿足}{まん|ぞく}しないで、
%
\ruby{永久}{えい|きう}に
\ruby{{\換字{進}}}{すゝ}んで
\ruby{{\換字{飽}}}{あ}くことを
\ruby{知}{し}らぬ
\ruby{歟}{か}、
%
\ruby{感心}{かん|しん}に
\ruby{自{\換字{分}}}{じ|ぶん}は
\ruby{自{\換字{分}}}{じ|ぶん}の
\ruby{勉{\換字{強}}}{べん|きやう}を
\ruby{仕}{し}て
\ruby{居}{ゐ}るさうだ。
%
\換字{志}て
\ruby{見}{み}りやあ
\ruby{容貌}{きり|やう}も
\ruby{佳}{よ}いし、
%
\ruby{心掛}{こゝろ|がけ}も
\ruby{可}{よ}いし、
%
\ruby{別}{べつ}に
\ruby{難}{なん}は
\ruby{無}{な}い
\ruby{女}{をんな}なんだ。
%
\ruby{左樣}{さ|う}いふ
\ruby{女}{をんな}が
\ruby{現}{あらは}れたので、
%
\ruby{學校}{がく|かう}の
\ruby{内}{うち}でも
\ruby{外}{そと}でも
\ruby{珍}{めづ}らしがつて、
%
\ruby{何}{なん}とか
\ruby{彼}{か}とか
\ruby{{\換字{評}}{\換字{判}}}{ひやう|ばん}が
\ruby{立}{た}つて
\ruby{居}{ゐ}たが、
%
\ruby{其内}{その|うち}に
\ruby[g]{水野}{みづの}が
\ruby{{\換字{迷}}}{まよ}ひ
\ruby{出}{だ}した。
%
\ruby{何樣}{ど|う}いふ
\ruby{機會}{は|め}から
\ruby[g]{水野}{みづの}の
\ruby{心}{こゝろ}が
\ruby{其女}{その|をんな}に
\ruby{傾}{かたむ}いたかは
\ruby{解}{わか}らないが、
%
\ruby{乃公}{お|れ}が
\ruby{思}{おも}ふにやあ
\ruby{別}{べつ}な
\ruby{事}{こと}はない。
%
\ruby{淨瑠璃}{じやう|る|り}の
\ruby{{\換字{文}}句}{もん|く}にある
\ruby{{\換字{通}}}{とほ}り、
%
\ruby{琥珀}{こ|はく}の
\ruby{塵}{ちり}や
\ruby{磁石}{じ|しやく}の
\ruby{針}{はり}で、
%
\ruby{眼}{め}に
\ruby{見}{み}えて
\ruby{何處}{ど|こ}が
\ruby{何樣}{ど|う}と
いふ
\ruby{事}{こと}は
\ruby{無}{な}いが、
%
たゞ
\ruby{譯}{わけ}も
\ruby{無}{な}く
\ruby{引}{ひ}き
\ruby{寄}{よ}せられて、
%
\ruby{心}{こゝろ}が
\ruby{其處}{そ|こ}へ
\ruby{行}{ゆ}くのが
\ruby{戀}{こひ}の
\ruby{{\換字{習}}}{なら}ひだ。
%
こりあ
\ruby{俗物}{ぞく|ぶつ}でも
\ruby{仙骨}{せん|こつ}でも
\ruby{同}{おな}じ
\ruby{事}{こと}、
%
いくら
\ruby[g]{水野}{みづの}が
\ruby{俊才}{すぐれ|もの}だつて、
%
\原本頁{33-1}%
\ruby{生血}{なま|ち}を
\ruby{包}{つゝ}んだ
\ruby{五尺}{ご|しやく}の
\ruby{身體}{から|だ}を、
%
\ruby{抱}{かゝ}へて
\ruby{居}{ゐ}るのだもの
\ruby{無理}{む|り}も
\ruby{無}{な}い、
%
\ruby{矢張}{やつ|ぱ}り
\ruby{年齡}{と|し}が
\ruby{年齡}{と|し}だから
\ruby{{\換字{迷}}}{まよ}つたんだらう。
%
\換字{志}かし
\ruby{相手}{あひ|て}も
\ruby{商賣人}{しやう|ばい|にん}ぢあ
\ruby{無}{な}し、
%
\ruby[g]{水野}{みづの}も
\ruby{獨身}{ひとり|み}で
\ruby{居}{ゐ}なけりあ
ならぬといふので
\ruby{無}{な}いから、
%
\ruby{全}{まつた}く
\ruby{深}{ふか}く
\ruby{思}{おも}ひ
\ruby{{\換字{込}}}{こ}んだものならば、
%
\ruby{緣}{えん}を
\ruby{纏}{まと}めりやあ
\ruby{其}{それ}で
\ruby{可}{い}いのだが、
%
さあ、
%
\ruby[g]{水野}{みづの}の
\ruby{不仕合}{ふ|し|あはせ}といふのは
\ruby{其處}{そ|こ}の
\ruby{事}{こと}で、
%
\ruby{俗}{ぞく}にいふ
\ruby{蟲}{むし}が
\ruby{{\換字{嫌}}}{きら}ふと
いふものでゞもあらうか、
%
\ruby{其女}{その|をんな}が
\ruby[g]{水野}{みづの}の
\ruby{眞心}{ま|ごゝろ}を
\ruby{受}{う}け
\ruby{納}{い}れぬので、
%
それで
\ruby[g]{水野}{みづの}は
\ruby{懊惱}{あう|なう}して
\ruby{居}{ゐ}ると
いふのだ。
%
もつとも
\ruby[g]{水野}{みづの}が
\ruby{明}{あか}らさまに、
%
\ruby{其女}{その|をんな}に
\ruby{何事}{なに|ごと}を
\ruby{云}{い}つたでも
あるまいから、
\ruby{其女}{その|をんな}も
\ruby[g]{水野}{みづの}に
\ruby{明}{あか}らさまに
\ruby{何事}{なに|ごと}を
\ruby{云}{い}つたでも
あるまいが、
%
これは
\ruby{世間}{せ|けん}に
\ruby{老}{お}いた
\ruby[g]{山路}{やまぢ}の
\ruby[g]{老夫}{ぢゞい}が、
%
\ruby[g]{水野}{みづの}の
\ruby{樣子}{やう|す}を
\ruby{見}{み}て
\ruby{察}{さつ}しての
\ruby{話}{はなし}だ。
%
さて
\ruby{其}{それ}にしたところで
\ruby{其限}{それ|ぎ}りの
\ruby{事}{こと}なら、
%
\ruby{芥火}{あくた|び}の
\ruby{燃}{も}えるやうに
ぶすりぶすりと、% 原本は踊り字にしていない
%
\原本頁{34-1}%
\ruby[g]{水野}{みづの}が
\ruby{物}{もの}を
\ruby{思}{おも}つて
\ruby{居}{ゐ}るだけで
\ruby{濟}{す}むのだが、
%
こゝに
\ruby{其}{そ}の
\ruby[g]{五十子}{いそこ}の
\ruby{親}{おや}に
お
\ruby{關}{せき}といふ、
%
\ruby{可憎}{い|や}な
\ruby{{\換字{強}}欲}{がう|よく}な
\ruby{惡婆}{あく|ば}がある。
%
\ruby{勿論}{もち|ろん}
\ruby{生}{うみ}の
\ruby{母}{はゝ}では
\ruby{無}{な}くつて、
%
\ruby[g]{五十子}{いそこ}とは
\ruby{別々}{べつ|〳〵}に
\ruby{住}{す}んで
\ruby{居}{ゐ}るほど、
%
\ruby{氣性}{き|しやう}も
\ruby{合}{あ}はねば
\ruby{仲}{なか}も
\ruby{惡}{わる}いのだが、
%
\ruby{時々}{とき|〴〵}
\ruby[g]{五十子}{いそこ}の
ところへ
\ruby{來}{き}ては
\ruby{無理}{む|り}を
\ruby{云}{い}つて、
%
\ruby{無}{な}け
\ruby{無}{な}しの
\ruby{金}{かね}を
\ruby{絞}{しぼ}つて
\ruby{行}{ゆ}く。
%
\ruby{其奴}{そ|いつ}が
\ruby[g]{水野}{みづの}の
\ruby{腹}{はら}を
\ruby{見}{み}て
\ruby{取}{と}つて、
%
\ruby{其}{そ}の
\ruby{初心}{う|ぶ}な
ところに
\ruby{付}{つ}け
\ruby{{\換字{込}}}{こ}んで、
%
いろいろ
さまざまな
\ruby{事}{こと}を
\ruby{云}{い}ひ
\ruby{散}{ち}らしちやあ、
%
つまり
\ruby{幾干}{いく|ら}かづゝ
\ruby{捲}{ま}き
\ruby{上}{あ}げるさうだ。
%
\ruby{金}{かね}は
\ruby{些少}{わづ|か}の
\ruby{事}{こと}だから
\ruby{仔細}{し|さい}は
\ruby{無}{な}いが、
%
\ruby{金}{かね}を
\ruby{取}{と}らう
\ruby{爲}{ため}ばつかりに
\ruby{其}{その}
\ruby[|j>]{婆}{ばゞあ}めが、
%
\ruby{好}{い}い
\ruby{加減}{か|げん}な
\ruby{事}{こと}を
\ruby{云}{い}つて
\ruby{煽}{あふ}り
\ruby{立}{た}つて
\ruby{燃}{も}え
\ruby{立}{た}たする。
%
ところが
\ruby{一方}{いつ|ぱう}ぢやあ
\ruby{{\換字{又}}}{また}、
%
\ruby{肝心}{かん|じん}の
\ruby{人}{ひと}に
よそ〳〵しく
\ruby{冷}{ひや}つこく
\ruby{待{\換字{遇}}}{あし|ら}はれる。
%
\ruby{火}{ひ}に
あひ
\ruby{水}{みづ}に
あふのだから
\ruby{敵}{かな}はない、
%
\ruby[g]{水野}{みづの}の
\ruby{心}{こゝろ}の
\ruby{靜穩}{しづ|か}なことは、
%
\ruby{今}{いま}は
\ruby{一時}{いつ|とき}でも
\ruby{有}{あ}りさうも
\ruby{無}{な}い
\ruby{譯}{わけ}。
%
そこで
\ruby{今}{いま}までの
\ruby{行狀}{みも|ち}とは
\ruby{打}{う}つて
\ruby{變}{かは}つて、
%
\ruby{家}{うち}に
\ruby{居}{ゐ}る
\ruby{時}{とき}は
\ruby{鬱々}{うつ|〳〵}として、
%
たゞ
\ruby{沈}{しづ}みきつて
\ruby{物}{もの}も
\ruby{言}{い}はず、
%
\ruby{机}{つくゑ}に
\ruby{對}{むか}つても
\ruby{書}{ほん}は
\ruby{讀}{よ}まずに、
%
\ruby[g]{長太息}{ためいき}を
\ruby{吐}{つ}く
\ruby{時}{とき}のみ
\ruby{多}{おほ}く、
%
\ruby{{\換字{朝}}}{あさ}は
\ruby{心}{こゝろ}よく
\ruby{起}{お}きる
\ruby{日}{ひ}も
\ruby{無}{な}く、
%
\ruby{夜}{よ}も
\ruby{寐苦}{ね|ぐる}しく
\ruby{{\換字{過}}}{すご}すさうだ。
%
これは
\ruby{乃公}{お|れ}が
\ruby[g]{老夫}{おやぢ}から
\ruby{聞}{き}いたゞけで、
%
\ruby{無論}{む|ろん}
\ruby[g]{山路}{やまぢ}の
\ruby[g]{老夫}{おやぢ}の
つもりでは、
%
\ruby{乃公}{お|ら}に
\ruby{意見}{い|けん}して
\ruby{{\換字{遣}}}{や}れと
いふのだつた。
%
\換字{志}かし
\ruby{乃公}{お|れ}は
\ruby{乃公}{お|れ}の
\ruby{考}{かんがへ}で、
%
\ruby[g]{水野}{みづの}のためには
\ruby{幾干}{いく|ら}でも、
%
\ruby{盡力}{つ|く}したいと
\ruby{思}{おも}つて
\ruby{居}{ゐ}ることは
\ruby{思}{おも}つて
\ruby{居}{ゐ}るが、
%
\ruby{意見}{い|けん}を
\ruby{仕}{し}て
\ruby{利益}{た|め}に
なりさうな
\ruby{筋}{すぢ}では
\ruby{無}{な}いと、
%
\ruby{見切}{み|き}つて
つい
\ruby{其儘}{その|まゝ}に
\ruby{{\換字{過}}}{す}ごして
\ruby{來}{き}たのだ。
』% 其三での島木の最後の語りが終えたところ

\原本頁{35-10}%
\ruby{辛}{から}くも
\ruby{此時}{こ|ゝ}まで
\ruby{堪}{こら}へたりし
\ruby[g]{日方}{ひかた}は
\ruby{再}{ふたゝ}び
\ruby{叫}{さけ}び
\ruby{出}{いだ}しぬ。

\原本頁{35-11}%
『
\ruby{何故}{な|ぜ}
\ruby{意見}{い|けん}を
\ruby{仕}{し}ても
\ruby{利益}{た|め}にならん?。
%
\ruby{意見}{い|けん}を
\ruby{仕無}{し|な}いで
\ruby{何}{なん}と
\ruby{爲}{す}るんだ?。
%
\原本頁{36-1}%
\ruby{何樣}{ど|う}して
\ruby[g]{水野}{みづの}の
\ruby{爲}{ため}に
\ruby{盡力}{つ|く}す?。
』

\原本頁{36-2}%
『
\ruby{乃公}{お|ら}あ
\ruby{出來}{で|き}る
\ruby{事}{こと}なら
\ruby[g]{水野}{みづの}の
\ruby{思}{おも}ひの、
%
\ruby{徹}{とほ}るやうに
\ruby{爲}{し}て
\ruby{{\換字{遣}}}{や}らうと
\ruby{思}{おも}つて
\ruby{居}{ゐ}るのだ。
』

\原本頁{36-4}%
『
\ruby{何}{なん}だと、
%
\ruby{馬鹿野郎}{ば|か|や|らう}ツ!、
%
\ruby{愚}{ぐ}にもつかん!。
%
そんな
\ruby{下}{くだ}らんことがあるものか、
%
\ruby{貴樣}{き|さま}は
\ruby{一體}{いつ|たい}
\ruby{腐敗}{ふ|はい}して
\ruby{居}{ゐ}る!。
』

\原本頁{36-6}%
『また
\ruby{馬鹿}{ば|か}
\ruby{呼}{よば}はりを
するナ!。
%
\ruby{汝}{きさま}こそ
\ruby{馬鹿}{ば|か}だ。
%
\ruby{意見}{い|けん}して
\ruby{役}{やく}に
\ruby{立}{た}つ
\ruby{位}{くらゐ}なら
\ruby{乃公}{お|れ}が
\ruby{爲}{す}るは。
%
\ruby{人}{ひと}は
\ruby{銘々}{めい|〳〵}に
\ruby{{\換字{所}}考}{かん|がへ}が
ある。
%
\ruby{乃公}{お|れ}は
\ruby{乃公}{お|れ}、
%
\ruby{汝}{きさま}は
\ruby{汝}{きさま}で
\ruby{可矣}{い|ゝ}ぢやあ
\ruby{無}{ね}えか。
%
\ruby{意見}{い|けん}が
\ruby{仕}{し}たけりやあ
\ruby[<j|]{汝}{きさま}
\ruby{爲}{し}ろ。
』

\原本頁{36-9}%
『
\ruby{勿論}{もち|ろん}だ。
%
\ruby{諫}{いさ}めて
\ruby{{\換字{遣}}}{や}らないで
\ruby{何樣}{ど|う}するものか。
%
\ruby{女}{をんな}が
\ruby{美}{よ}くつても
\ruby{惡}{わる}くつても、
%
\ruby{何}{なん}だ!、
%
\ruby{女}{をんな}が!。
%
\ruby{苟}{いやし}くも
\ruby{大{\換字{丈}}夫}{だい|ぢやう|ぶ}たるものが
\ruby{高}{たか}が
\ruby{一{\換字{婦}}人}{いち|ぷ|じん}に、
%
\ruby{志}{こゝろざし}を
\ruby{喪}{うしな}ふとは
\ruby{何}{なん}たる
\ruby{事}{こつ}た。
%
\ruby{實}{じつ}に
\ruby{怪}{け}しからん、
%
はがゆい
\ruby{奴}{やつ}だ。
%
\原本頁{37-1}%
\ruby{是非}{ぜ|ひ}
\ruby{{\換字{尋}}}{たづ}ねて
\ruby{行}{い}つて
\ruby{大}{おほい}に
\ruby{諫}{いさ}める。
』

\原本頁{37-2}%
\ruby{二人}{ふた|り}の
\ruby{問答}{もん|だふ}は
こゝに
\ruby{已}{や}んで、
%
\ruby[g]{山瀬}{やませ}は
\ruby{爽}{さわ}やかに
\ruby{口}{くち}を
\ruby{開}{ひら}きぬ。

\原本頁{37-3}%
『
\ruby{僕}{ぼく}は
\ruby{他人}{ひ|と}の
\ruby{意志}{い|し}
\ruby{感{\換字{情}}}{かん|じやう}の
\ruby{自由}{じ|いう}を
\ruby{{\換字{尊}}重}{そん|ちよう}するから、
%
\ruby{立入}{たち|い}つては
\ruby{敢}{あへ}て
\ruby{兎角}{と|かく}を
\ruby{言}{い}はぬ。
%
\換字{志}かし
これは
\ruby[g]{水野}{みづの}
\ruby{君}{くん}のために
\ruby{不利益}{ふ|り|えき}と
\ruby{思}{おも}ふから、
%
\ruby{一應}{いち|おう}は
\ruby{忠告}{ちゆう|こく}を
\ruby{試}{こゝろ}みるつもりだ。
』

\原本頁{37-6}%
\ruby{人}{ひと}
\ruby{皆}{みな}
\ruby{語}{かた}れども
\ruby[g]{羽{\換字{勝}}}{はがち}は
\ruby{語}{かた}らず、
%
たゞ
\ruby{僅}{わづか}に
\ruby{吁然}{ほ|つ}と
\ruby{息}{いき}つけば、
%
\ruby{手}{て}にせし
\ruby{卷{\換字{煙}}草}{た|ば|こ}の
\ruby{{\換字{灰}}}{はい}の% ルビは「はひ」でなく原本通り
\ruby{長}{なが}く
\ruby{續}{つゞ}けるが、
%
ぼたりと
\ruby{膝}{ひざ}の
\ruby{上}{うへ}に
\ruby{落}{お}ちて
\ruby{脆}{もろ}く
\ruby{散}{ち}つたり。

\原本頁{37-9}%
\ruby{夜色}{や|しよく}は
\ruby{樓外}{ろう|ぐわい}に
\ruby{沈々}{ちん|〳〵}として、
%
\ruby{澄}{す}みわたりたる
\ruby{天}{そら}に
かゝれる
\ruby{星斗}{ほ|し}は
\ruby{爛然}{らん|ぜん}と
\ruby{明}{あき}らかに、
%
\ruby{明日}{あ|す}は
\ruby{風}{かぜ}にや
\ruby{其}{そ}の
\ruby{大}{おほき}なるは、
%
いづれも
\ruby{煌々}{ひか|〳〵}と
\ruby[g]{瞬目}{めはじき}して、
%
\ruby{光}{ひかり}の
\ruby{芒}{のぎ}は
\ruby{搖}{ゆら}ぎに
\ruby{搖}{ゆら}げり。
