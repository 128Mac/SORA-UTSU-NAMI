\Entry{其六}

% メモ 校正終了 2024-03-29 2024-05-23 2024-06-15
\原本頁{38-2}%
\ruby[g]{山瀬}{やませ }が
\ruby{催}{もよほ}せし
\ruby[g]{小集}{せうしふ}の、
%
\ruby[g]{竹芝}{たけしば}の
\ruby{浦}{うら}に
\ruby{開}{ひら}かれし
\ruby{日}{ひ}なり、
%
これは
\ruby[<j||]{東}{とう }% 行末行頭の境界付近なので特例処置を施す
\ruby[<j||]{京}{きやう}
% \ruby{東京}{とう|きやう}
\原本頁{38-3}\改行%
を
\ruby[g]{丑寅}{うしとら}に
\ruby{離}{はな}れし
\ruby{東武線}{とう|ぶ|せん}の
\ruby[||j>]{鐘}{かねが}
\ruby[||j>]{淵}{ ふち}の
% \ruby{鐘淵}{かねが|ふち}の
\ruby[g]{停車}{ていしや}
\ruby{場}{じやう}より、% 原文通り「場」
%
\ruby{上}{のぼ}り
\ruby[g]{滊車}{き しや}の
\ruby{今}{いま}や
\ruby{出}{い}でんとするに
\ruby{駈}{か}け
\ruby{付}{つ}けて、
%
\ruby{辛}{から}くも
\ruby{乘}{の}り
\ruby{{\換字{込}}}{こ}みし
\ruby[g]{水野}{みづの }
\ruby{靜十郎}{せい|じふ|らう}は、
%
\ruby[g]{車室}{しやしつ}の
\ruby[g]{一隅}{いちぐう}に
\ruby{身}{み}を
おちつけて、
%
\ruby{煎}{い}りつくが
\ruby{如}{ごと}き
\ruby{急}{せ}き
\ruby{心}{ごゝろ}に
\ruby{少}{すくな}からぬ
\原本頁{38-6}\改行%
\ruby[g]{路程}{みちのり}を
\ruby{走}{はし}り
\ruby{來}{きた}りし
\ruby{胸}{むね}の
\ruby{轟}{とゞろ}きを
\ruby{纔}{わづか}に
\ruby{息}{やす}めぬ。

\原本頁{38-7}%
\ruby[g]{車窓}{しやそう}の
\ruby{外}{そと}は、
%
\ruby{目}{め}に
\ruby{障}{さは}るものも
\ruby{無}{な}く
\ruby[g]{廣々}{ひろ〴〵}としたる
\ruby[g]{葛{\換字{飾}}}{かつしか}の
\ruby{秋}{あき}の
\ruby[g]{稻田}{いなだ }に、
%
\ruby{黄金色}{こ|がね|いろ}の
\ruby[g]{夕陽}{ゆふひ }の
\ruby[g]{光線}{ひかり }
\ruby{明}{あか}るく
\ruby{斜}{なゝめ}に
\ruby{落}{お}ちて、
%
\ruby[g]{折々}{をり〳〵}
ばつと
\ruby{立}{た}つ
\ruby[||j>]{群}{ぐん}
\ruby[||j>]{雀}{じやく}の
% \ruby{群雀}{ぐん|じやく}の
\ruby{{\換字{空}}}{そら}に
\ruby{散}{ち}る
\ruby[g]{景色}{け しき}も、
%
\ruby[g]{土用}{ど よう}の
\ruby{旱}{てり}の
\ruby{足}{た}りて
\ruby{豊}{ゆたか}なる
\ruby{年}{とし}の
\換字{志}るしと
\ruby{好}{この}もしく、
%
\ruby{暑}{あつ}かりし
\ruby{夏}{なつ}の
\ruby{日}{ひ}の
\ruby{汗}{あせ}の
\ruby{滴}{しづく}は、
%
\ruby{今}{いま}
\ruby{皆}{みな}
やがて
\ruby[g]{粒々}{りふ〳〵}の
\ruby{實}{み}となつて
\ruby{現}{あらは}るべき
\ruby[<j>]{快}{こゝろよ}き
\ruby[g]{眺望}{ながめ }なり。

\原本頁{39-2}%
されば
\ruby{乗}{の}り
\ruby{合}{あ}はせし
\ruby[g]{人々}{ひと〴〵}も
\ruby{欣}{よろこ}び
\ruby{顏}{がほ}して、

\原本頁{39-3}%
『
\ruby{先}{ま}づ
\ruby{此}{こ}の
\ruby{{\換字{分}}}{ぶん}に
\ruby{行}{ゆ}きやあ
\ruby[g]{豐年}{ほうねん}でがす。
』

\原本頁{39-4}%
と
\ruby[g]{股引}{もゝひき}に
\ruby{草鞋穿}{わら|ぢ|ば}きの
\ruby[||j|]{農}{ひやく}% ルビ調整(原本通り)
\ruby[||j|]{夫}{しやう}らしきが
% \ruby{農夫}{ひやく|しやう}らしきが
\ruby[g]{眞先}{まつさき}に
\ruby{云}{い}ひ
\ruby{出}{だ}せば、

\原本頁{39-5}%
『
さうです、
%
\ruby{風}{かぜ}さへ
\ruby{無}{な}きやあ
\ruby{既}{もう}
\ruby{大{\換字{丈}}夫}{だい|ぢやう|ぶ}です。
%
おほかた
\ruby{不景氣}{ふ|けい|き}も
\ruby{直}{なほ}るでがせう。
』

\原本頁{39-7}%
と
\ruby{同}{おな}じ
\ruby{風}{ふう}の
\ruby{男}{をとこ}が
\ruby{云}{い}ふ。
%
その
\ruby{後}{あと}より
\ruby{髮}{かみ}の
\ruby{毛}{け}を
\ruby[g]{綺麗}{き れい}に
\ruby{{\換字{分}}}{わ}けたる
\ruby{生意氣}{なま|い|き}の
\ruby{{\換字{若}}}{わか}き
\ruby{男}{をとこ}の、
%
これは
\ruby[||j>]{商}{しやう}
\ruby[||j>]{人}{ にん}と
% \ruby{商人}{しやう|にん}と
\ruby{見}{み}えたるが、

\原本頁{39-9}%
『
\ruby{何}{なん}にしろ
\ruby[g]{此夏}{このなつ}の
\ruby[g]{暑氣}{あつさ }の
おかげですもの、
%
\ruby[||j>]{此}{この}
\ruby[||j>]{位}{ぐらゐ}の
% \ruby{此位}{この|ぐらゐ}の
\ruby{事}{こと}あ
\ruby{無}{な}くちやあ
なりませんや。
%
\ruby{暑}{あつ}かつた
\ruby{事}{こと}あ
\ruby[g]{無法}{む はふ}に
\ruby{暑}{あつ}うございましたが、
%
\ruby[g]{何樣}{ど う }でしやう
\ruby[g]{全國}{ぜんこく}ぢやあ
\ruby{其}{それ}がために、
%
\ruby[g]{去年}{きよねん}に
\ruby{比}{くら}べりやあ
\ruby{一千萬石}{いつ|せん|まん|ごく}も
\ruby[g]{餘計}{よ けい}に
\ruby{穫}{と}れる
\ruby[g]{算盤}{そろばん}だつて
\ruby{云}{い}ふんですからなア!。
%
\原本頁{40-1}%
\ruby[g]{一石}{いつこく}
\ruby[g]{十圓}{じふゑん}としても
\ruby{一億圓}{いち|おく|ゑん}、
%
\ruby{四千萬人}{よん|せん|まん|にん}に
\ruby{割}{わ}つて
みると、
%
\ruby{一人{\換字{前}}}{いち|にん|まへ}が
\ruby{二圓五十錢}{に|ゑん|ご|じふ|せん}
\ruby{宛}{づゝ}、
%
\ruby[g]{畢竟}{つ まり}
それだけ
\ruby{宛}{づゝ}
\ruby[g]{暑氣}{あつさ }の
\ruby[g]{堪{\換字{忍}}}{が まん}
\ruby{賃}{ちん}に
\ruby{貰}{もら}つたやうな
\ruby{譯}{わけ}に
\ruby{當}{あた}りますから、
%
\ruby[g]{隨{\換字{分}}}{ずゐぶん}
\ruby{暑}{あつ}かつたのも
\ruby[g]{無理}{む り }は
\ruby{有}{あ}りません。
%
\ruby{併}{しか}し
\ruby[g]{如是}{か う }なつて
\ruby{見}{み}りやあ
\ruby{有}{あ}り
\ruby{{\換字{難}}}{がた}いもんで、
%
\ruby[g]{屹度}{きつと }
\ruby[g]{景氣}{けいき }も
\ruby{好}{よ}くなりまさあネ。
』

\原本頁{40-6}%
などゝ
\ruby[g]{口々}{くち〴〵}に
\ruby{語}{かた}り
あへど、
%
\ruby[||j>]{思}{おもひ}
\ruby[||j>]{有}{ あ}る
\ruby{身}{み}の
\ruby[g]{水野}{みづの }
\ruby[g]{一人}{ひとり }は、
%
\ruby[g]{景色}{け しき}も
\ruby{眼}{め}に
\原本頁{40-7}\改行%
\ruby{{\換字{更}}}{さら}に
\ruby{見}{み}ざるが
ごとく、
%
\ruby[g]{談話}{はなし }も
\ruby{耳}{みゝ}に
\ruby{{\換字{更}}}{さら}に
\ruby{聞}{き}かぬが
\ruby{如}{ごと}く、
%
\ruby{身}{み}じろぎ
\原本頁{40-8}\改行%
も
\ruby{多}{おほ}くはせで
\ruby[||j>]{寂}{じやく}
\ruby[||j>]{然}{ ねん}と
% \ruby{寂然}{じやく|ねん}と
\ruby{坐}{すわ}りつ、
%
たゞ
\ruby{帶}{おび}の
\ruby{間}{あひだ}より
\ruby[g]{時計}{と けい}を
\ruby{出}{いだ}して、
%
\ruby[<j||]{恰}{あだか}% 恰も「あ(だ)かも」% 行末行頭の境界付近なので特例処置を施す
\原本頁{40-9}\改行%
も
\ruby[g]{滊車}{き しや}の
\ruby[g]{{\換字{速}}力}{はやさ }を
\ruby{疑}{うたが}ふやうに、
%
\ruby[g]{幾度}{いくたび}か
\ruby{其}{そ}の
\ruby{鍼}{はり}を
\ruby[g]{甲{\換字{斐}}}{か ひ }
\ruby{無}{な}く
\ruby[g]{視詰}{み つ }めぬ
\改行% 校正作業の簡略化のため
。
%
\原本頁{40-10}\改行%
\ruby[g]{淺黑}{あさぐろ}き
\ruby{其}{そ}の
\ruby{面}{おもて}は
\ruby{底}{そこ}に
\ruby[g]{蒼色}{あをみ }を
\ruby{帶}{お}びて、
%
\ruby[g]{鳳眼}{ほうがん}とやらん
\ruby{人}{ひと}のいふ
\ruby{魚尾上}{し|り|あが}りの
\ruby{眼}{め}は、
%
どんよりと
\ruby{曇}{くも}りて
\ruby{光}{ひか}り
\ruby{澱}{よど}み、
%
やゝ
\ruby{狭}{せま}き
\ruby{鼻}{はな}は
つんと
\原本頁{41-1}\改行%
\ruby{高}{たか}くして、
%
\ruby{血}{ち}の
\ruby[g]{色薄}{いろうす}き
\ruby{一}{いち}の
\ruby{字}{じ}
\ruby{口}{ぐち}の
\ruby[<j>]{唇}{くちびる}は、
%
\ruby{復}{ふたゝ}び
\ruby{開}{ひら}かるゝ
\ruby{時}{とき}の
\ruby{無}{な}からん
\ruby{如}{ごと}くに
\ruby{{\換字{飽}}}{あく}まで
\ruby{緊}{きび}しく
\ruby{閉}{とぢ}られたり。
%
\ruby{眼鼻立}{め|はな|だち}は
\ruby{醜}{あし}きに
あらぬ
\ruby[<j||]{男}{をとこ}ながら、
%
\ruby[g]{水野}{みづの }が
\ruby{今}{いま}の
\ruby{顏}{かほ}の
\ruby[g]{氣色}{やうす }は、
%
\ruby[g]{稚兒}{をさなご}は
\ruby{之}{これ}を
\ruby{望}{のぞ}まば
\ruby{怖}{おそ}れて
\ruby{泣}{な}くべし。

\原本頁{41-5}%
\ruby[g]{滊車}{き しや}の
やがて
\ruby{吾妻橋}{あづ|ま|ばし}% ルビ調整(原本通り)
\ruby[g]{停車}{ていしや}
\ruby{場}{じやう}に% 原文通り「場」
% 吾妻橋停車場 とうきょうスカイツリー駅
% 1902年(明治35年)に北千住駅から吾妻橋駅(現・とうきょうスカイツリー駅)へ延伸開業
% なので、ここで下車して吾妻橋を渡ることになる
\ruby{着}{つき}し
\ruby{時}{とき}には、
%
\ruby{暮}{く}れやすき
\ruby{秋}{あき}の
\ruby{日}{ひ}は
\原本頁{41-6}\改行%
\ruby{既}{はや}
\ruby{沒}{い}りて、
%
\ruby{千點萬點}{せん|てん|ばん|てん}の
\ruby[g]{燈火}{ともしび}に
\ruby{{\換字{飾}}}{かざ}られたる
\ruby{夜}{よる}の
\ruby[||j>]{東}{とう}
\ruby[||j>]{京}{きやう}は
% \ruby{東京}{とう|きやう}は
\ruby{眼}{め}の
\ruby{{\換字{前}}}{まへ}に
\ruby{現}{あら}はれぬ。

\原本頁{41-8}%
\ruby[g]{水野}{みづの }は
\ruby{人}{ひと}を
\ruby{突}{つ}き
\ruby{{\換字{退}}}{の}くるまでに
\ruby{忙}{いそ}がはしく
\ruby{歩}{あゆ}みて、
%
\ruby{忽}{たちま}ち
\ruby[g]{停車}{ていしや}
\ruby{場}{じやう}を% 原文通り「場」
\原本頁{41-9}\改行%
\ruby{出}{い}で、
%
\ruby{忽}{たちま}ち
\ruby{吾妻橋}{あづ|ま|ばし}を% ルビ調整(原本通り)
\ruby{越}{こ}え、
%
\ruby{忽}{たちま}ち
\ruby{茶屋町}{ちや|ゝ|まち}を% 現町名:台東区雷門二丁目
\ruby{{\換字{過}}}{す}ぎ、
%
\ruby{忽}{たちま}ち
\ruby[g]{並木}{なみき }を% 現町名:台東区雷門二丁目
\ruby{經}{へ}て
\改行% 校正作業の簡略化のため
、
%
\原本頁{41-10}\改行%
\ruby{忽}{たちま}ち
\ruby[g]{藏{\換字{前}}}{くらまへ}に
\ruby{至}{いた}り、
%
\ruby[g]{其處}{そ こ }に
\ruby{住}{すま}へる
\ruby[g]{月日}{つきひ }は
\ruby{未}{いま}だ
\ruby{長}{なが}からねど、
%
\ruby[g]{淺草}{あさくさ}
\ruby{一}{いち}
\原本頁{41-11}\改行%
との
\ruby{噂}{うはさ}を
\ruby{得}{え}たる
\ruby[g]{醫學士}{い がくし }
\ruby[g]{相良}{さがら }
\ruby[g]{公{\換字{平}}}{こうへい}の
\ruby[||j>]{玄}{げん}
\ruby[||j>]{關}{くわん}に
% \ruby{玄關}{げん|くわん}に
\ruby{至}{いた}り、

\原本頁{42-1}%
『
\ruby{頼}{たの}む。
』

\原本頁{42-2}%
と
\ruby[g]{一聲}{いつせい}
\ruby{音}{おと}づれたり。
