\Entry{其六}

% メモ 校正終了 2024-3-29
\原本頁{38-1}
\ruby[g]{山瀬}{やませ}が
\ruby{催}{もよほ}せし
\ruby{小集}{せう|しふ}の、
%
\ruby{竹芝}{たけ|しば}の
\ruby{浦}{うら}に
\ruby{開}{ひら}かれし
\ruby{日}{ひ}なり、
%
これは
\ruby[g]{東京}{とうきやう}を
\ruby{丑寅}{うし|とら}に
\ruby{離}{はな}れし
\ruby[g]{東武線}{とうぶせん}の
\ruby[g]{鐘淵}{かねがふち}の
\ruby[g]{停車場}{ていしやじやう}より、% 原文通り「場」
%
\ruby{上}{のぼ}り
\ruby{滊車}{き|しや}の
\ruby{今}{いま}や
\ruby{出}{い}でんとするに
\ruby{駈}{か}け
\ruby{付}{つ}けて、
%
\ruby{辛}{から}くも
\ruby{乘}{の}り
\ruby{{\換字{込}}}{こ}みし
\ruby[g]{水野}{みづの}
\ruby{靜十郎}{せい|じう|らう}は、
%
\ruby{車室}{しや|しつ}の
\ruby{一隅}{いち|ぐう}に
\ruby{身}{み}を
おちつけて、
%
\ruby{煎}{い}りつくが
\ruby{如}{ごと}き
\ruby{急}{せ}き
\ruby{心}{ごゝろ}に
\ruby{少}{すくな}からぬ
\ruby{路程}{みち|のり}を
\ruby{走}{はし}り
\ruby{來}{きた}りし
\ruby{胸}{むね}の
\ruby{轟}{とゞろ}きを
\ruby{纔}{わづか}に
\ruby{息}{やす}めぬ。

\原本頁{38-7}
\ruby{車窓}{しや|そう}の
\ruby{外}{そと}は、
%
\ruby{目}{め}に
\ruby{障}{さは}るものも
\ruby{無}{な}く
\ruby{廣々}{ひろ|〴〵}としたる
\ruby{葛{\換字{飾}}}{かつ|しか}の
\ruby{秋}{あき}の
\ruby{稻田}{いな|だ}に、
%
\ruby{黄金色}{こ|がね|いろ}の
\ruby{夕陽}{ゆふ|ひ}の
\ruby[g]{光線}{ひかり}
\ruby{明}{あか}るく
\ruby{斜}{なゝめ}に
\ruby{落}{お}ちて、
%
\ruby{折々}{をり|〳〵}
ばつと
\ruby{立}{た}つ
\ruby{群雀}{ぐん|じやく}の
\ruby{{\換字{空}}}{そら}に
\ruby{散}{ち}る
\ruby{景色}{け|しき}も、
%
\ruby{土用}{ど|よう}の
\ruby{旱}{てり}の
\ruby{足}{た}りて
\ruby{豊}{ゆたか}なる
\ruby{年}{とし}の
\換字{志}るしと
\ruby{好}{この}もしく、
%
\ruby{暑}{あつ}かりし
\ruby{夏}{なつ}の
\ruby{日}{ひ}の
\ruby{汗}{あせ}の
\ruby{滴}{しづく}は、
%
\ruby{今}{いま}
\ruby{皆}{みな}
やがて
\ruby{粒々}{りふ|〳〵}の
\ruby{實}{み}となつて
\ruby{現}{あらは}るべき
\ruby{快}{こゝろよ}き
\ruby{眺望}{なが|め}なり。

\原本頁{39-2}
されば
\ruby{乗}{の}り
\ruby{合}{あ}はせし
\ruby{人々}{ひと|〴〵}も
\ruby{欣}{よろこ}び
\ruby{顏}{がほ}して、

\原本頁{39-3}
『
\ruby{先}{ま}づ
\ruby{此}{こ}の
\ruby{{\換字{分}}}{ぶん}に
\ruby{行}{ゆ}きやあ
\ruby{豐年}{ほう|ねん}でがす。
』

\原本頁{39-4}
と
\ruby{股引}{もゝ|ひき}に
\ruby[g]{草鞋穿}{わらぢば}きの
\ruby[g]{農夫}{ひやくしやう}らしきが
\ruby{眞先}{まつ|さき}に
\ruby{云}{い}い
\ruby{出}{だ}せば、

\原本頁{39-5}
『さうです、
%
\ruby{風}{かぜ}さへ
\ruby{無}{な}きやあ
\ruby{既}{もう}
\ruby{大{\換字{丈}}夫}{だい|ぢやう|ぶ}です。
%
おほかた
\ruby{不景氣}{ふ|けい|き}も
\ruby{直}{なほ}るでがせう。
』

\原本頁{39-7}
と
\ruby{同}{おな}じ
\ruby{風}{ふう}の
\ruby{男}{をとこ}が
\ruby{云}{い}ふ。
%
その
\ruby{後}{あと}より
\ruby{髮}{かみ}の
\ruby{毛}{け}を
\ruby{綺麗}{き|れい}に
\ruby{{\換字{分}}}{わ}けたる
\ruby{生意氣}{なま|い|き}の
\ruby{{\換字{若}}}{わか}き
\ruby{男}{をとこ}の、
%
これは
\ruby{商人}{しやう|にん}と
\ruby{見}{み}えたるが、

\原本頁{39-9}
『
\ruby{何}{なに}にしろ
\ruby{此夏}{この|なつ}の
\ruby[g]{暑氣}{あつさ}の
おかげですもの、
%
\ruby{此位}{この|ぐらゐ}の
\ruby{事}{こと}あ
\ruby{無}{な}くちやあ
なりませんや。
%
\ruby{暑}{あつ}かつた
\ruby{事}{こと}あ
\ruby{無法}{む|はふ}に
\ruby{暑}{あつ}うございましたが、
%
\ruby{何樣}{ど|う}でしやう
\ruby{全國}{ぜん|こく}ぢやあ
\ruby{其}{それ}がために、
%
\ruby{去年}{きよ|ねん}に
\ruby{比}{くら}べりやあ
\ruby{一千萬石}{いつ|せん|まん|ごく}も
\ruby{餘計}{よ|けい}に
\ruby{穫}{と}れる
\ruby{算盤}{そろ|ばん}だつて
\ruby{云}{い}ふんですからなア!。
%
\原本頁{40-1}
\ruby{一石}{いつ|こく}
\ruby{十圓}{じう|ゑん}としても
\ruby{一億圓}{いち|おく|ゑん}、
%
\ruby{四千萬人}{よん|せん|まん|にん}に
\ruby{割}{わ}つて
みると、
%
\ruby{一人{\換字{前}}}{いち|にん|まへ}が
\ruby{二圓五十錢}{に|ゑん|ご|じう|せん}
\ruby{宛}{づゝ}、
%
\ruby[g]{畢竟}{つまり}
それだけ
\ruby{宛}{づゝ}
\ruby[g]{暑氣}{あつさ}の
\ruby{堪{\換字{忍}}賃}{が|まん|ちん}に
\ruby{貰}{もら}つたやうな
\ruby{譯}{わけ}に
\ruby{當}{あ}たりますから、
%
\ruby{隨{\換字{分}}}{ずゐ|ぶん}
\ruby{暑}{あつ}かつたのも
\ruby{無理}{む|り}は
\ruby{有}{あ}りません。
%
\ruby{併}{しか}し
\ruby{如是}{か|う}なつて
\ruby{見}{み}りやあ
\ruby{有}{あ}り
\ruby{難}{がた}いもんで、
%
\ruby{屹度}{きつ|と}
\ruby{景氣}{けい|き}も
\ruby{好}{よ}くなりまさあネ。
』

\原本頁{40-6}
などゝ
\ruby{口々}{くち|〴〵}に
\ruby{語}{かた}り
あへど、
%
\ruby[<h||]{思}{おもひ}
\ruby{有}{あ}る
\ruby{身}{み}の
\ruby[g]{水野}{みづの}
\ruby{一人}{ひと|り}は、
%
\ruby{景色}{け|しき}も
\ruby{眼}{め}に
\ruby{{\換字{更}}}{さら}に
\ruby{見}{み}ざるが
ごとく、
%
\ruby{談話}{はな|し}も
\ruby{耳}{みゝ}に
\ruby{{\換字{更}}}{さら}に
\ruby{聞}{き}かぬが
\ruby{如}{ごと}く、
%
\ruby{身}{み}じろぎも
\ruby{多}{おほ}くはせで
\ruby{寂然}{じやく|ねん}と
\ruby{坐}{すわ}りつ、
%
たゞ
\ruby{帶}{おび}の
\ruby{間}{あひだ}より
\ruby{時計}{と|けい}を
\ruby{出}{いだ}して、
%
\ruby{恰}{あだか}も% 恰も「あ(だ)かも」
\ruby{滊車}{き|しや}の
\ruby{{\換字{速}}力}{はや|さ}を
\ruby{疑}{うたが}ふやうに、
%
\ruby{幾度}{いく|たび}か
\ruby{其}{そ}の
\ruby{鍼}{はり}を
\ruby{甲{\換字{斐}}}{か|ひ}
\ruby{無}{な}く
\ruby{視詰}{み|つ}めぬ。
%
\ruby{淺黑}{あさ|ぐろ}き
\ruby{其}{そ}の
\ruby{面}{おもて}は
\ruby{底}{そこ}に
\ruby{蒼色}{あを|み}を
\ruby{帶}{お}びて、
%
\ruby{鳳眼}{ほう|がん}とやらん
\ruby{人}{ひと}のいふ
\ruby{魚尾上}{し|り|あが}りの
\ruby{眼}{め}は、
%
どんよりと
\ruby{曇}{くも}りて
\ruby{光}{ひか}り
\ruby{澱}{よど}み、
%
やゝ
\ruby{狭}{せま}き
\ruby{鼻}{はな}は
つんと
\ruby{高}{たか}くして、
%
\原本頁{41-1}
\ruby{血}{ち}の
\ruby{色薄}{いろ|うす}き
\ruby{一}{いち}の
\ruby{字}{じ}
\ruby{口}{ぐち}の
\ruby{唇}{くちびる}は、
%
\ruby{復}{ふたゝ}び
\ruby{開}{ひら}かるゝ
\ruby{時}{とき}の
\ruby{無}{な}からん
\ruby{如}{ごと}くに
\ruby{{\換字{飽}}}{あく}まで
\ruby{緊}{きび}しく
\ruby{閉}{とぢ}られたり。
%
\ruby{眼鼻立}{め|はな|だち}は
\ruby{醜}{あし}きに
あらぬ
\ruby{男}{をとこ}ながら、
%
\ruby[g]{水野}{みづの}が
\ruby{今}{いま}の
\ruby{顏}{かほ}の
\ruby{氣色}{やう|す}は、
%
\ruby[g]{稚兒}{をさなご}は
\ruby{之}{これ}を
\ruby{望}{のぞ}まば
\ruby{怖}{おそ}れて
\ruby{泣}{な}くべし。

\原本頁{41-5}
\ruby{滊車}{き|しや}の
やがて
\ruby[g]{吾妻橋}{あづまばし}
\ruby[g]{停車場}{ていしやじやう}に% 原文通り「場」
\ruby{着}{つ}きし
\ruby{時}{とき}には、
%
\ruby{暮}{く}れやすき
\ruby{秋}{あき}の
\ruby{日}{ひ}は
\ruby{既}{はや}
\ruby{沒}{い}りて、
%
\ruby{千點萬點}{せん|てん|ばん|てん}の
\ruby{燈火}{とも|しび}に
\ruby{{\換字{飾}}}{かざ}られたる
\ruby{夜}{よる}の
\ruby[g]{東京}{とうきやう}は
\ruby{眼}{め}の
\ruby{{\換字{前}}}{まへ}に
\ruby{現}{あら}はれぬ。

\原本頁{41-8}
\ruby[g]{水野}{みづの}は
\ruby{人}{ひと}を
\ruby{突}{つ}き
\ruby{{\換字{退}}}{の}くるまでに
\ruby{忙}{いそ}がはしく
\ruby{歩}{あゆ}みて、
%
\ruby{忽}{たちま}ち
\ruby[g]{停車場}{ていしやじやう}を% 原文通り「場」
\ruby{出}{い}で、
%
\ruby{忽}{たちま}ち
\ruby[g]{吾妻橋}{あづまばし}を
\ruby{越}{こ}え、
%
\ruby{忽}{たちま}ち
\ruby[g]{茶屋町}{ちやゝまち}を
\ruby{{\換字{過}}}{す}ぎ、
%
\ruby{忽}{たちま}ち
\ruby[g]{並木}{なみき}を
\ruby{經}{へ}て、
%
\ruby{忽}{たちま}ち
\ruby[g]{藏{\換字{前}}}{くらまへ}に
\ruby{至}{いた}り、
%
\ruby{其處}{そ|こ}に
\ruby{住}{すま}へる
\ruby{月日}{つき|ひ}は
\ruby{未}{いま}だ
\ruby{長}{なが}からねど、
%
\ruby{淺草}{あさ|くさ}
\ruby{一}{いち}との
\ruby{噂}{うはさ}を
\ruby{得}{え}たる
\ruby[g]{醫學士}{いがくし}
\ruby[g]{相良}{さがら}
\ruby[g]{公{\換字{平}}}{こうへい}の
\ruby{玄關}{げん|くわん}に
\ruby{至}{いた}り、

\原本頁{42-1}
『
\ruby{頼}{たの}む。
』

\原本頁{42-2}
と
\ruby{一聲}{いつ|せい}
\ruby{音}{おと}づれたり。
