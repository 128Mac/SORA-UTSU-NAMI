\Entry{其七}

\ruby{應}{おう}と
\ruby{答}{こた}へて
\ruby{出}{い}で
\ruby{來}{きた}れるは、
\ruby{盤臺面}{ばん|だい|づら}の
\ruby{鼻}{はな}の
\ruby{下}{した}に
\ruby{薄髭}{うす|ひげ}しよぼ〳〵と
\ruby{{\換字{煙}}}{けむり}の
\ruby{如}{ごと}く
\ruby{生}{は}えたる、
\ruby{二十七八}{に|じう|しち|はち}の
\ruby{物體}{もつ|たい}ぶつた
\ruby{男}{をとこ}なり。
\ruby[g]{水野}{みづの}が
\ruby{紺飛白}{こん|が|すり}の
\ruby{單衣}{ひとへ|もの}に、
\ruby{着皺}{き|じわ}も
\ruby{見}{み}ゆる
\ruby{薄{\換字{羽}}織}{うす|ば|おり}といふ
\ruby{身}{み}の
\ruby[g]{周圍}{まはり}を
\ruby{見}{み}て、
\ruby{突立}{つヽ|た}ちたるまヽ
\ruby{{\換字{尊}}大}{おほ|ふう}に、

『もう
\ruby{診察}{しん|さつ}の
\ruby{時間}{じ|かん}は
\ruby{濟}{す}んだが。
』

と
\ruby{云}{い}ひかけしが、また
\ruby{其}{そ}の
\ruby{{\GWI{u984f-j}}色}{かほ|いろ}の
\ruby{好}{よ}からぬを
\ruby{見}{み}て、

『お
\ruby{{\換字{前}}}{まへ}さんかネ。
』

と
\ruby{僅}{わずか}に
\ruby[g]{愛想}{あいそ}あり。

\ruby[g]{水野}{みづの}は
\ruby{叮嚀}{てい|ねい}に
\ruby{會釋}{ゑ|しやく}して、

『イヤ
\ruby{私}{わたくし}ではございません。
\ruby{御書留}{お|かき|とめ}
\ruby{置}{お}き
\ruby{下}{くだ}すつたといふ
\ruby{事}{こと}ですが、
\ruby{昨日}{さく|じつ}
\ruby[g]{使丁}{つかひ}を
\ruby{以}{も}つて
\ruby{願}{ねが}ひました
\ruby{四木村}{よ|つ|ぎ}の
\ruby{{\換字{平}}井}{ひら|ゐ}と
\ruby{申}{まを}す
\ruby{者}{もの}の
\ruby{方}{かた}の
\ruby{病人}{びやう|にん}、
\ruby{岩崎五十}{いは|さき|い|そ}といふものを
\ruby{御來診}{ご|らい|しん}
\ruby{願}{ねが}ひたいので
\ruby{出}{で}ましたのです。
』

と
\ruby{云}{い}へば、

『アゝ、
\ruby{其}{そ}の\ %空白有り
\ruby{四}{よ}ツ
\ruby{木}{ぎ}とかいふところは、
\ruby{非常}{ひ|じやう}に
\ruby{{\換字{遠}}}{とほ}いところぢやさうだナ。
\ruby{知}{し}らんものだから
\ruby{仕方}{し|かた}が
\ruby{無}{な}い、
\ruby{小梅}{こ|うめ}か
\ruby{{\換字{請}}地}{うけ|ぢ}の
\ruby[g]{{\換字{近}}傍}{ちかく}かと
\ruby{思}{おも}うて、ムヽ
\ruby{可矣}{よ|し}
\ruby{願}{ねが}つて
\ruby{置}{お}いて
\ruby{{\GWI{u9063-k}}}{や}ると
\ruby{僕}{ぼく}が
\ruby{{\換字{受}}合}{うけ|あ}つたが、
\ruby{後}{あと}で
\ruby{先生}{せん|せい}に
\ruby{酷}{ひど}く
\ruby{叱}{しか}られた!。
\ruby{重病人}{ぢゆう|びやう|にん}や
\ruby{長病人}{ちやう|びやう|にん}を
\ruby{澤山}{たく|さん}に
\ruby{扣}{ひか}へて
\ruby{居}{ゐ}られるから、
\ruby{中々}{なか|〳〵}
\ruby{其様}{そ|ん}な
\ruby{{\換字{遠}}}{とほ}いところへ
\ruby{御往診}{お|い|で}にはなりかねるといふことだ。
どうか
\ruby{他家}{よ|そ}へ
\ruby{行}{い}つて
\ruby{頼}{たの}んで
\ruby{見}{み}てくれ。
』

と、
\ruby{實}{じつ}に
\ruby{酷}{ひど}く
\ruby{叱}{しか}られや
\ruby{仕}{し}けむ、
\ruby{其}{そ}の
\ruby{時}{とき}の
\ruby[g]{不{\換字{平}}}{ふへい}は
\ruby{今}{いま}の
\ruby{{\GWI{u984f-j}}}{かほ}に
\ruby{膨}{ふく}れ
\ruby{出}{だ}して、
\ruby{{\換字{逐}}拂}{おつ|ぱら}つて
\ruby{仕舞}{し|ま}ふつもりの
\ruby{物言}{もの|い}ひ
\ruby[g]{仁慈}{なさけ}
\ruby{無}{な}し。

\ruby[g]{二三度}{にさんど}
\ruby[g]{四五度}{しごど}
\ruby{呼}{よ}びに
\ruby{{\GWI{u9063-k}}}{や}りける、といふ
\ruby{前句}{まへ|く}に、
\ruby{引}{ひ}く
\ruby{息}{いき}の
\ruby{{\換字{絕}}}{た}ゆるに
\ruby{醫者}{い|しや}のおどろかず、と
\ruby{付}{つ}けたるを、
\ruby{西鶴}{さい|くわく}が
\ruby{撰}{えら}みし
\ruby{其}{そ}の
\ruby[g]{疇昔}{むかし}より、
\ruby{世}{よ}に
\ruby{勢威}{いき|ほひ}ある
\ruby{醫者}{い|しや}を、
\ruby{富}{とみ}も
\ruby{無}{な}く
\ruby{名}{な}も
\ruby{無}{な}き
\ruby{賤人}{し|づ}が
\ruby{伏屋}{ふせ|や}に
\ruby{{\換字{請}}}{しやう}じ
\ruby{入}{い}れんとするほど、
\ruby{心}{こヽろ}に
\ruby{任}{まか}せで
\ruby{口惜}{くち|をし}きは
\ruby{無}{な}し。
\ruby[g]{相良}{さがら}が
\ruby{書生}{しよ|せい}の
\ruby{冷}{ひや}やかなる
\ruby{言葉}{こと|ば}も、
\ruby{今}{いま}さら
\ruby{珍}{めづ}しからぬ
\ruby{{\換字{浮}}世}{うき|よ}の
\ruby{態}{さま}なれば、
\ruby{腹}{はら}は
\ruby{立}{た}てねども
\ruby{差當}{さし|あた}つて
\ruby{恨}{うら}めしく
\ruby{悲}{かな}しく、
\ruby[g]{水野}{みづの}は

『
\ruby{左様}{さ|う}
\ruby{仰}{おつし}あつては
\ruby{當惑}{たう|わく}いたします。
\ruby{實}{じつ}は
\ruby{昨日}{さく|じつ}から
\ruby{今}{いま}
\ruby{御來臨}{お|い|で}か
\ruby{今}{いま}
\ruby{御來臨}{お|い|で}かと
\ruby{御待}{お|ま}ち
\ruby{申}{まを}して
\ruby{居}{をり}ました
\ruby{様}{やう}な
\ruby{譯}{わけ}でございますから。
』

と
\ruby{云}{い}ひかくるを、
\ruby{書生}{しよ|せい}は
\ruby{面倒}{めん|だう}と
\ruby{云}{い}はぬばかりに、

『だから、うつかり
\ruby{受合}{うけ|あ}つた
\ruby{段}{だん}は
\ruby{僕}{ぼく}が
\ruby[g]{謝罪}{あやま}る。
たゞし
\ruby{先生}{せん|せい}は
\ruby{御忙}{お|いそ}がしくつて
\ruby{御來診}{お|い|で}になられんといふのぢやから
\ruby{仕方}{し|かた}が
\ruby{無}{な}いぢや
\ruby{無}{な}いか。
』

と
\ruby{後}{あと}を
\ruby{言}{い}はせぬやうに
\ruby{壓}{お}し
\ruby{被}{かぶ}せて
\ruby{云}{い}ふ。
それを
\ruby[g]{此方}{こなた}は
\ruby{押返}{おし|かへ}して、

『では
\ruby{御座}{ご|ざ}いませうが
\ruby{其處}{そ|こ}を
\ruby[g]{何卒}{どうぞ}
、もう
\ruby{一度}{いち|ど}
\ruby{御願}{お|ねが}ひ
\ruby{下}{くだ}すつて
\ruby{見}{み}て
\ruby{頂}{いたゞ}きたいのです。
\ruby{先生}{せん|せい}より
\ruby{他}{ほか}の
\ruby{方}{かた}を
\ruby{願}{ねが}はう
\ruby{氣}{き}は
\ruby{無}{な}くつて、かうして
\ruby{態々}{わざ|〳〵}
\ruby{四}{よ}ツ
\ruby{木}{ぎ}から、
\ruby{御願}{お|ねが}ひに
\ruby{出}{で}たのでございますから。
』

と、
\ruby{低}{ひく}き
\ruby[g]{聲音}{こわね}に
\ruby[g]{顫動}{ふるひ}をさへ
\ruby{帶}{お}びて、
\ruby{思}{おも}ひ
\ruby{入}{い}つて
\ruby{頭}{かうべ}を
\ruby{下}{さ}げて{\GWI{u1b048}}み〴〵と
\ruby{頼}{たの}み
\ruby{聞}{きこ}えぬ。
\ruby{見}{み}れば
\ruby{其面}{その|おもて}は
\ruby{深}{ふか}き
\ruby{憂愁}{うれ|ひ}の
\ruby{陰雲}{く|も}に
\ruby{生氣}{せい|き}を
\ruby{{\GWI{u93bb-g}}}{とざ}されて、
\ruby{疑懼}{ぎ|く}に
\ruby{潤}{うる}める
\ruby{眼}{め}の
\ruby{中}{うち}には、
\ruby{限無}{かぎり|な}き
\ruby{悲痛}{ひ|つう}の
\ruby{色}{いろ}を
\ruby{{\換字{浮}}}{うか}めたり。
\ruby[g]{至誠}{まこと}に
\ruby{動}{うご}かされて
\ruby{爭}{あらそ}ひかねたる
\ruby{書生}{しよ|せい}は
\ruby{是非}{ぜ|ひ}
\ruby{無}{な}く
\ruby{立}{な}ち
\ruby{上}{あが}がって、

『それぢやあ
\ruby{先}{まあ }
\ruby{伺}{うかが}つて
\ruby{見}{み}て
\ruby{上}{あ}げやうから、
\ruby{其處}{そ|こ}へ
\ruby{上}{あが}がつて
\ruby{待}{ま}つて
\ruby{居}{ゐ}なさい。
』

と、
\ruby{{\換字{猶}}}{なほ}
\ruby[g]{水野}{みづの}を
\ruby{田舎漢}{ゐな|か|もの}あしらひにして
\ruby{奥}{おく}へ
\ruby{行}{ゆ}きぬ。

\ruby{丁度}{ちやう|ど}
\ruby{人}{ひと}の
\ruby{{\換字{途}}{\換字{絕}}}{と|だ}えし
\ruby{夜食}{や|しよく}の
\ruby{頃}{ころ}とて、
\ruby{人}{ひと}も
\ruby{無}{な}き
\ruby{玄關}{げん|くわん}にたゞ
\ruby{我}{われ}ひとり、
\ruby{兀然}{つヽ|くり}として
\ruby{坐}{すわ}り
\ruby{居}{を}れば、
\ruby{我}{わ}が
\ruby{影子}{か|げ}
\ruby{淋}{さび}しく
\ruby{古畳}{ふる|だヽみ}に
\ruby{浸}{し}みて、
\ruby{偶然}{ふ|と}
\ruby{見}{み}れば
\ruby{低}{ひく}く
\ruby{吊}{つ}りたる
\ruby{電燈}{でん|とう}の
\ruby{蓋裏}{かさ|うら}に、
\ruby{{\換字{弱}}々}{よわ|〳〵}としたる
\ruby{白}{しろ}き
\ruby{蛾}{が}の、
\ruby{蝶}{てふ}といふほども
\ruby{無}{な}く
\ruby{小}{ちひさ}なるが、やがて
\ruby{力盡}{ちから|つ}きての
\ruby{身}{み}の
\ruby{果}{はて}をも
\ruby{思}{おも}はず、
\ruby{飛}{と}んでは
\ruby{止}{と}まり、
\ruby{止}{と}まつては
\ruby{飛}{と}びて
\ruby{狂}{くる}ひ
\ruby{居}{を}れり。

\ruby{待}{ま}つこと
\ruby[g]{少時}{しばし}して
\ruby{間}{あひ}の
\ruby{劃}{しきり}の
\ruby{唐紙}{から|かみ}をがらりと
\ruby{明}{あ}けて、
\ruby{書生}{しよ|せい}は
\ruby{復}{ふたヽ}び
\ruby{入}{い}り
\ruby{來}{きた}りぬ。

『
\ruby{何様}{ど|う}も
\ruby{他}{ほか}の
\ruby{病家}{びよう|か}の
\ruby{都合}{つ|がふ}もあつて出られぬと
\ruby{仰}{おつし}ある。
\ruby{氣}{き}の
\ruby{毒}{どく}だけれども
\ruby{他}{ほか}へ
\ruby{行}{い}つて
\ruby{下}{くだ}さい。
』

\ruby{言葉}{こと|ば}の
\ruby{柔}{やさ}しくなりたるだけに
\ruby{拒{\換字{絕}}}{きよ|ぜつ}の
\ruby{意}{こヽろ}はいよ〳〵
\ruby{堅}{かた}し。
さりとて
\ruby{病}{や}める
\ruby{五十子}{い|そ|こ}が
\ruby{曾}{かつ}てより
\ruby{信}{しん}じて、
\ruby{苦悶}{く|もん}の
\ruby{床}{とこ}の
\ruby{上}{うへ}の
\ruby{獨語}{ひとり|ごと}に
\ruby{頼}{たの}みたしといひしは、たゞ
\ruby{此}{こ}の
\ruby{家}{いへ}の
\ruby[g]{主人}{あるじ}なるを、いづくにか
\ruby{行}{ゆ}き
\ruby{他人}{ひ|と}を
\ruby{頼}{たの}まん。
\ruby[g]{水野}{みづの}はほとほと
\ruby{行}{ゆ}き
\ruby{詰}{つ}まりて、
\ruby[g]{言葉}{ことば}も
\ruby{無}{な}く
\ruby{力}{ちから}も
\ruby{無}{な}く
\ruby{首}{かうべ}を
\ruby{垂}{た}れしが、
\ruby{搏}{はたゝ}き
\ruby{已}{や}めぬ
\ruby{彼}{か}の
\ruby{白}{しろ}き
\ruby{蛾}{が}の、
\ruby[g]{電燈}{あかり}の
\ruby[g]{周圍}{まはり}を
\ruby{飛}{と}び
\ruby{廻}{めぐ}る
\ruby{其}{そ}の
\ruby{陰翳眼}{か|げ|め}の
\ruby{前}{まへ}にちら〳〵と
\ruby{落}{お}つれば、
\ruby{噫}{あゝ}、
\ruby{我}{われ}も
\ruby{取}{と}りかぬる
\ruby{燈}{ひ}の
\ruby{{\換字{近}}傍}{かた|はら}を、
\ruby{{\換字{猶}}}{なほ}
\ruby{去}{さ}らぬ
\ruby{蟲}{むし}と
\ruby{愚}{ぐ}にも
\ruby{愚}{ぐ}なれど、
\ruby{甲斐無}{か|ひ|な}くも
\ruby{飛}{と}び
\ruby{直}{なほ}し〳〵するごとく、
\ruby[g]{言葉}{ことば}を
\ruby{換}{か}へて
\ruby{頼}{たの}みて
\ruby{見}{み}んと、
\ruby{其場}{その|ば}は
\ruby{立}{た}たんともせざる
\ruby{折}{をり}から、
\ruby{奥}{おく}の
\ruby{方}{はう}より
\ruby{丁}{ちやう}といふ
\ruby{石子}{い|し}の
\ruby{響}{ひゞ}き、
\ruby{確}{たしか}に
\ruby{人}{ひと}の
\ruby{碁}{ご}を
\ruby{打}{う}てる
\ruby{音}{おと}の、
\ruby{幽}{かすか}に
\ruby[g]{此方}{こなた}に
\ruby{聞}{きこ}えたり。

