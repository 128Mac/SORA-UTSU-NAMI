\Entry{其七}

% メモ 校正終了 2024-03-30
\原本頁{42-4}%
\ruby{應}{おう}と
\ruby{答}{こた}へて
\ruby{出}{い}で
\ruby{來}{きた}れるは、
%
\ruby{盤臺面}{ばん|だい|づら}の
\ruby{鼻}{はな}の
\ruby{下}{した}に
\ruby{薄髭}{うす|ひげ}
しよぼ〳〵と
\ruby{{\換字{煙}}}{けむり}の
\ruby{如}{ごと}く
\ruby{生}{は}えたる、
%
\ruby{二十七八}{に|じう|しち|はち}の
\ruby{物體}{もつ|たい}ぶつた
\ruby{男}{をとこ}なり。
%
\ruby{水野}{みづ|の}が
\ruby[g]{紺飛白}{こんがすり}の
\ruby{單衣}{ひとへ|もの}に、
%
\ruby{着皺}{き|じわ}も
\ruby{見}{み}ゆる
\ruby{薄羽織}{うす|ば|おり}といふ
\ruby{身}{み}の
\ruby{周圍}{まは|り}を
\ruby{見}{み}て、
%
\ruby{突立}{つゝ|た}ちたる
まゝ
\ruby{{\換字{尊}}大}{おほ|ふう}に、

\原本頁{42-8}%
『もう
\ruby{診察}{しん|さつ}の
\ruby{時間}{じ|かん}は
\ruby{濟}{す}んだが。
』

\原本頁{42-9}%
と
\ruby{云}{い}ひかけしが、
%
また
\ruby{其}{そ}の
\ruby{顏色}{かほ|いろ}の
\ruby{好}{よ}からぬを
\ruby{見}{み}て、

\原本頁{42-10}%
『お
\ruby{{\換字{前}}}{まへ}さんかネ。
』

\原本頁{43-1}%
と
\ruby{僅}{わづか}に
\ruby{愛想}{あい|そ}あり。

\原本頁{43-2}%
\ruby{水野}{みづ|の}は
\ruby{叮嚀}{てい|ねい}に
\ruby{會釋}{ゑ|しやく}して、

\原本頁{43-3}%
『イヤ
\ruby{私}{わたくし}では
ございません。
%
\ruby{御書{\換字{留}}}{お|かき|とめ}
\ruby{置}{お}き
\ruby{下}{くだ}すつたといふ
\ruby{事}{こと}ですが、
%
\ruby{昨日}{さく|じつ}
\ruby{使丁}{つか|ひ}を
\ruby{以}{も}つて
\ruby{願}{ねが}ひました
\ruby[g]{四木村}{よつぎ}の
\ruby{{\換字{平}}井}{ひら|ゐ}と
\ruby{申}{まを}す
\ruby{者}{もの}の
\ruby{方}{かた}の
\ruby{病人}{びやう|にん}、
%
\ruby{岩崎}{いは|さき}% 原本のこの部分は「いわさき」
\ruby{五十}{い|そ}といふものを
\ruby{御來診}{ご|らい|しん}
\ruby{願}{ねが}ひたいので
\ruby{出}{で}ましたのです。
』

\原本頁{43-7}%
と
\ruby{云}{い}へば、

\原本頁{43-8}%
『アヽ、
%
\ruby{其}{そ}の
\ruby[g]{四ッ木}{よ ぎ}% TODO 小さい「ッ」となっているがどうしようか
とかいふところは、
%
\ruby{非常}{ひ|じやう}に
\ruby{{\換字{遠}}}{とほ}い
ところぢやさうだナ。
%
\ruby{知}{し}らんものだから
\ruby{仕方}{し|かた}が
\ruby{無}{な}い、
%
\ruby{小梅}{こ|うめ}か
\ruby{{\換字{請}}地}{うけ|ぢ}の
\ruby{{\換字{近}}傍}{ちか|く}かと
\ruby{思}{おも}うて、
%
ムヽ
\ruby{可矣}{よ|し}
\ruby{願}{ねが}つて
\ruby{置}{お}いて
\ruby{{\換字{遣}}}{や}ると
\ruby{僕}{ぼく}が
\ruby{受合}{うけ|あ}つたが、
%
\ruby{後}{あと}で
\ruby{先生}{せん|せい}に
\ruby{酷}{ひど}く
\ruby{叱}{しか}られた!。
%
\ruby[g]{重病人}{ぢゆうびやうにん}や% 原本通り「重(ぢゆう)」
\ruby[g]{長病人}{ちやうびやうにん}を
\ruby{澤山}{たく|さん}に
\ruby{扣}{ひか}へて
\ruby{居}{ゐ}られるから、
%
\原本頁{44-1}%
\ruby{中々}{なか|〳〵}
\ruby{其樣}{そ|ん}な
\ruby{{\換字{遠}}}{とほ}いところへ
\ruby[g]{御往診}{おいで}には
なりかねるといふことだ。
%
どうか
\ruby{他家}{よ|そ}へ
\ruby{行}{い}つて
\ruby{頼}{たの}んで
\ruby{見}{み}てくれ。
』

\原本頁{44-3}%
と、
%
\ruby{實}{まこと}に
\ruby{酷}{ひど}く
\ruby{叱}{しか}られや
\ruby{仕}{し}けむ、
%
\ruby{其}{そ}の
\ruby{時}{とき}の
\ruby{不{\換字{平}}}{ふ|へい}は
\ruby{今}{いま}の
\ruby{顏}{かほ}に
\ruby{膨}{ふく}れ
\ruby{出}{だ}して、
%
\ruby{{\換字{逐}}拂}{おつ|ぱら}つて
\ruby{仕舞}{し|ま}ふ
つもりの
\ruby{物言}{もの|い}ひ
\ruby{仁慈}{なさ|け}
\ruby{無}{な}し。

\原本頁{44-5}%
\ruby{二三度}{に|さん|ど}
\ruby{四五度}{し|ご|ど}
\ruby{呼}{よ}びに
\ruby{{\換字{遣}}}{や}りける、
%
といふ
\ruby{{\換字{前}}}{まへ}
\ruby{句}{く}に、
%
\ruby{引}{ひ}く
\ruby{息}{いき}の
\ruby{{\換字{絕}}}{た}ゆるに
\ruby{醫者}{い|しや}の
おどろかず、
%
と
\ruby{付}{つ}けたるを、
%
\ruby{西鶴}{さい|くわく}が
\ruby{撰}{えら}みし
\ruby{其}{そ}の
\ruby{疇昔}{むか|し}より、
%
\ruby{世}{よ}に
\ruby{勢威}{いき|ほひ}ある
\ruby{醫者}{い|しや}を、
%
\ruby{富}{とみ}も
\ruby{無}{な}く
\ruby{名}{な}も
\ruby{無}{な}き
\ruby{賤人}{し|づ}が
\ruby{伏屋}{ふせ|や}に
\ruby{{\換字{請}}}{しやう}じ
\ruby{入}{い}れんとするほど、
%
\ruby{心}{こゝろ}に
\ruby{任}{まか}せで
\ruby{口惜}{くち|をし}きは
\ruby{無}{な}し。
%
\ruby{相良}{さが|ら}が
\ruby{書生}{しよ|せい}の
\ruby{冷}{ひや}やかなる
\ruby{言葉}{こと|ば}も、
%
\ruby{今}{いま}さら
\ruby{珍}{めづ}しからぬ
\ruby{{\換字{浮}}世}{うき|よ}の
\ruby{態}{さま}なれば、
%
\ruby{腹}{はら}は
\ruby{立}{た}てねども
\ruby{差當}{さし|あた}つて
\ruby{恨}{うら}めしく
\ruby{悲}{かな}しく、
%
\ruby{水野}{みづ|の}は

\原本頁{44-11}%
『
\ruby{左樣}{さ|う}
\ruby{仰}{おつし}あつては
\ruby{當惑}{たう|わく}いたします。
%
\ruby{實}{じつ}は
\ruby{昨日}{さく|じつ}から
\ruby{今}{いま}
\ruby{御來臨}{お|い|で}か
\ruby{今}{いま}
\ruby{御來臨}{お|い|で}かと
\ruby{御待}{お|ま}ち
\ruby{申}{まを}して
\ruby{居}{をり}ました
\ruby{樣}{やう}な
\ruby{譯}{わけ}で
ございますから。
』

\原本頁{45-2}%
と
\ruby{云}{い}ひかくるを、
%
\ruby{書生}{しよ|せい}は
\ruby{面倒}{めん|だう}と
\ruby{云}{い}はぬばかりに、

\原本頁{45-3}%
『だから、
%
うつかり
\ruby{受合}{うけ|あ}つた
\ruby{段}{だん}は
\ruby{僕}{ぼく}が
\ruby{謝罪}{あや|ま}る。
%
たゞし
\ruby{先生}{せん|せい}は
\ruby{御忙}{お|いそ}がしくつて
\ruby[g]{御來診}{おいで}に
なられんといふのぢやから
\ruby{仕方}{し|かた}が
\ruby{無}{な}いぢや
\ruby{無}{な}いか。
』

\原本頁{45-6}%
と
\ruby{後}{あと}を
\ruby{言}{い}はせぬやうに
\ruby{壓}{お}し
\ruby{被}{かぶ}せて
\ruby{云}{い}ふ。
%
それを
\ruby{此方}{こな|た}は
\ruby{押{\換字{返}}}{おし|かへ}して、

\原本頁{45-7}%
『では
\ruby{御座}{ご|ざ}いませうが
\ruby{其處}{そ|こ}を
\ruby{何卒}{どう|ぞ}、
%
もう
\ruby{一度}{いち|ど}
\ruby{御願}{お|ねが}ひ
\ruby{下}{くだ}すつて
\ruby{見}{み}て
\ruby{頂}{いたゞ}きたいのです。
%
\ruby{先生}{せん|せい}より
\ruby{他}{ほか}の
\ruby{方}{かた}を
\ruby{願}{ねが}はう
\ruby{氣}{き}は
\ruby{無}{な}くつて、
%
かうして
\ruby{態々}{わざ|〳〵}
\ruby[g]{四ッ木}{よ ぎ}から、% TODO 小さい「ッ」となっているがどうしようか
%
\ruby{御願}{お|ねが}ひに
\ruby{出}{で}たので
ございますから。
』

\原本頁{45-10}%
と、
%
\ruby{低}{ひく}き
\ruby{聲音}{こわ|ね}に
\ruby{顫動}{ふる|ひ}をさへ
\ruby{帶}{お}びて、
%
\ruby{思}{おも}ひ
\ruby{入}{い}つて
\ruby{頭}{かうべ}を
\ruby{下}{さ}げて
{\換字{志}}みじみ% TODO 要検討踊り字表記/原本は行頭禁足で非踊り字表記だった
\ruby{頼}{たの}み
\ruby{聞}{きこ}えぬ。
%
\ruby{見}{み}れば
\ruby{其面}{その|おもて}は
\ruby{深}{ふか}き〳〵
\ruby{憂愁}{うれ|ひ}の
\ruby{陰雲}{く|も}に
\ruby{生氣}{せい|き}を
\ruby{{\換字{鎖}}}{とざ}されて、
%
\ruby{疑懼}{ぎ|く}に
\ruby{潤}{うる}める
\ruby{眼}{め}の
\ruby{中}{うち}には、
%
\原本頁{46-1}%
\ruby{限無}{かぎり|な}き
\ruby{悲痛}{ひ|つう}の
\ruby{色}{いろ}を
\ruby{{\換字{浮}}}{うか}めたり。
%
\ruby{至誠}{まこ|と}に
\ruby{動}{うご}かされて
\ruby{爭}{あらそ}ひかねたる
\ruby{書生}{しよ|せい}は
\ruby{是非}{ぜ|ひ}
\ruby{無}{な}く
\ruby{立}{た}ち
\ruby{上}{あが}がつて、

\原本頁{46-3}%
『それぢやあ
\ruby{先}{まあ}
\ruby[|j>]{伺}{うかゞ}つて
\ruby{見}{み}て
\ruby{上}{あ}げやうから、
%
\ruby{其處}{そ|こ}へ
\ruby{上}{あが}がつて
\ruby{待}{ま}つて
\ruby{居}{ゐ}なさい。
』

\原本頁{46-5}%
と、
%
\ruby{{\換字{猶}}}{なほ}
\ruby{水野}{みづ|の}を
\ruby{田舎}{ゐな|か}
\ruby{{\換字{漢}}}{もの}
あしらひにして
\ruby{奧}{おく}へ
\ruby{行}{ゆ}きぬ。

\原本頁{46-6}%
\ruby{丁度}{ちやう|ど}
\ruby{人}{ひと}の
\ruby{{\換字{途}}{\換字{絕}}}{と|だ}えし
\ruby{夜食}{や|しよく}の
\ruby{頃}{ころ}とて、
%
\ruby{人}{ひと}も
\ruby{無}{な}き
\ruby{玄關}{げん|くわん}に
たゞ
\ruby{我}{われ}
ひとり、
%
\ruby{兀然}{つゝ|くり}として
\ruby{坐}{すわ}り
\ruby{居}{を}れば、
%
\ruby{我}{わ}が
\ruby{影子}{か|げ}
\ruby{淋}{さび}しく
\ruby{{\換字{古}}疊}{ふる|だゝみ}に
\ruby{浸}{し}みて、
%
\ruby{偶然}{ふ|と}
\ruby{見}{み}れば
\ruby{低}{ひく}く
\ruby{吊}{つ}りたる
\ruby{電燈}{でん|とう}の
\ruby{蓋裏}{かさ|うら}に、
%
\ruby{{\換字{弱}}々}{よわ|〳〵}としたる
\ruby{白}{しろ}き
\ruby{蛾}{が}の、
%
\ruby{蝶}{てふ}といふほども
\ruby{無}{な}く
\ruby{小}{ちひさ}なるが、
%
やがて
\ruby[<j|]{力}{ちから}
\ruby{盡}{つ}きての
\ruby{身}{み}の
\ruby{果}{はて}をも
\ruby{思}{おも}はず、
%
\ruby{飛}{と}んでは
\ruby{止}{と}まり、
%
\ruby{止}{と}まつては
\ruby{飛}{と}びて
\ruby{狂}{くる}ひ
\ruby{居}{を}れり。

\原本頁{46-11}%
\ruby{待}{ま}つこと
\ruby{少時}{しば|し}して
\ruby{間}{あひ}の
\ruby{劃}{しきり}の
\ruby{唐紙}{から|かみ}を
がらりと
\ruby{明}{あ}けて、
%
\ruby{書生}{しよ|せい}は
\ruby{復}{ふたゝ}び
\ruby{入}{い}り
\ruby{來}{きた}りぬ。

\原本頁{47-2}%
『
\ruby{何樣}{ど|う}も
\ruby{他}{ほか}の
\ruby{病家}{びやう|か}の
\ruby{都合}{つ|がふ}も
あつて
\ruby{出}{で}られぬと
\ruby{仰}{おつし}ある。
%
\ruby{氣}{き}の
\ruby{毒}{どく}だけれども
\ruby{他}{ほか}へ
\ruby{行}{い}つて
\ruby{下}{くだ}さい。
』

\原本頁{47-4}%
\ruby{言葉}{こと|ば}の
\ruby{柔}{やさ}しくなりたるだけに
\ruby{拒{\換字{絕}}}{きよ|ぜつ}の
\ruby{意}{こゝろ}は
いよ〳〵
\ruby{堅}{かた}し。
%
さりとて
\ruby{病}{や}める
\ruby{五十子}{い|そ|こ}が
\ruby{曾}{かつ}てより
\ruby{信}{しん}じて、
%
\ruby{苦悶}{く|もん}の
\ruby{床}{とこ}の
\ruby{上}{うへ}の
\ruby[<j||]{獨}{ひとり}
\ruby{語}{ごと}に
\ruby{頼}{たの}みたしと
いひしは、
%
たゞ
\ruby{此}{こ}の
\ruby{家}{いへ}の
\ruby{主人}{ある|じ}なるを、
%
いづくにか
\ruby{行}{ゆ}き
\ruby{他人}{ひ|と}を
\ruby{頼}{たの}まん。
%
\ruby{水野}{みづ|の}は
ほとほと% TODO 畳文字にすべきかな
\ruby{行}{ゆ}き
\ruby{詰}{つ}まりて、
%
\ruby{言葉}{こと|ば}も
\ruby{無}{な}く
\ruby{力}{ちから}も
\ruby{無}{な}く
\ruby{首}{かうべ}を
\ruby{垂}{た}れしが、
%
\ruby{搏}{はたゝ}き
\ruby{已}{や}めぬ
\ruby{彼}{か}の
\ruby{白}{しろ}き
\ruby{蛾}{が}の、
%
\ruby{電燈}{あか|り}の
\ruby{周圍}{まは|り}を
\ruby{飛}{と}び
\ruby{{\換字{廻}}}{めぐ}る
\ruby{其}{そ}の
\ruby{陰翳眼}{か|げ|め}の
\ruby{{\換字{前}}}{まへ}に
ちら〳〵と
\ruby{落}{お}つれば、
%
\ruby{噫}{あゝ}、
%
\ruby{我}{われ}も
\ruby{取}{と}りかぬる
\ruby{燈}{ひ}の
\ruby{{\換字{近}}傍}{かた|はら}を、
%
\ruby{{\換字{猶}}}{なほ}
\ruby{去}{さ}らぬ
\ruby{蟲}{むし}と
\ruby{愚}{おろか}にも
\ruby{愚}{おろか}なれど、
%
\ruby{甲{\換字{斐}}}{か|ひ}
\ruby{無}{な}くも
\ruby{飛}{と}び
\ruby{直}{なほ}し〳〵するごとく、
%
\ruby{言葉}{こと|ば}を
\ruby{換}{か}へて
\ruby{頼}{たの}みて
\ruby{見}{み}んと、
%
\ruby{其}{その}
\ruby{場}{ば}は% 原文通り「場」
\ruby{立}{た}たんともせざる
\ruby{折}{をり}から、
%
\原本頁{48-1}%
\ruby{奧}{おく}の
\ruby{方}{はう}より
\ruby{丁}{ちやう}といふ
\ruby{石子}{い|し}の
\ruby{響}{ひゞ}き、
%
\ruby{確}{たしか}に
\ruby{人}{ひと}の
\ruby{碁}{ご}を
\ruby{打}{う}てる
\ruby{音}{おと}の、
%
\ruby{幽}{かすか}に
\ruby{此方}{こな|た}に
\ruby{聞}{きこ}えたり。
