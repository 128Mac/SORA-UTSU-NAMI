\Entry{其八}

\原本頁{}
\ruby{人}{ひと}おの〳〵
\ruby{我}{わ}が
\ruby{娯樂}{たの|しみ}に
\ruby{使}{つか}はれるは
\ruby{無}{な}し。
%
\ruby{中}{なか}にも
\ruby{碁好}{ご|ずき}は
\ruby{聖}{せい}に
\ruby{{\換字{近}}}{ちか}く
\ruby{愚}{ぐ}に
\ruby{{\換字{近}}}{ちか}く、
%
\ruby{假}{かり}の
\ruby{與奪}{やり|とり}の
\ruby{白黑}{しろ|くろ}の
\ruby{石}{いし}に、
%
\ruby{氣}{き}を
\ruby{{\換字{遣}}}{つか}ひ
\ruby{心}{こゝろ}を
\ruby{苦}{くるし}めて
\ruby{一切}{いつ|さい}を
\ruby{忘}{わす}れ
\ruby{果}{は}て、
%
\ruby{一寸}{いつ|すん}の
\ruby{暇}{ひま}を
\ruby{偸}{ぬす}んで
\ruby{始}{はじ}めし
\ruby{爭戰}{あら|そひ}にも、
%
\ruby{思}{おも}はず
\ruby{{\換字{半}}日}{はん|にち}の
\ruby{尻}{しり}を
\ruby{腐}{くさ}らせて
\ruby{悔}{くや}まぬが
\ruby{常}{つね}なり。
%
されば
\ruby{殆}{ほとん}ど
\ruby{一日}{いち|にち}の
\ruby{忙}{せは}しき
\ruby{業務}{つと|め}を
\ruby{{\換字{終}}}{を}へし
\ruby{擧句}{あげ|く}、
%
\ruby[<h||]{心}{こゝろ}
\ruby{蘇生}{よみ|が}へる
\ruby{晩餐}{ばん|さん}の
\ruby{小{\換字{酌}}}{せう|しやく}の
\ruby{後}{のち}に、
%
\ruby{憎}{にく}くも
\ruby{可愛}{あは|ゆ}くもある% ルビは原本通り
\ruby{其敵}{その|てき}を
\ruby{得}{え}て、
%
\ruby{罪無}{つみ|な}き
\ruby{樂}{たのし}みを
\ruby{取}{と}る
\ruby{一手}{いつ|て}
\g詰めruby{々々}{〳〵}の、
%
\ruby{興}{きよう}の
\ruby{極}{きは}めて
\ruby{旺}{さかん}なるところへ、
%
\ruby{熟知}{なじ|み}にもあらぬ
\ruby{病家}{びやう|か}の、\換字{志}かも
\ruby{普{\換字{通}}}{な|み}
\ruby{外}{はづ}れて
\ruby{{\換字{遠}}}{とほ}きより、
%
\ruby{夜陰}{や|いん}に
\ruby{及}{およ}びて
\ruby{呼}{よ}び
\ruby{{\換字{迎}}}{むか}へんとするとも、
%
\ruby{門{\換字{前}}}{もん|ぜん}の
\ruby{雀羅}{じやく|ら}、
%
\ruby{藥局}{やく|きよく}の
\ruby{蛛網}{しゆ|まう}、
%
\ruby{客}{きやく}に
\ruby{饑}{う}ゑきつたる
\ruby{庸醫}{よう|い}はいざ
\ruby{知}{し}らず、
%
\ruby{苟}{いやし}くも
\ruby{名}{な}の
\ruby{{\換字{通}}}{とほ}つたるほどの
\ruby{人}{ひと}の
\ruby{應}{おう}ぜざるべきは、
%
\ruby{思}{おも}へば
\ruby{無理}{む|り}も
\ruby{無}{な}き
\ruby{事{\換字{情}}}{わ|け}なりと、
%
\ruby{鈍}{にぶ}からぬ
\ruby[g]{水野}{みづの}は
\ruby{早}{はや}くも
\ruby{悟}{さと}りしが、
%
\ruby{物}{もの}に
\ruby{脆}{もろ}からぬ
\ruby{性質}{せい|しつ}の
\ruby{{\換字{猶}}}{なほ}
\ruby{思}{おも}ひ
\ruby{棄}{す}てず、
%
\ruby{何}{なに}をか
\ruby{考}{かんが}へ
\ruby{得}{え}しや
\ruby{此度}{こ|たび}は
\ruby{氣輕}{き|がる}く、

\原本頁{}
『ヤ、
%
たび〳〵
\ruby{御面倒}{ご|めん|だう}を
\ruby{願}{ねが}ひまして、
%
\ruby{有}{あ}り
\ruby{難}{がた}うございました。
』

\原本頁{}
と、
%
\ruby{云}{い}ひながら
\ruby{多少錢}{いく|ら|か}を
\ruby{手早}{て|ばや}く
\ruby{白色包}{か|み|づゝみ}にして、

\原本頁{}
『
\ruby{{\換字{煙}}草}{たば|こ}でも
\ruby{購}{と}つて
\ruby{參}{まゐ}つて
\ruby{獻}{あ}げるべきですが。
』

\原本頁{}
と、
%
\ruby{言葉}{こと|ば}を
\ruby{{\換字{飾}}}{かざ}つて
\ruby{取}{と}りつくろひ、
%
\ruby{流石}{さす|が}
\ruby{手}{て}を
\ruby{出}{いだ}しては
\ruby{取}{と}りかぬるを
\ruby{無理}{む|り}やりに
\ruby{握}{にぎ}らすれば、
%
まさかに
\ruby{投}{な}げ
\ruby{{\換字{返}}}{かへ}すこともせず、

\原本頁{}
『どうも
\ruby{御氣}{お|き}の
\ruby{毒}{どく}で、
』

\原本頁{}
と、
%
\ruby{我}{わ}が
\ruby{師}{し}の
\ruby{{\換字{迎}}}{むかへ}に
\ruby{應}{おう}ぜぬが
\ruby{氣}{き}の
\ruby{毒}{どく}なやら、
%
\ruby{我}{わ}が
\ruby{錢}{ぜに}
\ruby{使}{つか}はせしが
\ruby{氣}{き}の
\ruby{毒}{どく}なやら、
%
どちら
\ruby{付}{つ}かぬ
\ruby{挨拶}{あい|さつ}して、
%
うぢ〳〵と
\ruby{取}{と}りぬ。

\原本頁{}
\ruby{印}{いん}を
\ruby{結}{むす}び、
%
\ruby{呪}{じゆ}を
\ruby{誦}{じゆ}すること、
%
\ruby{今}{いま}は
\ruby{流行}{は|や}らず、
%
\ruby{世}{よ}にたゞ
\ruby{錢{\換字{術}}}{せん|じゆつ}ありて
\ruby{神}{かみ}に
\ruby{{\換字{通}}}{つう}ずるを、
%
\ruby{知}{し}らぬほど
\ruby{迂闊}{う|くわつ}にはあらざりし
\ruby[g]{水野}{みづの}は、
%
\ruby{書生}{しよ|せい}が
\ruby{我}{わ}が
\ruby{人{\換字{情}}錢}{こゝ|ろ|づけ}を
\ruby{收}{おさ}めしを
\ruby{見}{み}て、

\原本頁{}
『
\ruby{何樣}{ど|う}いふものでございましやう?
\ruby{病人}{びやう|にん}が
\ruby{思}{おも}ひ
\ruby{{\換字{込}}}{こ}んで
\ruby{居}{を}るのでございますから、
%
\ruby{一度}{いち|ど}だけなりと
\ruby{診}{み}て
\ruby{戴}{いたゞ}く
\ruby{譯}{わけ}には
\ruby{參}{まゐ}りますまいか。
%
こちらの
\ruby{先生}{せん|せい}の
\ruby{事}{こと}でございますから、
%
\ruby{澤山}{たく|さん}の
\ruby{御病家}{ご|びやう|か}の
\ruby{御都合}{ご|つ|がふ}もあつて、
%
\ruby{御暇}{お|ひま}の
\ruby{少}{すく}ないのは
\ruby{承知}{しよう|ち}して
\ruby{居}{を}りますから、
%
\ruby{始{\換字{終}}}{し|ゞう}
\ruby{來}{き}て
\ruby{戴}{いたゞ}きたいとは
\ruby{申}{まを}しますまいが、
%
\ruby[<h||]{只}{たつた}
\ruby{一度}{いち|ど}おいでなすつて
\ruby{下}{くだ}さるほどの
\ruby{事}{こと}なら、
%
\ruby{然程}{さ|ほど}
\ruby{御暇}{お|ひま}の
\ruby{取}{と}れるでは
\ruby{無}{な}し、
%
\ruby{御都合}{ご|つ|がふ}の
\ruby{出來}{で|き}ぬでも
\ruby{無}{な}からうと
\ruby{存}{ぞん}じます。
%
\ruby{一度}{いち|ど}でも
\ruby{御診察}{ご|しん|さつ}
\ruby{下}{くだ}すつて、
%
そして
\ruby{御指揮}{お|さし|づ}を
\ruby{仕}{し}て
\ruby{戴}{いたゞ}いたら、
%
あとは
\ruby{村醫}{そん|い}でも
\ruby{間}{ま}に
\ruby{合}{あ}はうかと
\ruby{存}{ぞん}じますが、
%
\ruby{病人}{びやう|にん}も
\ruby{信}{しん}じて
\ruby{居}{を}りませぬ
\ruby{村醫}{そん|い}ばかりでは、
%
\ruby{實以}{じつ|もつ}て
\ruby{傍觀}{わき|め}にも
\ruby{案}{あん}じられまして、
%
\ruby{癒}{なほ}るものも
\ruby{癒}{なほ}るまいかと
\ruby{心配}{しん|ぱい}
\ruby{致}{いた}します。
%
\ruby{貴君}{あな|た}には
\ruby{御無理}{ご|む|り}を
\ruby{申}{まを}して
\ruby{濟}{す}みませんが、
%
\ruby{折}{を}り
\ruby{入}{い}つて
\ruby{一}{ひと}つ
\ruby{此}{こ}の
\ruby{譯}{わけ}を
\ruby{仰}{おつし}あつて、
%
も
\ruby{一度}{いち|ど}
\ruby{何卒}{どう|ぞ}
\ruby{御願}{お|ねが}ひなすつて
\ruby{見}{み}てはいたゞけますまいか。
』

\原本頁{}
と
\ruby{泣}{な}かぬばかりに
\ruby{掻}{かき}
\ruby{口說}{く|ど}けば、
%
\ruby{書生}{しよ|せい}の
\ruby{面}{おもて}には
\ruby{難色}{なん|しよく}
\ruby{見}{み}えしが、
%
\ruby{既}{すで}に
\ruby{毒}{どく}を
\ruby{盛}{も}られたれば
\ruby{爭}{あらそ}ひ
\ruby{難}{がた}く、
%
\ruby{無下}{む|げ}に
\ruby{酷}{むご}くは
\ruby{斥}{しりぞ}けかねて、

\原本頁{}
『では
\ruby{始{\換字{終}}}{し|ゞう}
\ruby{病人}{びやう|にん}を
\ruby{受合}{うけ|あ}つて
\ruby{吳}{く}れといふのでは
\ruby{無}{な}くつて、
%
\ruby{診斷}{しん|だん}だけで
\ruby{好}{い}いからといふのぢやネ。
』

\原本頁{}
『ハイ、
%
それで
\ruby{滿足}{まん|ぞく}
\ruby{致}{いた}しませうと
\ruby{申}{まを}しますのですから、
%
\ruby{何卒}{どう|か}
\ruby{枉}{ま}げて
\ruby{御聞入}{お|きゝ|い}れ
\ruby{下}{くだ}さるやうに
\ruby{御願}{お|ねが}ひなすつて。
』

\原本頁{}
と
\ruby{一問一答}{いち|もん|いつ|たふ}の
\ruby{果}{は}てし
\ruby{後}{のち}、
%
\ruby{澁}{しぶ}る〳〵
\ruby{{\換字{弱}}}{よわ}つた
\ruby{氣色}{け|しき}して
\ruby{奧}{おく}へ
\ruby{行}{ゆ}きぬ。
%
\ruby[g]{水野}{みづの}は
\ruby{病}{や}める
\ruby{我}{わ}が
\ruby[g]{五十子}{いそこ}が
\ruby{物憂}{もの|う}げに、
%
\ruby{此}{こ}の
\ruby{廣}{ひろ}き
\ruby{世}{よ}に
\ruby{只}{たゞ}
\ruby{一人}{ひと|り}の
\ruby{誠意}{まこ|と}ある
\ruby{介抱者}{かい|はう|しや}をも
\ruby{有}{も}たずして、
%
\ruby{頼}{たの}み
\ruby{少}{すくな}き
\ruby{村醫}{そん|い}の
\ruby{怪}{あや}しき
\ruby{藥}{くすり}をのみ
\ruby{力}{ちから}としつゝ、
%
\ruby{心淋}{こゝろ|さび}しくも
\ruby{秋}{あき}の
\ruby{夜}{よ}
\ruby{悲}{かな}しき
\ruby{田舎家}{ゐ|なか|や}の
\ruby{一室}{ひと|ま}の
\ruby{内}{うち}に
\ruby{横}{よこた}はれる
\ruby{光景}{あり|さま}を
\ruby{胸}{むね}の
うちに
\ruby{描}{ゑが}きながら、
%
こたびの
\ruby{{\換字{返}}事}{へん|じ}は
\ruby{如何}{い|か}にぞと、
%
\ruby{聞}{き}く
\ruby{耳}{みゝ}
\ruby{立}{た}てゝ
\ruby{意}{こゝろ}を
\ruby{注}{つ}くれば、

\原本頁{}
『うるさい!。
%
\換字{志}つゝこい!。
』

\原本頁{}
と
\ruby{叱}{しか}る
\ruby{聲}{こゑ}に
\ruby{次}{つ}いで、
%
\ruby{負}{ま}けかゝりたるに
\ruby{怒}{いかり}をや
\ruby{含}{ふく}みけん、
%
パチリと
\ruby{{\換字{強}}}{つよ}く
\ruby{石}{いし}を
\ruby{下}{くだ}す
\ruby{音}{おと}して、
%
やがて
\ruby{書生}{しよ|せい}は
\ruby{膨}{ふく}れかへつて
\ruby{出}{い}で
\ruby{來}{きた}りぬ。
%
\ruby{挨拶}{あい|さつ}は
\ruby{聞}{き}かずとも
\ruby{既}{はや}
\ruby{解}{わか}りたり。
%
されど
\ruby{如是}{か|く}ても
\ruby[g]{水野}{みづの}は
\ruby{屈}{くつ}せず、
%
\ruby{書生}{しよ|せい}が
\ruby{何}{なに}を
\ruby{云}{い}ひしやらも
\ruby{知}{し}らずに、
%
\ruby{如何}{い|か}にしてか
\ruby{我}{わ}が
\ruby{念}{おもひ}を
\ruby{{\換字{遂}}}{と}げんと
\ruby{考}{かんが}へ
\ruby{沈}{しづ}みし
\ruby{後}{のち}、
%
\ruby{思}{おも}ひ
\ruby{得}{え}しところやありけん
\ruby{頭}{かうべ}を
\ruby{擡}{あ}げしが、
%
\ruby{其}{そ}の
\ruby{面}{おもて}は
\ruby{何時}{い|つ}か
\ruby{聊}{いさゝ}か
\ruby{色}{いろ}ざし
\ruby{來}{きた}り、
%
\ruby{其}{そ}の
\ruby{眼}{め}よりは
\ruby{今}{いま}まで
\ruby{潛}{ひそ}み% 【潛 u6f5b 「先先」】【潜 u6f5c 「夫夫」】併用されている
\ruby{居}{ゐ}たりし
\ruby{烱々}{けい|〳〵}たる
\ruby{光}{ひかり}の
\ruby{閃}{ひらめ}き
\ruby{出}{い}でゝ、
%
\ruby{見}{み}る〳〵
\ruby{如何}{い|か}なる
\ruby{任務}{つと|め}にも
\ruby{堪}{た}ふべく、
%
\ruby{如何}{い|か}なる
\ruby{人}{ひと}にも
\ruby{爭}{あらそ}つて
\ruby{{\換字{勝}}}{か}つべき
\ruby{峻烈}{しゆん|れつ}の
\ruby{氣象}{き|しやう}を
\ruby{現}{あらは}し
\ruby{出}{いだ}しぬ。
%
\ruby{折}{をり}から
\ruby{一}{ひと}つの
\ruby{彼}{か}の
\ruby{小}{ちひ}さき
\ruby{蛾}{が}は、
%
\ruby[<h||]{力}{ちから}
\ruby{盡}{つ}き
\ruby{翼傷}{つばさ|きず}つきて
\ruby{翩々}{ひら|〳〵}として、
%
\ruby{落花}{らく|くわ}の
\ruby{枝}{えだ}を
\ruby{辭}{じ}せしが
\ruby{如}{ごと}くに、
%
あはれにも
\ruby[g]{水野}{みづの}が
\ruby{膝}{ひざ}の
\ruby{{\換字{前}}}{まへ}に
\ruby{墜}{お}ちぬ。

