\Entry{其九}

\ruby{頭}{かうべ}を
\ruby{下}{さ}げ
\ruby{言葉}{こと|ば}を
\ruby{低}{ひく}くして、
\ruby{頼}{たの}むほどは
\ruby{頼}{たの}み
\ruby{盡}{つく}せしを、
\ruby{膠無}{にべ|な}く
\ruby{色}{いろ}なく
\ruby{斷}{ことわ}りに
\ruby{斷}{ことわ}られたり。
\ruby{今}{いま}は
\ruby{復言}{また|い}うべき
\ruby{餘地}{よ|ち}も
\ruby{無}{な}からんを、
\ruby[g]{水野}{みづの}はそも〳〵
\ruby{何}{なん}とせんとかする。

\ruby{水}{みづ}をもて
\ruby{解}{と}くべからざるものは
\ruby{火}{ひ}をもて
\ruby{熔}{と}かすべし、
\ruby{刀}{たう}をもて
\ruby{截}{き}り
\ruby{難}{がた}きものは
\ruby{槌}{つち}をもて
\ruby{碎}{くだ}き
\ruby{得}{え}ん。
\ruby{求}{もと}めて
\ruby{已}{や}まぬ
\ruby[g]{願望}{ねがひ}の
\ruby{心}{こヽろ}あれば、おのづと
\ruby{働}{はたら}く
\ruby{智慧}{ち|ゑ}の
\ruby{眼}{まなこ}は、
\ruby{我}{わ}が
\ruby{思}{おも}へる
\ruby{地}{ち}に
\ruby{到}{いた}らんとするに
\ruby{{\換字{平}}和}{おだ|やか}なる
\ruby{路}{みち}を
\ruby{取}{と}ることの
\ruby{甲斐無}{か|ひ|な}きを
\ruby{悟}{さと}りたらん
\ruby{曉}{あかつき}、いかで
\ruby{{\換字{猶}}}{なほ}
\ruby{別}{べつ}に
\ruby{峻}{さか}しき
\ruby{一}{ひ}ト
\ruby{條}{すぢ}の
\ruby{徑}{こみち}ありて
\ruby{其處}{そ|こ}に
\ruby{{\GWI{u901a-k}}}{つう}ずるを
\ruby{見出}{み|いだ}さゞらんや。
\ruby[g]{水野}{みづの}は
\ruby{今}{いま}その
\ruby{峻}{さか}しきを
\ruby{見出}{み|いだ}して
\ruby{攀}{よ}ぢ
\ruby{上}{のぼ}らんとするなり。
\ruby{火}{ひ}の
\ruby{力}{ちから}、
\ruby{槌}{つち}の
\ruby{力}{ちから}を
\ruby{試}{こヽろ}みんとするなり。

\ruby{其}{そ}の
\ruby{顏}{かほ}つきの
\ruby{變}{かは}れる
\ruby{如}{ごと}くに、
\ruby[g]{言葉}{ことば}の
\ruby[g]{調子}{てうし}も
\ruby{俄}{にはか}に
\ruby{變}{かは}り、
\ruby{聲}{こゑ}も\GWI{u1b048}たたかに
\ruby{大}{おおき}くなりぬ。

『いよ〳〵
\ruby{先生}{せん|せい}は
\ruby{御來臨}{お|い|で}
\ruby{下}{くだ}さらんと
\ruby{仰}{おつし}あるのですか。
イヤ、それは
\ruby{失禮}{しつ|れい}ながら
\ruby{左様}{さ|う}ではございますまい、
\ruby{御取次}{お|とり|つぎ}の
\ruby{御言葉}{お|こと|ば}が
\ruby{足}{た}らんので、
\ruby{先生}{せん|せい}に
\ruby[g]{御理解}{おわかり}が
\ruby{無}{な}いのでしやう。
\ruby{{\GWI{u9060-k}}方}{えん|ぱう}だから
\ruby{行}{い}つて
\ruby{{\GWI{u9063-k}}}{や}らぬと、そんな
\ruby{事}{こと}を
\ruby{仰}{おつし}ある
\ruby{先生}{せん|せい}では
\ruby{無}{な}い、そんな
\ruby{無慈悲}{む|じ|ひ}な
\ruby{先生}{せん|せい}では
\ruby{無}{な}い。
…… 』

と、
\ruby{今}{いま}までは
\ruby{頭}{あたま}の
\ruby{低}{ひく}かりし
\ruby{男}{をとこ}の、
\ruby[g]{居丈高}{ゐたけだか}になつて、
\ruby{思}{おも}ひの
\ruby{外}{ほか}なる
\ruby{{\換字{強}}言}{しひ|ごと}を
\ruby{云}{い}い
\ruby{出}{いだ}せば、
\ruby{書生}{しよ|せい}は
\ruby{其}{そ}の
\ruby{意外}{いぐ|わい}なるに
\ruby{度}{ど}を
\ruby{失}{うしな}つて、
\ruby[g]{狼狽}{うろた}へながらも
\ruby{怫然}{ふつ|ぜん}として、
\ruby{急}{きふ}に
\ruby{遮}{さへぎ}り
\ruby{止}{とゞ}めんと、

『バ、バ、
\ruby{馬鹿}{ば|か}な
\ruby{事}{こと}を、』

と、
\ruby[g]{眞赤}{まつか}になりて
\ruby[g]{抗辯}{あらが}はんとしけるが、
\ruby{紫電閃}{し|でん|ひら}めきて
\ruby{出}{い}づるが
\ruby{如}{ごと}き
\ruby[g]{水野}{みづの}の
\ruby{恐}{おそ}ろしき
\ruby{眼}{め}に
\ruby{眼}{め}を
\ruby{見合}{み|あは}せて、
\ruby{睨}{にら}み
\ruby{殺}{ころ}さんばかりに
\ruby{我}{われ}を
\ruby{見据}{み|す}ゑたる
\ruby{其}{そ}の
\ruby{異}{あや}しき
\ruby{力}{ちから}に
\ruby[g]{所以無}{いはれな}くも
\ruby{氣壓}{け|お}され、
\ruby{云}{い}ひ
\ruby{甲斐無}{か|ひ|な}くも
\ruby{當}{あた}り
\ruby{難}{がた}くおぼえて、
\ruby{我知}{われ|し}らず
\ruby{面}{おもて}を
\ruby{背向}{そ|む}け
\ruby{言葉}{こと|ば}を
\ruby{吞}{の}みたり。
\ruby[g]{水野}{みづの}は
\ruby{相手}{あひ|て}のたぢろぎしに
\ruby{{\換字{緩}}}{ゆる}みを
\ruby{{\換字{呉}}}{く}れず、
\ruby{往來}{わう|らい}にも
\ruby{鳴}{な}り
\ruby{渡}{わた}れ、
\ruby{奥}{おく}にも
\ruby{響}{ひび}けと、いよ〳〵
\ruby{聲}{こゑ}を
\ruby{高}{たか}め、
\ruby{言葉}{こと|ば}を
\ruby{荒}{あら}くして、

『
\ruby{御當家}{こ|ち|ら}の
\ruby{先生}{せん|せい}は
\ruby{仁慈深}{な|さけ|ぶか}い
\ruby{先生}{せん|せい}だ、
\ruby{取次}{とり|つぎ}の
\ruby{君}{きみ}がまだ
\ruby{新參}{しん|ざん}で、
\ruby{御當家}{こ|ち|ら}の
\ruby{御風儀}{ご|ふう|ぎ}を
\ruby{知}{し}らんので、
\ruby{中{\GWI{u9014-k}}}{ちゆう|と}で
\ruby{間{\GWI{u9055-k}}}{ま|ちが}つた
\ruby{忠義立}{ちゆう|ぎ|だて}で
\ruby{計}{はか}らつて、
\ruby{其様}{そ|ん}な
\ruby{好}{い}い
\ruby{加減}{か|げん}な
\ruby{事}{こと}を
\ruby{御言}{お|い}ひのだ。
\ruby{御慈悲深}{お|じ|ひ|ぶか}い
\ruby[g]{此方}{こちら}の
\ruby{先生}{せん|せい}だもの、
\ruby{{\GWI{u9060-k}}方}{ゑん|ぽう}だつて
\ruby{來}{き}て
\ruby{下}{くだ}さるのだ。
\ruby{世間}{せ|けん}にも
\ruby{有}{あ}り
\ruby{觸}{ふ}れた
\ruby{藥賣}{くす|りう}り
\ruby{坊主}{ばう|ず}と、
\ruby[g]{此方}{こちら}の
\ruby{先生}{せん|せい}とは
\ruby{譯}{わけ}が
\ruby{{\GWI{u9055-k}}}{ちが}ふ。
\ruby{商賣}{しやう|ばい}づくばかりで
\ruby{病人}{びやう|にん}をいぢる、
\ruby{其様}{そ|ん}な
\ruby{卑劣}{ひ|れつ}くさい
\ruby{先生}{せん|せい}では
\ruby{無}{な}いのだ、
\ruby{先生}{せん|せい}の
\ruby{御性分}{ご|しやう|ぶん}の
\ruby{美}{うつく}しい
\ruby{御慈悲深}{お|じ|ひ|ぶか}いのは
\ruby{誰}{だれ}だつて
\ruby{知}{し}つて
\ruby{居}{ゐ}る。
\ruby{他人}{ひ|と}も
\ruby{知}{し}つて
\ruby{居}{ゐ}る、
\ruby{自分}{じ|ぶん}も
\ruby{知}{し}つて
\ruby{居}{ゐ}る。
\ruby{先生}{せん|せい}で
\ruby{無}{な}くちやあならんと
\ruby{云}{い}つて、
\ruby{御願}{お|ねが}ひ
\ruby{申}{まを}すのに
\ruby{來}{き}て
\ruby{下}{くだ}さらん、そんな
\ruby{仁慈}{な|さけ}の
\ruby{無}{な}い
\ruby{先生}{せん|せい}では
\ruby{無}{な}い。
\ruby{先生}{せん|せい}の
\ruby{御氣性}{ご|き|しやう}も
\ruby{知}{し}らないで、
\ruby{何}{なに}を
\ruby[g]{{\換字{寝}}惚}{ねとぼ}けた
\ruby{挨拶}{あい|さつ}をするのだ。
』

と、
\ruby{口}{くち}も
\ruby{開}{あ}かせず
\ruby{疊}{たヽ}みかけて、
\ruby{{\換字{猶}}}{なほ}も
\ruby{止}{と}め
\ruby{度}{ど}
\ruby{無}{な}く
\ruby{罵}{のゝし}らんとす。
\ruby{此}{こ}の
\ruby{時}{とき}
\ruby{藥局}{やく|きよく}の
\ruby{内}{うち}こと〳〵と
\ruby{音}{おと}して、
\ruby{物騒}{もの|さわ}がしき
\ruby{此場}{この|ば}の
\ruby[g]{様子}{やうす}を、
\ruby{何事}{なに|ごと}かと
\ruby{他}{た}の
\ruby{書生}{しよ|せい}の
\ruby{覗}{うかゞ}ひに
\ruby{來}{き}しとおぼしく、
\ruby{{\換字{叉}}}{また}
\ruby{今}{いま}の
\ruby{間}{ま}に
\ruby{來}{き}し
\ruby{二三人}{に|さん|にん}の
\ruby{藥取}{くすり|と}りは、こそ〳〵と
\ruby{隅}{すみ}の
\ruby{方}{かた}に
\ruby{潛}{ひそ}み
\ruby{居}{ゐ}て
\ruby{成行}{なり|ゆき}を
\ruby{見}{み}、はや
\ruby{門}{もん}の
\ruby{外}{そと}にはちらりほらりと、
\ruby{人}{ひと}さへ
\ruby{立}{た}ちて
\ruby{見居}{み|ゐ}るさまなり。

\ruby{書生}{しよ|せい}は
\ruby{心}{こヽろ}も
\ruby{心}{こヽろ}ならず、

『マア
\ruby{左様}{そ|う}
\ruby{大}{おほき}な
\ruby{聲}{こゑ}を
\ruby{立}{た}てゝは
\ruby{困}{こま}るぢや
\ruby{無}{な}いか。
』

と
\ruby{制}{せい}すれども
\ruby{耳}{みヽ}にも
\ruby{入}{い}るればこそ、

『つまり
\ruby{君}{きみ}のやうな
\ruby{取次}{とり|つぎ}は
\ruby{先生}{せん|せい}の
\ruby{不利{\換字{益}}}{ふ|た|め}だ、
\ruby{先生}{せん|せい}の
\ruby{{\換字{評}}{\換字{判}}}{ひやう|ばん}を
\ruby{惡}{わる}くする。
\ruby{{\換字{技}}{\換字{術}}}{わ|ざ}ばかり
\ruby{良}{よ}い
\ruby{先生}{せん|せい}では
\ruby{無}{な}い、
\ruby{御優}{お|やさ}しいので
\ruby{人徳}{にん|とく}のある
\ruby{先生}{せん|せい}をそれぢやあ
\ruby{臺無}{だい|な}しに
\ruby{仕}{し}て
\ruby{仕舞}{し|ま}ふでは
\ruby{無}{な}いか。
さつさと
\ruby{{\換字{猶}}一度}{も|いち|ど}
\ruby{奧}{おく}へ
\ruby{行}{い}つて
\ruby{願}{ねが}つて
\ruby{來}{き}てくれ。
\ruby{願}{ねが}ひ
\ruby{直}{なほ}して
\ruby{{\換字{呉}}}{く}れなければ
\ruby{此處}{こ|ゝ}は
\ruby{動}{うご}かん。
\ruby{病人}{びやう|にん}が
\ruby{先生}{せん|せい}で
\ruby{無}{な}ければと
\ruby{云}{い}つて
\ruby{首}{くび}を
\ruby{{\換字{延}}}{の}ばして
\ruby{待}{ま}つて
\ruby{居}{ゐ}るのだ、
\ruby{先生}{せん|せい}のお
\ruby{供}{とも}を
\ruby{仕}{し}て
\ruby{歸}{かへ}らなけりやあ
\ruby{此處}{こ|ゝ}は
\ruby{動}{うご}かん。
\ruby{書生}{しよ|せい}の
\ruby{癖}{くせ}に
\ruby{有}{あ}る
\ruby{間敷事}{ま|じき|こと}だ。
\ruby{碁}{ご}なぞに
\ruby{凝}{こ}つて
\ruby{居}{ゐ}るやうだから
\ruby{取次}{とり|つぎ}が
\ruby{間違}{ま|ちが}ふのだ。
さあ
\ruby{確乎}{しつ|かり}として
\ruby{先生}{せん|せい}に
\ruby{願}{ねが}つて見て
\ruby{{\換字{呉}}}{く}れ。
うるさい、\GWI{u1b048}つゝこい、とは
\ruby{何}{なん}の
\ruby{事}{こと}だ。
\GWI{u1b048}つゝこい
\ruby{人間}{にん|げん}に
\ruby{恨}{うら}まれたら、
\ruby{先生}{せん|せい}に
\ruby{飛}{と}んだ
\ruby{御{\GWI{u8ff7-k}}惑}{ご|めい|わく}が
\ruby{掛}{かヽ}らう、
\ruby{祟}{たゝ}りかね
\ruby{無}{な}いものだと
\ruby{思}{おも}ふか。
』

と、
\ruby[g]{次第}{しだい}〳〵に
\ruby{聲高}{こわ|だか}に
\ruby{云}{い}へば、
\ruby{門外}{もん|ぐわい}に
\ruby{人}{ひと}は
\ruby{愈々}{いよ|〳〵}
\ruby{嵩}{かさ}みて、
\ruby{奧}{おく}の
\ruby{方}{かた}は
\ruby{人}{ひと}の
\ruby{氣}{け}もせず
\ruby[g]{{\GWI{u975c-j}}謐}{しづか}になりぬ。

\ruby{時}{とき}に
\ruby{此室}{こ|ヽ}と
\ruby{奧}{おく}との
\ruby{劃域}{し|きり}はするりと
\ruby{開}{あ}いて、
\ruby{立出}{たち|いで}でたる
\ruby{{\換字{猶}}}{なほ}
\ruby{若}{わか}き
\ruby{此家}{この|や}の
\ruby{主人}{しゆ|じん}は、
\ruby{福々}{ふく|〴〵}しく
\ruby{肥}{ふと}りたる
\ruby{其顔}{その|かほ}に、
\ruby{莞爾}{にこ|やか}なる
\ruby{笑}{えみ}をつくりて、

『ヤ、
\ruby{取次}{とり|つぎ}のものを
\ruby{御叱}{お|しか}りでは
\ruby{恐}{おそ}れ
\ruby{入}{い}る。
\ruby{直}{すぐ}と
\ruby{今}{いま}から
\ruby{出}{で}ますから、さあ\kundoku{一}{ひ}{ト}{}\kundoku{足}{あし}{}{}
\ruby{御先}{お|さき}へ。
\ruby{相田}{あひ|だ}!、
\ruby{所}{ところ}は
\ruby{分}{わか}かつて
\ruby{居}{ゐ}るだらうな、ムゝ
\ruby{左様}{さ|う}か、
\ruby{直}{すぐ}と
\ruby{車}{くるま}の
\ruby{{\換字{支}}度}{し|たく}をさせろ。
』

と、
\ruby{卒直}{そつ|ちよく}に
\ruby[g]{水野}{みづの}に
\ruby{滿足}{まん|ぞく}を
\ruby{與}{あた}へぬ。

\ruby[g]{水野}{みづの}は、
\ruby{此}{こ}の
\ruby{己}{おのれ}に
\ruby{克}{か}つことを
\ruby{知}{し}つて
\ruby{非}{ひ}を
\ruby{{\GWI{u9042-k}}}{と}げんともせざる
\ruby{良醫}{りやう|い}の
\ruby{前}{まへ}に、
\ruby{心}{こゝろ}よりの
\ruby{感謝}{かん|しや}の
\ruby{禮}{れい}を
\ruby{深々}{ふか|〴〵}と
\ruby{施}{ほどこ}して、
\ruby{欣}{よろこ}び
\ruby{勇}{いさ}んで
\ruby[g]{室外}{おもて}に
\ruby{出}{い}でぬ。

\ruby{惡}{あ}しき
\ruby{兆}{しるし}かと
\ruby{忌}{いま}はしかりし
\ruby{彼}{か}の
\ruby{蛾}{が}の
\ruby{弄}{なぶ}りし
\ruby{電燈}{でん|とう}の
\ruby{下}{した}は
\ruby{去}{さ}つて、
\ruby{藍色滴}{らん|しよく|したヽ}るが
\ruby{如}{ごと}き
\ruby{澄}{す}みたる
\ruby{天}{そら}に、
\ruby{星}{ほし}は
\ruby{梨子地}{な|し|じ}を
\ruby{描}{か}きたらんやうに
\ruby{光}{ひか}り
\ruby{輝}{かがや}けるを、
\ruby{振}{ふ}り
\ruby{仰}{あお}ぎて
\ruby{眺}{なが}めたる
\ruby{可憐}{か|れん}の
\ruby[g]{水野}{みづの}は、
\ruby{我}{わ}が
\ruby{意}{こヽろ}の
\ruby{中}{うち}の
\ruby{其人}{その|ひと}のために、
\ruby{思}{おも}ふ
\ruby{事}{こと}
\ruby{{\GWI{u9042-k}}}{と}げたる
\ruby{嬉}{うれ}しさに
\ruby{頭}{かしら}
\ruby{高}{ たか}き
\ruby{心地}{こヽ|ち}して、
\ruby{水色}{みず|いろ}の
\ruby{光}{ひか}り
\ruby{特}{こと}に
\ruby{優}{すぐ}れたる
\ruby{一}{ひと}つの
\ruby{星}{ほし}に
\ruby{眼}{まなこ}を
\ruby{止}{とど}めて、
\ruby[g]{少時}{しばし}は
\ruby{人知}{ひと|し}らぬ
\ruby{胸}{むね}の
\ruby{{\換字{涼}}}{すゞ}しさを
\ruby{味}{あじは}ひたり。

