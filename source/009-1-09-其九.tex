\Entry{其九}

% メモ 校正終了 2024-03-30 2024-05-23 2024-06-17
\原本頁{53-9}%
\ruby[||j>]{頭}{かしら}を
\ruby{下}{さ}げ
\ruby[g]{言葉}{ことば }を
\ruby{低}{ひく}くして、
%
\ruby{頼}{たの}む
ほどは
\ruby{頼}{たの}み
\ruby{盡}{つく}せしを、
%
\ruby[g]{膠無}{にべな }く
\ruby{色}{いろ}なく
\ruby[g]{斷り}{ことわ }に
\ruby{斷}{ことわ}られたり。
%
\ruby{今}{いま}は
\ruby{復}{また}
\ruby[||j>]{言}{ものい}ふべき
\ruby[g]{餘地}{よ ち }も
\ruby{無}{な}からんを、
%
\ruby[g]{水野}{みづの }は
そも〳〵
\ruby{何}{なん}とせんとかする。

\原本頁{54-2}%
\ruby{水}{みづ}をもて
\ruby{解}{と}くべからざるものは
\ruby{火}{ひ}をもて
\ruby{熔}{と}かすべし、
%
\ruby{刀}{たう}をもて
\ruby{截}{き}り
\ruby{{\換字{難}}}{がた}きものは
\ruby{槌}{つち}をもて
\ruby{碎}{くだ}き
\ruby{得}{え}ん。
%
\ruby{求}{もと}めて
\ruby{已}{や}まぬ
\ruby[g]{願望}{ねがひ }の
\ruby{心}{こゝろ}あれば、
%
おのづと
\ruby{働}{はたら}く
\ruby[g]{智慧}{ち ゑ }の
\ruby{眼}{まなこ}は、
%
\ruby{我}{わ}が
\ruby{思}{おも}へる
\ruby{地}{ち}に
\ruby{到}{いた}らんとするに
\ruby[g]{{\換字{平}}和}{おだやか}なる
\ruby{路}{みち}を
\ruby{取}{と}ることの
\ruby[g]{甲{\換字{斐}}}{か ひ }
\ruby{無}{な}きを
\ruby{悟}{さと}りたらん
\ruby[<j>]{曉}{あかつき}、
%
いかで
\ruby{{\換字{猶}}}{なほ}
\ruby{別}{べつ}に
\ruby{峻}{さか}しき
\ruby{一}{ひ}ト
\ruby{條}{すぢ}の
\ruby{徑}{こみち}ありて
\ruby[g]{其處}{そ こ }に
\ruby{{\換字{通}}}{つう}ずるを
\ruby[g]{見出}{み いだ}さゞらんや。
%
\ruby[g]{水野}{みづの }は
\ruby{今}{いま}その
\ruby{峻}{さか}しきを
\ruby[g]{見出}{み いだ}して
\ruby{攀}{よ}ぢ
\ruby{上}{のぼ}らんとするなり。
%
\ruby{火}{ひ}の
\ruby{力}{ちから}、
%
\ruby{槌}{つち}の
\ruby{力}{ちから}を
\ruby{試}{こゝろ}みんとするなり。

\原本頁{54-9}%
\ruby{其}{そ}の
\ruby{顏}{かほ}つきの
\ruby{變}{かは}れるが
\ruby{如}{ごと}くに、
%
\ruby[g]{言葉}{ことば }の
\ruby[g]{調子}{てうし }も
\ruby{俄}{にはか}に
\ruby{變}{かは}り、
%
\ruby{聲}{こゑ}も
\換字{志}たゝかに
\ruby{大}{おおき}くなりぬ。

% \原本頁{54-11}%
『
いよ〳〵
\ruby[g]{先生}{せんせい}は
\ruby{御來臨}{お|い|で}
\ruby{下}{くだ}さらんと
\ruby{仰}{おつし}あるのですか。
%
イヤ、
%
それは
\ruby[g]{失禮}{しつれい}ながら
\ruby[g]{左樣}{さ う }では
ございますまい、
%
\ruby{御取次}{お|とり|つぎ}の
\ruby{御言葉}{お|こと|ば}が
\ruby{足}{た}らんので、
%
\ruby[g]{先生}{せんせい}に
\ruby{御理解}{お|わか|り}が
\ruby{無}{な}いのでしやう。
%
\ruby[g]{{\換字{遠}}方}{ゑんぱう}だから
\ruby{行}{い}つて
\ruby{{\換字{遣}}}{や}らぬと、
%
そんな
\ruby{事}{こと}を
\ruby{仰}{おつし}ある
\ruby[g]{先生}{せんせい}では
\ruby{無}{な}い、
%
そんな
\ruby{無慈悲}{む|じ|ひ}な
\ruby[g]{先生}{せんせい}では
\ruby{無}{な}い。
%
‥‥』

\原本頁{55-5}%
と、
%
\ruby{今}{いま}までは
\ruby{頭}{あたま}の
\ruby{低}{ひく}かりし
\ruby{男}{をとこ}の、
%
\ruby{居{\換字{丈}}高}{ゐ|たけ|だか}になつて、
%
\ruby{思}{おも}ひの
\ruby{外}{ほか}なる
\ruby[g]{{\換字{強}}言}{しひごと}を
\ruby{云}{い}ひ
\ruby{出}{いだ}せば、
%
\ruby[g]{書生}{しよせい}は
\ruby{其}{そ}の
\ruby[g]{意外}{いぐわい}なるに
\ruby{度}{ど}を
\ruby{失}{うしな}つて、
%
\ruby[g]{{\換字{狼}}狽}{うろた }へながらも
\ruby[g]{怫然}{ふつぜん}として、
%
\ruby{急}{きふ}に
\ruby{{\換字{遮}}}{さへぎ}り
\ruby{止}{とゞ}めんと、

\原本頁{55-8}%
『
バ、
%
バ、
%
\ruby[g]{馬鹿}{ば か }な
\ruby{事}{こと}を、
』

\原本頁{55-9}%
と、
%
\ruby[g]{眞赤}{まつか }になりて
\ruby[g]{抗辯}{あらが }はんとしけるが、% 弁 瓣 辦 辧 辨 辩 (辯)
%
\ruby[g]{紫電}{し でん}
\ruby{閃}{ひら}めきて
\ruby{出}{い}づるが
\ruby{如}{ごと}き
\ruby[g]{水野}{みづの }が
\ruby{恐}{おそ}ろしき
\ruby{眼}{め}に
\ruby{眼}{め}を
\ruby[g]{見合}{み あは}せて、
%
\ruby{睨}{にら}み
\ruby{殺}{ころ}さん
ばかりに
\ruby{我}{われ}を
\ruby[g]{見据}{み す }ゑたる
\ruby{其}{そ}の
\ruby{異}{あや}しき
\ruby{力}{ちから}に
\ruby[g]{{\換字{所}}以}{いはれ }
\ruby{無}{な}くも
\ruby[g]{氣壓}{け お }され、
%
\ruby{云}{い}ひ
\ruby[g]{甲{\換字{斐}}}{が ひ }
\ruby{無}{な}くも
\ruby{當}{あた}り
\ruby{{\換字{難}}}{がた}く
おぼえて、
%
\ruby{我}{われ}
\ruby{知}{し}らず
\ruby{面}{おもて}を
\ruby[g]{背向}{そ む }け
\ruby[g]{言葉}{ことば }を
\ruby{吞}{の}みたり。
%
\原本頁{56-2}%
\ruby[g]{水野}{みづの }は
\ruby[g]{相手}{あひて }の
たぢろぎしに
\ruby{{\換字{緩}}}{ゆる}みを
\ruby{吳}{く}れず、
%
\ruby[g]{往來}{わうらい}にも
\ruby{鳴}{な}り
\ruby{渡}{わた}れ、
%
\ruby{奧}{おく}にも
\ruby{響}{ひゞ}けと、
%
いよ〳〵
\ruby{聲}{こゑ}を
\ruby{高}{たか}め、
%
\ruby[g]{言葉}{ことば }を
\ruby{荒}{あら}くして、

\原本頁{56-4}%
『
\ruby{御當家}{こ|ち|ら}の
\ruby[g]{先生}{せんせい}は
\ruby[g]{仁慈}{なさけ }
\ruby{深}{ぶか}い
\ruby[g]{先生}{せんせい}だ、
%
\ruby[g]{取次}{とりつぎ}の
\ruby{君}{きみ}がまだ
\ruby[g]{新參}{しんざん}で、
%
\ruby{御}{こ}
\ruby{當}{ち}
\ruby{家}{ら}の
\ruby{御風儀}{ご|ふう|ぎ}を
\ruby{知}{し}らんので、
%
\ruby[g]{中{\換字{途}}}{ちゆうと}で
\ruby[g]{間{\換字{違}}}{ま ちが}つた
\ruby[g]{忠義}{ちゆうぎ}% 原本通り(ちゆう)(国会図書館 コマ番号 32/134 p56 l5)
\ruby{立}{だて}で
\ruby{計}{はか}らつて
\改行% 校正作業の簡略化のため
、
%
\原本頁{56-6}\改行%
\ruby[g]{其樣}{そ ん }な
\ruby{好}{い}い
\ruby[g]{加減}{か げん}な
\ruby{事}{こと}を
\ruby[g]{御言}{お い }ひのだ。
%
\ruby{御慈悲}{お|じ|ひ}
\ruby{深}{ぶか}い
\ruby[g]{此方}{こちら }の% ルビ調整(原本通り)
\ruby[g]{先生}{せんせい}だもの、
%
\ruby[g]{{\換字{遠}}方}{ゑんぱう}だつて
\ruby{來}{き}て
\ruby{下}{くだ}さるのだ。
%
\ruby[g]{世間}{せ けん}に
\ruby{有}{あ}り
\ruby{觸}{ふ}れた
\ruby[g]{藥賣}{くすりう}り
\ruby[g]{坊主}{ばうず }
\原本頁{56-8}\改行%
と、
%
\ruby[g]{此方}{こちら }の% ルビ調整(原本通り)
\ruby[g]{先生}{せんせい}とは
\ruby{譯}{わけ}が
\ruby{{\換字{違}}}{ちが}ふ。
%
\ruby[||j>]{商}{しやう}
\ruby[||j>]{賣}{ ばい}づく
% \ruby{商賣}{しやう|ばい}づく
ばかりで
\ruby[||j>]{病}{びやう}
\ruby[||j>]{人}{ にん}を
% \ruby{病人}{びやう|にん}を
いぢる
\改行% 校正作業の簡略化のため
、
%
\原本頁{56-9}\改行%
\ruby[g]{其樣}{そ ん }な
\ruby[g]{卑劣}{ひ れつ}くさい
\ruby[g]{先生}{せんせい}では
\ruby{無}{な}いのだ、
%
\ruby[g]{先生}{せんせい}の
\ruby{御性{\換字{分}}}{ご|しやう|ぶん}の
\ruby{美}{うつく}しい
\ruby{御慈悲}{お|じ|ひ}
\ruby{深}{ぶか}いのは
\ruby{誰}{たれ}でも% 国会図書館 コマ番号 32/134 p56 l10
\ruby{知}{し}つて
\ruby{居}{ゐ}る。
%
\ruby[g]{他人}{ひ と }も
\ruby{知}{し}つて
\ruby{居}{ゐ}る、
%
\ruby[g]{自{\換字{分}}}{じ ぶん}も
\ruby{知}{し}つて
\ruby{居}{ゐ}る。
%
\ruby[g]{先生}{せんせい}で
\ruby{無}{な}くちやあ
ならんと
\ruby{云}{い}つて、
%
\ruby[g]{御願}{お ねが}ひ
\ruby{申}{まを}すのに
\原本頁{57-1}\改行%
\ruby{來}{き}て
\ruby{下}{くだ}さらん、
%
そんな
\ruby[g]{仁慈}{なさけ }の
\ruby{無}{な}い
\ruby[g]{先生}{せんせい}では
\ruby{無}{な}い。
%
\ruby[g]{先生}{せんせい}の
\ruby{御氣性}{ご|き|しやう}も
\ruby{知}{し}らないで、
%
\ruby{何}{なに}を
\ruby[g]{寢惚}{ね とぼ}けた
\ruby[g]{挨拶}{あいさつ}を
するのだ。
』

\原本頁{57-3}%
と、
%
\ruby{口}{くち}も
\ruby{開}{あ}かせず
\ruby{疊}{たゝ}みかけて、
%
\ruby{{\換字{猶}}}{なほ}も
\ruby{止}{と}め
\ruby{度}{ど}
\ruby{無}{な}く
\ruby{罵}{のゝし}らんとす。
%
\ruby{此}{こ}の
\ruby{時}{とき}
\ruby[||j>]{藥}{やく}
\ruby[||j>]{局}{きよく}の
% \ruby{藥局}{やく|きよく}の
\ruby{内}{うち}
こと〳〵と
\ruby{音}{おと}して、
%
\ruby[g]{物騷}{ものさわ}がしき
\ruby[g]{此場}{このば }の% 原文通り「場」
\ruby[g]{樣子}{やうす }を、
%
\ruby[g]{何事}{なにごと}かと
\ruby{他}{た}の
\ruby[g]{書生}{しよせい}の
\ruby{覗}{うかゞ}ひに
\ruby{來}{き}しと
おぼしく、
%
\ruby{{\換字{又}}}{また}
\ruby{今}{いま}の
\ruby{間}{ま}に
\ruby{來}{き}し
\ruby{二三人}{に|さん|にん}の
\ruby[g]{藥取}{くすりと}りは、
%
こそ〳〵と
\ruby{隅}{すみ}の
\ruby{方}{かた}に
\ruby{潛}{ひそ}み% 【潛 u6f5b 「先先」】【潜 u6f5c 「夫夫」】併用されている
\ruby{居}{ゐ}て
\ruby[g]{成行}{なりゆき}を
\ruby{見}{み}、
%
はや
\ruby{門}{もん}の
\ruby{外}{そと}には
ちらり
ほらりと、
%
\ruby{人}{ひと}さへ
\ruby{立}{た}ちて
\ruby[g]{見居}{み ゐ }るさまなり。

\原本頁{57-8}%
\ruby[g]{書生}{しよせい}は
\ruby{心}{こゝろ}も
\ruby{心}{こゝろ}ならず、

\原本頁{57-9}%
『
マア
\ruby[g]{左樣}{そ う }
\ruby{大}{おほき}な
\ruby{聲}{こゑ}を
\ruby{立}{た}てゝは
\ruby{困}{こま}るぢや
\ruby{無}{な}いか。
』

\原本頁{57-10}%
と、
\ruby{制}{せい}すれども
\ruby{耳}{みゝ}にも
\ruby{入}{い}るれば
こそ、

\原本頁{57-11}%
『
つまり
\ruby{君}{きみ}のやうな
\ruby[g]{取次}{とりつぎ}は
\ruby[g]{先生}{せんせい}の
\ruby{不利益}{ふ|た|め}だ、
%
\ruby[g]{先生}{せんせい}の
\ruby[||j>]{{\換字{評}}}{ひやう}
\ruby[||j>]{{\換字{判}}}{ ばん}を
% \ruby{{\換字{評}}{\換字{判}}}{ひやう|ばん}を
\ruby{惡}{わる}くする。
%
\ruby[g]{{\換字{技}}{\換字{術}}}{わ ざ }ばかり
\ruby{良}{よ}い
\ruby[g]{先生}{せんせい}では
\ruby{無}{な}い、
%
\ruby[g]{御優}{お やさ}しいので
\ruby[g]{人徳}{にんとく}のある
\原本頁{58-2}\改行%
\ruby[g]{先生}{せんせい}を
それぢやあ
\ruby[g]{臺無}{だいな }しに
\ruby{仕}{し}て
\ruby[g]{仕舞}{し ま }ふでは
\ruby{無}{な}いか、
%
さつさと
\ruby{{\換字{猶}}}{も}
\ruby[g]{一度}{いちど }
\ruby{奧}{おく}へ
\ruby{行}{い}つて
\ruby{願}{ねが}つて
\ruby{來}{き}てくれ。
%
\ruby{願}{ねが}ひ
\ruby{直}{なほ}して
\ruby{吳}{く}れなければ
\ruby[g]{此處}{こ ゝ }は
\ruby{動}{うご}かん。
%
\ruby[||j>]{病}{びやう}
\ruby[||j>]{人}{ にん}が
% \ruby{病人}{びやう|にん}が
\ruby[g]{先生}{せんせい}で
\ruby{無}{な}ければと
\ruby{云}{い}つて
\ruby{首}{くび}を
\ruby{{\換字{延}}}{の}ばして
\ruby{待}{ま}つて
\原本頁{58-5}\改行%
\ruby{居}{ゐ}るのだ、
%
\ruby[g]{先生}{せんせい}の
\ruby[g]{御供}{お とも}を
\ruby{仕}{し}て
\ruby{歸}{かへ}らなけりやあ
\ruby[g]{此處}{こ ゝ }は
\ruby{動}{うご}かん。
%
\ruby[g]{書生}{しよせい}の
\ruby{癖}{くせ}に
\ruby{有}{あ}る
\ruby[g]{間敷}{ま じき}
\ruby{事}{こと}だ。
%
\ruby{碁}{ご}なぞに
\ruby{凝}{こ}つて
\ruby{居}{ゐ}るやうだから
\ruby[g]{取次}{とりつぎ}が
\原本頁{58-7}\改行%
\ruby[g]{間{\換字{違}}}{ま ちが}ふのだ。
%
さあ
\ruby[g]{確乎}{しつかり}として
\ruby[g]{先生}{せんせい}に
\ruby{願}{ねが}つて
\ruby{見}{み}て
\ruby{吳}{く}れ。
%
うるさい
\改行% 校正作業の簡略化のため
、
%
\原本頁{58-8}\改行%
\換字{志}つゝこい、
%
とは
\ruby{何}{なん}の
\ruby{事}{こと}だ。
%
\換字{志}つゝこい
\ruby[g]{人間}{にんげん}に
\ruby{恨}{うら}まれたら、
%
\ruby[g]{先生}{せんせい}に
\ruby{飛}{と}んだ
\ruby{御{\換字{迷}}惑}{ご|めい|わく}が
\ruby{掛}{かゝ}らう、
%
\ruby{祟}{たゝ}りかね
\ruby{無}{な}いものだと
\ruby{思}{おも}ふか。
』

\原本頁{58-10}%
と、
%
\ruby[g]{次第}{し だい}〳〵に
\ruby[g]{聲高}{こわだか}に
\ruby{云}{い}へば、
%
\ruby[||j>]{門}{もん}
\ruby[||j>]{外}{ぐわい}に
% \ruby{門外}{もん|ぐわい}に
\ruby{人}{ひと}は
\ruby[g]{愈々}{いよ〳〵}
\ruby{嵩}{かさ}みて、
%
\ruby{奧}{おく}の
\ruby{方}{かた}は
\ruby{人}{ひと}の
\ruby{氣}{け}もせず
\ruby[g]{靜謐}{しづか }になりぬ。

\原本頁{59-1}%
\ruby{時}{とき}に
\ruby[g]{此室}{こ ゝ }と
\ruby{奧}{おく}との
\ruby[g]{劃域}{しきり }は
するりと
\ruby{開}{あ}いて、
%
\ruby[g]{立出}{たちい }でたる
\ruby{{\換字{猶}}}{なほ}
\ruby{{\換字{若}}}{わか}き
\ruby{此}{この}
\原本頁{59-2}\改行%
\ruby{家}{や}の
\ruby[g]{主人}{しゆじん}は、
%
\ruby[g]{福々}{ふく〴〵}しく
\ruby{肥}{ふと}りたる
\ruby{其}{その}
\ruby{顏}{かほ}に、
%
\ruby[g]{莞爾}{にこやか}なる
\ruby{笑}{ゑみ}を
つくりて
\改行% 校正作業の簡略化のため
、

\原本頁{59-3}%
『
ヤ、
%
\ruby[g]{取次}{とりつぎ}のものを
\ruby[g]{御叱}{お しか}りでは
\ruby{恐}{おそ}れ
\ruby{入}{い}る。
%
\ruby{直}{すぐ}と
\ruby{今}{いま}から
\ruby{出}{で}ますから、
%
さあ
\ruby{一}{ひ}
ト
\ruby{足}{あし}
\ruby[g]{御先}{お さき}へ。
%
\ruby[g]{相田}{あひだ }!、
%
\ruby{{\換字{所}}}{ところ}は
\ruby{{\換字{分}}}{わか}つて
\ruby{居}{ゐ}るだらうな、
%
ム
\原本頁{59-5}\改行%
ヽ
\ruby[g]{左樣}{さ う }か、
%
\ruby{直}{すぐ}と
\ruby{車}{くるま}の
\ruby[g]{支度}{し たく}を
させろ。
』

\原本頁{59-6}%
と、
%
\ruby[||j>]{卒}{そつ}
\ruby[||j>]{直}{ちよく}に
% \ruby{卒直}{そつ|ちよく}に
\ruby[g]{水野}{みづの }に
\ruby[g]{滿足}{まんぞく}を
\ruby{與}{あた}へぬ。

\原本頁{59-7}%
\ruby[g]{水野}{みづの }は、
%
\ruby{此}{こ}の
\ruby{己}{おのれ}に
\ruby{克}{か}つことを
\ruby{知}{し}つて
\ruby{非}{ひ}を
\ruby{{\換字{遂}}}{と}げん
ともせざる
\ruby[g]{良醫}{りやうい}の
\ruby{{\換字{前}}}{まへ}に、
%
\ruby{心}{こゝろ}よりの
\ruby[g]{{\換字{感}}謝}{かんしや}の
\ruby{禮}{れい}を
\ruby[g]{深々}{ふか〴〵}と
\ruby{施}{ほどこ}して、
%
\ruby{欣}{よろこ}び
\ruby{勇}{いさ}んで
\ruby[g]{室外}{おもて }に
\ruby{出}{い}でぬ。

\原本頁{59-10}%
\ruby{惡}{あし}き
\ruby{兆}{しるし}かと
\ruby{忌}{いま}はしかりし
\ruby{彼}{か}の
\ruby{蛾}{が}の
\ruby{弄}{なぶ}りし
\ruby[g]{電燈}{でんとう}の
\ruby{下}{した}は
\ruby{去}{さ}つて、
%
\ruby[||j>]{藍}{らん}
\ruby[||j>]{色}{しよく}
% \ruby{藍色}{らん|しよく}
\ruby[||g>]{滴}{したゝ}るが% ルビ調整(長いルビ対策)(る)を親文字に加える
\ruby{如}{ごと}き
\ruby{澄}{す}みたる
\ruby{天}{そら}に、
%
\ruby{星}{ほし}は
\ruby{梨子地}{な|し|じ}を
\ruby{描}{か}きたらんやうに
\ruby{光}{ひか}り
\ruby{輝}{かゞや}けるを、
%
\ruby{振}{ふ}り
\ruby{仰}{あふ}ぎて
\ruby{眺}{なが}めたる
\ruby[g]{可憐}{か れん}の
\ruby[g]{水野}{みづの }は、
%
\ruby{我}{わ}が
\ruby{意}{こゝろ}の
\ruby{中}{うち}の
\ruby{其}{その}
\ruby{人}{ひと}のために、
%
\ruby{思}{おも}ふ
\ruby{事}{こと}
\ruby{{\換字{遂}}}{と}げたる
\ruby{嬉}{うれ}しさに
\ruby[||j>]{頭}{かしら}
\ruby[||j>]{高}{ たか}き
\ruby[g]{心地}{こゝち }して、
%
\ruby[g]{水色}{みづいろ}の
\ruby{光}{ひか}り
\ruby{特}{こと}に
\ruby{優}{すぐ}れたる
\ruby{一}{ひと}つの
\ruby{星}{ほし}に
\ruby{眼}{まなこ}を
\ruby{止}{とゞ}めて、
%
\ruby[g]{少時}{しばし }は
\ruby[g]{人知}{ひとし }らぬ
\ruby{胸}{むね}の
\ruby{凉}{すゞ}しさを
\ruby{味}{あぢは}ひたり。
