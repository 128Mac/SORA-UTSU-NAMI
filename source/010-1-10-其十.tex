\Entry{其十}

\ruby{先挽後推}{さき|びき|あと|おし}の
\ruby{勢}{いきほひ}よく、
\ruby{矢}{や}を
\ruby{射}{い}る
\ruby{如}{ごと}くに
\ruby{走}{はし}れる
\ruby{相良}{さが|ら}の
\ruby{車}{くるま}は、
\ruby{長橋}{ちやう|けう}を
\ruby{東}{ひがし}に
\ruby{渡}{わた}つて
\ruby{小梅}{こ|うめ}にかゝり、
\ruby{引舟{\換字{通}}}{ひき|ふね|どほ}りを
\ruby{眞直}{まつ|すぐ}に
\ruby{北}{きた}へと、
\ruby{夜風}{よ|かぜ}のやや
\ruby{{\換字{寒}}}{さむ}きを
\ruby{衝}{つ}いて
\ruby{{\換字{進}}}{すゝ}みに
\ruby{{\換字{進}}}{すゝ}みぬ。
\ruby{{\換字{道}}}{みち}は
\ruby{砥}{と}の
\ruby{如}{ごと}し、
\ruby{人}{ひと}の
\ruby{往來}{ゆき|き}は
\ruby{無}{な}し、
\ruby{車夫}{しや|ふ}は
\ruby{脚一杯}{あし|いつ|ぱい}に
\ruby{駈}{か}くるほどに、おほよその
\ruby{二里}{に|り}を
\ruby{瞬}{またゝ}く
\ruby{間}{ま}に
\ruby{{\換字{過}}}{す}ぎて、
\ruby{忽地}{たち|まち}にして
\ruby{目}{め}ざす
\ruby{四}{よ}ツ
\ruby{木}{ぎ}へと
\ruby{着}{つ}きぬ。

\ruby{病人}{びやう|にん}の
\ruby{大切}{たい|せつ}さは
\ruby{貧富}{ひん|ぷ}に
\ruby{關}{かゝ}はらぬ
\ruby{事}{こと}ながら、
\ruby{市街}{ま|ち}
\ruby{離}{はな}れたる
\ruby{{\換字{遠}}}{とほ}きところより、
\ruby{夜}{よ}にさへ
\ruby{入}{い}りたるに
\ruby{無理{\換字{強}}}{む|り|じひ}に
\ruby{{\換字{強}}}{し}ひて、
\ruby{我}{わ}が
\ruby{先生}{せん|せい}を
\ruby{{\換字{迎}}}{むか}へたるは、
\ruby{田舎}{ゐな|か}とは
\ruby{云}{い}へ、
\ruby{定}{さだ}めし
\ruby{門構}{もん|がま}への
\ruby{立派}{りつ|ぱ}に、
\ruby{庭{\換字{前}}}{には|さき}
\ruby{廣}{ひろ}く、がつしりとしたる
\ruby{槻柱}{けやき|ばしら}の
\ruby{太}{ふと}きが、
\ruby{二尺}{に|しやく}も
\ruby{厚}{あつ}さのある
\ruby{茅葺屋根}{かや|ぶき|や|ね}のいと
\ruby{高}{たか}く
\ruby{大}{おほき}なるを
\ruby{支}{さゝ}へたるやうの
\ruby{家}{いへ}ならんと、
\ruby{車夫}{しや|ふ}は
\ruby{心}{こゝろ}の
\ruby{中}{うち}に
\ruby{算}{つも}り
\ruby{居}{ゐ}けるが、
\ruby{{\換字{分}}}{わか}り
\ruby{{\換字{兼}}}{か}ぬる
\ruby{闇}{やみ}の
\ruby{村逕}{むら|みち}を
\ruby{{\換字{迷}}}{まよ}ひ〳〵て、やうやくに
\ruby{{\換字{尋}}}{たづ}ね
\ruby{當}{あ}てたるは
\ruby{是}{これ}は
\ruby{如何}{い|か}な
\ruby{事}{こと}、
\ruby{{\換字{寒}}竹}{かん|ちく}の
\ruby{藪疊}{やぶ|だゝみ}の
\ruby{不體裁}{ぶ|ざ|ま}に
\ruby{歪}{ゆが}みたる
\ruby{其}{そ}の
\ruby{構}{かまへ}の
\ruby{中}{うち}こそは
\ruby{意外}{い|ぐわい}に
\ruby{濶}{ひろ}けれ、
\ruby{{\換字{空}}}{むな}しく
\ruby{明}{あ}け
\ruby{置}{お}く
\ruby{地}{ち}を
\ruby{惜}{をし}んでか、
\ruby{{\換字{通}}}{かよ}ひ
\ruby{路}{ぢ}をも
\ruby{埋}{うづ}むるまでに
\ruby{作}{つく}りたる
\ruby{芋}{いも}の
\ruby{圃}{はたけ}の
\ruby{奧}{おく}に、
\ruby{微}{かす}けき
\ruby{星}{ほし}のひかりを
\ruby{{\換字{浴}}}{あ}びて
\ruby{黑}{くろ}みて
\ruby{立}{た}てる、
\ruby{見}{み}るからが
\ruby{悲}{かな}しき
\ruby{草}{くさ}の
\ruby{屋}{や}なり。

\ruby{餘}{あま}りの
\ruby{思}{おも}はくの
\ruby{{\換字{違}}}{ちが}ひの
\ruby{忌々}{いま|〳〵}しくてや、
\ruby{車夫}{しや|ふ}は
\ruby{憚}{はゞか}り
\ruby{氣}{げ}
\ruby{無}{な}く
\ruby{人力車}{く|る|ま}を
\ruby{挽}{ひ}き
\ruby{入}{い}るれば、
\ruby{車輪}{しや|りん}に
\ruby{觸}{ふ}るゝ
\ruby{芋}{いも}の
\ruby{葉}{は}は
\ruby{左右}{さ|いう}に
\ruby{開}{ひら}けて、
\ruby{湛}{たゝ}へられし
\ruby{露}{つゆ}の
\ruby{珠}{たま}は
\ruby{墜}{お}ちて
\ruby{聲}{こゑ}あり。

\ruby{人}{ひと}ありや
\ruby{無}{な}しや
\ruby{岑閑}{しん|かん}として、たゞ
\ruby{燈}{ひ}のみ
\ruby{洩}{も}るゝ
\ruby{板{\換字{戸}}}{いた|ど}を
\ruby{敲}{たゝ}き
\ruby{驚}{おどろ}かしつゝ
\ruby{車夫}{しや|ふ}は
\ruby{聲}{こゑ}
\ruby{明}{あき}らかにそれと
\ruby{云}{い}ひ
\ruby{入}{い}るれば、
\ruby{何}{なに}を
\ruby{擱}{さしお}きても
\ruby{飛}{と}んで
\ruby{出}{い}でゝ、
\ruby{喜}{よろこ}び〳〵て
\ruby{{\換字{迎}}}{むか}へ
\ruby{入}{い}るべきを、
\ruby{是}{これ}はまた
\ruby{何}{なん}たる
\ruby{事}{こと}ぞ
\ruby{沈着}{おち|つ}き
\ruby{拂}{はら}つて、

『ハア、
\ruby{左樣}{さ|う}ですかい!。
』

と、
\ruby{田舎}{ゐな|か}
\ruby{詞}{ことば}の
\ruby{素氣無}{す|げ|な}く
\ruby{答}{こた}へたるのみにて
\ruby{嬉}{うれ}しき
\ruby{顏}{かほ}もせねば、
\ruby{{\換字{請}}}{しやう}じ
\ruby{入}{い}れんともせず、
\ruby{折}{をり}から
\ruby{自裂}{は|じ}け
\ruby{{\換字{兼}}}{か}ねたる
\ruby{大豆}{ま|め}の
\ruby{莢}{さや}を
\ruby{取}{と}るにやあらん、
\ruby{箕}{み}を
\ruby{{\換字{前}}}{まへ}にして
\ruby{乾}{かは}きたる
\ruby{豆}{まめ}を
\ruby{弄}{いぢ}り
\ruby{居}{ゐ}し
\ruby{婆}{ばゞ}の、
\ruby{面}{おもて}は
\ruby{赭黄色}{あか|き|いろ}く
\ruby{焦}{や}け
\ruby{皺}{しわ}びて、
\ruby{髮}{かみ}は
\ruby{天蠶糸屑}{て|ぐ|す|くず}の
\ruby{如}{ごと}く
\ruby{白}{しろ}く
\ruby{光}{ひか}るが
\ruby{{\換字{交}}}{まじ}れる、
\ruby{年}{とし}の
\ruby{頃}{ころ}は
\ruby{六十}{ろく|じう}ばかりなるが、
\ruby{不承不承}{ふ|しよう|ぶ|しよう}に
\ruby{身}{み}を
\ruby{起}{おこ}して
\ruby{{\換字{戸}}口}{と|ぐち}に
\ruby{立塞}{たち|ふさ}がり、

『
\ruby{病人}{びやう|にん}は
\ruby{此處}{こ|ゝ}には
\ruby{居}{を}りましねえ。
\ruby{別室}{はな|れ}の
\ruby{方}{はう}に
\ruby{寢}{ね}て
\ruby{居}{を}りますから、
\ruby{直}{すぐ}とそつちらへ
\ruby{御座}{ご|ざ}らしつて
\ruby{下}{くだ}さい。
\ruby{暗}{くら}くつて
\ruby{{\換字{分}}}{わか}りますまいが
\ruby{足元}{あし|もと}は
\ruby{好}{い}いでがす。
\ruby{家}{うち}へさへ
\ruby{付}{つ}いて
\ruby{{\換字{廻}}}{まは}れば
\ruby{直}{ぢき}でがすよ。
あ、\換字{志}かし
\ruby{{\換字{菜}}}{な}
\ruby{圃}{ばたけ}へでも
\ruby{轉}{ころ}げられると
\ruby{詰}{つま}らない。
\ruby{水野}{みづ|の}さんが
\ruby{後}{あと}になつたゞから
\ruby{仕方}{し|かた}が
\ruby{無}{な}い、
\ruby{妾}{わし}が
\ruby{案内}{あん|ない}を
\ruby{仕}{し}てあげやう。
ヤ、
\ruby{車夫}{くる|まや}さん、
\ruby{提灯}{ちやう|ちん}があるの、
\ruby{其}{そ}の
\ruby{提灯}{ちやう|ちん}を
\ruby{妾}{わし}に
\ruby{貸}{か}さつせえ。
さあ
\ruby{先生}{せん|せい}さん、
\ruby{妾}{わし}に
\ruby{隨}{つ}いて
\ruby{御坐}{ご|ざ}らつせえ。
』

と、
\ruby{藁草履}{わら|ざう|り}つゝかけて
\ruby{先}{さき}に
\ruby{立}{た}つたり。
\ruby{相良}{さが|ら}は
\ruby{是非無}{ぜ|ひ|な}く
\ruby{後}{あと}に
\ruby{隨}{つ}きて、
\ruby{家}{いへ}の
\ruby{横手}{よこ|て}を
\ruby{斜}{なゝめ}に
\ruby{奧}{おく}へ、
\ruby{此方}{こな|た}には
\ruby{燃料}{たき|れう}の
\ruby{柴木}{しば|き}の
\ruby{積}{つ}まれ、
\ruby{彼方}{かな|た}には
\ruby{玉蜀黍幹}{たう|もろ|こし|がら}の
\ruby{埒無}{らち|な}く
\ruby{置}{お}かれなどしたる
\ruby{間}{あひだ}を
\ruby{縫}{ぬ}ひて、さて、
\ruby{下}{した}は
\ruby{夏蒔}{なつ|まき}の
\ruby{{\換字{菜}}}{な}の
\ruby{圃}{はたけ}の
\ruby{細徑}{ほそ|みち}の
\ruby{滑}{すべ}り
\ruby{易}{やす}く、
\ruby{上}{うへ}は
\ruby{{\換字{柿}}}{かき}の
\ruby{樹}{き}の
\ruby{幾本}{いく|ほん}の
\ruby{枝低}{えだ|ひく}くして
\ruby{帽子}{ばう|し}
\ruby{危}{あやふ}きところを
\ruby{{\換字{過}}}{す}ぐれば、
\ruby{{\換字{前}}}{まへ}の
\ruby{家}{いへ}よりは
\ruby{彼是}{かれ|これ}
\ruby{二十間餘}{に|じう|けん|あま}りも
\ruby{距}{はな}れたりとおぼしきところに、
\ruby{椎}{しひ}の
\ruby{樹}{き}ならん
\ruby{眞黑}{まつ|くろ}に
\ruby{見}{み}ゆる
\ruby{{\換字{丈}}矮}{たけ|ひく}き
\ruby{樹}{き}のいと
\ruby{大}{おほい}なるを
\ruby{後楯}{うしろ|だて}に
\ruby{取}{と}りて、
\ruby{僅}{わづか}に
\ruby{二}{ふ}タ
\ruby{室}{ま}ほどなるべき
\ruby{離屋}{はな|れや}
\ruby{立}{た}てり。

『さあ
\ruby{此處}{こ|ゝ}でがあす、
\ruby{上}{あが}つて
\ruby{下}{くだ}さい。
』

と、
\ruby{婆}{ばゞ}は
\ruby{{\換字{戸}}}{と}を
\ruby{引}{ひ}き
\ruby{明}{あ}けてつか〳〵と
\ruby{上}{あが}りぬ。

『お
\ruby{{\換字{前}}}{まへ}さまが
\ruby{頼}{たの}み
\ruby{度}{た}いと
\ruby{云}{い}つた
\ruby{先生}{せん|せい}がござらしつた。
』

と、
\ruby{云}{い}ひながら
\ruby{次}{つぎ}の
\ruby{室}{ま}の
\ruby{長四疊}{なが|よ|でふ}を
\ruby{{\換字{過}}}{す}ぎて、
\ruby{六疊}{ろく|でふ}の
\ruby{其}{そ}の
\ruby{室}{ま}に
\ruby{至}{いた}りたれど、
\ruby{熱}{ねつ}の
\ruby{一}{ひ}ト
\ruby{{\換字{退}}}{ひき}
\ruby{{\換字{退}}}{ひ}きし
\ruby{汐合}{しほ|あひ}の
\ruby{時}{とき}にや、
\ruby{病人}{びやう|にん}は
\ruby{答}{こた}へも
\ruby{無}{な}く
\ruby{音}{おと}も
\ruby{無}{な}く
\ruby{眠}{ねむ}り
\ruby{居}{を}れり。

\ruby{醫師}{い|し}は
\ruby{婆}{ばゞ}につゞきて
\ruby{上}{あが}りけるが、
\ruby{先}{ま}ず
\ruby{此}{こ}の
\ruby{室}{ま}に
\ruby{籠}{こも}りたる
\ruby{不快}{ふ|くわい}の
\ruby{臭氣}{にほ|い}に、
\ruby{不審}{ふ|しん}の
\ruby{眉}{まゆ}を
\ruby{顰}{ひそ}めて\換字{志}ろりと
\ruby{見渡}{み|わた}せば、
\ruby{廣}{ひろ}からぬ
\ruby{一室}{ひと|ま}の
\ruby{内}{うち}
\ruby{法外}{はふ|ぐわい}に
\ruby{明}{あか}るく、
\ruby{病人}{びやう|にん}が
\ruby{枕上}{まくら|もと}の
\ruby{洋燈}{らん|ぷ}は
\ruby{何時}{いつ|か}か
\ruby{燃}{も}え
\ruby{高}{かう}じて、
\ruby{其}{そ}の
\ruby{火屋}{ほ|や}の
\ruby{上}{うへ}の
\ruby{方}{かた}は
\ruby{眞黑}{まつ|くろ}に
\ruby{煤}{すゝ}け、
\ruby{毒々}{どく|〴〵}しき
\ruby{黑}{くろ}き
\ruby{油{\換字{煙}}}{ゆ|えん}は
\ruby{今}{いま}やしたゝかに
\ruby{舞}{ま}ひ
\ruby{上}{あが}り
\ruby{居}{を}れり。

『オーヤ、
\ruby{洋燈}{らん|ぷ}が
\ruby{出{\換字{過}}}{で|す}ぎて
\ruby{居}{ゐ}る!。
\ruby{何}{なん}とマア
\ruby{危}{あぶな}い
\ruby{事}{こと}だつた!。
いくら
\ruby{病人}{びやう|にん}だつて、
\ruby{意氣地}{い|く|ぢ}が
\ruby{無}{な}いつて、ハア、
\ruby{此樣}{こ|ん}な
\ruby{事}{こと}つて
\ruby{有}{あ}る
\ruby{譯}{わけ}で
\ruby{無}{な}い。
』

と
\ruby{婆}{ばゞ}は
\ruby{獨語}{ひとり|ごと}して
\ruby{其}{そ}の
\ruby{心}{しん}を
\ruby{引{\換字{込}}}{ひつ|こ}ませぬ。

\ruby{臭氣}{にほ|い}の
\ruby{源}{もと}は
\ruby{仔細無}{し|さい|な}き
\ruby{事}{こと}なりけるが、
\ruby{惱}{なや}み
\ruby{疲}{つか}れし
\ruby{後}{のち}の
\ruby{睡}{ねむ}りたる
\ruby{間}{ま}に、
\ruby{洋燈}{らん|ぷ}はおのづと
\ruby{燃}{も}え
\ruby{高}{かう}じて、\換字{志}たゝかに
\ruby{憫然}{あは|れ}なる
\ruby{人}{ひと}に
\ruby{惡氣}{あく|き}をや
\ruby{吸}{す}はせけん。
\ruby{相良}{さが|ら}は
\ruby{眼}{ま}のあたりに
\ruby{見}{み}たる
\ruby{此}{こ}の
\ruby{一事}{ひと|こと}と、
\ruby{婆}{ばゞ}が
\ruby{今}{いま}
\ruby{洩}{も}らしたる
\ruby{其}{そ}の
\ruby{一語}{ひと|こと}とに、
\ruby{誰}{たれ}
\ruby{看護}{み|まも}るものも
\ruby{無}{な}き
\ruby{此}{こ}の
\ruby{病人}{びやう|にん}の、
\ruby{何病}{なに|びやう}に
\ruby{惱}{なや}めるかはいざ
\ruby{知}{し}らず、
\ruby{萬般}{よろ|づ}のあはれさ
\ruby{推}{お}し
\ruby{測}{はか}り
\ruby{知}{し}られて、
\ruby{他}{ひと}の
\ruby{憂}{うき}を
\ruby{見}{み}るには
\ruby{馴}{な}れたる
\ruby{身}{み}も、
\ruby{先}{ま}づ
\ruby{惻然}{そく|ぜん}として
\ruby{心}{こゝろ}を
\ruby{動}{うご}かしぬ。
