\Entry{其十}

% メモ 校正終了 2024-03-30 2024-05-23 2024-06-17
\原本頁{60-6}%
\ruby[g]{先挽}{さきびき}
\ruby[g]{後推}{あとおし}の
\ruby[<j>]{勢}{いきほひ}よく、
%
\ruby{矢}{や}を
\ruby{射}{い}る
\ruby{如}{ごと}くに
\ruby{走}{はし}れる
\ruby[g]{相良}{さがら }の
\ruby{車}{くるま}は、
%
\ruby[||j>]{長}{ちやう}
\ruby[||j>]{橋}{ けう}を
% \ruby{長橋}{ちやう|けう}を
\ruby{東}{ひがし}に
\ruby{渡}{わた}つて
\ruby[g]{小梅}{こ うめ}に
かゝり、
%
\ruby{引舟{\換字{通}}}{ひき|ふね|どほ}りを
\ruby[g]{眞直}{まつすぐ}に
\ruby{北}{きた}へと、
%
\ruby[g]{夜風}{よ かぜ}の
や
\原本頁{60-8}\改行%
や
\ruby{{\換字{寒}}}{さむ}きを
\ruby{衝}{つ}いて
\ruby{{\換字{進}}}{すゝ}みに
\ruby{{\換字{進}}}{すゝ}みぬ。
%
\ruby{{\換字{道}}}{みち}は
\ruby{砥}{と}の
\ruby{如}{ごと}し、
%
\ruby{人}{ひと}の
\ruby[g]{往來}{ゆきき }は
\ruby{無}{な}し
\改行% 校正作業の簡略化のため
、
%
\原本頁{60-9}\改行%
\ruby[g]{車夫}{しやふ }は
\ruby{脚一杯}{あし|いつ|ぱい}に
\ruby{駈}{か}くるほどに、
%
おほよその
\ruby[g]{二里}{に り }を
\ruby{瞬}{またゝ}く
\ruby{間}{ま}に
\ruby{{\換字{過}}}{す}ぎて、
%
\ruby[g]{忽地}{たちまち}にして
\ruby{目}{め}ざす
\ruby{四ツ木}{よ| |ぎ}へと
\ruby{着}{つ}きぬ。

\原本頁{61-1}%
\ruby[||j>]{病}{びやう}
\ruby[||j>]{人}{ にん}の
% \ruby{病人}{びやう|にん}の
\ruby[g]{大切}{たいせつ}さは
\ruby[g]{{\換字{貧}}富}{ひんぷ }には
\ruby{關}{かゝ}はらぬ
\ruby{事}{こと}ながら、
%
\ruby[g]{市街}{ま ち }
\ruby{離}{はな}れたる
\ruby{{\換字{遠}}}{とほ}き
ところより、
%
\ruby{夜}{よ}にさへ
\ruby{入}{い}りたるに
\ruby[g]{無理}{む り }
\ruby{{\換字{強}}}{じひ}に
\ruby{{\換字{強}}}{し}ひて、
%
\ruby{我}{わ}が
\ruby[g]{先生}{せんせい}を
\原本頁{61-4}\改行%
\ruby{{\換字{迎}}}{むか}へたるは、
%
\ruby[g]{田舎}{ゐなか }とは
\ruby{云}{い}へ、
%
\ruby{定}{さだ}めし
\ruby[g]{門構}{もんがま}への
\ruby[g]{立派}{りつぱ }に、
%
\ruby[g]{庭{\換字{前}}}{にはさき}
\ruby{廣}{ひろ}く
\改行% 校正作業の簡略化のため
、
%x
\原本頁{61-4}\改行%
がつしりとしたる
\ruby[<j||]{槻}{けやき}
\ruby[||j>]{柱}{ばしら}
の
\ruby{太}{ふと}きが、
%
\ruby[g]{二尺}{にしやく}も
\ruby{厚}{あつ}さのある
\ruby{茅葺屋根}{かや|ぶき|や|ね}の
いと
\ruby{高}{たか}く
\ruby{大}{おほき}なるを
\ruby{支}{さゝ}へたるやうの
\ruby{家}{いへ}ならんと、
%
\ruby[g]{車夫}{しやふ }は
\ruby{心}{こゝろ}の
\ruby{中}{うち}に
\原本頁{61-6}\改行%
\ruby{算}{つも}り
\ruby{居}{ゐ}けるが、
%
\ruby{{\換字{分}}}{わか}り
\ruby{{\換字{兼}}}{か}ぬる
\ruby{闇}{やみ}の
\ruby[g]{村逕}{むらみち}を
\ruby{{\換字{迷}}}{まよ}ひ〳〵て、
%
やうやくに
\ruby{{\換字{尋}}}{たづ}ね
\ruby{當}{あ}てたるは
\ruby{是}{これ}は
\ruby[g]{如何}{い か }な
\ruby{事}{こと}、
%
\ruby[g]{{\換字{寒}}竹}{かんちく}の
\ruby[||j>]{藪}{やぶ}
\ruby[||j>]{疊}{だゝみ}の
% \ruby{藪疊}{やぶ|だゝみ}の
\ruby{不體裁}{ぶ|ざ|ま}に
\ruby{歪}{ゆが}みたる
\原本頁{61-8}\改行%
\ruby{其}{そ}の
\ruby{構}{かまへ}の
\ruby{中}{うち}こそは
\ruby[g]{意外}{いぐわい}に
\ruby{濶}{ひろ}けれ、
%
\ruby{{\換字{空}}}{むな}しく
\ruby{明}{あ}け
\ruby{置}{お}く
\ruby{地}{ち}を
\ruby{惜}{をし}んでか
\改行% 校正作業の簡略化のため
、
%
\原本頁{61-9}\改行%
\ruby{{\換字{通}}}{かよ}ひ
\ruby{路}{ぢ}をも
\ruby{埋}{うづ}むるまでに
\ruby{作}{つく}りたる
\ruby{芋}{いも}の
\ruby{圃}{はたけ}の
\ruby{奧}{おく}に、
%
\ruby{微}{かす}けき
\ruby{星}{ほし}の
\ruby[<j||]{光}{ひかり}を% 行末行頭の境界付近なので特例処置を施す
\ruby{{\換字{浴}}}{あ}びて
\ruby{黑}{くろ}みて
\ruby{立}{た}てる、
%
\ruby{見}{み}るからが
\ruby{悲}{かな}しき
\ruby{草}{くさ}の
\ruby{屋}{や}なり。

\原本頁{61-11}%
\ruby{餘}{あま}りの
\ruby{思}{おも}はく
\ruby{{\換字{違}}}{ちが}ひの
\ruby[g]{忌々}{いま〳〵}しくてや、
%
\ruby[g]{車夫}{しやふ }は
\ruby{憚}{はゞか}り% 「憚 は(ゞ)か」
\ruby{氣}{げ}
\ruby{無}{な}く
\ruby{人力車}{く|る|ま}を
\ruby{挽}{ひ}き
\ruby{入}{い}るれば、
%
\ruby[g]{車輪}{しやりん}に
\ruby{觸}{ふ}るゝ
\ruby{芋}{いも}の
\ruby{葉}{は}は
\ruby[g]{左右}{さ いう}に
\ruby{開}{ひら}けて、
%
\ruby{湛}{たゝ}へられし
\ruby{露}{つゆ}の
\ruby{珠}{たま}は
\ruby{墜}{お}ちて
\ruby{聲}{こゑ}あり。

\原本頁{62-3}%
\ruby{人}{ひと}ありや
\ruby{無}{な}しや
\ruby[g]{岑閑}{しんかん}として、
%
たゞ
\ruby{燈}{ひ}のみ
\ruby{洩}{も}るゝ
\ruby[g]{板{\換字{戸}}}{いたど }を
\ruby{敲}{たゝ}き
\ruby{驚}{おどろ}かしつゝ
\ruby[g]{車夫}{しやふ }は
\ruby{聲}{こゑ}
\ruby{明}{あき}らかに
それと
\ruby{云}{い}ひ
\ruby{入}{い}るれば、
%
\ruby{何}{なに}を
\ruby{擱}{さしお}きても
\ruby{飛}{と}んで
\ruby{出}{い}でゝ、
%
\ruby{喜}{よろこ}び〳〵て
\ruby{{\換字{迎}}}{むか}へ
\ruby{入}{い}るべきを、
%
\ruby{是}{これ}は
また
\ruby{何}{なん}たる
\ruby{事}{こと}ぞ
\ruby[g]{沈着}{おちつ }き
\ruby{拂}{はら}つて、

\原本頁{62-7}%
『
ハア、
%
\ruby[g]{左樣}{さ う }ですかい!。
』

\原本頁{62-8}%
と、
%
\ruby[g]{田舎}{ゐなか }
\ruby{詞}{ことば}の
\ruby{素氣無}{す|げ|な}く
\ruby{答}{こた}へたる
のみにて、
\ruby{嬉}{うれ}しき
\ruby{顏}{かほ}も
せねば、
%
\ruby{{\換字{請}}}{しやう}じ
\ruby{入}{い}れん
ともせず、
%
\ruby{折}{をり}から
\ruby[g]{自裂}{は じ }け
\ruby{{\換字{兼}}}{か}ねたる
\ruby[g]{大豆}{ま め }の
\ruby{莢}{さや}を
\ruby{取}{と}るにやあらん、
%
\ruby{箕}{み}を
\ruby{{\換字{前}}}{まへ}にして
\ruby{乾}{かは}きたる
\ruby{豆}{まめ}を
\ruby{弄}{いぢ}り
\ruby{居}{ゐ}し
\ruby{婆}{ばゞ}の、
%
\ruby{面}{おもて}は
\ruby{赭}{あか}% 行末行頭の境界付近なので特例処置を施す、親文字分解
\ruby{黄}{き}
\原本頁{62-11}\改行%
\ruby{色}{いろ}く
\ruby{焦}{や}け
\ruby{皺}{しわ}びて、
%
\ruby{髮}{かみ}は
\ruby{天蠶糸屑}{て|ぐ|す|くず}の
\ruby{如}{ごと}く
\ruby{白}{しろ}く
\ruby{光}{ひか}るが
\ruby{{\換字{交}}}{まじ}れる、
%
\ruby{年}{とし}の
\原本頁{63-1}\改行%
\ruby{頃}{ころ}は
\ruby[g]{六十}{ろくじふ}ばかりなるが、
%
\ruby{不承不承}{ふ|しよう|ぶ|しよう}に% 非踊り字表記
\ruby{身}{み}を
\ruby{起}{おこ}して
\ruby[g]{{\換字{戸}}口}{と ぐち}に
\ruby[g]{立塞}{たちふさ}がり
\改行% 校正作業の簡略化のため
、

\原本頁{63-2}%
『
\ruby[||j>]{病}{びやう}
\ruby[||j>]{人}{ にん}は
% \ruby{病人}{びやう|にん}は
\ruby[g]{此處}{こ ゝ }には
\ruby{居}{を}りましねえ。
%
\ruby[g]{別室}{はなれ }の
\ruby{方}{はう}に
\ruby{寢}{ね}て
\ruby{居}{を}りますから、
%
\ruby{直}{すぐ}と
そつちらへ
\ruby[g]{御坐}{ご ざ }らしつて
\ruby{下}{くだ}さい。
%
\ruby{暗}{くら}くつて
\ruby{{\換字{分}}}{わか}りますまいが
\原本頁{63-4}\改行%
\ruby[g]{足元}{あしもと}は
\ruby{好}{い}いでがす。
%
\ruby{家}{うち}へさへ
\ruby{付}{つ}いて
\ruby{{\換字{廻}}}{まは}れば
\ruby{直}{ぢき}でがすよ。
%
あ、
%
\換字{志}かし
\ruby[g]{{\換字{菜}}圃}{なばたけ}へでも
\ruby{轉}{ころ}げられると
\ruby{詰}{つま}らない。
%
\ruby[g]{水野}{みづの }さんが
\ruby{後}{あと}になつた
\原本頁{63-6}\改行%
だから
\ruby[g]{仕方}{し かた}が
\ruby{無}{な}い、
%
\ruby{妾}{わし}が
\ruby[g]{案内}{あんない}を
\ruby{仕}{し}てあげやう。
%
ヤ、
%
\ruby[g]{車夫}{くるまや}さん、
%
\ruby[||j>]{提}{ちやう}
\ruby[||j>]{灯}{ ちん}があるの、
% \ruby{提灯}{ちやう|ちん}があるの、
%
\ruby{其}{そ}の
\ruby[||j>]{提}{ちやう}
\ruby[||j>]{灯}{ ちん}を
% \ruby{提灯}{ちやう|ちん}を
\ruby{妾}{わし}に
\ruby{貸}{か}さつせえ。
%
さあ
\ruby[g]{先生}{せんせい}さん、
%
\ruby{妾}{わし}に
\ruby{隨}{つ}いて
\ruby[g]{御坐}{ご ざ }らつせえ。
』

\原本頁{63-9}%
と、
%
\ruby{藁草履}{わら|ざう|り}
つゝかけて
\ruby{先}{さき}に
\ruby{立}{た}つたり。
%
\ruby[g]{相良}{さがら }は
\ruby{是非無}{ぜ|ひ|な}く
\ruby{後}{あと}に
\ruby{隨}{つ}きて、
%
\ruby{家}{いへ}の
\ruby[g]{横手}{よこて }を
\ruby{斜}{なゝめ}に
\ruby{奧}{おく}へ、
%
\ruby[g]{此方}{こなた }には% ルビ調整(原本通り)
\ruby[g]{燃料}{たきれう}の
\ruby[g]{柴木}{しばき }の
\ruby{積}{つ}まれ、
%
\ruby[g]{彼方}{かなた }には
\ruby{玉蜀黍幹}{たう|もろ|こし|がら}の
\ruby[g]{埒無}{らちな }く
\ruby{置}{お}かれなどしたる
\ruby{間}{あひだ}を
\ruby{縫}{ぬ}ひて、
%
さて、
%
\ruby{下}{した}は
\ruby[g]{夏蒔}{なつまき}の
\ruby{{\換字{菜}}}{な}の
\ruby{圃}{はたけ}の
\ruby{中}{なか}の
\ruby[g]{細徑}{ほそみち}の
\ruby{滑}{すべ}り
\ruby{易}{やす}く、
%
\ruby{上}{うへ}は
\ruby{柹}{かき}の
\ruby{樹}{き}の
\ruby[g]{幾本}{いくほん}の
\ruby{枝}{えだ}
\原本頁{64-2}\改行%
\ruby{低}{ひく}くして
\ruby[g]{帽子}{ばうし }
\ruby{危}{あやふ}きところを
\ruby{{\換字{過}}}{す}ぐれば、
%
\ruby{{\換字{前}}}{まへ}の
\ruby{家}{いへ}よりは
\ruby{彼}{かれ}
\ruby{是}{これ}
\ruby{二十間}{に|じふ|けん}
\ruby{餘}{あま}りも
\ruby{距}{はな}れたりと
おぼしきところに、
%
\ruby{椎}{しひ}の
\ruby{樹}{き}ならん
\ruby[g]{眞黑}{まつくろ}に
\ruby{見}{み}ゆる
\ruby[g]{{\換字{丈}}矮}{たけひく}き
\ruby{樹}{き}のいと
\ruby{大}{おほい}なるを
\ruby[||j>]{後}{うしろ}
\ruby[||j>]{楯}{ だて}に
% \ruby{後楯}{うしろ|だて}に
\ruby{取}{と}りて、
%
\ruby{僅}{わづか}に
\ruby{二}{ふ}タ
\ruby{室}{ま}ほど
なるべき
\ruby[g]{離屋}{はなれや}
\ruby{立}{た}てり。

\原本頁{64-6}%
『
さあ
\ruby[g]{此處}{こ ゝ }でがあす、
%
\ruby{上}{あが}つて
\ruby{下}{くだ}さい。
』

\原本頁{64-7}%
と、
%
\ruby{婆}{ばゞ}は
\ruby{{\換字{戸}}}{と}を
\ruby{引}{ひ}き
\ruby{明}{あ}けて
つか〳〵と
\ruby{上}{あが}りぬ。

\原本頁{64-8}%
『
お
\ruby{{\換字{前}}}{まへ}さまが
\ruby{頼}{たの}み
\ruby{度}{た}いと
\ruby{云}{い}つた
\ruby[g]{先生}{せんせい}が
ござらしつた。
』

\原本頁{64-9}%
と、
%
\ruby{云}{い}ひながら
\ruby{次}{つぎ}の
\ruby{室}{ま}の
\ruby{長四疊}{なが|よ|でふ}を
\ruby{{\換字{過}}}{す}ぎて、
%
\ruby[g]{六疊}{ろくでふ}の
\ruby{其}{そ}の
\ruby{室}{ま}に
\ruby{至}{いた}りたれど、
%
\ruby{熱}{ねつ}の
\ruby[g]{一{\換字{退}}}{ひとひき}
\ruby{{\換字{退}}}{ひ}きし
\ruby[g]{汐合}{しほあひ}の
\ruby{時}{とき}にや、
%
\ruby[||j>]{病}{びやう}
\ruby[||j>]{人}{ にん}は
% \ruby{病人}{びやう|にん}は
\ruby{答}{こた}へも
\ruby{無}{な}く
\ruby{音}{おと}も
\ruby{無}{な}く
\ruby{眠}{ねむ}り
\ruby{居}{を}れり。

\原本頁{65-1}%
\ruby[g]{醫師}{い し }は
\ruby{婆}{ばゞ}に
つゞきて
\ruby{上}{あが}りけるが、
%
\ruby{先}{ま}ず
\ruby{此}{こ}の
\ruby{室}{ま}に
\ruby{籠}{こも}りたる
\ruby[g]{不快}{ふくわい}の
\ruby[g]{臭氣}{にほひ }に、
%
\ruby[g]{不審}{ふ しん}の
\ruby{眉}{まゆ}を
\ruby{顰}{ひそ}めて
\換字{志゛}ろりと% 「志」+「濁点」
\ruby[g]{見渡}{み わた}せば、
%
\ruby{廣}{ひろ}からぬ
\ruby[g]{一室}{ひとま }の
\ruby{内}{うち}
\ruby[||j>]{法}{はふ}
\ruby[||j>]{外}{ぐわい}に
% \ruby{法外}{はふ|ぐわい}に
\ruby{明}{あか}るく、
%
\ruby[||j>]{病}{びやう}
\ruby[||j>]{人}{ にん}が
% \ruby{病人}{びやう|にん}が
\ruby[||j>]{枕}{まくら}
\ruby[||j>]{上}{ もと}の
% \ruby{枕上}{まくら|もと}の
\ruby[g]{洋燈}{らんぷ }は
\ruby[g]{何時}{い つ }か
\ruby{燃}{も}え
\ruby{高}{かう}じて、
%
\ruby{其}{そ}の
\ruby[g]{火屋}{ほ や }の
\ruby{上}{うへ}の
\ruby{方}{かた}は
\ruby[g]{眞黑}{まつくろ}に
\ruby{煤}{すゝ}け、
%
\ruby[g]{毒々}{どく〴〵}しき
\ruby{黑}{くろ}き
\ruby[g]{油{\換字{煙}}}{ゆ えん}は
\ruby{今}{いま}や
したゝかに
\ruby{舞}{ま}ひ
\ruby{上}{あが}り
\ruby{居}{を}れり。

\原本頁{65-6}%
『
オーヤ、
%
\ruby[g]{洋燈}{らんぷ }が
\ruby[g]{出{\換字{過}}}{で す }ぎて
\ruby{居}{ゐ}る!。
%
\ruby{何}{なん}と
マア
\ruby{危}{あぶな}い
\ruby{事}{こと}だつた!。
%
いくら
\ruby[||j>]{病}{びやう}
\ruby[||j>]{人}{ にん}だつて、
% \ruby{病人}{びやう|にん}だつて、
%
\ruby{意氣地}{い|く|ぢ}が
\ruby{無}{な}いつて、
%
ハア、
%
\ruby[g]{此樣}{こ ん }な
\ruby{事}{こと}つて
\ruby{有}{あ}る
\ruby{譯}{わけ}で
\ruby{無}{な}い。
』

\原本頁{65-9}%
と
\ruby{婆}{ばゞ}は
\ruby[||j>]{獨}{ひとり}
\ruby[||j>]{語}{ ごと}して
\ruby{其}{そ}の
\ruby{心}{しん}を
\ruby[g]{引{\換字{込}}}{ひつこ }ませぬ。

\原本頁{65-10}%
\ruby[g]{臭氣}{にほひ }の
\ruby{源}{もと}は
\ruby[g]{仔細}{し さい}
\ruby{無}{な}き
\ruby{事}{こと}なりけるが、
%
\ruby{惱}{なや}み
\ruby{疲}{つか}れし
\ruby{後}{のち}の
\ruby{睡}{ねむ}りたる
\ruby{間}{ま}に、
%
\ruby[g]{洋燈}{らんぷ }は
おのづと
\ruby{燃}{も}え
\ruby{高}{かう}じて、
\換字{志}たゝかに
\ruby[g]{憫然}{あはれ }なる
\ruby{人}{ひと}に
\ruby[g]{惡氣}{あくき }をや
\ruby{吸}{す}はせけん。
%
\ruby[g]{相良}{さがら }は
\ruby{眼}{ま}のあたり
\ruby{見}{み}たる
\ruby{此}{こ}の
\ruby[g]{一事}{ひとこと}と、
%
\ruby{婆}{ばゞ}が
\ruby{今}{いま}
\原本頁{66-2}\改行%
\ruby{洩}{も}らしたる
\ruby{其}{そ}の
\ruby[g]{一語}{ひとこと}とに、
%
\ruby{誰}{たれ}
\ruby[g]{看護}{み まも}るものも
\ruby{無}{な}き
\ruby{此}{こ}の
\ruby[||j>]{病}{びやう}
\ruby[||j>]{人}{ にん}の、
% \ruby{病人}{びやう|にん}の、
%
\ruby[||j>]{何}{なに}
\ruby[||j>]{病}{びやう}に
% \ruby{何病}{なに|びやう}に
\ruby{惱}{なや}めるかは
いざ
\ruby{知}{し}らず、
%
\ruby[g]{萬般}{よろづ }の
あはれさ
\ruby{推}{お}し
\ruby{測}{はか}り
\ruby{知}{し}られて
\改行% 校正作業の簡略化のため
、
%
\原本頁{66-4}\改行%
\ruby{他}{ひと}の
\ruby{憂}{うき}を
\ruby{見}{み}るには
\ruby{馴}{な}れたる
\ruby{身}{み}も、
%
\ruby{先}{ま}づ
\ruby[g]{惻然}{そくぜん}として
\ruby{心}{こゝろ}を
\ruby{動}{うご}かしぬ
\改行% 校正作業の簡略化のため
。
