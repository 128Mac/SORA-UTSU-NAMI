\Entry{其十一}

% メモ 校正終了 2024-03-31 2024-05-23 2024-06-17
\原本頁{66-6}%
\ruby{思}{おも}ふまゝに
\ruby{世}{よ}を
\ruby{振舞}{ふる|ま}ふは
\ruby{下人}{げ|にん}の
\ruby{常}{つね}なり。
%
\ruby{相良}{さが|ら}の
\ruby{車夫等}{しや|ふ|ら}は
\ruby{此}{こ}の
\ruby{狀態}{やう|す}に
\ruby{呆}{あき}れ
\ruby{果}{は}てゝ、
%
せめては
\ruby{番茶}{ばん|ちや}なりと
\ruby{飮}{の}んで
\ruby{寢}{ね}ころんで
\ruby{寛}{くつろ}がんと、
%
\ruby{母家}{おも|や}を
さして
\ruby{戾}{もど}りけるが、

\原本頁{66-9}%
『
\ruby{何}{なん}と
\ruby{飛}{と}んだ
ところへ
\ruby{來}{き}たぢや
\ruby{無}{ね}えか。
%
とても
\ruby{眼}{め}も
\ruby{鼻}{はな}も
\ruby{明}{あ}きさうぢや
\ruby{無}{ね}えぜ。
』

\原本頁{67-1}%
『
ハヽヽ、
%
あの
\ruby{婆}{ばあ}さんは
\ruby{大方}{おほ|かた}、
%
\ruby{御醫者}{お|い|しや}さんの
\ruby{御抱}{お|かゝへ}は
\ruby{澤山}{たく|さん}
\ruby{給金}{きふ|きん}を
\ruby{取}{と}るだらう
\ruby{位}{ぐらゐ}に
\ruby{思}{おも}つて
\ruby{居}{ゐ}るだらうよ。
』

\原本頁{67-3}%
『
ウツ、
%
\ruby{{\換字{違}}}{ちげ}へ
\ruby{無}{ね}え。
%
\ruby{一體}{いつ|たい}
\ruby{吾家}{う|ち}の
\ruby{先生}{せん|せい}は
\ruby{人}{ひと}が
\ruby{好{\換字{過}}}{よ|す}ぎるからナア。
%
\ruby{此方等}{こち|と|ら}あ% 原本のルビ配置に合わせる
\ruby{何樣}{ど|う}したつて
\ruby{取}{と}るものあ
\ruby{取}{と}るが、
%
\ruby{先生}{せん|せい}が
\ruby{第一}{だい|いち}
\ruby{馬鹿}{ば|か}を
\ruby{見}{み}らあ。
』

\原本頁{67-6}%
と
\ruby{闇}{やみ}に
はびこる
\ruby{胴{\換字{魔}}聲}{どう|ま|ごゑ}
\ruby{太}{ふと}く、
%
\ruby{{\換字{遠}}慮}{ゑん|りよ}も
\ruby{無}{な}く
\ruby{二人}{ふた|り}して
\ruby{喚}{わめ}き
\ruby{散}{ち}らした
\改行% 校正作業の簡略化のため
り。

\原本頁{67-8}%
\ruby{婆}{ばゞ}は
\ruby{此等}{これ|ら}の
\ruby{聲}{こゑ}を
\ruby{聞}{き}かざりしと
\ruby{見}{み}ゆ。
%
いたはり
\ruby{氣}{げ}も
\ruby{無}{な}く
\ruby[||j>]{病}{びやう}
\ruby[||j>]{人}{ にん}を
% \ruby{病人}{びやう|にん}を
\ruby{搖}{ゆ}り
\ruby{起}{おこ}こして、

\原本頁{67-10}%
『
お
\ruby{{\換字{前}}}{めへ}さまが
\ruby{頼}{たの}みたい
\ruby{云}{い}つた
\ruby{先生}{せん|せい}が
\ruby{御坐}{ご|ざ}らしつたよ。
』

\原本頁{67-11}%
と、
%
\ruby{同}{おな}じ
\ruby{言葉}{こと|ば}を
\ruby{冷}{ひや}やかに
\ruby{繰}{く}り
\ruby{{\換字{返}}}{かへ}しつ、
%
\ruby{重}{おも}き
\ruby{眶}{まぶた}を
\ruby[||j>]{力}{ちから}
\ruby[||j>]{無}{ な}く
\ruby{擧}{あ}げて、
%
\原本頁{68-1}%
\ruby{微}{かすか}に
\ruby{點頭}{うな|づく}を
\ruby{見}{み}るより、

\原本頁{68-2}%
『
さあ
\ruby{先生樣}{せん|せい|さん}、
%
\ruby{見}{み}てやつて
\ruby{下}{くだ}さい。
%
\ruby{濟}{す}んだらば
\ruby{別}{べつ}に
\ruby{水}{みづ}は
あげ
\ruby{無}{な}いから、
%
\ruby{其處}{そ|こ}の
\ruby{椽先}{えん|さき}の
\ruby{手水鉢}{てう|づ|ばち}で、
%
\ruby{{\換字{勝}}手}{かつ|て}に
\ruby{手}{て}を
\ruby{洗}{あら}ふが
\ruby{可}{い}いでがあす。
%
ナアニ
\ruby{一昨日}{をと|ゝ|ひ}
\ruby{汲}{く}んだばかりで、
%
\ruby{誰}{だれ}も
\ruby{使}{つか}はないから
\ruby{奇麗}{き|れい}でがあすよ。
%
そして
\ruby{彼方}{あつ|ち}へ
\ruby{寄}{よ}つて
\ruby{溫茶}{ぬる|ちや}でも
\ruby{上}{あが}らつしやい。
%
どれ
\ruby{妾}{わたし}は
\ruby{先}{さき}へ
\ruby{行}{い}つて
\ruby{火}{ひ}でも
\ruby{燃}{た}きましやう。
』

\原本頁{68-7}%
と、
%
\ruby{他人同士}{た|にん|どう|し}とは
\ruby{本}{もと}より
\ruby{一目}{ひと|め}にも
\ruby{知}{し}れわたりたれど、
%
さりとては
\ruby{乾}{かわ}き
\ruby{切}{き}つたる
\ruby{心}{こゝろ}の
\ruby{鬼々}{おに|〳〵}しくも
\ruby{人{\換字{情}}}{なさ|け}
\ruby{無}{な}き
\ruby{婆}{ばゞ}かな、
%
と
\ruby{竊}{ひそか}に
\ruby{驚}{おどろ}ける
\ruby{相良}{さが|ら}を
\ruby{後}{あと}にして、
%
\ruby{恰}{あだか}も% 恰も「あ(だ)かも」
\ruby{機關仕掛}{ぜん|まい|じ|かけ}の
\ruby[||j>]{人}{にん}
\ruby[||j>]{形}{ぎやう}か
% \ruby{人形}{にん|ぎやう}か
なんぞの
\ruby{動}{うご}くやうに、
%
\ruby{四圍}{あた|り}への
\ruby[||j>]{斟}{しん}
\ruby[||j>]{{\換字{酌}}}{しやく}も
% \ruby{斟{\換字{酌}}}{しん|しやく}も
\ruby{氣{\換字{兼}}}{き|がね}も
\ruby{無}{な}く、
%
\ruby{我}{わ}が
\ruby{行}{ゆ}かんとする
\ruby{方}{かた}へ
\ruby{早{\換字{速}}}{さつ|さ}と
\ruby{行}{ゆ}き
\改行% 校正作業の簡略化のため
ぬ。

\原本頁{69-1}%
『
\ruby{水野}{みづ|の}さんが
\ruby{居}{ゐ}ないで、
%
ハア
\ruby{餘計}{よ|けい}な
\ruby{暇潰}{ひま|つぶ}しな。% TODO 原本では「ひまつ?ぶ」となっていて「?」が踊り字のように扱っていたが踊り字に見えないので除去
%
アヽ
\ruby{江{\換字{戸}}}{え|ど}の
\ruby{人}{ひと}と
\ruby{挨拶}{あい|さつ}するのは
\ruby{面倒}{めん|だう}な。
』

\原本頁{69-3}%
と、
%
つぶやきながら
\ruby{婆}{ばゞ}は
\ruby{火}{ひ}を
\ruby{焚}{た}き
はじめたり。

\原本頁{69-4}%
\ruby{急}{いそ}ぎに
\ruby{急}{いそ}ぎて
\ruby{今}{いま}
\ruby{歸}{かへ}り
\ruby{來}{きた}れる
\ruby{水野}{みづ|の}は、
%
\ruby{額}{ひたひ}に
\ruby{汗}{あせ}の
\ruby{玉}{たま}を
\ruby{散}{ち}らして、
%
\ruby{蒸}{む}されたるが
\ruby{如}{ごと}くに
なりたる
\ruby{面}{おもて}は、
%
\ruby[<j||]{薄}{うす }
\ruby[<j>]{紅}{くれなゐ}に
\ruby{血}{ち}の
\ruby{色}{いろ}
\ruby{潮}{さ}したれば、
%
\ruby{引}{ひき}
\原本頁{69-6}\改行%
\ruby{立}{た}つて
\ruby{見}{み}ゆる
\ruby{眉目}{び|もく}の
あたりに
\ruby{淸秀}{せい|しう}の
\ruby{氣}{き}
\ruby{滿}{み}ち
\ruby{溢}{あふ}れて、
%
これこそ
\ruby{水野}{みづ|の}が
\ruby{往時}{むか|し}の
\ruby{面貌}{おも|わ}かと、
%
\ruby{天晴}{あつ|ぱ}れ
\ruby{美}{うつく}しく
\ruby{生々}{いき|〳〵}としたり。
%
\ruby{早}{はや}くも
\ruby{既}{すで}に
\原本頁{69-8}\改行%
\ruby{相良}{さが|ら}の
\ruby{見}{み}えたるに
\ruby{欣}{よろこ}び
\ruby{悅}{よろこ}び、
%
\ruby{取}{と}り
\ruby{敢}{あ}へず
\ruby{先}{ま}ず
\ruby{車夫}{しや|ふ}を
\ruby{犒}{ねぎら}ひて
\ruby{手當}{て|あて}を
\ruby{與}{あた}へ、
%
\ruby{{\換字{更}}}{さら}に
\ruby[||j>]{病}{びやう}
\ruby[||j>]{室}{ しつ}へは
% \ruby{病室}{びやう|しつ}へは
\ruby{行}{ゆ}かんともせずして、
%
こゝに
\ruby{數{\換字{分}}間}{すう|ふん|かん}の
\ruby{後}{のち}
\原本頁{69-10}\改行%
\ruby{我}{わ}が
\ruby{受}{う}くべき
\ruby[||j>]{吉}{きつ}
\ruby[||j>]{凶}{きよう}
% \ruby{吉凶}{きつ|きよう}
いづれかの
\ruby{報告}{しら|せ}の、
%
\ruby{醫}{い}によつて
\ruby{齎}{もた}らさるべきを
\ruby{恐}{おそ}る
\ruby{懼}{おそ}る
\ruby{待}{ま}ちたり。

\原本頁{70-1}%
\ruby{程}{ほど}
\ruby{經}{へ}て
\ruby{相良}{さが|ら}は
\ruby{歸}{かへ}り
\ruby{來}{きた}りぬ。
%
むさくろしき
\ruby{此}{こ}の
\ruby{婆}{ばゞ}が
\ruby{茶}{ちや}の
\ruby{間}{ま}の
\ruby{中}{うち}に
\原本頁{70-2}\改行%
て、
%
\ruby{水野}{みづ|の}と
\ruby{互}{たがひ}に
\ruby{挨拶}{あい|さつ}して、
%
さて
\ruby{婆}{ばゞ}と
\ruby{水野}{みづ|の}とに
\ruby{向}{むか}つて
\ruby{徐}{おもむ}ろに、
%
\ruby[<j||]{病}{びやう}% 行末行頭の境界付近なので特例処置を施す
\原本頁{70-3}\改行% 直前に読点があるせいか「病」で改行されてしまう
\ruby[||j>]{人}{にん}の
% \ruby{病人}{びやう|にん}の
\ruby{中々}{なか|〳〵}に
\ruby[||j>]{重}{ぢゆう}
\ruby[||j>]{體}{ たい}なる% 原本通り「重(ぢゆう)」
% \ruby{重體}{ぢゆう|たい}なる% 原本通り「重(ぢゆう)」
\ruby{事}{こと}、
%
\ruby[||j>]{徴}{ちよう}
\ruby[||j>]{候}{ こう}の
% \ruby{徴候}{ちよう|こう}の
\ruby{不完全}{ふ|くわん|ぜん}なるを
もて
\ruby{今}{いま}までの
\ruby{醫}{い}は
\改行% 校正作業の簡略化のため
、
%
\原本頁{70-4}\改行%
\ruby{何}{なん}と
\ruby{診斷}{しん|だん}したるか
\ruby{知}{し}らざれども、
%
\ruby{病氣}{やま|ひ}は
\ruby{全}{まつた}く
\ruby[||j>]{腸}{ちやう}
\ruby[||j>]{窒扶斯}{ ち|ぶ|す}なる
\ruby{事}{こと}、
%
\原本頁{70-5}\改行%
\ruby{傳染}{でん|せん}の
\ruby{{\換字{虞}}}{おそれ}ある
\ruby{病}{やまひ}なれば
\ruby{其}{そ}の
\ruby{心}{こゝろ}すべき
\ruby{事}{こと}、
%
\ruby[||j>]{患}{くわん}
\ruby[||j>]{者}{ じや}の
% \ruby{患者}{くわん|じや}の
ためには
\ruby{設備}{せつ|び}
\ruby{宜}{よろ}しき
\ruby[||j>]{病}{びやう}
\ruby[||j>]{院}{ ゐん}に
% \ruby{病院}{びやう|ゐん}に
\ruby{入}{い}らしむるを
\ruby{良}{よ}しとする
\ruby{事}{こと}、
%
されども
\ruby{{\換字{遠}}路}{ゑん|ろ}を
\ruby{{\換字{伴}}}{ともな}ひ
\ruby{行}{ゆ}かんも
\ruby{{\換字{難}}儀}{なん|ぎ}にして、
%
\ruby{聊}{いさゝ}か
\ruby{懸念}{け|ねん}も
\ruby{無}{な}きに
あらねば、
%
\ruby{一軒建}{いつ|けん|だち}の
\ruby{離}{はな}れ
\原本頁{70-8}\改行%
\ruby{家}{や}なるを
\ruby{幸}{さいは}ひ、
%
\ruby{彼處}{かし|こ}にて
\ruby{療養}{れう|やう}さするも
\ruby{惡}{あし}からぬ
\ruby{事}{こと}、
%
たゞし
\ruby{此}{こ}の
\ruby{病}{やまひ}は
\ruby{藥劑}{くす|り}よりも
\ruby{寧}{むし}ろ
\ruby{看護}{かん|ご}の
\ruby{良否}{よし|あし}によりて、
%
\ruby[||j>]{囘}{くわい}
\ruby[||j>]{復}{ ふく}すると% 原本通り「囘」
% \ruby{囘復}{くわい|ふく}すると% 原本通り「囘」
\ruby{爲}{せ}ざるとも
\ruby{生}{しやう}ずるもの
\ruby{故}{ゆゑ}、
%
\ruby{今}{いま}の
\ruby{如}{ごと}き
\ruby{狀}{さま}にては
\ruby{宜}{よろ}しからぬ
\ruby{事}{こと}、
%
\ruby{彼處}{かし|こ}にて
\ruby{其}{その}
\原本頁{70-11}\改行%
\ruby{儘}{まゝ}
\ruby{療養}{れう|やう}せんには
\ruby{是非}{ぜ|ひ}とも
\ruby{智識}{ち|しき}
\ruby{經驗}{けい|けん}の
\ruby{十{\換字{分}}}{じふ|ぶん}なる
\ruby{良看護{\換字{婦}}}{りやう|かん|ご|ふ}を
\ruby{添}{そ}ふべき
\ruby{事}{こと}、
%
\原本頁{71-1}%
くれ〴〵も
\ruby[||j>]{患}{くわん}
\ruby[||j>]{者}{ じや}をして
% \ruby{患者}{くわん|じや}をして
\ruby{{\換字{強}}}{つよ}き
\ruby{身動}{み|うご}きなど
\ruby{爲}{せ}しめざるやう、
%
\原本頁{71-2}\改行%
\ruby[||j>]{取}{とり}
\ruby[||j>]{扱}{あつか}ひも
% \ruby{取扱}{とり|あつか}ひも
\ruby{極}{きは}めて
\ruby{手柔}{て|やはら}かにすべき
\ruby{事}{こと}、
%
\ruby{看護}{かん|ご}の
\ruby[||j>]{力}{ちから}
\ruby[||j>]{足}{ た}らねば
\ruby{危}{あやふ}き
\ruby{事}{こと}、
%
\原本頁{71-3}\改行%
\ruby{今}{いま}まで
\ruby{投劑}{とう|ざい}し
\ruby{居}{を}れる
\ruby{醫}{い}に
\ruby{此由}{この|よし}を
\ruby{語}{かた}りて、
%
\ruby{其}{そ}のつもりの
\ruby{處方}{しよ|はう}を
\ruby{乞}{こ}ひ、
%
\ruby{且}{か}つ
\ruby{種々}{いろ|〳〵}の
\ruby{注意}{ちゆう|い}を% 原本通り「ち(ゆ)うい」
\ruby{受}{う}くべき
\ruby{事}{こと}、
%
\ruby{其}{その}
\ruby{他}{た}
さし
\ruby{當}{あた}つての
\ruby{樣々}{さま|〴〵}の
\ruby{處置}{しよ|ち}など、
%
\ruby{我}{わ}が
\ruby{職{\換字{分}}}{つと|め}の
\ruby{上}{うへ}より
\ruby{云}{い}ふべきほどの
\ruby{事}{こと}は、
%
\ruby{一々}{いち|〳〵}
\ruby{物柔}{もの|やは}らかに
\ruby{言}{い}ひ
\ruby{盡}{つく}して、
%
\ruby{御大切}{ご|たい|せつ}にと
\ruby{靜々}{しづ|〳〵}と
\ruby{歸}{かへ}りぬ。

\原本頁{71-7}%
\ruby{醫師}{い|し}が
\ruby{親切}{しん|せつ}の
\ruby{長々}{なが|〳〵}しき
\ruby{物語}{もの|がた}りの
\ruby{間}{あひだ}に、
%
\ruby{雲間}{くも|ま}の
\ruby{月}{つき}の
\ruby{如}{ごと}
\ruby{唯}{たゞ}
\ruby{僅}{わづか}の
\ruby{間}{あひだ}だけ
\ruby{美}{うつく}\換字{志}かりし
\ruby{水野}{みづ|の}は、
%
\ruby{其}{そ}の
\ruby{往時}{むか|し}の
\ruby[<j>]{俤}{おもかげ}も
いづくへやら、
%
\ruby[<j>]{唇}{くちびる}は
\ruby{微}{かすか}に
\ruby{顚}{ふる}へて
\ruby{自然}{ひとり|で}に
\ruby{戰}{おのゝ}き、
%
\ruby{眼}{め}は
\ruby{洞然}{どう|ぜん}として
\ruby{何處}{いづ|く}を
\ruby{見}{み}るとも
\ruby{無}{な}く
\ruby{据}{すわ}りたるに、
%
\ruby{引}{ひき}かへて
\ruby{冷酷}{れい|こく}なる
\ruby{主人}{ある|じ}の
\ruby{婆}{ばゞ}は、
%
\ruby{哂}{しや}れ
\ruby{{\換字{古}}}{ふる}したる
\ruby{木彫}{き|ぼり}の
\ruby{假面}{め|ん}の、
%
いづくにも
\ruby{潤}{うるほ}ひの
\ruby{無}{な}きが
\ruby{如}{ごと}き
\ruby{顏}{かほ}して、

\原本頁{72-1}%
『
\ruby{傳染病}{うつ|り|やまひ}ぢやあ
ハア
\ruby{大變}{たい|へん}な
\ruby{事}{こと}だ。
%
\ruby{死}{し}なれでも
\ruby{仕}{し}たらまあ、
%
オヽ% TODO 原本の活字が気になる(p72 1行目)
\ruby{厭}{いや}な
\ruby{事}{こと}だ。
%
\ruby{早{\換字{速}}}{さつ|そく}に
\ruby{{\換字{逐}}}{ぼ}ひ% TODO 原本では「ぼ」に見えるので
\ruby{出}{だ}して
\ruby{仕舞}{し|ま}は
\ruby{無}{な}けりやあ。
』

\原本頁{72-3}%
と、
%
\ruby{慈悲}{じ|ひ}も
\ruby{人{\換字{情}}}{なさ|け}も
\ruby{無}{な}く
\ruby{云}{い}ひ
\ruby{出}{いで}しさまは、
%
たゞ
\ruby{地獄物語}{ぢ|ごく|もの|がたり}の
\ruby{奪衣婆}{だつ|え|ば}を、% 奪衣婆(だつえば)は、三途川(葬頭河)で亡者の衣服を剥ぎ取る老婆の鬼。
%
\ruby{今}{いま}
\ruby{眼}{め}の
\ruby{{\換字{前}}}{まへ}に
\ruby{見}{み}るが
\ruby{如}{ごと}し。
