\Entry{其十二}

% メモ 校正終了 2024-04-01 2024-05-24 2024-06-17
\原本頁{72-6}%
\ruby{重}{おも}き
\ruby[g]{風邪}{か ぜ }なりと
\ruby{村}{むら}の
\ruby{醫}{い}の
\ruby[g]{尾竹}{を だけ}の
\ruby{云}{い}ひし
\ruby{時}{とき}だに、
%
\ruby{其}{そ}の
\ruby[g]{容態}{ようだい}の
\ruby[g]{傍觀}{わきめ }にも
たゞならぬに、
%
\ruby{淺}{あさ}からず
\ruby{心}{こゝろ}をも
\ruby{使}{つか}ひ
\ruby{氣}{き}をも
\ruby{揉}{も}みしものを、
%
\ruby[g]{淺草}{あさくさ}
\ruby[g]{以北}{い ほく}にては
\ruby[g]{上無}{うへな }き
\ruby{人}{ひと}に
\ruby{頼}{たの}み
おもへる
\ruby[g]{相良}{さがら }に
\ruby{今}{いま}、
%
\ruby{病}{やまひ}は
これこれなり、
%
\ruby[g]{看護}{かんご }
\ruby{行}{ゆ}き
\ruby{屆}{とゞ}かずば% 「屆」「届」 原本通り「屆」
\ruby{危}{あやふ}からんと
\ruby{云}{い}はれては、
%
\ruby[g]{愕然}{がくぜん}として
\ruby[||j>]{打}{うち}
\ruby[||j>]{驚}{おどろ}きつ、
% \ruby{打驚}{うち|おどろ}きつ、
%
\ruby{胸}{むね}の
たゞ
\ruby{中}{なか}に
\ruby[g]{鐵槌}{てつゝゐ}の
\ruby[g]{一撃}{いちげき}を
\ruby{受}{う}けたるやう
おぼえて、
%
\原本頁{73-1}%
\ruby{我}{われ}
\ruby{先}{ま}づ
\ruby{死}{し}にも
すべく
\ruby{惱}{なや}ましきに、
%
\ruby[g]{垂死}{すゐし }の
\ruby{人}{ひと}を
\ruby{{\換字{逐}}}{お}ひ
\ruby{出}{いだ}さんと
いふ
\ruby[g]{苛酷}{いらひど}き
\ruby{婆}{ばゞ}の
\ruby[g]{言葉}{ことば }を
\ruby{聞}{き}きては、
%
\ruby[g]{怒火}{いかり }
\ruby[g]{心頭}{しんとう}に
\ruby{起}{おこ}つて
\ruby{堪}{た}ふるにも
\ruby{堪}{た}へられず、
%
\ruby{思}{おも}はず
\ruby{目}{め}に
\ruby{稜角立}{か|ど|た}てゝ
\ruby{峻}{けは}しく
\ruby{睨}{にら}みしが、
%
ハツト
\ruby{心}{こゝろ}づきて
\ruby{自}{みづか}ら
\ruby{警}{いまし}め、
%
\ruby{燃}{も}え
\ruby{立}{た}つ
\ruby[g]{瞋恚}{いかり }を
\ruby[g]{押鎭}{おしゝづ}め
\ruby[g]{押鎭}{おしゝづ}めて、
%
わざと
\ruby[g]{何氣}{なにげ }
\ruby{無}{な}く
\ruby{粧}{よそほ}ふ
\ruby[g]{言葉}{ことば }つき
\ruby[g]{{\換字{平}}穩}{なだらか}に、

\原本頁{73-6}%
『
そんな
\ruby{酷}{むご}らしいことを
\ruby{云}{い}つたつて
\ruby[g]{仕樣}{し やう}が
\ruby{無}{な}いぢや
\ruby{無}{な}いか、
%
\ruby{歩}{ある}けも
\ruby[g]{仕無}{し な }い
\ruby[||j>]{病}{びやう}
\ruby[||j>]{人}{ にん}を
% \ruby{病人}{びやう|にん}を
\ruby{{\換字{逐}}}{おひ}
\ruby{出}{た}すなんて。
』

\原本頁{73-8}%
と、
%
\ruby[g]{打碎}{うちくだ}けて
\ruby{云}{い}へど
\ruby{婆}{ばゞ}は
\ruby{應}{おう}ぜず、

\原本頁{73-9}%
『
\ruby{歩}{ある}けても
\ruby{歩}{ある}けないでも
\ruby{構}{かま}ひは
\ruby{有}{あ}りましねえ。
%
そんな
\ruby{病}{やまひ}で
\ruby{死}{し}なれた
\ruby{日}{ひ}には、
%
\ruby{彼}{あ}の
\ruby{家}{うち}へ
\ruby{入}{はい}る
\ruby{人}{ひと}は
\ruby{無}{な}くなつて、
%
\ruby{後}{あと}が
\ruby[g]{廃物}{だ め }になつて
\ruby[g]{仕舞}{し ま }ひます。
%
\ruby[g]{早{\換字{速}}}{さつさ }と
\ruby{出}{で}て
\ruby{貰}{もら}つて
\ruby[g]{掃除}{さうぢ }を
\ruby{仕}{し}て、
%
\ruby[g]{行者さ}{ぎやうじや }
んにでも
\ruby{淸}{きよ}めて
\ruby{貰}{もら}ひます。
%
\原本頁{74-1}%
\makeatletter
\@ifundefined{デバッグ@ビルド}{%
  \ruby[g]{行者さ}{ぎやうじや }
  んを
}{%
  \ruby[||j>]{行}{ぎやう}
  \ruby[||j>]{者}{ じや}さんを
}%
\makeatother
% \ruby{行者}{ぎやう|じや}さんを
\ruby{喚}{よ}ぶだけは
\ruby{痛}{いた}みになるが、
%
それだけは
\ruby{時}{とき}の
\ruby[g]{不祥}{ふしやう}と
\ruby[g]{勘辨}{かんべん}するでがあす。% 弁 瓣 辦 辧 (辨) 辩 辯
』

\原本頁{74-3}%
と、
%
\ruby{{\換字{飽}}}{あく}まで
\ruby[g]{我欲}{が よく}の
\ruby{云}{い}ひ
\ruby{草}{ぐさ}なり。

\原本頁{74-4}%
『
だつて
\ruby[||j>]{病}{びやう}
\ruby[||j>]{人}{ にん}が
% \ruby{病人}{びやう|にん}が
\ruby[g]{自{\換字{分}}}{じ ぶん}で
\ruby{出}{で}て
\ruby{行}{ゆ}きやうは
\ruby{無}{な}し、
%
\ruby{{\換字{又}}}{また}
\ruby{五十子}{い|そ|こ}さんの
お
\ruby{母}{つか}さんは、
%
\ruby{汝}{おまへ}の
\ruby{知}{し}つて
\ruby{居}{ゐ}る
\ruby{{\換字{通}}}{とほ}りの
\ruby[g]{自{\換字{分}}}{じ ぶん}
\ruby[g]{{\換字{勝}}手}{かつて }ばかりの
\ruby[g]{繼母}{まゝはゝ}さんで
\改行% 校正作業の簡略化のため
、
%
\原本頁{74-6}\改行%
\ruby[g]{{\換字{平}}常}{ふだん }から
\ruby{五十子}{い|そ|こ}さんには
\ruby[g]{無理}{む り }を
\ruby{云}{い}ふけれど、
%
\ruby{五十子}{い|そ|こ}さんの
\ruby[g]{世話}{せ わ }
\原本頁{74-7}\改行%
は
\ruby[g]{毫末}{ちつと }も
\ruby[g]{仕無}{し な }い、
%
\ruby{酷}{ひど}い〳〵
\ruby[||j>]{人}{にん}
\ruby[||j>]{{\換字{情}}}{じやう}の
% \ruby{人{\換字{情}}}{にん|じやう}の
\ruby{無}{な}い
\ruby{人}{ひと}ぢあ
\ruby{無}{な}いか。
%
\ruby[g]{今度}{こんど }の
\ruby[<j||]{病}{びやう}
\原本頁{74-8}\改行%
\ruby{氣}{き}を
\ruby{知}{し}らせて
\ruby{{\換字{遣}}}{や}つても、
%
\ruby{顏}{かほ}も
\ruby{出}{だ}さ
\ruby{無}{な}けりあ、
%
\ruby[g]{手紙}{て がみ}
\ruby{一}{ひと}つ
\ruby{{\換字{遣}}}{よこ}さない
\原本頁{74-9}\改行%
\ruby{位}{くらゐ}の
\ruby{人}{ひと}だもの、
%
\ruby[||j>]{病}{びやう}
\ruby[||j>]{人}{ にん}を
% \ruby{病人}{びやう|にん}を
\ruby[g]{引取}{ひきと }らうとは
\ruby{云}{い}ふまいぢあ
\ruby{無}{な}いか。
』

\原本頁{74-10}%
『
けれども
\ruby{親}{おや}は
\ruby{親}{おや}でがあす、
%
\ruby{引}{ひ}き
\ruby{取}{と}らないとは
\ruby{云}{い}はせましねえ。
%
\ruby{親}{おや}が
\ruby[g]{引取}{ひきと }らないほどの
\ruby{厄介者}{やく|かい|もの}を、
%
\ruby[g]{他人}{た にん}の
\ruby{婆}{ばゞあ}が
ハア
\ruby{擔}{かつ}がう
\ruby[g]{理由}{わ け }は
\原本頁{75-1}\改行%
\ruby{有}{あ}りましねえ。
%
たつて
\ruby[g]{引取}{ひきと }ら
\ruby{無}{な}けりやあ、
%
ナアニ
\ruby{譯}{わけ}は
\ruby{無}{な}い、
%
\ruby[<j||]{{\換字{巡}}}{じゆん}% 行末行頭の境界付近なので特例処置を施す
\ruby{査}{さ}さん
\ruby{頼}{たの}んで
\ruby[g]{引取}{ひきと }らせるだ。
%
ハア、
%
\ruby[g]{{\換字{道}}理}{す ぢ }の
\ruby{{\換字{違}}}{ちが}つた
\ruby{事}{こと}
\ruby{云}{い}はない
\ruby{婆}{ばゞ}
\原本頁{75-3}\改行%
だよ。
%
\ruby{婆}{ばゞ}は
\ruby[g]{他人}{た にん}だよ、
%
\ruby[g]{身寄}{み より}で
\ruby{無}{な}いだよ、
%
\ruby[g]{錢金}{ぜにかね}づくで
\ruby{彼}{あ}の
\ruby{家}{うち}に
\ruby{置}{お}いたばかりだよ。
%
\ruby[g]{貯金}{たくはへ}も
\ruby{有}{あ}るか
\ruby{無}{な}いか
\ruby{知}{し}れない
\ruby[||j>]{病}{びやう}
\ruby[||j>]{人}{ にん}を
% \ruby{病人}{びやう|にん}を
\ruby{預}{あづ}かる、
%
\原本頁{75-5}\改行%
{---}{---}%
\換字{志}かも
\ruby{傳染病}{うつ|り|やまひ}の
\ruby[||j>]{大}{たい}% ルビ調整(原本通り)
\ruby[||j>]{病}{びやう}
\ruby[||j>]{人}{ にん}
を
\ruby{預}{あづ}かる、
%
\ruby[g]{其樣}{そ ん }な
\ruby{鈍}{どん}くさい
\ruby{事}{こと}
\ruby[g]{出來}{で き }ないだよ。
%
お
\ruby[g]{{\換字{前}}樣}{めへさま}も
\ruby[||j>]{病}{びやう}
\ruby[||j>]{人}{ にん}には
% \ruby{病人}{びやう|にん}には
\ruby[g]{他人}{た にん}で
\ruby{無}{な}いか、
%
\ruby{恨}{うら}みつぽい
\ruby[g]{其樣}{そ ん }な
\ruby[<j||]{眼}{まなこ}
\原本頁{75-7}\改行%
つきをして
\ruby{何}{なに}も
\ruby{此}{この}
\ruby{婆}{ばゞあ}を
\ruby{視}{み}さつしやることは
\ruby{無}{な}い。
』

\原本頁{75-8}%
『
なるほど
\ruby{其}{それ}は
\ruby[g]{左樣}{さ う }でもあらうが、
%
いくら
\ruby[g]{他人}{た にん}でも
\ruby[||j>]{病}{びやう}
\ruby[||j>]{人}{ にん}を
% \ruby{病人}{びやう|にん}を
\ruby[g]{突出}{つきだ }さうといふのは、
%
それは
\ruby{餘}{あま}り
\ruby{酷}{むご}いぢやあ
\ruby{無}{な}いか。
』

\原本頁{75-10}%
『
\ruby[||j>]{病}{びやう}
\ruby[||j>]{人}{ にん}だから
% \ruby{病人}{びやう|にん}だから
\ruby{{\換字{逐}}}{お}ひ
\ruby{出}{だ}さうといふので、
%
\ruby{酷}{むご}かあ
\ruby{酷}{むご}いに
\ruby{仕}{し}て
\ruby{置}{お}かつしやい。
』

\原本頁{76-1}%
『
お
\ruby{婆}{ばあ}さん、
%
お
\ruby{{\換字{前}}}{まへ}、
%
そんな
\ruby{事}{こと}を
\ruby{云}{い}つたつて、
%
\ruby[g]{人間}{ひ と }には
\ruby[g]{人{\換字{道}}}{み ち }といふものがある。
%
\ruby{動}{うご}かしてさへ
\ruby{惡}{わる}いと
\ruby[g]{醫者}{い しや}の
\ruby{云}{い}つた
\ruby[||j>]{病}{びやう}
\ruby[||j>]{人}{ にん}を
% \ruby{病人}{びやう|にん}を
\ruby{{\換字{逐}}}{お}ひ
\ruby{出}{だ}さうとは
\ruby[g]{非{\換字{道}}}{ひ だう}では
\ruby{無}{な}いか。
』

\原本頁{76-4}%
『
\ruby[g]{非{\換字{道}}}{ひ だう}なら
\ruby[g]{非{\換字{道}}}{ひ だう}に
\ruby{仕}{し}て
\ruby{置}{お}かつしやい。
%
\ruby{金}{かね}の
\ruby[g]{出處}{で どこ}の
\ruby{覺束無}{おぼ|つか|な}い
\ruby[g]{其樣}{そ ん }な
\ruby{大病人}{たい|びやう|にん}を、
%
\ruby[g]{世話}{せ わ }を
して
\ruby{損}{そん}を
するのは
\ruby{婆}{ばゞあ}は
\ruby{{\換字{嫌}}}{きら}ひだ。
』

\原本頁{76-6}%
『
でも
\ruby{有}{あ}らうが
\ruby{汝}{おまへ}が
\ruby{今}{いま}
\ruby{{\換字{逐}}}{お}ひ
\ruby{出}{だ}して
\ruby[g]{仕舞}{し ま }へば、
%
よしんば
\ruby[g]{繼母}{おつか }さんが
\ruby{引}{ひ}き
\ruby{取}{と}るにしても、
%
あちこち
\ruby{持}{も}ち
\ruby{{\換字{廻}}}{まは}られるのは
\ruby[||j>]{病}{びやう}
\ruby[||j>]{人}{ にん}の
% \ruby{病人}{びやう|にん}の
\ruby{不利益}{ふ|た|め}
\改行% 校正作業の簡略化のため
、
%
\原本頁{76-8}\改行%
\換字{志}かも
\ruby[g]{何樣}{ど う }して
\ruby{彼}{あ}の
\ruby[g]{繼母}{おつか }さんが、
%
\ruby{碌}{ろく}な
\ruby[g]{世話}{せ わ }をする
\ruby{事}{こと}では
\ruby{無}{な}い。
%
\原本頁{76-9}\改行%
\ruby{仕}{し}て
\ruby{見}{み}れば
\ruby[g]{看護}{かんご }が
\ruby{惡}{わる}けりやあ
\ruby{危}{あぶな}いといふ
\ruby[g]{病氣}{びやうき}だもの、
%
\ruby{十}{とほ}に
\ruby{一}{ひと}つも
\ruby{助}{たすか}る
\ruby{瀬}{せ}は
\ruby{無}{な}い、
%
\ruby{見}{み}す〳〵
\ruby[||j>]{病}{びやう}
\ruby[||j>]{人}{ にん}は
% \ruby{病人}{びやう|にん}は
\ruby{殺}{ころ}されるやうなもの!。
%
\ruby{譯}{わけ}の
\原本頁{76-11}\改行%
\ruby{{\換字{分}}}{わか}らない
\ruby{汝}{おまへ}でも
\ruby{無}{な}し、
%
こゝのところを
\ruby{考}{かんが}へて、
%
\ruby{私}{わたし}が
\ruby{此}{こ}の
\ruby{{\換字{通}}}{とほ}り
\ruby{手}{て}を
ついて
\ruby{頼}{たの}むから、
%
\原本頁{77-1}%
どうか
\ruby[g]{左樣}{さ う }
あこぎな
\ruby{事}{こと}を
\ruby{云}{い}はないで、
%
\ruby[g]{當{\換字{分}}}{たうぶん}
\改行% 校正作業の簡略化のため
‥‥。
』

\原本頁{77-3}%
『
イヽエ、
%
あこぎな
\ruby{事}{こと}を
\ruby{云}{い}ふでがあすよ。
%
\ruby{手}{て}をついて
\ruby{頼}{たの}んだつて、
%
\ruby[g]{芋塊}{い も }が
\ruby{一}{ひと}つ
\ruby[g]{自然}{ひとりで}に
\ruby[g]{出來}{で き }て
\ruby{來}{く}るものぢやあ
ござら
\ruby{無}{な}い。
%
\ruby{頼}{たの}むなら
\ruby{頼}{たの}むやうにして
\ruby{頼}{たの}まつしやい。
』

\原本頁{77-6}%
『
\ruby{頼}{たの}むやうに
\ruby{仕}{し}ろつて、
%
\ruby[g]{何樣}{ど う }すれば
\ruby{好}{い}いと
\ruby{云}{い}ふのかえ。
』

\原本頁{77-7}%
『
\ruby{婆}{ばゞあ}は
\ruby{年}{とし}を
とつて
\ruby{氣}{き}が
\ruby{短}{みじか}い、
%
\ruby[g]{打撒}{ぶちま }けて
\ruby[||j>]{汝}{おまへ}
\ruby[||j>]{樣}{ さま}に
% \ruby{汝樣}{おまへ|さま}に
\ruby{云}{い}つて
\ruby{上}{あ}げやう。
%
\ruby[||j>]{病}{びやう}
\ruby[||j>]{人}{ にん}の
% \ruby{病人}{びやう|にん}の
\ruby[g]{月々}{つき〴〵}のものは
\ruby{今}{いま}まで
\ruby{{\換字{通}}}{どほ}りに
\ruby{屹}{きつ}と
お
\ruby[g]{{\換字{前}}樣}{めへさま}が
\ruby[g]{受合}{うけあ }つて、
%
それから
\ruby[||j>]{病}{びやう}
\ruby[||j>]{人}{ にん}が
% \ruby{病人}{びやう|にん}が
いけなかつたら、
%
\ruby{後}{あと}の
\ruby[g]{始末}{し まつ}は
\ruby{皆}{みな}
\ruby{此}{この}
\ruby{婆}{ばゞあ}に
\ruby[g]{{\換字{迷}}惑}{めいわく}を
\ruby{掛}{か}けないで、
%
そして
\ruby[g]{座敷}{ざ しき}に
\ruby[g]{死穢}{け がれ}を
\ruby{付}{つ}けた
\ruby[g]{謝罪}{あやまり}に
\ruby{二十兩}{に|じふ|りやう}、
%
\ruby{癒}{なほ}つたら
\ruby{祝}{いはひ}に
\ruby[||j>]{十}{じふ}
\ruby[||j>]{兩}{りやう}
% \ruby{十兩}{じふ|りやう}
\ruby[||j>]{{\換字{遣}}}{ よこ}すと、
%
\ruby[g]{確乎}{しつかり}
\ruby{御{\換字{前}}樣}{お|めへ|さま}が
\ruby[g]{呑{\換字{込}}}{のみこ }んで、
%
\ruby{先}{ま}づ
\ruby[||j>]{十}{じふ}
\ruby[||j>]{兩}{りやう}だけ
% \ruby{十兩}{じふ|りやう}だけ
\ruby{渡}{わた}して
\ruby{置}{お}かつしやい。
%
\原本頁{78-1}%
その
\ruby{代}{かは}り
\ruby[||j>]{病}{びやう}
\ruby[||j>]{人}{ にん}には
% \ruby{病人}{びやう|にん}には
\ruby{構}{かま}はないから、
%
どうなりと
\ruby[g]{{\換字{勝}}手}{かつて }に
\ruby[g]{介抱}{かいはう}さつしやい。
%
さ、
%
お
\ruby[g]{{\換字{前}}樣}{めへさま}も
あかの
\ruby[g]{他人}{た にん}、
%
これだけ
\ruby[g]{踏{\換字{込}}}{ふみこ }んで
\ruby[g]{世話}{せ わ }も
なるまい。
%
それとも
\ruby[||j>]{病}{びやう}
\ruby[||j>]{人}{ にん}が
% \ruby{病人}{びやう|にん}が
\ruby[||j>]{愍}{かは}
\ruby[||j>]{然}{いさう}で、% 「愍然 か(は)いさう」
% \ruby{愍然}{かは|いさう}で、% 「愍然 か(は)いさう」
%
\ruby{金}{かね}を
\ruby{出}{だ}してもと
\ruby{云}{い}はつしやるか、
%
どつちでも
お
\ruby[g]{{\換字{前}}樣}{めへさま}の
\ruby{好}{すき}に
さつしやい。
』

\原本頁{78-5}%
『
ムヽ
』

\原本頁{78-6}%
\ruby{手}{て}に
\ruby{在}{あ}らば
\ruby{千金萬金}{せん|きん|ばん|きん}も
\ruby{何}{なに}
\ruby{惜}{をし}かるべきを、
%
\ruby{及}{およ}ぶことの
\ruby{及}{およ}ばぬに
\ruby{口}{くち}
\ruby{惜}{をし}きは
\ruby{金沙汰}{かね|さ|た}なり。
%
\ruby[g]{水野}{みづの }は
\ruby{生}{うま}れて
はじめて
\ruby[g]{日頃}{ひ ごろ}
\ruby{此}{この}
\ruby[|g|]{阿堵物}{もの}を% 「阿堵物(あとぶつ)」お金のこと
\ruby{卑}{いやし}みしを
\ruby{悔}{く}いぬ。
