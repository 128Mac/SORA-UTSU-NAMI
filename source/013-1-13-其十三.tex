\Entry{其十三}

% メモ 校正終了 2024-04-04
\原本頁{78-10}%
おのづから
\ruby{横}{よこ}さまに
\ruby{降}{ふ}る
\ruby{雨}{あめ}はあらじ、
%
\ruby{風}{かぜ}の
\ruby{添}{そ}はるにこそ、
%
\ruby{音}{おと}
あらけなく
\ruby{夜}{よる}の
\ruby{窓}{まど}をも
\ruby{打}{う}つなれ、
%
と
\ruby{胸}{むね}
ゆたかなる
\ruby{{\換字{古}}}{むかし}の
\ruby{人}{ひと}の
\ruby{云}{い}ひける。
%
かゝる
\ruby{鬼}{おに}くさき
\ruby{婆}{ばゞ}も、
%
\ruby{齡}{とし}の
\ruby{十七八}{じう|なな|はち}には、
%
\ruby{女}{をんな}の
\ruby{本性}{うまれ|つき}とて、
%
\ruby{臙脂}{べ|に}
\ruby{白{\換字{粉}}}{おし|ろい}に
\ruby{色}{いろ}つくりて、
%
\ruby{人}{ひと}に
\ruby{悅}{よろこ}ばれんと
\ruby{願}{ねが}ひたる
\ruby{日}{ひ}もあるべきに、
%
\ruby{其}{そ}の
\ruby{後}{のち}
\ruby{如何}{い|か}なる
\ruby{世}{よ}の
\ruby{風}{かぜ}に
\ruby{吹}{ふ}き
\ruby{曲}{ゆが}められてか、
%
\ruby{今}{いま}は
\ruby{如是}{か|く}
\ruby{直}{すぐ}ならず
\ruby{人}{ひと}には
\ruby{當}{あた}るならん。
%
\ruby[g]{水野}{みづの}は
\ruby{一度}{ひと|たび}は
\ruby{此}{こ}の
\ruby{婆}{ばゞ}を
\ruby{憎}{にく}しと
\ruby{見}{み}しかど、
%
\ruby{憎}{にく}む
\ruby{心}{こゝろ}は
\ruby{忽}{たちま}ちに
\ruby{失}{う}せて、
%
\ruby{且}{かつ}は
\ruby{其}{そ}の
\ruby{欲深}{よく|ふか}きに
\ruby{呆}{あき}れ、
%
\ruby{且}{かつ}は
\ruby{其}{そ}の
\ruby[h|]{意}{こゝろ}
\ruby{剛}{たけ}きを
\ruby{怪}{あやし}み、
%
\ruby{且}{かつ}は
\ruby{其}{そ}の
\ruby{人}{ひと}
らしからぬまでに
\ruby{{\換字{尊}}}{たつと}き
\ruby[g]{愛{\換字{情}}}{こゝろ}の
\ruby{既}{すで}に
\ruby{壞}{やぶ}れ
\ruby{盡}{つく}して、
%
\ruby{卑}{いや}しき
\ruby{我}{が}のみの
\ruby{殘}{のこ}りて
\ruby{高}{たか}ぶれるを、
%
\ruby{哀}{かなし}み
\ruby{愍}{あはれ}みて
\ruby{打見}{うち|み}やりたり。

\原本頁{79-10}%
されど
\ruby{今}{いま}は
\ruby{他人}{ひ|と}を
\ruby{愍}{あはれ}みて
あるべき
\ruby{時}{とき}ならねば、
%
\ruby[g]{水野}{みづの}は
\ruby{直}{たゞち}に
\ruby{差}{さ}し
\ruby{當}{あた}つての
\ruby{我}{わ}が
\ruby{上}{うへ}に
\ruby{掛}{かゝ}れる
\ruby{事}{こと}に
\ruby{心}{こゝろ}を
\ruby{惱}{なや}ましめぬ。

\原本頁{80-1}%
\ruby[g]{五十子}{いそこ}を
\ruby{如是}{か|く}
\ruby{忌}{いま}はしく
\ruby{親切}{しん|せつ}
\ruby{無}{な}き
\ruby{婆}{ばゞ}の
\ruby{家}{いへ}に
\ruby{在}{あ}らせんよりは、
%
\ruby{良}{よ}き
\ruby{病院}{びやう|ゐん}に
\ruby{移}{うつ}さんかた
\ruby{萬般}{すべ|て}に
\ruby{就}{つ}けて
\ruby{心地}{こゝ|ち}よし
とは
\ruby{思}{おも}ひながらも、
%
\ruby{今{\換字{宵}}}{こ|よひ}の
\ruby{如}{ごと}く
\ruby{穩}{おだ}やかに
\ruby{晴}{は}れてのみ
あるべくは
あらぬ
\ruby{秋}{あき}の
\ruby[g]{天候}{そら}の
\ruby{{\換字{習}}}{ならひ}なれば、
%
\ruby{時}{とき}に
\ruby{臨}{のぞ}みて
\ruby{如何}{い|か}なる
\ruby{雨}{あめ}
\ruby{風}{かぜ}の
\ruby{妨{\換字{害}}}{さま|たげ}に
\ruby{{\換字{遇}}}{あ}はんも
\ruby{知}{し}るべからず、
%
\ruby{{\換字{又}}}{また}
\ruby{然無}{さ|な}きだに
\ruby{{\換字{遠}}路}{ゑん|ろ}を
\ruby{{\換字{伴}}}{ともな}ひ
\ruby{行}{ゆ}く
\ruby{{\換字{途}}上}{と|じやう}は
\ruby{病人}{びやう|にん}も
\ruby{特}{こと}に
\ruby{心惱}{こゝろ|なや}ましかるべく、
%
それがために
\ruby{萬一}{まん|いち}
\ruby{惡}{あし}き
\ruby{事}{こと}もやとの
\ruby{懸念}{け|ねん}も
\ruby{少}{すくな}からぬに、
%
\ruby{由無}{よし|な}き
\ruby{金錢}{きん|せん}を
\ruby{婆}{ばゞ}に
\ruby{貪}{むさぼ}らるゝは
\ruby{愚}{おろか}なるに
\ruby{似}{に}たれど、
%
これも
\ruby{病}{や}める
\ruby{人}{ひと}のためと
\ruby{{\換字{忍}}}{しの}ばんには
\ruby{露}{つゆ}
\ruby{厭}{いと}はしからずと、
%
\ruby[g]{水野}{みづの}は
\ruby{{\換字{終}}}{つひ}に
\ruby{意}{こゝろ}を
\ruby{決}{けつ}して、
%
\ruby{彼}{か}の
\ruby{離}{はな}れ
\ruby{室}{や}に
\ruby{置}{お}きたるまゝ
\ruby{介抱}{かい|はう}する
\ruby{事}{こと}と
\ruby{定}{さだ}めたり。
%
もとより
\ruby{一}{ひと}つには
\ruby{其}{そ}の
\ruby{奧深}{おく|ふか}き
\ruby{底}{そこ}の
\ruby{底}{そこ}の
\ruby{心}{こゝろ}に、
%
\ruby[g]{五十子}{いそこ}と
\ruby{我}{われ}との
\ruby{相距}{あひ|さ}らざらんを
\ruby{望}{のぞ}む
\ruby{思}{おもひ}の
\ruby{潜}{ひそ}めばなるべし。% 【潛 u6f5b 「先先」】【潜 u6f5c 「夫夫」】併用されている
%
たとひ
\ruby{自己}{お|の}が
\ruby{身}{み}は
\ruby{如何}{い|か}なる
\ruby{故}{ゆゑ}にか
\原本頁{81-1}%
\ruby[g]{五十子}{いそこ}に
\ruby{{\換字{嫌}}}{きら}はれて、
%
\ruby{特}{こと}に
\ruby{病}{やまひ}のため
\ruby{癇}{かん}の
\ruby{高}{たか}ぶりて
\ruby{我}{が}の
\ruby{{\換字{強}}}{つよ}くなれる
\ruby{此}{こ}の
\ruby{頃}{ごろ}の
\ruby[g]{彼女}{かれ}には、
%
\ruby{面}{おもて}を
\ruby{會}{あ}はすを
さへ
\ruby{厭}{いと}はるゝより、
%
\ruby{自}{みづか}ら
\ruby{病床}{びやう|しやう}に
\ruby{{\換字{近}}}{ちか}づきて
\ruby{問}{と}ひ
\ruby{慰}{なぐさ}めも
\ruby{仕度}{し|た}く、
%
\ruby{看護}{せ|わ}も
\ruby{仕}{し}て
\ruby{{\換字{遣}}}{や}りたき
\ruby{心}{こゝろ}の、
%
\ruby{{\換字{遣}}}{や}る
\ruby{方}{かた}も
\ruby{無}{な}く
\ruby{逸}{はや}るを
\ruby{抑}{おさ}へに
\ruby{抑}{おさ}へて、
%
\ruby{裏面}{う|ら}にて
こそ
\ruby{力}{ちから}の
\ruby{及}{およ}ぶ
\ruby{限}{かぎ}りを
\ruby{盡}{つく}して
\ruby{駈}{か}けも
\ruby{走}{はし}りもすれ、
%
\ruby{病人}{びやう|にん}の
\ruby{氣}{き}に
\ruby{{\換字{逆}}}{さから}はじと
\ruby{其}{そ}の
\ruby{{\換字{前}}}{まへ}には
\ruby[g]{身影}{かげ}をさへ
\ruby{見}{み}することも
\ruby{無}{な}くて、
%
たゞ
\ruby{竊}{ひそか}に
\ruby{外}{そと}に
\ruby{立}{た}つて、
%
\ruby{細}{ほそ}りたる
\ruby{聲}{こゑ}の
\ruby{孱{\換字{弱}}}{か|よわ}きを
\ruby{聞}{き}き、
%
\ruby{或}{あるひ}は
\ruby{物}{もの}の
\ruby{罅隙}{す|き}より
\ruby{窶}{やつ}れたる
\ruby{其}{そ}の
\ruby{面貌}{おも|かげ}の
\ruby{悲}{かな}しきを
\ruby{見}{み}ては、
%
\ruby{男兒}{をと|こ}たる
\ruby{身}{み}の
\ruby{人目}{ひと|め}はづかしくも、
%
にじみ
\ruby{來}{く}る
\ruby{涙}{なみだ}を
\ruby{止}{とゞ}め
かねて、
%
\ruby{神}{かみ}も
\ruby{我}{わ}が
\ruby{誠心}{まこ|と}を
\ruby{憐}{あは}れませ
たまひて、
%
\ruby{此}{こ}の
\ruby{人}{ひと}の
\ruby{病苦}{びやう|く}を
\ruby{救}{すく}はせ
たまへ、
%
と
\ruby{何}{なん}の
\ruby{神}{かみ}に
\ruby{祈}{いの}るとも
\ruby{無}{な}く、
%
\ruby{何時}{い|つ}か
\ruby{我}{われ}
\ruby{知}{し}らず
\ruby{祈}{いの}り
\ruby{居}{ゐ}る、
%
\ruby{思}{おも}へば
\ruby{愚}{おろか}しき
\ruby{{\換字{朝}}夕}{あさ|ゆふ}に
\ruby{甘}{あま}んじて、
%
\ruby{{\換字{猶}}}{なほ}これより
\ruby{幾日}{いく|か}と
\ruby{定}{さだ}まらぬ
\原本頁{82-1}%
\ruby{其}{そ}の
\ruby{間}{あひだ}を、
%
せめてもの
\ruby{果敢}{は|か}なき
\ruby{心{\換字{遣}}}{こゝろ|や}りに、
%
\ruby{{\換字{猶}}}{なほ}
\ruby{其}{そ}の
おろかしき
\ruby{振舞}{ふる|まひ}を
\ruby{續}{つゞ}けんとは
するなり。

\原本頁{82-3}%
『では
\ruby{汝}{おまへ}の
\ruby{云}{い}ふ
\ruby{{\換字{通}}}{とほ}りに
\ruby{仕}{し}やう。
%
\ruby{一切}{いつ|さい}
\ruby{私}{わたし}が
\ruby{受合}{うけ|あ}つて
\ruby{置}{お}く。
』

\原本頁{82-4}%
と、
%
\ruby{決然}{けつ|ぜん}として
\ruby[g]{水野}{みづの}は
\ruby{云}{い}へど、

\原本頁{82-5}%
『たゞ
\ruby{受合}{うけ|あ}つても
いけましねえ、
%
\ruby{何時}{い|つ}
その
\ruby{十兩}{じう|りやう}は
\ruby{渡}{わた}して
\ruby{吳}{く}れ
さつしやる。
』

\原本頁{82-7}%
と、
%
\ruby{婆}{ばゞ}は
\ruby{手}{て}に
\ruby{握}{にぎ}らぬことには
\ruby{人}{ひと}を
\ruby{信}{しん}ぜず。

\原本頁{82-8}%
『
\ruby{明日}{あ|す}の
\ruby{{\換字{朝}}}{あさ}
\ruby{渡}{わた}す。
』

\原本頁{82-9}%
『
\ruby{大{\換字{丈}}夫}{だい|ぢやう|ぶ}かね。
』

\原本頁{82-10}%
『
\ruby{大{\換字{丈}}夫}{だい|ぢやう|ぶ}だ。
』

\原本頁{82-11}%
『
\ruby{看病人}{かん|びやう|にん}
はエ。
』

\原本頁{83-1}%
『
\ruby{矢張}{やつ|ぱり}
\ruby{私}{わたし}が
\ruby{雇}{やと}つて
\ruby{付}{つ}ける。
%
\ruby[g]{相良}{さがら}さんに
\ruby{良}{い}いのを
\ruby{世話}{せ|わ}をして
\ruby{貰}{もら}ふ。
%
\換字{志}かし
\ruby{一切}{いつ|さい}
かういふ
\ruby{事}{こと}を、
%
\ruby{私}{わたし}が
\ruby{爲}{し}たのだと
\ruby{病人}{びやう|にん}に
\ruby{云}{い}つては
ならぬ。
%
\ruby{病人}{びやう|にん}が
\ruby{私}{わたし}の
\ruby{世話}{せ|わ}になるのを
\ruby{厭}{いや}がつて
\ruby{居}{ゐ}るから、
%
たゞ
\ruby{學校}{がく|かう}の
\ruby{人{\換字{達}}}{ひと|たち}が
\ruby{爲}{す}るのだと
\ruby{云}{い}つて
\ruby{置}{おい}てくれ。
』

\原本頁{83-5}%
『はア、ようがす、
%
それは
\ruby{無益}{む|だ}な
\ruby{口}{くち}きく
\ruby{婆}{ばゞあ}でない
でがあす。
%
\換字{志}かし
\ruby{甚}{えら}い
\ruby{金}{かね}が
かゝりませうに、
%
\ruby{親切}{しん|せつ}な
\ruby{事}{こと}だネ。
』

\原本頁{83-7}%
と、
%
\ruby{冷}{ひや}やかに
\ruby{笑}{わら}ふ
\ruby{口}{くち}の
\ruby{左右}{さ|いう}に、
%
\ruby{深}{ふか}き
\ruby{皺}{しわ}
あらはれて
\ruby{物凄}{もの|すさま}じく、
%
さも〳〵
\ruby[g]{水野}{みづの}が
\ruby{爲}{な}す
\ruby{一切}{いつ|さい}の
\ruby{事}{こと}の、
%
やがては
\ruby{{\換字{朝}}}{あした}の
\ruby{霜}{しも}の
\ruby{柱}{はしら}を
\ruby{{\換字{彩}}色}{いろ|ど}り
\ruby{夕}{ゆふべ}の
\ruby{露}{つゆ}の
\ruby{珠}{たま}を
\ruby{綴}{つゞ}らんとする
\ruby{痴}{おろか}なる
\ruby{企畫}{くは|だて}の
\ruby{如}{ごと}く
\ruby{甲{\換字{斐}}}{か|ひ}
\ruby{無}{な}く
\ruby{{\換字{終}}}{おは}らんを
\ruby{見徹}{み|ぬ}きて
\ruby{知}{し}りたりと
\ruby{云}{い}はぬばかりの
\ruby{面色}{かほ|つき}したり。

\原本頁{83-11}%
『
\ruby{快}{よ}くなるまで
みんな
\ruby{御{\換字{前}}樣}{お|まへ|さま}が
\ruby{一人}{ひと|り}で
\ruby{爲}{さ}つしやるかネ。
』

\原本頁{84-1}%
『ム。
』

\原本頁{84-2}%
『
\ruby{百兩位}{ひやく|りやう|ぐらゐ}では
\ruby{{\換字{追}}付}{おつ|つ}きましねえかも
\ruby{知}{し}れましねえヨ。
』

\原本頁{84-3}%
『ム。
』

\原本頁{84-4}%
『ホー
\ruby{御{\換字{前}}樣}{お|まへ|さま}は
\ruby{學校}{がく|かう}の
\ruby{敎員}{けう|ゐん}でもつて、
%
\ruby{其樣}{そん|な}に
\ruby{御金}{お|かね}
\ruby{有}{も}つてるだかネ。
』

\原本頁{84-6}%
\ruby[g]{水野}{みづの}は
\ruby{苦}{にが}りきつて
\ruby{答}{こたへ}をもせず、

\原本頁{84-7}%
『
\ruby{何}{なん}でも
\ruby{可}{よ}い、
%
\ruby{其樣}{そ|ん}なことを
\ruby{云}{い}つて
\ruby{居}{ゐ}る
\ruby{暇}{ひま}は
\ruby{無}{な}い。
%
わたしは
これから
\ruby[g]{尾竹}{をだけ}の
ところへ
\ruby{行}{ゆ}く。
』

\原本頁{84-9}%
\ruby{突}{つ}と
\ruby{立上}{たち|あが}つたる
\ruby[g]{水野}{みづの}は
\ruby{此處}{こ|ゝ}を
\ruby{出}{い}でゝ、
%
\ruby{村}{むら}の
\ruby{醫}{い}を
\ruby{問}{と}ひて
\ruby[g]{相良}{さがら}の
\ruby{言}{ことば}を
\ruby{傳}{つた}へ、
%
\ruby{手}{て}ぬかり
\ruby{無}{な}きやう
\ruby{十{\換字{分}}}{じう|ぶん}に
\ruby{其}{そ}の
\ruby{職{\換字{分}}}{つと|め}を
\ruby{盡}{つく}さんことを
\ruby{乞}{こ}ひ
\ruby{求}{もと}め、
%
これより
\ruby{直}{すぐ}にも
\ruby{見舞}{み|ま}はん
といふ
\ruby{親切}{しん|せつ}
\ruby{籠}{こも}れる
\ruby{答}{こたへ}を
\ruby{聞}{き}きて、
%
\原本頁{85-1}%
はじめて
\ruby{我}{わ}が
\ruby{宿}{やど}とせる
\ruby[g]{山路}{やまぢ}が
\ruby{方}{かた}に
\ruby{歸}{かへ}りぬ。

\原本頁{85-2}%
\ruby{物}{もの}の
\ruby{味}{あぢ}さへ
\ruby{知}{し}るや
\ruby{知}{し}らずや、
%
\ruby{湯漬}{ゆ|づ}け
\ruby{飯}{めし}
\ruby{忙}{せは}しく
\ruby{夜食}{や|しよく}を
\ruby{濟}{す}ませて、
%
\ruby{長}{なが}き
\ruby{夜}{よ}も
\ruby{既}{はや}
\ruby{{\換字{更}}}{ふ}けて
\ruby{何時}{なん|じ}かを
\ruby{打}{う}つ
\ruby{時計}{と|けい}の
\ruby{音}{おと}の
\ruby{折}{をり}から
\ruby{聞}{きこ}ゆるを
\ruby{數}{かぞ}へも
\ruby{敢}{あ}へず、
%
\ruby{急}{いそ}ぎ
\ruby{周章}{あ|は}てゝ
\ruby{{\換字{又}}}{また}
\ruby{{\換字{戸}}外}{そ|と}へ
\ruby{出}{い}でんと
すれば、

\原本頁{85-5}%
『
\ruby[g]{水野}{みづの}さん、
%
\ruby{何處}{ど|こ}へ
\ruby{今}{いま}から
\ruby{御出}{お|いで}に
なります?。
』

\原本頁{85-6}%
と、
%
\ruby{低}{ひく}く
\ruby{沈}{しづ}める
\ruby{聲音}{こわ|ね}の
\ruby{呼}{よ}び
\ruby{止}{と}めたり。
