\Entry{其十四}

% メモ 校正終了 2024-04-05
\原本頁{85-8}%
\ruby{云}{い}はゞ
\ruby{我}{わ}が
\ruby{假}{かり}の
\ruby{宿{\換字{所}}}{や|ど}の
\ruby{主人}{ある|じ}なりと
\ruby{云}{い}ふまで
なれど、
%
\ruby{東京}{とう|けい}あたりに% 原本「東京」を「とうけい」としている
\ruby{黒塗}{くろ|ぬり}の
\ruby{小札}{こ|ふだ}
\ruby{懸}{か}け
ならべたる
\ruby{商賣}{しやう|ばい}づくの
\ruby{下宿屋}{げ|しゆく|や}と
いふにはあらで、
%
\ruby{我}{わ}が
\ruby{校長}{かう|ちやう}の
\ruby{高田}{たか|だ}と
\ruby{懇意}{こん|い}なる
\ruby{間柄}{あひだ|がら}なるより、
%
\ruby{其}{そ}の
\ruby{云}{い}ひ
\ruby{入}{いれ}に
よりて、
%
\原本頁{86-1}%
\ruby{唯}{たゞ}
\ruby{我}{われ}
\ruby{一人}{ひと|り}を
\ruby{賓客}{きや|く}
\ruby{同樣}{どう|やう}に、
%
\ruby{萬般}{よろ|づ}
\ruby{親切}{しん|せつ}に
\ruby{世話}{せ|わ}し
\ruby{吳}{く}るゝ
\ruby{此家}{こ|ゝ}の
\ruby{老夫}{おや|ぢ}の
\ruby[g]{吉右衛門}{きちゑもん}に
\ruby{呼}{よ}び
\ruby{{\換字{留}}}{と}められては、
%
\ruby{心}{こゝろ}の
\ruby{急}{せ}いたる
\ruby{折}{をり}からとて、
%
あらずもがなには
\ruby{思}{おも}ひながら、
%
\ruby{後}{あと}
\ruby{振}{ふ}り
\ruby{反}{かへ}りて
\ruby{立停}{たち|とゞ}まり、

\原本頁{86-4}%
『ア、
%
\ruby{一寸}{ちよ|いと}
\ruby{濱町}{はま|ちやう}まで
\ruby{行}{い}つて
\ruby{來}{き}ます。
%
\ruby{何程}{いく|ら}
\ruby{急}{いそ}いでも
\ruby{遲}{おそ}くは
ならうが、
%
\ruby{歸}{かへ}ることは
\ruby{屹度}{きつ|と}
\ruby{歸}{かへ}ります。
%
\ruby{濟}{す}まんけれど
\ruby{敲}{たゝ}きますから、
%
\ruby{關}{かま}はず
\ruby{{\換字{戸}}締}{し|ま}りを
\ruby{仕}{し}て
\ruby{仕舞}{し|ま}つて
\ruby{寢}{やす}んで
\ruby{下}{くだ}さい。
』

\原本頁{86-7}%
と
\ruby{云}{い}ひつゝ
\ruby{燈火}{あか|り}さす
\ruby{茶}{ちや}の
\ruby{室}{ま}を
\ruby{覗}{うかゞ}へば、
%
\ruby{讀}{よ}みさしたる
\ruby{新聞}{しん|ぶん}を
\ruby{傍}{かたへ}に
\ruby{置}{お}きて、
%
\ruby{兀}{は}げたる
\ruby{頭}{かしら}の
\ruby{澤々}{つや|〳〵}と
\ruby{光}{ひか}れる
\ruby[g]{吉右衛門}{きちゑもん}は、
%
\ruby{眞鍮緣}{しん|ちゆう|ぶち}の
\ruby{鏡玉}{た|ま}
\ruby{圓}{まろ}き
\ruby{{\換字{古}}風}{むか|し}
\ruby{眼鏡}{め|がね}を
\ruby{掛}{か}けたる、
%
\ruby{淸}{きよ}らなる
\ruby{赤}{あか}ら
\ruby{顏}{がほ}を
\ruby{此方}{こな|た}に
\ruby{向}{む}けたる
\ruby{其}{そ}の
\ruby{右}{みぎ}の
\ruby{方}{かた}には、
%
\ruby{孫娘}{まご|むすめ}の
\ruby{一昨年}{を|と|ゝし}
\ruby{小學}{せう|がく}を
\ruby{卒}{を}へたる
ばかりなるが、
%
\ruby{何}{なに}を
\ruby{讀}{よ}める
ならんか
\ruby{燈火}{とも|しび}の
\ruby{下}{した}に
\ruby{身}{み}を
\ruby{低}{ひく}く
\ruby{俯}{ふ}して、
%
\ruby{疊}{たゝみ}に
\ruby{置}{お}ける
\ruby{書}{しよ}に
\原本頁{87-1}%
\ruby{餘念}{よ|ねん}
\ruby{無}{な}く
\ruby{讀}{よ}み
\ruby{入}{い}つたる、
%
\ruby{其}{そ}の
\ruby{黑}{くろ}き
\ruby{頭髮}{かし|ら}に
\ruby{何}{なに}やら
\ruby{紅}{あか}き
\ruby{巾}{きれ}
\ruby{美}{うつく}しく、
%
\ruby{一幅}{いつ|ぷく}の
\ruby{{\換字{平}}和}{へい|わ}の
\ruby{夜}{よる}の
\ruby{圖}{づ}は
\ruby{眼}{め}の
\ruby{{\換字{前}}}{まへ}に
\ruby{現}{あら}はれて、
%
\ruby{身}{み}の
\ruby{疲}{つか}れ
\ruby{心}{こゝろ}の
\ruby{勞}{つか}れを
\ruby{休}{やす}むる
\ruby{間}{ま}も
\ruby{無}{な}き
\ruby{水野}{みづ|の}をして、
%
\ruby{人}{ひと}は
\ruby{斯}{か}く
\ruby{無邪氣}{む|じや|き}に
\ruby{世}{よ}を
\ruby{{\換字{送}}}{おく}るもあるをと、
%
そゞろに
\ruby{其}{そ}の
\ruby{無事}{ぶ|じ}の
\ruby{淸福}{せい|ふく}の
\ruby{價値}{あた|ひ}
\ruby{貴}{たつと}きを
\ruby{思}{おも}はしめぬ。

\原本頁{87-5}%
『ハア、
%
\ruby{左樣}{そ|う}でございますか、
%
\ruby{宜}{よろ}しうございますとも。
%
\換字{志}かし
\ruby{大變}{たい|へん}
せか〳〵して
いらつしやいますが、
%
\ruby{氣}{き}を
\ruby{御付}{お|つ}けなさいまし、
%
\ruby{爭}{あらそ}ひなんぞ
\ruby{爲}{な}すつては
いけませんぜ。
%
\ruby{{\換字{平}}井}{ひら|ゐ}の
お
\ruby[<j|]{澤}{さは}
\ruby{婆}{ばゞあ}の
ところへ
\ruby{御出}{お|いで}なすつたと
\ruby{聞}{き}きましたが、
%
あの
\ruby{婆}{ばゞあ}と
\ruby{物言}{もの|いひ}なんぞ
\ruby{爲}{な}さりやあ
\ruby{仕}{し}ますまいネ、
%
\ruby{彼奴}{あい|つ}はどうせ
\ruby{人}{ひと}ぢやあ
\ruby{無}{な}いのですから。
%
それは
\ruby{左樣}{そ|う}と
\ruby{岩崎}{いは|ざき}さんは
\ruby{何樣}{ど|う}でございます?。
』

\原本頁{87-11}%
『
\ruby{岩崎}{いは|ざき}は
どうも
いよ〳〵
\ruby{惡}{わる}い。
%
ナーニ
お
\ruby{澤}{さは}
\ruby{婆}{ばあ}さんには
\ruby{此方}{こつ|ち}で
\ruby{負}{ま}けて
\ruby{居}{ゐ}るから
\ruby{論}{ろん}は
\ruby{無}{な}いよ。
%
\ruby{爭}{あらそ}ひ
なんぞ
\ruby{仕}{し}て
\ruby{來}{き}たのでは
\ruby{無}{な}い。
%
たゞ
\ruby{早}{はや}く
\ruby{濱町}{はま|ちやう}へ
\ruby{行}{ゆ}かうと
\ruby{思}{おも}つて
\ruby{居}{ゐ}るので
\ruby{急}{いそ}いで
\ruby{居}{ゐ}るので。
』

\原本頁{88-3}%
『
\ruby{濱町}{はま|ちやう}は
\ruby{島木}{しま|き}さんの
ところへで
\ruby{御座}{ご|ざ}いますか。
』

\原本頁{88-4}%
『アヽ
\ruby{左樣}{そ|う}、
%
\ruby{島木}{しま|き}の
ところへだ。
』

\原本頁{88-5}%
『それぢやあ
\ruby{路}{みち}は
\ruby{{\換字{遠}}}{とほ}いし、
%
\ruby{御會話}{お|はな|し}は
\ruby{長}{なが}くなりませうし、
%
\ruby{御歸}{お|かへ}りは
\ruby{大變}{たい|へん}
\ruby{遲}{おそ}くなりましやうが、
%
なんなら
\ruby{明日}{あ|す}に
なすつては
\ruby{何樣}{ど|う}で
ございます?。
』

\原本頁{88-8}%
『
\ruby{明日}{あ|す}と
\ruby{云}{い}つて
\ruby{居}{ゐ}るわけには
\ruby{行}{い}かないのだから。
』

\原本頁{88-9}%
\ruby{此時}{この|とき}
\ruby{娘}{むすめ}は
\ruby{書}{しよ}を
\ruby{棄}{す}てゝ、
%
\ruby{急}{きふ}に
\ruby{頭}{かうべ}を
\ruby{擡}{もた}げたるが、
%
さつと
\ruby{燈火}{あか|り}を
\ruby{{\換字{浴}}}{あ}びたる
\ruby{面}{おもて}の、
%
\ruby{色}{いろ}は
\ruby{初花}{はつ|はな}の
\ruby{日}{ひ}に
\ruby{匂}{にほ}ふかと
\ruby{麗}{うる}はしく、
%
\ruby{細}{ほそ}けれど
\ruby{鮮}{あざ}やかなる
\ruby{眉}{まゆ}、
%
\ruby{小}{ちひさ}けれども
はつきりと
\ruby{仕}{し}たる
\ruby{眼}{め}つき、
%
まだ
\ruby{罪}{つみ}も
\ruby{無}{な}く
\ruby{慾}{よく}も
\ruby{無}{な}く、
%
たゞ
\ruby{生々}{いき|〳〵}と
\ruby{愛度}{あ|ど}なく
\ruby{美}{うつく}しきが、
%
\ruby{突}{つ}と
\ruby{立上}{たち|あが}りて
\ruby{走}{はし}り
\ruby{出}{い}で、

\原本頁{89-2}%
『なぜ
\ruby{其樣}{そん|な}に
\ruby{他{\換字{所}}}{よ|そ}へばかし
\ruby{入}{い}らつしやるの!。
%
\ruby{{\換字{戸}}外}{そ|と}は
もう
\ruby{眞闇}{まつ|くら}で、
%
いけませんわ。
%
\ruby[<j|]{妾}{わたし}
\ruby{御願}{お|ねが}ひだから
\ruby{御止}{お|よ}しなさいよ。
』

\原本頁{89-4}%
と、
%
\ruby{甘}{あま}へたる
\ruby{調子}{てう|し}に
\ruby{云}{い}ひ〳〵
\ruby{水野}{みづ|の}を
\ruby{扯}{ひ}きて、
%
はや
\ruby{女}{をんな}づくるべき
\ruby{齡}{とし}なれど
\ruby{{\換字{猶}}}{なほ}
\ruby{兒童}{こ|ども}くさく、
%
\ruby{{\換字{遠}}慮}{ゑん|りよ}も
\ruby{無}{な}く
\ruby{此方}{こな|た}へ
\ruby{扯}{ひ}き
\ruby{入}{い}れんとすれば、
%
\ruby{水野}{みづ|の}は
おのづと
\ruby{催}{もよほ}さるゝ
\ruby{笑}{わら}ひの
\ruby{顏}{かほ}を
\ruby{顰}{しか}めながら、
%
そつと
\ruby{其}{その}
\ruby{手}{て}をはづして、

\原本頁{89-8}%
『マア
お
\ruby{濱}{はま}ちやん、
%
\ruby{堪{\換字{忍}}}{か|に}して% 原文通り「堪忍」
お
\ruby{吳}{く}れ、
%
どうしても
\ruby{行}{い}つて
\ruby{來}{こ}なくてはならない
\ruby{事}{こと}だから。
』

\原本頁{89-10}%
と、
%
\ruby{周章}{あ|は}てゝ
\ruby{土間}{ど|ま}へ
\ruby{下}{お}りて
\ruby{出}{い}でかゝるに、
%
\ruby{媚}{なまめ}ける
\ruby{笑}{わら}ひを
\ruby{帶}{お}びたる
\ruby[<j|]{聲}{こゑ}
\ruby{美}{うつく}しく
\ruby{我}{わ}が
\ruby{背後}{うし|ろ}に
\ruby{當}{あた}つて、

\原本頁{90-1}%
『あら、
%
いやな
\ruby{人}{ひと}!、
%
きつと
\ruby{{\換字{又}}}{また}
\ruby{五十子}{い|そ|こ}さんの
\ruby{事}{こと}で
\ruby{心配}{しん|ぱい}して
\ruby{居}{ゐ}るのよ!。
』

\原本頁{90-3}%
と、
%
\ruby{{\換字{婦}}人}{をん|な}は
\ruby{口頭}{くち|さき}より
\ruby{先}{ま}づませて、
%
\ruby{戀}{こひ}
\ruby{知}{し}り
\ruby{顏}{がほ}に
\ruby{獨語}{ひとり|ご}つが
\ruby{聞}{きこ}えぬ。
%
\ruby{心}{こゝろ}も
こゝに
あらず
\ruby{思}{おもひ}の
\ruby{忙}{せは}しければ、
%
\ruby{{\換字{平}}生}{ひご|ろ}は
いと
\ruby{可愛}{か|はゆ}しと
\ruby{思}{おも}へる
\ruby{濱子}{はま|こ}が
\ruby{言葉}{こと|ば}をも、
%
\ruby{我}{わ}が
\ruby{胸}{むね}の
\ruby{中}{うち}に
\ruby{{\換字{留}}}{とゞ}むる
\ruby{暇無}{いとま|な}くて、
%
\ruby{急}{きふ}に
\ruby{村徑}{むら|みち}の
\ruby{闇}{やみ}を
\ruby{衝}{つ}いて
\ruby{歩}{ある}き
\ruby{出}{いだ}せば、
%
\ruby{門}{かど}を
\ruby{出}{い}づるや
\ruby{否}{いな}や
\ruby{足元}{あし|もと}
\ruby{{\換字{近}}}{ちか}き
\ruby{蓮田}{はす|だ}の% TODO 暫定で「蓮 uf999」とする(参考「蓮 uu84ee」)
\ruby{中}{うち}より、
%
\ruby{人}{ひと}に
\ruby{驚}{おどろ}ける
\ruby{五位鷺}{ご|ゐ|さぎ}の
\ruby{其}{その}
\ruby{聲}{こゑ}
\ruby{淋}{さび}しく
\ruby{人}{ひと}を
\ruby{驚}{おどろ}かして、
%
ぎやあと
\ruby{鳴}{な}きつつ
\ruby{立}{た}つて
\ruby{去}{さ}りたり。
