\Entry{其十四}

% メモ 校正終了 2024-04-05 2024-05-24 2024-06-17
\原本頁{85-8}%
\ruby{云}{い}はゞ
\ruby{我}{わ}が
\ruby{假}{かり}の
\ruby[g]{宿{\換字{所}}}{や ど }の
\ruby[g]{主人}{あるじ }なりと
\ruby{云}{い}ふまで
なれど、
%
\ruby[g]{東京}{とうけい}あたりに% 原本「東京」を「とうけい」としている
\ruby[g]{黒塗}{くろぬり}の
\ruby[g]{小札}{こ ふだ}
\ruby{懸}{か}け
ならべたる
\ruby[||j>]{商}{しやう}
\ruby[||j>]{賣}{ ばい}づくの
% \ruby{商賣}{しやう|ばい}づくの
\ruby{下宿屋}{げ|しゆく|や}と
いふにはあらで、
%
\ruby{我}{わ}が
\ruby[||j>]{校}{かう}
\ruby[||j>]{長}{ちやう}の
% \ruby{校長}{かう|ちやう}の
\ruby[g]{高田}{たかだ }と
\ruby[g]{懇意}{こんい }なる
\ruby[||j>]{間}{あひだ}
\ruby[||j>]{柄}{ がら}なるより、
% \ruby{間柄}{あひだ|がら}なるより、
%
\ruby{其}{そ}の
\ruby{云}{い}ひ
\ruby{入}{いれ}に
よりて、
%
\原本頁{86-1}%
\ruby{唯}{たゞ}
\ruby{我}{われ}
\ruby[g]{一人}{ひとり }を
\ruby[g]{賓客}{きやく }
\ruby[g]{同樣}{どうやう}に、
%
\ruby[g]{萬般}{よろづ }
\ruby[g]{親切}{しんせつ}に
\ruby[g]{世話}{せ わ }し
\ruby{吳}{く}るゝ
\ruby[g]{此家}{こ ゝ }の
\原本頁{86-2}\改行%
\ruby[g]{老夫}{おやぢ }の
\ruby{吉右衛門}{きち||ゑ|もん}に
\ruby{呼}{よ}び
\ruby{{\換字{留}}}{と}められては、
%
\ruby{心}{こゝろ}の
\ruby{急}{せ}いたる
\ruby{折}{をり}からとて
\改行% 校正作業の簡略化のため
、
%
\原本頁{86-3}\改行%
あらずもがなには
\ruby{思}{おも}ひながら、
%
\ruby{後}{あと}
\ruby{振}{ふ}り
\ruby{反}{かへ}りて
\ruby[g]{立停}{たちとゞ}まり、

\原本頁{86-4}%
『
ア、
%
\ruby[g]{一寸}{ちよいと}
\ruby[||j>]{濱}{はま}
\ruby[||j>]{町}{ちやう}まで
% \ruby{濱町}{はま|ちやう}まで
\ruby{行}{い}つて
\ruby{來}{き}ます。
%
\ruby[g]{何程}{いくら }
\ruby{急}{いそ}いでも
\ruby{遲}{おそ}くは
ならうが、
%
\ruby{歸}{かへ}ることは
\ruby[g]{屹度}{きつと }
\ruby{歸}{かへ}ります。
%
\ruby{濟}{す}まんけれど
\ruby{敲}{たゝ}きますから、
%
\ruby{關}{かま}はず
\ruby[g]{{\換字{戸}}締}{し ま }りを
\ruby{仕}{し}て
\ruby[g]{仕舞}{し ま }つて
\ruby{寢}{やす}んで
\ruby{下}{くだ}さい。
』

\原本頁{86-7}%
と
\ruby{云}{い}ひつゝ
\ruby[g]{燈火}{あかり }さす
\ruby{茶}{ちや}の
\ruby{室}{ま}を
\ruby{覗}{うかゞ}へば、
%
\ruby{讀}{よ}みさしたる
\ruby[g]{新聞}{しんぶん}を
\ruby{傍}{かたへ}に
\ruby{置}{お}きて、
%
\ruby{兀}{は}げたる
\ruby{頭}{かしら}の
\ruby[g]{澤々}{つや〳〵}と
\ruby{光}{ひか}れる
\ruby{吉右衛門}{きち||ゑ|もん}は、
%
\ruby[||j>]{眞}{しん}
\ruby[||j>]{鍮}{ちゆう}
\ruby[||j>]{緣}{ ぶち}の
\ruby[g]{鏡玉}{た ま }
\原本頁{86-9}\改行%
\ruby{圓}{まろ}き
\ruby[g]{{\換字{古}}風}{むかし }
\ruby[g]{眼鏡}{め がね}を
\ruby{掛}{か}けたる、
%
\ruby{淸}{きよ}らなる
\ruby{赤}{あか}ら
\ruby{顏}{がほ}を
\ruby[g]{此方}{こなた }に% ルビ調整(原本通り)
\ruby{向}{む}けたる
\ruby{其}{そ}の
\ruby{右}{みぎ}の
\ruby{方}{かた}には、
%
\ruby[||j>]{孫}{まご}
\ruby[||j>]{娘}{むすめ}の
% \ruby{孫娘}{まご|むすめ}の
\ruby{一昨年}{をと|ゝ|し}% ルビ調整(原本通り)
\ruby[g]{小學}{せうがく}を
\ruby{卒}{を}へたる
ばかりなるが、
%
\ruby{何}{なに}を
\ruby{讀}{よ}める
ならんか
\ruby[g]{燈火}{ともしび}の
\ruby{下}{した}に
\ruby{身}{み}を
\ruby{低}{ひく}く
\ruby{俯}{ふ}して、
%
\ruby{疊}{たゝみ}に
\ruby{置}{お}ける
\ruby{書}{しよ}に
\原本頁{87-1}\改行%
\ruby[g]{餘念}{よ ねん}
\ruby{無}{な}く
\ruby{讀}{よ}み
\ruby{入}{い}つたる、
%
\ruby{其}{そ}の
\ruby{黑}{くろ}き
\ruby[g]{頭髮}{かしら }に
\ruby{何}{なに}やら
\ruby{紅}{あか}き
\ruby{巾}{きれ}
\ruby{美}{うつく}しく、
%
\ruby[g]{一幅}{いつぷく}の
\ruby[g]{{\換字{平}}和}{へいわ }の
\ruby{夜}{よる}の
\ruby{圖}{づ}は
\ruby{眼}{め}の
\ruby{{\換字{前}}}{まへ}に
\ruby{現}{あら}はれて、
%
\ruby{身}{み}の
\ruby{疲}{つか}れ
\ruby{心}{こゝろ}の
\ruby{勞}{つか}れを
\原本頁{87-3}\改行%
\ruby{休}{やす}むる
\ruby{間}{ま}も
\ruby{無}{な}き
\ruby[g]{水野}{みづの }をして、
%
\ruby{人}{ひと}は
\ruby{斯}{か}く
\ruby{無邪氣}{む|じや|き}に
\ruby{世}{よ}を
\ruby{{\換字{送}}}{おく}るもあるをと、
%
そゞろに
\ruby{其}{そ}の
\ruby[g]{無事}{ぶ じ }の
\ruby[g]{淸福}{せいふく}の
\ruby[g]{價値}{あたひ }
\ruby[<j||]{貴}{たつと}きを% ルビ調整(原本通り)
\ruby{思}{おも}はしめぬ。

\原本頁{87-5}%
『
ハア、
%
\ruby[g]{左樣}{そ う }でございますか、
%
\ruby{宜}{よろ}しうございますとも。
%
\換字{志}かし
\原本頁{87-6}\改行%
\ruby[g]{大變}{たいへん}
せか〳〵して
いらつしやいますが、
%
\ruby{氣}{き}を
\ruby[g]{御付}{お つ }けなさいまし
\改行% 校正作業の簡略化のため
、
%
\原本頁{87-7}\改行%
\ruby{爭}{あらそ}ひなんぞ
\ruby{爲}{な}すつては
いけませんぜ。
%
\ruby[g]{{\換字{平}}井}{ひらゐ }の
お
\ruby{澤}{さは}
\ruby{婆}{ばゞあ}の
ところへ
\原本頁{87-8}\改行%
\ruby[g]{御出}{お いで}なすつたと
\ruby{聞}{き}きましたが、
%
あの
\ruby{婆}{ばゞあ}と
\ruby[g]{物言}{ものいひ}なんぞ
\ruby{爲}{な}さりやあ
\ruby{仕}{し}ますまいネ、
%
\ruby[g]{彼奴}{あいつ }はどうせ
\ruby{人}{ひと}ぢやあ
\ruby{無}{な}いのですから。
%
それは
\ruby[g]{左樣}{そ う }と
\ruby[g]{岩崎}{いはざき}さんは
\ruby[g]{何樣}{ど う }でございます?。
』

\原本頁{87-11}%
『
\ruby[g]{岩崎}{いはざき}は
どうも
いよ〳〵
\ruby{惡}{わる}い。
%
ナーニ
お
\ruby{澤}{さは}
\ruby{婆}{ばあ}さんには
\ruby[g]{此方}{こつち }で% ルビ調整(原本通り)
\ruby{負}{ま}けて
\ruby{居}{ゐ}るから
\ruby{論}{ろん}は
\ruby{無}{な}いよ、
%
\ruby{爭}{あらそ}ひ
なんぞ
\ruby{仕}{し}て
\ruby{來}{き}たのでは
\ruby{無}{な}い。
%
た
\原本頁{88−2}\改行%
ゞ
\ruby{早}{はや}く
\ruby[||j>]{濱}{はま}
\ruby[||j>]{町}{ちやう}へ
% \ruby{濱町}{はま|ちやう}へ
\ruby{行}{ゆ}かうと
\ruby{思}{おも}つて
\ruby{居}{ゐ}るので
\ruby{急}{いそ}いで
\ruby{居}{ゐ}るので。
』

\原本頁{88-3}%
『
\ruby[||j>]{濱}{はま}
\ruby[||j>]{町}{ちやう}は
% \ruby{濱町}{はま|ちやう}は
\ruby[g]{島木}{しまき }さんの
ところへで
\ruby[g]{御座}{ご ざ }いますか。
』

\原本頁{88-4}%
『
アヽ
\ruby[g]{左樣}{そ う }、
%
\ruby[g]{島木}{しまき }の
ところへだ。
』

\原本頁{88-5}%
『
それぢやあ
\ruby{路}{みち}は
\ruby{{\換字{遠}}}{とほ}いし、
%
\ruby{御會話}{お|はな|し}は
\ruby{長}{なが}くなりませうし、
%
\ruby[g]{御歸}{お かへ}りは
\ruby[g]{大變}{たいへん}
\ruby{遲}{おそ}くなりましやうが、
%
なんなら
\ruby[g]{明日}{あ す }に
なすつては
\ruby[g]{何樣}{ど う }で
ございます?。
』

\原本頁{88-8}%
『
\ruby[g]{明日}{あ す }と
\ruby{云}{い}つて
\ruby{居}{ゐ}るわけには
\ruby{行}{い}かないのだから。
』

\原本頁{88-9}%
\ruby[g]{此時}{このとき}
\ruby{娘}{むすめ}は
\ruby{書}{しよ}を
\ruby{棄}{す}てゝ、
%
\ruby{急}{きふ}に
\ruby{頭}{かうべ}を
\ruby{擡}{もた}げたるが、
%
さつと
\ruby[g]{燈火}{あかり }を
\ruby{{\換字{浴}}}{あ}びたる
\ruby{面}{おもて}の、
%
\ruby{色}{いろ}は
\ruby[g]{初花}{はつはな}の
\ruby{日}{ひ}に
\ruby{匂}{にほ}ふかと
\ruby{麗}{うる}はしく、
%
\ruby{細}{ほそ}けれど
\ruby{鮮}{あざ}やかなる
\ruby{眉}{まゆ}、
%
\ruby{小}{ちひさ}けれども
はつきりと
\ruby{仕}{し}たる
\ruby{眼}{め}つき、
%
まだ
\ruby{罪}{つみ}も
\ruby{無}{な}く
\ruby{慾}{よく}
\原本頁{89-1}\改行%
も
\ruby{無}{な}く、
%
たゞ
\ruby[g]{生々}{いき〳〵}と
\ruby[g]{愛度}{あ ど }なく
\ruby{美}{うつく}しきが、
%
\ruby{突}{つ}と
\ruby[g]{立上}{たちあが}りて
\ruby{走}{はし}り
\ruby{出}{い}て
\改行% 校正作業の簡略化のため
、

\原本頁{89-2}%
『
なぜ
\ruby[g]{其樣}{そんな }に
\ruby[g]{他{\換字{所}}}{よ そ }へばかし
\ruby{入}{い}らつしやるの!。
%
\ruby[g]{{\換字{戸}}外}{そ と }は
もう
\ruby[g]{眞闇}{まつくら}で、
%
いけませんわ。
%
\ruby[||j>]{妾}{わたし}
\ruby[g]{御願}{おねが}ひだから% ルビ調整(原本通り)
\ruby[g]{御止}{お よ }しなさいよ。
』

\原本頁{89-4}%
と、
%
\ruby{甘}{あま}へたる
\ruby[g]{調子}{てうし }に
\ruby{云}{い}ひ〳〵
\ruby[g]{水野}{みづの }を
\ruby{扯}{ひ}きて、
%
はや
\ruby{女}{をんな}づくるべき
\原本頁{89-5}\改行%
\ruby{齡}{とし}なれど
\ruby{{\換字{猶}}}{なほ}
\ruby[g]{兒童}{こ ども}くさく、
%
\ruby[g]{{\換字{遠}}慮}{ゑんりよ}も
\ruby{無}{な}く
\ruby[g]{此方}{こ なた}へ% ルビ調整(原本通り)
\ruby{扯}{ひ}き
\ruby{入}{い}れんとすれば
\改行% 校正作業の簡略化のため
、
%
\原本頁{89-6}\改行%
\ruby[g]{水野}{みづの }は
おのづと
\ruby{催}{もよほ}さるゝ
\ruby{笑}{わら}ひの
\ruby{顏}{かほ}を
\ruby{顰}{しか}めながら、
%
そつと
\ruby{其}{その}
\ruby{手}{て}をはづして、

\原本頁{89-8}%
『
マア
お
\ruby{濱}{はま}ちやん、
%
\ruby[g]{堪{\換字{忍}}}{か に }して% 原文通り「堪忍」
お
\ruby{吳}{く}れ、
%
どうしても
\ruby{行}{い}つて
\ruby{來}{こ}なくてはならない
\ruby{事}{こと}だから。
』

\原本頁{89-10}%
と、
%
\ruby[g]{周章}{あ は }てゝ
\ruby[g]{土間}{ど ま }へ
\ruby{下}{お}りて
\ruby{出}{い}でかゝるに、
%
\ruby{媚}{なまめ}ける
\ruby{笑}{わら}ひを
\ruby{帶}{お}びたる
\ruby{聲}{こゑ}
\ruby{美}{うつく}しく
\ruby{我}{わ}が
\ruby[g]{背後}{うしろ }に
\ruby{當}{あた}つて、

\原本頁{90-1}%
『
あら、
%
いやな
\ruby{人}{ひと}!、
%
きつと
\ruby{{\換字{又}}}{また}
\ruby{五十子}{い|そ|こ}さんの
\ruby{事}{こと}で
\ruby[g]{心配}{しんぱい}して
\ruby{居}{ゐ}るのよ!。
』

\原本頁{90-3}%
と、
%
\ruby[g]{{\換字{婦}}人}{をんな }は
\ruby[g]{口頭}{くちさき}より
\ruby{先}{ま}づませて、
%
\ruby{戀}{こひ}
\ruby{知}{し}り
\ruby{顏}{がほ}に
\ruby[g]{獨語}{ひとりご}つが
\ruby{聞}{きこ}えぬ。
%
\ruby{心}{こゝろ}も
こゝに
あらず
\ruby{思}{おもひ}の
\ruby{忙}{せは}しければ、
%
\ruby[g]{{\換字{平}}生}{ひごろ }は% ルビ調整(原本通り)
いと
\ruby[g]{可愛}{か はゆ}しと
\ruby{思}{おも}へる
\ruby[g]{濱子}{はまこ }が
\ruby[g]{言葉}{ことば }をも、
%
\ruby{我}{わ}が
\ruby{胸}{むね}の
\ruby{中}{うち}に
\ruby{{\換字{留}}}{とゞ}むる
\ruby[g]{暇無}{いとまな}くて、
%
\ruby{急}{きふ}に
\ruby[g]{村徑}{むらみち}の
\ruby{闇}{やみ}
\原本頁{90-6}\改行%
を
\ruby{衝}{つ}いて
\ruby{歩}{ある}き
\ruby{出}{いだ}せば、
%
\ruby{門}{かど}を
\ruby{出}{い}づるや
\ruby{否}{いな}や
\ruby[g]{足元}{あしもと}
\ruby{{\換字{近}}}{ちか}き
\ruby[g]{蓮田}{はすだ }の% 「蓮 uf999」(参考「蓮 u84ee」)
\ruby{中}{うち}より
\改行% 校正作業の簡略化のため
、
%
\原本頁{90-7}\改行%
\ruby{人}{ひと}に
\ruby{驚}{おどろ}ける
\ruby{五位鷺}{ご|ゐ|さぎ}の
\ruby{其}{その}
\ruby{聲}{こゑ}
\ruby{淋}{さび}しく
\ruby{人}{ひと}を
\ruby{驚}{おどろ}かして、
%
ぎやあと
\ruby{鳴}{な}きつつ
\ruby{立}{た}つて
\ruby{去}{さ}りたり。
