\Entry{其十五}

% メモ 校正終了 2024-04-06
\原本頁{90-10}%
\ruby{勞}{らう}を
\ruby{厭}{いと}ひてにはあらず、
%
\ruby{時}{とき}を
\ruby{惜}{をし}みて、
%
\ruby{勸}{すゝ}むる
\ruby{人力車}{く|る|ま}の
ありしまま、
%
\原本頁{91-1}%
さるところより
\ruby{其車}{そ|れ}には
\ruby{乘}{の}りしが、
%
やうやく
\ruby{濱町}{はま|ちやう}に
\ruby{着}{つ}きし
\ruby{時}{とき}には、
%
\ruby{流石}{さす|が}に
\ruby{人}{ひと}の
\ruby{家}{いへ}を
\ruby{音}{おと}づれんは
\ruby{後目痛}{うしろ|め|た}きほど
\ruby{{\換字{更}}}{ふ}けに
\ruby{{\換字{更}}}{ふ}けたり。
%
\ruby{日頃}{ひ|ごろ}
\ruby{心{\換字{遣}}}{こゝろ|づか}ひの
\ruby{鹵莾}{おろ|か}ならぬ
\ruby{水野}{みづ|の}は、
%
\ruby{{\換字{鎖}}}{とざ}し
\ruby{固}{かた}めたる
\ruby{{\換字{戸}}}{と}を
\ruby{思}{おも}ひ
\ruby{{\換字{遣}}}{や}りなく
\ruby{打敲}{うち|たゝ}きて、
%
\ruby{{\換字{近}}隣}{あた|り}の
\ruby{寢耳}{ね|みゝ}をまで
\ruby{驚}{おどろ}かさんことを
\ruby{憚}{はゞか}り、% 「憚 は(ゞ)か」
%
\ruby{聊}{いさゝ}か
\ruby{自}{みづか}ら
\ruby{躊躇}{ため|ら}ひしが、
%
\ruby{愚}{おろか}なり、
%
\ruby{臆}{おく}して
\ruby{已}{や}むべきには
あらぬものをと、
%
\ruby{手}{て}を
\ruby{擧}{あ}げて
ほと〳〵と
\ruby{星}{ほし}の
\ruby{下}{した}に
\ruby{敲}{たゝ}きぬ。

\原本頁{91-7}%
\ruby{心}{こゝろ}の
\ruby{優}{やさ}しさに
おのづから
\ruby{手}{て}も
\ruby{柔軟}{やはら|か}に
\ruby{當}{あた}りて、
%
\ruby{其}{そ}の
\ruby{音}{おと}は
\ruby{左}{さ}まで
\ruby{{\換字{強}}}{つよ}からざりしが、
%
\ruby{幸}{さいはひ}にして
\ruby{未}{ま}だ
\ruby{睡}{ねむ}らざりし
\ruby{女}{をんな}の
ありけむ、
%
ハイと
\ruby{明}{あき}らかに
\ruby{答}{こた}ふる
\ruby{聲}{こゑ}して、

\原本頁{91-10}%
『
\ruby{誰樣}{どな|た}?。
%
\ruby{伊東}{い|とう}さん?。
』

\原本頁{91-11}%
と、
%
\ruby{云}{い}ひながら
\ruby{開}{あ}けに
かゝりたり。

\原本頁{92-1}%
\ruby{伊東}{い|とう}とは
\ruby{島木}{しま|き}を
\ruby{外}{ほか}にして
\ruby{唯}{たゞ}
\ruby{一人}{ひと|り}の
\ruby{此}{こ}の
\ruby{家}{や}の
\ruby{止宿者}{き|や|く}にて、
%
\ruby{無類}{む|るゐ}の
\ruby{極樂蜻蛉}{ごく|らく|とん|ぼ}なるよしを
\ruby{島木}{しま|き}より
\ruby{聞}{き}きしが、
%
さては
\ruby{今{\換字{宵}}}{こ|よひ}は
その
\ruby{男}{をとこ}の、
%
\ruby{何處}{いづ|く}の
\ruby{花}{はな}の
\ruby{陰}{かげ}にか
\ruby{憩}{いこ}ひて、
%
\ruby{{\換字{更}}}{ふ}けて
\ruby{{\換字{猶}}}{なほ}
\ruby{今}{いま}に
\ruby{歸}{かへ}り
\ruby{來}{きた}らざるを、
%
\ruby{婢}{をんな}の
\ruby{待}{ま}ち
\ruby{居}{ゐ}たりしならんと
\ruby{早}{はや}くも
\ruby{猜}{すゐ}しぬ。

\原本頁{92-5}%
『イヽエ、
%
\ruby{島木}{しま|き}さんを
\ruby{急用}{きふ|よう}で
\ruby{{\換字{尋}}}{たづ}ねて
\ruby{來}{き}ました。
%
わたしは
\ruby{水野}{みづ|の}と
いふものです。
』

\原本頁{92-7}%
と、
%
\ruby{云}{い}ふ
\ruby{間}{ま}に
\ruby{雨{\換字{戸}}}{あま|ど}は
\ruby{一枚}{いち|まい}
\ruby{繰}{く}り
\ruby{明}{あ}けられて、
%
\ruby{細帶姿}{ほそ|おび|すがた}の
\換字{志}どけ
\ruby{無}{な}く
\ruby{背後}{うし|ろ}の
\ruby{上}{あが}り
\ruby{端}{はな}に
\ruby{置}{お}きたる
\ruby{小}{こ}
\ruby{洋燈}{らん|ぷ}の
\ruby{光}{ひかり}の
\ruby{中}{うち}に
\ruby{現}{あらは}れたるは、
%
\ruby{丸顏}{まる|がほ}の
\ruby{色白}{いろ|じろ}の
\ruby{氣}{き}さくものゝ、
%
\ruby{名}{な}は
\ruby{忘}{わす}れたれど
\ruby[g]{見記臆}{みおぼえ}ある% 原本通り「おぼえ」
\ruby{女}{をんな}なり。

\原本頁{92-10}%
『オヤ、
%
\ruby{水野}{みづ|の}さんでしたか、
%
\ruby{存}{ぞん}じてましたよ、
%
たしか
\ruby{彼}{あ}の
\ruby{菖蒲}{しやう|ぶ}のある
\ruby{四ッ木}{よ| |ぎ}とかの。% TODO 四ツ木
%
\ruby{能}{よ}くおぼえて
\ruby{居}{ゐ}たでしやう。
%
\ruby{褒}{ほ}めて
\ruby{頂戴}{ちやう|だい}な、
%
\原本頁{93-1}
ホヽヽ、
%
まあ
\ruby{御入}{お|はい}んなさい。
%
\ruby{大層}{たい|そう}
\ruby{遲}{おそ}く
\ruby{入}{い}らした
\ruby{事}{こと}ネ。
%
エエ、
%
\ruby{居}{ゐ}らつしやいますとも
\ruby{島木}{しま|き}さんは。
%
ハア、
%
イエ
\ruby{未}{ま}だ
\ruby{御睡}{お|よ}り
\ruby{就}{つ}きや
なさりますまい、
%
\ruby{今}{いま}しがた
\ruby{他{\換字{所}}}{よ|そ}から
\ruby{御歸}{お|かへ}りに
なつたばかりなんですから。
』

\原本頁{93-4}%
と、
%
\ruby{一人}{ひと|り}で
\ruby{饒舌}{しや|べ}りながら
\ruby{後}{あと}を
\ruby{{\換字{鎖}}}{し}めて、
%
やがて、

\原本頁{93-5}%
『ホヽ、
%
\ruby{此樣}{こ|ん}な
\ruby{姿}{なり}を
\ruby{仕}{し}て
\ruby{居}{ゐ}て、
%
\ruby{御免}{ご|めん}なさいましよ。
』

\原本頁{93-6}%
と、
%
\ruby{云}{い}ひ〳〵
\ruby{先}{さき}に
\ruby{立}{た}つて
\ruby{二階}{に|かい}へ
\ruby{導}{みち}びき、

\原本頁{93-7}%
『
\ruby{島木}{しま|き}さん、
%
さあ
\ruby{御起}{お|お}きなさいまし。
%
\ruby{貴下}{あな|た}の
\ruby{好}{すき}な
\ruby{水野}{みづ|の}さんが
\ruby{御來臨}{お|い|で}なすつてよ。
%
\ruby{明日}{あし|た}は
\ruby{驕}{おご}つて
\ruby{下}{くだ}さるでしようネ。
』

\原本頁{93-10}%
と、
%
\ruby{其}{その}
\ruby{室}{へや}に
\ruby{入}{い}つて
\ruby{{\換字{遠}}慮無}{ゑん|りよ|な}く
\ruby{洋燈}{らん|ぷ}の
\ruby{火}{ひ}を
\ruby{明}{あか}るくしたり。

\原本頁{93-11}%
『
\ruby{何}{なん}だ
\ruby{驕}{おご}つて
\ruby{下}{くだ}さるで
\換字{志}やうも
\ruby{無}{な}いもんだ、
%
\ruby{自{\換字{分}}}{じ|ぶん}が
\ruby{岡惚}{をか|ぼ}れて
\ruby{居}{ゐ}やがるんだ
\ruby{癖}{くせ}に。
』

\原本頁{94-2}%
と、
%
\ruby{輕}{かる}く
\ruby{罵}{のゝし}りながら
\ruby{島木}{しま|き}は
\ruby{起}{お}き
\ruby{出}{い}でしが、
%
\ruby{既}{はや}
\ruby{水野}{みづ|の}の
\ruby{{\換字{近}}々}{ちか|〴〵}と
\ruby{入}{い}り
\ruby{來}{きた}り
\ruby{居}{を}りて、
%
\ruby{今}{いま}の
\ruby{戲言}{たは|むれ}を
\ruby{聞}{き}きしや
\ruby{苦虫}{にが|むし}を% 原本通り「虫」
\ruby{噛}{か}みたる
\ruby{如}{ごと}き
\ruby{顏色}{かほ|つき}なせるを
\ruby{見}{み}て、

\原本頁{94-5}%
『ヤ、
%
\ruby{失敬}{しつ|けい}
\g詰めruby{々々}{〳〵}。
%
\ruby{戲言}{じやう|だん}だよ。
%
\ruby{大層}{たい|そう}
\ruby{遲}{おそ}く
\ruby{來}{き}たぢや
\ruby{無}{な}いか。
%
さあ
まあ
\ruby{此上}{こ|れ}に
\ruby{坐}{すわ}つて
\ruby{吳}{く}れたまへ。
』

\原本頁{94-7}%
と、
%
\ruby{慌}{あわ}てゝ
\ruby{敷物}{しき|もの}を
\ruby{出}{いだ}し、
%
\ruby{自己}{おの|れ}は
\ruby{手早}{て|ばや}く
\ruby{衣}{い}を
\ruby{改}{あらた}めたり。

\原本頁{94-8}%
『オイ
お
\ruby{作}{さく}さん、
%
\ruby{此處}{こ|ゝ}は
\ruby{乃公}{お|れ}が
\ruby{片}{かた}づけて
\ruby{仕舞}{し|ま}ふがネ、
%
もう
\ruby{火}{ひ}は
\ruby[<j|]{皆}{みんな}
\ruby{{\換字{消}}}{き}えて
\ruby{仕舞}{し|ま}つたかエ、
%
せめて
\ruby{御茶}{お|ちや}だけ
\ruby{欲}{ほし}いのだが。
』

\原本頁{94-10}%
『ハア、
%
もう
\ruby{樓下}{し|た}にも
ありませんが
\ruby{打火}{お|こ}して
あげましやう。
%
ナアニ
\ruby{別段}{べつ|だん}
\ruby{譯}{わけ}は
ありませんから。
』

\原本頁{95-1}%
\ruby{此家}{こ|ゝ}は
\ruby{家作}{や|づく}りも
\ruby{什器}{だう|ぐ}も
\ruby{淸潔}{きれ|い}に、
%
\ruby{四十五六}{し|じう|ご|ろく}の
\ruby{女}{をんな}
\ruby{主人}{ある|じ}と、
%
\ruby{此女}{こ|れ}と、
%
\ruby{下働}{した|ばたら}きの
\ruby{婢}{をんな}と
\ruby{三人}{さん|にん}して、
%
\ruby{客}{きやく}は
たゞ
\ruby{二人}{ふた|り}の
\ruby{島木}{しま|き}
\ruby{伊東}{い|とう}を
かしづく
\ruby{下宿屋}{げ|しゆく|や}
めかさぬ
\ruby{品}{ひん}の
\ruby{良}{よ}き
\ruby{家}{いへ}なれど、
%
\ruby{{\換字{又}}}{また}
\ruby{折々}{をり|〳〵}は
\ruby{骨牌}{は|な}に
\ruby{貸}{か}す
\ruby{窩}{あな}ともなり
\ruby{{\換字{兼}}}{か}ねぬほど、
%
\ruby{一切}{すべ|て}を
\ruby{金錢}{か|ね}の
\ruby{光}{ひかり}に
\ruby{美}{うつく}しく
\ruby{仕}{し}こなして
\ruby{見}{す}するところとは
\ruby{知}{し}りながら、
%
\ruby{深夜}{しん|や}に
\ruby{人}{ひと}を
\ruby{煩}{わづら}はすことの
\ruby{氣}{き}の
\ruby{毒}{どく}さに
\ruby{耐}{た}へかねて、

\原本頁{95-7}%
『マア
いゝさ
\ruby{島木}{しま|き}
\ruby{君}{くん}、
%
\ruby{茶}{ちや}なぞは
\ruby{要}{い}らんよ、
%
お
\ruby{作}{さく}さんは
もう
\ruby{寢}{やす}んで
\ruby{吳}{く}れたまへ。
』

\原本頁{95-9}%
と、
%
\ruby{水野}{みづ|の}は
\ruby{言葉}{こと|ば}を
\ruby{挿}{さしはさ}まざるを
\ruby{得}{え}ざりき。

\原本頁{95-10}%
\ruby{島木}{しま|き}は
\ruby{物}{もの}に
\ruby{滞}{とゞこほ}らずして、
%
\ruby{心}{こゝろ}の
\ruby{動}{うご}きの
\ruby{早}{はや}き
\ruby{男}{をとこ}なれば、

\原本頁{95-11}%
『ン、
%
それも
\ruby{左樣}{さ|う}だ。
%
ぢやあ
お
\ruby{作}{さく}さん
\ruby{茶}{ちや}は
いゝからね、
%
そら
\ruby{彼}{あ}の
\ruby{葡萄酒}{ぶ|だう|しゆ}と
\ruby[g]{乾燥牛肉}{ドライドビーフ}とを
\ruby{持}{も}つて
\ruby{來}{き}て
お
\ruby{吳}{く}れ。
』

\原本頁{96-2}%
と
\ruby{云}{い}へば、

\原本頁{96-3}%
『ハア、
%
\ruby{其}{そ}の
\ruby{方}{はう}が% 原本では「方」のルビが欠けているが他と合わせて「はう」
\ruby{却}{かへ}つて
\ruby{宜}{よろ}しう
\ruby{御座}{ご|ざ}んしやう。
』

\原本頁{96-4}%
と、
%
\ruby{婢}{をんな}は
\ruby{下}{した}に
\ruby{降}{お}り
\ruby{行}{ゆ}きしが、
%
\ruby{忽地}{たちま|ち}にして
\ruby{一}{ひと}つの
\ruby{廣}{ひろ}き
\ruby{{\換字{盆}}}{ぼん}に、
%
\ruby{燈}{ひ}を
\ruby{受}{う}けて
\ruby{美}{うつく}しき
ポカラの
\ruby{玻璃盞}{コ|ツ|プ}
\ruby{二}{ふた}つ、
%
\ruby{薄手}{うす|で}の
\ruby{白皿}{しろ|ざら}
\ruby{二}{ふた}つ、
%
ニツケルの
\ruby{栓拔器}{せん|ぬ|き}、
%
まだ
\ruby{開}{あ}けぬ
\ruby{薄}{うす}き
\ruby{罐詰}{くわん|づめ}、
%
\ruby{利休箸}{り|きう|ばし}を
\ruby{載}{の}せて、
%
\ruby{片手}{かた|て}に
\ruby{葡萄酒}{ぶ|だう|しゆ}の
\ruby{罎}{びん}を
\ruby{提}{ひつさ}げて
\ruby{來}{きた}りぬ。

\原本頁{96-8}%
『よし〳〵。
%
もうこれで
\ruby{好}{い}いから
\ruby{樓下}{し|た}へ
\ruby{行}{い}つて
\ruby{御就眠}{お|や|す}み。
%
\ruby{御客樣}{お|きやく|さま}が
\ruby{氣}{き}の
\ruby{{\換字{通}}}{とほ}つた
\ruby{方}{かた}だから
\ruby{御{\換字{酌}}}{お|しやく}には
\ruby{及}{およ}ばない。
%
\ruby{{\換字{勝}}手}{かつ|て}に
\ruby{御免}{ご|めん}を
\ruby{蒙}{かうむ}るさ。
』

\原本頁{96-11}%
『それぢやあ、
%
\ruby{御二人}{お|ふ|たり}で
\ruby{水入}{みづ|い}らずに
\ruby{御話}{お|はな}しなさいまし、
%
まあ
\ruby{御睦}{お|むつ}まじいこと、
%
\ruby{些}{ちと}
\ruby{妬}{や}けますネ。
%
ホヽヽヽ。
%
ですけれど
\ruby{島木}{しま|き}さん
\ruby{御用}{ご|よう}が
ありましたなら
\ruby{構}{かま}はないで
\ruby{呼}{よ}んで
\ruby{下}{くだ}さいましよ。
』

\原本頁{97-3}%
\ruby{婢}{をんな}は
\ruby{樓下}{し|た}に
\ruby{去}{さ}つて
\ruby{行}{ゆ}きたり。
%
\ruby{手早}{て|ばや}く
\ruby{片}{かた}づけられたる
\ruby{座敷}{ざ|しき}の
\ruby{好}{よ}き
\ruby{程}{ほど}に
\ruby{坐}{すわ}りて、
%
\ruby{島木}{しま|き}は
\ruby{葡萄酒}{ぶ|だう|しゆ}の
\ruby{栓}{くち}を
\ruby{拔}{ぬ}きながら
\ruby{水野}{みづ|の}の
\ruby{面}{おもて}を
\ruby{見}{み}て、

\原本頁{97-5}%
『
\ruby{君}{きみ}、
%
\ruby{大層}{たい|そう}
\ruby{顏色}{かほ|いろ}が
\ruby{惡}{わる}いぢや
\ruby{無}{な}いか。
%
\ruby{何樣}{ど|う}か
\ruby{仕}{し}はせんか、
%
\ruby{氣}{き}になるネ、
%
さあ、
%
まあ、
%
\ruby{飮}{や}つて
\ruby{吳}{く}れたまへナ。
』

\原本頁{97-7}%
と、
%
\ruby{詞}{ことば}の
\ruby{調子}{てう|し}こそ
\ruby{{\換字{猶}}}{なほ}
\ruby{冴}{さ}えたれ、
%
\ruby{顏}{かほ}には
\ruby{憂愁}{うれ|ひ}の
\ruby{曇}{くも}りを
\ruby{上}{のぼ}せて、
%
\ruby{友}{とも}を
\ruby{思}{おも}ふ
\ruby{{\換字{情}}}{こゝろ}の
\ruby{溫}{あたゝ}かくも
\ruby{溫}{あたゝ}かく、
%
\ruby{{\換字{強}}}{し}ひて
\ruby{玻璃盞}{コ|ツ|プ}を
\ruby{執}{と}らせて
\ruby{注}{つ}ぎて
\ruby{{\換字{遣}}}{や}りたる
\ruby{酒}{さけ}は
いつはり
\ruby{無}{な}き
\ruby{血}{ち}の
\ruby{色}{いろ}を
なしたり。
