\Entry{其十六}

% メモ 校正終了 2024-04-06 2024-05-25 2024-06-17
\原本頁{98-1}%
いつもながらの
\ruby{島木}{しま|き}が
\ruby{親切}{しん|せつ}の、
%
\ruby{今{\換字{宵}}}{こ|よひ}は
\ruby{別}{わ}けて
\ruby{身}{み}に
\ruby{染}{し}む
\ruby{心地}{こゝ|ち}して
\改行% 校正作業の簡略化のため
、
%
\原本頁{98-2}\改行%
\ruby{今}{いま}までには
\ruby{經驗}{おぼ|え}
\ruby{無}{な}き
\ruby{事}{こと}なるが、
%
おのずと
\ruby{脆}{もろ}くも
\ruby{涙}{なみだ}の
\ruby{湧}{わ}き
\ruby{上}{あが}るを
\改行% 校正作業の簡略化のため
、
%
\原本頁{98-3}\改行%
\ruby{水野}{みづ|の}は
\ruby{怪}{あやし}まれや
せんと
\ruby{竊}{そつ}と
\ruby{拭}{ぬぐ}ひて、
%
わざと
\ruby{眼}{め}の
\ruby{行}{ゆ}く
\ruby{方}{かた}を
\ruby{逸}{そ}らして
\ruby{床}{とこ}の
\ruby{間}{ま}を
\ruby{見}{み}つ、
%
\ruby{其處}{そ|こ}に
\ruby{掛}{かゝ}れる
\ruby{狩野風}{かの|う|ふう}の
\ruby{{\換字{達}}磨}{だる|ま}を、
%
たゞ
\ruby{譯}{わけ}も
\ruby{無}{な}く
\ruby{見}{み}つめながら、

\原本頁{98-6}%
『
ナニ
\ruby{何樣}{ど|う}も
\ruby{仕}{し}は
\ruby{仕無}{し|な}いよ、
%
\ruby{心配}{しん|ぱい}して
\ruby{吳}{く}れたまふな。
』

\原本頁{98-7}%
と、
%
\ruby{然}{さ}ばかり
\ruby{我}{わ}が
\ruby{胸}{むね}の
\ruby{中}{うち}の
\ruby{苦惱}{くるし|み}の
\ruby{色}{いろ}に
\ruby{出}{い}でゝ、
%
\ruby{人目}{ひと|め}に
\ruby{著}{しる}く
\ruby{現}{あら}はるゝかと
\ruby{驚}{おどろ}かるゝ
\ruby{心}{こゝろ}を
\ruby{押}{お}し
\ruby{隱}{かく}して
\ruby{答}{こた}へぬ。

\原本頁{98-9}%
『
\ruby{左樣}{さ|う}かエ。
%
それなら
\ruby{好}{い}いが
\ruby{餘}{あんま}り
\ruby{氣}{き}を
\ruby{使}{つか}つちやあ
いけないぜ、
%
\ruby{今日}{け|ふ}%
{---}{---}%
イヤ
\ruby{今日}{け|ふ}と
\ruby{云}{い}つちやあ
\ruby{既}{もう}
\ruby{十二時}{じふ|に|じ}
\ruby{{\換字{過}}}{す}ぎだから
をかしい。
%
\ruby{昨{\換字{宵}}}{ゆふ|べ}の
\ruby{會}{くわい}にも、
%
\ruby{君}{きみ}は
\ruby{幹事}{かん|じ}の
\ruby{山瀬}{やま|せ}のところへ、
%
\ruby{君}{きみ}の
\ruby{友人}{いう|じん}が
\ruby[||j>]{大}{たい}
\ruby[||j>]{病}{びやう}
% \ruby{大病}{たい|びやう}
で
\改行% 校正作業の簡略化のため
、
%
\原本頁{99-1}\改行%
\ruby{介抱}{かい|はう}の
\ruby{仕手}{し|て}も
\ruby{無}{な}いから
\ruby{其}{そ}の
\ruby{爲}{ため}に
\ruby{出}{で}ぬ、
%
と
\ruby{云}{い}つて
\ruby{{\換字{遣}}}{や}つたさうだが
\改行% 校正作業の簡略化のため
、
%
\原本頁{99-2}\改行%
\ruby{君}{きみ}は
\ruby{一體}{いつ|たい}
\ruby[||j>]{{\換字{情}}}{じやう}が
\ruby{深}{ふ}か
\ruby{{\換字{過}}}{す}ぎるから、
%
\ruby{餘計}{よ|けい}に
それで
\ruby{心勞}{しん|らう}でも
\ruby{仕}{し}や
\ruby{仕無}{し|な}いかと、
%
\ruby{一同}{みん|な}が
\ruby{君}{きみ}の
\ruby{爲}{ため}に
\ruby{心配}{しん|ぱい}して
ゐたよ。
』

\原本頁{99-4}%
\ruby{島木}{しま|き}が
\ruby{言葉}{こと|ば}には
\ruby{何}{なん}の
\ruby{事}{こと}も
\ruby{無}{な}けれど、
%
\ruby{水野}{みづ|の}が
\ruby{胸}{むね}には
\ruby{響}{ひゞ}くところあ
\改行% 校正作業の簡略化のため
り。

\原本頁{99-6}%
『
ムヽ、
%
\ruby{昨{\換字{宵}}}{ゆふ|べ}の
\ruby{羽{\換字{勝}}}{は|がち}
\ruby{君}{くん}の
\ruby{會}{くわい}に
\ruby{出無}{で|な}かつたのは、
%
\ruby{眞誠}{ほん|と}に
\ruby{諸君}{しよ|くん}に
\ruby{濟}{す}まなかつたが、
%
\ruby{實}{じつ}は
\ruby{如是}{か|う}して
まご〳〵して
\ruby{居}{ゐ}て、
%
\ruby{今}{いま}
\ruby{頃}{ごろ}
\ruby{君}{きみ}のところへ
\ruby{來}{く}る
\ruby{位}{くらゐ}だから、
%
\ruby{何樣}{ど|う}か
\ruby{察}{さつ}して
\ruby{赦}{ゆる}して
\ruby{吳}{く}れたまへ。
』

\原本頁{99-9}%
『
ナアニ
\ruby{赦}{ゆる}すも
\ruby{赦}{ゆる}さないも
\ruby{有}{あ}りや
\ruby{仕無}{し|な}いが
\ruby{君}{きみ}のその
\ruby{友人}{いう|じん}の
\ruby{上}{うへ}は
\ruby{兎}{と}に
\ruby{角}{かく}、
%
\ruby{一同}{みん|な}は
\ruby{眞誠}{ほん|と}に
たゞ
\ruby{君}{きみ}の
\ruby{上}{うへ}を
いろ〳〵に
\ruby{心配}{しん|ぱい}して
ゐたよ
\改行% 校正作業の簡略化のため
。
』

\原本頁{99-11}%
『
ヤ、
%
\ruby{眞}{しん}に
\ruby{諸君}{みん|な}の
\ruby{厚意}{こう|い}は
\ruby{深}{ふか}く
\ruby{謝}{しや}する。
%
\ruby{誰}{だれ}も
\ruby{僕}{ぼく}の
\ruby{不參}{ふ|さん}を
\ruby{怒}{おこ}りは
\ruby{仕無}{し|な}かつたかね。
%
\ruby{日方}{ひ|かた}
\ruby{君}{くん}は
\ruby{何}{なん}とも
\ruby{云}{い}は
\ruby{無}{な}かつたかね。
』

\原本頁{100-2}%
『
ムヽ、
%
\ruby{日方}{ひ|かた}は
\ruby{何}{なに}を
\ruby{言}{い}つたつて
\ruby{管}{かま}や
\ruby{仕無}{し|な}いがね、
%
\ruby{羽{\換字{勝}}}{は|がち}は
\ruby{君}{きみ}に
\ruby{會}{あ}
へなかつたのを、
%
\ruby{口}{くち}へは
\ruby{出}{だ}さなかつたが
\ruby{酷}{ひど}く
\ruby{殘念}{ざん|ねん}がつて
\ruby{居}{ゐ}たよ
\改行% 校正作業の簡略化のため
。
』

\原本頁{100-4}%
『
アヽ、
%
\ruby{羽{\換字{勝}}}{は|がち}
\ruby{君}{くん}には
\ruby{僕}{ぼく}も
\ruby{會}{あ}ひたかつたが、
%
\ruby{何}{なん}にしろ
\ruby{一方}{いつ|ぱう}の
\ruby{事}{こと}が
あつたので、
%
\ruby{懷}{なつ}かしくは
\ruby{思}{おも}ひながら
\ruby{意}{い}に
\ruby{任}{まか}せ
\ruby{無}{な}かつた。
%
アヽ
\ruby{僕}{ぼく}は
\ruby{羽{\換字{勝}}}{は|がち}
\ruby{君}{くん}に
\ruby{負}{そむ}いた、
%
\ruby{濟}{す}まなかつた。
』

\原本頁{100-7}%
\ruby{水野}{みづ|の}は
\ruby{{\換字{情}}}{じやう}に
\ruby{堪}{た}へざる
\ruby{如}{ごと}く、
\GWI{u1b048-u3099}つと% 「志」+「濁点」
\ruby{俯首}{うつ|む}きて
\ruby{眼}{め}を
\ruby{瞑}{ふさ}ぎつ、
%
\ruby[||j>]{獨}{ひとり}
\ruby[||j>]{語}{ ごと}のやうに
\ruby{{\換字{又}}}{また}
\ruby{再度}{ふた|ゝび}

\原本頁{100-9}%
『
アヽ、
%
\ruby{濟}{す}まなかつた。
』

\原本頁{100-10}%
と、
%
\ruby{繰}{く}り
\ruby{{\換字{返}}}{かへ}しぬ。
%
\ruby{島木}{しま|き}は
\ruby{其}{そ}の
いぢらしき
\ruby{樣子}{やう|す}を
\ruby{見}{み}て、
%
\ruby{此}{こ}の
\ruby[<j||]{{\換字{猶}}}{なほ}% 行末行頭の境界付近なので特例処置を施す
\ruby[<j||]{心}{こゝろ}の
\ruby{醇}{じゆん}なる
\ruby{年{\換字{若}}}{とし|わか}き
\ruby{友}{とも}を
\ruby{愛憐}{いと|ほし}む
\ruby{{\換字{情}}}{こゝろ}を
\ruby{起}{おこ}さゞるを
\ruby{得}{え}ざりき。

\原本頁{101-1}%
『
マア
\ruby{其}{そ}りやあ
\ruby{其}{そ}れで
\ruby{濟}{す}んだ
\ruby{事}{こと}として、
%
また
\ruby{羽{\換字{勝}}}{は|がち}に
\ruby{{\換字{遇}}}{あ}う
\ruby{時}{とき}も
\ruby{有}{あ}らうから
\ruby{好}{い}いぢやあ
\ruby{無}{な}いか。
%
さうして
\ruby{君}{きみ}の
わざ〳〵
\ruby{來}{き}た
\ruby{用事}{よう|じ}と
いふのは?。
』

\原本頁{101-4}%
\ruby{問}{と}はれて
\ruby{水野}{みづ|の}は
\ruby{猛然}{まう|ぜん}と
\ruby{我}{われ}に
\ruby{復}{かへ}り、
%
\ruby{夜}{よ}を
\ruby{冒}{をか}し
\ruby{{\換字{遠}}}{とほき}を
\ruby{歩}{あゆ}みて
\ruby{此處}{こ|ゝ}に
\ruby{來}{きた}れるも、
%
たゞ
\ruby{此}{こ}の
\ruby{一}{ひと}つの
\ruby{事}{こと}の
ためなるをやと、
%
\ruby{津}{わたり}に
\ruby{舟}{ふね}を
\ruby{得}{え}し
\ruby{心地}{こゝ|ち}して、
%
\ruby{自}{みづか}ら
\ruby{奮}{ふる}つて
\ruby{面}{おもて}を
\ruby{擡}{あ}げしが、
%
\ruby{慚}{は}づる
ところの
\ruby{有}{あ}ればにや
\ruby{直}{すぐ}に
\ruby{崩折}{くづ|を}れて、
%
\ruby{甲{\換字{斐}}}{か|ひ}
\ruby{無}{な}くも
\ruby{伏目}{ふし|め}に
なりて
\ruby{我}{わ}が
\ruby{膝}{ひざ}を
\ruby{見}{み}たり。

\原本頁{101-8}%
されど
\ruby{云}{い}はでは
\ruby{叶}{かな}はざる
ことゝて、

\原本頁{101-9}%
『
\ruby{深夜}{しん|や}に
\ruby{君}{きみ}を
\ruby{驚}{おどろ}かしたのは
\ruby{濟}{す}ま
\ruby{無}{な}かつたが、
%
かういふ
\ruby{譯}{わけ}だから
\ruby{聞}{き}いて
\ruby{吳}{く}れたまへ。
%
\ruby{實}{じつ}は
\ruby{僕}{ぼく}の
\ruby{出}{で}て
\ruby{居}{ゐ}る
\ruby{學校}{がく|かう}で、
%
\ruby{同}{おな}じ
\ruby{職}{しよく}を
\ruby{取}{と}つて
\ruby{居}{ゐ}るものに、
%
\ruby{僕}{ぼく}の
\ruby{新}{あたら}しい
\ruby{友人}{いう|じん}がある。
%
\ruby{其人}{そ|れ}は
\ruby{物}{もの}も
\ruby{出來}{で|き}れば
\ruby{氣立}{き|だて}
\原本頁{102-1}\改行%
も
\ruby{立派}{りつ|ぱ}な、
%
まことに
\ruby{得{\換字{難}}}{え|がた}い
\ruby{人物}{じん|ぶつ}なので、
%
\ruby{僕}{ぼく}は
\ruby{非常}{ひ|じやう}に
\ruby{大切}{たい|せつ}に
\ruby{思}{おも}つて
\ruby{居}{ゐ}る、
%
ところが
\ruby{其人}{そ|れ}が
\ruby[||j>]{大}{たい}
\ruby[||j>]{病}{びやう}に
% \ruby{大病}{たい|びやう}に
\ruby{罹}{かゝ}つた。
%
\ruby{一體}{いつ|たい}
\ruby{愍然}{あは|れ}な
\ruby{不幸}{ふ|かう}な
\ruby{人}{ひと}で
\改行% 校正作業の簡略化のため
、
%
\原本頁{102-3}\改行%
\ruby{母}{はゝ}は
\ruby{有}{あ}るけれども
\ruby{継}{まゝ}しい
\ruby{中}{なか}で、
%
\ruby{病氣}{びやう|き}を
\ruby{知}{し}らせて
\ruby{{\換字{遣}}}{や}つても
\ruby{振}{ふ}り
\ruby{顧}{かへ}つても
\ruby{見無}{み|な}い
\ruby{位}{くらゐ}、
%
それに
また
\ruby{家}{いへ}を
\ruby{貸}{か}して
\ruby{居}{ゐ}る
\ruby{婆}{ばゞあ}が
\ruby{殘酷}{ざん|こく}な
\ruby{奴}{やつ}で、
%
\原本頁{102-5}\改行%
\ruby{病}{や}み
\ruby{惱}{なや}んで
\ruby{居}{ゐ}るものを
\ruby{{\換字{逐}}}{お}ひ
\ruby{出}{だ}さうといふ
\ruby{位}{くらゐ}な
\ruby{非{\換字{道}}}{ひ|だう}さ。
%
\ruby{左樣}{さ|う}いふ
\ruby{中}{なか}に
\ruby{悶臥}{もん|ぐわ}して
\ruby{居}{ゐ}て、
%
\ruby{誰}{たれ}に
\ruby{世話}{せ|わ}を
されるといふ
\ruby{事}{こと}も
\ruby{無}{な}いので、
%
\ruby{可哀}{か|あい}さうに
\ruby[||j>]{病}{びやう}
\ruby[||j>]{人}{ にん}は
% \ruby{病人}{びやう|にん}は
\ruby{死}{し}を
\ruby{待}{ま}つ
ばかりになつて
\ruby{居}{ゐ}るのだ。
%
そこで
\ruby{何樣}{だ|う}しても
\ruby{餘{\換字{所}}}{よ|そ}に
\ruby{見{\換字{兼}}}{み|か}ねるから、
%
\ruby{僕}{ぼく}が
\ruby{奔走}{ほん|そう}して
\ruby{良}{い}い
\ruby{醫者}{い|しや}に
\ruby{見}{み}せて
\ruby{{\換字{遣}}}{や}ると、
%
\ruby{病}{やまひ}は
\ruby[||j>]{腸}{ちやう}
\ruby[||j>]{窒扶斯}{ ち|ぶ|す}だといふ% ルビ調整(原本通り)
\ruby{事}{こと}で、
%
\ruby{看護}{かん|ご}が
\ruby{行届}{ゆき|とゞ}か% 「屆」「届」 原本通り「届」
\ruby{無}{な}けりやあ
\ruby{無}{な}い
\ruby{生命}{いの|ち}だといふ。
%
\ruby{僕}{ぼく}は
\ruby{自{\換字{分}}}{じ|ぶん}の
\ruby{肉}{にく}を
\ruby{{\換字{削}}}{そ}いで
\ruby{食}{く}はせてなりと、
%
\ruby{何樣}{ど|う}かして
\ruby{助}{たす}けて
\ruby{{\換字{遣}}}{や}りたいと
\ruby{思}{おも}ふのだが、‥‥』

\原本頁{103-1}%
と、
%
\ruby{虛言}{う|そ}は
\ruby{少}{すこし}も
\ruby{無}{な}けれど
\ruby{忌}{い}むことは
\ruby{忌}{い}みて、
%
\ruby{此處}{こ|ゝ}までは
\ruby{云}{い}ひたりしが
\ruby{後}{あと}は
\ruby{言}{い}ひ
\ruby{澱}{よど}むを、
%
\ruby{其}{そ}の
\ruby{聲}{こゑ}の
\ruby{微}{かすか}に
\ruby{顫}{ふる}ふを
\ruby{聞}{き}き、
%
\ruby{其}{そ}の
\ruby{眼}{め}の
\ruby{濕}{ぬ}れ
\ruby{色}{いろ}なせるを
\ruby{見}{み}て、

\原本頁{103-4}%
『
アヽ、
%
\ruby{解}{わか}つたよ、
%
もう
\ruby{可}{い}いさ、
%
\ruby{君}{きみ}。
%
\ruby{金子}{か|ね}が
\ruby{先}{さき}に
\ruby{立}{た}つからと
\ruby{云}{い}ふのだらう。
%
\換字{志}て
\ruby{何}{ど}の
\ruby[||j>]{位}{くらゐ}
\ruby[||j>]{用}{ よう}
\ruby[||j>]{立}{ だ}てやうかエ。
』

\原本頁{103-6}%
と、
%
\ruby{輕々}{かろ|〴〵}と
\ruby{事}{こと}も
\ruby{無}{な}げに
\ruby{引取}{ひつ|と}つて
\ruby{云}{い}つて、
%
\ruby{云}{い}ひ
\ruby{{\換字{難}}}{にく}き
\ruby{口數}{くち|かず}を
\ruby{多}{おほ}くは
きかせぬ
\ruby[||j>]{同}{おもひ}
\ruby[||j>]{{\換字{情}}}{ やり}の
% \ruby{同{\換字{情}}}{おもひ|やり}の
\ruby{骨}{ほね}に
\ruby{徹}{てつ}するほど
\ruby{嬉}{うれ}し
\ruby{悲}{かな}しく、

\原本頁{103-8}%
『
\ruby{濟}{す}まないけれども
\ruby{一時}{いち|どき}で
\ruby{無}{な}くとも
\ruby{可}{い}いから
\ruby[||j>]{百}{ひやく}
\ruby[||j>]{圓}{ ゑん}ばかり、
% \ruby{百圓}{ひやく|ゑん}ばかり、
』

\原本頁{103-9}%
と、
%
\ruby{纔}{わづか}に
\ruby{口}{くち}を
\ruby{洩}{も}らせし
\ruby{限}{き}り、
%
あとは
\ruby{無言}{む|ごん}の
\ruby{頭}{かうべ}を
\ruby{低}{た}れて、
%
\ruby{深々}{ふか|〳〵}と
\ruby{頼}{たの}み
\ruby{入}{い}りたりしが、
%
\ruby{何時}{い|つ}より
\ruby{出}{い}で
\ruby{居}{ゐ}し
\ruby{涙}{なみだ}
なりけん、
%
\ruby{人}{ひと}の
\ruby{{\換字{情}}}{こゝろ}の
\ruby{凝}{こ}りて
\ruby{滴}{したゝ}る
\ruby{露}{つゆ}の
\ruby{眞玉}{ま|だま}は
ばらりと
\ruby{墜}{お}ちたり。

\原本頁{104-1}%
\ruby{誠}{まこと}
せめて
\ruby{人}{ひと}を
\ruby{頼}{たの}む
\ruby{心}{こゝろ}の
いぢらしくも、
%
\ruby{何時}{い|つ}の
\ruby{間}{ま}にか
\ruby{謹}{つゝし}みて
\ruby{律義}{りち|ぎ}に
\ruby{端座}{す|わ}り
\ruby{居}{ゐ}たる、
%
\ruby{水野}{みづ|の}が
\ruby{身}{み}を
\ruby{窄}{すぼ}めし
\ruby{姿}{すがた}の
\ruby{{\換字{寒}}}{さむ}げなるを
\ruby{見}{み}て、
%
\ruby{島木}{しま|き}は
\ruby{思}{おも}はず
\ruby{慨然}{がい|ぜん}として、

\原本頁{104-4}%
『
ナアニ
\ruby{可}{い}いさ。
%
\ruby{君}{きみ}、
%
それんばかりの
\ruby{事}{こと}を。
%
\ruby{宜}{よろ}しい、
%
\ruby{承知}{しよう|ち}した。
%
\ruby{今}{いま}
\ruby{直}{すぐ}
\ruby{献}{あ}げる。
』

\原本頁{104-6}%
と、
%
\ruby{確然}{しつ|かり}と
\ruby{明}{あき}らかに
\ruby{先}{ま}づ
\ruby{答}{こた}へつ、
%
\ruby{少時}{しば|し}
\ruby[<j||]{間}{あひだ}を% ルビ調整(原本通り)
\ruby{置}{お}きて、

\原本頁{104-7}%
『\換字{志}かし、
%
\ruby{君}{きみ}、
%
\ruby{僕}{ぼく}は
\ruby{何}{なに}も
\ruby{君}{きみ}に
\ruby{恨}{うら}みを
\ruby{云}{い}ふのでは
\ruby{無}{な}いが、
%
\ruby{何故}{な|ぜ}
\ruby{君}{きみ}は
\ruby{僕}{ぼく}に
\ruby{其}{そ}の
\ruby{友人}{いう|じん}の
\ruby{名}{な}を、
%
\ruby{岩崎}{いは|ざき}
\ruby{五十子}{い|そ|こ}と
いふものだとは
\ruby{云}{い}つて
\ruby{吳}{く}れぬ?。
%
イヤ、
%
\ruby{吃驚}{びつ|くり}しないでも
\ruby{宜}{い}い、
%
\ruby{意見}{い|けん}は
\ruby{云}{い}は
\ruby{無}{な}いが、
』

\原本頁{104-10}%
と、
%
\ruby{何事}{なに|ごと}をか
\ruby{徐}{しづか}に
\ruby{云}{い}ひ
\ruby{出}{だ}さんとすれば、
%
\ruby{水野}{みづ|の}が
\ruby{面}{おもて}は
たゞ
\ruby{火}{ひ}となつたり。
