\Entry{其十七}

% メモ 校正 2024-04-06
\原本頁{105-2}%
\ruby{自信}{じ|しん}は
\ruby{{\換字{強}}}{つよ}くとも、
%
\ruby{學問}{がく|もん}は
\ruby{博}{ひろ}くとも、
%
\ruby{氣}{き}の
\ruby{働}{はたら}きは
\ruby{八方}{はつ|ぱう}に
\ruby{{\換字{銳}}}{するど}くとも、
%
\ruby{未}{ま}だ
\ruby{世}{よ}に
\ruby{老}{お}いぬ
\ruby{心}{こゝろ}の
\ruby{柔輭}{やはら|か}に
\ruby{嫩}{わか}ければ、
%
\ruby{人}{ひと}には
\ruby{知}{し}らさず
\ruby{祕}{ひ}め
\ruby{置}{お}きたることを、
%
つけ〳〵と
\ruby{覿面}{てき|めん}に
\ruby{云}{い}ひ
\ruby{出}{いだ}されては、
%
\ruby{胸}{むね}の
\ruby{眞正中}{まつ|たゞ|なか}を
\換字{志}たゝかなる
\ruby{箭}{や}に、
%
\ruby{羽中}{は|なか}の
\ruby{{\換字{節}}}{ふし}せめて
\ruby{射{\換字{込}}}{い|こ}まれたる
\ruby{思}{おも}ひして、
%
ハツと
\ruby{驚}{おどろ}き
\ruby{惑}{まど}ひしが、
%
\ruby{元來}{も|と}
\ruby{底}{そこ}の
\ruby{{\換字{弱}}}{よわ}からぬ
\ruby{男}{をとこ}なり、
%
\ruby{忽}{たちま}ち
\ruby{我}{われ}に
\ruby{{\換字{返}}}{かへ}つて
\ruby{惡}{わる}びれず、
%
\ruby{靜}{しづか}かに
\ruby{我}{わ}が
\ruby{腔内}{む|ね}の
\ruby{血}{ち}の
\ruby{跳}{をど}りの
\ruby{鎭}{しづ}まるを
\ruby{待}{ま}ちながら、
%
\ruby{身動}{み|うご}きだに
せずして
\ruby{大人}{おと|な}しく、
%
\ruby{島木}{しま|き}の
いふところを
\ruby{聞}{き}かんと
\ruby{仕}{し}たり。

\原本頁{105-10}%
\ruby{島木}{しま|き}は
\ruby{人}{ひと}の
\ruby{{\換字{情}}}{こゝろ}の
\ruby{流}{なが}れの
\ruby{瀬}{せ}に、
%
\ruby{慣}{な}れきつたる
\ruby{鵜}{う}の
\ruby{目}{め}の
\ruby{働}{はたら}き
\ruby{敏捷}{す|ばや}く、
%
\原本頁{106-1}%
\ruby{日}{ひ}の
\ruby{光}{ひかり}の
\ruby{明}{あき}らかなるに
\ruby{我}{わ}が
\ruby{影}{かげ}を
\ruby{怯}{お}づる
\ruby{{\換字{若}}鮎}{わか|あゆ}の
\ruby{振舞}{ふる|まひ}の、
%
\ruby{優}{やさ}しくも
\換字{志}ほらしき
\ruby{水野}{みづ|の}が
\ruby{樣子}{やう|す}を
\ruby{見}{み}て
\ruby{取}{と}つて、
%
\ruby{曾}{かつ}て
\ruby{吉右衛門}{き|ち|ゑ|もん}より
\ruby{聞}{き}きしと、
%
\ruby{今}{いま}
\ruby{直接}{ぢ|か}に
\ruby{聞}{き}きしとの
\ruby{二}{ふた}つの
\ruby{談話}{はな|し}に
\ruby{照}{て}らし
\ruby{合}{あ}はせて、
%
\ruby{大槪}{おほ|よそ}の
\ruby{事}{こと}は
\ruby{曉}{さと}り
\ruby{盡}{つく}しつ、
%
\ruby{今}{いま}
\ruby{{\換字{更}}}{さら}に
また
\ruby{油然}{ゆう|ぜん}として
\ruby{愛憐}{いと|ほし}む
\ruby{心}{こゝろ}の
\ruby{起}{おこ}るに
\ruby{堪}{た}へぬが
\ruby{如}{ごと}く、
%
\ruby{言葉}{こと|ば}づかひも
\ruby{砕}{くだ}けて
\ruby{露}{つゆ}
\ruby{隔氣}{へだ|て}なく、
%
いと
\ruby{親}{した}しくも
\ruby{說}{と}き
\ruby{出}{いだ}したり。

\原本頁{106-7}%
『% 原本ではこの島木の語りの終わりである「』」は欠落している
ねえ
\ruby{君}{きみ}、
%
\ruby{可厭}{い|や}なものは、
%
\ruby{無心}{む|しん}を
\ruby{聽}{き}いた
\ruby{後}{あと}で
\ruby{意見}{い|けん}
\ruby{云}{い}ふ
\ruby{奴}{やつ}だと、
%
\ruby{{\換字{古}}}{むかし}から
\ruby{云}{い}つてあるぢやあ
\ruby{無}{な}いか。
%
ハヽヽ
まさかに
\ruby{僕}{ぼく}だつて
\ruby[||j>]{其}{その}
\ruby[||j>]{位}{くらゐ}な
% \ruby{其位}{その|くらゐ}な
\ruby{事}{こと}は
\ruby{知}{し}つて
\ruby{居}{ゐ}るから、
%
\ruby{此處}{こ|ゝ}で
\ruby{下手}{へ|た}な
\ruby{叔{\換字{父}}}{を|ぢ}さんの
\ruby{役}{やく}を
\ruby{{\換字{勤}}}{つと}めて、
%
\ruby{何}{なん}の
\ruby{彼}{か}のと
\ruby{{\換字{難}}}{むづ}かしい
\ruby{事}{こと}を
\ruby{云}{い}ふなあ
\ruby{自{\換字{分}}}{じ|ぶん}で
\ruby{願}{ねが}ひ
\ruby{下}{さ}げるし、
%
\ruby{{\換字{又}}}{また}
\ruby{理屈}{り|くつ}なんぞといふ
\ruby{野暮}{や|ぼ}なものを、
%
\ruby{餘}{あま}り
\ruby{有}{あ}り
\ruby{{\換字{難}}}{がた}いと
\ruby{思}{おも}つてゐる
\ruby{僕}{ぼく}でも
\ruby{無}{な}いから、
%
\原本頁{107-1}%
\ruby{君}{きみ}が
\ruby{何樣}{ど|う}
\ruby{仕}{し}やうと、
%
\ruby{斯樣}{か|う}
\ruby{仕}{し}やうと、
%
それを
\ruby{兎}{と}や
\ruby{角}{かく}いふ
\ruby{僕}{ぼく}ぢやあ
\ruby{無}{な}い。
%
\ruby{惡}{わる}い
\ruby{事}{こと}さへ
\ruby{仕無}{し|な}けりやあ、
%
\ruby{好}{すき}きな
\ruby{事}{こと}を
\ruby{仕}{し}て
\ruby{面白}{おも|しろ}く
\ruby{世}{よ}を
\ruby{渡}{わた}るのが、
%
\ruby{可}{い}いぢやあ
\ruby{無}{な}いかと
いふのが
\ruby{僕}{ぼく}の
\ruby{宗旨}{しゆう|し}なのは、
%
\ruby{君}{きみ}も
\ruby{知}{し}つて
\ruby{居}{ゐ}る
\ruby{{\換字{通}}}{とほ}りの
\ruby{事}{こと}だ。
%
だから
\ruby{意見}{い|けん}と
\ruby{思}{おも}つて
\ruby{聞}{き}いて
\ruby{吳}{く}れちやあ
\ruby{困}{こま}るが、
%
たつた
\ruby{一}{ひと}つ
\ruby{君}{きみ}に
\ruby{聞}{き}いて
\ruby{置}{お}いて
\ruby{貰}{もら}ひたい
\ruby{事}{こと}がある。
%
\ruby{下}{くだ}らない
\ruby{事}{こと}では
\ruby{有}{あ}らうが、
%
\ruby{聞}{き}いて
\ruby{吳}{く}れたまへ。
%
\ruby{僕}{ぼく}は
\ruby{隨{\換字{分}}}{ずゐ|ぶん}
\ruby{今}{いま}までの
\ruby{品行}{み|もち}が、
%
\ruby{疵瑕}{き|ず}だらけの
\ruby{大馬鹿}{おほ|ば|か}な
\ruby{奴}{やつ}なんだから、
%
\ruby{當世}{たう|せい}で
よく
\ruby{云}{い}ふ
\ruby{神聖}{しん|せい}な
\ruby{戀愛}{れん|あい}、
{---}{---}
そんな
\ruby[||j>]{上}{じやう}
\ruby[||j>]{品}{ ひん}なものあ
% \ruby{上品}{じやう|ひん}なものあ
\ruby{知}{し}らないが、
%
\ruby{戀愛}{れん|あい}も
\ruby{惚}{ほ}れた
はれたも
\ruby{同}{おな}じ
\ruby{事}{こと}として、
%
マア
\ruby{僕}{ぼく}だけで
\ruby{云}{い}つて
\ruby{見}{み}りやあ、
%
\ruby{戀愛}{れん|あい}は
\ruby{可怖}{こ|は}いものぢやあ
\ruby{無}{な}いが、
%
\ruby{戀愛}{れん|あい}に
\ruby{隨}{つ}いて
\ruby{來}{く}る
\ruby{隨{\換字{伴}}者}{お|と|も}は
\ruby{怖}{こは}い、
%
と
つく〴〵
\ruby{身}{み}に
\ruby{染}{し}みて
\ruby{覺}{おぼ}えて
\ruby{居}{ゐ}るんだ。
%
そこで
\ruby{君}{きみ}に
\ruby{其}{そ}の
\ruby{隨{\換字{伴}}者}{お|と|も}だけにやあ
\ruby{戒愼}{よう|じん}して
\ruby{貰}{もら}ひたいと
\ruby{思}{おも}ふ。
%
\原本頁{108-1}%
\ruby{云}{い}つて
\ruby{置}{お}きたいと
\ruby{云}{い}ふのは
\ruby{只}{たゞ}これ
\ruby{一}{ひと}つだ。
%
いゝカエ、
%
\ruby{惚}{ほ}れた
はれたの
\ruby{其}{そ}の
\ruby{{\換字{迷}}}{まよ}ひは、
%
\ruby{些}{ちつと}も
\ruby{可怖}{こ|は}い
\ruby{事}{こと}は
\ruby{無}{な}いが、
%
それに
\ruby{付}{つ}いて
\ruby{來}{く}る
\ruby{隨{\換字{伴}}者}{お|と|も}は
\ruby{怖}{こは}い
\ruby{危險}{き|けん}ものだといふのだよ。
』% 原本では島木の語りの終わりである「』」は欠落している
