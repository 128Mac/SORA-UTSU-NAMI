\Entry{其十七}

% メモ 校正 2024-04-06 2024-05-25 2024-06-17
\原本頁{105-2}%
\ruby[g]{自信}{じ しん}は
\ruby{{\換字{強}}}{つよ}くとも、
%
\ruby[g]{學問}{がくもん}は
\ruby{博}{ひろ}くとも、
%
\ruby{氣}{き}の
\ruby{働}{はたら}きは
\ruby[g]{八方}{はつぱう}に
\ruby{{\換字{銳}}}{するど}くとも
\改行% 校正作業の簡略化のため
、
%
\原本頁{105-3}\改行%
\ruby{未}{ま}だ
\ruby{世}{よ}に
\ruby{老}{お}いぬ
\ruby{心}{こゝろ}の
\ruby[g]{柔輭}{やはらか}に
\ruby{嫩}{わか}ければ、
%
\ruby{人}{ひと}には
\ruby{知}{し}らさず
\ruby{祕}{ひ}め
\ruby{置}{お}きたることを、
%
つけ〳〵と
\ruby[g]{覿面}{てきめん}に
\ruby{云}{い}ひ
\ruby{出}{いだ}されては、
%
\ruby{胸}{むね}の
\ruby{眞正中}{まつ|たゞ|なか}を
\原本頁{105-5}\改行%
\換字{志}たゝかなる
\ruby{箭}{や}に、
%
\ruby[g]{羽中}{は なか}の
\ruby{{\換字{節}}}{ふし}せめて
\ruby[g]{射{\換字{込}}}{い こ }まれたる
\ruby{思}{おも}ひして、
%
ハツと
\ruby{驚}{おどろ}き
\ruby{惑}{まど}ひしが、
%
\ruby[g]{元來}{も と }
\ruby{底}{そこ}の
\ruby{{\換字{弱}}}{よわ}からぬ
\ruby{男}{をとこ}なり、
%
\ruby{忽}{たちま}ち
\ruby{我}{われ}に
\ruby{{\換字{返}}}{かへ}つて
\原本頁{105-7}\改行%
\ruby{惡}{わる}びれず、
%
\ruby{靜}{しづか}に
\ruby{我}{わ}が
\ruby[g]{腔内}{む ね }の
\ruby{血}{ち}の
\ruby{跳}{をど}りの
\ruby{鎭}{しづ}まるを
\ruby{待}{ま}ちながら、
%
\ruby{身}{み}
\ruby{動}{うご}きだに
せずして
\ruby[g]{大人}{おとな }しく、
%
\ruby[g]{島木}{しまき }の
いふところを
\ruby{聞}{き}かんと
\ruby{仕}{し}た
\改行% 校正作業の簡略化のため
り。

\原本頁{105-10}%
\ruby[g]{島木}{しまき }は
\ruby{人}{ひと}の
\ruby{{\換字{情}}}{こゝろ}の
\ruby{流}{なが}れの
\ruby{瀬}{せ}に、
%
\ruby{慣}{な}れきつたる
\ruby{鵜}{う}の
\ruby{目}{め}の
\ruby{働}{はたら}き
\ruby[g]{敏捷}{す ばや}く
\改行% 校正作業の簡略化のため
、
%
\原本頁{106-1}\改行%
\ruby{日}{ひ}の
\ruby{光}{ひかり}の
\ruby{明}{あき}らかなるに
\ruby{我}{わ}が
\ruby{影}{かげ}を
\ruby{怯}{お}づる
\ruby[g]{{\換字{若}}鮎}{わかあゆ}の
\ruby[g]{振舞}{ふるまひ}の、
%
\ruby{優}{やさ}しくも
\原本頁{106-2}\改行%
\換字{志}ほらしき
\ruby[g]{水野}{みづの }が
\ruby[g]{樣子}{やうす }を
\ruby{見}{み}て
\ruby{取}{と}つて、
%
\ruby{曾}{かつ}て
\ruby{吉右衛門}{きち||ゑ|もん}より
\ruby{聞}{き}きしと、
%
\ruby{今}{いま}
\ruby[g]{直接}{ぢ か }に
\ruby{聞}{き}きしとの
\ruby{二}{ふた}つの
\ruby[g]{談話}{はなし }に
\ruby{照}{て}らし
\ruby{合}{あ}はせて、
%
\ruby[g]{大槪}{おほよそ}の
\原本頁{106-4}\改行%
\ruby{事}{こと}は
\ruby{曉}{さと}り
\ruby{盡}{つく}しつ、
%
\ruby{今}{いま}
\ruby{{\換字{更}}}{さら}に
また
\ruby[g]{油然}{ゆうぜん}として
\ruby[g]{愛憐}{いとほし}む
\ruby{心}{こゝろ}の
\ruby{起}{おこ}るに
\ruby{堪}{た}へぬが
\ruby{如}{ごと}く、
%
\ruby[g]{言葉}{ことば }づかひも
\ruby{砕}{くだ}けて
\ruby{露}{つゆ}
\ruby[g]{隔氣}{へだて }なく、
%
いと
\ruby{親}{した}しくも
\ruby{說}{と}き
\原本頁{106-6}\改行%
\ruby{出}{いだ}したり。

\原本頁{106-7}%
『% 原本ではこの島木の語りの終わりである「』」は欠落している
ねえ
\ruby{君}{きみ}、
%
\ruby[g]{可厭}{い や }なものは、
%
\ruby[g]{無心}{む しん}を
\ruby{聽}{き}いた
\ruby{後}{あと}で
\ruby[g]{意見}{い けん}
\ruby{云}{い}ふ
\ruby{奴}{やつ}だと、
%
\ruby{{\換字{古}}}{むかし}から
\ruby{云}{い}つてあるぢや
\ruby{無}{な}いか。
%
ハヽヽ
まさかに
\ruby{僕}{ぼく}だつて
\ruby[<j||]{其}{その }% 行末行頭の境界付近なので特例処置を施す
\ruby[<j||]{位}{くらゐ}な
% \ruby{其位}{その|くらゐ}な
\原本頁{106-9}\改行%
\ruby{事}{こと}は
\ruby{知}{し}つて
\ruby{居}{ゐ}るから、
%
\ruby[g]{此處}{こ ゝ }で
\ruby[g]{下手}{へ た }な
\ruby[g]{叔{\換字{父}}}{を ぢ }さんの
\ruby{役}{やく}を
\ruby{{\換字{勤}}}{つと}めて、
%
\ruby{何}{なん}の
\ruby{彼}{か}のと
\ruby{{\換字{難}}}{むづ}かしい
\ruby{事}{こと}を
\ruby{云}{い}ふなあ
\ruby[g]{自{\換字{分}}}{じ ぶん}で
\ruby{願}{ねが}ひ
\ruby{下}{さ}げるし、
%
\ruby{{\換字{又}}}{また}
\ruby[g]{理屈}{り くつ}な
\原本頁{106-11}\改行%
んぞといふ
\ruby[g]{野暮}{や ぼ }なものを、
%
\ruby{餘}{あま}り
\ruby{有}{あ}り
\ruby{{\換字{難}}}{がた}いと
\ruby{思}{おも}つてゐる
\ruby{僕}{ぼく}でも
\ruby{無}{な}いから、
%
\ruby{君}{きみ}が
\ruby[g]{何樣}{ど う }
\ruby{仕}{し}やうと、
%
\ruby[g]{斯樣}{か う }
\ruby{仕}{し}やうと、
%
それを
\ruby{兎}{と}や
\ruby{角}{かく}いふ
\原本頁{107-2}\改行%
\ruby{僕}{ぼく}ぢやあ
\ruby{無}{な}い。
%
\ruby{惡}{わる}い
\ruby{事}{こと}さへ
\ruby[g]{仕無}{し な }けりやあ、
%
\ruby{好}{す}きな
\ruby{事}{こと}を
\ruby{仕}{し}て
\ruby[g]{面白}{おもしろ}
\原本頁{107-3}\改行%
く
\ruby{世}{よ}を
\ruby{渡}{わた}るのが、
%
\ruby{可}{い}いぢやあ
\ruby{無}{な}いかと
いふのが
\ruby{僕}{ぼく}の
\ruby[g]{宗旨}{しゆうし}なのは
\改行% 校正作業の簡略化のため
、
%
\原本頁{107-4}\改行%
\ruby{君}{きみ}も
\ruby{知}{し}つて
\ruby{居}{ゐ}る
\ruby{{\換字{通}}}{とほ}りの
\ruby{事}{こと}だ。
%
だから
\ruby[g]{意見}{い けん}と
\ruby{思}{おも}つて
\ruby{聞}{き}いて
\ruby{吳}{く}れちやあ
\ruby{困}{こま}るが、
%
たつた
\ruby{一}{ひと}つ
\ruby{君}{きみ}に
\ruby{聞}{き}いて
\ruby{置}{お}いて
\ruby{貰}{もら}ひたい
\ruby{事}{こと}がある。
%
\原本頁{107-6}\改行%
\ruby{下}{くだ}らない
\ruby{事}{こと}では
\ruby{有}{あ}らうが、
%
\ruby{聞}{き}いて
\ruby{吳}{く}れたまへ。
%
\ruby{僕}{ぼく}は
\ruby[g]{隨{\換字{分}}}{ずゐぶん}
\ruby{今}{いま}までの
\ruby[g]{品行}{み もち}が、
%
\ruby[g]{疵瑕}{き ず }だらけの
\ruby{大馬鹿}{おほ|ば|か}な
\ruby{奴}{やつ}なんだから、
%
\ruby[g]{當世}{たうせい}で
よく
\ruby{云}{い}ふ
\ruby[g]{神聖}{しんせい}な
\ruby[g]{戀愛}{れんあい}、%
{---}{---}%
そんな
\ruby[||j>]{上}{じやう}
\ruby[||j>]{品}{ ひん}なものあ
% \ruby{上品}{じやう|ひん}なものあ
\ruby{知}{し}らないが、
%
\ruby[g]{戀愛}{れんあい}も
\ruby{惚}{ほ}
\原本頁{107-9}\改行%
れた
はれたも
\ruby{同}{おな}じ
\ruby{事}{こと}として、
%
マア
\ruby{僕}{ぼく}だけで
\ruby{云}{い}つて
\ruby{見}{み}りやあ、
%
\ruby[g]{戀愛}{れんあい}は
\ruby[g]{可怖}{こ は }いものぢやあ
\ruby{無}{な}いが、
%
\ruby[g]{戀愛}{れんあい}に
\ruby{隨}{つ}いて
\ruby{來}{く}る
\ruby{隨{\換字{伴}}者}{お|と|も}は
\ruby{怖}{こは}い
\改行% 校正作業の簡略化のため
、
%
\原本頁{107-11}\改行%
と
つく〴〵
\ruby{身}{み}に
\ruby{染}{し}みて
\ruby{覺}{おぼ}えて
\ruby{居}{ゐ}るんだ。
%
そこで
\ruby{君}{きみ}に
\ruby{其}{そ}の
\ruby{隨{\換字{伴}}者}{お|と|も}
\原本頁{108-1}\改行%
だけにやあ
\ruby[g]{戒愼}{ようじん}して
\ruby{貰}{もら}ひたいと
\ruby{思}{おも}ふ。
%
\ruby{云}{い}つて
\ruby{置}{お}きたいと
\ruby{云}{い}ふのは
\ruby{只}{たゞ}これ
\ruby{一}{ひと}つだ。
%
いゝカエ、
%
\ruby{惚}{ほ}れた
はれたの
\ruby{其}{そ}の
\ruby{{\換字{迷}}}{まよ}ひは、
%
\ruby{些}{ちつと}も
\原本頁{108-3}\改行%
\ruby[g]{可怖}{こ は }い
\ruby{事}{こと}は
\ruby{無}{な}いが、
%
それに
\ruby{付}{つ}いて
\ruby{來}{く}る
\ruby{隨{\換字{伴}}者}{お|と|も}は
\ruby{怖}{こは}い
\ruby[g]{危險}{あぶない}ものだといふのだよ。
』% 原本では島木の語りの終わりである「』」は欠落している
