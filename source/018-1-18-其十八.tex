\Entry{其十八}

% メモ 校正 2024-04-06
\原本頁{108-6}%
『
\ruby{僕}{ぼく}は
\ruby{元}{もと}から
\ruby{學問}{がく|もん}は
\ruby{{\換字{嫌}}}{きら}ひだし、
%
\ruby{身}{み}に
\ruby{浸}{し}みて
\ruby{書}{ほん}を
\ruby{讀}{よん}んだ
\ruby{事}{こと}も
\ruby{無}{な}いから、
%
どうせ
\ruby{僕}{ぼく}の
\ruby{云}{い}ふ
\ruby{事}{こと}なぞは
\ruby{下}{くだ}ら
\ruby{無}{な}からうが、
%
まんざら
\ruby[g]{正中}{つぼ}に
\ruby{外}{はづ}れたことも
\ruby{云}{い}は
\ruby{無}{な}いつもりだ。
%
かういふ
\ruby{理屈}{り|くつ}だ、
%
\ruby{聞}{き}いて
\ruby{吳}{く}れたまへ。
%
\ruby{僕}{ぼく}に
\ruby{云}{い}はせりやあ
\ruby{色戀}{いろ|こひ}といふ
\ruby{奴}{やつ}あ、
%
\ruby{人間}{にん|げん}が
\ruby{一人並}{いち|にん|なみ}に
\原本頁{108-10}\改行%
\ruby{成熟}{でき|あが}ると、
%
\ruby{一度}{いち|ど}は
\ruby{屹度}{きつ|と}
\ruby{發}{はつ}しる
\ruby{熱病}{ね|つ}なので、
%
\ruby{身體}{から|だ}の
\ruby{中}{なか}から
\ruby{自然}{ひと|りで}に
\原本頁{109-1}\改行%
\ruby{湧}{わ}く
\ruby{奴}{やつ}だ、
%
\ruby{各自}{めい|〳〵}の
\ruby{料簡}{れう|けん}から
\ruby{出}{で}て
\ruby{來}{く}るんぢやあ
\ruby{無}{な}い。
%
そりやあ
\ruby{其}{そ}の
\ruby{當人}{たう|にん}から
\ruby{云}{い}つて
\ruby{見}{み}りやあ、
%
\ruby{彼處}{あす|こ}が% 原本では「あ(す)こ」に見えるので
\ruby{好}{い}いとか、
%
\ruby{此處}{こ|ゝ}が
\ruby{好}{い}いとか、
%
それ〴〵に
\ruby{理由}{わ|け}が
\ruby{有}{あ}つて
\ruby{惚}{ほ}れるのでも
\ruby{有}{あ}らうが、
%
ナアニ
\ruby{年齡}{と|し}が
\原本頁{109-4}\改行%
\ruby{爲}{さ}せるんだよ、
%
\ruby{年齡}{と|し}が
\ruby{爲}{さ}せるんだよ。
%
\ruby{彼}{あ}の
\ruby{女}{をんな}あ
\ruby{好}{い}いから
サア
\ruby{惚}{ほ}れて
\ruby{{\換字{遣}}}{や}らうと、
%
\ruby{{\換字{分}}別}{ふん|べつ}を
つけてから
\ruby{惚}{ほ}れる
\ruby{奴}{やつ}は
\ruby{無}{な}い。
%
\ruby{誰}{だれ}の
\ruby{戀路}{こひ|ぢ}も
\原本頁{109-6}\改行%
\ruby{同}{おんな}じ
\ruby{事}{こと}で、
%
\ruby{其}{そ}の
\ruby{眞實}{ほん|たう}の
ところを
\ruby{云}{い}やあ、
%
\ruby{自{\換字{分}}}{じ|ぶん}にも
\ruby{理由}{わ|け}は
\ruby{{\換字{分}}}{わか}らないけれど、
%
\ruby{何}{なん}だか
\ruby{知}{し}ら
\ruby{無}{な}いが
\ruby{自然}{ひと|りで}に
\ruby{好}{す}く、
%
それが
\ruby{抑々}{そも|〳〵}の
\ruby[g]{發端}{はじまり}で、
%
\原本頁{109-8}\改行%
\ruby{其}{そ}の
\ruby{人}{ひと}の
\ruby{笑顏}{ゑ|がほ}なんぞが
\ruby{何時}{い|つ}の
\ruby{間}{ま}にか
\ruby{眼}{め}に
\ruby{染}{し}み
\ruby{付}{つ}いて
\ruby{{\換字{遺}}}{のこ}つたり、
%
\原本頁{109-9}\改行%
\ruby{物}{もの}を
\ruby{云}{い}つた
\ruby{聲}{こゑ}の
\ruby{色}{いろ}が
\ruby{耳}{みゝ}に
\ruby{{\換字{遺}}}{のこ}つたりして、
%
\ruby{{\換字{終}}}{しまひ}には
すつかりと
\ruby{其}{その}
\ruby{人}{ひと}が
\ruby{自{\換字{分}}}{じ|ぶん}の
\ruby{胸}{むね}の
\ruby{中}{うち}に
\ruby{在}{あ}るやうになる、
%
サア
\ruby{忘}{わす}れやうと
\ruby{思}{おも}つても
\ruby{忘}{わす}れられない、
%
\ruby[g]{始{\換字{終}}}{しじゆう}
\ruby{其}{その}
\ruby{人}{ひと}の
\ruby{傍}{そば}に
\ruby{居}{ゐ}て
\ruby{見}{み}たくなる、
%
\ruby{離}{はな}れて
\ruby{居}{ゐ}ちやあ
\原本頁{110-1}\改行%
\ruby{物悲}{もの|がな}しくつて、
%
\ruby{何}{なん}と
\ruby{無}{な}く
\ruby{氣}{き}が
\ruby{濟}{す}まないやうな
\ruby{心持}{こゝろ|もち}がする、
%
\ruby{自{\換字{分}}}{じ|ぶん}が
\ruby{其}{その}
\ruby{人}{ひと}を
\ruby{思}{おも}ふやうに、
%
\ruby{其}{その}
\ruby{人}{ひと}にも
\ruby{自{\換字{分}}}{じ|ぶん}を
\ruby{思}{おも}つて
\ruby{貰}{もら}ひたくなる、
%
それから
\ruby{段々}{だん|〴〵}と
\ruby{泣}{な}いたり
\ruby{笑}{わら}つたりが
\ruby{始}{はじ}まる、
%
まあ
\ruby{斯樣}{か|う}
\ruby{云}{い}つた
\ruby{順立}{じゆん|だて}ぢやあ
\ruby{無}{な}いか。
%
\換字{志}て
\ruby{見}{み}りやあ
\ruby{自然}{ひと|りで}に
\ruby{好}{す}くといふのが
\ruby{戀}{こひ}の
\ruby{水上}{みな|かみ}だが、
%
\ruby[g]{自然}{ひとりで}の
\ruby{好惡}{すき|きらひ}だもの、
%
\ruby{理屈}{り|くつ}は
\ruby{有}{あ}りや
\ruby{仕無}{し|な}い、
%
みんな
\ruby{年齡}{と|し}が
\ruby{爲}{さ}せるんだ、
%
\ruby{年齡}{と|し}が
\ruby{爲}{さ}せるんだ。
%
\ruby[g]{懷姙者}{みもち}は
\ruby{酸}{す}いものを
\ruby[g]{自然}{ひとりで}に
\ruby{好}{す}く、
%
\原本頁{110-7}\改行%
\ruby{溜飮持}{りう|いん|もち}は
\ruby[g]{香物}{かう〳〵}で
\ruby[g]{茶漬飮}{ちやづけ}を
\ruby[g]{自然}{ひとりで}に
\ruby{好}{す}く、
%
\ruby{其}{そ}の
\ruby[g]{自然}{ひとりで}に
\ruby{好}{す}くのは
\ruby{誰}{たれ}がさせる?、
%
\ruby{惡阻}{つわ|り}が
\ruby{爲}{さ}せるんだ、
%
\ruby{溜飮}{りう|いん}が
\ruby{爲}{さ}せるんだ。
%
\ruby{戀路}{こひ|ぢ}の
\ruby{{\換字{迷}}惑}{まよ|ひ}は
\ruby{年齡}{と|し}が
\ruby{爲}{さ}せるんだ。
%
\ruby{男兒}{をと|こ}が
\ruby{男兒}{をと|こ}づくる
\ruby{頃}{ころ}にやあ
\ruby{髭鬚}{ひ|げ}が
\ruby{生}{は}えて
\原本頁{110-10}\改行%
\ruby{來}{く}る、
%
\ruby{髭鬚}{ひ|げ}の
\ruby{生}{は}えるのは
\ruby{年齡}{と|し}が
\ruby{爲}{さ}せるんだもの、
%
それに
\ruby{善}{い}いも
\原本頁{110-11}\改行%
\ruby{惡}{わる}いも
\ruby{有}{あ}りやうは
\ruby{無}{な}い、
%
\ruby{口}{くち}の
\ruby{周圍}{まは|り}に
\ruby{出}{で}て
\ruby{來}{く}る
\ruby{髭鬚}{ひ|げ}も、
%
\ruby{心}{こゝろ}の
\ruby{上}{うへ}に
\原本頁{111-1}\改行%
\ruby{萌}{めぐ}む
\ruby{戀}{こひ}も、
%
\原本頁{111-1}%
\ruby{年端}{と|し}が
\ruby{爲}{さ}せるに
\ruby[g]{差異}{ちがひ}は
\ruby{無}{な}い、
%
\ruby{丁度}{ちやう|ど}
\ruby[|j>]{同}{おんな}じ
\ruby{事}{こと}だもの、
%
ナニ
\ruby{戀愛}{こ|ひ}を
\ruby{善}{い}いとも
\ruby{惡}{わる}いとも
\ruby{云}{い}はう
\ruby{譯}{わけ}は
\ruby{無}{な}い。
%
たゞ
\ruby{年齡}{と|し}が
\ruby{爲}{さ}せる
\原本頁{111-3}\改行%
\ruby[g]{熱病}{ねつ}を
すらりと
\ruby{濟}{すま}せて
\ruby[g]{仕無}{しま}へば、
%
\ruby{疱瘡}{はう|さう}や
\ruby{痲疹}{はし|か}が
\ruby{濟}{す}んだと
\ruby{同}{おな}じに、
%
\原本頁{111-4}\改行%
つまり
\ruby{芽出度}{め|で|たい}と
\ruby{云}{い}へば
\ruby{云}{い}へるので、
%
\ruby{戀}{こひ}は
\ruby{怖}{おそ}ろしいものでも
\ruby{何}{なん}でも
\ruby{無}{な}い。
%
\ruby{併}{しか}し
\ruby{{\換字{又}}}{また}、
%
\ruby{君}{きみ}は
\ruby{學問}{がく|もん}もあり
\ruby{思慮}{し|りよ}もあるから、
%
\ruby{萬々}{ばん|〳〵}
\ruby{承知}{しよう|ち}
\ruby{仕}{し}て
\ruby{居}{ゐ}やうが、
%
お
\ruby{互}{たがひ}に
\ruby{男兒}{をと|こ}といふ
\ruby{奴}{やつ}は、
%
\ruby{戀愛}{こ|ひ}の
\ruby[g]{奴隷}{けらい}に
\ruby{生}{うま}れて
\ruby{居}{ゐ}るものでも
\ruby{何}{なん}でも
\ruby{無}{な}い、
%
それ〴〵
\ruby{男子}{をと|こ}
\ruby{一匹}{いつ|ぴき}
\ruby{{\換字{前}}}{まへ}の
\ruby{目的}{もく|てき}のために
\ruby{意氣地}{い|き|ぢ}を
\ruby{磨}{みが}いて
\ruby[<j>]{一生}{いつ|しやう}を
\ruby{働}{はたら}いて
\ruby{行}{ゆ}かうといふ
\ruby{身}{み}、
%
\ruby{戀}{こひ}に
\ruby{捲}{ま}き
\ruby{倒}{たふ}されちやあ
ならねえ
\ruby{身體}{から|だ}だ、
%
\ruby{其}{そ}の
\ruby{熱病}{ね|つ}に
\ruby{身體}{から|だ}を
\ruby{{\換字{遣}}}{や}る
\ruby{譯}{わけ}にやあ
いかねえ
\ruby{約束}{やく|そく}がある。
%
\ruby{病}{やまひ}にも
\ruby{輕}{かる}い
\ruby{重}{おも}いは
あり、
%
\ruby{戀}{こひ}にも
\ruby{深}{ふか}い
\ruby{淺}{あさ}いは
\ruby{有}{あ}らうが、
%
\原本頁{111-11}\改行%
\ruby{如何}{い|か}に
\ruby{戀}{こひ}に
\ruby{惱}{なや}んでも
\ruby{苦}{くる}しんでも、
%
\ruby{吐}{つ}く
\ruby{息}{いき}が
\ruby{火}{ひ}に
なつて
\ruby{燃}{も}えるほどに
\ruby{狂}{くる}はうとも、
%
\ruby{戀}{こひ}に
\ruby{負}{ま}けて
\ruby{死}{し}んぢやあ
\ruby{男子}{をと|こ}たる
\ruby{身}{み}の、
%
\ruby{眼}{め}が
\ruby{瞑}{ふさ}げねえ
\ruby{筈}{はず}だ。
%
イヤ
\ruby{瞑}{ふさ}げねえ、
%
どうしても
\ruby{死}{しに}きれねえ、
%
\ruby{死}{し}ね
\ruby{無}{ね}え
\原本頁{112-3}\改行%
\ruby{筈}{はず}だ。
%
\ruby{乃公}{お|ら}あ
\ruby{死}{し}な
\ruby{無}{ね}え、
%
\ruby{死}{し}にも
\ruby{仕無}{し|ね}えが、
%
\ruby{汝}{おめへ}も
\ruby{死}{し}ねめえ、
%
\ruby{死}{し}にも
すめえナ。
%
\ruby{知}{し}れ
\ruby{切}{き}つた
\ruby{事}{こと}だが、
%
ナア
\ruby[g]{水野}{みづの}、
%
お
\ruby{互}{たがひ}に
\ruby{幾干}{いく|そ}
\ruby{{\換字{若}}干}{ばく|そ}の
\原本頁{112-5}\改行%
\ruby{苦勞}{く|らう}を
\ruby{仕}{し}て、
%
\ruby{今日}{け|ふ}まで
\ruby{{\換字{遣}}}{や}つて
\ruby{來}{き}たなあ
\ruby{何}{なん}の
\ruby{爲}{ため}だ?。
%
\ruby[<g>]{志}{こゝろざし}こそ
\ruby{異}{ちが}ふけれど、
%
\ruby{男兒}{をと|こ}と
\ruby{生}{うま}れた
\ruby{生}{うま}れ
\ruby{甲{\換字{斐}}}{が|ひ}にやあ、
%
\ruby{各自}{めい|〳〵}の
\ruby{念願}{おも|ひ}を
\ruby{{\換字{遂}}}{と}げやうと、
%
それ
ばつかりの
\ruby{爲}{ため}ぢやあ
\ruby{無}{ね}えか。
%
\ruby{特}{こと}さら
\ruby{汝}{おめへ}は
\ruby{乃公}{お|れ}から
\ruby{云}{い}やあ、
%
マア
\ruby{慾}{よく}の
\ruby{無}{な}さすぎる
\ruby{偏人}{へん|じん}で、
%
\ruby{取}{と}れる
\ruby{錢}{ぜに}も
\ruby{取}{と}らず
\ruby{出世}{しゆつ|せ}も
\ruby{望}{のぞ}まず、
%
\ruby{大根}{だい|こん}
\ruby{人參}{にん|じん}の
\ruby{尻尾}{しつ|ぽ}を
\ruby{咬}{かじ}つて、
%
それで
\ruby{濟}{す}まして
\ruby{居}{ゐ}るやうな
\ruby{{\換字{遣}}}{や}り
\原本頁{112-10}\改行%
\ruby{方}{かた}。
%
アヽ
\ruby{世}{よ}の
\ruby{中}{なか}は
いろ〳〵のもんだ、
%
\ruby[g]{水野}{みづの}だつて
\ruby{不味}{ま|づ}いものあ
\原本頁{112-11}\改行%
\ruby{不味}{ま|づ}く、
%
\ruby{美味}{う|ま}いものは
\ruby{旨}{うま}からうが、
%
\ruby{其}{それ}にも
\ruby{此}{これ}にも
\ruby{頓着無}{とん|ぢやく|な}く、
%
\ruby{{\換字{若}}}{わか}い
\ruby{身}{み}そらで
\ruby{色氣}{いろ|け}も
\ruby{無}{な}く、
%
\ruby{下手}{へ|た}な
\ruby{律僧}{りつ|そう}は
\ruby{及}{およ}ばぬ
\ruby{身持}{み|もち}で、
%
たゞ
\ruby{學問}{がく|もん}に
\ruby{凝}{こ}つて
\ruby{居}{ゐ}る、
%
アヽ
\ruby{聖人}{せい|じん}と
\ruby{云}{い}ふなあ
\ruby{彼樣}{あ|ん}な
\ruby{男}{をとこ}の
\ruby{事}{こと}か
\ruby{知}{し}らん、
%
\ruby{餘{\換字{所}}目}{よ|そ|め}から
\ruby{見}{み}ては
\ruby{氣}{き}が
\ruby{竭}{つ}きて、
%
\ruby{何}{なん}だか
\ruby{憫然}{かあ|いさう}なやうな% 「憫然 か(あ)いさう」
\ruby{氣}{き}がすると、
%
\原本頁{113-4}\改行%
\ruby{思}{おも}つた
\ruby{位}{くらゐ}に
\ruby{月日}{つき|ひ}を
\ruby{經}{へ}て
\ruby{來}{き}た、
%
\ruby{其}{そ}の
\ruby{汝}{おめへ}の
\ruby{難行}{なん|ぎやう}
\ruby{苦行}{く|ぎやう}も
\ruby{何}{なん}の
\ruby{爲}{ため}だ。
%
やつぱり
\ruby{何時}{い|つ}か
\ruby{一度}{いち|ど}は
\ruby{汝}{おめへ}は
\ruby{汝}{おめへ}で、
%
\ruby{男兒}{をと|こ}
\ruby{甲{\換字{斐}}}{が|ひ}のある
\ruby{仕事}{し|ごと}を
\ruby{仕}{し}やうためばかりの
\ruby{事}{こと}ぢやあ
\ruby{無}{な}いか。
%
その
\ruby{木食坊主}{もく|じき|ばう|ず}か
なんぞのやうな、
%
\原本頁{113-8}\改行%
\ruby{味}{あぢ}の
\ruby{無}{な}い
\ruby{長}{なが}い
\ruby{月日}{つき|ひ}の
\ruby{生活}{くら|し}さへも、
%
\ruby{笑}{わら}つて
\ruby{仕}{し}て
\ruby{來}{き}た
\ruby{汝}{おめへ}だもの、
%
\ruby{何樣}{ど|ん}な
\ruby{苦}{くる}しい
\ruby{戀}{こひ}に
\ruby{落}{お}ちても、
%
よもや
\ruby{本心}{ほん|しん}を
\ruby{失}{うしな}つて、
%
\ruby[g]{熱病}{ねつ}に
\ruby{負}{ま}けて
\原本頁{113-10}\改行%
\ruby{仕舞}{し|ま}ふやうなことは
\ruby{有}{あ}るめえが、
%
さあ、
%
\ruby{戀愛}{こ|ひ}は
\ruby{怖}{こは}かあ
\ruby{無}{ね}えが
\ruby{隨{\換字{伴}}者}{お|と|も}が
\ruby{怖}{こは}い、
%
\ruby{案}{あん}じられて
ならねえところが
\ruby{其處}{そ|こ}にある!。
』% 島木の語りがほと段落だと思うが阮甫には閉じカッコ欠落
