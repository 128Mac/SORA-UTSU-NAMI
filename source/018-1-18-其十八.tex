\Entry{其十八}

% メモ 校正 2024-04-06 2024-05-25 2024-06-18
\原本頁{108-6}%
『
\ruby{僕}{ぼく}は
\ruby{元}{もと}から
\ruby[g]{學問}{がくもん}は
\ruby{{\換字{嫌}}}{きら}ひだし、
%
\ruby{身}{み}に
\ruby{浸}{し}みて
\ruby{書}{ほん}を
\ruby{讀}{よ}んだ
\ruby{事}{こと}も
\ruby{無}{な}いから、
%
どうせ
\ruby{僕}{ぼく}の
\ruby{云}{い}ふ
\ruby{事}{こと}なぞは
\ruby{下}{くだ}ら
\ruby{無}{な}からうが、
%
まんざら
\ruby[g]{正中}{つ ぼ }に
\ruby{外}{はづ}れたことも
\ruby{云}{い}は
\ruby{無}{な}いつもりだ。
%
かういふ
\ruby[g]{理屈}{り くつ}だ、
%
\ruby{聞}{き}いて
\ruby{吳}{く}れたまへ。
%
\ruby{僕}{ぼく}に
\ruby{云}{い}はせりやあ
\ruby[g]{色戀}{いろこひ}といふ
\ruby{奴}{やつ}あ、
%
\ruby[g]{人間}{にんげん}が
\ruby{一人並}{いち|にん|なみ}に
\原本頁{108-10}\改行%
\ruby[g]{成熟}{できあが}ると、
%
\ruby[g]{一度}{いちど }は
\ruby[g]{屹度}{きつと }
\ruby{發}{はつ}しる
\ruby[g]{熱病}{ね つ }なので、
%
\ruby[g]{身體}{からだ }の
\ruby{中}{なか}から
\ruby[g]{自然}{ひとりで}に
\原本頁{109-1}\改行%
\ruby{湧}{わ}く
\ruby{奴}{やつ}だ、
%
\ruby[g]{各自}{めい〳〵}の
\ruby[g]{料簡}{れうけん}から
\ruby{出}{で}て
\ruby{來}{く}るんぢやあ
\ruby{無}{な}い。
%
そりやあ
\ruby{其}{そ}の
\ruby[g]{當人}{たうにん}から
\ruby{云}{い}つて
\ruby{見}{み}りやあ、
%
\ruby[g]{彼處}{あすこ }が% 原本では「あ(す)こ」に見えるので
\ruby{好}{い}いとか、
%
\ruby[g]{此處}{こ ゝ }が
\ruby{好}{い}いとか
\改行% 校正作業の簡略化のため
、
%
\原本頁{109-3}\改行%
それ〴〵に
\ruby[g]{理由}{わ け }が
\ruby{有}{あ}つて
\ruby{惚}{ほ}れるのでも
\ruby{有}{あ}らうが、
%
ナアニ
\ruby[g]{年齡}{と し }が
\原本頁{109-4}\改行%
\ruby{爲}{さ}せるんだよ、
%
\ruby[g]{年齡}{と し }が
\ruby{爲}{さ}せるんだよ。
%
\ruby{彼}{あ}の
\ruby{女}{をんな}あ
\ruby{好}{い}いから
サア
\ruby{惚}{ほ}れて
\ruby{{\換字{遣}}}{や}らうと、
%
\ruby[g]{{\換字{分}}別}{ふんべつ}を
つけてから
\ruby{惚}{ほ}れる
\ruby{奴}{やつ}は
\ruby{無}{な}い。
%
\ruby{誰}{だれ}の
\ruby[g]{戀路}{こひぢ }も
\原本頁{109-6}\改行%
\ruby{同}{おんな}じ
\ruby{事}{こと}で、
%
\ruby{其}{そ}の
\ruby[g]{眞實}{ほんたう}の
ところを
\ruby{云}{い}やあ、
%
\ruby[g]{自{\換字{分}}}{じ ぶん}にも
\ruby[g]{理由}{わ け }は
\ruby{{\換字{分}}}{わか}らな
\原本頁{109-7}\改行%
いけれど、
%
\ruby{何}{なん}だか
\ruby{知}{し}ら
\ruby{無}{な}いが
\ruby[g]{自然}{ひとりで}に
\ruby{好}{す}く、
%
それが
\ruby[g]{抑々}{そも〳〵}の
\ruby[g]{發端}{はじまり}で
\改行% 校正作業の簡略化のため
、
%
\原本頁{109-8}\改行%
\ruby{其}{そ}の
\ruby{人}{ひと}の
\ruby[g]{笑顏}{ゑ がほ}なんぞが
\ruby[g]{何時}{い つ }の
\ruby{間}{ま}にか
\ruby{眼}{め}に
\ruby{染}{し}み
\ruby{付}{つ}いて
\ruby{{\換字{遺}}}{のこ}つたり、
%
\原本頁{109-9}\改行%
\ruby{物}{もの}を
\ruby{云}{い}つた
\ruby{聲}{こゑ}の
\ruby{色}{いろ}が
\ruby{耳}{みゝ}に
\ruby{{\換字{遺}}}{のこ}つたりして、
%
\ruby{{\換字{終}}}{しまひ}には
すつかりと
\ruby{其}{その}
\ruby{人}{ひと}が
\ruby[g]{自{\換字{分}}}{じ ぶん}の
\ruby{胸}{むね}の
\ruby{中}{うち}に
\ruby{在}{あ}るやうになる、
%
サア
\ruby{忘}{わす}れやうと
\ruby{思}{おも}つても
\ruby{忘}{わす}れられない、
%
\ruby[||j>]{始}{し}% ルビ調整(原本通り)
\ruby[||j>]{{\換字{終}}}{じゆう}
\ruby[||j>]{其}{ その}
\ruby[||j>]{人}{ ひと}の
\ruby{傍}{そば}に
\ruby{居}{ゐ}て
\ruby{見}{み}たくなる、
%
\ruby{離}{はな}れて
\ruby{居}{ゐ}ちやあ
\原本頁{110-1}\改行%
\ruby[g]{物悲}{ものがな}しくつて、
%
\ruby{何}{なん}と
\ruby{無}{な}く
\ruby{氣}{き}が
\ruby{濟}{す}まないやうな
\ruby[||j>]{心}{こゝろ}
\ruby[||j>]{持}{ もち}がする、
% \ruby{心持}{こゝろ|もち}がする、
%
\ruby[g]{自{\換字{分}}}{じ ぶん}が
\ruby{其}{その}
\ruby{人}{ひと}を
\ruby{思}{おも}ふやうに、
%
\ruby{其}{その}
\ruby{人}{ひと}にも
\ruby[g]{自{\換字{分}}}{じ ぶん}を
\ruby{思}{おも}つて
\ruby{貰}{もら}ひたくなる、
%
それから
\ruby[g]{段々}{だん〴〵}と
\ruby{泣}{な}いたり
\ruby{笑}{わら}つたりが
\ruby{始}{はじ}まる、
%
まあ
\ruby[g]{斯樣}{か う }
\ruby{云}{い}つた
\makeatletter
\@ifundefined{デバッグ@ビルド}{%
  \ruby[||j>]{順}{じゆん}
  \ruby[||j>]{立}{ だて}ぢやあ
}{%
  \ruby[<j||]{順}{じゆん}% 行末行頭の境界付近なので特例処置を施す
  \ruby[||j>]{立}{だて}ぢやあ
}%
\makeatother
% \ruby{順立}{じゆん|だて}ぢやあ
\ruby{無}{な}いか。
%
\換字{志}て
\ruby{見}{み}りやあ
\ruby[g]{自然}{ひとりで}に
\ruby{好}{す}くといふのが
\ruby{戀}{こひ}の
\ruby[g]{水上}{みなかみ}だが、
%
\ruby[g]{自然}{ひとりで}の
\ruby[||j>]{好}{すき}
\ruby[||j>]{惡}{きらひ}だもの、
% \ruby{好惡}{すき|きらひ}だもの、
%
\ruby[g]{理屈}{り くつ}は
\ruby{有}{あ}りや
\ruby[g]{仕無}{し な }い、
%
みんな
\ruby[g]{年齡}{と し }が
\ruby{爲}{さ}
\原本頁{110-6}\改行%
せるんだ、
%
\ruby[g]{年齡}{と し }が
\ruby{爲}{さ}せるんだ。
%
\ruby{懷姙者}{み|も|ち}は
\ruby{酸}{す}いものを
\ruby[g]{自然}{ひとりで}に
\ruby{好}{す}く
\改行% 校正作業の簡略化のため
、
%
\原本頁{110-7}\改行%
\ruby{溜飮持}{りう|いん|もち}は
\ruby[g]{香物}{かう〳〵}で
\ruby{茶漬飮}{ちや|づ|け}を
\ruby[g]{自然}{ひとりで}に
\ruby{好}{す}く、
%
\ruby{其}{そ}の
\ruby[g]{自然}{ひとりで}に
\ruby{好}{す}くのは
\ruby{誰}{たれ}がさせる?、
%
\ruby[g]{惡阻}{つわり }が
\ruby{爲}{さ}せるんだ、
%
\ruby[g]{溜飮}{りういん}が
\ruby{爲}{さ}せるんだ。
%
\ruby[g]{戀路}{こひぢ }の
\ruby[g]{{\換字{迷}}惑}{まよひ }は
\ruby[g]{年齡}{と し }が
\ruby{爲}{さ}せるんだ。
%
\ruby[g]{男兒}{をとこ }が
\ruby[g]{男兒}{をとこ }づくる
\ruby{頃}{ころ}にやあ
\ruby[g]{髭鬚}{ひ げ }が
\ruby{生}{は}えて
\原本頁{110-10}\改行%
\ruby{來}{く}る、
%
\ruby[g]{髭鬚}{ひ げ }の
\ruby{生}{は}えるのは
\ruby[g]{年齡}{と し }が
\ruby{爲}{さ}せるんだもの、
%
それに
\ruby{善}{い}いも
\原本頁{110-11}\改行%
\ruby{惡}{わる}いも
\ruby{有}{あ}りやうは
\ruby{無}{な}い、
%
\ruby{口}{くち}の
\ruby[g]{周圍}{まはり }に
\ruby{出}{で}て
\ruby{來}{く}る
\ruby[g]{髭鬚}{ひ げ }も、
%
\ruby{心}{こゝろ}の
\ruby{上}{うへ}に
\原本頁{111-1}\改行%
\ruby{萌}{めぐ}む
\ruby{戀}{こひ}も、
%
\ruby[g]{年端}{と し }が
\ruby{爲}{さ}せるに
\ruby[g]{差異}{ちがひ }は
\ruby{無}{な}い、
%
\ruby[g]{丁度}{ちやうど}
\ruby[||j>]{同}{おんな}じ
\ruby{事}{こと}だもの、
%
ナニ
\ruby[g]{戀愛}{こ ひ }を
\ruby{善}{い}いとも
\ruby{惡}{わる}いとも
\ruby{云}{い}はう
\ruby{譯}{わけ}は
\ruby{無}{な}い。
%
たゞ
\ruby[g]{年齡}{と し }が
\ruby{爲}{さ}せる
\原本頁{111-3}\改行%
\ruby[g]{熱病}{ね つ }を
すらりと
\ruby{濟}{すま}せて
\ruby[g]{仕無}{し ま }へば、
%
\ruby[g]{疱瘡}{はうさう}や
\ruby[g]{痲疹}{はしか }が
\ruby{濟}{す}んだと
\ruby{同}{おな}じに
\改行% 校正作業の簡略化のため
、
%
\原本頁{111-4}\改行%
つまり
\ruby{芽出度}{め|で|たい}と
\ruby{云}{い}へば
\ruby{云}{い}へるので、
%
\ruby{戀}{こひ}は
\ruby{怖}{おそ}ろしいものでも
\ruby{何}{なん}でも
\ruby{無}{な}い。
%
\ruby{併}{しか}し
\ruby{{\換字{又}}}{また}、
%
\ruby{君}{きみ}は
\ruby[g]{學問}{がくもん}もあり
\ruby[g]{思慮}{し りよ}もあるから、
%
\ruby[g]{萬々}{ばん〳〵}
\ruby[g]{承知}{しようち}
\ruby{仕}{し}て
\ruby{居}{ゐ}やうが、
%
お
\ruby{互}{たがひ}に
\ruby[g]{男兒}{をとこ }といふ
\ruby{奴}{やつ}は、
%
\ruby[g]{戀愛}{こ ひ }の
\ruby[g]{奴隷}{け らい}に
\ruby{生}{うま}れて
\ruby{居}{ゐ}るものでも
\ruby{何}{なん}でも
\ruby{無}{な}い、
%
それ〴〵
\ruby[g]{男子}{をとこ }
\ruby[g]{一匹}{いつぴき}
\ruby{{\換字{前}}}{まへ}の
\ruby[g]{目的}{もくてき}のために
\ruby{意氣地}{い|き|ぢ}を
\ruby{磨}{みが}いて
\ruby[||j>]{一}{いつ}
\ruby[||j>]{生}{しやう}を
% \ruby{一生}{いつ|しやう}を
\ruby{働}{はたら}いて
\ruby{行}{ゆ}かうといふ
\ruby{身}{み}、
%
\ruby{戀}{こひ}に
\ruby{捲}{ま}き
\ruby{倒}{たふ}されちやあ
ならねえ
\ruby[g]{身體}{からだ }だ、
%
\ruby{其}{そ}の
\ruby[g]{熱病}{ね つ }に
\ruby[g]{身體}{からだ }を
\ruby{{\換字{遣}}}{や}る
\ruby{譯}{わけ}にやあ
いかねえ
\ruby[g]{約束}{やくそく}がある。
%
\ruby{病}{やまひ}にも
\ruby{輕}{かる}い
\ruby{重}{おも}いは
あり、
%
\ruby{戀}{こひ}にも
\ruby{深}{ふか}い
\ruby{淺}{あさ}いは
\ruby{有}{あ}らうが
\改行% 校正作業の簡略化のため
、
%
\原本頁{111-11}\改行%
\ruby[g]{如何}{い か }に
\ruby{戀}{こひ}に
\ruby{惱}{なや}んでも
\ruby{苦}{くる}しんでも、
%
\ruby{吐}{つ}く
\ruby{息}{いき}が
\ruby{火}{ひ}に
なつて
\ruby{燃}{も}えるほどに
\ruby{狂}{くる}はうとも、
%
\ruby{戀}{こひ}に
\ruby{負}{ま}けて
\ruby{死}{し}んぢやあ
\ruby[g]{男子}{をとこ }たる
\ruby{身}{み}の、
%
\ruby{眼}{め}が
\ruby{瞑}{ふさ}げねえ
\ruby{筈}{はず}だ。
%
イヤ
\ruby{瞑}{ふさ}げねえ、
%
どうしても
\ruby{死}{しに}きれねえ、
%
\ruby{死}{し}ね
\ruby{無}{ね}え
\原本頁{112-3}\改行%
\ruby{筈}{はず}だ。
%
\ruby[g]{乃公}{お ら }あ
\ruby{死}{し}な
\ruby{無}{ね}え、
%
\ruby{死}{し}にも
\ruby[g]{仕無}{し ね }えが、
%
\ruby{汝}{おめへ}も
\ruby{死}{し}ねめえ、
%
\ruby{死}{し}にも
すめえナ。
%
\ruby{知}{し}れ
\ruby{切}{き}つた
\ruby{事}{こと}だが、
%
ナア
\ruby[g]{水野}{みづの }、
%
お
\ruby{互}{たがひ}に
\ruby[g]{幾干}{いくそ }
\ruby[g]{{\換字{若}}干}{ばくそ }の
\原本頁{112-5}\改行%
\ruby[g]{苦勞}{く らう}を
\ruby{仕}{し}て、
%
\ruby[g]{今日}{け ふ }まで
\ruby{{\換字{遣}}}{や}つて
\ruby{來}{き}たなあ
\ruby{何}{なん}の
\ruby{爲}{ため}だ?。
%
\ruby[<j||]{志}{こゝろ}%
\ruby[||j>]{こ}{ざし}そ
% \ruby[<j>]{志}{こゝろざし}こそ
\ruby{異}{ちが}ふけれど、
%
\ruby[g]{男兒}{をとこ }と
\ruby{生}{うま}れた
\ruby{生}{うま}れ
\ruby[g]{甲{\換字{斐}}}{が ひ }にやあ、
%
\ruby[g]{各自}{めい〳〵}の
\ruby[g]{念願}{おもひ }を
\ruby{{\換字{遂}}}{と}げやうと、
%
それ
ばつかりの
\ruby{爲}{ため}ぢやあ
\ruby{無}{ね}えか。
%
\ruby{特}{こと}さら
\ruby{汝}{おめへ}は
\ruby[g]{乃公}{お れ }から
\ruby{云}{い}やあ、
%
マア
\ruby{慾}{よく}の
\ruby{無}{な}さすぎる
\ruby[g]{偏人}{へんじん}で、
%
\ruby{取}{と}れる
\ruby{錢}{ぜに}も
\ruby{取}{と}らず
\ruby[g]{出世}{しゆつせ}も
\ruby{望}{のぞ}まず、
%
\ruby[g]{大根}{だいこん}
\ruby[g]{人參}{にんじん}の
\ruby[g]{尻尾}{しつぽ }を
\ruby{咬}{かじ}つて、
%
それで
\ruby{濟}{す}まして
\ruby{居}{ゐ}るやうな
\ruby{{\換字{遣}}}{や}り
\原本頁{112-10}\改行%
\ruby{方}{かた}。
%
アヽ
\ruby{世}{よ}の
\ruby{中}{なか}は
いろ〳〵のもんだ、
%
\ruby[g]{水野}{みづの }だつて
\ruby[g]{不味}{ま づ }いものあ
\原本頁{112-11}\改行%
\ruby[g]{不味}{ま づ }く、
%
\ruby[g]{美味}{う ま }いものは
\ruby{旨}{うま}からうが、
%
\ruby{其}{それ}にも
\ruby{此}{これ}にも
\ruby{頓着無}{とん|ぢやく|な}く、
%
\ruby{{\換字{若}}}{わか}い
\ruby{身}{み}そらで
\ruby[g]{色氣}{いろけ }も
\ruby{無}{な}く、
%
\ruby[g]{下手}{へ た }な
\ruby[g]{律僧}{りつそう}は
\ruby{及}{およ}ばぬ
\ruby[g]{身持}{み もち}で、
%
たゞ
\ruby[g]{學問}{がくもん}に
\ruby{凝}{こ}つて
\ruby{居}{ゐ}る、
%
アヽ
\ruby[g]{聖人}{せいじん}と
\ruby{云}{い}ふなあ
\ruby[g]{彼樣}{あ ん }な
\ruby{男}{をとこ}の
\ruby{事}{こと}か
\ruby{知}{し}らん、
%
\ruby{餘{\換字{所}}目}{よ|そ|め}から
\ruby{見}{み}ては
\ruby{氣}{き}が
\ruby{竭}{つ}きて、
%
\ruby{何}{なん}だか
\ruby[||j>]{憫}{かあ}
\ruby[||j>]{然}{いさう}なやうな% 「憫然 か(あ)いさう」
% \ruby{憫然}{かあ|いさう}なやうな% 「憫然 か(あ)いさう」
\ruby{氣}{き}がすると、
%
\原本頁{113-4}\改行%
\ruby{思}{おも}つた
\ruby{位}{くらゐ}に
\ruby[g]{月日}{つきひ }を
\ruby{經}{へ}て
\ruby{來}{き}た、
%
\ruby{其}{そ}の
\ruby{汝}{おめへ}の
\ruby[||j>]{{\換字{難}}}{なん}
\ruby[||j>]{行}{ぎやう}
% \ruby{{\換字{難}}行}{なん|ぎやう}
\ruby[||j>]{苦}{ く}
\ruby[||j>]{行}{ぎやう}も
\ruby{何}{なん}の
\ruby{爲}{ため}だ。
%
やつぱり
\ruby[g]{何時}{い つ }か
\ruby[g]{一度}{いちど }は
\ruby{汝}{おめへ}は
\ruby{汝}{おめへ}で、
%
\ruby[g]{男兒}{をとこ }
\ruby[g]{甲{\換字{斐}}}{が ひ }のある
\ruby[g]{仕事}{し ごと}を
\ruby{仕}{し}やうためばかりの
\ruby{事}{こと}ぢやあ
\ruby{無}{な}いか。
%
その
\ruby{木食坊主}{もく|じき|ばう|ず}か
なんぞのやうな、
%
\原本頁{113-8}\改行%
\ruby{味}{あぢ}の
\ruby{無}{な}い
\ruby{長}{なが}い
\ruby[g]{月日}{つきひ }の
\ruby[g]{生活}{くらし }さへも、
%
\ruby{笑}{わら}つて
\ruby{仕}{し}て
\ruby{來}{き}た
\ruby{汝}{おめへ}だもの、
%
\ruby[g]{何樣}{ど ん }な
\ruby{苦}{くる}しい
\ruby{戀}{こひ}に
\ruby{落}{お}ちても、
%
よもや
\ruby[g]{本心}{ほんしん}を
\ruby{失}{うしな}つて、
%
\ruby[g]{熱病}{ね つ }に
\ruby{負}{ま}けて
\原本頁{113-10}\改行%
\ruby[g]{仕舞}{し ま }ふやうなことは
\ruby{有}{あ}るめえが、
%
さあ、
%
\ruby[g]{戀愛}{こ ひ }は
\ruby{怖}{こは}かあ
\ruby{無}{ね}えが
\ruby{隨{\換字{伴}}者}{お|と|も}が
\ruby{怖}{こは}い、
%
\ruby{案}{あん}じられて
ならねえところが
\ruby[g]{其處}{そ こ }にある!。
』% 島木の語りがほと段落だと思うが阮甫には閉じカッコ欠落
