\Entry{其十九}

% メモ 校正 2024-04-07
\原本頁{114-2}%
『
\ruby{隨{\換字{伴}}者}{お|と|も}と
\ruby{云}{い}ふなあ
\ruby{他}{ほか}ぢやあ
\ruby{無}{ね}えが、
%
\ruby{戀}{こひ}に
\ruby{隨}{つ}いて
\ruby{來}{く}る
\ruby{心氣}{し|ん}の
\ruby{疲勞}{つか|れ}だ。
%
お
\ruby{互}{たがひ}に
\ruby{覺}{おぼ}えのある
\ruby{事}{こと}だが、
%
\ruby{男}{をとこ}の
\ruby{兒}{こ}といふ
\ruby{奴}{やつ}あ
\ruby{十三四}{じう|さん|し}から、
%
\原本頁{114-4}\改行%%
そろ〳〵
\ruby{野心}{や|しん}が
\ruby{燃}{も}えたつて
\ruby{來}{き}て、
%
\ruby{威張}{ゐ|ば}つて
\ruby{見}{み}たい、
%
\ruby{人}{ひと}に
\ruby{{\換字{勝}}}{か}ちたい、
%
\ruby{功}{てがら}が
\ruby{立}{た}てたい、
%
\ruby{名}{な}が
\ruby{立}{た}てたい、
%
\ruby{天下}{てん|か}が
\ruby{取}{と}りたい、
%
と
\ruby{氣象}{き|しやう}
\ruby{相應}{さう|おう}の
\ruby{望}{のぞ}みを
\ruby{起}{おこ}すが、
%
それでも
\ruby{其}{そ}の
\ruby{時{\換字{分}}}{じ|ぶん}の
\ruby{腹中}{はらん|なか}は
\ruby[g]{淸潔}{きれい}なもので、
%
たゞ
\ruby[g]{醇醉}{いつぽんぎ}の
\ruby[g]{大望心}{たいまう}が
あるばかり、
%
\ruby{乃公}{お|ら}あ
\ruby{太閤}{たい|かふ}だぞ、
%
\ruby[g]{拿破崙}{なぽれおん}だぞと、
%
\原本頁{114-8}\改行%
\ruby[g]{各自}{てん〴〵}に
\ruby{力}{りき}む
\ruby{其勢}{その|いきほひ}で、
%
\ruby{伸}{の}びも
\ruby{育}{そだ}ちも
\ruby{仕}{し}て
\ruby{來}{く}るが、
%
\ruby{遲}{おそ}かれ
\ruby{{\換字{速}}}{はや}かれ
\ruby{時{\換字{節}}}{と|き}が
\ruby{來}{き}て、
%
\ruby{戀}{こひ}という
\ruby{奴}{やつ}に
\ruby{魅入}{み|い}られちやあ、
%
さあ
\ruby{腹}{はら}の
\ruby{中}{なか}が
\ruby{揉}{も}めて
\原本頁{114-10}\改行%
\ruby{來}{く}る。
%
\ruby[g]{大望心}{たいまう}は
\ruby[g]{大望心}{たいまう}で
\ruby{居}{ゐ}しかつて
\ruby{居}{ゐ}る、
%
\ruby{戀}{こひ}の
\ruby{心}{こゝろ}は
\ruby{戀}{こひ}の
\ruby{心}{こゝろ}で
\ruby{自由}{ま|ま}に
\ruby{働}{はたら}く。
%
\原本頁{115-1}%
\ruby{双方}{さう|はう}が
\ruby{頭}{かしら}は
\ruby{下}{さ}げないから、
%
\ruby[g]{衝突}{ぶつか}りやあ
\ruby{何樣}{ど|う}しても
\ruby{忽}{たちま}ち
\ruby[g]{戰爭}{たゝかひ}で、
%
\ruby{那方}{どつ|ち}が
\ruby{{\換字{勝}}}{か}つにしても
\ruby{負}{ま}けるにしても、
%
なか〳〵
\ruby{樂}{らく}な
\原本頁{115-3}\改行%
\ruby{爭鬩}{せり|あひ}ぢやあ
\ruby{無}{な}い。
%
\ruby{戀}{こひ}が
\ruby{負}{ま}けて
\ruby{倒}{たふ}れりやあ
\ruby{其}{そ}の
\ruby{傷口}{きず|ぐち}から、
%
\ruby{溢}{こぼ}れる
\原本頁{115-4}\改行%
\ruby{血潮}{ち|しほ}が
\ruby{急}{きふ}にやあ
\ruby{止}{と}まらず、
%
\ruby[g]{大望心}{たいまう}が
\ruby{負}{ま}けりやあ
\ruby{其}{そ}の
\ruby{英氣}{えい|き}は、
%
\ruby{未練氣}{み|れん|げ}
\ruby{無}{な}く
\ruby{去}{さ}つて
\ruby{仕舞}{し|ま}つて
\ruby{呼}{よ}んでも
\ruby{{\換字{還}}}{かへ}らねえ。
%
つまり
\ruby{何樣}{ど|う}なつても
\ruby{根}{ね}が
\ruby{同士討}{どう|し|うち}の、
%
\ruby{酷}{ひど}い
\ruby[g]{戰爭}{たゝかひ}に
\ruby{國土}{く|に}は
\ruby{荒}{あ}れて、
%
\ruby{{\換字{遺}}}{のこ}るものは
\ruby{怖}{おそ}ろしい
\ruby{心氣}{し|ん}の
\ruby{疲勞}{つか|れ}!。
%
\ruby{櫻色}{さくら|いろ}して
\ruby{居}{ゐ}た
\ruby{面}{かほ}は
\ruby{白}{しら}けて、
%
\ruby{葛}{くず}の
\ruby{葉裏}{は|うら}を
\ruby{見}{み}るやうになり、
%
\ruby{眼}{め}は
\ruby{冴}{さ}えなくなる、
%
\ruby{白髮}{しら|が}は
さす、
%
\ruby{{\換字{強}}}{つよ}い
\ruby{奴}{やつ}は
\ruby{癇癪持}{かん|しやく|もち}になる。
%
\ruby{{\換字{弱}}}{よわ}い
\ruby{奴}{やつ}は
\ruby{萎縮{\換字{漢}}}{いぢ|け|もの}になる。
%
\ruby{筋骨}{すぢ|ぼね}は
\ruby{弛}{ゆる}んで
\ruby{仕舞}{し|ま}ふ、
%
\ruby{勞苦{\換字{嫌}}}{ほね|をり|ぎら}ひになる。
%
\ruby{其}{そ}の
\ruby{位}{くらゐ}なのは
\ruby{未}{ま}だ
\ruby{可}{い}い
\ruby{{\換字{分}}}{ぶん}で、
%
\ruby{隨{\換字{分}}}{ずゐ|ぶん}
\ruby{怖}{おそ}ろしい
\ruby{病氣}{びやう|き}さへも
\原本頁{115-11}\改行%
\ruby{引出}{ひき|だ}す。
%
よしんば
\ruby[g]{大望心}{たいまう}と
\ruby{戀愛}{こ|ひ}とが
\ruby[g]{衝突}{ぶつか}らないで、
%
\ruby{腹}{はら}の
\ruby{中}{なか}が
それほどには
\ruby{揉}{も}め
\ruby{無}{な}いでも、
%
\ruby{向}{むこ}ふに
\ruby{的}{まと}の
\ruby{無}{な}い
\ruby{戀}{こひ}は
\ruby{無}{な}いから、
%
\ruby{星}{ほし}に
\原本頁{116-2}\改行%
\ruby{中}{あた}る
\ruby{中}{あた}らぬは
\ruby{時}{とき}の
\ruby{{\換字{運}}}{うん}
\ruby{身}{み}の
\ruby{{\換字{運}}}{うん}!。
%
\ruby{相手}{あひ|て}と
\ruby{馬}{うま}が
\ruby{合}{あ}ふ
\ruby{合}{あ}はぬもあるし、
%
\原本頁{116-3}\改行%
\ruby{相手}{あひ|て}とは
\ruby{死}{し}ぬほどに
\ruby{好}{す}き
\ruby{合}{あ}つても、
%
\ruby{自{\換字{分}}}{じ|ぶん}たち
ばかりのために
\ruby{出來}{で|き}て
\ruby{居}{ゐ}る
\ruby{世界}{せ|かい}ぢやあ
\ruby{無}{な}いもの、
%
\ruby{何}{なに}がさて
\ruby{外{\換字{道}}}{げ|だう}も
\ruby{居}{ゐ}る、
%
\ruby{惡{\換字{魔}}}{あく|ま}も
\ruby{居}{ゐ}る、
%
\ruby{敵}{てき}も
\ruby{居}{ゐ}る、
%
おせつかいも
\ruby{居}{ゐ}る、
%
\ruby{義理}{ぎ|り}もある、
%
\ruby{人{\換字{情}}}{にん|じやう}もある、
%
\ruby{時}{とき}もある、
%
\ruby{場合}{ば|あひ}もあつて、% 原文通り「場」
%
\ruby{隨意}{ま|ゝ}ならぬ
\ruby{憂}{う}き
\ruby{世}{よ}を
\ruby{泣}{な}くものが
\ruby{多}{おほ}い。
%
\原本頁{116-7}\改行%
\ruby{左樣}{さ|う}で
\ruby{無}{な}くつてさへ
\ruby{戀}{こひ}を
\ruby{知}{し}るなあ
\ruby{涙}{なみだ}を
\ruby{知}{し}る
\ruby{始}{はじめ}で、
%
\ruby{氣}{き}が
\ruby{優}{やさ}しくなる、
%
\ruby{脆}{もろ}くなる、
%
\ruby{感}{かん}じが
\ruby{早}{はや}くなる、
%
\ruby{深}{ふか}くなる、
%
\ruby{何}{なん}でも
\ruby{無}{な}い
\ruby{事}{こと}に
ハツと
\ruby{思}{おも}つたり、
%
\ruby{小}{ちひさ}な
\ruby{事}{こと}を
くよ〳〵と
\ruby{案}{あん}じたり、
%
\ruby{{\換字{前}}表}{ぜん|ぺう}といふやうな
\ruby{事}{こと}を
\ruby{氣}{き}にしたり、
%
\ruby{何}{なに}かにつけて
\ruby{思}{おも}ひ
\ruby{{\換字{過}}}{すご}しを
\ruby{仕}{し}たり、
%
\ruby{寢}{ね}るべき
\原本頁{116-11}\改行%
\ruby{時}{とき}に
\ruby{寢}{ね}られなかつたりする。
%
そこで
\ruby{段々}{だん|〴〵}と
\ruby{心氣}{し|ん}が
\ruby{{\換字{弱}}}{よわ}る。
%
\ruby{心氣}{し|ん}が
\原本頁{117-1}\改行%
\ruby{{\換字{弱}}}{よわ}りやあ
\ruby{愈々}{いよ|〳〵}
\ruby{氣}{き}が
\ruby{脆}{もろ}くなる、
%
\ruby{感}{かん}じが
\ruby{{\換字{強}}}{つよ}くなる。
%
\ruby{氣}{き}が
\ruby{脆}{もろ}く、
%
\ruby{感}{かん}じが
\ruby{{\換字{強}}}{つよ}くなりやあ
\ruby{{\換字{又}}}{また}
\ruby{心氣}{し|ん}が
\ruby{{\換字{弱}}}{よわ}る。
%
\ruby{雁齒鑢}{がん|ぎ|やすり}が
かゝるやうなものだから
\ruby{堪}{たま}らう
\ruby{譯}{わけ}は
\ruby{無}{な}い。
%
\ruby{一日}{いち|にち}
\ruby{一日}{いち|にち}に
\ruby{{\換字{弱}}}{よわ}つた
\ruby{擧句}{あげ|く}は、
%
\ruby{魂魄}{たま|しひ}が
\ruby{薄手}{うす|で}に
なりきつて、
%
\ruby{觸}{さは}るものさへ
あれば
\ruby{砕}{くだ}けたがる
\ruby[g]{玻璃}{びいどろ}か
なんぞのやうになつて
\ruby{仕舞}{し|ま}ふ。
%
よく
\ruby{世間}{せ|けん}にある
\ruby{戀路}{こひ|ぢ}の
\ruby{果}{はて}の、
%
\ruby{飛}{と}んでも
\ruby{無}{な}い
\ruby{不幸福}{ふ|しあ|はせ}は%「幸福」 ここは「は」
\ruby{皆}{みな}
\ruby{其處}{そ|こ}で
\ruby{出來}{で|き}る。
%
たとひ
\ruby{{\換字{嫌}}}{きら}はれても
\ruby{{\換字{嫌}}}{きら}はれても、
%
\ruby{好}{す}かれたいのが
\ruby{戀}{こひ}の
\ruby{慾}{よく}で、
%
\ruby{{\換字{又}}}{また}
\ruby{憂}{う}いも
\ruby{辛}{つら}いも
\ruby{堪{\換字{忍}}}{しん|ばう}して% 原文通り「堪忍」
\ruby{添}{そ}ひ
\ruby{{\換字{遂}}}{と}げたいのが
\原本頁{117-8}\改行%
\ruby{戀}{こひ}の
\ruby{意地}{い|ぢ}だ。
%
\換字{志}て
\ruby{見}{み}りやあ
\ruby{戀}{こひ}に
\ruby{生命}{いの|ち}の
\ruby{捨}{す}てやうは
\ruby{無}{な}い、
%
\ruby{戀}{こひ}は
\ruby{生々}{いき|いき}と
\ruby{美}{うつく}しいものだ。
%
\ruby{世}{よ}の
\ruby{不幸福}{ふ|しあ|せ}な%「幸福」 ここは「は」欠落
\ruby{人}{ひと}を
\ruby{見}{み}りやあ、
%
\ruby{戀}{こひ}で
\ruby{死}{し}ぬものは
\ruby{一人}{ひと|り}も
\ruby{無}{な}く、
%
\ruby[<j|]{皆}{みんな}
\ruby{心氣}{し|ん}の
\ruby{疲勞}{つか|れ}に
\ruby{堪}{こら}へ
\ruby{切}{き}れ
\ruby{無}{な}くなつて、
%
おのが
\ruby{魂魄}{たま|しひ}を
\ruby{碎}{くだ}いて
\ruby{仕舞}{し|ま}うのだが、
%
\ruby{{\換字{避}}}{さ}けやうにも
\ruby{{\換字{避}}}{さ}け
\ruby{{\換字{難}}}{にく}いのは
\ruby{此}{こ}の
\ruby{隨{\換字{伴}}者}{お|と|も}だから、
%
\ruby{戀}{こひ}は
\ruby{毫末}{ちつ|と}も
\ruby{怖}{こは}かあ
\ruby{無}{な}いが、
%
\ruby{其}{そ}の
\ruby{隨{\換字{伴}}者}{お|と|も}の
\ruby{心氣}{し|ん}の
\ruby{疲勞}{つか|れ}は
\ruby{恐}{おそ}ろしい。
%
\ruby{實}{じつ}を
\ruby{云}{い}やあ
\ruby{僕}{ぼく}が
\ruby{君}{きみ}の
\ruby{事}{こと}を
\ruby{素破拔}{すつ|ぱ|ぬ}いて
\ruby{饒舌}{しや|べ}つたから、
%
\原本頁{118-3}\改行%
\ruby{羽{\換字{勝}}}{は|がち}も
\ruby{日方}{ひ|かた}も
\ruby{君}{きみ}のために、
%
\ruby{二人}{ふた|り}とも
\ruby{甚}{ひど}く
\ruby{心配}{しん|ぱい}して
\ruby{居}{ゐ}る。
%
\ruby{特}{こと}に
\ruby{日方}{ひ|かた}は
\ruby{彼}{あ}の
\ruby{氣性}{き|しやう}だから、
%
\ruby{{\換字{強}}}{きつ}い
\ruby{意見}{い|けん}を
\ruby{云}{い}ひに
\ruby{行}{ゆ}かうかも
\ruby{知}{し}れないが、
%
\原本頁{118-5}\改行%
\ruby{乃公}{お|ら}あ
\ruby{何}{なんに}も
\ruby{意見}{い|けん}は
\ruby{云}{い}はない。
%
\ruby{何}{なん}も
\ruby{彼}{か}も
\ruby{解}{わか}つて
\ruby{居}{ゐ}る
\ruby{君}{きみ}の
\ruby{事}{こと}だもの、
%
\原本頁{118-6}\改行%
\ruby{君}{きみ}が
\ruby{詰}{つま}ら
\ruby{無}{な}い
\ruby{事}{こと}を
\ruby{仕}{し}やう
\ruby{氣{\換字{遣}}}{き|づか}ひは
\ruby{無}{な}いが、
%
たゞ
\ruby{心氣}{し|ん}の
\ruby{疲勞}{つか|れ}に
\ruby{負}{ま}けぬやうにと、
%
これだけを
\ruby{君}{きみ}に
\ruby{頼}{たの}んで
\ruby{置}{お}く。
%
\ruby{見}{み}りやあ
\ruby{顏色}{かほ|つき}と
\ruby{云}{い}ひ
\ruby{容態}{よう|す}といひ、
%
\ruby{心氣}{し|ん}が
\ruby{疲}{つか}れて
\ruby{居}{ゐ}ないやうでも
\ruby{無}{な}い、
%
\ruby{氣}{き}をつけて
\原本頁{118-9}\改行%
\ruby{吳}{く}れ
\ruby{無}{な}くちやあ
いけないぜ。
%
\ruby{何時}{い|つ}かは
\ruby{云}{い}はう〳〵と
\ruby{思}{おも}つて
\ruby{居}{ゐ}たので、
%
つい
\ruby{圖}{づ}に
\ruby{乘}{の}つて
\ruby{長}{なが}く
\ruby{饒舌}{しや|べ}つて、
%
\ruby{言葉}{こと|ば}さへ
\ruby{亂暴}{らん|ばう}に
\ruby{言}{い}ひ
\ruby{{\換字{過}}}{す}ごしたが、
%
\ruby{意}{こゝろ}だけは
\ruby{是非}{ぜ|ひ}とも
\ruby{汲}{く}んで
\ruby{吳}{く}れたまへ。
%
\ruby{千言萬言}{せん|げん|ばん|げん}
\ruby{饒舌}{しや|べ}つても、
%
\ruby{身體}{から|だ}を
\ruby{大切}{たい|せつ}に
\ruby{仕}{し}て
\ruby{吳}{く}れろといふ、
%
たゞの
\ruby{一句}{いつ|く}に
\ruby{止}{とゞ}まるのだ。
%
\ruby{飯}{めし}の
\ruby{不味}{ま|づ}い
\ruby{時}{とき}も
\ruby{堪{\換字{忍}}}{がま|ん}して% 原文通り「堪忍」
\ruby{食}{く}つて、
%
\ruby{成}{な}るたけ
\ruby{精々}{せい|〴〵}
\ruby{身體}{から|だ}を
\ruby{使}{つか}つて、
%
\ruby{寢}{ね}るべき
\ruby{時}{とき}にやあ
\ruby{整然}{ちや|ん}と
\ruby{寢}{ね}て、
%
\ruby[<j|]{力}{ちから}
\ruby{足}{あし}を
\ruby{踏}{ふ}んで
\ruby{確乎}{しつ|かり}と、
%
\ruby{快活}{き|さく}に
\ruby{日}{ひ}を
\ruby{{\換字{送}}}{おく}つて
\ruby{貰}{もら}ひたいのだ。
%
\ruby{君}{きみ}の
\ruby{氣}{き}に
\ruby{入}{い}つたほどの
\ruby{人}{ひと}だもの、
%
\原本頁{119-5}\改行%
\ruby{僕}{ぼく}は
\ruby{其}{そ}の
\ruby{人}{ひと}を
\ruby{知}{し}らないが、
%
\ruby{屹度}{きつ|と}
\ruby{好}{い}い
\ruby{人}{ひと}だらうと
\ruby{思}{おも}つて
\ruby{居}{ゐ}て、
%
\ruby{君}{きみ}の
\ruby{{\換字{運}}命}{う|ん}の
\ruby{好}{い}いやうにと
ばかり
\ruby{祈}{いの}つて
\ruby{居}{ゐ}る。
%
\ruby{僕}{ぼく}の
\ruby{力}{ちから}の
\ruby{要}{い}る
\ruby{事}{こと}が
あらば、
%
\ruby{何}{なん}なりと
\ruby{{\換字{遠}}慮無}{ゑん|りよ|な}く
\ruby{云}{い}つて
\ruby{吳}{く}れたまへ、
%
\ruby{君}{きみ}のために
\ruby{幸福}{しあ|はせ}になる%「幸福」ここは「は」
\ruby{事}{こと}ならば、
%
\ruby{何樣}{ど|ん}な
\ruby{事}{こと}を
\ruby{仕}{し}ても
\ruby{僕}{ぼく}は
\ruby{厭}{いと}はない。
%
\ruby{馬}{うま}にでも
\ruby{牛}{うし}にでもなつて
\ruby{働}{はたら}かうが、
%
\ruby{其}{そ}の
\ruby{代}{かは}り
\ruby{今}{いま}
\ruby{言}{い}つた
\ruby{戀}{こひ}の
\ruby{隨{\換字{伴}}者}{お|と|も}にやあ
\ruby{必}{かなら}ず
\原本頁{119-10}\改行%
\ruby{負}{ま}けて
\ruby{吳}{く}れたまふな。
%
\ruby{世界}{せ|かい}に
\ruby{人間}{ひ|と}は
\ruby{多}{おほ}いけれど、
{---}{---}
そりやあ
\原本頁{119-11}\改行%
\ruby{偉}{えら}い
\ruby{人}{ひと}も
\ruby{多}{おほ}からうが、
%
\ruby{此}{こ}の
\ruby{何年}{なん|ねん}を
\ruby{{\換字{過}}}{すご}して
\ruby{來}{き}た、
%
\ruby{君}{きみ}の
\ruby{行狀}{おこ|なひ}の
\ruby{殊{\換字{勝}}}{しゆ|しよう}さを
\ruby{見}{み}ては、
%
アヽ、
%
\ruby{眞似}{ま|ね}たつて
\ruby{眞似}{ま|ね}られない
\ruby{事}{こと}だ、
%
あゝいふ
\ruby{男}{をとこ}は
\ruby{今}{いま}の
\ruby{世}{よ}には、
%
\ruby{中々}{なか|〳〵}
\ruby{二人}{ふ|たり}とは
\ruby{有}{あ}りはすまい、
%
\ruby{島木}{しま|き}
\ruby{萬五郎}{まん|ご|らう}は
\ruby{俗物}{ぞく|ぶつ}だが、
%
\ruby{朋友}{とも|だち}にやあ
\ruby{幸福}{しあ|はせ}にも%「幸福」ここは「は」
\ruby{心}{こゝろ}の
\ruby{氣高}{け|だか}い
\ruby[g]{水野}{みづの}のやうな
\ruby{人}{ひと}を
\ruby{持}{も}つて
\原本頁{120-4}\改行%
\ruby{居}{ゐ}ると、
%
\ruby{天}{てん}にも
\ruby{地}{ち}にも
\ruby[<j|]{唯}{たつた}
\ruby{一人}{ひと|り}の
\ruby{大切}{たい|せつ}な
\ruby{朋友}{とも|だち}に
\ruby{思}{おも}つて
\ruby{居}{ゐ}る
\ruby{君}{きみ}の
\ruby{事}{こと}だから、
%
どうか
\ruby{身體}{から|だ}を
\ruby{大切}{たい|せつ}に
\ruby{仕}{し}て
\ruby{吳}{く}れたまへ、
%
\ruby{君}{きみ}の
\ruby{其}{そ}の
\ruby{顏}{かほ}つきを
\ruby{見}{み}ちやあ
\ruby{案}{あん}じられて
ならない。
%
くどいやうだが
\ruby{今}{いま}
\ruby{言}{い}つた
\ruby{事}{こと}を
\ruby{能}{よ}く
\ruby{聽}{き}いて
\ruby{置}{お}いて
\ruby{吳}{く}れたまへ。
』

\原本頁{120-8}%
と、
%
\ruby{眞{\換字{情}}}{ま|ごゝろ}こめて
\ruby{云}{い}ひ
\ruby{{\換字{終}}}{をは}りたり。
