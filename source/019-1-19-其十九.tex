\Entry{其十九}

% メモ 校正 2024-04-07 2024-05-25 2024-06-18
\原本頁{114-2}%
『
\ruby{隨{\換字{伴}}者}{お|と|も}と
\ruby{云}{い}ふなあ
\ruby{他}{ほか}ぢやあ
\ruby{無}{ね}えが、
%
\ruby{戀}{こひ}に
\ruby{隨}{つ}いて
\ruby{來}{く}る
\ruby[g]{心氣}{し ん }の
\ruby[g]{疲勞}{つかれ }だ。
%
お
\ruby{互}{たがひ}に
\ruby{覺}{おぼ}えのある
\ruby{事}{こと}だが、
%
\ruby{男}{をとこ}の
\ruby{兒}{こ}といふ
\ruby{奴}{やつ}あ
\ruby{十三四}{じふ|さん|し}から
\改行% 校正作業の簡略化のため
、
%
\原本頁{114-4}\改行%%
そろ〳〵
\ruby[g]{野心}{や しん}が
\ruby{燃}{も}えたつて
\ruby{來}{き}て、
%
\ruby[g]{威張}{ゐ ば }つて
\ruby{見}{み}たい、
%
\ruby{人}{ひと}に
\ruby{{\換字{勝}}}{か}ちた
\原本頁{114-5}\改行%
い、
%
\ruby{功}{てがら}が
\ruby{立}{た}てたい、
%
\ruby{名}{な}が
\ruby{立}{た}てたい、
%
\ruby[g]{天下}{てんか }が
\ruby{取}{と}りたい、
%
と
\ruby[g]{氣象}{きしやう}
\ruby[g]{相應}{さうおう}の
\ruby{望}{のぞみ}を
\ruby{起}{おこ}すが、
%
それでも
\ruby{其}{そ}の
\ruby[g]{時{\換字{分}}}{じ ぶん}の
\ruby[||j>]{腹}{はらん}
\ruby[||j>]{中}{ なか}は
% \ruby{腹中}{はらん|なか}は
\ruby[g]{淸潔}{きれい }なもので、
%
た
\原本頁{114-7}\改行%
ゞ% ルビ調整(原本通り)非踊り字表記(行末行頭の境界付近)
\ruby[||j>]{醇}{いつ}
\ruby[||j>]{醉}{ぽんぎ}
の
\ruby{大望心}{た|い|まう}が% ルビ調整(原本通り)
あるばかり、
%
\ruby[g]{乃公}{お ら }あ
\ruby[g]{太閤}{たいかふ}だぞ、
%
\ruby{拿破崙}{な|ぽれ|おん}だぞと
\改行% 校正作業の簡略化のため
、
%
\原本頁{114-8}\改行%
\ruby[g]{各自}{てん〴〵}に
\ruby{力}{りき}む
\ruby[||j>]{其}{その}
\ruby[||j>]{勢で}{いきほひ|}、% ルビ調整(原本通り)
%
\ruby{伸}{の}びも
\ruby{育}{そだ}ちも
\ruby{仕}{し}て
\ruby{來}{く}るが、
%
\ruby{遲}{おそ}かれ
\ruby{{\換字{速}}}{はや}かれ
\ruby[g]{時{\換字{節}}}{と き }が
\ruby{來}{き}て、
%
\ruby{戀}{こひ}という
\ruby{奴}{やつ}に
\ruby[g]{魅入}{み い }られちやあ、
%
さあ
\ruby{腹}{はら}の
\ruby{中}{なか}が
\ruby{揉}{も}めて
\原本頁{114-10}\改行%
\ruby{來}{く}る。
%
\ruby{大望心}{たい|ま|う}は% ルビ調整(原本通り)
\ruby{大望心}{たい|ま|う}で% ルビ調整(原本通り)
\ruby{居}{ゐ}しかつて
\ruby{居}{ゐ}る、
%
\ruby{戀}{こひ}の
\ruby{心}{こゝろ}は
\ruby{戀}{こひ}の
\ruby{心}{こゝろ}で
\ruby[g]{自由}{ま ま }に
\ruby{働}{はたら}く。
%
\原本頁{115-1}%
\ruby[g]{双方}{さうはう}が
\ruby{頭}{かしら}は
\ruby{下}{さ}げないから、
%
\ruby[g]{衝突}{ぶつか }りやあ
\ruby[g]{何樣}{ど う }しても
\ruby[<j||]{忽}{たちま}ち% 行末行頭の境界付近なので特例処置を施す
\ruby[g]{戰爭}{たゝかひ}で、
%
\ruby[g]{那方}{どつち }が
\ruby{{\換字{勝}}}{か}つにしても
\ruby{負}{ま}けるにしても、
%
なか〳〵
\ruby{樂}{らく}な
\原本頁{115-3}\改行%
\ruby[g]{爭鬩}{せりあひ}ぢやあ
\ruby{無}{な}い。
%
\ruby{戀}{こひ}が
\ruby{負}{ま}けて
\ruby{倒}{たふ}れりやあ
\ruby{其}{そ}の
\ruby[g]{傷口}{きずぐち}から、
%
\ruby{溢}{こぼ}れる
\原本頁{115-4}\改行%
\ruby[g]{血潮}{ち しほ}が
\ruby{急}{きふ}にやあ
\ruby{止}{と}まらず、
%
\ruby{大望心}{たい|ま|う}が% ルビ調整(原本通り)
\ruby{負}{ま}けりやあ
\ruby{其}{そ}の
\ruby[g]{英氣}{えいき }は、
%
\ruby{未練氣}{み|れん|げ}
\ruby{無}{な}く
\ruby{去}{さ}つて
\ruby[g]{仕舞}{し ま }つて
\ruby{呼}{よ}んでも
\ruby{{\換字{還}}}{かへ}らねえ。
%
つまり
\ruby[g]{何樣}{ど う }なつても
\ruby{根}{ね}が
\ruby{同士討}{ど|し|うち}の、% (どうし)でなく原本通り(どし)
%
\ruby{酷}{ひど}い
\ruby[g]{戰爭}{たゝかひ}に
\ruby[g]{國土}{く に }は
\ruby{荒}{あ}れて、
%
\ruby{{\換字{遺}}}{のこ}るものは
\ruby{怖}{おそ}ろしい
\ruby[g]{心氣}{し ん }の
\ruby[g]{疲勞}{つかれ }!。
%
\ruby[||j>]{櫻}{さくら}
\ruby[||j>]{色}{ いろ}して
% \ruby{櫻色}{さくら|いろ}して
\ruby{居}{ゐ}た
\ruby{面}{かほ}は
\ruby{白}{しら}けて、
%
\ruby{葛}{くず}の
\ruby[g]{葉裏}{は うら}を
\ruby{見}{み}るやうになり、
%
\ruby{眼}{め}は
\ruby{冴}{さ}えなくなる、
%
\ruby[g]{白髮}{しらが }は
さす、
%
\ruby{{\換字{強}}}{つよ}い
\ruby{奴}{やつ}は
\ruby{癇癪持}{かん|しやく|もち}になる。
%
\ruby{{\換字{弱}}}{よわ}い
\ruby{奴}{やつ}は
\ruby{萎縮{\換字{漢}}}{い|ぢけ|もの}になる。
%
\ruby[g]{筋骨}{すぢぼね}は
\ruby{弛}{ゆる}んで
\ruby[g]{仕舞}{し ま }ふ、
%
\ruby{勞苦{\換字{嫌}}}{ほね|をり|ぎら}ひになる。
%
\ruby{其}{そ}の
\ruby{位}{くらゐ}なのは
\ruby{未}{ま}だ
\ruby{可}{い}い
\ruby{{\換字{分}}}{ぶん}で、
%
\ruby[g]{隨{\換字{分}}}{ずゐぶん}
\ruby{怖}{おそ}ろしい
\ruby[g]{病氣}{びやうき}さへも
\原本頁{115-11}\改行%
\ruby[g]{引出}{ひきだ }す。
%
よしんば
\ruby{大望心}{たい|ま|う}と% ルビ調整(原本通り)
\ruby[g]{戀愛}{こ ひ }とが
\ruby[g]{衝突}{ぶつか }らないで、
%
\ruby{腹}{はら}の
\ruby{中}{なか}が
それほどには
\ruby{揉}{も}め
\ruby{無}{な}いでも、
%
\ruby{向}{むこ}ふに
\ruby{的}{まと}の
\ruby{無}{な}い
\ruby{戀}{こひ}は
\ruby{無}{な}いから、
%
\ruby{星}{ほし}に
\原本頁{116-2}\改行%
\ruby{中}{あた}る
\ruby{中}{あた}らぬは
\ruby{時}{とき}の
\ruby{{\換字{運}}}{うん}
\ruby{身}{み}の
\ruby{{\換字{運}}}{うん}!。
%
\ruby[g]{相手}{あひて }と
\ruby{馬}{うま}が
\ruby{合}{あ}ふ
\ruby{合}{あ}はぬもあるし
\改行% 校正作業の簡略化のため
、
%
\原本頁{116-3}\改行%
\ruby[g]{相手}{あひて }とは
\ruby{死}{し}ぬほどに
\ruby{好}{す}き
\ruby{合}{あ}つても、
%
\ruby[g]{自{\換字{分}}}{じ ぶん}たち
ばかりのために
\ruby[g]{出來}{で き }て
\ruby{居}{ゐ}る
\ruby[g]{世界}{せ かい}ぢやあ
\ruby{無}{な}いもの、
%
\ruby{何}{なに}がさて
\ruby[g]{外{\換字{道}}}{げ だう}も
\ruby{居}{ゐ}る、
%
\ruby[g]{惡{\換字{魔}}}{あくま }も
\ruby{居}{ゐ}
\原本頁{116-5}\改行%
る、
%
\ruby{敵}{てき}も
\ruby{居}{ゐ}る、
%
おせつかいも
\ruby{居}{ゐ}る、
%
\ruby[g]{義理}{ぎ り }もある、
%
\ruby[||j>]{人}{にん}
\ruby[||j>]{{\換字{情}}}{じやう}もある、
% \ruby{人{\換字{情}}}{にん|じやう}もある、
%
\原本頁{116-6}\改行%
\ruby{時}{とき}もある、
%
\ruby[g]{場合}{ば あひ}もあつて、% 原文通り「場」
%
\ruby[g]{隨意}{ま ゝ }ならぬ
\ruby{憂}{う}き
\ruby{世}{よ}を
\ruby{泣}{な}くものが
\ruby{多}{おほ}い
\改行% 校正作業の簡略化のため
。
%
\原本頁{116-7}\改行%
\ruby[g]{左樣}{さ う }で
\ruby{無}{な}くつてさへ
\ruby{戀}{こひ}を
\ruby{知}{し}るなあ
\ruby{涙}{なみだ}を
\ruby{知}{し}る
\ruby{始}{はじめ}で、
%
\ruby{氣}{き}が
\ruby{優}{やさ}しくなる、
%
\ruby{脆}{もろ}くなる、
%
\ruby{{\換字{感}}}{かん}じが
\ruby{早}{はや}くなる、
%
\ruby{深}{ふか}くなる、
%
\ruby{何}{なん}でも
\ruby{無}{な}い
\ruby{事}{こと}に
ハツと
\ruby{思}{おも}つたり、
%
\ruby{小}{ちひさ}な
\ruby{事}{こと}を
くよ〳〵と
\ruby{案}{あん}じたり、
%
\ruby[g]{{\換字{前}}表}{ぜんぺう}といふやうな
\ruby{事}{こと}を
\ruby{氣}{き}にしたり、
%
\ruby{何}{なに}かにつけて
\ruby{思}{おも}ひ
\ruby{{\換字{過}}}{すご}しを
\ruby{仕}{し}たり、
%
\ruby{寢}{ね}るべき
\原本頁{116-11}\改行%
\ruby{時}{とき}に
\ruby{寢}{ね}られなかつたりする。
%
そこで
\ruby[g]{段々}{だん〴〵}と
\ruby[g]{心氣}{し ん }が
\ruby{{\換字{弱}}}{よわ}る。
%
\ruby[g]{心氣}{し ん }が
\原本頁{117-1}\改行%
\ruby{{\換字{弱}}}{よわ}りやあ
\ruby[g]{愈々}{いよ〳〵}
\ruby{氣}{き}が
\ruby{脆}{もろ}くなる、
%
\ruby{{\換字{感}}}{かん}じが
\ruby{{\換字{強}}}{つよ}くなる。
%
\ruby{氣}{き}が
\ruby{脆}{もろ}く、
%
\ruby{{\換字{感}}}{かん}じが
\ruby{{\換字{強}}}{つよ}くなりやあ
\ruby{{\換字{又}}}{また}
\ruby[g]{心氣}{し ん }が
\ruby{{\換字{弱}}}{よわ}る。
%
\ruby{雁齒鑢}{がん|ぎ|やすり}が
かゝるやうなものだから
\ruby{堪}{たま}らう
\ruby{譯}{わけ}は
\ruby{無}{な}い。
%
\ruby[g]{一日}{いちにち}
\ruby[g]{一日}{いちにち}に
\ruby{{\換字{弱}}}{よわ}つた
\ruby[g]{擧句}{あげく }は、
%
\ruby[g]{魂魄}{たましひ}が
\ruby[g]{薄手}{うすで }に
なりきつて、
%
\ruby{觸}{さは}るものさへ
あれば
\ruby{砕}{くだ}けたがる
\ruby[g]{玻璃}{びいどろ}か
なんぞのやうになつて
\ruby[g]{仕舞}{し ま }ふ。
%
よく
\ruby[g]{世間}{せ けん}にある
\ruby[g]{戀路}{こひぢ }の
\ruby{果}{はて}の、
%
\ruby{飛}{と}んでも
\ruby{無}{な}い
\ruby{不幸福}{ふ|しあ|はせ}は%「幸福」 ここは「は」
\ruby{皆}{みな}
\ruby[g]{其處}{そ こ }で
\ruby[g]{出來}{で き }る。
%
たとひ
\ruby{{\換字{嫌}}}{きら}はれても
\ruby{{\換字{嫌}}}{きら}はれても、
%
\ruby{好}{す}かれたいのが
\ruby{戀}{こひ}の
\ruby{慾}{よく}で、
%
\ruby{{\換字{又}}}{また}
\ruby{憂}{う}いも
\ruby{辛}{つら}いも
\ruby[g]{堪{\換字{忍}}}{しんばう}して% 原文通り「堪忍」
、
%
\ruby{添}{そ}ひ
\ruby{{\換字{遂}}}{とげ}たいのが
\原本頁{117-8}\改行%
\ruby{戀}{こひ}の
\ruby[g]{意地}{い ぢ }だ。
%
\換字{志}て
\ruby{見}{み}りやあ
\ruby{戀}{こひ}に
\ruby[g]{生命}{いのち }の
\ruby{捨}{す}てやうは
\ruby{無}{な}い、
%
\ruby{戀}{こひ}は
\ruby[g]{生々}{いきいき}と
\ruby{美}{うつく}しいものだ。
%
\ruby{世}{よ}の
\ruby{不幸福}{ふ|しあ|せ}な%「幸福」 ここは「は」欠落
\ruby{人}{ひと}を
\ruby{見}{み}りやあ、
%
\ruby{戀}{こひ}で
\ruby{死}{し}ぬものは
\ruby[g]{一人}{ひとり }も
\ruby{無}{な}く、
%
\ruby[||j>]{皆}{みんな}
\ruby[g]{心氣}{し ん }の
\ruby[g]{疲勞}{つかれ }に
\ruby{堪}{こら}へ
\ruby{切}{き}れ
\ruby{無}{な}くなつて、
%
おのが
\ruby[g]{魂魄}{たましひ}を
\ruby{碎}{くだ}いて
\ruby[g]{仕舞}{し ま }うのだが、
%
\ruby{{\換字{避}}}{さ}けやうにも
\ruby{{\換字{避}}}{さ}け
\ruby{{\換字{難}}}{にく}いのは
\ruby{此}{こ}の
\ruby{隨{\換字{伴}}者}{お|と|も}だから、
%
\ruby{戀}{こひ}は
\ruby[g]{毫末}{ちつと }も
\ruby{怖}{こは}かあ
\ruby{無}{な}いが、
%
\ruby{其}{そ}の
\ruby{隨{\換字{伴}}者}{お|と|も}の
\ruby[g]{心氣}{し ん }の
\ruby[g]{疲勞}{つかれ }
\原本頁{118-2}\改行%
は
\ruby{恐}{おそ}ろしい。
%
\ruby{實}{じつ}を
\ruby{云}{い}やあ
\ruby{僕}{ぼく}が
\ruby{君}{きみ}の
\ruby{事}{こと}を
\ruby{素破拔}{すつ|ぱ|ぬ}いて
\ruby[g]{饒舌}{しやべ }つたから
\改行% 校正作業の簡略化のため
、
%
\原本頁{118-3}\改行%
\ruby[g]{羽{\換字{勝}}}{は がち}も
\ruby[g]{日方}{ひ かた}も
\ruby{君}{きみ}のために、
%
\ruby[g]{二人}{ふたり }とも
\ruby{甚}{ひど}く
\ruby[g]{心配}{しんぱい}して
\ruby{居}{ゐ}る。
%
\ruby{特}{こと}に
\ruby[g]{日方}{ひ かた}は
\ruby{彼}{あ}の
\ruby[g]{氣性}{きしやう}だから、
%
\ruby{{\換字{強}}}{きつ}い
\ruby[g]{意見}{い けん}を
\ruby{云}{い}ひに
\ruby{行}{ゆ}かうかも
\ruby{知}{し}れないが
\改行% 校正作業の簡略化のため
、
%
\原本頁{118-5}\改行%
\ruby[g]{乃公}{お ら }あ
\ruby{何}{なんに}も
\ruby[g]{意見}{い けん}は
\ruby{云}{い}はない。
%
\ruby{何}{なに}も
\ruby{彼}{か}も
\ruby{解}{わか}つて
\ruby{居}{ゐ}る
\ruby{君}{きみ}の
\ruby{事}{こと}だもの
\改行% 校正作業の簡略化のため
、
%
\原本頁{118-6}\改行%
\ruby{君}{きみ}が
\ruby{詰}{つま}ら
\ruby{無}{な}い
\ruby{事}{こと}を
\ruby{仕}{し}やう
\ruby[g]{氣{\換字{遣}}}{き づか}ひは
\ruby{無}{な}いが、
%
たゞ
\ruby[g]{心氣}{し ん }の
\ruby[g]{疲勞}{つかれ }に
\ruby{負}{ま}けぬやうにと、
%
これだけを
\ruby{君}{きみ}に
\ruby{頼}{たの}んで
\ruby{置}{お}く。
%
\ruby{見}{み}りやあ
\ruby[g]{顏色}{かほつき}と
\ruby{云}{い}ひ
\ruby[g]{容態}{ようす }といひ、
%
\ruby[g]{心氣}{し ん }が
\ruby{疲}{つか}れて
\ruby{居}{ゐ}ないやうでも
\ruby{無}{な}い、
%
\ruby{氣}{き}をつけて
\原本頁{118-9}\改行%
\ruby{吳}{く}れ
\ruby{無}{な}くちやあ
いけないぜ。
%
\ruby[g]{何時}{い つ }かは
\ruby{云}{い}はう〳〵と
\ruby{思}{おも}つて
\ruby{居}{ゐ}たので、
%
つい
\ruby{圖}{づ}に
\ruby{乘}{の}つて
\ruby{長}{なが}く
\ruby[g]{饒舌}{しやべ }つて、
%
\ruby[g]{言葉}{ことば }さへ
\ruby[g]{亂暴}{らんばう}に
\ruby{言}{い}ひ
\ruby{{\換字{過}}}{す}ごしたが、
%
\ruby{意}{こゝろ}だけは
\ruby[g]{是非}{ぜ ひ }とも
\ruby{汲}{く}んで
\ruby{吳}{く}れたまへ。
%
\ruby{千言萬言}{せん|げん|ばん|げん}
\ruby[g]{饒舌}{しやべ }つても、
%
\ruby[g]{身體}{からだ }を
\ruby[g]{大切}{たいせつ}に
\ruby{仕}{し}て
\ruby{吳}{く}れろといふ、
%
たゞの
\ruby[g]{一句}{いつく }に
\ruby{止}{とゞ}まるのだ。
%
\ruby{飯}{めし}の
\ruby[g]{不味}{ま づ }い
\ruby{時}{とき}も
\ruby[g]{堪{\換字{忍}}}{が まん}して% 原文通り「堪忍」
\ruby{食}{く}つて、
%
\ruby{成}{な}るたけ
\ruby[g]{精々}{せい〴〵}
\ruby[g]{身體}{からだ }を
\ruby{使}{つか}つて、
%
\ruby{寢}{ね}るべき
\ruby{時}{とき}にやあ
\ruby[g]{整然}{ちやん }と
\ruby{寢}{ね}て、
%
\ruby[||j>]{力}{ちから}
\ruby[||j>]{足}{ あし}を
\ruby{踏}{ふ}んで
\ruby[g]{確乎}{しつかり}と、
%
\ruby[g]{快活}{き さく}に
\ruby{日}{ひ}を
\ruby{{\換字{送}}}{おく}つて
\ruby{貰}{もら}ひたいのだ。
%
\ruby{君}{きみ}の
\ruby{氣}{き}に
\ruby{入}{い}つたほどの
\ruby{人}{ひと}だもの、
%
\原本頁{119-5}\改行%
\ruby{僕}{ぼく}は
\ruby{其}{そ}の
\ruby{人}{ひと}を
\ruby{知}{し}らないが、
%
\ruby[g]{屹度}{きつと }
\ruby{好}{い}い
\ruby{人}{ひと}だらうと
\ruby{思}{おも}つて
\ruby{居}{ゐ}て、
%
\ruby{君}{きみ}の
\ruby[g]{{\換字{運}}命}{う ん }の
\ruby{好}{い}いやうにと
ばかり
\ruby{祈}{いの}つて
\ruby{居}{ゐ}る。
%
\ruby{僕}{ぼく}の
\ruby{力}{ちから}の
\ruby{要}{い}る
\ruby{事}{こと}が
あらば、
%
\ruby{何}{なん}なりと
\ruby{{\換字{遠}}慮無}{ゑん|りよ|な}く
\ruby{云}{い}つて
\ruby{吳}{く}れたまへ、
%
\ruby{君}{きみ}のために
\ruby[g]{幸福}{しあはせ}になる%「幸福」ここは「は」
\ruby{事}{こと}ならば、
%
\ruby[g]{何樣}{ど ん }な
\ruby{事}{こと}を
\ruby{仕}{し}ても
\ruby{僕}{ぼく}は
\ruby{厭}{いと}はない。
%
\ruby{馬}{うま}にでも
\ruby{牛}{うし}にでもなつて
\ruby{働}{はたら}かうが、
%
\ruby{其}{そ}の
\ruby{代}{かは}り
\ruby{今}{いま}
\ruby{言}{い}つた
\ruby{戀}{こひ}の
\ruby{隨{\換字{伴}}者}{お|と|も}にやあ
\ruby{必}{かなら}ず
\原本頁{119-10}\改行%
\ruby{負}{ま}けて
\ruby{吳}{く}れたまふな。
%
\ruby[g]{世界}{せ かい}に
\ruby[g]{人間}{ひ と }は
\ruby{多}{おほ}いけれど、%
{---}{---}%
そりやあ
\原本頁{119-11}\改行%
\ruby{偉}{えら}い
\ruby{人}{ひと}も
\ruby{多}{おほ}からうが、
%
\ruby{此}{こ}の
\ruby[g]{何年}{なんねん}を
\ruby{{\換字{過}}}{すご}して
\ruby{來}{き}た、
%
\ruby{君}{きみ}の
\ruby[g]{行狀}{おこなひ}の
\ruby[<j||]{殊}{しゆ }% 行末行頭の境界付近なので特例処置を施す
\ruby[<j||]{{\換字{勝}}}{しよう}
% \ruby{殊{\換字{勝}}}{しゆ|しよう}
さを
\ruby{見}{み}ては、
%
アヽ、
%
\ruby[g]{眞似}{ま ね }たつて
\ruby[g]{眞似}{ま ね }られない
\ruby{事}{こと}だ、
%
あゝいふ
\ruby[<j||]{男}{をとこ}は
\ruby{今}{いま}の
\ruby{世}{よ}には、
%
\ruby[g]{中々}{なか〳〵}
\ruby[g]{二人}{ふ たり}とは
\ruby{有}{あ}りはすまい、
%
\ruby[g]{島木}{しまき }
\ruby{萬五郎}{まん|ご|らう}は
\ruby[g]{俗物}{ぞくぶつ}だが、
%
\ruby[g]{朋友}{ともだち}にやあ
\ruby[g]{幸福}{しあはせ}にも%「幸福」ここは「は」
\ruby{心}{こゝろ}の
\ruby[g]{氣高}{け だか}い
\ruby[g]{水野}{みづの }のやうな
\ruby{人}{ひと}を
\ruby{持}{も}つて
\原本頁{120-4}\改行%
\ruby{居}{ゐ}ると、
%
\ruby{天}{てん}にも
\ruby{地}{ち}にも
\ruby[||j>]{唯}{たつた}
\ruby[g]{一人}{ひとり}の
\ruby[g]{大切}{たいせつ}な
\ruby[g]{朋友}{ともだち}に
\ruby{思}{おも}つて
\ruby{居}{ゐ}る
\ruby{君}{きみ}の
\ruby{事}{こと}だから、
%
どうか
\ruby[g]{身體}{からだ }を
\ruby[g]{大切}{たいせつ}に
\ruby{仕}{し}て
\ruby{吳}{く}れたまへ、
%
\ruby{君}{きみ}の
\ruby{其}{そ}の
\ruby{顏}{かほ}つきを
\ruby{見}{み}ちやあ
\ruby{案}{あん}じられて
ならない。
%
くどいやうだが
\ruby{今}{いま}
\ruby{言}{い}つた
\ruby{事}{こと}を
\ruby{能}{よ}く
\ruby{聽}{き}いて
\ruby{置}{お}いて
\ruby{吳}{く}れたまへ。
』

\原本頁{120-8}%
と、
%
\ruby[g]{眞{\換字{情}}}{まごゝろ}こめて
\ruby{云}{い}ひ
\ruby{{\換字{終}}}{をは}りたり。
