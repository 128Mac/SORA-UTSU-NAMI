\Entry{其二十}

% メモ 校正終了 2024-04-08
\原本頁{120-10}%
\ruby{磊落}{らい|らく}なれども
\ruby{思{\換字{遣}}}{おもひ|や}りあり、
%
\ruby{粗}{あら}きが
\ruby{如}{ごと}くなれども
\ruby{精細}{こま|か}なるところある
\ruby{島木}{しま|き}が
\ruby{長々}{なが|〳〵}しき
\ruby[||j>]{物}{もの}
\ruby[||j>]{語}{がたり}は、
% \ruby{物語}{もの|がたり}は、
%
わざと
\ruby{我}{わ}が
\ruby{上}{うへ}には
\ruby{貼}{つ}かぬように
\ruby{云}{い}ひたりとは
\ruby{聞}{きこ}えたれど、
%
その
\ruby{言葉}{こと|ば}の
\ruby{中}{うち}の
\ruby{{\換字{節}}々}{ふし|〴〵}には、
%
\ruby{既}{はや}
\ruby{全然}{すつ|かり}と
\ruby{我}{わ}が
\ruby{{\換字{近}}來}{ちか|ごろ}の
\ruby{狀態}{あり|さま}を
\ruby{知}{し}り
\ruby{盡}{つく}くして
\ruby{言}{い}ふと
\ruby{思}{おぼ}しくて、
%
ひし〳〵と
\ruby{身}{み}に
\ruby{徹}{こた}ふる
ところの
\ruby{少}{すくな}からぬに、
%
\ruby{氣息}{い|き}を
さへ
\ruby{潜}{ひそ}めて% 【潛 u6f5b 「先先」】【潜 u6f5c 「夫夫」】併用されている
\ruby{聞}{き}き
\ruby{居}{ゐ}たりし
\ruby{水野}{みづ|の}は、
%
\ruby{胸}{むね}の
\ruby{中}{うち}は
\ruby{石川}{いし|かは}の
\ruby{淸}{きよ}き
\ruby{瀬}{せ}を
\ruby{流}{なが}るゝ
\ruby{水}{みづ}と
\ruby{爽快}{さわや|か}にして、
%
\ruby{底}{そこ}の
\ruby{心}{こゝろ}は
\原本頁{121-6}\改行%
\ruby{春}{はる}と
\ruby[<j>]{溫}{あたゝか}き
\ruby{我}{わ}が
\ruby{友}{とも}が、
%
\ruby{虛僞}{いつ|はり}ならず
\ruby{我}{われ}を
\ruby{思}{おも}ひ
\ruby{吳}{く}るゝ
\ruby{其}{そ}の
\ruby{眞{\換字{情}}}{ま|ごゝろ}に、
%
\ruby{其}{それ}と
\ruby{指}{さ}しては
\ruby{捉}{とら}へ
\ruby{{\換字{難}}}{がた}き
\ruby{香氣}{にほ|ひ}の
\ruby{物}{もの}を
\ruby{罩}{こ}むるが
\ruby{如}{ごと}くに
\ruby{我}{わ}が
\ruby{身心}{しん|〴〵}の
\ruby{全部}{すべ|て}が
\ruby{引}{ひ}き
\ruby{包}{つゝ}まれたるを
\ruby{覺}{おぼ}えて、
%
\ruby{嗚呼}{あ|ゝ}
\ruby{我}{われ}
\ruby{不幸福}{ふ|しあ|はせ}の%「幸福」ここは「は」
\ruby{月日}{つき|ひ}の
\ruby{下}{した}に
\ruby{生}{うま}れて、
%
\原本頁{121-9}\改行%
\ruby{物}{もの}の
\ruby{心}{こゝろ}も
\ruby{知}{し}らぬ
\ruby{頃}{ころ}より、
%
\ruby{{\換字{父}}}{ちゝ}をも
\ruby{母}{はゝ}をも
\ruby{失}{うしな}ひて、
%
\ruby{兄}{あに}も
\ruby{無}{な}ければ
\ruby{姊}{あね}も
\ruby{無}{な}く、
%
\ruby{世}{よ}の
\ruby{剩}{あま}され
\ruby{物}{もの}と
なつて
\ruby{生長}{そ|だ}ちし
まゝ、
%
\ruby{幼}{をさな}き
\ruby{時}{とき}の
\ruby{心}{こゝろ}にも、
%
\原本頁{121-11}\改行%
\ruby{丁稚奉公}{でつ|ち|ぼう|こう}せし
\ruby{家}{いへ}に、
%
\ruby{巢}{す}くひし
\ruby{燕}{つばめ}の
\ruby{親鳥}{おや|どり}の、
%
\ruby{日}{ひ}に
\ruby{百度}{もゝ|たび}も
\ruby{千度}{ち|たび}も
\ruby{飛}{と}んで
\ruby{去}{さ}つては
\ruby{飛}{と}んで
\ruby{{\換字{返}}}{かへ}つて、
%
まだ
\ruby{{\換字{弱}}}{よわ}き
\ruby{雛}{ひな}に
\ruby{餌}{ゑ}を
\ruby{{\換字{運}}}{はこ}ぶを
\ruby{見}{み}て、
%
\ruby{顏}{かほ}も
おぼえぬ
\ruby{吾}{わ}が
\ruby{母}{はゝ}
\ruby{戀}{こひ}しく、
%
\ruby{親}{おや}のある
\ruby{子}{こ}の
\ruby{羨}{うらや}ましさに、
%
\換字{志}く〳〵
\原本頁{122-3}\改行%
\ruby{泣}{な}いたる
\ruby{事}{こと}の
\ruby{記臆}{おぼ|え}さへ、% 原本通り「おぼえ」
%
まざ〳〵と
\ruby{今}{いま}に
\ruby{{\換字{遺}}}{のこ}れるなるが、
%
それには
\原本頁{122-4}\改行%
\ruby{引換}{ひき|か}へて
\ruby{幸{\換字{運}}}{しあ|はせ}にも、
%
アヽ%「幸運」ここは「は」
\ruby{我}{われ}
\ruby{何}{なん}の
\ruby{福}{ふく}のあつてか、
%
\ruby{自然}{し|ぜん}
\g詰めruby{々々}{〳〵}に
\ruby{知}{し}り
\原本頁{122-5}\改行%
\ruby{合}{あ}つたる
\ruby{六人}{ろく|にん}の
\ruby{良}{よ}き
\ruby{友}{とも}の
\ruby{其}{そ}の
\ruby{中}{うち}にも、
%
\ruby{{\換字{分}}}{わ}けて
\ruby{親}{した}しき
\ruby{羽{\換字{勝}}}{は|がち}
\ruby{島木}{しま|き}、
%
\原本頁{122-6}\改行%
\ruby{特}{こと}に
\ruby{島木}{しま|き}が
\ruby{眼}{ま}の
\ruby{{\換字{前}}}{あたり}の
\ruby{友{\換字{情}}}{なさ|け}!。
%
お
\ruby{澤}{さは}
\ruby[||j>]{婆}{ばゞあ}の
\ruby{言葉}{こと|ば}の
\ruby{{\換字{通}}}{とほ}り、
%
\ruby{手}{て}をついて
\原本頁{122-7}\改行%
\ruby{頼}{たの}んだつて
\ruby{芋塊}{い|も}
\ruby{一}{ひと}つも、
%
\ruby{自然}{ひと|りで}には
\ruby{出}{で}て
\ruby{來}{こ}ない
\ruby{此}{こ}の
\ruby{世}{よ}の
\ruby{中}{なか}に、
%
いづれ
\ruby{身}{み}の
\ruby[||j>]{油}{あぶら}
\ruby[||j>]{汗}{ あせ}が
% \ruby{油汗}{あぶら|あせ}が
\ruby{化}{ば}けたに
\ruby{{\換字{違}}}{ちが}ひ
\ruby{無}{な}い
\ruby{多額}{おほ|く}の
\ruby{金子}{か|ね}をも、
%
\ruby{紙}{かみ}の
\ruby{一枚}{いち|まい}でも
\ruby{吳}{く}れるやうに、
%
\ruby{惜}{をし}む
\ruby{色}{いろ}さへ
\ruby{無}{な}く
\ruby[<j>]{快}{こゝろよ}く
\ruby{吳}{く}れて、
\換字{志}かも
\ruby{君}{きみ}の
ためになる
\ruby{事}{こと}ならば、
%
\ruby{馬}{うま}にでも
\ruby{牛}{うし}にでもなつて
\ruby{働}{はたら}いて
\ruby{{\換字{遣}}}{や}らうと、
%
\ruby{身}{み}を
\ruby{入}{い}れて
\ruby{吳}{く}れる
\ruby{其}{そ}の
\ruby{俠氣}{をとこ|ぎ}!。
%
\ruby{人世}{うき|よ}の
\ruby{場数}{ば|かず}を% 原文通り「場」
\ruby{踏}{ふ}んで
\ruby{來}{き}た
\ruby{人}{ひと}には、
%
\原本頁{123-1}\改行%
\ruby{隨{\換字{分}}}{ずゐ|ぶん}
\ruby{幼稚}{こ|ども}にも
\ruby[||j>]{{\換字{若}}}{じやく}
\ruby[||j>]{輩}{ はい}にも
% \ruby{{\換字{若}}輩}{じやく|はい}にも
\ruby{思}{おも}はれようか
\ruby{知}{し}れぬ
\ruby{事}{こと}なるに、
%
\ruby{我}{わ}が
\ruby{{\換字{情}}緖}{おも|ひ}の
\ruby{上}{うへ}に
\ruby{就}{つ}いては
\ruby{咎}{とが}め
\ruby{立}{だ}てもせず、
%
\ruby{年齡}{と|し}の
\ruby{{\換字{所}}爲}{せ|ゐ}にして
\ruby{仕舞}{し|ま}つて
\ruby{一}{ひ}
ト
\原本頁{123-3}\改行%
\ruby{言}{こと}も
\ruby{云}{い}はぬ
\ruby{寛大}{おほ|やう}さ!。
%
たゞ
\ruby{身體}{から|だ}を
\ruby{大切}{だい|じ}に
\ruby{仕}{し}て
\ruby{吳}{く}れろと
\ruby{云}{い}つて
\ruby{吳}{く}れる
\ruby{其}{そ}の
\ruby{親切}{しん|せつ}!。
%
\ruby{嗚呼}{あ|ゝ}
\ruby{兄}{あに}と
\ruby{云}{い}はうか、
%
\ruby{姊}{あね}と
\ruby{云}{い}はうか、
%
\ruby{兄}{あに}も
\ruby{姊}{あね}も
\原本頁{123-5}\改行%
\ruby{中々}{なか|〳〵}
かうばかりはあるまい。
%
まして
\ruby{朋友}{とも|だち}と
\ruby{云}{い}はうには
\ruby{勿體無}{もつ|たい|な}いほど。
%
\ruby{人}{ひと}に
\ruby{云}{い}はれぬ
\ruby{苦悶}{く|るし}みを
\ruby{抱}{いだ}けば、
%
\ruby{何}{なに}につけ
\ruby{彼}{か}につけて
\ruby{此}{こ}の
\原本頁{123-7}\改行%
\ruby{世}{よ}の
\ruby{中}{なか}を、
%
\ruby{味氣}{あぢ|き}
\ruby{無}{な}く
\ruby{思}{おも}ふ
\ruby{時}{とき}のみ
\ruby{此頃}{この|ごろ}は
\ruby{多}{おほ}かりしが、
%
あゝ
\ruby{有}{あ}り
\ruby{{\換字{難}}}{がた}き
\ruby{天}{てん}の
\ruby{恩惠}{めぐ|み}、
%
\ruby{水野}{みづ|の}
\ruby{靜十郎}{せい|じう|らう}
\ruby{幸福}{さい|はひ}にして、%「幸福」ここは「は」
%
かゝる
\ruby{信義}{しん|ぎ}の
\ruby{友}{とも}にも
\ruby{未}{ま}だ
\原本頁{123-9}\改行%
\ruby{棄}{す}てられねば、
%
アヽ
\ruby{思}{おも}へば
\ruby{我}{われ}は
\ruby{世}{よ}にも
\ruby{稀}{まれ}なる
\ruby{幸{\換字{運}}}{しあ|はせ}を%「幸運」ここは「は」
\ruby{受}{う}け
\ruby{得}{え}たる
\原本頁{123-10}\改行%
\ruby{身}{み}なるかな、
%
\ruby{我}{わ}が
\ruby{行末}{ゆく|すゑ}も
\ruby{光}{ひかり}ありて、
%
\ruby{{\換字{強}}}{あなが}ち
\ruby{黑闇}{や|み}のみならず
\ruby{見}{み}ゆ、
%
と
\ruby{悅}{よろこ}ぶにも
\ruby{先}{ま}づ
\ruby{涙}{なみだ}にて、
%
\ruby{謝}{しや}する
\ruby{言葉}{こと|ば}も
たど〳〵しく、

\原本頁{124-1}%
『アヽ
\ruby{島木}{しま|き}
\ruby{君}{くん}、
%
\ruby{感謝}{かん|しや}する。
%
\ruby{免}{ゆる}して
\ruby{吳}{く}れたまへ、
%
\ruby{僕}{ぼく}は
\ruby{何}{なん}にも
\ruby{言}{い}ふことが
\ruby{出來無}{で|き|な}い。
%
\ruby{言}{い}ひたい
\ruby[||j>]{{\換字{情}}}{こゝろ}
\ruby[||j>]{懷}{ もち}は
% \ruby{{\換字{情}}懷}{こゝろ|もち}は
\ruby{澤山}{たん|と}あるが
\ruby{胸}{むね}が
\ruby{張}{は}つて
\ruby{居}{ゐ}て
\原本頁{124-3}\改行%
\ruby{何}{なん}にも
\ruby{言}{い}へない。
%
\ruby{實}{じつ}に
\g詰めruby{々々}{〳〵}
\ruby{君}{きみ}の
\ruby{親切}{しん|せつ}は
\ruby{深}{ふか}く
\ruby{謝}{しや}する。
%
\ruby{君}{きみ}の
\ruby{談}{はなし}は
\ruby{骨}{ほね}に
\ruby{浸}{し}みて
\ruby{解}{わか}つた。
%
\ruby{決}{けつ}して
\ruby{忘}{わす}れ
\ruby{無}{な}い、
%
\ruby{忘}{わす}れ
\ruby{無}{な}い!。
%
\ruby{成程}{なる|ほど}
\ruby{何}{なん}に
\ruby{卷}{ま}き
\原本頁{124-5}\改行%
\ruby{倒}{たふ}されては
\ruby{濟}{す}まない
\ruby{身體}{から|だ}だ!。
%
\ruby{僕}{ぼく}も
\ruby{果敢}{は|か}ない
\ruby{思}{おもひ}に
\ruby{死}{し}にたかあ
\ruby{無}{な}い!。
%
いや
\ruby{僕}{ぼく}は
\ruby{何樣}{ど|う}
まかり
\ruby{間{\換字{違}}}{ま|ちが}つても
\ruby{脆}{もろ}くは
\ruby{死}{し}なゝい!。
%
\ruby{戀{\換字{情}}}{じよ|う}は
\ruby{戀{\換字{情}}}{じよ|う}だけれど、
%
\ruby{大望心}{たい|ま|う}は
\ruby{大望心}{たい|ま|う}だ!。
%
\ruby{身體}{から|だ}も
\ruby{必}{かなら}ず
\ruby{大切}{たい|せつ}にする。
』

\原本頁{124-8}%
と、
%
\ruby{{\換字{強}}}{しひ}て
\ruby{勉}{つと}めて
\ruby{答}{こた}へたり。

\原本頁{124-9}%
\ruby{夜}{よ}は
\ruby{彼}{かれ}
\ruby{一句}{いつ|く}
\ruby{此}{これ}
\ruby{一句}{いつ|く}の
\ruby{二人}{ふた|り}が
\ruby{親}{した}しき
\ruby[||j>]{物}{もの}
\ruby[||j>]{語}{がたり}に
% \ruby{物語}{もの|がたり}に
\ruby{漸}{やうや}く
\ruby{盡}{つ}きて、
%
\ruby{早}{はや}くも
\ruby{暁天}{あ|け}
\ruby{{\換字{近}}}{ちか}く
ならんとすれば、
%
\ruby{水野}{みづ|の}は
\ruby{{\換字{終}}}{つひ}に
\ruby{島木}{しま|き}が
\ruby{許}{もと}を
\ruby{辭}{じ}して、
%
\ruby{{\換字{情}}中}{ふと|ころ}に
\ruby{阿堵物}{も||の}あるに% 「阿堵物(あとぶつ)」お金のこと
\ruby{勢}{いきほ}ひ
\ruby{好}{よ}く、
%
\ruby{紫色}{むら|さき}
\ruby{立}{だ}てる
\ruby{天}{そら}の
\ruby{星}{ほし}
\ruby{薄}{うす}れ
\ruby{行}{ゆ}きて
\ruby{{\換字{朝}}風}{あさ|かぜ}の
\ruby{徐々}{おも|むろ}に
\ruby{吹}{ふ}き
\ruby{出}{だ}す
\ruby{頃}{ころ}、
%
\ruby{相良}{さが|ら}が
\ruby{家}{いへ}を
\ruby{敲}{たゝ}き
\ruby{起}{おこ}して
\ruby{昨日}{きの|ふ}の
\ruby{恩}{おん}を
\ruby{謝}{しや}し、
%
\ruby{{\換字{猶}}}{なほ}
\ruby{信頼}{た|の}むに
\ruby{足}{た}るべき
\ruby{看護{\換字{婦}}}{かん|ご|ふ}を
\ruby{世話}{せ|わ}せん
ことを
\ruby{乞}{こ}ひ
\ruby{求}{もと}めて、
%
\ruby{其}{そ}の
\ruby[<j>]{快}{こゝろよ}く
\ruby{諾}{うけが}ひ
\ruby{吳}{く}れたるに
\ruby[||j>]{心}{こゝろ}
\ruby[||j>]{勇}{ いさ}み、
% \ruby{心勇}{こゝろ|いさ}み、
%
\ruby{足}{あし}
\ruby{輕}{かろ}く
\ruby{歸路}{かへ|り}を
\ruby{急}{いそ}ぎて、
%
\ruby{淺草}{あさ|くさ}の
\ruby{雷神門{\換字{前}}}{かみ|なり|もん|まへ}に
さしかゝりぬ。
