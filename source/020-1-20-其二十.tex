\Entry{其二十}

% メモ 校正終了 2024-04-08 2024-05-25 2024-06-18
\原本頁{120-10}%
\ruby[g]{磊落}{らいらく}なれども
\ruby[g]{思{\換字{遣}}}{おもひや}りあり、
%
\ruby{粗}{あら}きが
\ruby{如}{ごと}くなれども
\ruby[g]{精細}{こまか }なるところある
\ruby[g]{島木}{しまき }が
\ruby[g]{長々}{なが〳〵}しき
\ruby[||j>]{物}{もの}
\ruby[||j>]{語}{がたり}は、
% \ruby{物語}{もの|がたり}は、
%
わざと
\ruby{我}{わ}が
\ruby{上}{うへ}には
\ruby{貼}{つ}かぬように
\ruby{云}{い}ひたりとは
\ruby{聞}{きこ}えたれど、
%
その
\ruby[g]{言葉}{ことば }の
\ruby{中}{うち}の
\ruby[g]{{\換字{節}}々}{ふし〴〵}には、
%
\ruby{既}{はや}
\ruby[g]{全然}{すつかり}と
\ruby{我}{わ}が
\ruby[g]{{\換字{近}}來}{ちかごろ}の
\ruby[g]{狀態}{ありさま}を
\ruby{知}{し}り
\ruby{盡}{つく}して
\ruby{言}{い}ふと
\ruby{思}{おぼ}しくて、
%
ひし〳〵と
\ruby{身}{み}に
\ruby{徹}{こた}ふる
ところの
\ruby{少}{すくな}からぬに、
%
\ruby[g]{氣息}{い き }を
さへ
\ruby{潜}{ひそ}めて% 【潛 u6f5b 「先先」】【潜 u6f5c 「夫夫」】併用されている
\ruby{聞}{き}き
\ruby{居}{ゐ}たりし
\ruby[g]{水野}{みづの }は、
%
\ruby{胸}{むね}の
\ruby{中}{うち}は
\ruby[g]{石川}{いしかは}の
\ruby{淸}{きよ}き
\ruby{瀬}{せ}を
\ruby{流}{なが}るゝ
\ruby{水}{みづ}と
\ruby[g]{爽快}{さわやか}にして、
% 和歌「 いし河や せみの小川の 清き瀬も 水音たかし 五月雨の頃」 宗中詠之書之
%  (いし河や せみの小川)→石川や瀬見の小川=賀茂川の異名
%
\ruby{底}{そこ}の
\ruby{心}{こゝろ}は
\原本頁{121-6}\改行%
\ruby{春}{はる}と
\ruby[<j>]{溫}{あたゝか}き
\ruby{我}{わ}が
\ruby{友}{とも}が、
%
\ruby[g]{虛僞}{いつはり}ならず
\ruby{我}{われ}を
\ruby{思}{おも}ひ
\ruby{吳}{く}るゝ
\ruby{其}{そ}の
\ruby[g]{眞{\換字{情}}}{まごゝろ}に、
%
\ruby{其}{それ}と
\ruby{指}{さ}しては
\ruby{捉}{とら}へ
\ruby{{\換字{難}}}{がた}き
\ruby[g]{香氣}{に ほひ}の
\ruby{物}{もの}を
\ruby{罩}{こ}むるが
\ruby{如}{ごと}くに
\ruby{我}{わ}が
\ruby[g]{身心}{しん〴〵}の
\ruby[g]{全部}{すべて }
\原本頁{121-8}\改行%
が
\ruby{引}{ひ}き
\ruby{包}{つゝ}まれたるを
\ruby{覺}{おぼ}えて、
%
\ruby[g]{嗚呼}{あ ゝ }
\ruby{我}{われ}
\ruby{不幸福}{ふ|しあ|はせ}の%「幸福」ここは「は」
\ruby[g]{月日}{つきひ }の
\ruby{下}{した}に
\ruby{生}{うま}れて
\改行% 校正作業の簡略化のため
、
%
\原本頁{121-9}\改行%
\ruby{物}{もの}の
\ruby{心}{こゝろ}も
\ruby{知}{し}らぬ
\ruby{頃}{ころ}より、
%
\ruby{{\換字{父}}}{ちゝ}をも
\ruby{母}{はゝ}をも
\ruby{失}{うしな}ひて、
%
\ruby{兄}{あに}も
\ruby{無}{な}ければ
\ruby{姊}{あね}
\原本頁{121-10}\改行%
も
\ruby{無}{な}く、
%
\ruby{世}{よ}の
\ruby{剩}{あま}され
\ruby{物}{もの}と
なつて
\ruby[g]{生長}{そ だ }ちし
まゝ、
%
\ruby{幼}{をさな}き
\ruby{時}{とき}の
\ruby{心}{こゝろ}にも
\改行% 校正作業の簡略化のため
、
%
\原本頁{121-11}\改行%
\ruby{丁稚奉公}{でつ|ち|ぼう|こう}せし
\ruby{家}{いへ}に、
%
\ruby{巢}{す}くひし
\ruby{燕}{つばめ}の
\ruby[g]{親鳥}{おやどり}の、
%
\ruby{日}{ひ}に
\ruby[g]{百度}{もゝたび}も
\ruby[g]{千度}{ち たび}も
\ruby{飛}{と}んで
\ruby{去}{さ}つては
\ruby{飛}{と}んで
\ruby{{\換字{返}}}{かへ}つて、
%
まだ
\ruby{{\換字{弱}}}{よわ}き
\ruby{雛}{ひな}に
\ruby{餌}{ゑ}を
\ruby{{\換字{運}}}{はこ}ぶを
\ruby{見}{み}て、
%
\ruby{顏}{かほ}も
おぼえぬ
\ruby{吾}{わ}が
\ruby{母}{はゝ}
\ruby{戀}{こひ}しく、
%
\ruby{親}{おや}のある
\ruby{子}{こ}の
\ruby{羨}{うらや}ましさに、
%
\換字{志}く〳〵
\原本頁{122-3}\改行%
\ruby{泣}{ない}たる
\ruby{事}{こと}の
\ruby[g]{記臆}{おぼえ }さへ、% 原本通り「おぼえ」
%
まざ〳〵と
\ruby{今}{いま}に
\ruby{{\換字{遺}}}{のこ}れるなるが、
%
それには
\原本頁{122-4}\改行%
\ruby[g]{引換}{ひきか }へて
\ruby[g]{幸{\換字{運}}}{しあはせ}にも、
%
アヽ%「幸運」ここは「は」
\ruby{我}{われ}
\ruby{何}{なん}の
\ruby{福}{ふく}のあつてか、
%
\makeatletter
\@ifundefined{デバッグ@ビルド}{%
  \ruby[g]{自然}{し ぜん}
  \ruby[g]{自然}{し ぜん}に
}{%
  \ruby[g]{自然}{し ぜん}
  \ruby[g]{々々}{ 〳〵 }に
}%
\makeatother
\ruby{知}{し}り
\原本頁{122-5}\改行%
\ruby{合}{あ}つたる
\ruby[g]{六人}{ろくにん}の
\ruby{良}{よ}き
\ruby{友}{とも}の
\ruby{其}{そ}の
\ruby{中}{うち}にも、
%
\ruby{{\換字{分}}}{わ}けて
\ruby{親}{した}しき
\ruby[g]{羽{\換字{勝}}}{は がち}
\ruby[g]{島木}{しまき }、
%
\原本頁{122-6}\改行%
\ruby{特}{こと}に
\ruby[g]{島木}{しまき }が
\ruby{眼}{ま}の
\ruby{{\換字{前}}}{あたり}の
\ruby[g]{友{\換字{情}}}{なさけ }!。
%
お
\ruby{澤}{さは}
\ruby[||j>]{婆}{ばゞあ}の
\ruby[g]{言葉}{ことば }の
\ruby{{\換字{通}}}{とほ}り、
%
\ruby{手}{て}をついて
\原本頁{122-7}\改行%
\ruby{頼}{たの}んだつて
\ruby[g]{芋塊}{い も }
\ruby{一}{ひと}つも、
%
\ruby[g]{自然}{ひとりで}には
\ruby{出}{で}て
\ruby{來}{こ}ない
\ruby{此}{こ}の
\ruby{世}{よ}の
\ruby{中}{なか}に、
%
いづれ
\ruby{身}{み}の
\ruby[||j>]{油}{あぶら}
\ruby[||j>]{汗}{ あせ}が
% \ruby{油汗}{あぶら|あせ}が
\ruby{化}{ば}けたに
\ruby{{\換字{違}}}{ちが}ひ
\ruby{無}{な}い
\ruby[g]{多額}{おほく }の
\ruby[g]{金子}{か ね }をも、
%
\ruby{紙}{かみ}の
\ruby[g]{一枚}{いちまい}でも
\ruby{吳}{く}れるやうに、
%
\ruby{惜}{をし}む
\ruby{色}{いろ}さへ
\ruby{無}{な}く
\ruby[<j>]{快}{こゝろよ}く
\ruby{吳}{く}れて、
\換字{志}かも
\ruby{君}{きみ}の
ためになる
\ruby{事}{こと}ならば、
%
\ruby{馬}{うま}にでも
\ruby{牛}{うし}にでもなつて
\ruby{働}{はたら}いて
\ruby{{\換字{遣}}}{や}らうと、
%
\ruby{身}{み}を
\ruby{入}{い}れて
\ruby{吳}{く}れる
\ruby{其}{そ}の
\ruby[g]{俠氣}{をとこぎ}!。
%
\ruby[g]{人世}{うきよ }の
\ruby[g]{場数}{ば かず}を% 原文通り「場」
\ruby{踏}{ふ}んで
\ruby{來}{き}た
\ruby{人}{ひと}には、
%
\原本頁{123-1}\改行%
\ruby[g]{隨{\換字{分}}}{ずゐぶん}
\ruby[g]{幼稚}{こども }にも
\ruby[||j>]{{\換字{若}}}{じやく}
\ruby[||j>]{輩}{ はい}にも
% \ruby{{\換字{若}}輩}{じやく|はい}にも
\ruby{思}{おも}はれようか
\ruby{知}{し}れぬ
\ruby{事}{こと}なるに、
%
\ruby{我}{わ}が
\ruby[g]{{\換字{情}}緖}{おもひ }の
\ruby{上}{うへ}に
\ruby{就}{つ}いては
\ruby{咎}{とが}め
\ruby{立}{だて}もせず、
%
\ruby[g]{年齡}{と し }の
\ruby[g]{{\換字{所}}爲}{せ ゐ }にして
\ruby[g]{仕舞}{し ま }つて
\ruby{一}{ひ}
ト
\原本頁{123-3}\改行%
\ruby{言}{こと}も
\ruby{云}{い}はぬ
\ruby[g]{寛大}{おほやう}さ!。
%
たゞ
\ruby[g]{身體}{からだ }を
\ruby[g]{大切}{だいじ }に
\ruby{仕}{し}て
\ruby{吳}{く}れろと
\ruby{云}{い}つて
\ruby{吳}{く}れる
\ruby{其}{そ}の
\ruby[g]{親切}{しんせつ}!。
%
\ruby[g]{嗚呼}{あ ゝ }
\ruby{兄}{あに}と
\ruby{云}{い}はうか、
%
\ruby{姊}{あね}と
\ruby{云}{い}はうか、
%
\ruby{兄}{あに}も
\ruby{姊}{あね}も
\原本頁{123-5}\改行%
\ruby[g]{中々}{なか〳〵}
かうばかりはあるまい。
%
まして
\ruby[g]{朋友}{ともだち}と
\ruby{云}{い}はうには
\ruby{勿體無}{もつ|たい|な}いほど。
%
\ruby{人}{ひと}に
\ruby{云}{い}はれぬ
\ruby[g]{苦悶}{くるし }みを
\ruby{抱}{いた}けば、
%
\ruby{何}{なに}につけ
\ruby{彼}{か}につけて
\ruby{此}{こ}の
\原本頁{123-7}\改行%
\ruby{世}{よ}の
\ruby{中}{なか}を、
%
\ruby[g]{味氣}{あぢき }
\ruby{無}{な}く
\ruby{思}{おも}ふ
\ruby{時}{とき}のみ
\ruby[g]{此頃}{このごろ}は
\ruby{多}{おほ}かりしが、
%
あゝ
\ruby{有}{あ}り
\ruby{{\換字{難}}}{がた}き
\ruby{天}{てん}の
\ruby[g]{恩惠}{めぐみ }、
%
\ruby[g]{水野}{みづの }
\ruby{靜十郎}{せい|じふ|らう}
\ruby[g]{幸福}{さいはひ}にして、%「幸福」ここは「は」
%
かゝる
\ruby[g]{信義}{しんぎ }の
\ruby{友}{とも}にも
\ruby{未}{ま}だ
\原本頁{123-9}\改行%
\ruby{棄}{す}てられねば、
%
アヽ
\ruby{思}{おも}へば
\ruby{我}{われ}は
\ruby{世}{よ}にも
\ruby{稀}{まれ}なる
\ruby[g]{幸{\換字{運}}}{しあはせ}を%「幸運」ここは「は」
\ruby{受}{う}け
\ruby{得}{え}たる
\原本頁{123-10}\改行%
\ruby{身}{み}なるかな、
%
\ruby{我}{わ}が
\ruby[g]{行末}{ゆくすゑ}も
\ruby{光}{ひかり}ありて、
%
\ruby{{\換字{強}}}{あなが}ち
\ruby[g]{黑闇}{や み }のみならず
\ruby{見}{み}ゆ、
%
と
\ruby{悅}{よろこ}ぶにも
\ruby{先}{ま}づ
\ruby{涙}{なみだ}にて、
%
\ruby{謝}{しや}する
\ruby[g]{言葉}{ことば }も
たど〳〵しく、

\原本頁{124-1}%
『
アヽ
\ruby[g]{島木}{しまき }
\ruby{君}{くん}、
%
\ruby[g]{{\換字{感}}謝}{かんしや}する。
%
\ruby{免}{ゆる}して
\ruby{吳}{く}れたまへ、
%
\ruby{僕}{ぼく}は
\ruby{何}{なん}にも
\ruby{言}{い}ふことが
\ruby{出來無}{で|き|な}い。
%
\ruby{言}{い}ひたい
\ruby[||j>]{{\換字{情}}}{こゝろ}
\ruby[||j>]{懷}{ もち}は
% \ruby{{\換字{情}}懷}{こゝろ|もち}は
\ruby[g]{澤山}{たんと }
あるが、
\ruby{胸}{むね}が
\ruby{張}{は}つて
\ruby{居}{ゐ}て
\原本頁{124-3}\改行%
\ruby{何}{なん}にも
\ruby{言}{い}へない。
%
\ruby{實}{じつ}に
\ruby[g]{々々}{ 〳〵 }
\ruby{君}{きみ}の
\ruby[g]{親切}{しんせつ}は
\ruby{深}{ふか}く
\ruby{謝}{しや}する。
%
\ruby{君}{きみ}の
\ruby{談}{はなし}は
\ruby{骨}{ほね}に
\ruby{浸}{し}みて
\ruby{解}{わか}つた。
%
\ruby{決}{けつ}して
\ruby{忘}{わす}れ
\ruby{無}{な}い、
%
\ruby{忘}{わす}れ
ない!。
%
\ruby[g]{成程}{なるほど}
\ruby{何}{なん}に
\ruby{卷}{ま}き
\原本頁{124-5}\改行%
\ruby{倒}{たふ}されては
\ruby{濟}{す}まない
\ruby[g]{身體}{からだ }だ!。
%
\ruby{僕}{ぼく}も
\ruby[g]{果敢}{は か }ない
\ruby{思}{おもひ}に
\ruby{死}{し}にたかあ
\ruby{無}{な}い!。
%
いや
\ruby{僕}{ぼく}は
\ruby[g]{何樣}{ど う }
まかり
\ruby[g]{間{\換字{違}}}{ま ちが}つても
\ruby{脆}{もろ}くは
\ruby{死}{し}なゝい!。
%
\ruby[g]{戀{\換字{情}}}{じよう }
\原本頁{124-7}\改行%
は
\ruby[g]{戀{\換字{情}}}{じよう }
だけれど、
%
\ruby{大望心}{たい|ま|う}は
\ruby{大望心}{たい|ま|う}だ!。
%
\ruby[g]{身體}{からだ }も
\ruby{必}{かなら}ず
\ruby[g]{大切}{たいせつ}にする
\改行% 校正作業の簡略化のため
。
』

\原本頁{124-8}%
と、
%
\ruby{{\換字{強}}}{しひ}て
\ruby{勉}{つと}めて
\ruby{答}{こた}へたり。

\原本頁{124-9}%
\ruby{夜}{よ}は
\ruby{彼}{かれ}
\ruby[g]{一句}{いつく }
\ruby{此}{これ}
\ruby[g]{一句}{いつく }の
\ruby[g]{二人}{ふたり }が
\ruby{親}{した}しき
\ruby[||j>]{物}{もの}
\ruby[||j>]{語}{がたり}に
% \ruby{物語}{もの|がたり}に
\ruby{漸}{やうや}く
\ruby{盡}{つ}きて、
%
\ruby{早}{はや}くも
\ruby[g]{暁天}{あ け }
\ruby{{\換字{近}}}{ちか}く
ならんとすれば、
%
\ruby[g]{水野}{みづの }は
\ruby{{\換字{終}}}{つひ}に
\ruby[g]{島木}{しまき }が
\ruby{許}{もと}を
\ruby{辭}{じ}して、
%
\ruby[g]{{\換字{情}}中}{ふところ}に
\ruby[|g|]{阿堵物}{もの}あるに% 「阿堵物(あとぶつ)」お金のこと
\ruby{勢}{いきほ}ひ
\ruby{好}{よ}く、
%
\ruby[g]{紫色}{むらさき}
\ruby{立}{だ}てる
\ruby{天}{そら}の
\ruby{星}{ほし}
\ruby{薄}{うす}れ
\ruby{行}{ゆ}きて
\ruby[g]{{\換字{朝}}風}{あさかぜ}の
\ruby[g]{徐徐}{おもむろ}に% ルビ調整(原本通り)非踊り字表記(行末行頭の境界付近)
\ruby{吹}{ふ}き
\ruby{出}{だ}す
\ruby{頃}{ころ}、
%
\ruby[g]{相良}{さがら }が
\ruby{家}{いへ}を
\ruby{敲}{たゝ}き
\ruby{起}{おこ}して
\ruby[g]{昨日}{きのふ }の
\ruby{恩}{おん}を
\ruby{謝}{しや}し、
%
\ruby{{\換字{猶}}}{なほ}
\ruby[g]{信頼}{た の }むに
\ruby{足}{た}るべき
\ruby{看護{\換字{婦}}}{かん|ご|ふ}を
\ruby[g]{世話}{せ わ }せん
ことを
\ruby{乞}{こ}ひ
\ruby{求}{もと}めて、
%
\ruby{其}{そ}の
\ruby[<j>]{快}{こゝろよ}く
\ruby{諾}{うけが}ひ
\ruby{吳}{く}れたるに
\ruby[||j>]{心}{こゝろ}
\ruby[||j>]{勇}{ いさ}み、
% \ruby{心勇}{こゝろ|いさ}み、
%
\ruby{足}{あし}
\ruby{輕}{かろ}く
\ruby[g]{歸路}{かへり }を
\ruby{急}{いそ}ぎて、
%
\ruby[g]{淺草}{あさくさ}の
\ruby{雷神門{\換字{前}}}{かみ|なり|もん|まへ}に
さしかゝりぬ。
