\Entry{其二十一}

% メモ 校正終了 2024-04-08 2024-05-25 2024-06-18
\原本頁{125-6}%
おもふ
\ruby{人}{ひと}の
\ruby{病}{やまひ}は
\ruby{篤}{あつ}けれども、
%
\ruby{思}{おも}ひし
\ruby{事}{こと}は
\ruby{皆}{みな}
\ruby{爲}{な}し
\ruby{得}{え}たり、
%
\ruby[g]{相良}{さがら }も
\原本頁{125-7}\改行%
\ruby{今}{いま}
\ruby[g]{一度}{いちど }
\ruby[g]{見舞}{み ま }ひて
\ruby[g]{尾竹}{を たけ}に
あひて
\ruby[g]{種々}{くさ〴〵}の
\ruby[||j>]{心}{こゝろ}
\ruby[||j>]{添}{ ぞへ}をも
% \ruby{心添}{こゝろ|ぞへ}をも
なし
\ruby{置}{お}かんと
\ruby{云}{い}ひ
\改行% 校正作業の簡略化のため
、
%
\原本頁{125-8}\改行%
\ruby{良}{よ}き
\ruby{看護{\換字{婦}}}{かん|ご|ふ}をも
\ruby{晝}{ひる}までとは
\ruby{{\換字{過}}}{すご}さず
\ruby{四ツ木}{よ| |ぎ}に
\ruby{{\換字{遣}}}{や}り
\ruby{吳}{く}るゝ
\ruby[g]{手筈}{て はず}に
\ruby{定}{さだ}まりたり、
%
この
\ruby{上}{うへ}は
たゞ
\ruby{健}{まめ}やかなる
\ruby[||j>]{婢}{をんな}
\ruby[||j>]{一人}{ ひと|り}を
\ruby{看護{\換字{婦}}}{かん|ご|ふ}の
\ruby[g]{指揮}{さしづ }の
\ruby{下}{しも}につけて
\ruby[g]{雜事}{ざつじ }に
\ruby{當}{あた}らすれば、
%
もとより
\ruby[g]{介抱}{かいはう}の
\ruby[g]{此上}{このうへ}
\ruby{無}{な}く
\ruby[g]{行屆}{ゆきとゞ}きて% 「屆」「届」 原本通り「屆」
\原本頁{126-1}\改行%
\ruby{善}{ぜん}を
\ruby{盡}{つく}したりと
\ruby{云}{い}ふべきには
あらねど、
%
\ruby{今}{いま}の
\ruby{身}{み}にての
\ruby{我}{わ}が
\ruby{心}{こゝろ}の
\原本頁{126-2}\改行%
\ruby{及}{およ}ぶほどだけは
\ruby{盡}{つく}したるなり、
%
と
\ruby{思}{おも}ふにつけて
\ruby{人}{ひと}
\ruby{知}{し}らず
\ruby{樂}{たの}しく
\改行% 校正作業の簡略化のため
、
%
\原本頁{126-3}\改行%
\ruby[||j>]{愁}{うれひ}の
\ruby{中}{なか}にも
\ruby{幽}{かすか}なる
\ruby{笑}{ゑみ}の
\ruby{催}{もよほ}さるゝ
\ruby[g]{心地}{こゝち }して、
%
\ruby{願}{ねが}はくは
\ruby{我}{わ}が
\ruby{五十子}{い|そ|こ}の
\ruby{病}{やまひ}の
\ruby{漸}{やうや}く
\ruby{痊}{おこた}りて、
%
\ruby[||j>]{心}{こゝろ}
\ruby[||j>]{盡}{ づく}しの
% \ruby{心盡}{こゝろ|づく}しの
\ruby[g]{甲{\換字{斐}}}{か ひ }も
あれかし、
%
\ruby{暴}{あら}き
\ruby{雨}{あめ}
\ruby{風}{かぜ}に
\ruby{根}{ね}を
\原本頁{126-5}\改行%
\ruby{搖}{ゆら}がされて
\ruby[g]{敢無}{あへな }くも
\ruby[g]{天壽}{いのち }ならず
\ruby{枯}{か}れんとする
\ruby{樹}{き}を、
%
おぼつか
\ruby{無}{な}きながら
\ruby{支}{さゝ}へ
\ruby{培}{つちか}ひて、
%
\ruby{復}{ふたゝ}び
\ruby{花}{はな}
\ruby{{\換字{咲}}}{さ}く
\ruby{春}{はる}の
\ruby{曉}{あした}に、
%
\ruby[g]{丹誠}{たんせい}の
\ruby[g]{甲{\換字{斐}}}{か ひ }ありて
\原本頁{126-7}\改行%
\ruby[||j>]{美}{うつく}しく
\ruby{日}{ひ}に
\ruby{匂}{にほ}ふを
\ruby{見}{み}ば、
%
\ruby[g]{如何}{い か }ばかりか
\ruby{心}{こゝろ}の
\ruby{嬉}{うれ}しからん、
%
それにつけても
\ruby[g]{昨日}{きのふ }よりの
\ruby{長}{なが}き
\ruby{夜}{よ}
\ruby[g]{一夜}{ひとよ }を、
%
\ruby{我}{わ}が
\ruby{五十子}{い|そ|こ}は
\ruby[g]{如何}{い か }なる
\ruby[g]{狀態}{やうす }に
\ruby{{\換字{送}}}{おく}りたらん、
%
\ruby{熱}{ねつ}の
\ruby{烈}{はげ}しく
\ruby{{\換字{進}}}{さ}すことは
\ruby{無}{な}かりしか、
%
\ruby{{\換字{強}}}{つよ}く
\ruby{苦}{くるし}む
\ruby{事}{こと}は
\ruby{無}{な}かりしか、
%
ともすれば
\ruby[g]{心臓}{しんざう}
\ruby[g]{肺臓}{はいざう}の
\ruby{此}{こ}の
\ruby{病}{やまひ}には
\ruby{惡}{あし}くなるものと
\ruby{聞}{き}きたるが、
\ruby[g]{其等}{それら }の
\ruby{凶}{あし}きことは
\ruby{無}{な}かりし
\ruby{歟}{か}、
%
\ruby[g]{尾竹}{を たけ}も
\ruby[<g>]{親切}{しんせつ}の
\makeatletter
\@ifundefined{デバッグ@ビルド}{%
  \ruby{男}{をとこ}
}{%
  \ruby[<j||]{男}{をとこ}% 行末行頭の境界付近なので特例処置を施す
  \原本頁{127-1}\改行%
}%
\makeatother
なれば、
%
\ruby[g]{容態}{ようだい}
\ruby{惡}{あし}くば
\ruby{附}{つ}きゝりに
\ruby{附}{つ}きても
\ruby{居}{ゐ}ては
\ruby{吳}{く}れたるべけれど、
%
\ruby{氷}{こほり}より
\ruby{冷}{つめた}い
\ruby{心}{こゝろ}の
\ruby{彼}{あ}の
お
\ruby{澤}{さは}
\ruby{婆}{ばゞ}、
%
くれ〴〵も
\ruby{頼}{たの}み
\ruby{置}{お}きたる
\makeatletter
\@ifundefined{デバッグ@ビルド}{%
  \ruby[||j>]{氷}{ひよう}
  \ruby[||j>]{嚢}{ なう}
}{%
  \ruby[<j||]{氷}{ひよう}
  \ruby[<j||]{嚢}{なう}
  \原本頁{127-3}\改行%
}%
\makeatother
の
% \ruby{氷嚢}{ひよう|なう}の
\ruby[g]{世話}{せ わ }さへ、
%
\ruby{既}{すで}に
\ruby[|g|]{一昨日}{をとゝひ}
といひ
\ruby[g]{昨日}{きのふ }と
\ruby{云}{い}ひ、
%
\ruby{碌}{ろく}に
\ruby{身}{み}に
\ruby{染}{し}みても
\原本頁{127-4}\改行%
\ruby{爲}{し}て
\ruby{吳}{く}れざりし、
%
あゝいふ
\ruby{不}{ふ}
\ruby[<j>]{幸}{しあはせ}の%「不幸」ここは「は」
\ruby{處}{ところ}に
\ruby[g]{居合}{ゐ あ }はせたる
\makeatletter
\@ifundefined{デバッグ@ビルド}{%
  \ruby[<j||]{病}{びやう}
  \ruby[<j||]{人}{にん}の、
}{%
  \ruby[||j>]{病}{びやう}
  \ruby[||j>]{人}{ にん}の、
}%
\makeatother
% \ruby{病人}{びやう|にん}の、
%
\ruby{思}{おも}へば
\ruby[g]{一夜}{ひとよ }が
\ruby[g]{氣{\換字{遣}}}{き づか}はるゝ、
%
と
\ruby[g]{偶然}{ふ と }
\ruby[g]{思念}{おもひ }の
\ruby[g]{其處}{そ こ }に
\ruby[g]{片荷}{かたに }づゝては
\ruby{矢}{や}も
\原本頁{127-6}\改行%
\ruby{楯}{たて}も
\ruby{堪}{たま}らず、
%
\ruby[g]{物淋}{ものさび}しく
\ruby[g]{薄暗}{うすくら}き
\ruby{離}{はな}れ
\ruby{屋}{や}の
\ruby{中}{うち}の、
%
\ruby[g]{孤燈}{こ とう}
\ruby[||j>]{力}{ちから}
\ruby[||j>]{無}{ な}く% ルビ調整(原本通り)
\ruby{照}{て}らす
\原本頁{127-7}\改行%
\ruby[||j>]{光}{ひかり}の
\ruby{下}{もと}に、
%
\ruby[g]{頭髮}{か み }は
\ruby[g]{亂菊}{らんぎく}の
\ruby[g]{花瓣}{はなびら}の% 弁 (瓣) 辦 辧 辨 辩 辯
\ruby{霜}{しも}に
\ruby{傷}{いた}める
\ruby{姿}{すがた}と
\ruby{崩}{くづ}れて、
%
\ruby{悶}{もだ}え
\ruby{悶}{もだ}えつゝ
\ruby[g]{埒無}{らちな }く
\ruby{病}{や}み
\ruby{臥}{ふ}せる
\ruby{態}{さま}の、
%
\ruby{眼}{め}の
\ruby{{\換字{前}}}{まへ}に
あり〳〵と
\ruby{{\換字{浮}}}{うか}み
\ruby{來}{く}るやう
\ruby{覺}{おぼ}えて、
%
\ruby[g]{島木}{しまき }が
\ruby{寓}{やど}を
\ruby{敲}{たゝ}きたりし
\ruby{折}{をり}、
%
\ruby{頭}{かうべ}を
\ruby{反}{かへ}して
\ruby[g]{偶然}{ふ と }
\ruby{見}{み}し
\ruby{北}{きた}の
\原本頁{127-10}\改行%
\ruby{{\換字{空}}}{そら}に、
%
\ruby{大}{おほき}なる
\ruby{美}{うつく}しき
\ruby{星}{ほし}の
\ruby[g]{長々}{なが〳〵}と
\ruby{光}{ひかり}を
\ruby{曳}{ひ}いて
\ruby{流}{なが}れて
\ruby{{\換字{消}}}{き}えしも、
%
\ruby{思}{おも}ひ
\ruby{合}{あは}されて
\ruby[g]{今{\換字{更}}}{いまさら}
\ruby{急}{きふ}に
\ruby{何}{なん}と
\ruby{無}{な}く
\ruby{忌}{いま}はしく、
%
おもはず
\ruby[g]{慄然}{りつぜん}として
\ruby{天}{てん}を
\ruby{偸}{ぬす}み
\ruby{見}{み}たり。

\原本頁{128-2}%
\ruby{天}{そら}は
\ruby{今}{いま}
\ruby{白}{しら}み
わたりて
\ruby{靜}{しづか}に、
%
\ruby[g]{星辰}{ほ し }は
\ruby{潛}{ひそ}みつ、% 【潛 u6f5b 「先先」】【潜 u6f5c 「夫夫」】併用されている
%
\ruby[g]{瑠璃}{る り }の
\ruby[||j>]{盤}{ばん}
\ruby[||j>]{上}{じやう}に
% \ruby{盤上}{ばん|じやう}に
\ruby[g]{金砂}{きんしや}を
\原本頁{128-3}\改行%
\ruby{撒}{ま}きし
\ruby{數時間{\換字{前}}}{すう|じ|かん|まへ}の
\ruby[g]{光景}{ありさま}は
\ruby{痕}{あと}も
\ruby{無}{な}く
\ruby{{\換字{消}}}{き}え
\ruby{去}{さ}つて、
%
また
ありし
おもかげを
\ruby{{\換字{忍}}}{しの}ぶべくも
あらぬ
\ruby{狀}{さま}なるに、
%
おのづと
\ruby{新}{あたら}しき
\ruby[g]{淸旦}{あした }の
\ruby{氣}{き}を
\原本頁{128-5}\改行%
\ruby{受}{う}けて
\ruby{胸}{むね}も
\ruby{開}{ひら}き、
%
アヽ
\ruby[g]{{\換字{前}}表}{ぜんぺう}と
いふやうなる
\ruby{事}{こと}を
\ruby{氣}{き}に
\ruby{仕}{し}たる
\makeatletter
\@ifundefined{デバッグ@ビルド}{%
  \ruby[<g>]{愚さ}{おろか }、
}{%
  \ruby{愚}{おろか}さ
  \改行% 校正作業の簡略化のため
  、
}%
\makeatother
%
\原本頁{128-6}\改行%
\ruby[g]{島木}{しまき }の
\ruby[g]{言葉}{ことば }にも
\ruby{羞}{はづ}か
\換字{志}かりし、
%
と
\ruby{私}{ひそか}に
\ruby{自}{みづか}ら
\ruby[g]{女々}{め ゝ }しきを
\ruby{慚}{は}ぢたり
\改行% 校正作業の簡略化のため
。
%
\原本頁{128-8}\改行%
されど
\ruby{心}{こゝろ}は
\ruby[g]{一度}{ひとたび}
\ruby{動}{うご}きて
\ruby{復}{また}
\ruby{安}{やす}まらず。
%
\ruby{曉}{あした}に
\ruby{{\換字{消}}}{き}えし
\ruby{星}{ほし}は
\ruby[g]{再度}{ふたゝび}
\ruby[||j>]{夕}{ゆふべ}に
\ruby{見}{み}るべけれども、
%
\ruby[g]{一度}{ひとたび}
\ruby{去}{さ}つては
\ruby{行}{ゆ}く
\ruby{方}{かた}
\ruby{知}{し}れぬ
\ruby{人}{ひと}の
\ruby{身}{み}の、
%
\ruby[g]{死生}{し せい}の
\ruby[g]{抑々}{そも〳〵}
\ruby{何}{なに}に
\ruby{繫}{かゝ}りて、
%
\ruby[g]{禍福}{くわふく}の
\ruby[g]{將{\換字{又}}}{はたまた}
\ruby{何}{なに}に
\ruby{本}{もと}づくかも
\ruby{{\換字{分}}}{わか}らぬ
\ruby[g]{茫々}{ばう〳〵}たる
\ruby[g]{劫{\換字{運}}}{ごふうん}の
\ruby{測}{はか}り
\ruby{{\換字{難}}}{がた}く
\ruby{窺}{うかゞ}ひ
\ruby{{\換字{難}}}{がた}きに
\ruby{思}{おも}ひ
\ruby{到}{いた}りては、
%
あゝ
\ruby{頼}{たの}まれぬ
\ruby{人}{ひと}の
\ruby{世}{よ}なるかな、
%
\ruby{我}{わ}が
\ruby{心}{こゝろ}の
\ruby{膏}{あぶら}を
\ruby{燃}{も}やし、
%
\ruby{骨}{ほね}の
\ruby{髓}{ずゐ}を% u9ad3 骨 左 月 辶
\ruby{焚}{た}きて、
%
\ruby[g]{願望}{ねがひ }は
\ruby{大}{おほい}ならぬ
\原本頁{129-1}\改行%
\ruby{我}{わ}が
\ruby{身}{み}の
\ruby[g]{周圍}{まはり }に、
%
\ruby{聊}{いさゝ}かの
\ruby[g]{光明}{ひかり }を
\ruby{得}{え}んと
\ruby{願}{ねが}ふも、
%
\ruby[g]{{\換字{運}}命}{うんめい}の
\ruby{風}{かぜ}の
\ruby[g]{容赦}{ようしや}
\原本頁{129-2}\改行%
\ruby{無}{な}く
\ruby{吹}{ふ}き
\ruby{荒}{すさ}まんには、
%
\ruby{頼}{たの}む
\ruby{影}{かげ}なき
\ruby[g]{裸火}{はだかび}の、
%
\ruby{脆}{もろ}くも
\ruby{忽}{たちま}ち
\ruby{吹}{ふ}き
\ruby{滅}{け}さ
\原本頁{129-3}\改行%
れて、
%
\ruby[g]{天地}{てんち }は
\ruby[||j>]{{\換字{情}}}{なさけ}
\ruby[||j>]{無}{ な}き
\ruby{闇}{やみ}と
なるべし。
%
おもへば
\ruby{小}{ちひさ}きは
\ruby{人}{ひと}の
\ruby{力}{ちから}なり
\改行% 校正作業の簡略化のため
。
%
\原本頁{129-4}\改行%
かほどに
\ruby{身}{み}を
\ruby{勞}{つか}らせ
\ruby{心}{こゝろ}を
\ruby{盡}{つく}して、
%
\ruby{我}{わ}が
\ruby{思}{おも}ふ
\ruby{人}{ひと}
\ruby{好}{よ}かれと
\ruby{我}{われ}は
\ruby{願}{ねが}へど、
%
\ruby[g]{慈悲}{なさけ }
\ruby{有}{あ}りや
\ruby{無}{な}しやも
おぼつかなき、
%
\ruby[g]{{\換字{運}}命}{うんめい}と
いふものゝ
\makeatletter
\@ifundefined{デバッグ@ビルド}{%
  \ruby[||j>]{意}{こゝろ}
  \ruby[||j>]{任}{ まか}せ!、
}{%
  \ruby[<j||]{意}{こゝろ}
  \ruby{任}{まか}せ!、
}%
\makeatother
%
\ruby{其}{そ}の
\ruby{意}{こゝろ}が
\ruby[g]{人{\換字{情}}}{なさけ }を
\ruby{知}{し}つて
\ruby{吳}{く}れうでも
\ruby{無}{な}ければ、
%
\ruby{思}{おも}へば〳〵
\原本頁{129-7}\改行%
\ruby{悲}{かな}しきは
\ruby{人}{ひと}の
\ruby{世}{よ}!。
%
\ruby[g]{{\換字{平}}生}{ひごろ }は% ルビ調整(原本通り)
\ruby{天}{そら}
\ruby{{\換字{翔}}}{か}ける
\ruby{事}{こと}も
\ruby{爲}{な}さば
\ruby{爲}{な}すべき
\ruby[g]{雄心}{をごゝろ}
\ruby{持}{も}
\原本頁{129-8}\改行%
ちし
\ruby{我}{われ}なりしが、
%
\ruby{身}{み}に
\ruby{染}{し}みて
\ruby{今}{いま}ぞ
\ruby[g]{人間}{にんげん}の
\ruby[g]{甲{\換字{斐}}}{か ひ }
\ruby{無}{な}きを
\ruby{知}{し}りつる!
\改行% 校正作業の簡略化のため
。
%
\原本頁{129-9}\改行%
\ruby{天}{てん}は
\ruby{限}{かぎ}り
\ruby{無}{な}く
\ruby{大}{おほい}なるに、
%
\ruby{我}{われ}は
\ruby[g]{糠星}{ぬかぼし}の
\ruby{其}{それ}より
\ruby{微}{かす}けく、
%
\ruby{地}{ち}は
\ruby{涯}{はて}も
\ruby{無}{な}く
\ruby{廣}{ひろ}やかなるに、
%
\ruby{身}{み}は
\ruby{塵}{ちり}
\ruby{土}{ひぢ}と
\ruby{小}{ちひさ}なる、
%
\ruby{此}{こ}の
\ruby[<g>]{某甲}{なにがし}が
\ruby{懷}{いだ}ける
\ruby{念}{おもひ}の、
%
\原本頁{129-11}\改行%
\ruby[g]{{\換字{運}}命}{うんめい}に
\ruby{對}{むか}へる
\ruby{其}{そ}の
\ruby[g]{眞態}{ありさま}は、
%
\ruby{譬}{たと}へば
\ruby[g]{一縷}{いちる }の
\ruby{細}{ほそ}き〳〵、
%
\ruby{毛}{け}の
\ruby{如}{ごと}く
\ruby[g]{蜘蛛}{く も }の
\ruby{圍}{ゐ}の
ごとき
\ruby{絲}{いと}を、
%
\ruby{千萬馬力}{せん|まん|ば|りき}もて
\makeatletter
\@ifundefined{デバッグ@ビルド}{%
  \ruby[<g>]{轟き}{とゞろ }
}{%
  \ruby{轟}{とゞろ}き
}%
\makeatother
\ruby{{\換字{廻}}}{まは}れる
\ruby{大車輪}{だい|しや|りん}に
\ruby{繫}{か}けて
\改行% 校正作業の簡略化のため
、
%
\原本頁{130-2}\改行%
\ruby{其}{そ}の
\ruby[g]{車輪}{しやりん}の
\ruby{我}{わ}が
\ruby{願}{ねが}ふ
\ruby{方}{かた}に
\ruby{{\換字{廻}}}{まは}らんことを、
%
\ruby{竊}{ひそか}に
\ruby{願}{ねが}ひ
\ruby{求}{もと}むるが
\ruby{如}{ごと}し
\改行% 校正作業の簡略化のため
。
%
\原本頁{130-3}\改行%
\ruby[g]{嗚呼}{あ ゝ }、
%
\ruby{我}{わ}が
\ruby{願}{ねが}ひの
\ruby{聽}{き}かるべき
\ruby[<g>]{や\換字{?!}}{}。% 「?!」が行頭にならないよう特殊処理
%
\ruby[||j>]{心}{こゝろ}
\ruby[||j>]{細}{ ぼそ}くも
% \ruby{心細}{こゝろ|ぼそ}くも
また
\ruby[||j>]{心}{こゝろ}
\ruby[||j>]{細}{ ぼそ}くて、
% \ruby{心細}{こゝろ|ぼそ}くて、
%
\makeatletter
\@ifundefined{デバッグ@ビルド}{%
  \ruby[g]{{\換字{情}}無}{なさけな}くも
}{%
  \ruby[<j||]{{\換字{情}}}{なさけ}% 行末行頭の境界付近なので特例処置を施す
  \原本頁{130-4}\改行%
  \ruby{無}{な}くも
}%
\makeatother
\ruby{物}{もの}のみの
\ruby{思}{おも}はるゝ
\ruby{世}{よ}かな!。
%
\ruby{我}{わ}が
\ruby[g]{智慧}{ち ゑ }の
\ruby{今}{いま}
\ruby{効}{かひ}
\ruby{無}{な}きを
\ruby{知}{し}り
\改行% 校正作業の簡略化のため
、
%
\原本頁{130-5}\改行%
\ruby{我}{わ}が
\ruby[g]{意念}{おもひ }の
\ruby{今}{いま}
\ruby[g]{孱{\換字{弱}}}{か よわ}きを
\ruby{知}{し}り、
%
\ruby{斷}{た}えぬ
\ruby{泉}{いづみ}と
\ruby{湧}{わ}き
\ruby{上}{あが}る
\ruby{戀}{こひ}の
\ruby{誠}{まこと}に
\ruby{洗}{あら}はれて、
%
\ruby{心}{こゝろ}は
\ruby[g]{無垢}{む く }の
\ruby[g]{往時}{むかし }に
\ruby{{\換字{返}}}{かへ}りぬ。
%
アヽ
\ruby{今}{いま}
\ruby{我}{われ}は
\ruby[g]{嬰兒}{みどりご}なり!。
%
\ruby[g]{天地}{てんち }の
\ruby[g]{那處}{いづく }に
\ruby[g]{慈母}{は ゝ }の
\ruby[g]{御坐}{お は }
\ruby[<g>]{す\換字{?!}}{}。% 「?!」が行頭にならないよう特殊処理
%
\ruby{泣}{な}きて
\ruby{呼}{よ}び
\ruby{度}{た}き
\ruby[g]{心地}{こゝち }ぞする。
%
と
\ruby[g]{曉天}{あかつき}の
\ruby{{\換字{猶}}}{なほ}
\ruby[g]{靜寂}{しづか }にして
\ruby{人}{ひと}の
\ruby{{\換字{通}}}{とほ}りも
\ruby[g]{稀少}{まばら }なるに、
%
\ruby{深}{ふか}くも
\ruby{心}{こゝろ}の
\ruby{奧}{おく}に
\ruby{思}{おも}ひ
\ruby{入}{い}つたる
\ruby[g]{水野}{みづの }は、
%
ふつと
\ruby{我}{われ}に
\ruby{{\換字{返}}}{かへ}つて
\ruby{頭}{かうべ}を
\ruby{擡}{あ}ぐれば、
%
\ruby{身}{み}は
\ruby[g]{何時}{い つ }の
\ruby{程}{ほど}にか
\ruby{來}{きた}りけん、
%
\ruby[g]{塵埃}{ち り }
\ruby{無}{な}き
\ruby{{\換字{朝}}}{あした}の
\ruby{露}{つゆ}けき
\ruby[g]{石路}{せきろ }の、
%
\ruby[g]{長々}{なが〳〵}しきを
\ruby{知}{し}らぬ
\原本頁{130-11}\改行%
\ruby{間}{ま}に
\ruby{{\換字{過}}}{す}ぎて、
%
\ruby{今}{いま}や
\makeatletter
\@ifundefined{デバッグ@ビルド}{%
  \ruby[<g>]{淺草寺}{せんさうじ }の
}{%
  \ruby{淺草寺}{せん|さう|じ}の
}%
\makeatother
\ruby[g]{山門}{さんもん}を、
%
\ruby{既}{すで}に
\ruby{{\換字{半}}}{なかば}は
\ruby{潛}{くゞ}り% 【潛 u6f5b 「先先」】【潜 u6f5c 「夫夫」】併用されている
\ruby{居}{ゐ}たり。

\原本頁{131-1}%
\ruby[g]{晝間}{ひ る }は
\ruby{賑}{にぎ}やかなる
\ruby[g]{中店}{なかみせ}も、
%
\ruby{{\換字{猶}}}{なほ}
\ruby{寂々}{じやく|〳〵}として
\ruby{物}{もの}の
\ruby{響}{ひゞき}を
\ruby{傳}{つた}へず、
%
\ruby[g]{御{\換字{扉}}}{みとびら}を
\ruby{今}{いま}
\ruby{開}{ひら}きしばかりの、
%
\ruby[g]{御堂}{み だう}の
\ruby{内}{うち}は
\ruby[g]{仄暗}{ほのぐら}きに、
%
\ruby{御燈明}{み|あか|し}の
\ruby[g]{煌々}{きら〳〵}と
\makeatletter
\@ifundefined{デバッグ@ビルド}{%
  \ruby[<g>]{黄金色}{こがねいろ}
}{%
  \ruby{黄}{こ}
  \原本頁{131-3}\改行%
  \ruby{金}{がね}
  \ruby{色}{いろ}
}%
\makeatother
に
\ruby{見}{み}えて、
%
\ruby[g]{{\換字{朝}}{\換字{勤}}}{あさづと}めの
\ruby[g]{讀經}{どきやう}の
\ruby{聲}{こゑ}は
\ruby[||j>]{殊}{しゆ}
\ruby[||j>]{{\換字{勝}}}{しよう}に
% \ruby{殊{\換字{勝}}}{しゆ|しよう}に
\ruby{澄}{す}み
\ruby{渡}{わた}り、
%
\ruby[g]{御堂}{み だう}の
\ruby[<j||]{甍}{いらか}
\原本頁{131-4}\改行%
は
\ruby{天}{そら}に
\ruby{聳}{そび}えて、
%
そこ
\ruby[g]{此處}{こ ゝ }に
\ruby{立}{た}てる
\ruby[g]{老樹}{おいき }の
\ruby[g]{銀杏}{い てふ}は、
%
まだ
\ruby{下}{お}り
\ruby{立}{た}たぬ
\ruby[g]{鳩雞}{はととり}を
\ruby{宿}{やど}して、
%
\ruby{睡}{ねむ}れるが
\ruby{如}{ごと}く
\ruby{靜}{しづか}に
\ruby{秋}{あき}の
\ruby{曙}{あした}の
\ruby{色}{いろ}を
\ruby{見}{み}せたり。

\原本頁{131-6}%
\ruby[g]{水野}{みづの }は
あはれにも
\ruby{頭}{かうべ}を
\ruby{下}{さ}げて、
%
かつて
\ruby{拜}{をが}みしことなき
\ruby{觀世音菩薩}{くわ|んぜ|おん|ぼ|さつ}を、
%
\ruby[g]{此日}{このひ }
はじめて
\ruby{涙}{なみだ}の
\ruby{眼}{め}を
\ruby{閉}{と}ぢ、
%
\ruby[g]{一心}{いつしん}に
\ruby{拜}{をが}み
\ruby[<j>]{奉}{たてまつ}りたり。
