\Entry{其二十一}

% メモ 校正終了 2024-04-08
\原本頁{125-6}%
おもふ
\ruby{人}{ひと}の
\ruby{病}{やまひ}は
\ruby{篤}{あつ}けれども、
%
\ruby{思}{おも}ひし
\ruby{事}{こと}は
\ruby{皆}{みな}
\ruby{爲}{な}し
\ruby{得}{え}たり、
%
\ruby{相良}{さが|ら}も
\原本頁{125-7}\改行%
\ruby{今}{いま}
\ruby{一度}{いち|ど}
\ruby{見舞}{み|ま}ひて
\ruby{尾竹}{を|たけ}に
あひて
\ruby{種々}{くさ|〴〵}の
\ruby{心添}{こゝろ|ぞへ}をも
なし
\ruby{置}{お}かんと
\ruby{云}{い}ひ、
%
\原本頁{125-8}\改行%
\ruby{良}{よ}き
\ruby{看護{\換字{婦}}}{かん|ご|ふ}をも
\ruby{晝}{ひる}までとは
\ruby{{\換字{過}}}{すご}さず
\ruby[g]{四ッ木}{よ ぎ}に% TODO 四ツ木
\ruby{{\換字{遣}}}{や}り
\ruby{吳}{く}るゝ
\ruby{手筈}{て|はず}に
\ruby{定}{さだ}まりたり、
%
この
\ruby{上}{うへ}は
たゞ
\ruby{健}{まめ}やかなる
\ruby[<j||]{婢}{をんな}
\ruby{一人}{ひと|り}を
\ruby{看護{\換字{婦}}}{かん|ご|ふ}の
\ruby{指揮}{さし|づ}の
\ruby{下}{しも}につけて
\ruby{雜事}{ざつ|じ}に
\ruby{當}{あた}らすれば、
%
もとより
\ruby{介抱}{かい|はう}の
\ruby{此上}{この|うへ}
\ruby{無}{な}く
\ruby{行屆}{ゆき|とゞ}きて% 「屆」「届」 原本通り「屆」
\原本頁{126-1}\改行%
\ruby{善}{ぜん}を
\ruby{盡}{つく}したりと
\ruby{云}{い}ふべきには
あらねど、
%
\ruby{今}{いま}の
\ruby{身}{み}にての
\ruby{我}{わ}が
\ruby{心}{こゝろ}の
\原本頁{126-2}\改行%
\ruby{及}{およ}ぶほどだけは
\ruby{盡}{つく}したるなり、
%
と
\ruby{思}{おも}ふにつけて
\ruby{人}{ひと}
\ruby{知}{し}らず
\ruby{樂}{たの}しく、
%
\原本頁{126-3}\改行%
\ruby[||j>]{愁}{うれひ}の
\ruby{中}{なか}にも
\ruby{幽}{かすか}なる
\ruby{笑}{ゑみ}の
\ruby{催}{もよほ}さるゝ
\ruby{心地}{こゝ|ち}して、
%
\ruby{願}{ねが}はくは
\ruby{我}{わ}が
\ruby[g]{五十子}{いそこ}の
\ruby{病}{やまひ}の
\ruby{漸}{やうや}く
\ruby{痊}{おこた}りて、
%
\ruby{心盡}{こゝろ|づく}しの
\ruby{甲{\換字{斐}}}{か|ひ}も
あれかし、
%
\ruby{暴}{あら}き
\ruby{雨}{あめ}
\ruby{風}{かぜ}に
\ruby{根}{ね}を
\原本頁{126-5}\改行%
\ruby{搖}{ゆら}がされて
\ruby{敢無}{あへ|な}くも
\ruby{天壽}{いの|ち}ならず
\ruby{枯}{か}れんとする
\ruby{樹}{き}を、
%
おぼつか
\ruby{無}{な}きながら
\ruby{支}{さゝ}へ
\ruby{培}{つちか}ひて、
%
\ruby{復}{ふたゝ}び
\ruby{花}{はな}
\ruby{{\換字{咲}}}{さ}く
\ruby{春}{はる}の
\ruby{曉}{あした}に、
%
\ruby{丹誠}{たん|せい}の
\ruby{甲{\換字{斐}}}{か|ひ}ありて
\原本頁{126-7}\改行%
\ruby[||j>]{美}{うつく}しく
\ruby{日}{ひ}に
\ruby{匂}{にほ}ふを
\ruby{見}{み}ば、
%
\ruby{如何}{い|か}ばかりか
\ruby{心}{こゝろ}の
\ruby{嬉}{うれ}しからん、
%
それにつけても
\ruby{昨日}{きの|ふ}よりの
\ruby{長}{なが}き
\ruby{夜}{よ}
\ruby{一夜}{ひと|よ}を、
%
\ruby{我}{わ}が
\ruby[g]{五十子}{いそこ}は
\ruby{如何}{い|か}なる
\ruby{狀態}{やう|す}に
\ruby{{\換字{送}}}{おく}りたらん、
%
\ruby{熱}{ねつ}の
\ruby{烈}{はげ}しく
\ruby{{\換字{進}}}{さ}すことは
\ruby{無}{な}かりしか、
%
\ruby{{\換字{強}}}{つよ}く
\ruby{苦}{くるし}む
\ruby{事}{こと}は
\ruby{無}{な}かりしか、
%
ともすれば
\ruby{心臓}{しん|ざう}
\ruby{肺臓}{はい|ざう}の
\ruby{此}{こ}の
\ruby{病}{やまひ}には
\ruby{惡}{あし}くなるものと
\ruby{聞}{き}きたるが、
\ruby{其等}{それ|ら}の
\ruby{凶}{あし}きことは
\ruby{無}{な}かりし
\ruby{歟}{か}、
%
\ruby{尾竹}{を|たけ}も
\ruby{親切}{しん|せつ}の
\ruby{男}{をとこ}なれば、
%
\ruby{容態}{よう|だい}
\ruby{惡}{あし}くば
\ruby{附}{つ}きゝりに
\ruby{附}{つ}きても
\ruby{居}{ゐ}ては
\ruby{吳}{く}れたるべけれど、
%
\ruby{氷}{こほり}より
\ruby{冷}{つめた}い
\ruby{心}{こゝろ}の
\ruby{彼}{あ}の
お
\ruby{澤}{さは}
\ruby{婆}{ばゞ}、
%
くれ〴〵も
\ruby{頼}{たの}み
\ruby{置}{お}きたる
\ruby{氷嚢}{ひよう|なう}の
\ruby{世話}{せ|わ}さへ、
%
\ruby{既}{すで}に
\ruby[g]{一昨日}{をとゝひ}といひ
\ruby{昨日}{きの|ふ}と
\ruby{云}{い}ひ、
%
\ruby{碌}{ろく}に
\ruby{身}{み}に
\ruby{染}{し}みても
\原本頁{127-4}\改行%
\ruby{爲}{し}て
\ruby{吳}{く}れざりし、
%
あゝいふ
\ruby{不幸}{ふ|しあはせ}の%「不幸」ここは「は」
\ruby{處}{ところ}に
\ruby{居合}{ゐ|あ}はせたる
\ruby{病人}{びやう|にん}の、
%
\ruby{思}{おも}へば
\ruby{一夜}{ひと|よ}が
\ruby{氣{\換字{遣}}}{き|づか}はるゝ、
%
と
\ruby{偶然}{ふ|と}
\ruby{思念}{おも|ひ}の
\ruby{其處}{そ|こ}に
\ruby{片荷}{かた|に}づゝては
\ruby{矢}{や}も
\原本頁{127-6}\改行%
\ruby{楯}{たて}も
\ruby{堪}{たま}らず、
%
\ruby{物淋}{もの|さび}しく
\ruby{薄暗}{うす|くら}き
\ruby{離}{はな}れ
\ruby{屋}{や}の
\ruby{中}{うち}の、
%
\ruby{孤燈}{こ|とう}
\ruby[<j|]{力}{ちから}
\ruby{無}{な}く
\ruby{照}{て}らす
\原本頁{127-7}\改行%
\ruby[||j>]{光}{ひかり}の
\ruby{下}{もと}に、
%
\ruby{頭髮}{か|み}は
\ruby{亂菊}{らん|ぎく}の
\ruby{花瓣}{はな|びら}の% 弁 (瓣) 辦 辧 辨 辩 辯
\ruby{霜}{しも}に
\ruby{傷}{いた}める
\ruby{姿}{すがた}と
\ruby{崩}{くづ}れて、
%
\ruby{悶}{もだ}え
\ruby{悶}{もだ}えつゝ
\ruby{埒無}{らち|な}く
\ruby{病}{や}み
\ruby{臥}{ふ}せる
\ruby{態}{さま}の、
%
\ruby{眼}{め}の
\ruby{{\換字{前}}}{まへ}に
あり〳〵と
\ruby{{\換字{浮}}}{うか}み
\ruby{來}{く}るやう
\ruby{覺}{おぼ}えて、
%
\ruby{島木}{しま|き}が
\ruby{寓}{やど}を
\ruby{敲}{たゝ}きたりし
\ruby{折}{をり}、
%
\ruby{頭}{かうべ}を
\ruby{反}{かへ}して
\ruby{偶然}{ふ|と}
\ruby{見}{み}し
\ruby{北}{きた}の
\原本頁{127-10}\改行%
\ruby{{\換字{空}}}{そら}に、
%
\ruby{大}{おほき}なる
\ruby{美}{うつく}しき
\ruby{星}{ほし}の
\ruby{長々}{なが|〳〵}と
\ruby{光}{ひかり}を
\ruby{曳}{ひ}いて
\ruby{流}{なが}れて
\ruby{{\換字{消}}}{き}えしも、
%
\ruby{思}{おも}ひ
\ruby{合}{あは}されて
\ruby{今{\換字{更}}}{いま|さら}
\ruby{急}{きふ}に
\ruby{何}{なん}と
\ruby{無}{な}く
\ruby{忌}{いま}はしく、
%
おもはず
\ruby{慄然}{りつ|ぜん}として
\ruby{天}{てん}を
\ruby{偸}{ぬす}み
\ruby{見}{み}たり。

\原本頁{128-2}%
\ruby{天}{そら}は
\ruby{今}{いま}
\ruby{白}{しら}み
わたりて
\ruby{靜}{しづか}に、
%
\ruby{星辰}{ほ|し}は
\ruby{潛}{ひそ}みつ、% 【潛 u6f5b 「先先」】【潜 u6f5c 「夫夫」】併用されている
%
\ruby{瑠璃}{る|り}の
\ruby{盤上}{ばん|じやう}に
\ruby{金砂}{きん|しや}を
\原本頁{128-3}\改行%
\ruby{撒}{ま}きし
\ruby{數時間{\換字{前}}}{すう|じ|かん|まへ}の
\ruby{光景}{あり|さま}は
\ruby{痕}{あと}も
\ruby{無}{な}く
\ruby{{\換字{消}}}{き}え
\ruby{去}{さ}つて、
%
また
ありし
おもかげを
\ruby{{\換字{忍}}}{しの}ぶべくも
あらぬ
\ruby{狀}{さま}なるに、
%
おのづと
\ruby{新}{あたら}しき
\ruby{淸旦}{あし|た}の
\ruby{氣}{き}を
\原本頁{128-5}\改行%
\ruby{受}{う}けて
\ruby{胸}{むね}も
\ruby{開}{ひら}き、
%
アヽ
\ruby{{\換字{前}}表}{ぜん|ぺう}と
いふやうなる
\ruby{事}{こと}を
\ruby{氣}{き}に
\ruby{仕}{し}たる
\ruby{愚}{おろか}さ、
%
\原本頁{128-6}\改行%
\ruby{島木}{しま|き}の
\ruby{言葉}{こと|ば}にも
\ruby{羞}{はづ}か
\換字{志}かりし、
%
と
\ruby{私}{ひそか}に
\ruby{自}{みづか}ら
\ruby{女々}{め|ゝ}しきを
\ruby{慚}{は}ぢたり。
%
\原本頁{128-8}\改行%
されど
\ruby{心}{こゝろ}は
\ruby{一度}{ひと|たび}
\ruby{動}{うご}きて
\ruby{復}{また}
\ruby{安}{やす}まらず。
%
\ruby{曉}{あした}に
\ruby{{\換字{消}}}{き}えし
\ruby{星}{ほし}は
\ruby{再度}{ふた|ゝび}
\ruby[|j>]{夕}{ゆふべ}に
\ruby{見}{み}るべけれども、
%
\ruby{一度}{ひと|たび}
\ruby{去}{さ}つては
\ruby{行}{ゆ}く
\ruby{方}{かた}
\ruby{知}{し}れぬ
\ruby{人}{ひと}の
\ruby{身}{み}の、
%
\ruby{死生}{し|せい}の
\ruby{抑々}{そも|〳〵}
\ruby{何}{なに}に
\ruby{繫}{かゝ}りて、
%
\ruby{禍福}{くわ|ふく}の
\ruby{將{\換字{又}}}{はた|また}
\ruby{何}{なに}に
\ruby{本}{もと}づくかも
\ruby{{\換字{分}}}{わか}らぬ
\ruby{茫々}{ばう|〳〵}たる
\ruby{劫{\換字{運}}}{ごふ|うん}の
\ruby{測}{はか}り
\ruby{{\換字{難}}}{がた}く
\ruby{窺}{うかゞ}ひ
\ruby{{\換字{難}}}{がた}きに
\ruby{思}{おも}ひ
\ruby{到}{いた}りては、
%
あゝ
\ruby{頼}{たの}まれぬ
\ruby{人}{ひと}の
\ruby{世}{よ}なるかな、
%
\ruby{我}{わ}が
\ruby{心}{こゝろ}の
\ruby{膏}{あぶら}を
\ruby{燃}{も}やし、
%
\ruby{骨}{ほね}の
\ruby{髓}{ずゐ}を% u9ad3 骨 左 月 辶
\ruby{焚}{た}きて、
%
\ruby{願望}{ねが|ひ}は
\ruby{大}{おほい}ならぬ
\原本頁{129-1}\改行%
\ruby{我}{わ}が
\ruby{身}{み}の
\ruby{周圍}{まは|り}に、
%
\ruby{聊}{いさゝ}かの
\ruby{光明}{ひか|り}を
\ruby{得}{え}んと
\ruby{願}{ねが}ふも、
%
\ruby{{\換字{運}}命}{うん|めい}の
\ruby{風}{かぜ}の
\ruby{容赦}{よう|しや}
\原本頁{129-2}\改行%
\ruby{無}{な}く
\ruby{吹}{ふ}き
\ruby{荒}{すさ}まんには、
%
\ruby{頼}{たの}む
\ruby{影}{かげ}なき
\ruby{裸火}{はだ|かび}の、
%
\ruby{脆}{もろ}くも
\ruby{忽}{たちま}ち
\ruby{吹}{ふ}き
\ruby{滅}{け}されて、
%
\ruby{天地}{てん|ち}は
\ruby{{\換字{情}}}{なさけ}
\ruby{無}{な}き
\ruby{闇}{やみ}と
なるべし。
%
おもへば
\ruby{小}{ちひさ}きは
\ruby{人}{ひと}の
\ruby{力}{ちから}なり。
%
\原本頁{129-4}\改行%
かほどに
\ruby{身}{み}を
\ruby{勞}{つか}らせ
\ruby{心}{こゝろ}を
\ruby{盡}{つく}して、
%
\ruby{我}{わ}が
\ruby{思}{おも}ふ
\ruby{人}{ひと}
\ruby{好}{よ}かれと
\ruby{我}{われ}は
\ruby{願}{ねが}へど、
%
\ruby{慈悲}{なさ|け}
\ruby{有}{あ}りや
\ruby{無}{な}しやも
おぼつかなき、
%
\ruby{{\換字{運}}命}{うん|めい}と
いふものゝ
\ruby[<j|]{意}{こゝろ}
\ruby{任}{まか}せ!、
%
\ruby{其}{そ}の
\ruby{意}{こゝろ}が
\ruby{人{\換字{情}}}{なさ|け}を
\ruby{知}{し}つて
\ruby{吳}{く}れうでも
\ruby{無}{な}ければ、
%
\ruby{思}{おも}へば〳〵
\原本頁{129-7}\改行%
\ruby{悲}{かな}しきは
\ruby{人}{ひと}の
\ruby{世}{よ}!。
%
\ruby{{\換字{平}}生}{ひご|ろ}は
\ruby{天}{そら}
\ruby{{\換字{翔}}}{か}ける
\ruby{事}{こと}も
\ruby{爲}{な}さば
\ruby{爲}{な}すべき
\ruby{雄心}{を|ごゝろ}
\ruby{持}{も}ちし
\ruby{我}{われ}なりしが、
%
\ruby{身}{み}に
\ruby{染}{し}みて
\ruby{今}{いま}ぞ
\ruby{人間}{にん|げん}の
\ruby{甲{\換字{斐}}}{か|ひ}
\ruby{無}{な}きを
\ruby{知}{し}りつる!。
%
\原本頁{129-9}\改行%
\ruby{天}{てん}は
\ruby{限}{かぎ}り
\ruby{無}{な}く
\ruby{大}{おほい}なるに、
%
\ruby{我}{われ}は
\ruby{糠星}{ぬか|ぼし}の
\ruby{其}{それ}より
\ruby{微}{かす}けく、
%
\ruby{地}{ち}は
\ruby{涯}{はて}も
\ruby{無}{な}く
\ruby{廣}{ひろ}やかなるに、
%
\ruby{身}{み}は
\ruby{塵}{ちり}
\ruby{土}{ひぢ}と
\ruby{小}{ちひさ}なる、
%
\ruby{此}{こ}の
\ruby{某甲}{なに|がし}が
\ruby{懷}{いだ}ける
\ruby{念}{おもひ}の、
%
\原本頁{129-11}\改行%
\ruby{{\換字{運}}命}{うん|めい}に
\ruby{對}{むか}へる
\ruby{其}{そ}の
\ruby{眞態}{あり|さま}は、
%
\ruby{譬}{たと}へば
\ruby{一縷}{いち|る}の
\ruby{細}{ほそ}き〳〵、
%
\ruby{毛}{け}の
\ruby{如}{ごと}く
\ruby{蜘蛛}{く|も}の
\ruby{圍}{い}のごとき
\ruby{絲}{いと}を、
%
\ruby{千萬馬力}{せん|まん|ば|りき}もて
\ruby{轟}{とゞろ}き
\ruby{{\換字{廻}}}{まは}れる
\ruby{大車輪}{だい|しや|りん}に
\ruby{繫}{か}けて、
%
\原本頁{130-2}\改行%
\ruby{其}{そ}の
\ruby{車輪}{しや|りん}の
\ruby{我}{わ}が
\ruby{願}{ねが}ふ
\ruby{方}{かた}に
\ruby{{\換字{廻}}}{まは}らんことを、
%
\ruby{竊}{ひそか}に
\ruby{願}{ねが}ひ
\ruby{求}{もと}むるが
\ruby{如}{ごと}し。
%
\原本頁{130-3}\改行%
\ruby{嗚呼}{あ|ゝ}、
%
\ruby{我}{わ}が
\ruby{願}{ねが}ひの
\ruby{聽}{き}かるべきや\換字{?!}。
%
\ruby{心細}{こゝろ|ぼそ}くも
また
\ruby{心細}{こゝろ|ぼそ}くて、
%
\ruby{{\換字{情}}無}{なさけ|な}くも
\ruby{物}{もの}のみの
\ruby{思}{おも}はるゝ
\ruby{世}{よ}かな!。
%
\ruby{我}{わ}が
\ruby{智慧}{ち|ゑ}の
\ruby{今}{いま}
\ruby{効}{かひ}
\ruby{無}{な}きを
\ruby{知}{し}り、
%
\原本頁{130-5}\改行%
\ruby{我}{わ}が
\ruby{意念}{おも|ひ}の
\ruby{今}{いま}
\ruby{孱{\換字{弱}}}{か|よわ}きを
\ruby{知}{し}り、
%
\ruby{斷}{た}えぬ
\ruby{泉}{いづみ}と
\ruby{湧}{わ}き
\ruby{上}{あが}る
\ruby{戀}{こひ}の
\ruby{誠}{まこと}に
\ruby{洗}{あら}はれて、
%
\ruby{心}{こゝろ}は
\ruby{無垢}{む|く}の
\ruby{往時}{むか|し}に
\ruby{{\換字{返}}}{かへ}りぬ。
%
アヽ
\ruby{今}{いま}
\ruby{我}{われ}は
\ruby{嬰兒}{みどり|ご}なり!。
%
\ruby{天地}{てん|ち}の
\ruby{那處}{いづ|く}に
\ruby{慈母}{は|ゝ}の
\ruby{御坐}{お|は}す\換字{?!}。
%
\ruby{泣}{な}きて
\ruby{呼}{よ}び
\ruby{度}{た}き
\ruby{心地}{こゝ|ち}ぞする。
%
と
\ruby{曉天}{あか|つき}の
\ruby{{\換字{猶}}}{なほ}
\ruby{靜寂}{しづ|か}にして
\ruby{人}{ひと}の
\ruby{{\換字{通}}}{とほ}りも
\ruby{稀少}{まば|ら}なるに、
%
\ruby{深}{ふか}くも
\ruby{心}{こゝろ}の
\ruby{奧}{おく}に
\ruby{思}{おも}ひ
\ruby{入}{い}つたる
\ruby{水野}{みづ|の}は、
%
ふつと
\ruby{我}{われ}に
\ruby{{\換字{返}}}{かへ}つて
\ruby{頭}{かうべ}を
\ruby{擡}{あ}ぐれば、
%
\ruby{身}{み}は
\ruby{何時}{い|つ}の
\ruby{程}{ほど}にか
\ruby{來}{きた}りけん、
%
\ruby{塵埃}{ち|り}
\ruby{無}{な}き
\ruby{{\換字{朝}}}{あした}の
\ruby{露}{つゆ}けき
\ruby{石路}{せき|ろ}の、
%
\ruby{長々}{なが|〳〵}しきを
\ruby{知}{し}らぬ
\原本頁{130-11}\改行%
\ruby{間}{ま}に
\ruby{{\換字{過}}}{す}ぎて、
%
\ruby{今}{いま}や
\ruby{淺草寺}{せん|さう|じ}の
\ruby{山門}{さん|もん}を、
%
\ruby{既}{すで}に
\ruby{{\換字{半}}}{なかば}は
\ruby{潛}{くゞ}り% 【潛 u6f5b 「先先」】【潜 u6f5c 「夫夫」】併用されている
\ruby{居}{ゐ}たり。

\原本頁{131-1}%
\ruby{晝間}{ひ|る}は
\ruby{賑}{にぎ}やかなる
\ruby{中店}{なか|みせ}も、
%
\ruby{{\換字{猶}}}{なほ}
\ruby{寂々}{じやく|〳〵}として
\ruby{物}{もの}の
\ruby{響}{ひゞき}を
\ruby{傳}{つた}へず、
%
\ruby{御{\換字{扉}}}{み|とびら}を
\ruby{今}{いま}
\ruby{開}{ひら}きしばかりの、
%
\ruby{御堂}{み|だう}の
\ruby{内}{うち}は
\ruby{仄暗}{ほの|ぐら}きに、
%
\ruby{御燈明}{み|あか|し}の
\ruby{煌々}{きら|〳〵}と
\ruby{黄金色}{こ|がね|いろ}に
\ruby{見}{み}えて、
%
\ruby{{\換字{朝}}{\換字{勤}}}{あさ|づと}めの
\ruby{讀經}{ど|きやう}の
\ruby{聲}{こゑ}は
\ruby{殊{\換字{勝}}}{しゆ|しよう}に
\ruby{澄}{す}み
\ruby{渡}{わた}り、
%
\ruby{御堂}{み|だう}の
\ruby{甍}{いらか}は
\ruby{天}{そら}に
\ruby{聳}{そび}えて、
%
そこ
\ruby{此處}{こ|ゝ}に
\ruby{立}{た}てる
\ruby{老樹}{おい|き}の
\ruby{銀杏}{い|てふ}は、
%
まだ
\ruby{下}{お}り
\ruby{立}{た}たぬ
\ruby{鳩雞}{はと|とり}を
\ruby{宿}{やど}して、
%
\ruby{睡}{ねむ}れるが
\ruby{如}{ごと}く
\ruby{靜}{しづか}かに
\ruby{秋}{あき}の
\ruby{曙}{あした}の
\ruby{色}{いろ}を
\ruby{見}{み}せたり。

\原本頁{131-6}%
\ruby{水野}{みづ|の}は
あはれにも
\ruby{頭}{かうべ}を
\ruby{下}{さ}げて、
%
かつて
\ruby{拜}{をが}みしことなき
\ruby{觀世音菩薩}{くわ|んぜ|おん|ぼ|さつ}を、
%
\ruby{此日}{この|ひ}
はじめて
\ruby{涙}{なみだ}の
\ruby{眼}{め}を
\ruby{閉}{と}ぢ、
%
\ruby{一心}{いつ|しん}に
\ruby{拜}{をが}み
\ruby{奉}{たてまつ}りたり。
