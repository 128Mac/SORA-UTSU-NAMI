\Entry{其二十二}

% メモ 校正終了 2024-04-09
\原本頁{131-9}%
\ruby{我}{わ}が
\ruby{戀}{こひ}
\ruby{叶}{かな}へかしとも
\ruby{祈}{いの}らばこそ、
%
たゞ
\ruby{人}{ひと}の
\ruby{命}{いのち}の
\ruby{暴風雨}{あ|ら|し}に
\ruby{揉}{も}まるる
\ruby{芭蕉葉}{ば|せう|ば}と
\ruby{危}{あやふ}きを
\ruby{悲}{かなし}みて、
%
\ruby{只}{ひた}
\ruby{管}{すら}に
\ruby{我}{わ}が
\ruby[g]{五十子}{いそこ}
\ruby{禍災}{わざ|はひ}
\ruby{無}{な}かれとのみ、
%
\原本頁{132-1}\改行%
\ruby{堪}{た}へがたき
\ruby{思}{おもひ}の
\ruby{誠}{まこと}を
\ruby{致}{いた}して、
%
\ruby{他念}{た|ねん}も
\ruby{無}{な}く
\ruby[g]{水野}{みづの}の
\ruby{願}{ねが}ひ
\ruby{奉}{たてまつ}れる
\ruby{折}{をり}から、
%
\ruby{我傍}{わが|かたへ}にも
\ruby{人}{ひと}ありて、
%
\ruby{先刻}{さ|き}より
\ruby{普門品}{ふ|もん|ぼん}を
ほそ〴〵と
\ruby{唱}{とな}へ
\ruby{居}{ゐ}けるが、
%
\ruby{既}{すで}に
\ruby{偈}{げ}の
ところにかゝりて
\ruby{漸}{やうや}く
\ruby{勢}{いきほひ}づき、
%
\ruby{弘誓深如海}{ぐ|ぜい|じん|によ|かい}の
\ruby{句}{く}
\原本頁{132-4}\改行%
あたりより
\ruby{嗄}{しはが}れたる
\ruby{聲}{こゑ}も
おのづから
\ruby{張}{は}り
\ruby{來}{きた}りて、
%
いま、
%
\ruby{或漂流巨海}{わく|へう|る|こ|かい}、
%
\ruby{龍魚諸鬼難}{りゆう|ぎよ|しよ|き|なん}、
%
\ruby{念彼觀音力}{ねん|ぴ|くわん|のん|りき}、% 「觀音」の読みは原本通り「くわん(の)ん」
%
\ruby{波浪不能沒}{は|らう|ふ|のう|もつ}と
\ruby{調子}{てう|し}に
\ruby{乘}{の}りて
\ruby{打誦}{うち|じゆ}せるを
\ruby{見}{み}たり。

\原本頁{132-7}%
たゞ
\ruby{一}{ひ}ト
\ruby{筋}{すぢ}に
\ruby{頼}{たの}み
\ruby{奉}{たてまつ}る
\ruby{思}{おもひ}は
\ruby{聲}{こゑ}の
\ruby{色}{いろ}にも
\ruby{現}{あらは}はれて% TODO 原本では「あら(はは)れて」となっている
\ruby{願}{ねが}ひ
\ruby{求}{もと}むる
\ruby{態}{さま}の
\原本頁{132-7}\改行%
\ruby{僞}{いつはり}ならず
\ruby{聞}{きこ}ゆるは、
%
\ruby{如何}{い|か}なる
\ruby{苦惱}{くる|しみ}のある
\ruby{人}{ひと}なるか、
%
と
\ruby{我}{わ}が
\ruby{胸}{むね}に
\ruby{疼痛}{いた|み}あれば
\ruby{他}{ひと}の
\ruby{胸}{むね}の
\ruby{疼痛}{いた|み}も
\ruby{餘{\換字{所}}}{よ|そ}ならず
\ruby{覺}{おぼ}えて
\ruby{自己}{おの|れ}
\ruby{念}{ねん}じ
\ruby{{\換字{終}}}{おは}りたる
\ruby[g]{水野}{みづの}は
\ruby{其人}{その|ひと}を
\ruby{見}{み}るに、
%
\ruby{衣服}{な|り}こそは
\ruby{見苦}{み|ぐる}しからね、
%
がりゝと
\ruby{痩}{や}せて
\ruby{手足}{て|あし}のみ
\ruby{徒}{いたづら}に
\ruby{長}{なが}う
\ruby{見}{み}えたる、
%
\ruby{髮}{かみ}は
\ruby{既}{すで}に
\ruby{薄}{うす}くして
\ruby{光澤}{つ|や}
\ruby{無}{な}き
\ruby{猫毛}{ねこ|げ}の
ほや〳〵と
\ruby{烟}{けむり}のやうに
\ruby{殘}{のこ}れる、
%
\ruby{脫}{ぬ}け
\ruby{上}{あが}りたる
\ruby{額}{ひたひ}の
\ruby{特}{こと}に
\ruby{廣}{ひろ}く、
%
\原本頁{133-2}\改行%
\ruby{下}{くだ}り
\ruby{長}{なが}き
\ruby{鼻}{はな}の
\ruby{細}{ほそ}くして
\ruby{淋}{さび}しき、
%
\ruby{下作}{げ|さく}には
あらねど
\ruby{甚}{いた}く
\ruby{{\換字{貧}}相}{ひん|さう}なる
\原本頁{133-3}\改行%
\ruby{男}{をとこ}の、
%
\ruby{眉間}{み|けん}に
\ruby{苦}{くる}しげなる
\ruby{八字}{はち|のじ}の
\ruby{皺}{しわ}を
\ruby{深々}{ふか|〴〵}と
\ruby{疊}{たゝ}みて、
%
\ruby{{\換字{猶}}}{なほ}
しきりに
\ruby{念彼觀音力}{ねん|ぴ|くわん|のん|りき}、% 「觀音」の読みは原本通り「くわん(の)ん」
%
\ruby{應時得{\換字{消}}散}{おう|じ|とく|せう|さん}
などゝ
\ruby{誦}{じゆ}し
つゞけたる
\ruby{狀態}{あり|さま}の、
%
\ruby{老}{お}いたる
\ruby{人}{ひと}だけに
\ruby{愍然}{あは|れ}さ
\ruby{{\換字{勝}}}{まさ}るのみならず、
%
\ruby{時々}{とき|〴〵}の
\ruby{聲}{こゑ}の
\ruby{曇}{くも}りて
\ruby{顫}{ふる}ふに、
%
\原本頁{133-6}\改行%
\ruby{其}{そ}の
\ruby{胸}{むね}の
\ruby{中}{うち}も
\ruby{推測}{おし|はか}られて
\ruby{物悲}{もの|がな}しく、
%
あゝ
\ruby{憂}{うれひ}を
\ruby{懷}{いだ}くものは
\ruby{我}{われ}ばかりには
あらざりけり、
%
\ruby{心}{こゝろ}の
\ruby[g]{痛苦}{くるしみ}に
\ruby{堪}{た}へかねて、
%
\ruby{此人}{この|ひと}も
\ruby{御佛}{み|ほとけ}を
\ruby{頼}{たの}むなるべし、
%
\ruby{妻}{つま}や
\ruby{病}{や}み
\ruby{臥}{ふ}せる、
%
\ruby{子}{こ}や
\ruby{患}{わづら}へる、
%
\ruby{或}{あるひ}は
\ruby{老}{お}いて
\ruby{子}{こ}の
\ruby{無}{な}き
\ruby{歟}{か}、
%
\ruby{子}{こ}ありて
\ruby{或}{あるひ}は
\ruby{不孝}{ふ|かう}なる
\ruby{歟}{か}、
%
いづれ
\ruby{悲}{かな}しき
\ruby{事{\換字{情}}}{わ|け}あらんと、
%
\原本頁{133-10}\改行%
そゞろに
\ruby{心惹}{こゝろ|ひ}かれて
\ruby{直}{すぐ}には
\ruby{見棄}{み|す}てかぬる
\ruby{思}{おもひ}したり。

\原本頁{133-11}%
\ruby[||j>]{旱歳}{ひでり|どし}の
\ruby{冷氣}{ひ|え}
\ruby{早}{はや}き
\ruby{秋}{あき}の
\ruby{曉天}{あか|つき}の
\ruby{事}{こと}とて、
%
\ruby{{\換字{寒}}}{さむ}きやうに
\ruby{廣々}{ひろ|〴〵}としたる
\ruby{御堂}{み|だう}の
\ruby{中}{うち}は、
%
\ruby{此人}{この|ひと}と
\ruby{我}{われ}との
ほかに
\ruby{人}{ひと}も
\ruby{見}{み}えず、
%
\ruby{香}{かう}の
\ruby{氣}{き}
しづかに
\ruby{薫}{くん}じて
\ruby{殊{\換字{勝}}}{しゆ|しよう}さ
\ruby{身}{み}に
\ruby{浸}{し}み
\ruby{渡}{わた}り、
%
\ruby{見上}{み|あ}ぐる
\ruby{眼}{め}を
\ruby{照}{て}らす
\ruby{施無畏}{せ|む|ゐ}の
\ruby{三大字}{さん|だい|じ}は、
%
\ruby{一世}{いつ|せ}に
\ruby{秀}{ひいで}し
\ruby{佐{\換字{文}}山}{さ|ぶん|ざん}が
\ruby{長櫃三個}{なが|もち|さん|さを}の
\ruby{反故}{ほ|ご}を
つくつて
\ruby{纔}{わづか}に
\ruby{書}{か}きしといふ
\ruby{傳說}{いひ|つたへ}さへ、
%
おのづと
\ruby{想}{おも}ひ
\ruby{起}{おこ}さるゝばかり
\ruby{筆勢遒麗}{ひつ|せい|しう|れい}に、
%
\原本頁{134-5}\改行%
\ruby{金光美}{きん|くわう|ゝつく}しく
\ruby{高}{たか}く
\ruby{懸}{かゝ}りて、
%
まことに
\ruby{人}{ひと}をして
\ruby{慈眼視衆生}{じ|げん|じ|しゆ|じやう}の
\ruby{菩薩}{ぼ|さつ}の
\原本頁{134-6}\改行%
\ruby{威力}{ゐ|りき}を
\ruby{仰}{あふ}がんとする
\ruby{心}{こゝろ}を
\ruby{發}{おこ}さしめ、
%
たま〳〵に
\ruby{鳩}{はと}の
はた〳〵と
\原本頁{134-7}\改行%
\ruby{飛}{と}んでは
\ruby{靜}{しづ}かさを
\ruby{破}{やぶ}るも
\ruby{却}{かへ}つて
\ruby{寂}{さ}びて、
%
\ruby{{\換字{平}}生}{ひご|ろ}の
\ruby{賑}{にぎ}はしさに
\ruby{引反}{ひき|か}へて
\ruby{今{\換字{朝}}}{け|さ}の
\ruby{此}{こ}の
\ruby{御堂}{み|だう}の
\ruby{神々}{かう|〴〵}しく
\ruby{{\換字{尊}}}{たつと}さに、
%
\ruby[g]{水野}{みづの}は
\ruby{今}{いま}まで
\ruby{知}{し}らざりし
\ruby{趣味}{おも|むき}を
おぼえたり。

\原本頁{134-10}%
\ruby{妙音觀世音}{めう|おん|くわん|ぜ|おん}、
%
\ruby{梵音海潮音}{ぼん|おん|かい|てう|おん}、
%
\ruby{{\換字{勝}}彼世間音}{しよう|ひ|せ|けん|おん}、
%
と
\ruby{老}{お}いたる
\ruby{人}{ひと}の
\ruby{誦}{じゆ}する
\原本頁{134-11}\改行%
\ruby{聲}{こゑ}は、
%
いよ〳〵
\ruby{眞心}{ま|ごゝろ}
\ruby{籠}{こも}りて
\ruby{澄}{す}み
\ruby{行}{ゆ}き、
%
\ruby{普門品}{ふ|もん|ぼん}は
\ruby{今}{いま}や
\ruby{{\換字{終}}}{をは}るに
\ruby{{\換字{近}}}{ちか}からんとす。

\原本頁{135-2}%
\ruby{時}{とき}に
\ruby{御堂}{み|だう}の
\ruby{内}{うち}
\ruby{俄}{にはか}に
\ruby{騷}{さわ}がしく、
%
がたごとゝ
\ruby{薩{\換字{摩}}下駄}{さつ|ま|げ|た}
\ruby{踏}{ふ}み
\ruby{鳴}{な}らす
\ruby{音}{おと}を
\ruby{憚}{はゞか}り% 「憚 は(ゞ)か」
\ruby{氣}{げ}
\ruby{無}{な}く
\ruby{伽藍}{が|らん}に
\ruby{響}{ひゞ}かせて、
%
\ruby{太}{ふと}き〳〵
\ruby[g]{洋{\換字{杖}}}{すてつき}
もて
\ruby{益}{えき}も
\ruby{無}{な}く
\ruby{床板}{ゆか|いた}を
\ruby{突}{つ}きちらし
\ruby{撲}{たゝ}きちらしながら、
%
\ruby{入}{い}り
\ruby{來}{きた}れる
\ruby{二人}{ふた|り}の
\ruby{書生}{しよ|せい}あり。
%
\原本頁{135-5}\改行%
\ruby{醉}{ゑひ}を% 「醉」は原本通り「ゑ」で調整
\ruby{帶}{お}びたりとは
\ruby{見}{み}えねど
\ruby[g]{反響}{こだま}の
\ruby{起}{おこ}るほどの
\ruby{馬鹿}{ば|か}
\ruby{聲}{ごゑ}を
あげて、

\原本頁{135-6}%
『ハツ、
オイ、
%
まだ
\ruby{此樣}{こ|ん}なものを
\ruby{本氣}{ほん|き}で
\ruby{禮拜}{らい|はい}して
\ruby{居}{ゐ}るものがあるぜ!。
』

\原本頁{135-8}%
と、
%
\ruby{紺絞}{こん|しぼり}の
\ruby{兵兒帶}{へ|こ|おび}を
\ruby{締}{し}めたるが
\ruby{云}{い}へば、

\原本頁{135-9}%
『ウン、
%
\ruby{可愍}{ふ|びん}なものさ、
%
\ruby{五六世紀}{ご|ろく|せい|き}も
\ruby{{\換字{前}}}{まへ}の
\ruby{思想}{し|さう}に
\ruby{養}{やしな}はれて
\ruby{居}{ゐ}るのだからナ。
』

\原本頁{135-11}%
と、
%
\ruby{白金巾}{しろ|がな|きん}の
\ruby{帶}{おび}したるが
\ruby{答}{こた}へたり。

\原本頁{136-1}%
『
\ruby{我輩}{わが|はい}の
\ruby{親{\換字{分}}}{おや|ぶん}は、
%
\ruby[g]{基督}{くりすと}が
\ruby{代表}{だい|へう}した
\ruby{馬鹿}{ば|か}
\ruby{思想}{し|さう}を
\ruby{奴隷}{ど|れい}
\ruby{{\換字{道}}徳}{だう|とく}と
\ruby{罵}{のゝし}つたが、
%
\ruby{我輩}{わが|はい}は
\ruby{法然日蓮}{はふ|ねん|にち|れん}の% TODO 暫定で「蓮 uf999」とする(参考「蓮 uu84ee」)
\ruby{代表}{だい|へう}した
\ruby{馬鹿}{ば|か}
\ruby{思想}{し|さう}を
\ruby{乞食}{こ|じき}
\ruby{{\換字{道}}徳}{だう|とく}と
\ruby{斷言}{だん|げん}するが、
%
\ruby{何樣}{ど|う}だ、
%
\ruby{可}{よ}からう。
』

\原本頁{136-4}%
『ウン、
%
\ruby{偉}{えら}い!。
%
\ruby{釋迦}{しや|か}が
\ruby{事實上}{じ|ゞつ|じやう}
\ruby{乞食}{こ|じき}だから
\ruby{{\換字{猶}}}{なほ}
\ruby{可笑}{を|か}しい。
%
それだのに、
%
\ruby{木佛金佛}{き|ぶつ|かな|ぶつ}を
\ruby{拜}{をが}む
\ruby{奴}{やつ}さへ
あるのだからナ。
%
ほんとに
\ruby{本能主義}{ほん|のう|しゆ|ぎ}の
\ruby{有}{あ}り
\ruby{難}{がた}い、
%
\ruby{大}{おほ}もての
\ruby{美的境界}{び|てき|きやう|がい}でも
\ruby{敎}{をし}へて
\ruby{{\換字{遣}}}{や}りたいナ。
%
ハヽヽハヽ。% TODO 行頭、行末禁則の影響で踊り字の組み合わせがおかしい
』

\原本頁{136-8}%
『ヤ、
%
\ruby{酷}{ひど}いところで
\ruby[g]{自惚}{のろけ}る
\ruby{奴}{やつ}だナ。
%
ハヽハヽ。
』

\原本頁{136-9}%
\ruby{十八間四面}{じう|はつ|けん|し|めん}の
\ruby{御堂}{み|だう}も
\ruby{動}{ゆら}ぐばかりに
\ruby{高笑}{たか|わら}ひして、
%
\ruby{繪}{ゑ}に
\ruby{見}{み}る
\ruby{惡鬼}{あく|き}
\ruby{羅刹}{ら|せつ}が
\ruby{持}{も}てる
\ruby{銕{\換字{杖}}}{てつ|ぢやう}の
\ruby{如}{ごと}き
\ruby{恐}{おそ}ろしき
\ruby{重}{おも}げなる
\ruby{{\換字{杖}}}{つゑ}もて、
%
\ruby{我}{わ}が
\ruby{踏}{ふ}める
\ruby{床}{ゆか}を、
%
\ruby{我}{わ}が
\ruby{威風}{ゐ|ふう}を
\ruby{見}{み}よと
ばかりに、
%
どしんと
\ruby{突}{つ}きたり。

\原本頁{137-1}%
\ruby{皆發無等々}{かい|ほつ|む |とう|〴〵}
\ruby{阿耨多羅三藐三菩提心}{あ |のく|た |ら |さん|みやく|さん|ぼ|だい|しん}と、
%
\ruby{念}{ねん}じ
\ruby{{\換字{終}}}{をは}りて
\ruby{禮拜}{らい|はい}し
\ruby{濟}{すま}したる
\ruby{老}{お}いたる
\ruby{男}{をとこ}は、
%
\ruby{頭}{かうべ}を
\ruby{擡}{もた}げて
\ruby[g]{水野}{みづの}と
\ruby{顏}{かほ}
\ruby{見合}{み|あ}はせて、
%
おもはず
\ruby{互}{たがひ}に
\ruby{眉}{まゆ}を
\ruby{顰}{ひそ}めざるを
\ruby{得}{え}ざりき。
