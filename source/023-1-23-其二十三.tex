\Entry{其二十三}

\ruby{天}{そら}の
\ruby{彼方}{あな|た}に
\ruby{颶風}{つむじ|かぜ}を
\ruby{起}{おこ}しゝニイチエが
\ruby{眞趣}{おも|むき}を
\ruby{實}{まこと}に
\ruby{知}{し}れりや、それも
\ruby{覺束無}{おぼ|つか|な}げなる
\ruby{書生}{しよ|せい}の
\ruby{放言}{はう|げん}の、
\ruby{餘}{あま}りの
\ruby{事}{こと}に
\ruby{傍痛}{かたは|らいた}くはおぼえたれど、
\ruby{意}{こゝろ}を
\ruby{動}{うご}かすほどにも
\ruby{至}{いた}らざりければ、
\ruby{他}{ひと}は
\ruby{他}{ひと}なり、
\ruby{我}{われ}は
\ruby{我}{われ}なり、
\ruby{關係無}{かけ|かまひ|な}き
\ruby{禽}{とり}の
\ruby{聲}{こゑ}の、それまでの
\ruby{事}{こと}なりと
\ruby{聞}{き}き
\ruby{捨}{す}てゝ、
\ruby{既}{すで}に『ツアラツウストラ
\ruby{如是{\換字{説}}}{によ|ぜ|せつ}』をも
\ruby{窺}{うかゞ}ひ
\ruby{讀}{よ}まぬにあらざりし
\ruby{水野}{みづ|の}は、\換字{志}ろりと
\ruby{冷}{ひや}やかに
\ruby{彼}{か}の
\ruby{二人}{ふ|たり}をば
\ruby{一瞥}{いち|べつ}せしのみに
\ruby{止}{とゞ}まりて、
\ruby{徐々此處}{やお|ら|こ|ゝ}を
\ruby{去}{さ}らんと
\ruby{歩}{あゆ}み
\ruby{出}{いだ}せば、
\ruby{彼}{か}の
\ruby{老}{お}いたる
\ruby{男}{をとこ}も
\ruby{一齊}{と|も}にと
\ruby{{\換字{随}}}{したが}へり。

\ruby{世}{よ}の
\ruby{態人}{さま|ひと}の
\ruby{{\換字{情}}}{こゝろ }\ %空白有り
\ruby{漸}{ やうや}く
\ruby{移}{うつ}りて、
\ruby{礎}{いしずゑ}は
\ruby{舊}{きう}に
\ruby{依}{よ}りて
\ruby{固}{かた}く、
\ruby{棟}{むね}は
\ruby{舊}{きう}に
\ruby{依}{よ}りて
\ruby{高}{たか}けれども、
\ruby{今}{いま}は
\ruby{此}{こ}の
\ruby{莊嚴}{さう|ごん}なる
\ruby{御堂}{み|だう}の
\ruby{内}{うち}にさへも、
\ruby{謗法毀佛}{はう|ばふ|き|ぶつ}の
\ruby{暴}{あば}れ
\ruby{聲起}{ごゑ|おこ}りて、
\ruby{譬喩}{たと|へ}を
\ruby{取}{と}りて
\ruby{云}{い}はゞ
\ruby{月黑}{つき|くろ}き
\ruby{夜}{よ}の
\ruby{大潮}{おほ|しほ}の、
\ruby{洲}{す}を
\ruby{吞}{の}み
\ruby{岩}{いは}を
\ruby{嚙}{か}みて
\ruby{漸}{やうや}く
\ruby{大地}{だい|ち}を
\ruby{犯}{をか}さんとするが
\ruby{如}{ごと}くに、
\ruby{何時}{い|つ}となく
\ruby{破壞}{は|くわい}の
\ruby{吶喊}{さけ|び}の
\ruby{押寄}{おし|よ}するは、
\ruby{{\換字{所}}謂末法澆季}{いは|ゆる|まつ|ぱふ|げう|き}の
\ruby{是非}{ぜ|ひ}も
\ruby{無}{な}き
\ruby{當時}{い|ま}の
\ruby{大勢}{いき|ほい}なり。
\ruby{書生}{しよ|せい}は
\ruby{{\換字{猶}}}{なほ}がたりごとりと、
\ruby{力足}{ちから|あし}を
\ruby{踏}{ふ}み
\ruby{{\換字{杖}}}{つゑ}を
\ruby{突}{つ}き
\ruby{立}{た}てゝて
\ruby{歩}{ある}き
\ruby{居}{ゐ}しが、
\ruby{紺絞}{こん|しぼり}の
\ruby{帶}{おび}したるは、
\ruby{急}{きふ}に
\ruby{虛空}{こ|くう}に
\ruby{{\換字{杖}}}{つゑ}を
\ruby{擧}{あ}げて、
\ruby{揭}{かゝ}げられたる
\ruby{額}{がく}の
\ruby{一}{ひと}つを
\ruby{指}{さ}しながら、

『
\ruby{面白}{おも|しろ}いナア、
\ruby{此}{こ}の
\ruby{一}{ひと}つ
\ruby{家}{や}の
\ruby{畫}{ゑ}は!、どうも
\ruby{巧}{よ}く
\ruby{出來}{で|き}て
\ruby{居}{ゐ}るナ、
\ruby{氣}{き}に
\ruby{入}{い}つたナア!。
』

と
\ruby{云}{い}へば、

『ムヽ、』

と、
\ruby{白}{しろ}き
\ruby{帶}{おび}したるは
\ruby{其意}{その|い}を
\ruby{得}{え}ぬげに
\ruby{應}{こた}へつ、

『\換字{志}かし
\ruby{御厩}{みう|まや}の
\ruby{喜三太}{き|さん|だ}も
\ruby{好}{い}いぢやあ
\ruby{無}{な}いか。
』

と
\ruby{附加}{つけ|くは}へたり。

『
\ruby{馬鹿}{ば|か}ツ!。
そりやあ
\ruby{{\換字{技}}{\換字{術}}}{ぎじ|ゆつ}だけの
\ruby{論}{ろん}だ。
\ruby{云}{い}ふなあ
\ruby{其處}{そ|こ}ぢやあ
\ruby{無}{な}い。
よく
\ruby{見}{み}ろ!
\ruby{吾輩}{わが|はい}の
\ruby{此}{こ}の
\ruby{一}{ひと}つ
\ruby{家}{や}の
\ruby{圖}{づ}を!。
\ruby{何樣}{ど|う}だ
\ruby{彼}{あ}の
\ruby{婆}{ばあ}さんの
\ruby{顏}{かほ}の
\ruby{立派}{りつ|ぱ}なこと!。
\ruby{實}{じつ}に
\ruby{立派}{りつ|ぱ}ぢやあ
\ruby{無}{な}いか、
\ruby{立派}{りつ|ぱ}ぢやあ
\ruby{無}{な}いか!。
\ruby{國家}{こく|か}の
\ruby{法律}{はふ|りつ}なんぞといふ
\ruby{奴}{やつ}ア
\ruby{踏}{ふ}み
\ruby{付}{つ}け
\ruby{切}{き}つた
\ruby{彼}{あ}の
\ruby{顏}{かほ}つき!。
\ruby{世間}{せ|けん}の
\ruby{善惡}{ぜん|あく}の
\ruby{沙汰}{さ|た}なんぞを
\ruby{寄}{よ}せつけも
\ruby{仕無}{し|な}い
\ruby{彼}{あ}の
\ruby{顏付}{かほ|つき}!。
\ruby{戀}{こひ}も
\ruby{人{\換字{情}}}{にん|じやう}も
\ruby{無}{な}い
\ruby{彼}{あ}の
\ruby{顏}{かほ}つき!。
\ruby{邪}{じや}でも
\ruby{非}{ひ}でもまかはない
\ruby{彼}{あ}の
\ruby{顏}{かほ}つき!。
おれが
\ruby{{\換字{勝}}手}{かつ|て}だぞといふ
\ruby{彼}{あ}の
\ruby{顏}{かほ}つき!。
\ruby{神}{かみ}でも
\ruby{佛}{ほとけ}でも
\ruby{對面}{むか|ふ}へまはつたら
\ruby{斫殺}{たゝ|つき}つて
\ruby{{\換字{遣}}}{や}らうといふ
\ruby{彼}{あ}の
\ruby{顏}{かほ}つき!。
あゝ
\ruby{何}{なん}と
\ruby{立派}{りつ|ぱ}な
\ruby{顏}{かほ}に
\ruby{書}{か}いてあるでは
\ruby{無}{な}いか。
\ruby{十{\換字{分}}}{じう|ぶん}に
\ruby{惡人}{あく|にん}の
\ruby{偉大}{ゐ|だい}な
\ruby{精神}{せい|しん}が
\ruby{發揮}{はつ|き}してある!。
\ruby{誰}{たれ}だつて
\ruby{此}{こ}の
\ruby{繪}{ゑ}を
\ruby{能}{よ}く
\ruby{見}{み}たらば、
\ruby{{\換字{強}}惡}{がう|あく}が
\ruby{美}{い}いものだといふ
\ruby{事}{こと}に
\ruby{氣}{き}が
\ruby{付}{つ}くだらう!。
\ruby{見}{み}ろ、
\ruby{彼}{あ}の
\ruby{娘}{むすめ}が
\ruby{卑小}{け|ち}な
\ruby{惡}{わる}びれた
\ruby{樣子}{やう|す}を!。
\ruby{人}{ひと}に
\ruby{縋}{すが}りたがるやうな、
\ruby{哀愍}{あは|れみ}を
\ruby{乞}{こ}うやうな、
\ruby{泣}{な}き
\ruby{出}{だ}しさうな、
\ruby{切}{せつ}なさうな、
\ruby{善惡}{ぜん|あく}の
\ruby{{\換字{道}}理}{だう|り}を
\ruby{怖}{こは}がつて
\ruby{居}{ゐ}るやうな、
\ruby{國家}{こく|か}の
\ruby{規律}{おき|て}なんぞにびく〳〵して
\ruby{居}{ゐ}るやうな、
\ruby{神佛}{しん|ぶつ}なんぞにおど〳〵して
\ruby{居}{ゐ}る、\換字{志}みつたれた、
\ruby{見}{み}つとも
\ruby{無}{な}い
\ruby{醜態}{ざ|ま}が、すつかり
\ruby{見}{み}えて
\ruby{居}{ゐ}る!。
\ruby{{\換字{所}}謂善人}{いは|ゆる|ぜん|にん}といふ
\ruby{奴}{やつ}が
\ruby{卑劣}{け|ち}なもので、
\ruby{下}{くだ}らないものだといふ
\ruby{事}{こと}は、
\ruby{何樣}{ど|ん}な
\ruby{馬鹿}{ば|か}な
\ruby{奴}{やつ}の
\ruby{眼}{め}にも
\ruby{暎}{うつ}るだらう!。
\ruby{何樣}{ど|う}だ、
\ruby{好}{い}いぢやあ
\ruby{無}{な}いか
\ruby{好}{い}い
\ruby{畫}{ゑ}ぢやあ
\ruby{無}{な}いか。
、
\ruby{何樣}{ど|う}だ、
\ruby{{\換字{分}}}{わ}かつたか、
\ruby{好}{い}いか、オイ、
\ruby{君}{きみ}!。
\ruby{此}{こ}の
\ruby{一}{ひと}つ
\ruby{家}{や}の
\ruby{御婆}{お|ばあ}さんが
\ruby{國王}{こく|わう}になりやあ、
\ruby{世界中}{せ|かい|ぢゆう}を
\ruby{斬}{き}り
\ruby{伏}{ふ}せて
\ruby{{\換字{寝}}酒}{ね|ざけ}の
\ruby{下物}{さか|な}に
\ruby{仕}{し}て
\ruby{{\換字{遣}}}{や}らうと、
\ruby{手}{て}に
\ruby{持}{も}つた
\ruby{利器}{え|もの}を
\ruby{振}{ふ}り
\ruby{舞}{ま}はすんだ!。
もし
\ruby{此}{こ}の
\ruby{娘}{むすめ}が
\ruby{國王}{こく|わう}になりやあ、
\ruby{彼方}{あつ|ち}へも
\ruby{此方}{こつ|ち}へも
\ruby{氣}{き}がねを
\ruby{仕}{し}て、
\ruby{一年中}{いち|ねん|ぢゆう}べそをかいて
\ruby{居}{ゐ}なけりやあならないんだ!。
\ruby{何樣}{ど|う}だ、
\ruby{{\換字{強}}惡}{がう|あく}に
\ruby{限}{かぎ}るだらう!。
\ruby{一體眞實}{いつ|たい|ほん|たう}の
\ruby{理屈}{り|くつ}から
\ruby{云}{い}やあ、
\ruby{此}{こ}の
\ruby{娘}{むすめ}の
\ruby{方}{はう}が
\ruby{善}{ぜん}なのだからナア。
』

『ウン、
\ruby{成程}{なる|ほど}
\ruby[g]{々々}{〳〵}。
\ruby{{\換字{強}}惡}{がう|あく}は
\ruby{眞實}{ほん|と}に
\ruby{偉}{えら}いナア!。
だけれど
\ruby{憫然}{かはい|さう}に
\ruby{今}{いま}の
\ruby[g]{世界}{せかい}ぢやあ、
\ruby{男子}{をの|こ}でも
\ruby{此}{こ}の
\ruby{娘}{むすめ}のやうな
\ruby{奴}{やつ}ばかり
\ruby{多}{おほ}いぜ!。
ハヽヽ。
』

『ハヽハヽヽ、
\ruby{左樣}{さ|う}だ、〳〵、
\ruby{笑}{わら}つて
\ruby{{\換字{遣}}}{や}れ、
\ruby{笑}{わら}つて
\ruby{{\換字{遣}}}{や}れ。
アツハツハツハヽヽヽ。
』

『アツハツハツハヽヽヽ。
』

\ruby{{\換字{朝}}詣}{あさ|まゐ}りする
\ruby{人}{ひと}のちらほらとは
\ruby{見}{み}え
\ruby{初}{そ}めたれど、
\ruby{{\換字{猶}}}{なほ}
\ruby{極}{きは}めて
\ruby{四邊}{あ|たり}の
\ruby{物靜}{もの|しづ}かなれば、
\ruby{聞}{き}けよがしに
\ruby{聲大}{こゑ|おほき}く
\ruby{語}{かた}らふ
\ruby{二人}{ふた|り}の
\ruby{談}{はなし}は、
\ruby{既}{すで}に
\ruby{御堂}{み|だう}を
\ruby{離}{はな}れて
\ruby{石路}{せき|ろ}を
\ruby{歩}{あゆ}める
\ruby{水野}{みづ|の}と
\ruby{彼}{か}の
\ruby{老}{お}いたる
\ruby{男}{をとこ}との
\ruby{背後}{うし|ろ}より
\ruby{響}{ひゞ}きて、
\ruby{態}{わざ}とらしき
\ruby{其}{そ}の
\ruby{嘲}{あざけ}り
\ruby{笑}{わら}ひも
\ruby{一々}{いち|〳〵}
\ruby{聞}{きこ}こえたり。
今しも
\ruby{水野}{みづ|の}と
\ruby{並}{なら}びて
\ruby{歩}{ある}ける
\ruby{彼}{か}の
\ruby{男}{をとこ}は
\ruby{再}{ふたた}び
\ruby{水野}{みづ|の}と
\ruby{面}{おもて}を
\ruby{見合}{み|あ}はせつ、
\ruby{{\換字{終}}}{つひ}に
\ruby{堪}{た}へ
\ruby{{\換字{兼}}}{か}ねてか
\ruby{口}{くち}を
\ruby{開}{ひら}き、

『
\ruby{大變}{たい|へん}な
\ruby{世}{よ}の
\ruby{中}{なか}になつてまゐりました!。
\ruby{私共}{わたし|ども}の
\ruby{倅}{せがれ}なんぞも
\ruby{學校}{ぐあ|かう}へ
\ruby{{\換字{遣}}}{や}つて
\ruby{置}{お}きましたら、まあ
\ruby{矢張}{や|は}り
\ruby{彼樣}{あ|ゝ}いつた
\ruby{調子}{てう|し}になりまして、
\ruby{人}{ひと}に
\ruby{苦勞}{く|らう}ばかりいたさせます。
\ruby{御参}{おま|ゐり}を
\ruby{致}{いた}しますのも、
\ruby{實}{じつ}を
\ruby{申}{まを}しますと、つまりは
\ruby{其樣}{そ|ん}な
\ruby{譯}{わけ}から
\ruby{起}{おこ}つた
\ruby{事}{こと}のためでございますが、……』

と、
\ruby{思}{おも}ひ
\ruby{餘}{あま}つたる
\ruby{憂}{う}さを
\ruby{漏}{も}らしかけしが、
\ruby{流石}{さす|が}に
\ruby{心}{こゝろ}づきて、
\ruby{馴染無}{な|じみ|な}き
\ruby{人}{ひと}に
\ruby{吾}{わ}が
\ruby{家内}{い|へ}の
\ruby{事}{こと}を
\ruby{言}{い}はんもはしたなしとてや、

『
\ruby{御利生}{ご| り |しやう}を
\ruby{現}{あら}はさうとして
\ruby{書}{か}きました
\ruby{額}{がく}を
\ruby{見}{み}て、
\ruby{一}{ひと}つ
\ruby{家}{や}の
\ruby{婆}{ばあ}さんの
\ruby{方}{はう}を
\ruby{褒}{ほ}めますなんて、ほんに
\ruby{淺草寺}{せん|さう|じ}はじまつて
\ruby{以來無}{この|かた|な}い
\ruby{事}{こと}でございましやう!。
まあ
\ruby{何}{なん}といふ
\ruby{間{\換字{違}}}{ま|ちが}つた
\ruby{事}{こと}で!。
』

と、
\ruby{談}{はなし}を
\ruby{横}{よこ}に
\ruby{{\換字{逸}}}{そ}らしたり。
\ruby{水野}{みづ|の}は
\ruby{當}{あ}たり
\ruby{障}{さは}らずに、

『まことに
\ruby{左樣}{さ|やう}でござります。
』

と、
\ruby{穩}{おだ}やかに
\ruby{答}{こた}へて
\ruby{多}{おほ}くは
\ruby{言}{ものい}はず、たゞ
\ruby{人}{ひと}の
\ruby{親}{おや}には
\ruby{{\換字{情}}}{なさけ}
\ruby{篤}{あつ}きが
\ruby{多}{おほ}きに、
\ruby{人}{ひと}の
\ruby{子}{こ}にはまた
\ruby{彼等二人}{かれ|ら|ふた|り}の
\ruby{如}{ごと}く
\ruby{心放縱}{こゝろ|ほしい|まゝ}なるが
\ruby{多}{おほ}き
\ruby{世}{よ}の
\ruby{相}{すがた}の、さま〴〵なるを
\ruby{思}{おも}ひて
\ruby{歎}{たん}じながらも、
\ruby{今}{いま}の
\ruby{書生}{しよ|せい}の
\ruby{笑}{わら}ひ
\ruby{聲}{ごゑ}には、
\ruby{少}{すくな}からず
\ruby{不快}{ふ|くわい}を
\ruby{覺}{おぼ}えたり。

\ruby{自}{みづか}ら
\ruby{知}{し}る
\ruby{我}{わ}が
\ruby[g]{昨夕}{ゆふべ}のありさまは、
\ruby{取}{と}りも
\ruby{直}{なほ}さず
\ruby{旅}{たび}の
\ruby{人}{ひと}を
\ruby{護}{かば}へる
\ruby{彼}{か}の
\ruby{娘}{むすめ}にも
\ruby{似}{に}て、
\ruby{病}{や}める
\ruby{五十子}{い|そ|こ}を
\ruby{恤}{いたは}らんがためとて、
\ruby{一}{ひと}つ
\ruby{家}{や}の
\ruby{婆}{ばゞ}にも
\ruby{似}{に}たらん
\ruby{彼}{か}の
お
\ruby{澤}{さは}
\ruby{婆}{ばゞあ}に、
\ruby{下}{さ}げがたき
\ruby{頭}{かしら}を
\ruby{幾度}{いく|たび}も
\ruby{{\換字{益}}無}{えき|な}く
\ruby{下}{さ}げて、\換字{志}かも
\ruby{{\換字{益}}無}{えき|な}く
\ruby{云}{い}ひ
\ruby{斥}{しりぞ}けられたる
\ruby{其事}{その|こと}の
\ruby{今}{いま}さら
\ruby{胸}{むね}に
\ruby{{\換字{浮}}}{うか}み
\ruby{來}{く}れば、
\ruby{當無}{あて|な}く
\ruby{放}{はな}ちたるには
\ruby{疑}{うたが}ひ
\ruby{無}{な}き
\ruby{嘲笑}{わら|ひ}の
\ruby{矢}{や}も、\換字{志}たゝかに
\ruby{我}{わ}が
\ruby{背}{そびら}に
\ruby{立}{た}てる
\ruby{心地}{こゝ|ち}して、
\ruby{厭}{いと}はしき
\ruby{思}{おもひ}の
\ruby{比}{たと}ふるに
\ruby{物無}{もの|な}く、
\ruby{身}{み}の
\ruby{内}{うち}を
\ruby{掻}{か}き
\ruby{挘}{むし}りたきやうなる
\ruby{感}{かん}じを
\ruby{{\換字{懐}}}{いだ}きつゝ、
\ruby{夢路}{ゆめ|ぢ}を
\ruby{辿}{たど}るが
\ruby{如}{ごと}く
\ruby{中店}{なか|みせ}を
\ruby{出}{で}はづるれば、

『ヤ、
\ruby{水野}{みづ|の}さん。
』

と、
\ruby{凉}{すゞ}しき
\ruby{聲}{こゑ}の
\ruby{玉}{たま}を
\ruby{轉}{まろ}ばすが
\ruby{如}{ごと}くに
\ruby{呼}{よ}びかけて、
\ruby{黑革}{くろ|かは}の
\ruby{眉庇付}{まび|さし|つ}きたる
\ruby{帽}{ぼう}を
\ruby{傾}{かたぶ}けつゝ、
\ruby{身}{み}を
\ruby{前屈}{まへ|かゞ}みにして
\ruby{走}{はし}り
\ruby{來}{きた}れる
\ruby{美少年}{び|せう|ねん}あり。
\ruby{彼}{か}の
\ruby{老}{お}いたる
\ruby{男}{をとこ}は
\ruby{既}{すで}に
\ruby{去}{さ}つて
\ruby{在}{あ}らず。

