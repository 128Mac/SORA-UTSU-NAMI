\Entry{其二十三}

% メモ 校正終了 2024-04-09 2024-05-26 2024-06-18
\原本頁{137-5}%
\ruby{天}{そら}の
\ruby[g]{彼方}{あなた }に
\ruby[||j>]{颶}{つむじ}
\ruby[||j>]{風}{ かぜ}を
% \ruby{颶風}{つむじ|かぜ}を
\ruby{起}{おこ}しゝ
ニイチエが
\ruby[g]{眞趣}{おもむき}を
\ruby{實}{まこと}に
\ruby{知}{し}れりや、
%
それも
\ruby[g]{覺束}{おぼつか}
\ruby{無}{な}げなる
\ruby[g]{書生}{しよせい}の
\ruby[g]{放言}{はうげん}の、
%
\ruby{餘}{あま}りの
\ruby{事}{こと}に
\ruby[<j>]{傍}{かたはら}
\ruby[||j>]{痛}{ いた}くは
おぼえたれど、
%
\ruby{意}{こゝろ}を
\ruby{動}{うご}かすほどにも
\ruby{至}{いた}らざりければ、
%
\ruby{他}{ひと}は
\ruby{他}{ひと}なり、
%
\ruby{我}{われ}は
\ruby{我}{われ}
\原本頁{137-8}\改行%
なり、
%
\ruby[||j>]{關}{かけ}
\ruby[||j>]{係}{かまひ}
% \ruby{關係}{かけ|かまひ}
\ruby{無}{ な }き% 「關係(かけかまひ)」が突き出て来ているので全角空白で補正
\ruby{禽}{とり}の
\ruby{聲}{こゑ}の、
%
それまでの
\ruby{事}{こと}なりと
\ruby{聞}{き}き
\ruby{捨}{す}てゝ、
%
\ruby{既}{すで}
\原本頁{137-9}\改行%
に
『
ツアラツウストラ
\ruby{如是說}{によ|ぜ|せつ}
』をも
\ruby{窺}{うかゞ}ひ
\ruby{讀}{よ}まぬに
あらざりし
\ruby[g]{水野}{みづの }は、
\換字{志゛}ろりと% 「志」+「濁点」
\ruby{冷}{ひや}やかに
\ruby{彼}{か}の
\ruby[g]{二人}{ふたり }をば
\ruby[g]{一瞥}{いちべつ}せしのみに
\ruby{止}{とゞ}まりて
\改行% 校正作業の簡略化のため
、
%
\原本頁{138-1}\改行%
\ruby[g]{徐々}{やおら}
\ruby[g]{此處}{こ ゝ }を
\ruby{去}{さ}らんと
\ruby{歩}{あゆ}み
\ruby{出}{いだ}せば、
%
\ruby{彼}{か}の
\ruby{老}{お}いたる
\ruby{男}{をとこ}も
\ruby[g]{一齊}{と も }にと
\ruby[<j||]{隨}{したが}へり。% 行末行頭の境界付近なので特例処置を施す

\原本頁{138-3}%
\ruby{世}{よ}の
\ruby[g]{態人}{さまひと}の
\ruby[<j||]{{\換字{情}}}{こゝろ}
\ruby[||j>]{漸}{やうや}く
\ruby{移}{うつ}りて、
%
\ruby[<j>]{礎}{いしずゑ}は
\ruby{舊}{きう}に
\ruby{依}{よ}りて
\ruby{固}{かた}く、
%
\ruby{棟}{むね}は
\ruby{舊}{きう}に
\ruby{依}{よ}りて
\ruby{高}{たか}けれども、
%
\ruby{今}{いま}は
\ruby{此}{こ}の
\ruby[g]{莊嚴}{さうごん}なる
\ruby[g]{御堂}{み だう}の
\ruby{内}{うち}にさへも、
%
\ruby[g]{謗法}{はうばふ}
\ruby[g]{毀佛}{き ぶつ}の
\ruby{暴}{あば}れ
\ruby{聲}{ごゑ}
\ruby{起}{おこ}りて、
%
\ruby[g]{譬喩}{たとへ }を
\ruby{取}{と}りて
\ruby{云}{い}はゞ
\ruby[g]{月黑}{つきくろ}き
\ruby{夜}{よ}の
\ruby[g]{大潮}{おほしほ}の、
%
\ruby{洲}{す}を
\原本頁{138-6}\改行%
\ruby{吞}{の}み
\ruby{岩}{いは}を
\ruby{嚙}{か}みて
\ruby{漸}{やうや}く
\ruby[g]{大地}{だいち }を
\ruby{犯}{をか}さんとするが
\ruby{如}{ごと}くに、
%
\ruby[g]{何時}{い つ }となく
\原本頁{138-7}\改行%
\ruby[g]{破壞}{はくわい}の
\ruby[g]{吶喊}{さけび }の
\ruby[g]{押寄}{おしよ }するは、
%
\ruby[g]{{\換字{所}}謂}{いはゆる}
\ruby[g]{末法}{まつぱふ}
\ruby[g]{澆季}{げうき }の% 人情が薄く、世の乱れた時代。末の世。
\ruby[g]{是非}{ぜ ひ }も
\ruby{無}{な}き
\ruby[g]{當時}{い ま }の
\ruby[g]{大勢}{いきほい}なり。
%
\ruby[g]{書生}{しよせい}は
\ruby{{\換字{猶}}}{なほ}
がたり
ごとりと、
%
\ruby[||j>]{力}{ちから}
\ruby[||j>]{足}{ あし}を
\ruby{踏}{ふ}み
\ruby{{\換字{杖}}}{つゑ}を
\ruby{突}{つ}き
\ruby{立}{た}てゝ
\原本頁{138-9}\改行%
\ruby{歩}{ある}き
\ruby{居}{ゐ}しが、
%
\ruby[||j>]{紺}{こん}
\ruby[||j>]{絞}{しぼり}の
% \ruby{紺絞}{こん|しぼり}の
\ruby{帶}{おび}したるは、
%
\ruby{急}{きふ}に
\ruby[g]{虛{\換字{空}}}{こ くう}に
\ruby{{\換字{杖}}}{つゑ}を
\ruby{擧}{あ}げて、
%
\ruby{揭}{かゝ}げられたる
\ruby{額}{がく}の
\ruby{一}{ひと}つを
\ruby{指}{さ}しながら、

\原本頁{138-11}%
『
\ruby[g]{面白}{おもしろ}いナア、
%
\ruby{此}{こ}の
\ruby{一}{ひと}つ
\ruby{家}{や}の
\ruby{畫}{ゑ}は!、
% 浅茅ヶ原の鬼婆(あさぢがはらのおにばば)は、東京都台東区花川戸に伝わる伝説。一つ家の鬼婆(ひとつやのおにばば)、
% 一つ家(ひとつや)あるいは土地の名前だけをとり浅茅ヶ原(あさぢがはら)とも称される。
% 浅草寺(東京都台東区)の観音菩薩にまつわる伝説として江戸時代以後には書籍や演芸・芝居なども取り上げられ、
% 広く知られていった。一軒家に棲む老女が宿泊する旅人をあやめて金品を奪っていたなどとする話は各地にみられ、
% これもそのうちの一例と見ることができる。
% 宿泊客を殺害し金品を奪う行為を咎めようとした娘が稚児に分して宿泊客となるが、その稚児(自分の娘)も殺す。
% 老婆が自分の行いを悔いていたところ、家を訪れていた稚児が現れた。実は稚児は浅草寺の観音菩薩の化身であり、
% 老婆に人道を説くために稚児の姿で家を訪れたのだった。
% その後、観音菩薩の力で竜と化した老婆が娘の亡骸とともに池へ消えたとも、
% 観音菩薩が娘の亡骸を抱いて消えた後、% 老婆が池に身を投げたとも、
% 老婆は仏門に入って死者たちを弔ったともいわれている
%
どうも
\ruby{巧}{よ}く
\ruby[g]{出來}{で き }て
\ruby{居}{ゐ}るナ、
%
\ruby{氣}{き}に
\ruby{入}{い}つたナア!。
』

\原本頁{139-2}%
と
\ruby{云}{い}へば、

\原本頁{139-3}%
『
ムヽ、
』

\原本頁{139-4}%
と、
%
\ruby{白}{しろ}き
\ruby{帶}{おび}したるは
\ruby{其}{その}
\ruby{意}{い}を
\ruby{得}{え}ぬげに
\ruby{應}{こた}へつ、

\原本頁{139-5}%
『\換字{志}かし
\ruby[g]{御厩}{みうまや}の
\ruby{喜三太}{き|さん|だ}も
% 御厩喜三太は、室町時代初期に成立した軍記・伝記物語『義経記』に登場する架空の人物。源義経の郎党の一人
\ruby{好}{い}いぢやあ
\ruby{無}{な}いか。
』

\原本頁{139-6}%
と
\ruby[g]{附加}{つけくは}へたり。

\原本頁{139-7}%
『
\ruby[g]{馬鹿}{ば か }ツ!。
%
そりやあ
\ruby[g]{{\換字{技}}{\換字{術}}}{ぎじゆつ}だけの
\ruby{論}{ろん}だ。
%
\ruby{云}{い}ふなあ
\ruby[g]{其處}{そ こ }ぢやあ
\ruby{無}{な}い。
%
よく
\ruby{見}{み}ろ!% \inhibitglue{}% ここは「空き」があるので
\,% 原本上でのアキを再現するため「3/18 em」空ける
\ruby[g]{吾輩}{わがはい}の
\ruby{此}{こ}の
\ruby{一}{ひと}つ
\ruby{家}{や}の
\ruby{圖}{づ}を!。
%
\ruby[g]{何樣}{ど う }だ
\ruby{彼}{あ}の
\ruby{婆}{ばあ}さ
\原本頁{139-9}\改行%
んの
\ruby{顏}{かほ}の
\ruby[g]{立派}{りつぱ }なこと!。
%
\ruby{實}{じつ}に
\ruby[g]{立派}{りつぱ }ぢやあ
\ruby{無}{な}いか、
%
\ruby[g]{立派}{りつぱ }ぢやあ
\ruby{無}{な}いか!。
%
\ruby[g]{國家}{こくか }の
\ruby[g]{法律}{はふりつ}
なんぞといふ
\ruby{奴}{やつ}ア
\ruby{踏}{ふ}み
\ruby{付}{つ}け
\ruby{切}{き}つた
\ruby{彼}{あ}の
\ruby{顏}{かほ}つ
\原本頁{139-11}\改行%
き!。
%
\ruby[g]{世間}{せ けん}の
\ruby[g]{善惡}{ぜんあく}の
\ruby[g]{沙汰}{さ た }
なんぞを
\ruby{寄}{よ}せつけも
\ruby[g]{仕無}{し な }い
\ruby{彼}{あ}の
\ruby[g]{顏付}{かほつき}!
\改行% 校正作業の簡略化のため
。
%
\原本頁{140-1}\改行%
\ruby{戀}{こひ}も
\ruby[||j>]{人}{にん}
\ruby[||j>]{{\換字{情}}}{じやう}も
% \ruby{人{\換字{情}}}{にん|じやう}も
\ruby{無}{な}い
\ruby{彼}{あ}の
\ruby{顏}{かほ}つき!。
%
\ruby{邪}{じや}でも
\ruby{非}{ひ}でも
まかはない% (かまはない)と思われるが原本通り
\footnote{「まかはない」は「かまはない」の誤植と思われるが原本通りとする
(国会図書館 コマ番号 74/134 p 140 l-1)}%
\ruby{彼}{あ}の
\ruby{顏}{かほ}つき!。
%
おれが
\ruby[g]{{\換字{勝}}手}{かつて }だぞといふ
\ruby{彼}{あ}の
\ruby{顏}{かほ}つき!。
%
\ruby{神}{かみ}でも
\ruby{佛}{ほとけ}でも
\ruby[g]{對面}{むかふ }へ
まはつたら
\ruby[g]{斫殺}{たゝつき}つて
\ruby{{\換字{遣}}}{や}らうといふ
\ruby{彼}{あ}の
\ruby{顏}{かほ}つき!。
%
あゝ
\ruby{何}{なん}と
\原本頁{140-4}\改行%
\ruby[g]{立派}{りつぱ }な
\ruby{顏}{かほ}に
\ruby{書}{か}いてあるでは
\ruby{無}{な}いか。
%
\ruby[g]{十{\換字{分}}}{じふぶん}に
\ruby[g]{惡人}{あくにん}の
\ruby[g]{偉大}{ゐ だい}な
\ruby[g]{精神}{せいしん}が
\原本頁{140-5}\改行%
\ruby[g]{發揮}{はつき }してある!。
%
\ruby{誰}{だれ}だつて
\ruby{此}{こ}の
\ruby{繪}{ゑ}を
\ruby{能}{よ}く
\ruby{見}{み}たらば、
%
\ruby[g]{{\換字{強}}惡}{がうあく}が
\ruby{美}{い}いものだといふ
\ruby{事}{こと}に
\ruby{氣}{き}が
\ruby{付}{つ}くだらう!。
%
\ruby{見}{み}ろ、
%
\ruby{彼}{あ}の
\ruby{娘}{むすめ}が
\ruby[g]{卑小}{け ち }な
\ruby{惡}{わる}びれた
\ruby[g]{樣子}{やうす }を!。
%
\ruby{人}{ひと}に
\ruby{縋}{すが}りたがるやうな、
%
\ruby[g]{哀愍}{あはれみ}を
\ruby{乞}{こ}ふやうな、
%
\原本頁{140-8}\改行%
\ruby{泣}{な}き
\ruby{出}{だ}しさうな、
%
\ruby{切}{せつ}なさうな、
%
\ruby[g]{善惡}{ぜんあく}の
\ruby[g]{{\換字{道}}理}{だうり }を
\ruby{怖}{こは}がつて
\ruby{居}{ゐ}るやう
\原本頁{140-9}\改行%
な、
%
\ruby[g]{國家}{こくか }の
\ruby[g]{規律}{おきて }なんぞに
びく〴〵して% 原本では判読しがたいが(びくびく)と思われるので繰り返し部分も濁点のものにした
\ruby{居}{ゐ}るやうな、
%
\ruby[g]{神佛}{しんぶつ}
なんぞ
\原本頁{140-10}\改行%
に
おど〳〵して
\ruby{居}{ゐ}る、
%
\換字{志}みつたれた、
%
\ruby{見}{み}つとも
\ruby{無}{な}い
\ruby[g]{醜態}{ざ ま }が、
%
す
\原本頁{140-11}\改行%
つかり
\ruby{見}{み}えて
\ruby{居}{ゐ}る!。
%
\ruby[g]{{\換字{所}}謂}{いはゆる}
\ruby[g]{善人}{ぜんにん}といふ
\ruby{奴}{やつ}が
\ruby[g]{卑劣}{け ち }なもので、
%
\ruby{下}{くだ}ら
\原本頁{141-1}\改行%
ないものだといふ
\ruby{事}{こと}は、
%
\ruby[g]{何樣}{ど ん }な
\ruby[g]{馬鹿}{ば か }な
\ruby{奴}{やつ}の
\ruby{眼}{め}にも
\ruby{暎}{うつ}る% 草冠付きの「暎」
だらう!
\改行% 校正作業の簡略化のため
。
%
\原本頁{141-2}\改行%
\ruby[g]{何樣}{ど う }だ、
%
\ruby{好}{い}いぢやあ
\ruby{無}{な}いか、
%
\ruby{好}{い}い
\ruby{畫}{ゑ}ぢやあ
\ruby{無}{な}いか。
%
\ruby[g]{何樣}{ど う }だ、
%
\ruby{{\換字{分}}}{わか}つたか、
%
\ruby{好}{い}いか、
%
オイ、
%
\ruby{君}{きみ}!。
%
\ruby{此}{こ}の
\ruby{一}{ひと}つ
\ruby{家}{や}の
\ruby[g]{御婆}{お ばあ}さんが
\ruby[g]{國王}{こくわう}になりやあ、
%
\ruby{世界中}{せ|かい|ぢゆう}を
\ruby{斬}{き}り
\ruby{伏}{ふ}せて
\ruby[g]{寢酒}{ね ざけ}の
\ruby[g]{下物}{さかな }に
\ruby{仕}{し}て
\ruby{{\換字{遣}}}{や}らうと、
%
\ruby{手}{て}に
\ruby{持}{も}つた
\ruby[g]{利器}{え もの}を
\ruby{振}{ふ}り
\ruby{舞}{ま}はすんだ!。
%
もし
\ruby{此}{こ}の
\ruby{娘}{むすめ}が
\ruby[g]{國王}{こくわう}に
なりやあ、
%
\ruby[g]{彼方}{あつち }へも
\ruby[g]{此方}{こつち }へも% ルビ調整(原本通り)
\ruby{氣}{き}がねを
\ruby{仕}{し}て、
%
\ruby{一年中}{いち|ねん|ぢゆう}
べそを
かいて
\ruby{居}{ゐ}なけりやあならないんだ!。
%
\ruby[g]{何樣}{ど う }だ、
%
\ruby[g]{{\換字{強}}惡}{がうあく}に
\ruby{限}{かぎ}るだらう!。
%
\ruby[g]{一體}{いつたい}
\原本頁{141-8}\改行%
\ruby[g]{眞實}{ほんたう}の
\ruby[g]{理屈}{り くつ}から
\ruby{云}{い}やあ、
%
\ruby{此}{こ}の
\ruby{娘}{むすめ}の
\ruby{方}{はう}が
\ruby{惡}{あく}で
\ruby{婆}{ばあ}さんの
\ruby{方}{はう}が
\ruby{善}{ぜん}なのだからナア。
』

\原本頁{141-10}%
『
ウン、
%
\ruby[g]{成程}{なるほど}
\ruby[g]{々々}{ 〳〵 }。
%
\ruby[g]{{\換字{強}}惡}{がうあく}は
\ruby[g]{眞實}{ほんと }に
\ruby{偉}{えら}いナア!。
%
だけれど
\ruby[||j>]{憫}{かは}
\ruby[||j>]{然}{いさう}に% 「憫然 か(は)いさう」
% \ruby{憫然}{かは|いさう}に% 「憫然 か(は)いさう」
\原本頁{141-11}\改行%
\ruby{今}{いま}の
\ruby[g]{世界}{せ かい}ぢやあ、
%
\ruby[g]{男子}{をとこ }でも
\ruby{此}{こ}の
\ruby{娘}{むすめ}のやうな
\ruby{奴}{やつ}ばかり
\ruby{多}{おほ}いぜ!。
%
\原本頁{142-1}\改行%
ハヽヽ。
』

\原本頁{142-2}%
『
ハヽハヽヽ、
%
\ruby[g]{左樣}{さ う }だ、
%
〳〵、
%
\ruby{笑}{わら}つて
\ruby{{\換字{遣}}}{や}れ、
%
\ruby{笑}{わら}つて
\ruby{{\換字{遣}}}{や}れ。
%
アツハツハツハヽヽヽ。
』

\原本頁{142-4}%
『
アツハツハツハヽヽヽ。
』

\原本頁{142-5}%
\ruby[g]{{\換字{朝}}詣}{あさまゐ}りする
\ruby{人}{ひと}の
ちらほらとは
\ruby{見}{み}え
\ruby{初}{そ}めたれど、
%
\ruby{{\換字{猶}}}{なほ}
\ruby{極}{きは}めて
\ruby[g]{四邊}{あたり }の
\原本頁{142-6}\改行%
\ruby[g]{物靜}{ものしづ}かなれば、
%
\ruby{聞}{き}けよがしに
\ruby{聲}{こゑ}
\ruby{大}{おほき}く
\ruby{語}{かた}らふ
\ruby[g]{二人}{ふたり }の
\ruby{談}{はなし}は、
%
\ruby{既}{すで}に
\ruby[g]{御堂}{み だう}を
\ruby{離}{はな}れて
\ruby[g]{石路}{せきろ }を
\ruby{歩}{あゆ}める
\ruby[g]{水野}{みづの }と
\ruby{彼}{か}の
\ruby{老}{お}いたる
\ruby{男}{をとこ}との
\ruby[g]{背後}{うしろ }より
\ruby{響}{ひゞ}きて、
%
\ruby{態}{わざ}とらしき
\ruby{其}{そ}の
\ruby{嘲}{あざけ}り
\ruby{笑}{わら}ひも
\ruby[g]{一々}{いち〳〵}
\ruby{聞}{きこ}えたり。
%
\ruby{今}{いま}しも
\ruby[g]{水野}{みづの }と
\原本頁{142-9}\改行%
\ruby{並}{なら}びて
\ruby{歩}{ある}ける
\ruby{彼}{か}の
\ruby{男}{をとこ}は
\ruby{再}{ふたゝ}び
\ruby[g]{水野}{みづの }と
\ruby{面}{おもて}を
\ruby[g]{見合}{み あ }はせつ、
%
\ruby{{\換字{終}}}{つひ}に
\ruby{堪}{た}へ
\ruby{{\換字{兼}}}{か}ねてか
\ruby{口}{くち}を
\ruby{開}{ひら}き、

\原本頁{142-11}%
『
\ruby[g]{大變}{たいへん}な
\ruby{世}{よ}の
\ruby{中}{なか}に
なつてまゐりました!。
%
\ruby[||j>]{私}{わたし}
\ruby[||j>]{共}{ ども}の
% \ruby{私共}{わたし|ども}の
\ruby{倅}{せがれ}なんぞも
\ruby[g]{學校}{がくかう}へ
\ruby{{\換字{遣}}}{や}つて
\ruby{置}{お}きましたら、
%
まあ
\ruby[g]{矢張}{や は }り
\ruby[g]{彼樣}{あ ゝ }いつた
\ruby[g]{調子}{てうし }に
なりま
\原本頁{143-2}\改行%
して、
%
\ruby{人}{ひと}に
\ruby[g]{苦勞}{く らう}ばかり
いたさせます。
%
\ruby[g]{御參}{おまゐり}を
\ruby{致}{いた}しますのも
%
\ruby{實}{じつ}を
\原本頁{143-3}\改行%
\ruby{申}{まを}しますと、
%
つまりは
\ruby[g]{其樣}{そ ん }な
\ruby{譯}{わけ}から
\ruby{起}{おこ}つた
\ruby{事}{こと}の
ためでございますが、‥‥』

\原本頁{143-5}%
と、
%
\ruby{思}{おも}ひ
\ruby{餘}{あま}つたる
\ruby{憂}{う}さを
\ruby{漏}{も}らしかけしが、
%
\ruby[g]{流石}{さすが }に
\ruby{心}{こゝろ}づきて、
%
\ruby[g]{馴染}{なじみ }% 「{馴染}{なじみ}」だと思うが原本通り
\ruby{無}{な}き
\ruby{人}{ひと}に
\ruby{吾}{わ}が
\ruby[g]{家内}{い へ }の
\ruby{事}{こと}を
\ruby{言}{い}はんも
はしたなしとてや、

\原本頁{143-7}%
『
\ruby{御}{ご}
\ruby[g]{利生}{りしやう}を
\ruby{現}{あら}はさうとして
\ruby{書}{か}きました
\ruby{額}{がく}を
\ruby{見}{み}て、
%
\ruby{一}{ひと}つ
\ruby{家}{や}の
\ruby{婆}{ばあ}さんの
\ruby{方}{はう}を
\ruby{褒}{ほ}めますなんて、
%
ほんに
\ruby{淺草寺}{せん|さう|じ}
はじまつて
\ruby[g]{以來}{このかた}
\ruby{無}{な}い
\ruby{事}{こと}で
ございましやう!。
%
まあ
\ruby{何}{なん}といふ
\ruby[g]{間{\換字{違}}}{ま ちが}つた
\ruby{事}{こと}で!。
』

\原本頁{143-10}%
と、
%
\ruby{談}{はなし}を
\ruby{横}{よこ}に
\ruby{逸}{そ}らしたり。
%
\ruby[g]{水野}{みづの }は
\ruby{當}{あた}り
\ruby{障}{さは}らずに、

\原本頁{143-11}%
『
まことに
\ruby[g]{左樣}{さ やう}で
ござります。
』

\原本頁{144-1}%
と、
%
\ruby{穩}{おだ}やかに
\ruby{答}{こた}へて
\ruby{多}{おほ}くは
\ruby{言}{ものい}はず、
%
たゞ
\ruby{人}{ひと}の
\ruby{親}{おや}には
\ruby[||j>]{{\換字{情}}}{なさけ}
\ruby[||j>]{篤}{ あつ}きが
\ruby{多}{おほ}きに、
%
\ruby{人}{ひと}の
\ruby{子}{こ}には
また
\ruby[g]{彼等}{かれら }
\ruby[g]{二人}{ふたり }の
\ruby{如}{ごと}く
\ruby[<j||]{心}{こゝろ}
\ruby[||j>]{放}{ほしい}
\ruby[||j>]{縱}{ まゝ}なるが
\ruby{多}{おほ}き
\ruby{世}{よ}の
\ruby[<j||]{相}{すがた}の、% 行末行頭の境界付近なので特例処置を施す
%
さま〴〵なるを
\ruby{思}{おも}ひて
\ruby{歎}{たん}じながらも、
%
\ruby{今}{いま}の
\ruby[g]{書生}{しよせい}の
\ruby{笑}{わら}ひ
\ruby{聲}{ごゑ}には、
%
\ruby{少}{すくな}からず
\ruby[g]{不快}{ふくわい}を
\ruby{覺}{おぼ}えたり。

\原本頁{144-5}%
\ruby{自}{みづか}ら
\ruby{知}{し}る
\ruby{我}{わ}が
\ruby[g]{昨夕}{ゆふべ }の
ありさまは、
%
\ruby{取}{と}りも
\ruby{直}{なほ}さず
\ruby{旅}{たび}の
\ruby{人}{ひと}を
\ruby{護}{かば}へる
\原本頁{144-6}\改行%
\ruby{彼}{か}の
\ruby{娘}{むすめ}にも
\ruby{似}{に}て、
%
\ruby{病}{や}める
\ruby{五十子}{い|そ|こ}を
\ruby{恤}{いたは}らんが
ためとて、
%
\ruby{一}{ひと}つ
\ruby{家}{や}の
\ruby{婆}{ばゞ}にも
\ruby{似}{に}たらん
\ruby{彼}{か}の
お
\ruby{澤}{さは}
\ruby[||j>]{婆}{ばゞあ}に、
%
\ruby{下}{さ}げがたき
\ruby{頭}{かしら}を
\ruby[g]{幾度}{いくたび}も
\ruby{益}{えき}
\ruby{無}{な}く
\ruby{下}{さ}げて、
%
\換字{志}かも
\ruby{益}{えき}
\ruby{無}{な}く
\ruby{云}{い}ひ
\ruby{{\換字{斥}}}{しりぞ}けられたる
\ruby{其}{その}
\ruby{事}{こと}の
\ruby{今}{いま}さら
\ruby{胸}{むね}に
\ruby{{\換字{浮}}}{うか}み
\ruby{來}{く}れば、
%
\ruby[g]{當無}{あてな }く
\ruby{放}{はな}ちたるには
\ruby{疑}{うたが}ひ
\ruby{無}{な}き
\ruby[g]{嘲笑}{わらひ }の
\ruby{矢}{や}も、
%
\換字{志}たゝかに
\ruby{我}{わ}が
\ruby{背}{そびら}に% 背中。 うしろ。 「背(そ)平(ひら)」の意
\ruby{立}{た}てる
\ruby[g]{心地}{こゝち }して、
%
\ruby{厭}{いと}はしき
\ruby{思}{おもひ}の
\ruby{比}{たと}ふるに
\ruby{物}{もの}
\ruby{無}{な}く、
%
\ruby{身}{み}の
\ruby{内}{うち}を
\ruby{掻}{か}き
\ruby{挘}{むし}りたきやうなる
\ruby{{\換字{感}}}{かん}じを
\ruby{懷}{いだ}きつゝ、
%
\ruby[g]{夢路}{ゆめぢ }を
\ruby{辿}{たど}るが
\ruby{如}{ごと}く
\ruby[g]{中店}{なかみせ}を
\ruby{出}{で}はづるれば、

\原本頁{145-2}%
『
ヤ、
%
\ruby[g]{水野}{みづの }さん。
』

\原本頁{145-3}%
と、
%
\ruby{凉}{すゞ}しき
\ruby{聲}{こゑ}の
\ruby{玉}{たま}を
\ruby{轉}{まろ}ばすが
\ruby{如}{ごと}くに
\ruby{呼}{よ}びかけて、
%
\ruby[g]{黑革}{くろかは}の
\ruby[g]{眉庇}{まびさし}
\ruby{付}{つ}きたる
\ruby{帽}{ばう}を
\ruby{傾}{かたぶ}けつゝ、
%
\ruby{身}{み}を
\ruby{{\換字{前}}}{まへ}
\ruby{屈}{かゞ}みにして
\ruby{走}{はし}り
\ruby{來}{きた}れる
\ruby{美少年}{び|せう|ねん}あり
\改行% 校正作業の簡略化のため
。
%
\原本頁{145-5}\改行%
\ruby{彼}{か}の
\ruby{老}{お}いたる
\ruby{男}{をとこ}は
\ruby{既}{すで}に
\ruby{去}{さ}つて
\ruby{在}{あ}らず。
