\Entry{其二十四}

% メモ 校正終了 2024-04-09
\原本頁{145-7}%
\ruby{{\換字{近}}}{ちか}づくや
\ruby{否}{いな}や
\ruby{帽}{ばう}を
\ruby{脫}{と}りて、
%
\ruby{眞{\換字{率}}}{しん|そつ}に
\ruby{頭}{かうべ}を
\ruby{下}{さ}げて
\ruby{挨拶}{あい|さつ}するは、
%
\ruby{林檎}{りん|ご}の
\ruby{如}{ごと}く
\ruby{美}{うつく}しき
\ruby{色澤}{いろ|つや}、
%
\ruby[||j>]{人}{にん}
\ruby[||j>]{形}{ぎやう}の
% \ruby{人形}{にん|ぎやう}の
\ruby{如}{ごと}き
\ruby{端正}{た|ゞ}しき
\ruby{眼鼻立}{め|はな|だち}、
%
\ruby{姊}{あね}の
\ruby{男}{をとこ}にしても
\ruby{見}{み}まはしく
\ruby{立派}{りつ|ぱ}なるには
\ruby{異}{かは}りて、
%
\ruby{此}{これ}は
\ruby{女}{をんな}にしても
\ruby{見}{み}たく
\ruby{可愛}{か|はい}らしと
\ruby{人}{ひと}に
\ruby{云}{い}はれたる
\ruby{五十子}{い|そ|こ}が
\ruby{弟}{おとゝ}の
\ruby{松之助}{まつ|の|すけ}なり。

\原本頁{146-1}%
\ruby{母}{はゝ}は
\ruby{有}{あ}りても
\ruby{繼}{まゝ}しき
\ruby{中}{なか}なり、
%
\ruby{財產}{ざい|さん}は
\ruby{繼母}{は|ゝ}に
\ruby{皆}{みな}
\ruby{奪}{と}られたり、
%
\ruby{姊}{あね}より
ほかに
\ruby{頼}{たの}むべき
\ruby{人}{ひと}を
\ruby{有}{も}たぬ
\ruby{松之助}{まつ|の|すけ}は、
%
\ruby{往時}{むか|し}の
\ruby{{\換字{乳}}母}{う|ば}なりしが
\ruby{今}{いま}は
\ruby{下谷}{した|や}の
\ruby{廣小路}{ひろ|こう|ぢ}
\ruby{{\換字{近}}}{ちか}くに、
%
\ruby{下梳}{した|すき}の
\ruby{二人}{ふた|り}も
\ruby{使}{つか}ふほどの
\ruby{女髮結}{か|み|ゆひ}となりて、
%
\ruby{堅}{かた}く
\ruby{身}{み}を
\ruby{持}{も}てる
\ruby{幸福}{しあ|はせ}には%「幸福」ここは「は」
\ruby{苦}{くる}しげ
\ruby{無}{な}く
\ruby{日}{ひ}を
\ruby{{\換字{送}}}{おく}れるが
\ruby{許}{もと}に、
%
\ruby{{\換字{留}}守番}{る|す|ばん}を
\ruby{{\換字{兼}}}{か}ねたる
\ruby{客寓人}{かゝ|り|びと}となりつ、
%
\ruby{月々}{つき|〴〵}
\ruby{姊}{あね}が
\ruby{取}{と}る
\ruby{僅少}{わづ|か}なる
\ruby{給料}{きふ|れう}の
\原本頁{146-6}\改行%
\ruby{内}{うち}より、
%
\ruby{{\換字{分}}}{わ}けて
\ruby{貰}{もら}ふ
\ruby{財布}{さい|ふ}の
\ruby{塵芥}{ご|み}ほどの
\ruby{金子}{か|ね}を、
%
\ruby{一{\換字{半}}}{なか|ば}は
\ruby{形式}{か|た}ばかりの
\ruby[||j>]{食}{しよく}
\ruby[||j>]{料}{ れう}として
% \ruby{食料}{しよく|れう}として
\ruby{入}{い}れ、
%
\ruby{一{\換字{半}}}{なか|ば}は
おのれの
\ruby{學資}{がく|し}として、
%
\ruby{責}{せ}めて
\ruby{某}{それ}の
\原本頁{146-8}\改行%
\ruby{學校}{がく|かう}の
\ruby{官費生}{くわん|ぴ|せい}となりて
\ruby{世}{よ}に
\ruby{立}{た}つ
\ruby{{\換字{道}}}{みち}の
\ruby{緖}{いとぐち}を
\ruby{得}{う}る
\ruby{迄}{まで}と、
%
\ruby{足}{た}らぬ
\ruby{{\換字{勝}}}{がち}なる
\ruby{中}{なか}にも
\ruby{心}{こゝろ}を
\ruby{勵}{はげ}まして、
%
\ruby{夜學}{や|がく}の
\ruby{歸路}{かへ|り}は
\ruby{辛}{つら}き
\ruby{{\換字{冬}}}{ふゆ}の
\ruby{{\換字{雪}}}{ゆき}、
%
\ruby{籠}{こも}り
\ruby{居}{ゐ}の
\ruby{夏}{なつ}は
\ruby{堪}{た}へ
\ruby{{\換字{難}}}{がた}き
\ruby{陋巷}{ろ|じ}の
\ruby{奧}{おく}の
\ruby{矮屋}{こ|いへ}の
\ruby{暑熱}{あつ|さ}にも、
%
\ruby{萎}{め}げず
\ruby{怯}{ひる}まずして
\ruby[||j>]{勉}{べん}
\ruby[||j>]{{\換字{強}}}{きやう}すれば、
% \ruby{勉{\換字{強}}}{べん|きやう}すれば、
%
\ruby{齡}{とし}は
\ruby{{\換字{猶}}}{なほ}
\ruby{數}{かぞ}へ
\ruby{年}{どし}の
\ruby{一七}{じう|しち}にして、
%
\ruby{思想}{かん|がへ}こそは
\ruby{世}{よ}に
\ruby{磨}{す}れざれ、
%
\原本頁{147-1}\改行%
\ruby{學問}{がく|もん}の
\ruby{出來}{で|き}は
いと
\ruby{佳}{よ}くして、
%
\ruby{行末}{ゆく|すゑ}
\ruby{發{\換字{達}}}{なり|い}づべく
\ruby{見}{み}ゆる
\ruby{少年}{せう|ねん}なり。
%
\原本頁{147-2}\改行%
\ruby{繼母}{は|ゝ}は
\ruby{不品行}{ふ|み|もち}にして
\ruby{心}{こゝろ}
\ruby{曲}{ゆが}み、
%
\ruby{有}{あ}りても
\ruby{却}{かへ}つて
\ruby{無}{な}きに
\ruby{劣}{おと}れば、
%
\ruby{天}{てん}にも
\ruby{地}{ち}にも
\ruby{頼}{たの}み
\ruby{頼}{たの}まるべきは
\ruby{只}{たゞ}
\ruby[||j>]{姊}{きよう}
\ruby[||j>]{弟}{ だい}と、
% \ruby{姊弟}{きよう|だい}と、
%
\ruby{深}{ふか}くも
\ruby{此}{こ}の
\ruby{弟}{おとゝ}の
\ruby{上}{うへ}を
のみ
\ruby{思}{おも}ひて、
%
\ruby{自己}{おの|れ}の
\ruby{今}{いま}の
\ruby{身}{み}は
\ruby{差}{さ}し
\ruby{當}{あた}りて
\ruby{田舎}{ゐな|か}の
\ruby{草萊}{く|さ}の
\ruby{間}{あひだ}に
\ruby{埋}{うづ}もれ
\原本頁{147-5}\改行%
\ruby{沒}{かく}るゝとも、
%
\ruby{如何}{い|か}にもして
\ruby{弟}{おとゝ}の
\ruby{{\換字{若}}}{わか}き
\ruby{時}{とき}を
\ruby{徒}{あだ}に
\ruby{{\換字{過}}}{すご}さしめず、
%
\ruby{出來}{で|き}ぬながらも
\ruby{人}{ひと}の
\ruby{後}{のち}に
\ruby{落}{お}ちぬほどには
\ruby{物學}{もの|まな}びを
させて、
%
\ruby{男兒}{をと|こ}
\ruby{一人{\換字{前}}}{いち|にん|まへ}には
\ruby{生}{おふ}し
\ruby{立}{た}て、
%
\ruby{我}{わ}が
\ruby{家}{いへ}の
\ruby{名}{な}をも
\ruby{擧}{あ}げさせん、
%
\ruby{弟}{おとゝ}のためには
\ruby{挿}{さ}したる
\ruby{掻頭}{かん|ざし}を
\ruby{賣}{う}り、
%
\ruby{着}{き}たる
\ruby{衣}{もの}を
\ruby{脫}{ぬ}ぐとも
\ruby{惜}{をし}まじとは、
%
\ruby{五十子}{い|そ|こ}が
\ruby{日頃}{ひ|ごろ}の
\ruby{念慮}{おも|ひ}なりき。

\原本頁{147-10}%
\ruby{秋風}{あき|かぜ}の
\ruby{中}{なか}に
\ruby{嬰兒}{あか|ご}の
\ruby{泣}{な}きても、
%
\ruby{拾}{ひろ}ふ
\ruby{人}{ひと}は
\ruby{少}{すくな}き
\ruby{此}{こ}の
\ruby{冷}{つめた}き
\ruby{世}{よ}に、
%
\ruby{女}{をんな}なり、
%
\ruby{少年}{せう|ねん}なりの、
%
\ruby{孱{\換字{弱}}}{か|よわ}き
\ruby{身}{み}をもて、
%
\ruby{屈}{くつ}すること
\ruby{無}{な}く
\ruby{凜々}{り|ゝ}しくも
\原本頁{148-1}\改行%
\ruby{立}{た}てる、
%
\ruby{此}{こ}の
\ruby{姊}{あね}
%\ %隙間調整 % TODO このフレーズパターンに似たものをどうするか
\ruby{此}{こ}の
\ruby{弟}{おとゝ}の
\ruby{潔}{いさぎよ}くも
\ruby{健}{けなげ}なる
\ruby[||j>]{心}{こゝろ}
\ruby[||j>]{掛}{ がけ}は、
% \ruby{心掛}{こゝろ|がけ}は、
%
\ruby{同}{おな}じく
\ruby{{\換字{貧}}苦}{ひん|く}と
\ruby{戰}{たゝか}ひ
\ruby{來}{きた}れる
\ruby{水野}{みづ|の}が
\ruby{心}{こゝろ}を
\ruby{少}{すくな}からず
\ruby{動}{うご}かして、
%
\ruby{深}{ふか}くも
\ruby{五十子}{い|そ|こ}を
\ruby{思}{おも}ひ
\ruby{思}{おも}ひて
\ruby{忘}{わす}るゝ
\ruby{能}{あた}はざるに
\ruby{至}{いた}りし
\ruby{原因}{いは|れ}の
\ruby{中}{うち}の、
%
\ruby[||j>]{力}{ちから}
\ruby[||j>]{{\換字{強}}}{ づよ}き
% \ruby{力{\換字{強}}}{ちから|づよ}き
\ruby{一}{ひと}つの
\ruby{個條}{か|でう}とはなりぬ。

\原本頁{148-5}%
されば
\ruby{我}{わ}が
\ruby{五十子}{い|そ|こ}が
\ruby{身}{み}にも
\ruby{代}{か}へじと
\ruby{深}{ふか}くも
\ruby{愛}{いつく}しめりと
\ruby{思}{おも}ふにつけて、
%
\ruby{水野}{みづ|の}も
\ruby[<j||]{自然}{おのづ|から}
\ruby{松之助}{まつ|の|すけ}を
\ruby{他}{よそ}ならず
おもへば、
%
\ruby{松之助}{まつ|の|すけ}も
また
\ruby{水野}{みづ|の}を
\ruby{他}{よそ}ならず
\ruby{思}{おも}ひ、
%
\ruby{五十子}{い|そ|こ}が
\ruby{許}{もと}にて
\ruby{相}{あひ}
\ruby{識}{し}りてより、
%
\ruby{四五度}{し|ご|たび}も
\ruby{面}{おもて}を
\ruby{會}{あ}はせたるには
\ruby{{\換字{過}}}{す}ぎねど、
%
\ruby{姊}{あね}の
\ruby{如何}{い|か}なる
\ruby{故}{ゆゑ}にか
\ruby{我}{われ}を
\ruby{好}{この}まざるに
\ruby{似}{に}ず、
%
\ruby{此兒}{こ|れ}は
\ruby{可愛}{か|はい}くも
\ruby{我}{われ}に
\ruby{睦}{むつ}みて、
%
\ruby{我}{われ}を
\ruby{眞}{まこと}の
\ruby{兄}{あに}
なんどの
\ruby{如}{ごと}くに
あしらひ、
%
\ruby{隔意}{へだて|ぎ}も
\ruby{無}{な}く
\ruby{打解}{うち|と}けて
\ruby{語}{かた}らふなり。

\原本頁{148-11}%
\ruby{我}{わ}が
\ruby{思}{おも}ふ
\ruby{人}{ひと}の
\ruby{弟}{おとゝ}と
\ruby{思}{おも}はんには、
%
たとひ
\ruby{色}{いろ}
\ruby{黑}{くろ}く
\ruby{醜}{みにく}くとも、
%
\ruby{{\換字{猶}}}{なほ}
\ruby{厭}{いと}はしき
\ruby{兒}{こ}とは
\ruby{見棄}{み|す}てざらんに、
%
まして
これは
\ruby{玉}{たま}の
\ruby{如}{ごと}く
\ruby{美}{うつく}しくして、
%
\原本頁{149-2}\改行%
\ruby{加之}{しか|も}
\ruby{我}{われ}に
\ruby{親}{したし}めるなり、
%
\ruby{今}{いま}
\ruby{其}{そ}の
\ruby{淸}{すゞ}しき
\ruby{眼}{め}を
\ruby{見張}{み|は}りて
\ruby{懷}{なつか}しげに
\ruby{我}{われ}を
\ruby{見}{み}ながら、

\原本頁{149-4}%
『
\ruby{君}{きみ}!、
%
\ruby{書狀}{てが|み}を
\ruby{有}{あ}り
\ruby{{\換字{難}}}{がた}う!。
%
\ruby{毫}{ちつと}も
\ruby{知}{し}らなかつた。
%
\ruby{僕}{ぼく}あ
\ruby{彼狀}{あ|れ}を
\ruby{見}{み}て
\ruby{吃驚}{びつ|くり}した!。
%
\ruby{郵便}{いう|びん}が
\ruby{昨夜}{ゆふ|べ}
\ruby{夜中}{よ|なか}に
\ruby{着}{つ}いたから、
%
それから
\ruby{今{\換字{朝}}}{け|さ}
\ruby{暗}{くら}い
\ruby{中}{うち}に
\ruby{飛}{とん}で
\ruby{出}{で}て
\ruby{來}{き}たんだ。
%
\ruby{姊}{ねえ}さんは
\ruby{何樣}{ど|ん}なだね、
%
エ、
%
\ruby{惡}{わる}いか?、
%
エヽエ。
』

\原本頁{149-8}%
と、
%
\ruby{我}{われ}を
\ruby{一家}{いつ|け}の
\ruby{人}{ひと}か
なんぞのやうに
\ruby[||j>]{心}{こゝろ}
\ruby[||j>]{易}{ やす}く
% \ruby{心易}{こゝろ|やす}く
\ruby{思}{おも}へる
\ruby{言葉}{こと|ば}つきの
\ruby{修{\換字{飾}}無}{かざ|り|な}く、
%
\ruby{姊}{あね}を
\ruby{思}{おも}へる
\ruby{{\換字{情}}}{こゝろ}の
\ruby{溢}{あふ}るゝ
ばかりに、
%
\ruby{取}{と}り
\ruby{繕}{つくろ}ひ
\ruby{氣}{げ}
\ruby{無}{な}く
\ruby{忙}{せは}しく
\ruby{問}{と}ふを
\ruby{見}{み}ては、
%
\ruby{今}{いま}まで
\ruby{胸}{むね}の
\ruby{中}{うち}に
もや〳〵としたる
\ruby{一切}{いつ|さい}の
\ruby{不快}{ふ|くわい}さ
\ruby{忌}{いま}はしさも、
%
\ruby{{\換字{朝}}日}{あさ|ひ}に
あひて
\ruby[||j>]{霜}{しも}
\ruby[||j>]{柱}{ばしら}の
% \ruby{霜柱}{しも|ばしら}の
\ruby{嵯牙}{さ|が}として
\ruby{立}{た}てるも
\ruby{忽}{たちま}ちに
\原本頁{150-1}\改行%
\ruby{摧}{くだ}き
\ruby{融}{と}かさるゝ
\ruby{心地}{こゝ|ち}して、
%
\ruby{水野}{みづ|の}は
\ruby{思}{おも}はずも
\ruby{其}{その}
\ruby{手}{て}を
\ruby{執}{と}りて、
%
\ruby{正}{たゞ}しく
\ruby{答}{こた}ふるよりは
\ruby{先一句}{まづ|いつ|く}、

\原本頁{150-3}%
『マア
\ruby{安心}{あん|しん}したまへ。
』

\原本頁{150-4}%
と
\ruby{慰}{なぐさ}めたり。
