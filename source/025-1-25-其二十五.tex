\Entry{其二十五}

% メモ 校正 2024-04-09
\原本頁{150-6}%
\ruby{氣{\換字{遣}}}{き|づか}はしさに
\ruby{堪}{た}へねばこそ
\ruby{知}{し}らず
\ruby{識}{し}らず
\ruby{大悲}{だい|ひ}の
\ruby{御誓願}{おん|ちか|ひ}を
\ruby{頼}{たの}みて、
%
\原本頁{150-7}\改行%
その
\ruby{爲}{ため}に
\ruby{書生}{しよ|せい}の
\ruby{嘲笑}{あざ|けり}をも
\ruby{受}{う}くるに
\ruby{至}{いた}りたるなれ、
%
それを
\ruby{今}{いま}
\ruby{此}{こ}の
\原本頁{150-8}\改行%
\ruby{少年}{せう|ねん}の
\ruby{姊}{あね}を
\ruby{思}{おも}ふ
\ruby{心根}{こゝろ|ね}の
いぢらしきとて、
%
\ruby{先}{ま}づ
\ruby{安心}{あん|しん}したまへと
\ruby[g]{眞實}{まこと}にもあらぬ
\ruby{氣休}{き|やす}めを
\ruby{云}{い}ひたるは
\ruby{何}{なん}の
\ruby{心}{こゝろ}ぞや、
%
\ruby{自}{みづか}ら
\ruby{欺}{あざむ}き
\ruby{人}{ひと}を
\ruby{欺}{あざむ}くとは
\ruby{此}{こ}の
\ruby{事}{こと}なりと、
%
\ruby[g]{水野}{みづの}は
はツと
\ruby{思}{おも}ひしかど、
%
\ruby{既}{すで}に
\ruby{口}{くち}を
すべらせたれば
\ruby{駟}{し}も
\ruby{及}{およ}ばず、
%
たゞ
\ruby[g]{四ッ木}{よ ぎ}に% TODO 四ツ木
\ruby{着}{つ}きても
\ruby[g]{松之助}{まつのすけ}が
\ruby{驚}{おどろ}く
\ruby{事}{こと}などの
\ruby{無}{な}からんをば、
%
\ruby{今{\換字{更}}}{いま|さら}
\ruby{{\換字{又}}}{また}
ひそかに
\ruby{切}{せつ}に
\ruby{念}{ねん}じたり。

\原本頁{151-3}%
\ruby[g]{松之助}{まつのすけ}は
\ruby{嬉}{うれ}しげに
\ruby[g]{水野}{みづの}を
\ruby{見}{み}て、

\原本頁{151-4}%
『では
\ruby{其樣}{そん|な}に
\ruby{甚}{ひど}くは
\ruby{無}{な}いの?、
%
あゝ
\ruby{有{\換字{難}}}{あり|がた}かつた!。
%
\ruby{僕}{ぼく}は
\ruby{何}{ど}の
\ruby[<j|]{位}{くらゐ}
\ruby{心配}{しん|ぱい}したか
\ruby{知}{し}れない。
%
\ruby{併}{\換字{志}か}し
\ruby{{\換字{平}}常}{た|ゞ}の
\ruby{風邪}{か|ぜ}では
\ruby{無}{な}いやうだつて、
%
\ruby{何病}{なに|びやう}だつたの?。
』

\原本頁{151-7}%
と、
%
\ruby{人}{ひと}の
\ruby{一句}{いつ|く}を
\ruby{直}{たゞち}に
\ruby{信}{しん}じて
\ruby{無邪氣}{む|じや|き}に
\ruby{悅}{よろこ}べる
さまの
\ruby{罪}{つみ}なさは、
%
\ruby{却}{かへ}つて
\ruby[g]{水野}{みづの}の
\ruby{眼}{め}に
\ruby{憫然}{あは|れ}に
\ruby{見}{み}えたり。

\原本頁{151-9}%
『
\ruby{病氣}{びやう|き}は
\ruby[g]{腸窒扶斯}{ちやうちぶす}といふ
\ruby{事}{こと}で、
%
なか〳〵
\ruby{輕}{かる}くは
\ruby{無}{な}い
\ruby{病患}{わづ|らひ}なのだよ。
%
\換字{志}かし
\ruby{醫師}{い|し}も
\ruby{信用}{しん|よう}の
\ruby{出來}{で|き}る
\ruby{人}{ひと}を
\ruby{頼}{たの}み、
%
\ruby{看護{\換字{婦}}}{かん|ご|ふ}も
\ruby{今日}{け|ふ}から
\ruby{來}{く}る
\ruby{手筈}{て|はず}に
なつて
\ruby{居}{ゐ}るから、
%
\ruby{決}{けつ}して
\ruby{無益}{む|えき}の
\ruby{心配}{しん|ぱい}は
\ruby{仕玉}{し|たま}ふな。
%
まあ
\ruby{大{\換字{丈}}夫}{だい|ぢやう|ぶ}だと
\ruby{僕}{ぼく}はおもふ。
』

\原本頁{152-2}%
『ナニ
\ruby[g]{窒扶斯}{ちぶす}だつて!。
%
\ruby{困}{こま}つたナア、
%
アヽ
\ruby{其}{そ}りやあ
\ruby{大變}{たい|へん}だ、
%
\ruby{大變}{たい|へん}だ!。
%
アヽ
\ruby{僕}{ぼく}あ
\ruby{何樣}{ど|う}したら
\ruby{好}{い}いんだらう!。
%
\ruby{左樣}{そ|う}して
\ruby{醫者}{い|しや}だの
\ruby{何}{なん}ぞは
\ruby{誰}{たれ}が
\ruby{仕}{し}て
\ruby{吳}{く}れたの?。
%
\ruby{姊}{ねえ}さんに
\ruby{其}{それ}だけの
\ruby{事}{こと}が
\ruby{自{\換字{分}}}{じ|ぶん}で
\ruby{出來}{で|き}たの?。
%
\ruby{姊}{ねえ}さんにやあ
\ruby{其樣}{そ|ん}な
\ruby{事}{こと}の
\ruby{出來}{で|き}さうも
\ruby{無}{な}いナア
\ruby{僕}{ぼく}が
\ruby{知}{し}つて
\ruby{居}{ゐ}る。
%
\ruby{誰}{だれ}が
\ruby{仕}{し}て
\ruby{吳}{く}れたの?。
%
\ruby{君}{きみ}が
\ruby{親切}{しん|せつ}に?。
』

\原本頁{152-7}%
\ruby{何}{なに}と
\ruby{無}{な}く
\ruby{感}{かん}じて
\ruby{知}{し}れる
\ruby{歟}{か}
\ruby{兒童心}{こ|ども|ごゝろ}の
\ruby{敏}{さと}くも、
%
はや
\ruby{眼}{め}の
\ruby{中}{うち}は
\ruby{涙}{なみだ}ぐみて、
%
\ruby{泣}{な}き
\ruby{出}{だ}さん
ばかりの
\ruby{顏}{かほ}つきの
\ruby{正直}{しやう|ぢき}にも、
%
\ruby{其}{そ}の
\ruby{然}{しか}りとの
\ruby{一語}{いち|ご}を
\ruby{聞}{き}きて
\ruby{直}{たゞち}に
\ruby{謝}{しや}せんと、
%
\ruby{待}{ま}ち
\ruby{設}{まう}けたる
\ruby{意中}{い|ちゆう}は
あり〳〵と
\ruby{見}{み}えぬ。
%
\原本頁{152-10}\改行%
\ruby[g]{水野}{みづの}は
\ruby{自己}{お|の}が
\ruby{此度}{こ|たび}の
\ruby{振舞}{ふる|まひ}の、
%
\ruby{恩}{おん}を
\ruby{賣}{う}るやうに
\ruby{取}{と}られん
\ruby{事}{こと}を
\ruby{心苦}{こゝろ|ぐる}しく
\ruby{思}{おも}ひ
\ruby{居}{ゐ}たれば、
%
\ruby{彼}{か}の
お
\ruby{澤}{さは}
\ruby{婆}{ばゞ}に
\ruby{對}{むか}ひて
\ruby{云}{い}ひ
\ruby{置}{お}ける
おもむきを、
%
\原本頁{153-1}\改行%
\ruby{{\換字{飽}}}{あ}くまで
\ruby{徹}{とほ}さんと
\ruby{思}{おも}へるなり。

\原本頁{153-2}%
『イヽエ。
』

\原本頁{153-3}%
\ruby{思}{おも}ひの
\ruby{外}{ほか}なる
\ruby[g]{水野}{みづの}が
\ruby{答}{こたへ}に
\ruby[g]{松之助}{まつのすけ}は
\ruby{合點}{が|てん}
\ruby{行}{ゆ}かぬ
ところあり。

\原本頁{153-4}%
『ぢやあ
\ruby{誰}{たれ}が
\ruby{仕}{し}て
\ruby{吳}{く}れたの?。
』

\原本頁{153-5}%
『
\ruby{學校}{がく|かう}の
\ruby{人}{ひと}たちが。
』

\原本頁{153-6}%
『
\ruby{君}{きみ}だの
\ruby{校長}{かう|ちやう}さんだのが?。
』

\原本頁{153-7}%
『マアそんなものだと
\ruby{思}{おも}つて
\ruby{居}{ゐ}たまへ。
』

\原本頁{153-8}%
『ア、
%
それぢやあ
\ruby{矢張}{やつ|ぱ}り
\ruby{君}{きみ}の
\ruby{親切}{しん|せつ}なんだ、
%
きつと
\ruby{左樣}{さ|う}に
\ruby{{\換字{違}}無}{ちがひ|な}い、
%
\原本頁{153-9}\改行%
\ruby{僕}{ぼく}は
\ruby{知}{し}つてゐる!。
%
ほんたうに
\ruby{君}{きみ}
\ruby{有}{あ}り
\ruby{{\換字{難}}}{がた}う!。
%
\ruby{僕}{ぼく}あ
\ruby{一生}{いつ|しやう}
おぼえて
\ruby{居}{ゐ}る!。
』

\原本頁{153-11}%
\ruby{淡泊}{たん|ぱく}にも
\ruby{頭}{かうべ}を
\ruby{下}{さ}げて
\換字{志}み〴〵と
\ruby{恩}{おん}を
\ruby{謝}{しや}せる
\ruby[g]{松之助}{まつのすけ}が
\ruby{心}{こゝろ}は
\ruby{其}{そ}の
\ruby{手}{て}に
\ruby{籠}{こも}りて、
%
\ruby[g]{水野}{みづの}は
\ruby{我}{わ}が
\ruby{手}{て}の
\ruby{緊}{きび}しく
\ruby{握}{にぎ}られたるを
\ruby{感}{かん}じぬ。
%
\ruby{談話}{はな|し}は
\原本頁{154-2}\改行%
\ruby{一}{ひ}ト
\ruby{先}{まづ}
\ruby{{\換字{終}}}{をは}りけるが、
%
\ruby{問答}{もん|だふ}は
\ruby{{\換字{又}}}{また}
\ruby{突}{とつ}として
\ruby{起}{おこ}りぬ。

\原本頁{154-3}%
『
\ruby{君}{きみ}は
こんなに
\ruby{夙}{はや}く
\ruby{何處}{ど|こ}へ
\ruby{行}{い}つたの?。
』

\原本頁{154-4}%
『
\ruby{少}{すこ}しばかり
\ruby{用}{よう}があつて
\ruby{出}{で}たんだが、
%
もう
\ruby{歸路}{かへ|り}なのだ。
』

\原本頁{154-5}%
『
\ruby{其}{そ}の
\ruby{次}{ついで}に
\ruby{觀音樣}{くわん|のん|さま}へ% 「觀音」の読みは原本通り「くわん(の)ん」
\ruby{詣}{まゐ}つたのかエ?。
』

\原本頁{154-6}%
『ムヽ。
』

\原本頁{154-7}%
『
\ruby{觀音樣}{くわん|のん|さま}に% 「觀音」の読みは原本通り「くわん(の)ん」
\ruby{何}{なん}の
\ruby{用}{よう}があつて?。
もし
\ruby{願}{ねが}ひ
\ruby{事}{ごと}でも
\ruby{爲}{し}て?。
』

\原本頁{154-8}%
『ムヽ。
』

\原本頁{154-9}%
『
\ruby{虛言}{う|そ}だらう。
%
そりやあ
\ruby{可笑}{を|か}しいナア、
%
ハヽ。
』

\原本頁{154-10}%
『
\ruby{何故}{な|ぜ}そんなに
\ruby{君}{きみ}にやあ
\ruby{可笑}{を|か}しいのかね?。
』

\原本頁{154-11}%
『だつて
\ruby{君}{きみ}、
%
\ruby{君}{きみ}は
いつか
\ruby{僕}{ぼく}に
\ruby{敎}{をし}へたぢやあ
\ruby{無}{な}いか。
%
ホラ、
%
\ruby{此}{こ}の
\原本頁{155-1}\改行%
\ruby{觀音}{くわん|のん}といふ% 「觀音」の読みは原本通り「くわん(の)ん」
\ruby{人}{ひと}は
\ruby{聞}{き}いて
\ruby{思}{おも}つて
\ruby{修}{をさ}めるといふ
\ruby{三}{みつ}つの
\ruby{學問}{がく|もん}の
\ruby{法則}{はふ|そく}を、
%
\ruby{敎}{をし}へて
\ruby{{\換字{遺}}}{のこ}した
\ruby{人}{ひと}なので、
%
\ruby{敬}{けい}すべき
\ruby{人}{ひと}には
\ruby{{\換字{違}}無}{ちがひ|な}いが、
%
\ruby{福}{ふく}を
\ruby{與}{あた}へるものなんぞとして
\ruby{拜}{をが}むのは、
%
\ruby{感心}{かん|しん}の
\ruby{出來}{で|き}ない
\ruby{卑}{いや}しい
\ruby{事}{こと}だと、
%
\原本頁{155-4}\改行%
\ruby{僕}{ぼく}が
\ruby{{\換字{習}}慣}{く|せ}でもつて
\ruby{拜}{をが}まうとしたら、
%
\ruby{敎}{をし}へて
\ruby{吳}{く}れた
\ruby{事}{こと}が
あつたもの!。
%
その
\ruby{君}{きみ}が
\ruby{願}{ねが}ひ
\ruby{事}{ごと}なんぞ
\ruby{仕}{し}やう
\ruby{譯}{わけ}は
\ruby{無}{な}いもの!。
』

\原本頁{155-6}%
\ruby{實}{げ}に
\ruby{嘗}{かつ}て
\ruby{此}{こ}の
\ruby{少年}{せう|ねん}が
\ruby[g]{四ッ木}{よ ぎ}よりの% TODO 四ツ木
\ruby{歸}{かへ}るさを
\ruby{{\換字{送}}}{おく}りがてら、
%
\ruby{共}{とも}に
\ruby{心}{こゝろ}
たのしく
\ruby{{\換字{遊}}}{あそ}び
あるきつゝ
\ruby{此處}{こ|ゝ}に
\ruby{來}{きた}りし
\ruby{時}{とき}、
%
\ruby{生}{なま}さかしくも
\ruby{然}{さ}る
\ruby{事}{こと}を
\ruby{說}{と}きて、
%
\ruby{幸福}{しあ|はせ}を%「幸福」ここは「は」
\ruby{得}{え}んとて
\ruby{佛}{ほとけ}を
\ruby{拜}{をが}む
\ruby{世}{よ}の
\ruby{人}{ひと}の
\ruby{心}{こゝろ}の
\ruby{卑}{いや}しさを
\ruby{笑}{わら}ひし
\ruby{事}{こと}ありしを、
%
\ruby{端}{はし}
\ruby{無}{な}くも
\ruby{今}{いま}
\ruby{云}{い}ひ
\ruby{出}{いだ}されて
\ruby{想}{おも}ひ
\ruby{起}{おこ}せば、
%
\ruby{{\換字{又}}}{また}
\ruby{新}{あらた}に
\ruby{毒箭}{どく|や}を
\ruby{胸板}{むな|いた}に
\ruby{射}{い}
\ruby{立}{た}てられし
\ruby{心地}{こゝ|ち}して、
%
\ruby{堪}{た}へがたき
\ruby{不快}{ふ|くわい}さを
\ruby{再度}{ふた|ゝび}
\ruby{覺}{おぼ}えつ。
%
おもへば
\ruby{其}{それ}のみには
あらざりし、
%
はじめて
\ruby{東京}{とう|きやう}にて
\ruby{羽{\換字{勝}}}{は|がち}
\原本頁{156-1}\改行%
\ruby{島木}{しま|き}
\ruby{等}{ら}
\ruby{七人}{なな|にん}
\ruby{打揃}{うち|そろ}ひて、
%
\ruby{詣}{まゐ}るとも
\ruby{無}{な}く
\ruby{此}{こ}の
\ruby{御堂}{み|だう}に
\ruby{參}{まゐ}りし
\ruby{折}{をり}、
%
\ruby{島木}{しま|き}と
\ruby[g]{楢井}{ならい}と
\ruby{羽{\換字{勝}}}{は|がち}とは
\ruby{手}{て}を
\ruby{合}{あは}せて
\ruby{拜}{をが}み、
%
\ruby[g]{日方}{ひかた}と
\ruby{山瀬}{やま|せ}と
\ruby{名倉}{なぐ|ら}とは
\ruby{三人}{さん|にん}を
\ruby{冷笑}{あざ|わら}ひしに、
%
おのれは
\ruby{拜}{をが}みもせねば
\ruby{冷笑}{あざ|わら}ひもせで、
%
\ruby{我}{われ}は
たゞ
\ruby{{\換字{古}}}{いにしへ}の
\ruby{賢人}{けん|じん}として
\ruby{大士}{だい|し}を
\ruby{待}{ま}たんと
\ruby{思}{おも}ふなりとて、
%
たゞ
\ruby{帽}{ばう}を
\ruby{脫}{ぬ}ぎて
\ruby{一禮}{いち|れい}したりし
\ruby{{\換字{古}}}{ふる}き
\ruby{事}{こと}まで
\ruby{心}{こゝろ}に
\ruby{{\換字{浮}}}{うか}べば、
%
\ruby{一腔}{いつ|こう}の
\ruby{中}{うち}は
\ruby{火}{ひ}の
\ruby{散}{ち}る
\ruby{如}{ごと}くに
\ruby{羞惡}{しう|を}の
\ruby[<j|]{{\換字{情}}}{こゝろ}
\ruby{燃}{も}え
\ruby{立}{た}つて、
%
\ruby{菩薩}{ぼ|さつ}の
\ruby{大威力}{だい|ゐ|りき}を
\ruby{假}{か}りたき
\ruby{念}{おもひ}は
\ruby{今}{いま}
\ruby{{\換字{猶}}}{なほ}
こゝに
ありながら、
%
\ruby{今}{いま}
こゝに
\ruby{我}{われ}を
\ruby{卑}{いや}しくして、
%
\ruby{世}{よ}の
\ruby{人並}{ひと|な}みに
\ruby{菩薩}{ぼ|さつ}を
\ruby{拜}{をが}みしを
\ruby{口惜}{くち|をし}く
おもふが
\ruby{如}{ごと}き
\ruby{感}{かん}じも
\ruby{起}{おこ}りて、
%
\ruby{不安}{ふ|あん}の
\ruby{色}{いろ}の
\ruby{面}{おもて}に
\ruby{出}{い}づらんを
\ruby{制}{せん}せんとして% 原本通り「制」を「せん」とした
\ruby{制}{せい}しがたきを
\ruby{覺}{おぼ}えたり。

\原本頁{156-10}%
『ハヽヽ、
%
そんな
\ruby{事}{こと}を
\ruby{云}{い}つた
\ruby{事}{こと}も
\ruby{成程}{なる|ほど}
\ruby{有}{あ}つた。
』

\原本頁{156-11}%
\ruby{辛}{から}くも
\ruby{自}{みづか}ら
\ruby{克}{か}つて
\ruby{塞}{ふさ}がる
\ruby{胸}{むね}より
\ruby{答}{こた}へ
\ruby{得}{え}たるは、
%
\ruby{全}{まつた}き
\ruby{意味}{い|み}も
\ruby{無}{な}き
\原本頁{157-1}\改行%
\ruby{言葉}{こと|ば}なり。

\原本頁{157-2}%
『さうして
\ruby{君}{きみ}は
\ruby{何}{なに}を
\ruby{願}{ねが}つたの?。
』

\原本頁{157-3}%
\ruby{心}{こゝろ}
\ruby{無}{な}く
\ruby{放}{はな}つ
\ruby{少年}{せう|ねん}の
\ruby{箭}{や}は、
%
またもや
\ruby[g]{水野}{みづの}が
\ruby{心窩}{む|ね}の
\ruby{眞正中}{まつ|たゞ|なか}に
\ruby{立}{た}ちぬ。
%
\原本頁{157-4}\改行%
されど
\ruby[g]{水野}{みづの}は
\ruby{痛手}{いた|で}を
\ruby{外}{よそ}にして、

\原本頁{157-5}%
『
\ruby{何}{なん}でも
\ruby{可}{い}いから
\ruby{急}{いそ}いで
\ruby{行}{ゆ}かう。
』

\原本頁{157-6}%
と、
%
\ruby[g]{松之助}{まつのすけ}と
\ruby{共}{とも}に
\ruby[g]{四ッ木}{よ ぎ}へと% TODO 四ツ木
\ruby{志}{こゝろざ}し、
%
\ruby{人}{ひと}の
\ruby{{\換字{運}}命}{う|ん}、
%
\ruby{我}{わ}が
\ruby{{\換字{運}}命}{う|ん}の
\ruby{測}{はか}り
\原本頁{157-7}\改行%
\ruby{{\換字{難}}}{がた}き
\ruby{{\換字{前}}{\換字{途}}}{ゆく|て}を
\ruby{見}{み}んと、
%
\ruby{心}{こゝろ}に
\ruby{幾枝}{いく|し}の
\ruby{箭}{や}を
\ruby{負}{お}ひながら、
%
\ruby{路}{みち}を
\ruby{急}{いそ}ぎて
\ruby{歩}{あゆ}み
\ruby{出}{いだ}しぬ。

\原本頁{157-9}%
\ruby{此}{こ}の
\ruby{時}{とき}
\ruby{日}{ひ}は
\ruby{漸}{やうや}く
\ruby{昇}{のぼ}ると
\ruby{共}{とも}に、
%
\ruby{狂風滾々}{きやう|ふう|こん|〳〵}と
\ruby{吹}{ふ}き
\ruby{出}{いだ}して、
%
\ruby{美}{うるは}しかりし
\ruby{{\換字{空}}}{そら}は
\ruby{何時}{い|つ}と
\ruby{無}{な}く
\ruby{黄}{き}ばみ、
%
\ruby[g]{暴風雨日}{あれび}
\ruby{{\換字{近}}}{ちか}き
\ruby{天}{てん}に
\ruby{氣味}{き|み}
あしき
\ruby{雲}{くも}の
おだやかならず
\ruby{湧}{わ}き
ひろごりて、
%
\ruby{昨夜}{ゆふ|べ}に
\ruby{變}{かは}れる
\ruby{今日}{け|ふ}の
\ruby{狀態}{やう|す}の、
%
そぞろに
\ruby{定}{さだ}め
\ruby{無}{な}き
\ruby{人間}{ひ|と}の
\ruby{上}{うへ}を
\ruby{示}{しめ}すが
\ruby{如}{ごと}く、
%
\ruby{首}{かうべ}を
\ruby{傾}{かたむ}けて
\ruby{{\換字{進}}}{すゝ}む
\ruby[g]{水野}{みづの}と
\原本頁{158-2}\改行%
\ruby[g]{松之助}{まつのすけ}との
\ruby{眞向}{まつ|かう}に
\ruby{烈}{はげ}しく
\ruby{當}{あた}る
\ruby{風}{かぜ}は、
%
\ruby{二人}{ふた|り}が
\ruby{心臓}{む|ね}をして
\ruby{騷}{さわ}ぎに
\ruby{騷}{さわ}がしめぬ。
