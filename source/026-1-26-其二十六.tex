\Entry{其二十六}

% メモ 校正終了 2024-04-09
\原本頁{158-5}%
\ruby{語}{かた}り
つゞけたる
\ruby{談話}{はな|し}の
\ruby{間}{うち}、
%
\ruby{息}{いき}
つぎ〳〵に
われ
\ruby{知}{し}らず
\ruby{飮}{の}みし
\ruby{葡萄酒}{ぶ|だう|しゆ}の
\ruby{量}{りやう}の
\ruby{少}{すくな}からで、
%
\ruby{既}{すで}に
\ruby{其}{そ}の
\ruby{六七{\換字{分}}}{ろく|しち|ぶ}を
\ruby{盡}{つく}したれば、
%
\ruby[||j>]{醉}{すゐ}
\ruby[||j>]{興}{きよう}% 「醉」は原本通り「ゐ」で調整
% \ruby{醉興}{すゐ|きよう}% 「醉」は原本通り「ゐ」で調整
おのづから
\ruby{發}{はつ}して
\ruby{獨}{ひと}り
\ruby{機{\換字{嫌}}}{き|げん}よく、
%
\ruby{不規律}{ふ|き|りつ}の
\ruby[||j>]{大}{たい}
\ruby[||j>]{將}{しやう}を
% \ruby{大將}{たい|しやう}を
もて
\ruby{自}{みづか}ら
\ruby{許}{ゆる}せるほど
ありて、
%
ふたゝび
\ruby{睡}{ねむ}りには
\ruby{就}{つ}かんともせず、
%
\ruby{島木}{しま|き}は
\ruby{{\換字{猶}}}{なほ}
ぐびりぐびりと% 原本は行末禁則箇所なので踊り字にはなっていない
\ruby[||j>]{獨}{どく}
\ruby[||j>]{{\換字{酌}}}{しやく}を
% \ruby{獨{\換字{酌}}}{どく|しやく}を
\ruby{續}{つゞ}けたり。

\原本頁{158-10}%
むつくりと
\ruby{肥}{こ}えたる
\ruby{身體}{から|だ}
ゆたかに
\ruby{胡坐}{あぐ|ら}
かきて、
%
\ruby{土}{つち}
\ruby{多}{おほ}き
\ruby{山}{やま}の
\ruby{岩}{いは}を
\原本頁{159-1}\改行%
\ruby{隱}{かく}せるが
\ruby{如}{ごと}くに、
%
\ruby{肉}{にく}
ふくらかにして
\ruby{骨}{ほね}を
\ruby{見}{み}せぬ
\ruby{丸々}{まる|〳〵}としたる
\ruby{顏}{かほ}の、
%
\ruby{其}{そ}の
\ruby{小}{ちひ}さなる
\ruby{眼}{め}の
あたりに
\ruby{笑}{ゑみ}を
\ruby{含}{ふく}み、
%
\ruby{今}{いま}しも
ぐつと
\ruby{一盞}{いつ|さん}を
\ruby{仰}{あふ}ぎたるが、

\原本頁{159-4}%
『もう
\ruby{出}{で}て
\ruby{來}{き}さうなものだがナ、
%
\ruby[||j>]{畜}{ちく}
\ruby[||j>]{生}{しやう}!、
% \ruby{畜生}{ちく|しやう}!、
%
まだかナ。
』

\原本頁{159-5}%
と、
%
\ruby{誰}{たれ}に
\ruby{云}{い}へるともなく
\ruby{自}{みづか}ら
\ruby{語}{かた}れり。

\原本頁{159-6}%
\ruby{島木}{しま|き}は
\ruby{水野}{みづ|の}が
\ruby{胸中}{む|ね}を
\ruby{知}{し}りたれど、
%
\ruby{水野}{みづ|の}は
\ruby{島木}{しま|き}が
\ruby{肚裏}{は|ら}を
\ruby{知}{し}らざりき。
%
\ruby{妻子}{さい|し}
\ruby[||j>]{兄}{きやう}
\ruby[||j>]{弟}{ だい}も
% \ruby{兄弟}{きやう|だい}も
\ruby{無}{な}く
\ruby{親}{おや}も
\ruby{無}{な}ければ、
%
\ruby{氣}{き}まゝなる
\ruby{寄寓}{かり|ずみ}の
\ruby{面倒}{めん|だう}
\ruby{無}{な}きを
\ruby{悅}{よろこ}びて、
%
\ruby{一家}{いつ|か}を
こそは
\ruby{{\換字{猶}}}{なほ}
\ruby{構}{かま}へざれ、
%
\ruby{幾度}{いく|たび}か
\ruby{{\換字{浮}}}{う}き
\ruby{幾度}{いく|たび}か
\ruby{沈}{しづ}みし
\原本頁{159-9}\改行%
\ruby{末}{すゑ}に、
%
\ruby{漸}{やうや}く
\ruby[||j>]{合}{がふ}
\ruby[||j>]{百}{ひやく}の% 「合百」証拠金を納めないで相場の上げ下げにて賭けをする一種の賭博。
% \ruby{合百}{がふ|ひやく}の% 「合百」証拠金を納めないで相場の上げ下げにて賭けをする一種の賭博。
\ruby{果敢無}{は|か|な}きより、
%
\ruby{今}{いま}は
\ruby{人}{ひと}の
\ruby{噂}{うはさ}にも
\ruby{上}{のぼ}るほどの
\ruby[||j>]{玉}{ぎよく}
\ruby[||j>]{高}{ だか}を
% \ruby{玉高}{ぎよく|だか}を
\ruby{動}{うご}かすに
\ruby{至}{いた}りし
\ruby{島木}{しま|き}も、
%
もとより
\ruby{右}{みぎ}は
\ruby{地獄}{ぢ|ごく}
\ruby{左}{ひだり}は
\ruby{極樂}{ごく|らく}の
\ruby{間}{あひだ}の
\ruby{綱}{つな}を
\ruby{渡}{わた}つて
\ruby{日}{ひ}を
\ruby{{\換字{送}}}{おく}る
\ruby{投機師}{とう|き|し}の
\ruby{身}{み}の
\ruby{上}{うへ}は、
%
\ruby{貨物}{くわ|ぶつ}を
\ruby{積}{つ}み
\ruby{問屋}{とひ|や}を
\ruby{控}{ひか}へて
\ruby{十}{じう}の
\ruby{一}{いち}
\ruby{十}{じう}の
\ruby{二}{に}の
\ruby{利}{り}を
\ruby{征}{と}りて
\ruby{行}{ゆ}く
\ruby{堅氣}{かた|ぎ}の
\ruby{商人}{あき|うど}とは
\ruby{異}{こと}なれば、
%
\ruby{此處}{こ|ゝ}
\ruby{一}{ひ}ト
\ruby{伸}{のし}と
\ruby{有}{あ}らん
\ruby{限}{かぎ}りの
\ruby[<j||]{力}{ちから}
\ruby{瘤}{こぶ}を
\ruby{入}{い}れて
\ruby{蒐}{かゝ}れる
\ruby{此}{こ}の
\ruby{秋}{あき}の、
%
\ruby{天候}{てん|こう}を
\原本頁{160-3}\改行%
\ruby{重}{おも}なる
\ruby{相場}{さう|ば}の% 原文通り「場」
\ruby{時季}{と|き}に、
%
\ruby{捉}{とら}へ
かねたる
\ruby{雲}{くも}の
\ruby[||j>]{心}{こゝろ}
\ruby[||j>]{風}{ かぜ}の
% \ruby{心風}{こゝろ|かぜ}の
\ruby{料簡}{れう|けん}は
\ruby{我}{わ}が
\ruby{思}{おも}はくと
\ruby{{\換字{違}}}{ちが}ひて、
%
\ruby{{\換字{追}}敷}{おひ|じき}
\g詰めruby{々々}{〳〵}と
\ruby{取}{と}り
\ruby{立}{た}てらるゝに
\ruby{懷中}{ふと|ころ}
\ruby{危}{あやふ}く、
%
\ruby{既}{すで}に
\ruby{其}{そ}の
\原本頁{160-5}\改行%
\ruby{剩}{あま}すところは
\ruby{幾何}{いく|ばく}も
あらぬ
\ruby[||j>]{端}{はした}
\ruby[||j>]{錢}{ がね}と
% \ruby{端錢}{はした|がね}と
なりて、
%
\ruby{{\換字{運}}}{うん}と
\ruby[<j>]{志}{こゝろざし}との
\ruby{今}{いま}
\ruby{少時}{しば|し}
\ruby{反}{そむ}かば、
%
またもや
\ruby{身}{み}の
\ruby{皮}{かは}も% 原本通り「皮 か(は)」
\ruby{無}{な}き
\ruby{赤裸々}{あか|はだ|か}となりて、
%
\ruby{賽}{さい}の
\ruby{河原}{か|はら}に
\ruby{積}{つ}める
\ruby{石}{いし}の
\ruby{{\換字{瓦}}落離}{ぐわ|ら|り}と
\ruby{崩}{くづ}れたる
\ruby{{\換字{情}}無}{なさけ|な}さを
\ruby{見}{み}るべしと、
%
\ruby{流石}{さす|が}に
\ruby{心}{こゝろ}も
おちつき
かぬるところへ、
%
\ruby{折}{をり}も
\ruby{折}{をり}とて
\ruby{水野}{みづ|の}の
\ruby{無心}{む|しん}なり。
%
\ruby{{\換字{運}}}{うん}を
\ruby{背負}{せ|お}へる
\ruby{時}{とき}には
\ruby{其}{そ}の
\ruby{二倍}{に|ばい}
\ruby{三倍}{さん|ばい}も
\ruby{與}{あた}ふるに
\ruby{易}{やす}けれど、
%
\ruby{夜明}{よ|あ}けての
\ruby{天地}{てん|ち}の
\原本頁{160-10}\改行%
\ruby{狀態}{やう|す}
\ruby{次第}{し|だい}にて
\ruby{我}{わ}が
\ruby{生命}{いの|ち}はと
さへ
\ruby{思}{おも}へる
\ruby{矢先}{や|さき}に
\ruby{云}{い}ひかけられては、
%
\原本頁{160-11}\改行%
\ruby[||j>]{敗}{まけ}
\ruby[||j>]{軍}{いくさ}の
% \ruby{敗軍}{まけ|いくさ}の
\ruby{{\換字{退}}}{ひ}き
\ruby{際}{ぎは}に
\ruby{頼}{たの}みきつたる
\ruby{持鎗}{もち|やり}を
\ruby{{\換字{所}}望}{しよ|まう}されたる
\ruby{心地}{こゝ|ち}して、
%
\ruby{流石}{さす|が}の
\ruby{島木}{しま|き}も
\ruby{行}{ゆ}き
\ruby{詰}{つま}りしが、
%
\ruby{竹}{たけ}を
\ruby{割}{わ}つたる
\ruby{如}{ごと}き
\ruby{持{\換字{前}}}{もち|まへ}の
\ruby{氣象}{き|しやう}は
\ruby{義}{ぎ}を
\原本頁{161-2}\改行%
\ruby{見}{み}て
\ruby{勇}{いさ}んで、
%
エヽ
どうせ
\ruby{曲}{まが}つて
\ruby{仕舞}{し|ま}えば
\ruby{無}{な}くなる
\ruby{金}{かね}を、
%
\ruby{今}{いま}
\ruby{{\換字{遣}}}{や}つて
\ruby{仕舞}{し|ま}へば
\ruby{友{\換字{達}}}{とも|だち}の
\ruby{利益}{た|め}!、
%
\ruby{踏張}{ふん|ば}れ〳〵
\ruby{男}{をとこ}の
\ruby{兒}{がき}だ、
%
\ruby{裸々}{はだ|か}になつても
\ruby{怖}{こは}くは
\ruby{無}{な}い、
%
\ruby{百兩}{ひやく|りやう}ばかりの
\ruby{鼻糞金}{はな|くそ|がね}を
\ruby{出}{だ}し
\ruby{悋}{をし}んでは、
%
\ruby{萬五郎}{まん|ご|らう}の
\原本頁{161-5}\改行%
\ruby{男}{をとこ}が
\ruby{廢}{す}たる!、
%
\ruby{{\換字{情}}無}{なさけ|な}い!、
%
\ruby{行末}{ゆく|すゑ}が
\ruby{見}{み}える!、
%
\ruby{百萬兩{\換字{分}}限}{ひやく|まん|りやう|ぶ|げん}になつた
\ruby{時}{とき}の
\ruby[||j>]{額}{むかふ}
\ruby[||j>]{疵}{ きず}になる!、
% \ruby{額疵}{むかふ|きず}になる!、
%
\ruby{握}{にぎ}つた
\ruby{錢}{ぜに}から
\ruby{{\換字{煙}}}{けむ}を
\ruby{出}{だ}すのは
\ruby{三{\換字{文}}野郎}{さん|もん|や|らう}のする
\ruby{事}{こと}だ、
%
と
\ruby{早}{はや}くも
\ruby[||j>]{決}{けつ}
\ruby[||j>]{着}{ちやく}して
% \ruby{決着}{けつ|ちやく}して
\ruby{臓腑}{ざう|ふ}を
\ruby{見}{み}せずに、
%
\ruby{奇麗}{き|れい}に
\ruby[<j>]{快}{こゝろよ}く
\ruby{用立}{よう|だ}てて% TODO 原本通り行末禁則で踊り字未対応
\ruby{歸}{かへ}しやりつ、
%
さて
\ruby{其}{それ}が
ためとにも
あらざるべけれど、
%
\ruby{何}{なん}と
\ruby{無}{な}く
\ruby{心}{こゝろ}に
\ruby{怡悅}{よろ|こび}を
\ruby{覺}{おぼ}えて、
%
\ruby{今}{いま}は
\ruby{氣}{き}も
\ruby{冴}{さ}え〴〵と
\ruby{飮}{の}み
\ruby{居}{を}れるなり。

\原本頁{161-10}%
『もう
\ruby{出}{で}て
\ruby{來}{き}さうなものだがナ、
%
まだかナ、
%
\ruby[||j>]{畜}{ちく}
\ruby[||j>]{生}{しやう}!。
% \ruby{畜生}{ちく|しやう}!。
』

\原本頁{161-11}%
ふたゝび
\ruby{獨}{ひと}り
ごちて
\ruby{酒盞}{さか|づき}を
\ruby{取}{と}りぬ。

\原本頁{162-1}%
『まだ
\ruby{出}{で}て
\ruby{來}{こ}ないかナ、
%
\ruby[||j>]{畜}{ちく}
\ruby[||j>]{生}{しやう}めツ!。
% \ruby{畜生}{ちく|しやう}めツ!。
』

\原本頁{162-2}%
\ruby{何}{なに}を
\ruby{待}{ま}てるにか
\ruby{三度}{み|たび}
\ruby{獨語}{ひとり|ご}ちしが、
%
\ruby{答}{こた}ふるものは
\ruby{有}{あ}るべくも
\ruby{無}{な}く、
%
\原本頁{162-3}\改行%
\ruby{室}{しつ}の
\ruby{一隅}{いち|ぐう}の
\ruby{小机}{こ|づくゑ}の
\ruby{上}{うへ}の
\ruby{懷中時計}{くわい|ちゆう|ど|けい}の% 「懷中(くわいちゆう)」「ゆ」有り
\ruby{音}{おと}のみの
\ruby{有}{あ}るか
\ruby{無}{な}きかに
\ruby{響}{ひゞ}けり。

\原本頁{162-5}%
\ruby{相手}{あひ|て}
\ruby{無}{な}き
\ruby{淋}{さび}しさに
\ruby{堪}{た}へかねてか、

\原本頁{162-6}%
『
\ruby[||j>]{畜}{ちく}
\ruby[||j>]{生}{しやう}ツ、
% \ruby{畜生}{ちく|しやう}ツ、
%
\ruby{出}{で}て
\ruby{來}{き}やがらなくつても
\ruby{仕方}{し|かた}が
\ruby{無}{な}いかナ。
%
ハヽヽ、
%
\原本頁{162-7}\改行%
\ruby{怒}{おこ}るほど
\ruby{乃公}{お|れ}も
\ruby{野暮}{や|ぼ}ぢやあ
いけねえ。
%
それは
さうと
\ruby{水野}{みづ|の}は
もう
\原本頁{162-8}\改行%
\ruby{大{\換字{分}}}{だい|ぶ}
\ruby{行}{い}つたらう。
%
\ruby[||j>]{愍}{かあ}
\ruby[||j>]{然}{いさう}に、% 「愍然 か(あ)いさう」
% \ruby{愍然}{かあ|いさう}に、% 「愍然 か(あ)いさう」
%
\ruby{堅}{かた}い
\ruby[||j>]{正}{しやう}
\ruby[||j>]{直}{ ぢき}な
% \ruby{正直}{しやう|ぢき}な
\ruby{男}{をとこ}だから、
%
\ruby{人一倍}{ひと|いち|ばい}
\ruby{何彼}{なに|か}につけて
\ruby[||j>]{物}{もの}
\ruby[||j>]{思}{おもひ}を
% \ruby{物思}{もの|おもひ}を
\ruby{仕}{し}て
\ruby{居}{ゐ}る!。

\原本頁{162-10}%
\換字{庵点}
\ruby{粋}{すゐ}な
\ruby{{\換字{浮}}世}{うき|よ}を
\ruby{戀}{こひ}
\ruby{故}{ゆゑ}に、
%
\ruby{野暮}{や|ぼ}に
\ruby{暮}{くら}すも
\ruby{心}{こゝろ}がら。
%
あゝ
\ruby{端唄}{は|うた}の
\ruby{{\換字{文}}句}{もん|く}ぢやあ
\ruby{無}{な}いが
\ruby{{\換字{迷}}}{まよ}つちやあ
\ruby{野暮}{や|ぼ}になる!。
%
フン、
%
ナンダ
\ruby{此方}{こつ|ち}やあ
\ruby{戀}{こひ}
\ruby{故}{ゆゑ}ぢやあ
\ruby{無}{ね}えで、
%
\ruby{慾}{よく}
\ruby{故}{ゆゑ}に
\ruby{野暮}{や|ぼ}になり
\ruby{切}{き}つて
\ruby{居}{ゐ}やがる!。
%
アヽ
もう
そろ〳〵
\ruby{出}{で}て
\ruby{來}{き}て
\ruby{吳}{く}れても
\ruby{好}{よ}さゝうなものだが、
%
チヨツ
\ruby{忌々}{いま|〳〵}しい、
\GWI{u1b048-u3099}れつたいナア。% 「志」+「濁点」
%
ア、
%
\ruby{豪氣}{がう|ぎ}に
\ruby{醉}{よ}つて% 「醉」は原本通り「よ」で調整
\ruby{來}{き}た、
%
\ruby{好}{い}い
\ruby[||j>]{心}{こゝろ}
\ruby[||j>]{持}{ もち}だ!。
% \ruby{心持}{こゝろ|もち}だ!。
%
\ruby{何}{なん}だか
もう
\ruby{出}{で}て
\ruby{來}{き}さうな
\ruby[||j>]{心}{こゝろ}
\ruby[||j>]{持}{ もち}がする!。
% \ruby{心持}{こゝろ|もち}がする!。
%
ヱヽト、

\原本頁{163-5}%
\換字{庵点}
\ruby{起}{お}きて
\ruby{見}{み}つ、
%
\ruby{寢}{ね}て
\ruby{見}{み}つ
\ruby{待}{ま}てど、
%
たより
\ruby{無}{な}く、
%
チン〳〵
チンチン、
%
\ruby{蚊屋}{か|や}の
\ruby{廣}{ひろ}さに
たゞ
\ruby{獨}{ひと}り、
%
ツンテン、
%
\ruby{蚊}{か}を
\ruby{焼}{や}く
\ruby{火}{ひ}より
\ruby{胸}{むね}の
\ruby{火}{ひ}の、
%
\ruby{燃}{も}ゆる
おもひを
\ruby{察}{さつ}しやんせカナ。
%
ハヽヽヽ。
』

\原本頁{163-8}%
\ruby{聲}{こゑ}は
\ruby{美}{うつく}しからず
\ruby{錆}{さ}びたれど、
%
\ruby{聞}{き}き
\ruby{記臆}{おぼ|え}なるべきには% 原本通り「おぼえ」
\ruby{似合}{に|あ}はず
\ruby{我流}{が|りう}の
\ruby{{\換字{節}}{\換字{廻}}}{ふし|まは}しにも
をかしきところありて、
%
\ruby{小聲}{こ|ゞゑ}に
\ruby{唱}{うた}ひ
\ruby{仕舞}{し|ま}ひつゝ、
%
\原本頁{163-10}\改行%
\ruby{今}{いま}
\ruby{將}{まさ}に
\ruby{一壜}{ひと|びん}の
\ruby{酒}{さけ}を
\ruby{盡}{つく}し
\ruby{果}{は}たさんとして、
%
\ruby{手}{て}に
\ruby{取}{と}り
\ruby{上}{あ}げて
\ruby{自}{みづか}ら
\ruby{{\換字{酌}}}{つ}がんと、
%
\ruby{其}{そ}の
\ruby{尻下}{しり|さが}がりの
\ruby{小}{ちひさ}き
\ruby{目}{め}を
\ruby{一}{ひ}トしほ
\ruby{下}{さ}げて、
%
\ruby{莞爾}{につ|こり}と
\ruby{樂}{たの}しげに
\ruby{笑}{わら}ひしが、
%
\ruby{何}{なに}をか
\ruby{聞}{き}きつけしや
\ruby{俄然}{が|ぜん}として、

\原本頁{164-2}%
『ヤツ、
%
\ruby{來}{き}たぞ!%\inhibitglue{}% ここは「空き」があるので
\,% 原本上でのアキを再現するため「3/18 em」空ける
\ruby{來}{き}て
\ruby{吳}{く}れたぞ!、
%
おいでなすつたぞ!。
%
\ruby{占}{し}めたナ!、
%
サア
\ruby{來}{こ}いだ!。
』

\原本頁{164-4}%
と
\ruby{飛}{と}び
\ruby{立}{た}つたり。

\原本頁{164-5}%
\ruby{投}{な}げ
\ruby{出}{だ}されたる
\ruby{壜}{びん}は
\ruby{飜筋斗}{とん|ぼ|がへり}して、
%
\ruby{疊}{たゝみ}に
\ruby{溢}{こぼ}れたる
\ruby{紅色}{くれ|なゐ}の
\ruby{餘瀝}{した|ゝり}は、
%
\原本頁{164-6}\改行%
まだ
\ruby{早}{はや}き
\ruby{紅葉}{もみ|ぢ}を
こゝに
\ruby{散}{ち}らしたり。
