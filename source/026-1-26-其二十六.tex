\Entry{其二十六}

% メモ 校正終了 2024-04-09 2024-05-26 2024-06-19
\原本頁{158-5}%
\ruby{語}{かた}り
つゞけたる
\ruby[g]{談話}{はなし }の
\ruby{間}{うち}、
%
\ruby{息}{いき}
つぎ〳〵に
われ
\ruby{知}{し}らず
\ruby{飮}{の}みし
\ruby{葡萄酒}{ぶ|だう|しゆ}の
\ruby{量}{りやう}の
\ruby{少}{すくな}からで、
%
\ruby{既}{すで}に
\ruby{其}{そ}の
\ruby{六七{\換字{分}}}{ろく|しち|ぶ}を% 原本には漢数字「七」のルビ無し
\ruby{盡}{つく}したれば、
%
\ruby[||j>]{醉}{すゐ}
\ruby[||j>]{興}{きよう}% 「醉」は原本通り「ゐ」で調整
% \ruby{醉興}{すゐ|きよう}% 「醉」は原本通り「ゐ」で調整
おのづから
\ruby{發}{はつ}して
\ruby{獨}{ひと}り
\ruby[g]{機{\換字{嫌}}}{き げん}よく、
%
\ruby{不規律}{ふ|き|りつ}の
\ruby[||j>]{大}{たい}
\ruby[||j>]{將}{しやう}を
% \ruby{大將}{たい|しやう}を
もて
\ruby{自}{みづか}ら
\ruby{許}{ゆる}せるほど
ありて、
%
ふたゝび
\ruby{睡}{ねむ}りには
\ruby{就}{つ}かんともせず、
%
\ruby[g]{島木}{しまき }は
\ruby{{\換字{猶}}}{なほ}
ぐびりぐびりと% ルビ調整(原本通り)非踊り字表記(行末行頭の境界付近)
\ruby[||j>]{獨}{どく}
\ruby[||j>]{{\換字{酌}}}{しやく}を
% \ruby{獨{\換字{酌}}}{どく|しやく}を
\ruby{續}{つゞ}けたり。

\原本頁{158-10}%
むつくりと
\ruby{肥}{こ}えたる
\ruby[g]{身體}{からだ }
ゆたかに
\ruby[g]{胡坐}{あぐら }
かきて、
%
\ruby{土}{つち}
\ruby{多}{おほ}き
\ruby{山}{やま}の
\ruby{岩}{いは}を
\原本頁{159-1}\改行%
\ruby{隱}{かく}せるが
\ruby{如}{ごと}くに、
%
\ruby{肉}{にく}
ふくらかにして
\ruby{骨}{ほね}を
\ruby{見}{み}せぬ
\ruby[g]{丸々}{まる〳〵}としたる
\ruby{顏}{かほ}の、
%
\ruby{其}{そ}の
\ruby{小}{ちひ}さなる
\ruby{眼}{め}の
あたりに
\ruby{笑}{ゑみ}を
\ruby{含}{ふく}み、
%
\ruby{今}{いま}しも
ぐつと
\ruby[g]{一盞}{いつさん}を
\ruby{仰}{あふ}ぎたるが、

\原本頁{159-4}%
『
もう
\ruby{出}{で}て
\ruby{來}{き}さうなものだがナ、
%
\ruby[||j>]{畜}{ちく}
\ruby[||j>]{生}{しやう}!、
% \ruby{畜生}{ちく|しやう}!、
%
まだかナ。
』

\原本頁{159-5}%
と、
%
\ruby{誰}{たれ}に
\ruby{云}{い}へるともなく
\ruby{自}{みづか}ら
\ruby{語}{かた}れり。

\原本頁{159-6}%
\ruby[g]{島木}{しまき }は
\ruby[g]{水野}{みづの }が
\ruby[g]{胸中}{む ね }を
\ruby{知}{し}りたれど、
%
\ruby[g]{水野}{みづの }は
\ruby[g]{島木}{しまき }が
\ruby[g]{肚裏}{は ら }を
\ruby{知}{し}らざりき。
%
\ruby[g]{妻子}{さいし }
\ruby[||j>]{兄}{きやう}
\ruby[||j>]{弟}{ だい}も
% \ruby{兄弟}{きやう|だい}も
\ruby{無}{な}く
\ruby{親}{おや}も
\ruby{無}{な}ければ、
%
\ruby{氣}{き}まゝなる
\ruby[g]{寄寓}{かりずみ}の
\ruby[g]{面倒}{めんだう}
\ruby{無}{な}きを
\ruby{悅}{よろこ}びて、
%
\ruby[g]{一家}{いつか }を
こそは
\ruby{{\換字{猶}}}{なほ}
\ruby{構}{かま}へざれ、
%
\ruby[g]{幾度}{いくたび}か
\ruby{{\換字{浮}}}{う}き
\ruby[g]{幾度}{いくたび}か
\ruby{沈}{しづ}みし
\原本頁{159-9}\改行%
\ruby{末}{すゑ}に、
%
\ruby{漸}{やうや}く
\ruby[||j>]{合}{がふ}
\ruby[||j>]{百}{ひやく}の
% \ruby{合百}{がふ|ひやく}の
% 合百(ごうひゃく)は、特定の株式や商品先物などの値動きを対象とする日本の伝統的な賭博。
% 合百の語源は、米相場の変動額に対して銭百文をかけたこと
\ruby{果敢無}{は|か|な}きより、
%
\ruby{今}{いま}は
\ruby{人}{ひと}の
\ruby{噂}{うはさ}にも
\ruby{上}{のぼ}るほどの
\ruby[<j||]{玉}{ぎよく}% 行末行頭の境界付近なので特例処置を施す
\ruby[||j>]{高}{だか}を
% \ruby{玉高}{ぎよく|だか}を
\ruby{動}{うご}かすに
\ruby{至}{いた}りし
\ruby[g]{島木}{しまき }も、
%
もとより
\ruby{右}{みぎ}は
\ruby[g]{地獄}{ぢ ごく}
\ruby{左}{ひだり}は
\ruby[g]{極樂}{ごくらく}の
\ruby{間}{あひだ}の
\原本頁{159-11}\改行%
\ruby{綱}{つな}を
\ruby{渡}{わた}つて
\ruby{日}{ひ}を
\ruby{{\換字{送}}}{おく}る
\ruby{投機師}{とう|き|し}の
\ruby{身}{み}の
\ruby{上}{うへ}は、
%
\ruby[g]{貨物}{くわぶつ}を
\ruby{積}{つ}み
\ruby[g]{問屋}{とひや }を
\ruby{控}{ひか}へて
\ruby{十}{じふ}の
\ruby{一}{いち}
\ruby{十}{じふ}の
\ruby{二}{に}の
\ruby{利}{り}を
% 「十の一二」(じゅうのいちに):わずか
\ruby{征}{と}りて
\ruby{行}{ゆ}く
\ruby[g]{堅氣}{かたぎ }の
\ruby[g]{商人}{あきうど}とは
\ruby{異}{こと}なれば、
%
\ruby[g]{此處}{こ こ }% ルビ調整(原本通り)非踊り字表記(行末行頭の境界付近)
\ruby{一}{ひ}ト
\ruby{伸}{のし}と
\ruby{有}{あ}らん
\ruby{限}{かぎ}りの
\ruby[||j>]{力}{ちから}
\ruby[||j>]{瘤}{ こぶ}を
\ruby{入}{い}れて
\ruby{蒐}{かゝ}れる
\ruby{此}{こ}の
\ruby{秋}{あき}の、
%
\ruby[g]{天候}{てんこう}を
\原本頁{160-3}\改行%
\ruby{重}{おも}なる
\ruby[g]{相場}{さうば }の% 原文通り「場」
\ruby[g]{時季}{と き }に、
%
\ruby{捉}{とら}へ
かねたる
\ruby{雲}{くも}の
\ruby[||j>]{心}{こゝろ}
\ruby[||j>]{風}{ かぜ}の
% \ruby{心風}{こゝろ|かぜ}の
\ruby[g]{料簡}{れうけん}は
\ruby{我}{わ}が
\ruby{思}{おも}はくと
\ruby{{\換字{違}}}{ちが}ひて、
%
\ruby[g]{{\換字{追}}敷}{おひじき}% →「追証」→「追加証拠金」
% 株式の信用取引や、外国為替証拠金取引(FX)において、証拠金が不足した際に差し入れるお金のこと。
\ruby[g]{々々}{ 〳〵 }と
\ruby{取}{と}り
\ruby{立}{た}てらるゝに
\ruby[g]{懷中}{ふところ}
\ruby{危}{あやふ}く、
%
\ruby{既}{すで}に
\ruby{其}{そ}の
\原本頁{160-5}\改行%
\ruby{剩}{あま}すところは
\ruby[g]{幾何}{いくばく}も
あらぬ
\ruby[||j>]{端}{はした}
\ruby[||j>]{錢}{ がね}と
% \ruby{端錢}{はした|がね}と
なりて、
%
\ruby{{\換字{運}}}{うん}と
\ruby[<g||]{志}{こゝろ}% ルビ調整(特殊処理)親文字1に対してルビ5文字
\ruby[||j>]{と}{ざし}
% \ruby[<j>]{志と}{こゝろざし}
の
\ruby{今}{いま}
\ruby[g]{少時}{しばし }
\ruby{反}{そむ}かば、
%
またもや
\ruby{身}{み}の
\ruby{皮}{かは}も% 原本通り「皮 か(は)」
\ruby{無}{な}き
\ruby{赤裸々}{あか|はだ|か}となりて、
%
\ruby{賽}{さい}の
\ruby[g]{河原}{か はら}に
\ruby{積}{つ}める
\ruby{石}{いし}の
\ruby{{\換字{瓦}}落離}{ぐわ|ら|り}と
\ruby{崩}{くづ}れたる
\ruby[g]{{\換字{情}}無}{なさけな}さを
\ruby{見}{み}るべしと、
%
\ruby[g]{流石}{さすが }に
\ruby{心}{こゝろ}も
おちつき
かぬるところへ、
%
\ruby{折}{をり}も
\ruby{折}{をり}とて
\ruby[g]{水野}{みづの }の
\ruby[g]{無心}{む しん}なり。
%
\ruby{{\換字{運}}}{うん}を
\ruby[g]{背負}{せ お }へる
\ruby{時}{とき}には
\ruby{其}{そ}の
\ruby[g]{二倍}{に ばい}
\ruby[g]{三倍}{さんばい}も
\ruby{與}{あた}ふるに
\ruby{易}{やす}けれど、
%
\ruby[g]{夜明}{よ あ }けての
\ruby[g]{天地}{てんち }の
\原本頁{160-10}\改行%
\ruby[g]{狀態}{やうす }
\ruby[g]{次第}{し たい}にて
\ruby{我}{わ}が
\ruby[g]{生命}{いのち }はと
さへ
\ruby{思}{おも}へる
\ruby[g]{矢先}{や さき}に
\ruby{云}{い}ひかけられては
\改行% 校正作業の簡略化のため
、
%
\原本頁{160-11}\改行%
\ruby[||j>]{敗}{まけ}
\ruby[||j>]{軍}{いくさ}の
% \ruby{敗軍}{まけ|いくさ}の
\ruby{{\換字{退}}}{ひ}き
\ruby{際}{ぎは}に
\ruby{頼}{たの}みきつたる
\ruby[g]{持鎗}{もちやり}を
\ruby[g]{{\換字{所}}望}{しよまう}されたる
\ruby[g]{心地}{こゝち }して、
%
\ruby[g]{流石}{さすが }の
\ruby[g]{島木}{しまき }も
\ruby{行}{ゆ}き
\ruby{詰}{つま}りしが、
%
\ruby{竹}{たけ}を
\ruby{割}{わ}つたる
\ruby{如}{ごと}き
\ruby[g]{持{\換字{前}}}{もちまへ}の
\ruby[g]{氣象}{きしやう}は
\ruby{義}{ぎ}を
\原本頁{161-2}\改行%
\ruby{見}{み}て
\ruby{勇}{いさ}んで、
%
エヽ
どうせ
\ruby{曲}{まが}つて
\ruby[g]{仕舞}{し ま }えば
\ruby{無}{な}くなる
\ruby{金}{かね}を、
%
\ruby{今}{いま}
\ruby{{\換字{遣}}}{や}つて
\ruby[g]{仕舞}{し ま }へば
\ruby[g]{友{\換字{達}}}{ともだち}の
\ruby[g]{利益}{た め }!、
%
\ruby[g]{踏張}{ふんば }れ〳〵
\ruby{男}{をとこ}の
\ruby{兒}{がき}だ、
%
\ruby[g]{裸々}{はだか}に
なつて
も
\ruby{怖}{こは}くは
\ruby{無}{な}い、
%
\ruby[<j||]{百}{ひやく}
\ruby[||j>]{兩}{りやう}
ばかりの
\ruby{鼻糞金}{はな|くそ|がね}を
\ruby{出}{だ}し
\ruby{悋}{をし}んでは、
%
\ruby{萬五郎}{まん|ご|らう}の
\原本頁{161-5}\改行%
\ruby{男}{をとこ}が
\ruby{廢}{す}たる!、
%
\ruby[g]{{\換字{情}}無}{なさけな}い!、
%
\ruby[g]{行末}{ゆくすゑ}が
\ruby{見}{み}える!、
%
\ruby[<j||]{百}{ひやく}% ルビ調整(配置位置調整)親文字に対してルビが多いので親文字毎分解
\ruby[||j>]{萬}{まん}
\ruby[||j>]{兩}{りやう}
\ruby[||j>]{{\換字{分}}}{ ぶ }
\ruby[||j>]{限}{げん}
% \ruby{百萬兩{\換字{分}}限}{ひやく|まん|りやう|ぶ|げん}
になつた
\ruby{時}{とき}の
\ruby[||j>]{額}{むかふ}
\ruby[||j>]{疵}{ きず}になる!、
% \ruby{額疵}{むかふ|きず}になる!、
%
\ruby{握}{にぎ}つた
\ruby{錢}{ぜに}から
\ruby{{\換字{煙}}}{けむ}を
\ruby{出}{だ}すのは
\ruby{三{\換字{文}}野郎}{さん|もん|や|らう}のする
\ruby{事}{こと}だ、
%
と
\ruby{早}{はや}くも
\ruby[||j>]{決}{けつ}
\ruby[||j>]{着}{ちやく}して
% \ruby{決着}{けつ|ちやく}して
\ruby[g]{臓腑}{ざうふ }を
\ruby{見}{み}せずに、
%
\ruby[g]{奇麗}{き れい}に
\ruby[<j>]{快}{こゝろよ}く
\ruby[g]{用立}{ようだ }てて% ルビ調整(原本通り)非踊り字表記(行末行頭の境界付近)
\ruby{歸}{かへ}しやりつ、
%
さて
\ruby{其}{それ}が
ためとにも
あらざるべけれど、
%
\ruby{何}{なん}と
\ruby{無}{な}く
\ruby{心}{こゝろ}に
\ruby[g]{怡悅}{よろこび}を
\ruby{覺}{おぼ}えて、
%
\ruby{今}{いま}は
\ruby{氣}{き}も
\ruby{冴}{さ}え〴〵と
\ruby{飮}{の}み
\ruby{居}{を}れるなり。

\原本頁{161-10}%
『
もう
\ruby{出}{で}て
\ruby{來}{き}さうなものだがナ、
%
まだかナ、
%
\ruby[||j>]{畜}{ちく}
\ruby[||j>]{生}{しやう}!。
% \ruby{畜生}{ちく|しやう}!。
』

\原本頁{161-11}%
ふたゝび
\ruby{獨}{ひと}り
ごちて
\ruby[g]{酒盞}{さかづき}を
\ruby{取}{と}りぬ。

\原本頁{162-1}%
『
まだ
\ruby{出}{で}て
\ruby{來}{こ}ないかナ、
%
\ruby[||j>]{畜}{ちく}
\ruby[||j>]{生}{しやう}めツ!。
% \ruby{畜生}{ちく|しやう}めツ!。
』

\原本頁{162-2}%
\ruby{何}{なに}を
\ruby{待}{ま}てるにか
\ruby[g]{三度}{み たび}
\ruby[g]{獨語}{ひとりご}ちしが、
%
\ruby{答}{こた}ふるものは
\ruby{有}{あ}るべくも
\ruby{無}{な}く
\改行% 校正作業の簡略化のため
、
%
\原本頁{162-3}\改行%
\ruby{室}{しつ}の
\ruby[g]{一隅}{いちぐう}の
\ruby[g]{小机}{こづくゑ}の
\ruby{上}{うへ}の
%
\ruby[||j>]{懷}{くわい}% 「懷中(くわいちゆう)」「ゆ」有り
\ruby[||j>]{中}{ ちゆ}% ルビ調整(原本通り)親文字とルビがバラバラになっているが
\ruby[||j>]{時}{ うど}
\ruby[||j>]{計}{ けい}
% \ruby{懷中時計}{くわい|ちゆう|ど|けい}% 「懷中(くわいちゆう)」「ゆ」有り
の
\ruby{音}{おと}のみの
\ruby{有}{あ}るか
\ruby{無}{な}きかに
\ruby{響}{ひゞ}け
\改行% 校正作業の簡略化のため
り。

\原本頁{162-5}%
\ruby[g]{相手}{あひて }
\ruby{無}{な}き
\ruby{淋}{さび}しさに
\ruby{堪}{た}へかねてか、

\原本頁{162-6}%
『
\ruby[||j>]{畜}{ちく}
\ruby[||j>]{生}{しやう}ツ、
% \ruby{畜生}{ちく|しやう}ツ、
%
\ruby{出}{で}て
\ruby{來}{き}やがらなくつても
\ruby[g]{仕方}{し かた}が
\ruby{無}{な}いかナ。
%
ハヽヽ、
%
\ruby{怒}{おこ}るほど
\ruby[g]{乃公}{お れ }も
\ruby[g]{野暮}{や ぼ }ぢやあ
いけねえ。
%
それは
さうと
\ruby[g]{水野}{みづの }は
もう
\原本頁{162-8}\改行%
\ruby[g]{大{\換字{分}}}{だいぶ }
\ruby{行}{い}つたらう。
%
\ruby[||j>]{愍}{かあ}
\ruby[||j>]{然}{いさう}に、% 「愍然 か(あ)いさう」
% \ruby{愍然}{かあ|いさう}に、% 「愍然 か(あ)いさう」
%
\ruby{堅}{かた}い
\ruby[||j>]{正}{しやう}
\ruby[||j>]{直}{ ぢき}な
% \ruby{正直}{しやう|ぢき}な
\ruby{男}{をとこ}だから、
%
\ruby{人一倍}{ひと|いち|ばい}
\ruby[g]{何彼}{なにか }につけて
\ruby[||j>]{物}{もの}
\ruby[||j>]{思}{おもひ}を
% \ruby{物思}{もの|おもひ}を
\ruby{仕}{し}て
\ruby{居}{ゐ}る!。

\原本頁{162-10}%
\換字{庵点}
\ruby{粋}{すゐ}な
\ruby[g]{{\換字{浮}}世}{うきよ }を
\ruby{戀}{こひ}
\ruby{故}{ゆゑ}に、
%
\ruby[g]{野暮}{や ぼ }に
\ruby{暮}{くら}すも
\ruby{心}{こゝろ}がら。
%
あゝ
\ruby[g]{端唄}{は うた}の
\ruby[g]{{\換字{文}}句}{もんく }ぢやあ
\ruby{無}{な}いが
\ruby{{\換字{迷}}}{まよ}つちやあ
\ruby[g]{野暮}{や ぼ }になる!。
%
フン、
%
ナンダ
\ruby[g]{此方}{こつち }やあ% ルビ調整(原本通り)
\ruby{戀}{こひ}
\ruby{故}{ゆゑ}ぢやあ
\ruby{無}{ね}えで、
%
\ruby{慾}{よく}
\ruby{故}{ゆゑ}に
\ruby[g]{野暮}{や ぼ }になり
\ruby{切}{き}つて
\ruby{居}{ゐ}やがる!。
%
アヽ
もう
そろ〳〵
\ruby{出}{で}て
\ruby{來}{き}て
\ruby{吳}{く}れても
\ruby{好}{よ}さゝうなものだが、
%
チヨツ
\ruby[g]{忌々}{いま〳〵}
\makeatletter
\@ifundefined{デバッグ@ビルド}{%
  \par%
}{%
  \relax%
}%
\makeatother
しい、
\換字{志゛}れつたいナア。% 「志」+「濁点」
%
ア、
%
\ruby[g]{豪氣}{がうぎ }に
\ruby{醉}{よ}つて% 「醉」は原本通り「よ」で調整
\ruby{來}{き}た、
%
\ruby{好}{い}い
\ruby[<j||]{心}{こゝろ}
\ruby[<j||]{持}{もち}だ
% \ruby{心持}{こゝろ|もち}だ
!
\改行% 校正作業の簡略化のため
。
%
\原本頁{163-4}\改行%
\ruby{何}{なん}だか
もう
\ruby{出}{で}て
\ruby{來}{き}さうな
\ruby[||j>]{心}{こゝろ}
\ruby[||j>]{持}{ もち}がする!。
%
ヱヽト、

\原本頁{163-5}%
\換字{庵点}
\ruby{起}{お}きて
\ruby{見}{み}つ、
%
\ruby{寢}{ね}て
\ruby{見}{み}つ
\ruby{待}{ま}てど、
%
たより
\ruby{無}{な}く、
%
チン〳〵
チンチン、
%
\ruby[g]{蚊屋}{か や }の
\ruby{廣}{ひろ}さに
たゞ
\ruby{獨}{ひと}り、
%
ツンテン、
%
\ruby{蚊}{か}を
\ruby{焼}{や}く
\ruby{火}{ひ}より
\ruby{胸}{むね}の
\ruby{火}{ひ}の、
%
\ruby{燃}{も}ゆる
おもひを
\ruby{察}{さつ}しやんせカナ。
%
ハヽヽヽ。
』

\原本頁{163-8}%
\ruby{聲}{こゑ}は
\ruby{美}{うつく}しからず
\ruby{錆}{さ}びたれど、
%
\ruby{聞}{き}き
\ruby[g]{記臆}{おぼえ }なるべきには% 原本通り「おぼえ」
\ruby[g]{似合}{に あ }はず
\ruby{我}{が}
\原本頁{163-9}\改行%
\ruby{流}{りう}の
\ruby[g]{{\換字{節}}{\換字{廻}}}{ふしまは}しにも
をかしきところありて、
%
\ruby[g]{小聲}{こ ゞゑ}に
\ruby{唱}{うた}ひ
\ruby[g]{仕舞}{し ま }ひつゝ
\改行% 校正作業の簡略化のため
、
%
\原本頁{163-10}\改行%
\ruby{今}{いま}
\ruby{將}{まさ}に
\ruby[g]{一壜}{いちびん}の% (ひと)と読ませるなら「一ト」のように記したと思う
\ruby{酒}{さけ}を
\ruby{盡}{つく}し
\ruby{果}{は}たさんとして、
%
\ruby{手}{て}に
\ruby{取}{と}り
\ruby{上}{あ}げて
\ruby{自}{みづか}ら
\ruby{{\換字{酌}}}{つ}がんと、
%
\ruby{其}{そ}の
\ruby[g]{尻下}{しりさが}りの
\ruby{小}{ちひさ}き
\ruby{目}{め}を
\ruby{一}{ひ}トしほ
\ruby{下}{さ}げて、
%
\ruby[g]{莞爾}{につこり}と
\ruby{樂}{たの}しげ
\原本頁{164-1}\改行%
に
\ruby{笑}{わら}ひしが、
%
\ruby{何}{なに}をか
\ruby{聞}{き}きつけしや
\ruby[g]{俄然}{が ぜん}として、

\原本頁{164-2}%
『
ヤツ、
%
\ruby{來}{き}たぞ!% \inhibitglue{}% ここは「空き」があるので
\,% 原本上でのアキを再現するため「3/18 em」空ける
\ruby{來}{き}て
\ruby{吳}{く}れたぞ!、
%
おいでなすつたぞ!。
%
\ruby{占}{し}めたナ!、
%
サア
\ruby{來}{こ}いだ!。
』

\原本頁{164-4}%
と
\ruby{飛}{と}び
\ruby{立}{た}つたり。

\原本頁{164-5}%
\ruby{投}{な}げ
\ruby{出}{だ}されたる
\ruby{壜}{びん}は
\ruby[g]{飜筋}{とんぼ }
\ruby{斗}{がへり}して、
% 翻筋斗(モンドリ)空中でからだを1回転させること。とんぼ返り。宙返り。
%
\ruby{疊}{たゝみ}に
\ruby{溢}{こぼ}れたる
\ruby[g]{紅色}{くれなゐ}の
\ruby[g]{餘瀝}{したゝり}は、
%
\原本頁{164-6}\改行%
まだ
\ruby{早}{はや}き
\ruby[g]{紅葉}{もみぢ }を
こゝに
\ruby{散}{ち}らしたり。
