\Entry{其二十七}

% メモ 校正終了 2024-04-10
\原本頁{164-7}%
がらりと
\ruby{樓}{ろう}の
\ruby{雨{\換字{戸}}}{あま|ど}を
\ruby{繰}{く}り
\ruby{開}{あ}くれば、
%
\ruby{白}{しら}みわたれる
\ruby[<j>]{曉}{あかつき}の
\ruby{天}{そら}より、
%
\原本頁{164-8}\改行%
\ruby{蓬々然}{ほう|〳〵|ぜん}として
\ruby{下}{おろ}し
\ruby{來}{く}る
\ruby{風}{かぜ}は、
%
おもむろに
\ruby{面}{おもて}を
\ruby{撲}{う}ち
\ruby{胸}{むね}を
\ruby{撲}{う}つて、
%
\原本頁{164-9}\改行%
\ruby{昨日}{きの|ふ}の
\ruby{夜}{よ}の
\ruby{靜穩}{おだ|やか}なりし
\ruby[<j>]{俤}{おもかげ}は
\ruby{{\換字{猶}}}{なほ}
\ruby{{\換字{遺}}}{のこ}れども、
%
\ruby{日}{ひ}の
\ruby{將}{まさ}に
\ruby{出}{い}でんとする
\原本頁{165-1}\改行%
\ruby{方}{かた}の
\ruby{雲}{くも}の
\ruby{色}{いろ}
\ruby{峻}{けは}しく、
%
\ruby{何}{なに}と
\ruby{無}{な}く
\ruby{物}{もの}
\ruby{凄}{すさま}じき
\ruby{景象}{やう|す}は
\ruby{見}{み}る〳〵
\ruby{動}{うご}き
\ruby{展}{の}びて、
%
やがて
\ruby{恐}{おそ}ろしくも
\ruby{一}{ひ}ト
\ruby{暴風}{あ|れ}の、
%
\ruby{暴}{あ}れ
\ruby{立}{た}たんとする
\ruby[<j>]{勢}{いきほひ}は
\ruby{現}{あら}はれたり。

\原本頁{165-4}%
\ruby{昔語}{むかし|がたり}の
\ruby{海坊主}{うみ|ばう|ず}の
\ruby{如}{ごと}く、
%
ヌツと
\ruby{突立}{つゝ|た}つたるまゝ
\ruby{四邊}{あた|り}を
\ruby{見{\換字{廻}}}{み|まは}せる
\ruby{島木}{しま|き}は、
%
\ruby{刻一刻}{こく|いつ|こく}に
\ruby{吹募}{ふき|つの}る
\ruby{風}{かぜ}の、
%
\ruby{袂}{たもと}を
\ruby{揚}{あ}げ
\ruby{裾}{すそ}を
\ruby{{\換字{扇}}}{あふ}るをも
\ruby{知}{し}らぬやうに、
%
\ruby{身}{み}じろぎもせずして
\ruby{居}{ゐ}たりしが、
%
\ruby{{\換字{終}}}{つひ}には
\ruby{此}{こ}の
\ruby{風}{かぜ}の
\ruby{高}{こう}じに
\ruby{高}{こう}じて、
%
\ruby{老木}{おい|き}の
\ruby{枝}{えだ}を
\ruby{裂}{さ}き、
%
\ruby{{\換字{若}}樹}{わか|ぎ}の
\ruby{根}{ね}を
\ruby{拔}{ぬ}き、
%
\ruby{沙}{すな}を
\ruby{舞}{ま}はせ
\ruby{石}{いし}を
\ruby{躍}{をど}らすに
\ruby{至}{いた}るべき
さまの、
%
\ruby{十{\換字{分}}}{じう|ぶん}に
\ruby{想}{おも}ひやらるゝに
\ruby{及}{およ}びて、
%
\ruby{大浪}{おほ|なみ}の
うねりて
\ruby{寄}{よ}するが
\ruby{如}{ごと}くに、
%
\ruby{肥}{ふと}つたる
\ruby[||j>]{顏}{かほ}
\ruby[||j>]{中}{ぢゆう}を
% \ruby{顏中}{かほ|ぢゆう}を
\ruby{笑}{わらひ}に
\ruby{動}{うご}かして、

\原本頁{165-10}%
『ウフ、ウフ、
%
ウアツハツハヽハヽ。
%
とう〳〵
\ruby{來}{き}やがつたナ!%\inhibitglue{}% ここは原本では行末なので「空き」はないが
\,% 原本上でのアキを再現するため「3/18 em」空ける
%\原本頁{165-11}\改行%
ヤイ
\ruby{風}{かぜ}の
\ruby{神}{かみ}!。
%
\ruby[||j>]{男振}{をとこ|ぶり}が
\ruby{好}{い}いぞ!%\inhibitglue{}% ここは「空き」があるので
\,% 原本上でのアキを再現するため「3/18 em」空ける
\換字{志}つかり
\ruby{{\換字{遺}}}{や}れ。
%
\ruby{雨}{あめ}の
\ruby{隨}{つ}いて
\ruby{來}{き}やがらねえのは
\ruby{忌々}{いま|〳〵}しいが、
%
\ruby{仕方}{し|かた}が
\ruby{無}{ね}え、
%
\ruby{汝}{てめへ}だけで
ウンと
\ruby{働}{はたら}け。
%
\原本頁{166-2}\改行%
\ruby[|j>]{男}{をとこ}
\ruby{振}{ ぶり}が% 全角空白は「男(をとこ)」のルビが突出為の対策
\ruby{好}{い}いぞ、
%
〳〵!。
』

\原本頁{166-3}%
と、
%
\ruby[||j>]{打}{うち}
\ruby[||j>]{戲}{たはむ}れて
% \ruby{打戲}{うち|たはむ}れて
\ruby{引{\換字{返}}}{ひつ|かへ}せば、
%
\ruby{洋燈}{らん|ぷ}は
\ruby{既}{すで}に
\ruby{風}{かぜ}に
\ruby{{\換字{消}}}{け}されて、
%
\ruby{室}{しつ}に
\ruby{滿}{み}てる
\原本頁{166-4}\改行%
\ruby[||j>]{曙}{しよ}
\ruby[||j>]{色}{しよく}は
% \ruby{曙色}{しよ|しよく}は
\ruby{其}{その}
\ruby[||j>]{光}{ひかり}に
\ruby{代}{かは}り
\ruby{居}{ゐ}たり。

\原本頁{166-5}%
\ruby{島木}{しま|き}が
\ruby{待}{ま}ちに
\ruby{待}{ま}つたるは、
%
\ruby{我}{われ}を
\ruby{訪}{と}ひ
\ruby{來}{こ}ん
\ruby{{\換字{婦}}}{をんな}にも
あらねば、
%
\ruby{他{\換字{所}}}{よ|そ}より
\ruby{入}{い}るべき
\ruby{金}{かね}にも
あらず、
%
\ruby{唯}{たゞ}
\ruby{此}{こ}の
\ruby{野}{の}を
\ruby{拂}{はら}ひ
\ruby{禾}{くわ}を
\ruby{偃}{ふ}すの
\ruby{風}{かぜ}なりけり。
%
\ruby{數日{\換字{前}}}{すう|じつ|ぜん}より、
%
\ruby{乾坤一擲}{けん|こん|いつ|てき}と
\ruby{試}{こゝろ}みたる
\ruby[||j>]{丁}{ちやう}
\ruby[||j>]{{\換字{半}}}{ はん}の、
% \ruby{丁{\換字{半}}}{ちやう|はん}の、
%
\ruby{時}{とき}、
%
\ruby{利}{り}あらずして
\ruby{思}{おも}ふ
\ruby{目}{め}は
\ruby{出}{い}でず、
%
\換字{志}きりに
\ruby{敵}{てき}に
\ruby{切}{き}り
\ruby{捲}{まく}られて、
%
\ruby{踏}{ふ}み
\ruby{耐}{こた}へ
\ruby{踏}{ふ}み
\ruby{耐}{こた}へては
\ruby{戰}{たゝか}ふものゝ、
%
\ruby{既}{すで}に
\ruby{味方}{み|かた}は
\ruby{崩}{くづ}れ
\ruby{立}{た}つて
\ruby{討死手負}{うち|じに|て|おひ}の
\ruby{數}{かず}を
\原本頁{166-10}\改行%
\ruby{知}{し}らず、
%
\ruby{大勢}{たい|せい}の
ほゞ
\ruby{定}{さだ}まりたるに、
%
\ruby{無念}{む|ねん}の
\ruby{牙}{きば}を
\ruby{咬}{か}み
\ruby{血眼}{ち|まなこ}を
\ruby{瞋}{いか}らして、
%
\ruby[||j>]{大}{おほ}
\ruby[||j>]{童}{わらは}に
% \ruby{大童}{おほ|わらは}に
なつて
\ruby{奮闘}{ふん|とう}すれども、
%
\ruby{疲}{つか}れきつたる
\ruby{身}{み}の
\ruby{思}{おも}ふに
\ruby{任}{まか}せねば、
%
\原本頁{167-1}%
\ruby{天{\換字{運}}}{てん|うん}
いよ〳〵
\ruby{我}{われ}に
\ruby{惠}{めぐ}まずば
\ruby{屍}{かばね}を
\ruby{原頭}{げん|とう}に
\ruby{曝}{さら}すも
\ruby{今}{いま}の
\ruby{間}{ま}ならんと、
%
\ruby{覺悟}{かく|ご}の
\ruby{臍}{ほぞ}を
\ruby{固}{かた}めつゝも、
%
あはれ
\ruby{一}{ひ}ト
\ruby{暴風}{あ|れ}も
あれかしと
\ruby{祈}{いの}り
\ruby{居}{ゐ}けるに、
%
\ruby{昨日}{きの|ふ}の
\ruby{芝浦}{しば|うら}の
\ruby{會}{くわい}の
\ruby[||j>]{席}{せき}
\ruby[||j>]{上}{じやう}より、
% \ruby{席上}{せき|じやう}より、
%
\ruby{羽{\換字{勝}}}{は|がち}の
\ruby{豫言}{こと|ば}と
いひ
\ruby{星}{ほし}の
\ruby{光}{ひかり}と
\ruby{云}{い}ひ、
%
\ruby{頼}{たの}もしく
\ruby{思}{おも}はるゝ
\ruby{{\換字{節}}}{ふし}の
\ruby{少}{すくな}からぬを
\ruby{知}{し}つて、
%
\ruby{危}{あや}ぶみ
ながらも
\ruby{待}{ま}ち
\ruby{居}{ゐ}たりし
\ruby{其}{その}
\ruby{風}{かぜ}の、
%
\ruby{果}{はた}して
\ruby{獵々颯々}{れふ|〳〵|さつ|〳〵}として
\ruby{吹}{ふ}き
\ruby{出}{いだ}したるに、
%
\ruby{今}{いま}
\ruby{見}{み}よ
\ruby{敗}{やぶれ}を
\ruby{轉}{てん}じて
\ruby{{\換字{勝}}}{かち}と
なさんは
\ruby{瞬}{またゝ}く
\ruby{間}{ま}なり、
%
\ruby{盛}{せ}り
\ruby{{\換字{返}}}{かへ}して
\ruby[||j>]{鏖}{みな}
\ruby[||j>]{殺}{ごろし}にして
% \ruby{鏖殺}{みな|ごろし}にして
\ruby{吳}{く}れんと、
%
\ruby{駒}{こま}の
\ruby{頭}{かしら}を
\ruby{立直}{たて|なほ}して
\ruby{鞍蓋}{くら|かさ}に
\ruby{突立上}{つゝ|たち|あが}つたる
\ruby[||j>]{將}{しやう}
\ruby[||j>]{軍}{ ぐん}の
% \ruby{將軍}{しやう|ぐん}の
\ruby{意氣}{い|き}
\ruby{既}{すで}に
\ruby{疾}{はや}く
\ruby{敵}{てき}を
\ruby{吞}{の}んで
\ruby{槊}{さく}を
\ruby{横}{よこた}へて
\ruby{眼}{め}も
\ruby{遙}{はるか}に
\ruby{睥睨}{へい|げい}するが
\ruby{如}{ごと}く、
%
\ruby{勃々}{ぼつ|〳〵}たる
\ruby{英氣}{えい|き}と
\ruby{限}{かぎ}り
\ruby{無}{な}き
\ruby{活力}{くわつ|りよく}との、
%
\ruby{溢}{あふ}るゝ
ばかり
\ruby{身}{み}に
\原本頁{167-10}\改行%
\ruby{湧}{わ}くを
\ruby{覺}{おぼ}えて、
%
\ruby{流石}{さす|が}の
\ruby{島木}{しま|き}も
\ruby{押包}{おし|つゝ}み
\ruby{{\換字{兼}}}{か}ねつ、
%
\ruby{數聲}{すう|せい}の
\ruby{笑}{わらひ}を
\ruby{漏}{も}ら
\換字{志}しなりけり。

\原本頁{168-1}%
\ruby{死生存亡}{し|せい|そん|ばう}
\ruby{此}{こ}の
\ruby{一擧}{いつ|きよ}と、
%
\ruby{鎬}{しのぎ}を
\ruby{{\換字{削}}}{けづ}つて
\ruby{爭}{あらそ}ふべき
\ruby{戰鬪}{たゝ|かひ}は、
%
\ruby{今}{いま}より
\ruby{二三時間}{に|さん|じ|かん}の
\ruby{後}{のち}に
\ruby{逼}{せま}り
\ruby{居}{を}れり。
%
\ruby{島木}{しま|き}は
\ruby{重}{おも}げなる
\ruby{身}{み}を
\ruby{無{\換字{造}}作}{む|ざう|さ}に
\ruby{動}{うご}かして、
%
\原本頁{168-3}\改行%
\ruby{自}{みづか}ら
\ruby{押入}{おし|いれ}より
\ruby{夜具}{や|ぐ}
\ruby{取}{と}り
\ruby{出}{いだ}しつ、
%
ごろりと
\ruby{其}{それ}に
くるまりて、
%
\ruby{横}{よこ}に
なるが
\ruby{早}{はや}きか
\ruby{頓}{やが}て
\ruby{睡}{ねむ}りぬ。
%
\ruby{島木}{しま|き}は
\ruby{自}{みづか}ら
\ruby{敎}{をし}へ
\ruby{自}{みづか}ら
\ruby{養}{やしな}ひて、
%
\ruby{敎}{をし}へ
\ruby{得}{え}
\原本頁{168-5}\改行%
\ruby{養}{やしな}ひ
\ruby{得}{え}たるところある
\ruby{男}{をとこ}なりけり。

\原本頁{168-6}%
\ruby{風}{かぜ}は
\ruby{次第}{し|だい}に
\ruby{烈}{はげ}しくなりぬ。
%
\ruby{鼾}{いびき}は
\ruby{漸}{やうや}く
\ruby{盛}{さかん}になりぬ。
%
\ruby{風}{かぜ}の
\ruby{息}{や}む
\ruby{時}{とき}、
%
\原本頁{168-7}\改行%
\ruby{鼾}{いびき}の
\ruby{聲}{こゑ}あり、
%
\ruby{鼾}{いびき}の
\ruby{無}{な}き
\ruby{時}{とき}、
%
\ruby{風}{かぜ}の
\ruby{音}{おと}あり。
%
\ruby{開}{ひら}き
\ruby{放}{はな}されたる
\ruby{押入}{おし|いれ}、
%
\原本頁{168-8}\改行%
\ruby{投出}{なげ|だ}されたる
\ruby{酒瓶}{とつ|くり}、
%
\ruby{{\換字{消}}}{き}えたる
\ruby{洋燈}{らん|ぷ}、
%
\ruby{{\換字{空}}虛}{か|ら}の
\ruby{罐}{くわん}、
%
\ruby{歪}{いびつ}に
\ruby{展}{の}べられたる
\ruby{蒲團}{ふ|とん}、
%
\ruby{明}{あ}けかけたる
\ruby{雨{\換字{戸}}}{あま|ど}、
%
\ruby{雷}{らい}の
\ruby{如}{ごと}き
\ruby{鼾聲}{い|びき}、
%
\ruby{波濤}{な|み}と
\ruby{轟}{とゞろ}く
\ruby{風}{かぜ}の
\ruby{音}{おと}、
%
\原本頁{168-10}\改行%
\ruby{埒}{らち}
\ruby{無}{な}しとも
\ruby{{\換字{狼}}{\換字{藉}}}{らう|ぜき}とも
\ruby{亂暴}{らん|ばう}とも、
%
\ruby{云}{い}ふべき
\ruby{言葉}{こと|ば}は
\ruby{無}{な}き
\ruby{一室}{いつ|しつ}の
\ruby{狀}{さま}なり。
