\Entry{其二十八}

% メモ 校正終了 2024-04-10 2024-05-27 2024-06-19
\原本頁{169-2}%
『
\ruby{甚}{ひど}く
\ruby[g]{寢{\換字{込}}}{ね こ }んで
\ruby{居}{ゐ}たぢや
\ruby{無}{な}いか。
』

\原本頁{169-3}%
と、
%
\ruby{其}{その}
\ruby[||j>]{頭}{かしら}に
\ruby{黄金細工}{き|ん|ざい|く}の
\ruby{施}{ほどこ}しある
\ruby{美}{うつく}しき
\ruby[g]{琥珀}{こ はく}の
パイプを
\ruby{口}{くち}より
\ruby{放}{はな}しさまに
\ruby{云}{い}ひたるは、
%
\ruby[g]{梅幸}{ばいかう}の% 歌舞伎役者の尾上梅幸のことと思われる
\ruby[g]{伊東}{い とう}と
\ruby[g]{渾名}{あだな }
\ruby{呼}{よ}ばるゝも
\ruby[g]{無理}{む り }ならず
\原本頁{169-5}\改行%
\ruby{見}{み}ゆる
\ruby{其}{その}
\ruby{人}{ひと}を
\ruby{其}{その}
\ruby{儘}{まゝ}の
\ruby[g]{面立}{おもだち}の、
%
\ruby{三十三四}{さん|じふ|さん|し}の
\ruby{色}{いろ}
\ruby{白}{しろ}き
\ruby{男}{をとこ}にて、
%
\ruby[g]{昨夜}{ゆふべ }を
\原本頁{169-6}\改行%
\ruby[g]{何處}{いづく }にてか
\ruby{{\換字{過}}}{すご}しての
\ruby[g]{今{\換字{朝}}}{け さ }、
%
\ruby{何}{なに}か
\ruby[g]{用事}{ようじ }ありて
\ruby[g]{此家}{こ ゝ }に
たち
\ruby{戾}{もど}りしが
\改行% 校正作業の簡略化のため
、
%
\原本頁{169-7}\改行%
\ruby{今}{いま}や
\ruby{既}{すで}に
\ruby[g]{{\換字{朝}}食}{あさげ }を
\ruby{濟}{す}ませて
\ruby{{\換字{率}}}{いざ}
これよりと、
%
\ruby[g]{戰鬪}{たゝかひ}の
\ruby{場}{ば}へ% 原文通り「場」
\ruby{赴}{おもむ}かんとする
\ruby{{\換字{前}}}{まへ}の
\ruby[g]{僅少}{わづか }の
\ruby{暇}{いとま}を、
%
\ruby[g]{{\換字{煙}}草}{たばこ }
\ruby{休}{やす}みに
\ruby[g]{島木}{しまき }の
\ruby{室}{しつ}に
\ruby{來}{き}て、
%
\ruby[g]{昨日}{きのふ }
\ruby[g]{今日}{け ふ }
こそ
\原本頁{169-10}\改行%
\ruby{敵}{てき}
\ruby[g]{味方}{み かた}と
\ruby[g]{立別}{たちわか}れて
こそあれ
\ruby{同}{おな}じ
\ruby[g]{修羅}{しゆら }の
\ruby{巷}{ちまた}の
\ruby{友}{とも}の
\ruby[g]{相語}{あひかた}らへるなり
\改行% 校正作業の簡略化のため
。

\原本頁{169-11}%
『
アヽ、
%
ちよいと
\ruby{寢}{ね}やうと
\ruby{思}{おも}つたけが、
%
つい
ぐつすりと
\ruby{寢}{ね}て
\ruby[g]{仕舞}{し ま }つた。
』

\原本頁{170-2}%
『
\ruby{{\換字{宵}}}{よひ}にやあ
\ruby{汝}{おめへ}
\ruby[||j>]{寢}{ ね }られなかつたナ。
』

\原本頁{170-3}%
\ruby{微}{かすか}に
\ruby[g]{冷笑}{あざわら}ふ
\ruby[g]{樣子}{やうす }の
\ruby[<j>]{唇}{くちびる}の
\ruby{端}{はた}に
\ruby{見}{み}ゆるを、
%
\ruby{見}{み}て
\ruby{取}{と}つたる
\ruby[g]{島木}{しまき }は
\ruby[g]{一寸}{いつすん}も
\ruby{{\換字{退}}}{ひ}けては
\ruby{居}{ゐ}ず。

\原本頁{170-5}%
『
\ruby[g]{馬鹿}{ば か }あ
\ruby{云}{い}へ。
%
\ruby{汝}{おめへ}のやうな
\ruby[g]{繊細}{か ぼそ}い
\ruby[g]{野郎}{や らう}ぢやあ
\ruby{有}{あ}るめえし、
%
そんな
\ruby[g]{卑小}{け ち }な
\ruby[||j>]{根}{こん}
\ruby[||j>]{性}{じやう}は
% \ruby{根性}{こん|じやう}は
\ruby{持}{も}たねえ
\ruby{萬五郎}{まん|ご|らう}さまだ。
%
お
\ruby{作}{さく}に
\ruby{聞}{き}いて
\ruby{見}{み}りやあ
\原本頁{170-7}\改行%
\ruby{解}{わか}る
\ruby{事}{こと}だ。
』

\原本頁{170-8}%
『
ハヽヽ、
%
\ruby[g]{豪氣}{がうぎ }に
\ruby[g]{今{\換字{朝}}}{け さ }は
\ruby{氣}{き}が
\ruby{{\換字{強}}}{つよ}いナ。
%
\ruby[g]{背後}{うしろ }から
\ruby{風}{かぜ}が
\ruby{推}{お}してるからナア。
』

\原本頁{170-10}%
『
フン、
%
\ruby[g]{{\換字{嫌}}味}{いやみ }を
\ruby{言}{い}ひやがる!。
%
よ\換字{志}にしろ、
%
\ruby{男}{をとこ}が
\ruby{下}{さが}るぜ、
%
\ruby{下}{くだ}らねえ。
』

\原本頁{171-1}%
どつと
\ruby{吹}{ふ}く
\ruby{風}{かぜ}の
\ruby{音}{おと}、
%
ひゆーと
\ruby{鳴}{な}る
\ruby{物}{もの}の
\ruby[g]{叫聲}{さけび }、
%
\ruby[g]{二人}{ふたり }が
\ruby{居}{を}れる
\ruby{此}{この}
\ruby{樓}{ろう}も、
%
ゆらりと
\ruby{今}{いま}は
\ruby{一}{ひ}ト
\ruby{搖}{ゆら}ぎして、
%
\ruby[g]{一切}{いつさい}の
\ruby{物}{もの}
\ruby{皆}{みな}
\ruby{震}{ふる}ひ
\ruby{動}{うご}けば、
%
ものこそ
\ruby{言}{い}はね
\ruby[g]{伊東}{い とう}が
\ruby{眉}{まゆ}は
ぴりゝと
\ruby{縮}{ちゞ}みて、
%
\ruby{心}{こゝろ}の
\ruby{安}{やす}からぬを
\ruby{現}{あら}はしたり。

\原本頁{171-5}%
『
\ruby[g]{何樣}{ど う }した?% \inhibitglue{}% ここは「空き」があるので
\,% 原本上でのアキを再現するため「3/18 em」空ける
\ruby[g]{梅幸}{ばいかう}!。
%
\ruby{氣}{き}が
\ruby{揉}{も}めるか?。
』

\原本頁{171-6}%
\ruby{{\換字{前}}}{まへ}の
\ruby[g]{{\換字{返}}報}{しかへし}と
\ruby[g]{島木}{しまき }が
\ruby[g]{調戲}{からか }へば、
%
\ruby{此}{これ}も
\ruby[g]{男兒}{をとこ }なり、
%
\ruby[||j>]{癇}{かん}
\ruby[||j>]{癪}{しやく}らしく
% \ruby{癇癪}{かん|しやく}らしく
\ruby{{\換字{煙}}}{けむり}を
\ruby{吐}{は}きて、

\原本頁{171-8}%
『
\ruby{高}{たか}が
\ruby[g]{此樣}{こ ん }な
\ruby{無雨之風}{か|ら|つ|かぜ}!。
%
\ruby{何}{なに}が
\ruby{怖}{こは}い。
』

\原本頁{171-9}%
と、
%
\ruby{只}{たゞ}
\ruby[g]{一言}{ひとこと}に
\ruby{云}{い}ひ
\ruby{{\換字{消}}}{け}しつ、
%
\ruby{{\換字{強}}}{しひ}て
\ruby{笑}{わら}つて、

\原本頁{171-10}%
『
\換字{志}かし
\ruby[g]{中々}{なか〳〵}
\ruby{吹}{ふ}きやがるナ。
%
\ruby{汝}{おめへ}こそ
\ruby[g]{内々}{ない〳〵}
\ruby{嬉}{うれ}しからう!。
%
\ruby{曲}{まが}り
\ruby{屋}{や}さんが
\ruby[g]{立直}{たちなほ}つて
\ruby{來}{き}さうだぜ。
』

\原本頁{172-1}%
と、
%
\ruby{云}{い}ひ
\ruby{足}{た}したり。

\原本頁{172-2}%
『
\ruby[g]{左樣}{さ う }さ、
%
いつまで
\ruby{曲}{まが}り
つゞけで
\ruby{堪}{たま}るもんか。
%
\ruby{汝}{おめへ}
ばかりに
\ruby{當}{あた}つて
\ruby{居}{ゐ}られる
\ruby[g]{世界}{せ かい}ぢやあ
\ruby{無}{ね}え。
%
たまにやあ
\ruby[g]{此樣}{こ ん }な
\ruby{風}{かぜ}も
\ruby{吹}{ふ}いて
\ruby{吳}{く}れなくつちやあ!。
』

\原本頁{172-5}%
『
\ruby[||j>]{憫}{かは}
\ruby[||j>]{然}{いさう}に、% 「憫然 か(は)いさう」
% \ruby{憫然}{かは|いさう}に、% 「憫然 か(は)いさう」
%
\ruby{{\換字{空}}}{そら}を
\ruby{見}{み}ちやあ
\ruby[<j||]{百}{ひやく}
\ruby[||j>]{姓}{しやう}
なんぞが
\ruby[g]{何程}{いくら }
\ruby{泣}{な}いてるか
\ruby{知}{し}れや
\ruby{仕}{し}ない!。
%
\ruby{汝}{おめへ}は
もつと
\ruby{吹}{ふ}け
\ruby{位}{ぐらゐ}に
\ruby{思}{おも}つて
\ruby{居}{ゐ}るだらうが。
』

\原本頁{172-7}%
『
\ruby[||j>]{當}{あたり}
\ruby[||j>]{然}{ まへ}よ。
% \ruby{當然}{あたり|まへ}よ。
%
\ruby{吹}{ふ}いて
\ruby{吹}{ふ}いて
\ruby{吹}{ふ}き
\ruby{拔}{ぬ}けと
\ruby{思}{おも}つて
\ruby{居}{ゐ}るんだ!。
%
\ruby[<j||]{農}{ひやく}
\ruby[||j>]{夫}{しやう}が
\ruby{泣}{な}いたつて
\ruby{笑}{わら}つたつて
\ruby{構}{かま}ふもんか!。
%
\ruby[g]{早稻}{わ せ }も
\ruby[g]{晩稻}{おくて }も
\ruby{吹}{ふ}き
\ruby{飛}{と}んで
\原本頁{172-9}\改行%
\ruby[g]{仕舞}{し ま }へと
\ruby{思}{おも}つて
\ruby{居}{ゐ}るんだ。
』

\原本頁{172-10}%
『
いゝ
\ruby{蟲}{むし}だナア!。
%
\ruby{酷}{ひど}い
\ruby[g]{野郎}{や らう}だぞ!。
%
\ruby{他}{ひと}に
\ruby[<j||]{百}{ひやく}
\ruby[||j>]{兩}{りやう}の
\ruby{損}{そん}を
させても、
%
\ruby[g]{自己}{う ぬ }が
\ruby[||j>]{一}{いち}
\ruby[||j>]{兩}{りやう}
% \ruby{一兩}{いち|りやう}
\ruby{儲}{ まう}けりやあ% ルビ調整(長いルビ対策)「一兩(いちりよう)」のルビが一文字突き出てくる為
\ruby{好}{い}いといふ
\ruby[g]{料簡}{れうけん}
\ruby{方}{かだ}だ。% ルビ調整(原本通り)(か(だ))
』

\原本頁{173-1}%
『
ナンダ、
%
\ruby{惡}{わる}く
\ruby[g]{素人}{しろうと}くせえ
\ruby{事}{こと}を
\ruby{吐}{ぬ}かしやがる!。
%
\ruby{今}{いま}の
\ruby[g]{世界}{せ かい}で
\ruby{金}{かね}を
\ruby{儲}{まう}けて
\ruby[g]{大顏}{おほづら}を
\ruby{仕}{し}て
\ruby{居}{ゐ}る
\ruby{奴}{やつ}に、
%
\ruby{唯}{たゞ}の
\ruby[g]{一人}{ひとり }でも
\ruby{其}{そ}の
\ruby[g]{料簡}{れうけん}で
\ruby{無}{ね}え
\ruby{奴}{やつ}が
\ruby{有}{あ}るものかい!。
%
\ruby{大}{おほき}な
\ruby[||j>]{門}{もん}
\ruby[||j>]{構}{がまへ}を
% \ruby{門構}{もん|がまへ}を
\ruby{仕}{し}て
\ruby{居}{ゐ}る
\ruby{奴}{やつ}あ、
%
\ruby[g]{悉皆}{みんな }
いゝ
\ruby{蟲}{むし}に
\ruby{羽}{はね}が
\ruby{生}{は}へたのぢや
\ruby{無}{ね}えか!。
』

\原本頁{173-5}%
『
ハヽヽ、
%
\ruby[g]{{\換字{違}}無}{ちげへね}え!。
%
\ruby{言}{い}つて
\ruby{見}{み}りやあ
まあ
\ruby[g]{其樣}{そ ん }なもんだ。
%
\ruby{併}{{\換字{志}}か}し
\ruby{汝}{おめへ}は
\ruby[g]{{\換字{平}}常}{ふだん }から、
%
\ruby[||j>]{觀}{くわん}% 「觀音」の読みは原本通り「くわん(の)ん」
\ruby[||j>]{音}{ のん}
% \ruby{觀音}{くわん|のん}% 「觀音」の読みは原本通り「くわん(の)ん」
なんぞを
\ruby[g]{信心}{しん〴〵}して
\ruby{居}{ゐ}るが
\ruby{彼}{あり}やあ
\ruby{何}{なん}だ!、
%
\ruby[g]{矢張}{やつぱ }り
\ruby[||j>]{觀}{くわん}% 「觀音」の読みは原本通り「くわん(の)ん」
\ruby[||j>]{音}{ のん}
\ruby[||j>]{樣}{ さま}
% \ruby{觀音}{くわん|のん}% 「觀音」の読みは原本通り「くわん(の)ん」
を
\ruby[g]{取捉}{とつつか}めへても、
%
\ruby[g]{其樣}{そ ん }な
あこぎな
\ruby[g]{料簡}{れうけん}で
もつて、
%
\ruby[g]{金持}{かねもち}になるやうと
\ruby{祈}{いの}つて
\ruby{居}{ゐ}るのか?。
』

\原本頁{173-9}%
『
ムヽ、
%
\ruby{他}{ほか}に
\ruby{祈}{いの}らうことは
\ruby{無}{ね}えぢやあ
\ruby{無}{ね}えか!。
』

\原本頁{173-10}%
『
ぢやあ
\ruby{惡}{わる}い
\ruby[g]{暴風}{あ れ }も
\ruby{祈}{いの}りかね
\ruby{無}{ね}えが、
%
そんな
\ruby[g]{我欲}{が よく}の
\ruby{願}{ねがひ}を
\ruby{掛}{か}けたつて、
%
\ruby[||g>]{觀音}{くわんのん}% 「觀音」の読みは原本通り「くわん(の)ん」
% \ruby{觀音}{くわん|のん}% 「觀音」の読みは原本通り「くわん(の)ん」
は
\ruby[g]{正路}{しやうろ}の
\ruby{佛}{ほとけ}
ださうだぜ。
』

\原本頁{174-1}%
『
ナニ
\ruby[g]{乃公}{お ら }の
\ruby[||j>]{觀}{くわん}% 「觀音」の読みは原本通り「くわん(の)ん」
\ruby[||j>]{音}{ のん}
% \ruby{觀音}{くわん|のん}% 「觀音」の読みは原本通り「くわん(の)ん」
は
\ruby[g]{乃公}{お ら }の
\ruby[||j>]{觀}{くわん}% 「觀音」の読みは原本通り「くわん(の)ん」
\ruby[||j>]{音}{ のん}
% \ruby{觀音}{くわん|のん}% 「觀音」の読みは原本通り「くわん(の)ん」
だ!。
%
\ruby{汝}{おめへ}の
\ruby[||j>]{觀}{くわん}% 「觀音」の読みは原本通り「くわん(の)ん」
\ruby[||j>]{音}{ のん}
% \ruby{觀音}{くわん|のん}% 「觀音」の読みは原本通り「くわん(の)ん」
たあ
\ruby{異}{ちが}つたつて
\ruby{管}{かま}はねえ。
%
\ruby[g]{乃公}{お ら }あ
\ruby[g]{乃公}{お ら }で
\ruby{濟}{す}んでるんだから、
%
これで
\ruby{可}{い}いんだ。
』

\原本頁{174-3}%
『
\ruby{何}{なん}だか
\ruby[g]{{\換字{道}}理}{す ぢ }が
\ruby{{\換字{通}}}{とほ}らねえやうだが、
アツ、
%
また
\ruby{吹}{ふ}きやあがる、
%
\原本頁{174-4}\改行%
\ruby{甚}{ひど}くなつて
\ruby{來}{き}たぞ。
%
オヽ
\ruby{塀}{へい}が
\ruby{飛}{と}んだぞ、
%
\ruby[||j>]{棟}{むな}
\ruby[||j>]{{\換字{瓦}}}{がはら}が
% \ruby{棟{\換字{瓦}}}{むな|がはら}が
\ruby{落}{お}ちたぞ!。
』

\原本頁{174-5}%
『
どうだ
\ruby[g]{{\換字{情}}無}{なさけな}いか、
%
\ruby[g]{心配}{しんぱい}か!。
』

\原本頁{174-6}%
『
\ruby[g]{馬鹿}{ば か }あ
\ruby{云}{い}ふな、
%
\ruby[g]{篦棒}{べらぼう}ナ!。
%
\ruby[g]{天{\換字{運}}}{う ん }は
\ruby[g]{何樣}{ど う }
\ruby[g]{循環}{ま は }つたつて
\ruby[g]{手腕}{う で }は
\ruby[g]{手腕}{う で }だ!。
%
\ruby[||j>]{{\換字{逆}}}{むかひ}
\ruby[||j>]{風}{ かぜ}を
% \ruby{{\換字{逆}}風}{むかひ|かぜ}を
\ruby{乘}{の}つ
\ruby{切}{き}つて
\ruby[g]{腕{\換字{前}}}{うでまへ}を
\ruby{見}{み}せてやらあ。
%
\ruby[g]{此方}{こちら }あ% ルビ調整(原本通り)
\ruby[g]{昨夜}{ゆふべ }
\ruby{辨天樣}{べん|てん|さま}に、% 弁 瓣 辦 辧 (辨) 辩 辯
\換字{志}たゝか
お
\ruby[g]{賽錢}{さいせん}を
\ruby{献}{あ}げて
\ruby{來}{き}たんだ、
%
はゞかりながら
\ruby{辨天樣}{べん|てん|さま}が% 弁 瓣 辦 辧 (辨) 辩 辯
\ruby{付}{つい}て
\ruby{居}{ゐ}るんだ!。
』

\原本頁{174-10}%
\ruby{笑}{わら}ひながら
\ruby{云}{い}ひたる
\ruby{末}{すゑ}の
\ruby[g]{言葉}{ことば }は、
%
\ruby{暗}{あん}に
\ruby[g]{自己}{おのれ }が
\ruby[g]{昨夜}{きのふ }の
\ruby[g]{豪{\換字{遊}}}{がういう}を
\ruby{誇}{ほこ}つて、
%
\ruby[g]{{\換字{遊}}謔}{たはむれ}の
\ruby{中}{うち}にも
\ruby{威}{ゐ}を
\ruby{張}{は}りて、
%
\ruby{聊}{いさゝ}か
\ruby{自}{みづか}ら
\ruby{{\換字{強}}}{つよ}うせるなり。

\原本頁{175-1}%
『
\ruby{何}{なん}だ、
%
\ruby{薄}{うす}ら
\ruby[<j>]{腥}{なまぐさ}い
\ruby[g]{辨天}{べんてん}が% 弁 瓣 辦 辧 (辨) 辩 辯
\ruby{何}{なに}が
ありがたい\換字{!?}。
%
\ruby[g]{此方}{こつち }あ% ルビ調整(原本通り)
\ruby[||j>]{淸}{しやう}
\ruby[||j>]{淨}{ 〴〵}な
\ruby[g]{仙人}{せんにん}に
お
\ruby[g]{初穂}{はつほ }が
\ruby{献}{あ}げてあるんだ!。
%
\ruby[g]{今日}{け ふ }は
\ruby{此}{こ}の
\ruby[g]{乃公}{お ら }が
\ruby[g]{大當}{おほあた}りだ!。
』

\原本頁{175-3}%
これも
\ruby{私}{ひそか}に
\ruby{自}{みづか}ら
\ruby{快}{こゝろ}よし
とするところあるなり。

\原本頁{175-4}%
『
ナニ
\ruby{此}{こ}の
\ruby{鼻}{はな}が
\ruby[g]{矢張}{やつぱ }り
\ruby{當}{あた}る!。
』

\原本頁{175-5}%
『
ナニ
\ruby{此}{こ}の
\ruby{乃公樣}{お|れ|さま}が
\ruby[g]{屹度}{きつと }
\ruby{當}{あた}る!。
』

\原本頁{175-6}%
『
おれが、
』

\原本頁{175-7}%
『
おれが、
』

\原本頁{175-8}%
『
ハヽハヽ、
』

\原本頁{175-9}%
『
ハヽハヽ、
』

\原本頁{175-10}%
『
なにも
\ruby[g]{此處}{こ ゝ }で
\ruby[g]{喧嘩}{けんくわ}あ
\ruby{爲}{す}る
\ruby{事}{こと}も
\ruby{無}{ね}え。
』

\原本頁{175-11}%
『
\ruby[g]{二人}{ふたり }とも
\ruby{當}{あた}らう!。
』

\原本頁{176-1}%
『
\ruby[g]{夕方}{ゆふがた}までだ!。
』

\原本頁{176-2}%
\ruby{風}{かぜ}は
いよ〳〵
\ruby{狂}{くる}へる
\ruby{中}{なか}を、
%
\ruby[g]{二人}{ふたり }は
おのれおのれが% 原本でも通り字になっていないのでそのままとする
\ruby[g]{本陣}{ほんじん}へと、
%
\ruby[g]{勇威}{いきほひ}を
\ruby{含}{ふく}んで
\ruby[g]{立出}{たちい }でたり。
%
\ruby{風}{かぜ}も
\ruby{慾}{よく}に
\ruby{使}{つか}はるゝ
\ruby{人}{ひと}の
\ruby{世}{よ}の
\ruby{中}{なか}や。
