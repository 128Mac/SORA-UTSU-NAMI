\Entry{其二十九}

\原本頁{}
\ruby{其}{そ}の
\ruby{日}{ひ}
\ruby{晝}{ひる}を
\ruby{{\換字{過}}}{す}ぎて
\ruby{風}{かぜ}いよ〳〵
\ruby{烈}{はげ}しく、
%
\ruby{天}{そら}は
\ruby{塵埃}{ぢん|あい}に
\ruby{濁}{にご}れるが
\ruby{如}{ごと}くに
\ruby{一面}{いち|めん}の
\ruby{黄雲}{くわう|ゝん}に
\ruby{包}{つゝ}まれて、
%
\ruby{常}{つね}ならぬ
\ruby{{\換字{暖}}氣}{あたゝ|かさ}の
\ruby{氣味}{き|み}
\ruby{惡}{あし}ければ、
%
\ruby{人皆安}{ひと|みな|やす}き
\ruby{心}{こゝろ}も
\ruby{無}{な}くて、
%
\ruby{{\換字{若}}}{も}し
\ruby{此上}{この|うへ}に
\ruby{雨}{あめ}も
\ruby{混}{まじ}らばと
\ruby{氣{\換字{遣}}}{き|づか}ふ
\ruby{折}{をり}しも、
%
\ruby{頭上}{づ|じやう}の
\ruby{雲}{くも}やうやく
\ruby{墨色}{すみ|いろ}さして、
%
\ruby{蔽}{おほ}ひかぶさる
\ruby{樣}{よう}に
\ruby{昏}{くら}くなれば、
%
\ruby{如何}{い|か}になり
\ruby{行}{ゆ}く
\ruby{{\換字{魔}}日}{ま|び}ぞと
\ruby{誰}{たれ}しも
\ruby{恐}{おそ}れあひぬ。
%
\ruby{事}{こと}
\ruby{無}{な}くて
\ruby{家}{いへ}にある
\ruby{爺媼}{ぢゞ|ばゞ}さへ
\ruby{是}{かく}の
\ruby{如}{ごと}くなれば、
%
まして、
%
\ruby{{\換字{遣}}}{や}らん
\ruby{買}{か}はんの
\ruby{呼}{よ}び
\ruby{聲}{ごゑ}は
\ruby{戰場}{せん|じやう}の% 原文通り「場」
\ruby{矢叫}{や|さけ}びと
\ruby{入}{い}り
\ruby{亂}{みだ}れて、
%
\ruby{打振}{うち|ふ}る
\ruby{兩手}{りやう|て}は
\ruby{浪}{なみ}
\ruby{寄}{よ}る
\ruby{尾花}{を|ばな}と
\ruby{{\換字{空}}}{そら}に
\ruby{揉}{も}まるゝ
\ruby{其場}{その|ば}の% 原本通り「場」
\ruby{混亂}{こん|らん}は、
%
\ruby{猜}{すゐ}するにも
\ruby{{\換字{猶}}}{なほ}
\ruby{餘}{あまり}あり。

\原本頁{}
\ruby{伊東}{い|とう}はいづれへ
\ruby{逸}{そ}れしにや
\ruby{歸}{かへ}り
\ruby{來}{きた}らねど、
%
\ruby{雨}{あめ}
\ruby{下}{お}りんとして
\ruby{下}{お}りず
\ruby{風}{かぜ}
\ruby{衰}{おとろ}へぬ
\ruby{夕{\換字{近}}}{ゆうべ|ちか}く、
%
\ruby{島木}{しま|き}は
\ruby{悠然}{いう|ぜん}として
\ruby{歸}{かへ}り
\ruby{來}{きた}りぬ。
%
\ruby{島木}{しま|き}につゞきて
\ruby{上}{あが}り
\ruby{來}{きた}れる
\ruby{婢}{をんな}は、
%
\ruby{例}{れい}となり
\ruby{居}{を}れると
\ruby{見}{み}えて
\ruby{茶}{ちや}を
\ruby{入}{い}れて
\ruby{薦}{すゝ}めつ。

\原本頁{}
『
\ruby{伊東}{い|とう}さんは?、
%
\ruby{御存知}{ご|ぞん|ぢ}
\ruby{無}{な}くつて?。
』

\原本頁{}
『
\ruby{知}{し}らねえよ、
%
\ruby{一{\換字{所}}}{いつ|しよ}ぢやあ
\ruby{無}{ね}えから。
%
\換字{志}かしおほかた
\ruby{彼女}{あ|れ}のところだらう。
』

\原本頁{}
『ほんとに
\ruby{凝}{こ}つて
\ruby{行}{い}らつしやるのネ!。
%
\ruby{幸{\換字{運}}}{い|ゝ}につけても、
%
\ruby{惡{\換字{運}}}{わる|い}に
\ruby{付}{つ}けてもネエ!。
』

\原本頁{}
『ウン。
%
ハヽ、
%
\ruby{今日}{け|ふ}は
\ruby{幸{\換字{運}}}{い|ゝ}につけてもぢやあ
\ruby{無}{な}さゝうだ!。
%
でも
\ruby{彼女}{あ|れ}の
\ruby{方}{はう}でも
\ruby{招}{よ}ぶやうだから
\ruby{堪}{たま}らねえや。
%
\ruby{汝}{おめへ}も
\ruby{女}{をんな}の
\ruby{端}{はし}くれだ、
%
どうだ、
%
\ruby{些}{ちつと}あ
\ruby{妬}{や}けるかい?。
』

\原本頁{}
『
\ruby{何}{なん}ですつて、
%
\ruby{端}{はし}くれですつて?。
%
あんまり
\ruby{酷}{ひど}い
\ruby{事}{こと}ネ。
%
ようござんすよ、
%
たんと
\ruby{惡口}{わる|くち}を
\ruby{仰}{おつしや}いまし、
%
\ruby{告訴}{いゝ|つけ}て
\ruby{{\換字{遣}}}{や}るとこを
\ruby{知}{し}つてますから。
%
ア、
%
そりやあ
\ruby{左樣}{そ|う}と
\ruby{貴君}{あな|た}は
\ruby{今日}{け|ふ}は
\ruby{大當}{おほ|あた}りでしやう。
%
あなたも
\ruby{男兒}{をと|こ}の
\ruby{端}{はし}くれだ、
%
\ruby{些}{ちつと}あ
\ruby{氣{\換字{前}}}{き|まへ}を
\ruby{見}{み}せて
\ruby{御奢}{お|おご}んなさいな。
%
\ruby{風}{かぜ}の
\ruby{音}{おと}を
\ruby{聞}{き}いちやあ
\ruby{主{\換字{婦}}}{おか|み}さんと
\ruby{一日}{いち|にち}
\ruby{云}{い}ひ
\ruby{暮}{く}らして
\ruby{居}{ゐ}ましたよ。
』

\原本頁{}
『
\ruby{左樣}{さ|う}かい、
%
\ruby{其奴}{そ|いつ}あ
\ruby{頼}{たの}もしかつた!。
%
\ruby{奢}{おご}つて
\ruby{{\換字{遣}}}{や}らう。
』

\原本頁{}
『オヤ、
%
\ruby{其}{それ}あ
\ruby{早{\換字{速}}}{さつ|そく}に
\ruby{有}{あ}り
\ruby{難}{がた}う!。
%
さうして
\ruby{何}{なに}を
\ruby{奢}{おご}つて
\ruby{下}{くだ}さる?。
』

\原本頁{}
『
\ruby{生憎劇塲}{あひ|にく|しば|ゐ}は% 原文通り「塲」
\ruby{好}{い}いところが
\ruby{開}{あ}いて
\ruby{居}{ゐ}ねえナ。
』

\原本頁{}
『さうネエ。
』

\原本頁{}
『
\ruby{秋草}{あき|くさ}も
\ruby{今日}{け|ふ}の
\ruby{此}{こ}の
\ruby{風}{かぜ}ぢやあもう。
』

\原本頁{}
『さうネエ。
』

\原本頁{}
『
\ruby{矢張}{やつ|ぱ}り
\ruby{下卑}{げ|び}でも
\ruby{甘}{あま}い
\ruby{物}{もの}といふところで
\ruby{堪{\換字{忍}}}{かん|にん}して% 原文通り「堪忍」
\ruby{貰}{もら}はう。
』

\原本頁{}
『さうねエ。
%
それぢやあ、
%
あの、
%
\ruby{何}{なに}を?。
』

\原本頁{}
『
\ruby{今}{いま}
\ruby{川燒}{がは|やき}きの
\ruby{皮}{かは}の% 原本通り「皮 か(は)」
\ruby{厚}{あつ}い
\ruby{冷}{つめ}たいのでも。
%
ハヽハヽ。
』

\原本頁{}
『エヽ
\ruby{悔}{くや}しいヨ、
%
おぼえて
\ruby{居}{ゐ}らつしやい。
%
もう
\ruby{貴君}{あな|た}の
\ruby{云}{い}ふ
\ruby{事}{こと}は
\ruby{當}{あて}に
\ruby{仕}{し}やしない。
』

\原本頁{}
『オイ〳〵
\ruby{左樣}{さ|う}ぶり〳〵しちやあ
\ruby{困}{こま}る。
%
\ruby{頼}{たの}む
\ruby{事}{こと}があるんだ、
%
\ruby{大}{おほ}まじめだ。
』

\原本頁{}
『ヘイ〳〵、
%
\ruby{澤山}{たん|と}
お
\ruby{使}{つか}ひなさいまし!。
%
\ruby{何}{なん}の
\ruby{御用}{ご|よう}?。
』

\原本頁{}
『
\ruby{惡}{わる}く
\ruby{角}{かく}ばるナ、
%
\ruby{怒}{おこ}つちやあいけねえ。
%
\ruby{好}{い}いかエ、
%
\ruby{客}{きやく}が
\ruby{一人來}{ひと|り|く}る
\ruby{筈}{はず}に
\ruby{招}{よ}んであるんだ。
%
\ruby{汝}{おめへ}の
\ruby{見}{み}はからひで、
%
\ruby{例}{いつも}の
\ruby{家}{うち}へでも
\ruby{電話}{でん|わ}をかけて、
%
\ruby{手一杯}{て|いつ|ぱい}に
\ruby{御馳走}{ご|ち|そう}を
\ruby{仕}{し}て
\ruby{貰}{もら}ひてえのだ。
%
\ruby{他家}{わ|き}へ
\ruby{行}{い}くなあ
\ruby{不妙}{ま|づ}いのだから。
%
ヨ、
%
\ruby{頼}{たの}むよ。
%
\ruby{客}{きやく}が
\ruby{堅人}{かた|じん}で、
%
\ruby{話}{はなし}が
\ruby{堅}{かた}いと
\ruby{來}{き}て
\ruby{居}{ゐ}るんだから。
』

\原本頁{}
『ハア、
%
\ruby{左樣}{さ|う}。
%
ようござんす。
%
\ruby{御酒}{ご|しゆ}は?。
%
\ruby[g]{麦酒}{びーる}?。
%
\ruby[g]{葡萄酒}{いつもの}?。
%
さうして
\ruby{直}{ぢき}に
\ruby{御入來}{お|い|で}ですか。% 国会図書館では「おいで」、国文学研究資料館では「おい 」
』

\原本頁{}
『ウン、
%
もうそろ〳〵
\ruby{來}{く}る
\ruby{時{\換字{分}}}{じ|ぶん}だから
\ruby{急}{いそ}いでネ。
』

\原本頁{}
『あの
\ruby[g]{水野}{みづの}さんとか
\ruby{仰}{おつし}ある
\ruby{方}{かた}?。
』

\原本頁{}
『ソラ
\ruby{惚}{ほ}れてやがるもんだから
\ruby{兎角}{と|かく}
\ruby{名}{な}をいふ!。
%
お
\ruby{生憎樣}{あひ|にく|さま}!。
』

\原本頁{}
『
\ruby[g]{水野}{みづの}ぢやあ
\ruby{無}{ね}え、
%
\ruby{羽{\換字{勝}}}{は|がち}といふんだ。
%
\換字{志}かし
\ruby{色}{いろ}の
\ruby{白}{しろ}い、
%
\ruby{眼}{め}の
\ruby{優}{やさ}しい、
%
\ruby{滅法}{めつ|ぱふ}に
\ruby{好}{い}い
\ruby{男}{をとこ}だから、
%
\ruby{{\換字{又}}}{また}
\ruby{汝}{おめへ}は
\ruby{直}{すぐ}と
\ruby{惚}{ほ}れるだらう。
』

\原本頁{}
『
\ruby{他聞}{ひと|ぎゝ}の
\ruby{惡}{わる}い!。
%
よしても
\ruby{下}{くだ}さいよ。
%
\ruby{妾}{わたし}や
\ruby{男}{をとこ}の
\ruby{美}{い}いのに
\ruby{惚}{ほ}れるやうな
\ruby{耄碌}{まう|ろく}ぢやあ
\ruby{有}{あ}りませんよ。
%
ホヽホヽ。
』

\原本頁{}
『オヤ
\ruby{異}{おつ}なたんかを
\ruby{切}{き}りやあがる。
%
それぢやあ
\ruby{何樣}{ど|ん}な
\ruby{男}{をとこ}に
\ruby{惚}{ほ}れるんだ?。
』

\原本頁{}
『
\ruby{知}{し}れた
\ruby{事}{こと}でさアネ、
%
\ruby{明治}{めい|じ}ツ
\ruby{子}{こ}ですよ。
%
\ruby{成功者}{あた|り|や}さんばつかりに
\ruby{惚}{ほ}れるんですわネ。
』

\原本頁{}
『
\ruby{畜生}{ちき|しやう}ツ、
%
\ruby{甚}{ひど}く
\ruby{當世}{たう|せい}なことを
\ruby{吐}{ぬか}しやあがる。
%
\ruby{此奴}{こい|つ}は
\ruby{今}{いま}
\ruby{川焼}{がは|やき}の
\ruby{讐}{かたき}を
\ruby{打}{う}たれた。
%
ハヽハヽ。
』

\原本頁{}
『ホヽホヽ。
』

\原本頁{}
お
\ruby{作}{さく}の
\ruby{笑}{わら}つて
\ruby{樓}{にかい}を
\ruby{下}{お}りきつたる
\ruby{時}{とき}、
%
がらりと
\ruby{格子}{かう|し}の
\ruby{明}{あ}く
\ruby{音}{おと}して、
%
\ruby{頼}{たの}むといふ
\ruby{聲}{こゑ}の
\ruby{此家}{こ|ゝ}の
\ruby{客}{きやく}には
\ruby{似合}{に|あ}はしからず
\ruby{堅}{かた}く、
%
\ruby{洋服}{やう|ふく}
\ruby{姿}{すがた}のきりゝとしたる、
%
\ruby{日}{ひ}に
\ruby{焦}{や}けきつたる
\ruby{顏}{かほ}の
\ruby{恐}{おそ}ろしく
\ruby{赭}{あか}く、
%
\ruby{潮風}{しほ|かぜ}に
\ruby{晒}{さ}らされてか
\ruby{眼}{め}さへ
\ruby{赤色}{あか|いろ}を
\ruby{帶}{お}びたる
\ruby{鐵}{てつ}づくりの
\ruby{如}{ごと}き
\ruby{男}{をとこ}は
\ruby{入}{い}り
\ruby{來}{きた}りぬ。

\原本頁{}
お
\ruby{作}{さく}は
\ruby{受}{う}け
\ruby{取}{と}りたる
\ruby{名刺}{めい|し}の
\ruby{表}{おもて}に
\ruby{羽{\換字{勝}}}{は|がち}
\ruby{千{\換字{造}}}{せん|ざう}といふ
\ruby{四{\換字{文}}字}{よん|も|じ}の
\ruby{記}{しる}されたるを
\ruby{見}{み}ぬ。
