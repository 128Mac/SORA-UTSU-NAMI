\Entry{其二十九}

% メモ 校正終了 2024-04-10 2024-05-27 2024-06-19
\原本頁{176-5}%
\ruby{其}{そ}の
\ruby{日}{ひ}
\ruby{晝}{ひる}を
\ruby{{\換字{過}}}{す}ぎて
\ruby{風}{かぜ}
いよ〳〵
\ruby{烈}{はげ}しく、
%
\ruby{天}{そら}は
\ruby[g]{塵埃}{ぢんあい}に
\ruby{濁}{にご}れるが
\ruby{如}{ごと}くに
\ruby[g]{一面}{いちめん}の
\ruby[||j>]{黄}{くわう}
\ruby[||j>]{雲}{ ゝん}に
% \ruby{黄雲}{くわう|ゝん}に
\ruby{包}{つゝ}まれて、
%
\ruby{常}{つね}
ならぬ
\ruby[||j>]{{\換字{暖}}}{あた}
\ruby[||j>]{氣}{ゝかさ}の
% \ruby{{\換字{暖}}氣}{あた|ゝかさ}の
\ruby[g]{氣味}{き み }
\ruby{惡}{あし}ければ、
%
\ruby{人}{ひと}
\ruby{皆}{みな}
\原本頁{176-7}\改行%
\ruby{安}{やす}き
\ruby{心}{こゝろ}も
\ruby{無}{な}くて、
%
\ruby{{\換字{若}}}{も}し
\ruby{此}{この}
\ruby{上}{うへ}に
\ruby{雨}{あめ}も
\ruby{混}{まじ}らばと
\ruby[g]{氣{\換字{遣}}}{き づか}ふ
\ruby{折}{をり}しも、
%
\ruby[g]{頭上}{づじやう}の
\ruby{雲}{くも}
やうやく
\ruby[g]{墨色}{すみいろ}
さして、
%
\ruby{蔽}{おほ}ひかぶさる
\ruby{樣}{よう}に
\ruby{昏}{くら}くなれば、
%
\ruby[g]{如何}{い か }になり
\ruby{行}{ゆ}く
\ruby[g]{{\換字{魔}}日}{ま び }ぞと
\ruby{誰}{たれ}しも
\ruby{恐}{おそ}れあひぬ。
%
\ruby{事}{こと}
\ruby{無}{な}くて
\ruby{家}{いへ}にある
\ruby{爺}{ぢゞ}
\ruby{媼}{ばゞ}さへ
\ruby{是}{かく}の
\ruby{如}{ごと}くなれば、
%
まして、
%
\ruby{{\換字{遣}}}{や}らん
\ruby{買}{か}はんの
\ruby{呼}{よ}び
\ruby{聲}{ごゑ}は
\ruby[||j>]{戰}{せん}
\ruby[||j>]{場}{じやう}の% 原文通り「場」
% \ruby{戰場}{せん|じやう}の% 原文通り「場」
\原本頁{177-1}\改行%
\ruby[g]{矢叫}{や さけ}びと
\ruby{入}{い}り
\ruby{亂}{みだ}れて、
%
\ruby[g]{打振}{うちふ }る
\ruby[g]{兩手}{りやうて}は
\ruby{浪}{なみ}
\ruby{寄}{よ}る
\ruby[g]{尾花}{を ばな}と
\ruby{{\換字{空}}}{そら}に
\ruby{揉}{も}まるゝ
\原本頁{177-2}\改行%
\ruby{其}{その}
\ruby{場}{ば}の% 原本通り「場」
\ruby[g]{混亂}{こんらん}は、
%
\ruby{猜}{すゐ}するにも
\ruby{{\換字{猶}}}{なほ}
\ruby{餘}{あまり}あり。

\原本頁{177-3}%
\ruby[g]{伊東}{い とう}は
いづれへ
\ruby{逸}{そ}れしにや
\ruby{歸}{かへ}り
\ruby{來}{きた}らねど、
%
\ruby{雨}{あめ}
\ruby{下}{お}りんとして
\ruby{下}{お}り
\原本頁{177-4}\改行%
ず
\ruby{風}{かぜ}
\ruby[||j>]{衰}{おとろ}へんとして
\ruby{衰}{おとろ}へぬ
\ruby[||j>]{夕}{ゆうべ}
\ruby[||j>]{{\換字{近}}}{ ちか}く、
% \ruby{夕{\換字{近}}}{ゆうべ|ちか}く、
%
\ruby[g]{島木}{しまき }は
\ruby[g]{悠然}{いうぜん}として
\ruby{歸}{かへ}り
\ruby{來}{きた}りぬ
\改行% 校正作業の簡略化のため
。
%
\原本頁{177-5}\改行%
\ruby[g]{島木}{しまき }に
つゞきて
\ruby{上}{あが}り
\ruby{來}{きた}れる
\ruby{婢}{をんな}は、
%
\ruby{例}{れい}となり
\ruby{居}{を}れると
\ruby{見}{み}えて
\ruby{茶}{ちや}を
\原本頁{177-6}\改行%
\ruby{入}{い}れて
\ruby{薦}{すゝ}めつ。

\原本頁{177-7}%
『
\ruby[g]{伊東}{い とう}さんは?、
%
\ruby{御存知}{ご|ぞん|ぢ}
\ruby{無}{な}くつて?。
』

\原本頁{177-8}%
『
\ruby{知}{し}らねえよ、
%
\ruby[g]{一{\換字{所}}}{いつしよ}ぢやあ
\ruby{無}{ね}えから。
%
\換字{志}かし
おほかた
\ruby[g]{彼女}{あ れ }の
ところだらう。
』

\原本頁{177-10}%
『
ほんとに
\ruby{凝}{こ}つて
\ruby{行}{い}らつしやるのネ!。
%
\ruby[g]{幸{\換字{運}}}{い ゝ }に
つけても、
%
\ruby[g]{惡{\換字{運}}}{わるい }に
\ruby{付}{つ}けてもネエ!。
』

\原本頁{178-1}%
『
ウン。
%
ハヽ、
%
\ruby[g]{今日}{け ふ }は
\ruby[g]{幸{\換字{運}}}{い ゝ }に
つけてもぢやあ
\ruby{無}{な}さゝうだ!。
%
でも
\ruby[g]{彼女}{あ れ }の
\ruby{方}{はう}でも
\ruby{招}{よ}ぶやうだから
\ruby{堪}{たま}らねえや。
%
\ruby{汝}{おめへ}も
\ruby{女}{をんな}の
\ruby{端}{はし}くれだ
\改行% 校正作業の簡略化のため
、
%
\原本頁{178-3}\改行%
どうだ、
%
\ruby{些}{ちつと}あ
\ruby{妬}{や}けるかい?。
』

\原本頁{178-4}%
『
\ruby{何}{なん}ですつて、
%
\ruby{端}{はし}くれですつて?。
%
あんまり
\ruby{酷}{ひど}い
\ruby{事}{こと}ネ。
%
ようござんすよ、
%
たんと
\ruby[g]{惡口}{わるくち}を
\ruby[<j>]{仰}{おつしや}いまし、
%
\ruby[g]{告訴}{いつけ }て
\ruby{{\換字{遣}}}{や}るとこを
\ruby{知}{し}つてますから。
%
ア、
%
そりやあ
\ruby[g]{左樣}{そ う }と
\ruby[g]{貴君}{あなた }は
\ruby[g]{今日}{け ふ }は
\ruby[g]{大當}{おほあた}りでしやう。
%
あなたも
\ruby[g]{男兒}{をとこ }の
\ruby{端}{はし}くれだ、
%
\ruby{些}{ちつと}あ
\ruby[g]{氣{\換字{前}}}{き まへ}を
\ruby{見}{み}せて
\ruby[g]{御奢}{お おご}んなさいな。
%
\ruby{風}{かぜ}の
\ruby{音}{おと}を
\ruby{聞}{き}いちやあ
\ruby[g]{主{\換字{婦}}}{おかみ }さんと
\ruby[g]{一日}{いちにち}
\ruby{云}{い}ひ
\ruby{暮}{く}らして
\ruby{居}{ゐ}ましたよ。
』

\原本頁{178-9}%
『
\ruby[g]{左樣}{さ う }かい、
%
\ruby[g]{其奴}{そ いつ}あ
\ruby{頼}{たの}もしかつた!。
%
\ruby{奢}{おご}つて
\ruby{{\換字{遣}}}{や}らう。
』

\原本頁{178-10}%
『
オヤ、
%
\ruby{其}{それ}あ
\ruby[g]{早{\換字{速}}}{さつそく}に
まあ
\ruby{有}{あ}り
\ruby{{\換字{難}}}{がた}う!。
%
さうして
\ruby{何}{なに}を
\ruby{奢}{おご}つて
\ruby{下}{くだ}さ
\改行% 校正作業の簡略化のため
る?。
』

\原本頁{179-1}%
『
\ruby[g]{生憎}{あひにく}
\ruby[g]{劇塲}{しばゐ }は% 原文通り「塲」
\ruby{好}{い}いところが
\ruby{開}{あ}いて
\ruby{居}{ゐ}ねえナ。
』

\原本頁{179-2}%
『
さうネエ。
』

\原本頁{179-3}%
『
\ruby[g]{秋草}{あきくさ}も
\ruby[g]{今日}{け ふ }の
\ruby{此}{こ}の
\ruby{風}{かぜ}ぢやあもう。
』

\原本頁{179-4}%
『
さうネエ。
』

\原本頁{179-5}%
『
\ruby[g]{矢張}{やつぱ }り
\ruby[g]{下卑}{げ び }でも
\ruby{甘}{あま}い
\ruby{物}{もの}と
いふところで
\ruby[g]{堪{\換字{忍}}}{かんにん}して% 原文通り「堪忍」
\ruby{貰}{もら}はう。
』

\原本頁{179-6}%
『
さうねエ。
%
それぢやあ、
%
あの、
%
\ruby{何}{なに}を?。
』

\原本頁{179-7}%
『
\ruby{今川燒}{いま|がは|やき}の
\ruby{皮}{かは}の% 原本通り「皮 か(は)」
\ruby{厚}{あつ}い
\ruby{冷}{つめ}たいのでも。
%
ハヽハヽ。
』

\原本頁{179-8}%
『
エヽ
\ruby{悔}{くや}しいヨ、
%
おぼえて
\ruby{居}{ゐ}らつしやい。
%
もう
\ruby[g]{貴君}{あなた }の
\ruby{云}{い}ふ
\ruby{事}{こと}は
\ruby{當}{あて}に
\ruby{仕}{し}やしない。
』

\原本頁{179-10}%
『
オイ〳〵
\ruby[g]{左樣}{さ う }
ぷり〳〵しちやあ
\ruby{困}{こま}る。
%
\ruby{頼}{たの}む
\ruby{事}{こと}が
あるんだ、
%
\ruby{大}{おほ}まじめだ。
』

\原本頁{180-1}%
『
ヘイ〳〵、
%
\ruby[g]{澤山}{たんと }
お
\ruby{使}{つか}ひなさいまし!。
%
\ruby{何}{なん}の
\ruby[g]{御用}{ご よう}?。
』

\原本頁{180-2}%
『
\ruby{惡}{わる}く
\ruby{角}{かく}ばるナ、
%
\ruby{怒}{おこ}つちやあ
いけねえ。
%
\ruby{好}{い}いかエ、
%
\ruby{客}{きやく}が
\ruby[g]{一人}{ひとり }
\ruby{來}{く}る
\ruby{筈}{はず}に
\ruby{招}{よ}んで
あるんだ。
%
\ruby{汝}{おめへ}の
\ruby{見}{み}はからひで、
%
\ruby{例}{いつも}の
\ruby{家}{うち}へでも
\ruby[g]{電話}{でんわ }を
かけて、
%
\ruby{手一杯}{て|いつ|ぱい}に
\ruby{御馳走}{ご|ち|そう}を
\ruby{仕}{し}て
\ruby{貰}{もら}ひてえのだ。
%
\ruby[g]{他家}{わ き }へ
\ruby{行}{い}くなあ
\ruby[g]{不妙}{ま づ }いのだから。
%
ヨ、
%
\ruby{頼}{たの}むよ。
%
\ruby{客}{きやく}が
\ruby[g]{堅人}{かたじん}で、
%
\ruby{話}{はなし}が
\ruby{堅}{かた}いと
\ruby{來}{き}て
\原本頁{180-6}\改行%
\ruby{居}{ゐ}るんだから。
』

\原本頁{180-7}%
『
ハア、
%
\ruby[g]{左樣}{さ う }。
%
ようござんす。
%
\ruby[g]{御酒}{ご しゆ}は?。
%
\ruby[g]{麦酒}{びーる}?。
%
\ruby{葡萄酒}{いつ|も|の}?。
%
\原本頁{180-8}\改行%
さうして
\ruby{直}{ぢき}に
\ruby{御入來}{お|い|で}ですか。% 国会図書館では「おいで」、国文学研究資料館では「おい 」
』

\原本頁{180-9}%
『
ウン、
%
もう
そろ〳〵
\ruby{來}{く}る
\ruby[g]{時{\換字{分}}}{じ ぶん}だから
\ruby{急}{いそ}いでネ。
』

\原本頁{180-10}%
『
あの
\ruby[g]{水野}{みづの }さんとか
\ruby{仰}{おつし}ある
\ruby{方}{かた}?。
』

\原本頁{180-11}%
『
ソラ
\ruby{惚}{ほ}れて
やがるもんだから
\ruby[g]{兎角}{と かく}
\ruby{名}{な}を
いふ!。
%
お
\ruby{生憎樣}{あひ|にく|さま}!。
%
\ruby[g]{水野}{みづの }ぢやあ
\ruby{無}{ね}え、
%
\ruby[g]{羽{\換字{勝}}}{は がち}と
いふんだ。
%
\換字{志}かし
\ruby{色}{いろ}の
\ruby{白}{しろ}い、
%
\ruby{眼}{め}の
\ruby{優}{やさ}しい、
%
\ruby[g]{滅法}{めつぱふ}に
\ruby{好}{い}い
\ruby{男}{をとこ}だから、
%
\ruby{{\換字{又}}}{また}
\ruby[||j>]{汝}{おめへ}は
\ruby{直}{すぐ}と
\ruby{惚}{ほ}れるだらう。
』

\原本頁{181-3}%
『
\ruby[g]{他聞}{ひとぎゝ}の
\ruby{惡}{わる}い!。
%
よしても
\ruby{下}{くだ}さいよ。
%
\ruby{妾}{わたし}や
\ruby{男}{をとこ}の
\ruby{美}{い}いのに
\ruby{惚}{ほ}れるやうな
\ruby[g]{耄碌}{まうろく}ぢやあ
\ruby{有}{あ}りませんよ。
%
ホヽホヽ。
』

\原本頁{181-5}%
『
オヤ
\ruby{異}{おつ}な
たんかを
\ruby{切}{き}りやあがる。
%
それぢやあ
\ruby[g]{何樣}{ど ん }な
\ruby{男}{をとこ}に
\ruby{惚}{ほ}れるんだ?。
』

\原本頁{181-7}%
『
\ruby{知}{し}れた
\ruby{事}{こと}でさアネ、
%
\ruby[g]{明治}{めいぢ }ツ
\ruby{子}{こ}ですよ。
%
\ruby{成功者}{あた|り|や}さん
ばつかりに
\ruby{惚}{ほ}れるんですわネ。
』

\原本頁{181-9}%
『
\ruby[||j>]{畜}{ちき}
\ruby[||j>]{生}{しやう}ツ、
% \ruby{畜生}{ちき|しやう}ツ、
%
\ruby{甚}{ひど}く
\ruby[g]{當世}{たうせい}なことを
\ruby{吐}{ぬか}しやあがる。
%
\ruby[g]{此奴}{こいつ }は
\ruby{今川燒}{いま|がは|やき}の
\ruby[<j||]{讐}{かたき}を
\ruby{打}{う}たれた。
%
ハヽハヽ。
』

\原本頁{181-11}%
『
ホヽホヽ。
』

\原本頁{182-1}%
お
\ruby{作}{さく}の
\ruby{笑}{わら}つて
\ruby{樓}{にかい}を
\ruby{下}{お}りきつたる
\ruby{時}{とき}、
%
がらりと
\ruby[g]{格子}{かうし }の
\ruby{明}{あ}く
\ruby{音}{おと}して
\改行% 校正作業の簡略化のため
、
%
\原本頁{182-2}\改行%
\ruby{頼}{たの}む
といふ
\ruby{聲}{こゑ}の
\ruby[g]{此家}{こ ゝ }の
\ruby{客}{きやく}には
\ruby[g]{似合}{に あ }はしからず
\ruby{堅}{かた}く、
%
\ruby[g]{洋服}{やうふく}
\ruby{姿}{すがた}の
きりゝとしたる、
%
\ruby{日}{ひ}に
\ruby{焦}{や}けきつたる
\ruby{顏}{かほ}の
\ruby{恐}{おそ}ろしく
\ruby{赭}{あか}く、
%
\ruby[g]{潮風}{しほかぜ}に
\ruby{晒}{さ}らされてか
\ruby{眼}{め}さへ
\ruby[g]{赤色}{あかみ }を
\ruby{帶}{お}びたる
\ruby{鐵}{てつ}づくりの
\ruby{如}{ごと}き
\ruby{男}{をとこ}は
\ruby{入}{い}り
\ruby{來}{きた}り
\改行% 校正作業の簡略化のため
ぬ。

\原本頁{182-6}%
お
\ruby{作}{さく}は
\ruby{受}{う}け
\ruby{取}{と}りたる
\ruby[g]{名刺}{めいし }の
\ruby{表}{おもて}に
\ruby[g]{羽{\換字{勝}}}{は がち}
\ruby[g]{千{\換字{造}}}{せんざう}といふ
\ruby{四{\換字{文}}字}{よん|もん|じ}の% ルビ調整(原稿通り)
\ruby{記}{しる}されたるを
\ruby{見}{み}ぬ。
