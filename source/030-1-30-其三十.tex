\Entry{其三十}

\ruby{酒}{さけ}は
\ruby{舊友}{きう|いう}と
\ruby{{\GWI{hkcs_m98f2}}}{の}むより
\ruby{甘}{うま}きは
\ruby{無}{な}く、
\ruby{談}{だん}は
\ruby{{\換字{半}}醉}{はん|すゐ}の
\ruby{時}{とき}より
\ruby{熱}{ねつ}するは
\ruby{無}{な}し、
\ruby{雞黍}{けい|しよ}の
\ruby{設}{まう}け
\ruby{粗薄}{そ|はく}なりとも、
\ruby{膠漆}{かう|しつ}の
\ruby{{\換字{情}}}{じやう}の
\ruby{殷厚}{いん|こう}ならんには、
\ruby{杯}{さかづき}を
\ruby{手}{て}にして
\ruby{相見}{あい|み}て
\ruby{笑}{わら}ふ
\ruby{一眄}{いち|べん}の
\ruby{中}{うち}にも
\ruby{限無}{かぎり|な}き
\ruby{味}{あぢはひ}は
\ruby{有}{あ}るべきを、ましてこれは
\ruby{范張陳雷}{はん|ちやう|ちん|らい}の
\ruby{語}{かた}らひのみならで、
\ruby{野心}{や|しん}に
\ruby{燃}{も}ゆる
\ruby{若}{わか}き
\ruby{男}{をとこ}の、
\ruby{志}{こヽろざし}は
\ruby{各々異}{おの|〳〵|こと}なれども
\ruby{事}{こと}を
\ruby{一}{いつ}にして
\ruby{功}{こう}を
\ruby{擧}{あ}げんとする
\ruby{相談}{さう|だん}に、
\ruby{意氣}{い|き}は
\ruby{齊}{ひと}しく
\ruby{昻}{あが}りて
\ruby{興}{きよう}は
\ruby{湧}{わ}くが
\ruby{如}{ごと}し。

\ruby{亭主八杯}{てい|しゆ|はち|はい}の
\ruby{諺}{ことわざ}に
\ruby{洩}{も}れず、
\ruby[g]{{\GWI{u7fbd-k}\換字{勝}}}{はがち}より
\ruby{先}{ま}ず
\ruby{島木}{しま|き}は
\ruby{醉}{よ}ひて、
\ruby{其}{そ}の
\ruby{肥}{ふと}つたる
\ruby[g]{身體}{からだ}を
\ruby{柱}{はしら}に
\ruby{靠}{もた}せながら、
\ruby{腫}{は}れたるが
\ruby{如}{ごと}き
\ruby{顏}{かほ}に
\ruby{笑}{ゑみ}を
\ruby{{\GWI{u6d6e-k}}}{うか}めつゝ、

『
\ruby{兎}{と}も
\ruby{角}{かく}も
\ruby{其}{それ}ぢやあ
\ruby[g]{一萬二千圓}{いちまんにせんゑん}だけは
\ruby{君}{きみ}の
\ruby{權利}{けん|り}の
\ruby{内}{うち}に
\ruby{置}{お}くと
\ruby{決}{き}めた。
\ruby{{\GWI{u8239-k}}}{ふね}も
\ruby{借}{か}りるなら
\ruby{借}{か}りるが
\ruby{好}{い}い、
\ruby{買}{か}ふならばまた
\ruby{買}{か}ふが
\ruby{好}{い}い。
\ruby{一切君}{いつ|さい|きみ}の
\ruby[g]{考次第}{かんがへしだい}に
\ruby{任}{まか}せる。
\ruby{一艘}{いつ|ぱい}
\ruby{仕立}{し|た}てるとも
\ruby{二艘三艘}{に|はい|さん|ばい}
\ruby{仕立}{し|た}てるとも、それも
\ruby{君次第}{きみ|し|だい}で
\ruby{論}{ろん}は
\ruby{無}{な}い。
\ruby{乃公}{お|ら}あ
\ruby{素人}{しろ|うと}だ、
\ruby{君}{きみ}は
\ruby{黑人}{くろ|うと}だ。
\ruby{乃公}{お|ら}あ
\ruby{何}{なに}も
\ruby{彼}{か}も
\ruby{分}{わか}らないんだ。
おらあたゞ
\ruby{焔{\GWI{u785d-k}}}{えん|せう}と
\ruby{彈丸}{た|ま}とを
\ruby{出}{だ}すんだ。
\ruby{狙}{ねら}つて
\ruby{撃}{う}つて
\ruby{鳥}{とり}を
\ruby{穫}{と}るなあ
\ruby{君}{きみ}の
\ruby{手腕一}{う|で|いつ}ぱいに
\ruby{仕}{し}て
\ruby{貰}{もら}うんだ。
\ruby{後}{うしろ}から
\ruby{臂}{ひぢ}に
\ruby{觸}{さは}るやうな
\ruby{野暮}{や|ぼ}は
\ruby{仕}{し}ねえ。
\ruby{乃公}{お|ら}あ
\ruby{資金}{か|ね}を
\ruby{出}{だ}す、
\ruby{君}{きみ}は
\ruby{手腕}{う|で}を
\ruby{貸}{か}す。
\ruby[g]{利{\GWI{u76ca-k}}}{まうけ}は
\ruby{笑}{わら}つて
\ruby{山分}{やま|わけ}に
\ruby{仕}{し}やうが、
\ruby{損}{そん}は
\ruby{泣言}{なき|ごと}を
\ruby{云}{い}ひつこ
\ruby{無}{な}しで、
\ruby{氣持好}{き|もち|よ}く
\ruby{骰子}{さ|い}を
\ruby{轉}{ころ}がして
\ruby{見}{み}やうと
\ruby{云}{い}うんだ。
\GWI{koseki-900370}かし
\ruby{僕}{ぼく}も
\ruby[g]{商人}{あきんど}だ、
\ruby{算盤}{そろ|ばん}だけは
\ruby{合點}{がつ|てん}の
\ruby{行}{ゆ}く
\ruby{男}{をとこ}だから、
\ruby{大}{おほ}づもりのところだけは
\ruby[g]{都度々々}{つど〳〵}
\ruby{聞}{き}きたい。
\ruby{其他}{その|ほか}にやあ
\ruby{何}{なに}も
\ruby{注文}{ちゆう|もん}は
\ruby{無}{な}いんだ。
\ruby{全}{まつた}く
\ruby{君}{きみ}の
\ruby{料簡次第}{れう|けん|し|だい}だ。
なあに
\ruby{一}{ぴん}と
\ruby{出}{で}やうと
\ruby{六}{ろく}と
\ruby{出}{で}やうと
\ruby{口惜}{くや|し}かあ
\ruby{無}{ね}え、
\ruby{事業}{し|ごと}の
\ruby{巧}{うま}く
\ruby{行}{い}くのと
\ruby{行}{い}かないのは、
\ruby{{\換字{半}}分}{はん|ぶん}は
\ruby{手腕}{う|で}で
\ruby{{\換字{半}}分}{はん|ぶん}は
\ruby{耳朶}{みゝつ|たぶ}だ!。
\ruby[g]{{\GWI{u9063-k}}付}{やつつ}けるだけ
\ruby[g]{{\GWI{u9063-k}}付}{やつつ}けて
\ruby{貰}{もら}やあ、
\ruby{何様}{ど|う}なつたつて
\ruby{驚}{おどろ}かねえんだから、
\ruby{斟酌無}{しん|しやく|な}く
\ruby{存分}{ぞん|ぶん}に
\ruby[g]{{\GWI{u9063-k}}}{や}つて
\ruby{{\換字{呉}}}{く}れたまへ。
\ruby{今}{いま}も
\ruby{話}{はな}した
\ruby{{\GWI{u901a-k}}}{とほ}り
\ruby{此}{こ}の
\ruby{風}{かぜ}が
\ruby{出無}{で|な}かつたら、
\ruby{擴}{ひろ}げられるだけ
\ruby{戰線}{せん|〳〵}を
\ruby{擴}{ひろ}げて
\ruby{置}{お}いた
\ruby{此}{こ}の
\ruby{萬五郎}{まん|ご|らう}は、
\ruby{今}{いま}ごろは
\ruby{何處}{ど|こ}へケシ
\ruby{飛}{と}んでるか
\ruby{分}{わか}からないんだが、
\ruby{其}{そ}の
\ruby{危}{あぶ}ない
\ruby{瀬}{せ}を
\ruby{渡}{わた}つて
\ruby{揉}{も}み
\ruby{合}{あ}つたゞけに、とう〳〵
\ruby{切}{き}り
\ruby{{\換字{勝}}}{か}つて
\ruby{一}{ひ}ト
\ruby{伸}{のし}
\ruby{伸}{の}して、
\ruby{如是}{こ|う}した
\ruby{話}{はなし}も
\ruby{出來}{で|き}るんだもの!。
お
\ruby{互}{たがひ}に
\ruby{度胸}{ど|きよう}と
\ruby{腕}{うで}とに
\ruby{掛}{か}けて
\ruby{敗}{ひけ}を
\ruby{取}{と}ら
\ruby{無}{な}きやあ、
\ruby{少}{すこ}し
\ruby{{\GWI{u904b-k}}}{うん}さへ
\ruby{添}{そ}やあ
\ruby{{\GWI{u9020-k}}作}{ざう|さ}は
\ruby{無}{ね}え。
\ruby{三井}{みつ|ゐ}や
\ruby{岩崎}{いは|さき}を
\ruby{尻目}{しり|め}に
\ruby{見}{み}て、
\ruby{笑}{わら}つて
\ruby{一杯{\GWI{hkcs_m98f2}}}{いつ|ぱい|の}ま
\ruby{無}{な}くつちやあ!。
\ruby{米}{こめ}や
\ruby{株}{かぶ}ばかり
\ruby{打}{たヽ}いて
\ruby{居}{ゐ}るのも
\ruby{智慧}{ち|ゑ}が
\ruby{足}{た}り
\ruby{無}{ね}えから、
\ruby{乃公}{お|ら}あ
\ruby{大蛸}{おほ|だこ}になつて
\ruby{八方}{はつ|ぽう}へ
\ruby{手}{て}を
\ruby{出}{だ}す!。
\ruby{五分}{ご|ぶ}や
\ruby{七分}{しち|ぶ}の
\ruby{口錢}{こう|せん}にヘイコラヘイコラと
\ruby{頭}{あたま}を
\ruby{下}{さ}げてこしらへた
\ruby{身上}{しん|じやう}ぢやあ
\ruby{無}{な}し、
\ruby{根}{ね}が
\ruby{泡沫錢}{あぶ|く|ぜに}だもの、
\ruby{{\GWI{u6d88-k}}}{き}えたつて
\ruby{未練}{み|れん}は
\ruby{無}{ね}えが、
\ruby{何}{なに}か
\ruby{知}{し}ら
\ruby[g]{那方}{どつち}かの
\ruby{手}{て}で
\ruby{攫}{つか}むつもりだ。
\ruby{思}{おも}ひ
\ruby{出}{だ}しやあソレ
\ruby{四五年前}{し|ご|ねん|まへ}の
\ruby{事}{こと}だつけ、
\ruby{七人揃}{しち|にん|そろ}つた
\ruby{其時}{その|とき}に、おれが
\ruby{例}{いつも}の
\ruby{法螺話}{ほ|ら|ばなし}の
\ruby{末}{すゑ}、お
\ruby{互}{たがい}に
\ruby{那}{ど}の
\ruby{路}{みち}にせよ
\ruby{世}{よ}を
\ruby{渡}{わた}るにやあ、
\ruby{跣足}{は|だし}ぢやあ
\ruby{歩}{ある}けねえ、
\ruby{草鞋}{わら|ぢ}が
\ruby{要}{い}る。
おれが
\ruby{一番巧}{いち|ばん|うま}く
\ruby{當}{あた}りやあ、
\ruby[g]{一同}{みんな}に
\ruby{一萬兩}{いち|まん|りやう}づゝの
\ruby{草鞋}{わら|ぢ}を
\ruby{穿}{は}かせて、
\ruby{世}{よ}の
\ruby{石高路}{いし|だか|みち}を
\ruby{歩}{ある}かせて
\ruby{{\GWI{u9063-k}}}{や}ると
\ruby{云}{い}つたら、
\ruby{馬鹿}{ば|か}に
\ruby{誰}{だれ}も
\ruby{彼}{かれ}も
\ruby{怒}{おこ}りやあがつて、あの
\ruby[g]{溫和}{おとな}しい
\ruby[g]{水野}{みづの}までが、
\ruby{僕}{ぼく}は
\ruby{踏}{ふ}み
\ruby{抜}{ぬ}きを
\ruby{仕}{し}たつて
\ruby{其様}{そ|ん}な
\ruby{草鞋}{わら|ぢ}は
\ruby{貰}{もら}はないと
\ruby{云}{い}ふし、
\ruby[g]{日方}{ひかた}はおらが
\ruby{背中}{せ|なか}を
\ruby{擲}{なぐ}りやがるし、
\ruby{楢井}{なら|い}や
\ruby{山瀬}{やま|せ}や
\ruby{名倉}{な|ぐら}までが、
\ruby{失敬}{しつ|けい}だ〳〵と
\ruby{腹}{はら}を
\ruby{立}{た}つたが、
\ruby{其時君}{その|とき|きみ}はたつた
\ruby{一人}{ひと|り}、なあに
\ruby{島木}{しま|き}が
\ruby{親切}{しん|せつ}で
\ruby{{\換字{呉}}}{く}れやうといふなら
\ruby{貰}{もら}ふが
\ruby{好}{い}いぢや
\ruby{無}{な}いか、
\ruby{氣}{き}が
\ruby{狭}{せま}い!、
\ruby{成程世}{なる|ほど|よ}を
\ruby{渡}{わた}るにやあ
\ruby{草鞋}{わら|ぢ}が
\ruby{要}{い}る、と
\ruby[g]{沈着}{おちつ}いて
\ruby{云}{い}つて
\ruby{{\換字{呉}}}{く}れた
\ruby{時}{とき}あ
\ruby{嬉}{うれ}しかつたよ。
それでと
\ruby{云}{い}ふ
\ruby{譯}{わけ}ぢやあ
\ruby{更}{さら}に
\ruby{無}{ね}えが、
\ruby{云}{い}はゞ
\ruby{其時云}{その|とき|い}つた
\ruby{其}{その}
\ruby{草鞋}{わら|ぢ}を、
\ruby{今日}{け|ふ}から
\ruby{君}{きみ}に
\ruby{穿}{は}いて
\ruby{貰}{もら}つて、
\ruby{君}{きみ}だけに
\ruby{歩}{ある}いて
\ruby{貰}{もら}ふやうになつたなあァ、
\ruby{嬉}{うれ}しい!。
サア
\ruby[g]{{\GWI{u7fbd-k}\換字{勝}}君}{はがちくん}!、これからだ。
ウンと
\ruby{大股}{おほ|また}に
\ruby{踏張}{ふん|ば}つてくれ!。
\ruby{君}{きみ}の
\ruby{腿骨}{すねつ|ぽね}の
\ruby{{\GWI{u9054-k}}者}{たつ|しや}なところと、
\ruby{男兒振}{をと|こ|ぶ}りの
\ruby{好}{い}いところを
\ruby{見}{み}せて
\ruby{{\換字{呉}}}{く}れたまへ。
ナア
\ruby[g]{{\GWI{u7fbd-k}\換字{勝}}君}{はがちくん}!。
』

と、これは
\ruby{{\GWI{u98fd-k}}}{あく}まで
\ruby{醉}{よひ}に
\ruby{乘}{じよう}じて
\ruby{碎}{くだ}けて
\ruby{云}{い}へど、
\ruby[g]{{\GWI{u7fbd-k}\換字{勝}}}{はがち}は
\ruby{醉}{よ}うて
\ruby{醉}{よ}はぬ
\ruby{姿勢}{し|せい}さへ
\ruby{正}{ただ}しく、
\ruby{堅固}{けん|ご}の
\ruby{言葉}{こと|ば}つき
\ruby{力{\換字{強}}}{ちから|づよ}く、

『ム。
\ruby{悉皆了解}{しつ|かい|れう|かい}した。
\ruby{確}{たしか}に
\ruby{承諾}{しよう|だく}した。
\ruby{面白}{おも|しろ}い。
\ruby{行}{や}れるだけは
\ruby{屹}{きつ}と
\ruby{行}{や}る
\ruby[g]{{\GWI{u7fbd-k}\換字{勝}}}{はがち}だ!。
\ruby{{\GWI{u904b-k}}}{うん}が
\ruby{逃}{に}げれば
\ruby{{\GWI{u904b-k}}}{うん}を
\ruby[g]{{\GWI{u8ffd-k}}尾}{おつか}ける!。
たとひ
\ruby{草鞋}{わら|ぢ}は
\ruby{穿}{は}き
\ruby{切}{き}つても、
\ruby{歩}{ある}きだしたら
\ruby{必}{かなら}ず
\ruby{歩}{ある}く。
\ruby{中{\GWI{u9014-k}}}{ちゆ|うと}では
\ruby{休}{やす}まぬ、
\ruby{{\GWI{u904b-k}}}{うん}は
\ruby{摑}{つか}む!。
\ruby{其代}{その|かは}り
\ruby{悉皆屹度任}{みん|な|きつ|と|まか}せて
\ruby{{\換字{呉}}}{く}れ。
』

と、
\ruby{云}{い}ひも
\ruby{{\GWI{u7d42-ue0101}}}{おは}らぬに
\ruby{島木}{しま|き}は
\ruby{烈}{はげ}しく、

『オヽ、
\ruby{任}{まか}せないで
\ruby{何}{なん}とするもんだ。
\ruby{屹度頼}{きつ|と|たの}んだぞ!。
』

と、
\ruby{口}{くち}を
\ruby{衝}{つ}いて
\ruby{答}{こた}へたり。

『ムッ、
\ruby{頼}{たの}まれたぞ。
』

『オヽ、
\ruby{頼}{たの}んだぞ。
』

『さあ
\ruby{始}{はじ}まつたぞ!。
』

『
\ruby{双六}{すご|ろく}が』

『ハヽハヽ。
』

『ハヽハヽ。
』

\ruby[g]{玻璃戔}{こつぷ}は
\ruby[g]{玻璃戔}{こつぷ}とカチリと
\ruby{觸}{あた}つて、
\ruby{酒}{さけ}は
\ruby{二人}{ふ|たり}に
\ruby{一時}{いちじ|}に
\ruby{仰}{あふ}がれたり。

