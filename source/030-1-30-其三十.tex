\Entry{其三十}

% メモ 校正終了 2024-04-10
\原本頁{182-9}%
\ruby{酒}{さけ}は
\ruby{舊友}{きう|いう}と
\ruby{飮}{の}むより
\ruby{甘}{うま}きは
\ruby{無}{な}く、
%
\ruby{談}{だん}は
\ruby{{\換字{半}}醉}{はん|すゐ}の
\ruby{時}{とき}より
\ruby{熱}{ねつ}するは
\ruby{無}{な}し、
%
\ruby{雞黍}{けい|しよ}の%
% 雞(ニワトリ)の肉を羹(あつもの)とし、
% 黍(きび)を炊いて飯とすること。
% 多く、客をもてなすことにいう語。饗応
\ruby{設}{まう}け
\ruby{粗薄}{そ|はく}なりとも、
%
\ruby{膠漆}{かう|しつ}の
\ruby{{\換字{情}}}{じやう}の
\ruby{殷厚}{いん|こう}ならんには、
%
\ruby[<j||]{杯}{さかづき}を% 行末行頭の境界付近なので特例処置を施す
\ruby{手}{て}にして
\ruby{相}{あひ}
\ruby{見}{み}て
\ruby{笑}{わら}ふ
\ruby{一眄}{いち|べん}の
\ruby{中}{うち}にも
\ruby{限}{かぎり}
\ruby{無}{な}き
\ruby[<j>]{味}{あぢはひ}は
\ruby{有}{あ}るべきを、
%
まして
これは
\ruby{范張陳雷}{はん|ちやう|ちん|らい}の
\ruby{語}{かた}らひ
のみならで、
%
\ruby{野心}{や|しん}に
\ruby{燃}{も}ゆる
\ruby{{\換字{若}}}{わか}き
\ruby{男}{をとこ}の、
%
\ruby[<j>]{志}{こゝろざし}は
\ruby{各々}{おの|〳〵}
\ruby{異}{こと}なれども
\ruby{事}{こと}を
\ruby{一}{いつ}にして
\ruby{功}{こう}を
\ruby{擧}{あ}げんとする
\ruby{相談}{さう|だん}に、
%
%\原本頁{183-4}\改行%
\ruby{意氣}{い|き}は
\ruby{齊}{ひと}しく
\ruby{昻}{あが}りて
\ruby{興}{きよう}は
\ruby{湧}{わ}くが
\ruby{如}{ごと}し。

\原本頁{183-5}%
\ruby{亭主八杯}{てい|しゆ|はち|はい}の
% 亭主八盃、客三盃
% 亭主 ていしゅ 八盃 はっぱい、客 きゃく 三盃 さんばい
% 酒席で、主人が客よりも多く酒を飲むこと。客をだしにして主人が酒を飲むこと。
\ruby[<j>]{諺}{ことわざ}に
\ruby{洩}{も}れず、
%
\ruby{羽{\換字{勝}}}{は|がち}より
\ruby{先}{ま}ず
\ruby{島木}{しま|き}は
\ruby{醉}{よ}ひて、% 「醉」は原本通り「よ」で調整
%
\ruby{其}{そ}の
\ruby{肥}{ふと}つたる
\ruby{身體}{から|だ}を
\ruby{柱}{はしら}に
\ruby{靠}{もた}せながら、
%
\ruby{腫}{は}れたるが
\ruby{如}{ごと}き
\ruby{顏}{かほ}に
\ruby{笑}{ゑみ}を
\ruby{{\換字{浮}}}{うか}めつゝ、

\原本頁{183-7}%
『
\ruby{兎}{と}も
\ruby{角}{かく}も
\ruby{其}{それ}ぢやあ
\ruby{一萬二千圓}{いち|まん|に|せん|ゑん}だけは
\ruby{君}{きみ}の
\ruby{權利}{けん|り}の
\ruby{内}{うち}に
\ruby{置}{お}くと
\ruby{決}{き}めた。
%
\ruby{{\換字{船}}}{ふね}も
\ruby{借}{か}りるなら
\ruby{借}{か}りるが
\ruby{好}{い}い、
%
\ruby{買}{か}ふならば
また
\ruby{買}{か}ふが
\ruby{好}{い}
い。
%
\ruby{一切}{いつ|さい}
\ruby{君}{きみ}の
\ruby{考次第}{かんがへ|し|だい}に
\ruby{任}{まか}せる。
%
\ruby{一艘}{いつ|ぱい}
\ruby{仕立}{し|た}てるとも
\ruby{二艘}{に|はい}
\ruby{三艘}{さん|ばい}
\ruby{仕立}{し|た}てるとも、
%
それも
\ruby{君}{きみ}
\ruby{次第}{し|だい}で
\ruby{論}{ろん}は
\ruby{無}{な}い。
%
\ruby{乃公}{お|ら}あ
\ruby{素人}{しろう|と}だ、
%
\ruby{君}{きみ}は
\ruby{黑人}{くろう|と}だ。
%
\ruby{乃公}{お|ら}あ
\ruby{何}{なに}も
\ruby{彼}{か}も
\ruby{{\換字{分}}}{わか}らないんだ。
%
おらあ
たゞ
\ruby{焔{\換字{硝}}}{えん|せう}と
\ruby{彈丸}{た|ま}とを
\原本頁{184-1}\改行%
\ruby{出}{だ}すんだ。
%
\ruby{狙}{ねら}つて
\ruby{撃}{う}つて
\ruby{鳥}{とり}を
\ruby{穫}{と}るなあ
\ruby{君}{きみ}の
\ruby{手腕}{う|で}
\ruby{一}{いつ}ぱいに
\ruby{仕}{し}て
\ruby{貰}{もら}ふんだ。
%
\ruby{後}{うしろ}から
\ruby{臂}{ひぢ}に
\ruby{觸}{さは}るやうな
\ruby{野暮}{や|ぼ}は
\ruby{仕}{し}ねえ。
%
\ruby{乃公}{お|ら}あ
\ruby{資金}{か|ね}を
\ruby{出}{だ}す、
%
\ruby{君}{きみ}は
\ruby{手腕}{う|で}を
\ruby{貸}{か}す。
%
\ruby{利益}{まう|け}は
\ruby{笑}{わら}つて
\ruby{山{\換字{分}}}{やま|わけ}に
\ruby{仕}{し}やうが、
%
\ruby{損}{そん}は
\ruby{泣言}{なき|ごと}を
\ruby{云}{い}ひつこ
\ruby{無}{な}しで、
%
\ruby{氣持}{き|もち}
\ruby{好}{よ}く
\ruby{骰子}{さ|い}を
\ruby{轉}{ころ}がして
\ruby{見}{み}やうと
\ruby{云}{い}ふんだ。
%
\原本頁{184-5}\改行%
\換字{志}かし
\ruby{僕}{ぼく}も
\ruby{商人}{あき|んど}だ、
%
\ruby{算盤}{そろ|ばん}だけは
\ruby{合點}{が|てん}の
\ruby{行}{ゆ}く
\ruby{男}{をとこ}だから、
%
\ruby{大}{おほ}づもりの
ところだけは
\ruby{都度}{つ|ど}
\ruby[g]{々々}{ 〳〵 }
\ruby{聞}{き}きたい。
%
\ruby{其}{その}
\ruby{他}{ほか}にやあ
\ruby{何}{なに}も
\ruby[||j>]{注}{ちゆう}
\ruby[||j>]{{\換字{文}}}{ もん}は
% \ruby{注{\換字{文}}}{ちゆう|もん}は
\ruby{無}{な}いんだ。
%
\ruby{全}{まつた}く
\ruby{君}{きみ}の
\ruby{料簡}{れう|けん}
\ruby{次第}{し|だい}だ。
%
なあに
\ruby{一}{ぴん}と
\ruby{出}{で}やうと
\ruby{六}{ろく}と
\ruby{出}{で}やうと
\原本頁{184-8}\改行%
\ruby{口惜}{く|やし}かあ
\ruby{無}{ね}え、
%
\ruby{事業}{し|ごと}の
\ruby{巧}{うま}く
\ruby{行}{い}くのと
\ruby{行}{い}かないのは、
%
\ruby{{\換字{半}}{\換字{分}}}{はん|ぶん}は
\ruby{手腕}{う|で}で
\ruby{{\換字{半}}{\換字{分}}}{はん|ぶん}は
\ruby[||j>]{耳}{みゝつ}
\ruby[||j>]{朶}{ たぶ}だ!。
% \ruby{耳朶}{みゝつ|たぶ}だ!。
%
\ruby{{\換字{遣}}付}{やつ|つ}けるだけ
\ruby{{\換字{遣}}付}{やつ|つ}けて
\ruby{貰}{もら}やあ、
%
\ruby{何樣}{ど|う}なつたつて
\ruby{驚}{おどろ}かねえんだから、
%
\ruby[||j>]{斟}{しん}
\ruby[||j>]{{\換字{酌}}}{しやく}
% \ruby{斟{\換字{酌}}}{しん|しやく}
\ruby{無}{な}く
\ruby{存{\換字{分}}}{ぞん|ぶん}に
\ruby{{\換字{遣}}}{や}つて
\ruby{吳}{く}れたまへ。
%
\ruby{今}{いま}も
\ruby{話}{はな}した
\ruby{{\換字{通}}}{とほ}り
\ruby{此}{こ}の
\ruby{風}{かぜ}が
\ruby{出無}{で|な}かつたら、
%
\ruby{擴}{ひろ}げられるだけ
\ruby{戰線}{せん|〳〵}を
\ruby{擴}{ひろ}げて
\ruby{置}{お}いた
\ruby{此}{こ}の
\ruby{萬五郎}{まん|ご|らう}は、
%
\ruby{今}{いま}ごろは
\ruby{何處}{ど|こ}へ
ケシ
\ruby{飛}{と}んでるか
\ruby{{\換字{分}}}{わか}らないんだが、
%
\ruby{其}{そ}の
\ruby{危}{あぶ}ない
\ruby{瀬}{せ}を
\ruby{渡}{わた}つて
\ruby{揉}{も}み
\ruby{合}{あ}つたゞけに、
%
とう〳〵
\原本頁{185-3}\改行%
\ruby{切}{き}り
\ruby{{\換字{勝}}}{か}つて
\ruby{一}{ひ}
ト
\ruby{伸}{のし}
\ruby{伸}{の}して、
%
\ruby{如是}{こ|う}した
\ruby{話}{はなし}も
\ruby{出來}{で|き}るんだもの!。
%
お
\ruby{互}{たがひ}に
\ruby{度胸}{ど|きよう}と
\ruby{腕}{うで}とに
\ruby{掛}{か}けて
\ruby{敗}{ひけ}を
\ruby{取}{と}ら
\ruby{無}{な}きやあ、
%
\ruby{少}{すこ}し
\ruby{{\換字{運}}}{うん}さへ
\ruby{添}{そ}やあ
\原本頁{185-5}\改行%
\ruby{{\換字{造}}作}{ざう|さ}は
\ruby{無}{ね}え。
%
\ruby{三井}{みつ|ゐ}や
\ruby{岩崎}{いは|さき}を% 原本のこの部分は「いわさき」
\ruby{尻目}{しり|め}に
\ruby{見}{み}て、
%
\ruby{笑}{わら}つて
\ruby{一杯}{いつ|ぱい}
\ruby{飮}{の}ま
\ruby{無}{な}くつちやあ!。
%
\ruby{米}{こめ}や
\ruby{株}{かぶ}ばかり
\ruby{打}{たゝ}いて
\ruby{居}{ゐ}るのも
\ruby{智慧}{ち|ゑ}が
\ruby{足}{た}り
\ruby{無}{ね}えから、
%
\原本頁{185-7}\改行%
\ruby{乃公}{お|ら}あ
\ruby{大蛸}{おほ|だこ}になつて
\ruby{八方}{はつ|ぽう}へ
\ruby{手}{て}を
\ruby{出}{だ}す!。
%
\ruby{五{\換字{分}}}{ご|ぶ}や
\ruby{七{\換字{分}}}{しち|ぶ}の
\ruby{口錢}{こう|せん}に
ヘイコラ
ヘイコラと
\ruby{頭}{あたま}を
\ruby{下}{さ}げて
こしらへた
\ruby[||j>]{身}{しん}
\ruby[||j>]{上}{しやう}ぢやあ
% \ruby{身上}{しん|しやう}ぢやあ
\ruby{無}{な}し、
%
\ruby{根}{ね}が
\原本頁{185-9}\改行%
\ruby{泡沫錢}{あぶ|く|ぜに}だもの、
%
\ruby{{\換字{消}}}{き}えたつて
\ruby{未練}{み|れん}は
\ruby{無}{ね}えが、
%
\ruby{何}{なに}か
\ruby{知}{し}ら
\ruby{那方}{どつ|ち}かの
\ruby{手}{て}で
\ruby{攫}{つか}むつもりだ。
%
\ruby{思}{おも}ひ
\ruby{出}{だ}しやあ
ソレ
\ruby{四五年}{し|ご|ねん}
\ruby{{\換字{前}}}{まへ}の
\ruby{事}{こと}だつけ、
%
\ruby{七人}{なな|にん}
\原本頁{185-11}\改行%
\ruby{揃}{そろ}つた
\ruby{其}{その}
\ruby{時}{とき}に、
%
おれが
\ruby{例}{いつも}の
\ruby{法螺話}{ほ|ら|ばなし}の
\ruby{末}{すゑ}、
%
お
\ruby{互}{たがひ}に
\ruby{那}{ど}の
\ruby{路}{みち}にせよ
\ruby{世}{よ}を
\ruby{渡}{わた}るにやあ、
%
\ruby{跣足}{はだ|し}ぢやあ
\ruby{歩}{ある}けねえ、
%
\ruby{草鞋}{わら|ぢ}が
\ruby{要}{い}る。
%
おれが
\ruby{一番}{いち|ばん}
\原本頁{186-2}\改行%
\ruby{巧}{うま}く
\ruby{當}{あた}りやあ、
%
\ruby{一同}{みん|な}に
\ruby{一萬兩}{いち|まん|りやう}づゝの
\ruby{草鞋}{わら|ぢ}を
\ruby{穿}{は}かせて、
%
\ruby{世}{よ}の
\ruby{石高路}{いし|だか|みち}を
\ruby{歩}{ある}かせて
\ruby{{\換字{遣}}}{や}ると
\ruby{云}{い}つたら、
%
\ruby{馬鹿}{ば|か}に
\ruby{誰}{だれ}も
\ruby{彼}{かれ}も
\ruby{怒}{おこ}りやあがつて、
%
\原本頁{186-4}\改行%
あの
\ruby{溫和}{おと|な}しい
\ruby{水野}{みづ|の}までが、
%
\ruby{僕}{ぼく}は
\ruby{踏}{ふ}み
\ruby{拔}{ぬ}きを
\ruby{仕}{し}たつて
\ruby{其樣}{そ|ん}な
\ruby{草鞋}{わら|ぢ}は
\ruby{貰}{もら}はないと
\ruby{云}{い}ふし、
%
\ruby{日方}{ひ|かた}は
おらが
\ruby{背中}{せ|なか}を
\ruby{擲}{なぐ}りやがるし、
%
\ruby{楢井}{なら|い}や
\ruby{山瀬}{やま|せ}や
\ruby{名倉}{なぐ|ら}までが、
%
\ruby{失敬}{しつ|けい}だ〳〵と
\ruby{腹}{はら}を
\ruby{立}{た}つたが、
%
\ruby{其}{その}
\ruby{時}{とき}
\ruby{君}{きみ}は
たつた
\ruby{一人}{ひと|り}、
%
なあに
\ruby{島木}{しま|き}が
\ruby{親切}{しん|せつ}で
\ruby{吳}{く}れやう
といふなら
\ruby{貰}{もら}ふが
\ruby{好}{い}いぢや
\ruby{無}{な}いか、
%
\ruby{氣}{き}が
\ruby{狭}{せま}い!、
%
\ruby{成程}{なる|ほど}
\ruby{世}{よ}を
\ruby{渡}{わた}るにやあ
\ruby{草鞋}{わら|ぢ}が
\ruby{要}{い}る、
%
と
\原本頁{186-9}\改行%
\ruby{沈着}{おち|つ}いて
\ruby{云}{い}つて
\ruby{吳}{く}れた
\ruby{時}{とき}あ
\ruby{嬉}{うれ}しかつたよ。
%
それでと
\ruby{云}{い}ふ
\ruby{譯}{わけ}ぢやあ
\ruby{{\換字{更}}}{さら}に
\ruby{無}{ね}えが、
%
\ruby{云}{い}はゞ
\ruby{其}{その}
\ruby{時}{とき}
\ruby{云}{い}つた
\ruby{其}{その}
\ruby{草鞋}{わら|ぢ}を、
%
\ruby{今日}{け|ふ}から
\ruby{君}{きみ}に
\ruby{穿}{は}いて
\ruby{貰}{もら}つて、
%
\ruby{君}{きみ}だけに
\ruby{歩}{ある}いて
\ruby{貰}{もら}ふやうに
なつたなあア、
%
\ruby{嬉}{うれ}しい!。
%
\原本頁{187-1}\改行%
サア
\ruby{羽{\換字{勝}}}{は|がち}
\ruby{君}{くん}!、
%
これからだ。
%
ウンと
\ruby{大跨}{おほ|また}に
\ruby{踏張}{ふん|ば}つてくれ!。
%
\ruby{君}{きみ}の
\ruby[||j>]{腿}{すねつ}
\ruby[||j>]{骨}{ ぽね}の
% \ruby{腿骨}{すねつ|ぽね}の
\ruby{{\換字{達}}者}{たつ|しや}な
ところと、
%
\ruby{男兒}{をと|こ}
\ruby{振}{ぶ}りの
\ruby{好}{い}い
ところを
\ruby{見}{み}せて
\ruby{吳}{く}れたまへ。
%
ナア
\ruby{羽{\換字{勝}}}{は|がち}
\ruby{君}{くん}!。
』

\原本頁{187-4}%
と、
%
これは
\ruby{{\換字{飽}}}{あく}まで
\ruby{醉}{よひ}に% 「醉」は原本通り「よ」で調整
\ruby{乘}{じよう}じて
\ruby{碎}{くだ}けて
\ruby{云}{い}へど、
%
\ruby{羽{\換字{勝}}}{は|がち}は
\ruby{醉}{よ}うて% 「醉」は原本通り「よ」で調整
\ruby{醉}{よ}はぬ% 「醉」は原本通り「よ」で調整
\ruby{姿勢}{し|せい}さへ
\ruby{正}{たゞ}しく、
%
\ruby{堅固}{けん|ご}の
\ruby{言葉}{こと|ば}つき
\ruby[||j>]{力}{ちから}
\ruby[||j>]{{\換字{強}}}{ づよ}く、
% \ruby{力{\換字{強}}}{ちから|づよ}く、

\原本頁{187-6}%
『ム。
%
\ruby{悉皆}{しつ|かい}
\ruby{了解}{れう|かい}した。
%
\ruby{確}{たしか}に
\ruby[||j>]{承}{しよう}
\ruby[||j>]{諾}{ だく}した。
% \ruby{承諾}{しよう|だく}した。
%
\ruby{面白}{おも|しろ}い。
%
\ruby{行}{や}れるだけは
\ruby{屹}{きつ}と
\ruby{行}{や}る
\ruby{羽{\換字{勝}}}{は|がち}だ!。
%
\ruby{{\換字{運}}}{うん}が
\ruby{{\換字{逃}}}{に}げれば
\ruby{{\換字{運}}}{うん}を
\ruby{{\換字{追}}尾}{おつ|か}ける!。
%
たとひ
\ruby{草鞋}{わら|ぢ}は
\原本頁{187-8}\改行%
\ruby{穿}{は}き
\ruby{切}{き}つても、
%
\ruby{歩}{ある}きだしたら
\ruby{必}{かなら}ず
\ruby{歩}{ある}く。
%
\ruby{中{\換字{途}}}{ちゆう|と}では
\ruby{休}{やす}まぬ、
%
\ruby{{\換字{運}}}{うん}は
\原本頁{187-9}\改行%
\ruby{摑}{つか}む!。
%
\ruby{其}{その}
\ruby{代}{かは}り
\ruby{悉皆}{みん|な}
\ruby{屹度}{きつ|と}
\ruby{任}{まか}せて
\ruby{吳}{く}れ。
』

\原本頁{187-10}%
と、
%
\ruby{云}{い}ひも
\ruby{{\換字{終}}}{おは}らぬに
\ruby{島木}{しま|き}は
\ruby{烈}{はげ}しく、

\原本頁{187-11}%
『オヽ、
%
\ruby{任}{まか}せないで
\ruby{何}{なん}とするもんだ。
%
\ruby{屹度}{きつ|と}
\ruby{頼}{たの}んだぞ!。
』

\原本頁{188-1}%
と、
%
\ruby{口}{くち}を
\ruby{衝}{つ}いて
\ruby{答}{こた}へたり。

\原本頁{188-2}%
『ムツ、
%
\ruby{頼}{たの}まれたぞ。
』

\原本頁{188-3}%
『オヽ、
%
\ruby{頼}{たの}んだぞ。
』

\原本頁{188-4}%
『さあ
\ruby{始}{はじ}まつたぞ!。
』

\原本頁{188-5}%
『
\ruby{双六}{すご|ろく}が!。
』

\原本頁{188-6}%
『ハヽハヽ。
』

\原本頁{188-7}%
『ハヽハヽ。
』

\原本頁{188-8}%
\ruby{玻璃盞}{こ|つ|ぷ}は
\ruby{玻璃盞}{こ|つ|ぷ}と
カチリと
\ruby{觸}{あた}つて、
%
\ruby{酒}{さけ}は
\ruby{二人}{ふた|り}に
\ruby{一時}{いち|じ}に
\ruby{仰}{あふ}がれたり。
