\Entry{其三十一}

% メモ 校正終了 2024-04-11 2024-05-27 2024-06-19
\原本頁{189-1}%
\ruby[g]{授業}{じゆげふ}も
\ruby{爲}{な}し
\ruby{{\換字{難}}}{がた}く
\ruby{見}{み}えたるほどの
\ruby[g]{暴風}{あ れ }の
\ruby[g]{一日}{いちにち}の
\ruby[<j||]{生}{なま }
\ruby[<j>]{{\換字{暖}}}{あたゝか}きに、
%
\ruby[g]{{\換字{平}}生}{いつも }の% ルビ調整(原本通り)
\ruby{如}{ごと}く
\ruby[g]{敎鞭}{けうべん}を
\ruby{執}{と}りて
\ruby[g]{太郎}{た らう}
\ruby[g]{次郎}{じ らう}を
\ruby[g]{相手}{あひて }に
\ruby{仕}{し}たりし
\ruby[g]{水野}{みづの }は、
%
\ruby{我}{が}の
\ruby{{\換字{強}}}{つよ}き
ところあれば
\ruby[g]{職務}{つとめ }を
\ruby{怠}{おこた}りこそは
\ruby{爲}{せ}ざりつれ、
%
\ruby[g]{{\換字{前}}日}{ぜんじつ}よりの
\ruby[g]{身心}{しん〴〵}の
\ruby{疲}{つか}れに、
%
\ruby[g]{五體}{ご たい}の
\ruby{綿}{わた}の
\ruby{如}{ごと}くなれるを
\ruby{我}{われ}と
\ruby{覺}{おぼ}えつゝ、
%
やうやく
\ruby[g]{午後}{ご ゞ }
\原本頁{189-6}\改行%
\ruby[g]{何時}{なんじ }の
\ruby{今}{いま}、
%
\ruby{始}{はじ}めて
\ruby{我}{わ}が
\ruby{身}{み}の
\ruby{我}{わ}が
\ruby{物}{もの}と
なりたる
\ruby[g]{心地}{こゝち }する
\ruby{氣}{き}の
\ruby{{\換字{緩}}}{ゆる}みに、
%
\ruby[||j>]{歩}{あるき}
\ruby[||j>]{調}{ つき}さへ
% \ruby{歩調}{あるき|つき}さへ
\ruby[g]{遲々}{ち ゝ }として、
%
\ruby[g]{脫力}{がつかり}して
\ruby{歸}{かへ}り
\ruby{來}{きた}りれり。

\原本頁{189-7}%
\ruby{手}{て}を
\ruby{掛}{か}けたるには
あらねど
\ruby{小}{ちひ}さき
\ruby{樹}{き}
\ruby{草}{くさ}など
\ruby{好}{よ}きほどに
\ruby{生}{は}えたれば
おのづからの
\ruby{庭}{には}と
なりたる
\ruby[g]{{\換字{空}}地}{あきち }を
\ruby{{\換字{前}}}{まへ}に、
%
\ruby{南}{みなみ}を
\ruby{受}{う}けたる
\ruby{長}{なが}き
\ruby[g]{一棟}{ひとむね}の、
%
\ruby{其}{そ}の
\ruby{奧}{おく}の
\ruby[g]{一間}{ひとま }は
\ruby{我}{わ}が
\ruby{起}{おき}
\ruby{臥}{ふし}の
ところと
\ruby{定}{さだ}まり
たるなり。
%
\ruby[g]{水野}{みづの }は
\ruby{常}{つね}の
\ruby{如}{ごと}く
\ruby[g]{庭先}{にはさき}を
\ruby{家}{いへ}に
\ruby{{\換字{沿}}}{そ}ひて
\ruby{{\換字{廻}}}{まは}りて、
%
\ruby{椽}{えん}より
\ruby{直}{たゞち}に
\ruby[g]{座敷}{ざ しき}に
\ruby{上}{あが}らんと
するに、
%
\ruby[g]{今日}{け ふ }は
\ruby{烈}{はげ}しき
\ruby{風}{かぜ}を
\ruby{厭}{いと}ひて、
%
\ruby[g]{雨{\換字{戸}}}{あまど }さへ
\ruby[g]{幾枚}{いくまい}か
\ruby{引}{ひ}かれ
\原本頁{190-1}\改行%
\ruby{居}{ゐ}たり。

\原本頁{190-2}%
『
ア、
%
\ruby{風}{かぜ}が
\ruby{甚}{ひど}いので
\ruby[g]{雨{\換字{戸}}}{あまど }を
\ruby{引}{ひ}いて
\ruby{置}{お}きました。
%
\ruby[||j>]{薄}{うすつ}
\ruby[||j>]{暗}{ くら}くつて
% \ruby{薄暗}{うすつ|くら}くつて
お
\ruby{{\換字{嫌}}}{いや}なら
\ruby{明}{あ}けて
あげませう。
%
\ruby[g]{方向}{む き }が
\ruby{好}{い}いので
\ruby[g]{此家}{こ ゝ }は
\ruby[g]{其程}{それほど}ぢやあ
\ruby{有}{あ}りませんが、
%
\ruby{何}{なに}にしろ
\ruby{甚}{ひど}い
\ruby{{\換字{嫌}}}{いや}な
\ruby{風}{かぜ}です。
』

\原本頁{190-5}%
\ruby{我}{わ}が
\ruby[g]{跫音}{あしおと}を
\ruby{聞}{きゝ}つけての
\ruby{吉右衛門}{きち||ゑ|もん}が
\ruby[g]{言葉}{ことば }に、

\原本頁{190-6}%
『
なあに
\ruby[g]{今日}{け ふ }は
\ruby{別}{べつ}に
\ruby[g]{細字}{こまかい}
\ruby{書}{ほん}を
\ruby{讀}{よ}まうとも
\ruby{思}{おも}はないから、
%
\ruby[g]{矢張}{やつぱ }り
\ruby[g]{此儘}{このまゝ}にして!。
』

\原本頁{190-8}%
と
\ruby{云}{い}ひながら
\ruby[g]{水野}{みづの }は
\ruby{身}{み}を
\ruby{側}{そば}めて、
%
\ruby{隙}{す}かして
\ruby{引}{ひ}かれたる
\ruby{{\換字{戸}}}{と}の
\ruby{間}{すき}より
\ruby{上}{あが}り、

\原本頁{190-10}%
『
ほんとに
\ruby[g]{氣持}{き もち}の
\ruby{惡}{わる}い、
%
\ruby{頭}{あたま}の
\ruby{痛}{いた}くなるやうな
\ruby{風}{かぜ}で、
‥‥
\ruby{早}{はや}く
\ruby{止}{や}んで
\ruby{吳}{く}れなくちやあ
\ruby[g]{仕方}{し かた}が
\ruby{無}{な}い。
』

\原本頁{191-1}%
と
\ruby[g]{座敷}{ざ しき}に
\ruby{入}{い}りつゝ
\ruby[g]{言葉}{ことば }を
\ruby{足}{た}せば、

\原本頁{191-2}%
『
\ruby[g]{左樣}{さ う }で
ございます。
%
\ruby{雨}{あめ}が
\ruby{隨}{つ}いて
\ruby{來}{こ}ないで
\ruby[g]{先々}{まあ〳〵}ですが、
%
\ruby[g]{土地}{ところ }に
よつちやあ
\ruby[g]{餘程}{よ ほど}の
\ruby[g]{損{\換字{害}}}{いたみ }です。
%
この
\ruby{{\換字{嫌}}}{いや}に
\ruby[<j>]{{\換字{暖}}}{あたゝか}い
\ruby{事}{こと}は
\ruby[g]{何樣}{ど う }でしやう。
%
% \原本頁{191-4}\改行%
\ruby[||j>]{病}{びやう}
\ruby[||j>]{人}{ にん}
% \ruby{病人}{びやう|にん}
なんぞにやあ
\ruby{{\換字{感}}}{き}きますネ。
%
オ、
%
\ruby[||j>]{病}{びやう}
\ruby[||j>]{人}{ にん}と
% \ruby{病人}{びやう|にん}と
\ruby{云}{い}やあ
\ruby[g]{今{\換字{朝}}}{け さ }
お
\ruby{頼}{たの}みの
\ruby{婢}{をんな}は、
%
\ruby{私}{わたし}の
\ruby[g]{本家}{う ち }の
\ruby{方}{はう}の
\ruby{小作人}{こ|さく|にん}の
\ruby{娘}{むすめ}で、
%
がせいに
\ruby{能}{よ}く
\ruby{働}{はたら}くのが
ありましたから、
%
\ruby{能}{よ}く
\ruby{云}{い}ひつけて
\ruby{其}{それ}を
\ruby{{\換字{遣}}}{や}つて
\ruby{置}{お}きました。
%
\ruby{看護{\換字{婦}}}{かん|ご|ふ}さんも
\ruby{來}{き}たさうです。
』

\原本頁{191-8}%
と、
%
\ruby{間}{あひ}の
\ruby{襖}{ふすま}は
\ruby{開}{ひら}き
\ruby{居}{ゐ}たる
\ruby{中}{なか}の
\ruby{間}{ま}に
ありて
\ruby[g]{敷居}{しきゐ }
\ruby{越}{ご}しの
\ruby[g]{挨拶}{あいさつ}なり。
%
\ruby[g]{水野}{みづの }は
\ruby[g]{床{\換字{近}}}{とこちか}く
\ruby{置}{お}きたる
\ruby{机}{つくゑ}の
\ruby{{\換字{前}}}{まへ}に
\ruby{坐}{すわ}りて、
%
\ruby{始}{はじ}めて
\ruby[g]{昨日}{きのふ }
\ruby[g]{以來}{い らい}の
\ruby[g]{疲勞}{つかれ }を
\ruby{息}{やす}めつゝ、

\原本頁{191-11}%
『
アヽ、
%
\ruby{今}{いま}
\ruby[g]{一寸}{ちよつと}
\ruby[g]{歸路}{かへり }に
\ruby[g]{立寄}{たちよ }つて
\ruby{來}{き}ました。
%
いろ〳〵
お
\ruby[g]{世話}{せ わ }を
\ruby{有}{あ}り
\ruby{{\換字{難}}}{がた}かつた。
%
\ruby{先}{まあ}
これで
\ruby[g]{一切}{いつさい}
\ruby{思}{おも}ふやうになつた。
』

\原本頁{191-2}%
と、
%
\ruby[g]{重荷}{おもに }を
\ruby{卸}{おろ}したるが
\ruby{如}{ごと}き
\ruby[g]{顏色}{かほつき}すれば、
%
\ruby{例}{れい}の
\ruby[g]{眼鏡}{め がね}の
\ruby{中}{うち}より
\ruby[g]{一寸}{ちよつと}
\ruby{見}{み}て、

\原本頁{192-4}%
『
\ruby[g]{昨夜}{ゆふべ }は
\ruby{碌}{ろく}に
\ruby{御睡眠}{お|よ|り}は
なさりますまいのに、
%
\ruby[g]{今日}{け ふ }は
\ruby{{\換字{又}}}{また}
\ruby[g]{{\換字{平}}生}{いつも }の% ルビ調整(原本通り)
\ruby{{\換字{通}}}{とほ}り
\ruby{御{\換字{勤}}務}{お|つ|とめ}では、
%
\ruby[g]{大抵}{たいてい}な
\ruby{御疲勞}{お|くた|びれ}では
ありますまい。
%
\ruby[g]{今夜}{こんや }は
まあ
\ruby{早}{はや}く
\ruby{御睡眠}{お|やす|み}なさいまし。
』

\原本頁{192-7}%
と、
%
\ruby{云}{い}ひさして
\ruby{茶}{ちや}の
\ruby{間}{ま}の
\ruby{方}{かた}を
\ruby{顧}{かへり}みて
\ruby{聲}{こゑ}
\ruby[||j>]{大}{おほき}く、

\原本頁{192-8}%
『
お
\ruby{濱}{はま}や。
%
また
\ruby[g]{其樣}{そ ん }なに
\ruby{書}{ほん}に
ばかり
\ruby[g]{取付}{とつつ }いて
\ruby{居}{ゐ}ちやあいけない。
%
\ruby[g]{先生}{せんせい}が
お
\ruby{歸}{かへ}り
なすつたぢやあ
\ruby{無}{な}いか、
%
\ruby[g]{御茶}{お ちや}を
\ruby{持}{も}つて
\ruby{來}{こ}ないか。
』

\原本頁{192-10}%
と、
%
\ruby[g]{悠然}{ゆつくり}と
したる
\ruby[g]{調子}{てうし }に
\ruby{呼}{よ}ばゝつたるは、
%
\ruby[g]{言葉}{ことば }つきなども
\ruby{異}{をか}しからぬほど
\ruby[g]{江{\換字{戸}}}{え ど }の
\ruby{水}{みづ}も
\ruby{飮}{の}んだる
\ruby{果}{はて}の
\ruby[g]{老夫}{おやぢ }なれど、
%
\ruby[g]{流石}{さすが }は
\ruby{根}{ね}が
\ruby{此}{こ}の
\ruby{邊}{あたり}の
\ruby[g]{田舎}{ゐ なか}% ルビ調整(原本通り)
\ruby{風}{ふう}なり。

\原本頁{193-2}%
\ruby{小}{ちひ}さき
\ruby[g]{刳{\換字{盆}}}{くりぼん}に
\ruby{大}{おほき}なる
\ruby{筒茶碗}{つゝ|ぢあ|わん}
\ruby{載}{の}せて、
%
\ruby[g]{嫣然}{につこり}と
\ruby{笑}{ゑ}みて
\ruby[g]{持出}{もちい }でたる
お
\ruby{濱}{はま}は、
%
\ruby[g]{水野}{みづの }が
\ruby[g]{膝{\換字{近}}}{ひざちか}く
それを
\ruby{置}{お}きて、
%
おのれは
\ruby[g]{祖{\換字{父}}}{ぢ ゞ }の
\ruby{傍}{かたへ}に
\ruby{甘}{あま}えるやうに
\ruby{坐}{すわ}り。

\原本頁{193-5}%
『
\ruby[g]{昨夜}{ゆふべ }は
\ruby{怖}{こは}かつた
でしようねえ、
%
\ruby[g]{眞闇}{まつくら}で!。
%
あれから
\ruby[||j>]{妾}{わたし}
\ruby[||j>]{床}{ とこ}へ
\ruby{入}{はい}つたら、
%
\ruby[g]{先生}{せんせい}の
\ruby{行}{いら}しつた
\ruby{方}{はう}の、
%
\ruby{{\換字{遠}}}{とほ}くの
\ruby{{\換字{遠}}}{とほ}くから、
%
\ruby{狗}{いぬ}の
\ruby{鳴}{な}く
\ruby{聲}{こゑ}が
\原本頁{193-7}\改行%
\ruby{聞}{きこ}えて
\ruby{來}{き}て、
%
\ruby{淋}{さび}しかつたわ!。
』

\原本頁{193-8}%
と
\ruby{云}{い}ひ
\ruby{出}{だ}せば、

\原本頁{193-9}%
『
ハヽヽ、
%
\ruby{何}{な}んだ
\ruby{下}{くだ}らない、
%
\ruby[g]{叩頭}{おじぎ }も
\ruby{仕}{し}ないで!。
%
\ruby[g]{突然}{いきなり}と
\ruby[g]{其樣}{そ ん }な
\ruby{事}{こと}を
\ruby{云}{い}ひ
\ruby{出}{だ}すよ。
%
\ruby{狗}{いぬ}が
\ruby{鳴}{な}いたつて
\ruby{何}{なに}
\ruby{淋}{さみ}しい
\ruby{奴}{やつ}が
あるもんか。
』

\原本頁{193-11}%
と
\ruby{笑}{わらひ}を
\ruby{帶}{お}びて
\ruby{吉右衛門}{きち||ゑ|もん}は
\ruby{叱}{しか}るを、
%
\ruby[g]{眞赤}{まつか }なる
\ruby[g]{番茶}{ばんちや}の
\ruby{味}{あぢ}も
\ruby{無}{な}く
\ruby{香}{か}も
\ruby{無}{な}けれど、
%
\ruby{熱}{あつ}き
のみに
\ruby{人}{ひと}の
\ruby{{\換字{情}}}{なさけ}は
\ruby{有}{あ}るを
\ruby{啜}{すゝ}れる
\ruby[g]{水野}{みづの }は、

\原本頁{194-2}%
『
ハヽヽ、
%
お
\ruby{濱}{はま}ちやんは
いつでも
\ruby[g]{面白}{おもしろ}い
\ruby{事}{こと}を
\ruby{云}{い}ふ!。
%
そして
\ruby[g]{昨夜}{ゆふべ }は
\makeatletter
\@ifundefined{デバッグ@ビルド}{%
  \ruby[<g||]{一生}{いつしやう}
  \ruby[g]{懸命}{けんめい}に
}{%
  \ruby[||j>]{一}{いつ }
  \ruby[||j>]{生}{しやう}
  \ruby[||j>]{懸}{ けん}
  \ruby[||j>]{命}{ めい}に
}%
\ruby{書}{ほん}を
\ruby{讀}{よ}んで
\ruby{居}{ゐ}たぢや
\ruby{無}{な}いか、
%
あれは
\ruby[g]{一體}{いつたい}
\ruby{何}{なん}の
\ruby{本}{ほん}
\原本頁{194-4}\改行%
だえ。
』

\原本頁{194-5}%
と
\ruby{問}{と}ふに、
%
お
\ruby{濱}{はま}は
\ruby{忽}{たちま}ち
\ruby[g]{不足}{ふ そく}らしき
\ruby{恨}{うら}みを
\ruby{其}{その}
\ruby{色}{いろ}に
\ruby{現}{あら}はしたり。

\原本頁{194-6}%
『
だつて
\ruby{先}{せん}の
\ruby{中}{うち}は
\ruby[g]{毎晩}{まいばん}
\ruby[g]{々々}{ 〳〵 }
いろんな
\ruby[g]{面白}{おもしろ}い
お
\ruby{譚}{はなし}を
\ruby{仕}{し}て
\ruby{聞}{き}かして
\ruby{下}{くだ}すつたのに、
%
\ruby{此}{この}
\ruby{{\換字{節}}}{せつ}は
\ruby{毫}{ちつと}も
\ruby{御談話}{お|はな|し}
なんぞして
\ruby{下}{くだ}さらないんだもの。
%
\ruby{妾}{わたし}は
ほんとに
\ruby{詰}{つま}らなくつて、
%
\ruby[g]{仕方}{し かた}が
ないから
\ruby[g]{本家}{う ち }から
\ruby{書}{ほん}を
\ruby{持}{も}つて
\ruby{來}{き}て
\ruby{讀}{よ}んで
\ruby{居}{ゐ}るのよ。
』

\原本頁{194-10}%
『
でも
\ruby{書}{ほん}が
おもしろけりやあ
\ruby{可}{いゝ}ぢやあ
\ruby{無}{な}いか、
%
\ruby{私}{わたし}の
\ruby{無器用}{ぶ|き|よう}な
\ruby[g]{談話}{はなし }
なんぞより。
』

\原本頁{195-1}%
\ruby[g]{頭髮}{か み }も
ゆら〳〵と
\ruby{頭}{かうべ}を
\ruby{振}{ふ}つて、

\原本頁{195-2}%
『
イヽエ、
%
\ruby[g]{矢張}{やつぱ }り
\ruby{御談話}{お|はな|し}の
\ruby{方}{はう}が
\ruby[||j>]{妾}{わたし}
\ruby[||j>]{好}{ す}きなのよ。
%
あの
\ruby{本}{ほん}は
\ruby[g]{面白}{おもしろ}い
\ruby{事}{こと}は
\ruby[g]{面白}{おもしろ}いけれど、
%
むづかしくつて
いけないところが
\ruby{有}{あ}るんですもの!。
%
\ruby[g]{今夜}{こんや }は
\ruby[g]{何處}{どつこ }へも
\ruby{行}{い}かないで
\ruby[g]{御話}{おはなし}を
\ruby{仕}{し}て。
%
ネ、
%
\ruby[g]{御願}{おねがひ}
ですから
\ruby{泣}{な}くやうなのを!。
%
\ruby[||j>]{妾}{わたし}
\ruby[||j>]{泣}{ な}くやうな
\ruby[g]{御話}{おはなし}が
\ruby[g]{大好}{だいす }きなのよ。
』

\原本頁{195-6}%
と
\ruby[g]{{\換字{遠}}慮}{ゑんりよ}も
\ruby{無}{な}く
\ruby[g]{{\換字{強}}{\換字{請}}}{ね だ }れば
\ruby{吉右衛門}{きち||ゑ|もん}は
\ruby{苦}{にが}りて、

\原本頁{195-7}%
『
また
\ruby[g]{其樣}{そ ん }なに
\ruby{直}{ぢき}
\ruby[||j>]{汝}{おまへ}は
\ruby{甘}{あま}つたれるよ!。
%
そんな
\ruby[g]{氣樂}{き らく}な
\ruby{事}{こと}
どころぢやあ
\ruby{無}{な}くつて
ゐらつしやるのだ。
』

\原本頁{195-9}%
と、
%
\ruby{少}{すこ}し
\ruby{叱}{しか}り
\ruby[g]{氣味}{ぎ み }に
\ruby{{\換字{遮}}}{さへぎ}り
\ruby{止}{とゞ}むるに、

\原本頁{195-10}%
『
アヽ
\ruby[||j>]{妾}{わたし}
\ruby[||j>]{知}{ し}つてますよ、
%
\ruby[g]{五十子}{いそこ}さんが
\ruby{惡}{わる}いから\換字{!?}。
%
\ruby[||j>]{妾}{わたし}
\ruby[||j>]{今日}{ け|ふ}
\ruby{見}{み}て
\ruby{來}{き}てよ
\ruby[g]{五十子}{いそこ}さんを。
%
ほんとに
\ruby[||j>]{憫}{かは}
\ruby[||j>]{然}{いさう}に% 「憫然 か(は)いさう」
% \ruby{憫然}{かは|いさう}に% 「憫然 か(は)いさう」
\ruby[g]{病重}{わ る }いのねえ。
』

\原本頁{196-1}%
と、
%
\ruby{然}{さ}も
\ruby[g]{心配}{しんぱい}
\ruby{氣}{げ}に
\ruby{艶}{つや}やかなる
\ruby{面}{おもて}の
\ruby{美}{うつく}しき
\ruby{眉}{まゆ}を
\ruby[g]{打顰}{うちひそ}めたる、
%
\ruby{云}{い}ふに
\ruby{云}{い}はれぬ
\ruby[g]{可愛}{か はい}さ
ありて、
%
\ruby[g]{此室}{こ ゝ }ばかりには
\ruby{騷}{さわ}がしき
\ruby{風}{かぜ}も
\ruby{吹}{ふ}かぬが
\ruby{如}{ごと}し。
