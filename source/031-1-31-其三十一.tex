\Entry{其三十一}

\ruby{授業}{じゆ|げふ}も
\ruby{爲}{な}し
\ruby{難}{がた}く
\ruby{見}{み}えたるほどの
\ruby{暴風}{あ|れ}の
\ruby{一日}{いち|にち}の
\ruby{生{\換字{暖}}}{なま|あたゝか}きに、
\ruby{{\換字{平}}生}{いつ|も}の
\ruby{如}{ごと}く
\ruby{敎鞭}{けう|べん}を
\ruby{執}{と}りて
\ruby{太郎次郎}{た|らう|じ|らう}を
\ruby{相手}{あひ|て}に
\ruby{仕}{し}たりし
\ruby{水野}{みづ|の}は、
\ruby{我}{が}の
\ruby{{\換字{強}}}{つよ}きところあれば
\ruby{職務}{つと|め}を
\ruby{怠}{おこた}りこそは
\ruby{爲}{せ}ざりつれ、
\ruby{{\換字{前}}日}{ぜん|じつ}よりの
\ruby{身心}{しん|〴〵}の
\ruby{疲}{つか}れに、
\ruby{五體}{ご|たい}の
\ruby{綿}{わた}の
\ruby{如}{ごと}くなれるを
\ruby{我}{われ}と
\ruby{覺}{おぼ}えつゝ、やうやく
\ruby{午後}{ご|ゞ}
\ruby{何時}{なん|じ}の
\ruby{今}{いま}、
\ruby{始}{はじ}めて
\ruby{我}{わ}が
\ruby{身}{み}の
\ruby{我}{わ}が
\ruby{物}{もの}となりたる
\ruby{心地}{こゝ|ち}する
\ruby{氣}{き}の
\ruby{{\換字{緩}}}{ゆる}みに、
\ruby{歩調}{あるき|つき}さへ
\ruby{遲々}{ち|ゝ}として、
\ruby{脫力}{がつ|かり}して
\ruby{歸}{かへ}り
\ruby{來}{きた}れり。

\ruby{手}{て}を
\ruby{掛}{か}けたるにはあらねど
\ruby{小}{ちひ}さき
\ruby{樹草}{き|くさ}など
\ruby{好}{よ}きほどに
\ruby{生}{は}えたればおのづからの
\ruby{庭}{には}となりたる
\ruby{{\換字{空}}地}{あき|ち}を
\ruby{{\換字{前}}}{まへ}に、
\ruby{南}{みなみ}を
\ruby{受}{う}けたる
\ruby{長}{なが}き
\ruby{一棟}{ひと|むね}の、
\ruby{其}{そ}の
\ruby{奧}{おく}の
\ruby{一間}{ひと|ま}は
\ruby{我}{わ}が
\ruby{起臥}{おき|ふし}のところと
\ruby{定}{さだ}まりたるなり。
\ruby{水野}{みづ|の}は
\ruby{常}{つね}の
\ruby{如}{ごと}く
\ruby{庭先}{には|さき}を
\ruby{家}{いへ}に
\ruby{{\換字{沿}}}{そ}ひて
\ruby{{\換字{廻}}}{まは}りて、
\ruby{椽}{えん}より
\ruby{直}{たゞち}に
\ruby{座敷}{ざ|しき}に
\ruby{上}{あが}らんとするに、
\ruby{今日}{け|ふ}は
\ruby{烈}{はげ}しき
\ruby{風}{かぜ}を
\ruby{厭}{いと}ひて、
\ruby{雨{\換字{戸}}}{あま|ど}さへ
\ruby{幾枚}{いく|まい}か
\ruby{引}{ひ}かれ
\ruby{居}{ゐ}たり。

『ア、
\ruby{風}{かぜ}が
\ruby{甚}{ひど}いので
\ruby{雨{\換字{戸}}}{あま|ど}を
\ruby{引}{ひ}いて
\ruby{置}{お}きました。
\ruby{薄暗}{うすつ|くら}くつて
お
\ruby{{\換字{嫌}}}{いや}なら
\ruby{明}{あ}けてあげませう。
\ruby{方向}{む|き}が
\ruby{好}{い}いので
\ruby{此家}{こ|ゝ}は
\ruby{其程}{それ|ほど}ぢやあ
\ruby{有}{あ}りませんが、
\ruby{何}{なに}にしろ
\ruby{甚}{ひど}い
\ruby{{\換字{嫌}}}{いや}な
\ruby{風}{かぜ}です。
』

\ruby{我}{わ}が
\ruby{跫音}{あし|おと}を
\ruby{聞}{きゝ}つけての
\ruby[g]{吉右衛門}{きちゑもん}が
\ruby{言葉}{こと|ば}に、

『なあに
\ruby{今日}{け|ふ}は
\ruby{別}{べつ}に
\ruby{細字書}{こま|かい|ほん}を
\ruby{讀}{よ}まうとも
\ruby{思}{おも}はないから、
\ruby{矢張}{やつ|ぱ}り
\ruby{此儘}{この|まゝ}にして!。
』

と
\ruby{云}{い}いながら
\ruby{水野}{みづ|の}は
\ruby{身}{み}を
\ruby{側}{そば}めて、
\ruby{隙}{す}かして
\ruby{引}{ひ}かれたる
\ruby{{\換字{戸}}}{と}の
\ruby{間}{すそ}より
\ruby{上}{あが}り、

『ほんとに
\ruby{氣持}{き|もち}の
\ruby{惡}{わる}い、
\ruby{頭}{あたま}の
\ruby{痛}{いた}くなるやうな
\ruby{風}{かぜ}で、{---}
\ruby{早}{はや}く
\ruby{止}{や}んで
\ruby{吳}{く}れなくちやあ
\ruby{仕方}{し|かた}が
\ruby{無}{な}い。
』

と
\ruby{座敷}{ざ|しき}に
\ruby{入}{い}りつゝ
\ruby{言葉}{こと|ば}を
\ruby{足}{た}せば、

『
\ruby{左樣}{さ|う}でございます。
\ruby{雨}{あめ}が
\ruby{隨}{つ}いて
\ruby{來}{こ}ないで
\ruby{先々}{まあ|〳〵}ですが、
\ruby{土地}{とこ|ろ}によつちやあ
\ruby{餘程}{よ|ほど}の
\ruby{損{\換字{害}}}{いた|み}です。
この
\ruby{{\換字{嫌}}}{いや}に
\ruby{{\換字{暖}}}{あたゝか}い
\ruby{事}{こと}は
\ruby{何樣}{ど|う}でしやう。
\ruby{病人}{びやう|にん}なんぞにやあ
\ruby{感}{き}きますネ。
オ、
\ruby{病人}{びやう|にん}と
\ruby{云}{い}やあ
\ruby{今{\換字{朝}}}{け|さ}
お
\ruby{頼}{たの}みの
\ruby{婢}{をんな}は、
\ruby{私}{わたし}の
\ruby{本家}{う|ち}の
\ruby{方}{はう}の
\ruby{小作人}{こ|さく|にん}の
\ruby{娘}{むすめ}で、がせいに
\ruby{能}{よ}く
\ruby{働}{はたら}くのがありましたから、
\ruby{能}{よ}く
\ruby{云}{い}ひつけて
\ruby{其}{それ}を
\ruby{{\換字{遣}}}{や}つて
\ruby{置}{お}きました。
\ruby{看護{\換字{婦}}}{かん|ご|ふ}さんも
\ruby{來}{き}たさうです。
』

と、
\ruby{間}{あひ}の
\ruby{襖}{ふすま}は
\ruby{開}{ひら}き
\ruby{居}{ゐ}たる
\ruby{中}{なか}の
\ruby{間}{ま}にありて
\ruby{敷居}{しき|ゐ}
\ruby{越}{ご}しの
\ruby{挨拶}{あい|さつ}なり。
\ruby{水野}{みづ|の}は
\ruby{床{\換字{近}}}{とこ|ちか}く
\ruby{置}{お}きたる
\ruby{机}{つくゑ}の
\ruby{{\換字{前}}}{まへ}に
\ruby{坐}{すわ}りて、
\ruby{始}{はじ}めて
\ruby{昨日}{きの|ふ}
\ruby{以來}{い|らい}の
\ruby{疲勞}{つか|れ}を
\ruby{息}{やす}めつゝ、

『アヽ、
\ruby{今}{いま}
\ruby{一寸}{ちよ|つと}
\ruby{歸路}{かへ|り}に
\ruby{立寄}{たち|よ}つて
\ruby{來}{き}ました。
いろ〳〵お
\ruby{世話}{せ|わ}を
\ruby{有}{あ}り
\ruby{難}{がた}かつた。
\ruby{先}{まあ}これで
\ruby{一切}{いつ|さい}
\ruby{思}{おも}ふやうになつた。
』

と、
\ruby{重荷}{おも|に}を
\ruby{卸}{おろ}したるが
\ruby{如}{ごと}き
\ruby{顏色}{かほ|つき}すれば、
\ruby{例}{れい}の
\ruby{眼鏡}{め|がね}の
\ruby{中}{うち}より
\ruby{一寸}{ちよ|つと}
\ruby{見}{み}て、

『
\ruby{昨夜}{ゆふ|べ}は
\ruby{碌}{ろく}に
\ruby{御睡眠}{お|よ|り}はなさりますまいのに、
\ruby{今日}{け|ふ}は
\ruby{{\換字{又}}}{また}
\ruby{{\換字{平}}生}{いつ|も}の
\ruby{{\換字{通}}}{とほ}り
\ruby{御{\換字{勤}}務}{お|つ|とめ}では、
\ruby{大抵}{たい|てい}な
\ruby{御疲勞}{お|くた|びれ}ではありますまい。
\ruby{今夜}{こん|や}はまあ
\ruby{早}{はや}く
\ruby{御睡眠}{お|やす|み}なさいまし。
』

と、
\ruby{云}{い}ひさして
\ruby{茶}{ちや}の
\ruby{間}{ま}の
\ruby{方}{かた}を
\ruby{顧}{かへり}みて
\ruby{聲}{こゑ}
\ruby{大}{おほき}く、

『お
\ruby{濱}{はま}や。
また
\ruby{其樣}{そ|ん}なに
\ruby{書}{ほん}にばかり
\ruby{取付}{とつ|つ}いて
\ruby{居}{ゐ}ちやあいけない。
\ruby{先生}{せん|せい}が
お
\ruby{歸}{かへ}りなすつたぢやあ
\ruby{無}{な}いか、
\ruby{御茶}{お|ちや}を
\ruby{持}{も}つて
\ruby{來}{こ}ないか。
』

と、
\ruby{悠然}{ゆつ|くり}としたる
\ruby{調子}{てう|し}に
\ruby{呼}{よ}ばゝつたるは、
\ruby{言葉}{こと|ば}つきなども
\ruby{異}{をか}しからぬほど
\ruby{江{\換字{戸}}}{え|ど}の
\ruby{水}{みづ}も
\ruby{飮}{の}んだる
\ruby{果}{はて}の
\ruby{老夫}{おや|ぢ}なれど、
\ruby{流石}{さす|が}は
\ruby{根}{ね}が
\ruby{此}{こ}の
\ruby{邊}{あたり}の
\ruby{田舎}{ゐな|か}
\ruby{風}{ふう}なり。

\ruby{小}{ちひ}さき
\ruby{刳{\換字{盆}}}{くり|ぼん}に
\ruby{大}{おほき}なる
\ruby{筒茶碗載}{つゝ|ぢあ|わん|の}せて、
\ruby{嫣然}{につ|こり}と
\ruby{笑}{ゑ}みて
\ruby{持出}{もち|い}でたる
お
\ruby{濱}{はま}は、
\ruby{水野}{みづ|の}が
\ruby{膝{\換字{近}}}{ひざ|ちか}くそれを
\ruby{置}{お}きて、おのれは
\ruby{祖{\換字{父}}}{ぢ|ゞ}の
\ruby{傍}{かたへ}に
\ruby{甘}{あま}えるやうに
\ruby{坐}{すわ}り。

『
\ruby{昨夜}{ゆふ|べ}は
\ruby{怖}{こは}かつたでしようねえ、
\ruby{眞闇}{まつ|くら}で!。
あれから
\ruby{妾}{わたし}
\ruby{床}{とこ}へ
\ruby{入}{はい}つたら、
\ruby{先生}{せん|せい}の
\ruby{行}{いら}しつた
\ruby{方}{はう}の、
\ruby{{\換字{遠}}}{とほ}くの
\ruby{{\換字{遠}}}{とほ}くから、
\ruby{狗}{いぬ}の
\ruby{鳴}{な}く
\ruby{聲}{こゑ}が
\ruby{聞}{きこ}えて
\ruby{來}{き}て、
\ruby{淋}{さび}しかつたわ!。
』

と
\ruby{云}{い}ひ
\ruby{出}{だ}せば、

『ハヽヽ、
\ruby{何}{な}んだ
\ruby{下}{くだ}らない、
\ruby{叩頭}{おじ|ぎ}も
\ruby{仕}{し}ないで!。
\ruby{突然}{いき|なり}と
\ruby{其樣}{そ|ん}な
\ruby{事}{こと}を
\ruby{云}{い}ひ
\ruby{出}{だ}すよ。
\ruby{狗}{いぬ}が
\ruby{鳴}{な}いたつて
\ruby{何淋}{なに|さみ}しい
\ruby{奴}{やつ}があるもんか。
』

と
\ruby{笑}{わらひ}を
\ruby{帶}{お}びて
\ruby[g]{吉右衛門}{きちゑもん}は
\ruby{叱}{しか}るを、
\ruby{眞赤}{まつ|か}なる
\ruby{番茶}{ばん|ちや}の
\ruby{味}{あぢ}も
\ruby{無}{な}く
\ruby{香}{か}も
\ruby{無}{な}けれど、
\ruby{熱}{あつ}きのみに
\ruby{人}{ひと}の
\ruby{{\換字{情}}}{なさけ}は
\ruby{有}{あ}るを
\ruby{啜}{すゝ}れる
\ruby{水野}{みづ|の}は、

『ハヽヽ、お
\ruby{濱}{はま}ちやんはいつでも
\ruby{面白}{おも|しろ}い
\ruby{事}{こと}を
\ruby{云}{い}ふ!。
そして
\ruby{昨夜}{ゆふ|べ}は
\ruby{一生懸命}{いつ|しやう|けん|めい}に
\ruby{書}{ほん}を
\ruby{讀}{よ}んで
\ruby{居}{ゐ}たぢやあ
\ruby{無}{な}いか、あれは
\ruby{一體}{いつ|たい}
\ruby{何}{なん}の
\ruby{本}{ほん}だえ。
』

と
\ruby{問}{と}ふに、
お
\ruby{濱}{はま}は
\ruby{忽}{たちま}ち
\ruby{不足}{ふ|そく}らしき
\ruby{恨}{うら}みを
\ruby{其色}{その|いろ}に
\ruby{現}{あら}はしたり。

『だつて
\ruby{先}{せん}の
\ruby{中}{うち}は
\ruby{毎晩}{まい|ばん}
\g詰めruby{々々}{〳〵}いろんな
\ruby{面白}{おも|しろ}い
お
\ruby{譚}{はなし}を
\ruby{仕}{し}て
\ruby{聞}{き}かして
\ruby{下}{くだ}すつたのに、
\ruby{此{\換字{節}}}{この|せつ}は
\ruby{毫}{ちつと}も
\ruby{御談話}{お|はな|し}なんぞして
\ruby{下}{くだ}さらないんだもの。
\ruby{妾}{わたし}はほんとに
\ruby{詰}{つ}まらなくつて、
\ruby{仕方}{し|かた}がないから
\ruby{本家}{う|ち}から
\ruby{書}{ほん}を
\ruby{持}{も}つて
\ruby{來}{き}て
\ruby{讀}{よ}んで
\ruby{居}{ゐ}るのよ。
』

『でも
\ruby{書}{ほん}がおもしろけりやあ
\ruby{可}{いゝ}ぢやあ
\ruby{無}{な}いか、
\ruby{私}{わたし}の
\ruby{不器用}{ぶ|き|よう}な
\ruby{談話}{はな|し}なんぞより。
』

\ruby{頭髮}{か|み}もゆら〳〵と
\ruby{頭}{かうべ}を
\ruby{振}{ふ}つて、

『イヽエ、
\ruby{矢張}{やつ|ぱ}り
\ruby{御談話}{お|はな|し}の
\ruby{方}{はう}が
\ruby[<h||]{妾}{わたし}
\ruby{好}{す}きなのよ。
あの
\ruby{本}{ほん}は
\ruby{面白}{おも|しろ}い
\ruby{事}{こと}は
\ruby{面白}{おも|しろ}いけれど、むづかしくつていけないところが
\ruby{有}{あ}るんですもの!。
\ruby{今夜}{こん|や}は
\ruby{何處}{どつ|こ}へも
\ruby{行}{い}かないで
\ruby{御話}{お|はなし}を
\ruby{仕}{し}て。
ネ、
\ruby{御願}{お|ねがひ}ですから
\ruby{泣}{な}くやうなのを!。
\ruby[<h||]{妾}{わたし}
\ruby{泣}{な}くやうな
\ruby{御話}{お|はなし}が
\ruby{大好}{だい|す}きなのよ。
』

と
\ruby{{\換字{遠}}慮}{ゑん|りよ}も
\ruby{無}{な}く
\ruby{{\換字{強}}{\換字{請}}}{ね|だ}れば
\ruby[g]{吉右衛門}{きちゑもん}は
\ruby{苦}{にが}りて、

『また
\ruby{其樣}{そ|ん}なに
\ruby{直}{ぢき}
\ruby{汝}{おまへ}は
\ruby{甘}{あま}つたれるよ!。
そんな
\ruby{氣樂}{き|らく}な
\ruby{事}{こと}どころぢやあ
\ruby{無}{な}くつてゐらつしやるのだ。
』

と、
\ruby{少}{すこ}し
\ruby{叱}{しか}り
\ruby{氣味}{ぎ|み}に
\ruby{{\換字{遮}}}{さへぎ}り
\ruby{止}{とゞ}むるに、

『アヽ
\ruby[<h||]{妾}{わたし}
\ruby{知}{し}つてますよ、
\ruby{五十子}{い|そ|こ}さんが
\ruby{惡}{わる}いから\換字{?!}。
\ruby[<h||]{妾}{わたし}
\ruby{今日}{け|ふ}
\ruby{見}{み}て
\ruby{來}{き}てよ
\ruby{五十子}{い|そ|こ}さんを。
ほんとに
\ruby{憫然}{かは|いさう}に% 「憫然 か(は)いさう」
\ruby{病重}{わ|る}いのねえ。
』

と、
\ruby{然}{さ}も
\ruby{心配}{しん|ぱい}
\ruby{氣}{げ}に
\ruby{艶}{つや}やかなる
\ruby{面}{おもて}の
\ruby{美}{うつく}しき
\ruby{眉}{まゆ}を
\ruby{打顰}{うち|しか}めたる、
\ruby{云}{い}ふに
\ruby{云}{い}はれぬ
\ruby{可愛}{か|はい}さありて、
\ruby{此室}{こ|ゝ}ばかりには
\ruby{騷}{さわ}がしき
\ruby{風}{かぜ}も
\ruby{吹}{ふ}かぬが
\ruby{如}{ごと}し。
