\Entry{其三十二}

『
\ruby{人}{ひと}にも
\ruby{云}{い}はないで
\ruby{何時}{い|つ}の
\ruby{間}{ま}に
\ruby{岩崎}{いは|さき}さんのところへ
\ruby{行}{い}つて
\ruby{見}{み}たのだ。
\ruby{彼方}{あち|ら}ぢやあ
\ruby{御煩}{お|うるさ}く
\ruby{御思}{おお|も}ひだらうに!。
』

『いゝえ
\ruby{祖父}{お|ぢい}さん、
\ruby{一寸}{ちよ|つと}
\ruby{行}{い}つたばかしで、
\ruby{上}{あが}りも
\ruby{何}{なに}も
\ruby{仕}{し}やあ
\ruby{仕}{し}ないのよ!。
たゞそーつと
\ruby{外}{そと}から
\ruby{見}{み}たばかりなの。
だけれども
\ruby{肥}{ふと}つた
\ruby{看護婦}{かん|ご|ふ}さんも
\ruby{見}{み}たし、
\ruby{丁度松}{ちよ|うど|まつ}ちやんにも
\ruby{會}{あ}つて
\ruby{話}{はなし}を
\ruby{仕}{し}て
\ruby{來}{き}たのよ。
\ruby{松}{まつ}ちやんは
\ruby{曩日吾家}{いつ|か|う|ち}で
\ruby{一{\換字{所}}}{いつ|しよ}に
\ruby{{\換字{遊}}}{あそ}んだ
\ruby{時}{とき}なんかとは
\ruby{{\換字{違}}}{ちが}つて、
\ruby{泣}{な}きさうな
\ruby{顏}{かほ}を
\ruby{仕}{し}て
\ruby{居}{ゐ}るんだもの、
\ruby{妾}{わたし}ほんとに
\ruby{憫然}{かはい|さう}になつちまつたの!。
だもんだから
\ruby{彼}{あ}の
\ruby{椎}{しひ}の
\ruby{樹}{き}の
\ruby{傍}{そば}で、
\ruby{二人}{ふた|り}でつい
\ruby{泣}{な}いて
\ruby{話}{はなし}を
\ruby{仕}{し}て
\ruby{居}{ゐ}たら、
\ruby{彼家}{あす|こ}の
お
\ruby{澤婆}{さは|ばゞあ}つたら
\ruby{眞箇}{ほん|と}に
\ruby{憎}{にく}らしい!、
お
\ruby{濱子}{はま|つこ}!、
\ruby{汝}{おめへ}まで
\ruby{心配}{しん|ぱい}して
\ruby{居}{ゐ}るだけえ?、だけれど
\ruby{泣}{な}いたつて
\ruby{無{\換字{益}}}{だ|め}なこんだ!。
\ruby{心配}{しん|ぱい}で
\ruby{癒}{なほ}る
\ruby{病氣}{びや|うき}あ
\ruby{無}{ね}えだから、つて
\ruby{{\換字{菜}}圃}{はた|け}の
\ruby{對}{むかふ}から
\ruby{大}{おほき}な
\ruby{聲}{こゑ}をして
\ruby{怒鳴}{ど|な}るんだもの!。
\ruby{妾}{わたし}ほんとに
\ruby{口惜}{く|や}しくつて
\ruby{口惜}{く|や}しくつて、
\ruby{風}{かぜ}の
\ruby{中}{なか}を
\ruby{駈}{か}け
\ruby{出}{だ}して
\ruby{歸}{かへ}つて
\ruby{來}{き}て
\ruby{一人}{ひと|り}で
\ruby{怒}{おこ}つて
\ruby{泣}{な}いたわ。
ほんとに
\ruby{彼樣}{あ|ん}な
\ruby{意地惡}{い|ぢ|わる}な
\ruby{婆}{ばゞあ}つたら
\ruby{有}{あ}りや
\ruby{仕}{し}ない!。
\ruby{今度}{こん|ど}また
\ruby{彼樣}{あ|ん}な
\ruby{事}{こと}を
\ruby{云}{い}つたら
\ruby{引爬}{ひつ|か}いて
\ruby{{\換字{遣}}}{や}らなくつちやあ。
』

『ハヽヽ、また
\ruby{其樣}{そ|ん}な
お
\ruby{轉婆}{てん|ば}な
\ruby{事}{こと}をいふよ!。
\ruby{何樣}{ど|う}して〳〵
\ruby{彼}{あ}の
\ruby{婆}{ばあ}さんにやあ
\ruby{汝}{おまへ}なんぞの
\ruby{爪}{つめ}も
\ruby{立}{た}つもんぢやあ
\ruby{無}{な}い。
\ruby{婆}{ばあ}さんを
\ruby{引爬}{ひつ|か}きやあ
\ruby{汝}{おまへ}の
\ruby{爪}{つめ}は
\ruby{悉皆}{みん|な}
\ruby{{\換字{脱}}}{と}れたつて、
\ruby{彼方}{むか|ふ}にやあ
\ruby{蚯蚓脹}{みゝ|ず|ばれ}も
\ruby{出來}{で|き}や
\ruby{仕}{し}ない。
そんな
\ruby{事}{こと}はまあ
\ruby{何樣}{ど|う}でも
\ruby{可}{い}いが、もうそろ〳〵と
\ruby{日}{ひ}が
\ruby{暮}{く}れかゝる、
お
\ruby{鍋}{なべ}が
\ruby{何}{なに}かことつかせて
\ruby{居}{ゐ}る、
\ruby{汝}{おまへ}も
\ruby{彼方}{あつ|ち}へ
\ruby{行}{い}つて
\ruby{夕方}{ゆふ|がた}の
\ruby{事}{こと}を、
\ruby{些}{ちつと}は
\ruby{傍}{そば}から
\ruby{手傳}{て|つだ}つて
\ruby{{\換字{遣}}}{や}りナ。
』

『
\ruby{先生}{せん|せい}が
\ruby{今夜面白}{こん|や|おも|しろ}い
\ruby{御話}{お|はなし}を
\ruby{仕}{し}て
\ruby{下}{くだ}さるなら。
』

『
\ruby{祖父}{お|ぢい}さんが
\ruby{命令}{いひ|つけ}るのに
\ruby{先生}{せん|せい}のところへ
\ruby{掛}{かゝ}つて
\ruby{行}{い}くとは、
\ruby{何}{なん}だか
\ruby{理由}{わ|け}の
\ruby{{\換字{分}}}{わか}らない
\ruby{理屈合}{り|くつ|あひ}だナ。
サアマア
\ruby{何}{なん}でも
\ruby{可}{い}いから
\ruby{御働}{お|はら}き、
お
\ruby{働}{はたら}き!』

『ハイ。
ぢやあ
\ruby{先生}{せん|せい}
\ruby{屹度}{きつ|と}
\ruby{後刻}{の|ち}に
\ruby{先日}{この|あひだ}の
\ruby{御話}{お|はなし}の
\ruby{續}{つゞ}きをネ。
』

\ruby{頭}{くび}を
\ruby{曲}{ま}げて
\ruby{水野}{みづ|の}の
\ruby{顏}{かほ}を
\ruby{覗}{のぞ}き
\ruby{{\換字{込}}}{こ}むやうにして
\ruby{自己}{お|の}が
\ruby{{\換字{勝}}手}{かつ|て}を
\ruby{云}{い}ひつつ
お
\ruby{濱}{はま}は
\ruby{纔}{わづか}に
\ruby{彼方}{かな|た}に
\ruby{去}{さ}りたり。

\ruby{祖父}{ぢ|ゞ}は
\ruby{孫娘}{まご|むすめ}の
\ruby{背姿}{うしろ|すがた}を
\ruby{見}{み}おくりながら、

『
\ruby{身長}{せ|い}ばかり
\ruby{彼樣}{あ|ん}なに
\ruby{大}{おほ}きくなつて、いつまで
\ruby{彼樣}{あ|ん}な
\ruby{調子}{てう|し}で
\ruby{居}{ゐ}るのでしやう!。
もう
\ruby{少}{すこ}しは
\ruby{女}{をんな}らしくなりさうなものですのに、あゝやんちやんでは
\ruby{仕方}{し|かた}が
\ruby{有}{あ}りません。
いくらお
\ruby{澤婆}{さは|ばあ}さんが
\ruby{憎}{にく}いと
\ruby{云}{い}つたつて、
\ruby{引爬}{ひつ|か}いて
\ruby{{\換字{遣}}}{や}らうなんて、ハヽハヽハヽ。
』

と
\ruby{獨語}{ひとり|ごと}の
\ruby{如}{ごと}く
\ruby{{\換字{又}}}{また}
\ruby{辯護}{べん|ご}の
\ruby{如}{ごと}く
\ruby{云}{い}へば、
\ruby{其}{そ}の
\ruby{語}{ことば}に
\ruby{隨}{つ}いて、

『\換字{志}かしお
\ruby{澤}{さは}といふ
\ruby{婆}{ばあ}さんは
\ruby{眞箇}{ほん|と}に
\ruby{甚}{ひど}い!。
\ruby{何樣}{ど|う}した
\ruby{人}{ひと}だか
\ruby{知}{し}らないが、
\ruby{全}{まる}で
\ruby{普{\換字{通}}}{ひと|なみ}ぢやあ
\ruby{無}{な}い、
\ruby{先鬼婆}{まあ|おに|ばゞあ}だから、
\ruby{誰}{たれ}だつて
\ruby{何樣}{ど|う}か
\ruby{仕}{し}て
\ruby{{\換字{遣}}}{や}りたい
\ruby{位}{ぐらゐ}には
\ruby{思}{おも}はうぢやあ
\ruby{無}{な}いか。
』

と、
\ruby{水野}{みづ|の}は
\ruby{我}{わ}が
\ruby{思}{おも}へるところを
\ruby{打}{う}ち
\ruby{出}{いだ}したり。

『
\ruby{貴君}{あな|た}も
\ruby{何}{なに}かで
\ruby{御腹立}{お|はら|だち}でしたネ。
\ruby{其}{そり}やあ
\ruby{左樣}{そ|う}でございますとも、
\ruby{普{\換字{通}}}{な|み}ぢやあ
\ruby{有}{あ}りません!。
\ruby{仰}{おつし}ある
\ruby{{\換字{通}}}{とほ}り
\ruby{鬼}{おに}になつて
\ruby{居}{ゐ}るのですから!。
あれでも
\ruby{舊}{もと}は
\ruby{人}{ひと}の
\ruby{好}{い}い
\ruby{婆}{ばあ}さんでしたが、
\ruby{親一人}{おや|ひと|り}
\ruby{娘一人}{こ|ひと|り}の
\ruby{秘蔵娘}{ひ|ざう|むすめ}の、
お
\ruby{里}{さと}といふのに
\ruby{婿}{むこ}を
\ruby{取}{と}つたー
\ruby{其婿}{その|むこ}が
\ruby{惡}{わる}かつたところから
\ruby{彼樣}{あ|ゝ}なつたのです。
』

『フーン。
』

『
\ruby{婿}{むこ}は
\ruby{兵作}{ひやう|さく}といふ
\ruby{惡}{わる}い
\ruby{奴}{やつ}で、
\ruby{今}{いま}は
\ruby{東京}{とう|きやう}の
\ruby{牛込}{うし|ごめ}あたりに、
\ruby{樂}{らく}な
\ruby{生活}{くら|し}を
\ruby{仕}{し}て
\ruby{居}{ゐ}るさうですが、
\ruby{出}{で}は
\ruby{二合{\換字{半}}領}{に|がふ|はん|りやう}の
\ruby{可成}{か|なり}な
\ruby{大盡}{だい|じん}の
\ruby{二番生}{に|ばん|ばえ}で、
\ruby{男振}{をとこ|ぶり}の
\ruby{惡}{わる}くない
\ruby{應對}{おう|たい}の
\ruby{上手}{じや|うず}な
\ruby{男}{をとこ}です。
\ruby{婆}{ばあ}さんの
\ruby{家}{うち}は
\ruby{村}{むら}でも
\ruby{指折}{ゆび|をり}の
\ruby{物持}{もの|もち}でしたが、
\ruby{其}{そ}の
\ruby{兵作}{ひやう|さく}といふのが
\ruby{猫}{ねこ}を
\ruby{被}{かぶ}つた
\ruby{狼}{おほかみ}でして、
\ruby{何}{なに}を
\ruby{爲}{す}る、
\ruby{彼}{か}を
\ruby{爲}{す}ると
\ruby{云}{い}つては
\ruby{金}{かね}を
\ruby{持出}{もち|だ}し、
\ruby{{\換字{終}}}{しまひ}には
\ruby{家屋敷}{いへ|や|しき}まで
\ruby{抵當}{てい|たう}に
\ruby{打込}{ぶち|こ}んだのです。
\換字{志}かし
\ruby{其}{それ}が
\ruby{眞實}{ほん|と}に
\ruby{商賣事}{しやう|ばい|ごと}で
\ruby{損}{そん}を
\ruby{仕}{し}たといふなら
\ruby{未}{ま}だ
\ruby{好}{よ}うございますが、
\ruby{實}{じつ}は
\ruby{婿}{むこ}になる
\ruby{前}{まへ}から
\ruby{他}{ほか}に
\ruby{{\換字{情}}婦}{をん|な}が
\ruby{有}{あ}つて、
\ruby{其方}{その|はう}に
\ruby{悉皆}{みん|な}こかしたのです。
\ruby{左樣}{そ|う}して
\ruby{置}{お}いて
\ruby{{\換字{平}}井}{ひら|ゐ}の
\ruby{家}{うち}に
\ruby{塵}{ちり}ツ
\ruby{葉}{ぱ}
\ruby{一}{ひと}つ
\ruby{無}{な}くなつた
\ruby{時{\換字{分}}}{じ|ぶん}に、さあ
\ruby{自{\換字{分}}}{じ|ぶん}が
\ruby{{\換字{逐}}出}{おひ|だ}されて
\ruby{仕舞}{し|ま}ふ
\ruby{心算}{つも|り}で、
\ruby{彼}{あ}の
\ruby{婆}{ばあ}さん
\ruby{親子}{おや|こ}に
\ruby{無理}{む|り}ばかり
\ruby{云}{い}つて、
\ruby{打}{ぶ}ちます、
\ruby{蹴}{け}ます、
\ruby{暴}{あば}れます、
\ruby{散々}{さん|〴〵}に
\ruby{酷}{ひど}い
\ruby{事}{こと}を
\ruby{致}{いた}しました。
それが
\ruby{爲}{ため}に
お
\ruby{里}{さと}が
\ruby{癆瘵氣質}{らう|さい|かた|ぎ}になつて、
\ruby{氣}{き}は
\ruby{異}{をか}\換字{志}くなるし、
\ruby{生}{い}きながら
\ruby{幽靈}{いう|れい}のやうに
\ruby{痩}{や}せて、
\ruby{苦}{くる}しんで〳〵
\ruby{居}{を}りましたが、
\ruby{其中}{その|なか}を
\ruby{畢竟}{とう|〳〵}
\ruby{別}{わか}れ
\ruby{話}{ばなし}を
\ruby{仕}{し}て、
\ruby{兵作}{ひやう|さく}は
\ruby{身}{み}を
\ruby{{\換字{退}}}{の}いて
\ruby{仕舞}{し|ま}ひました。
』

『ヤ、それは
\ruby{恐}{おそ}ろしい
\ruby{酷}{むご}い
\ruby{談}{はなし}で。
』

『それこれでお
\ruby{里}{さと}は
\ruby{死}{し}んで
\ruby{仕舞}{し|ま}ひます。
\ruby{婆}{ばあ}さんは
\ruby{住}{す}んで
\ruby{居}{ゐ}た
\ruby{家}{うち}も
\ruby{{\換字{逐}}出}{おつ|た}てられて、
\ruby{他人}{ひ|と}の
\ruby{物置小屋}{もの|おき|ご|や}を
\ruby{假}{か}りて
\ruby{入}{はい}るやうな
\ruby{始末}{し|まつ}にもなりましたが、それから
\ruby{彼}{あ}の
\ruby{婆}{ばあ}さんは
\ruby{鬼}{おに}のやうになりまして、
\ruby{誰彼}{たれ|かれ}の
\ruby{見}{み}さかひ
\ruby{無}{な}く
\ruby{人}{ひと}を
\ruby{疑}{うたが}ひ、
\ruby{一生懸命}{いつ|しやう|けん|めい}に
\ruby{挊}{かせ}いでは
\ruby{一文二文}{いち|もん|に|もん}を
\ruby{溜}{た}めて、
\ruby{其錢}{その|ぜに}を
\ruby{苛}{ひど}い
\ruby{高利}{かう|り}で
\ruby{貸}{か}し
\ruby{出}{だ}しました。
\ruby{左樣}{さ|う}して
\ruby{五年六年}{ご|ねん|ろく|ねん}と
\ruby{立}{た}つ
\ruby{内}{うち}に
\ruby{段々太}{だん|〴〵|ふと}りまして、
\ruby{舊}{もと}の
\ruby{自{\換字{分}}}{じ|ぶん}の
\ruby{家}{うち}を
\ruby{取}{と}り
\ruby{{\換字{返}}}{かへ}して
\ruby{手}{て}に
\ruby{入}{い}れたのです。
\ruby{他手}{ひと|で}に
\ruby{渡}{わた}つて
\ruby{居}{ゐ}る
\ruby{中}{うち}に
\ruby{焼}{や}けましたので、
\ruby{母屋}{おも|や}や
\ruby{藏}{くら}は
\ruby{殘}{のこ}つて
\ruby{居}{ゐ}ませんが、
\ruby{丁度今岩崎}{ちや|うど|いま|いは|ざき}さんの
\ruby{借}{か}りて
\ruby{居}{い}る
\ruby{室}{へや}が、
\ruby{兵作}{ひやう|さく}を
\ruby{婿}{むこ}に
\ruby{取}{と}つた
\ruby{其初}{その|はじめ}に、
\ruby{老人}{とし|より}は
\ruby{若}{わか}い
\ruby{夫婦}{ふう|ふ}に
\ruby{香}{こう}ばしく
\ruby{有}{あ}るまいからつて、
\ruby{自{\換字{分}}}{じ|ぶん}の
\ruby{隱居{\換字{所}}}{いん|きよ|じよ}にと
\ruby{建}{た}てた
\ruby{別室}{はな|れ}で、
\ruby{今}{いま}
\ruby{自{\換字{分}}}{じ|ぶん}の
\ruby{入}{はい}つて
\ruby{居}{ゐ}る
\ruby{汚}{きたな}い
\ruby{家}{うち}は、
\ruby{{\換字{平}}井}{ひら|ゐ}の
\ruby{家}{うち}の
\ruby{榮}{さか}えて
\ruby{居}{ゐ}た
\ruby{頃}{ころ}の
\ruby{雑物小屋}{ざふ|もつ|ご|や}です。
\ruby{左樣}{さ|う}いふ
\ruby{婆}{ばあ}さんですから、
\ruby{今}{いま}ぢやあたゞ、
\ruby{金}{かね}より
\ruby{外}{ほか}に
\ruby{味方}{み|かた}は
\ruby{無}{な}いと
\ruby{思}{おも}つて、まるで
\ruby{鬼}{おに}のやうになり
\ruby{切}{き}つて
\ruby{居}{ゐ}て、
\ruby{村}{むら}の
\ruby{者}{もの}にも
\ruby{憎}{にく}がられりやあ、
\ruby{自{\換字{分}}}{じ|ぶん}も
\ruby{村}{むら}の
\ruby{者}{もの}を
\ruby{對敵}{むか|ふ}にして
\ruby{居}{ゐ}るので
\ruby{云}{い}つて
\ruby{見}{み}りやあ
\ruby{愍然}{かはい|さう}な
\ruby{筋}{すぢ}もあるのです。
』

『
\ruby{大}{おほ}きに、
\ruby{成程}{なる|ほど}!。
』

\ruby{水野}{みづ|の}は
\ruby{此談}{この|はなし}を
\ruby{聞}{き}きて
\ruby{黯然}{あん|ぜん}として、
\ruby{{\換字{情}}}{こゝろ}の
\ruby{傷}{きずつ}ける
\ruby{人}{ひと}の
\ruby{末路}{す|ゑ}の
\ruby{恐}{おそ}ろしさを
\ruby{思}{おも}ひつゝ
\ruby{歎}{たん}ずるところへ、
\ruby{忙}{あはた}だしく
\ruby{人}{ひと}の
\ruby{駈}{か}け
\ruby{來}{く}る
\ruby{跫音}{あし|おと}して、
\ruby{椽{\換字{前}}}{えん|さき}より、

『
\ruby{水野}{みづ|の}さん!
\ruby{水野}{みづ|の}さん!。
』

と
\ruby{呼}{よ}ぶは
\ruby{他人}{ほ|か}ならず
\ruby{松之助}{まつ|の|すけ}なり。

\ruby{其}{その}おろ〳〵したる
\ruby{悲}{かな}しき
\ruby{聲音}{こわ|ね}を
\ruby{聞}{き}くより、
\ruby{何}{なん}とは
\ruby{無}{な}しに
\ruby{胸潰}{むね|つぶ}れて、

『ど、
\ruby{何樣}{ど|う}かしたか?、
\ruby{惡}{わる}いのかえ?、
\ruby{姊}{ねえ}さんが。
』

と、サツと
\ruby{障子}{しやう|じ}を
\ruby{開}{ひら}けば、
\ruby{{\換字{暖}}}{あたゝか}き
\ruby{不快}{ふ|くわい}の
\ruby{風}{かぜ}はムツと
\ruby{吹}{ふ}きて、
\ruby{黄昏}{たそ|がれ}の
\ruby{空}{そら}の
\ruby{光線}{ひか|り}の
\ruby{{\換字{弱}}}{よわ}きに、
\ruby{恐怖}{おそ|れ}を
\ruby{{\換字{懐}}}{いだ}ける
\ruby{松之助}{まつ|の|すけ}の
\ruby{顏}{かほ}は
\ruby{影}{かげ}さへ
\ruby{淋}{さみ}しく
\ruby{薄々}{うす|〳〵}と
\ruby{白}{しら}みて
\ruby{見}{み}えたり。

『
\ruby{大變}{たい|へん}に
\ruby{惡}{わる}い!。
いけないかも
\ruby{知}{し}れ……。
アヽ、
\ruby{僕}{ぼく}あ
\ruby{何樣}{ど|う}したら
\ruby{宣}{よ}からう!。
』

\ruby{既泣}{はや|な}き
\ruby{聲}{ごゑ}の、\換字{志}どろもどろの
\ruby{其}{その}
\ruby{言葉}{こと|ば}を
\ruby{聞}{き}くや
\ruby{聞}{き}かずや、
\ruby{水野}{みづ|の}は
\ruby{忽}{たちま}ち
\ruby{全身}{ぜん|しん}に
\ruby{氷}{こほり}の
\ruby{水}{みづ}を
\ruby{浴}{あ}びし
\ruby{心地}{こゝ|ち}して、アツとばかりに
\ruby{仆}{たふ}れんとしけるが、
\ruby{辛}{から}くも
\ruby{堪}{た}へて
\ruby{自}{みづか}ら
\ruby{保}{たも}ち、
\ruby{次}{つ}いで
\ruby{烈}{はげ}しき
\ruby[g]{戰慄}{ふるひ}の
\ruby{止}{と}めても
\ruby{止}{と}まらず
\ruby{起}{おこ}り
\ruby{來}{く}るを
\ruby{{\換字{強}}}{し}ひて
\ruby{制}{せい}しつ、

『ナニ、そんな
\ruby{事}{こと}が……、
\ruby{大丈夫}{だい|ぢやう|ぶ}だ!。
』

と、
\ruby{我}{わ}が
\ruby{耳}{みゝ}にも
\ruby{知}{し}るゝ
\ruby{顫聲}{ふるひ|ごゑ}に
\ruby{云}{い}いさま、
\ruby{我知}{われ|し}らず
\ruby{我}{わ}が
\ruby{座}{ざ}より
\ruby{飛}{と}び
\ruby{立}{た}つて、
\ruby{踵}{かゝと}も
\ruby{地}{ち}に
\ruby{着}{つ}かぬ
\ruby{跣足}{は|だし}の
\ruby{危}{あやふ}く、
\ruby{轉}{まろ}ぶが
\ruby{如}{ごと}くに
\ruby{走去}{はせ|さ}つたり。

