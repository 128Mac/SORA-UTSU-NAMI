\Entry{其三十二}

% メモ 校正終了 2024-04-11 2024-05-27 2024-06-20
\原本頁{196-5}%
『
\ruby{人}{ひと}にも
\ruby{云}{い}はないで
\ruby[g]{何時}{い つ }の
\ruby{間}{ま}に
\ruby[g]{岩崎}{いはざき}さんの
ところへ
\ruby{行}{い}つて
\ruby{見}{み}たのだ。
%
\ruby[g]{彼方}{あちら }ぢやあ
\ruby[g]{御煩}{おうるさ}く
\ruby[g]{御思}{お おも}ひだらうのに!。
』

\原本頁{196-7}%
『
いゝえ
\ruby[g]{祖{\換字{父}}}{お ぢい}さん、
%
\ruby[g]{一寸}{ちよつと}
\ruby{行}{い}つたばかしで、
%
\ruby{上}{あが}りも
\ruby{何}{なに}も
\ruby{仕}{し}やあ
\ruby{仕}{し}ないのよ!。
%
たゞ
そ{---}{---}つと
\ruby{外}{そと}から
\ruby{見}{み}たばかりなの。
%
だけども
\ruby{肥}{ふと}つた
\ruby{看護{\換字{婦}}}{かん|ご|ふ}さんも
\ruby{見}{み}たし、
%
\ruby[g]{丁度}{ちやうど}
\ruby{松}{まつ}ちやんにも
\ruby{會}{あ}つて
\ruby{話}{はなし}を
\ruby{仕}{し}て
\ruby{來}{き}たのよ。
%
\ruby{松}{まつ}ちやんは
\ruby[g]{曩日}{いつか }
\ruby[g]{吾家}{う ち }で
\ruby[g]{一{\換字{所}}}{いつしよ}に
\ruby{{\換字{遊}}}{あす}んだ% ルビ調整(原本通り)
\ruby{時}{とき}
なんかとは
\ruby{{\換字{違}}}{ちが}つて、
%
\ruby{泣}{な}きさうな
\ruby{顏}{かほ}を
\ruby{仕}{し}て
\ruby{居}{ゐ}るんだもの、
%
\ruby{妾}{わたし}
ほんとに
\ruby[||j>]{憫}{かは}
\ruby[||j>]{然}{いさう}に% 「憫然 か(は)いさう」
% \ruby{憫然}{かは|いさう}に% 「憫然 か(は)いさう」
なつちまつたの!。
%
だもんだから
\ruby{彼}{あ}の
\ruby{椎}{しひ}の
\ruby{樹}{き}の
\ruby{傍}{そば}で、
%
\ruby[g]{二人}{ふたり }で
つい
\ruby{泣}{な}いて
\ruby{話}{はなし}を
\ruby{仕}{し}て
\ruby{居}{ゐ}たら、
%
\ruby[g]{彼家}{あすこ }の
お
\ruby{澤}{さは}
\ruby[||j>]{婆}{ばゞあ}つたら
\ruby[g]{眞箇}{ほんと }に
\ruby{憎}{にく}らしい!
\改行% 校正作業の簡略化のため
、
%
\原本頁{197-4}\改行%
お
\ruby[g]{濱子}{はまつこ}!、
%
\ruby{汝}{おめへ}まで
\ruby[g]{心配}{しんぱい}して
\ruby{居}{ゐ}るだけえ?、
%
だけれど
\ruby{泣}{な}いたつて
\原本頁{197-5}\改行%
\ruby[g]{無益}{だ め }なこんだ!、
%
\ruby[g]{心配}{しんぱい}で
\ruby{癒}{なほ}る
\ruby[g]{病氣}{びやうき}あ
\ruby{無}{ね}えだから、
%
つて
\ruby[g]{{\換字{菜}}圃}{はたけ }の
\ruby[<j||]{對}{むかふ}% 行末行頭の境界付近なので特例処置を施す
\原本頁{197-6}\改行%
から
\ruby{大}{おほき}な
\ruby{聲}{こゑ}をして
\ruby[g]{怒鳴}{ど な }るんだもの!。
%
\ruby{妾}{わたし}ほんとに
\ruby[g]{口惜}{く やし}くつて
\ruby[g]{口惜}{く やし}くつて、
%
\ruby{風}{かぜ}の
\ruby{中}{なか}を
\ruby{駈}{か}け
\ruby{出}{だ}して
\ruby{歸}{かへ}つて
\ruby{來}{き}て
\ruby[g]{一人}{ひとり }で
\ruby{怒}{おこ}つて
\ruby{泣}{な}いたわ。
%
ほんとに
\ruby[g]{彼樣}{あ ん }な
\ruby{意地惡}{い|ぢ|わる}な
\ruby{婆}{ばゞあ}つたら
\ruby{有}{あ}りや
\ruby{仕}{し}ない!。
%
\ruby[g]{今度}{こんど }
また
\ruby[g]{彼樣}{あ ん }な
\ruby{事}{こと}を
\ruby{云}{い}つたら
\ruby[g]{引爬}{ひつか }いて
\ruby{{\換字{遣}}}{や}らなくつちやあ。
』

\原本頁{197-10}%
『
ハヽヽ、
%
また
\ruby[g]{其樣}{そ ん }な
\ruby{御轉婆}{お|てん|ば}な
\ruby{事}{こと}を
\ruby{云}{い}ふよ!。
%
\ruby[g]{何樣}{ど う }して〳〵
\ruby{彼}{あ}の
\ruby{婆}{ばあ}さんにやあ
\ruby{汝}{おまへ}なんぞの
\ruby{爪}{つめ}も
\ruby{立}{た}つもんぢや
\ruby{無}{な}い。
%
\ruby{婆}{ばあ}さんを
\ruby[g]{引爬}{ひつか }きやあ
\ruby{汝}{おまへ}の
\ruby{爪}{つめ}は
\ruby[g]{悉皆}{みんな }
\ruby{脫}{と}れたつて、
%
\ruby[g]{彼方}{むかふ }にやあ
\ruby{蚯蚓脹}{みゝ|ず|ばれ}も
\ruby[g]{出來}{で き }や
\ruby{仕}{し}ない。
%
そんな
\ruby{事}{こと}は
まあ
\ruby[g]{何樣}{ど う }でも
\ruby{可}{い}いが、
%
もう
そろ〳〵と
\ruby{日}{ひ}が
\ruby{暮}{く}れ
かゝる、
%
お
\ruby{鍋}{なべ}が
\ruby{何}{なに}か
ことつかせて
\ruby{居}{ゐ}る、
%
\ruby{汝}{おまへ}も
\ruby[g]{彼方}{あつち }へ
\ruby{行}{い}つて
\ruby[g]{夕方}{ゆふがた}の
\ruby{事}{こと}を、
%
\ruby{些}{ちつと}は
\ruby{傍}{そば}から
\ruby[g]{手傳}{て つだ}つて
\ruby{{\換字{遣}}}{や}りナ。
』

\原本頁{198-5}%
『
\ruby[g]{先生}{せんせい}が
\ruby[g]{今夜}{こんや }
\ruby[g]{面白}{おもしろ}い
\ruby[g]{御話}{おはなし}を
\ruby{仕}{し}て
\ruby{下}{くだ}さるなら。
』

\原本頁{198-6}%
『
\ruby[g]{祖{\換字{父}}}{お ぢい}さんが
\ruby[g]{命令}{いひつけ}るのに
\ruby[g]{先生}{せんせい}の
ところへ
\ruby{掛}{かゝ}つて
\ruby{行}{い}くとは、
%
\ruby{何}{なん}だか
\ruby[g]{理由}{わ け }の
\ruby{{\換字{分}}}{わか}らない
\ruby[g]{理屈}{り くつ}
\ruby{合}{あひ}だナ。
%
サアマア
\ruby{何}{なん}でも
\ruby{可}{い}いから
\ruby[g]{御働}{おはたら}き
\改行% 校正作業の簡略化のため
、
%
\原本頁{198-8}\改行%
お
\ruby{働}{はたら}き!』

\原本頁{198-9}%
『
ハイ。
%
ぢやあ
\ruby[g]{先生}{せんせい}
\ruby[g]{屹度}{きつと }
\ruby[g]{後刻}{の ち }に
\ruby[||j>]{先}{この}
\ruby[||j>]{日}{あひだ}の
% \ruby{先日}{この|あひだ}の
\ruby[g]{御話}{おはなし}の
\ruby{續}{つゞ}きをネ。
』

\原本頁{198-10}%
\ruby{頭}{くび}を
\ruby{曲}{ま}げて
\ruby[g]{水野}{みづの }の
\ruby{顏}{かほ}を
\ruby{覗}{のぞ}き
\ruby{{\換字{込}}}{こ}むやうにして
\ruby[g]{自己}{お の }が
\ruby[g]{{\換字{勝}}手}{かつて }を
\ruby{云}{い}ひつつ
お
\ruby{濱}{はま}は
\ruby{纔}{わづか}に
\ruby[g]{彼方}{かなた }に
\ruby{去}{さ}りたり。

\原本頁{199-1}%
\ruby[g]{祖{\換字{父}}}{ぢ ゞ }は
\ruby[||j>]{孫}{まご}
\ruby[||j>]{娘}{むすめ}の
% \ruby{孫娘}{まご|むすめ}の
\ruby[<j||]{背}{うしろ}
\ruby[||j>]{姿}{すがた}
を
\ruby{見}{み}おくりながら、

\原本頁{199-2}%
『
\ruby[g]{身長}{せ い }ばかり
\ruby[g]{彼樣}{あ ん }なに
\ruby{大}{おほ}きくなつて、
%
いつまで
\ruby[g]{彼樣}{あ ん }な
\ruby[g]{調子}{てうし }で
\ruby{居}{ゐ}
るのでしやう!。
%
もう
\ruby{少}{すこ}しは
\ruby{女}{をんな}らしく
なりさうなものですのに
\改行% 校正作業の簡略化のため
、
%
\原本頁{199-4}\改行%
あゝ
やんちやんでは
\ruby[g]{仕方}{し かた}が
\ruby{有}{あ}りません。
%
いくら
お
\ruby{澤}{さは}
\ruby{婆}{ばあ}さんが
\ruby{憎}{にく}いと
\ruby{云}{い}つたつて、
%
\ruby[g]{引爬}{ひつか }いて
\ruby{{\換字{遣}}}{や}らうなんて、
%
ハヽハヽハヽ。
』

\原本頁{199-6}%
と
\ruby[||j>]{獨}{ひとり}
\ruby[||j>]{語}{ ごと}の
\ruby{如}{ごと}く
\ruby{{\換字{又}}}{また}
\ruby[g]{辯護}{べんご }の% 弁 瓣 辦 辧 辨 辩 (辯)
\ruby{如}{ごと}く
\ruby{云}{い}へば、
%
\ruby{其}{そ}の
\ruby{語}{ことば}に
\ruby{隨}{つ}いて、

\原本頁{199-7}%
『
\換字{志}かし
お
\ruby{澤}{さは}といふ
\ruby{婆}{ばあ}さんは
\ruby[g]{眞箇}{ほんと }に
\ruby{甚}{ひど}い!。
%
\ruby[g]{何樣}{ど う }した
\ruby{人}{ひと}だか
\ruby{知}{し}らないが、
%
\ruby{全}{まる}で
\ruby[g]{普{\換字{通}}}{ひとなみ}ぢやあ
\ruby{無}{な}い、
%
\ruby{先}{まあ}
\ruby[||j>]{鬼}{おに}
\ruby[||j>]{婆}{ばゞあ}
% \ruby{鬼婆}{おに|ばゞあ}
だから、
%
\ruby{誰}{だれ}だつて
\ruby[g]{何樣}{ど う }か
\ruby{仕}{し}て
\ruby{{\換字{遣}}}{や}りたい
\ruby{位}{ぐらゐ}には
\ruby{思}{おも}はうぢやあ
\ruby{無}{な}いか。
』

\原本頁{199-10}%
と、
%
\ruby[g]{水野}{みづの }は
\ruby{我}{わ}が
\ruby{思}{おも}へる
ところを
\ruby{打}{う}ち
\ruby{出}{いだ}したり。

\原本頁{199-11}%
『
\ruby[g]{貴君}{あなた }も
\ruby{何}{なに}かで
\ruby{御腹立}{お|はら|だち}でしたネ。
%
\ruby{其}{そり}やあ
\ruby[g]{左樣}{そ う }で
ございますとも、
%
% \原本頁{200-1}\改行%
\ruby[g]{普{\換字{通}}}{な み }ぢやあ
\ruby{有}{あ}りません!。
%
\ruby{仰}{おつし}ある
\ruby{{\換字{通}}}{とほ}り
\ruby{鬼}{おに}に
なつて
\ruby{居}{ゐ}るのですから!。
%
あれでも
\ruby{舊}{もと}は
\ruby{人}{ひと}の
\ruby{好}{い}い
\ruby{婆}{ばあ}さんでしたが、
%
\ruby{親一人}{おや|ひと|り}
\ruby{娘一人}{こ|ひと|り}の
\ruby[g]{秘蔵}{ひ ざう}
\ruby{娘}{むすめ}の、
%
お
\ruby{里}{さと}と
いふのに
\ruby{聟}{むこ}を% 婿 5a7f (聟 805f)
\ruby{取}{と}つた%
{---}{---}%
\ruby{其}{その}
\ruby{聟}{むこ}が% 婿 5a7f (聟 805f)
\ruby{惡}{わる}かつた
とこ
\原本頁{200-4}\改行%
ろから
\ruby[g]{彼樣}{あ ゝ }なつたのです。
』

\原本頁{200-5}%
『
フーン。
』

\原本頁{200-6}%
『
\ruby{聟}{むこ}は% 婿 5a7f (聟 805f)
\ruby[||j>]{兵}{ひやう}
\ruby[||j>]{作}{ さく}といふ
% \ruby{兵作}{ひやう|さく}といふ
\ruby{惡}{わる}い
\ruby{奴}{やつ}で、
%
\ruby{今}{いま}は
\ruby[||j>]{東}{とう}
\ruby[||j>]{京}{きやう}の
% \ruby{東京}{とう|きやう}の
\ruby[g]{牛{\換字{込}}}{うしごめ}
あたりに、
%
\ruby{樂}{らく}な
\ruby[g]{生活}{くらし }を
\ruby{仕}{し}て
\ruby{居}{ゐ}るさうですが、
%
\ruby{出}{で}は
\ruby[g]{二合}{に がふ}
\ruby{{\換字{半}}}{はん}
\ruby{領}{りやう}の
\ruby[g]{可成}{か なり}な
\ruby[g]{大盡}{だいじん}の
\ruby{二番生}{に|ばん|ばえ}で、
%
\ruby[||j>]{男}{をとこ}
\ruby[||j>]{振}{ ぶり}の
\ruby{惡}{わる}くない
\ruby[g]{應對}{おうたい}の
\ruby[g]{上手}{じやうず}な
\ruby{男}{をとこ}です。
%
\ruby{婆}{ばあ}さんの
\ruby{家}{うち}は
\ruby{村}{むら}でも
\原本頁{200-9}\改行%
\ruby[g]{指折}{ゆびをり}の
\ruby[g]{物持}{ものもち}でしたが、
%
\ruby{其}{そ}の
\ruby[||j>]{兵}{ひやう}
\ruby[||j>]{作}{ さく}と
% \ruby{兵作}{ひやう|さく}と
いふのが
\ruby{猫}{ねこ}を
\ruby{被}{かぶ}つた
\ruby[<j>]{{\換字{狼}}}{おほかみ}でして
\改行% 校正作業の簡略化のため
、
%
\原本頁{200-10}\改行%
\ruby{何}{なに}を
\ruby{爲}{す}る、
%
\ruby{彼}{か}を
\ruby{爲}{す}ると
\ruby{云}{い}つては
\ruby{金}{かね}を
\ruby[g]{持出}{もちだ }し、
%
\ruby{{\換字{終}}}{しまひ}には
\ruby{家屋敷}{いへ|や|しき}まで
\ruby[g]{抵當}{ていたう}に
\ruby[g]{打{\換字{込}}}{ぶちこ }んだのです。
%
\換字{志}かし
\ruby{其}{それ}が
\ruby[g]{眞實}{ほんと }に
\ruby[||j>]{商}{しやう}
\ruby[||j>]{賣}{ ばい}
\ruby[||j>]{事}{ ごと}で
\ruby{損}{そん}を
\ruby{仕}{し}た
と
いふのなら
\ruby{未}{ま}だ
\ruby{好}{よ}うございますが、
%
\ruby{實}{じつ}は
\ruby{聟}{むこ}になる% 婿 5a7f (聟 805f)
\ruby{{\換字{前}}}{まへ}から
\ruby{他}{ほか}に
\ruby[g]{{\換字{情}}{\換字{婦}}}{をんな }が
\ruby{有}{あ}つて、
%
\ruby{其}{その}
\ruby{方}{はう}に
\ruby[g]{悉皆}{みんな }
こかしたのです。
%
\ruby[g]{左樣}{そ う }して
\ruby{置}{お}いて
\ruby[g]{{\換字{平}}井}{ひらゐ }
\原本頁{201-3}\改行%
の
\ruby{家}{うち}に
\ruby{塵}{ちり}ツ
\ruby{葉}{ぱ}
\ruby{一}{ひと}つ
\ruby{無}{な}くなつた
\ruby[g]{時{\換字{分}}}{じ ぶん}に、
%
さあ
\ruby[g]{自{\換字{分}}}{じ ぶん}が
\ruby[g]{{\換字{逐}}出}{おひだ }されて
\ruby[g]{仕舞}{し ま }ふ
\ruby[g]{心算}{つもり }で、
%
\ruby{彼}{あ}の
\ruby{婆}{ばあ}さん
\ruby[g]{親子}{おやこ }に
\ruby[g]{無理}{む り }
ばかり
\ruby{云}{い}つて、
%
\ruby{打}{ぶ}ちます、
%
\ruby{蹴}{け}ます、
%
\ruby{暴}{あば}れます、
%
\ruby[g]{散々}{さん〴〵}に
\ruby{酷}{ひど}い
\ruby{事}{こと}を
\ruby{致}{いた}しました。
%
それが
\ruby{爲}{ため}に
お
\ruby{里}{さと}が
\ruby[g]{癆瘵}{らうがい}% ルビ調整(原本通り)
\ruby[g]{氣質}{かたぎ }になつて、
%
\ruby{氣}{き}は
\ruby{異}{をか}\換字{志}くなるし、
%
\ruby{生}{い}きながら
\ruby[g]{幽靈}{いうれい}のやうに
\ruby{痩}{や}せて、
%
\ruby{苦}{くる}しんで〳〵
\ruby{居}{を}りましたが、
%
\ruby{其}{その}
\ruby{中}{なか}を
\ruby[g]{畢竟}{とう〳〵}
\ruby{別}{わか}れ
\ruby[<j||]{話}{ばなし}% 行末行頭の境界付近なので特例処置を施す
\原本頁{201-8}\改行%
を
\ruby{仕}{し}て、
%
\ruby[||j>]{兵}{ひやう}
\ruby[||j>]{作}{ さく}は
% \ruby{兵作}{ひやう|さく}は
\ruby{身}{み}を
\ruby{{\換字{退}}}{の}いて
\ruby[g]{仕舞}{し ま }ひました。
』

\原本頁{201-9}%
『
ヤ、
%
それは
\ruby{恐}{おそ}ろしい
\ruby{酷}{むご}い
\ruby{談}{はなし}で。
』

\原本頁{201-10}%
『
それ
これで
お
\ruby{里}{さと}は
\ruby{死}{し}んで
\ruby[g]{仕舞}{し ま }ひます、
%
\ruby{婆}{ばあ}さんは
\ruby{住}{す}んで
\ruby{居}{ゐ}た
\ruby{家}{うち}をも
\ruby[g]{{\換字{逐}}出}{おつた }てられて、
%
\ruby[g]{他人}{ひ と }の
\ruby{物置小屋}{もの|おき|ご|や}を
\ruby{假}{か}りて
\ruby{入}{はい}るやうな
\ruby[g]{始末}{し まつ}に
\原本頁{202-1}\改行%
も
なりましたが、
%
それから
\ruby{彼}{あ}の
\ruby{婆}{ばあ}さんは
\ruby{鬼}{おに}のやうに
なりまして
\改行% 校正作業の簡略化のため
、
%
\原本頁{202-2}\改行%
\ruby{誰}{だれ}
\ruby{彼}{かれ}の
\ruby{見}{み}さかひ
\ruby{無}{な}く
\ruby{人}{ひと}を
\ruby{疑}{うたが}ひ、
%
\ruby[||j>]{一}{いつ}
\ruby[||j>]{生}{しやう}
\ruby[||j>]{懸}{ けん}
\ruby[||j>]{命}{ めい}に
\ruby{挊}{かせ}いでは
\ruby[g]{一{\換字{文}}}{いちもん}
\ruby[g]{二{\換字{文}}}{に もん}を
\原本頁{202-3}\改行%
\ruby{溜}{た}めて、
%
\ruby{其}{その}
\ruby{錢}{ぜに}を
\ruby{苛}{ひど}い
\ruby[g]{高利}{かうり }で
\ruby{貸}{か}し
\ruby{出}{だ}しました。
%
\ruby[g]{左樣}{さ う }して
\ruby[g]{五年}{ご ねん}
\ruby[g]{六年}{ろくねん}と
\ruby{立}{た}つ
\ruby{内}{うち}に
\ruby[g]{段々}{だん〴〵}
\ruby{太}{ふと}りまして、
%
\ruby{舊}{もと}の
\ruby[g]{自{\換字{分}}}{じ ぶん}の
\ruby{家}{うち}を
\ruby{取}{と}り
\ruby{{\換字{返}}}{かへ}して
\ruby{手}{て}に
\ruby{入}{い}れたのです。
%
\ruby[g]{他手}{ひとで }に
\ruby{渡}{わた}つて
\ruby{居}{ゐ}る
\ruby{中}{うち}に
\ruby{焼}{や}けましたので、
%
\ruby[g]{母屋}{おもや }や
\ruby{藏}{くら}
\原本頁{202-6}\改行%
は
\ruby{殘}{のこ}つて
\ruby{居}{ゐ}ませんが、
%
\ruby[g]{丁度}{ちやうど}
\ruby{今}{いま}
\ruby[g]{岩崎}{いはざき}さんの
\ruby{借}{か}りて
\ruby{居}{ゐ}る
\ruby{室}{へや}が、
%
\ruby[<j||]{兵}{ひやう}
\ruby[<j||]{作}{さく}
% \ruby{兵作}{ひやう|さく}
\原本頁{202-7}\改行%
を
\ruby{聟}{むこ}に% 婿 5a7f (聟 805f)
\ruby{取}{と}つた
\ruby{其}{その}
\ruby{初}{はじめ}に、
%
\ruby[g]{老人}{としより}は
\ruby{{\換字{若}}}{わか}い
\ruby[g]{夫{\換字{婦}}}{ふうふ }に
\ruby{香}{かう}ばしく
\ruby{有}{あ}るまいからつて、
%
\ruby[g]{自{\換字{分}}}{じ ぶん}の
\ruby{隱居{\換字{所}}}{いん|きよ|じよ}にと
\ruby{建}{た}てた
\ruby[g]{別室}{はなれ }で、
%
\ruby{今}{いま}
\ruby[g]{自{\換字{分}}}{じ ぶん}の
\ruby{入}{はい}つて
\ruby{居}{ゐ}る
\ruby[<j||]{汚}{きたな}い% 行末行頭の境界付近なので特例処置を施す
\ruby{家}{うち}は、
%
\ruby[g]{{\換字{平}}井}{ひらゐ }の
\ruby{家}{うち}の
\ruby{榮}{さか}えて
\ruby{居}{ゐ}た
\ruby{頃}{ころ}の
\ruby{雜物小屋}{ざふ|もつ|ご|や}です。
%
\ruby[g]{左樣}{さ う }いふ
\ruby{婆}{ばあ}
\原本頁{202-10}\改行%
さんですから、
%
\ruby{今}{いま}ぢやあ
たゞ、
%
\ruby{金}{かね}より
\ruby{外}{ほか}に
\ruby[g]{味方}{み かた}は
\ruby{無}{な}いと
\ruby{思}{おも}つて
\改行% 校正作業の簡略化のため
、
%
\原本頁{202-11}\改行%
まるで
\ruby{鬼}{おに}のやうになり
\ruby{切}{き}つて
\ruby{居}{ゐ}て、
%
\ruby{村}{むら}の
\ruby{者}{もの}にも
\ruby{憎}{にく}がられりやあ
\改行% 校正作業の簡略化のため
、
%
\原本頁{203-1}\改行%
\ruby[g]{自{\換字{分}}}{じ ぶん}も
\ruby{村}{むら}の
\ruby{者}{もの}を
\ruby[g]{對敵}{むかふ }にして
\ruby{居}{ゐ}るので
\ruby{云}{い}つて
\ruby{見}{み}りやあ
\ruby[||j>]{愍}{かは}
\ruby[||j>]{然}{いさう}な% 「愍然 か(は)いさう」
% \ruby{愍然}{かは|いさう}な% 「愍然 か(は)いさう」
\ruby{筋}{すぢ}もあるのです。
』

\原本頁{203-2}%
『
\ruby{大}{おほ}きに、
%
\ruby[g]{成程}{なるほど}!。
』

\原本頁{203-3}%
\ruby[g]{水野}{みづの }は
\ruby{此}{この}
\ruby{談}{はなし}を
\ruby{聞}{き}きて
\ruby[g]{黯然}{あんぜん}として、
%
\ruby{{\換字{情}}}{こゝろ}の
\ruby{傷}{きずつ}ける
\ruby{人}{ひと}の
\ruby[g]{末路}{す ゑ }の
\ruby{恐}{おそ}ろしさを
\ruby{思}{おも}ひつゝ
\ruby{歎}{たん}ずる
ところへ、
%
\ruby{忙}{あはた}だしく
\ruby{人}{ひと}の
\ruby{駈}{か}け
\ruby{來}{く}る
\ruby[g]{跫音}{あしおと}して
\改行% 校正作業の簡略化のため
、
%
\原本頁{203-6}\改行%
\ruby[g]{椽{\換字{前}}}{えんさき}より、

\原本頁{203-7}%
『
\ruby[g]{水野}{みづの }さん!% \inhibitglue{}% ここは「空き」があるので
\,% 原本上でのアキを再現するため「3/18 em」空ける
\ruby[g]{水野}{みづの }さん!。
』

\原本頁{203-8}%
と
\ruby{呼}{よ}ぶは
\ruby[g]{他人}{ほ か }ならず
\ruby{松之助}{まつ|の|すけ}なり。

\原本頁{203-9}%
\ruby{其}{その}
おろ〳〵したる
\ruby{悲}{かな}しき
\ruby[g]{聲音}{こわね }を
\ruby{聞}{き}くより、
%
\ruby{何}{なん}とは
\ruby{無}{な}しに
\ruby{胸}{むね}
\ruby{潰}{つぶ}れて
\改行% 校正作業の簡略化のため
、

\原本頁{203-11}%
『
ど、
%
\ruby[g]{何樣}{ど う }かしたか?、
%
\ruby{惡}{わる}いのかえ?、
%
\ruby{姊}{ねえ}さんが。
』

\原本頁{204-1}%
と、
%
サツと
\ruby[g]{障子}{しやうじ}を
\ruby{開}{ひら}けば、
%
\ruby[<j>]{{\換字{暖}}}{あたゝか}き
\ruby[g]{不快}{ふくわい}の
\ruby{風}{かぜ}は
ムツと
\ruby{吹}{ふ}きて、
%
\ruby[g]{黄昏}{たそがれ}の
\ruby{{\換字{空}}}{そら}の
\ruby[g]{光線}{ひかり }の
\ruby{{\換字{弱}}}{よわ}きに、
%
\ruby[g]{恐怖}{おそれ }を
\ruby{懷}{いだ}ける
\ruby{松之助}{まつ|の|すけ}の
\ruby{顏}{かほ}は
\ruby{影}{かげ}さへ
\ruby{淋}{さみ}しく
\ruby[g]{薄々}{うす〳〵}と
\ruby{白}{しら}みて
\ruby{見}{み}えたり。

\原本頁{204-4}%
『
\ruby[g]{大變}{たいへん}に
\ruby{惡}{わる}い!。
%
いけないかも
\ruby{知}{し}れ‥‥。
%
アヽ、
%
\ruby{僕}{ぼく}あ
\ruby[g]{何樣}{ど う }したら
\ruby{宜}{よ}からう!。
』

\原本頁{204-6}%
\ruby{既}{はや}
\ruby{泣}{な}き
\ruby{聲}{ごゑ}の、
\換字{志}どろもどろの
\ruby{其}{その}
\ruby[g]{言葉}{ことば }を
\ruby{聞}{き}くや
\ruby{聞}{き}かずや、
%
\ruby[g]{水野}{みづの }は
\ruby{忽}{たちま}ち
\ruby[g]{全身}{ぜんしん}に
\ruby{氷}{こほり}の
\ruby{水}{みづ}を
\ruby{{\換字{浴}}}{あ}びし
\ruby[g]{心地}{こゝち }して、
%
アツとばかりに
\ruby{仆}{たふ}れんと
しけるが、
%
\ruby{辛}{から}くも
\ruby{堪}{た}へて
\ruby{自}{みづか}ら
\ruby{保}{たも}ち、
%
\ruby{次}{つ}いで
\ruby{烈}{はげ}しき
\ruby[g]{戰慄}{ふるひ }の
\ruby{止}{と}めても
\ruby{止}{と}まらず
\ruby{起}{おこ}り
\ruby{來}{く}るを
\ruby{{\換字{強}}}{し}ひて
\ruby{制}{せい}しつ、

\原本頁{204-10}%
『
ナニ、
%
そんな
\ruby{事}{こと}が‥‥、
%
\ruby{大{\換字{丈}}夫}{だい|ぢやう|ぶ}だ!。
』

\原本頁{204-11}%
と、
%
\ruby{我}{わ}が
\ruby{耳}{みゝ}にも
\ruby{知}{し}るゝ
\ruby[||j>]{顫}{ふるひ}
\ruby[||j>]{聲}{ ごゑ}に
% \ruby{顫聲}{ふるひ|ごゑ}に
\ruby{云}{い}ひさま、
%
\ruby{我}{われ}
\ruby{知}{し}らず
\ruby{我}{わ}が
\ruby{座}{ざ}より
\ruby{飛}{と}び
\ruby{立}{た}つて、
%
\ruby{踵}{かゝと}も
\ruby{地}{ち}に
\ruby{着}{つ}かぬ
\ruby[g]{跣足}{はだし }の
\ruby{危}{あやふ}く、
%
\ruby{轉}{まろ}ぶが
\ruby{如}{ごと}くに
\ruby[g]{走去}{はせさ }つた
\改行% 校正作業の簡略化のため
り。
