\Entry{其三十三}

% メモ 校正終了 2024-04-11 2024-05-28
\原本頁{205-4}%
\ruby{槇籬}{まき|がき}
\ruby{隣}{とな}る
\ruby{木槿籬}{むく|げ|がき}、
%
\ruby{杉籬}{すぎ|がき}
つゞく
\ruby[||j>]{藪}{やぶ}
\ruby[||j>]{疊}{だゝみ}の、
% \ruby{藪疊}{やぶ|だゝみ}の、
%
\ruby{村徑}{むら|みち}の
\ruby{黄昏}{たそ|がれ}を
\ruby{息急}{いき|せは}しく
\ruby{走}{はし}る
\ruby{水野}{みづ|の}は、
%
\ruby{後}{あと}より
\ruby{{\換字{追}}}{お}ひ
\ruby{縋}{すが}れる
\ruby{松之助}{まつ|の|すけ}の
\ruby{手}{て}を
\ruby{引立}{ひつ|た}てゝ、
%
\ruby{夢}{ゆめ}に
\ruby{高}{たか}き
ところより
\ruby{落}{お}つるが
\ruby{如}{ごと}き
\ruby{膽}{きも}
\ruby{縮}{すく}む
\ruby{思}{おも}ひに、
%
\ruby{何}{なん}の
\ruby{{\換字{分}}別}{ふん|べつ}も
\ruby{無}{な}く
\ruby{駈}{か}けに
\ruby{駈}{か}けたり。

\原本頁{20}%
\ruby{今{\換字{朝}}}{け|さ}よりの
\ruby{風}{かぜ}に
\ruby{葉}{は}は
\ruby{裂}{さ}け
\ruby{茎}{くき}は
\ruby{折}{を}れ
\ruby{伏}{ふ}して、
%
\ruby{滿目}{まん|もく}の
\ruby{光景}{あり|さま}
\ruby{忌}{いま}はしく
\ruby{{\換字{狼}}{\換字{藉}}}{らう|ぜき}たる
\ruby{芋圃}{いも|ばた}の
\ruby{間}{あひだ}を、
%
\ruby{突}{つ}と
\ruby{行}{ゆ}き
\ruby{拔}{ぬ}けて、
%
\ruby{例}{れい}の
\ruby{婆}{ばゞ}が
\ruby{家}{いへ}の
\ruby{横}{よこ}を
\ruby{奧}{おく}へと
\ruby{{\換字{通}}}{とほ}らんとすれば、
%
\ruby{折}{をり}しも
\ruby{例}{れい}の
お
\ruby{澤}{さは}
\ruby{婆}{ばゞ}は、
%
\ruby{風}{かぜ}に
\ruby{捥}{も}がれたる
\ruby{柹}{かき}の
\原本頁{206-1}\改行%
\ruby{實}{み}の、
%
\ruby{或}{あるひ}は
\ruby{{\換字{猶}}}{なほ}
\ruby{靑}{あを}く、
%
\ruby{或}{あるひ}は
\ruby[||j>]{{\換字{半}}}{なかば}
\ruby[||j>]{黄}{ き}ばみ
\ruby{赤}{あか}らめるを、
\換字{志}たゝかに
\ruby{取}{と}り
\ruby{入}{い}れたる
\ruby{重}{おも}げなる
\ruby{箕}{み}に、
%
\ruby{枯柴}{かれ|しば}の
\ruby{如}{ごと}く
\ruby{骨立}{ほね|だ}つたる
\ruby[||j>]{兩}{りやう}
\ruby[||j>]{腕}{ うで}を% ルビ調整(原本通り)非踊り字表記
% \ruby{兩腕}{りやう|うで}を% ルビ調整(原本通り)非踊り字表記
\ruby{長}{なが}く
\ruby{露}{あらは}して
\ruby{掛}{か}けつ、
%
\ruby{一}{ひ}
ト
\ruby{歩}{あし}
\ruby{一}{ひ}
ト
\ruby{歩}{あし}に
\ruby{{\換字{強}}欲}{がう|よく}の
\ruby{力}{ちから}を
\ruby{入}{い}れて
\ruby{辛}{から}くも
\ruby{吾家}{わが|や}に
\ruby{{\換字{運}}}{はこ}ばんと、
%
\ruby{未}{ま}だ
\ruby{止}{や}まぬ
\ruby{風}{かぜ}に
\ruby{霜}{しも}の
\ruby{薄}{すゝき}と
\ruby{騷立}{さわ|だ}つ
\ruby{白髮}{しら|が}を
\ruby{吹}{ふ}き
\ruby{立}{た}たせながら
\原本頁{206-5}\改行%
\ruby{此方}{こな|た}へ% ルビ調整(原本通り)
\ruby{來}{き}かゝりしが、
%
\ruby{水野}{みづ|の}が
\ruby{慌}{あわ}て
\ruby{{\換字{狼}}狽}{うろ|た}へて
\ruby{入}{い}り
\ruby{來}{きた}れる
\ruby{態}{さま}を、
%
\ruby[<j||]{圓}{つぶら}なる% 行末行頭の境界付近なので特例処置を施す
\ruby{眼}{め}に
ぎろりと
\ruby{見}{み}て、
%
さも
\ruby{心地}{こゝ|ち}よげに
\ruby{冷笑}{あざ|わら}ひ、

\原本頁{206-7}%
『
とう〳〵
\ruby[<j||]{廿}{にじふ}
\ruby[||j>]{兩}{りやう}
に
なつて
\ruby{來}{き}たゞかネ?。
』

\原本頁{206-8}%
と、
%
\ruby{恰}{あだか}も% 恰も「あ(だ)かも」
\ruby{病}{や}める
\ruby{人}{ひと}の
\ruby{疾}{と}く
\ruby{死}{し}なんことを
\ruby{待設}{まち|まう}け
\ruby{居}{を}りし
\ruby{其}{そ}の
\ruby{甲{\換字{斐}}}{か|ひ}ありて、
%
\ruby{今}{いま}や
\ruby{我}{わ}が
\ruby{望}{のぞ}める
\ruby{時機}{と|き}の
\ruby{至}{いた}らんとするに、
%
\ruby{自}{みづか}ら
\ruby{先}{ま}づ
\ruby{聲}{こゑ}を
\ruby{揚}{あ}げて
\ruby{祝}{しゆく}し
\ruby{悅}{よろこ}べるが
\ruby{如}{ごと}く
\ruby{云}{い}ひぬ。

\原本頁{206-11}%
おのが
\ruby{手}{て}に
\ruby{些少}{すこ|し}ばかりの
\ruby{金子}{か|ね}の
\ruby{落}{お}ちんことを
\ruby{希}{ねが}ふ
\ruby{意}{こゝろ}より、
%
\ruby{他}{ひと}の
\ruby{生命}{いの|ち}
\ruby{掛}{か}けて
\ruby{思}{おも}へる
\ruby{人}{ひと}をも
\ruby{死}{し}ねがしに
\ruby{云}{い}ひなしたる
\ruby{此}{こ}の
\ruby{老婆}{ば|ゞ}の
\ruby{面}{つら}の
\ruby{憎}{にく}さ!。
%
\ruby{人}{ひと}には
あらずと
\ruby{豫}{かね}てより
\ruby{思}{おも}ひ
\ruby{居}{ゐ}たれど、
%
まのあたりに
\ruby{骨}{ほね}を
\ruby{刺}{さ}す
\ruby{此}{こ}の
\ruby{酷毒}{こく|どく}の
\ruby{語}{ことば}を
\ruby{{\換字{浴}}}{あび}せられては、
%
\ruby{頭脳}{あた|ま}の
\ruby{眞中}{まん|なか}より
\ruby{烈火}{れつ|くわ}の
\ruby{奔}{はし}る
\ruby{心地}{こゝ|ち}して、
%
おのれ
\ruby{憎}{につく}き
\ruby{獸畜}{けだ|もの}め、
%
たゞ
\ruby{一}{ひ}
ト
\ruby{攫}{つかみ}に
\ruby{引攫}{ひつ|つか}んで、
%
\原本頁{207-5}\改行%
\ruby{天狗裂}{てん|ぐ|ざ}きに
\ruby{裂}{さ}きて
\ruby{木}{き}の
\ruby{股}{また}
\ruby{高}{たか}く
\ruby{掛}{か}けて
\ruby{吳}{く}れんと、
%
むら〳〵と
\ruby{恐}{おそ}ろしき
\ruby{忿怒}{いか|り}の
\ruby{衝}{つ}き
\ruby{上}{あが}り
\ruby{來}{き}て、
%
\ruby{流石}{さす|が}に
\ruby{堪}{こら}へ
\ruby[||j>]{{\換字{情}}}{じやう}
\ruby[||j>]{{\換字{強}}}{ つよ}き
\ruby{水野}{みづ|の}も
\ruby{眞靑}{まつ|さを}に
なりたり。
