\Entry{其三十四}

% メモ 校正終了 2024-04-11 2024-05-28 2024-06-23 2024-06-25
\原本頁{207-9}%
\ruby{吉右衛門}{きち||ゑ|もん}が
\ruby[||j>]{物}{もの}
\ruby[||j>]{語}{がたり}に
% \ruby{物語}{もの|がたり}に
よりて
\ruby{此}{こ}の
\ruby{婆}{ばゞ}が
\ruby{身}{み}の
\ruby{上}{うへ}を
\ruby{聞}{き}かざりせば、
%
\ruby{或}{あるひ}は
\ruby{走}{はし}り
かゝりて
\ruby{一}{ひ}ト
\ruby{踢}{け}に
\ruby{踢}{け}
\ruby{倒}{たふ}すか、
%
\ruby{左}{さ}なくば
\ruby{其}{その}
\ruby{面}{おもて}に
\ruby{唾}{つばき}して
\ruby{罵}{のゝし}る
ほどの
\ruby{事}{こと}は
\ruby{爲}{し}たらんを、
%
\ruby{其}{そ}の
\ruby{如是}{か|く}
\ruby{鬼々}{おに|〳〵}しくなれる
\ruby{{\換字{所}}以}{ゆ|ゑん}を
\ruby{思}{おも}ひ
\ruby{{\換字{浮}}}{うか}むると、
%
\ruby{且}{かつ}は
\ruby{如是}{かゝ|る}
\ruby{老婆}{ば|ゞ}を
\ruby{相手}{あひ|て}に
\ruby{取}{と}りて
\ruby{何}{なに}と
なすべき、
%
\ruby{田}{た}に
\ruby{生}{うま}れて
\ruby{田}{た}に
\ruby{死}{し}する
\ruby{蟲}{むし}にも
\ruby{等}{ひと}しき
\ruby{田舎}{ゐ|なか}% ルビ調整(原本通り)
\ruby{婆}{ばゞ}の
\ruby{一言}{ひと|こと}に、
%
\ruby{氣}{き}を
\ruby{動}{うご}かして
\ruby{我}{われ}を
\ruby{忘}{わす}れん
としたるは
\ruby{愚}{おろか}なりと、
%
\ruby{{\換字{飽}}}{あく}まで
\ruby{{\換字{強}}}{つよ}く
\ruby{見下}{み|さ}げたるとに、
%
おのづと
\ruby{心}{こゝろ}も
\ruby{{\換字{緩}}}{ゆる}み
\ruby{和}{やはら}ぎて、
%
\ruby{水野}{みづ|の}は
\ruby{滿腔}{まん|こう}の
% 満腔(まんこう)
% 全身。体中。または、心底。心から。
% 「満腔の思い」「満腔の怒り」などのように用い、情念が全身全霊を包むさまを表現する。
% 「満腔の謝意」など、心が込もっている意にも用いられる。
\ruby{燃}{も}ゆる
\ruby{忿恚}{いか|り}を
\ruby{僅}{わづか}に
\ruby{怪}{あや}しき
\ruby{侮蔑}{いや|しみ}の
\ruby{笑}{わらひ}に
\ruby{洩}{も}らして、
%
\ruby{言葉}{こと|ば}も
\ruby{無}{な}く
\ruby{突}{つ}と
\ruby{擦}{す}れ
\ruby{{\換字{違}}}{ちが}つて
\ruby{去}{さ}り
\ruby{行}{ゆ}けば、
%
\ruby{婆}{ばゞ}は
\ruby{{\換字{猶}}}{なほ}
\ruby{其}{そ}の
\ruby[<j||]{後}{うしろ}
\ruby[||j>]{姿}{すがた}を
\ruby{見{\換字{送}}}{み|おく}つて、

\原本頁{208-8}%
『
\ruby[<j>]{怖}{おつかな}い
\ruby{顏}{かほ}して
\ruby{怒}{おこ}つたつて
\ruby{無益}{だ|め}な
\ruby{事}{こん}だ。
%
そんなに
\ruby{怒}{おこ}つて
\ruby{歩}{ある}いて
\ruby{柹實}{か|き}を
\ruby{踏}{ふ}み
\ruby{潰}{つぶ}しては
ならねえだよ。
%
ハヽハヽ。
』

\原本頁{208-10}%
と、
%
\ruby{侮}{あなど}り
\ruby{笑}{わら}ひぬ。

\原本頁{208-11}%
\ruby[||j>]{面}{おもて}
を
\ruby{對}{あは}せたる
\ruby{時}{とき}にだに
\ruby{既}{すで}に
\ruby{{\換字{忍}}}{しの}びたれば、
%
\ruby{背後}{はい|ご}の
\ruby{笑}{わらひ}には
\ruby{耳}{みゝ}をも
\ruby{假}{か}さず、
%
\ruby{柹}{かき}の
\ruby{樹}{き}
\ruby{幾本}{いく|もと}の
\ruby{下}{した}を
\ruby{潜}{くぐ}りて、% 【潛 u6f5b 「先先」】【潜 u6f5c 「夫夫」】併用されている
%
\ruby{我}{わ}が
\ruby{五十子}{い|そ|こ}の
\ruby{病}{や}みて
\ruby{臥}{ふ}せる
\ruby{別室}{はな|れ}
\ruby{{\換字{近}}}{ちか}く
\ruby{到}{いた}れば、
%
\ruby{風}{かぜ}の
\ruby{騷}{さわ}がしきを
\ruby{厭}{いと}ひたりと
\ruby{見}{み}えて、
%
はや
\ruby{{\換字{戸}}}{と}を
\ruby{引}{ひ}きたるが、
%
\ruby{中}{なか}には
\ruby{燈}{ひ}の
\ruby[||j>]{光}{ひかり}
\ruby[||j>]{{\換字{弱}}}{ よわ}く
\ruby{籠}{こも}りて、
%
\ruby{人}{ひと}の
\ruby{動}{うご}ける
\ruby{影}{かげ}の
ちら〳〵としたり。

\原本頁{209-5}%
\ruby{今}{いま}までは
\ruby{先}{さき}に
\ruby{立}{た}ちて
\ruby{來}{きた}れる
\ruby{水野}{みづ|の}の、
%
\ruby{此處}{こ|ゝ}に
\ruby{至}{いた}りて
\ruby{俄}{にはか}に
\ruby{歩}{あゆ}み
\ruby{鈍}{にぶ}れば、
%
\ruby{松之助}{まつ|の|すけ}の
\ruby{方}{かた}、
%
\ruby{先}{さき}に
なりて、
%
\ruby{既}{すで}に
\ruby{沓脫}{くつ|ぬぎ}に
\ruby{一}{ひ}ト
\ruby{足}{あし}
\ruby{踏}{ふ}み
\ruby{入}{い}るゝに、
%
\ruby{水野}{みづ|の}は
\ruby{其}{そ}の
\ruby{執}{と}りたる
\ruby{手}{て}を
\ruby[||j>]{力}{ちから}
\ruby[||j>]{無}{ な}く
\ruby{放}{はな}して、
%
\ruby{續}{つゞ}いて
\ruby{入}{い}らん
ともせず
\原本頁{209-8}\改行%
\ruby{立{\換字{迷}}}{たち|まよ}ひ
\ruby{居}{ゐ}たり。

\原本頁{209-9}%
\ruby{此}{こ}の
\ruby{心得{\換字{難}}}{こゝろ|え|がた}き
\ruby{擧動}{ふる|まひ}の
\ruby{意}{こゝろ}を、
%
\ruby{松之助}{まつ|の|すけ}は
\ruby{{\換字{更}}}{さら}に
\ruby{解}{と}く
\ruby{由無}{よし|な}ければ、
%
\ruby{振顧}{ふり|かへ}りて
\ruby{此度}{こ|たび}は
\ruby{我}{わ}が
\ruby{手}{て}に
\ruby{水野}{みづ|の}の
\ruby{手}{て}を
\ruby{執}{と}り、
%
\ruby{疾}{と}く
\ruby{此方}{こな|た}へ% ルビ調整(原本通り)
\ruby{上}{あが}れよと
\ruby{眼}{め}に
\ruby{云}{い}はせて
\ruby{引張}{ひつ|ぱ}つたり。

\原本頁{210-1}%
\ruby{言}{い}はず
\ruby{語}{かた}らずの
\ruby{我}{わ}が
\ruby{誠}{まこと}の
\ruby{{\換字{情}}}{こゝろ}は、
%
\ruby{知}{し}らず
\ruby{識}{し}らずに
\ruby{他}{ひと}の
\ruby{優}{やさ}しき
\ruby{胸}{むね}に
\ruby{響}{ひゞ}きては、
%
\ruby{可憐}{か|はゆ}き
\ruby{我}{わ}が
\ruby{松之助}{まつ|の|すけ}は
\ruby{我}{われ}を
\ruby{兄}{あに}などの
やうに
\ruby{思}{おも}ひ
\ruby{做}{な}し
\ruby{取}{と}
\原本頁{210-3}\改行%
り
\ruby{做}{な}して、
%
\ruby{泣}{な}き
\ruby{顏}{がほ}に
\ruby{姊}{あね}が
\ruby{急}{きふ}を
\ruby{訴}{うつた}へに
\ruby{來}{きた}りし
それに
\ruby{釣}{つ}り
\ruby{{\換字{込}}}{こ}まれて
\改行% 校正作業の簡略化のため
、
%
\原本頁{210-4}\改行%
ハツと
\ruby{驚}{おどろ}きし
\ruby{餘}{あま}りに
\ruby{何}{なに}といふ
\ruby{考}{かんが}へも
\ruby{無}{な}く、
%
\ruby{走}{はし}り
\ruby{出}{い}でゝ
\ruby{此處}{こ|ゝ}へは
\ruby{來}{きた}りしものゝ、
%
\ruby{如何}{い|か}なる
\ruby{宿世}{しゆ|くせ}の
\ruby{仇}{あだ}の
ありてか、
%
\ruby{我}{わ}が
\ruby{五十子}{い|そ|こ}を
\ruby{思}{おも}ふ
\ruby{心}{こゝろ}の
\ruby{募}{つの}るだけに、
%
\ruby{五十子}{い|そ|こ}の
\ruby{我}{われ}を
\ruby{厭}{いと}ふ
\ruby{{\換字{情}}}{こゝろ}も
\ruby{漸}{やうや}く
\ruby{募}{つの}りて、
%
\ruby{特}{こと}に
\ruby[<j||]{病}{びやう}% 行末行頭の境界付近なので特例処置を施す
\ruby[||j>]{氣}{き}の
\ruby{爲}{さ}する
\ruby{癇}{かん}の
\ruby{{\換字{所}}爲}{わ|ざ}とは
\ruby{云}{い}へ、
%
\ruby{此}{こ}の
\ruby{頃}{ごろ}は
\ruby{我}{わ}が
\ruby{面}{おもて}を
\ruby{見}{み}るをさ
\ruby[<g>]{へ甚}{はなはだ}しく% 行末行頭の境界付近なので特例処置を施す
\ruby{忌}{い}み
\ruby{{\換字{嫌}}}{きら}ふやうになり
\ruby{居}{を}れるなれば、
%
\ruby{我}{われ}は
こそ
\ruby{其}{そ}の
\ruby{人}{ひと}の
\ruby{傍}{そば}に
\ruby{在}{あ}りて
\ruby{兎}{と}も
\ruby{角}{かく}も
なるを
\ruby{見果}{み|はて}んと
\ruby{願}{ねが}へ、
%
\ruby{今}{いま}
その
\ruby{病狀}{やう|す}の
\ruby{凶}{あし}き
\ruby{盛}{さか}りに
\原本頁{210-10}\改行%
\ruby{我}{わ}が
\ruby{面}{おもて}を
\ruby{見}{み}せて、
%
その
\ruby{人}{ひと}に
\ruby[<j>]{快}{こゝろよ}からぬ
\ruby{思}{おもひ}させんことは、
%
たとへば
また
\ruby{復}{ふたゝ}び
\ruby{戀}{こひ}しき
\ruby{人}{ひと}の
\ruby{此}{こ}の
\ruby{世}{よ}の
\ruby{顏}{かほ}を
\ruby{見}{み}るを
\ruby{得}{え}ざるに
\ruby{至}{いた}らん
\ruby{其}{そ}の
\ruby{悲}{かな}しさは、
%
\ruby{能}{よ}く
\ruby{{\換字{忍}}}{しの}ぶべし
とするも、
%
これは
\ruby{{\換字{忍}}}{しの}ぶに
\ruby{{\換字{忍}}}{しの}びがたき
とこ
\原本頁{211-2}\改行%
ろなり。
%
\ruby{特}{こと}に
われは
\ruby{死}{し}を
\ruby{起}{おこ}し
\ruby{生}{せい}を
\ruby{囘}{かへ}すの% 原本通り「囘」
\ruby{{\換字{道}}}{みち}を
\ruby{知}{し}れるにも
あらず
\改行% 校正作業の簡略化のため
、
%
\原本頁{211-3}\改行%
また
\ruby{我}{わ}が
\ruby{岩崎}{いは|さき}% 原本のこの部分は「いわさき」
\ruby{氏}{うぢ}に
\ruby{何}{なん}の
\ruby{因緣}{ゆ|かり}もあるにもあらず、
%
\ruby{云}{い}はゞ
\ruby{赤}{あか}の
\ruby{他人}{た|にん}
の
\ruby{身}{み}をもて、
%
\ruby{然}{さ}らぬだに
\ruby{生}{い}くる
\ruby{死}{し}ぬるの
\ruby{境}{さかひ}に
\ruby{惱}{なや}める
\ruby{人}{ひと}の
\ruby{枕頭}{まくら|べ}に
\原本頁{211-5}\改行%
\ruby{見}{あらは}れて、
%
\ruby{其}{そ}の
\ruby{人}{ひと}に
\ruby{忌}{いま}はしき
\ruby{思}{おもひ}を
さするほかには
\ruby{何}{なん}の
\ruby{能}{のう}も
\ruby{無}{な}き
\ruby[<j||]{面}{おもて}
\原本頁{211-6}\改行%
を
\ruby{差}{さ}し
\ruby{出}{だ}さん
\ruby{心無}{こゝろ|な}さは、
%
\ruby{我}{わが}
\ruby{爲}{な}し
\ruby{得}{う}べきところならんや。
%
\ruby{痩}{や}せたる
\ruby{其}{そ}の
\ruby{人}{ひと}の
\ruby{手}{て}をも
\ruby{執}{と}り、
%
\ruby{冷}{ひ}えんとする
\ruby{其}{その}
\ruby{人}{ひと}の
\ruby{身}{み}をも
\ruby{溫}{あたゝ}めて、
%
\ruby{及}{およ}ばぬまでも
\ruby[||j>]{心}{こゝろ}
\ruby[||j>]{限}{ かぎ}りの
% \ruby{心限}{こゝろ|かぎ}りの
\ruby{介抱}{かい|はう}を
\ruby{仕}{し}たき
\ruby{望}{のぞみ}は
\ruby{熾盛}{さか|ん}なれども、
%
\ruby{因緣}{いん|ねん}の
\ruby{恨}{うら}めしくも
\ruby{悲}{かな}しくも
\ruby{厭}{いと}ひ
\ruby{{\換字{嫌}}}{きら}はれたる
\ruby{身}{み}の
\ruby{其}{それ}も
\ruby{叶}{かな}はず、
%
たゞ
\ruby{{\換字{戸}}}{と}の
\ruby{外}{そと}に
\ruby{泣}{な}き
\ruby{惑}{まど}ひて、
%
あだに
\ruby{物}{もの}を
\ruby{思}{おも}ひ
\ruby{心}{こゝろ}を
\ruby{苦}{くるし}めん
ためばかりに
\ruby{此處}{こ|ゝ}に
\原本頁{211-11}\改行%
\ruby{來}{きた}りし
\ruby{冥利}{みや|うり}の
\ruby{拙}{つたな}さ!、
%
\ruby{我}{わ}が
\ruby{愚}{おろか}さ!。
%
\ruby{思}{おも}へば
\ruby{何}{なん}とせん
\ruby{意}{こゝろ}にて
\ruby{此處}{こ|ゝ}に
\ruby{走}{はし}りては
\ruby{來}{きた}りしぞや。
%
\ruby{甲{\換字{斐}}}{か|ひ}なくも
\ruby{甲{\換字{斐}}}{か|ひ}
\ruby{無}{な}く
\ruby{氣}{き}を
\ruby{揉}{も}みて、
%
たゞ% ルビ調整(原本通り)踊り字表記(行末行頭の境界付近)
たゞ
\ruby{亂}{みだ}れて
\ruby{絲}{いと}の
\ruby{如}{ごと}き
\ruby{思}{おもひ}に、
%
\ruby{獨}{ひと}り
\ruby{泣}{な}くより
ほかには
\ruby{爲}{な}すべき
\ruby{我}{わ}が
\ruby{事}{こと}もあらざる
\ruby[<j||]{{\換字{情}}}{なさけ}
\ruby{無}{な}さを
\ruby{如何}{い|か}にせん。

\原本頁{212-4}%
と
\ruby{松之助}{まつ|の|すけ}の
\ruby{手}{て}を
そつと
\ruby{拂}{はら}つて、
%
\ruby{面}{おもて}を
かくしつゝ
\ruby{逸}{そ}れたる
\ruby{水野}{みづ|の}は
\改行% 校正作業の簡略化のため
、
%
\原本頁{212-5}\改行%
\ruby{家}{いへ}の
\ruby{背後}{うし|ろ}の
\ruby{椎}{しひ}の
\ruby{老樹}{おい|き}の
\ruby{幹}{みき}に
\ruby{頭}{かうべ}を
\ruby{埋}{うづ}めて、
%
こんもりとしたる
\ruby{其}{その}
\ruby{陰}{かげ}には、
%
はや
\ruby{夕闇}{ゆふ|やみ}の
\ruby{逼}{せま}りて
\ruby{昏}{くら}くなれるが
\ruby{中}{なか}に
\ruby{立盡}{たち|つく}せり。

\原本頁{212-7}%
\ruby{風}{かぜ}は
\ruby{{\換字{猶}}}{なほ}
\ruby{吹}{ふ}けど
やゝ
\ruby{衰}{おとろ}へて、
『
\ruby{四十七士}{し|じふ|しち|し}の
\ruby{墓}{はか}どころ、
%
\ruby{{\換字{雪}}}{ゆき}は
\ruby{{\換字{消}}}{き}えても
%
\原本頁{212-9}\改行%
\ruby{名}{な}は
\ruby{殘}{のこ}る、
』
% 鉄道唱歌 東海道篇 二 の一部から
% 右は高輪泉岳寺           / 四十七士の墓どころ       / 雪は消えても消えのこる   / 名は千載の後までも
% みぎはたかなわせんがくじ / しじふしちしのはかどころ / ゆきはきえてもきえのこる / なはせんざいののちまでも
と、
%
\ruby{村}{むら}の
\ruby{兒}{こ}が
\ruby{{\換字{遠}}方}{とほ|く}にて
\ruby{唱}{うた}ふ
\ruby{金切聲}{かな|きり|ごゑ}の
\ruby{幽}{かすか}に
\ruby{聞}{きこ}えくるも
\原本頁{212-10}\改行%
\ruby{時}{とき}に
\ruby{取}{と}りて
\ruby{忌}{いま}はしく、
%
\ruby{塒}{ねぐら}に
\ruby{急}{いそ}ぐ
\ruby{歸}{かへ}り
\ruby{鴉}{がらす}の
\ruby{二三羽}{に|さん|ば}
\ruby{鳴}{な}きつれたるも
\ruby{耳立}{みゝ|だ}つて
\ruby{淋}{さび}しく、
%
\ruby{其}{その}
\ruby{後}{ゝち}は
\ruby{物音}{もの|おと}も
\ruby{無}{な}く
\ruby{日}{ひ}は
\ruby{暮}{く}れんとす。
