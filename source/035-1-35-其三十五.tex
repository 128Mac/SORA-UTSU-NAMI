\Entry{其三十五}

% メモ 校正終了 2024-04-11 2024-05-28
\原本頁{213-2}%
\ruby{我}{われ}は
\ruby{今}{いま}
\ruby{何}{なに}として
\ruby{來}{きた}りけん
\ruby{我}{われ}
\ruby{知}{し}らず、
%
\ruby{我}{われ}は
\ruby{今}{いま}
\ruby{何}{なに}となさば
\ruby{宜}{よ}からん
\ruby{我}{われ}
\ruby{知}{し}らず、
%
\ruby{我}{われ}は
たゞ
\ruby{此處}{こ|ゝ}に
\ruby{來}{こ}では
\ruby{叶}{かな}はざるやう
\ruby{思}{おも}ひて
\ruby{此處}{こ|ゝ}に
\ruby{來}{きた}り、
%
\ruby{我}{われ}は
たゞ
\ruby{此處}{こ|ゝ}を
\ruby{去}{さ}りがたき
\ruby{心地}{こゝ|ち}するばかりに
\ruby{此處}{こ|ゝ}に
\ruby{在}{あ}るなり、
%
\ruby{來}{きた}れるが
\ruby{他}{ひと}の
\ruby{益}{やく}にも
\ruby{立}{た}たず、
%
\ruby{在}{あ}るが
\ruby{思}{おも}ひの
\ruby{晴}{は}るゝ
\ruby{業}{わざ}にも
あらざるを、
%
\ruby{女々}{め|ゝ}しくも
\ruby{男兒}{をと|こ}らしからぬ
\ruby{振舞}{ふる|まひ}をするかな!。
%
\ruby[<j||]{愚}{おろか}な
\原本頁{213-7}\改行%
りとも
\ruby{日頃}{ひ|ごろ}の
\ruby{我}{われ}は
\ruby{如是}{か|く}は
あらざりしものを、
%
\ruby{意氣地}{い|く|ぢ}
\ruby{無}{な}くも
\ruby{崩折}{くづ|を}れたる
\ruby{心}{こゝろ}の
\ruby{何}{なに}を
\ruby{待}{ま}てるぞや!。
%
\ruby{醫藥}{い|やく}の
\ruby{力}{ちから}は
\ruby{限}{かぎり}あり、
%
\ruby[<j||]{定}{ぢやう}
\ruby[||j>]{命}{みやう}
は
\ruby{如何}{いか|ん}とも
\ruby{爲}{な}しがたければ、
%
その
\ruby{人}{ひと}の
\ruby{魂魄}{た|ま}の
\ruby[||j>]{{\換字{情}}}{なさけ}
\ruby[||j>]{無}{ な }くも
\ruby{天}{そら}に
\ruby{去}{さ}つて、
%
\ruby{松之助}{まつ|の|すけ}の
\ruby{泣聲}{なき|ごゑ}の
わつと
\ruby{起}{おこ}らん
\ruby{時}{とき}、
%
\ruby{我}{われ}は
\ruby{其}{そ}の
\ruby{聲}{こゑ}を
\ruby{聞}{き}いて
\ruby{世}{よ}を
\ruby{思}{おも}い
\ruby{切}{き}
\原本頁{214-1}\改行%
り、
%
\ruby{此}{こ}の
\ruby{椎}{しひ}の
\ruby{幹}{みき}の
\ruby{岩}{いは}の
ごときに、
%
\ruby{額}{ひたひ}を
\ruby{打付}{うち|つ}け
\ruby{頭顱}{なづ|き}を
\ruby{破}{わ}つて、
%
よ
\原本頁{214-2}\改行%
しや
\ruby{身}{み}は
\ruby{輪{\換字{廻}}}{りん|ね}の
\ruby{闇}{やみ}に
\ruby{{\換字{迷}}}{まよ}ひ
\ruby{入}{い}るとも、
%
\ruby[g]{一念}{おもひ}は
\ruby{芳魂}{はう|こん}の
\ruby{行方}{ゆく|へ}を
\ruby{{\換字{追}}}{お}ひて、
%
\原本頁{214-3}\改行%
\ruby{紫雲}{し|うん}の
\ruby{{\換字{空}}}{そら}の
\ruby{遙}{はる}けくも
あれ、
%
\ruby[||j>]{黄}{こわう}
\ruby[||j>]{泉}{ せん}の
% \ruby{黄泉}{こわう|せん}の
\ruby{涯}{はて}の
\ruby{{\換字{遠}}}{とほ}くも
あれ、
%
つれなき
\ruby{風}{かぜ}
\原本頁{214-4}\改行%
の
\ruby{持}{も}て
\ruby{去}{さ}れる
\ruby{花}{はな}の
\ruby{香}{かをり}に
\ruby{引}{ひ}かされて、
%
あくがれ
\ruby{漂泊}{さま|よ}ふ
\ruby{蝶}{てふ}の
\ruby{如}{ごと}くに、
%
\原本頁{214-5}\改行%
\ruby{{\換字{飽}}}{あく}まで
\ruby{戀}{こひ}しき
\ruby{人}{ひと}に
\ruby{{\換字{伴}}}{ともな}はんとて、
%
こゝには
\ruby{{\換字{空}}}{むな}しく
\ruby{佇}{たゝず}める
\ruby{歟}{か}。
%
\ruby{或}{ある}は
\原本頁{214-6}\改行%
\ruby{{\換字{又}}}{また}
\ruby{{\換字{強}}}{つよ}く
\ruby{忌}{い}み
\ruby{{\換字{嫌}}}{きら}はれたるより、
%
\ruby{堪}{た}へがたき
\ruby{苦悶}{も|だえ}に
\ruby{自}{みづか}ら
\ruby{堪}{た}へて、
%
\ruby{其}{その}
\ruby{人}{ひと}に
\ruby{{\換字{近}}}{ちか}づきもせず
\ruby{{\換字{過}}}{すご}し% 国会図書館では「すご」、国文学研究資料館では「 ご」
\ruby{居}{ゐ}けるが、
%
\ruby{{\換字{若}}}{も}し
\ruby{不幸}{ふ|かう}にして
\ruby{其}{そ}の
\ruby{{\換字{遠}}慮}{ゑん|りよ}の
\ruby[<j||]{俄}{にはか}に% 行末行頭の境界付近なので特例処置を施す
\ruby{失}{う}すべき
\ruby{時}{とき}にも
\ruby{至}{いた}らば、
%
\ruby{先}{ま}ず
\ruby{枕}{まくら}の
\ruby{邊}{ほとり}に
\ruby{走}{はし}り
\ruby{寄}{よ}つて、
%
\ruby{我}{わ}が
\ruby{火}{ひ}と
\原本頁{214-9}\改行%
\ruby{熱}{あつ}き
\ruby{萬石}{ばん|こく}の
\ruby{涙}{なみだ}を、
%
せめては
\ruby{其}{そ}の
\ruby{冷}{つめた}き
\ruby{骸}{かばね}に
\ruby{親}{した}しく
\ruby{濺}{そゝ}ぎ、% 国文学研究資料館のは印字不鮮明で判読できず、国会図書館のを採用。
%
\ruby{{\換字{情}}}{つれ}
\ruby{無}{な}かりし
\ruby{其}{そ}の
\ruby{人}{ひと}の
\ruby{手}{て}を
\ruby{執}{と}り
\ruby{搖}{ゆさ}ぶりて、
%
\ruby{心}{こゝろ}ゆく
ばかり
\ruby{號哭}{がう|こく}せんとて
\ruby{此處}{こ|ゝ}には
\ruby{居}{ゐ}るにや。
%
それにもあらねば、
%
これにもあらず、
%
\ruby{何}{なに}せん
\ruby[<j||]{心}{こゝろ}は% 行末行頭の境界付近なので特例処置を施す
\ruby{{\換字{更}}}{さら}に
\ruby{無}{な}くして、
%
\ruby{我}{われ}にも
\ruby{我}{われ}の
\ruby{解}{わか}らぬ
\ruby{感想}{おも|ひ}に、
%
たゞ
\ruby{此處}{こ|ゝ}を
\ruby{去}{さ}りかねて
\ruby{水野}{みづ|の}は
\ruby{{\換字{猶}}}{なほ}
\ruby{立}{た}てり。

\原本頁{215-3}%
\ruby{暮}{く}るゝに
\ruby{{\換字{連}}}{つ}れて
\ruby{風}{かぜ}は
\ruby{收}{をさ}まり、
%
\ruby{闇}{やみ}は
\ruby{葉}{は}の
\ruby{密}{こ}みたる
\ruby{椎}{しひ}の
\ruby{{\換字{梢}}}{こずゑ}より
\ruby{廣}{ひろ}がつて、
%
\ruby{{\換字{終}}}{つひ}に
\ruby{其}{その}
\ruby{黑}{くろ}き
\ruby[<j>]{懷}{ふところ}の
\ruby{中}{うち}に
\ruby{四邊}{あた|り}を
\ruby{包}{つゝ}みぬ。

\原本頁{215-5}%
\ruby{森々}{しん|〳〵}と
\ruby{靜}{しづか}なる
\ruby{此}{こ}の
\ruby{日}{ひ}
\ruby{此}{こ}の
\ruby[||j>]{{\換字{宵}}}{ゆふべ}
\ruby[||j>]{天}{ てん}に
\ruby{星}{ほし}
\ruby{無}{な}し、
%
\ruby{星}{ほし}は
\ruby{死}{し}したるならん、
%
\ruby{地}{ち}に
\ruby{風}{かぜ}は
\ruby{{\換字{弱}}}{よわ}りぬ、
%
\ruby{風}{かぜ}は
\ruby{今}{いま}
おのが
\ruby{墓穴}{はか|あな}を
\ruby{{\換字{尋}}}{たづ}ねて
\ruby{永}{なが}く
\ruby{休}{やす}まんとせり。
%
\原本頁{215-7}\改行%
\ruby{{\換字{古}}}{ふ}りたる
\ruby{椎}{しひ}の
\ruby{木}{き}は
\ruby{忽然}{こつ|ぜん}として
\ruby{人}{ひと}の
\ruby{聲}{こゑ}をなし、

\原本頁{215-8}%
『
\ruby{衆生被困厄}{しゆ|じやう|び|こん|やく}、
%
\ruby{無量苦逼身}{む|りやう|く|ひつ|しん}、
%
\ruby{觀音妙智力}{くわん|のん|めう|ち|りき}、% 「觀音」の読みは原本通り「くわん(の)ん」
%
\ruby{能救世間苦}{のう|ぐ|せ|けん|く}、
』

\原本頁{215-9}%
と
\ruby{囁}{さゝや}くが
\ruby{如}{ごと}くに
\ruby{誦}{じゆ}し
\ruby{出}{いだ}せり。

\原本頁{215-10}%
\ruby{椎}{しひ}の
\ruby{那處}{いづ|く}に
\ruby{彼}{か}の
\ruby[||j>]{額}{ひたひ}
\ruby[||j>]{廣}{ ひろ}く
% \ruby{額廣}{ひたひ|ひろ}く
\ruby{鼻細}{はな|ほそ}き
\ruby{老}{お}いたる
\ruby{男}{をとこ}の
\ruby{潛}{ひそ}み% 【潛 u6f5b 「先先」】【潜 u6f5c 「夫夫」】併用されている
\ruby{居}{を}れりや、
%
\ruby{聲}{こゑ}は
\ruby{全}{まつた}く
\ruby{其}{そ}の
\ruby{聲}{こゑ}なりけり。

\原本頁{216-1}%
\ruby{愚}{おろか}なり!、
%
こは
\ruby{我}{わ}が
\ruby{招}{よ}ばずして
\ruby{我}{わ}が
\ruby{記臆}{き|おく}の
\ruby{現}{あらは}れ
\ruby{來}{きた}れるには
\ruby{{\換字{過}}}{す}ぎざるものをと
\ruby{水野}{みづ|の}が
\ruby{冷}{ひや}やかに
\ruby{聞}{き}きし
\ruby{時}{とき}は、
%
\ruby{其}{その}
\ruby{聲}{こゑ}は
\ruby{既}{はや}
\ruby{失}{う}せて
\ruby{{\換字{遺}}響}{ひゞ|き}も
\ruby{無}{な}かりしが、
%
\ruby{當時}{その|とき}
\ruby{椎}{しひ}の
\ruby{大木}{おほ|き}は
\ruby{忽}{たちま}ち
\ruby{二}{ふた}つに
\ruby{裂}{さ}けて、
%
\ruby{其處}{そ|こ}に
\ruby{明}{あき}らかなる
\ruby{世界}{せ|かい}の
\ruby{朗}{ほが}らかに
\ruby{現}{あらは}れたるが
\ruby{中}{うち}に、
%
\ruby{年齡}{と|し}は
\ruby{二十四五}{に|じふ|し|ご}なる
\ruby[<j||]{男}{をとこ}の% 行末行頭の境界付近なので特例処置を施す
\ruby{戀}{こひ}に
\ruby{窶}{やつ}れたる
\ruby{顏}{かほ}の
\ruby{勇威}{いき|ほひ}
\ruby{無}{な}く
\ruby{光釆}{ひか|り}
\ruby{無}{な}く、
%
\ruby{五月雨}{さ|み|だれ}の
\ruby{檐}{のき}の
\ruby{雫}{しづく}と
\ruby{涙}{なみだ}を
\原本頁{216-6}\改行%
\ruby{放}{はふ}らし
\ruby{落}{おと}し
\ruby{居}{を}れるさまの
\ruby{醜}{みにく}くも
\ruby{醜}{みにく}きを、
%
\ruby{右}{みぎ}の
\ruby{肩}{かた}には
\ruby{恐}{おそ}ろしき
\ruby{猛鷲}{あら|わし}を
\ruby{宿}{と}まらしめ、
%
\ruby{後}{うしろ}には
\ruby{凄}{すさま}じき
\ruby{大蛇}{だい|じや}を
\ruby{隨}{したが}へたる
\ruby{氣味}{き|み}
\ruby{惡}{あ}しき
\ruby[<j||]{大}{おほ }
\ruby[<j||]{男}{をとこ}
% \ruby{大男}{おほ|をとこ}
\原本頁{216-8}\改行%
の、
%
\ruby{神}{かみ}に
\ruby{似}{に}て
\ruby{神}{かみ}の
\ruby{威}{ゐ}
\ruby{無}{な}く、
%
\ruby{人}{ひと}かと
\ruby{見}{み}れば
\ruby{人}{ひと}らしからぬが、
%
\ruby{憐}{あはれ}む
\原本頁{216-9}\改行%
が
\ruby{如}{ごと}く
\ruby{侮}{あなど}るが
\ruby{如}{ごと}き
\ruby{眼}{め}して
\ruby{見詰}{み|つ}め
\ruby{居}{ゐ}たるが
\ruby{{\換字{分}}明}{あり|〳〵}と
\ruby{見}{み}えぬ。
