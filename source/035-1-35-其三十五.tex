\Entry{其三十五}

\ruby{我}{われ}は
\ruby{今何}{いま|なん}として
\ruby{來}{きた}りけん
\ruby{我知}{われ|し}らず、
\ruby{我}{われ}は
\ruby{今何}{いま|なん}となさば
\ruby{宣}{よ}からん
\ruby{我知}{われ|し}らず、
\ruby{我}{われ}はたゞ
\ruby{此處}{こ|ゝ}に
\ruby{來}{こ}では
\ruby{叶}{かな}はざるやう
\ruby{思}{おも}ひて
\ruby{此處}{こ|ゝ}に
\ruby{來}{きた}り、
\ruby{我}{われ}はたゞ
\ruby{此處}{こ|ゝ}を
\ruby{去}{さ}りがたき
\ruby{心地}{こヽ|ち}するばかりに
\ruby{此處}{こ|ゝ}に
\ruby{在}{あ}るなり、
\ruby{來}{きた}れるが
\ruby{他}{ひと}の
\ruby{{\換字{益}}}{やく}にも
\ruby{立}{た}たず、
\ruby{在}{あ}るが
\ruby{思}{おも}ひの
\ruby{晴}{は}るゝ
\ruby{業}{わざ}にもあらざるを、
\ruby{女々}{め|ゝ}しくも
\ruby{男兒}{をと|こ}らしからぬ
\ruby{振舞}{ふる|まひ}をするかな!。
\ruby{愚}{おろか}かなりとも
\ruby{日頃}{ひ|ごろ}の
\ruby{我}{われ}は
\ruby{如是}{か|く}はあらざりしものを、
\ruby{意気地無}{い|く|じ|な}くも
\ruby{崩折}{くづ|を}れたる
\ruby{心}{こヽろ}の
\ruby{何}{なに}を
\ruby{待}{ま}てるぞや!。
\ruby{醫藥}{い|やく}の
\ruby{力}{ちから}は
\ruby{限}{かぎり}あり、
\ruby{定命}{ぢやう|みやう}は
\ruby{如何}{いか|ん}とも
\ruby{爲}{な}しがたければ、その
\ruby{人}{ひと}の
\ruby{魂魄}{た|ま}の
\ruby{{\換字{情}}無}{なさけ|な}くも
\ruby{天}{そら}に
\ruby{去}{さ}つて、
\ruby{松之助}{まつ|の|すけ}の
\ruby{泣聲}{なき|ごゑ}のわつと
\ruby{起}{おこ}らん
\ruby{時}{とき}、
\ruby{我}{われ}は
\ruby{其}{そ}の
\ruby{聲}{こゑ}を
\ruby{聞}{き}いて
\ruby{世}{よ}を
\ruby{思}{おも}い
\ruby{切}{き}り、
\ruby{此}{こ}の
\ruby{椎}{しひ}の
\ruby{幹}{みき}の
\ruby{岩}{いは}のごときに、
\ruby{額}{ひたひ}を
\ruby{打付}{うち|つ}け
\ruby{頭顱}{なづ|き}を
\ruby{破}{わ}つて、よしや
\ruby{身}{み}は
\ruby{輪廻}{りん|ね}の
\ruby{闇}{やみ}に
\ruby{{\GWI{u8ff7-k}}}{まよ}ひ
\ruby{入}{い}るとも、
\ruby{一念}{おも|ひ}は
\ruby{芳魂}{はう|こん}の
\ruby{行方}{ゆく|へ}を
\ruby{{\GWI{u8ffd-k}}}{お}ひて、
\ruby{紫雲}{し|うん}の
\ruby{空}{そら}の
\ruby{遙}{はる}けくもあれ、
\ruby{黄泉}{こわう|せん}の
\ruby{涯}{はて}の
\ruby{{\GWI{u9060-k}}}{とほ}くもあれ、つれなき
\ruby{風}{かぜ}の
\ruby{持}{も}て
\ruby{去}{さ}れる
\ruby{花}{はな}の
\ruby{香}{かをり}に
\ruby{引}{ひ}かされて、あくがれ
\ruby{漂泊}{さま|よ}ふ
\ruby{蝶}{てふ}の
\ruby{如}{ごと}くに、
\ruby{{\GWI{u98fd-k}}}{あく}まで
\ruby{戀}{こひ}しき
\ruby{人}{ひと}に
\ruby{伴}{ともな}はんとて、こゝには
\ruby{空}{むな}しく
\ruby{佇}{たゝず}める
\ruby{歟}{か}。
或は
\ruby{{\換字{叉}}{\換字{強}}}{また|つよ}く
\ruby{忌}{い}み
\ruby{{\換字{嫌}}}{きら}はれたるより、
\ruby{堪}{た}へがたき
\ruby[g]{苦悶}{もだえ}に
\ruby{自}{みずか}ら
\ruby{堪}{た}へて、
\ruby{其人}{その|ひと}に
\ruby{{\換字{近}}}{ちか}づきもせず
\ruby{{\GWI{u904e-k}}}{そご}し
\ruby{居}{ゐ}けるが、
\ruby{若}{も}し
\ruby{不幸}{ふ|かう}にして
\ruby{其}{そ}の
\ruby{{\GWI{u9060-k}}慮}{ゑん|りよ}の
\ruby{俄}{にはか}に
\ruby{失}{う}すべき
\ruby{時}{とき}にも
\ruby{至}{いた}らば、
\ruby{先}{ま}ず
\ruby{枕}{まくら}の
\ruby{邊}{ほとり}に
\ruby{走}{はし}り
\ruby{寄}{よ}つて、
\ruby{我}{わ}が
\ruby{火}{ひ}と
\ruby{熱}{あつ}き
\ruby{萬石}{ばん|こく}の
\ruby{淚}{なみだ}を、せめては
\ruby{其}{そ}の
\ruby{冷}{つめた}き
\ruby{骸}{かばね}に
\ruby{親}{した}しく
\ruby{濺}{そゝ}ぎ、
\ruby{{\換字{情}}無}{つれ|な}かりし
\ruby{其}{そ}の
\ruby{人}{ひと}の
\ruby{手}{て}を
\ruby{執}{と}り
\ruby{搖}{ゆさ}ぶりて、
\ruby{心}{こヽろ}ゆくばかり
\ruby{號哭}{がう|こく}せんとて
\ruby{此處}{こ|ゝ}には
\ruby{居}{ゐ}るにや。
それにもあらねば、これにもあらず、
\ruby{何}{なに}せん
\ruby{心}{こヽろ}は
\ruby{更}{さら}に
\ruby{無}{な}くして、
\ruby{我}{われ}にも
\ruby{我}{われ}の
\ruby{解}{わか}らぬ
\ruby[g]{感想}{おもひ}に、たゞ
\ruby{此處}{こ|ゝ}を
\ruby{去}{さ}りかねて
\ruby[g]{水野}{みづの}は
\ruby{{\換字{猶}}}{なほ}
\ruby{立}{た}てり。

\ruby{暮}{く}るゝに
\ruby{{\GWI{u9023-k}}}{つ}れて
\ruby{風}{かぜ}は
\ruby{収}{をさ}まり、
\ruby{闇}{やみ}は
\ruby{葉}{は}の
\ruby{密}{こ}みたる
\ruby{椎}{しひ}の
\ruby{{\換字{梢}}}{こずゑ}より
\ruby{廣}{ひろ}がつて、
\ruby{{\換字{終}}}{つひ}に
\ruby{其黑}{その|くろ}き
\ruby{懷}{ふところ}の
\ruby{中}{うち}に
\ruby[g]{四邊}{あたり}を
\ruby{包}{つゝ}みぬ。

\ruby{森々}{しん|〳〵}と
\ruby{靜}{しづか}なる
\ruby{此}{こ}の
\ruby{日此}{ひ|こ}の
\ruby{{\換字{宵}}天}{ゆふべ|てん}に
\ruby{星無}{ほし|な}し、
\ruby{星}{ほし}は
\ruby{死}{し}したるならん、
\ruby{地}{ち}には
\ruby{風}{かぜ}は
\ruby{{\換字{弱}}}{よわ}りぬ、
\ruby{風}{かぜ}は
\ruby{今}{いま}おのが
\ruby{墓穴}{はか|あな}を
\ruby{{\換字{尋}}}{たづ}ねて
\ruby{永}{なが}く
\ruby{休}{やす}まんとせり。
\ruby{古}{ふ}りたる
\ruby{椎}{しい}の
\ruby{木}{き}は
\ruby{忽然}{こつ|ぜん}として
\ruby{人}{ひと}の
\ruby{聲}{こゑ}をなし、

『
\ruby{衆生被困厄}{しゆ|じやう|び|こん|やく}、
\ruby{無量苦逼身}{む|りやう|く|ひつ|しん}、
\ruby{觀音妙智力}{くわん|のん|めう|ち|りき}、
\ruby{能救世間苦}{のう|ぐ|せ|けん|く}、』

と
\ruby{囁}{さゝや}くが
\ruby{如}{ごと}くに
\ruby{誦}{じゆ}し
\ruby{出}{いだ}せり。

\ruby{椎}{しい}の
\ruby[g]{那處}{いづく}に
\ruby{彼}{か}の
\ruby{額廣}{ひたひ|ひろ}く
\ruby{鼻細}{はな|ほそ}き
\ruby{老}{お}いたる
\ruby{男}{をとこ}の
\ruby{潛}{ひそ}み
\ruby{居}{を}れりや、
\ruby{聲}{こゑ}は
\ruby{全}{まつた}く
\ruby{其}{そ}の
\ruby{聲}{こゑ}なりけり。

\ruby{愚}{おろか}なり!、こは
\ruby{我}{わ}が
\ruby{招}{よ}ばずして
\ruby{我}{わ}が
\ruby{記憶}{き|おく}の
\ruby{現}{あらは}れ
\ruby{來}{きた}れるには
\ruby{{\GWI{u904e-k}}}{す}ぎざるものをと
\ruby[g]{水野}{みづの}が
\ruby{冷}{ひや}やかに
\ruby{聞}{き}きし
\ruby{時}{とき}は、
\ruby{其聲}{その|こゑ}は
\ruby{既失}{はや|う}せて
\ruby{{\GWI{u907a-k}}響}{ひゞ|き}も
\ruby{無}{な}かりしが、
\ruby{當時椎}{その|とき|しひ}の
\ruby{大木}{おほ|き}は
\ruby{忽}{たちま}ち
\ruby{二}{ふた}つに
\ruby{裂}{さ}けて、
\ruby{其處}{そ|こ}に
\ruby{明}{あき}らかなる
\ruby{世界}{せ|かい}の
\ruby{朗}{ほが}らかに
\ruby{現}{あらは}れたるが
\ruby{中}{うち}に、
\ruby{年齡}{と|し}は
\ruby{二十四五}{に|じう|し|ご}なる
\ruby{男}{をとこ}の
\ruby{戀}{こひ}に
\ruby{窶}{やつ}れたる
\ruby{顏}{かほ}の
\ruby{勇威無}{いき|ほい|な}く
\ruby{光釆}{ひか|り}
\ruby{無}{な}く、
\ruby[g]{五月雨}{さみだれ}の
\ruby{檐}{のき}の
\ruby{雫}{しづく}と
\ruby{淚}{なみだ}を
\ruby{放}{はふ}らし
\ruby{落}{おと}し
\ruby{居}{を}れるさまの
\ruby{醜}{みにく}くも
\ruby{醜}{みにく}きを、
\ruby{右}{みぎ}の
\ruby{肩}{かた}には
\ruby{恐}{おそ}ろしき
\ruby{猛鷲}{あら|わし}を
\ruby{宿}{と}まらしめ、
\ruby{後}{うしろ}には
\ruby{凄}{すさま}じき
\ruby{大蛇}{だい|じや}を
\ruby{{\GWI{u96a8-j}}}{したが}へたる
\ruby{氣味惡}{き|み|あ}しき
\ruby{大男}{おほ|をとこ}の、
\ruby{神}{かみ}に
\ruby{似}{に}て
\ruby{神}{かみ}の
\ruby{威無}{ゐ|な}く、
\ruby{人}{ひと}かと
\ruby{見}{み}れば
\ruby{人}{ひと}らしからぬが、
\ruby{憐}{あはれ}むが
\ruby{如}{ごと}く
\ruby{侮}{あなど}るが
\ruby{如}{ごと}き
\ruby{眼}{め}して
\ruby{見詰}{み|つ}め
\ruby{居}{ゐ}たるが
\ruby{分明}{あり|〳〵}と
\ruby{見}{み}えぬ。

