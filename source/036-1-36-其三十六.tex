\Entry{其三十六}

% メモ 校正終了 2024-05-28 2024-06-26
\原本頁{217-1}%
\ruby[g]{落葉}{おちば }を
\ruby{誘}{さそ}ふ
\ruby{山}{やま}
\ruby{下}{おろ}しの
\ruby{風}{かぜ}を
\ruby{其}{その}
\ruby{儘}{まゝ}なる
\ruby[g]{猛鷲}{あらわし}の
\ruby[g]{打翥}{うちはぶ}く
\ruby{音}{おと}の
\ruby{中}{うち}には、
% ここから暫くは、原本作成時の植字環境と jlreq の違いによるとおもわれる
『
\ruby[g]{神明}{か み }は
\ruby{殪}{たふ}れたり、
』
『
\ruby[g]{佛陀}{ほとけ }は
\ruby{死}{し}したり、
』といふ
\ruby{響}{ひゞき}の
\ruby{聞}{きこ}え、
%
\ruby{首}{かうべ}を
\ruby{擡}{あ}げて
\ruby[g]{蜿蜒}{う ね }る
\ruby[g]{大蛇}{だいじや}の
ざわ〳〵と
\ruby[g]{木茅}{き かや}を
\ruby{倒}{たふ}し
\ruby{行}{ゆ}く
\ruby{音}{おと}の
\ruby{中}{うち}には、
%
『
\ruby[g]{神明}{か み }は
\ruby[g]{想像}{さうざう}のみ、
』
『
\ruby[g]{佛陀}{ほとけ }は
\ruby[g]{假說}{か せつ}のみ、
』といふ
\ruby{聲}{こゑ}あり。
% ここまでは「『」「』」が複数あり、行末行頭の境界付近なものもあるため原本と若干異なる行処理が行われている

\原本頁{217-5}%
\ruby[g]{水野}{みづの }は
\ruby{自}{みづ}から
\ruby{思}{おも}はずして
\ruby{自}{おのづ}から
\ruby[g]{如是}{か く }
\ruby{想}{おも}ひ、
%
\ruby{外}{そと}に
\ruby[g]{見聞}{み きゝ}せずして
\ruby{内}{うち}に
\ruby[g]{如是}{か く }
\ruby[g]{見聞}{み き }きせる
\ruby{時}{とき}、
%
\ruby{靜}{しづか}なる
\ruby{五十子}{い|そ|こ}が
\ruby{家}{いへ}の
\ruby{方}{かた}にて、
%
かたりと
\ruby[<j||]{微}{かすか}
\原本頁{217-7}\改行%
の
\ruby[g]{物音}{ものおと}の
\ruby{仕}{し}たるを
\ruby{聞}{き}きつけ、
%
\ruby[||j>]{豁}{くわつ}
\ruby[||j>]{然}{ ぜん}として
% \ruby{豁然}{くわつ|ぜん}として
われに
\ruby{{\換字{返}}}{かへ}れば、
%
\ruby{我}{わ}が
\ruby{止}{と}め
\ruby{{\換字{途}}}{ど}
\ruby{無}{な}かりし
\ruby{涙}{なみだ}の
\ruby[g]{何時}{い つ }か
\ruby{乾}{かわ}き、
%
\ruby{我}{わ}が
\ruby{疲}{つか}れたる
\ruby{心}{こゝろ}の
\ruby[g]{何時}{い つ }か
\ruby{奮}{ふる}ひて
\改行% 校正作業の簡略化のため
、
%
\原本頁{217-9}\改行%
\ruby{倚}{よ}りかゝりたる
\ruby{椎}{しひ}の
\ruby{幹}{みき}を
\ruby{離}{はな}れ、
%
そを
\ruby[g]{背向}{そ がひ}にして
\ruby[g]{挺然}{ていぜん}と
\ruby{獨}{ひと}り
\ruby[g]{樹蔭}{こ かげ}の
\ruby{闇}{やみ}に
\ruby{立}{た}ちつ、
%
\ruby{{\換字{魔}}}{ま}の
\ruby{如}{ごと}くに
\ruby{來}{きた}り
\ruby{{\換字{魔}}}{ま}の
\ruby{如}{ごと}くに
\ruby{去}{さ}る
\ruby[g]{蝙蝠}{かはほり}の、
%
ひらひらと% ルビ調整(原本通り)非踊り字表記(行末行頭の境界付近)
\ruby{{\換字{梢}}}{こずゑ}の
\ruby[g]{盡頭}{はずれ }を
\ruby{飛}{と}びかへれるを、
%
\ruby[g]{雲{\換字{透}}}{くもずき}に
\換字{志゛}つと% 「志」+「濁点」
\ruby[g]{打見}{うちみ }やりたり。

\原本頁{218-1}%
\ruby{有}{あ}りや
\ruby[||j>]{神}{かみ}
\ruby[||j>]{佛}{ほとけ}の?、
% \ruby{神佛}{かみ|ほとけ}の?、
%
\ruby{有}{あ}るにも
\ruby{似}{に}たるかな!。
%
\ruby{無}{な}しや
\ruby[||j>]{神}{かみ}
\ruby[||j>]{佛}{ほとけ}の?、
% \ruby{神佛}{かみ|ほとけ}の?、
%
\ruby{無}{な}きにも
\ruby{似}{に}たるかな!。

\原本頁{218-3}%
\ruby{有}{あ}るには
\ruby{無}{な}きの
\ruby[<j>]{疑}{うたがひ}あり、
%
\ruby{無}{な}きには
\ruby{有}{あ}るの
\ruby[<j>]{疑}{うたがひ}あり、
%
\ruby{有}{あ}りとも
\ruby[g]{爲{\換字{難}}}{し がた}く、
%
\ruby{無}{な}しとも
\ruby[g]{爲{\換字{難}}}{し がた}し。
%
\ruby[g]{有無}{う む }の
いづれは
\ruby{今}{いま}
\ruby{知}{し}らねども、
%
\ruby{世}{よ}に
\ruby{無}{な}き
\ruby{方}{かた}の
\ruby[g]{眞實}{ま こと}ならば、% ルビ調整(原本通り)「眞實(まこと)」
%
\ruby[g]{男兒}{をとこ }の
\ruby{頭}{かうべ}を
\ruby{下}{さ}げて
\ruby[g]{祈願}{きぐわん}を
\ruby{捧}{さゝ}げんことの
\ruby{羞}{はづか}しくも
\ruby[g]{口惜}{くちを }しく、
%
\ruby{{\換字{若}}}{も}し
\ruby{世}{よ}に
\ruby{在}{おは}す
\ruby{事}{こと}の
\ruby{定}{ぢやう}ならば、
%
\ruby{身}{み}をも
\ruby[g]{魂魄}{たましひ}をも
\ruby[g]{犠牲}{いけにへ}にして、
%
\ruby[||j>]{廣}{くわう}
\ruby[||j>]{大}{ だい}の
% \ruby{廣大}{くわう|だい}の
\ruby{御慈悲}{おん|じ|ひ}を
\ruby{頼}{たの}み
\ruby[<j>]{奉}{たてまつ}らんと
\ruby{思}{おも}ふ
\ruby{此}{こ}の
\ruby[g]{人間}{ひ と }の
\ruby{心}{こゝろ}のみぞ
\原本頁{218-8}\改行%
\ruby{僞}{いつは}り
\ruby{無}{な}き
\ruby[g]{眞實}{まこと }なる!。% ルビ調整(原本通り)「眞實(まこと)」
%
\ruby{二}{ふ}タ
\ruby{路}{みち}かけて
\ruby[g]{取舎}{しゆしや}し
わづらひつゝ、
%
\ruby{利}{よき}に
\ruby{就}{つ}かんとする
\ruby{此}{こ}の
\ruby[g]{{\換字{分}}別}{ふんべつ}の
\ruby{醜}{みにく}さよ、
%
\ruby[g]{智慧}{ち ゑ }の
\ruby[g]{狡猾}{かしこ }さよ!。
%
あゝ
\ruby[g]{人間}{ひ と }
\原本頁{218-10}\改行%
は
\ruby[g]{卑劣}{さ も }しくも
\ruby{怯}{きたな}き
\ruby{心}{こゝろ}を
\ruby{有}{も}てるかな!。
%
されど
\ruby{此}{こ}の
\ruby{疑}{うたが}ひ
\ruby{惑}{まど}ひて
\ruby[<j||]{苦}{くるし}めるこそは、
%
\ruby[g]{人間}{ひ と }の
\ruby{僞}{いつは}り
\ruby{無}{な}き
\ruby[g]{眞實}{まこと }の% ルビ調整(原本通り)「眞實(まこと)」
\ruby[g]{{\換字{情}}狀}{さ ま }
なるべけれ。
%
\ruby{我}{われ}
こゝに
\ruby{在}{あ}り、
%
われ
こゝに
\ruby{思}{おも}ふ。
%
\ruby{思}{おも}はるゝものゝ
\ruby{有}{あ}り
\ruby{無}{な}しは
\ruby{定}{さだ}かならず
\改行% 校正作業の簡略化のため
、
%
\原本頁{219-2}\改行%
\ruby{思}{おも}ふ
\ruby{我}{われ}の
\ruby{在}{あ}る
\ruby{事}{こと}が
\ruby[g]{眞實}{まこと }なるのみ。% ルビ調整(原本通り)「眞實(まこと)」
%
\ruby[g]{菩薩}{ぼ さつ}の
\ruby[g]{言葉}{ことば }、
%
\ruby{鷲}{わし}の
\ruby[g]{言葉}{ことば }、
%
\ruby[g]{妙典}{めうてん}の
\ruby{敎}{をしへ}、
%
\ruby[g]{大蛇}{だいじや}の
\ruby{敎}{をしへ}、
%
\ruby{我}{われ}に
いづれを
\ruby{取}{と}り
\ruby[g]{那方}{いづれ }を
\ruby{捨}{す}つる
\ruby[||j>]{力}{ちから}
\ruby[||j>]{無}{ な}し、
%
たゞ
\ruby[g]{那方}{いづれ }をも
\ruby{取}{と}り
\ruby{惱}{なや}み、
%
また
いづれをも
\ruby{捨}{す}て
\ruby{惱}{なや}む
\ruby{其}{その}
\ruby{事}{こと}のみぞ
\ruby{我}{わ}が
\ruby[g]{眞{\換字{情}}}{ま こと}なる!。% ルビ調整(原本通り)「眞情(まこと)」
%
\ruby[g]{神明}{か み }
\ruby[g]{佛陀}{ほとけ }を
\ruby{頼}{たの}み
\ruby[<j>]{奉}{たてまつ}りたき
\ruby[g]{心地}{こゝち }のするも、
\ruby{我}{わ}が
\ruby{欺}{あざむ}かぬ
\原本頁{219-6}\改行%
\ruby[g]{眞{\換字{情}}}{ま こと}なり、% ルビ調整(原本通り)「眞情(まこと)」
%
\ruby[g]{神明}{か み }
\ruby[g]{佛陀}{ほとけ }をも
\ruby{肯}{うけが}はずして、
%
\ruby[g]{智慧}{ち ゑ }の
\ruby[g]{鋼鐵}{はがね }の
\ruby{{\換字{杖}}}{つゑ}に
\ruby{頼}{よ}つて
\ruby{此}{こ}の
\ruby[g]{戰鬪}{たゝかひ}の
\ruby{世}{よ}に
\ruby{立}{た}たんとするも
\ruby{我}{わ}が
\ruby{欺}{あざむ}かぬ
\ruby[g]{眞{\換字{情}}}{ま こと}なり、% ルビ調整(原本通り)「眞情(まこと)」
%
\ruby{獸}{けもの}にもあらず
\ruby{鳥}{とり}にもあらで、
%
\ruby[g]{光明}{ひかり }の
\ruby{國}{くに}
\ruby[g]{黑闇}{や み }の
\ruby{國}{くに}の
\ruby{境}{さかひ}を
\ruby{飛}{と}ぶ
\ruby{彼}{あ}の
\ruby[g]{{\換字{魔}}魅}{ま もの}の
\ruby{如}{ごと}き
\ruby[g]{蝙蝠}{かはほり}の、
%
\ruby{世}{よ}にも
\ruby{厭}{いと}はしく
\ruby{醜}{みにく}きは、
%
\ruby{我}{わ}が
\ruby{胸}{むね}の
\ruby{中}{うち}の
\makeatletter
\@ifundefined{デバッグ@ビルド}{%
  \ruby[<j||]{怪}{くわい}
  \ruby[||j>]{物}{ぶつ}の、
}{%
  \ruby[||j>]{怪}{くわい}
  \ruby[||j>]{物}{ ぶつ}の、
}%
\makeatother
% \ruby{怪物}{くわい|ぶつ}の、
%
\ruby{化}{な}りて
\ruby{出}{い}でしかとも
\ruby{思}{おも}はれて、
%
\ruby{何}{なに}とも
\ruby{云}{い}へぬ
\ruby{忌}{いま}はしき
\ruby{氣}{き}のする!。
%
\原本頁{219-11}\改行%
されど、
%
されど、
%
\ruby{是}{こ}は
\ruby[g]{眞實}{まこと }なり、% ルビ調整(原本通り)「眞實(まこと)」
%
\ruby{我}{われ}は
\ruby{僞}{いつは}らず、
%
\ruby{我}{われ}は
\ruby{矯}{た}めず、
%
\ruby{我}{われ}は
\ruby{{\換字{飾}}}{かざ}らず、
\原本頁{220-1}% 120 ページになっているが
%
\ruby{恐}{おそ}るゝ
ところ
\ruby{無}{な}し。
%
われ
こゝに
\ruby{思}{おも}ふ!。
%
\ruby{我}{われ}
こゝに
\ruby{在}{あ}り!。
%
\ruby{天}{てん}
\ruby{我}{わ}が
\ruby{戀}{おも}へる
\ruby{人}{ひと}を
\ruby{何}{なに}と
せんとはする\換字{!?}、
%
\ruby{天}{てん}
そも〳〵
\ruby{我}{われ}を
\ruby{何}{なに}と
なれとかする\換字{!?}。

\原本頁{220-4}% 120 ページになっているが
と
\ruby[g]{淺草}{あさくさ}の
\ruby[g]{御堂}{み だう}に
\ruby{身}{み}を
\ruby{投}{な}げ
\ruby{伏}{ふ}して
\ruby{涙}{なみだ}に
くれし
\ruby[<j>]{曉}{あかつき}には
\ruby{引}{ひき}かへ、
%
\ruby{一{\換字{文}}字口}{いち|もん|じ|ぐち}
\ruby{緊}{きび}しく
\ruby[g]{引締}{ひきし }めて、
%
\ruby{{\換字{猶}}}{なほ}
\ruby[g]{石人}{せきじん}の
\ruby{如}{ごと}く
\ruby[g]{突立}{つゝた }てる
\ruby{時}{とき}、
%
\ruby[g]{尾竹}{を たけ}と
\ruby{松之助}{まつ|の|すけ}とは
\ruby{家}{いへ}の
\ruby{中}{うち}より
\ruby{現}{あらは}れ
\ruby{出}{い}でゝ、

\原本頁{220-7}%
『
そこに
\ruby{居}{ゐ}らつしやるのは
\ruby[g]{水野}{みづの }さんで?。
%
ア、
%
\ruby[g]{御入}{お はい}んなされば
\ruby{宜}{よろ}しかつたものを。
』

\原本頁{220-9}%
と
\ruby[g]{尾竹}{を たけ}の
\ruby{云}{い}ふに
\ruby{續}{つゞ}いて
\ruby{松之助}{まつ|の|すけ}は、

\原本頁{220-10}%
『
そこに
\ruby{居}{ゐ}たの?。
%
\ruby{僕}{ぼく}は
\ruby{君}{きみ}は
\ruby{何}{なに}か
\ruby{思}{おも}ひ
\ruby{出}{だ}して
\ruby{歸}{かへ}つたのかと
\ruby{思}{おも}つた!。
%
\ruby[g]{水野}{みづの }
\ruby{君}{くん}、
%
\ruby{君}{きみ}は
\ruby{變}{へん}な
\ruby{人}{ひと}だネ。
』

\原本頁{221-1}%
と、
%
\ruby{我}{わ}が
\ruby{姊}{あね}の
\ruby[g]{水野}{みづの }を
\ruby{{\換字{嫌}}}{きら}へる
\ruby{事}{こと}の
\ruby[g]{如何}{い か }
ばかり
\ruby{其}{そ}の
\ruby{人}{ひと}を
\ruby{苦}{くるし}め
\ruby{居}{を}るかをも
\ruby{知}{し}らずして
\ruby{云}{い}ふ。

\原本頁{221-3}%
\ruby[g]{尾竹}{を たけ}は
また
\ruby{直}{たゞち}に
\ruby[g]{引取}{ひつと }つて、

\原本頁{221-4}%
『
\ruby{定}{さだ}めし
\ruby{案}{あん}じて
\ruby{居}{ゐ}て
\ruby{下}{くだ}さる
だらうと
いふので、
%
\ruby{今}{いま}
\ruby[g]{御宅}{お たく}へ
\ruby[g]{一寸}{ちよつと}
\ruby[g]{樣子}{やうす }を
\ruby{申}{まを}しに
\ruby{上}{あが}らうとした
ところで
ござりました。
%
\ruby{熱}{ねつ}が
\ruby{甚}{ひど}く
\ruby{發}{はつ}して
\ruby[g]{譫語}{せんご }が
\ruby{{\換字{強}}}{つよ}かつたり
なんぞ
したので、
%
\ruby{傍}{そば}の
\ruby{人}{ひと}は
\ruby[g]{一時}{いちじ }
\ruby{驚}{おどろ}いたのでしたが、
%
\ruby{別}{べつ}の
\ruby{事}{こと}も
\ruby{無}{な}くつて
まあ
\ruby{濟}{す}みました。
%
\ruby{肺}{はい}も
\ruby[g]{心臓}{しんざう}も
\ruby[g]{故障}{こしやう}は
\ruby{無}{な}し、
%
まづ
\ruby{今}{いま}の
ところでは
\ruby{怖}{こは}くは
\ruby{無}{な}いです。
%
\換字{志}かし
\ruby{二三日}{に|さん|にち}は
まだ
\ruby[g]{此樣}{こ ん }な
\ruby{事}{こと}も
ありましやうよ、
%
\ruby[g]{此處}{こ ゝ }
\ruby{二三日}{に|さん|にち}が
\ruby{峠}{たうげ}ですから。% ルビ調整(原本通り)(たうげ)
』

\原本頁{221-11}%
と、
%
いと
\ruby[g]{親切}{しんせつ}に
\ruby{語}{かた}り
\ruby{聞}{きか}せたり。
