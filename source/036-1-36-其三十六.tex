\Entry{其三十六}

\ruby{落葉}{おち|ば}を
\ruby{誘}{さそ}ふ
\ruby{山下}{やま|おろ}しの
\ruby{風}{かぜ}を
\ruby{其儘}{その|まゝ}なる
\ruby{猛鷲}{あら|わし}の
\ruby{打翥}{うち|はぶ}く
\ruby{音}{おと}の
\ruby{中}{うち}には、『
\ruby{神明}{か|み}は
\ruby{殪}{たふ}れたり、』『
\ruby{佛陀}{ほと|け}は
\ruby{死}{し}したり、』といふ
\ruby{響}{ひゞき}の
\ruby{聞}{きこ}え、
\ruby{首}{かうべ}を
\ruby{擡}{あ}げて
\ruby{蜿蜒}{う|ね}る
\ruby{大蛇}{だい|じや}のざわ〳〵と
\ruby{木茅}{き|かや}を
\ruby{倒}{たふ}し
\ruby{行}{ゆ}く
\ruby{音}{おと}の
\ruby{中}{うち}には、『
\ruby{神明}{か|み}は
\ruby{想像}{さう|ざう}のみ、』『
\ruby{佛陀}{ほと|け}は
\ruby{假{\換字{説}}}{か|せつ}のみ、』といふ
\ruby{聲}{こゑ}あり。

\ruby{水野}{みづ|の}は
\ruby{自}{みづ}から
\ruby{思}{おも}はずして
\ruby{自}{おのづ}から
\ruby{如是}{か|く}
\ruby{想}{おも}ひ、
\ruby{外}{そと}に
\ruby{見聞}{み|きゝ}せずして
\ruby{内}{うち}に
\ruby{如是}{か|く}
\ruby{見聞}{み|き}きせる
\ruby{時}{とき}、
\ruby{靜}{しづか}なる
\ruby{五十子}{い|そ|こ}が
\ruby{家}{いへ}の
\ruby{方}{かた}にて、かたりと
\ruby{微}{かすか}の
\ruby{物音}{もの|おと}の
\ruby{仕}{し}たるを
\ruby{聞}{き}きつけ、
\ruby{豁然}{くわつ|ぜん}としてわれに
\ruby{{\換字{返}}}{かへ}れば、
\ruby{我}{わ}が
\ruby{止}{と}め
\ruby{{\換字{途}}}{ど}
\ruby{無}{な}かりし
\ruby{淚}{なみだ}の
\ruby{何時}{い|つ}か
\ruby{乾}{かわ}き、
\ruby{我}{わ}が
\ruby{疲}{つか}れたる
\ruby{心}{こゝろ}の
\ruby{何時}{い|つ}か
\ruby{奮}{ふる}ひて、
\ruby{倚}{よ}りかかりたる
\ruby{椎}{しい}の
\ruby{幹}{みき}を
\ruby{離}{はな}れ、そを
\ruby{背向}{そ|がひ}にして
\ruby{挺然}{てい|ぜん}と
\ruby{獨}{ひと}り
\ruby{樹陰}{こ|かげ}の
\ruby{闇}{やみ}に
\ruby{立}{た}ちつ、
\ruby{魔}{ま}の
\ruby{如}{ごと}くに
\ruby{來}{きた}り
\ruby{魔}{ま}の
\ruby{如}{ごと}くに
\ruby{去}{さ}る
\ruby{蝙蝠}{かは|ほり}の、ひらひらと
\ruby{{\換字{梢}}}{こずゑ}の
\ruby{盡頭}{はず|れ}を
\ruby{飛}{とび}かへれるを、
\ruby{雲透}{くも|ずき}に\換字{志}つと
\ruby{打見}{うち|み}やりたり。

\ruby{有}{あ}りや
\ruby{神佛}{かみ|ほとけ}の?、
\ruby{有}{あ}るにも
\ruby{似}{に}たるかな!。
\ruby{無}{な}しや
\ruby{神佛}{かみ|ほとけ}の?、
\ruby{無}{な}きにも
\ruby{似}{に}たるかな!。

\ruby{有}{あ}るには
\ruby{無}{な}きの
\ruby{疑}{うたがひ}あり、
\ruby{有}{あ}りとも
\ruby{爲難}{し|がた}く、
\ruby{無}{な}しとも
\ruby{爲難}{し|がた}し。
\ruby{有無}{う|む}のいづれは
\ruby{今}{いま}
\ruby{知}{し}らねども、
\ruby{世}{よ}に
\ruby{無}{な}き
\ruby{方}{かた}の
\ruby{眞實}{ま|こと}ならば、
\ruby{男兒}{をの|こ}の
\ruby{頭}{かうべ}を
\ruby{下}{さ}げて
\ruby{祈願}{き|ぐわん}を
\ruby{捧}{さゝ}げんことの
\ruby{羞}{はづか}しくも
\ruby{口惜}{くち|を}しく、
\ruby{若}{も}し
\ruby{世}{よ}に
\ruby{在}{おは}す
\ruby{事}{こと}の
\ruby{定}{ぢやう}ならば、
\ruby{身}{み}をも
\ruby{魂魄}{たま|しひ}をも
\ruby{犠牲}{いけ|にへ}にして、
\ruby{廣大}{くわう|だい}の
\ruby{御慈悲}{おん|じ|ひ}を
\ruby{頼}{たの}み
\ruby{奉}{たてまつ}らんと
\ruby{思}{おも}ふ
\ruby{此}{こ}の
\ruby{人間}{ひ|と}の
\ruby{心}{こゝろ}のみぞ
\ruby{僞}{いつは}り
\ruby{無}{な}き
\ruby{眞實}{ま|こと}なる!。
\ruby{二}{ふ}タ
\ruby{路}{みち}かけて
\ruby{取舎}{しゆ|しや}しわづらひつゝ、
\ruby{利}{よき}に
\ruby{就}{つ}かんとする
\ruby{此}{こ}の
\ruby{{\換字{分}}別}{ふん|べつ}の
\ruby{醜}{みにく}さよ、
\ruby{智慧}{ち|ゑ}の
\ruby{狡猾}{かし|こ}さよ!。
あゝ
\ruby{人間}{ひ|と}は
\ruby{卑劣}{さ|も}しくも
\ruby{怯}{きたな}き
\ruby{心}{こゝろ}を
\ruby{有}{も}てるかな!。
されど
\ruby{此}{こ}の
\ruby{疑}{うたが}ひ
\ruby{惑}{まど}ひて
\ruby{苦}{くるし}めるこそは、
\ruby{人間}{ひ|と}の
\ruby{僞}{いつは}り
\ruby{無}{な}き
\ruby{眞實}{まこ|と}の
\ruby{{\換字{情}}狀}{さ|ま}なるべけれ。
\ruby{我}{われ}こゝに
\ruby{在}{あ}り、われこゝに
\ruby{思}{おも}ふ。
\ruby{思}{おも}はるゝものゝ
\ruby{有}{あ}り
\ruby{無}{な}しは
\ruby{定}{さだ}かならず、
\ruby{思}{おも}ふ
\ruby{我}{われ}の
\ruby{在}{あ}る
\ruby{事}{こと}が
\ruby{眞實}{まこ|と}なるのみ。
\ruby{菩薩}{ぼ|さつ}の
\ruby{言葉}{こと|ば}、
\ruby{鷲}{わし}の
\ruby{言葉}{こと|ば}、
\ruby{妙典}{めう|てん}の
\ruby{敎}{をしへ}、
\ruby{大蛇}{だい|じや}の
\ruby{敎}{をしへ}、
\ruby{我}{われ}にいづれを
\ruby{取}{と}り
\ruby{那方}{いづ|れ}を
\ruby{捨}{す}つる
\ruby{力無}{ちから|な}し、たゞ
\ruby{那方}{いづ|れ}をも
\ruby{取}{と}り
\ruby{惱}{なや}み、またいづれをも
\ruby{捨}{す}て
\ruby{惱}{なや}む
\ruby{其事}{その|こと}のみぞ
\ruby{我}{わ}が
\ruby{眞實}{まこ|と}なる!。
\ruby{神明}{か|み}
\ruby{佛陀}{ほと|け}をも
\ruby{肯}{うけが}はずして、
\ruby{智慧}{ち|ゑ}の
\ruby{鋼鐵}{はが|ね}の
\ruby{杖}{つゑ}に
\ruby{頼}{よ}つて
\ruby{此}{こ}の
\ruby{戰鬪}{たゝ|かひ}の
\ruby{世}{よ}に
\ruby{立}{た}たんとするも
\ruby{我}{わ}が
\ruby{欺}{あざむ}かぬ
\ruby{眞實}{まこ|と}なり。
\ruby{獸}{けもの}にもあらず
\ruby{鳥}{とり}にもあらで、
\ruby{光明}{ひか|り}の
\ruby{國}{くに}
\ruby{黑闇}{や|み}の
\ruby{國}{くに}の
\ruby{境}{さかひ}を
\ruby{飛}{と}ぶ
\ruby{彼}{あ}の
\ruby{魔魅}{ま|もの}の
\ruby{如}{ごと}き
\ruby{蝙蝠}{かは|ほり}の、
\ruby{世}{よ}にも
\ruby{厭}{いと}はしく
\ruby{醜}{みにく}きは、
\ruby{我}{わ}が
\ruby{胸}{むね}の
\ruby{中}{うち}の
\ruby{怪物}{くわい|ぶつ}の、
\ruby{化}{な}りて
\ruby{出}{い}でしかとも
\ruby{思}{おも}はれて、
\ruby{何}{なに}とも
\ruby{云}{い}へぬ
\ruby{忌}{いま}はしき
\ruby{氣}{き}のする!。
されど、されど、
\ruby{是}{こ}は
\ruby{眞實}{まこ|と}なり、
\ruby{我}{われ}は
\ruby{僞}{いつは}らず、
\ruby{我}{われ}は
\ruby{矯}{た}めず、
\ruby{我}{われ}は
\ruby{{\換字{飾}}}{かざ}らず、
\ruby{恐}{おそ}るゝところ
\ruby{無}{な}し。
われこゝに
\ruby{思}{おも}ふ!。
\ruby{我}{われ}こゝに
\ruby{在}{あ}り!。
\ruby{天我}{てん|わ}が
\ruby{戀}{おも}へる
\ruby{人}{ひと}を
\ruby{何}{なに}とせんとはする\換字{?!}。
\ruby{天}{てん}そも〳〵
\ruby{我}{われ}を
\ruby{何}{なに}となれとかする\換字{?!}。

と
\ruby{淺草}{あさ|くさ}の
\ruby{御堂}{み|だう}に
\ruby{身}{み}を
\ruby{投}{な}げ
\ruby{伏}{ふ}して
\ruby{淚}{なみだ}にくれし
\ruby{曉}{あかつき}には
\ruby{引}{ひき}かへ、
\ruby{一{\換字{文}}字口緊}{いち|もん|じ|ぐち|きび}しく
\ruby{引締}{ひき|し}めて、
\ruby{{\換字{猶}}}{なほ}
\ruby{石人}{せき|じん}の
\ruby{如}{ごと}く
\ruby{突立}{つゝ|た}てる
\ruby{時}{とき}、
\ruby{尾竹}{を|たけ}と
\ruby{松之助}{まつ|の|すけ}とは
\ruby{家}{いへ}の
\ruby{中}{うち}より
\ruby{現}{あらは}れ
\ruby{出}{い}でゝ、

『そこに
\ruby{居}{ゐ}らつしやるのは
\ruby{水野}{みづ|の}さんで?。
ア、
\ruby{御入}{お|はい}んなされば
\ruby{宜}{よろ}しかつたものを。
』

と
\ruby{尾竹}{を|たけ}の
\ruby{云}{い}ふに
\ruby{續}{つゞ}いて
\ruby{松之助}{まつ|の|すけ}は、

『そこに
\ruby{居}{ゐ}たの?。
\ruby{僕}{ぼく}は
\ruby{君}{きみ}は
\ruby{何}{なに}か
\ruby{思}{おも}ひ
\ruby{出}{だ}して
\ruby{歸}{かへ}つたのかと
\ruby{思}{おも}つた!。
\ruby{水野君}{みづ|の|くん}、
\ruby{君}{きみ}は
\ruby{變}{へん}な
\ruby{人}{ひと}だネ。
』

と、
\ruby{我}{わ}が
\ruby{姊}{あね}の
\ruby{水野}{みづ|の}を
\ruby{{\換字{嫌}}}{きら}へる
\ruby{事}{こと}の
\ruby{如何}{い|か}ばかり
\ruby{其}{そ}の
\ruby{人}{ひと}を
\ruby{苦}{くるし}め
\ruby{居}{を}るかをも
\ruby{知}{し}らずして
\ruby{云}{い}ふ。

\ruby{尾竹}{を|たけ}はまた
\ruby{直}{たゞち}に
\ruby{引取}{ひつ|と}つて、

『
\ruby{定}{さだ}めし
\ruby{案}{あん}じて
\ruby{居}{ゐ}て
\ruby{下}{くだ}さるだらうといふので、
\ruby{今}{いま}
\ruby{御宅}{お|たく}へ
\ruby{一寸}{ちよ|つと}
\ruby{樣子}{やう|す}を
\ruby{申}{まを}しに
\ruby{上}{あが}らうとしたところでござりました。
\ruby{熱}{ねつ}が
\ruby{甚}{ひど}く
\ruby{發}{はつ}して
\ruby{譫語}{せん|ご}が
\ruby{{\換字{強}}}{つよ}かつたりなんぞしたので、
\ruby{傍}{そば}の
\ruby{人}{ひと}は
\ruby{一時}{いち|じ}
\ruby{驚}{おどろ}いたのでしたが、
\ruby{別}{べつ}の
\ruby{事}{こと}も
\ruby{無}{な}くつてまあ
\ruby{濟}{す}みました。
\ruby{肺}{はい}も
\ruby{心臓}{しん|ざう}も
\ruby{故障}{こ|しやう}は
\ruby{無}{な}し、まづ
\ruby{今}{いま}のところでは
\ruby{怖}{こは}くは
\ruby{無}{な}いです。
\換字{志}かし
\ruby{二三日}{に|さん|にち}はまだ
\ruby{此樣}{こ|ん}な
\ruby{事}{こと}もありましやうよ、
\ruby{此處}{こ|ゝ}
\ruby{二三日}{に|さん|にち}が
\ruby{峠}{たうげ}ですから。
』

と、いと
\ruby{親切}{しん|せつ}に
\ruby{語}{かた}り
\ruby{聞}{きか}せたり。
