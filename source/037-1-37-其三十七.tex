\Entry{其三十七}

% メモ 校正終了 2024-04-12 2024-05-28 2024-06-27
\原本頁{222-2}%
『
あら
お
\ruby{止}{よし}なさいよ、
%
\ruby[g]{頭髮}{か み }が
\ruby{壞}{こは}れまさあネ。
%
いやですよ、
%
ほんとに、
%
\ruby{人}{ひと}を
\ruby[g]{馬鹿}{ば か }にしたツ!。
%
そんな
\ruby{事}{こと}は
\ruby{妾}{わたし}や
\ruby{{\換字{嫌}}}{きら}ひ
ですつてば、
%
\ruby{大}{おほ}きな
\ruby{聲}{こゑ}を
\ruby{出}{だ}しますよ。
%
ほら、
%
ほら
\ruby{御師匠}{おつ|し|よ}さんの
\ruby[g]{下駄}{げ た }の
\ruby{音}{おと}ぢや
ありませんか。
』

\原本頁{222-6}%
\ruby[||j>]{男}{をとこ}の
\ruby{力}{ちから}の
\ruby{{\換字{緩}}}{ゆる}む
\ruby{間}{ひま}に
\ruby{辛}{から}くも
\ruby{{\換字{逃}}}{のが}れて、
\換字{志}どけ
\ruby{無}{な}く
\ruby{亂}{みだ}れたる
\ruby[g]{衣服}{な り }の
\ruby{{\換字{前}}}{まへ}を
\ruby[g]{引直}{ひきなほ}しつ、
%
\ruby{膳}{ぜん}の
\ruby{先}{さき}に
\ruby{{\換字{遠}}}{とほ}く
\ruby{離}{はな}れて
\ruby{坐}{すわ}つたるは、
%
さして
\ruby{美}{うつく}し
といふには
あらねど、
%
\ruby{光}{ひか}り
\ruby{流}{なが}るゝが
\ruby{如}{ごと}き
\ruby{眼}{め}の
\ruby{中}{なか}に
\ruby[||j>]{{\換字{情}}}{なさけ}
\ruby[||j>]{有}{ あ}つて、
%
\ruby{世}{よ}に
いふ
\ruby[||j>]{男}{をとこ}
\ruby[||j>]{好}{ ずき}
のする
\ruby[g]{何處}{ど こ }と
\ruby{無}{な}く
\ruby{仇}{あだ}つぽき
\ruby[g]{廿歳}{はたち }ばかりの
すらりとしたる
\原本頁{222-11}\改行%
\ruby{女}{をんな}にて、
%
\ruby[g]{人{\換字{前}}}{ひとまへ}は
\ruby[g]{此家}{こ ゝ }の
\ruby{女主人}{あ|る|じ}の
\ruby{内弟子}{うち|で|し}なり、
%
\ruby[||j>]{娘}{むすめ}
\ruby[||j>]{{\換字{分}}}{ ぶん}なり
% \ruby{娘{\換字{分}}}{むすめ|ぶん}なり
なれど、
%
\ruby{人}{ひと}の
\ruby{見}{み}ぬ
\ruby{時}{とき}は
\ruby{水仕業}{みづ|し|わざ}%【水仕事】料理・洗濯(せんたく)などの、水を使ってする家庭の仕事。
も
\ruby{爲}{さ}せらるゝ、
%
\ruby{寄食者}{かゝ|りう|ど}ともつかず
\ruby[g]{下婢}{はした }ともつかぬ
\ruby{怪}{あや}しきものなれば、
%
\ruby{置}{お}く
\ruby{方}{かた}にも
\ruby{置}{お}かるゝ
\ruby{方}{かた}にも、
%
いづれ
\ruby[g]{一寸}{ちよつと}したる
\ruby[g]{關係}{あ や }は
\ruby{潜}{ひそ}める% 【潛 u6f5b 「先先」】【潜 u6f5c 「夫夫」】併用されている
なるべし。
%
\ruby{男}{をとこ}は
\ruby{顏}{かほ}の
\ruby{色}{いろ}
\ruby{黑}{くろ}く
\ruby[g]{{\換字{強}}壯}{ぢやうぶ}さうに
\ruby[||j>]{膩}{あぶら}
\ruby[||j>]{光}{ でり}の
% \ruby{膩光}{あぶら|でり}の
したる、
%
\ruby{四十餘歳}{し|じふ|いく|つ}の
\ruby[g]{品格}{ひ ん }の
\ruby{無}{な}きなるが、
%
\ruby{膳}{ぜん}を
\ruby{{\換字{前}}}{まへ}にして
\ruby[g]{胡坐}{あぐら }
\ruby{組}{く}めり。

\原本頁{223-6}%
\ruby{格子{\換字{戸}}}{かう|し|ど}は
\ruby{輕}{かろ}く
からりと
\ruby{開}{あ}きて、
%
やがて
\ruby{入}{い}り
\ruby{來}{きた}れるは
\ruby{果}{はた}して
\ruby{女主人}{あ|る|じ}なり。
%
\ruby[g]{五十}{ご じふ}に
\ruby{{\換字{近}}}{ちか}きには
\ruby{疑}{うたが}ひ
\ruby{無}{な}けれど、
%
ぶつてりと
\ruby{肥}{ふと}つたる
\ruby[g]{{\換字{平}}顏}{ひらがほ}の、
%
\ruby{特}{こと}に
\ruby{今}{いま}は
\ruby[g]{{\換字{浴}}後}{ゆあがり}とて
\ruby{照}{て}らつきて
\ruby{赤}{あか}きに、
%
\ruby{絲}{いと}の
\ruby{如}{ごと}く
\ruby{剃}{す}りつけたる
\ruby{眉}{まゆ}の
\ruby[g]{{\換字{嫌}}味}{いやみ }たらしく
\ruby{細}{ほそ}く、
%
\ruby[g]{髮際}{はえぎは}
\ruby[g]{異樣}{ことやう}に
\ruby{濃}{こ}き
\ruby{髮}{かみ}を、
\換字{志}たゝかに
\原本頁{223-10}\改行%
\ruby{油}{あぶら}つけて
\ruby{銀杏{\換字{返}}}{い|てふ|がへ}しに
\ruby{結}{ゆ}ひたる、
%
みづからは
\ruby{未}{ま}だ
\ruby{老}{お}い
\ruby{{\換字{込}}}{こ}まぬ
\ruby[g]{意氣}{い き }を
\ruby{示}{しめ}したる
なるべけれど、
%
\ruby{人}{ひと}は
\ruby{見}{み}るより
\ruby{恐}{おそ}れて
\ruby{{\換字{逃}}}{にげ}
\ruby{走}{はし}るべき
\ruby{態}{さま}な
\原本頁{224-1}\改行%
り。

\原本頁{224-2}%
\ruby{女主人}{あ|る|じ}は
\ruby[||j>]{糠}{ぬか}
\ruby[||j>]{袋}{ぶくろ}の
% \ruby{糠袋}{ぬか|ぶくろ}の
\ruby{絲}{いと}を
\ruby{口}{くち}に
しつゝ、
%
\ruby[g]{手拭}{てぬぐひ}を
ばたりと
\ruby[g]{一度}{ひとたび}
\ruby{鳴}{な}らして
\改行% 校正作業の簡略化のため
、
%
\原本頁{224-3}\改行%
\換字{志゛}ろりと% 「志」+「濁点」
\ruby{白}{しら}けたる
\ruby[g]{此場}{このば }の% 原文通り「場」
\ruby{狀}{さま}を
\ruby{見}{み}れば、
%
\ruby{男}{をとこ}は
\ruby{何}{なに}
\ruby{喰}{く}はぬ
\ruby{顏}{かほ}して
\ruby{酒}{さけ}
\ruby{無}{な}き
\ruby[g]{猪口}{ちよく }を
\ruby{吸}{す}ひ、
%
\ruby{女}{をんな}は
\ruby[g]{徳利}{とくり }に
\ruby{手}{て}は
\ruby{觸}{ふ}れ
ながら
\ruby{{\換字{酌}}}{しやく}を
せんとも
\ruby{爲}{せ}で
\ruby{護}{まも}り
\ruby{居}{ゐ}たる
\ruby{其}{そ}の
\ruby[g]{呼吸}{い き }は
\ruby{{\換字{猶}}}{なほ}
はづみて
\ruby[g]{事實}{まこと }を% ルビ調整(原本通り)
\ruby{語}{かた}れり。

\原本頁{224-6}%
\ruby[g]{十{\換字{分}}}{じふぶん}に
\ruby{男}{をとこ}の
\ruby{何}{なに}と
\ruby{爲}{し}たりしかを
\ruby{猜}{すゐ}したる
\ruby{女主人}{あ|る|じ}の
\ruby{顏}{かほ}は、
%
\ruby{見}{み}る〳〵
\ruby[g]{紫色}{むらさき}に
\ruby{脹}{は}れたるが
\ruby{如}{ごと}くなりて、

\原本頁{224-8}%
『
\ruby{何}{なに}を
\ruby{仕}{し}て
おいでだつたエ、
%
\ruby[g]{貴郞}{おまへ }さんは。
』

\原本頁{224-9}%
と、
%
\ruby{先}{ま}づ
\ruby[g]{一句}{いつく }
\ruby[||j>]{男}{をとこ}の
\ruby{顏}{かほ}を
\ruby{見}{み}て
\ruby{詰}{なじ}りしが、

\原本頁{224-10}%
『
\ruby{先}{さき}へ
\ruby{始}{はじ}めたなあ
\ruby{惡}{わる}かつたが、
%
\ruby{飮}{や}つてた
ばかりだわナ、
%
\ruby[g]{堪{\換字{忍}}}{か に }しねえナ。% 原文通り「堪忍」
』

\原本頁{225-1}%
と、
%
\ruby{男}{をとこ}も
さるもの、
%
\ruby{穩}{おだ}やかに
\ruby{澱}{よど}まず
\ruby{云}{い}ひ
\ruby{流}{なが}すを
\ruby{聞}{き}きて、
%
いよいよ% ルビ調整(原本通り)非踊り字表記(行末行頭の境界付近)
\ruby{眼}{まなこ}を
\ruby{嶮}{けは}\換字{志}くし、

\原本頁{225-3}%
『
\ruby[g]{左樣}{さ う }かい!。
%
そりやあ
\ruby[g]{堪{\換字{忍}}}{か に }するも% 原文通り「堪忍」
\ruby{何}{なに}も
ありやあ
\ruby{仕}{し}ない。
』

\原本頁{225-4}%
と
\ruby{冷}{ひや}やかに
\ruby{云}{い}ひ
\ruby{切}{き}りつ、
%
\ruby{間}{あひだ}を
\ruby{隔}{お}きて、

\原本頁{225-5}%
『
だつて
\ruby[g]{盗賊}{どろばう}
\ruby{猫}{ねこ}が
\ruby{暴}{あば}れた
やうだからサ。
%
\ruby{{\換字{留}}守番}{る|す|ばん}
\ruby[g]{甲{\換字{斐}}}{が ひ }が
\ruby{無}{な}いと
\ruby{思}{おも}つて
\ruby{聞}{き}いたんだよ。
%
お
\ruby{龍}{りゆう}、
%
お
\ruby{{\換字{前}}}{まへ}、
%
\ruby{氣}{き}を
つけ
\ruby{無}{な}くつちやあ
いけないよ。

\原本頁{225-8}%
ほんとに
\ruby{碌}{ろく}で
\ruby{無}{な}しの
\ruby[g]{盗賊}{どろばう}
\ruby{猫}{ねこ}が
\ruby{居}{ゐ}るんだからネ。
%
\ruby{恐}{おそ}ろしい
\ruby[g]{圖々}{づう〴〵}しい
\ruby{奴}{やつ}なんだからネ。
%
\ruby[g]{油斷}{ゆ だん}も
\ruby{隙}{すき}も
なりや
\ruby{仕}{し}ない。
%
\ruby{捕}{つかま}へたら
\ruby{鼻}{はな}づらを
\ruby[g]{引擦}{ひつこす}つて
\ruby{{\換字{遣}}}{や}りたいぢや
\ruby{無}{な}いか。
』

\原本頁{225-11}%
と、
%
\ruby{云}{い}ひながら
\ruby{男}{をとこ}の
\ruby[g]{對面}{むかふ }へ、
%
むずと
\ruby{坐}{すわ}つたり。

\原本頁{226-1}%
\ruby{男}{をとこ}は
\ruby{困}{こう}じたる
\ruby{顏}{かほ}に
\ruby[||j>]{苦}{にが}
\ruby[||j>]{笑}{わらひ}して
% \ruby{苦笑}{にが|わらひ}して
\ruby{横}{よこ}を
\ruby{向}{む}けり。
