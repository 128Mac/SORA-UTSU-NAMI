\Entry{其三十七}

『あらお
\ruby{止}{よし}なさいよ、
\ruby{頭髪}{か|み}が
\ruby{壊}{こは}れまさあネ。
いやですよ。
ほんとに、人を
\ruby{馬鹿}{ば|か}にしたツ!。
そんな
\ruby{事}{こと}は
\ruby{妾}{わたし}や
\ruby{{\換字{嫌}}}{きら}ひですつてば、
\ruby{大}{おほ}きな
\ruby{聲}{こゑ}を
\ruby{出}{だ}しますよ。
ほら、ほら
\ruby{御師匠}{おつ|し|よ}さんの
\ruby{下駄}{げ|た}の
\ruby{音}{おと}ぢやありませんか。
』

\ruby{男}{をとこ}の
\ruby{力}{ちから}の
\ruby{{\換字{緩}}}{ゆる}む
\ruby{間}{ひま}に
\ruby{辛}{から}くも
\ruby{逃}{のが}れて、\換字{志}どけ
\ruby{無}{な}く
\ruby{亂}{みだ}れたる
\ruby{衣服}{な|り}の
\ruby{前}{まへ}を
\ruby{引直}{ひき|なほ}しつ、
\ruby{膳}{ぜん}の
\ruby{先}{さき}に
\ruby{{\換字{遠}}}{とほ}く
\ruby{離}{はな}れて
\ruby{坐}{すわ}つたるは、さして
\ruby{美}{うつく}しといふにあらねど、
\ruby{光}{ひか}り
\ruby{流}{なが}るゝが
\ruby{如}{ごと}き
\ruby{眼}{め}の
\ruby{中}{なか}に
\ruby{{\換字{情}}}{なさけ}
\ruby{有}{\ あ}つて、
\ruby{世}{よ}にいふ
\ruby{男好}{をとこ|ずき}のする
\ruby{何處}{ど|こ}と
\ruby{無}{な}く
\ruby{仇}{あだ}つぽき
\ruby[g]{廿歳}{はたち}ばかりのすらりとしたる
\ruby{女}{をんな}にて、
\ruby{人前}{ひと|まへ}は
\ruby{此家}{こ|ゝ}の
\ruby{女主人}{あ|る|じ}の
\ruby{内弟子}{うち|で|し}なり、
\ruby{娘{\換字{分}}}{むすめ|ぶん}なりなれど、
\ruby{人}{ひと}の
\ruby{見}{み}ぬ
\ruby{時}{とき}は
\ruby{水仕業}{みづ|し|わざ}も
\ruby{爲}{さ}せらるゝ、
\ruby{寄食者}{かゝ|りう|ど}ともつかず
\ruby{下婢}{はし|た}ともつかぬ
\ruby{怪}{あや}しきものなれば、
\ruby{置}{お}く
\ruby{方}{かた}にも
\ruby{置}{お}かるゝ
\ruby{方}{かた}にも、いづれ
\ruby{一寸}{ちよ|つと}したる
\ruby{關係}{あ|や}は
\ruby{潜}{ひそ}めるなるべし。
\ruby{男}{をとこ}は
\ruby{顏}{かほ}の
\ruby{色黑}{いろ|くろ}く
\ruby{{\換字{強}}壯}{ぢや|うぶ}さうに
\ruby{膩光}{あぶら|てり}のしたる、
\ruby{四十餘歳}{し|じう|いく|つ}の
\ruby{品格}{ひ|ん}の
\ruby{無}{な}きなるが、
\ruby{膳}{ぜん}を
\ruby{前}{まへ}にして
\ruby{胡坐組}{あ|ぐら|く}めり。

\ruby{格子戸}{かう|し|ど}は
\ruby{輕}{かろ}くからりと
\ruby{開}{あ}きて、やがて
\ruby{入}{い}り
\ruby{來}{きた}れるは
\ruby{果}{はた}して
\ruby{女主人}{あ|る|じ}なり。
\ruby{五十}{ご|じう}に
\ruby{{\換字{近}}}{ちか}きには
\ruby{疑}{うたが}ひ
\ruby{無}{な}けれど、ぶつてりと
\ruby{肥}{ふと}つたる
\ruby{{\換字{平}}顏}{ひら|がほ}の、
\ruby{特}{こと}に
\ruby{今}{いま}は
\ruby{浴後}{ゆあ|がり}とて
\ruby{照}{て}らつきて
\ruby{赤}{あか}きに、
\ruby{絲}{いと}の
\ruby{如}{ごと}く
\ruby{剃}{す}りつけたる
\ruby{眉}{まゆ}の
\ruby{{\換字{嫌}}味}{いや|み}たらしく
\ruby{細}{ほそ}く、
\ruby{髪際異樣}{はえ|ぎは|こと|やう}に
\ruby{濃}{こ}き
\ruby{髪}{かみ}を、\換字{志}たゝかに
\ruby{油}{あぶら}つけて
\ruby{銀杏{\換字{返}}}{い|てふ|がへ}しに
\ruby{結}{ゆ}ひたる、みづからは
\ruby{未}{ま}だ
\ruby{老}{お}い
\ruby{{\換字{込}}}{こ}まぬ
\ruby{意氣}{い|き}を
\ruby{示}{しめ}したるなるべけれど、
\ruby{人}{ひと}は
\ruby{見}{み}るより
\ruby{恐}{おそ}れて
\ruby{逃走}{にげ|はし}るべき
\ruby{態}{さま}なり。

\ruby{女主人}{あ|る|じ}は
\ruby{糠袋}{ぬか|ぶくろ}の
\ruby{絲}{いと}を
\ruby{口}{くち}にしつゝ、
\ruby{手拭}{てぬ|ぐひ}をばたりと
\ruby{一度鳴}{いち|ど|な}らして、\換字{志}ろりと
\ruby{白}{しら}けたる
\ruby{此場}{この|ば}の
\ruby{狀}{さま}を
\ruby{見}{み}れば、
\ruby{男}{をとこ}は
\ruby{何喰}{なに|く}はぬ
\ruby{顏}{かほ}して
\ruby{酒無}{さけ|な}き
\ruby[g]{猪口}{ちよく}を
\ruby{吸}{す}ひ、
\ruby{女}{をんな}は
\ruby{徳利}{とく|り}に
\ruby{手}{て}は
\ruby{觸}{ふ}れながら
\ruby{酌}{しやく}をせんとも
\ruby{爲}{せ}で
\ruby{護}{まも}り
\ruby{居}{ゐ}たる
\ruby{其}{そ}の
\ruby{呼吸}{い|き}は
\ruby{{\換字{猶}}}{なほ}はづみて
\ruby{事實}{ま|こと}を
\ruby{語}{かた}れり。

\ruby{十{\換字{分}}}{じう|ぶん}に
\ruby{男}{をとこ}の
\ruby{何}{なに}と
\ruby{爲}{し}たりしかを
\ruby{猜}{すひ}したる
\ruby{女主人}{あ|る|じ}の
\ruby{顏}{かほ}は、
\ruby{見}{み}る〳〵
\ruby{紫色}{むら|さき}に
\ruby{脹}{は}れたるが
\ruby{如}{ごと}くなりて、

『
\ruby{何}{なに}を
\ruby{仕}{し}ておいでだつたエ、
\ruby{貴郞}{おま|へ}さんは。
』

と、
\ruby{先}{ま}づ
\ruby{一句男}{いつ|く|をとこ}の
\ruby{顔}{かほ}を
\ruby{見}{み}て
\ruby{詰}{なじ}りしが、

『
\ruby{先}{さき}へ
\ruby{始}{はじ}めたなあ
\ruby{惡}{わる}かつたが、
\ruby{飮}{や}つたばかりだわナ、
\ruby{堪忍}{か|に}しねえナ。
』

と、
\ruby{男}{をとこ}もさるもの、
\ruby{穩}{おだ}やかに
\ruby{澱}{よど}まず
\ruby{云}{い}ひ
\ruby{流}{なが}すを
\ruby{聞}{き}きて、いよいよ
\ruby{眼}{まなこ}を
\ruby{嶮}{けは}\換字{志}くし、

『
\ruby{左樣}{さ|う}かい!。
そりやあ
\ruby{堪忍}{か|に}するも
\ruby{何}{なに}もありやあ
\ruby{仕}{し}ない。
』

と
\ruby{冷}{ひや}やかに
\ruby{云}{い}ひ
\ruby{切}{き}りつ、
\ruby{間}{あひだ}を
\ruby{隔}{お}きて、

『だつて
\ruby{盗賊猫}{どろ|ばう|ねこ}が
\ruby{暴}{あば}れたやうだからサ。
\ruby[g]{{\換字{留}}守番甲斐}{るすばんがひ}が
\ruby{無}{な}いと
\ruby{思}{おも}つて
\ruby{聞}{き}いたんだよ。
お
\ruby{龍}{りゆう}、
お
\ruby{前}{まへ}、
\ruby{氣}{き}をつけ
\ruby{無}{な}くつちやあいけないよ。

ほんとに
\ruby{碌}{ろく}で
\ruby{無}{な}しの
\ruby{盗賊猫}{どろ|ばう|ねこ}が
\ruby{居}{ゐ}るんだからネ。
\ruby{恐}{おそ}ろしい
\ruby{圖々}{づう|〴〵}しい
\ruby{奴}{やつ}なんだからネ。
\ruby{油斷}{ゆ|だん}も
\ruby{隙}{すき}もなりや
\ruby{仕}{し}ない。
\ruby{捕}{つか}まへたら
\ruby{鼻}{はな}づらを
\ruby{引擦}{ひつ|こす}つて
\ruby{{\換字{遣}}}{や}りたいぢや
\ruby{無}{な}いか。
』

と、
\ruby{云}{い}ひながら
\ruby{男}{をとこ}の
\ruby[g]{對面}{むかふ}へ、むずと
\ruby{坐}{すわ}つたり。

\ruby{男}{をとこ}は
\ruby{困}{こう}じたる
\ruby{顏}{かほ}に
\ruby{苦笑}{にが|わらひ}して
\ruby{横}{よこ}を
\ruby{向}{む}けり。

