\Entry{其三十八}

\ruby{見馴}{み|な}れ
\ruby{聞}{き}き
\ruby{馴}{な}れたるにさまでは
\ruby{感}{かん}ぜねど、
\ruby{何}{なん}と
\ruby{挨拶}{あい|さつ}すべき
\ruby{言葉}{こと|ば}を
\ruby{知}{し}らねば、
お
\ruby{龍}{りう}は
\ruby{手拭糠袋}{てぬ|ぐひ|ぬか|ぶくろ}を
\ruby{手渡}{て|わた}しされたるを
\ruby{機}{き}に、
\ruby{其}{そ}を
\ruby{臺所近}{だい|どころ|ちか}き
\ruby{掛竿}{かけ|ざを}に
\ruby{叮嚀}{てい|ねい}に
\ruby{懸}{か}けて、わざと
\ruby{暇取}{ひま|と}りて
\ruby{此方}{こな|た}へ
\ruby{來}{く}れば、
\ruby{膳}{ぜん}の
\ruby{上}{うへ}に
\ruby{伏}{ふ}せありたる
\ruby{我}{わ}が
\ruby[g]{猪口}{ちよく}を、
\ruby[g]{不興氣}{ふきようげ}に
\ruby{取}{と}り
\ruby{上}{あ}げたる
\ruby{主人}{ある|じ}に
\ruby{向}{むか}ひて、
\ruby{男}{をとこ}は
\ruby{自}{みづか}ら
\ruby{徳利}{とく|り}を
\ruby{手}{て}にして、
\ruby{諂}{へつら}ひ
\ruby{笑}{わらひ}を
\ruby{面}{おもて}に
\ruby{{\換字{浮}}}{うか}べつゝ、
\ruby{今}{いま}や
\ruby{酌}{しやく}して
\ruby{{\換字{遣}}}{や}らんとしたる
\ruby{其}{そ}の
\ruby{狀態}{あり|さま}の、たとへば
\ruby{女主人}{あ|る|じ}は
\ruby{怒}{いか}つたる
\ruby{蝦蟇}{ひき|がへる}の
\ruby{如}{ごと}く、
\ruby{男}{をとこ}は
\ruby{{\換字{又}}地}{また|ち}に
\ruby{下}{お}りたる
\ruby{狡猾}{わる|がしこ}き
\ruby{烏}{からす}の
\ruby{如}{ごと}くなるに、
\ruby{思}{おも}はずも
\ruby{安芝居}{やす|しば|ゐ}の
\ruby{安役者}{やす|やく|しや}が
\ruby{出}{だ}せる
\ruby{世話物}{せ|わ|もの}の、
\ruby{下卑}{げ|び}たる
\ruby{一}{ひ}ト
\ruby{場}{ば}を
\ruby{見}{み}る
\ruby{心地}{こゝ|ち}して、おのれもまた
\ruby{其}{そ}の
\ruby{同}{おな}じ
\ruby{此}{こ}の
\ruby{舞臺}{ぶ|たい}に
\ruby{{\換字{交}}}{まじ}りて
\ruby{一}{ひ}ト
\ruby{役}{やく}を
\ruby{演}{す}ることかと、
\ruby{身}{み}に
\ruby{染}{し}みてつく〴〵と
\ruby{嬉}{うれ}しからず
\ruby{思}{おも}ひしが、
\ruby{漸}{やうや}く
\ruby{二人}{ふた|り}の
\ruby{仲}{なか}の
\ruby{治}{をさ}まり
\ruby{行}{ゆ}かんとするさまなれば、
\ruby{差當}{さし|あた}り
\ruby{先}{ま}づ
\ruby{其事}{そ|れ}を
\ruby{{\換字{悅}}}{よろ}びて
\ruby{坐}{ざ}に
\ruby{戾}{もど}り、
\ruby{膳}{ぜん}の
\ruby{上}{うへ}の
\ruby{聊}{いさゝ}か
\ruby{淋}{さび}しきを
\ruby{見}{み}て、

『お
\ruby{師匠}{し|よ}さん、あの
\ruby{傳}{でん}さんの
\ruby{下}{くだ}すつたものを
\ruby{開}{あ}けましやうか。
』

と、
\ruby{機{\換字{嫌}}取}{き|げん|と}り
\ruby{顏}{がほ}に
\ruby{優}{やさし}しく
\ruby{云}{い}へば、
\ruby{主人}{ある|じ}も
\ruby{此女}{こ|れ}に
\ruby{對}{むか}つては
\ruby{言葉}{こと|ば}を
\ruby{和}{やは}らげつ。

『アヽ、たしか
\ruby{雀燒}{すゞめ|やき}だつたネ、ぢやあ
\ruby{開}{あ}けておくれ!。
オヤありやあ
\ruby{汝}{おまへ}につて
\ruby{彼人}{あの|ひと}が
\ruby{{\換字{呉}}}{く}れたんだつたのに。
』

『あらいやな、そんな
\ruby{事}{こと}を!。
どうだつて
\ruby{好}{い}いぢやあありませんか。
』

『
\ruby{左樣}{さ|う}かい。
ぢやあ、まあ、
\ruby{貰}{もら}ふよ。
\ruby{面倒}{めん|ど}くさいから
\ruby{取}{と}り
\ruby{{\換字{分}}}{わ}けずともだよ。
あゝ
\ruby{左樣}{さ|う}さ、
\ruby{其儘}{その|まゝ}で
\ruby{好}{い}いやネ、
\ruby{構}{かま}やあしないよ。
』

\ruby{大}{おほき}からぬ
\ruby{杉折}{すぎ|おり}は
\ruby{膳}{ぜん}の
\ruby{傍}{かたはら}に
\ruby{出}{いだ}されたり。

『オヤ
\ruby{此}{これ}あ
\ruby{千住}{せん|じゆ}のだよ、\換字{志}かも
\ruby{鮒}{ふな}だ、
\ruby{自轉車天狗}{じ|てん|しや|てん|ぐ}が
\ruby{物}{もの}を
\ruby{{\換字{呉}}}{く}れると、いつでも
\ruby{奇妙}{き|めう}に
\ruby{{\換字{遠}}}{とほ}い
\ruby{{\換字{所}}}{ところ}のものばかりだから
\ruby{可笑}{を|か}しいのさ、
\ruby{帝釋}{たい|しやく}さまの
お
\ruby{水}{みづ}を
\ruby{何}{なん}でも
\ruby{無}{な}い
\ruby{日}{ひ}に
\ruby{持}{も}つて
\ruby{來}{き}て
\ruby{{\換字{呉}}}{く}れたりなんぞするのは、
\ruby{自轉車乘}{じ|てん|しや|の}りで
\ruby{無}{な}くつちやあ
\ruby{出來}{で|き}ない
\ruby{事}{こと}だよ。
ン、
\ruby{中々}{なか|〳〵}おいしいよ、
\ruby{汝}{おまへ}も
お
\ruby{食}{あが}りな、
\ruby[g]{一杯}{ひとつ}あげやう。
』

『イヽエ
\ruby{妾}{わたし}は。
』

『ハヽヽ、ちつとも
\ruby{飮}{や}らないだけは、ほんとに
\ruby{汝}{おまへ}にも
\ruby{似合}{に|あ}はないよ。
だけれど、
\ruby[g]{其行狀}{それ}で
\ruby{飮}{や}られちやあ
\ruby{大變}{たい|へん}だからネ、
\ruby{其}{それ}も
\ruby{可}{い}いかも
\ruby{知}{し}れないよ。
』

『あらまあ
\ruby{甚}{ひど}い
\ruby{事}{こと}を。
』

『だつてお
\ruby{酒}{さけ}まで
\ruby{好}{すき}だつた
\ruby{日}{ひ}にやあ
\ruby{何樣}{ど|う}したつて
お
\ruby{前}{まへ}は、
\ruby{紀伊国屋}{き|の|くに|や}が
\ruby{演}{し}さうな
\ruby{肌}{はだ}の
\ruby{女}{をんな}になるからねえ!。
\ruby{折角妾}{せつ|かく|わたし}の
\ruby{名跡}{あ|と}を
\ruby{取}{と}つて
\ruby{貰}{もら}はうと
\ruby{思}{おも}つて
\ruby{居}{ゐ}たつて、
\ruby{何樣}{ど|ん}な
\ruby{場}{ば}を
お
\ruby{前}{まへ}が
\ruby{出}{だ}して
\ruby{仕舞}{し|ま}ふか
\ruby{知}{し}れやしないもの!。
』

『いやですよ、お
\ruby{師匠}{し|よ}さん、そんな
\ruby{事}{こと}を
\ruby{云}{い}つちやあ、
\ruby{妾}{わたし}はもう
\ruby{澤山凝}{たん|と|こ}りて
\ruby{居}{ゐ}るんですもの、いつまでもおとなしく
\ruby{仕}{し}て
\ruby{居}{ゐ}て
\ruby{一生獨身}{いつ|しやう|どく|しん}で、
お
\ruby{師匠}{し|よ}さんの
\ruby{傍}{そば}にばかり
\ruby{居}{ゐ}るつもりなんですから。
』

『
\ruby{嬉}{うれ}しいねえ。
お
\ruby{前}{まへ}が
\ruby{左樣}{さ|う}いふ
\ruby{氣}{き}で
\ruby{居}{ゐ}て
\ruby{{\換字{呉}}}{く}れりやあ
\ruby{妾}{わたし}あ
\ruby{此上無}{この|うへ|な}しさ。
いよ〳〵
\ruby{左樣}{さ|う}なら
\ruby{妾}{わたし}の
\ruby{事}{こと}をネ、これから
お
\ruby{母}{つか}さん
お
\ruby{母}{つか}さんと
\ruby{呼}{よ}んでも
\ruby{可}{い}いよ。
\ruby{妾}{わたし}の
\ruby{方}{はう}ぢやあ
\ruby{疾}{とう}から
\ruby{既實}{もう|じつ}の
\ruby{娘}{こ}のやうに
\ruby{思}{おも}つて
\ruby{居}{ゐ}るんだから。
』

『お
\ruby{師匠}{し|よ}さん、そりやあ
\ruby{本當}{ほん|たう}なの、きつと
\ruby{本當}{ほん|たう}なの?。
お
\ruby{母}{つか}さんと
\ruby{云}{い}つても
\ruby{惡}{わる}かあ
\ruby{無}{な}くつて?。
』

『あゝ
\ruby{可}{いゝ}ともさ。
\ruby{妾}{わたし}あ
\ruby{何樣}{ど|ん}なに
\ruby{嬉}{うれ}しいか
\ruby{知}{し}れやしないよ。
』

\ruby{男}{をとこ}は
\ruby{此時}{この|とき}まで
\ruby{手持無}{て|もち|な}くて、
\ruby{二人}{ふた|り}が
\ruby[g]{對話}{はなし}を
\ruby{聞}{き}き
\ruby{居}{ゐ}たりしが、こゝにむぐ〳〵と
\ruby{口}{くち}を
\ruby{動}{うご}かして、

『お
\ruby{母}{つか}さんにしちやあ
\ruby{變}{へん}に
\ruby{若}{わか}いナ。
』

と、
\ruby{阿諛}{あ|ゆ}に
\ruby{似}{に}たる
\ruby{語}{ご}を
\ruby{挿}{さしはさ}めば、
\ruby{女主人}{あ|る|じ}は
\ruby{忽}{たちま}ち、

『
\ruby{何}{なん}だとエ、
\ruby{餘計}{よ|けい}な
\ruby{御世話}{お|せ|わ}だよ。
\ruby{黙}{だま}つておいで!。
』

と、たしなめは
\ruby{仕}{し}たけれど
\ruby{腹}{はら}は
\ruby{立}{た}てぬ
\ruby{顏}{かほ}なり。

『
\ruby{妾}{わたし}もネ、
お
\ruby{前}{まへ}は
\ruby{知}{し}るまいが
\ruby{子}{こ}はあるけれども、 --- もつとも
\ruby{義理}{ぎ|り}だけで
\ruby{根}{ね}は
\ruby{他人}{た|にん}なのさ、だもんだから
お
\ruby{前}{まへ}、
\ruby{妾}{わたし}を
\ruby{馬鹿}{ば|か}にして、
\ruby{一人}{ひと|り}は
\ruby{女}{をんな}の
\ruby{癖}{くせ}に
\ruby{生意氣}{なま|い|き}に
\ruby{{\換字{敎}}員}{けう|ゐん}なんぞになりやがつて、
\ruby{{\換字{近}}在}{きん|ざい}に
\ruby{一人}{ひと|り}で
\ruby{暮}{くら}して
\ruby{居}{ゐ}るし、
\ruby{其弟}{その|おとうと}は
\ruby{書生}{しよ|せい}を
\ruby{仕}{し}て
\ruby{居}{ゐ}るが、
\ruby{二人}{ふ|たり}とも
\ruby{妾}{わたし}を
\ruby{馬鹿}{ば|か}に
\ruby{仕}{し}きつて
\ruby{居}{ゐ}て、
\ruby{此家}{こ|ゝ}なんぞへは
\ruby{寄}{よ}りつきも
\ruby{仕}{し}ないんだが、ほんとにまあ
\ruby{何樣}{ど|ん}なに
\ruby{高慢}{かう|まん}な
\ruby{憎}{にく}らしい
\ruby[g]{奴等}{やつら}だらう!。
だから
\ruby{妾}{わたし}も
\ruby{其等}{そい|ら}を
\ruby{子}{こ}だとは
\ruby{思}{おも}つて
\ruby{居}{ゐ}やしないのさ。
\ruby{同}{おな}じ
\ruby{他人}{た|にん}なら
\ruby{妾}{わたし}は
お
\ruby{前}{まへ}を、ほんたうに
\ruby{妾}{わたし}の
\ruby{娘}{むすめ}にして、
\ruby{何樣}{ど|ん}なにでも
\ruby{好}{よ}くして
\ruby{{\換字{遣}}}{や}りたいよ。
なあに
\ruby{何}{なん}にも
\ruby{有}{あ}りや
\ruby{仕}{し}ないけれど、それでも
お
\ruby{前}{まへ}、
\ruby{妾}{わたし}は
\ruby{妾}{わたし}
\ruby{一人}{ひと|り}でもつて、どうやら
\ruby{斯樣}{こ|う}やら
\ruby{{\換字{遣}}}{や}つて
\ruby{來}{き}て
\ruby{居}{ゐ}るんだからネ、それだけの
\ruby{事}{こと}は
お
\ruby{前}{まへ}に
\ruby{譲}{ゆづ}るつもりなのさ。
エ、
\ruby{其}{そ}の
\ruby{娘}{むすめ}かエ、
\ruby{五十}{い|そ}と
\ruby{云}{い}つてネ、
\ruby[g]{容貌}{きりやう}も
\ruby{惡}{わる}かあ
\ruby{無}{な}いが、
\ruby{愛}{あい}の
\ruby{無}{な}い、
\ruby{矢張}{やつ|ぱ}りあの
\ruby{妾}{わたし}の
\ruby{大{\換字{嫌}}}{だい|きら}ひな
\ruby{海老茶}{え|び|ちや}の
\ruby{袋}{ふくろ}を
\ruby{穿}{は}いてる
\ruby{奴}{やつ}なのさ。
\ruby{男}{をとこ}の
\ruby{子}{こ}は
\ruby{松之助}{まつ|の|すけ}といつて、
\ruby{直}{ぢき}そこの
\ruby{下谷}{した|や}に
\ruby{居}{ゐ}るのだがネ、
\ruby{此}{こ}の
\ruby{方}{はう}はまだしも
\ruby{素直}{す|なほ}な
\ruby{性質}{た|ち}だから
\ruby{手}{て}なづけては
\ruby{居}{ゐ}るけれど、やつぱし
\ruby{姊}{あね}びいきだから
\ruby{妾}{わたし}の
\ruby{爲}{ため}にやあ、
\ruby{末始{\換字{終}}}{すゑ|し|じゆう}は
\ruby{爲}{な}りさうもない
\ruby{奴}{やつ}なのさ。
\ruby{此樣}{こ|ふ}いふ
\ruby{譯}{わけ}なんだから、
お
\ruby{前次第}{まへ|し|だい}で、ほんとに
お
\ruby{前}{まへ}が
\ruby{妾}{わたし}の
\ruby{後}{あと}を
\ruby{取}{と}る
\ruby{氣}{き}になつて
お
\ruby{{\換字{呉}}}{く}れなら、どんなにでも
\ruby{妾}{わたし}は
お
\ruby{前}{まへ}に
\ruby{肩}{かた}を
\ruby{入}{い}れるよ。
\ruby{其代}{その|かは}り
お
\ruby{前}{まへ}\換字{志}つかりしてネ、よその
\ruby{下}{くだ}らない
\ruby{猫}{ねこ}なんぞに
\ruby{手}{て}をかけられたりなんぞ
\ruby{仕}{し}ないやうに
\ruby{仕}{し}て
お
\ruby{{\換字{呉}}}{く}れで
\ruby{無}{な}くちやいけないよ。
ハヽヽ。
おや、
\ruby{暗}{くら}くなつて
\ruby{來}{き}たネ、
\ruby[g]{洋燈}{らんぷ}さへ
\ruby{準備}{し|たく}が
\ruby{仕}{し}てあるなら
\ruby{構}{かま}はないから、
お
\ruby{湯}{ゆ}へ
\ruby{行}{い}つておいでな。
\ruby{妾}{わたし}あ
お
\ruby{前}{まへ}が
\ruby{美麗}{き|れい}だつて
\ruby{云}{い}はれると
\ruby{眞實}{ほ|んと}に
\ruby{天狗}{てん|ぐ}なんだから、いくらでも
\ruby{悠々磨}{ゆつ|くり|みが}いておいで!。
』

