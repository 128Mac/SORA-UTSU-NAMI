\Entry{其三十九}

\原本頁{}
『あら
\ruby{虛言}{う|そ}ばつかり!。
%
いくら
\ruby{磨}{みが}いたつて、
%
どうせ
\ruby{美麗}{き|れい}になんか
\ruby{成}{な}りやあ
\ruby{仕}{し}ませんよ。
』

\原本頁{}
とは
\ruby{云}{い}ひたれど
\ruby{師匠}{し|ゝやう}が
\ruby{言葉}{こと|ば}に
\ruby{悅}{よろこ}べるさまは、
%
\ruby{掩}{おほ}はんとして
\ruby{掩}{おほ}ひきれず、
%
\ruby{愛嬌}{あい|けう}
\ruby{溢}{こぼ}るゝ
\ruby{眼}{め}のしほに
\ruby{見}{み}えたり。
%
\ruby{女主人}{あ|る|じ}はこれを
\ruby{見}{み}て
\ruby{取}{と}りて、
%
\ruby{此}{これ}もおなじく
\ruby{笑顏}{ゑ|がほ}つくり、

\原本頁{}
『ナニ
\ruby{妾}{わたし}が
お
\ruby{茶々羅}{ちや|〳〵|ら}を
\ruby{云}{い}ふもんかネ。
%
\ruby{傳}{でん}さんだつて
\ruby{淸}{せい}さんだつて
\ruby{{\換字{勝}}}{かつ}さんだつて、
%
みんな
お
\ruby{{\換字{前}}}{まへ}が
\ruby{美麗}{き|れい}だもんだから
\ruby{大騷}{おほ|さわ}ぎ
\ruby{{\換字{遣}}}{や}つてるんだあネ。
%
\ruby{虛言}{う|そ}だと
\ruby{思}{おも}ふなら
\ruby{聞}{き}いて
\ruby{御覧}{ご|らん}!。
』

\原本頁{}
と、
%
\ruby{重}{かさ}ねて
\ruby{復}{また}も
\ruby{悅}{よろこ}ばせにかゝれば、

\原本頁{}
『あら、
%
あんまりだわ
\ruby{御師匠}{お|し|よ}さん!。
%
たんと
\ruby{御嬲}{お|なぶ}りなさいよ、
%
ようござんすわ。
』

\原本頁{}
と、
%
\ruby{此度}{こ|たび}はつんとして
\ruby{横}{よこ}を
\ruby{向}{む}きしが、
%
\ruby{媚}{なまめ}きながら
\ruby{微瞋}{やゝ|いか}れる
\ruby{顏}{かほ}は、
%
\ruby{女主人}{あ|る|じ}が
\ruby{言葉}{こと|ば}もいつはりならず
\ruby{艶}{えん}なり。% 原本通り「えん」

\原本頁{}
やゝありて
\ruby{思}{おも}ひ
\ruby{出}{だ}したるやうに、

\原本頁{}
『
\ruby{少}{すこ}し
\ruby{早}{はや}くつても
\ruby{洋燈}{らん|ぷ}を
\ruby{點}{つ}けましやう。
』

\原本頁{}
と、
%
\ruby{云}{い}ひさまに
\ruby{立}{た}つて
お
\ruby{龍}{りゆう}は
\ruby{去}{さ}りつ、
%
\ruby{何}{なに}をなせるにや
\ruby{少時}{しば|らく}
\ruby{其姿}{その|すがた}を
\ruby{見}{み}せざりしが、
%
\ruby{火}{ひ}を
\ruby{點}{てん}じたる
\ruby{釣洋燈}{つり|らん|ぷ}を
\ruby{持}{も}ち
\ruby{來}{きた}りて、
%
\ruby{座敷}{ざ|しき}の
\ruby{中央}{ま|なか}に
\ruby{高}{たか}く
\ruby{吊}{つ}りし
\ruby{時}{とき}には、
%
\ruby{今}{いま}までのほつれかゝりたる
\ruby{髷}{まげ}のあとかたも
\ruby{無}{な}く、
%
\ruby{其}{そ}の
\ruby{頭髮}{か|み}は
\ruby{早}{はや}くも
\ruby{結}{ゆ}ひかへられて、
%
さつぱりとしたる
\ruby{束髮}{そく|はつ}の
\ruby{美}{うつく}しきが、
%
\ruby{燈}{ひ}の
\ruby{光}{ひかり}に
\ruby{鮮}{あざ}やかに
\ruby{映}{うつ}し
\ruby{出}{いだ}されたり。

\原本頁{}
『オヤ
\ruby{早變}{はや|がは}りだネエ、
%
\ruby{吃驚}{びつ|くり}させられたよ。
%
チヨイと
\ruby{彼方}{あつ|ち}を
\ruby{向}{む}いて
\ruby{御見}{お|み}せナ、
%
ヘーエそれが
\ruby{花月卷}{か|げつ|まき}とやらかエ?。
』

\原本頁{}
『ハア、
%
\ruby{左樣}{さ|う}ですの。
%
\ruby{似合}{に|あ}はなくつて?。
』

\原本頁{}
『イヽエ
\ruby{似合}{に|あ}はないどころぢあ
\ruby{無}{な}いよ、
%
これは
\ruby{此}{これ}でもつて、
%
いつそ
\ruby{{\換字{又}}}{また}
\ruby{好}{い}いよ。
%
お
\ruby{{\換字{前}}}{まへ}は
\ruby{徳}{とく}な
\ruby{顏立}{かほ|だち}で、
%
\ruby{何}{なん}に
\ruby{結}{ゆ}つても
\ruby{似合}{に|あ}ふのが
\ruby{妙}{めう}だネ。
%
だが
\ruby{束髮}{そく|はつ}も
\ruby{此頃}{この|ごろ}は
\ruby{考}{かんが}へたネ、
%
\ruby{一}{ひ}ト\換字{志}きり
\ruby{人}{ひと}が
\ruby{爲}{し}た
\ruby{蝸牛}{まひ〳〵|つぶろ}の
\ruby{親方見}{おや|かた|み}たやうなのなんざあ、
%
\ruby{堪}{たま}らなく
\ruby{可厭}{い|や}なもんだつたがねえ、
%
ハヽヽ。
』

\原本頁{}
『ホヽヽ、
%
\ruby{御師匠}{お|し|よ}さんの
\ruby{口}{くち}には
\ruby{叶}{かな}いませんわ。
%
ぢやあ
\ruby{一寸}{ちよ|つと}
\ruby{御湯}{お|ゆう}へ。
』

\原本頁{}
『あゝ
\ruby{可}{い}いとも!。
%
さあ〳〵
\ruby{髮}{かみ}も
\ruby{出來}{で|き}たし、
%
\ruby{行}{い}つておいで、
%
\ruby{行}{い}つておいで!。
』

\原本頁{}
『ぢやあ
\ruby{一寸}{ちよ|いと}。
』

\原本頁{}
\ruby{云}{い}ひながら
\ruby{會釋}{ゑ|しやく}して
\ruby{身}{み}を
\ruby{起}{おこ}し、
%
やがて
\ruby{徐}{しづか}に
\ruby{出}{で}て
\ruby{行}{ゆ}きけるが
\ruby{輕}{かろ}らかなる
\ruby{下駄}{げ|た}の
\ruby{音}{おと}は
\ruby{幾程}{いく|ほど}も
\ruby{無}{な}く
\ruby{{\換字{消}}}{き}えぬ。

\原本頁{}
『
\ruby{大{\換字{分}}}{だい|ぶ}
\ruby{念入}{ねん|い}りにあやなすぢやあ
\ruby{無}{ね}えか。
』

\原本頁{}
\ruby{男}{をとこ}は
\ruby{女主人}{あ|る|じ}が
お
\ruby{龍}{りゆう}に
\ruby{對}{たい}する
\ruby{擧動}{ふる|まひ}を
\ruby{怪}{あや}しむやうに
\ruby{云}{い}へば、
%
やゝ
\ruby{醉}{ゑ}ひたる
\ruby{女主人}{あ|る|じ}はそれには
\ruby{關}{かま}はず、
%
\ruby{今}{いま}
\ruby{迄}{まで}は
\ruby{他}{ひと}の
\ruby{見}{み}る
\ruby{目}{め}を
\ruby{{\換字{兼}}}{か}ねて
\ruby{堪}{こら}へ
\ruby{居}{ゐ}しが、
%
\ruby{今}{いま}は
\ruby{憚}{はゞか}るところも% 「憚 は(ゞ)か」
\ruby{無}{な}きに、
%
\ruby{突然}{いき|なり}
\ruby{手}{て}あたり
\ruby{任}{まか}せに
\ruby{男}{をとこ}の
\ruby{口}{くち}の
\ruby{端}{はた}をいやといふほど
\ruby{捻}{つね}りて、

\原本頁{}
『あやなすぢやあ
\ruby{無}{ね}えかも
\ruby{無}{な}いもんだ。
%
\ruby{人}{ひと}の
\ruby{居}{ゐ}ない
\ruby{中}{うち}
\ruby{何}{なに}を
\ruby{爲}{し}やうと
\ruby{仕}{し}たんだエ。
』

\原本頁{}
と、
%
\ruby{新}{あらた}に
\ruby{罪}{つみ}を
\ruby{糺}{たゞ}さんとする
\ruby{其勢}{その|いきほひ}なか〳〵
\ruby{當}{あた}りがたければ
\ruby{男}{をとこ}はこれに
\ruby{辟易}{へき|えき}して
\ruby{聊}{いさゝ}か
\ruby{身}{み}を
\ruby{{\換字{退}}}{ひ}きぬ。

\原本頁{}
『ナニたゞ
\ruby{調戲}{から|か}つたばかりだよ、
%
\ruby{戲談}{じやう|だん}だわナ。
』

\原本頁{}
『フン、
%
\ruby{戲談}{じやう|だん}から
\ruby{駒}{こま}が
\ruby{出無}{で|な}くつて
\ruby{御仕合}{お|し|あはせ}さ。
』

\原本頁{}
\ruby{長{\換字{煙}}管}{なが|ぎせ|る}は
\ruby{忽}{たちま}ち
\ruby{烈}{はげ}しく
\ruby{膝頭}{ひざ|がしら}を
\ruby{突}{つ}きぬ。
%
\ruby{男}{をとこ}はいよ〳〵
\ruby{後}{あと}じさりするのみ。

\原本頁{}
『あやまつた〳〵。
%
いゝ
\ruby{加減}{か|げん}にして
\ruby{吳}{く}れ、
%
\ruby{痛}{いて}えやナ。
』

\原本頁{}
『
\ruby{痛}{いた}くつても
\ruby{關}{かま}ふもんか、
%
\ruby{碌}{ろく}で
\ruby{無}{な}しめ。
』

\原本頁{}
『あやまつたと
\ruby{云}{い}ふに
\ruby{執念深}{しふ|ねん|ぶか}いなあ。
』

\原本頁{}
『
\ruby{執念深}{しふ|ねん|ぶか}いなあ
\ruby{妾}{わたし}の
\ruby{性}{しやう}だよ。
%
ほんとに
\ruby{彼女}{あ|れ}なんぞに
\ruby{指}{ゆび}でもさして
\ruby{御覧}{ご|らん}、
%
\ruby{今度}{こん|ど}からたゞ
\ruby{置}{お}きやあ
\ruby{仕無}{し|な}いから。
%
\ruby{彼女}{あ|れ}あ
\ruby{妾}{わたし}が
\ruby{大事}{だい|じ}にかけてるんだもの。
』

\原本頁{}
『だから
\ruby{彼樣}{あ|ん}なに
\ruby{味}{あぢ}に
\ruby{{\換字{文}}}{あや}なして
\ruby{何樣}{ど|う}するんだと
\ruby{聞}{き}くのだ!。
』

\原本頁{}
『どうしたつて
\ruby{宜}{い}いよ、
%
\ruby{汝}{おまへ}の
\ruby{御世話}{お|せ|わ}にやあならない。
%
\ruby{妾}{わたし}も
\ruby{取}{と}る
\ruby{年}{とし}だし、
%
\ruby{子}{こ}は
\ruby{無}{な}いし、
%
どうせ
\ruby{汝}{おまへ}はちつとも
\ruby{當}{あて}にやあならないしするから、
%
\ruby{彼女}{あ|れ}に
\ruby{後}{あと}を
\ruby{{\換字{遣}}}{や}つて
\ruby{彼女}{あ|れ}にかゝるんだよ。
』

\原本頁{}
『フーム、
%
\ruby{{\換字{強}}氣}{がう|ぎ}に
\ruby{彼岸詣}{ひ|がん|まひ}りでも
\ruby{仕}{し}さうな
\ruby{風}{ふう}な
\ruby{事}{こと}をいふナ。
%
そりやあ
\ruby{眞實}{ほん|たう}かエ。
』

\原本頁{}
『さうさ、
%
ほんたうで
\ruby{無}{な}くつてサ。
』

\原本頁{}
『ハヽヽ、
%
\ruby{虛言}{う|そ}を
\ruby{云}{い}ひねえナ。
%
\ruby{止}{よ}しねえ〳〵!。
%
\ruby{繼子}{まゝ|こ}だつて
\ruby{何}{なん}だつて
\ruby{二人}{ふた|り}も
\ruby{子}{こ}もあるのに、
%
\ruby{其樣}{そ|ん}な
\ruby{事}{こと}がなんで
\ruby{出來}{で|き}るもんか。
』

\原本頁{}
『
\ruby{出來無}{で|き|な}いものかネ、
%
\ruby{爲}{す}るんだもの!。
%
\ruby{無理}{む|り}でも
\ruby{左樣}{さ|う}して
\ruby{妾}{わたし}やあ
\ruby{彼女}{あ|れ}にかゝるんだよ。
%
\ruby{相続人}{さう|ぞく|にん}になつてる
\ruby{五十}{い|そ}は
\ruby{死}{し}ぬかも
\ruby{知}{し}れないのだから。
』

\原本頁{}
『ハヽヽ、
%
\ruby{{\換字{強}}氣}{がう|ぎ}に
\ruby{老}{お}い
\ruby{{\換字{込}}}{こ}んだ
\ruby{事}{こと}をいふが、
%
\ruby{乃公}{お|れ}まで
\ruby{食}{く}はせやうと
\ruby{云}{い}ふなあ、
%
ちつと
\ruby{甚}{ひど}い!。
%
どうしてお
\ruby{{\換字{前}}}{めへ}が
\ruby{後}{あと}を
\ruby{案}{あん}じる
\ruby{風}{ふう}かエ。
%
\ruby{汝}{おめへ}は
\ruby{彼女}{あ|れ}をすつかり
\ruby{取}{と}り
\ruby{{\換字{込}}}{こ}んで、\換字{志}やぶつて
\ruby{{\換字{遣}}}{や}らうと
\ruby{云}{い}ふんだらう。
』

\原本頁{}
『
\ruby{何}{なん}だとエ?。
』

\原本頁{}
『
\ruby{知}{し}れた
\ruby{事}{こと}さ!。
%
\ruby{食物}{くひ|もの}に
\ruby{仕}{し}やうと
\ruby{云}{い}ふんだらう!。
%
\ruby{何}{なに}も
\ruby{一人}{ひと|り}で
\ruby{占}{し}めずともの
\ruby{事}{こと}だ、
%
\ruby{乃公}{お|れ}にも
\ruby{{\換字{半}}{\換字{分}}}{はん|ぶん}
\ruby{{\換字{遺}}}{よこ}しねえナ。
%
\ruby{圃}{はたけ}でこしらへたものぢやあ
\ruby{有}{あ}るまいし、
%
たゞ
\ruby{穫}{と}つた
\ruby{魚}{さかな}ぢやあ
\ruby{無}{ね}えか、
%
\ruby{吝}{おし}みなさんナ。
%
\ruby{其代}{その|かは}り
\ruby{骨}{ほね}つきの
\ruby{方}{はう}は
\ruby{其方}{そつ|ち}へ
\ruby{{\換字{遣}}}{や}らあ!。
』

\原本頁{}
『
\ruby{畜生}{ちく|しやう}!、
%
\ruby{惡徒}{あく|とう}め!、
%
えゝ
\ruby{仕方}{し|かた}が
\ruby{無}{な}い!。
%
それぢやあ
\ruby{片身}{かた|み}はあげるからネ、
%
\ruby{要}{い}る
\ruby{時}{とき}に
\ruby{何時}{い|つ}でも
\ruby{庖丁}{はう|ちやう}を
お
\ruby{貸}{か}し!。
』
