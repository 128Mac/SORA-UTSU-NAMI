\Entry{其三十九}

% メモ 校正終了 2024-04- 2024-06-27
\原本頁{233-4}%
『
あら
\ruby[g]{虛言}{う そ }
ばつかり!。
%
いくら
\ruby{磨}{みが}いたつて、
%
どうせ
\ruby[g]{美麗}{き れい}に
なんか
\ruby{成}{な}りやあ
\ruby{仕}{し}ませんよ。
』

\原本頁{233-6}%
とは
\ruby{云}{い}ひたれど
\ruby[g]{師匠}{ししやう}が% ルビ調整(原本通り)
\ruby[g]{言葉}{ことば }に
\ruby{悅}{よろこ}べる
さまは、
%
\ruby{掩}{おほ}はんとして
\ruby{掩}{おほ}ひきれず、
%
\ruby[g]{愛嬌}{あいけう}
\ruby{溢}{こぼ}るゝ
\ruby{眼}{め}の
しほに
\ruby{見}{み}えたり。
%
\ruby{女主人}{あ|る|じ}は
これを
\ruby{見}{み}て
\ruby{取}{と}りて、
%
\ruby{此}{これ}も
おなじく
\ruby[g]{笑顏}{ゑ がほ}つくり、

\原本頁{233-9}%
『
ナニ
\ruby{妾}{わたし}が
お
\ruby{茶々羅}{ちや|〳〵|ら}を
\ruby{云}{い}ふもんかネ。
%
\ruby{傳}{でん}さんだつて
\ruby{淸}{せい}さんだつて
\ruby{{\換字{勝}}}{かつ}さんだつて、
%
みんな
お
\ruby{{\換字{前}}}{まへ}が
\ruby[g]{美麗}{き れい}だもんだから
\ruby[g]{大騷}{おほさわ}ぎ
\ruby{{\換字{遣}}}{や}つてるんだあネ。
%
\ruby[g]{虛言}{う そ }だと
\ruby{思}{おも}ふなら
\ruby{聞}{き}いて
\ruby[g]{御覧}{ご らん}!。
』

\原本頁{234-2}%
と、
%
\ruby{重}{かさ}ねて
\ruby{復}{また}も
\ruby{悅}{よろこ}ばせに
かゝれば、

\原本頁{234-3}%
『
あら、
%
あんまりだわ
\ruby{御師匠}{お|し|よ}さん!。
%
たんと
\ruby[g]{御嬲}{お なぶ}りなさいよ、
%
ようござんすわ。
』

\原本頁{234-5}%
と、
%
\ruby[g]{此度}{こ たび}は
つんとして
\ruby{横}{よこ}を
\ruby{向}{む}きしが、
%
\ruby{媚}{なまめ}きながら
\ruby{微}{やゝ}
\ruby{瞋}{いか}れる
\ruby{顏}{かほ}は%
\改行% 校正作業の簡略化のため
、
%
\原本頁{234-6}\改行%
\ruby{女主人}{あ|る|じ}が
\ruby[g]{言葉}{ことば }も
いつはり
ならず
\ruby{艶}{えん}なり。% 原本通り「えん」

\原本頁{234-7}%
やゝ
ありて
\ruby{思}{おも}ひ
\ruby{出}{だ}したるやうに、

\原本頁{234-8}%
『
\ruby{少}{すこ}し
\ruby{早}{はや}くつても
\ruby[g]{洋燈}{らんぷ }を
\ruby{點}{つ}けましやう。
』

\原本頁{234-9}%
と、
%
\ruby{云}{い}ひさまに
\ruby{立}{た}つて
お
\ruby{龍}{りゆう}は
\ruby{去}{さ}りつ、
%
\ruby{何}{なに}を
なせるにや
\makeatletter
\@ifundefined{デバッグ@ビルド}{%
  \ruby[g]{少時}{しばらく}
  \ruby[||j>]{其}{その}
  \ruby[||j>]{姿}{すがた}
}{%
  \ruby[<j||]{少}{しば }% ルビ調整(特殊処理)ルビが重なるので
  \ruby[<j||]{時}{らく }
  \ruby[<j||]{其}{その }
  \ruby[<j||]{姿}{すがた}
}%
\makeatother
\原本頁{234-11}\改行%
を
\ruby{見}{み}せざりしが、
%
\ruby{火}{ひ}を
\ruby{點}{てん}じたる
\ruby{釣}{つり}
\ruby[g]{洋燈}{らんぷ }を
\ruby{持}{も}ち
\ruby{來}{きた}りて、
%
\ruby[g]{座敷}{ざ しき}の
\ruby[g]{中央}{ま なか}に
\ruby{高}{たか}く
\ruby{吊}{つ}りし
\ruby{時}{とき}には、
%
\ruby{今}{いま}までの
ほつれ
かゝりたる
\ruby{髷}{まげ}の
あとかたも
\ruby{無}{な}く、
%
\ruby{其}{そ}の
\ruby[g]{頭髮}{か み }は
\ruby{早}{はや}くも
\ruby{結}{ゆ}ひ
かへられて、
%
さつぱり
としたる
\ruby[g]{束髮}{そくはつ}の
\ruby{美}{うつく}しきが、
%
\ruby{燈}{ひ}の
\ruby[g]{光に}{ひかり }
\ruby{鮮}{あざ}やかに
\ruby{映}{うつ}し
\ruby{出}{いだ}されたり。

\原本頁{235-3}%
『
オヤ
\ruby[g]{早變}{はやがは}りだネエ、
%
\ruby[g]{吃驚}{びつくり}させられたよ。
%
チヨイと
\ruby[g]{彼方}{あつち }を
\ruby{向}{む}いて
\ruby[g]{御見}{お み }せナ、
%
ヘーエ
それが
\ruby{花月卷}{くわ|げつ|まき}とやらかエ?。
』

\原本頁{235-5}%
『
ハア、
%
\ruby[g]{左樣}{さ う }ですの。
%
\ruby[g]{似合}{に あ }はなくつて?。
』

\原本頁{235-6}%
『
イヽエ
\ruby[g]{似合}{に あ }はない
どころぢあ
\ruby{無}{な}いよ、
%
これは
\ruby{此}{これ}で
もつて、
%
いつそ
\ruby{{\換字{又}}}{また}
\ruby{好}{い}いよ。
%
お
\ruby{{\換字{前}}}{まへ}は
\ruby{徳}{とく}な
\ruby[g]{顏立}{かほだち}で、
%
\ruby{何}{なん}に
\ruby{結}{い}つても% 原本通り、ここや(ゆ)でなく(い)
\ruby[g]{似合}{に あ }ふのが
\ruby{妙}{めう}
\原本頁{235-8}\改行%
だネ。
%
だが
\ruby[g]{束髮}{そくはつ}も
\ruby[g]{此頃}{このごろ}は
\ruby{考}{かんが}へたネ、
%
\ruby{一}{ひ}ト\換字{志}きり
\ruby{人}{ひと}が
\ruby{爲}{し}た
\makeatletter
\@ifundefined{デバッグ@ビルド}{%
  \ruby[<-||]{蝸}{まひ〳〵}
  \ruby[||->]{牛}{つぶろ}の%
  %\ruby[<g>]{蝸牛}{まひ〳〵つぶろ}の%
}{%
  \ruby[<j>]{蝸}{まひ〳〵}
  \ruby[||j>]{牛}{ つぶろ}の% 行末行頭の境界付近なので特例処置を施す
}%
\makeatother
\原本頁{235-9}\改行%
\ruby[g]{親方}{おやかた}
\ruby{見}{み}たやうなのなんざあ、
%
\ruby{堪}{たま}らなく
\ruby[g]{可厭}{い や }なもんだつたがねえ
\改行% 校正作業の簡略化のため
、
%
\原本頁{235-10}\改行% デバッグのための改行は割愛
ハヽヽ。
』

\原本頁{235-11}%
『
ホヽヽ、
%
\ruby{御師匠}{お|し|よ}さんの
\ruby{口}{くち}には
\ruby{叶}{かな}いませんわ。
%
ぢやあ
\ruby[g]{一寸}{ちよつと}
\ruby[g]{御湯}{お ゆう}
\原本頁{236-1}\改行%
へ。
』

\原本頁{236-2}%
『
あゝ
\ruby{可}{い}いとも!。
%
さあ〳〵
\ruby{髮}{かみ}も
\ruby[g]{出來}{で き }たし、
%
\ruby{行}{い}つておいで、
%
\ruby{行}{い}つておいで!。
』

\原本頁{236-4}%
『
ぢやあ
\ruby[g]{一寸}{ちよいと}。
』

\原本頁{236-5}%
\ruby{云}{い}ひながら
\ruby[g]{會釋}{ゑしやく}して
\ruby{身}{み}を
\ruby{起}{おこ}し、
%
やがて
\ruby{徐}{しづか}に
\ruby{出}{で}て
\ruby{行}{ゆ}きけるが、
\ruby{輕}{かろ}らかなる
\ruby[g]{下駄}{げ た }の
\ruby{音}{おと}は
\ruby[g]{幾程}{いくほど}も
\ruby{無}{な}く
\ruby{{\換字{消}}}{き}えぬ。

\原本頁{236-7}%
『
\ruby[g]{大{\換字{分}}}{だいぶ }
\ruby[g]{念入}{ねんい }りに
あやなすぢやあ
\ruby{無}{ね}えか。
』

\原本頁{236-8}%
\ruby{男}{をとこ}は
\ruby{女主人}{あ|る|じ}が
お
\ruby{龍}{りゆう}に
\ruby{對}{たい}する
\ruby[g]{擧動}{ふるまひ}を
\ruby{怪}{あやし}むやうに
\ruby{云}{い}へば、
%
やゝ
\ruby{醉}{ゑ}ひたる% 「醉」は原本通り「ゑ」で調整
\ruby{女主人}{あ|る|じ}は
それには
\ruby{關}{かま}はず、
%
\ruby{今}{いま}
\ruby{迄}{まで}は
\ruby{他}{ひと}の
\ruby{見}{み}る
\ruby{目}{め}を
\ruby{{\換字{兼}}}{か}ねて
\ruby{堪}{こら}へ
\ruby{居}{ゐ}しが、
%
\ruby{今}{いま}は
\ruby{憚}{はゞか}るところも% 「憚 は(ゞ)か」
\ruby{無}{な}きに、
%
\ruby[g]{突然}{いきなり}
\ruby{手}{て}あたり
\ruby{任}{まか}せに
\ruby{男}{をとこ}の
\ruby{口}{くち}の
\ruby{端}{はた}を
いやといふほど
\ruby{捻}{つね}りて、

\原本頁{237-1}%
『
あやなすぢや
\ruby{無}{ね}えかも
\ruby{無}{な}いもんだ。
%
\ruby{人}{ひと}の
\ruby{居}{ゐ}ない
\ruby{中}{うち}
\ruby{何}{なに}を
\ruby{爲}{し}やう
と
\ruby{仕}{し}たんだエ。
』

\原本頁{237-3}%
と、
%
\ruby{新}{あらた}に
\ruby{罪}{つみ}を
\ruby{糺}{たゞ}さんとする
\ruby[<j||]{其}{その }% ルビ調整(特殊処理)ルビが重なるので
\ruby[<j>]{勢}{いきほひ}
なか〳〵
\ruby{當}{あた}りがたければ
\ruby{男}{をとこ}は
こ
れに
\ruby[g]{辟易}{へきえき}して
\ruby{聊}{いさゝ}か
\ruby{身}{み}を
\ruby{{\換字{退}}}{ひ}きぬ。

\原本頁{237-5}%
『
ナニ
たゞ
\ruby[g]{調戲}{からか }つたばかりだよ、
%
\ruby[||j>]{戲}{じやう}
\ruby[||j>]{談}{ だん}だわナ。
% \ruby{戲談}{じやう|だん}だわナ。
』

\原本頁{237-6}%
『
フン、
%
\ruby[||j>]{戲}{じやう}
\ruby[||j>]{談}{ だん}から
% \ruby{戲談}{じやう|だん}から
\ruby{駒}{こま}が
\ruby[g]{出無}{で な }くつて
\ruby{御}{お}
\ruby{仕}{し}
\ruby{合}{あはせ}さ。
』

\原本頁{237-7}%
\ruby{長{\換字{煙}}管}{なが|ぎせ|る}は
\ruby{忽}{たちま}ち
\ruby{烈}{はげ}しく
\ruby[||j>]{膝}{ひざ}
\ruby[||j>]{頭}{がしら}を
% \ruby{膝頭}{ひざ|がしら}を
\ruby{突}{つ}きぬ。
%
\ruby{男}{をとこ}は
いよ〳〵
\ruby{後}{あと}じさり
するのみ。

\原本頁{237-9}%
『
あやまつた〳〵。
%
いゝ
\ruby[g]{加減}{か げん}にして
\ruby{吳}{く}れ、
%
\ruby{痛}{いて}えやナ。
』

\原本頁{237-10}%
『
\ruby{痛}{いた}くつたつて
\ruby{關}{かま}ふもんか、
%
\ruby{碌}{ろく}で
\ruby{無}{な}しめ。
』

\原本頁{237-11}%
『
あやまつたと
\ruby{云}{い}ふに
\ruby{執念深}{しふ|ねん|ぶか}いなあ。
』

\原本頁{238-1}%
『
\ruby{執念深}{しふ|ねん|ぶか}いなあ
\ruby{妾}{わたし}の
\ruby{性}{しやう}だよ。
%
ほんとに
\ruby[g]{彼女}{あ れ }
なんぞに
\ruby{指}{ゆび}でも
さして
\ruby[g]{御覧}{ご らん}、
%
\ruby[g]{今度}{こんど }から
たゞ
\ruby{置}{お}きやあ
\ruby[g]{仕無}{し な }いから。
%
\ruby[g]{彼女}{あ れ }あ
\ruby{妾}{わたし}が
\ruby[g]{大切}{だいじ }に
かけてるんだもの。
』

\原本頁{238-4}%
『
だから
\ruby[g]{彼樣}{あ ん }なに
\ruby{味}{あぢ}に
\ruby{{\換字{文}}}{あや}なして
\ruby[g]{何樣}{ど う }するんだと
\ruby{聞}{き}くのだ!。
』

\原本頁{238-5}%
『
どうしたつて
\ruby{宜}{い}いよ、
%
\ruby{汝}{おまへ}の
\ruby{御世話}{お|せ|わ}にやあ
ならない。
%
\ruby{妾}{わたし}も
\ruby{取}{と}る
\ruby{年}{とし}だし、
%
\ruby{子}{こ}は
\ruby{無}{な}いし、
%
どうせ
\ruby{汝}{おまへ}は
ちつとも
\ruby{當}{あて}にやあ
ならないしするから、
%
\ruby[g]{彼女}{あ れ }に
\ruby{後}{あと}を
\ruby{{\換字{遣}}}{や}つて
\ruby[g]{彼女}{あ れ }に
かゝるんだよ。
』

\原本頁{238-8}%
『
フーム、
%
\ruby[g]{{\換字{強}}氣}{がうぎ }に
\ruby{彼岸詣}{ひ|がん|まひ}りでも
\ruby{仕}{し}さうな
\ruby{風}{ふう}な
\ruby{事}{こと}を
いふナ。
%
そりやあ
\ruby[g]{眞實}{ほんたう}かエ。
』

\原本頁{238-10}%
『
さうさ、
%
ほんたうで
\ruby{無}{な}くつてサ。
』

\原本頁{238-11}%
『
ハヽヽ、
%
\ruby[g]{虛言}{う そ }を
\ruby{云}{い}ひねえナ。
%
\ruby{止}{よ}しねえ〳〵!。
%
\ruby[g]{繼子}{まゝこ }だつて
\ruby{何}{なん}だつて
\ruby[g]{二人}{ふたり }も
\ruby{子}{こ}も
あるのに、
%
\ruby[g]{其樣}{そ ん }な
\ruby{事}{こと}がなんで
\ruby[g]{出來}{で き }るもんか。
』

\原本頁{239-2}%
『
\ruby{出來無}{で|き|な}いものかネ、
%
\ruby{爲}{す}るんだもの!。
%
\ruby[g]{無理}{む り }でも
\ruby[g]{左樣}{さ う }して
\ruby{妾}{わたし}やあ
\ruby[g]{彼女}{あ れ }に
かゝるんだよ。
%
\ruby{相続人}{さう|ぞく|にん}に
なつてる
\ruby[g]{五十}{い そ }は
\ruby{死}{し}ぬかも
\ruby{知}{し}れないのだから。
』

\原本頁{239-5}%
『
ハヽヽ、
%
\ruby[g]{{\換字{強}}氣}{がうぎ }に
\ruby{老}{お}い
\ruby{{\換字{込}}}{こ}んだ
\ruby{事}{こと}を
いふが、
%
\ruby[g]{乃公}{お れ }まで
\ruby{食}{く}はせやうと
\ruby{云}{い}ふなあ、
%
ちつと
\ruby{甚}{ひど}い!。
%
どうして
お
\ruby{{\換字{前}}}{めへ}が
\ruby{後}{あと}を
\ruby{案}{あん}じる
\ruby{風}{ふう}かエ
\改行% 校正作業の簡略化のため
。
%
\原本頁{239-7}\改行%
\ruby{汝}{おめへ}は
\ruby[g]{彼女}{あ れ }を
すつかりと
\ruby{取}{と}り
\ruby{{\換字{込}}}{こ}んで、
\換字{志}やぶつて
\ruby{{\換字{遣}}}{や}らうと
\ruby{云}{い}ふんだらう。
』

\原本頁{239-9}%
『
\ruby{何}{なん}だとエ?。
』

\原本頁{239-10}%
『
\ruby{知}{し}れた
\ruby{事}{こと}さ!。
%
\ruby[g]{食物}{くひもの}に
\ruby{仕}{し}やうと
\ruby{云}{い}ふんだらう!。
%
\ruby{何}{なに}も
\ruby[g]{一人}{ひとり }で
\ruby{占}{し}めずともの
\ruby{事}{こと}だ、
%
\ruby[g]{乃公}{お れ }にも
\ruby[g]{{\換字{半}}{\換字{分}}}{はんぶん}
\ruby{{\換字{遺}}}{よこ}しねえナ。
%
\ruby{圃}{はたけ}で
こしらへたものぢやあ
\ruby{有}{あ}るまいし、
%
たゞ
\ruby{穫}{と}つた
\ruby{魚}{さかな}ぢやあ
\ruby{無}{ね}えか、
%
\ruby{吝}{おし}みなさんナ。
%
\ruby{其}{その}
\ruby{代}{かは}り
\ruby{骨}{ほね}つきの
\ruby{方}{はう}は
\ruby[g]{其方}{そつち }へ
\ruby{{\換字{遣}}}{や}らあ!。
』

\原本頁{240-3}%
『
\ruby[||j>]{畜}{ちく}
\ruby[||j>]{生}{しやう}!、
% \ruby{畜生}{ちく|しやう}!、
%
\ruby[g]{惡徒}{あくとう}め!、
%
えゝ
\ruby[g]{仕方}{し かた}が
\ruby{無}{な}い!。
%
それぢやあ
\ruby[g]{片身}{かたみ }は
あげるからネ、
%
\ruby{要}{い}る
\ruby{時}{とき}に
\ruby[g]{何時}{い つ }でも
\ruby[||j>]{庖}{はう}
\ruby[||j>]{丁}{ちやう}を
% \ruby{庖丁}{はう|ちやう}を
お
\ruby{貸}{か}し!。
』
