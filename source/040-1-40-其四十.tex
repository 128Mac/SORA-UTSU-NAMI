\Entry{其四十}

\ruby{互}{たがひ}の
\ruby{胸中}{む|ね}に
\ruby{塊物}{も|の}はありながら、
\ruby{相酌}{あひ|じやく}の
\ruby{酒}{さけ}にいつしか
\ruby{解}{と}け
\ruby{合}{あ}つて、
\ruby{男}{をとこ}が
\ruby{勤}{つと}むる
\ruby{亭主役}{てい|しゆ|やく}、
\ruby{銚子}{てう|し}のかはり
\ruby{目間}{め|ま}を
\ruby{拔}{ぬ}けさせねば、
\ruby{女主人}{あ|る|じ}は
\ruby{湯上}{ゆあ|がり}の
\ruby{早}{はや}くも
\ruby{上機{\換字{嫌}}}{じやう|き|げん}となつて、

『そりやあ
\ruby{幾千}{いく|ら}でも
\ruby{働}{はたら}かうが、
\ruby{一體彼女}{いつ|たい|あ|れ}あ
\ruby{何樣}{ど|う}した
\ruby{譯}{わけ}の
\ruby{娘}{こ}なんだ?。
いつ
\ruby{聞}{き}いても
\ruby{些仔細}{ちと|し|さい}があつてとばかしで、
\ruby{聞}{き}かされないが。
』

と
\ruby{男}{をとこ}の
\ruby{云}{い}ふを
\ruby{聞}{き}いて
\ruby{舌}{した}なめずりしつ
\ruby{低聲}{こ|ごゑ}に
\ruby{{\換字{説}}出}{とき|いだ}したり。

『
\ruby{汝}{おまへ}は
\ruby{成程知}{なる|ほど|し}るまいがネ、
\ruby{一昨々年}{さき|を|と|ゝし}の
\ruby{春}{はる}までは
\ruby{彼女}{あ|れ}も
\ruby{矢張}{やつ|ぱ}り、
\ruby{妾}{わたし}のところへ
\ruby{稽古}{けい|こ}に
\ruby{來}{き}た
\ruby{娘}{こ}さ。
』

『ウン。
』

『
\ruby{内務省}{ない|む|しやう}とかの
\ruby{小吏}{こし|べん}の
\ruby{老人}{おぢい|さん}と、
\ruby{父子二人}{おや|こ|ふた|り}きりで
\ruby{暮}{くら}して
\ruby{居}{ゐ}たんだが、
お
\ruby{父}{とつ}さんが
\ruby{日光羊羹見}{につ|くわう|やう|かん|み}たやうに
\ruby{變}{へん}に
\ruby{乾固}{ひ|かた}まつた
\ruby{朴實}{こく|めい}な
\ruby{人}{ひと}だつたのには
\ruby{似合}{に|あ}はないで、あの
\ruby{子}{こ}は
\ruby{{\換字{蓮}}葉}{はす|は}でも
\ruby{無}{な}いが
\ruby{妙}{めう}に
\ruby{{\換字{浮}}氣}{うは|き}つぽい、
お
\ruby{狭}{きやん}な
\ruby{面白}{おも|しろ}いところのある、
\ruby{好}{す}いた
\ruby{男}{をとこ}になら
\ruby{生命}{いの|ち}でも
\ruby{抛}{はふ}り
\ruby{出}{だ}さうツてつたやうな
\ruby{肌合}{はだ|あひ}の
\ruby{娘}{こ}で、
\ruby{同}{おな}い
\ruby{齡}{どし}ぐらゐな
\ruby{娘{\換字{達}}}{こ|たち}が
\ruby{集}{よ}つて
\ruby{談話}{はな|し}を
\ruby{仕}{し}た
\ruby{時}{とき}、
お
\ruby{七}{しち}の
\ruby{爲}{し}た
\ruby{事}{こと}が
\ruby{{\換字{道}}理}{もつ|とも}だといつて
\ruby{一同}{みん|な}に
\ruby{笑}{わら}はれたつて、
\ruby{泣}{な}いて
\ruby{口惜}{く|やし}がつて
\ruby{怒}{おこ}つた
\ruby{事}{こと}がある
\ruby{程}{ほど}なのさ。
そんな
\ruby{調子}{てう|し}だつたもんだから
\ruby{年齡}{と|し}も
\ruby{行}{ゆ}かないのに、これも
\ruby{矢張}{やつ|ぱ}り
\ruby{吾家}{う|ち}へ
\ruby{來}{き}て
\ruby{居}{ゐ}た
\ruby{建具屋}{たて|ぐ|や}の
\ruby{息子}{むす|こ}の
\ruby{源}{げん}といふいなせな
\ruby{男}{をとこ}と
\ruby{人知}{ひと|し}れず
\ruby{出來}{で|き}て
\ruby{仕舞}{し|ま}つたのさ。
』

『フーム、なある
\ruby{程}{ほど}。
お
\ruby{前}{まへ}が
\ruby{撮合山}{とり|も|ち}を
\ruby{行}{や}つたんだナ。
\ruby{兩方}{りやう|はう}から
\ruby{拜}{おが}まれて
\ruby{錢}{ぜに}を
\ruby{取}{と}つたらう!。
\ruby{惡徒}{あく|とう}ツて
\ruby{云}{い}ふなあ
\ruby{其樣}{さ|う}いふのゝ
\ruby{事}{こと}だぜ。
』

『
\ruby{{\換字{交}}}{ま}ぜるなら
\ruby{後}{あと}を
\ruby{話}{はな}さないよ。
』

『あやまつた、あやまつた、それから。
』

『
\ruby{其}{そ}の
\ruby{中}{うち}に
\ruby{彼}{あ}の
\ruby{娘}{こ}の
お
\ruby{父}{とつ}さんが
\ruby{病}{わづら}ひついて、
\ruby{老齢}{と|し}だから
\ruby{叶}{かな}はない、
\ruby{死}{ごね}つちまつたんだ。
すると
\ruby{駿府}{すん|ぷ}とかゝら
\ruby{叔母}{を|ば}さんが
\ruby{出}{で}て
\ruby{來}{き}て、あの
\ruby{娘}{こ}を
\ruby{田舎}{ゐ|なか}へ
\ruby{{\換字{連}}}{つ}れて
\ruby{行}{い}かうといふのさ。
そら
\ruby{{\換字{情}}夫}{をと|こ}の
\ruby{一件}{いつ|けん}があるから
\ruby{行}{い}きたかあ
\ruby{無}{な}いが、まさか
\ruby{十七八}{じう|しち|はち}だから
\ruby{曝露}{さら|け}け
\ruby{出}{だ}して
\ruby{言}{い}ふことあ
\ruby{出來}{で|き}ず、
\ruby{自{\換字{分}}}{じ|ぶん}の
\ruby{家}{うち}に
\ruby{財産}{しん|だい}は
\ruby{無}{な}し、
\ruby{他}{ほか}に
\ruby{身寄}{み|より}も
\ruby{何}{なに}も
\ruby{無}{な}いから、
\ruby{楯}{たて}にして
\ruby{取}{と}る
\ruby{理屈}{り|くつ}が
\ruby{無}{な}いんで、とう〳〵
\ruby{駿府}{すん|ぷ}へ
\ruby{{\換字{連}}}{つ}れて
\ruby{行}{い}かれたアネ。
』

『だつて
\ruby{其}{それ}ぢやあ
\ruby{其}{そ}の
\ruby{建具屋}{たて|ぐ|や}の
\ruby{倅}{せがれ}が
\ruby{意氣地}{い|く|ぢ}が
\ruby{無}{な}さ
\ruby{{\換字{過}}}{す}ぎるぢやあ
\ruby{無}{ね}えか。
』

『それがお
\ruby{前}{まへ}、
\ruby{理由}{わ|け}があるからなんさ。
\ruby{其}{それ}あ
\ruby{其}{そ}の
\ruby{源}{げん}といふのにやあ
\ruby{嫁}{よめ}になる
\ruby{筈}{はず}の
\ruby{娘}{こ}が、
\ruby{親類内}{しん|るゐ|うち}に
\ruby{決定}{き|ま}つて
\ruby{居}{ゐ}たんで、つまり
\ruby{源}{げん}の
\ruby{方}{はう}ぢやあ
\ruby{初手}{しよ|て}から
\ruby{當座}{たう|ざ}の
\ruby{花}{はな}にしたんだネ。
だから
\ruby{彼}{あ}の
\ruby{娘}{こ}に
\ruby{捕}{つか}まへられて
\ruby{煮}{に}え
\ruby{詰}{つま}つた
\ruby{話}{はなし}をされる
\ruby{段}{だん}になりやあ、いつでも
\ruby{間}{ま}に
\ruby{合}{あは}せを
\ruby{云}{い}つて
\ruby{巧}{うま}く
\ruby{逃}{に}げて、とう〳〵
\ruby{逃}{に}げて〳〵
\ruby{惡}{わる}くも
\ruby{思}{おも}はれずに
\ruby{逃}{に}げおはせたんだよ。
』

『ヤ、そりやあ
\ruby{源}{げん}といふ
\ruby{奴}{やつ}あ
\ruby{酷}{むご}かつたナ、
お
\ruby{龍}{りゆう}こそ
\ruby{眞實}{ほん|と}に
\ruby{憫然}{かはい|さう}だ。
』

『ひどく
\ruby{御察}{お|さつ}しがいゝネ、
\ruby{何樣}{ど|う}かして
お
\ruby{{\換字{遣}}}{や}りナ。
』

『すぐと
\ruby{左樣皮肉}{さ|う|ひ|にく}を
\ruby{云}{い}はずともだ。
ウン、それから。
』

『そこで
\ruby{生木}{なま|き}を
\ruby{引裂}{ひき|さ}かれて
\ruby{駿府}{すん|ぷ}へ
\ruby{{\換字{連}}}{つ}れて
\ruby{行}{い}かれたんだから、
お
\ruby{龍}{りゆう}は
\ruby{矢}{や}も
\ruby{楯}{たて}も
\ruby{堪}{たま}りや
\ruby{仕}{し}ない、
\ruby{雨}{あめ}の
\ruby{降}{ふ}るやうに
\ruby{手紙}{て|がみ}を
\ruby{{\換字{遣}}}{よこ}したのさ。
ところが
\ruby{源}{げん}の
\ruby{方}{はう}が
\ruby{其心}{そ|れ}なんだから
\ruby{{\換字{返}}事}{へん|じ}も
\ruby{{\換字{遣}}}{や}らない。
\ruby{斷念}{あき|らめ}させやうといふんで
\ruby{關}{かま}はずに
\ruby{置}{お}くから、
お
\ruby{龍}{りゆう}は
\ruby{餘程恨}{よつ|ほど|うら}んだらしい。
それでも
\ruby{此方}{こつ|ち}ぢやあ
\ruby{關}{かま}はずに
\ruby{置}{お}くと、
\ruby{流石}{さす|が}は
\ruby{明治}{めい|じ}ツ
\ruby{子}{こ}だから
\ruby{氣}{き}が
\ruby{{\換字{強}}}{つよ}いネ、
\ruby{源}{げん}の
\ruby{家}{うち}へ
\ruby{押}{お}しかけやうつて
\ruby{云}{い}つて
\ruby{來}{き}たんだよ。
さあ、
\ruby{來}{こ}られちやあ
\ruby{大事}{おほ|ごと}だから
\ruby{源}{げん}は
\ruby{{\換字{弱}}}{よわ}つて、
\ruby{一{\換字{丈}}}{いち|ぢやう}もある
\ruby{手紙}{て|がみ}を
\ruby{三日}{みつ|か}もかゝつて
\ruby{書}{か}いて、
\ruby{親々}{おや|〳〵}の
\ruby{壓制}{おし|つけ}で
\ruby{仕方}{し|かた}が
\ruby{無}{な}くつて、
お
\ruby{前}{まへ}にやあ
\ruby{濟}{す}まないが
\ruby{實}{じつ}は
\ruby{既女房}{もう|によう|ばう}を
\ruby{貰}{もら}つた。
\ruby{腹}{はら}も
\ruby{立}{た}つだらうが
\ruby{何樣}{ど|う}か
\ruby{堪忍}{か|に}して
\ruby{{\換字{呉}}}{く}れ、
\ruby{二人}{ふ|たり}の
\ruby{中}{なか}は
\ruby{無}{な}い
\ruby{緣}{えん}と
\ruby{諦}{あきら}めて、
\ruby{汝}{おまへ}も
\ruby{叔母}{お|ば}さん
\ruby{次第}{し|だい}に
\ruby{好}{い}い
\ruby{婿}{むこ}を
\ruby{取}{と}つて
\ruby{榮}{さか}えてくれろ、と
\ruby{哀}{あは}れつぽく
\ruby{巧}{うま}く
\ruby{虛言}{う|そ}をついたネ。
』

『やれ〳〵!。
いよ〳〵
\ruby{酷}{むご}いナア、
\ruby{惡}{わる}い
\ruby{奴}{やつ}だ。
』

『するとお
\ruby{前}{まへ}、よく〳〵だつたと
\ruby{見}{み}えて、
\ruby{怖}{こは}い
\ruby{話}{はなし}さ!、
\ruby{忘}{わす}れもしない
\ruby{去年}{きよ|ねん}の
\ruby{一月}{いち|ぐわつ}の
\ruby{十三日}{じう|さん|にち}、
\ruby{{\換字{寒}}}{かん}の
\ruby{眞中}{さ|なか}の
\ruby{{\換字{雪}}}{ゆき}のふるのに、
\ruby{安倍川}{あ|べ|かは}とかいふ
\ruby{大}{おほき}な
\ruby{川}{かは}へ
\ruby{飛}{と}び
\ruby{{\換字{込}}}{こ}まうとしたさうさ。
\ruby{幸福}{しあ|はせ}に
\ruby{助}{たす}けられたから
\ruby{可}{い}いやうなものゝ、
\ruby{死}{し}なれりやあ
\ruby{差}{さ}し
\ruby{詰}{づ}め
\ruby{源}{げん}は
\ruby{取}{と}り
\ruby{憑}{つ}かれ
\ruby{無}{な}くちやあならないんだつたのさ。
』

『フム、それから。
』

『まあお
\ruby{待}{ま}ち。
さぞ
\ruby{湯}{ゆ}の
\ruby{中}{なか}で
\ruby{噴嚏}{くし|やみ}を
\ruby{仕}{し}て
\ruby{居}{ゐ}るだらう、
\ruby{憫然}{かはい|さう}に。
ハヽヽ。

\ruby{話}{はな}しながら
\ruby{飮}{や}るんで
\ruby{大層發}{たい|そう|はつ}したよ。
\ruby{駿府}{すん|ぷ}へ
\ruby{行}{い}つたのが
\ruby{一昨年}{をと|と|し}の
\ruby{夏}{なつ}の
\ruby{末}{すゑ}で、
\ruby{飛}{と}び
\ruby{{\換字{込}}}{こ}んだのが
\ruby{去年}{きよ|ねん}の
\ruby{一月}{いち|ぐわつ}だから、
\ruby{其間}{その|あひだ}の
\ruby{彼}{あ}の
\ruby{女}{こ}の
\ruby{事}{こと}を
\ruby{思}{おも}ふと
\ruby{實}{じつ}は
\ruby{憫然}{ふ|びん}さね。
だが
\ruby{驚}{おどろ}いたのは
\ruby{源}{げん}さ。
\ruby{離}{はな}れて
\ruby{居}{ゐ}る
\ruby{土地}{と|ち}だから
\ruby{助}{たす}かつたのか
\ruby{助}{たす}からなかつたも
\ruby{知}{し}りやうは
\ruby{無}{な}いし、とても
\ruby{生}{い}きて
\ruby{居}{ゐ}ても
\ruby{詰}{つま}らないから
\ruby{死}{し}んで
\ruby{仕舞}{し|ま}ふから
\ruby{憫然}{あ|はれ}と
\ruby{思}{おも}つて、
\ruby{一片}{いつ|ぺん}の
\ruby{囘向}{ゑ|かう}でも
\ruby{仕}{し}て
\ruby{{\換字{呉}}}{く}れろといふ
\ruby{淚}{なみだ}の
\ruby{痕}{あと}の
\ruby{一}{いつ}ぱいにある
\ruby{不氣味}{ぶ|き|み}な
\ruby{手紙}{て|がみ}を
\ruby{受取}{うけ|と}つたのだから、
\ruby{眞靑}{まつ|さを}になつて
\ruby{慄}{ふる}へて
\ruby{仕舞}{し|ま}つて、いよ〳〵
\ruby{死}{し}んで
\ruby{{\換字{終}}}{しま}つたものなら
\ruby{仕方}{し|かた}が
\ruby{無}{な}い、
\ruby{陰}{かげ}ながら
\ruby{法事}{はう|じ}でも
\ruby{仕}{し}て
\ruby{祟}{たゝ}りの
\ruby{來}{こ}ないやうに
\ruby{仕}{し}やうと、
\ruby{彼地}{あつ|ち}の
\ruby{新聞}{しん|ぶん}を
\ruby{取}{と}つて
\ruby{調}{しら}べて
\ruby{見}{み}ると、
\ruby{丁度其}{ちや|うど|そ}の
\ruby{手紙}{て|がみ}の
\ruby{日付}{ひ|づけ}の
\ruby{{\換字{翌}}日}{よく|じつ}の
\ruby{新聞}{しん|ぶん}に、
\ruby{美人}{び|じん}の
\ruby{投身}{みな|げ}といふ
\ruby{標題}{み|だし}があつて、
\ruby{彼}{あ}の
\ruby{名}{な}が
\ruby{見}{み}えたから
\ruby[g]{捼然}{ぎよつ}としたが、
\ruby{助}{たす}かつて
\ruby{叔母}{を|ば}の
\ruby{家}{うち}へ
\ruby{引渡}{ひき|わた}された、
\ruby{仔細}{し|さい}は
\ruby{解}{わか}らないが
\ruby{發狂}{はつ|きやう}した
\ruby{{\換字{所}}爲}{せ|ゐ}だらう、と
\ruby{書}{か}いてあつたのでホツと
\ruby{氣息}{い|き}を
\ruby{吐}{つ}いたネ。
』

『ン、そこで
\ruby{源}{げん}といふ
\ruby{奴}{やつ}は
\ruby{何樣}{ど|う}したエ。
』

\vspace{4zw}
\Large{天うつ浪 {\normalsize 第一{\換字{終}}}}
