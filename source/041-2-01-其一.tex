\makeatletter
\@ifundefined{全三巻@一括ビルド}{%
{\huge
\ruby{天}{そら} う つ %空白有り
\ruby{浪}{なみ}}  {\normalsize 第二}
\vspace*{3zw}
\Entry{其一}
}
\makeatother

% メモ 校正終了 2024-04-12
\原本頁{1-3}%
『
それからと
いふものは
\ruby{當人}{たう|にん}も、
%
\ruby{一旦}{いつ|たん}
\ruby{死}{し}ぬとまで
\ruby{云}{い}つて
\ruby{{\換字{遺}}}{よこ}して
\原本頁{1-4}\改行%
\ruby{置}{お}いて
\ruby{死損}{しに|そこな}つたんだから、
%
\ruby{流石}{さす|が}に
きまりが
\ruby{惡}{わる}いか
\ruby{羞}{はづか}しいかして、
%
パツタリと
\ruby{音沙汰}{おと|さ|た}を
\ruby{聞}{き}かせなかつたが、
%
\ruby{何}{なん}でも
\ruby{後}{あと}で
\ruby{聞}{き}いた
\ruby{談話}{はな|し}の
\ruby{模樣}{も|やう}で
\ruby{考}{かんが}へて
\ruby{見}{み}ると、
%
\ruby{一霎時}{しば|ら|く}の
\ruby{間}{あひだ}は
\ruby{茫然}{ぼう|つ}と
\ruby{仕}{し}て
\ruby{居}{ゐ}たんだ\換字{子}。
%
\原本頁{1-8}\改行%
\ruby{其}{その}
\ruby{中}{うち}に
\ruby{段々}{だん|〴〵}
\ruby{氣}{き}が
\ruby{鎭}{しづ}まつて、
%
\ruby{斷念}{あき|らめ}が
\ruby{自然}{ひと|りで}に
ついて
\ruby{來}{く}ると、
%
\ruby{源}{げん}の
\ruby{事}{こと}は
\ruby{捨}{す}てたものと
\ruby{仕}{し}て
\ruby{仕舞}{し|ま}つた
\ruby{樣子}{やう|す}だが、
%
さあ
\ruby{持{\換字{前}}}{もち|まへ}の
\ruby{氣性}{き|しやう}のある
ところへ
\ruby{捨鉢}{すて|ばち}が
\ruby{加勢}{か|せい}したから、
%
\ruby{輪}{わ}をかけて
お
\ruby{狭}{きやん}に
なつたと
\ruby{見}{み}えて、
%
\ruby{叔母}{を|ば}が
\ruby{婿}{むこ}にと% (婿 5a7f) 聟 805f
\ruby{定}{き}めた
\ruby{男}{をとこ}を
\ruby{{\換字{嫌}}}{きら}つて、
%
{---}{---}
\ruby{何}{なん}でも
\ruby{其}{そ}の
\ruby{男}{をとこ}が
\ruby{挊}{かせ}いで
\ruby{金錢}{か|ね}を
\ruby{溜}{た}めるより
ほかにやあ
\ruby{何}{なに}も
\ruby{知}{し}らない、
%
\ruby{人}{ひと}の
\ruby{{\換字{情}}合}{じやう|あひ}や
\ruby{意氣}{い|き}
\ruby{張}{はり}も
\ruby{{\換字{分}}}{わか}らない
\ruby{牛}{うし}の
やうな
\ruby{男}{をとこ}に
なると
いふので
だゝを
\ruby{捏}{こね}たのだから、
%
\原本頁{2-4}\改行%
とう〳〵
\ruby{大{\換字{紛}}紜}{おほ|ごた|〳〵}が
\ruby{持}{も}ち
\ruby{上}{あが}つたのさ。
%
\ruby{叔母}{を|ば}は
\ruby{{\換字{古}}風}{むかし|ふう}
だから
\ruby{嵩}{かさ}に
かゝつて、
%
\ruby{妾}{わたし}が
\ruby{可}{い}いと
\ruby{定}{き}めて
\ruby{先方}{む|かふ}へも
\ruby{話}{はな}したのだから、
%
\ruby{今{\換字{更}}}{いま|さら}
\ruby{變改}{へん|がい}は
\原本頁{2-6}\改行%
\ruby{出來}{で|き}も
\ruby{仕}{し}ないし、
%
また
\ruby{金}{かね}は
あり
\ruby{人物}{じん|ぶつ}は
\ruby{堅}{かた}し、
%
\ruby{婿}{むこ}にして% (婿 5a7f) 聟 805f
\ruby{不足}{ふ|そく}のある
\ruby{男}{をとこ}でも
\ruby{無}{な}いのに、
%
\ruby{厭}{いや}の
\ruby{{\換字{嫌}}}{きら}ひの
といふのは
\ruby{我儘}{わが|まゝ}だと
\ruby{叱}{しか}れば、
%
\ruby{男}{をとこ}の
\ruby{方}{はう}からも
\ruby{喧}{やかま}しく
\ruby{逼}{せま}つて
\ruby{來}{く}るので、
%
お
\ruby{龍}{りう}も
\ruby{氣}{き}の
\ruby{{\換字{弱}}}{よわ}い
\ruby{娘}{こ}なら
\ruby{折}{を}れて
\ruby{仕舞}{し|ま}つて、
%
\ruby{其男}{そ|れ}を
\ruby{婿}{むこ}にして% (婿 5a7f) 聟 805f
\ruby{身}{み}を
\ruby{固}{かた}める
ところだつたが、
%
\ruby{毅然}{しつ|かり}として
\ruby{{\換字{分}}別}{ふん|べつ}が
あると
\ruby{云}{い}ふんでは
\ruby{無}{な}いけれど、
%
\ruby{妙}{めう}に
\ruby{氣}{き}の
\ruby{冴}{さ}えた、
%
\原本頁{2-11}\改行%
\ruby{萎}{ひ}けて
\ruby{居}{ゐ}ない
\ruby{娘}{こ}だから、
%
たとへ
\ruby{金}{かね}が
\ruby{無}{な}くつて
\ruby{人物}{じん|ぶつ}が
\ruby{堅}{かた}くなくつて
\ruby{一眼}{めつ|かち}で
\ruby{跛足}{びつ|こ}で
\ruby{有}{あ}らうとも、
%
\ruby{其}{そ}の
\ruby{心意氣}{こゝろ|い|き}さへ
\ruby{妾}{わたし}の
\ruby{氣}{き}に
\ruby{入}{い}りやあ、
%
\原本頁{3-2}\改行%
\ruby{妾}{わたし}あ
\ruby{亭主}{てい|しゆ}にでも
\ruby{何}{なん}にでも
するが、
%
\ruby{味}{あぢ}の
\ruby{無}{な}い
\ruby{石{\換字{瓦}}}{いし|かはら}のやうな
\ruby{人}{ひと}に
\ruby{添}{そ}ふ
\ruby{事}{こと}あ
\ruby{出來}{で|き}ません。
%
\ruby{第一}{だい|いち}
\ruby[||j>]{妾}{わたし}あ
\ruby{人}{ひと}の
\ruby{緣合}{{\換字{𛀁}}ん|あひ}の
\ruby{談}{はなし}に、
%
\ruby{目上}{め|うへ}の
\ruby{者}{もの}が
\ruby{壓制}{おし|つけ}
\原本頁{3-4}\改行%
わざを
\ruby{仕}{し}やうとするのは
\ruby{蟲}{むし}が
\ruby{{\換字{嫌}}}{きら}つて
なりません。
%
\ruby{大}{おほ}きな
\ruby{御世話}{お|せ|わ}です。
%
\ruby{要}{い}らない
\ruby{事}{こと}です。
%
\ruby{妾}{わたし}あ
もう
\ruby{人}{ひと}の
\ruby{内君}{おかみ|さん}
なんぞに
なれなくつたつて
\ruby{構}{かま}はない
\ruby{身體}{から|だ}です。
%
\ruby{好}{す}きな
\ruby{人}{ひと}になら
\ruby{妾}{めかけ}にでも
\ruby{{\換字{情}}{\換字{婦}}}{い|ろ}にでも
\原本頁{3-7}\改行%
なつて
\ruby{與}{や}る
\ruby{代}{かは}り、
%
\ruby{{\換字{嫌}}}{いや}な
\ruby{人}{ひと}になら
\ruby{奧樣}{おく|さん}になれ
\ruby{御臺{\換字{所}}}{み|だい|どころ}になれ
つて
\ruby{云}{い}はれたつて
\ruby{{\換字{嫌}}}{いや}な
\ruby{事}{こと}です。
%
と
\ruby{恐}{おそ}ろしい
\ruby{亂暴}{らん|ばう}を
\ruby{云}{い}つて
\ruby{叔母}{を|ば}と
\ruby{舌戰}{やり|あ}つたさうだよ。
』

\原本頁{3-10}%
『なる
\ruby{程}{ほど}なア!。
%
\ruby{考}{かんが}へて
\ruby{見}{み}りやあ
\ruby{愍然}{あは|れ}な
ところが
\ruby{心底}{しん|そこ}には
あるぜ!。
%
しかし
\ruby{餘程}{よつ|ぽど}
\ruby{異樣}{お|つ}な
\ruby{出來}{で|き}の
\ruby{娘}{こ}だナ!。
』

\原本頁{4-1}%
『マア
\ruby{左樣}{さ|う}さ。
%
\ruby{自棄}{や|け}から
\ruby{出}{で}て
\ruby{居}{ゐ}る
\ruby{料簡}{れう|けん}なんだから
\ruby{云}{い}つて
\ruby{見}{み}りやあ
\ruby{愍然}{かあ|いさう}な% 「愍然 か(あ)いさう」
ところもあるのさ。
%
それでも
\ruby{先方}{む|かふ}の
\ruby{男}{をとこ}が
\ruby{氣}{き}が
\ruby{{\換字{廻}}}{まは}ら
\ruby{無}{な}くつて、
%
\ruby{話}{はなし}が
\ruby{中々}{なか|〳〵}
\ruby{壞}{こは}れないので、
%
そこで
\ruby{彼}{あ}の
\ruby{娘}{こ}は
\ruby{癇癪}{かん|しやく}を
\ruby{起}{おこ}して、
%
\原本頁{4-4}\改行%
\ruby{今年}{こ|とし}の
\ruby{三月}{さん|ぐわつ}に
\ruby{駿府}{すん|ぷ}を
\ruby{脫}{ぬ}け
\ruby{出}{だ}し、
%
\ruby{何}{なん}といふ
\ruby{目的}{め|あて}が
あつたのじやあ
\ruby{無}{な}いが、
%
\ruby{何}{なに}を
するにしても
\ruby{自{\換字{分}}}{じ|ぶん}の
\ruby{{\換字{勝}}手}{かつ|て}に
\ruby{世}{よ}を
\ruby{{\換字{送}}}{おく}らう
といふんで、
%
ふらりと
\ruby{東京}{とう|きやう}へ
\ruby{{\換字{遣}}}{や}つて
\ruby{來}{き}たのさ。
%
すると
\ruby{銀座}{ぎん|ざ}の
\ruby{往來}{わう|らい}で
もつて、
%
ひよつくりと
\ruby{源}{げん}に
\ruby{會}{あ}つたゞらうぢや
\ruby{無}{な}いか!。
』

\原本頁{4-8}%
『ヤ、
%
そりやあ
\ruby{大變}{だい|へん}だ!。% 原本でも濁音の「だいへん」
%
おもしろい、
%
\ruby{面白}{おも|しろ}い!。
%
さうして、
』

\原本頁{4-9}%
『
\ruby{源}{げん}は
\ruby{只}{たゞ}
\ruby{無暗}{む|やみ}に
\ruby{雜沓}{ひと|ごみ}へ
\ruby{入}{はい}つて
\ruby{{\換字{逃}}}{に}げて
\ruby{仕舞}{し|ま}つたんだが、
%
\ruby{其}{そ}の
\ruby{時}{とき}の
\ruby{樣子}{やう|す}を
\ruby{見}{み}て
\ruby{可怪}{を|かしい}と% ここのルビは「をかしい」
\ruby{勘付}{かん|づ}いたから、
%
さあ
\ruby{彼娘}{あ|れ}は
\ruby{竊}{そつ}と
\ruby{源}{げん}の
\ruby{樣子}{やう|す}を
\ruby{内内}{ない|〳〵}で% 原本通り、行末禁則で非踊り字
\ruby{捜}{さぐ}つたんだね。
%
すると
\ruby{源}{げん}の
\ruby{心中}{は|ら}が
\ruby{眞實}{ほん|とう}に
\ruby{讀}{よ}めたから、
%
どんなにか
\ruby{口惜}{く|やし}がつた
\ruby{事}{こと}だらうさ!。
』

\原本頁{5-2}%
『ン、
%
\ruby{{\換字{道}}理}{もつ|とも}だ、
%
\ruby{口惜}{く|やし}かつたらうさ!。
』

\原本頁{5-3}%
『
\ruby{甚}{ひど}く
\ruby{力}{ちから}を
\ruby{入}{い}れるね、
%
\ruby{可怪}{を|かし}いよ。% ここのルビは「をかし」
%
それから
お
\ruby{{\換字{前}}}{まへ}
\ruby{何處}{ど|こ}に
\ruby{何樣}{ど|う}して
\ruby{買}{か}つたんだか、
%
\ruby{人}{ひと}を
\ruby{騙}{だま}して
\ruby{取}{と}りでも
\ruby{仕}{し}たんだか、
%
\ruby{袂}{たもと}の
\ruby{中}{なか}に
\ruby{短銃}{ぴす|とる}を
\ruby{秘}{かく}してね。
』

\原本頁{5-6}%
『ヨーツ!、
%
\ruby{凄}{すご}いナア。
』

\原本頁{}%
『たしか
\ruby{銀鼠}{ぎん|ねず}だつたと
\ruby{思}{おも}つたが
\ruby{薄}{うす}い
\ruby{色}{いろ}の
\ruby{頭巾}{づ|きん}を
\ruby{深}{ふか}く
\ruby{被}{かぶ}つて、
%
\ruby{四日}{よ|つか}の
お
\ruby{月樣}{つき|さま}の
\ruby{丁度}{ちやう|ど}
\ruby{出}{で}て
\ruby{居}{ゐ}た
\ruby{暮合}{くれ|あひ}の
\ruby{點燈頃}{ひ|ともし|ごろ}を、
%
\ruby{源}{げん}の
\ruby{家}{うち}の
\ruby{横丁}{よこ|ちやう}の
\ruby{角}{かど}に
\ruby{立}{た}つて
\ruby{居}{ゐ}たのが
\ruby{此}{こ}の
\ruby{四月}{し|ぐわつ}だア{\換字{子}}。
%
ほんとに
\ruby{考}{かんが}へて
\ruby{見}{み}ると
\ruby{怖}{こは}い
\ruby{事}{こと}さ{\換字{子}}。
』

\原本頁{5-11}%
『ン、ン、
』

\原本頁{6-1}%
『それを
ちらりと
\ruby{見}{み}た
\ruby{妾}{わたし}あ、
%
\ruby{銀座}{ぎん|ざ}で
\ruby{會}{あ}つた
といふ
\ruby{話}{はなし}も
\ruby{源}{げん}から
\ruby{聞}{き}
\原本頁{6-2}\改行%
いてたから、
%
こりやあと
\ruby{氣取}{け|ど}つて
\ruby{仕舞}{し|ま}つて
\ruby{話}{はなし}を
\ruby{仕掛}{し|か}けて、
%
\ruby{其}{その}
\ruby{晩}{ばん}は
\ruby{吾家}{う|ち}へ
\ruby{寢}{ね}させて
\ruby{置}{お}いて、
%
\ruby{源夫{\換字{婦}}}{げん|ふう|ふ}に
\ruby{内{\換字{通}}}{ない|つう}を
\ruby{仕}{し}て
\ruby{{\換字{遣}}}{や}つたんで、
%
\ruby{何事}{なに|ごと}も
\ruby{起}{おこ}らずに
\ruby{濟}{す}んで
\ruby{仕舞}{し|ま}つたんだよ。
』

\原本頁{6-5}%
『アヽ
\ruby{惜}{をし}い
\ruby{事}{こと}を
\ruby{仕}{し}た。
%
ドンと
\ruby{{\換字{遣}}}{や}らせりやあ
\ruby{好}{よ}かつたのに!。
』

\原本頁{6-6}%
『
\ruby{戱談}{じやう|だん}ぢや
\ruby{無}{な}いよ、
%
\ruby{下}{くだ}らない!。
%
\ruby{源夫{\換字{婦}}}{げん|ふう|ふ}は
\ruby{怖}{こは}がつて〳〵{\換字{子}}、
%
\ruby{弟子}{で|し}のあるのを
\ruby{幸}{さいは}ひに
\ruby{仙臺}{せん|だい}へ
\ruby{竊}{そつ}と
\ruby{挊}{かせ}ぎに
\ruby{行}{い}つて
\ruby{仕舞}{し|ま}つたのさ。
%
そ
\原本頁{6-8}\改行%
れから
いろ〳〵
\ruby{理解}{り|かい}を
\ruby{云}{い}つて
\ruby{聞}{き}かせて、
%
とう〳〵
\ruby{宅}{うち}へ
\ruby{置}{お}くやうに
\ruby{仕}{し}たんだが、
%
\ruby{左樣}{さ|う}いつた
\ruby{氣性}{き|しやう}だから
また
\ruby{男}{をとこ}が
\ruby{好}{す}いて、
%
\ruby{段々}{だん|〴〵}と
\ruby{妾}{わたし}の
ためになり
\ruby{出}{だ}したね。
』

\原本頁{6-11}%
『そりやあ
\ruby{左樣}{さ|う}
だらう、
%
\ruby{中々}{なか|〳〵}
\ruby{價}{ね}の
\ruby{踏}{ふ}める
\ruby{奇貨}{しろ|もの}だわエ。
』

\原本頁{7-1}%
『そこで
\ruby{{\換字{前}}{\換字{途}}}{さき|〴〵}の
\ruby{考案}{かん|がへ}も
あるから、
%
すつかり
\ruby{妾}{わたし}のものに
\ruby{仕}{し}て
\ruby{仕舞}{し|ま}はうと
\ruby{思}{おも}ふんだが\換字{子}、
%
まあ
\ruby{第一}{だい|いち}に
チヤンと
\ruby{關係}{ひつ|かゝり}を
\ruby{切}{き}つちまは
なけりやあ
ならないのが
\ruby{叔母}{を|ば}の
\ruby{方}{はう}だ\換字{子}。
』

\原本頁{7-4}%
『ン、
%
そりやあ
\ruby{譯}{わけ}は
\ruby{無}{ね}え、
%
\ruby{乃公}{お|れ}が
\ruby{法}{はふ}をかいて
\ruby{{\換字{遣}}}{や}らあ。
』
