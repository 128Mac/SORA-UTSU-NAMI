\Entry{其二}

% メモ 校正終了 2024-04-12 2024-05-28 2024-06-28
\原本頁{7-6}%
『
\ruby{眞實}{ほん|と}に
\ruby{譯無}{わけ|な}く
\ruby{此方}{こつ|ち}のものに% ルビ調整(原本通り)
\ruby{出來}{で|き}やうか\換字{子}。
』

\原本頁{7-7}%
『
\ruby{出來}{で|き}るともさ!。
%
\ruby{併}{しか}し
\ruby{{\換字{戸}}籍}{せ|き}まで
といふ
\ruby{譯}{わけ}にやあ
いかねえ。
%
\ruby{籍}{せき}まで
\ruby{御{\換字{前}}}{お|まへ}の
\ruby{娘}{むすめ}にしやうと
\ruby{云}{い}ふにやあ、
%
お
\ruby{{\換字{前}}}{まへ}が
\ruby{岩崎}{いは|さき}の% 原本のこの部分は「いわさき」
\ruby{家}{うち}を
\ruby{{\換字{退}}}{の}いて
\ruby{仕舞}{し|ま}つて
\ruby{一本立}{いつ|ぽん|だち}に
なるか、
%
\ruby{二人}{ふた|り}の
\ruby{子}{こ}を
\ruby[||j>]{順}{じゆん}
\ruby[||j>]{々}{ 〳〵}に
\ruby{白痴}{ば|か}か
\ruby{瘋癲}{きち|がひ}かに
\ruby{云}{い}
\原本頁{7-10}\改行%
ひ
\ruby{立}{た}てゝ、% 踊り字調整「〻(二の字点、揺すり点)に見えるが(ゝ)」
%
\ruby{相續權}{さう|ぞく|けん}の
\ruby{無}{な}いやうに
\ruby{仕}{し}て
\ruby{仕舞}{し|ま}はなけりや
ならねえが
\改行% 校正作業の簡略化のため
、
%
\原本頁{8-1}\改行%
そんな
\ruby{事}{こと}は
\ruby{迚}{とて}も
\ruby{出來}{で|き}ることぢやあ
\ruby{無}{ね}え。
』

\原本頁{8-2}%
『
そんな
\ruby{事}{こと}は
\ruby{損}{そん}になるから
\ruby{詰}{つま}らないや\換字{子}。
%
いくら
\ruby{氣}{き}に
\ruby{入}{い}らない
% \原本頁{8-3}\改行%
\ruby{子}{こ}だつて
\ruby{何}{なん}だつて、
%
\ruby{{\換字{若}}}{わか}い
\ruby{者}{もの}あ
\ruby{何樣}{ど|ん}なに
\ruby{出世}{しゆつ|せ}を
するか
\ruby{知}{し}れや
\ruby{仕}{し}ないんだもの!、
%
\ruby{金錢}{お|あし}が
\ruby{要}{い}らないなら
\ruby{其}{そ}の
\ruby{親}{おや}に
なつてる
\ruby{方}{はう}が
\ruby{利}{り}に
\ruby{當}{あた}るぢや
\ruby{無}{な}いか。
』

\原本頁{8-6}%
『
ハヽヽ、
%
\ruby{繼子}{まゝ|つこ}% 踊り字調整「〻(二の字点、揺すり点)に見えるが(ゝ)」
\ruby{二人}{ふた|り}の
\ruby{親}{おや}に
なつてるのを、
%
\ruby{抽籤}{く|じ}
\ruby{{\換字{前}}}{まへ}の
\ruby{勸業銀行債{\換字{券}}}{くわ|ん|ぎ|ん|さい|けん}
でも
\ruby{持}{も}つてるやうに
\ruby{思}{おも}つてるのか。
%
ハヽヽ
お
\ruby{{\換字{前}}}{まへ}は
\ruby{眞實}{ほん|と}に
\ruby{怜悧}{り|かう}だ、% ルビ調整(原本通り)(りかう)
%
\ruby{好}{い}い
\ruby{料簡}{れう|けん}だ、
%
\ruby{{\換字{感}}心}{かん|しん}した。
』

\原本頁{8-9}%
『
お
\ruby{冷}{ひ}やかしで
\ruby{無}{な}いよ、
%
\ruby{馬鹿}{ば|か}にしてツ!。
』

\原本頁{8-10}%
『
それだつて
\ruby{二人}{ふた|り}も
\ruby{子}{こ}があるのに
\ruby{其}{その}
\ruby{上}{うへ}に
\ruby{{\換字{又}}}{また}、
%
あの
お
\ruby{龍}{りう}も
\ruby{眞實}{ほん|と}の
\ruby{養女}{やう|ぢよ}に
\ruby{爲}{し}やうなんて、
%
あんまり
\ruby{蟲}{むし}が
\ruby{好}{よ}さ
\ruby{{\換字{過}}}{す}ぎるからナ。
』

\原本頁{9-1}%
『
ぢやあ
\ruby{何樣}{ど|う}すりやあ
\ruby{可}{いゝ}と% 踊り字調整「〻(二の字点、揺すり点)に見えるが(ゝ)」
いふのかエ。
』

\原本頁{9-2}%
『
\ruby{何樣}{ど|う}するも
\ruby{此樣}{こ|う}するも
\ruby{要}{い}る
\ruby{事}{こと}ぢやあ
\ruby{無}{な}い。
%
つまり
\ruby{叔母}{を|ば}
といふ
\ruby{奴}{やつ}が
\ruby{頭}{あたま}さへ
\ruby{出}{だ}して
\ruby{來}{こ}ないやうに
すりやあ
\ruby{好}{い}いのだらう。
』

\原本頁{9-4}%
『
\ruby{左樣}{さ|う}さ!。
%
\ruby{彼女}{あ|れ}を
\ruby{囮}{をとり}にして
\ruby{穫}{と}つた
\ruby{禽}{とり}を、
%
\ruby{他}{ひと}の
\ruby{手}{て}へ
\ruby{取}{と}られるやうな
\ruby{事}{こと}さへ
\ruby{無}{な}きやあ
\ruby{畢竟}{つま|り}
いゝのさ。% 踊り字調整「〻(二の字点、揺すり点)に見えるが(ゝ)」
』

\原本頁{9-6}%
『
だから
\ruby{譯}{わけ}は
\ruby{無}{ね}えといふのだ、
%
\ruby{矢筈}{や|はず}にかけるんだナ!。
』

\原本頁{9-7}%
『
\ruby{矢筈}{や|はず}にかけるつて、
%
\ruby{何樣}{ど|う}いふやうに?。
』

\原本頁{9-8}%
『
\ruby{叔母}{を|ば}の
ところへ
ポーンと
\ruby{一本}{いつ|ぽん}
\ruby{手紙}{て|がみ}を
\ruby{{\換字{遣}}}{や}つて、
%
\ruby{斯樣}{か|う}いふことを
\ruby{云}{い}つて
\ruby{{\換字{遣}}}{や}るのだ。
%
\ruby{妾}{わたし}は
\ruby{師匠}{し|ゝやう}と% 踊り字調整「〻(二の字点、揺すり点)に見えるが(ゝ)」
\ruby{弟子}{で|し}との
\ruby{緣}{{\換字{𛀁}}ん}で、
%
\ruby{其方}{そつ|ち}の
お
\ruby{龍}{りう}さんを
\makeatletter
\@ifundefined{デバッグ@ビルド}{%
  \ruby[<j||]{何}{なん }
  \ruby[<j||]{月}{ぐわつ}
}{%
  \ruby[||j|]{何}{なん}% 三単語の間に送り仮名ゼロのため、若干原本のルビ配置とは異なる
  \ruby[|j||]{月}{ぐわつ}
}%
\makeatother
% \ruby{何月}{なん|ぐわつ}
\ruby{以來}{この|かた}
\ruby[||j>]{食}{かゝり}% 踊り字調整「〻(二の字点、揺すり点)に見えるが(ゝ)」
\ruby[||j>]{客}{ うど}に
% \ruby{食客}{かゝり|うど}に% 踊り字調整「〻(二の字点、揺すり点)に見えるが(ゝ)」
\ruby{仕}{し}てゐます。
%
\ruby{聞}{き}けば
お
\ruby{龍}{りう}さんは
\ruby{複雜}{いり|く}んだ
\ruby{譯}{わけ}で、
%
\ruby{其方}{そつ|ち}を
\ruby{無言}{む|ごん}で
\ruby{出}{で}て
\ruby{來}{き}たのださうだが、
%
\ruby{一季}{いつ|き}
\ruby{{\換字{半}}季}{はん|き}の
\ruby{奉公人}{ほう|こう|にん}でも、
%
\原本頁{10-1}\改行%
\ruby{定}{き}める
ところは
\ruby{確然}{しや|ん}と
\ruby{定}{き}める
\ruby{{\換字{習}}}{ならひ}だから、
%
\ruby{何}{なん}の
\ruby{定}{きまり}も
\ruby{無}{な}しに
\ruby{無際限}{む|さい|げん}に
\ruby{置}{お}く
\ruby{譯}{わけ}にはいか
\ruby{無}{な}い。
%
\ruby{當人}{たう|にん}の
\ruby{料簡}{れう|けん}ぢやあ
\ruby{其方}{そつ|ち}へは
\ruby{歸}{かへ}りたく
\ruby{無}{な}い、
%
\ruby{此地}{こち|ら}で
\ruby{藝}{げい}の
\ruby{師匠}{し|ゝやう}でも% 踊り字調整「〻(二の字点、揺すり点)に見えるが(ゝ)」
\ruby{仕}{し}て
\ruby{暮}{くら}したいと
\ruby{云}{い}ふ
\ruby{事}{こと}だし、
%
\ruby{妾}{わたし}の
\ruby{目}{め}で
\原本頁{10-4}\改行%
\ruby{見}{み}ても
\ruby{當人}{たう|にん}の
\ruby{藝}{げい}の
\ruby{性質}{た|ち}に
\ruby{見{\換字{込}}}{み|こみ}が
あるから、
%
\ruby{{\換字{若}}}{も}し
\ruby{全}{まつた}く
お
\ruby{龍}{りう}さんを
\ruby{妾}{わたし}の
\ruby[||j>]{娘}{むすめ}
\ruby[||j>]{{\換字{分}}}{ ぶん}にして、
% \ruby{娘{\換字{分}}}{むすめ|ぶん}にして、
%
\ruby{妾}{わたし}の
\ruby{跡}{あと}を
\ruby{襲}{つ}がせても
\ruby{宜}{い}いと
\ruby{云}{い}ふのなら、
%
\ruby{今}{いま}までも
\ruby{世話}{せ|わ}を
\ruby{仕}{し}たが
\ruby{{\換字{猶}}}{なほ}
\ruby{此}{この}
\ruby{上}{うへ}とも
%
\ruby{立派}{りつ|ぱ}に
\ruby{藝}{げい}の
\ruby{成就}{でき|あが}るまでは
\ruby{何年}{なん|ねん}でも
\原本頁{10-7}\改行%
\ruby{世話}{せ|わ}を
\ruby{爲}{し}やうが、
%
そちらが
\ruby{左樣}{さ|う}いふ
\ruby{氣}{き}で
\ruby{無}{な}ければ
\ruby{此地}{こち|ら}でも
\ruby{困}{こま}る
\改行% 校正作業の簡略化のため
。
%
\原本頁{10-8}\改行%
たゞ% 踊り字調整「〻(二の字点、揺すり点)に濁点に見えるが(ゞ)」
べん〳〵とは
\ruby{世話}{せ|わ}も
\ruby{出來}{で|き}ぬから、
%
\ruby{今}{いま}
\ruby{迄}{ゝで}% 踊り字調整「〻(二の字点、揺すり点)に見えるが(ゝ)」
\ruby{世話}{せ|わ}を
\ruby{爲}{し}た
\ruby{食雜用}{くひ|ざふ|よう}を
\ruby{入}{い}れて、
%
\ruby{其方}{そつ|ち}へ
\ruby{引取}{ひき|と}つて
\ruby{貰}{もら}ひたいものだ。
%
しかし
\ruby{當人}{たう|にん}は
\ruby{何樣}{ど|う}
いふものだか、
%
\ruby{甚}{ひど}く
\ruby{其方}{そち|ら}の
\ruby{事}{こと}を
\ruby{惡}{わる}く
\ruby{云}{い}つて、
%
\ruby{田舎}{ゐな|か}へ
\ruby{{\換字{返}}}{かへ}される
\ruby[<j||]{位}{くらゐ}% 行末行頭の境界付近なので特例処置を施す
な
\原本頁{10-11}\改行%
ら
\ruby{舌}{した}を
\ruby{咬}{か}んで
\ruby{死}{し}ぬなぞと、
%
\ruby{無茶}{む|ちや}な
\ruby{事}{こと}を
\ruby{云}{い}つて
\ruby{居}{ゐ}て
\ruby{眞}{まこと}に
\ruby{困}{こま}ります
\改行% 校正作業の簡略化のため
。
%
\原本頁{11-1}\改行%
と
\ruby{斯樣}{か|う}いふやうに
\ruby{掛合}{かけ|あ}ふのだ。
%
\ruby{{\換字{遣}}}{よこ}すなら
\ruby{緣切}{{\換字{𛀁}}ん|きり}にしろ、
%
\ruby{{\換字{返}}}{かへ}せなら
\ruby{食雜用}{くひ|ざふ|よう}を
\ruby{入}{い}れろと、
%
\ruby{金額}{かね|だか}を
\ruby{大袈裟}{おゝ|げ|さ}にして% 踊り字調整「〻(二の字点、揺すり点)に見えるが(ゝ)」
どうだ〳〵で
\ruby{責}{せ}めるのさ。
%
さう
すりやあ
\ruby{大槪}{たい|がい}
\ruby{姪一人}{めひ|ゝと|り}% 踊り字調整「〻(二の字点、揺すり点)に見えるが(ゝ)」
\ruby{捨}{す}てた
\ruby{氣}{き}にならうぜ。
』

\原本頁{11-4}%
『
でも
\ruby{食雜用}{くひ|ざふ|よう}ぢや
\ruby{月十圓}{つき|じふ|ゑん}に
したつて
\ruby{知}{し}れたもんだから、
%
\ruby{大袈裟}{おゝ|げ|さ}% 踊り字調整「〻(二の字点、揺すり点)に見えるが(ゝ)」
に
\ruby{爲}{し}やうも
\ruby{無}{な}いぢや
\ruby{無}{な}いか。
』

\原本頁{11-6}%
『
\ruby{智慧}{ち|ゑ}の
\ruby{無}{な}い
\ruby{事}{こと}を
\ruby{云}{い}つたものだ!。
\ruby{衣服}{き|もの}や
\ruby{髮{\換字{飾}}}{かみ|かざ}りを
\ruby{少}{すこ}し
\ruby{買}{か}つて
\ruby{{\換字{遣}}}{や}つて
\ruby{置}{お}きやあ、
%
\ruby{大}{たい}した
\ruby{金額}{かね|だか}に
\ruby{註加}{つけ|かけ}が
\ruby{出來}{で|き}らあナ。
』

\原本頁{11-8}%
『
\ruby{成程}{なる|ほど}ネ。
%
それでも
\ruby{{\換字{連}}}{つ}れて
\ruby{歸}{かへ}つたらば?。
』

\原本頁{11-9}%
『
その
\ruby{時}{とき}はまた
\ruby{後}{あと}で
\ruby{策}{さく}を
\ruby{爲}{す}るとして、
%
\ruby{食雜用}{くひ|ざふ|よう}と
\ruby{緣切}{{\換字{𛀁}}ん|きり}とで
\ruby[<j||]{一}{ちよ }% 行末行頭の境界付近なので特例処置を施す
\ruby[<j||]{寸}{つと }
\ruby[<j||]{{\換字{暖}}}{あたゝ}まつて% 踊り字調整「〻(二の字点、揺すり点)に見えるが(ゝ)」% 行末行頭の境界付近なので特例処置を施す
\ruby{湯治}{たう|じ}とでも
\ruby{洒落}{しや|れ}たが
\ruby{宜}{い}い。
』

\原本頁{11-11}%
『
\ruby{緣切}{{\換字{𛀁}}ん|きり}とはエ?。
』

\原本頁{12-1}%
『
お
\ruby{龍}{りう}が
\ruby{駿府}{すん|ぷ}へ
\ruby{{\換字{連}}}{つれ}て
\ruby{行}{ゆ}かれると
\ruby{定}{きま}つたら、
%
お
\ruby{{\換字{前}}}{まへ}が
\ruby{源}{げん}の
\ruby{親{\換字{父}}}{おや|ぢ}へ
\ruby{衝突}{ぶつ|か}つて、
%
\ruby{此方}{こち|ら}の% ルビ調整(原本通り)
\ruby{息子}{むす|こ}さんが
\ruby{惡}{わる}いのだから、
%
\ruby[<j||]{{\換字{空}}}{にぎり}
\ruby[||j>]{{\換字{拳}}}{こぶし}
では
\ruby{話}{はなし}は
\ruby{濟}{す}みますまい、
%
いくらかの
\ruby{手切金}{て|ぎ|れ}を
\ruby{御與}{お|や}んなすつて、
%
\ruby{彼}{あ}の
\ruby{娘}{こ}を
\ruby{駿府}{すん|ぷ}へ
\原本頁{12-4}\改行%
\ruby{歸}{かへ}らせた
\ruby{方}{はう}が
\ruby{宜}{よ}う
ございましやう、
%
\ruby{左樣}{さ|う}
\ruby{爲}{し}ないと
\ruby{何時}{い|つ}までも
\ruby[||j>]{關}{ひつ}
\ruby[||j>]{係}{かゝり}があつて、% 踊り字調整「〻(二の字点、揺すり点)に見えるが(ゝ)」
% \ruby{關係}{ひつ|かゝり}があつて、% 踊り字調整「〻(二の字点、揺すり点)に見えるが(ゝ)」
%
\ruby{何樣}{ど|ん}な
\ruby{事}{こと}が
\ruby{起}{おこ}るか
\ruby{知}{し}れませんから、
%
と
\ruby{少}{すこ}し
\ruby{巧}{うま}く
\ruby{口}{くち}を
きゝやあ% 踊り字調整「〻(二の字点、揺すり点)に見えるが(ゝ)」
\ruby{必定}{きつ|と}
\ruby{取}{と}れらあナ。
%
\ruby{源}{げん}の
\ruby{家}{うち}ぢやあ
\ruby{怖}{こは}がりきつて
\ruby{居}{ゐ}やうから、
%
\ruby{出}{だ}さうぢやあ
\ruby{無}{ね}えか。
%
\ruby{其金}{そ|れ}を
\ruby{此方}{こつ|ち}の% ルビ調整(原本通り)
\ruby{懷中}{ふと|ころ}へ
そつくり
\ruby{入}{い}れて、
%
お
\ruby{龍}{りう}は
\ruby{叔母}{を|ば}に
\ruby{{\換字{連}}}{つ}れさせて
\ruby{歸}{かへ}しちまふなざあ、
%
まんざら
\ruby{野暮}{や|ぼ}ぢやあ
\ruby{無}{ね}えか。
』

\原本頁{12-10}%
『
さうさねえ。
%
\ruby{成程}{なる|ほど}
\ruby{野暮}{や|ぼ}
ぢやあ
\ruby{無}{ね}え
ぢやあ
\ruby{無}{ね}えかだ\換字{子}!。
%
ハヽ
\改行% 校正作業の簡略化のため
ヽ、
%
これだから
お
\ruby{{\換字{前}}}{めへ}は
\ruby{惡徒}{あく|とう}だつて
\ruby{云}{い}ふんだよ。
』

\原本頁{13-1}%
『
\ruby{笑}{わら}はせやがる!。
%
\ruby{番毎}{ばん|こ}に
\ruby{惡口}{わる|くち}だ。
』

\原本頁{13-2}%
『
ナニ
\ruby{褒}{ほ}めたんだよ。
』

\原本頁{13-3}%
『
\ruby{碌}{ろく}でも
\ruby{無}{ね}え
\ruby{褒}{ほ}めやうだナア、
%
\ruby{有}{あ}り
\ruby{{\換字{難}}}{がた}くも
\ruby{無}{ね}え。
%
そりやあ
\ruby{其樣}{さ|う}と
お
\ruby{龍}{りう}は
もう
\ruby{全}{まつた}く
\ruby{源}{げん}に
\ruby{未練}{み|れん}は
\ruby{無}{ね}えか。
』

\原本頁{13-5}%
『
いろ〳〵
\ruby{理解}{り|かい}を
\ruby{云}{い}つて
\ruby{聞}{き}かせたから、
%
\ruby{今}{いま}ぢや
\ruby{怒}{おこ}つては
\ruby{居}{ゐ}るやうだが、
%
\ruby{思}{おも}つては
\ruby{居}{ゐ}ない\換字{子}。
』

\原本頁{13-7}%
『
\ruby{先刻}{さつ|き}の
\ruby{言}{くち}の
\ruby{{\換字{通}}}{とほ}り
\ruby{男}{をとこ}にやあ
\ruby{懲}{こ}りてるか?。
』

\原本頁{13-8}%
『
ナアニ
\ruby{彼樣}{あ|ゝ}は% 踊り字調整「〻(二の字点、揺すり点)に見えるが(ゝ)」
\ruby{云}{い}つてるが、
%
\ruby{今}{いま}ぢやあ
もう、
%
\ruby{張}{は}りに
\ruby{來}{く}る
\ruby{{\換字{若}}}{わか}い
\ruby[<j||]{男}{をとこ}
たちに
ちやほや
\ruby{云}{い}はれるのを、
%
\ruby{可笑}{をか|し}がつて
\ruby{{\換字{遊}}}{あそ}んで
\ruby{居}{ゐ}る
\ruby{位}{くらゐ}だもの
\改行% 校正作業の簡略化のため
、
%
\原本頁{13-10}\改行%
そして
\ruby{{\換字{又}}}{また}
\ruby{{\換字{前}}々}{まへ|〳〵}からの
\ruby[||j>]{性}{しやう}
\ruby[||j>]{{\換字{分}}}{ ぶん}ぢやあ
% \ruby{性{\換字{分}}}{しやう|ぶん}ぢやあ
\ruby{有}{あ}るが、
%
\ruby{身}{み}だしなみを
\ruby{氣}{き}にして
\改行% 校正作業の簡略化のため
、
%
\原本頁{13-11}\改行%
\ruby{髮}{かみ}
なんぞも
\ruby{髮結}{かみ|ゆひ}に
\ruby{結}{い}はせる
\ruby{時}{とき}の
\ruby{間}{あひだ}にやあ、
%
やれ
\ruby{何}{なん}の、
%
\ruby{彼}{か}のと、
%
\ruby{流行}{はや|り}を
\ruby{{\換字{追}}}{お}つて
\ruby{束髮}{そく|はつ}の
\ruby{異}{おつ}なのまで
\ruby{仕}{し}て、
%
\ruby{男}{をとこ}たちに
\ruby{好}{い}いとか
\ruby{惡}{わる}いとか
\ruby{可笑}{をか|し}いとか
\ruby{云}{い}はれて、
%
おもしろ
さうに
\ruby{笑}{わら}つて
\ruby{騷}{さわ}ぐのだもの、
%
\ruby{一寸}{ちよ|つと}
\ruby{氣}{き}に
\ruby{入}{い}つた
\ruby{男}{をとこ}にでも
\ruby{逢}{あ}つた
\ruby{日}{ひ}にやあ、
%
\ruby{合點}{が|てん}で
\ruby{一}{ひ}ト
\ruby{苦勞}{く|らう}して
\ruby{見}{み}やうと
\ruby{云}{い}つたやうな
\ruby{調子}{てう|し}が
\ruby{見}{み}えるね。
』

\原本頁{14-5}%
『
フーン。
』

\原本頁{14-6}%
『
だから
\ruby{吾家}{う|ち}へ
\ruby{來}{く}る
\ruby{{\換字{若}}藏}{わか|ざう}たちの
\ruby{中}{なか}で、
%
\ruby{傳}{でん}でも
\ruby{淸}{せい}でも
\ruby{關}{かま}はないが、
%
\ruby{誰}{だれ}かと
\ruby{出來}{で|き}りやあ
\ruby{宜}{い}いと
\ruby{思}{おも}つてるのサ。
』

\原本頁{14-8}%
『
\ruby{解}{わか}らねえナ、
%
\ruby{何故}{な|ぜ}?。
』

\原本頁{14-9}%
『
\ruby{何故}{な|ぜ}つて
\ruby{{\換字{情}}夫}{い|ろ}が
\ruby{出來}{で|き}りやあ
\ruby{金錢}{おか|ね}が
\ruby{要}{い}るは\換字{子}、
%
\ruby{金錢}{おか|ね}が
\ruby{要}{い}りやあ
\ruby{自然}{ひと|りで}に
\ruby{欲}{ほ}しがるは\換字{子}。
%
\ruby{金錢}{おか|ね}を
\ruby{欲}{ほ}しがらない
\ruby{我儘者}{わが|まゝ|もの}にやあ% 踊り字調整「〻(二の字点、揺すり点)に見えるが(ゝ)」
\ruby{困}{こま}るけれど、
%
\ruby{金錢}{おか|ね}を
\ruby{欲}{ほ}しがる
\ruby{奴}{やつ}なら
\ruby{何樣}{ど|ん}な
\ruby{事}{こと}でも
\ruby{爲}{さ}せられるから\換字{子}!
\改行% 校正作業の簡略化のため
。
』

\原本頁{15-1}%
『
\ruby{{\換字{違}}無}{ちげへ|ね}え!。
%
\ruby{其樣}{そ|ん}な
\ruby{急處}{きふ|しよ}を
\ruby{捕}{つかめ}へやうと
\ruby{思}{おも}つて
\ruby{待構}{まち|かま}へて
\ruby{居}{ゐ}るのか
\原本頁{15-2}\改行%
エ?。
%
オヽ
\ruby[<j>]{怖}{おつかな}い!。
%
\ruby{何}{なん}の
\ruby{事}{こと}は
\ruby{無}{ね}え、
%
\ruby{他}{ひと}の
\ruby{色戀}{いろ|こひ}は
\ruby{汝}{おめへ}の
\ruby{餌食}{ゑ|じき}だナー
\改行% 校正作業の簡略化のため
。
』

\原本頁{15-3}%
『
ハヽヽ、
%
\ruby{云}{い}つて
\ruby{見}{み}りやあ
\ruby{其樣}{そ|ん}なものだ\換字{子}。
%
\ruby{一體}{いつ|たい}
\ruby{流行}{はや|り}も
\ruby{仕無}{し|な}い
\ruby{三絃}{べん|〳〵}の
\ruby{御師匠}{お|し|よ}さんで、
%
\ruby{澄}{す}まして
\ruby{{\換字{遣}}}{や}つて
\ruby{行}{ゆ}かれるのは、
%
\ruby{餌食}{ゑ|じき}になる
\ruby{奴}{やつ}が
ザラに
\ruby{有}{あ}るからだアネ。
%
つまり
\ruby{男}{をとこ}さへ
\ruby{見}{み}りやあ
べろつく
\ruby{娘}{むすめ}や、
%
\ruby{女}{をんな}さへ
\ruby{見}{み}りやあ
でれつく
\ruby{男}{をとこ}が、
%
\ruby{世}{よ}の
\ruby{中}{なか}に
\ruby{澤山}{たく|さん}
\ruby{有}{あ}る
\ruby{中}{うち}あ、
%
\ruby{下}{くだ}らない
\ruby{小說}{こ|ほん}でも
\ruby{御客樣}{お|きやく|さま}は
\ruby{絶}{た}えないし、
%
\ruby{彈}{ひ}けも
\ruby{仕}{し}ない
お
\ruby{師匠樣}{し|よ|さん}でも
\ruby{斯樣}{か|う}して
\ruby{御酒}{お|さけ}も
\ruby{飮}{の}めるんだから、
%
フン
\ruby{有}{あ}り
\ruby{{\換字{難}}}{がた}く
\ruby{出來}{で|き}てる
\ruby{世界}{せ|かい}さネ。
%
アヽ
お
\ruby{龍}{りう}も
ゝう% 踊り字調整「〻(二の字点、揺すり点)に見えるが(ゝ)」
\ruby{歸}{かへ}つても
\ruby{來}{く}るだらうし、
%
\ruby{物}{もの}も
\ruby{持}{も}つて
\ruby{來}{き}て
\ruby{吳}{く}れりやあ
\ruby{水}{みづ}も
\ruby{汲}{く}んで
\ruby{吳}{く}れる
といふ
\ruby[||j>]{重}{ちよう}
\ruby[||j>]{寶}{ ほう}な
% \ruby{重寶}{ちよう|ほう}な
\ruby{人{\換字{達}}}{ひと|たち}も
そろ〳〵
\ruby{來}{く}る
\ruby{時{\換字{分}}}{じ|ぶん}だ。
%
お
\ruby{{\換字{前}}}{まへ}
\ruby{一}{ひ}ト
\ruby{足先}{あし|さき}へ
また
\ruby{寄席}{よ|せ}へ
お
\ruby{出}{いで}ナ。
%
\ruby{妾}{わたし}も
お
\ruby{龍}{りう}を
\ruby{置}{おい}て
\原本頁{16-1}\改行%
\ruby{後}{あと}から
\ruby{出掛}{で|かけ}るよ。
%
\ruby{左樣}{さ|う}すると
また
\ruby{其}{その}
\ruby{塲}{ば}に% 原文通り「塲」
\ruby{居合}{ゐ|あは}せた
\ruby{{\換字{若}}}{わか}い
\ruby{奴}{やつ}に
\ruby{有}{あ}り
\ruby{{\換字{難}}}{がた}がられるのだから
をかしい!。
』

\原本頁{16-3}%
\ruby{住處}{すみ|か}も
\ruby{業體}{げふ|てい}も
\ruby{明}{あき}らかならぬ
\ruby{男}{をとこ}は
\ruby{點頭}{うな|づ}きて
\ruby{去}{さ}り、
%
\ruby{引{\換字{違}}}{ひき|ちが}へて
お
\ruby{龍}{りう}は
\ruby{歸}{かへ}り
\ruby{來}{きた}りぬ。

\原本頁{16-5}%
もとより
\ruby{色白}{いろ|じろ}の、
%
\ruby{特}{こと}に
\ruby{浴上}{ゆ|あが}り% ここだけ原本は「浴」のまま
なれば、
%
\ruby{少}{すこ}し
\ruby{上氣}{じやう|き}して
\ruby[<j>]{紅}{くれなゐ}
\ruby{潮}{ さ}
したる
\ruby{面}{おもて}の
\ruby{一}{ひ}トしほ
\ruby{麗}{うるは}しく、
%
\ruby{嫣然}{にこ|り}と
\ruby{笑}{ゑ}める
\ruby{頰}{ほゝ}に% 踊り字調整「〻(二の字点、揺すり点)に見えるが(ゝ)」
\ruby{笑靨}{ゑ|くぼ}
\ruby{少}{すこ}しよりて、
%
これが
\ruby{短銃}{ぴす|とる}を
\ruby{袂}{たもと}にして
\ruby{{\換字{情}}無}{つれ|な}き
\ruby{男}{をとこ}を
\ruby{撃}{う}たんとしたる
\ruby{恐}{おそ}ろしき
\ruby{女}{をんな}とは
\ruby{{\換字{更}}}{さら}に
\ruby{見}{み}えず、
%
たゞ% 踊り字調整「〻(二の字点、揺すり点)に濁点に見えるが(ゞ)」
これ
\ruby{垂絲}{し|だれ}
\ruby{櫻}{ざくら}の
\ruby{艶}{{\換字{𛀁}}ん}に% 原本通り「𛀁ん」
\ruby{{\換字{咲}}}{さ}き
ほこつて、
%
\ruby{吹}{ふ}けよ
\ruby{春風}{はる|かぜ}、
%
\ruby{吹}{ふ}かば
\ruby{狂}{くる}はん、
%
\ruby{降}{ふ}れよ
\ruby{春雨}{はる|さめ}、
%
\ruby{降}{ふ}らば
\ruby{濡}{ぬ}れんと、
%
\ruby{春}{はる}は
\ruby{十{\換字{分}}}{じふ|ぶん}の
\ruby{花}{はな}の
\ruby{色香}{いろ|か}に、
%
\ruby{溢}{こぼ}るゝ% 踊り字調整「〻(二の字点、揺すり点)に見えるが(ゝ)」
ばかりの
\ruby{優}{やさ}しき
\ruby{{\換字{情}}}{なさけ}の
\ruby{{\換字{浮}}}{うか}めるを
\ruby{見}{み}るが
\ruby{如}{ごと}し。

\原本頁{16-11}%
\ruby{夜}{よ}は
\ruby[||j>]{男}{をとこ}
\ruby[||j>]{弟子}{ で|し}の
\ruby{世界}{せ|かい}なり。
%
やがて
\ruby{淸}{せい}と
いへるが
\ruby{入}{い}り
\ruby{來}{きた}れる
\ruby{時}{とき}、
%
\ruby{女}{あ}
\原本頁{17-1}\改行%
\ruby{主人}{る|じ}は
\ruby{稽{\換字{古}}}{けい|こ}を
お
\ruby{龍}{りう}に
\ruby{托}{たく}して、
%
\ruby{用事}{よう|じ}
ありと
\ruby{云}{い}ひて
\ruby{寄席}{よ|せ}に
\ruby{去}{さ}りしが
\改行% 校正作業の簡略化のため
、
%
\原本頁{17-2}\改行%
それより
\ruby{傳}{でん}も
\ruby{來}{きた}り
\ruby{{\換字{勝}}}{かつ}も
\ruby{來}{きた}り、
%
\ruby{誰}{たれ}も
\ruby{彼}{かれ}も
\ruby{來}{きた}りて、
%
\ruby{皆}{みな}
お
\ruby{龍}{りう}が
\ruby{機{\換字{嫌}}}{き|げん}
とり〴〵に、
%
\ruby{富士}{ふ|じ}の
\ruby{白}{しろ}く
\ruby{優}{やさ}しきを
\ruby{取卷}{とり|ま}く
\ruby{夏}{なつ}の
\ruby{山々}{やま|〳〵}と、
%
いかつき
\ruby{身體}{から|だ}の
\ruby{背}{せ}を
くゞめ% 踊り字調整「〻(二の字点、揺すり点)に濁点に見えるが(ゞ)」
\ruby{頭}{かうべ}を
\ruby{低}{ひく}くして
しほらしく
したるも
をかし。
