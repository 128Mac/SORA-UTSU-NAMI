\Entry{其二}

% メモ 校正終了 2024-04-12 2024-05-28 2024-06-28
\原本頁{7-6}%
『
\ruby[g]{眞實}{ほんと }に
\ruby[g]{譯無}{わけな }く
\ruby[g]{此方}{こつち }のものに% ルビ調整(原本通り)
\ruby[g]{出來}{で き }やうか\換字{子}。
』

\原本頁{7-7}%
『
\ruby[g]{出來}{で き }るともさ!。
%
\ruby{併}{しか}し
\ruby[g]{{\換字{戸}}籍}{せ き }まで
といふ
\ruby{譯}{わけ}にやあ
いかねえ。
%
\ruby{籍}{せき}まで
\ruby[g]{御{\換字{前}}}{お まへ}の
\ruby{娘}{むすめ}にしやうと
\ruby{云}{い}ふにやあ、
%
お
\ruby{{\換字{前}}}{まへ}が
\ruby[g]{岩崎}{いはさき}の% 原本のこの部分は「いわさき」
\ruby{家}{うち}を
\ruby{{\換字{退}}}{の}いて
\ruby[g]{仕舞}{し ま }つて
\ruby{一本立}{いつ|ぽん|だち}に
なるか、
%
\ruby[g]{二人}{ふたり }の
\ruby{子}{こ}を
\ruby[||j>]{順}{じゆん}
\ruby[||j>]{々}{ 〳〵}に
\ruby[g]{白痴}{ば か }か
\ruby[g]{瘋癲}{きちがひ}かに
\ruby{云}{い}
\原本頁{7-10}\改行%
ひ
\ruby{立}{た}てゝ、% 踊り字調整「〻(二の字点、揺すり点)に見えるが(ゝ)」
%
\ruby{相續權}{さう|ぞく|けん}の
\ruby{無}{な}いやうに
\ruby{仕}{し}て
\ruby[g]{仕舞}{し ま }はなけりや
ならねえが
\改行% 校正作業の簡略化のため
、
%
\原本頁{8-1}\改行%
そんな
\ruby{事}{こと}は
\ruby{迚}{とて}も
\ruby[g]{出來}{で き }ることぢやあ
\ruby{無}{ね}え。
』

\原本頁{8-2}%
『
そんな
\ruby{事}{こと}は
\ruby{損}{そん}になるから
\ruby{詰}{つま}らないや\換字{子}。
%
いくら
\ruby{氣}{き}に
\ruby{入}{い}らない
% \原本頁{8-3}\改行%
\ruby{子}{こ}だつて
\ruby{何}{なん}だつて、
%
\ruby{{\換字{若}}}{わか}い
\ruby{者}{もの}あ
\ruby[g]{何樣}{ど ん }なに
\ruby[g]{出世}{しゆつせ}を
するか
\ruby{知}{し}れや
\ruby{仕}{し}ないんだもの!、
%
\ruby[g]{金錢}{お あし}が
\ruby{要}{い}らないなら
\ruby{其}{そ}の
\ruby{親}{おや}に
なつてる
\ruby{方}{はう}が
\ruby{利}{り}に
\ruby{當}{あた}るぢや
\ruby{無}{な}いか。
』

\原本頁{8-6}%
『
ハヽヽ、
%
\ruby[g]{繼子}{まゝつこ}% 踊り字調整「〻(二の字点、揺すり点)に見えるが(ゝ)」
\ruby[g]{二人}{ふたり }の
\ruby{親}{おや}に
なつてるのを、
%
\ruby[g]{抽籤}{く じ }
\ruby{{\換字{前}}}{まへ}の
\ruby{勸業銀行債{\換字{券}}}{くわ|ん|ぎ|ん|さい|けん}
でも
\ruby{持}{も}つてるやうに
\ruby{思}{おも}つてるのか。
%
ハヽヽ
お
\ruby{{\換字{前}}}{まへ}は
\ruby[g]{眞實}{ほんと }に
\ruby[g]{怜悧}{り かう}だ、% ルビ調整(原本通り)(りかう)
%
\ruby{好}{い}い
\ruby[g]{料簡}{れうけん}だ、
%
\ruby[g]{{\換字{感}}心}{かんしん}した。
』

\原本頁{8-9}%
『
お
\ruby{冷}{ひ}やかしで
\ruby{無}{な}いよ、
%
\ruby[g]{馬鹿}{ば か }にしてツ!。
』

\原本頁{8-10}%
『
それだつて
\ruby[g]{二人}{ふたり }も
\ruby{子}{こ}があるのに
\ruby{其}{その}
\ruby{上}{うへ}に
\ruby{{\換字{又}}}{また}、
%
あの
お
\ruby{龍}{りう}も
\ruby[g]{眞實}{ほんと }の
\ruby[g]{養女}{やうぢよ}に
\ruby{爲}{し}やうなんて、
%
あんまり
\ruby{蟲}{むし}が
\ruby{好}{よ}さ
\ruby{{\換字{過}}}{す}ぎるからナ。
』

\原本頁{9-1}%
『
ぢやあ
\ruby[g]{何樣}{ど う }すりやあ
\ruby{可}{いゝ}と% 踊り字調整「〻(二の字点、揺すり点)に見えるが(ゝ)」
いふのかエ。
』

\原本頁{9-2}%
『
\ruby[g]{何樣}{ど う }するも
\ruby[g]{此樣}{こ う }するも
\ruby{要}{い}る
\ruby{事}{こと}ぢやあ
\ruby{無}{な}い。
%
つまり
\ruby[g]{叔母}{を ば }
といふ
\ruby{奴}{やつ}が
\ruby{頭}{あたま}さへ
\ruby{出}{だ}して
\ruby{來}{こ}ないやうに
すりやあ
\ruby{好}{い}いのだらう。
』

\原本頁{9-4}%
『
\ruby[g]{左樣}{さ う }さ!。
%
\ruby[g]{彼女}{あ れ }を
\ruby{囮}{をとり}にして
\ruby{穫}{と}つた
\ruby{禽}{とり}を、
%
\ruby{他}{ひと}の
\ruby{手}{て}へ
\ruby{取}{と}られるやうな
\ruby{事}{こと}さへ
\ruby{無}{な}きやあ
\ruby[g]{畢竟}{つまり }
いゝのさ。% 踊り字調整「〻(二の字点、揺すり点)に見えるが(ゝ)」
』

\原本頁{9-6}%
『
だから
\ruby{譯}{わけ}は
\ruby{無}{ね}えといふのだ、
%
\ruby[g]{矢筈}{や はず}にかけるんだナ!。
』

\原本頁{9-7}%
『
\ruby[g]{矢筈}{や はず}にかけるつて、
%
\ruby[g]{何樣}{ど う }いふやうに?。
』

\原本頁{9-8}%
『
\ruby[g]{叔母}{を ば }の
ところへ
ポーンと
\ruby[g]{一本}{いつぽん}
\ruby[g]{手紙}{て がみ}を
\ruby{{\換字{遣}}}{や}つて、
%
\ruby[g]{斯樣}{か う }いふことを
\ruby{云}{い}つて
\ruby{{\換字{遣}}}{や}るのだ。
%
\ruby{妾}{わたし}は
\ruby[g]{師匠}{しゝやう}と% 踊り字調整「〻(二の字点、揺すり点)に見えるが(ゝ)」
\ruby[g]{弟子}{で し }との
\ruby{緣}{{\換字{𛀁}}ん}で、
%
\ruby[g]{其方}{そつち }の
お
\ruby{龍}{りう}さんを
\makeatletter
\@ifundefined{デバッグ@ビルド}{%
  \ruby[||g|]{何月}{なんぐわつ}
}{%
  \ruby[||j|]{何}{なん}% 三単語の間に送り仮名ゼロのため、若干原本のルビ配置とは異なる
  \ruby[|j||]{月}{ぐわつ}
}%
\makeatother
% \ruby{何月}{なん|ぐわつ}
\ruby[g]{以來}{このかた}
\ruby[||j>]{食}{かゝり}% 踊り字調整「〻(二の字点、揺すり点)に見えるが(ゝ)」
\ruby[||j>]{客}{ うど}に
% \ruby{食客}{かゝり|うど}に% 踊り字調整「〻(二の字点、揺すり点)に見えるが(ゝ)」
\ruby{仕}{し}てゐます。
%
\ruby{聞}{き}けば
お
\ruby{龍}{りう}さんは
\ruby[g]{複雜}{いりく }んだ
\ruby{譯}{わけ}で、
%
\ruby[g]{其方}{そつち }を
\ruby[g]{無言}{む ごん}で
\ruby{出}{で}て
\ruby{來}{き}たのださうだが、
%
\ruby[g]{一季}{いつき }
\ruby[g]{{\換字{半}}季}{はんき }の
\ruby{奉公人}{ほう|こう|にん}でも、
%
\原本頁{10-1}\改行%
\ruby{定}{き}める
ところは
\ruby[g]{確然}{しやん }と
\ruby{定}{き}める
\ruby{{\換字{習}}}{ならひ}だから、
%
\ruby{何}{なん}の
\ruby{定}{きまり}も
\ruby{無}{な}しに
\ruby{無際限}{む|さい|げん}に
\ruby{置}{お}く
\ruby{譯}{わけ}にはいか
\ruby{無}{な}い。
%
\ruby[g]{當人}{たうにん}の
\ruby[g]{料簡}{れうけん}ぢやあ
\ruby[g]{其方}{そつち }へは
\ruby{歸}{かへ}りたく
\ruby{無}{な}い、
%
\ruby[g]{此地}{こちら }で
\ruby{藝}{げい}の
\ruby[g]{師匠}{しゝやう}でも% 踊り字調整「〻(二の字点、揺すり点)に見えるが(ゝ)」
\ruby{仕}{し}て
\ruby{暮}{くら}したいと
\ruby{云}{い}ふ
\ruby{事}{こと}だし、
%
\ruby{妾}{わたし}の
\ruby{目}{め}で
\原本頁{10-4}\改行%
\ruby{見}{み}ても
\ruby[g]{當人}{たうにん}の
\ruby{藝}{げい}の
\ruby[g]{性質}{た ち }に
\ruby[g]{見{\換字{込}}}{み こみ}が
あるから、
%
\ruby{{\換字{若}}}{も}し
\ruby{全}{まつた}く
お
\ruby{龍}{りう}さんを
\ruby{妾}{わたし}の
\ruby[||j>]{娘}{むすめ}
\ruby[||j>]{{\換字{分}}}{ ぶん}にして、
% \ruby{娘{\換字{分}}}{むすめ|ぶん}にして、
%
\ruby{妾}{わたし}の
\ruby{跡}{あと}を
\ruby{襲}{つ}がせても
\ruby{宜}{い}いと
\ruby{云}{い}ふのなら、
%
\ruby{今}{いま}までも
\ruby[g]{世話}{せ わ }を
\ruby{仕}{し}たが
\ruby{{\換字{猶}}}{なほ}
\ruby{此}{この}
\ruby{上}{うへ}とも
%
\ruby[g]{立派}{りつぱ }に
\ruby{藝}{げい}の
\ruby[g]{成就}{できあが}るまでは
\ruby[g]{何年}{なんねん}でも
\原本頁{10-7}\改行%
\ruby[g]{世話}{せ わ }を
\ruby{爲}{し}やうが、
%
そちらが
\ruby[g]{左樣}{さ う }いふ
\ruby{氣}{き}で
\ruby{無}{な}ければ
\ruby[g]{此地}{こちら }でも
\ruby{困}{こま}る
\改行% 校正作業の簡略化のため
。
%
\原本頁{10-8}\改行%
たゞ% 踊り字調整「〻(二の字点、揺すり点)に濁点に見えるが(ゞ)」
べん〳〵とは
\ruby[g]{世話}{せ わ }も
\ruby[g]{出來}{で き }ぬから、
%
\ruby{今}{いま}
\ruby{迄}{ゝで}% 踊り字調整「〻(二の字点、揺すり点)に見えるが(ゝ)」
\ruby[g]{世話}{せ わ }を
\ruby{爲}{し}た
\ruby{食雜用}{くひ|ざふ|よう}を
\ruby{入}{い}れて、
%
\ruby[g]{其方}{そつち }へ
\ruby[g]{引取}{ひきと }つて
\ruby{貰}{もら}ひたいものだ。
%
しかし
\ruby[g]{當人}{たうにん}は
\ruby[g]{何樣}{ど う }
いふものだか、
%
\ruby{甚}{ひど}く
\ruby[g]{其方}{そちら }の
\ruby{事}{こと}を
\ruby{惡}{わる}く
\ruby{云}{い}つて、
%
\ruby[g]{田舎}{ゐなか }へ
\ruby{{\換字{返}}}{かへ}される
\makeatletter
\@ifundefined{デバッグ@ビルド}{%
  \ruby{位}{くらゐ}なら
}{%
  \ruby[<j||]{位}{くらゐ}% 行末行頭の境界付近なので特例処置を施す
  な
  \原本頁{10-11}\改行%
  ら
}%
\makeatother
\ruby{舌}{した}を
\ruby{咬}{か}んで
\ruby{死}{し}ぬなぞと、
%
\ruby[g]{無茶}{む ちや}な
\ruby{事}{こと}を
\ruby{云}{い}つて
\ruby{居}{ゐ}て
\ruby{眞}{まこと}に
\ruby{困}{こま}ります
\改行% 校正作業の簡略化のため
。
%
\原本頁{11-1}\改行%
と
\ruby[g]{斯樣}{か う }いふやうに
\ruby[g]{掛合}{かけあ }ふのだ。
%
\ruby{{\換字{遣}}}{よこ}すなら
\ruby[g]{緣切}{{\換字{𛀁}}んきり}にしろ、
%
\ruby{{\換字{返}}}{かへ}せなら
\ruby{食雜用}{くひ|ざふ|よう}を
\ruby{入}{い}れろと、
%
\ruby[g]{金額}{かねだか}を
\ruby{大袈裟}{おゝ|げ|さ}にして% 踊り字調整「〻(二の字点、揺すり点)に見えるが(ゝ)」
どうだ〳〵で
\ruby{責}{せ}めるのさ。
%
さう
すりやあ
\ruby[g]{大槪}{たいがい}
\ruby{姪一人}{めひ|ゝと|り}% 踊り字調整「〻(二の字点、揺すり点)に見えるが(ゝ)」
\ruby{捨}{す}てた
\ruby{氣}{き}にならうぜ。
』

\原本頁{11-4}%
『
でも
\ruby{食雜用}{くひ|ざふ|よう}ぢや
\ruby{月十圓}{つき|じふ|ゑん}に
したつて
\ruby{知}{し}れたもんだから、
%
\ruby{大袈裟}{おゝ|げ|さ}% 踊り字調整「〻(二の字点、揺すり点)に見えるが(ゝ)」
に
\ruby{爲}{し}やうも
\ruby{無}{な}いぢや
\ruby{無}{な}いか。
』

\原本頁{11-6}%
『
\ruby[g]{智慧}{ち ゑ }の
\ruby{無}{な}い
\ruby{事}{こと}を
\ruby{云}{い}つたものだ!。
\ruby[g]{衣服}{き もの}や
\ruby[g]{髮{\換字{飾}}}{かみかざ}りを
\ruby{少}{すこ}し
\ruby{買}{か}つて
\ruby{{\換字{遣}}}{や}つて
\ruby{置}{お}きやあ、
%
\ruby{大}{たい}した
\ruby[g]{金額}{かねだか}に
\ruby[g]{註加}{つけかけ}が
\ruby[g]{出來}{で き }らあナ。
』

\原本頁{11-8}%
『
\ruby[g]{成程}{なるほど}ネ。
%
それでも
\ruby{{\換字{連}}}{つ}れて
\ruby{歸}{かへ}つたらば?。
』

\原本頁{11-9}%
『
その
\ruby{時}{とき}はまた
\ruby{後}{あと}で
\ruby{策}{さく}を
\ruby{爲}{す}るとして、
%
\ruby{食雜用}{くひ|ざふ|よう}と
\ruby[g]{緣切}{{\換字{𛀁}}んきり}とで
\ruby[<j||]{一}{ちよ }% 行末行頭の境界付近なので特例処置を施す
\ruby[<j||]{寸}{つと }
\ruby[<j||]{{\換字{暖}}}{あたゝ}まつて% 踊り字調整「〻(二の字点、揺すり点)に見えるが(ゝ)」% 行末行頭の境界付近なので特例処置を施す
\ruby[g]{湯治}{たうじ }とでも
\ruby[g]{洒落}{しやれ }たが
\ruby{宜}{い}い。
』

\原本頁{11-11}%
『
\ruby[g]{緣切}{{\換字{𛀁}}んきり}とはエ?。
』

\原本頁{12-1}%
『
お
\ruby{龍}{りう}が
\ruby[g]{駿府}{すんぷ }へ
\ruby{{\換字{連}}}{つれ}て
\ruby{行}{ゆ}かれると
\ruby{定}{きま}つたら、
%
お
\ruby{{\換字{前}}}{まへ}が
\ruby{源}{げん}の
\ruby[g]{親{\換字{父}}}{おやぢ }へ
\ruby[g]{衝突}{ぶつか }つて、
%
\ruby[g]{此方}{こちら }の% ルビ調整(原本通り)
\ruby[g]{息子}{むすこ }さんが
\ruby{惡}{わる}いのだから、
%
\ruby[<j||]{{\換字{空}}}{にぎり}
\ruby[||j>]{{\換字{拳}}}{こぶし}
では
\ruby{話}{はなし}は
\ruby{濟}{す}みますまい、
%
いくらかの
\ruby{手切金}{て|ぎ|れ}を
\ruby[g]{御與}{お や }んなすつて、
%
\ruby{彼}{あ}の
\ruby{娘}{こ}を
\ruby[g]{駿府}{すんぷ }へ
\原本頁{12-4}\改行%
\ruby{歸}{かへ}らせた
\ruby{方}{はう}が
\ruby{宜}{よ}う
ございましやう、
%
\ruby[g]{左樣}{さ う }
\ruby{爲}{し}ないと
\ruby[g]{何時}{い つ }までも
\ruby[||j>]{關}{ひつ}
\ruby[||j>]{係}{かゝり}があつて、% 踊り字調整「〻(二の字点、揺すり点)に見えるが(ゝ)」
% \ruby{關係}{ひつ|かゝり}があつて、% 踊り字調整「〻(二の字点、揺すり点)に見えるが(ゝ)」
%
\ruby[g]{何樣}{ど ん }な
\ruby{事}{こと}が
\ruby{起}{おこ}るか
\ruby{知}{し}れませんから、
%
と
\ruby{少}{すこ}し
\ruby{巧}{うま}く
\ruby{口}{くち}を
きゝやあ% 踊り字調整「〻(二の字点、揺すり点)に見えるが(ゝ)」
\ruby[g]{必定}{きつと }
\ruby{取}{と}れらあナ。
%
\ruby{源}{げん}の
\ruby{家}{うち}ぢやあ
\ruby{怖}{こは}がりきつて
\ruby{居}{ゐ}やうから、
%
\ruby{出}{だ}さうぢやあ
\ruby{無}{ね}えか。
%
\ruby[g]{其金}{そ れ }を
\ruby[g]{此方}{こつち }の% ルビ調整(原本通り)
\ruby[g]{懷中}{ふところ}へ
そつくり
\ruby{入}{い}れて、
%
お
\ruby{龍}{りう}は
\ruby[g]{叔母}{を ば }に
\ruby{{\換字{連}}}{つ}れさせて
\ruby{歸}{かへ}しちまふなざあ、
%
まんざら
\ruby[g]{野暮}{や ぼ }ぢやあ
\ruby{無}{ね}えぢやあ
\ruby{無}{ね}えか。
』

\原本頁{12-10}%
『
さうさねえ。
%
\ruby[g]{成程}{なるほど}
\ruby[g]{野暮}{や ぼ }
ぢやあ
\ruby{無}{ね}え
ぢやあ
\ruby{無}{ね}えかだ\換字{子}!。
%
ハヽ
\改行% 校正作業の簡略化のため
ヽ、
%
これだから
お
\ruby{{\換字{前}}}{めへ}は
\ruby[g]{惡徒}{あくとう}だつて
\ruby{云}{い}ふんだよ。
』

\原本頁{13-1}%
『
\ruby{笑}{わら}はせやがる!。
%
\ruby[g]{番毎}{ばんこ }に
\ruby[g]{惡口}{わるくち}だ。
』

\原本頁{13-2}%
『
ナニ
\ruby{褒}{ほ}めたんだよ。
』

\原本頁{13-3}%
『
\ruby{碌}{ろく}でも
\ruby{無}{ね}え
\ruby{褒}{ほ}めやうだナア、
%
\ruby{有}{あ}り
\ruby{{\換字{難}}}{がた}くも
\ruby{無}{ね}え。
%
そりやあ
\ruby[g]{其樣}{さ う }と
お
\ruby{龍}{りう}は
もう
\ruby{全}{まつた}く
\ruby{源}{げん}に
\ruby[g]{未練}{み れん}は
\ruby{無}{ね}えか。
』

\原本頁{13-5}%
『
いろ〳〵
\ruby[g]{理解}{り かい}を
\ruby{云}{い}つて
\ruby{聞}{き}かせたから、
%
\ruby{今}{いま}ぢや
\ruby{怒}{おこ}つては
\ruby{居}{ゐ}るやうだが、
%
\ruby{思}{おも}つては
\ruby{居}{ゐ}ない\換字{子}。
』

\原本頁{13-7}%
『
\ruby[g]{先刻}{さつき }の
\ruby{言}{くち}の
\ruby{{\換字{通}}}{とほ}り
\ruby{男}{をとこ}にやあ
\ruby{懲}{こ}りてるか?。
』

\原本頁{13-8}%
『
ナアニ
\ruby[g]{彼樣}{あ ゝ }は% 踊り字調整「〻(二の字点、揺すり点)に見えるが(ゝ)」
\ruby{云}{い}つてるが、
%
\ruby{今}{いま}ぢやあ
もう、
%
\ruby{張}{は}りに
\ruby{來}{く}る
\ruby{{\換字{若}}}{わか}い
\ruby[<j||]{男}{をとこ}
たちに
ちやほや
\ruby{云}{い}はれるのを、
%
\ruby[g]{可笑}{をかし }がつて
\ruby{{\換字{遊}}}{あそ}んで
\ruby{居}{ゐ}る
\ruby{位}{くらゐ}だもの
\改行% 校正作業の簡略化のため
、
%
\原本頁{13-10}\改行%
そして
\ruby{{\換字{又}}}{また}
\ruby[g]{{\換字{前}}々}{まへ〳〵}からの
\ruby[||j>]{性}{しやう}
\ruby[||j>]{{\換字{分}}}{ ぶん}ぢやあ
% \ruby{性{\換字{分}}}{しやう|ぶん}ぢやあ
\ruby{有}{あ}るが、
%
\ruby{身}{み}だしなみを
\ruby{氣}{き}にして
\改行% 校正作業の簡略化のため
、
%
\原本頁{13-11}\改行%
\ruby{髮}{かみ}
なんぞも
\ruby[g]{髮結}{かみゆひ}に
\ruby{結}{い}はせる
\ruby{時}{とき}の
\ruby{間}{あひだ}にやあ、
%
やれ
\ruby{何}{なん}の、
%
\ruby{彼}{か}のと、
%
\ruby[g]{流行}{はやり }を
\ruby{{\換字{追}}}{お}つて
\ruby[g]{束髮}{そくはつ}の
\ruby{異}{おつ}なのまで
\ruby{仕}{し}て、
%
\ruby{男}{をとこ}たちに
\ruby{好}{い}いとか
\ruby{惡}{わる}いとか
\ruby[g]{可笑}{をかし }いとか
\ruby{云}{い}はれて、
%
おもしろ
さうに
\ruby{笑}{わら}つて
\ruby{騷}{さわ}ぐのだもの、
%
\ruby[g]{一寸}{ちよつと}
\ruby{氣}{き}に
\ruby{入}{い}つた
\ruby{男}{をとこ}にでも
\ruby{逢}{あ}つた
\ruby{日}{ひ}にやあ、
%
\ruby[g]{合點}{が てん}で
\ruby{一}{ひ}ト
\ruby[g]{苦勞}{く らう}して
\ruby{見}{み}やうと
\ruby{云}{い}つたやうな
\ruby[g]{調子}{てうし }が
\ruby{見}{み}えるね。
』

\原本頁{14-5}%
『
フーン。
』

\原本頁{14-6}%
『
だから
\ruby[g]{吾家}{う ち }へ
\ruby{來}{く}る
\ruby[g]{{\換字{若}}藏}{わかざう}たちの
\ruby{中}{なか}で、
%
\ruby{傳}{でん}でも
\ruby{淸}{せい}でも
\ruby{關}{かま}はないが、
%
\ruby{誰}{だれ}かと
\ruby[g]{出來}{で き }りやあ
\ruby{宜}{い}いと
\ruby{思}{おも}つてるのサ。
』

\原本頁{14-8}%
『
\ruby{解}{わか}らねえナ、
%
\ruby[g]{何故}{な ぜ }?。
』

\原本頁{14-9}%
『
\ruby[g]{何故}{な ぜ }つて
\ruby[g]{{\換字{情}}夫}{い ろ }が
\ruby[g]{出來}{で き }りやあ
\ruby[g]{金錢}{おかね }が
\ruby{要}{い}るは\換字{子}、
%
\ruby[g]{金錢}{おかね }が
\ruby{要}{い}りやあ
\ruby[g]{自然}{ひとりで}に
\ruby{欲}{ほ}しがるは\換字{子}。
%
\ruby[g]{金錢}{おかね }を
\ruby{欲}{ほ}しがらない
\ruby{我儘者}{わが|まゝ|もの}にやあ% 踊り字調整「〻(二の字点、揺すり点)に見えるが(ゝ)」
\ruby{困}{こま}るけれど、
%
\ruby[g]{金錢}{おかね }を
\ruby{欲}{ほ}しがる
\ruby{奴}{やつ}なら
\ruby[g]{何樣}{ど ん }な
\ruby{事}{こと}でも
\ruby{爲}{さ}せられるから\換字{子}!
\改行% 校正作業の簡略化のため
。
』

\原本頁{15-1}%
『
\ruby[g]{{\換字{違}}無}{ちげへね}え!。
%
\ruby[g]{其樣}{そ ん }な
\ruby[g]{急處}{きふしよ}を
\ruby{捕}{つかめ}へやうと
\ruby{思}{おも}つて
\ruby[g]{待構}{まちかま}へて
\ruby{居}{ゐ}るのか
\原本頁{15-2}\改行%
エ?。
%
オヽ
\ruby[<j>]{怖}{おつかな}い!。
%
\ruby{何}{なん}の
\ruby{事}{こと}は
\ruby{無}{ね}え、
%
\ruby{他}{ひと}の
\ruby[g]{色戀}{いろこひ}は
\ruby{汝}{おめへ}の
\ruby[g]{餌食}{ゑ じき}だナー
\改行% 校正作業の簡略化のため
。
』

\原本頁{15-3}%
『
ハヽヽ、
%
\ruby{云}{い}つて
\ruby{見}{み}りやあ
\ruby[g]{其樣}{そ ん }なものだ\換字{子}。
%
\ruby[g]{一體}{いつたい}
\ruby[g]{流行}{はやり }も
\ruby[g]{仕無}{し な }い
\makeatletter
\@ifundefined{デバッグ@ビルド}{%
  \ruby[g]{三絃}{べんべん}の
}{%
  \ruby[g]{三絃}{べん〳〵}の
}%
\makeatother
\ruby{御師匠}{お|し|よ}さんで、
%
\ruby{澄}{す}まして
\ruby{{\換字{遣}}}{や}つて
\ruby{行}{ゆ}かれるのは、
%
\ruby[g]{餌食}{ゑ じき}になる
\ruby{奴}{やつ}が
ザラに
\ruby{有}{あ}るからだアネ。
%
つまり
\ruby{男}{をとこ}さへ
\ruby{見}{み}りやあ
べろつく
\ruby{娘}{むすめ}や、
%
\ruby{女}{をんな}さへ
\ruby{見}{み}りやあ
でれつく
\ruby{男}{をとこ}が、
%
\ruby{世}{よ}の
\ruby{中}{なか}に
\ruby[g]{澤山}{たくさん}
\ruby{有}{あ}る
\ruby{中}{うち}あ、
%
\ruby{下}{くだ}らない
\ruby[g]{小說}{こ ほん}でも
\ruby{御客樣}{お|きやく|さま}は
\ruby{絶}{た}えないし、
%
\ruby{彈}{ひ}けも
\ruby{仕}{し}ない
お
\ruby{師匠樣}{し|よ|さん}でも
\ruby[g]{斯樣}{か う }して
\ruby[g]{御酒}{お さけ}も
\ruby{飮}{の}めるんだから、
%
フン
\ruby{有}{あ}り
\ruby{{\換字{難}}}{がた}く
\ruby[g]{出來}{で き }てる
\ruby[g]{世界}{せ かい}さネ。
%
アヽ
お
\ruby{龍}{りう}も
ゝう% 踊り字調整「〻(二の字点、揺すり点)に見えるが(ゝ)」
\ruby{歸}{かへ}つても
\ruby{來}{く}るだらうし、
%
\ruby{物}{もの}も
\ruby{持}{も}つて
\ruby{來}{き}て
\ruby{吳}{く}れりやあ
\ruby{水}{みづ}も
\ruby{汲}{く}んで
\ruby{吳}{く}れる
といふ
\ruby[||j>]{重}{ちよう}
\ruby[||j>]{寶}{ ほう}な
% \ruby{重寶}{ちよう|ほう}な
\ruby[g]{人{\換字{達}}}{ひとたち}も
そろ〳〵
\ruby{來}{く}る
\ruby[g]{時{\換字{分}}}{じ ぶん}だ。
%
お
\ruby{{\換字{前}}}{まへ}
\ruby{一}{ひ}ト
\ruby[g]{足先}{あしさき}へ
また
\ruby[g]{寄席}{よ せ }へ
お
\ruby{出}{いで}ナ。
%
\ruby{妾}{わたし}も
お
\ruby{龍}{りう}を
\ruby{置}{おい}て
\原本頁{16-1}\改行%
\ruby{後}{あと}から
\ruby[g]{出掛}{で かけ}るよ。
%
\ruby[g]{左樣}{さ う }すると
また
\ruby{其}{その}
\ruby{塲}{ば}に% 原文通り「塲」
\ruby[g]{居合}{ゐ あは}せた
\ruby{{\換字{若}}}{わか}い
\ruby{奴}{やつ}に
\ruby{有}{あ}り
\ruby{{\換字{難}}}{がた}がられるのだから
をかしい!。
』

\原本頁{16-3}%
\ruby[g]{住處}{すみか }も
\ruby[g]{業體}{げふてい}も
\ruby{明}{あき}らかならぬ
\ruby{男}{をとこ}は
\ruby[g]{點頭}{うなづ }きて
\ruby{去}{さ}り、
%
\ruby[g]{引{\換字{違}}}{ひきちが}へて
お
\ruby{龍}{りう}は
\ruby{歸}{かへ}り
\ruby{來}{きた}りぬ。

\原本頁{16-5}%
もとより
\ruby[g]{色白}{いろじろ}の、
%
\ruby{特}{こと}に
\ruby[g]{浴上}{ゆ あが}り% ここだけ原本は「浴」のまま
なれば、
%
\ruby{少}{すこ}し
\ruby[g]{上氣}{じやうき}して
\ruby[<j>]{紅}{くれなゐ}
\ruby{潮}{ さ}
したる
\ruby{面}{おもて}の
\ruby{一}{ひ}トしほ
\ruby{麗}{うるは}しく、
%
\ruby[g]{嫣然}{にこり }と
\ruby{笑}{ゑ}める
\ruby{頰}{ほゝ}に% 踊り字調整「〻(二の字点、揺すり点)に見えるが(ゝ)」
\ruby[g]{笑靨}{ゑ くぼ}
\ruby{少}{すこ}しよりて、
%
これが
\ruby[g]{短銃}{ぴすとる}を
\ruby{袂}{たもと}にして
\ruby[g]{{\換字{情}}無}{つれな }き
\ruby{男}{をとこ}を
\ruby{撃}{う}たんとしたる
\ruby{恐}{おそ}ろしき
\ruby{女}{をんな}とは
\ruby{{\換字{更}}}{さら}に
\ruby{見}{み}えず、
%
たゞ% 踊り字調整「〻(二の字点、揺すり点)に濁点に見えるが(ゞ)」
これ
\ruby[g]{垂絲}{し だれ}
\ruby{櫻}{ざくら}の
\ruby{艶}{{\換字{𛀁}}ん}に% 原本通り「𛀁ん」
\ruby{{\換字{咲}}}{さ}き
ほこつて、
%
\ruby{吹}{ふ}けよ
\ruby[g]{春風}{はるかぜ}、
%
\ruby{吹}{ふ}かば
\ruby{狂}{くる}はん、
%
\ruby{降}{ふ}れよ
\ruby[g]{春雨}{はるさめ}、
%
\ruby{降}{ふ}らば
\ruby{濡}{ぬ}れんと、
%
\ruby{春}{はる}は
\ruby[g]{十{\換字{分}}}{じふぶん}の
\ruby{花}{はな}の
\ruby[g]{色香}{いろか }に、
%
\ruby{溢}{こぼ}るゝ% 踊り字調整「〻(二の字点、揺すり点)に見えるが(ゝ)」
ばかりの
\ruby{優}{やさ}しき
\ruby{{\換字{情}}}{なさけ}の
\ruby{{\換字{浮}}}{うか}めるを
\ruby{見}{み}るが
\ruby{如}{ごと}し。

\原本頁{16-11}%
\ruby{夜}{よ}は
\ruby[||j>]{男}{をとこ}
\ruby[||j>]{弟子}{ で|し}の
\ruby[g]{世界}{せ かい}なり。
%
やがて
\ruby{淸}{せい}と
いへるが
\ruby{入}{い}り
\ruby{來}{きた}れる
\ruby{時}{とき}、
%
\ruby{女}{あ}
\原本頁{17-1}\改行%
\ruby[g]{主人}{る じ }は
\ruby[g]{稽{\換字{古}}}{けいこ }を
お
\ruby{龍}{りう}に
\ruby{托}{たく}して、
%
\ruby[g]{用事}{ようじ }
ありと
\ruby{云}{い}ひて
\ruby[g]{寄席}{よ せ }に
\ruby{去}{さ}りしが
\改行% 校正作業の簡略化のため
、
%
\原本頁{17-2}\改行%
それより
\ruby{傳}{でん}も
\ruby{來}{きた}り
\ruby{{\換字{勝}}}{かつ}も
\ruby{來}{きた}り、
%
\ruby{誰}{たれ}も
\ruby{彼}{かれ}も
\ruby{來}{きた}りて、
%
\ruby{皆}{みな}
お
\ruby{龍}{りう}が
\ruby[g]{機{\換字{嫌}}}{き げん}
とり〴〵に、
%
\ruby[g]{富士}{ふ じ }の
\ruby{白}{しろ}く
\ruby{優}{やさ}しきを
\ruby[g]{取卷}{とりま }く
\ruby{夏}{なつ}の
\ruby[g]{山々}{やま〳〵}と、
%
いかつき
\ruby[g]{身體}{からだ }の
\ruby{背}{せ}を
くゞめ% 踊り字調整「〻(二の字点、揺すり点)に濁点に見えるが(ゞ)」
\ruby{頭}{かうべ}を
\ruby{低}{ひく}くして
しほらしく
したるも
をかし。
