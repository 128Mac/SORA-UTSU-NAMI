\Entry{其三}

\ruby{春闌}{はる|た}けたる
\ruby{上野}{うへ|の}の
\ruby{夜}{よ}は
\ruby{深}{ふか}く
\ruby{人}{ひと}は
\ruby{稀}{まれ}にして、
\ruby{白}{しろ}き
\ruby{綿雲}{わた|ぐも}の
\ruby{地}{ち}に
\ruby{宿}{やど}れるが
\ruby{如}{ごと}く
\ruby{爛漫}{らん|まん}と
\ruby{{\換字{咲}}}{さ}き
\ruby{亂}{みだ}れたる
\ruby{櫻}{さくら}の
\ruby{{\換字{梢}}}{こずゑ}に、おぼろ
\ruby{月}{づき}の
\ruby{光薄}{ひかり|うつす}りと
\ruby{照}{て}らして、
\ruby{一場}{いち|ぢやう}の
\ruby{景色}{け|しき}は
\ruby{夢}{ゆめ}のやうに
\ruby{淡}{あは}し。

『あら
\ruby{源}{げん}さん、
\ruby{酷}{ひど}いよ、
\ruby{御待}{お|ま}ちつてば、
\ruby{御待}{お|ま}ちつて
\ruby{云}{い}ふのに!。
』

\ruby{男}{をとこ}は
\ruby{妾}{わ}が
\ruby[g]{言葉}{ことば}を
\ruby{耳}{みみ}にも
\ruby{入}{い}れず。
\ruby{振返}{ふり|かへ}りもせずして
\ruby{唯走}{ただ|はし}りに
\ruby{走}{はし}り
\ruby{去}{さ}る
\ruby{{\換字{情}}無}{つれ|な}さ
\ruby{味氣無}{あじ|き|な}さ。
\ruby{其}{そ}の
\ruby{後姿}{うしろ|すがた}は
\ruby{幾本}{いく|もと}の
\ruby{櫻}{さくら}の
\ruby{幹}{みき}より
\ruby{隱}{かく}れつ
\ruby{見}{あら}はれつして、
\ruby{見}{み}る〳〵
\ruby{{\換字{遠}}}{とほ}く
\ruby{花}{はな}の
\ruby{蔭}{かげ}の
\ruby{糢糊}{ぼ|つ}と
\ruby{白}{しろ}きが
\ruby{中}{なか}に
\ruby{{\換字{消}}}{き}え
\ruby{行}{ゆ}かんとすれば、
\ruby{心}{こゝろ}も
\ruby{{\換字{更}}}{さら}に
\ruby{心}{こゝろ}ならず、
\ruby[g]{御召縮緬}{おめし}の
\ruby{着物}{き|もの}の
\ruby{生憎}{あひ|にく}に
\ruby{足}{あし}に
\ruby{纏繞}{ま|つ}はるを
\ruby{煩}{うる}さしと
\ruby{苛}{いら}ちながら、
\ruby[g]{芝翫下駄}{しくわんげた}も
\ruby{踏}{ふ}みかへしたるまゝ
\ruby{{\換字{脱}}}{ぬ}ぎ
\ruby{捨}{す}てゝ、
\ruby[g]{足袋徒跣}{たびはだし}の
\ruby{脛}{はぎ}あらはなるさまの
\ruby{我羞}{われ|はづか}しきを
\ruby{厭}{いと}ふに
\ruby{暇無}{ひま|な}く、
\ruby{跳}{をど}る
\ruby{胸}{むね}の
\ruby[g]{氣息苦}{いきぐる}しさを
\ruby{堪}{こら}へ、

『
\ruby{源}{げん}さーん』

と{\換字{叉}}
\ruby{一}{ひ}ㇳ
\ruby{聲呼}{こゑ|よ}ぶに、
\ruby{男}{をとこ}は
\ruby{{\換字{猶}}}{なほ}
\ruby{心{\換字{強}}}{こゝろ|づよ}くも
\ruby{走}{はし}つて
\ruby{已}{や}まず、
\ruby{{\換字{返}}響}{こ|だま}のみ
\ruby{我}{わ}が
\ruby{耳}{みゝ}に、

『
\ruby{源}{げん}さーん』

と
\ruby{悲}{かな}しく
\ruby{聞}{き}こえて、
\ruby{天地}{てん|ち}は
\ruby{{\換字{情}}無}{つれ|な}くしん〳〵と
\ruby{物寂}{もの|さび}しく、
\ruby{月}{つき}もぼんやり、
\ruby{花}{はな}も
\ruby{朦朧}{ぼん|やり}、
\ruby{何}{なに}とも
\ruby{云}{い}へず
\ruby{只靜}{ただ|しづか}にして、
\ruby{我}{われ}のみの
\ruby{騒}{さわ}ぎ
\ruby{悶}{もだ}ゆるを
\ruby{笑}{わら}へるが
\ruby{如}{ごと}し。

『
\ruby{源}{げん}さーん』

\ruby{堪}{た}へかねて
\ruby{{\換字{叉}}}{また}
\ruby{一度呼}{ひと|たび|よ}べば、

『
\ruby{源}{げん}さーん』

と
\ruby{花}{はな}の
\ruby{間}{なか}より
\ruby{{\換字{返}}響}{こ|だま}のみ
\ruby{{\換字{叉}}}{また}
\ruby{一度繰}{ひと|たび|く}り
\ruby{{\換字{返}}}{かへ}したる
\ruby{其}{そ}の
\ruby{聲}{こゑ}の
\ruby{響}{ひゞ}くに
\ruby{{\換字{連}}}{つ}れて
\ruby{我}{わ}が
\ruby{頭上}{づ|じやう}なる
\ruby{花}{はな}はちら〳〵と
\ruby{散}{ち}りかゝりて、
\ruby{忽然}{こつ|ぜん}として
\ruby[g]{眞實}{まこと}の
\ruby{{\換字{雪}}}{ゆき}となり、
\ruby{見}{み}やる
\ruby[g]{彼方}{かなた}には
\ruby{廣々}{ひろ|〴〵}としたる
\ruby{川原}{か|はら}の
\ruby{見}{あら}はれて、
\ruby{其處}{そ|こ}を
\ruby{流}{なが}るゝ
\ruby{水}{みづ}の
\ruby{勢{\換字{強}}}{いきほ|ひつよ}きに、
\ruby{渡舟無}{わた|し|な}く
\ruby{橋無}{はし|な}ければ
\ruby{男}{をとこ}は
\ruby{{\換字{逃}}}{に}げまどひて、
\ruby{哀憫}{あは|れみ}を
\ruby{乞}{こ}ふが
\ruby{如}{ごと}く
\ruby{此方}{こ|なた}を
\ruby{振}{ふ}り
\ruby{{\換字{返}}}{かへ}りぬ。
\ruby{戀}{こひ}しかりしは
\ruby{先刻}{さ|き}の
\ruby{程}{ほど}なり、
\ruby{今}{いま}は
\ruby{憎}{にく}さ
\ruby{恨}{うら}めしさのむら〳〵と
\ruby{湧}{わ}き
\ruby{上}{あが}りて、
\ruby{思}{おも}はずも
\ruby{手}{て}にしたる
\ruby[g]{短銃}{ぴすとる}の
\ruby{引金}{ひき|がね}を
\ruby{引}{ひ}けば、どんと
\ruby{云}{い}ふ
\ruby{音}{おと}の
\ruby{中}{うち}に
\ruby{白煙}{しろ|けむり}ぱつと
\ruby{立}{た}つて、
\ruby{源}{げん}は
\ruby{朱}{あけ}になりつ
\ruby{摚}{どう}と
\ruby{倒}{たふ}れたるが、
\ruby{源}{げん}の
\ruby{倒}{たふ}るゝと
\ruby{同時}{どう|じ}に
\ruby{其}{そ}の
\ruby[g]{身後}{うしろ}に、
\ruby{記臆}{お|ぼゑ}も
\ruby{無}{な}く
\ruby{名}{な}も
\ruby{知}{し}らぬ
\ruby{若}{わか}き
\ruby{男}{をとこ}の、
\ruby{明}{あき}らかに
\ruby[g]{此方}{こなた}を
\ruby{向}{む}きて
\ruby{悠然}{いう|ぜん}として
\ruby{岸}{きし}に
\ruby{立}{た}てるが
\ruby{見}{み}えたり。
\ruby[g]{流石}{さすが}に
\ruby{人}{ひと}を
\ruby{殺}{ころ}したる
\ruby{身}{み}の
\ruby{罪}{つみ}に、
\ruby{心}{こゝろ}は
\ruby{度}{ど}を
\ruby{失}{うしな}ひて
\ruby{悸}{おそ}れ
\ruby{戰}{をのゝ}けるを、
\ruby{彼}{か}の
\ruby{男}{をとこ}は
\ruby{寛大}{おほ|やう}に
\ruby{{\換字{清}}}{すゞ}しき
\ruby{聲}{こゑ}して、

『
\ruby{赦}{ゆる}す、
\ruby{赦}{ゆる}してやる。
』

と
\ruby{優}{やさ}しく
\ruby{云}{い}ひたる
\ruby{其聲}{その|こゑ}の、
\ruby{何故}{なに|ゆゑ}とは
\ruby{無}{な}けれど
\ruby{身}{み}に
\ruby{沁}{し}みて
\ruby{嬉}{うれ}しく、
\ruby{骨}{ほね}も
\ruby{溶}{と}くるやうに
\ruby{悦}{よろこ}ばしと
\ruby{思}{おも}ふにつれて、
\ruby{忽地今}{たち|まち|いま}までの
\ruby{妾}{わ}が
\ruby{振舞}{ふる|まひ}のはした
\ruby{無}{な}かりしが
\ruby{口惜}{く|や}しく
\ruby{慚}{はづか}しく、
\ruby{顏}{かほ}に
\ruby{火}{ひ}の
\ruby{照}{て}るおもひして、
\ruby{何}{なに}とか
\ruby{言}{ものい}はん
\ruby{言}{ものい}はんとすれば、
\ruby{舌}{した}も
\ruby{結}{むす}ぼゝれ
\ruby{唇}{くち}も
\ruby{動}{うご}かず、
\ruby{有}{あ}り
\ruby{餘}{あま}る
\ruby{胸}{むね}の
\ruby{思}{おも}ひを
\ruby{現}{あらは}すに
\ruby{由無}{よし|な}く、
\ruby{苦}{くる}しみ〳〵て
\ruby{氣息塞}{い|き|つま}りたり。

『お
\ruby{龍}{りう}ちやん、お
\ruby{龍}{りう}ちやん、
\ruby{何様}{ど|う}お
\ruby{爲}{し}だよ。
お
\ruby{龍}{りう}。
\ruby{大層魘}{たい|そう|ゝな}されて
\ruby{居}{ゐ}るぢや
\ruby{無}{な}いか。
』

『ア、
\ruby{御師匠}{お|し|よ}さん!。
』

\ruby{覺}{さ}めたれども
\ruby{{\換字{猶}}}{なほ}
\ruby{茫然}{ばう|ぜん}として、
\ruby{星眼}{せい|がん}うつとりと
\ruby{懶}{ものう}げに
\ruby{動}{うご}かず。

『
\ruby{汝何}{おまへ|なに}か
\ruby{怖}{おそ}ろしい
\ruby{夢}{ゆめ}でも
\ruby{見}{み}たかエ。
お
\ruby{廉}{やす}くない
\ruby{夢}{ゆめ}かなんぞぢやあ
\ruby{無}{な}いか。
』

『あらお
\ruby{師匠}{し|よ}さん、
\ruby{{\換字{嫌}}}{いや}な!。
\ruby{何}{なに}か
\ruby{言}{い}つて?。
』

『
\ruby{何}{なん}だか
\ruby{分}{わか}らなかつたよ、
\ruby{妾}{わたし}も
\ruby{今目}{いま|め}が
\ruby{覺}{さ}めたんだもの。
\ruby{夢}{ゆめ}は
\ruby{五臓}{し|ん}の
\ruby[g]{疲勞}{つかれ}だつて
\ruby{云}{い}ふぢや
\ruby{無}{な}いか。
\ruby{昨夜妾}{ゆう|べ|わたし}が
\ruby{寄席}{よ|せ}から
\ruby{歸}{かへ}つて、それからまたお
\ruby{五十}{い|そ}の
\ruby{談}{はなし}やなんぞを
\ruby{遲}{おそ}くまで
\ruby{仕}{し}たもんだから、
\ruby[g]{屹度}{きつと}お
\ruby{前五臓}{まへ|し|ん}が
\ruby{疲}{つか}れたんだよ。
それで
\ruby{魘}{うな}されたりなんぞ
\ruby{仕}{し}たんだらうよ。
』

『そんな
\ruby{事}{こと}かも
\ruby{知}{し}れませんよ。
オヤツ、
\ruby{今朝}{け|さ}はお
\ruby{師匠}{し|よ}さんの
\ruby{代}{かは}りに
\ruby{四}{よ}ツ
\ruby{木}{ぎ}へいつて
\ruby{御病氣見舞}{ご|びや|うき|み|まひ}を
\ruby{爲}{す}る
\ruby{筈}{はず}でしたつけ。
\ruby{斯様}{か|う}しちやあ
\ruby{居}{ゐ}られないんでした、まあ
\ruby{起}{お}きましやう。
しかし
\ruby{何}{なん}だか
\ruby{可怪}{をか|し}な
\ruby{夢}{ゆめ}を
\ruby{妾}{わたし}あ
\ruby{見}{み}ましたよ。
』

\ruby{起}{お}きんとして
\ruby{起}{お}きず、
\ruby{枕}{まくら}に
\ruby{俯臥}{うつ|ぶ}して、
\ruby{美}{うつく}しき
\ruby{頸脚}{えり|あし}を
\ruby{惜氣}{おし|げ}も
\ruby{無}{な}く
\ruby{見}{み}せつ、
\ruby{名}{な}も
\ruby{知}{し}らず
\ruby{顏}{かほ}も
\ruby{定}{さだ}かならで
\ruby{聲}{こゑ}のみを
\ruby{聞}{き}きたる
\ruby{夢}{ゆめ}の
\ruby{中}{なか}の
\ruby{其人}{その|ひと}を
\ruby{思}{おも}ふにやあらん、
\ruby{凝然}{じ|つ}として
\ruby[g]{少時思想}{しばしおもひ}に
\ruby{耽}{ふけ}りたるが、
\ruby{寐}{ね}みだれる
\ruby{髪}{かみ}のほつれてかゝれる
\ruby{横顏}{よこ|がほ}ふくよかに
\ruby{白}{しろ}くして
\ruby{艶}{ゑん}なり。

