\Entry{其三}

% メモ 校正終了 2024-04-13
\原本頁{17-6}%
\ruby{春}{はる}
\ruby{闌}{た}けたる
\ruby{上野}{うへ|の}の
\ruby{夜}{よ}は
\ruby{深}{ふか}く
\ruby{人}{ひと}は
\ruby{稀}{まれ}にして、
%
\ruby{白}{しろ}き
\ruby{綿雲}{わた|ぐも}の
\ruby{地}{ち}に
\ruby{宿}{やど}れるが
\ruby{如}{ごと}く
\ruby{爛{\換字{熳}}}{らん|まん}と
\ruby{{\換字{咲}}}{さ}き
\ruby{亂}{みだ}れたる
\ruby{櫻}{さくら}の
\ruby{{\換字{梢}}}{こずゑ}に、
%
おぼろ
\ruby{月}{づき}の
\ruby[<j||]{光}{ひかり}
\ruby[||j>]{薄}{うつす}りと
\ruby{照}{て}らして、
%
\ruby{一塲}{いち|ぢやう}の% 原文通り「塲」
\ruby{景色}{け|しき}は
\ruby{夢}{ゆめ}の
やうに
\ruby{淡}{あは}し。

\原本頁{17-9}%
『あら
\ruby{源}{げん}さん、
%
\ruby{酷}{ひど}いよ、
%
\ruby{御待}{お|ま}ちつてば
\ruby{御待}{お|ま}ちつて
\ruby{云}{い}ふのに!。
』

\原本頁{17-10}%
\ruby[||j>]{男}{をとこ}は
\ruby{妾}{わ}が
\ruby{言葉}{こと|ば}を
\ruby{耳}{み〻}にも%原本通り「〻(二の字点、揺すり点)」
\ruby{入}{い}れず。
%
\ruby{振{\換字{返}}}{ふり|かへ}りも
せずして
\ruby{唯}{たゞ}% TODO 原本の「二の字点、揺すり点」に濁点のグリフが見つからないので「ゞ」
\ruby{走}{はし}りに
\ruby{走}{はし}り
\原本頁{18-1}\改行%
\ruby{去}{さ}る
\ruby{{\換字{情}}無}{つれ|な}さ
\ruby{味氣無}{あぢ|き|な}さ。
%
\ruby{其}{そ}の
\ruby[<j||]{後}{うしろ}
\ruby[||j>]{姿}{すがた}は
\ruby{幾本}{いく|もと}の
\ruby{櫻}{さくら}の
\ruby{幹}{みき}より
\ruby{隱}{かく}れつ
\ruby{見}{あら}はれつ
して、
%
\ruby{見}{み}る〳〵
\ruby{{\換字{遠}}}{とほ}く
\ruby{花}{はな}の
\ruby{蔭}{かげ}の
\ruby{糢糊}{ぼ|つ}と
\ruby{白}{しろ}きが
\ruby{中}{なか}に
\ruby{{\換字{消}}}{き}え
\ruby{行}{ゆ}かんと
すれば、
%
\ruby{心}{こ〻ろ}も%原本通り「〻(二の字点、揺すり点)」
\ruby{{\換字{更}}}{さら}に
\ruby{心}{こ〻ろ}ならず、%原本通り「〻(二の字点、揺すり点)」
%
\ruby{御召縮緬}{お|め||し}の
\ruby{着物}{き|もの}の
\ruby{生憎}{あひ|にく}に
\ruby{足}{あし}に
\ruby{纏繞}{ま|つ}はるを
\ruby{煩}{うる}さしと
\ruby{苛}{いら}ち
ながら、
%
\ruby{芝翫下駄}{し|くわん|げ|た}も
\ruby{踏}{ふ}みかへしたる
ま〻%原本通り「〻(二の字点、揺すり点)」
\ruby{脫}{ぬ}ぎ
\ruby{捨}{す}て〻、%原本通り「〻(二の字点、揺すり点)」
%
\ruby{足袋}{た|び}
\ruby{徒跣}{は|だし}の
\ruby{脛}{はぎ}
あらはなる
さまの
\ruby{我}{われ}
\ruby{羞}{はづか}しきを
\ruby{厭}{いと}ふに
\ruby{暇無}{ひま|な}く、
%
\ruby{跳}{をど}る
\ruby{胸}{むね}の
\ruby{氣息}{い|き}
\ruby{苦}{ぐる}しさを
\ruby{堪}{こら}へ、

\原本頁{18-7}%
『
\ruby{源}{げん}さーん
』

\原本頁{18-8}%
と
\ruby{{\換字{又}}}{また}
\ruby{一}{ひ}ト
\ruby{聲}{こゑ}
\ruby{呼}{よ}ぶに、
%
\ruby{男}{をとこ}は
\ruby{{\換字{猶}}}{なほ}
\ruby{心{\換字{強}}}{こ〻ろ|づよ}くも%原本通り「〻(二の字点、揺すり点)」
\ruby{走}{はし}つて
\ruby{已}{や}まず、
%
\ruby{{\換字{返}}響}{こ|だま}
のみ
\ruby{我}{わ}が
\ruby{耳}{み〻}に、%原本通り「〻(二の字点、揺すり点)」

\原本頁{18-10}%
『
\ruby{源}{げん}さーん』

\原本頁{18-11}%
と
\ruby{悲}{かな}しく
\ruby{聞}{きこ}えて、
%
\ruby{天地}{てん|ち}は
\ruby{{\換字{情}}無}{つれ|な}く
しん〳〵と
\ruby{物寂}{もの|さび}しく、
%
\ruby{月}{つき}も
ぼんやり、
%
\ruby{花}{はな}も
\ruby{朦朧}{ぼん|やり}、
%
\ruby{何}{なに}とも
\ruby{云}{い}へず
\ruby{只}{たゞ}% TODO 原本の「二の字点、揺すり点」に濁点のグリフが見つからないので「ゞ」
\ruby{靜}{しづか}に
して、
%
\ruby{我}{われ}
のみの
\ruby{騷}{さわ}ぎ
\ruby{悶}{もだ}ゆるを
\ruby{笑}{わら}へるが
\ruby{如}{ごと}し。

\原本頁{19-3}%
『
\ruby{源}{げん}さーん』

\原本頁{19-4}%
\ruby{堪}{た}へかねて
\ruby{{\換字{又}}}{また}
\ruby{一度}{ひと|たび}
\ruby{呼}{よ}べば、

\原本頁{19-5}%
『
\ruby{源}{げん}さーん』

\原本頁{19-6}%
と
%
\ruby{花}{はな}の
\ruby{間}{なか}より
\ruby{{\換字{返}}響}{こ|だま}
のみ
\ruby{{\換字{又}}}{また}
\ruby{一度}{ひと|たび}
\ruby{繰}{く}り
\ruby{{\換字{返}}}{かへ}したる
\ruby{其}{そ}の
\ruby{聲}{こゑ}の
\ruby{響}{ひゞ}くに% TODO 原本の「二の字点、揺すり点」に濁点のグリフが見つからないので「ゞ」
\ruby{{\換字{連}}}{つ}れて
\ruby{我}{わ}が
\ruby{頭上}{づ|じやう}なる
\ruby{花}{はな}は
ちら〳〵と
\ruby{散}{ち}り
か〻りて、%原本通り「〻(二の字点、揺すり点)」
%
\ruby{忽然}{こつ|ぜん}として
\ruby{眞實}{まこ|と}の
\ruby{{\換字{雪}}}{ゆき}となり、
%
\ruby{見}{み}やる
\ruby{彼方}{かな|た}には
\ruby{廣々}{ひろ|〴〵}と
したる
\ruby{川原}{かは|ら}の
\ruby{見}{あら}はれて、
%
\ruby{其處}{そ|こ}を
\ruby{流}{なが}る〻%原本通り「〻(二の字点、揺すり点)」
\ruby{水}{みづ}の
\ruby[<j||]{勢}{いきほひ}
\ruby{{\換字{強}}}{つよ}きに、
%
\ruby{渡舟}{わた|し}
\ruby{無}{な}く
\ruby{橋}{な}
\ruby{無}{な}ければ
%
\ruby{男}{をとこ}は
\ruby{{\換字{逃}}}{に}げ
まどひて、
%
\ruby{哀憫}{あは|れみ}を
\ruby{乞}{こ}ふが
\ruby{如}{ごと}く
\ruby{此方}{こな|た}を
\ruby{振}{ふ}り
\ruby{{\換字{返}}}{かへ}りぬ。
%
\ruby{戀}{こひ}しかりしは
\ruby{先刻}{さ|き}の
\原本頁{19-11}\改行%
\ruby{程}{ほど}なり、
%
\ruby{今}{いま}は
\ruby{憎}{にく}さ
\ruby{恨}{うら}めしさの
むら〳〵と
\ruby{湧}{わ}き
\ruby{上}{あが}りて、
%
\ruby{思}{おも}はずも
\原本頁{20-1}\改行%
\ruby{手}{て}にしたる
\ruby{短銃}{ぴす|とる}の
\ruby{引金}{ひき|がね}を
\ruby{引}{ひ}けば、
%
どんと
\ruby{云}{い}ふ
\ruby{音}{おと}の
\ruby{中}{うち}に
\ruby{白}{しろ}
\ruby[||j>]{{\換字{煙}}}{けむり}
ぱつと
\ruby{立}{た}つて、
%
\ruby{源}{げん}は
\ruby{朱}{あけ}に
なりつ
\ruby{摚}{どう}と
\ruby{倒}{たふ}れたるが、
%
\ruby{源}{げん}の
\ruby{倒}{たふ}る〻と%原本通り「〻(二の字点、揺すり点)」
\ruby{同時}{どう|じ}に
\ruby{其}{そ}の
\ruby{身後}{うし|ろ}に、
%
\ruby[g]{記臆}{おぼ{{\換字{𛀁}}}}も% 原本通り「おぼ𛀁」
\ruby{無}{な}く
\ruby{名}{な}も
\ruby{知}{し}らぬ
\ruby{{\換字{若}}}{わか}き
\ruby{男}{をとこ}の、
%
\ruby{明}{あき}らかに
\ruby{此方}{こな|た}を
\ruby{向}{む}きて
\ruby{悠然}{いう|ぜん}として
\ruby{岸}{きし}に
\ruby{立}{た}てるが
\ruby{見}{み}えたり。
%
\ruby{流石}{さす|が}に
\ruby{人}{ひと}を
\ruby{殺}{ころ}したる
\ruby{身}{み}の
\ruby{罪}{つみ}に、
%
\ruby{心}{こ〻ろ}は%原本通り「〻(二の字点、揺すり点)」
\ruby{度}{ど}を
\ruby{失}{うしな}ひて
\ruby{悸}{おそ}れ
\ruby{戰}{わな〻}けるを、%原本通り「〻(二の字点、揺すり点)」
%
\ruby{彼}{か}の
\ruby{男}{をとこ}は
\ruby{寛大}{おほ|やう}に
\ruby{淸}{すゞ}しき% TODO 原本の「二の字点、揺すり点」に濁点のグリフが見つからないので「ゞ」
\ruby{聲}{こゑ}して、

\原本頁{20-7}%
『
\ruby{赦}{ゆる}す、
%
\ruby{赦}{ゆる}してやる。
』

\原本頁{20-8}%
と
\ruby{優}{やさ}しく
\ruby{云}{い}ひたる
\ruby{其}{その}
\ruby{聲}{こゑ}の、
%
\ruby{何故}{なに|ゆゑ}とは
\ruby{無}{な}けれど
\ruby{身}{み}に
\ruby{沁}{し}みて
\ruby{嬉}{うれ}しく、
%
\ruby{骨}{ほね}も
\ruby{溶}{と}くるやうに
\ruby{悅}{よろこ}ばしと
\ruby{思}{おも}ふに
つれて、
%
\ruby{忽地}{たちま|ち}
\ruby{今}{いま}までの
\ruby{妾}{わ}が
\ruby{振舞}{ふる|まひ}の
はした
\ruby{無}{な}かりしが
\ruby{口惜}{く|や}しく
\ruby{慚}{はづか}しく、
%
\ruby{顏}{かほ}に
\ruby{火}{ひ}の
\ruby{照}{て}る
おもひして、
%
\ruby{何}{なに}とか
\ruby{言}{ものい}はん
\ruby{言}{ものい}はんと
すれば、
%
\ruby{舌}{した}も
\ruby{結}{むす}ぼ〻れ%原本通り「〻(二の字点、揺すり点)」
\ruby{唇}{くち}も
\ruby{動}{うご}かず、
%
\原本頁{21-1}\改行%
\ruby{有}{あ}り
\ruby{餘}{あま}る
\ruby{胸}{むね}の
\ruby{思}{おも}ひを
\ruby{現}{あらは}すに
\ruby{由}{よし}
\ruby{無}{な}く、
%
\ruby{苦}{くる}しみ〳〵て
\ruby{氣息}{い|き}
\ruby{塞}{つま}りたり。

\原本頁{21-2}%
『お
\ruby{龍}{りう}ちやん、
%
お
\ruby{龍}{りう}ちやん、
%
\ruby{何樣}{ど|う}
お
\ruby{爲}{し}だよ、
%
お
\ruby{龍}{りう}。
%
\ruby{大層}{たい|そう}
\ruby{魘}{〻な}されて%原本では「〻(二の字点、揺すり点)となっているが
\ruby{居}{ゐ}るぢや
\ruby{無}{な}いか。
』

\原本頁{21-4}%
『ア、
%
\ruby{御師匠}{お|し|よ}さん!。
』

\原本頁{21-5}%
\ruby{覺}{さ}めたれども
\ruby{{\換字{猶}}}{なほ}
\ruby{茫然}{ばう|ぜん}として、
%
\ruby{星眼}{せい|がん}
うつとりと
\ruby{懶}{ものう}げに
\ruby{動}{うご}かず。

\原本頁{21-6}%
『
\ruby[<j|]{汝}{おまへ}
\ruby{何}{なに}か
\ruby{怖}{おそ}ろしい
\ruby{夢}{ゆめ}でも
\ruby{見}{み}たかエ。
%
お
\ruby{{\換字{廉}}}{やす}くない
\ruby{夢}{ゆめ}か
なんぞぢやあ
\ruby{無}{な}いか。
』

\原本頁{21-8}%
『あら
お
\ruby{師匠}{し|よ}さん、
%
\ruby{{\換字{嫌}}}{いや}な!。
%
\ruby{何}{なに}か
\ruby{言}{い}つて?。
』

\原本頁{21-9}%
『
\ruby{何}{なん}だか
\ruby{{\換字{分}}}{わか}らなかつたよ、
%
\ruby{妾}{わたし}も
\ruby{今}{いま}
\ruby{目}{め}が
\ruby{覺}{さ}めたんだもの。
%
\ruby{夢}{ゆめ}は
\ruby{五臓}{し|ん}の
\ruby{疲勞}{つか|れ}だつて
\ruby{云}{い}ふぢや
\ruby{無}{な}いか。
%
\ruby{昨夜}{ゆふ|べ}
\ruby{妾}{わたし}が
\ruby{寄席}{よ|せ}から
\ruby{歸}{かへ}つて、
%
それから
また
お
\ruby{五十}{い|そ}の
\ruby{談}{はなし}や
なんぞを
\ruby{遲}{おそ}くまで
\ruby{仕}{し}たもんだから、
%
\ruby{屹度}{きつ|と}
お
\ruby{{\換字{前}}}{まへ}
\ruby{五臓}{し|ん}が
\ruby{疲}{つか}れたんだよ。
%
それで
\ruby{魘}{うな}されたり
なんぞ
\ruby{仕}{し}たんだらうよ。
』

\原本頁{22-3}%
『そんな
\ruby{事}{こと}かも
\ruby{知}{し}れませんよ。
%
オヤツ、
%
\ruby{今{\換字{朝}}}{け|さ}は
お
\ruby{師匠}{し|よ}さんの
\ruby{代}{かは}りに
\ruby{四ッ木}{よ| |ぎ}へ% TODO 四ツ木
いつて
\ruby{御病氣}{ご|びやう|き}
\ruby{見舞}{み|まひ}を
\ruby{爲}{す}る
\ruby{筈}{はず}でしたつけ。
%
\ruby{斯樣}{か|う}しちやあ
\ruby{居}{ゐ}られないんでした、
%
まあ
\ruby{起}{お}きましやう。
%
しかし
\ruby{何}{なん}だか
\ruby{可怪}{を|かし}な% ルビは「をかし」
\ruby{夢}{ゆめ}を
\ruby{妾}{わたし}あ
\ruby{見}{み}ましたよ。
』

\原本頁{22-8}%
\ruby{起}{お}きんとして
\ruby{起}{お}きず
\ruby{枕}{まくら}に
\ruby{俯伏}{うつ|ぶ}して、
%
\ruby{美}{うつく}しき
\ruby{頸脚}{{\換字{𛀁}}り|あし}を
\ruby{惜氣}{をし|げ}も
\ruby{無}{な}く
\ruby{見}{み}せつ、
%
\ruby{名}{な}も
\ruby{知}{し}らず
\ruby{顏}{かほ}も
\ruby{定}{さだ}かならで
\ruby{聲}{こゑ}のみを
\ruby{聞}{き}きたる
\ruby{夢}{ゆめ}の
\ruby{中}{なか}の
\ruby{其}{その}
\ruby{人}{ひと}を
\ruby{思}{おも}ふにやあらん、
%
\ruby{凝然}{じ|つ}として
\ruby{少時}{しば|し}
\ruby{思想}{おも|ひ}に
\ruby{耽}{ふけ}りたるが、
%
\ruby{寐}{ね}
みだれる
\ruby{髮}{かみ}の
ほつれて
か〻れる%原本通り「〻(二の字点、揺すり点)」
\ruby{横顏}{よこ|がほ}
ふくよかに
\ruby{白}{しろ}くして
\ruby{艶}{{\換字{𛀁}}ん}なり。% 原本通り「𛀁ん」
