\Entry{其四}

\ruby{{\換字{近}}}{ちか}く
\ruby{窓外}{まど|そと}を
\ruby{{\換字{過}}}{す}ぐる
\ruby{物賣}{もの|う}りの
\ruby{聲}{こゑ}は
\ruby{尾}{を}を
\ruby{引}{ひ}いて
\ruby{長}{なが}く、
\ruby{少}{すこ}し
\ruby{隔}{へだ}たりて
\ruby{聞}{きこ}ゆる
\ruby{大{\換字{通}}}{おほ|どほ}りの
\ruby{車馬}{しや|ば}の
\ruby{響}{ひゞき}は
\ruby{一}{ひと}ツになりてがやつき
\ruby{出}{だ}す
\ruby{日本橋}{に|ほん|ばし}は
\ruby{本銀町}{ほん|しろがね|ちやう}あたりの
\ruby{某}{それ}の
\ruby{横丁}{よこ|ちやう}の
\ruby{{\換字{朝}}景色}{あさ|げ|しき}、
\ruby{建}{た}ちならべる
\ruby{家々}{いへ|〳〵}に
\ruby{家々}{いへ|〳〵}の
\ruby{聲}{こゑ}あり
\ruby{物音}{もの|おと}ありて、
\ruby{子供}{こ|ども}あるところは
\ruby{先}{ま}づ
\ruby{騷々}{さう|〴〵}しく、
\ruby{若佼}{わか|き}が
\ruby{多}{おほ}きところは
\ruby{笑多}{わらひ|おほ}く、
\ruby{火}{ひ}の
\ruby{燃}{も}ゆる
\ruby{音}{おと}、
\ruby{水使}{みづ|つか}ふ
\ruby{音}{おと}、
\ruby{夜明}{よ|あ}けより
\ruby{一二時間}{いち|に|じ|かん}ばかりが
\ruby{程}{ほど}の
\ruby{一}{ひ}トしきり
\ruby{賑}{にぎ}やかなるは
\ruby{家}{いへ}ごみの
\ruby{市中}{まち|なか}の
\ruby{常}{つね}の
\ruby{態}{さま}なり。

いつもの
\ruby{晏起}{おそ|おき}には
\ruby{似}{に}ず
\ruby{今日}{け|ふ}は
\ruby{早起}{はや|おき}して、
お
\ruby{關}{せき}の
\ruby{家}{いへ}の
\ruby{{\換字{朝}}食}{あさ|めし}は
\ruby{疾}{とく}に
\ruby{濟}{す}みぬ。
\ruby{既}{すで}に
\ruby{髪}{かみ}を
\ruby{理}{をさ}め
\ruby{身}{み}じまひしたる
お
\ruby{龍}{りう}は、
\ruby{今}{いま}また
\ruby{衣}{い}を
\ruby{更}{あらた}め
\ruby{帶}{おび}を
\ruby{換}{か}へて、これより
\ruby{四}{よ}ツ
\ruby{木}{ぎ}へ
\ruby{赴}{おもむ}かんとはするなり。

\ruby{女主人}{あ|る|じ}は
\ruby{帶止}{おび|ど}めの
\ruby{美}{うつく}しきを
お
\ruby{龍}{りう}に
\ruby{渡}{わた}して、

『
\ruby{一寸}{ちよ|い}と
\ruby{見}{み}ておくれ、
\ruby{此品}{こ|れ}あ
\ruby{妾}{わたし}が
\ruby{汝}{おまへ}にあげやうと
\ruby{思}{おも}つて
\ruby{取}{と}つて
\ruby{來}{き}たんだよ。
\ruby{昨夜直}{ゆう|べ|す}ぐあげやうと
\ruby{思}{おも}つて
\ruby{居}{ゐ}たが、つい
\ruby{忘}{わす}れて
\ruby{仕舞}{し|ま}つた。
\ruby{夜}{よる}だつたもんだから、
\ruby{能}{よ}く
\ruby{{\換字{分}}}{わか}らなくつて、
\ruby{今}{いま}
\ruby{見}{み}ると
\ruby{色}{いろ}が
\ruby{何}{なん}だか
\ruby{思}{おも}つたやうぢや
\ruby{無}{な}いが、
\ruby{汝厭}{おまへ|いや}で
\ruby{無}{な}けりやあ
\ruby{締}{し}めておくれナ。
』

と
\ruby{云}{い}へば
お
\ruby{龍}{りう}は
\ruby{嬉}{うれ}しげに
\ruby{見}{み}ながら、

『あら
\ruby{勿體}{もつ|たい}ない、
\ruby{佳}{い}い
\ruby{色}{いろ}ですわ。
ちつとも
\ruby{可厭}{い|や}な
\ruby{事}{こと}なんぞありあ
\ruby{仕}{し}ませんが、ほんとに
\ruby{此品}{こ|り}あ
\ruby{戴}{いたゞ}いても
\ruby{宣}{い}いの?。
』

と、
\ruby{我}{われ}を
\ruby{愛}{あい}し
\ruby{吳}{く}るゝ
\ruby{女主人}{あ|る|じ}が
\ruby{{\換字{情}}}{なさけ}を、
\ruby{深}{ふか}くも
\ruby{悅}{よろこ}べる
\ruby{其}{そ}の
\ruby{眼色}{め|いろ}に、
\ruby{少}{すくな}からぬ
\ruby{感謝}{かん|しや}の
\ruby{意}{こゝろ}は
\ruby{表}{あらは}れたり。

『いゝともさ!お
\ruby{{\換字{前}}}{まへ}にあげやうつて
\ruby{買}{か}つて
\ruby{來}{き}たんだもの!。
それぢやあ
\ruby{御苦勞}{ご|く|らう}だけれども
\ruby{行}{い}つて
\ruby{來}{き}ておくれ。
いゝかエ、
\ruby{吾妻橋}{あ|づま|ばし}から
\ruby{直}{すぐ}
\ruby{{\換字{滊}}車}{き|しや}に
\ruby{乘}{の}つて、
\ruby{鐘}{かね}が
\ruby{淵}{ふち}といふので
\ruby{下}{お}りて
\ruby{右}{みぎ}の
\ruby{方}{はう}へ
\ruby{眞直}{まつ|すぐ}に
\ruby{行}{い}きさへすりやあ
\ruby{{\換字{造}}作}{ざう|さ}ないんだよ。
だけど
\ruby{田舎}{ゐな|か}
\ruby{{\換字{道}}}{みち}だから
\ruby{聞}{き}き
\ruby{聞}{き}き
\ruby{行}{い}かないと
\ruby{損}{そん}をするよ。
』

『ハイ、ようく
\ruby{{\換字{分}}}{わか}りました。
\ruby{狐}{きつね}に
\ruby{魅}{ばか}されないやうに
\ruby{參}{まゐ}りますよ。
ホヽヽ。
』

『ハヽヽ、ほんとに
\ruby{田舎}{ゐな|か}
\ruby{{\換字{道}}}{みち}でまごつく
\ruby{位器量}{くらゐ|きり|やう}の
\ruby{惡}{わる}い
\ruby{事}{こと}あ
\ruby{無}{な}いから\換字{子}、よく
\ruby{魅}{ばか}されないやうに
お
\ruby{仕}{し}よ。
ハヽヽ。
それから、あの、
\ruby{忘}{わす}れても
お
\ruby{五十}{い|そ}のところへ
\ruby{行}{い}くんぢやないよ。
\ruby{傳染}{う|つ}つた
\ruby{日}{ひ}にやあ
\ruby{間尺}{まし|やく}に
\ruby{合}{あ}はないから\換字{子}。
たゞ
\ruby{水野}{みづ|の}つて
\ruby{云}{い}ふのが
\ruby{世話}{せ|わ}を
\ruby{仕}{し}て
\ruby{居}{ゐ}やうから\換字{子}、
\ruby{其男}{そ|れ}に
\ruby{會}{あ}つて
\ruby{見舞}{み|まひ}の
\ruby{口上}{こう|じやう}を
\ruby{昨夜}{ゆふ|べ}
\ruby{敎}{をし}へて
\ruby{置}{お}いた
\ruby{{\換字{通}}}{とほ}りに
\ruby{云}{い}やあ
\ruby{宣}{い}いんだよ。
つまり
\ruby{病人}{びやう|にん}は
\ruby{何樣}{ど|う}だつて
\ruby{構}{かま}はないんだが、その
\ruby{水野}{みづ|の}つて
\ruby{男}{をとこ}への
\ruby{義理}{ぎ|り}でもつて、
お
\ruby{{\換字{前}}}{まへ}に
\ruby{行}{い}つて
\ruby{貰}{もら}ふやうな
\ruby{譯}{わけ}なんだから\換字{子}。
』

『ハイ、
\ruby{何}{なん}だか
\ruby{能}{よ}く
\ruby{{\換字{分}}}{わか}りませんけど、
\ruby{宣}{い}い
\ruby{加減}{か|げん}に
\ruby{申}{まを}して
\ruby{置}{お}きやあ
\ruby{宣}{い}いのでございましやう\換字{子}エ。
』

『ハヽヽ、
\ruby{左樣}{さ|う}さ、
\ruby{左樣}{さ|う}さ、それで
\ruby{宣}{い}いとも!。
\ruby{妾}{わたし}が
\ruby{顏}{かを}を
\ruby{出}{だ}しやあ
\ruby{何程{\換字{嫌}}}{いく|ら|いや}でも
\ruby{直接}{ぢ|か}に
お
\ruby{五十}{い|そ}を
\ruby{見舞}{み|ま}つて
\ruby{{\換字{遣}}}{や}らなきやならないんだから\換字{子}。
\ruby{{\換字{平}}生}{ふだ|ん}
\ruby{{\換字{交}}{\換字{情}}}{な|か}の
\ruby{惡}{わる}い
\ruby{奴}{やつ}の
\ruby{疫病}{やく|びやう}なんぞを、
\ruby{四}{よ}ツ
\ruby{木}{ぎ}くんだりへ
\ruby{見舞}{み|まひ}に
\ruby{行}{い}くなんて、
\ruby{可厭}{い|や}な
\ruby{事}{こツ}ちや
\ruby{無}{な}いか、
\ruby{馬鹿}{ば|か}
\g詰めruby{々々}{〳〵}しいわ\換字{子}。

だから
\ruby{妾}{わたし}あ
\ruby{寸白}{す|ばく}が
\ruby{起}{おこ}つて
\ruby{居}{ゐ}るんで
\ruby{出}{で}られないからとか
\ruby{何}{なん}とか
\ruby{云}{い}つて\換字{子}、
\ruby{娘}{むすめ}が
\ruby{生}{い}きても
\ruby{死}{し}んでも
\ruby{構}{かま}はないか、あんまりな
\ruby{人}{ひと}だと、
\ruby{水野}{みづ|の}に
\ruby{思}{おも}はれないやうに
\ruby{云}{い}つて
\ruby{置}{お}いて
\ruby{吳}{く}れさへすりやあ
\ruby{其}{それ}で
\ruby{宣}{い}いんだよ。
\ruby{水野}{みづ|の}に
\ruby{惡}{わる}く
\ruby{思}{おも}はれないやうにして
\ruby{置}{お}くと、また
\ruby{好}{い}い
\ruby{事}{こと}があるかも
\ruby{知}{し}れないんだから。
』

『ハイ、
\ruby{宣}{よろ}しうございます。
ぢやあ
\ruby{水野}{みづ|の}さんて
\ruby{仰}{おつし}あるのは、
\ruby{畢竟}{つま|り}
お
\ruby{五十}{い|そ}さんの
\ruby{御婿}{お|むこ}さんになる
\ruby{筈}{はず}の
\ruby{方}{かた}なんですか?。
』

『ナアに
\ruby{左樣}{さ|う}ぢやあ
\ruby{無}{な}いんだよ、
\ruby{何}{なん}でも
\ruby{無}{な}いんだよ。
お
\ruby{五十}{い|そ}には
\ruby{散々}{さん|〴〵}に
\ruby{{\換字{嫌}}}{きら}はれてゐるのさ。
』

『ヘエー、
\ruby{何}{なん}だか
\ruby{譯}{わけ}が
\ruby{{\換字{分}}}{わか}らないの\換字{子}。
それぢや
\ruby{御師匠樣}{お|し|よ|さん}の
\ruby{方}{はう}で
お
\ruby{五十}{い|そ}さんの
\ruby{御婿}{お|むこ}さんになさらうと
\ruby{思}{おも}つて
\ruby{居}{ゐ}らつしやる
\ruby{方}{かた}なの?。
』

『いゝえ、
\ruby{左樣}{さ|う}といふんでも
\ruby{無}{な}いんだよ。
\ruby{妾}{わたし}あそんな
\ruby{餘計}{よ|けい}な
\ruby{世話燒}{せ|わ|やき}きなんか
\ruby{{\換字{嫌}}}{いや}な
\ruby{事}{こつ}た\換字{子}。
』

『ヘエー、
\ruby{妙}{めう}\換字{子}エ。
\ruby{些}{ちつと}も
\ruby{譯}{わけ}が
\ruby{{\換字{分}}}{わか}らないの\換字{子}。
そしてその
\ruby{水野}{みづ|の}さんて
\ruby{怖}{こは}い
\ruby{人}{ひと}ですか。
』

『
\ruby{何}{なん}だ\換字{子}。
もう
\ruby{男}{をとこ}を
\ruby{怖}{こは}がる
\ruby{筈}{はず}の
お
\ruby{{\換字{前}}}{まへ}でも
\ruby{無}{な}いぢやあ
\ruby{無}{な}いか。
\ruby{高}{たか}が
\ruby{書}{ほん}を
\ruby{讀}{よ}んでるばかりの
\ruby{書生坊}{しよ|せい|ばう}で、
\ruby{柔}{やはら}かいんだか
\ruby{硬}{かた}いんだか
\ruby{何}{なん}だか、
\ruby{恰}{まる}で
\ruby{赤小豆}{あ|づ|き}の
\ruby{煮}{に}えこじけたやうな
\ruby{變}{へん}な
\ruby{可厭}{い|や}な
\ruby{男}{をとこ}さ。
』

『ヘエー、
\ruby{兎}{と}も
\ruby{角}{かく}もまあ
\ruby{行}{い}つてまゐりましやう。
ぢやあ
\ruby{食後片付}{あ|と|かた|づ}けもいたしませんが…………』

『いゝよ、お
\ruby{構}{かま}ひでない、さあ
\ruby{早}{はや}くおいで。
\ruby{今{\換字{朝}}桂庵}{け|さ|けい|あん}が
\ruby{婢}{をんな}を
\ruby{{\換字{連}}}{つ}れて
\ruby{來}{く}る
\ruby{筈}{はず}だから。
』

『ぢやあ、
\ruby{行}{い}つてまゐります。
』

『
\ruby{氣}{き}をつけておいで。
』

\ruby{見舞品}{み|まひ|もの}にや
\ruby{風呂敷包}{ふ|ろ|しき|づゝみ}の
\ruby{小}{ちひさ}きを
\ruby{持}{も}つて、
\ruby{街}{おもて}へ
\ruby{立出}{たち|い}でたる
\ruby{色白}{いろ|じろ}の
お
\ruby{龍}{りう}が、
\ruby{小}{こ}ざつぱりしたる
\ruby{着付}{き|つけ}、すらりとしたる
\ruby{姿}{すがた}は、
\ruby{忽}{たちま}ち
\ruby{往來}{わう|らい}の
\ruby{職人}{しよく|にん}の
\ruby{眼}{め}を
\ruby{惹}{ひ}きて、

『
\ruby{吉}{きち}や、
\ruby{見}{み}ねエ、
\ruby{小股}{こ|また}の
\ruby{切}{き}り
\ruby{上}{あが}がつた
\ruby{好}{い}い
\ruby{新造}{しん|ぞ}だナア。
』

『ウン、
\ruby{打殺}{ぶつ|ち}めて
\ruby{{\換字{遣}}}{や}りてえナ。
』

と
\ruby{{\換字{叫}}}{さけ}び
\ruby{出}{いだ}さしめぬ。
