\Entry{其四}

% メモ 校正終了 2024-04-16 2024-05-29 2024-06-29
\原本頁{23-2}%
\ruby{{\換字{近}}}{ちか}く
\ruby{窓}{まど}
\ruby{外}{そと}を
\ruby{{\換字{過}}}{す}ぐる
\ruby{物賣}{もの|う}りの
\ruby{聲}{こゑ}は
\ruby{尾}{を}を
\ruby{引}{ひ}いて
\ruby{長}{なが}く、
%
\ruby{少}{すこ}し
\ruby{隔}{へだ}たりて
\ruby{聞}{きこ}ゆる
\ruby{大{\換字{通}}}{おほ|どほ}りの
\ruby{車馬}{しや|ば}の
\ruby{響}{ひゞき}は% TODO 原本の「二の字点、揺すり点」に濁点のグリフが見つからないので「ゞ」
\ruby{一}{ひと}ツに
なりて
がやつき
\ruby{出}{だ}す
\ruby{日本橋}{に|ほん|ばし}は
\原本頁{23-4}\改行%
\ruby{本}{ほん}
\ruby[|-|]{銀}{しろかね}
\ruby{町}{ちやう}
% \ruby{本銀町}{ほん|しろかね|ちやう}
あたりの
\ruby{某}{それ}の
\ruby[||j>]{横}{よこ}
\ruby[||j>]{丁}{ちやう}の
% \ruby{横丁}{よこ|ちやう}の
\ruby{{\換字{朝}}景色}{あさ|げ|しき}、
%
\ruby{建}{た}ち
ならべる
\ruby{家々}{いへ|〳〵}に
\ruby{家々}{いへ|〳〵}の% ルビ調整(原本通り)非踊り字表記(行末行頭の境界付近)
\ruby{聲}{こゑ}あり
\ruby{物音}{もの|おと}ありて、
%
\ruby{子供}{こ|ども}ある
ところは
\ruby{先}{ま}づ
\ruby{騷々}{さう|〴〵}しく、
%
\ruby{{\換字{若}}佼}{わか|き}が
\原本頁{23-6}\改行%
\ruby{多}{おた}き
ところは
\ruby[||j>]{笑}{わらひ}
\ruby[||j>]{多}{ おほ}く、
%
\ruby{火}{ひ}の
\ruby{燃}{も}ゆる
\ruby{音}{おと}、
%
\ruby{水使}{みづ|〻つ}ふ% ルビ調整(原本通り)「〻(二の字点、揺すり点)」
%%%%%%%%%%%%%%%%%%「使」は「つか」だと思うが原本通り「つつ」としておく
\ruby{音}{おと}、
%
\ruby{夜明}{よ|あけ}より
\ruby{一二時間}{いち|に|じ|かん}
ばかりが
\ruby{程}{ほど}の
\ruby{一}{ひ}トしきり
\ruby{賑}{にぎ}やかなるは
\ruby{家}{いへ}ごみの
\ruby{市中}{まち|なか}の
\ruby{常}{つね}の
\ruby{態}{さま}なり。

\原本頁{23-9}%
いつもの
\ruby{晏起}{おそ|おき}には
\ruby{似}{に}ず
\ruby{今日}{け|ふ}は
\ruby{早起}{はや|おき}して、
%
お
\ruby{關}{せき}の
\ruby{家}{いへ}の
\ruby{{\換字{朝}}食}{あさ|めし}は
\ruby{疾}{とく}に
\ruby{濟}{す}みぬ。
%
\ruby{既}{すで}に
\ruby{髮}{かみ}を
\ruby{理}{をさ}め
\ruby{身}{み}じまひ
したる
お
\ruby{龍}{りう}は、
%
\ruby{今}{いま}また
\ruby{衣}{い}を
\ruby{{\換字{更}}}{あらた}め
\原本頁{24-1}\改行%
\ruby{帶}{おび}を
\ruby{換}{か}へて、
%
これより
\ruby{四ツ木}{よ| |ぎ}へ
\ruby{赴}{おもむ}かんとは
するなり。

\原本頁{24-2}%
\ruby{女主人}{あ|る|じ}は
\ruby{帶止}{おび|ど}めの
\ruby{美}{うつく}しきを
お
\ruby{龍}{りう}に
\ruby{渡}{わた}して、

\原本頁{24-3}%
『
\ruby{一寸}{ちよ|い}と
\ruby{見}{み}ておくれ、
%
\ruby{此品}{こ|れ}あ
\ruby{妾}{わたし}が
\ruby{汝}{おまへ}に
あげやうと
\ruby{思}{おも}つて
\ruby{取}{と}つて
\ruby{來}{き}たんだよ。
%
\ruby{昨夜}{ゆふ|べ}
\ruby{直}{す}ぐ
あげやうと
\ruby{思}{おも}つて
\ruby{居}{ゐ}たが、
%
つい
\ruby{忘}{わす}れて
\ruby{仕舞}{し|ま}つた。
%
\ruby{夜}{よる}だつた
もんだから、
%
\ruby{能}{よ}く
\ruby{{\換字{分}}}{わか}らなくつて、
%
\ruby{今}{いま}
\ruby{見}{み}ると
\ruby{色}{いろ}が
\ruby{何}{なん}だか
\ruby{思}{おも}つたやうぢや
\ruby{無}{な}いが、
%
\ruby[||j>]{汝}{おまへ}
\ruby[||j>]{厭}{ いや}で
\ruby{無}{な}けりやあ
\ruby{締}{し}めて
おくれナ。
』

\原本頁{24-8}%
と
\ruby{云}{い}へば
お
\ruby{龍}{りう}は
\ruby{嬉}{うれ}しげに
\ruby{見}{み}ながら、

\原本頁{24-9}%
『
あら
\ruby{勿體}{もつ|たい}ない、
%
\ruby{佳}{い}い
\ruby{色}{いろ}ですわ。
%
ちつとも
\ruby{可厭}{い|や}な
\ruby{事}{こと}なんぞ
ありあ
\ruby{仕}{し}ませんが、
%
ほんとに
\ruby{此品}{こ|り}あ
\ruby{戴}{いたゞ}いても% TODO 原本の「二の字点、揺すり点」に濁点のグリフが見つからないので「ゞ」
\ruby{宜}{い}いの?。
』

\原本頁{24-11}%
と、
%
\ruby{我}{われ}を
\ruby{愛}{あい}し
\ruby{吳}{く}る〻% TODO % ルビ調整(原本通り)「〻(二の字点、揺すり点)」
\ruby{女主人}{あ|る|じ}が
\ruby{{\換字{情}}}{なさけ}を、
%
\ruby{深}{ふか}くも
\ruby{悅}{よろこ}べる
\ruby{其}{そ}の
\ruby{眼色}{め|いろ}に、
%
\ruby{少}{すくな}からぬ
\ruby{{\換字{感}}謝}{かん|しや}の
\ruby{意}{こ〻ろ}は% TODO % ルビ調整(原本通り)「〻(二の字点、揺すり点)」
\ruby{表}{あらは}れたり。

\原本頁{25-2}%
『
い〻ともさ% TODO % ルビ調整(原本通り)「〻(二の字点、揺すり点)」
!\inhibitglue{}%
お
\ruby{{\換字{前}}}{まへ}に
あげやうつて
\ruby{買}{か}つて
\ruby{來}{き}たんだもの!。
%
それぢやあ
\ruby{御苦勞}{ご|く|らう}だけれども
\ruby{行}{い}つて
\ruby{來}{き}ておくれ。
%
い〻かエ、% TODO % ルビ調整(原本通り)「〻(二の字点、揺すり点)」
%
\ruby{吾妻橋}{あづ|ま|ばし}から% ルビ調整(原本通り)
\ruby{直}{じき}
\ruby{滊車}{〻|しや}に% TODO % ルビ調整(原本通り)「〻(二の字点、揺すり点)」
\ruby{乘}{の}つて、
%
\ruby{鐘が淵}{かね||ふち}といふので
\ruby{下}{お}りて
\ruby{右}{みぎ}の
\ruby{方}{はう}へ
\ruby{眞直}{まつ|すぐ}に
\ruby{行}{い}きさへすりやあ
\ruby{{\換字{造}}作}{ざう|さ}ないんだよ。
%
だけど
\ruby{田舎}{ゐな|か}
\ruby{{\換字{道}}}{みち}だから
\ruby{聞}{き}き
\ruby{聞}{き}き% ルビ調整(原本通り)非踊り字表記(行末行頭の境界付近)
\ruby{行}{い}かないと
\ruby{損}{そん}をするよ。
』

\原本頁{25-7}%
『
ハイ、
%
ようく
\ruby{{\換字{分}}}{わか}りました。
%
\ruby{狐}{きつね}に
\ruby{魅}{ばか}されないやうに
\ruby{參}{まゐ}りますよ。
%
ホヽヽ。
』

\原本頁{25-8}%
『
ハヽヽ、
%
ほんとに
\ruby{田舎}{ゐな|か}
\ruby{{\換字{道}}}{みち}で
まごつく
\ruby[||j>]{位}{くらゐ}
\ruby[||j>]{器}{ きり}
\ruby[||j>]{量}{ やう}の
\ruby{惡}{わる}い
\ruby{事}{こと}あ
\ruby{無}{な}いか
ら\換字{子}、
%
よく
\ruby{魅}{ばか}されない
やうに
お
\ruby{仕}{し}よ。
%
ハヽヽ。
%
それから、
%
あの
\改行% 校正作業の簡略化のため
、
%
\原本頁{25-11}\改行% 空行が入ってしまうがデバック優先
\ruby{忘}{わす}れても
お
\ruby{五十}{い|そ}の
ところへ
\ruby{行}{い}くんぢやないよ。
%
\ruby{傳染}{う|つ}つた
\ruby{日}{ひ}にや
\原本頁{26-1}\改行%
あ
\ruby{間尺}{まし|やく}に
\ruby{合}{あ}はないから\換字{子}。
%
たゞ% 原本の「二の字点、揺すり点」に濁点のグリフが見つからないので「ゞ」
\ruby{水野}{みづ|の}つて
\ruby{云}{い}ふのが
\ruby{世話}{せ|わ}を
\ruby{仕}{し}て
\ruby{居}{ゐ}やうからネ、
%
\ruby{其男}{そ|れ}に
\ruby{會}{あ}つて
\ruby{見舞}{み|まひ}の
\ruby[||j>]{口}{こう}
\ruby[||j>]{上}{じやう}を
% \ruby{口上}{こう|じやう}を
\ruby{昨夜}{ゆふ|べ}
\ruby{敎}{をし}へて
\ruby{置}{お}いた
\ruby{{\換字{通}}}{とほ}り
\原本頁{26-3}\改行%
に
\ruby{云}{い}やあ
\ruby{宜}{い}いんだよ。
%
つまり
\ruby[||j>]{病}{びやう}
\ruby[||j>]{人}{ にん}は
% \ruby{病人}{びやう|にん}は
\ruby{何樣}{ど|う}だつて
\ruby{構}{かま}はないんだが
\改行% デバッグ用の改行
、
%
\原本頁{26-4}\改行%
その
\ruby{水野}{みづ|の}つて
\ruby{男}{をとこ}への
\ruby{義理}{ぎ|り}で
もつて、
%
お
\ruby{{\換字{前}}}{まへ}に
\ruby{行}{い}つて
\ruby{貰}{もら}ふやうな
\ruby{譯}{わけ}なんだから\換字{子}。
』

\原本頁{26-6}%
『
ハイ、
%
\ruby{何}{なん}だか
\ruby{能}{よ}く
\ruby{{\換字{分}}}{わか}りませんけど、
%
\ruby{宜}{い}い
\ruby{加減}{か|げん}に
\ruby{申}{まを}して
\ruby{置}{お}きやあ
\ruby{宜}{い}いので
ございましやう\換字{子}エ。
』

\原本頁{26-8}%
『
ハヽヽ、
%
\ruby{左樣}{さ|う}さ、
%
\ruby{左樣}{さ|う}さ、
%
それで
\ruby{宜}{い}いとも!。
%
\ruby{妾}{わたし}が
\ruby{顏}{かほ}を
\ruby{出}{だ}し
やあ
\ruby{何程}{いく|ら}
\ruby{{\換字{嫌}}}{いや}でも
\ruby{直接}{ぢ|か}に
お
\ruby{五十}{い|そ}を
\ruby{見舞}{み|ま}つて
\ruby{{\換字{遣}}}{や}らなきや
ならないん
だから\換字{子}。
%
\ruby{{\換字{平}}生}{ふだ|ん}% ルビ調整(原本通り)
\ruby{{\換字{交}}{\換字{情}}}{な|か}の
\ruby{惡}{わる}い
\ruby{奴}{やつ}の
\ruby[||j>]{疫}{やく}
\ruby[||j>]{病}{びやう}なんぞを、
% \ruby{疫病}{やく|びやう}なんぞを、
%
\ruby{四ツ木}{よ| |ぎ}
くんだりへ
\ruby{見舞}{み|まひ}に
\ruby{行}{い}くなんて、
%
\ruby{可厭}{い|や}な
\ruby{事}{こツ}ちや
\ruby{無}{な}いか、
%
\ruby{馬鹿}{ば|か}
\ruby[g]{々々}{ 〳〵 }しいわ\換字{子}
\改行% デバッグのため
。
%
\原本頁{27-1}\改行%
だから
\ruby{妾}{わたし}あ
\ruby{寸白}{す|ばく}が
% 寸白 1 条虫・回虫などの、人体の寄生虫。また、それによって起こる病気。すんばく。
%      2 《1によると考えられたところから》婦人の腰痛や生殖器の病気の総称。
\ruby{起}{おこ}つて
\ruby{居}{ゐ}るんで
\ruby{出}{で}られないからとか
\ruby{何}{なん}とか
\ruby{云}{い}つて\換字{子}、
%
\ruby{娘}{むすめ}が
\ruby{生}{い}きても
\ruby{死}{し}んでも
\ruby{構}{かま}はないか、
%
あんまりな
\ruby{人}{ひと}だと
\原本頁{27-3}\改行%
、
%
\原本頁{27-3}\改行%
\ruby{水野}{みづ|の}に
\ruby{思}{おも}はれないやうに
\ruby{云}{い}つて
\ruby{置}{お}いて
\ruby{吳}{く}れさへすりやあ
\ruby{其}{それ}で
\ruby{宜}{い}いんだよ。
%
\ruby{水野}{みづ|の}に
\ruby{惡}{わる}く
\ruby{思}{おも}はれない
やうにして
\ruby{置}{お}くと、
%
また
\ruby{好}{い}い
\ruby{事}{こと}が
あるかも
\ruby{知}{し}れないんだから。
』

\原本頁{27-6}%
『
ハイ、
%
\ruby{宜}{よろ}しうございます。
%
ぢやあ
\ruby{水野}{みづ|の}さんて
\ruby{仰}{おつし}あるのは、
%
\ruby{畢竟}{つま|り}
お
\ruby{五十}{い|そ}さんの
\ruby{御婿}{お|むこ}さんになる% (婿 5a7f) 聟 805f
\ruby{筈}{はず}の
\ruby{方}{かた}なんですか?。
』

\原本頁{27-8}%
『
ナアに
\ruby{左樣}{さ|う}ぢやあ
\ruby{無}{な}いんだよ、
%
\ruby{何}{なん}でも
\ruby{無}{な}いんだよ。
%
お
\ruby{五十}{い|そ}には
\ruby{散々}{さん|〴〵}に
\ruby{{\換字{嫌}}}{きら}はれて
ゐるのさ。
』

\原本頁{27-10}%
『
ヘエー、
%
\ruby{何}{なん}だか
\ruby{譯}{わけ}が
\ruby{{\換字{分}}}{わか}らないの\換字{子}。
%
それぢや
\ruby{御師匠樣}{お|し|よ|さん}の
\ruby{方}{はう}で
お
\ruby{五十}{い|そ}さんの
\ruby{御聟}{お|むこ}さんに% 婿 5a7f (聟 805f)
なさらうと
\ruby{思}{おも}つて
\ruby{居}{ゐ}らつしやる
\ruby{方}{かた}な
\原本頁{28-1}\改行%
の?。
』

\原本頁{28-2}%
『
い〻え、% TODO % ルビ調整(原本通り)「〻(二の字点、揺すり点)」
%
\ruby{左樣}{さ|う}と
いふんでも
\ruby{無}{な}いんだよ。
%
\ruby{妾}{わたし}あ
そんな
\ruby{餘計}{よ|けい}な
\ruby{世話燒}{せ|わ|やき}
なんか
\ruby{{\換字{嫌}}}{いや}な
\ruby{事}{こと}た\換字{子}。
』

\原本頁{28-4}%
『
ヘエー、
%
\ruby{妙}{めう}\換字{子}エ。
%
\ruby{些}{ちつと}も
\ruby{譯}{わけ}が
\ruby{{\換字{分}}}{わか}らないの\換字{子}。
%
そして
その
\ruby{水野}{みづ|の}さんて
\ruby{怖}{こは}い
\ruby{人}{ひと}ですか。
』

\原本頁{28-6}%
『
\ruby{何}{なん}だ\換字{子}。
%
もう
\ruby{男}{をとこ}を
\ruby{怖}{こは}がる
\ruby{筈}{はず}の
お
\ruby{{\換字{前}}}{まへ}でも
\ruby{無}{な}いぢやあ
\ruby{無}{な}いか。
%
\ruby{高}{たか}が
\ruby{書}{ほん}を
\ruby{讀}{よ}んでる
ばかりの
\ruby{書生}{しよ|せい}
\ruby{坊}{つぽう}で、
%
\ruby{柔}{やはら}かいんだか
\ruby{硬}{かた}いんだか
\ruby{何}{なん}だか、
%
\ruby{恰}{まる}で
\ruby{赤小豆}{あ|づ|き}の
\ruby{煮}{に}え
こじけたやうな
\ruby{變}{へん}な
\ruby{可厭}{い|や}な
\ruby{男}{をとこ}さ。
』

\原本頁{28-9}%
『
ヘエー、
%
\ruby{兎}{と}も
\ruby{角}{かく}もまあ
\ruby{行}{い}つてまゐりましやう。
%
ぢやあ
\ruby{食後}{あ|と}
\ruby{片付}{かた|づ}けも
\ruby{致}{いた}しませんが
‥‥‥
』

\原本頁{28-11}%
『
い〻よ、% TODO % ルビ調整(原本通り)「〻(二の字点、揺すり点)」
%
お
\ruby{構}{かま}ひでない、
%
さあ
\ruby{早}{はや}く
おいで。
%
\ruby{今{\換字{朝}}}{け|さ}
\ruby{桂庵}{けい|あん}が
\ruby{婢}{をんな}を
\ruby{{\換字{連}}}{つ}れて
\ruby{來}{く}る
\ruby{筈}{はず}だから。
』

\原本頁{29-2}%
『
ぢやあ、
%
\ruby{行}{い}つて
まゐります。
』

\原本頁{29-3}%
『
\ruby{氣}{き}を
つけておいで。
』

\原本頁{29-4}%
\ruby{見舞品}{み|まひ|もの}にや
\ruby{風呂}{ふ|ろ}
\ruby[||j>]{敷}{しき}
\ruby[||j>]{包}{づ〻み}の% TODO % ルビ調整(原本通り)「〻(二の字点、揺すり点)」
\ruby{小}{ちひさ}きを
\ruby{持}{も}つて、
%
\ruby{街}{おもて}へ
\ruby{立出}{たち|い}でたる
\ruby{色白}{いろ|じろ}の
お
\ruby{龍}{りう}が、
%
\ruby{小}{こ}ざつぱり
したる
\ruby{着付}{き|つけ}、
%
すらりとしたる
\ruby{姿}{すがた}は、
%
\ruby{忽}{たちま}ち
\ruby{往來}{わう|らい}の
\ruby[||j>]{職}{しよく}
\ruby[||j>]{人}{ にん}の
% \ruby{職人}{しよく|にん}の
\ruby{眼}{め}を
\ruby{惹}{ひ}きて、

\原本頁{29-7}%
『
\ruby{吉}{きち}や、
%
\ruby{見}{み}ねエ、
%
\ruby{小股}{こ|また}の
\ruby{切}{き}り
\ruby{上}{あが}つた
\ruby{好}{い}い
\ruby{新{\換字{造}}}{しん|ぞ}だナア。
』

\原本頁{29-8}%
『
ウン、
%
\ruby{打殺}{ぶつ|ち}めて
\ruby{{\換字{遣}}}{や}りてえナ。
』

\原本頁{29-9}%
と
\ruby{叫}{さけ}び
\ruby{出}{いだ}さしめぬ。
