\Entry{其五}

% メモ 校正終了 2024-04-16 2024-05-29 2024-06-29
\原本頁{30-1}%
\ruby{我}{わ}が
\ruby{五十子}{い|そ|こ}に
さしたる
\ruby[g]{異狀}{いじやう}
\ruby{無}{な}し
といふ
\ruby[g]{尾竹}{を だけ}が
\ruby[g]{言葉}{ことば }に
\ruby[||j>]{心}{こゝろ}% 踊り字調整「〻(二の字点、揺すり点)に見えるが(ゝ)」
\ruby[||j>]{安}{ おち}
\ruby[||j>]{堵}{ つ}きて
\改行% 校正作業の簡略化のため
、
%
\原本頁{30-2}\改行%
\ruby[g]{徐々}{おもむろ}に
\ruby{我}{わ}が
\ruby{寓}{やど}に
\ruby{歸}{かへ}れる
\ruby[g]{水野}{みづの }は、
%
\ruby[g]{主人}{あるじ }の
\ruby{吉右衛門}{きち||ゑ|もん}が
\ruby[g]{老實}{まめやか}なる
\makeatletter
\@ifundefined{デバッグ@ビルド}{%
  \ruby[||j>]{注}{こゝろ}
  \ruby[||j>]{意}{ づけ}
}{%
  \ruby[<j||]{注}{こゝろ}% 踊り字調整「〻(二の字点、揺すり点)に見えるが(ゝ)」
  \ruby[||j>]{意}{づけ}
}%
\makeatother
% \ruby{注意}{こゝろ|づけ}
\原本頁{30-3}\改行%
に
\ruby{任}{まか}せて、
%
\ruby{其}{その}
\ruby{夜}{よ}は
\ruby{早}{はや}くより
\ruby[g]{臥床}{ふしど }に
\ruby{入}{い}りけるが、
%
\ruby{疲}{つか}れきつたるが
\ruby{故}{ゆゑ}にや
\ruby{却}{かへ}つて
\ruby{睡}{ねむ}りかねたり。

\原本頁{30-5}%
\ruby{一}{ひ}ト
\ruby{間}{ま}を
\ruby{隔}{へだ}てたる
\ruby{茶}{ちや}の
\ruby{室}{ま}の
\ruby{燈}{ひ}の
\ruby{下}{もと}に、
%
\ruby[g]{老夫}{おやぢ }は
\ruby[g]{悠々}{いう〳〵}と
\ruby[g]{{\換字{煙}}草}{たばこ }を
\ruby{喫}{ふか}せば、
%
\ruby[||j>]{孫}{まご}
\ruby[||j>]{娘}{むすめ}の
% \ruby{孫娘}{まご|むすめ}の
お
\ruby{濱}{はま}は
また
\ruby[g]{一心}{いつしん}に
\ruby{何}{なん}の
\ruby{書}{しよ}をか
\ruby{讀}{よ}める
さまの、
%
\ruby[g]{折々}{をり〳〵}
\ruby{其}{そ}の
\ruby{{\換字{煙}}草管}{き|せ|る}を
はたく
\ruby{音}{おと}、
%
\ruby{書}{ほん}を
\ruby{開}{あ}け
\ruby[<j>]{飜}{ひるがへ}す
\ruby{音}{おと}の
\ruby{耳}{みゝ}に% 踊り字調整「〻(二の字点、揺すり点)に見えるが(ゝ)」
\ruby{入}{い}る
\ruby{度}{たび}、
%
いと
\ruby{明}{あき}らかに
\ruby{思}{おも}ひ
\ruby{{\換字{遣}}}{や}られつ、
%
それに
\ruby{打}{うち}
\ruby{{\換字{交}}}{まじ}へて
\ruby{五十子}{い|そ|こ}が
\ruby{病}{やまひ}、
%
\ruby[g]{島木}{しまき }が
\ruby{{\換字{情}}}{なさけ}、
%
お
\ruby{澤}{さは}
\ruby[||j>]{婆}{ばゝあ}が% 踊り字調整「〻(二の字点、揺すり点)に見えるが(ゝ)」
\ruby{憎}{にく}さ、
%
\ruby[||j>]{觀}{くわん}% 「觀音」の読みは原本通り「くわん(の)ん」
\ruby[||j>]{音}{ のん}
\ruby[||j>]{堂}{ だう}
の
\ruby{{\換字{朝}}}{あさ}の
\ruby{{\換字{感}}}{かん}じ、
%
\ruby{椎}{しひ}の
\ruby[g]{樹蔭}{こ かげ}の
\ruby{夕}{ゆふべ}の
\ruby{思}{おもひ}など、
%
\ruby{{\換字{廻}}}{まは}り
\原本頁{30-10}\改行%
\ruby[g]{燈籠}{どうろう}の
\ruby{其}{その}
\ruby[g]{影像}{か げ }の
\ruby{如}{ごと}く
\ruby[g]{繰{\換字{返}}}{くりかへ}し〳〵
\ruby{胸}{むね}に
\ruby{現}{あら}はるゝに、% 踊り字調整「〻(二の字点、揺すり点)に見えるが(ゝ)」
%
\ruby[g]{幾度}{いくたび}か
\ruby[g]{幾度}{いくたび}か
\ruby[g]{寢{\換字{返}}}{ね がへ}り
\ruby{打}{う}ち
\ruby[g]{寢{\換字{返}}}{ね がへ}り
\ruby{打}{うつ}て
\ruby{睡}{ねむ}らんとしても
\ruby{睡}{ねむ}られず、
%
ほと〳〵
\ruby{自}{みづか}ら
\原本頁{31-1}\改行%
\ruby{困}{こう}じけるが、
%
やがて
\ruby[g]{何時}{い つ }と
\ruby{無}{な}く
\ruby{心鈍}{こゝろ|にぶ}りて、% 踊り字調整「〻(二の字点、揺すり点)に見えるが(ゝ)」
%
\ruby[g]{天地}{てんち }を
\ruby[g]{薄霧}{うすぎり}に
\ruby{包}{つゝ}み% 踊り字調整「〻(二の字点、揺すり点)に見えるが(ゝ)」
\ruby{行}{ゆ}かるゝが% 踊り字調整「〻(二の字点、揺すり点)に見えるが(ゝ)」
\ruby{如}{ごと}き
\ruby{思}{おもひ}を
しつゝ、% 踊り字調整「〻(二の字点、揺すり点)に見えるが(ゝ)」
%
\ruby{辛}{から}くも
\ruby[g]{我我}{われ〳〵}を% 原本通り非踊り字表記
おぼえぬ
\ruby{境}{さかひ}に
\ruby{入}{い}りぬ。
%
\原本頁{31-3}\改行%
\ruby[g]{疲勞}{つかれ }は
\ruby[g]{名殘}{な ごり}
\ruby{無}{な}く
\ruby[g]{一睡}{いつすゐ}に
\ruby{{\換字{消}}}{き}えて、
%
\ruby{明}{あ}けての
\ruby{其}{そ}の
\ruby{{\換字{朝}}}{あさ}は
\ruby{我}{わ}が
\ruby{心}{こゝろ}の% 踊り字調整「〻(二の字点、揺すり点)に見えるが(ゝ)」
いと
\ruby[g]{淸々}{すが〳〵}しきに、
%
お
\ruby{澤}{さは}が
\ruby{許}{もと}に
\ruby{置}{お}ける
\ruby{婢}{をんな}の
お
\ruby{鹽}{しほ}と
いふより、
%
\ruby{五十子}{い|そ|こ}が
\makeatletter
\@ifundefined{デバッグ@ビルド}{%
  \ruby[g]{病も}{やまひ }
}{%
  \ruby{病}{やまひ}も
}%
\makeatother
\ruby{{\換字{平}}}{たひら}なりとの
\ruby[g]{報知}{しらせ }を
さへ
\ruby{得}{{\換字{𛀁}}}たれば、
%
\ruby[g]{水野}{みづの }は
\ruby[g]{此頃}{このごろ}に
おぼ{{\換字{𛀁}}}
\ruby{無}{な}く
\原本頁{31-6}\改行%
\ruby[g]{氣合}{き あひ}
\ruby[g]{冴々}{さ{{\換字{𛀁}}}〴〵}しく、
%
先づ
\ruby[g]{島木}{しまき }に
\ruby{當}{あ}てゝの% 踊り字調整「〻(二の字点、揺すり点)に見えるが(ゝ)」
\ruby[||j>]{謝}{れい}
\ruby[||j>]{狀}{じやう}を
% \ruby{謝狀}{れい|じやう}を
\ruby{書}{か}き、
%
次に
\ruby[g]{羽{\換字{勝}}}{は がち}に
\ruby{當}{あ}てゝ% 踊り字調整「〻(二の字点、揺すり点)に見えるが(ゝ)」
\ruby{{\換字{過}}}{す}ぐる
\ruby{日}{ひ}の
\ruby[g]{會に}{くわい }
\ruby[g]{不參}{ふ さん}したる
\ruby[g]{理由}{いはれ }を
\ruby{書}{か}きて
\ruby{我}{わ}が
\ruby{心}{こゝろ}の% 踊り字調整「〻(二の字点、揺すり点)に見えるが(ゝ)」
\ruby{變}{かは}り
\ruby{無}{な}きことを
\ruby{云}{い}ひ
\ruby{{\換字{遣}}}{や}り、
%
また
\ruby{五十子}{い|そ|こ}が
\ruby[g]{繼母}{まゝはゝ}の% 踊り字調整「〻(二の字点、揺すり点)に見えるが(ゝ)」
お
\ruby{關}{せき}に
\ruby{對}{むか}つては、
%
\ruby{五十子}{い|そ|こ}が
\ruby[<j||]{病}{びやう}
\ruby[||j>]{狀}{じやう}
の
\ruby[g]{槪略}{あらまし}と
\ruby[g]{手當}{て あて}の
\ruby[g]{模樣}{も やう}とを
\ruby{知}{し}らせやりて、
%
さて
\ruby[g]{{\換字{朝}}食}{あさめし}を
\ruby{濟}{す}ませて
\ruby[g]{立出}{たちい }でつ、
%
\ruby{常}{つね}の
\ruby{如}{ごと}く
\ruby{正}{たゞ}しく% 踊り字調整「〻(二の字点、揺すり点)に濁点に見えるが(ゞ)」
おのが
\ruby[g]{職務}{つとめ }を
\ruby{執}{と}りぬ。

\原本頁{31-11}%
\ruby{此}{こ}の
\ruby{日}{ひ}
\ruby{天}{そら}は
\ruby{曇}{くも}りたれども
\ruby{風}{かぜ}
\ruby{無}{な}く
\ruby{雨}{あめ}
\ruby{無}{な}く、
%
\ruby{五十子}{い|そ|こ}の
\ruby[g]{容態}{ようだい}は
\ruby{晝}{ひる}も
\ruby{佳}{よ}く
\ruby[g]{黄昏}{く れ }も
\ruby{佳}{よ}かりければ、
%
\ruby[g]{水野}{みづの }は
\ruby{愁}{うれひ}の
\ruby{眉}{まゆ}をも
\ruby{聊}{いさゝ}か% 踊り字調整「〻(二の字点、揺すり点)に見えるが(ゝ)」
\ruby{開}{ひら}きて、
%
\ruby{憂}{う}きが
\原本頁{32-2}\改行%
\ruby{中}{なか}にも
\makeatletter
\@ifundefined{デバッグ@ビルド}{%
  \ruby[<j||]{心}{こゝろ}% 踊り字調整「〻(二の字点、揺すり点)に見えるが(ゝ)」
  \ruby{樂}{たの}しさを
}{%
  \ruby[||j>]{心}{こゝろ}% 踊り字調整「〻(二の字点、揺すり点)に見えるが(ゝ)」
  \ruby[||j>]{樂}{ たの}しさを
}%
\makeatother
おぼえ、
%
\ruby{特}{こと}に
\ruby[g]{明日}{あ す }は
\ruby{休暇日}{やす|み|び}の
\ruby[g]{土曜}{ど{\換字{𛀁}}う}といふに、
%
ひとしほ
ゆつたりと
\ruby{氣}{き}を
\ruby{寛}{くつろ}げて、
%
\ruby{夜}{よ}は
\ruby{靜}{しづか}なる
\ruby[g]{草屋}{くさのや}の
\ruby{秋}{あき}に、
%
\ruby[g]{熒々}{けい〳〵}たる
\ruby[g]{孤燈}{こ とう}の
\ruby{{\換字{前}}}{まへ}、
%
\ruby{机}{つくゑ}に
\ruby{慿}{よ}つて
\ruby[g]{端座}{たんざ }し、
%
\ruby[g]{萬斛}{ばんこく}の
\ruby{胸}{むね}の
\ruby{思}{おもひ}を
\ruby{忘}{わす}れんとてや、
%
\ruby[||j>]{一}{いつ}
\ruby[||j>]{卷}{くわん}の
% \ruby{一卷}{いつ|くわん}の
\ruby{書}{しよ}に
\ruby[g]{精神}{こゝろ }を% 踊り字調整「〻(二の字点、揺すり点)に見えるが(ゝ)」
\ruby{潜}{ひそ}めて、
%
つく〴〵と% 【潛 u6f5b 「先先」】【潜 u6f5c 「夫夫」】併用されている
\ruby{讀}{よ}み
\ruby{入}{い}つたる
\ruby{其}{そ}の
\ruby{風}{ふ}
\原本頁{32-6}\改行%
\ruby{{\換字{情}}}{ぜい}
は、
%
\ruby[g]{雷電}{らいでん}
こゝに% 踊り字調整「〻(二の字点、揺すり点)に見えるが(ゝ)」
\ruby{落}{お}ちかゝるとも% 踊り字調整「〻(二の字点、揺すり点)に見えるが(ゝ)」
\ruby{露}{つゆ}
\ruby{知}{し}らで
\ruby{{\換字{過}}}{す}ごすべき
\ruby[g]{狀態}{ありさま}にて
\改行% 校正作業の簡略化のため
、
%
\原本頁{32-7}\改行%
\ruby{身}{み}は
\ruby[g]{深山}{しんざん}の
\ruby[g]{岩室}{いはむろ}に
\ruby[||j>]{入}{にふ}
\ruby[||j>]{定}{ぢやう}したる
% \ruby{入定}{にふ|ぢやう}したる
\ruby{昔}{むかし}の
\ruby[g]{權者}{ごんじや}の、
%
\ruby[g]{形骸}{かたち }
\ruby{壞}{くづ}れず
\ruby{在}{あ}るが
\ruby{如}{ごと}くに
\ruby{動}{うご}かず、
%
\ruby{眼}{まなこ}は
\ruby[g]{{\換字{寒}}潬}{かんたん}に
\ruby{影}{かげ}を
\ruby{宿}{やど}せる
\ruby[g]{霜夜}{しもよ }の
\ruby{星}{ほし}と
\ruby{光}{ひか}り
\ruby{澄}{す}みつ、
%
\ruby{世}{よ}に
\ruby{何}{なに}
\ruby{物}{もの}のあるをも
\ruby{忘}{わす}れて、
%
\ruby{花}{はな}
\ruby{{\換字{咲}}}{さ}かば
\ruby{{\換字{咲}}}{さ}け、
%
\ruby{花}{はな}をも
\ruby{眺}{なが}めじ、
%
\ruby{{\換字{雪}}}{ゆき}
ふらば
ふれ、
%
\ruby{{\換字{雪}}}{ゆき}にも
\ruby{興}{きよう}ぜじと
\ruby{云}{い}はぬ
ばかりに
\ruby[<j||]{念}{おもひ}を
\ruby[<j>]{專}{もつぱら}にし、
%
\ruby[g]{{\換字{平}}生}{ひごろ }の% ルビ調整(原本通り)
\原本頁{32-11}\改行%
\ruby[g]{水野}{みづの }
\ruby[<j>]{某}{なにがし}の
\ruby[g]{性質}{もちまへ}を
\ruby{現}{あらは}して、
%
\ruby{凍}{こほ}りたる
\ruby{水}{みづ}の
\ruby{流}{なが}れぬが
\ruby{如}{ごと}く、
%
いつまでも
かくて
あるべき
\ruby[g]{樣子}{やうす }に
\ruby{見}{み}えたり。
