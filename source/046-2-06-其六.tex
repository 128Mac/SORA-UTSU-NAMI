\Entry{其六}

% メモ 校正終了 2024-04-16 2024-05-29 2024-06-29
\原本頁{33-3}%
『
せーんせい!。
』

\原本頁{33-4}%
\ruby[g]{水野}{みづの }は
\ruby{振}{ふり}
\ruby{{\換字{返}}}{かへ}りて
\ruby{見}{み}れば
\ruby{間}{あひ}の
\ruby{襖}{ふすま}は
\ruby{開}{あ}き
\ruby{居}{ゐ}て、
%
そこに
\ruby[g]{身體}{からだ }を
\ruby[g]{{\換字{半}}{\換字{分}}}{はんぶん}
\ruby[g]{此方}{こなた }の% ルビ調整(原本通り)
\ruby{燈}{ひ}に
\ruby{見}{み}せつ、
%
お
\ruby{濱}{はま}は
\ruby{我}{わ}が
\ruby{方}{かた}を
\ruby{打}{うち}
\ruby{護}{まも}り
\ruby{居}{ゐ}たり。

\原本頁{33-6}%
『
あゝ% 踊り字調整「〻(二の字点、揺すり点)に見えるが(ゝ)」
\ruby[g]{吃驚}{びつくり}した!。
%
\ruby{何}{なん}だ{{\換字{𛀁}}}?、
%
お
\ruby{濱}{はま}ちやん、
%
\ruby[g]{突然}{だしぬけ}に
\ruby[g]{其樣}{そ ん }な
\ruby{大}{おほき}な
\ruby{聲}{こゑ}をして!。
』

\原本頁{33-8}%
\ruby[g]{頭髮}{か み }を
\ruby{結}{むす}ばずして
\ruby[g]{後方}{うしろ }に
\ruby{下}{さ}げたれば、
%
ひとしほ
\ruby[g]{兒童}{こ ども}らしく
\ruby[<j||]{活}{くわつ}
\ruby[||j>]{潑}{ぱつ}に% 行末行頭の境界付近なので特例処置を施す
% \ruby{活潑}{くわつ|ぱつ}に
\ruby{見}{み}ゆる
\ruby{面}{おもて}の、
%
\ruby{小}{ちひ}さけれど
\ruby{淸}{すゞ}しき% 踊り字調整「〻(二の字点、揺すり点)に濁点に見えるが(ゞ)」
\ruby{眼}{め}を
\ruby[g]{出來}{で き }るたけ
\ruby[g]{見張}{み は }りて、

\原本頁{33-10}%
『
あら、
%
\ruby[g]{先生}{せんせい}〳〵つて
\ruby[g]{幾度}{いくど }
\ruby{呼}{よ}んだか
\ruby{知}{し}れやしませんのに、
%
ホヽ
\ruby[g]{先生}{せんせい}が
\ruby{{\換字{又}}}{また}
\ruby[g]{夢中}{む ちう}になつて
\ruby{居}{ゐ}らしつたんだは。
』

\原本頁{34-2}%
と、
%
お
\ruby{濱}{はま}は
\ruby{憚}{はゞか}り% 「憚 は(ゞ)か」% 踊り字調整「〻(二の字点、揺すり点)に濁点に見えるが(ゞ)」
\ruby{無}{な}く
\ruby[g]{事實}{ま こと}を
\ruby{語}{かた}りて、
%
\ruby{却}{かへ}つて
\ruby[g]{水野}{みづの }を
\ruby{{\換字{難}}}{なん}じ
\ruby{反}{かへ}しぬ。

\原本頁{34-3}%
『
\ruby[g]{左樣}{さ う }かエ、
%
それぢやあ
\ruby{私}{わたし}が
\ruby{惡}{わる}かつた、
%
\ruby[g]{勘{\換字{忍}}}{かんにん}〳〵!。% 原文通り「勘忍」
%
そして
\ruby{何}{なに}か
\ruby{用}{よう}?、
%
\ruby{用}{よう}ぢや
\ruby{無}{な}いの?。
』

\原本頁{34-5}%
『
\ruby[g]{御爺}{お ぢい}さんが\換字{子}、
%
\ruby[g]{番茶}{ばんちや}ですが
\ruby[g]{出來}{で き }ましたから
\ruby[g]{御飮}{お あが}りなさいませんか、
\ruby[g]{御茶}{お ちや}
\ruby{受}{うけ}は
\ruby[g]{柴栗}{しばぐり}の
\ruby{煠}{ゆ}でたの
ばつかり
ですけれども、
%
\ruby[g]{御茶}{お ちや}でも
あがつて、
%
そして
\ruby{餘}{あんま}り
\ruby[g]{根氣}{こ ん }を
\ruby[g]{御詰}{お つ }めなさらないで、
%
もう
\ruby{御休息}{お|やす|み}なすつた
\ruby{方}{はう}が
\ruby{宜}{よ}うございましやうツて!。
』

\原本頁{34-9}%
『
\ruby[g]{左樣}{さ う }!。
%
そりやあ
\ruby{有}{あ}り
\ruby{{\換字{難}}}{がた}う!。
%
それぢや
\ruby[g]{其方}{そつち }へ
\ruby{行}{い}つて
\ruby{御馳走}{ご|ち|そう}にならうが、
%
\ruby{栗}{くり}は
お
\ruby{濱}{はま}ちやんが
\ruby{剝}{む}いて
\ruby{吳}{く}れるのかエ。
』

\原本頁{34-11}%
『
いやよ、
%
ずるい
\ruby{事}{こと}\換字{子}エ
\ruby[g]{先生}{せんせい}は。
%
アヽ
\ruby{好}{い}いは、
%
\ruby{妾}{わたし}が
\ruby{剝}{む}いたのは
\ruby[g]{先生}{せんせい}に
あげますから、
%
\ruby[g]{先生}{せんせい}も
\ruby{妾}{わたし}に
\ruby{剝}{む}いて
\ruby[||j>]{頂}{ちやう}
\ruby[||j>]{戴}{ だい}ナ。
% \ruby{頂戴}{ちやう|だい}ナ。
』

\原本頁{35-2}%
\ruby{互}{たがひ}に
\ruby{戱}{たはむ}れて
\ruby{言}{ものい}ひながら、
%
お
\ruby{濱}{はま}は
\ruby{縋}{すが}るやうに
\ruby[g]{水野}{みづの }の
\ruby{手}{て}を
\ruby{取}{と}つて
\ruby[<j||]{誘}{いざな}へば、% 行末行頭の境界付近なので特例処置を施す
%
\ruby[g]{水野}{みづの }は
また
\ruby{扶}{たす}くるが
\ruby{如}{ごと}く
お
\ruby{濱}{はま}を
あしらひて、
%
\ruby{共}{とも}に
\ruby{直}{たゞち}に% 踊り字調整「〻(二の字点、揺すり点)に濁点に見えるが(ゞ)」
\ruby{茶}{ちや}の
\ruby{間}{ま}に
\ruby{至}{いた}るに、
%
\ruby{果}{はた}して
\ruby{焙}{ほう}じたる
\ruby{茶}{ちや}の
\ruby{香}{かほり}は
\ruby[g]{一室}{いつしつ}に
\ruby{充}{み}ち
\ruby{滿}{み}ちたり。
%
\原本頁{35-5}\改行%
\ruby[g]{三人}{さんにん}は
\ruby{一}{ひと}ツ
\ruby{燈}{ひ}の
\ruby{下}{もと}に
\ruby{鼎}{かなへ}に
\ruby{坐}{すわ}りて、
%
\ruby{互}{たがひ}に
\ruby{其}{そ}の
\ruby{淸}{きよ}らに
\ruby{和}{やさ}しき
\ruby{心}{こゝろ}より% 踊り字調整「〻(二の字点、揺すり点)に見えるが(ゝ)」
\ruby{溢}{あふ}るゝ% 踊り字調整「〻(二の字点、揺すり点)に見えるが(ゝ)」
\ruby{何}{なん}とは
\ruby{無}{な}しの
\ruby[g]{微笑}{ほゝゑみ}を% 踊り字調整「〻(二の字点、揺すり点)に見えるが(ゝ)」
\ruby{取}{と}り
\ruby{換}{かは}しつ、
%
\ruby{言}{い}はず
\ruby{語}{かた}らずの
\ruby{中}{うち}に
\ruby[g]{何事}{なにごと}も
\ruby{無}{な}き
\ruby[g]{此夜}{このよ }の
\ruby{靜}{しづか}さを
\ruby{相}{あひ}
\ruby{悅}{よろこ}べり。

\原本頁{35-8}%
もとより
\ruby{廣}{ひろ}からぬ
\ruby{家}{いへ}の
\ruby{事}{こと}なり、
%
\ruby{吉右衛門}{きち||ゑ|もん}は
\ruby[g]{二人}{ふたり }の
\ruby[<j||]{應}{うけ }
\ruby[<j||]{答}{こたへ}を
% \ruby{應答}{うけ|こたへ}を
\ruby[<j>]{悉}{こと〴〵}く
\ruby{聞}{き}きたれば、

\原本頁{35-10}%
『
また
\ruby[g]{先生}{せんせい}に
\ruby{甘}{あま}つたれるよ。
%
\ruby[g]{先生}{せんせい}に
\ruby{剝}{む}いて
\ruby{戴}{いたゞ}いて% 踊り字調整「〻(二の字点、揺すり点)に濁点に見えるが(ゞ)」
\ruby{食}{た}べやうなん
\原本頁{35-11}\改行%
て、
%
お
\ruby{{\換字{前}}}{まへ}のやうに
\ruby[g]{{\換字{遠}}慮}{ゑんりよ}を
\ruby{知}{し}らない
\ruby{女}{こ}は
\ruby{有}{あ}りやあ
\ruby{仕}{し}ない!。
%
ハヽ
\原本頁{36-1}\改行%
ヽヽ、
%
さあ
お
\ruby{茶}{ちや}を
\ruby{御}{お}あげ、
%
\ruby{栗}{くり}も
\ruby[||j>]{汝}{おまへ}
\ruby[||j>]{巧}{ うま}く
\ruby{剝}{む}けるなら
\ruby{剝}{む}いて
おあげ
\改行% 校正作業の簡略化のため
。
』

\原本頁{36-2}%
と、
%
\ruby[g]{一寸}{ちよいと}
\ruby{眞面目}{ま|じ|め}には
\ruby{窘}{たしな}めながら、
%
\ruby{叱}{しか}るが
\ruby[g]{矢張}{や はり}
\ruby[g]{笑顏}{ゑ がほ}にて、
%
\ruby{{\換字{更}}}{さら}に
\ruby{叱}{しか}るには
ならぬも
をかし。

\原本頁{36-4}%
『
イヤ、
%
ほんとは
\ruby{栗}{くり}は
\ruby{剝}{む}いて
\ruby{貰}{もら}は
なくつても
\ruby[g]{澤山}{たくさん}だよ。
%
お
\ruby{濱}{はま}ちやん!。
%
\ruby{危}{あぶな}い
\ruby{手}{て}つきか
\ruby{何}{なん}かで
もつて
\ruby{剝}{む}いて
\ruby{貰}{もら}つて、
%
\ruby{指}{ゆび}でも
\ruby[g]{負傷}{け が }を
されやうもんなら
\ruby[g]{大變}{たいへん}
だから\換字{子}エ。
』

\原本頁{36-7}%
かくいふ
\ruby{間}{ま}に
お
\ruby{濱}{はま}は
\ruby{其}{そ}の
\ruby{香}{かう}ばしき
\ruby{茶}{ちや}を
\ruby[g]{茶碗}{ちやわん}に
\ruby{注}{つ}ぎて、
%
\ruby[g]{一箇}{ひとつ }は% 「箇(7B87)」
\ruby[g]{水野}{みづの }の
\ruby{{\換字{前}}}{まへ}、
%
\ruby[g]{一箇}{ひとつ }は% 「箇(7B87)」
\ruby[g]{祖{\換字{父}}}{ぢ ゞ }の% 踊り字調整「〻(二の字点、揺すり点)に濁点に見えるが(ゞ)」
\ruby{{\換字{前}}}{まへ}に
\ruby{差}{さ}し
\ruby{置}{お}けば、

\原本頁{36-9}%
『
ぢやあ
\ruby{御{\換字{勝}}手}{ご|かつ|て}に、
』

\原本頁{36-10}%
と、
%
\ruby{小}{ちひさ}き
\ruby[g]{笊籬}{ざ る }に
\ruby{入}{い}れたる
\ruby[g]{栗實}{く り }の
\ruby{今}{いま}
\ruby{煠}{ゆ}で
\ruby{上}{あ}げし
ばかりと
\ruby{見}{み}{\換字{𛀁}}て
\ruby{{\換字{猶}}}{なほ}
\ruby{其}{そ}の
\ruby{皮}{かわ}の% 原本通り「皮 か(わ)」
\ruby[g]{蒸氣}{ゆ げ }に
\ruby{濕}{しめ}れるに
\ruby[g]{小刀}{こがたな}
\ruby{添}{そ}へて
\ruby{{\換字{盆}}}{ぼん}に
\ruby{載}{の}せたるを
\ruby[g]{主人}{あるじ }は
\ruby{差}{さ}し
\ruby{出}{だ}しぬ。

\原本頁{37-2}%
『
いゝわ、% 踊り字調整「〻(二の字点、揺すり点)に見えるが(ゝ)」
%
\ruby[g]{先生}{せんせい}!\inhibitglue{}%
そんな
\ruby{事}{こと}を
\ruby{云}{い}つて!。
%
\ruby[g]{澤山}{たくさん}でも
\ruby{何}{なん}でも
\ruby{剝}{む}いて
\ruby{上}{あ}げますよ。
%
\ruby{危}{あぶな}つかしい
\ruby{手}{て}つきだなんて
\ruby{云}{い}つたから
\ruby{{\換字{猶}}}{なほ}
\ruby{剝}{む}いて
あげるわ。
%
さうして
\ruby{{\換字{若}}}{もし}
\ruby[g]{萬一}{ひよつと}
\ruby[g]{負傷}{け が }を
\ruby{仕}{し}て
\ruby{血}{ち}でも
\ruby{出}{で}たらば、
%
その
\ruby{血}{ち}の
\ruby{着}{つ}いたのも
あげるから
いゝわ。% 踊り字調整「〻(二の字点、揺すり点)に見えるが(ゝ)」
』

\原本頁{37-6}%
『
あゝ、% 踊り字調整「〻(二の字点、揺すり点)に見えるが(ゝ)」
%
もう
あやまつた、
%
\ruby{怒}{おこ}つちやあ
いけない。
%
\ruby{私}{わたし}が
\ruby{二}{ふた}ツ
\ruby{三}{み}ツ
\ruby{剝}{む}いて
あげるから
\ruby[g]{中直}{なかなお}り
\ruby[g]{中直}{なかなお}り!。
』

\原本頁{37-8}%
『
ナアに
\ruby{優}{やさ}しくなさると
\ruby{{\換字{猶}}}{なほ}
\ruby[||j>]{增}{ぞう}
\ruby[||j>]{長}{ちやう}します。
% \ruby{增長}{ぞう|ちやう}します。
%
そんな
\ruby{下}{くだ}らない
\ruby{事}{こと}を
\ruby{云}{い}つたのを
とツこに、
%
\ruby[g]{指先}{ゆびさき}が
\ruby{痛}{いた}くなつて
\ruby{困}{こま}る
\ruby[||j>]{位}{くらゐ}
\ruby[||j>]{剝}{ む}かせて
\ruby[g]{御{\換字{遣}}}{お や }んなさる
\ruby{方}{はう}が
\ruby{宜}{よ}うございますのに。
%
ハヽヽ。
』

\原本頁{37-11}%
『
ハヽヽ、
%
\ruby[||j>]{憫}{かあ}
\ruby[||j>]{然}{いさう}に!。% 「憫然 か(あ)いさう」
% \ruby{憫然}{かあ|いさう}に!。% 「憫然 か(あ)いさう」
%
お
\ruby{濱}{はま}ちやんも
\ruby[g]{御爺}{お ぢい}さんに
\ruby{會}{あ}つちやあ
\ruby{敵}{かな}はない\換字{子}。
』

\原本頁{38-2}%
『
いや
もう
\ruby[g]{然樣}{さ う }では
ございません、
%
\ruby[g]{此女}{こ れ }には
\ruby[g]{老夫}{おやぢ }の
\ruby{方}{はう}が
\ruby[g]{始{\換字{終}}}{し じう}% ルビ調整(原本通り)「ゆ」無し
\ruby{{\換字{弱}}}{よわ}らされます。
%
\ruby[g]{談話}{はなし }を
しろ
\ruby[g]{談話}{はなし }を
\ruby{仕}{し}ろつて
\ruby[g]{{\換字{強}}{\換字{請}}}{せ が }みまして\換字{子}。
%
\ruby[g]{自{\換字{分}}}{じ ぶん}が
\ruby[g]{散々}{さん〴〵}に
\ruby{書}{ほん}を
\ruby{讀}{よ}んで
\ruby{置}{お}いて、
%
まだ
\ruby{其}{その}
\ruby{上}{うへ}に
\ruby{其}{そ}の
\ruby[g]{談話}{はなし }を
\ruby{仕}{し}ろつて
\ruby{責}{せ}めるんですもの。
』

\原本頁{38-6}%
『
あら
\ruby[g]{御爺}{お ぢい}さん、
%
そりやあ
\ruby[g]{{\換字{過}}日}{こなひだ}の
\ruby{晩}{ばん}
ばかりだは。
%
ありやあ
\ruby{書}{ほん}が
むづかしくつて
\ruby{妾}{わたし}にやあ
\ruby{{\換字{分}}}{わか}らなかつた
からだは。
』

\原本頁{38-8}%
『
\ruby[g]{一體}{いつたい}
\ruby{何}{なん}の
\ruby{書}{ほん}だつたの?。
』

\原本頁{38-9}%
『
いやな
\ruby{書}{ほん}だつたの!。
』

\原本頁{38-10}%
『
\ruby{{\換字{嫌}}}{いや}な
\ruby{書}{ほん}てまあ、
%
\ruby{何}{なん}といふ
\ruby{書}{ほん}?。
』

\原本頁{38-11}%
『
お
\ruby{爺}{ぢい}さん、
%
\ruby{默}{だま}つて
\ruby{居}{ゐ}てよ。
%
\ruby{云}{い}はないで
\ruby{居}{ゐ}てよ!。
\ruby{妾}{わたし}あ
たゞ% 踊り字調整「〻(二の字点、揺すり点)に濁点に見えるが(ゞ)」
\ruby[g]{本家}{ほんけ }から
\ruby[g]{手當}{て あた}り
\ruby[g]{次第}{し だい}に
\ruby{持}{も}つて
\ruby{來}{き}たばかしで、
%
\ruby{別}{べつ}に
\ruby[g]{彼書}{あ れ }を
\ruby{讀}{よ}もうつて
\ruby{持}{も}つて
\ruby{來}{き}たんぢや
\ruby{無}{な}かつたんだから。
』

\原本頁{39-3}%
『
ハテナ、
%
\ruby{匿}{かく}されると
\ruby{{\換字{猶}}}{なほ}
\ruby{聞}{き}きたいが
\ruby{何}{なん}の
\ruby{書}{ほん}だらう?。
』

\原本頁{39-4}%
『
イヤ
\ruby{新}{あたら}しい
\ruby[||j>]{活}{くわつ}
\ruby[||j>]{版}{ ぱん}
\ruby[||j>]{刷}{ ずり}の
\ruby{西洋綴}{せい|やう|とぢ}の
\ruby{書}{ほん}にやあ
\ruby[g]{彼樣}{あ ん }なものは
よもや
\ruby{入}{はい}つて
\ruby{居}{ゐ}まいと
\ruby{思}{おも}つて
\ruby{居}{ゐ}ましたが。
%
\ruby{飛}{とん}でも
\ruby{無}{な}い
\ruby{書}{ほん}が
\ruby{入}{はい}つて
\ruby{居}{ゐ}ましたのさ。
%
あの
\makeatletter
\@ifundefined{全三巻@一括ビルド}{%
  \ruby[g]{帝國}{ていこく}
  \ruby[g]{{\換字{文}}庫}{ぶんこ }
}{%
  \ruby{帝國{\換字{文}}庫}{てい|こく|ぶん|こ}
}%
\makeatother
とかいふ
\ruby{大}{おほき}な
\ruby{本}{ほん}にでさア。
』
