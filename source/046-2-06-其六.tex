\Entry{其六}

『せーんせい!。
』

\ruby[g]{水野}{みづの}は
\ruby{振{\GWI{u8fd4-k}}}{ふり|かへ}りて
\ruby{見}{み}れば
\ruby{間}{あひ}の
\ruby{襖}{ふすま}は
\ruby{開}{あ}き
\ruby{居}{ゐ}て、そこに
\ruby[g]{身體}{からだ}を
\ruby{{\換字{半}}分此方}{はん|ぶん|こな|た}の
\ruby{燈}{ひ}に
\ruby{見}{み}せつ、お
\ruby{濱}{はま}は
\ruby{我}{わ}が
\ruby{方}{かた}を
\ruby{打護}{うち|まも}り
\ruby{居}{ゐ}たり。

『あヽ
\ruby{吃驚}{びつ|くり}した!。
\ruby{何}{なん}だ{\GWI{u1b001}}?、お
\ruby{濱}{はま}ちやん、
\ruby{突然}{だし|ぬけ}に
\ruby{其様}{そ|ん}な
\ruby{大}{おほき}な
\ruby{聲}{こゑ}をして!。
』

\ruby{頭髪}{か|み}を
\ruby{結}{むす}ばずして
\ruby[g]{後方}{うしろ}に
\ruby{下}{さ}げたれば、ひとしほ
\ruby[g]{兒童}{こども}らしく
\ruby{活潑}{くわつ|ぱつ}に
\ruby{見}{み}ゆる
\ruby{面}{おもて}の、
\ruby{小}{ちひ}さけれど
\ruby{{\GWI{u6df8-jv}}}{すヾ}しき
\ruby{眼}{め}を
\ruby{出來}{で|き}るたけ
\ruby{見張}{み|は}りて、

『あら、
\ruby{先生}{せん|せい}〳〵つて
\ruby{幾度呼}{いく|ど|よ}んだか
\ruby{知}{し}れやしませんのに、ホヽ
\ruby{先生}{せん|せい}が
\ruby{{\換字{叉}}夢中}{また|む|ちう}になつて
\ruby{居}{ゐ}らしつたんだは。
』

と、お
\ruby{濱}{はま}は
\ruby{憚}{はヾか}り
\ruby{無}{な}く
\ruby{事實}{まこ|と}を
\ruby{語}{かた}りて、
\ruby{却}{かへ}つて
\ruby[g]{水野}{みづの}を
\ruby{難}{なん}じ
\ruby{反}{かへ}しぬ。

『
\ruby{左様}{さ|う}かェ、それぢやあ
\ruby{私}{わたし}が
\ruby{惡}{わる}かつた、
\ruby{堪忍}{かん|にん}〳〵!。
そして
\ruby{何}{なに}か
\ruby{用}{よう}?、
\ruby{用}{よう}ぢや
\ruby{無}{な}いの?。
』

『
\ruby[g]{御爺}{おぢい}さんが\GWI{u1b098}、
\ruby{番茶}{ばん|ちや}ですが
\ruby{出來}{で|き}ましたから
\ruby{御{\GWI{hkcs_m98f2}}}{お|あが}りなさいませんか
\ruby{御茶受}{お|ちや|うけ}は
\ruby{柴栗}{しば|ぐり}の
\ruby{煠}{ゆ}でたのばつかりですけれども、
\ruby{御茶}{お|ちや}でもあがつて、そして
\ruby{餘}{あんま}り
\ruby{根氣}{こ|ん}を
\ruby{御詰}{お|つ}めなさらないで、もう
\ruby{御休息}{お|やす|み}なすつた
\ruby{方}{はう}が
\ruby{宣}{よろし}うございましやうッて!。
』

『
\ruby{左様}{さ|う}!。
そりやあ
\ruby{有}{あ}り
\ruby{難}{がた}う!。
それぢや
\ruby[g]{其方}{そつち}へ
\ruby{行}{い}つて
\ruby{御馳走}{ご|ち|そう}にならうが、
\ruby{栗}{くり}はお
\ruby{濱}{はま}ちやんが
\ruby{剥}{む}いて
\ruby{{\換字{呉}}}{く}れるのかェ。
』

『いやよ、ずるい
\ruby{事}{こと}\GWI{u1b098}ェ
\ruby{先生}{せん|せい}は。
アヽ
\ruby{好}{い}いは、
\ruby{妾}{わたし}が
\ruby{剥}{む}いたのは
\ruby{先生}{せん|せい}にあげますから、
\ruby{先生}{せん|せい}も
\ruby{妾}{わたし}に
\ruby{剥}{む}いて
\ruby{頂戴}{ちやう|だい}ナ。
』

\ruby{互}{たがひ}に
\ruby{戯}{たはむ}れて
\ruby{言}{ものい}ひながら、お
\ruby{濱}{はま}は
\ruby{縋}{すが}るやうに
\ruby[g]{水野}{みづの}の
\ruby{手}{て}を
\ruby{取}{と}つて
\ruby{誘}{いざな}へば、
\ruby[g]{水野}{みづの}はまた
\ruby{扶}{たす}くるが
\ruby{如}{ごと}くお
\ruby{濱}{はま}をあしらひて、
\ruby{共}{とも}に
\ruby{直}{たヾち}に
\ruby{茶}{ちや}の
\ruby{間}{ま}に
\ruby{至}{いた}るに、
\ruby{果}{はた}して
\ruby{焙}{はう}じたる
\ruby{茶}{ちや}の
\ruby{香}{かほり}は
\ruby{一室}{いつ|しつ}に
\ruby{充}{み}ち
\ruby{満}{み}ちたり。

\ruby{三人}{さん|にん}は
\ruby{一}{ひと}ッ
\ruby{燈}{ひ}の
\ruby{下}{もと}に
\ruby{鼎}{かなへ}に
\ruby{坐}{すわ}りて、
\ruby{互}{たがひ}に
\ruby{其}{そ}の
\ruby{{\GWI{u6df8-jv}}}{きよ}らに
\ruby{和}{やさ}しき
\ruby{心}{こヽろ}より
\ruby{溢}{あふ}るヽ
\ruby{何}{なん}とは
\ruby{無}{な}しの
\ruby{微笑}{ほヽ|ゑみ}を
\ruby{取}{と}り
\ruby{換}{かは}しつ、
\ruby{言}{い}はず
\ruby{語}{かた}らずの
\ruby{中}{うち}に
\ruby{何事}{なに|ごと}も
\ruby{無}{な}き
\ruby{此夜}{この|よ}の
\ruby{靜}{しづか}さを
\ruby{相{\GWI{u6085-jv}}}{あい|よろこ}べり。

もとより
\ruby{廣}{ひろ}からぬ
\ruby{家}{いへ}の
\ruby{事}{こと}なり、
\ruby[g]{吉右衛門}{きちゑもん}は
\ruby{二人}{ふた|り}の
\ruby{應答}{うけ|こたへ}を
\ruby{悉}{こと〴〵}く
\ruby{聞}{き}きたれば、

『また
\ruby{先生}{せん|せい}に
\ruby{甘}{あま}つたれるよ。
\ruby{先生}{せん|せい}に
\ruby{剥}{む}いて
\ruby{戴}{いたヾ}いて
\ruby{食}{た}べやうなんて、お
\ruby{前}{まへ}のやうに
\ruby{{\GWI{u9060-k}}慮}{ゑん|りよ}を
\ruby{知}{し}らない
\ruby{女}{こ}は
\ruby{有}{あ}りやあ
\ruby{仕}{し}ない!。
ハヽヽヽ、さあお
\ruby{茶}{ちや}を
\ruby{御}{お}あげ、
\ruby{栗}{くり}も
\ruby{汝巧}{おまへ|うま}く
\ruby{剥}{む}けるなら
\ruby{剥}{む}いておあげ。
』

と、
\ruby{一寸眞面目}{ちよ|いと|ま|じ|め}には
\ruby{窘}{たしな}めながら、
\ruby{叱}{しか}るが
\ruby{矢張笑顏}{や|はり|ゑ|がほ}にて、
\ruby{更}{さら}に
\ruby{叱}{しか}るにはならぬもをかし。

『イヤ、ほんとは
\ruby{栗}{くり}は
\ruby{剥}{む}いて
\ruby{貰}{もら}はなくつても
\ruby{澤山}{たく|さん}だよ。
お
\ruby{濱}{はま}ちやん!。
\ruby{危}{あぶな}い
\ruby{手}{て}つきか
\ruby{何}{なん}かでもつて
\ruby{剥}{む}いて
\ruby{貰}{もら}つて、
\ruby{指}{ゆび}でも
\ruby{負傷}{け|が}をされやうもんなら
\ruby{大變}{たい|へん}だから\GWI{u1b098}ェ。
』

かくいふ
\ruby{間}{ま}にお
\ruby{濱}{はま}は
\ruby{其}{そ}の
\ruby{香}{かう}ばしき
\ruby{茶}{ちや}を
\ruby{茶碗}{ちや|わん}に
\ruby{注}{つ}ぎて、
\ruby{一個}{ひと|つ}は
\ruby[g]{水野}{みづの}の
\ruby{前}{まへ}、
\ruby{一個}{ひと|つ}は
\ruby{祖父}{ぢ|ヾ}の
\ruby{前}{まへ}に
\ruby{差}{さ}し
\ruby{置}{お}けば、

『ぢやあ
\ruby{御{\換字{勝}}手}{ご|かつ|て}に、』

と、
\ruby{小}{ちいさ}き
\ruby{笊籬}{ざ|る}に
\ruby{入}{い}れたる
\ruby{栗實}{く|り}の
\ruby{今煠}{いま|ゆ}で
\ruby{上}{あ}げしばかりと
\ruby{見}{み}{\GWI{u1b001}}て
\ruby{{\GWI{u7336-k}}}{なほ}
\ruby{其}{そ}の
\ruby{皮}{かわ}の
\ruby{蒸氣}{ゆ|げ}に
\ruby{濕}{しめ}れるに
\ruby{小刀添}{こが|たな|そ}へて
\ruby{{\GWI{u76c6-k}}}{ぼん}に
\ruby{載}{の}せたるを
\ruby{主人}{ある|じ}は
\ruby{差}{さ}し
\ruby{出}{だ}しぬ。

『いヽわ、
\ruby{先生}{せん|せい}!そんな
\ruby{事}{こと}を
\ruby{云}{い}つて!。
\ruby{澤山}{たく|さん}でも
\ruby{何}{なん}でも
\ruby{剥}{む}いて
\ruby{上}{あ}げますよ。
\ruby{危}{あぶな}つかしい
\ruby{手}{て}つきだなんて
\ruby{云}{い}つたから
\ruby{{\GWI{u7336-k}}}{なほ}
\ruby{剥}{む}いてあげるわ。
さうして
\ruby{若萬一}{もし|ひよ|つと}
\ruby{負傷}{け|が}を
\ruby{仕}{し}て
\ruby{血}{ち}でも
\ruby{出}{で}たらば、その
\ruby{血}{ち}の
\ruby{着}{つ}いたのもあげるからいヽわ。
』

『あヽ、もうあやまつた、
\ruby{怒}{おこ}つちやあいけない。
\ruby{私}{わたし}が
\ruby{二}{ふた}ッ
\ruby{三}{み}ッ
\ruby{剥}{む}いてあげるから
\ruby{中直}{なか|なお}り
\ruby{中直}{なか|なお}り!。
』

『ナァに
\ruby{優}{やさ}しくなさると
\ruby{{\GWI{u7336-k}}}{なほ }
\ruby{增長}{ぞう|ちやう}します。
そんな
\ruby{下}{くだ}らない
\ruby{事}{こと}を
\ruby{云}{い}つたのをとツこに、
\ruby{指先}{ゆび|さき}が
\ruby{痛}{いた}くなつて
\ruby{困}{こま}る
\ruby{位}{くらゐ}
\ruby{剥}{ む}かせて
\ruby{御{\GWI{u9063-k}}}{お|や}んなさる
\ruby{方}{はう}が
\ruby{宣}{よ}うございますのに。
ハヽヽ。
』

『ハヽヽ、
\ruby{憫然}{かあい|さう}に!。
お
\ruby{濱}{はま}ちやんも
\ruby{御爺}{お|ぢい}さんに
\ruby{會}{あ}つちやあ
\ruby{敵}{かな}はない\GWI{u1b098}。
』

『いやもう
\ruby{然様}{さ|う}ではございません、
\ruby{此女}{こ|れ}には
\ruby{老夫}{おや|ぢ}の
\ruby{方}{はう}が
\ruby{始{\GWI{u7d42-ue0101}}}{し|ヾう}
\ruby{{\GWI{u5f31-k}}}{よわ}らされます。
\ruby{談話}{はな|し}をしろ
\ruby{談話}{はな|し}を
\ruby{仕}{し}ろつて
\ruby{{\GWI{u5f3a-g}}{\GWI{u8acb-k}}}{せ|が}みまして\GWI{u1b098}。
\ruby{自分}{じ|ぶん}が
\ruby{散々}{さん|〴〵}に
\ruby{書}{ほん}を
\ruby{讀}{よ}んで
\ruby{置}{お}いて、まだ
\ruby{其上}{その|うへ}に
\ruby{其}{そ}の
\ruby{談話}{はな|し}を
\ruby{仕}{し}ろつて
\ruby{責}{せ}めるんですもの。
』

『あら
\ruby{御爺}{お|ぢい}さん、そりやあ
\ruby{{\GWI{u904e-k}}日}{こな|ひだ}の
\ruby{晩}{ばん}ばかりだは。
ありやあ
\ruby{書}{ほん}がむづかしくつて
\ruby{妾}{わたし}にやあ
\ruby{分}{わか}らなかつたからだは。
』

『
\ruby{一體何}{いつ|たい|なん}の
\ruby{書}{ほん}だつたの?。
』

『いやな
\ruby{書}{ほん}だつたの!。
』

『
\ruby{{\GWI{u5acc-k}}}{いや}な
\ruby{書}{ほん}てまあ、
\ruby{何}{なん}といふ
\ruby{書}{ほん}?。
』

『お
\ruby{爺}{ぢい}さん、
\ruby{黙}{だま}つて
\ruby{居}{ゐ}てよ。
\ruby{云}{い}はないで
\ruby{居}{ゐ}てよ!
\ruby{妾}{わたし}あたヾ
\ruby{本家}{ほん|け}から
\ruby{手當}{て|あた}り
\ruby{次第}{し|だい}に
\ruby{持}{も}つて
\ruby{來}{き}たばかしで、
\ruby{別}{べつ}に
\ruby{彼書}{あ|れ}を
\ruby{持}{も}つて
\ruby{來}{き}たんぢや
\ruby{無}{な}かつたんだから。
』

『ハテナ、
\ruby{匿}{かく}されると
\ruby{{\GWI{u7336-k}}}{なほ}
\ruby{聞}{き}きたいが
\ruby{何}{なん}の
\ruby{書}{ほん}だらう?。
』

『イヤ
\ruby{新}{あたら}しい
\ruby{活版刷}{くわつ|ぱん|ずり}の
\ruby{西洋綴}{せい|やう|とぢ}の
\ruby{書}{ほん}にやあ
\ruby{彼様}{あ|ん}なものはよもや
\ruby{入}{はい}つて
\ruby{居}{ゐ}まいと
\ruby{思}{おも}つて
\ruby{居}{ゐ}ましたが。
\ruby{飛}{と}んでも
\ruby{無}{な}い
\ruby{書}{ほん}が
\ruby{入}{はい}つて
\ruby{居}{ゐ}ましたのさ。
あの
\ruby{帝国文庫}{てい|こく|ぶん|こ}とかいふ
\ruby{大}{おほき}な
\ruby{本}{ほん}にでさア。
』

