\Entry{其六}

\原本頁{}%
『せーんせい!。
』

\原本頁{}%
\ruby[g]{水野}{みづの}は
\ruby{振{\換字{返}}}{ふり|かへ}りて
\ruby{見}{み}れば
\ruby{間}{あひ}の
\ruby{襖}{ふすま}は
\ruby{開}{あ}き
\ruby{居}{ゐ}て、
%
そこに
\ruby{身體}{から|だ}を
\ruby{{\換字{半}}{\換字{分}}}{はん|ぶん}
\ruby[g]{此方}{こなた}の
\ruby{燈}{ひ}に
\ruby{見}{み}せつ、
%
お
\ruby{濱}{はま}は
\ruby{我}{わ}が
\ruby{方}{かた}を
\ruby{打護}{うち|まも}り
\ruby{居}{ゐ}たり。

\原本頁{}%
『あゝ
\ruby{吃驚}{びつ|くり}した!。
%
\ruby{何}{なん}だ{\換字{𛀁}}?、
%
お
\ruby{濱}{はま}ちやん、
%
\ruby{突然}{だし|ぬけ}に
\ruby{其樣}{そ|ん}な
\ruby{大}{おほき}な
\ruby{聲}{こゑ}をして!。
』

\原本頁{}%
\ruby{頭髮}{か|み}を
\ruby{結}{むす}ばずして
\ruby{後方}{うし|ろ}に
\ruby{下}{さ}げたれば、
%
ひとしほ
\ruby{兒童}{こ|ども}らしく
\ruby{活潑}{くわつ|ぱつ}に
\ruby{見}{み}ゆる
\ruby{面}{おもて}の、
%
\ruby{小}{ちひ}さけれど
\ruby{淸}{すゞ}しき
\ruby{眼}{め}を
\ruby{出來}{で|き}るたけ
\ruby{見張}{み|は}りて、

\原本頁{}%
『あら、
%
\ruby{先生}{せん|せい}〳〵つて
\ruby{幾度}{いく|ど}
\ruby{呼}{よ}んだか
\ruby{知}{し}れやしませんのに、
%
ホヽ
\ruby{先生}{せん|せい}が
\ruby{{\換字{又}}}{また}
\ruby{夢中}{む|ちう}になつて
\ruby{居}{ゐ}らしつたんだは。
』

\原本頁{}%
と、
%
お
\ruby{濱}{はま}は
\ruby{憚}{はゞか}り% 「憚 は(ゞ)か」
\ruby{無}{な}く
\ruby{事實}{ま|こと}を
\ruby{語}{かた}りて、
%
\ruby{却}{かへ}つて
\ruby[g]{水野}{みづの}を
\ruby{難}{なん}じ
\ruby{反}{かへ}しぬ。

\原本頁{}%
『
\ruby{左樣}{さ|う}かエ、
%
それぢやあ
\ruby{私}{わたし}が
\ruby{惡}{わる}かつた、
%
\ruby{勘{\換字{忍}}}{かん|にん}〳〵!。% 原文通り「勘忍」
%
そして
\ruby{何}{なに}か
\ruby{用}{よう}?、
%
\ruby{用}{よう}ぢや
\ruby{無}{な}いの?。
』

\原本頁{}%
『
\ruby{御爺}{お|ぢい}さんが\換字{子}、
%
\ruby{番茶}{ばん|ちや}ですが
\ruby{出來}{で|き}ましたから
\ruby{御飮}{お|あが}りなさいませんか
\ruby{御茶受}{お|ちや|うけ}は
\ruby{柴栗}{しば|ぐり}の
\ruby{煠}{ゆ}でたのばつかりですけれども、
%
\ruby{御茶}{お|ちや}でもあがつて、
%
そして
\ruby{餘}{あんま}り
\ruby{根氣}{こ|ん}を
\ruby{御詰}{お|つ}めなさらないで、
%
もう
\ruby{御休息}{お|やす|み}なすつた
\ruby{方}{はう}が
\ruby{宜}{よ}うございましやうツて!。
』

\原本頁{}%
『
\ruby{左樣}{さ|う}!。
%
そりやあ
\ruby{有}{あ}り
\ruby{難}{がた}う!。
%
それぢや
\ruby{其方}{そつ|ち}へ
\ruby{行}{い}つて
\ruby{御馳走}{ご|ち|そう}にならうが、
%
\ruby{栗}{くり}は
お
\ruby{濱}{はま}ちやんが
\ruby{剝}{む}いて
\ruby{吳}{く}れるのかエ。
』

\原本頁{}%
『いやよ、
%
ずるい
\ruby{事}{こと}\換字{子}エ
\ruby{先生}{せん|せい}は。
%
アヽ
\ruby{好}{い}いは、
%
\ruby{妾}{わたし}が
\ruby{剝}{む}いたのは
\ruby{先生}{せん|せい}にあげますから、
%
\ruby{先生}{せん|せい}も
\ruby{妾}{わたし}に
\ruby{剝}{む}いて
\ruby{頂戴}{ちやう|だい}ナ。
』

\原本頁{}%
\ruby{互}{たがひ}に
\ruby{戱}{たはむ}れて
\ruby{言}{ものい}ひながら、
%
お
\ruby{濱}{はま}は
\ruby{縋}{すが}るやうに
\ruby[g]{水野}{みづの}の
\ruby{手}{て}を
\ruby{取}{と}つて
\ruby{誘}{いざな}へば、
%
\ruby[g]{水野}{みづの}はまた
\ruby{扶}{たす}くるが
\ruby{如}{ごと}く
お
\ruby{濱}{はま}をあしらひて、
%
\ruby{共}{とも}に
\ruby{直}{たゞち}に
\ruby{茶}{ちや}の
\ruby{間}{ま}に
\ruby{至}{いた}るに、
%
\ruby{果}{はた}して
\ruby{焙}{ほう}じたる
\ruby{茶}{ちや}の
\ruby{香}{かほり}は
\ruby{一室}{いつ|しつ}に
\ruby{充}{み}ち
\ruby{滿}{み}ちたり。

\原本頁{}%
\ruby{三人}{さん|にん}は
\ruby{一}{ひと}ツ
\ruby{燈}{ひ}の
\ruby{下}{もと}に
\ruby{鼎}{かなへ}に
\ruby{坐}{すわ}りて、
%
\ruby{互}{たがひ}に
\ruby{其}{そ}の
\ruby{淸}{きよ}らに
\ruby{和}{やさ}しき
\ruby{心}{こゝろ}より
\ruby{溢}{あふ}るゝ
\ruby{何}{なん}とは
\ruby{無}{な}しの
\ruby{微笑}{ほゝ|ゑみ}を
\ruby{取}{と}り
\ruby{換}{かは}しつ、
%
\ruby{言}{い}はず
\ruby{語}{かた}らずの
\ruby{中}{うち}に
\ruby{何事}{なに|ごと}も
\ruby{無}{な}き
\ruby{此夜}{この|よ}の
\ruby{靜}{しづか}さを
\ruby{相悅}{あひ|よろこ}べり。

\原本頁{}%
もとより
\ruby{廣}{ひろ}からぬ
\ruby{家}{いへ}の
\ruby{事}{こと}なり、
%
\ruby[g]{吉右衛門}{きちゑもん}は
\ruby{二人}{ふた|り}の
\ruby{應答}{うけ|こたへ}を
\ruby{悉}{こと〴〵}く
\ruby{聞}{き}きたれば、

\原本頁{}%
『また
\ruby{先生}{せん|せい}に
\ruby{甘}{あま}つたれるよ。
%
\ruby{先生}{せん|せい}に
\ruby{剝}{む}いて
\ruby{戴}{いたゞ}いて
\ruby{食}{た}べやうなんて、
%
お
\ruby{{\換字{前}}}{まへ}のやうに
\ruby{{\換字{遠}}慮}{ゑん|りよ}を
\ruby{知}{し}らない
\ruby{女}{こ}は
\ruby{有}{あ}りやあ
\ruby{仕}{し}ない!。
%
ハヽヽヽ、
%
さあお
\ruby{茶}{ちや}を
\ruby{御}{お}あげ、
%
\ruby{栗}{くり}も
\ruby[h|]{汝}{おまへ}
\ruby{巧}{うま}く
\ruby{剝}{む}けるなら
\ruby{剝}{む}いておあげ。
』

\原本頁{}%
と、
%
\ruby{一寸}{ちよ|いと}
\ruby{眞面目}{ま|じ|め}には
\ruby{窘}{たしな}めながら、
%
\ruby{叱}{しか}るが
\ruby{矢張}{や|はり}
\ruby{笑顏}{ゑ|がほ}にて、
%
\ruby{{\換字{更}}}{さら}に
\ruby{叱}{しか}るにはならぬもをかし。

\原本頁{}%
『イヤ、
%
ほんとは
\ruby{栗}{くり}は
\ruby{剝}{む}いて
\ruby{貰}{もら}はなくつても
\ruby{澤山}{たく|さん}だよ。
%
お
\ruby{濱}{はま}ちやん!。
%
\ruby{危}{あぶな}い
\ruby{手}{て}つきか
\ruby{何}{なん}かでもつて
\ruby{剝}{む}いて
\ruby{貰}{もら}つて、
%
\ruby{指}{ゆび}でも
\ruby{負傷}{け|が}をされやうもんなら
\ruby{大變}{たい|へん}だから\換字{子}エ。
』

\原本頁{}%
かくいふ
\ruby{間}{ま}に
お
\ruby{濱}{はま}は
\ruby{其}{そ}の
\ruby{香}{かう}ばしき
\ruby{茶}{ちや}を
\ruby{茶碗}{ちや|わん}に
\ruby{注}{つ}ぎて、
%
\ruby{一個}{ひと|つ}は
\ruby[g]{水野}{みづの}の
\ruby{{\換字{前}}}{まへ}、
%
\ruby{一個}{ひと|つ}は
\ruby{祖{\換字{父}}}{ぢ|ゞ}の
\ruby{{\換字{前}}}{まへ}に
\ruby{差}{さ}し
\ruby{置}{お}けば、

\原本頁{}%
『ぢやあ
\ruby{御{\換字{勝}}手}{ご|かつ|て}に、
』

\原本頁{}%
と、
%
\ruby{小}{ちひさ}き
\ruby{笊籬}{ざ|る}に
\ruby{入}{い}れたる
\ruby{栗實}{く|り}の
\ruby{今}{いま}
\ruby{煠}{ゆ}で
\ruby{上}{あ}げしばかりと
\ruby{見}{み}\換字{𛀁}て
\ruby{{\換字{猶}}}{なほ}
\ruby{其}{そ}の
\ruby{皮}{かわ}の% 原本通り「皮 か(わ)」
\ruby{蒸氣}{ゆ|げ}に
\ruby{濕}{しめ}れるに
\ruby{小刀添}{こ|がたな|そ}へて
\ruby{{\換字{盆}}}{ぼん}に
\ruby{載}{の}せたるを
\ruby{主人}{ある|じ}は
\ruby{差}{さ}し
\ruby{出}{だ}しぬ。

\原本頁{}%
『いゝわ、
%
\ruby{先生}{せん|せい}!そんな
\ruby{事}{こと}を
\ruby{云}{い}つて!。
%
\ruby{澤山}{たく|さん}でも
\ruby{何}{なん}でも
\ruby{剝}{む}いて
\ruby{上}{あ}げますよ。
%
\ruby{危}{あぶな}つかしい
\ruby{手}{て}つきだなんて
\ruby{云}{い}つたから
\ruby{{\換字{猶}}}{なほ}
\ruby{剝}{む}いてあげるわ。
%
さうして
\ruby{{\換字{若}}萬一}{もし|ひよ|つと}
\ruby{負傷}{け|が}を
\ruby{仕}{し}て
\ruby{血}{ち}でも
\ruby{出}{で}たらば、
%
その
\ruby{血}{ち}の
\ruby{着}{つ}いたのもあげるからいゝわ。
』

\原本頁{}%
『あゝ、
%
もうあやまつた、
%
\ruby{怒}{おこ}つちやあいけない。
%
\ruby{私}{わたし}が
\ruby{二}{ふた}ツ
\ruby{三}{み}ツ
\ruby{剝}{む}いてあげるから
\ruby{中直}{なか|なお}り
\ruby{中直}{なか|なお}り!。
』

\原本頁{}%
『ナアに
\ruby{優}{やさ}しくなさると
\ruby{{\換字{猶}}}{なほ}
\ruby{增長}{ぞう|ちやう}します。
%
そんな
\ruby{下}{くだ}らない
\ruby{事}{こと}を
\ruby{云}{い}つたのをとツこに、
%
\ruby{指先}{ゆび|さき}が
\ruby{痛}{いた}くなつて
\ruby{困}{こま}る
\ruby[h|]{位}{くらゐ}
\ruby{剝}{む}かせて
\ruby{御{\換字{遣}}}{お|や}んなさる
\ruby{方}{はう}が
\ruby{宜}{よ}うございますのに。
%
ハヽヽ。
』

\原本頁{}%
『ハヽヽ、
%
\ruby{憫然}{かあ|いさう}に!。% 「憫然 か(あ)いさう」
%
お
\ruby{濱}{はま}ちやんも
\ruby{御爺}{お|ぢい}さんに
\ruby{會}{あ}つちやあ
\ruby{敵}{かな}はない\換字{子}。
』

\原本頁{}%
『いやもう
\ruby{然樣}{さ|う}ではございません、
%
\ruby{此女}{こ|れ}には
\ruby{老夫}{おや|ぢ}の
\ruby{方}{はう}が
\ruby[g]{始{\換字{終}}}{しじう}% ルビは原本通り「ゆ」無し
\ruby{{\換字{弱}}}{よわ}らされます。
%
\ruby{談話}{はな|し}をしろ
\ruby{談話}{はな|し}を
\ruby{仕}{し}ろつて
\ruby{{\換字{強}}{\換字{請}}}{せ|が}みまして\換字{子}。
%
\ruby{自{\換字{分}}}{じ|ぶん}が
\ruby{散々}{さん|〴〵}に
\ruby{書}{ほん}を
\ruby{讀}{よ}んで
\ruby{置}{お}いて、
%
まだ
\ruby{其上}{その|うへ}に
\ruby{其}{そ}の
\ruby{談話}{はな|し}を
\ruby{仕}{し}ろつて
\ruby{責}{せ}めるんですもの。
』

\原本頁{}%
『あら
\ruby{御爺}{お|ぢい}さん、
%
そりやあ
\ruby{{\換字{過}}日}{こな|ひだ}の
\ruby{晩}{ばん}ばかりだは。
%
ありやあ
\ruby{書}{ほん}がむづかしくつて
\ruby{妾}{わたし}にやあ
\ruby{{\換字{分}}}{わか}らなかつたからだは。
』

\原本頁{}%
『
\ruby{一體}{いつ|たい}
\ruby{何}{なん}の
\ruby{書}{ほん}だつたの?。
』

\原本頁{}%
『いやな
\ruby{書}{ほん}だつたの!。
』

\原本頁{}%
『
\ruby{{\換字{嫌}}}{いや}な
\ruby{書}{ほん}てまあ、
%
\ruby{何}{なん}といふ
\ruby{書}{ほん}?。
』

\原本頁{}%
『お
\ruby{爺}{ぢい}さん、
%
\ruby{默}{だま}つて
\ruby{居}{ゐ}てよ。
%
\ruby{云}{い}はないで
\ruby{居}{ゐ}てよ!
\ruby{妾}{わたし}あたゞ
\ruby{本家}{ほん|け}から
\ruby{手當}{て|あた}り
\ruby{次第}{し|だい}に
\ruby{持}{も}つて
\ruby{來}{き}たばかしで、
%
\ruby{別}{べつ}に
\ruby{彼書}{あ|れ}を
\ruby{持}{も}つて
\ruby{來}{き}たんぢや
\ruby{無}{な}かつたんだから。
』

\原本頁{}%
『ハテナ、
%
\ruby{匿}{かく}されると
\ruby{{\換字{猶}}}{なほ}
\ruby{聞}{き}きたいが
\ruby{何}{なん}の
\ruby{書}{ほん}だらう?。
』

\原本頁{}%
『イヤ
\ruby{新}{あたら}しい
\ruby{活版刷}{くわつ|ぱん|ずり}の
\ruby{西洋綴}{せい|やう|とぢ}の
\ruby{書}{ほん}にやあ
\ruby{彼樣}{あ|ん}なものはよもや
\ruby{入}{はい}つて
\ruby{居}{ゐ}まいと
\ruby{思}{おも}つて
\ruby{居}{ゐ}ましたが。
%
\ruby{飛}{と}んでも
\ruby{無}{な}い
\ruby{書}{ほん}が
\ruby{入}{はい}つて
\ruby{居}{ゐ}ましたのさ。
%
あの
\ruby{帝國{\換字{文}}庫}{てい|こく|ぶん|こ}とかいふ
\ruby{大}{おほき}な
\ruby{本}{ほん}にでさア。
』
