\Entry{其七}

% メモ 校正終了 2024-04-17
\原本頁{39-8}%
『いや
\ruby{{\換字{古}}}{ふる}い
\ruby{本}{ほん}が
\ruby{新}{あたら}しくなつて
\ruby{澤山}{たく|さん}
\ruby{出}{で}るからね。
%
\ruby{左樣}{さ|う}して
\ruby{其}{そ}の
\ruby{書}{ほん}は
\ruby{何}{なん}と
\ruby{云}{い}ふ
\ruby{書}{ほん}だつたの?。
』

\原本頁{39-10}%
『ナアニ、
%
\ruby{私}{わたし}
なんぞが
\ruby{面皰}{にき|び}の
\ruby{出}{で}て
\ruby{居}{ゐ}た
\ruby{二才}{に|さい}の
\ruby{時{\換字{分}}}{じ|ぶん}
\ruby{貸本屋}{かし|ほん|や}で
\ruby{借}{か}りて
\ruby{讀}{よ}んだことのある
\ruby{人{\換字{情}}本}{にん|じやう|ぼん}で、
%
\ruby{初}{はじめ}は
\ruby{甚}{ひど}く
\ruby{{\換字{嫌}}}{きら}はれて
\ruby{居}{ゐ}た
\ruby{男}{をとこ}の、
%
\ruby{其}{そ}の
\ruby{親切}{しん|せつ}が
\ruby{{\換字{通}}}{つう}じて
\ruby{思}{おも}ひ
\ruby{思}{おも}はれる
やうになるといふ
\ruby{趣向}{ゆく|たて}を
\ruby{書}{か}いた
\ruby{下}{くだ}らない
ものでした。
』

\原本頁{40-3}%
『ハヽア、
%
それぢやあ
\ruby{二筋{\換字{道}}}{ふた|すぢ|みち}といふのぢやあ
\ruby{無}{な}いか、
%
そんなら
\ruby{何樣}{ど|う}して
\ruby{百年}{ひやく|ねん}も
\ruby{{\換字{前}}}{まへ}の
\ruby{{\換字{古}}}{ふる}いものだから、
%
いくら
\ruby{總傍訓}{そう|ふり|がな}が
あつたにしても、
%
こりやあ
お
\ruby{濱}{はま}ちやんには
\ruby{些}{ちつと}も
\ruby{{\換字{分}}}{わか}らなかつたろう。
%
\ruby{私等}{わたし|ら}に
さへ
\ruby{明瞭}{はつ|きり}とは
\ruby{解}{わか}らないところが
あるんだもの!。
』

\原本頁{40-8}%
『ハヽヽ、
%
\ruby{彼樣}{あ|ん}な
\ruby{書}{もの}がまあ
\ruby{左樣}{さ|う}ですか\換字{子}エ。
%
\ruby{成程}{なる|ほど}
いくら
\ruby{物}{もの}を
\ruby{知}{し}つて
\ruby{居}{ゐ}らしつても
\ruby{御{\換字{若}}}{お|わか}いから
\ruby{何樣}{ど|う}も
\ruby{仕方}{し|かた}が
ありません、
%
\ruby{御維新}{ご|いつ|しん}
\ruby{此方}{この|かた}
\ruby{物事}{もの|ごと}が
\ruby{全然}{すつ|かり}
\ruby{異}{ちが}つて
\ruby{參}{まゐ}りましたから\換字{子}。
%
さうすると
\ruby{{\換字{古}}}{むかし}の
\ruby{人{\換字{情}}本}{にん|じやう|ぼん}の
\ruby{精}{よ}く
\ruby{{\換字{分}}}{わか}るのは、
%
\ruby{此席}{こ|〻}ぢやあ% 原本通り「〻(二の字点、揺すり点)」
\ruby{私}{わたし}
ばつかりといふ
\ruby{譯}{わけ}ですか。
%
ハヽヽ、
%
\ruby{老夫}{おや|ぢ}も
たまにあ
\ruby{貴下}{あな|た}より
\ruby{{\換字{強}}}{つよ}い
ところが
ありますカ\換字{子}。
』

\原本頁{41-2}%
『
\ruby{詰}{つま}らない
\ruby{自慢}{じ|まん}を
\ruby{仕}{し}て!。
%
をかしな
\ruby{御爺}{お|ぢい}さん!。
%
どうせ
\ruby{御大名}{お|だい|みやう}の
\ruby{行列}{ぎやう|れつ}
なんぞ
\ruby{知}{し}つて
\ruby{居}{ゐ}るのも
\ruby{御爺}{お|ぢい}さん
ばかりよ。
』

\原本頁{41-4}%
『ハヽヽ、
%
また
\ruby{此}{こ}の
\ruby{老夫}{おぢい|さん}を
やりこめるよ。
%
どうも
\ruby{左樣}{さ|う}
\ruby{此頃}{この|ごろ}のやうに
\ruby{威勢}{いき|ほひ}が
\ruby{{\換字{強}}}{つよ}くなつては
\ruby{敵}{かな}はないナ。
%
もう
\ruby{談話}{はな|し}も
\ruby{何}{なに}も
\ruby{仕}{し}て
やらないからい〻。% 原本通り「〻(二の字点、揺すり点)」
』

\原本頁{41-7}%
『い〻わ、% 原本通り「〻(二の字点、揺すり点)」
%
あんな
\ruby{昔風}{むかし|ふう}の
\ruby{御談話}{お|はな|し}よりも、
%
\ruby[g]{一昨日}{をと〻ひ}から% 原本通り「〻(二の字点、揺すり点)」
\ruby{讀}{よ}んで
\ruby{居}{ゐ}る
\ruby{魯敏孫}{ろ|びん|そん}の
\ruby{御話}{お|はなし}の
\ruby{方}{はう}が
いくら
\ruby{面白}{おも|しろ}いか
\ruby{知}{し}れや
\ruby{仕}{し}ない。
』

\原本頁{41-9}%
『
\ruby{魯敏孫}{ろ|びん|そん}の
\ruby{談話}{はな|し}つて、
%
あの
\ruby{漂流記}{へう|りう|き}?。
』

\原本頁{41-10}%
『え〻% 原本通り「〻(二の字点、揺すり点)」
\ruby{左樣}{さ|う}よ、
%
あの
\ruby{魯敏孫}{ろ|びん|そん}
\ruby{漂流記}{へう|りう|き}よ。
』

\原本頁{41-11}%
『
\ruby{左樣}{さ|う}!。
%
さうして
\ruby{彼書}{あ|れ}が
\ruby{其樣}{そん|な}に
お
\ruby{濱}{はま}ちやんには
\ruby{面白}{おも|しろ}いの?。
』

\原本頁{42-1}%
『
\ruby{何故}{な|ぜ}?。
%
\ruby{先生}{せん|せい}にやあ
\ruby{彼書}{あ|れ}が
\ruby{面白}{おも|しろ}くないの!。
%
\ruby{先生}{せん|せい}は
\ruby{魯敏孫}{ろ|びん|そん}を
\ruby{偉}{えら}いとは
\ruby{思}{おも}はなくつて?。
%
\ruby{妾}{わたし}あ
\ruby{眞實}{ほん|と}に
\ruby{彼}{あ}の
\ruby{人}{ひと}が
\ruby{好}{すき}だわ。
%
\ruby{海}{うみ}の
\ruby{中}{なか}の
\ruby{小島}{こ|じま}に
\ruby[<j|]{唯}{たつた}
\ruby{一人}{ひと|り}で、
%
\ruby{立派}{りつ|ぱ}に
\ruby{生}{い}きて
\ruby{行}{ゆ}くなあ
\ruby{偉}{えら}いぢや
ありませんか。
%
\ruby{妾}{わたし}あ
\ruby{彼}{あ}の
\ruby{書}{ほん}を
\ruby{讀}{よ}んで
\ruby{斯}{か}う
\ruby{思}{おも}つたわ。
』

\原本頁{42-5}%
『おもしろい\換字{子}エ。
%
\ruby{何樣}{ど|ん}な
\ruby{事}{こと}を
\ruby{思}{おも}つたエ。
』

\原本頁{42-6}%
『
\ruby{妾}{わたし}も
\ruby{何樣}{ど|う}かした
\ruby{譯}{わけ}で
\ruby{其}{そ}の
\ruby{島}{しま}へ
\ruby{行}{い}つて\換字{子}、
%
さうして
\ruby{彼}{あ}の
\ruby{魯敏孫}{ろ|びん|そん}と
\ruby{一處}{いつ|しよ}に
\ruby{棲}{す}んで、
%
\ruby{荒}{あら}い
\ruby{事}{こと}は
\ruby{魯敏孫}{ろ|びん|そん}に
\ruby{仕}{し}て
\ruby{貰}{もら}ふ
\ruby{代}{かは}り、
%
こま〳〵とした
\ruby{事}{こと}は
\ruby{妾}{わたし}が
\ruby{仕}{し}て
\ruby{{\換字{遣}}}{や}つて、
%
\ruby{晝間}{ひる|ま}は
\ruby{一生懸命}{いつ|しやう|けん|めい}に
\ruby{働}{はたら}いても、
%
\ruby{夜}{よる}や
\ruby{雨}{あめ}の
\ruby{降}{ふ}つた
\ruby{靜}{しづか}かな
\ruby{日}{ひ}には
お
\ruby{話}{はなし}なんぞ
\ruby{仕}{し}て
\ruby{{\換字{遊}}}{あそ}んで
\ruby{居}{ゐ}たらば、
%
ほんとに
\ruby{何樣}{ど|ん}なにか
\ruby{面白}{おも|しろ}からうと
\ruby{思}{おも}つたのよ。
』

\原本頁{42-11}%
『ハヽヽ。
%
また
\ruby{下}{くだ}らないことを
\ruby{云}{い}ひ
\ruby{出}{だ}したナ。
』

\原本頁{43-1}%
『ハヽ、
%
こりやあ
\ruby{面白}{おも|しろ}い
\ruby{面白}{おも|しろ}い!。
%
ぢやあ
お
\ruby{濱}{はま}ちやんは
\ruby{魯敏孫}{ろ|びん|そん}の
\ruby{夫人}{おく|さん}に
なりたいと
いふんだ\換字{子}。
』

\原本頁{43-3}%
『いやな
\ruby{先生}{せん|せい}\換字{子}エ、
%
\ruby{夫人}{おく|さん}だなんて!。
%
\ruby{妾}{わたし}あ
\ruby{他}{ひと}の
\ruby{夫人}{おく|さん}に
なつたり、
%
\ruby{他}{ひと}の
\ruby{良人}{ごてい|しゆ}に
なつたりする
\ruby{人}{ひと}は
\ruby{大{\換字{嫌}}}{だい|きら}ひだわ。
%
\ruby{妾}{わたし}あ
\ruby{唯}{たゞ}% TODO 原本の「二の字点、揺すり点」に濁点のグリフが見つからないので「ゞ」
\ruby{魯敏孫}{ろ|びん|そん}の
\ruby{朋友}{おとも|だち}になつて
\ruby{見度}{み|た}いつて
\ruby{云}{い}つたのだわ。
』

\原本頁{43-6}%
『ハヽ、
%
\ruby{成程}{なる|ほど}、
%
\ruby{{\換字{分}}}{わか}つたよ、
%
\ruby{面白}{おも|しろ}いねエ。
%
つまり
お
\ruby{濱}{はま}ちやんは
\ruby[<j|]{女}{をんな}
\ruby{魯敏孫}{ろ|びん|そん}に
なりたいのだらう。
』

\原本頁{43-8}%
『え〻、% 原本通り「〻(二の字点、揺すり点)」
%
\ruby{左樣}{さ|う}なのよ。
%
ほんとに
\ruby{左樣}{さ|う}なのよ。
%
\ruby{眞靑}{まつ|さを}で
\ruby{際涯}{は|て}の
\ruby{無}{な}い
\ruby{大}{おほき}な
\ruby{洋}{うみ}の、
%
\ruby{塵}{ちり}も
\ruby{何}{なんに}も
\ruby{無}{な}い
\ruby{奇麗}{き|れい}な
\ruby{島}{しま}の
\ruby{中}{なか}で、
%
あの
\ruby{男兒}{をと|こ}らしい
\ruby{魯敏孫}{ろ|びん|そん}と、
%
たつた
\ruby{二人}{ふた|り}で
\ruby{働}{はたら}いて
\ruby{居}{ゐ}たら、
%
\ruby{妾}{わたし}あ
\ruby{何樣}{ど|ん}なに
\ruby{好}{い}い
\ruby{心持}{こ〻ろ|もち}% 原本通り「〻(二の字点、揺すり点)」
だらうと
\ruby{思}{おも}つて
\ruby{居}{ゐ}るのよ。
』

\原本頁{44-1}%
『これですもの、
%
どうも、
%
\ruby{呆}{あき}れて
\ruby{仕舞}{し|ま}ひます!。
%
\ruby{此女}{こ|れ}は
\ruby{取}{と}り
\ruby{{\換字{分}}}{わ}け
\ruby{無茶}{む|ちや}なので
ございましやうが、
%
\ruby{大}{だい}なり
\ruby{小}{せう}なり
\ruby{明治}{めい|じ}の
\ruby{生兒}{うま|れ}は、
%
\ruby{悉皆}{みん|な}
\ruby{斯樣}{か|う}なので
ございましやうか、
%
まるで
\ruby{昔}{むかし}の
\ruby{女兒}{むすめ|つこ}とは
\ruby{異}{ちが}つて
\ruby{居}{を}ります。
%
\ruby{二筋{\換字{道}}}{ふた|すぢ|みち}の
\ruby{話}{はなし}を
\ruby{仕}{し}て
\ruby{聞}{き}かせるのも
\ruby{異}{い}なものでしたが、
%
あんまり
\ruby{何樣}{ど|う}いふ
\ruby{譯}{わけ}だ
\ruby{何樣}{ど|う}いふ
\ruby{譯}{わけ}だと
\ruby{煩}{うるさ}く
\ruby{聞}{き}かれましたから、
%
ほんの
ざつとした
\ruby{筋}{すぢ}だけを
\ruby{話}{はな}して
\ruby{{\換字{遣}}}{や}りましたのに、
%
\ruby{碌}{ろく}にも
\ruby{{\換字{遂}}}{と}げては
\ruby{聞}{き}きませんで、
%
\ruby{詰}{つま}らないと
\ruby{一}{ひ}ト
\ruby{口}{くち}に
\ruby{云}{い}つて
\ruby{仕舞}{し|ま}ひましたのも、
%
\ruby{一體}{いつ|たい}が
\ruby{斯樣}{か|う}いふ
\ruby{調子}{てう|し}ですから
\ruby{無理}{む|り}も
ありません。
%
\ruby{實}{じつ}に
\ruby{世}{よ}の
\ruby{中}{なか}は
\ruby{變}{かは}つて
まゐりました。
』

\原本頁{44-10}%
『だつて
\ruby{祖{\換字{父}}}{お|ぢい}さん!。
%
\ruby{二筋{\換字{道}}}{あ|の|ほん}の
\ruby{御話}{お|はなし}は、
%
\ruby{{\換字{嫌}}}{きら}ひな
\ruby{人}{ひと}が
\ruby{好}{すき}に
なるなんて、
%
\ruby{馬鹿}{ば|か}げて
\ruby{居}{ゐ}るんだもの!。
』

\原本頁{45-1}%
『でも
\ruby{其}{それ}が
\ruby{人{\換字{情}}}{にん|じやう}つて
\ruby{云}{い}ふものなんで、
%
まだ
\ruby{中々}{なか|〳〵}
\ruby{汝{\換字{達}}}{おまへ|たち}にやあ
\ruby{{\換字{分}}}{わか}らないんだよ。
』

\原本頁{45-3}%
『そんな、
%
\ruby{{\換字{嫌}}}{きら}ひなものが
\ruby{好}{すき}になる
\ruby{人{\換字{情}}}{にん|じやう}なんて、
%
そりやあ
お
\ruby{行列}{ぎやう|れつ}の
\ruby{時{\換字{分}}}{じ|ぶん}の
\ruby{人{\換字{情}}}{にん|じやう}ぢやなくつて?。
』

\原本頁{45-5}%
『
\ruby{生意氣}{なま|い|き}な!。
%
\ruby{何}{なに}が
\ruby{小児}{こ|ども}の
\ruby{汝}{おまへ}
なんぞに
\ruby{未}{ま}だ
\ruby{{\換字{分}}}{わか}るものか!。
』

\原本頁{45-6}%
『だつて
\ruby{幾歳}{いく|つ}に
なつたつて、
%
\ruby{妾}{わたし}にや
\ruby{{\換字{分}}}{わか}らないわ。
%
\ruby{妾}{わたし}や
\ruby{幾歳}{いく|つ}に
なつたつて、
%
\ruby{屹度}{きつ|と}
お
\ruby{澤}{さは}
\ruby{婆}{ば〻あ}は% 「ばゞ」のはずだが、原本通り「〻(二の字点、揺すり点)」
\ruby{{\換字{嫌}}}{きらひ}で
\ruby{先生}{せん|せい}は
\ruby{好}{すき}だわ。
%
\ruby{先生}{せん|せい}が
\ruby{{\換字{嫌}}}{きらひ}で
お
\ruby{澤}{さは}
\ruby{婆}{ば〻あ}が% 「ばゞ」のはずだが、原本通り「〻(二の字点、揺すり点)」
\ruby{好}{す}きにはなりやあ
\ruby{仕}{し}ないわ。
』
