\Entry{其七}

% メモ 校正終了 2024-04-17 2024-05-29 2024-06-30
\原本頁{39-8}%
『
いや
\ruby{{\換字{古}}}{ふる}い
\ruby{本}{ほん}が
\ruby{新}{あたら}しくなつて
\ruby[g]{澤山}{たくさん}
\ruby{出}{で}るからね。
%
\ruby[g]{左樣}{さ う }して
\ruby{其}{そ}の
\ruby{書}{ほん}は
\ruby{何}{なん}と
\ruby{云}{い}ふ
\ruby{書}{ほん}だつたの?。
』

\原本頁{39-10}%
『
ナアニ、
%
\ruby{私}{わたし}
なんぞが
\ruby[g]{面皰}{にきび }の
\ruby{出}{で}て
\ruby{居}{ゐ}た
\ruby[g]{二才}{に さい}
\footnote{%
検討対象の「\ruby[g]{二才}{に さい}」の記述について、
「二歳」ではなく「二才」と記す場合は「年齢」以外にも
「若くて未熟な人」の意味もあるので原本通りとする
(国会図書館 コマ番号24/160 p-039 l-10)}%
の
\ruby[g]{時{\換字{分}}}{じ ぶん}
\ruby{貸本屋}{かし|ぼん|や}で
\ruby{借}{か}りて
\ruby{讀}{よ}んだことのある
\ruby[||j>]{人}{にん}
\ruby[||j>]{{\換字{情}}}{じやう}
\ruby[||j>]{本}{ ぼん}で、
%
\ruby{初}{はじめ}は
\ruby{甚}{ひど}く
\ruby{{\換字{嫌}}}{きら}はれて
\ruby{居}{ゐ}た
\ruby{男}{をとこ}の、
%
\ruby{其}{そ}
% \原本頁{40-2}\改行%
の
\ruby[g]{親切}{しんせつ}が
\ruby{{\換字{通}}}{つう}じて
\ruby{思}{おも}ひ
\ruby{思}{おも}はれる
やうになるといふ
\ruby[g]{趣向}{ゆくたて}を
\ruby{書}{か}いた
\ruby{下}{くだ}
% \原本頁{40-3}\改行%
らない
ものでした。
』

\原本頁{40-4}%
『
ハヽア、
%
それぢやあ
\ruby{二筋{\換字{道}}}{ふた|すぢ|みち}といふのぢやあ
\ruby{無}{な}いか、
%
そんなら
\ruby[g]{何樣}{ど う }して
\ruby[<j||]{百}{ひやく}
\ruby[<j||]{年}{ねん}も
% \ruby{百年}{ひやく|ねん}も
\ruby{{\換字{前}}}{まへ}の
\ruby{{\換字{古}}}{ふる}いものだから、
%
いくら
\ruby{總傍訓}{そう|ふり|がな}が
あつたにしても、
%
こりやあ
お
\ruby{濱}{はま}ちやんには
\ruby{些}{ちつと}も
\ruby{{\換字{分}}}{わか}らなかつたろう。
%
\ruby[g]{私等}{わたしら}に
さへ
\ruby[g]{明瞭}{はつきり}とは
\ruby{解}{わか}らないところが
あるんだもの!。
』

\原本頁{40-8}%
『
ハヽヽ、
%
\ruby[g]{彼樣}{あ ん }な
\ruby{書}{もの}がまあ
\ruby[g]{左樣}{さ う }ですか\換字{子}エ。
%
\ruby[g]{成程}{なるほど}
いくら
\ruby{物}{もの}を
\ruby{知}{し}つて
\ruby{居}{ゐ}らしつても
\ruby[g]{御{\換字{若}}}{お わか}いから
\ruby[g]{何樣}{ど う }も
\ruby[g]{仕方}{し かた}が
ありません、
%
\ruby{御維新}{ご|いつ|しん}
\原本頁{40-10}\改行%
\ruby[g]{此方}{このかた}% ルビ調整(原本通り)
\ruby[g]{物事}{ものごと}が
\ruby[g]{全然}{すつかり}
\ruby{異}{ちが}つて
\ruby{參}{まゐ}りましたから\換字{子}。
%
さうすると
\ruby[<j||]{昔}{むかし}の
\ruby[<j||]{人}{にん }% 行末行頭の境界付近なので特例処置を施す
\ruby[<j||]{{\換字{情}}}{じやう}
\原本頁{40-11}\改行%
\ruby{本}{ぼん}の
\ruby{精}{よ}く
\ruby{{\換字{分}}}{わか}るのは、
%
\ruby[g]{此席}{こ ゝ }ぢやあ% 踊り字調整「〻(二の字点、揺すり点)に見えるが(ゝ)」
\ruby[<j>]{私}{わたくし}
ばつかりといふ
\ruby{譯}{わけ}ですか。
%
ハ
\改行% 校正作業の簡略化のため
ハヽ、% 原本では(「ハ」(改行)「ハヽ、」)
%
\ruby[g]{老夫}{おやぢ }も
たまにあ
\ruby[g]{貴下}{あなた }より
\ruby{{\換字{強}}}{つよ}い
ところが
ありますカ\換字{子}。
』

\原本頁{41-2}%
『
\ruby{詰}{つま}らない
\ruby[g]{自慢}{じ まん}を
\ruby{仕}{し}て!。
%
をかしな
\ruby[g]{御爺}{お ぢい}さん!。
%
どうせ
\ruby{御大名}{お|だい|みやう}の
\ruby[||j>]{行}{ぎやう}
\ruby[||j>]{列}{ れつ}
% \ruby{行列}{ぎやう|れつ}
なんぞ
\ruby{知}{し}つて
\ruby{居}{ゐ}るのも
\ruby[g]{御爺}{お ぢい}さん
ばかりよ。
』

\原本頁{41-4}%
『
ハヽヽ、
%
また
\ruby{此}{こ}の
\ruby[||j>]{老}{おぢい}
\ruby[||j>]{夫}{ さん}を
% \ruby{老夫}{おぢい|さん}を
やりこめるよ。
%
どうも
\ruby[g]{左樣}{さ う }
\ruby[g]{此頃}{このごろ}のやうに
\ruby[g]{威勢}{いきほひ}が
\ruby{{\換字{強}}}{つよ}くなつては
\ruby{敵}{かな}はないナ。
%
もう
\ruby[g]{談話}{はなし }も
\ruby{何}{なに}も
\ruby{仕}{し}て
やらないから
いゝ。% 踊り字調整「〻(二の字点、揺すり点)に見えるが(ゝ)」
』

\原本頁{41-7}%
『
いゝわ、% 踊り字調整「〻(二の字点、揺すり点)に見えるが(ゝ)」
%
あんな
\ruby[||j>]{{\換字{古}}}{むかし}
\ruby[||j>]{風}{ ふう}の
% \ruby{{\換字{古}}風}{むかし|ふう}の
\ruby{御談話}{お|はな|し}よりも、
%
\ruby{一昨日}{をと|ゝ|ひ}から% 踊り字調整「〻(二の字点、揺すり点)に見えるが(ゝ)」
\ruby{讀}{よ}んで
\ruby{居}{ゐ}る
\ruby{魯敏孫}{ろ|びん|そん}の
\ruby[g]{御話}{おはなし}の
\ruby{方}{はう}が
いくら
\ruby[g]{面白}{おもしろ}いか
\ruby{知}{し}れや
\ruby{仕}{し}ない。
』

\原本頁{41-9}%
『
\ruby{魯敏孫}{ろ|びん|そん}の
\ruby[g]{談話}{はなし }つて、
%
あの
\ruby{漂流記}{へう|りう|き}?。
』

\原本頁{41-10}%
『
えゝ% 踊り字調整「〻(二の字点、揺すり点)に見えるが(ゝ)」
\ruby[g]{左樣}{さ う }よ、
%
あの
\ruby{魯敏孫}{ろ|びん|そん}
\ruby{漂流記}{へう|りう|き}よ。
』

\原本頁{41-11}%
『
\ruby[g]{左樣}{さ う }!。
%
さうして
\ruby[g]{彼書}{あ れ }が
\ruby[g]{其樣}{そんな }に
お
\ruby{濱}{はま}ちやんには
\ruby[g]{面白}{おもしろ}いの?。
』

\原本頁{42-1}%
『
\ruby[g]{何故}{な ぜ }?。
%
\ruby[g]{先生}{せんせい}にやあ
\ruby[g]{彼書}{あ れ }が
\ruby[g]{面白}{おもしろ}くないの!。
%
\ruby[g]{先生}{せんせい}は
\ruby{魯敏孫}{ろ|びん|そん}を
\ruby{偉}{{\換字{𛀁}}ら}いとは
\ruby{思}{おも}はなくつて?。
%
\ruby{妾}{わたし}あ
\ruby[g]{眞實}{ほんと }に
\ruby{彼}{あ}の
\ruby{人}{ひと}が
\ruby{好}{すき}だわ。
%
\ruby{海}{うみ}の
\ruby{中}{なか}の
\ruby[g]{小島}{こ じま}に
\ruby[||j>]{唯}{たつた}
\ruby[||j>]{一人}{ ひと|り}で、
%
\ruby[g]{立派}{りつぱ }に
\ruby{生}{い}きて
\ruby{行}{ゆ}くなあ
\ruby{偉}{{\換字{𛀁}}ら}いぢや
ありませんか。
%
\ruby{妾}{わたし}あ
\ruby{彼}{あ}の
\ruby{書}{ほん}を
\ruby{讀}{よ}んで
\ruby{斯}{か}う
\ruby{思}{おも}つたわ。
』

\原本頁{42-5}%
『
おもしろい\換字{子}エ。
%
\ruby[g]{何樣}{ど ん }な
\ruby{事}{こと}を
\ruby{思}{おも}つたエ。
』

\原本頁{42-6}%
『
\ruby{妾}{わたし}も
\ruby[g]{何樣}{ど う }かした
\ruby{譯}{わけ}で
\ruby{其}{そ}の
\ruby{島}{しま}へ
\ruby{行}{い}つて\換字{子}、
%
さうして
\ruby{彼}{あ}の
\ruby{魯敏孫}{ろ|びん|そん}と
\ruby[g]{一處}{いつしよ}に
\ruby{棲}{す}んで、
%
\ruby{荒}{あら}い
\ruby{事}{こと}は
\ruby{魯敏孫}{ろ|びん|そん}に
\ruby{仕}{し}て
\ruby{貰}{もら}ふ
\ruby{代}{かは}り、
%
こま〳〵とした
\ruby{事}{こと}は
\ruby{妾}{わたし}が
\ruby{仕}{し}て
\ruby{{\換字{遣}}}{や}つて、
%
\ruby[g]{晝間}{ひるま }は
\ruby[||j>]{一}{いつ }
\ruby[||j>]{生}{しやう}
\ruby[||j>]{懸}{ けん}
\ruby[||j>]{命}{ めい}に
\ruby{働}{はたら}いても、
%
\ruby{夜}{よる}や
\ruby{雨}{あめ}の
\ruby{降}{ふ}つた
\ruby{靜}{しづか}な
\ruby{日}{ひ}には
お
\ruby{話}{はなし}なんぞ
\ruby{仕}{し}て
\ruby{{\換字{遊}}}{あそ}んで
\ruby{居}{ゐ}たらば、
%
ほんとに
\原本頁{42-10}\改行%
\ruby[g]{何樣}{ど ん }なにか
\ruby[g]{面白}{おもしろ}からうと
\ruby{思}{おも}つたのよ。
』

\原本頁{42-11}%
『
ハヽヽ。
%
また
\ruby{下}{くだ}らないことを
\ruby{云}{い}ひ
\ruby{出}{だ}したナ。
』

\原本頁{43-1}%
『
ハヽ、
%
こりやあ
\ruby[g]{面白}{おもしろ}い
\ruby[g]{面白}{おもしろ}い!。
%
ぢやあ
お
\ruby{濱}{はま}ちやんは
\ruby{魯敏孫}{ろ|びん|そん}の
\ruby[g]{夫人}{おくさん}に
なりたいと
いふんだ\換字{子}。
』

\原本頁{43-3}%
『
いやな
\ruby[g]{先生}{せんせい}\換字{子}エ、
%
\ruby[g]{夫人}{おくさん}だなんて!。
%
\ruby{妾}{わたし}あ
\ruby{他}{ひと}の
\ruby[g]{夫人}{おくさん}に
なつたり、
%
\ruby{他}{ひと}の
\makeatletter
\@ifundefined{デバッグ@ビルド}{%
  \ruby[g]{良人に}{ごていしゆ }
}{%
  \ruby[||j>]{良}{ごてい}
  \ruby[||j>]{人}{ しゆ}に
}%
\makeatother
% \ruby{良人}{ごてい|しゆ}に
なつたりする
\ruby{人}{ひと}は
\ruby[g]{大{\換字{嫌}}}{だいきら}ひだわ。
%
\ruby{妾}{わたし}あ
\ruby{唯}{たゞ}% 踊り字調整「〻(二の字点、揺すり点)に濁点に見えるが(ゞ)」
\ruby{魯敏孫}{ろ|びん|そん}の
\ruby[||j>]{朋}{おとも}
\ruby[||j>]{友}{ だち}
になつて
% \ruby{朋友}{おとも|だち}になつて
\ruby[g]{見度}{み た }いつて
\ruby{云}{い}つたのだわ。
』

\原本頁{43-6}%
『
ハヽ、
%
\ruby[g]{成程}{なるほど}、
%
\ruby{{\換字{分}}}{わか}つたよ、
%
\ruby[g]{面白}{おもしろ}いねエ。
%
つまり
お
\ruby{濱}{はま}ちやんは
\ruby[<j||]{女}{をんな}
\ruby{魯敏孫}{ろ|びん|そん}に
なりたいのだらう。
』

\原本頁{43-8}%
『
えゝ、% 踊り字調整「〻(二の字点、揺すり点)に見えるが(ゝ)」
%
\ruby[g]{左樣}{さ う }なのよ。
%
ほんとに
\ruby[g]{左樣}{さ う }なのよ。
%
\ruby[g]{眞靑}{まつさを}で
\ruby[g]{際涯}{は て }の
\ruby{無}{な}い
\ruby{大}{おほき}な
\ruby{洋}{うみ}の、
%
\ruby{塵}{ちり}も
\ruby{何}{なんに}も
\ruby{無}{な}い
\ruby[g]{奇麗}{き れい}な
\ruby{島}{しま}の
\ruby{中}{なか}で、
%
あの
\ruby[g]{男兒}{をとこ }らしい
\ruby{魯敏孫}{ろ|びん|そん}と、
%
たつた
\ruby[g]{二人}{ふたり }で
\ruby{働}{はたら}いて
\ruby{居}{ゐ}たら、
%
\ruby{妾}{わたし}あ
\ruby[g]{何樣}{ど ん }なに
\ruby{好}{い}い
\ruby{心持}{こゝろ|もち}% 踊り字調整「〻(二の字点、揺すり点)に見えるが(ゝ)」
だら
\原本頁{43-11}\改行%
うと
\ruby{思}{おも}つて
\ruby{居}{ゐ}るのよ。
』

\原本頁{44-1}%
『
これですもの、
%
どうも、
%
\ruby{呆}{あき}れて
\ruby[g]{仕舞}{し ま }ひます!。
%
\ruby[g]{此女}{こ れ }は
\ruby{取}{と}り
\ruby{{\換字{分}}}{わ}け
\ruby[g]{無茶}{む ちや}なので
ございましやうが、
%
\ruby{大}{だい}なり
\ruby{小}{せう}なり
\ruby[g]{明治}{めいぢ }の
\ruby[g]{生兒}{うまれ }は、
%
\ruby[g]{悉皆}{みんな }
\ruby[g]{斯樣}{か う }なので
ございましやうか、
%
まるで
\ruby{昔}{むかし}の
\ruby[||j>]{女}{むすめ}
\ruby[||j>]{兒}{ つこ}とは
% \ruby{女兒}{むすめ|つこ}とは
\ruby{異}{ちが}つて
\ruby{居}{を}ります。
%
\ruby{二筋{\換字{道}}}{ふた|すぢ|みち}の
\makeatletter
\@ifundefined{デバッグ@ビルド}{%
  \ruby[g]{話を}{はなし }
}{%
  \ruby{話}{はなし}を
}%
\makeatother
\ruby{仕}{し}て
\ruby{聞}{き}かせるのも
\ruby{異}{い}なものでしたが、
%
あんまり
\ruby[g]{何樣}{ど う }いふ
\ruby{譯}{わけ}だ
\ruby[g]{何樣}{ど う }いふ
\ruby{譯}{わけ}だと
\ruby{煩}{うる}さく
\ruby{聞}{き}かれましたから
\改行% 校正作業の簡略化のため
、
%
\原本頁{44-6}\改行%
ほんの
ざつとした
\ruby{筋}{すぢ}だけを
\ruby{話}{はな}して
\ruby{{\換字{遣}}}{や}りましたのに、
%
\ruby{碌}{ろく}にも
\ruby{{\換字{遂}}}{と}げては
\ruby{聞}{き}きませんで、
%
\ruby{詰}{つま}らないと
\ruby{一}{ひ}ト
\ruby{口}{くち}に
\ruby{云}{い}つて
\ruby[g]{仕舞}{し ま }ひましたのも、
%
\ruby[g]{一體}{いつたい}が
\ruby[g]{斯樣}{か う }いふ
\ruby[g]{調子}{てうし }ですから
\ruby[g]{無理}{む り }も
ありません。
%
\ruby{實}{じつ}に
\ruby{世}{よ}の
\ruby{中}{なか}は
\ruby{變}{かは}つて
まゐりました。
』

\原本頁{44-10}%
『
だつて
\ruby[g]{祖{\換字{父}}}{お ぢい}さん!。
%
\ruby{二筋{\換字{道}}}{あの|ほ|ん}の
\ruby[g]{御話}{おはなし}は、
%
\ruby{{\換字{嫌}}}{きら}ひな
\ruby{人}{ひと}が
\ruby{好}{すき}に
なるなんで、
%
\ruby[g]{馬鹿}{ば か }げて
\ruby{居}{ゐ}るんだもの!。
』

\原本頁{45-1}%
『
でも
\ruby{其}{それ}が
\ruby[||j>]{人}{にん}
\ruby[||j>]{{\換字{情}}}{じやう}つて
% \ruby{人{\換字{情}}}{にん|じやう}つて
\ruby{云}{い}ふものなんで、
%
まだ
\ruby[g]{中々}{なか〳〵}
\ruby[||j>]{汝}{おまへ}
\ruby[||j>]{{\換字{達}}}{ たち}にやあ
% \ruby{汝{\換字{達}}}{おまへ|たち}にやあ
\ruby{{\換字{分}}}{わか}らないんだよ。
』

\原本頁{45-3}%
『
そんな、
%
\ruby{{\換字{嫌}}}{きら}ひなものが
\ruby{好}{すき}になる
\ruby[||j>]{人}{にん}
\ruby[||j>]{{\換字{情}}}{じやう}なんて、
% \ruby{人{\換字{情}}}{にん|じやう}なんて、
%
そりやあ
お
\ruby[<j||]{行}{ぎやう}
\ruby[<j||]{列}{ れつ}の
% \ruby{行列}{ぎやう|れつ}の
\ruby[g]{時{\換字{分}}}{じ ぶん}の
\ruby[||j>]{人}{にん}
\ruby[||j>]{{\換字{情}}}{じやう}ぢやなくつて?。
% \ruby{人{\換字{情}}}{にん|じやう}ぢやなくつて?。
』

\原本頁{45-5}%
『
\ruby{生意氣}{なま|い|き}な!。
%
\ruby{何}{なに}が
\ruby[g]{小兒}{こ ども}の
\ruby{汝}{おまへ}
なんぞに
\ruby{未}{ま}だ
\ruby{{\換字{分}}}{わか}るものか!。
』

\原本頁{45-6}%
『
だつて
\ruby[g]{幾歳}{いくつ }に
なつたつて、
%
\ruby{妾}{わたし}にや
\ruby{{\換字{分}}}{わか}らないわ。
%
\ruby{妾}{わたし}や
\ruby[g]{幾歳}{いくつ }に
なつたつて、
%
\ruby[g]{屹度}{きつと }
お
\ruby[||j>]{澤}{さは}% 行末行頭の境界付近なので特例処置を施す
\ruby[||j>]{婆}{ばゝあ}は% 踊り字調整「〻(二の字点、揺すり点)に見えるが(ゝ)」
\ruby{{\換字{嫌}}}{きらひ}で
\ruby[g]{先生}{せんせい}は
\ruby{好}{すき}だわ。
%
\ruby[g]{先生}{せんせい}が
\ruby{{\換字{嫌}}}{きらひ}で
\makeatletter
\@ifundefined{デバッグ@ビルド}{%
  お
  \ruby[||j]{澤}{さは}
  \ruby[||j]{婆}{ばゝあ}が% 踊り字調整「〻(二の字点、揺すり点)に見えるが(ゝ)」
}{%
  お
  \ruby[<j||]{澤}{さは }
  \ruby[<j||]{婆}{ばゝあ}が% 踊り字調整「〻(二の字点、揺すり点)に見えるが(ゝ)」
}%
\makeatother
\ruby{好}{す}きにはなりやあ
\ruby{仕}{し}ないわ。
』
