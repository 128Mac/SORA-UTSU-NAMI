\Entry{其七}

『いや
\ruby{古}{ふる}い
\ruby{本}{ほん}が
\ruby{新}{あたら}しくなつて
\ruby{澤山出}{たく|さん|で}るからね。
\ruby{左樣}{さ|う}して
\ruby{其}{そ}の
\ruby{書}{ほん}は
\ruby{何}{なん}と
\ruby{云}{い}ふ
\ruby{書}{ほん}だつたの?。
』

『ナアニ、
\ruby{私}{わたし}なんぞが
\ruby{面皰}{にき|び}の
\ruby{出}{で}て
\ruby{居}{ゐ}た
\ruby{二才}{に|さい}の
\ruby{時{\換字{分}}貸本屋}{じ|ぶん|かし|ほん|や}で
\ruby{借}{か}りて
\ruby{讀}{よ}んだことのある
\ruby{人{\換字{情}}本}{にん|じやう|ぼん}で、
\ruby{初}{はじめ}は
\ruby{甚}{ひど}く
\ruby{{\換字{嫌}}}{きら}はれて
\ruby{居}{ゐ}た
\ruby{男}{をとこ}の、
\ruby{其}{そ}の
\ruby{親切}{しん|せつ}が
\ruby{通}{つう}じて
\ruby{思}{おも}ひ
\ruby{思}{おも}はれるやうになるといふ
\ruby{趣向}{ゆく|たて}を
\ruby{書}{か}いた
\ruby{下}{くだ}らないものでした。
』

『ハヽア、それぢやあ
\ruby{二筋{\換字{道}}}{ふた|すぢ|みち}といふのぢやあ
\ruby{無}{な}いか、そんなら
\ruby{何樣}{ど|う}して
\ruby{百年}{ひやく|ねん}も
\ruby{前}{まへ}の
\ruby{古}{ふる}いものだから、いくら
\ruby{{\換字{総}}傍訓}{そう|ふり|がな}があつたにしても、こりやあ
お
\ruby{濱}{はま}ちやんには
\ruby{些}{ちつと}も
\ruby{{\換字{分}}}{わか}らなかつたろう。
\ruby{私等}{わたし|ら}にさへ
\ruby{明瞭}{はつ|きり}とは
\ruby{解}{わか}らないところがあるんだもの!。
』

『ハヽヽ、
\ruby{彼樣}{あ|ん}な
\ruby{書}{もの}がまあ
\ruby{左樣}{さ|う}ですか\換字{子}エ。
\ruby{成程}{なる|ほど}いくら
\ruby{物}{もの}を
\ruby{知}{し}つて
\ruby{居}{ゐ}らしつても
\ruby{御若}{お|わか}いから
\ruby{何樣}{ど|う}も
\ruby{仕方}{し|かた}がありません、
\ruby{御維新此方物事}{ご|いつ|しん|この|かた|もの|ごと}が
\ruby{全然}{すつ|かり}
\ruby{異}{ちが}つて
\ruby{參}{まゐ}りましたから\換字{子}。
さうすると
\ruby{昔}{むかし}の
\ruby{人{\換字{情}}本}{にん|じやう|ぼん}の
\ruby{精}{よ}く
\ruby{{\換字{分}}}{わか}るのは、
\ruby{此席}{こ|ゝ}ぢやあ
\ruby{私}{わたし}ばつかりといふ
\ruby{譯}{わけ}ですか。
ハヽヽ、
\ruby{老夫}{おや|ぢ}もたまにあ
\ruby{貴下}{あな|た}より
\ruby{{\換字{強}}}{つよ}いところがありますカ\換字{子}。
』

『
\ruby{詰}{つ}まらない
\ruby{自慢}{じ|まん}を
\ruby{仕}{し}て!。
をかしな
\ruby{御爺}{お|ぢい}さん!。
どうせ
\ruby{御大名}{お|だい|みやう}の
\ruby{行列}{ぎやう|れつ}なんぞ
\ruby{知}{し}つて
\ruby{居}{ゐ}るのも
\ruby{御爺}{お|ぢい}さんばかりよ。
』

『ハヽヽ、また
\ruby{此}{こ}の
\ruby{老夫}{おぢい|さん}をやりこめるよ。
どうも
\ruby{左樣此頃}{さ|う|この|ごろ}のやうに
\ruby{威勢}{いき|ほひ}が
\ruby{{\換字{強}}}{つよ}くなつては
\ruby{敵}{かな}はないナ。
もう
\ruby{談話}{はな|し}も
\ruby{何}{なに}も
\ruby{仕}{し}てやらないからいゝ。
』

『いゝわ、あんな
\ruby{昔風}{むかし|ふう}の
\ruby{御談話}{お|はな|し}よりも、
\ruby{一昨日}{を|とゝ|ひ}から
\ruby{讀}{よ}んで
\ruby{居}{ゐ}る
\ruby{魯敏孫}{ろ|びん|そん}の
\ruby{御話}{お|はなし}の
\ruby{方}{はう}がいくら
\ruby{面白}{おも|しろ}いか
\ruby{知}{し}れや
\ruby{仕}{し}ない。
』

『
\ruby{魯敏孫}{ろ|びん|そん}の
\ruby{談話}{はな|し}つて、あの
\ruby{漂流記}{へう|りう|き}?。
』

『えゝ
\ruby{左樣}{さ|う}よ、あの
\ruby{魯敏孫}{ろ|びん|そん}
\ruby{漂流記}{へう|りう|き}よ。
』

『
\ruby{左樣}{さ|う}!。
さうして
\ruby{彼書}{あ|れ}が
\ruby{其樣}{そん|な}に
お
\ruby{濱}{はま}ちやんには
\ruby{面白}{おも|しろ}いの?。
』

『
\ruby{何故}{な|ぜ}?。
\ruby{先生}{せん|せい}にやあ
\ruby{彼書}{あ|れ}が
\ruby{面白}{おも|しろ}くないの!。
\ruby{先生}{せん|せい}は
\ruby{魯敏孫}{ろ|びん|そん}を
\ruby{偉}{えら}いとは
\ruby{思}{おも}はなくつて?。
\ruby{妾}{わたし}あ
\ruby{眞實}{ほん|と}に
\ruby{彼}{あ}の
\ruby{人}{ひと}が
\ruby{好}{す}きだわ。
\ruby{海}{うみ}の
\ruby{中}{なか}の
\ruby{小島}{こ|じま}に
\ruby{唯一人}{たつた|ひと|り}で、
\ruby{立派}{りつ|ぱ}に
\ruby{生}{い}きて
\ruby{行}{ゆ}くなあ
\ruby{偉}{えら}いぢやあありませんか。
\ruby{妾}{わたし}あ
\ruby{彼}{あ}の
\ruby{書}{ほん}を
\ruby{讀}{よ}んで
\ruby{斯}{か}う
\ruby{思}{おも}つたわ。
』

『おもしろい\換字{子}エ。
\ruby{何樣}{ど|ん}な
\ruby{事}{こと}を
\ruby{思}{おも}つたエ。
』

『
\ruby{妾}{わたし}も
\ruby{何樣}{ど|う}かした
\ruby{譯}{わけ}で
\ruby{其}{そ}の
\ruby{島}{しま}へ
\ruby{行}{い}つて\換字{子}、さうして
\ruby{彼}{あ}の
\ruby{魯敏孫}{ろ|びん|そん}と
\ruby{一處}{いつ|しよ}に
\ruby{棲}{す}んで、
\ruby{荒}{あら}い
\ruby{事}{こと}は
\ruby{魯敏孫}{ろ|びん|そん}に
\ruby{仕}{し}て
\ruby{貰}{もら}ふ
\ruby{代}{かは}り、こまこまとした
\ruby{事}{こと}は
\ruby{妾}{わたし}が
\ruby{仕}{し}て
\ruby{{\換字{遣}}}{や}つて、
\ruby{晝間}{ひる|ま}は
\ruby{一生懸命}{いつ|しやう|けん|めい}に
\ruby{働}{はたら}いても、
\ruby{夜}{よる}や
\ruby{雨}{あめ}の
\ruby{降}{ふ}つた
\ruby{靜}{しづか}かな
\ruby{日}{ひ}には
お
\ruby{話}{はなし}なんぞ
\ruby{仕}{し}て
\ruby{{\換字{遊}}}{あそ}んで
\ruby{居}{ゐ}たらば、ほんとに
\ruby{何樣}{ど|ん}なにか
\ruby{面白}{おも|しろ}からうかと
\ruby{思}{おも}つたのよ。
』

『ハヽヽ。
また
\ruby{下}{くだ}らないことを
\ruby{云}{い}ひ
\ruby{出}{だ}したナ。
』

『ハヽ、こりやあ
\ruby{面白}{おも|しろ}い
\ruby{面白}{おも|しろ}い!。
ぢやあお
\ruby{濱}{はま}ちやんは
\ruby{魯敏孫}{ろ|びん|そん}の
\ruby{夫人}{おく|さん}になりたいといふんだ\換字{子}。
』

『いやな
\ruby{先生}{せん|せい}\換字{子}エ。
\ruby{夫人}{おく|さん}だなんて!。
\ruby{妾}{わたし}あ
\ruby{他}{ひと}の
\ruby{夫人}{おく|さん}になつたり、
\ruby{他}{ひと}の
\ruby{良人}{ごてい|しゆ}になつたりする
\ruby{人}{ひと}は
\ruby{大{\換字{嫌}}}{だい|きら}ひだわ。
\ruby{妾}{わたし}あ
\ruby{唯}{たゞ}
\ruby{魯敏孫}{ろ|びん|そん}の
\ruby{朋友}{おとも|だち}になつて
\ruby{見度}{み|た}いつて
\ruby{云}{い}つたのだわ。
』

『ハヽ、
\ruby{成程}{なる|ほど}、
\ruby{{\換字{分}}}{わか}つたよ。
\ruby{面白}{おも|しろ}いねエ。
つまりお
\ruby{濱}{はま}ちやんは
\ruby{女}{をんな}
\ruby{魯敏孫}{ろ|びん|そん}になりたいのだらう。
』

『えゝ、
\ruby{左樣}{さ|う}なのよ。
ほんとに
\ruby{左樣}{さ|う}なのよ。
\ruby{眞靑}{まつ|さを}で
\ruby{際涯}{は|て}の
\ruby{無}{な}い
\ruby{大}{おほき}な
\ruby{洋}{うみ}の、
\ruby{塵}{ちり}も
\ruby{何}{なんに}も
\ruby{無}{な}い
\ruby{奇麗}{き|れい}な
\ruby{島}{しま}の
\ruby{中}{なか}で、あの
\ruby{男兒}{をと|こ}らしい
\ruby{魯敏孫}{ろ|びん|そん}と、たつた
\ruby{二人}{ふ|たり}で
\ruby{働}{はたら}いて
\ruby{居}{ゐ}たら、
\ruby{妾}{わたし}あ
\ruby{何樣}{ど|ん}なに
\ruby{好}{い}い
\ruby{心持}{こゝろ|もち}だらうと
\ruby{思}{おも}つて
\ruby{居}{ゐ}るのよ。
』

『これですもの、どうも、
\ruby{呆}{あき}れて
\ruby{仕舞}{し|ま}ひます!。
\ruby{此女}{こ|れ}は
\ruby{取}{と}り
\ruby{{\換字{分}}}{わ}け
\ruby{無茶}{む|ちや}なのでございましやうが、
\ruby{大}{だい}なり
\ruby{小}{しよう}なり
\ruby{明治}{めい|じ}の
\ruby{生兒}{うま|れ}は、
\ruby{悉皆}{みん|な}
\ruby{斯樣}{か|う}なのでございましやうか、まるで
\ruby{昔}{むかし}の
\ruby{女兒}{むすめ|つこ}とは
\ruby{異}{ちが}つて
\ruby{居}{を}ります。
\ruby{二筋{\換字{道}}}{ふた|すぢ|みち}の
\ruby{話}{はなし}を
\ruby{仕}{し}て
\ruby{聞}{き}かせるのも
\ruby{異}{い}なものでしたが、あんまり
\ruby{何樣}{ど|う}いふ
\ruby{譯}{わけ}だ
\ruby{何樣}{ど|う}いふ
\ruby{譯}{わけ}だと
\ruby{煩}{うるさ}く
\ruby{聞}{き}かれましたから、ほんのざつとした
\ruby{筋}{すぢ}だけを
\ruby{話}{はな}して
\ruby{{\換字{遣}}}{や}りましたのに、
\ruby{碌}{ろく}にも
\ruby{{\換字{遂}}}{と}げては
\ruby{聞}{き}きませんので、
\ruby{詰}{つま}らないと
\ruby{一}{ひ}ㇳ
\ruby{口}{くち}に
\ruby{云}{い}つて
\ruby{仕舞}{し|ま}ひましたのも、
\ruby{一體}{いつ|たい}が
\ruby{斯樣}{か|う}いふ
\ruby{調子}{てう|し}ですから
\ruby{無理}{む|り}もありません。
\ruby{實}{じつ}に
\ruby{世}{よ}の
\ruby{中}{なか}は
\ruby{變}{かは}つてまゐりました。
』

『だつて
\ruby{祖父}{お|ぢい}さん!。
\ruby{二筋{\換字{道}}}{あの|ほ|ん}の
\ruby{御話}{お|はなし}は、
\ruby{{\換字{嫌}}}{きら}ひな
\ruby{人}{ひと}が
\ruby{好}{すき}になるなんで、
\ruby{馬鹿}{ば|か}げて
\ruby{居}{ゐ}るんだもの!。
』

『でも
\ruby{其}{それ}が
\ruby{人{\換字{情}}}{にん|じやう}つて
\ruby{云}{い}ふものなんで、まだ
\ruby{中々汝{\換字{達}}}{なか|〳〵|おまへ|たち}にやあ
\ruby{{\換字{分}}}{わか}らないんだよ。
』

『そんな、
\ruby{{\換字{嫌}}}{きら}ひなものが
\ruby{好}{すき}になる
\ruby{人{\換字{情}}}{にん|じやう}なんて、そりやあ
お
\ruby{行列}{ぎやう|れつ}の
\ruby{時{\換字{分}}}{じ|ぶん}の
\ruby{人{\換字{情}}}{にん|じやう}ぢやなくつて?。
』

『
\ruby{生意氣}{なま|い|き}な!。
\ruby{何}{なに}が
\ruby{小児}{こ|ども}の
\ruby{汝}{おまへ}なんぞに
\ruby{未}{ま}だ
\ruby{{\換字{分}}}{わか}るものか!。
』

『だつて
\ruby{幾歳}{いく|つ}になつたつて、
\ruby{妾}{わたし}にや
\ruby{{\換字{分}}}{わか}らないわ。
\ruby{妾}{わたし}や
\ruby{幾歳}{いく|つ}になつたつて、
\ruby{屹度}{きつ|と}
お
\ruby{澤婆}{さは|ばゞあ}は
\ruby{{\換字{嫌}}}{きらひ}で
\ruby{先生}{せん|せい}は
\ruby{好}{す}きだわ。
\ruby{先生}{せん|せい}が
\ruby{{\換字{嫌}}}{きらひ}で
お
\ruby{澤婆}{さは|ばゞあ}が
\ruby{好}{す}きにはなりやあ
\ruby{仕}{し}ないわ。
』

