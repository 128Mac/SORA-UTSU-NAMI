\Entry{其八}

% メモ 校正終了 2024-04-17
\原本頁{45-10}%
『ハヽヽ。
%
\ruby{然樣}{さ|う}ムキになつて
\ruby{老夫}{おぢい|さん}に
\ruby{食}{く}つて
\ruby{掛}{か〻}ることは% 原本通り「〻(二の字点、揺すり点)」
\ruby{無}{な}いぢやあ
\ruby{無}{な}いか。
%
もう
\ruby{可}{い}い、
%
もう
\ruby{可}{い}い。
%
とても
\ruby{老夫}{おぢい|さん}は
\ruby{汝}{おまへ}にやあ
\ruby{敵}{かな}はないよ。
%
しかし
\ruby{汝}{おまへ}が
もう
\ruby{二三年}{に|さん|ねん}も
\ruby{年}{とし}を
とつて、
%
\ruby{物事}{もの|ごと}が
\ruby{善}{よ}く
\ruby{解}{わか}つて
\ruby{來}{く}ると、
%
お
\ruby{澤}{さは}
\ruby{婆}{ばゞあ}だつて% TODO 原本の「二の字点、揺すり点」に濁点のグリフが見つからないので「ゞ」
\ruby{其樣}{そん|な}に
\ruby{憎}{にく}くは
\ruby{無}{な}く
\ruby{思}{おも}ふやうに
なるかも
\原本頁{46-4}\改行%
\ruby{知}{し}れないよ。
%
\ruby{先生}{せん|せい}だつて
\ruby{{\換字{過}}日}{こな|ひだ}までとは
\ruby{異}{ちが}つて、
%
\ruby{今}{いま}ぢやあ
もう
お
\ruby{澤}{さは}
\ruby{婆}{ばゞあ}を% TODO 原本の「二の字点、揺すり点」に濁点のグリフが見つからないので「ゞ」
\ruby{憎}{にく}いと
ばかり
\ruby{思}{おも}つては
\ruby{居}{ゐ}らつしやらない
やうだもの。
%
まあ
\ruby{何}{なん}とでも
\ruby{云}{い}つて
\ruby{居}{ゐ}るが
\ruby{宜}{い}い、
%
\ruby{人{\換字{情}}}{にん|じやう}と
いふものは
\ruby{年齡}{と|し}さへ
\ruby{老}{と}りやあ
\ruby{解}{わか}る
\ruby{事}{こと}だから。
』

\原本頁{46-8}%
\ruby{我}{わ}が
\ruby{此}{この}
\ruby{上}{うへ}
\ruby{無}{な}く
\ruby{好}{す}きなる
\ruby{其}{その}
\ruby{人}{ひと}の、
%
\ruby{我}{わ}が
\ruby{此}{この}
\ruby{上}{うへ}
\ruby{無}{な}く
\ruby{{\換字{嫌}}}{きら}へる
\ruby{其}{その}
\ruby{婆}{ばゞ}を% TODO 原本の「二の字点、揺すり点」に濁点のグリフが見つからないので「ゞ」
\ruby{憎}{にく}しと
のみは
\ruby{思}{おも}ひ
\ruby{居}{を}らじと
\ruby{云}{い}へるを
\ruby{聞}{き}きて、
%
お
\ruby{濱}{はま}は
\ruby{且}{かつ}は
\ruby{驚}{おどろ}き、
%
\ruby{且}{かつ}は
\ruby{訝}{いぶか}り、
%
\ruby{疑惑}{うた|がひ}の
\ruby{眉}{まゆ}を
\ruby{可憐}{か|はゆ}らしく
\ruby{顰}{ひそ}め
\ruby{頸}{くび}を
\ruby{枉}{ま}げて
\ruby[g]{水野}{みづの}の
\ruby{面}{おもて}を
\ruby{覗}{のぞ}き
\ruby{{\換字{込}}}{こ}みつ〻、% 原本通り「〻(二の字点、揺すり点)」

\原本頁{47-1}%
『ほんとなの?、
%
\ruby{先生}{せん|せい}。
%
\ruby{先生}{せん|せい}
あんな
\ruby{意地惡}{い|ぢ|わる}な
\ruby{惡}{にく}らしい
\ruby{老婆}{おばあ|さん}が
\ruby{好}{すき}になつたの?。
』

\原本頁{47-3}%
と、
%
さも〳〵
\ruby{然}{しか}らずといふ
\ruby{答}{こたへ}を
\ruby{聞}{き}きて、
%
\ruby{改}{あらた}めて
\ruby{{\換字{又}}}{また}
\ruby{我}{わ}が
\ruby{祖{\換字{父}}}{ぢ|〻}に% 原本通り「〻(二の字点、揺すり点)」
\ruby{對}{むか}ひて
\ruby{{\換字{勝}}}{か}ち
\ruby{誇}{ほこ}りたげに
\ruby{{\換字{尋}}}{たづ}ねたり。

\原本頁{47-5}%
\ruby[g]{水野}{みづの}は
\ruby{先刻}{さつ|き}より
\ruby{小刀}{こ|がたな}を
もて
\ruby{心}{こ〻ろ}% 原本通り「〻(二の字点、揺すり点)」
\ruby{長}{なが}く
\ruby{叮嚀}{てい|ねい}に
\ruby{栗}{くり}を
\ruby{剝}{む}きつ〻、% 原本通り「〻(二の字点、揺すり点)」
%
\ruby{既}{すで}に
\ruby{世}{よ}に
\ruby{老}{お}いたる
\ruby{{\換字{翁}}}{おきな}と
\ruby{未}{ま}だ
\ruby{世}{よ}を
\ruby{知}{し}らぬ
\ruby{少女}{をと|め}との、
%
\ruby{彼方}{かな|た}は
\ruby{經驗}{おぼ|{\換字{𛀁}}}に
\ruby{頼}{よ}り
\ruby[g]{此方}{こなた}は
\ruby{{\換字{空}}想}{おも|ひ}に
\ruby{任}{まか}せて、
%
\ruby{相}{あひ}
\ruby{和}{わ}せぬ
\ruby{談}{はなし}を
\ruby{{\換字{交}}}{まじ}ふるをば、
%
おのづから
\ruby{催}{もよほ}さる〻% 原本通り「〻(二の字点、揺すり点)」
\ruby{微笑}{ほ〻|ゑみ}の% 原本通り「〻(二の字点、揺すり点)」
\ruby{間}{うち}に
\ruby{聞}{き}き
\ruby{居}{ゐ}たりしが、
%
\ruby{恰}{あたか}も% 恰も「あ(た)かも」
\ruby{此時}{この|とき}
\ruby{奇麗}{〻|れい}に% 原本通り「〻(二の字点、揺すり点)」
\ruby{剝}{む}き
\ruby{{\換字{終}}}{をは}りし
\ruby{一箇}{ひと|つ}の
\ruby{栗}{くり}を、
%
そつと
お
\ruby{濱}{はま}が
\ruby{掌}{て}の
\ruby{上}{うへ}に
\ruby{載}{の}せてやりつ、

\原本頁{47-10}%
『なにも
\ruby{好}{すき}に
なつたといふ
\ruby{事}{こと}は
\ruby{無}{な}いのだ
けれども、
%
そりやあ
\ruby{憎}{にく}いと
ばかりも
\ruby{思}{おも}つては
\ruby{居}{ゐ}ない。
%
\ruby{考}{かんが}へて
\ruby{見}{み}ると
\ruby{今}{いま}では
\ruby{憫然}{かあ|いさう}で% 「憫然 か(あ)いさう」
ならないやうな
\ruby{氣}{き}さへする
\ruby{位}{くらゐ}だから。
』

\原本頁{48-2}%
と
\ruby{優}{やさ}しく
\ruby{答}{こた}へて、

\原本頁{48-3}%
『お
\ruby{濱}{はま}ちやん
だつて
\ruby{今}{いま}に
\ruby{彼}{あ}の
お
\ruby{澤}{さは}の
\ruby{腹}{おなか}の
\ruby{中}{なか}が
\ruby{合點}{が|てん}が
\ruby{行}{ゆ}けば、
%
\ruby{彼婆}{あ|れ}を
\ruby{憎}{にく}らしいとは
\ruby{思}{おも}はないやうに
なるかも
\ruby{知}{し}れないよ。
』

\原本頁{48-5}%
と
\ruby{語}{ことば}を
\ruby{足}{た}したり。

\原本頁{48-6}%
\ruby[g]{水野}{みづの}が
\ruby{此}{この}
\ruby{語}{ことば}は
\ruby{如何}{い|か}ばかり
\ruby{思}{おもひ}の
\ruby{外}{ほか}なりけん、
%
お
\ruby{濱}{はま}は
\ruby{呆}{あき}れたる
\ruby{眼}{め}を
\ruby{{\換字{睜}}}{みは}つて
\ruby{默}{だま}りけるが、
%
\ruby[g]{吉右衛門}{きちゑもん}は
\ruby{待設}{まち|まう}けしやうに
\ruby{言}{ことば}を
\ruby{挿}{さしはさ}みぬ。

\原本頁{48-8}%
『それ
\ruby{御覧}{ご|らん}、
%
\ruby{老夫}{おぢい|さん}の
\ruby{言}{い}ふ
\ruby{事}{こと}も
\ruby{嘘}{うそ}ぢやあ
\ruby{有}{あ}るまい。
%
\ruby{好}{す}きなものが
\ruby{{\換字{嫌}}}{きらひ}になつたりもすれば
\ruby{{\換字{嫌}}}{きらひ}なものが
\ruby{好}{す}きになつたりもする、
%
それは
\ruby[<j|]{皆}{みんな}
\ruby{人{\換字{情}}}{にん|じやう}と
いふものが
\ruby{爲}{さ}せるんで、
%
まだ
\ruby{中々}{なか|〳〵}
\ruby{汝{\換字{達}}}{おまへ|たち}にやあ
\ruby{{\換字{分}}}{わか}らない
\ruby{事}{こと}なんだよ。
』

\原本頁{49-1}%
お
\ruby{濱}{はま}は
\ruby{祖{\換字{父}}}{ぢ|ゞ}が% 「ぢゞ」のはずだが、原本通り「〻(二の字点、揺すり点)」
\ruby{言}{ことば}を
\ruby{聞}{き}きも
せずして、
%
\ruby{今}{いま}
\ruby{貰}{もら}ひし
\ruby{栗}{くり}を
\ruby{無邪氣}{む|じや|き}に
\ruby{食}{た}べながら、
%
\ruby{何事}{なに|ごと}を
\ruby{思}{おも}ひ
\ruby{{\換字{廻}}}{めぐ}らせるならん、
%
あらぬ
\ruby{方}{かた}に
\ruby{眼}{め}を
\ruby{{\換字{留}}}{とゞ}めて% TODO 原本の「二の字点、揺すり点」に濁点のグリフが見つからないので「ゞ」
\ruby{一寸}{ちよ|つと}
\ruby{考}{かんが}へ
\ruby{居}{ゐ}れば、
%
\ruby[g]{水野}{みづの}は
\ruby{{\換字{又}}}{また}
\ruby{樂}{たの}しげに
\ruby{栗}{くり}を
\ruby{剝}{む}き
\ruby{居}{を}り、
%
\ruby[g]{吉右衛門}{きちゑもん}は
\ruby{{\換字{煙}}草}{たば|こ}を
\ruby{深}{ふか}く
\ruby{吸}{す}ひて
\ruby{{\換字{緩}}}{ゆる}やかに
\ruby{其}{そ}の
\ruby{烟}{けむり}を
\ruby{噴}{ふ}き
\ruby{出}{だ}し
\ruby{居}{を}れり。

\原本頁{49-5}%
\ruby{靜寂}{しづ|か}なりしは
たゞ% TODO 原本の「二の字点、揺すり点」に濁点のグリフが見つからないので「ゞ」
\ruby{一霎時}{し|ば|し}なりき。
%
お
\ruby{濱}{はま}は
\ruby{何}{なに}を
\ruby{思}{おも}ひ
\ruby{得}{{\換字{𛀁}}}しにや
\ruby{忽}{たちま}ち
\ruby{嬉}{うれ}しげなる
\ruby{聲}{こゑ}に
\ruby{淋}{さび}しさを
\ruby{破}{やぶ}つて、

\原本頁{49-7}%
『アヽ
\ruby[<j|]{妾}{わたし}
\ruby{{\換字{分}}}{わか}つてよ、
%
\ruby[<j|]{妾}{わたし}
\ruby{{\換字{分}}}{わか}つてよ。
%
\ruby[g]{五十子}{いそこ}さんが
\ruby{今}{いま}に
\ruby{快}{よ}くなるとネエ、
%
\ruby{屹度}{きつ|と}
\ruby{大變}{たい|へん}に
\ruby{先生}{せん|せい}が
\ruby{好}{すき}になるんでしやう、
ホヽヽ、
%
それが
\ruby{人{\換字{情}}}{にん|じやう}つて
\ruby{云}{い}ふものなんでしやう。
%
\ruby{左樣}{さ|う}ぢやあ
\ruby{無}{な}くつて?、
%
え、
%
\ruby{祖{\換字{父}}}{お|ぢい}さん!。
%
\ruby[g]{五十子}{いそこ}さんが
\ruby{先生}{せん|せい}を
\ruby{大好}{だい|すき}になる、
%
アヽ
\ruby{左樣}{さ|う}なると
\ruby{好}{い}いわ、
%
\ruby{早}{はや}く
\ruby{左樣}{さ|う}なると、
%
\ruby[<j|]{妾}{わたし}
\ruby[g]{五十子}{いそこ}さんを
\ruby{姉}{ね{{\換字{𛀁}}}}さんに
\ruby{爲}{し}つちまふから、
%
\ruby{先生}{せん|せい}が
\ruby{兄}{にい}さんで、
%
\ruby[g]{五十子}{いそこ}さんが
\ruby{姉}{ね{{\換字{𛀁}}}}さんで、
%
さうして
\ruby{妾}{わたし}が
\ruby{其}{その}
\ruby{傍}{そば}に
\ruby{貼}{つ}いて
\ruby{居}{ゐ}るんなら、
%
ほんとに
\ruby{何樣}{どん|な}に
\ruby{嬉}{うれ}しいか
\ruby{知}{し}れや
\ruby{仕}{し}ないわ。
%
\ruby{左樣}{さ|う}なれば
\ruby{妾}{わたし}あ
\ruby{魯敏孫}{ろ|びん|そん}の
\ruby{朋友}{おとも|だち}になるのは
\ruby{廃}{よ}して
\ruby{{\換字{終}}}{しま}ふは。
』

\原本頁{50-4}%
と、
%
\ruby{僞}{いつはり}ならず
\ruby{悅}{よろこ}びて
\ruby{云}{い}ひ
\ruby{出}{だ}したる、
%
\ruby{面}{おもて}は
\ruby{晴}{は}れやかにして
\ruby{月}{つき}は
\ruby{雲}{くも}なく、
%
\ruby{{\換字{情}}}{こ〻ろ}は% 原本通り「〻(二の字点、揺すり点)」
\ruby{優}{やさ}しくして
\ruby{花}{はな}に
\ruby{露}{つゆ}あり。

\原本頁{50-6}%
されど
お
\ruby{濱}{はま}は
\ruby{{\換字{又}}}{また}
たゞちに、% TODO 原本の「二の字点、揺すり点」に濁点のグリフが見つからないので「ゞ」

\原本頁{50-7}%
『だけれど、
』

\原本頁{50-8}%
と
\ruby{云}{い}ひさして
\ruby{祖{\換字{父}}}{ぢ|〻}の% 「ぢゞ」のはずだが、原本通り「〻(二の字点、揺すり点)」
\ruby{面}{おもて}を
\ruby{見}{み}たり。
%
\ruby[g]{水野}{みづの}は
お
\ruby{濱}{はま}の
\ruby{言}{ことば}を
\ruby{何}{なに}と
\ruby{聞}{き}きしや、
%
\ruby{何氣}{なに|げ}
\ruby{無}{な}き
\ruby{風}{ふう}に
\ruby{身}{み}をも
\ruby{動}{うご}かさず、
%
ひたすらに
\ruby{栗}{くり}を
\ruby{剝}{む}き
\ruby{居}{ゐ}たり。
