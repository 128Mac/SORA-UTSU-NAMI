\Entry{其八}

% メモ 校正終了 2024-04-17 2024-05-29
\原本頁{45-10}%
『
ハヽヽ。
%
\ruby[g]{然樣}{さ う }ムキになつて
\ruby[||j>]{老}{おぢい}
\ruby[||j>]{夫}{ さん}に
% \ruby{老夫}{おぢい|さん}に
\ruby{食}{く}つて
\ruby{掛}{かゝ}ることは% 踊り字調整「〻(二の字点、揺すり点)に見えるが(ゝ)」
\ruby{無}{な}いぢやあ
\ruby{無}{な}いか。
%
もう
\ruby{可}{い}い、
%
もう
\ruby{可}{い}い。
%
とても
\ruby[||j>]{老}{おぢい}
\ruby[||j>]{夫}{ さん}は
% \ruby{老夫}{おぢい|さん}は
\ruby{汝}{おまへ}にやあ
\ruby{敵}{かな}はないよ!。
%
しかし
\ruby{汝}{おまへ}が
もう
\ruby{二三年}{に|さん|ねん}も
\ruby{年}{とし}を
とつて、
%
\ruby[g]{物事}{ものごと}が
\ruby{善}{よ}く
\ruby{解}{わか}つて
\ruby{來}{く}ると、
%
お
\ruby{澤}{さは}
\makeatletter
\@ifundefined{デバッグ@ビルド}{%
  \ruby[g]{婆だ}{ばゞあ }つて% 踊り字調整「〻(二の字点、揺すり点)に濁点に見えるが(ゞ)」
}{%
  \ruby{婆}{ばゞあ}だつて% 踊り字調整「〻(二の字点、揺すり点)に濁点に見えるが(ゞ)」
}%
\makeatother
\ruby[g]{其樣}{そんな }に
\ruby{憎}{にく}くは
\ruby{無}{な}く
\ruby{思}{おも}ふやうに
なるかも
\原本頁{46-4}\改行%
\ruby{知}{し}れないよ。
%
\ruby[g]{先生}{せんせい}だつて
\ruby[g]{{\換字{過}}日}{こなひだ}までとは
\ruby{異}{ちが}つて、
%
\ruby{今}{いま}ぢやあ
もう
お
\ruby{澤}{さは}
\ruby{婆}{ばゞあ}を% 踊り字調整「〻(二の字点、揺すり点)に濁点に見えるが(ゞ)」
\ruby{憎}{にく}いと
ばかり
\ruby{思}{おも}つては
\ruby{居}{ゐ}らつしやらない
やうだもの。
%
まあ
\ruby{何}{なん}とでも
\ruby{云}{い}つて
\ruby{居}{ゐ}るが
\ruby{宜}{い}い、
%
\ruby[||j>]{人}{にん}
\ruby[||j>]{{\換字{情}}}{じやう}と
% \ruby{人{\換字{情}}}{にん|じやう}と
いふものは
\ruby[g]{年齡}{と し }さへ
\ruby{老}{と}りやあ
\ruby{解}{わか}る
\ruby{事}{こと}だから。
』

\原本頁{46-8}%
\ruby{我}{わ}が
\ruby{此}{この}
\ruby{上}{うへ}
\ruby{無}{な}く
\ruby{好}{す}きなる
\ruby{其}{その}
\ruby{人}{ひと}の、
%
\ruby{我}{わ}が
\ruby{此}{この}
\ruby{上}{うへ}
\ruby{無}{な}く
\ruby{{\換字{嫌}}}{きら}へる
\ruby{其}{その}
\ruby{婆}{ばゞ}を% 踊り字調整「〻(二の字点、揺すり点)に濁点に見えるが(ゞ)」
\ruby{憎}{にく}しと
のみは
\ruby{思}{おも}ひ
\ruby{居}{を}らじと
\ruby{云}{い}へるを
\ruby{聞}{き}きて、
%
お
\ruby{濱}{はま}は
\ruby{且}{かつ}は
\ruby{驚}{おどろ}き、
%
\ruby{且}{かつ}は
\ruby{訝}{いぶか}り、
%
\ruby[g]{疑惑}{うたがひ}の
\ruby{眉}{まゆ}を
\ruby[g]{可憐}{か はゆ}らしく
\ruby{顰}{ひそ}め
\ruby{頸}{くび}を
\ruby{枉}{ま}げて
\ruby[g]{水野}{みづの }の
\ruby{面}{おもて}を
\ruby{覗}{のぞ}き
\ruby{{\換字{込}}}{こ}みつゝ、% 踊り字調整「〻(二の字点、揺すり点)に見えるが(ゝ)」

\原本頁{47-1}%
『
ほんとなの?、
%
\ruby[g]{先生}{せんせい}。
%
\ruby[g]{先生}{せんせい}
あんな
\ruby{意地惡}{い|ぢ|わる}な
\ruby{惡}{にく}らしい
\ruby[||j>]{老}{おばあ}
\ruby[||j>]{婆}{ さん}が
% \ruby{老婆}{おばあ|さん}が
\ruby{好}{すき}になつたの?。
』

\原本頁{47-3}%
と、
%
さも〳〵
\ruby{然}{しか}らずといふ
\ruby{答}{こたへ}を
\ruby{聞}{き}きて、
%
\ruby{改}{あらた}めて
\ruby{{\換字{又}}}{また}
\ruby{我}{わ}が
\ruby[g]{祖{\換字{父}}}{ぢ ゝ }に% 踊り字調整「〻(二の字点、揺すり点)に見えるが(ゝ)」
\ruby{對}{むか}ひて
\ruby{{\換字{勝}}}{か}ち
\ruby{誇}{ほこ}りたげに
\ruby{{\換字{尋}}}{たづ}ねたり。

\原本頁{47-5}%
\ruby[g]{水野}{みづの }は
\ruby[g]{先刻}{さつき }より
\ruby[g]{小刀}{こがたな}を
もて
\ruby[||j>]{心}{こゝろ}% 踊り字調整「〻(二の字点、揺すり点)に見えるが(ゝ)」
\ruby{長}{ なが}く
\ruby[g]{叮嚀}{ていねい}に
\ruby{栗}{くり}を
\ruby{剝}{む}きつゝ、% 踊り字調整「〻(二の字点、揺すり点)に見えるが(ゝ)」
%
\ruby{既}{すで}に
\ruby{世}{よ}に
\ruby{老}{お}いたる
\ruby{{\換字{翁}}}{おきな}と
\ruby{未}{ま}だ
\ruby{世}{よ}を
\ruby{知}{し}らぬ
\ruby[g]{少女}{をとめ }との、
%
\ruby[g]{彼方}{かなた }は
\ruby[g]{經驗}{おぼ{\換字{𛀁}}}に
\ruby{頼}{よ}り
\ruby[g]{此方}{こなた }は% ルビ調整(原本通り)
\ruby[g]{{\換字{空}}想}{おもひ }に
\ruby{任}{まか}せて、
%
\ruby{相}{あひ}
\ruby{和}{わ}せぬ
\ruby{談}{はなし}を
\ruby{{\換字{交}}}{まじ}ふるをば、
%
おのづから
\ruby{催}{もよほ}さるゝ% 踊り字調整「〻(二の字点、揺すり点)に見えるが(ゝ)」
\ruby[g]{微笑}{ほゝゑみ}の% 踊り字調整「〻(二の字点、揺すり点)に見えるが(ゝ)」
\ruby{間}{うち}に
\ruby{聞}{き}き
\ruby{居}{ゐ}たりしが、
%
\ruby{恰}{あたか}も% 恰も「あ(た)かも」
\ruby[g]{此時}{このとき}
\ruby[g]{奇麗}{ゝ れい}に% 踊り字調整「〻(二の字点、揺すり点)に見えるが(ゝ)」
\ruby{剝}{む}き
\ruby{{\換字{終}}}{をは}りし
\ruby[g]{一箇}{ひとつ }の
\ruby{栗}{くり}を、
%
そつと
お
\ruby{濱}{はま}が
\ruby{掌}{て}の
\ruby{上}{うへ}に
\ruby{載}{の}せてやりつ、

\原本頁{47-10}%
『
なにも
\ruby{好}{すき}に
なつたといふ
\ruby{事}{こと}は
\ruby{無}{な}いのだ
けれども、
%
そりやあ
\ruby{憎}{にく}いと
ばかりも
\ruby{思}{おも}つては
\ruby{居}{ゐ}ない。
%
\ruby{考}{かんが}へて
\ruby{見}{み}ると
\ruby{今}{いま}では
\ruby[||j>]{憫}{かあ}
\ruby[||j>]{然}{いさう}で% 「憫然 か(あ)いさう」
% \ruby{憫然}{かあ|いさう}で% 「憫然 か(あ)いさう」
ならないやうな
\ruby{氣}{き}さへする
\ruby{位}{くらゐ}だから。
』

\原本頁{48-2}%
と
\ruby{優}{やさ}しく
\ruby{答}{こた}へて、

\原本頁{48-3}%
『
お
\ruby{濱}{はま}ちやん
だつて
\ruby{今}{いま}に
\ruby{彼}{あ}の
お
\ruby{澤}{さは}の
\ruby{腹}{おなか}の
\ruby{中}{なか}が
\ruby[g]{合點}{が てん}が
\ruby{行}{ゆ}けば、
%
\ruby[g]{彼婆}{あ れ }を
\ruby{憎}{にく}らしいとは
\ruby{思}{おも}はないやうに
なるかも
\ruby{知}{し}れないよ。
』

\原本頁{48-5}%
と
\ruby{語}{ことば}を
\ruby{足}{た}したり。

\原本頁{48-6}%
\ruby[g]{水野}{みづの }が
\ruby{此}{この}
\ruby{語}{ことば}は
\ruby[g]{如何}{い か }ばかり
\ruby{思}{おもひ}の
\ruby{外}{ほか}なりけん、
%
お
\ruby{濱}{はま}は
\ruby{呆}{あき}れたる
\ruby{眼}{め}を
\ruby{{\換字{睜}}}{みは}つて
\ruby{默}{だま}りけるが、
%
\ruby{吉右衛門}{きち||ゑ|もん}は
\ruby[g]{待設}{まちまう}けしやうに
\ruby[<j||]{言}{ことば}を
\ruby[<j>]{挿}{さしはさ}みぬ。

\原本頁{48-8}%
『
それ
\ruby[g]{御覧}{ご らん}、
%
\ruby[||j>]{老}{おぢい}
\ruby[||j>]{夫}{ さん}の
% \ruby{老夫}{おぢい|さん}の
\ruby{言}{い}ふ
\ruby{事}{こと}も
\ruby{嘘}{うそ}ぢやあ
\ruby{有}{あ}るまい。
%
\ruby{好}{す}きなものが
\ruby{{\換字{嫌}}}{きらひ}になつたりもすれば
\ruby{{\換字{嫌}}}{きらひ}なものが
\ruby{好}{す}きになつたりもする、
%
それは
\ruby[<j||]{皆}{みんな}
\ruby[||j>]{人}{にん}
\ruby[||j>]{{\換字{情}}}{じやう}と
% \ruby{人{\換字{情}}}{にん|じやう}と
いふものが
\ruby{爲}{さ}せるんで、
%
まだ
\ruby[g]{中々}{なか〳〵}
\ruby[||j>]{汝}{おまへ}
\ruby[||j>]{{\換字{達}}}{ たち}にやあ
% \ruby{汝{\換字{達}}}{おまへ|たち}にやあ
\ruby{{\換字{分}}}{わか}らない
\ruby{事}{こと}なんだよ。
』

\原本頁{49-1}%
お
\ruby{濱}{はま}は
\ruby[g]{祖{\換字{父}}}{ぢ ゞ }が% 踊り字調整「〻(二の字点、揺すり点)に見えるが(ゝ)」
\ruby{言}{ことば}を
\ruby{聞}{き}きも
せずして、
%
\ruby{今}{いま}
\ruby{貰}{もら}ひし
\ruby{栗}{くり}を
\ruby{無邪氣}{む|じや|き}に
\ruby{食}{た}べながら、
%
\ruby[g]{何事}{なにごと}を
\ruby{思}{おも}ひ
\ruby{{\換字{廻}}}{めぐ}らせるならん、
%
あらぬ
\ruby{方}{かた}に
\ruby{眼}{め}を
\ruby{{\換字{留}}}{とゞ}めて% 踊り字調整「〻(二の字点、揺すり点)に濁点に見えるが(ゞ)」
\ruby[g]{一寸}{ちよつと}
\ruby{考}{かんが}へ
\ruby{居}{ゐ}れば、
%
\ruby[g]{水野}{みづの }は
\ruby{{\換字{又}}}{また}
\ruby{樂}{ゝの}しげに% 踊り字調整「〻(二の字点、揺すり点)に見えるが(ゝ)」
\ruby{栗}{くり}を
\ruby{剝}{む}き
\ruby{居}{を}り、
%
\ruby{吉右衛門}{きち||ゑ|もん}は
\ruby[g]{{\換字{煙}}草}{たばこ }を
\ruby{深}{ふか}く
\ruby{吸}{す}ひて
\ruby{{\換字{緩}}}{ゆる}やかに
\ruby{其}{そ}の
\ruby{烟}{けむり}を
\ruby{噴}{ふ}き
\ruby{出}{だ}し
\ruby{居}{を}れり。

\原本頁{49-5}%
\ruby[g]{靜寂}{しづか }なりしは
たゞ% 踊り字調整「〻(二の字点、揺すり点)に濁点に見えるが(ゞ)」
\ruby{一霎時}{し|ば|し}なりき。
%
お
\ruby{濱}{はま}は
\ruby{何}{なに}を
\ruby{思}{おも}ひ
\ruby{得}{{\換字{𛀁}}}しにや
\ruby{忽}{たちま}ち
\ruby{嬉}{うれ}しげなる
\ruby{聲}{こゑ}に
\ruby{淋}{さび}しさを
\ruby{破}{やぶ}つて、

\原本頁{49-7}%
『
アヽ
\ruby[||j>]{妾}{わたし}
\ruby[||j>]{{\換字{分}}}{ わか}つてよ、
%
\ruby[||j>]{妾}{わたし}
\ruby[||j>]{{\換字{分}}}{ わか}つてよ。
%
\ruby{五十子}{い|そ|こ}さんが
\ruby{今}{いま}に
\ruby{快}{よ}くなるとネエ、
%
\ruby[g]{屹度}{きつと }
\ruby[g]{大變}{たいへん}に
\ruby[g]{先生}{せんせい}が
\ruby{好}{すき}になるんでしやう、
ホヽヽ、
%
それが
\ruby[||j>]{人}{にん}
\ruby[||j>]{{\換字{情}}}{じやう}つて
% \ruby{人{\換字{情}}}{にん|じやう}つて
\ruby{云}{い}ふものなんでしやう。
%
\ruby[g]{左樣}{さ う }ぢやあ
\ruby{無}{な}くつて?、
%
え、
%
\ruby[g]{祖{\換字{父}}}{お ぢい}さん!。
%
\ruby{五十子}{い|そ|こ}さんが
\ruby[g]{先生}{せんせい}を
\ruby[g]{大好}{だいすき}になる、
%
アヽ
\ruby[g]{左樣}{さ う }なると
\ruby{好}{い}いわ、
%
\ruby{早}{はや}く
\ruby[g]{左樣}{さ う }なると、
%
\ruby[||j>]{妾}{わたし}
\ruby{五十子}{ い|そ|こ}さんを
\ruby{姉}{ね{{\換字{𛀁}}}}さんに
\ruby{爲}{し}つちまふから、
%
\ruby[g]{先生}{せんせい}が
\ruby{兄}{にい}さんで、
%
\ruby{五十子}{い|そ|こ}さんが
\ruby{姉}{ね{{\換字{𛀁}}}}さんで、
%
さうして
\ruby{妾}{わたし}が
\ruby{其}{その}
\ruby{傍}{そば}に
\ruby{貼}{つ}いて
\ruby{居}{ゐ}るんなら、
%
ほんとに
\ruby[g]{何樣}{どんな }に
\ruby{嬉}{うれ}しいか
\ruby{知}{し}れや
\ruby{仕}{し}ないわ。
%
\ruby[g]{左樣}{さ う }なれば
\ruby{妾}{わたし}あ
\ruby{魯敏孫}{ろ|びん|そん}の
\ruby[||j>]{朋}{おとも}
\ruby[||j>]{友}{ だち}になるのは
% \ruby{朋友}{おとも|だち}になるのは
\ruby{廃}{よ}して
\ruby{{\換字{終}}}{しま}ふは。
』

\原本頁{50-4}%
と、
%
\ruby[<j>]{僞}{いつはり}ならず
\ruby{悅}{よろこ}びて
\ruby{云}{い}ひ
\ruby{出}{だ}したる、
%
\ruby{面}{おもて}は
\ruby{晴}{は}れやかにして
\ruby{月}{つき}は
\ruby{雲}{くも}なく、
%
\ruby{{\換字{情}}}{こゝろ}は% 踊り字調整「〻(二の字点、揺すり点)に見えるが(ゝ)」
\ruby{優}{やさ}しくして
\ruby{花}{はな}に
\ruby{露}{つゆ}あり。

\原本頁{50-6}%
されど
お
\ruby{濱}{はま}は
\ruby{{\換字{又}}}{また}
たゞちに、% 踊り字調整「〻(二の字点、揺すり点)に濁点に見えるが(ゞ)」

\原本頁{50-7}%
『
だけれど、
』

\原本頁{50-8}%
と
\ruby{云}{い}ひさして
\ruby[g]{祖{\換字{父}}}{ぢ ゝ }の% 踊り字調整「〻(二の字点、揺すり点)に見えるが(ゝ)」
\ruby{面}{おもて}を
\ruby{見}{み}たり。
%
\ruby[g]{水野}{みづの }は
お
\ruby{濱}{はま}の
\ruby{言}{ことば}を
\ruby{何}{なに}と
\ruby{聞}{き}きし
\原本頁{50-9}\改行%
や、
%
\ruby[g]{何氣}{なにげ }
\ruby{無}{な}き
\ruby{風}{ふう}に
\ruby{身}{み}をも
\ruby{動}{うご}かさず、
%
ひたすらに
\ruby{栗}{くり}を
\ruby{剝}{む}き
\ruby{居}{ゐ}たり
\改行% 校正作業の簡略化のため
。
