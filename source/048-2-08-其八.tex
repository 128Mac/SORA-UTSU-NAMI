\Entry{其八}

『ハヽヽ。
\ruby{然様}{さ|う}ムキになつて
\ruby{老夫}{おぢい|さん}に
\ruby{食}{く}つて
\ruby{掛}{かヽ}ることは
\ruby{無}{な}いぢやあ
\ruby{無}{な}いか。
もう
\ruby{可}{い}い、
\ruby{可}{い}い。
とても
\ruby{老夫}{おぢい|さん}は
\ruby{汝}{おまへ}にやあ
\ruby{敵}{かな}はないよ。
しかし
\ruby{汝}{おまへ}がもう
\ruby{二三年}{に|さん|ねん}も
\ruby{年}{とし}をとつて、
\ruby{物事}{もの|ごと}が
\ruby{善}{よ}く
\ruby{解}{わか}つて
\ruby{來}{く}ると、お
\ruby{澤婆}{さは|ばヽあ}だつて
\ruby[g]{其様}{そんな}に
\ruby{憎}{にく}くは
\ruby{無}{な}く
\ruby{思}{おも}ふやうになるかも
\ruby{知}{し}れないよ。
\ruby{先生}{せん|せい}だつて
\ruby{{\GWI{u904e-k}}日}{こな|いだ}までとは
\ruby{異}{ちが}つて、
\ruby{今}{いま}ぢやあもうお
\ruby{澤婆}{さは|ばヽあ}を
\ruby{憎}{にく}いとばかり
\ruby{思}{おも}つては
\ruby{居}{ゐ}らつしやらないやうだもの。
まあ
\ruby{何}{なん}とでも
\ruby{云}{い}つて
\ruby{居}{ゐ}るが
\ruby{宣}{い}い、
\ruby{人{\GWI{u60c5-k}}}{にん|じやう}といふものは
\ruby{年齢}{と|し}さへ
\ruby{老}{と}りやあ
\ruby{解}{わか}る
\ruby{事}{こと}だから。
』

\ruby{我}{わ}が
\ruby{此上無}{この|うへ|な}く
\ruby{好}{す}きなる
\ruby{其人}{その|ひと}の、
\ruby{我}{わ}が
\ruby{此上無}{この|うへ|な}く
\ruby{{\GWI{u5acc-k}}}{きら}へる
\ruby{其婆}{その|ばヾ}を
\ruby{憎}{にく}しとのみは
\ruby{思}{おも}ひ
\ruby{居}{を}らじと
\ruby{云}{い}へるを
\ruby{聞}{き}きて、お
\ruby{濱}{はま}は
\ruby{且}{かつ}は
\ruby{驚}{おどろ}き、
\ruby{且}{かつ}は
\ruby{訝}{いぶか}り、
\ruby{疑惑}{うた|がひ}の
\ruby{眉}{まゆ}を
\ruby{可憐}{かは|ゆ}らしく
\ruby{顰}{ひそ}め
\ruby{頸}{くび}を
\ruby{枉}{ま}げて
\ruby[g]{水野}{みづの}の
\ruby{面}{おもて}を
\ruby{覗}{のぞ}き
\ruby{込}{こ}みつ……、

『ほんとなの?、
\ruby{先生}{せん|せい}。
\ruby{先生}{せん|せい}あんな
\ruby{意地惡}{い|じ|わる}な
\ruby{惡}{にく}らしい
\ruby{老婆}{おばあ|さん}が
\ruby{好}{すき}になつたの?。
』

と、さも〳〵
\ruby{然}{しか}らずといふ
\ruby{答}{こたへ}を
\ruby{聞}{き}きて、
\ruby{改}{あらた}めて
\ruby{{\GWI{u53c9-k}}}{また}
\ruby{我}{わ}が
\ruby{祖父}{そ|ふ}に
\ruby{對}{むか}ひて
\ruby{{\換字{勝}}}{か}ち
\ruby{誇}{ほこ}りたげに
\ruby{尋}{たづ}ねたり。

\ruby[g]{水野}{みづの}は
\ruby[g]{先刻}{さつき}より
\ruby{小刀}{こが|たな}をもて
\ruby{心長}{こヽろ|なが}く
\ruby{叮嚀}{てい|ねい}に
\ruby{栗}{くり}を
\ruby{剥}{む}きつヽ、
\ruby{既}{すで}に
\ruby{世}{よ}に
\ruby{老}{お}いたる
\ruby{{\GWI{u7fc1-k}}}{おきな}と
\ruby{未}{ま}だ
\ruby{世}{よ}を
\ruby{知}{し}らぬ
\ruby{少女}{をと|め}との、
\ruby{彼方}{かな|た}は
\ruby{經驗}{おぼ|{\GWI{u1b001}}}に
\ruby{頼}{よ}り
\ruby{此方}{こな|た}は
\ruby{{\GWI{u7a7a-ue0101}}想}{おも|ひ}に
\ruby{任}{まか}せて、
\ruby{相和}{あひ|わ}せぬ
\ruby{談}{はなし}を
\ruby{交}{まじ}ふるをば、おのづから
\ruby{催}{もよほ}さるヽ
\ruby{微笑}{ほヽ|ゑみ}の
\ruby{間}{うち}に
\ruby{聞}{き}き
\ruby{居}{ゐ}たりしが、
\ruby{恰}{あたか}も
\ruby{此時奇麗}{この|とき|ヽ|れい}に
\ruby{剥}{む}き
\ruby{{\GWI{u7d42-ue0101}}}{をは}りし
\ruby{一箇}{ひと|つ}の
\ruby{栗}{くり}を、そつとお
\ruby{濱}{はま}が
\ruby{掌}{て}の
\ruby{上}{うへ}に
\ruby{載}{の}せてやりつ、

『なにも
\ruby{好}{すき}になつたといふ
\ruby{事}{こと}は
\ruby{無}{な}いのだけれども、そりやあ
\ruby{憎}{にく}いとばかりも
\ruby{思}{おも}つては
\ruby{居}{ゐ}ない。
\ruby{考}{かんが}へて
\ruby{見}{み}ると
\ruby{今}{いま}では
\ruby{憫然}{かあい|さう}でならないやうな
\ruby{氣}{き}さへする
\ruby{位}{くらゐ}だから。
』

と
\ruby{優}{やさ}しく
\ruby{答}{こた}へて、

『お
\ruby{濱}{はま}ちやんだつて
\ruby{今}{いま}に
\ruby{彼}{あ}のお
\ruby{澤}{さは}の
\ruby{腹}{おなか}の
\ruby{中}{なか}が
\ruby{合點}{が|てん}が
\ruby{行}{ゆ}けば、
\ruby{彼婆}{あ|れ}を
\ruby{憎}{にく}らしいとは
\ruby{思}{おも}はないやうになるかも
\ruby{知}{し}れないよ。
』

と
\ruby{語}{ことば}を
\ruby{足}{た}したり。

\ruby[g]{水野}{みづの}が
\ruby{此語}{この|ことば}は
\ruby{如何}{い|か}ばかり
\ruby{思}{おもひ}の
\ruby{外}{ほか}なりけん、お
\ruby{濱}{はま}は
\ruby{呆}{あき}れたる
\ruby{眼}{め}を
\ruby{睜}{みは}つて
\ruby{默}{だま}りけるが、
\ruby[g]{吉右衛門}{きちゑもん}は
\ruby{待設}{まち|まう}けしやうに
\ruby{言}{ことば}を
\ruby{挿}{さしはさ}みぬ。

『それ
\ruby{御覧}{ご|らん}、
\ruby{老夫}{おぢい|さん}の
\ruby{言}{い}ふ
\ruby{事}{こと}も
\ruby{嘘}{うそ}ぢやあ
\ruby{有}{あ}るまい。
\ruby{好}{す}きなものが
\ruby{{\GWI{u5acc-k}}}{きらひ}になつたりもすれば
\ruby{{\GWI{u5acc-k}}}{きらひ}なものが
\ruby{好}{す}きになつたりもする、それは
\ruby{皆}{みんな}
\ruby{人{\GWI{u60c5-k}}}{ にん|じやう}といふものが
\ruby{爲}{さ}せるんで、まだ
\ruby{中々}{なか|〳〵}
\ruby{汝{\GWI{u9054-k}}}{おまへ|たち}にやあ
\ruby{{\換字{分}}}{わか}らない
\ruby{事}{こと}なんだよ。
』

お
\ruby{濱}{はま}は
\ruby{祖父}{ぢ|ヾ}が
\ruby{言}{ことば}を
\ruby{聞}{き}きもせずして、
\ruby{今貰}{いま|もら}ひし
\ruby{栗}{くり}を
\ruby{無邪氣}{む|じや|き}に
\ruby{食}{た}べながら、
\ruby{何事}{なに|ごと}を
\ruby{思}{おも}ひ
\ruby{{\GWI{u5efb-k}}}{めぐ}らせるならん、あらぬ
\ruby{方}{かた}に
\ruby{眼}{め}を
\ruby{留}{とど}めて
\ruby{一寸考}{ちよ|つと|かんが}へ
\ruby{居}{ゐ}れば、
\ruby[g]{水野}{みづの}は
\ruby{{\GWI{u53c9-k}}}{また}
\ruby{樂}{たの}しげに
\ruby{栗}{くり}を
\ruby{剥}{む}き
\ruby{居}{を}り、
\ruby[g]{吉右衛門}{きちゑもん}は
\ruby{{\GWI{u7159-k}}草}{たば|こ}を
\ruby{深}{ふか}く
\ruby{吸}{す}ひて
\ruby{{\GWI{u7de9-k}}}{ゆる}やかに
\ruby{其}{そ}の
\ruby{烟}{けむり}を
\ruby{噴}{ふ}き
\ruby{出}{だ}し
\ruby{居}{を}れり。

\ruby{靜寂}{しづ|か}なりしはたヾ
\ruby{一霎時}{し|ば|し}なりき。
お
\ruby{濱}{はま}は
\ruby{何}{なに}を
\ruby{思}{おも}ひ
\ruby{得}{{\GWI{u1b001}}}しにや
\ruby{忽}{たちま}ち
\ruby{嬉}{うれ}しげなる
\ruby{聲}{こゑ}に
\ruby{淋}{さび}しさを
\ruby{破}{やぶ}つて、

『アヽ
\ruby{妾}{わたし}
\ruby{{\換字{分}}}{ わか}つてよ、
\ruby{妾}{わたし}
\ruby{{\換字{分}}}{ わか}つてよ。
\ruby{五十子}{い|そ|こ}さんが
\ruby{今}{いま}に
\ruby{快}{よ}くなるとネェ、
\ruby{屹度大變}{きつ|と|たい|へん}に
\ruby{先生}{せん|せい}が
\ruby{好}{す}きになるんでしやう、ホヽヽ、それが
\ruby{人{\GWI{u60c5-k}}}{ にん|じやう}つて
\ruby{云}{い}ふものなんでしやう。
\ruby{左様}{さ|う}ぢやあ
\ruby{無}{な}くつて?、え、
\ruby{祖父}{おぢ|い}さん!。
\ruby{五十子}{い|そ|こ}さんが
\ruby{先生}{せん|せい}を
\ruby{大好}{だい|す}きになる、アヽ
\ruby{左様}{さ|う}なると
\ruby{好}{い}いわ、
\ruby{早}{はや}く
\ruby{左様}{さ|う}なると、
\ruby{妾}{わたし}
\ruby{五十子}{ い|そ|こ}さんを
\ruby{姉}{ね{\GWI{u1b001}}}さんに
\ruby{爲}{し}つちまふから、
\ruby{先生}{せん|せい}が
\ruby{兄}{にい}さんで、
\ruby{五十子}{い|そ|こ}さんが
\ruby{姉}{ね{\GWI{u1b001}}}さんで、さうして
\ruby{妾}{わたし}が
\ruby{其傍}{その|そば}に
\ruby{貼}{つ}いて
\ruby{居}{ゐ}るんなら、ほんとに
\ruby{何様}{どん|な}に
\ruby{嬉}{うれ}しいか
\ruby{知}{し}れや
\ruby{仕}{し}ないわ。
\ruby{左様}{さ|う}なれば
\ruby{妾}{わたし}あ
\ruby[g]{魯敏孫}{ろびんそん}の
\ruby{朋友}{おとも|だち}になるのは
\ruby{廃}{よ}して
\ruby{{\GWI{u7d42-ue0101}}}{しま}ふは。
』

と、
\ruby{僞}{いつはり}ならず
\ruby{{\GWI{u6085-jv}}}{よろこ}びて
\ruby{云}{い}ひ
\ruby{出}{だ}したる、
\ruby{面}{おもて}は
\ruby{晴}{は}れやかにして
\ruby{月}{つき}は
\ruby{雲}{くも}なく、
\ruby{{\GWI{u60c5-k}}}{こヽろ}は
\ruby{優}{やさ}しくして
\ruby{花}{はな}に
\ruby{露}{つゆ}あり。

されどお
\ruby{濱}{はま}は
\ruby{{\GWI{u53c9-k}}}{また}たヾちに、

『だけれど、』

と
\ruby{云}{い}ひさして
\ruby{祖父}{ぢ|ヽ}の
\ruby{面}{おもて}を
\ruby{見}{み}たり。
\ruby[g]{水野}{みづの}はお
\ruby{濱}{はま}の
\ruby{言}{ことば}を
\ruby{何}{なに}と
\ruby{聞}{き}きしや、
\ruby{何氣無}{なに|げ|な}き
\ruby{風}{ふう}に
\ruby{身}{み}をも
\ruby{動}{うご}かさず、ひたすらに
\ruby{栗}{くり}を
\ruby{剥}{む}き
\ruby{居}{ゐ}たり。

