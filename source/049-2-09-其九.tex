\Entry{其九}

% メモ 校正終了 2024-04-17
\原本頁{51-1}%
『だけれども
\ruby{何}{なん}だエ?。
』

\原本頁{51-2}%
お
\ruby{濱}{はま}の
\ruby{言}{い}ひ
\ruby{澱}{よど}みたるを
\ruby{怪}{あやし}みて
\ruby[g]{吉右衛門}{きちゑもん}は
\ruby{輕}{かる}く
\ruby{問}{と}へば、

\原本頁{51-3}%
『だけれども、
%
\ruby{何}{なん}だか
\ruby{知}{し}らないけども
\ruby{妾}{わたし}にやあ\換字{子}エ、
%
\ruby{何樣}{ど|う}も
\ruby{左樣}{さ|う}なりさうも
\ruby{無}{な}いやうな
\ruby{氣}{き}が
\ruby{自然}{ひと|りで}に
するのよ。
%
\ruby[g]{五十子}{いそこ}さんは
\ruby{病氣}{びやう|き}が
\ruby{癒}{なほ}つたらば\換字{子}、
%
\ruby{{\換字{遠}}}{とほ}い
\ruby{{\換字{遠}}}{とほ}い
ところへでも
\ruby{行}{い}つて
お
\ruby{仕舞}{し|ま}ひなさりさうな
\ruby{氣}{き}がするのよ。
%
\ruby{而}{さう}して
\ruby{其}{その}
\ruby{後}{あと}で
\ruby{松}{まつ}ちやんと
\ruby{妾}{わたし}とが
\ruby{一緖}{いつ|しよ}に
\ruby{泣}{な}くやうな
\ruby{事}{こと}が
ありさうに
\ruby{思}{おも}ふのよ。
%
あの
\ruby{椎}{しひ}の
\ruby{樹}{き}の
\ruby{暗}{くら}い
\ruby{蔭}{かげ}に、
%
たつた
\ruby{二人}{ふた|り}で
\ruby{淋}{さみ}{---}しく
\ruby{殘}{のこ}つて、
%
\ruby{泣}{な}くやうな
\ruby{事}{こと}に
なりさうな
\ruby{氣}{き}がするのよ。
』

\原本頁{51-10}%
と
\ruby{{\換字{近}}傍}{あた|り}
\ruby{關}{かま}はず
\ruby{言}{い}ひ
\ruby{放}{はな}ちたり。

\原本頁{51-11}%
\ruby{嫩}{わか}き
\ruby{心}{こ〻ろ}の% 原本通り「〻(二の字点、揺すり点)」
\ruby{{\換字{前}}後}{あと|さき}を
\ruby{顧}{かへりみ}ずして、
%
おのが
\ruby{胸}{むね}に
\ruby{{\換字{浮}}}{うか}める
ま〻を% 原本通り「〻(二の字点、揺すり点)」
\ruby{憚}{はゞか}り% TODO 原本の「二の字点、揺すり点」に濁点のグリフが見つからないので「ゞ」
\ruby{氣}{げ}も
\ruby{無}{な}く
\ruby{云}{い}ひ
\ruby{出}{だ}したる
\ruby{其}{それ}は、
%
もとより
\ruby{取}{と}るに
\ruby{足}{た}らぬ
\ruby{{\換字{空}}想}{おも|ひ}
ながら、
%
\ruby{戀}{こひ}に
\原本頁{52-2}\改行%
\ruby{心}{こ〻ろ}の% 原本通り「〻(二の字点、揺すり点)」
\ruby{{\換字{弱}}}{よわ}れる
\ruby{人}{ひと}には、
%
\ruby{幸}{さい}
\ruby{先}{さき}
あしき
\ruby{如是}{か〻|る}% 原本通り「〻(二の字点、揺すり点)」
\ruby{一}{ひ}ト
\ruby{言}{こと}の
\ruby{如何}{い|か}ばかり
\ruby{氣}{き}に
\ruby{障}{さは}り
\ruby{胸}{むね}に
\ruby{徹}{こた}へやしけんと、
%
\ruby[g]{吉右衛門}{きちゑもん}は
そつと
\ruby[g]{水野}{みづの}を
\ruby{見}{み}るに、
%
\ruby{幸}{さいはひ}にして
\ruby{今}{いま}の
\ruby{言}{ことば}には
\ruby{別}{べつ}に
\ruby{心}{こ〻ろ}をも% 原本通り「〻(二の字点、揺すり点)」
\ruby{動}{うご}かさゞりしやうにて、% TODO 原本の「二の字点、揺すり点」に濁点のグリフが見つからないので「ゞ」
%
\ruby{{\換字{猶}}}{なほ}
\ruby{默々}{もく|〳〵}と
\ruby{栗}{くり}を
\ruby{剝}{む}きつゞけ% TODO 原本の「二の字点、揺すり点」に濁点のグリフが見つからないので「ゞ」
\ruby{居}{を}れば、
%
やうやく
\ruby{自{\換字{分}}}{おの|れ}も
\ruby{安}{やす}き
\ruby{思}{おもひ}して、

\原本頁{52-6}%
『イヤ、
%
\ruby{老夫}{おぢい|さん}には
\ruby{其樣}{そ|ん}な
\ruby{氣}{き}は
\ruby{仕}{し}ないよ。
%
\ruby[g]{五十子}{いそこ}さんが
\ruby{{\換字{遠}}}{とほ}い
ところへ
\ruby{行}{い}つて
\ruby{仕舞}{し|ま}ふなんて、
%
そりやあ
\ruby{汝}{おまへ}が
\ruby{魯敏孫}{ろ|びん|そん}とかの
\ruby{書}{ほん}を
\ruby{讀}{よ}んだ
せいで、
%
そんな
\ruby{下}{くだ}らない
\ruby{事}{こと}を
\ruby{思}{おも}ひ
ついたんだらう。
%
\ruby{老夫}{おぢい|さん}は
また
\ruby[g]{五十子}{いそこ}さんが
\ruby{癒}{なほ}つて、
%
\ruby{松}{まつ}ちやんだの、
%
\ruby{汝}{おまへ}だの、
%
\ruby{島木}{しま|き}さんだのと、
%
みんなが
\ruby{賑}{にぎ}やかに
\ruby{{\換字{遊}}}{あそ}ぶ
\ruby{事}{こと}が、
%
\ruby{屹度}{きつ|と}
あるやうに
\ruby{思}{おも}つて
\ruby{居}{ゐ}るよ。
』

\原本頁{52-11}%
と
\ruby{老人}{とし|より}の
\ruby{思}{おも}ひ
\ruby{{\換字{遣}}}{や}り
\ruby{深}{ふか}くも
\ruby{祝}{いは}ひ
\ruby{直}{なほ}したり。

\原本頁{53-1}%
\ruby{賢}{かしこ}けれども
\ruby{{\換字{猶}}}{なほ}
\ruby{年}{とし}
\ruby{{\換字{若}}}{わか}ければ、
%
\ruby{言外}{げん|ぐわい}の
\ruby{其}{その}
\ruby{意}{こ〻ろ}は% 原本通り「〻(二の字点、揺すり点)」
\ruby{汲}{く}みて
\ruby{知}{し}るに
\ruby{由}{よし}
\ruby{無}{な}く、

\原本頁{53-2}%
『イヽエ、
%
ちつとも
\ruby{漂流記}{へう|りう|き}の
\ruby{故}{せゐ}ぢやあ% せ(ゐ)
\ruby{無}{な}いわ。
%
\ruby{{\換字{過}}日}{こな|ひだ}
\ruby{松}{まつ}ちやんと
\ruby{二人}{ふた|り}で、
%
あの
\ruby{椎}{しひ}の
\ruby{樹}{き}の
\ruby{蔭}{かげ}で
\ruby{話}{はなし}を
\ruby{仕}{し}た
\ruby{其}{その}
\ruby{時}{とき}から、
%
\ruby{何}{なん}となく
\ruby{其樣}{そ|ん}な
\ruby{氣}{き}が
\ruby{仕}{し}はじめたのよ。
%
\ruby{御爺}{お|ぢい}さん
こそ
\ruby{屹度}{きつ|と}
\ruby[g]{二筋{\換字{道}}}{あのほん}が
\ruby{贔負}{ひゐ|き}だから、
%
\ruby{彼}{あ}の
\ruby{本}{ほん}のやうに
なると
ばつかし
\ruby{考}{かんが}へて
\ruby{居}{ゐ}るんだわ。
』

\原本頁{53-6}%
と、
お
\ruby{濱}{はま}が
\ruby{負}{ま}けじ
\ruby{心}{ご〻ろ}に% 原本通り「〻(二の字点、揺すり点)」
\ruby{云}{い}ひ
\ruby{爭}{あらそ}ふ
\ruby{時}{とき}、
%
\ruby{今}{いま}まで
\ruby{傍目}{よそ|め}
\ruby{訝}{いぶか}しきまで
\ruby{沈着}{おち|つき}に
\ruby{沈着}{おち|つ}き
\ruby{居}{ゐ}し
\ruby[g]{水野}{みづの}は、

\原本頁{53-8}%
『どつちでも
まあ
\ruby{宜}{い}いぢやあ
\ruby{無}{な}いか
お
\ruby{濱}{はま}ちやん!。
%
\ruby{明日}{あし|た}の
\ruby{事}{こと}は
\ruby{明日}{あし|た}の
お
\ruby{天{\換字{道}}樣}{てん|たう|さま}が
\ruby{見}{み}せて
\ruby{下}{くだ}さるわ\換字{子}。
%
ハヽヽ。
』

\原本頁{53-10}%
と
\ruby{悲}{かな}しげにも
\ruby{無}{な}ければ
\ruby{嬉}{うれ}しげにも
\ruby{無}{な}く、
%
もとより
\ruby{可笑}{を|かし}げにも
あらぬ
\ruby{聲}{こゑ}して
\ruby{笑}{わら}ひ
つ〻% 原本通り「〻(二の字点、揺すり点)」
\ruby{制}{せい}し、
%
\ruby{{\換字{又}}}{また}
その
\ruby{掌}{て}の
\ruby{上}{うへ}に
\ruby{剝}{む}きたる
\ruby{栗一}{くり|ひと}ツを、
%
\ruby{食}{た}べよと
ばかり
\ruby{優}{やさ}しく
\ruby{置}{お}き
\ruby{{\換字{遣}}}{や}りたり。

\原本頁{54-2}%
『コレ
\ruby{何}{なん}だ!。
%
\ruby{剝}{む}いたのを
\ruby{先生}{せん|せい}に
\ruby{戴}{いたゞ}く% TODO 原本の「二の字点、揺すり点」に濁点のグリフが見つからないので「ゞ」
といふものが
あるものか。
』

\原本頁{54-3}%
と
\ruby[g]{吉右衛門}{きちゑもん}が
\ruby{眼}{め}の
\ruby{見}{み}つけて
\ruby{叱}{しか}れるは
\ruby{遲}{おそ}く
\ruby{{\換字{緩}}}{ゆる}く、

\原本頁{54-4}%
『い〻わ\換字{子}エ、% 原本通り「〻(二の字点、揺すり点)」
%
\ruby{先生}{せん|せい}!、
%
\ruby{戴}{いたゞ}いたつて。% TODO 原本の「二の字点、揺すり点」に濁点のグリフが見つからないので「ゞ」
』

\原本頁{54-5}%
と
\ruby{云}{い}へる
\ruby{答}{こたへ}は
\ruby{短}{みじか}く
\ruby{捷}{はや}くして、
%
\ruby{栗}{くり}は
\ruby{既}{すで}に
\ruby{滿面}{まん|めん}に
\ruby{笑}{わらひ}を
\ruby{盛}{も}れる
お
\ruby{濱}{はま}が
\ruby{口裏}{く|ち}に
\ruby{隱}{かく}れたり。

\原本頁{54-7}%
されど
\ruby{何}{なん}としけん
お
\ruby{濱}{はま}は
\ruby{忽地}{たちま|ち}にして、
%
\ruby{其}{そ}の
\ruby{美}{うつく}しき
\ruby{眉}{まゆ}を
\ruby{顰}{ひそ}むれば、

\原本頁{54-8}%
『い〻% 原本通り「〻(二の字点、揺すり点)」
\ruby{氣味}{き|み}、
%
い〻% 原本通り「〻(二の字点、揺すり点)」
\ruby{氣味}{き|み}!。
%
\ruby{蟲}{むし}が
\ruby{居}{ゐ}たと
\ruby{見}{み}える。
』

\原本頁{54-9}%
と
\ruby{樣子}{やう|す}を
\ruby{見}{み}て
\ruby{取}{と}つて
\ruby[g]{吉右衛門}{きちゑもん}は
\ruby{可笑}{を|かし}がりて
\ruby{笑}{わら}ひ
\ruby{崩}{くづ}れぬ。
%
\ruby{蟲}{むし}は
あらぬ
\ruby{筈}{はず}なるを
\ruby{不思議}{ふ|し|ぎ}の
\ruby{事}{こと}かなと、
%
\ruby[g]{水野}{みづの}は
\ruby{氣}{き}の
\ruby{毒}{どく}さに
お
\ruby{濱}{はま}を
\ruby{打}{うち}
\ruby{護}{まも}れば、
%
お
\ruby{濱}{はま}は
また
\ruby{物}{もの}を
\ruby{捜}{さぐ}るが
\ruby{如}{ごと}くに
\ruby[g]{水野}{みづの}が
\ruby{手先}{て|さき}に
\ruby{眼}{め}を
\ruby{注}{そ〻}ぎ% 原本通り「〻(二の字点、揺すり点)」
\ruby{居}{ゐ}しが、
%
やがて
\ruby{口}{くち}の
\ruby{中}{なか}の
\ruby{物}{もの}を
\ruby{嚥}{の}み
\ruby{{\換字{終}}}{しま}ひて
\ruby{後}{のち}、
%
\ruby[g]{水野}{みづの}が
\ruby{手}{て}をば
\ruby{突然}{いき|なり}
\ruby{取}{と}りて、

\原本頁{55-3}%
『
\ruby{先生}{せん|せい}、
%
\ruby{負傷}{け|が}をして
\ruby{居}{ゐ}てよ!。
%
\ruby{痛}{いた}く
なくつて?。
』

\原本頁{55-4}%
と
\ruby{示}{しめ}したるを
\ruby{見}{み}れば、
%
\ruby{左}{ひだり}の
\ruby{拇指}{おや|ゆび}の
\ruby{其}{その}
\ruby{腹}{はら}に、
%
\ruby{鮮血}{せん|けつ}
いさ〻か% 原本通り「〻(二の字点、揺すり点)」
にじみて
\ruby{臙脂}{{\換字{𛀁}}ん|じ}
\ruby{微}{かすか}に
\ruby{湧}{わ}けり。
%
\ruby{何}{なに}に
\ruby{心}{こ〻ろ}を% 原本通り「〻(二の字点、揺すり点)」
\ruby{取}{と}られて、
%
\ruby{何時}{い|つ}の
\ruby{間}{ま}にか
\ruby{{\換字{過}}}{あやま}つて
\ruby{傷}{きず}つけて、
%
しかも
\ruby{今}{いま}までは
\ruby{知}{し}らざりけん、
%
\ruby{全}{まつた}く
\ruby{聊}{いさ〻}か% 原本通り「〻(二の字点、揺すり点)」
\ruby{此}{この}
\ruby{血}{ち}の
\ruby{着}{つ}きたるに
お
\ruby{濱}{はま}は
\ruby{栗}{くり}の
\ruby{味}{あぢはい}を
\ruby{怪}{あやし}みたるなり。

\原本頁{55-8}%
『アヽ
\ruby{穢}{きたな}い
\ruby{事}{こと}をした。
%
\ruby{惡}{わる}かつた!。
%
\ruby{勘{\換字{忍}}}{か|に}して% 原文通り「勘忍」
おくれよ
お
\ruby{濱}{はま}ちやん。
%
ほんとに
\ruby{毫}{すこし}も
\ruby{知}{し}らなかつたのだから。
』

\原本頁{55-10}%
『ナアニ
\ruby{毫}{ちつと}も
\ruby{穢}{きたな}かあ
\ruby{無}{な}いわ。
%
\ruby{最初}{さい|しよ}
\ruby{妾}{わたし}が
\ruby{血}{ち}の
\ruby{着}{つ}いたのを
あげるなんて、
%
\ruby{緣起}{{\換字{𛀁}}ん|ぎ}でも
\ruby{無}{な}い
\ruby{事}{こと}を
\ruby{云}{い}つたから
\ruby{惡}{わる}かつたのよ。
』

\原本頁{56-1}%
\ruby{瑣細}{さ|さい}の
\ruby{事}{こと}なれど、
%
\ruby{今}{いま}まで
\ruby{賑}{にぎ}やかに
\ruby{語}{かた}らひし
\ruby{談話}{はな|し}の
\ruby{腰}{こし}は
これに
\ruby{砕}{くだ}けて、
%
\ruby{何}{なん}となく
\ruby{淋}{さび}しく
\ruby{白}{しら}けたる
\ruby{一室}{ひと|ま}の
\ruby{内}{うち}には、
%
\ruby{今}{いま}
\ruby{沸}{たぎ}り
\ruby{初}{そ}めでも
\ruby{仕}{し}たるやうに
\ruby{鐵瓶}{てつ|びん}の
\ruby{煮}{に}ゆる
\ruby{音}{おと}の
\ruby{幽}{かすか}に
\ruby{響}{ひゞ}き% TODO 原本の「二の字点、揺すり点」に濁点のグリフが見つからないので「ゞ」
\ruby{出}{だ}して、
%
\ruby{靜}{しづ}まり
かへつたる
\ruby{村}{むら}の
\ruby{夜}{よる}の
\ruby{中}{なか}を、
%
\ruby{澁江村}{し|ぶ|{\換字{𛀁}}}との
\ruby{境界}{さ|かひ}
あたりにや
\ruby{狗}{いぬ}の
\ruby{吠}{ほ}ゆるが、
%
べう〳〵として
\ruby{遙}{はるか}に
\ruby{聞}{きこ}えぬ。
