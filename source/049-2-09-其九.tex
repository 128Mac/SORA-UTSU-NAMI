\Entry{其九}

『だけれども
\ruby{何}{なん}だエ?。
』

お
\ruby{濱}{はま}の
\ruby{言}{い}ひ
\ruby{澱}{よど}みたるを
\ruby{怪}{あやし}みて
\ruby[g]{吉右衛門}{きちゑもん}は
\ruby{輕}{かる}く
\ruby{問}{と}へば、

『だけれども、
\ruby{何}{なん}だか
\ruby{知}{し}らないけれども
\ruby{妾}{わたし}にやあ\換字{子}エ、
\ruby{何樣}{ど|う}も
\ruby{左樣}{さ|う}なりさうも
\ruby{無}{な}いやうな
\ruby{氣}{き}が
\ruby{自然}{ひと|りで}にするのよ。
\ruby{五十子}{い|そ|こ}さんは
\ruby{病氣}{びやう|き}が
\ruby{癒}{なほ}つたらば\換字{子}、
\ruby{{\換字{遠}}}{とほ}い
\ruby{{\換字{遠}}}{とほ}いところへでも
\ruby{行}{い}つて
お
\ruby{仕舞}{し|ま}ひなさりさうな
\ruby{氣}{き}がするのよ。
\ruby{而}{さう}して
\ruby{其後}{その|あと}で
\ruby{松}{まつ}ちやんと
\ruby{妾}{わたし}とが
\ruby{一緖}{いつ|しよ}に
\ruby{泣}{な}くやうな
\ruby{事}{こと}がありさうに
\ruby{思}{おも}ふのよ。
あの
\ruby{椎}{しい}の
\ruby{樹}{き}の
\ruby{暗}{くら}い
\ruby{蔭}{かげ}に、たつた
\ruby{二人}{ふた|り}で
\ruby{淋}{さみ}ーしく
\ruby{殘}{のこ}つて、
\ruby{泣}{な}くやうな
\ruby{事}{こと}になりさうな
\ruby{氣}{き}がするのよ。
』

と
\ruby{{\換字{近}}傍}{あた|り}
\ruby{關}{かま}はず
\ruby{言}{い}ひ
\ruby{放}{はな}ちたり。

\ruby{嫩}{わか}き
\ruby{心}{こゝろ}の
\ruby{{\換字{前}}後}{あと|さき}を
\ruby{顧}{かへりみ}ずして、おのが
\ruby{胸}{むね}に
\ruby{{\換字{浮}}}{うか}めるまゝを
\ruby{憚}{はゞか}り
\ruby{氣}{げ}も
\ruby{無}{な}く
\ruby{云}{い}ひ
\ruby{出}{だ}したる
\ruby{其}{それ}は、もとより
\ruby{取}{と}るに
\ruby{足}{た}らぬ
\ruby{{\換字{空}}想}{おも|ひ}ながら、
\ruby{戀}{こひ}に
\ruby{心}{こゝろ}の
\ruby{{\換字{弱}}}{よわ}れる
\ruby{人}{ひと}には、
\ruby{幸先}{さい|さき}あしき
\ruby{如是}{かゝ|る}
\ruby{一}{ひ}ト
\ruby{言}{こと}の
\ruby{如何}{い|か}ばかり
\ruby{氣}{き}に
\ruby{障}{さは}り
\ruby{胸}{むね}に
\ruby{徹}{こた}へやしけんと、
\ruby[g]{吉右衛門}{きちゑもん}はそつと
\ruby{水野}{みづ|の}を
\ruby{見}{み}るに、
\ruby{幸}{さいはひ}にして
\ruby{今}{いま}の
\ruby{言}{ことば}には
\ruby{別}{べつ}に
\ruby{心}{こゝろ}をも
\ruby{動}{うご}かさゞりしやうにて、
\ruby{{\換字{猶}}}{なほ}
\ruby{默々}{もく|〳〵}と
\ruby{栗}{くり}を
\ruby{剝}{む}きつゞけ
\ruby{居}{を}れば、やうやく
\ruby{自{\換字{分}}}{おの|れ}も
\ruby{安}{やす}き
\ruby{思}{おもひ}して、

『イヤ、
\ruby{老夫}{おぢい|さん}には
\ruby{其樣}{そん|な}な
\ruby{氣}{き}は
\ruby{仕}{し}ないよ。
\ruby{五十子}{い|そ|こ}さんが
\ruby{{\換字{遠}}}{とほ}いところへ
\ruby{行}{い}つて
\ruby{仕舞}{し|ま}ふなんて、そりやあ
\ruby{汝}{おまへ}が
\ruby{魯敏孫}{ろ|びん|そん}とかの
\ruby{書}{ほん}を
\ruby{讀}{よ}んだせいで、そんな
\ruby{下}{くだ}らない
\ruby{事}{こと}を
\ruby{思}{おも}ひついたんだらう。
\ruby{老夫}{おぢい|さん}はまた
\ruby{五十子}{い|そ|こ}さんが
\ruby{癒}{なほ}つて、
\ruby{松}{まつ}ちやんだの、
\ruby{汝}{おまへ}だの、
\ruby{島木}{しま|き}さんだのと、みんなが
\ruby{賑}{にぎ}やかに
\ruby{{\換字{遊}}}{あそ}ぶ
\ruby{事}{こと}が、あるやうに
\ruby{思}{おも}つて
\ruby{居}{ゐ}るよ。
』

と
\ruby{老人}{とし|より}の
\ruby{思}{おも}ひ
\ruby{{\換字{遣}}}{や}り
\ruby{深}{ふか}くも
\ruby{祝}{いは}ひ
\ruby{直}{なほ}したり。

\ruby{賢}{かしこ}けれども
\ruby{{\換字{猶}}}{なほ}
\ruby{年{\換字{若}}}{とし|わか}ければ、
\ruby{言外}{げん|ぐわい}の
\ruby{其意}{その|こゝろ}は
\ruby{汲}{く}みて
\ruby{知}{し}るに
\ruby{由無}{よし|な}く、

『イヽエ、ちつとも
\ruby{漂流記}{へう|りう|き}の
\ruby{故}{せい}ぢやあ
\ruby{無}{な}いわ。
\ruby{{\換字{過}}日}{こな|ひだ}
\ruby{松}{まつ}ちやんと
\ruby{二人}{ふた|り}で、あの
\ruby{椎}{しひ}の
\ruby{樹}{き}の
\ruby{蔭}{かげ}で
\ruby{話}{はなし}を
\ruby{仕}{し}た
\ruby{其時}{その|とき}から、
\ruby{何}{なん}となく
\ruby{其樣}{そ|ん}な
\ruby{氣}{き}が
\ruby{仕}{し}はじめたのよ。
\ruby{御爺}{お|ぢい}さんこそ
\ruby{屹度}{きつ|と}
\ruby{二筋{\換字{道}}}{ふた|すぢ|みち}が
\ruby{贔負}{ひゐ|き}だから、
\ruby{彼}{あ}の
\ruby{本}{ほん}のやうになるとばつかし
\ruby{考}{かんが}へて
\ruby{居}{ゐ}るんだわ。
』

とお
\ruby{濱}{はま}が
\ruby{負}{ま}けじ
\ruby{心}{ごゝろ}に
\ruby{云}{い}ひ
\ruby{爭}{あらそ}ふ
\ruby{時}{とき}、
\ruby{今}{いま}まで
\ruby{傍目訝}{よそ|め|いぶか}しきまで
\ruby{沈着}{おち|つき}に
\ruby{沈着}{おち|つ}き
\ruby{居}{ゐ}し
\ruby{水野}{みづ|の}は、

『どつちでもまあ
\ruby{宣}{い}いぢやあ
\ruby{無}{な}いか
お
\ruby{濱}{はま}ちやん!。
\ruby{明日}{あし|た}の
\ruby{事}{こと}は
\ruby{明日}{あし|た}の
お
\ruby{天{\換字{道}}樣}{てん|たう|さま}が
\ruby{見}{み}せて
\ruby{下}{くだ}さるわ\換字{子}。
ハヽヽ。
』

と
\ruby{悲}{かな}しげにも
\ruby{無}{な}ければ
\ruby{嬉}{うれ}しげにも
\ruby{無}{な}く、もとより
\ruby{可笑}{を|か}しげにもあらぬ
\ruby{聲}{こゑ}して
\ruby{笑}{わら}ひつゝ
\ruby{制}{せい}し、
\ruby{{\換字{又}}}{また}その
\ruby{掌}{て}の
\ruby{上}{うへ}に
\ruby{剝}{む}きたる
\ruby{栗一}{くり|ひと}ツを、
\ruby{食}{た}べよとばかり
\ruby{優}{やさ}しく
\ruby{置}{お}き
\ruby{{\換字{遣}}}{や}りたり。

『コレ
\ruby{何}{なん}だ!。
\ruby{剝}{む}いたのを
\ruby{先生}{せん|せい}に
\ruby{戴}{いたゞ}くといふものがあるものか。
』

と
\ruby[g]{吉右衛門}{きちゑもん}が
\ruby{眼}{め}の
\ruby{見}{み}つけて
\ruby{叱}{しか}れるは
\ruby{遲}{おそ}く
\ruby{{\換字{緩}}}{ゆる}く、

『いゝわ\換字{子}エ、
\ruby{先生}{せん|せい}!、
\ruby{戴}{いたゞ}いたつて。
』

と
\ruby{云}{い}へる
\ruby{答}{こたへ}は
\ruby{短}{みじか}く
\ruby{捷}{はや}くして、
\ruby{栗}{くり}は
\ruby{既}{すで}に
\ruby{滿面}{まん|めん}に
\ruby{笑}{わらひ}を
\ruby{盛}{も}れる
お
\ruby{濱}{はま}が
\ruby{口裏}{く|ち}に
\ruby{隱}{かく}れたり。

されど
\ruby{何}{なん}としけん
お
\ruby{濱}{はま}は
\ruby{忽地}{たちま|ち}にして、
\ruby{其}{そ}の
\ruby{美}{うつく}しき
\ruby{眉}{まゆ}を
\ruby{顰}{ひそ}むれば、

『いゝ
\ruby{氣味}{き|み}、いゝ
\ruby{氣味}{き|み}!。
\ruby{蟲}{むし}が
\ruby{居}{ゐ}たと
\ruby{見}{み}える。
』

と
\ruby{樣子}{やう|す}を
\ruby{見}{み}て
\ruby{取}{と}つて
\ruby[g]{吉右衛門}{きちゑもん}は
\ruby{可笑}{を|か}しがりて
\ruby{笑}{わら}ひ
\ruby{崩}{くづ}れぬ。
\ruby{蟲}{むし}はあらぬ
\ruby{筈}{はず}なるを
\ruby{不思議}{ふ|し|ぎ}の
\ruby{事}{こと}かなと、
\ruby{水野}{みづ|の}は
\ruby{氣}{き}の
\ruby{毒}{どく}さに
お
\ruby{濱}{はま}を
\ruby{打護}{うち|まも}れば、
お
\ruby{濱}{はま}はまた
\ruby{物}{もの}を
\ruby{捜}{さぐ}るが
\ruby{如}{ごと}くに
\ruby{水野}{みづ|の}が
\ruby{手先}{て|さき}に
\ruby{眼}{め}を
\ruby{注}{そゝ}ぎ
\ruby{居}{ゐ}しが、やがて
\ruby{口}{くち}の
\ruby{中}{なか}の
\ruby{物}{もの}を
\ruby{嚥}{の}み
\ruby{{\換字{終}}}{しま}ひて
\ruby{後}{のち}、
\ruby{水野}{みづ|の}が
\ruby{手}{て}をば
\ruby{突然}{いき|なり}
\ruby{取}{と}りて、

『
\ruby{先生}{せん|せい}、
\ruby{負傷}{け|が}をして
\ruby{居}{ゐ}てよ!。
\ruby{痛}{いた}くなくつて。
』

と
\ruby{示}{しめ}したるを
\ruby{見}{み}れば、
\ruby{左}{ひだり}の
\ruby{拇指}{おや|ゆび}の
\ruby{其腹}{その|はら}に、
\ruby{鮮血}{せん|けつ}いさゝかにじみて
\ruby{臙脂微}{\換字{𛀁}ん|じ|かすか}に
\ruby{湧}{わ}けり。
\ruby{何}{なに}に
\ruby{心}{こゝろ}をとられて、
\ruby{何時}{い|つ}の
\ruby{間}{ま}にか
\ruby{{\換字{過}}}{あやま}つて
\ruby{傷}{きず}つけて、しかも
\ruby{今}{いま}までは
\ruby{知}{し}らざりけん、
\ruby{全}{まつた}く
\ruby{聊}{いさゝか}か
\ruby{此血}{この|ち}の
\ruby{着}{つ}きたるに
お
\ruby{濱}{はま}は
\ruby{栗}{くり}の
\ruby{味}{あぢはい}を
\ruby{怪}{あやし}みたるなり。

『アヽ
\ruby{穢}{きたな}い
\ruby{事}{こと}をした
\ruby{惡}{わる}かつた!。
\ruby{勘{\換字{忍}}}{か|に}しておくれよ% 原文通り「勘忍」
お
\ruby{濱}{はま}ちやん。
ほんとに
\ruby{毫}{すこし}も
\ruby{知}{し}らなかつたのだから。
』

『ナアニ
\ruby{毫}{ちつと}も
\ruby{穢}{きたな}かあ
\ruby{無}{な}いわ。
\ruby{最初妾}{さい|しよ|わたし}が
\ruby{血}{ち}の
\ruby{着}{つ}いたのをあげるなんて、
\ruby{{\換字{縁}}起}{\換字{𛀁}ん|ぎ}でも
\ruby{無}{な}い
\ruby{事}{こと}を
\ruby{云}{い}つたから
\ruby{惡}{わる}かつたのよ。
』

\ruby{瑣細}{さ|さい}の
\ruby{事}{こと}なれど、
\ruby{今}{いま}まで
\ruby{賑}{にぎ}やかに
\ruby{語}{かた}らひし
\ruby{談話}{はな|し}の
\ruby{腰}{こし}はこれに
\ruby{砕}{くだ}けて、
\ruby{何}{なん}となく
\ruby{淋}{さび}しく
\ruby{白}{しら}けたる
\ruby{一室}{ひと|ま}の
\ruby{内}{うち}には、
\ruby{今}{いま}
\ruby{沸}{たぎ}り
\ruby{初}{そ}めでも
\ruby{仕}{し}たるやうに
\ruby{鐵瓶}{てつ|びん}の
\ruby{煮}{に}ゆる
\ruby{音}{おと}の
\ruby{幽}{かす}かに
\ruby{響}{ひゞ}き
\ruby{出}{だ}して、
\ruby{靜}{しづ}まりかへつたる
\ruby{村}{むた}の
\ruby{夜}{よる}の
\ruby{中}{なか}を、
\ruby{澁江村}{し|ぶ|\換字{𛀁}}との
\ruby{境界}{さ|かひ}あたりにや
\ruby{狗}{いぬ}の
\ruby{吠}{ほ}ゆるが、べう〳〵として
\ruby{遙}{はるか}に
\ruby{聞}{きこ}えぬ。

