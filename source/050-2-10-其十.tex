\Entry{其十}

% メモ 校正終了 2024-04-17 2024-05-30 2024-06-30
\原本頁{56-7}%
\ruby{水野}{みづ|の}
\ruby{語}{かた}らず
\ruby{吉右衛門}{きち||ゑ|もん}
\ruby{言}{ものい}はず、
%
\ruby{瞬}{またゝ}かざる% 踊り字調整「〻(二の字点、揺すり点)に見えるが(ゝ)」
\ruby{燈火}{とも|しび}の
\ruby[||j>]{光}{ひかり}% ルビの重なりを是正
\ruby[||j>]{白}{ しろ}
\ruby[||j>]{々}{ 〴〵}と
\ruby{冷}{ひや}やかに
\ruby{照}{て}らす
ところ、
%
お
\ruby{濱}{はま}が
\ruby{眼}{め}の
\ruby{{\換字{前}}}{まへ}に
\ruby{動}{うご}けるものは、
%
\ruby{水野}{みづ|の}が
\ruby{指端}{ゆび|さき}を
\ruby{卷}{ま}きたる
\ruby{白紙}{か|み}に、
%
\ruby{知}{し}れるか
\ruby{知}{し}れぬほどづゝ% 踊り字調整「〻(二の字点、揺すり点)に見えるが(ゝ)」
じりゝ〳〵と、
%
\ruby{浸潤}{に|じ}み
\ruby{出}{いだ}して
\ruby{廣}{ひろ}がり
\ruby{行}{ゆ}く
\ruby{鮮血}{せん|けつ}の
\ruby{紅色}{あか|き}のみ。

\原本頁{57-1}%
\ruby{淋}{さみ}しさは
\ruby{今}{いま}
\ruby{人々}{ひと|〴〵}を
\ruby{包}{つゝ}みぬ。% 踊り字調整「〻(二の字点、揺すり点)に見えるが(ゝ)」
%
べう〳〵と
\ruby{鳴}{な}く
\ruby{狗}{いぬ}の
\ruby{聲}{こゑ}は、
%
また
\ruby{遙}{はるか}に
\ruby{{\換字{遠}}}{とほ}くより
こゝに% 踊り字調整「〻(二の字点、揺すり点)に見えるが(ゝ)」
\ruby{聞}{きこ}え
\ruby{來}{き}ぬ。

\原本頁{57-3}%
お
\ruby{濱}{はま}は
\ruby{{\換字{終}}}{つひ}に
\ruby{淋}{さみ}しさに
\ruby{堪}{た}へ
かねてや、
%
\ruby{心細}{こゝろ|ぼそ}けなる% 踊り字調整「〻(二の字点、揺すり点)に見えるが(ゝ)」
\ruby{面色}{おも|ゝち}して、% 踊り字調整「〻(二の字点、揺すり点)に見えるが(ゝ)」

\原本頁{57-4}%
『
あの
\ruby{狗}{いぬ}は
ほんとうに
\ruby{可厭}{い|や}な
\ruby{狗}{いぬ}\換字{子}エー。
%
\ruby{{\換字{過}}日}{こな|ひだ}
\ruby{先生}{せん|せい}が
\ruby{出}{で}て
\ruby{行}{いら}つしやつた
\ruby{夜}{よる}も、
%
\ruby{矢張}{やつ|ぱ}り
\ruby{彼}{あ}の
\ruby{{\換字{通}}}{とほ}りの
\ruby{聲}{こゑ}をして、
%
\ruby{彼}{あ}の
\ruby{見當}{けん|たう}で
\ruby{鳴}{な}いて
\ruby{居}{ゐ}たのよ。
%
そして
\ruby{其}{その}
\ruby{時}{とき}
しーんとして
\ruby{聞}{きい}て
\ruby{居}{ゐ}たらば、
%
\ruby{妾}{わたし}
なんだか
\ruby{悲}{かな}あしく
なつて、
%
\ruby{大變}{たい|へん}に
\ruby{妙}{めう}な
\ruby[||j>]{心}{こゝろ}% 踊り字調整「〻(二の字点、揺すり点)に見えるが(ゝ)」
\ruby[||j>]{持}{ もち}
がしたのよ。
』

\原本頁{57-8}%
と
\ruby{云}{い}ひ
\ruby{出}{いだ}せば、

\原本頁{57-9}%
『
また
\ruby{何}{なに}か
\ruby{下}{くだ}らない
\ruby{事}{こと}を
いふ!。
』

\原本頁{57-10}%
と
\ruby{吉右衛門}{きち||ゑ|もん}は
\ruby{打{\換字{消}}}{うち|け}し、

\原本頁{57-11}%
『
\ruby{妙}{めう}な
\ruby[||j>]{心}{こゝろ}% 踊り字調整「〻(二の字点、揺すり点)に見えるが(ゝ)」
\ruby[||j>]{持}{ もち}
つて、
%
\ruby{何樣}{ど|ん}な
\ruby[||j>]{心}{こゝろ}% 踊り字調整「〻(二の字点、揺すり点)に見えるが(ゝ)」
\ruby[||j>]{持}{ もち}
?。
』

\原本頁{58-1}%
と、
%
\ruby{水野}{みづ|の}は
\ruby{談話}{はな|し}に
\ruby{話}{はな}し
\ruby{甲{\換字{斐}}}{が|ひ}
あらしめんとの
\ruby{意}{こゝろ}% 踊り字調整「〻(二の字点、揺すり点)に見えるが(ゝ)」
ばかりに、
%
\ruby{問}{と}はでもの
\ruby{事}{こと}とは
\ruby{思}{おも}ひながら
\ruby{問}{と}ひ
\ruby{{\換字{返}}}{かへ}しぬ。

\原本頁{58-3}%
『
あの\換字{子}、
%
\ruby{疇昔}{むか|し}\換字{子}、
%
\ruby{妾}{わたし}が
ずつと
\ruby{小}{ちひさ}かつた
\ruby{時}{とき}%
{---}{---}%
まだ
\ruby{三歳}{みつ|ゝ}% 踊り字調整「〻(二の字点、揺すり点)に見えるが(ゝ)」
\ruby{四歳}{よつ|ゝ}% 踊り字調整「〻(二の字点、揺すり点)に見えるが(ゝ)」
で
\改行% 校正作業の簡略化のため
、
%
\原本頁{58-4}\改行%
\ruby{妾}{わたし}の
\ruby{眞實}{ほん|とう}の
\ruby{御母}{お|つか}さんが
\ruby{生}{い}きて
\ruby{居}{ゐ}た
\ruby{時}{とき}に\換字{子}、
%
\ruby{妾}{わたし}が
お
\ruby{母}{かつ}さん
\footnote{「母」のルビを(かつ)としているのはここのみで違和感はあるが原本通りとする
(国会図書館 コマ番号34/160 p-058 l-04)}%
に
\ruby{抱}{だ}かれて
うと〳〵として
\ruby{居}{ゐ}ると、
%
\ruby{{\換字{遠}}}{とほ}くの
\ruby{{\換字{遠}}}{とほ}くの% ルビ調整(原本通り)非踊り字表記
\ruby{方}{はう}で
もつて
\ruby{狗}{いぬ}の
\ruby{鳴}{な}いたのが
\ruby{聞}{きこ}えたのよ。
%
まあ
\ruby{左樣}{さ|う}いふことが
\ruby{有}{あ}つたのだと
\ruby{思}{おも}つて
\ruby[<j||]{頂}{ちやう}% 行末行頭の境界付近なので特例処置を施す
\ruby[||j>]{戴}{だい}よ。
% \ruby{頂戴}{ちやう|だい}よ。
%
そいで\換字{子}エ、
%
\ruby{{\換字{過}}日}{こな|ひだ}の
\ruby{夜}{よる}
あの
\ruby{狗}{いぬ}の
\ruby{聲}{こゑ}を
\ruby{聞}{き}いて
\ruby{思}{おも}ひ
\ruby{出}{だ}して
\ruby{見}{み}ると、
%
あの
\ruby{狗}{いぬ}は
やつぱり
\ruby{其}{そ}の
\ruby{時}{とき}の
\ruby{狗}{いぬ}で、
%
あの
\ruby{聲}{こゑ}も
やつぱり
\ruby{當時}{その|とき}の
\ruby{聲}{こゑ}で、
%
\ruby{而}{さう}して
\ruby{彼}{あ}の
\ruby{狗}{いぬ}の
\ruby{聲}{こゑ}を
\ruby{聞}{き}いて、
%
\ruby{可厭}{い|やー}に
\ruby{淋}{さみ}しいと
\ruby{思}{おも}つた
\ruby{其}{そ}の
\ruby[||j>]{心}{こゝろ}% 踊り字調整「〻(二の字点、揺すり点)に見えるが(ゝ)」
\ruby[||j>]{持}{ もち}
も、
%
やつぱり
\ruby{其}{そ}の
\ruby{時}{とき}
\ruby{可厭}{い|やー}に
\ruby{淋}{さみ}しいと
\ruby{思}{おも}つた
\ruby{其}{そ}の
\ruby[||j>]{心}{こゝろ}% 踊り字調整「〻(二の字点、揺すり点)に見えるが(ゝ)」
\ruby[||j>]{持}{ もち}
だと
\改行% 校正作業の簡略化のため
、
%
\原本頁{58-11}\改行%
\ruby{思}{おも}へて〳〵
\ruby{仕方}{し|かた}が
\ruby{無}{な}かつたのよ。
』

\原本頁{59-1}%
『
なんだエ、
%
また
\ruby{下}{くだ}らない!。
%
そりやあ
\ruby{氣}{き}の
\ruby{{\換字{所}}爲}{せ|ゐ}と
いふものだ
\改行% 校正作業の簡略化のため
は。
』

\原本頁{59-3}%
\ruby{吉右衛門}{きち||ゑ|もん}が
かく
\ruby{云}{い}ひ
\ruby{{\換字{終}}}{をは}れる
\ruby{時}{とき}、
%
\ruby{狗}{いぬ}は
また
\ruby{遙}{はるか}に
べう〳〵と
\ruby{鳴}{な}けり
\改行% 校正作業の簡略化のため
。

\原本頁{59-4}%
『
ほーら
\ruby{{\換字{又}}}{また}
\ruby{鳴}{な}いてよ
お
\ruby{爺}{ぢい}さん!。
%
\ruby{氣}{き}の
\ruby{{\換字{所}}爲}{せ|ゐ}ぢやあ
\ruby{無}{な}くつてよ
\ruby{眞實}{ほん|と}の
\ruby{事}{こと}よ!。
%
\ruby{今}{いま}
\ruby{鳴}{な}いた
\ruby{彼狗}{あ|れ}は
\ruby{何樣}{ど|う}しても
\ruby{{\換字{過}}日}{こな|ひだ}
\ruby{鳴}{な}いたのよ。
%
\ruby{{\換字{過}}日}{こな|ひだ}
\原本頁{59-6}\改行%
\ruby{鳴}{な}いた
\ruby{彼狗}{あ|れ}は
また
\ruby{妾}{わたし}が
\ruby{大變}{たい|へん}に
\ruby{小}{ちひさ}かつた
\ruby{時}{とき}
\ruby{鳴}{な}いたのかも
\ruby{知}{し}れなく
\改行% 校正作業の簡略化のため
つてよ!。
%
\ruby{而}{さう}して
\ruby{何}{なん}だか
\ruby{妾}{わたし}あ、
%
\ruby{妾}{わたし}の
\ruby{{\換字{前}}}{まへ}の
\ruby{世}{よ}といふ
\ruby{時}{とき}にも、
%
\ruby{矢張}{やつ|ぱ}
\改行% 校正作業の簡略化のため
り
\ruby{此樣}{こ|ん}な
\ruby{淋}{さみ}しい
\ruby{晩}{ばん}に、
%
やつぱり
\ruby{彼樣}{あ|ん}な
\ruby{狗}{いぬ}の
\ruby{聲}{こゑ}を
\ruby{聞}{き}いて、
%
やつぱり
\ruby{妙}{めう}な
\ruby[||j>]{心}{こゝろ}% 踊り字調整「〻(二の字点、揺すり点)に見えるが(ゝ)」
\ruby[||j>]{持}{ もち}
が
\ruby{爲}{し}たやうな
\ruby{氣}{き}が
\ruby{仕}{し}てならないのよ!。
%
あゝ% 踊り字調整「〻(二の字点、揺すり点)に見えるが(ゝ)」
\ruby{何}{なん}だか
\原本頁{59-10}\改行%
\ruby{妾}{わたし}あ
ぞく〳〵するやうな
\ruby[||j>]{心}{こゝろ}% 踊り字調整「〻(二の字点、揺すり点)に見えるが(ゝ)」
\ruby[||j>]{持}{ もち}
がして、
%
\ruby{變}{へん}に
\ruby{氣味}{き|み}が
\ruby{惡}{わる}くなつて
\ruby{來}{き}て
\ruby{堪}{たま}らないのよ。
%
あら
また
\ruby{鳴}{な}くのネエ、
%
あゝ、% 踊り字調整「〻(二の字点、揺すり点)に見えるが(ゝ)」
%
\ruby{厭}{いや}だこと!。
%
\ruby{萬一}{ひよ|つと}
すると
\ruby{眞實}{ほん|と}に
\ruby{{\換字{前}}}{まへ}の
\ruby{世}{よ}つて
ことが
\ruby{有}{あ}るんぢや
\ruby{無}{な}いか
\ruby{知}{し}らん。
%
\ruby{{\換字{前}}}{まへ}
\原本頁{60-2}\改行%
の
\ruby{世}{よ}つて
いふものが
あるかと
\ruby{思}{おも}ふと、
%
\ruby{何}{なん}だか
\ruby{怖}{こは}いやうな
\ruby{氣}{き}が
するのネエ。
%
\ruby{先生}{せん|せい}は
\ruby{{\換字{前}}}{まへ}の
\ruby{世}{よ}の
あるやうな
\ruby[||j>]{心}{こゝろ}% 踊り字調整「〻(二の字点、揺すり点)に見えるが(ゝ)」
\ruby[||j>]{持}{ もち}
は
\ruby{仕}{し}なくつて?。
』

\原本頁{60-4}%
お
\ruby{濱}{はま}が
かく
\ruby{云}{い}ひたる
\ruby{時}{とき}の
\ruby{其}{そ}の
\ruby{面}{おもて}は、
%
\ruby[<j>]{僞}{いつはり}ならず
\ruby{惑}{まどひ}を
\ruby{帶}{お}び
\ruby{怖畏}{おそ|れ}を
\ruby{帶}{お}びて、
%
まことに
\ruby{{\換字{前}}世}{ぜん|せ}と
いふものゝ% 踊り字調整「〻(二の字点、揺すり点)に見えるが(ゝ)」
\ruby{{\換字{空}}}{むな}しからぬを
\ruby{{\換字{感}}}{かん}じて、
%
\ruby{其}{そ}の
\ruby{恐}{おそ}ろしさに
\ruby{魘}{おび}えたるが
\ruby{如}{ごと}し。

\原本頁{60-7}%
\ruby{實}{げ}に
\ruby{思}{おも}へば
\ruby{人}{ひと}は
\ruby{或}{ある}
\ruby{事}{こと}に
あへる
\ruby{時}{とき}、
%
かゝる% 踊り字調整「〻(二の字点、揺すり点)に見えるが(ゝ)」
\ruby{事}{こと}には
\ruby{往時}{むか|し}
\ruby{既}{すで}に
\ruby{一度}{ひと|たび}
\ruby{逢}{あ}ひたる
ことの
ありしと、
%
\ruby{思}{おも}はるゝ% 踊り字調整「〻(二の字点、揺すり点)に見えるが(ゝ)」
やうなる
\ruby{心地}{こゝ|ち}の% 踊り字調整「〻(二の字点、揺すり点)に見えるが(ゝ)」
\ruby{爲}{す}る
\ruby{事}{こと}も
\ruby{無}{な}きには
あらぬなり。
%
\ruby{既}{すで}に
\ruby{{\換字{兼}}好}{けん|かう}は
\ruby[||j>]{幾}{いく}
\ruby[||j>]{百}{ひやく}
\ruby[||j>]{年}{ねん}
の
\ruby{昔}{むかし}に、

\原本頁{60-10}%
% \begin{quote}% 原本では引用インデントされていない
『
\ruby{只}{たゞ}% 踊り字調整「〻(二の字点、揺すり点)に濁点に見えるが(ゞ)」
\ruby{今}{いま}
\ruby{人}{ひと}の
いふことも、
%
\ruby{目}{め}に
\ruby{見}{み}ゆる
ものも、
%
\ruby{我}{わ}が
\ruby{心}{こゝろ}の% 踊り字調整「〻(二の字点、揺すり点)に見えるが(ゝ)」
うちも、
%
かゝることの% 踊り字調整「〻(二の字点、揺すり点)に見えるが(ゝ)」
\ruby{何時}{い|つ}ぞや
\ruby{有}{あ}りしかと
おぼ{\換字{𛀁}}て、
%
いつとは
\ruby{思}{おも}ひ
\ruby{出}{い}でねども、
%
まさしく
ありし
\ruby{心地}{こゝ|ち}のする% 踊り字調整「〻(二の字点、揺すり点)に見えるが(ゝ)」
』
% \end{quote}% 原本では引用インデントされていない
% 徒然草(上)第71段「名を聞くより、やがて、面影は推し測らるゝ心地するを、‥」 の一説
% 今起こっていること、人の言っている事、目に見ているものなど、いつかもあったり、見たりしているというような思い。

\原本頁{61-2}%
とは
\ruby{云}{い}ひたらずや。

\原本頁{61-3}%
\ruby{生}{うま}れぬ
\ruby{{\換字{前}}}{まへ}の
\ruby{世}{よ}の
\ruby{有}{ある}
\ruby{無}{なし}なんどは、
%
もとより
\ruby{凡下}{ぼん|げ}の
\ruby{身}{み}の
\ruby{何}{なん}とも
\ruby{知}{し}らねば、
%
\ruby{吉右衛門}{きち||ゑ|もん}も
\ruby{横合}{よこ|あひ}よりは
\ruby{吻}{くち}を
\ruby{容}{い}れず、
%
\ruby{水野}{みづ|の}は
\ruby{物}{もの}を
\ruby{思}{おも}ひて
\ruby{{\換字{猶}}}{なほ}
\ruby{語}{かた}らざる
\ruby{時}{とき}、
%
ふたゝび% 踊り字調整「〻(二の字点、揺すり点)に見えるが(ゝ)」
べう〳〵と
\ruby{鳴}{な}く、
%
\ruby{狗}{いぬ}の
\ruby{聲}{こゑ}は、

\原本頁{61-6}%
% \begin{quote}% 原本では引用インデントされていない
『
\ruby{我}{われ}は
\ruby{方々}{かた|〴〵}の
\ruby{{\換字{前}}}{まへ}の
\ruby{世}{よ}より
\ruby{既}{すで}に
\ruby{知}{し}りたまへる
\ruby{狗}{いぬ}なるをや!。
』
% \end{quote}% 原本では引用インデントされていない

\原本頁{61-7}%
と
\ruby{告}{つ}ぐるが
\ruby{如}{ごと}くに
\ruby{聞}{きこ}え
\ruby{來}{きた}りぬ。
