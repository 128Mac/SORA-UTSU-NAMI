\Entry{其十}

\ruby{水野}{みづ|の}
\ruby{語}{かた}らず
\ruby{吉右衛門}{き|ち|ゑ|もん }
\ruby{言}{ものい}はず、
\ruby{瞬}{またゝ}かざる
\ruby{燈火}{とも|しび}の
\ruby{光白々}{ひかり|しろ|〴〵}と
\ruby{冷}{ひや}やかに
\ruby{照}{て}らすところ、
お
\ruby{濱}{はま}が
\ruby{眼}{め}の
\ruby{前}{まへ}に
\ruby{動}{うご}けるものは、
\ruby{水野}{みづ|の}が
\ruby{指端}{ゆび|さき}を
\ruby{{\換字{巻}}}{ま}きたる
\ruby{白紙}{か|み}に、
\ruby{知}{し}れるか
\ruby{知}{し}れぬほどづゝじりゝ〳〵と、
\ruby{浸潤}{に|じ}み
\ruby{出}{いだ}して
\ruby{廣}{ひろ}がり
\ruby{行}{ゆ}く
\ruby{鮮血}{せん|けつ}の
\ruby{紅色}{あか|き}のみ。

\ruby{淋}{さみ}しさは
\ruby{今}{いま}
\ruby{人々}{ひと|〴〵}を
\ruby{包}{つゝ}みぬ。
べう〳〵と
\ruby{鳴}{な}く
\ruby{狗}{いぬ}の
\ruby{聲}{こゑ}は、また
\ruby{遙}{はる}かに
\ruby{{\換字{遠}}}{とほ}くよりこゝに
\ruby{聞}{きこ}え
\ruby{來}{き}ぬ。

お
\ruby{濱}{はま}は
\ruby{{\換字{終}}}{つひ}に
\ruby{淋}{さみ}しさに
\ruby{堪}{た}へかねてや、
\ruby{心細}{こゝろ|ぼそ}けなる
\ruby{面色}{おも|ゝち}して、

『あの
\ruby{狗}{いぬ}はほんとうに
\ruby{可厭}{い|や}な
\ruby{狗}{いぬ}\換字{子}エー。
\ruby{{\換字{過}}日先生}{こな|いだ|せん|せい}が
\ruby{出}{で}て
\ruby{行}{いら}つしやつた
\ruby{夜}{よる}も、
\ruby{矢張}{やつ|ぱ}り
\ruby{彼}{あ}の
\ruby{{\換字{通}}}{とほ}りの
\ruby{聲}{こゑ}をして、
\ruby{彼}{あ}の
\ruby{見當}{けん|たう}で
\ruby{鳴}{な}いて
\ruby{居}{ゐ}たのよ。
そして
\ruby{其時}{その|とき}しーんとして
\ruby{聞}{き}いて
\ruby{居}{ゐ}たらば、
\ruby{妾}{わたし}なんだか
\ruby{悲}{かな}あしくなつて、
\ruby{大變}{たい|へん}に
\ruby{妙}{めう}な
\ruby{心持}{こゝろ|もち}がしたのよ。
』

と
\ruby{云}{い}ひ
\ruby{出}{いだ}せば、

『また
\ruby{何}{なに}か
\ruby{下}{くだ}らない
\ruby{事}{こと}をいふ!。
』

と
\ruby{吉右衛門}{き|ち|ゑ|もん}は
\ruby{打{\換字{消}}}{うち|け}し、

『
\ruby{妙}{めう}な
\ruby{心持}{こゝろ|もち}つて、
\ruby{何樣}{ど|ん}な
\ruby{心持}{こゝろ|もち}?。
』

と、
\ruby{水野}{みづ|の}は
\ruby{談話}{はな|し}に
\ruby{話}{はな}し
\ruby{甲斐}{が|ひ}あらしめんとの
\ruby{意}{こゝろ}ばかりに、
\ruby{問}{と}はでもの
\ruby{事}{こと}とは
\ruby{思}{おも}ひながら
\ruby{問}{と}ひ
\ruby{{\換字{返}}}{かへ}しぬ。

『あの\換字{子}、
\ruby{疇昔}{むか|し}\換字{子}、
\ruby{妾}{わたし}がずつと
\ruby{小}{ちひさ}かつた
\ruby{時}{とき}\------ まだ
\ruby{三歳四歳}{み|つゝ|よ|つゝ}で、
\ruby{妾}{わたし}の
\ruby{眞實}{ほん|と}の
\ruby{御母}{お|つか}さんが
\ruby{生}{い}きて
\ruby{居}{ゐ}た
\ruby{時}{とき}に\換字{子}、
\ruby{妾}{わたし}が
お
\ruby{母}{かつ}さんに
\ruby{抱}{だ}かれてうと〳〵として
\ruby{居}{ゐ}ると、
\ruby{{\換字{遠}}}{とほ}くの
\ruby{{\換字{遠}}}{とほ}くの
\ruby{方}{はう}でもつて
\ruby{狗}{いぬ}の
\ruby{鳴}{な}いたのが
\ruby{聞}{きこ}えたのよ。
まあ
\ruby{左樣}{さ|う}いふことが
\ruby{有}{あ}つたのだと
\ruby{思}{おも}つて
\ruby{頂戴}{ちやう|だい}よ。
そいで\換字{子}エ、
\ruby{{\換字{過}}日}{こな|いだ}の
\ruby{夜}{よる}あの
\ruby{狗}{いぬ}の
\ruby{聲}{こゑ}を
\ruby{聞}{き}いて
\ruby{思}{おも}ひ
\ruby{出}{だ}して
\ruby{見}{み}ると、あの
\ruby{狗}{いぬ}はやつぱり
\ruby{其}{そ}の
\ruby{時}{とき}の
\ruby{狗}{いぬ}で、あの
\ruby{聲}{こゑ}もやつぱり
\ruby{當時}{その|とき}の
\ruby{聲}{こゑ}で、
\ruby{而}{さう}して
\ruby{彼}{あ}の
\ruby{狗}{いぬ}の
\ruby{聲}{こゑ}を
\ruby{聞}{き}いて、
\ruby{可厭}{い|やー}に
\ruby{淋}{さみ}しいと
\ruby{思}{おも}つた
\ruby{其}{そ}の
\ruby{心持}{こゝろ|もち}も、やつぱり
\ruby{其}{そ}の
\ruby{時}{とき}
\ruby{可厭}{い|やー}に
\ruby{淋}{さみ}しいと
\ruby{思}{おも}つた
\ruby{其}{そ}の
\ruby{心持}{こゝろ|もち}だと、
\ruby{思}{おも}へて〳〵
\ruby{仕方}{し|かた}が
\ruby{無}{な}かつたのよ。
』

『なんだエ、また
\ruby{下}{くだ}らない!。
そりやあ
\ruby{氣}{き}の
\ruby{{\換字{所}}爲}{せ|ゐ}といふものだは。
』

\ruby{吉右衛門}{き|ち|ゑ|もん}がかく
\ruby{云}{い}ひ
\ruby{{\換字{終}}}{をは}れる
\ruby{時}{とき}、
\ruby{狗}{いぬ}はまた
\ruby{遙}{はるか}にべう〳〵と
\ruby{鳴}{な}けり。

『ほーら
\ruby{{\換字{又}}}{また}
\ruby{鳴}{な}いてよ
お
\ruby{爺}{ぢい}さん!。
\ruby{氣}{き}の
\ruby{{\換字{所}}爲}{せ|ゐ}ぢやあ
\ruby{無}{な}くつてよ
\ruby{眞實}{ほん|と}の
\ruby{事}{こと}よ!。
\ruby{今}{いま}
\ruby{鳴}{な}いた
\ruby{彼狗}{あ|れ}は
\ruby{何樣}{ど|う}しても
\ruby{{\換字{過}}日}{こな|いだ}
\ruby{鳴}{な}いたのよ。
\ruby{{\換字{過}}日}{こな|いだ}
\ruby{鳴}{な}いた
\ruby{彼狗}{あ|れ}はまた
\ruby{妾}{わたし}が
\ruby{大變}{たい|へん}に
\ruby{小}{ちひさ}かつた
\ruby{時}{とき}
\ruby{鳴}{な}いたのかも
\ruby{知}{し}れなくつてよ!。
\ruby{而}{さう}して
\ruby{何}{なん}だか
\ruby{妾}{わたし}あ、
\ruby{妾}{わたし}の
\ruby{前}{まへ}の
\ruby{世}{よ}といふ
\ruby{時}{とき}にも、
\ruby{矢張}{やつ|ぱ}り
\ruby{此樣}{こ|ん}な
\ruby{淋}{さみ}しい
\ruby{晩}{ばん}に、やつぱり
\ruby{彼樣}{あん|な}な
\ruby{狗}{いぬ}の
\ruby{聲}{こゑ}を
\ruby{聞}{き}いて、やつぱり
\ruby{妙}{めう}な
\ruby{心持}{こゝろ|もち}が
\ruby{爲}{し}たやうな
\ruby{氣}{き}が
\ruby{仕}{し}てならないのよ!。
あゝ
\ruby{何}{なん}だか
\ruby{妾}{わたし}あぞく〳〵するやうな
\ruby{心持}{こゝろ|もち}がして、
\ruby{變}{へん}に
\ruby{氣味}{き|み}が
\ruby{惡}{わる}くなつて
\ruby{來}{き}て
\ruby{堪}{たま}らないのよ。
あらまた
\ruby{鳴}{な}くのネエ、あゝ、
\ruby{厭}{いや}だこと!。
\ruby{萬一}{ひよ|つと}すると
\ruby{眞實}{ほん|と}に
\ruby{前}{まへ}の
\ruby{世}{よ}つていふものがあるかと
\ruby{思}{おも}ふと、
\ruby{何}{なん}だか
\ruby{怖}{こは}いやうな
\ruby{氣}{き}がするのネエ。
\ruby{先生}{せん|せい}は
\ruby{前}{まへ}の
\ruby{世}{よ}のあるやうな
\ruby{心持}{こゝろ|もち}は
\ruby{仕}{し}なくつて?。
』

お
\ruby{濱}{はま}がかく
\ruby{云}{い}ひたる
\ruby{時}{とき}の
\ruby{其}{そ}の
\ruby{面}{おもて}は、
\ruby{僞}{いつはり}ならず
\ruby{惑}{まどひ}を
\ruby{帶}{お}び
\ruby{怖畏}{おそ|れ}を
\ruby{帶}{お}びて、まことに
\ruby{前世}{ぜん|せ}といふものゝ
\ruby{空}{むな}しからぬを
\ruby{感}{かん}じて、
\ruby{其}{そ}の
\ruby{恐}{おそ}ろしさに
\ruby{魘}{おび}えたるが
\ruby{如}{ごと}し。

\ruby{實}{げ}に
\ruby{思}{おも}へば
\ruby{人}{ひと}は
\ruby{或事}{ある|こと}にあへる
\ruby{時}{とき}、かゝる
\ruby{事}{こと}には
\ruby{往時既}{むか|し|すで}に
\ruby{一度逢}{ひと|たび|あ}ひたることのありしと、
\ruby{思}{おも}はるゝやうなる
\ruby{心地}{こゝ|ち}の
\ruby{爲}{す}る
\ruby{事}{こと}も
\ruby{無}{な}きにはあらぬなり。
\ruby{既}{すで}に
\ruby{{\換字{兼}}好}{けん|かう}は
\ruby{幾百年}{いく|ひやく|ねん}の
\ruby{昔}{むかし}に、

\begin{quote}
『
\ruby{只今人}{たゞ|いま|ひと}のいふことも、
\ruby{目}{め}に
\ruby{見}{み}ゆるものも、
\ruby{我}{わ}が
\ruby{心}{こゝろ}のうちも、かゝることの
\ruby{何時}{い|つ}ぞや
\ruby{有}{あ}りしかとおぼ\換字{江}て、いつとは
\ruby{思}{おも}ひ
\ruby{出}{い}で ねども、まさしくありし
\ruby{心地}{こゝ|ち}のする』
\end{quote}

とは
\ruby{云}{い}ひたらずや。

\ruby{生}{うま}れぬ
\ruby{前}{まへ}の
\ruby{世}{よ}の
\ruby{有無}{ある|なし}なんどは、もとより
\ruby{凡下}{ぼん|げ}の
\ruby{身}{み}の
\ruby{何}{なん}とも
\ruby{知}{し}らねば、
\ruby{吉右衛門}{き|ち|ゑ|もん}も
\ruby{横合}{よこ|あひ}よりは
\ruby{吻}{くち}を
\ruby{容}{い}れず、
\ruby{水野}{みづ|の}は
\ruby{物}{もの}を
\ruby{思}{おも}ひて
\ruby{{\換字{猶}}}{なほ}
\ruby{語}{かた}らざる
\ruby{時}{とき}、ふたゝびべう〳〵と
\ruby{鳴}{な}く、
\ruby{狗}{いぬ}の
\ruby{聲}{こゑ}は、

\begin{quote}
『
\ruby{我}{われ}は
\ruby{方々}{かた|〴〵}の
\ruby{前}{まへ}の
\ruby{世}{よ}より
\ruby{既}{すで}に
\ruby{知}{し}りたまへる
\ruby{狗}{いぬ}なるをや!。
』
\end{quote}

と
\ruby{告}{つ}ぐるが
\ruby{如}{ごと}くに
\ruby{聞}{きこ}え
\ruby{來}{きた}りぬ。

