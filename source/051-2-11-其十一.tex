\Entry{其十一}

\原本頁{}%
\ruby{偶然}{ぐう|ぜん}の
\ruby{事}{こと}とすればそれまでなれども、
%
\ruby{奇}{あや}しとすれば
\ruby{奇}{あや}しくもあるかな。
%
かつて
\ruby{我}{わ}が
\ruby{讀}{よ}みし
\ruby{書}{しよ}の
\ruby{中}{うち}に『
\ruby{幻}{ヴイジヨン}と
\ruby{謎}{リツドル}と』といへる
\ruby{一章}{いつ|しやう}ありて、
%
\ruby{其}{そ}の
\ruby{幽怪}{ゆう|くわい}
\ruby{神異}{しん|い}の
\ruby{趣味}{おも|むき}は、
%
\ruby{骨身}{ほね|み}に
\ruby{沁}{し}みて
\ruby{忘}{わす}れ
\ruby{難}{がた}く、
%
\ruby{今}{いま}に
\ruby{鮮明}{あざ|やか}に
\ruby{心頭}{むな|さき}に
\ruby{{\換字{遺}}}{のこ}れる、
%
\ruby{其}{それ}を
お
\ruby{濱}{はま}の
\ruby{知}{し}るべくはあらねど、
%
\ruby{其}{そ}の
\ruby{言}{い}ふところを
\ruby{聞}{き}けば、
%
\ruby{何}{なん}ぞ
\ruby{彼}{か}の
\ruby{記}{しる}せるところと
\ruby{相}{あひ}
\ruby{似}{に}たるや。
%
たゞ
\ruby{彼}{かれ}は
\ruby{考慮}{かん|がへ}に
\ruby{老}{お}いたる
\ruby{人}{ひと}の
\ruby{言葉}{こと|ば}にして、
%
これは
\ruby{何}{なん}の
\ruby{思案}{し|あん}も
\ruby{無}{な}き
\ruby{少女}{こ|ども}の
\ruby{言葉}{こと|ば}なり、
%
\ruby{彼}{かれ}は
\ruby{先}{ま}づ
\ruby{思}{おも}ひて
\ruby{後}{のち}に
\ruby{狗}{いぬ}の
\ruby{聲}{こゑ}を
\ruby{聞}{き}き、
%
これは
\ruby{先}{ま}づ
\ruby{狗}{いぬ}の
\ruby{聲}{こゑ}を
\ruby{聞}{き}いて
\ruby{後}{のち}に
\ruby{思}{おも}ひ
\ruby{起}{おこ}せるの
\ruby{差異}{ちが|ひ}こそあれ、
%
おのづからに
\ruby{此}{こ}の
\ruby{年}{とし}ゆかぬ
\ruby{娘}{こ}の、
%
\ruby{誰}{たれ}
\ruby{敎}{をし}へぬにか〻る
\ruby{事}{こと}を
\ruby{想}{おも}ひ
\ruby{出}{いだ}せる
\ruby{不思議}{ふ|し|ぎ}さ!。
%
\ruby{月日}{つき|ひ}は
\ruby{誰}{たれ}の
\ruby{{\換字{所}}有}{も|の}としも
\ruby{無}{な}ければ、
%
\ruby{仰}{あふ}ぐものは
\ruby{皆}{みな}
\ruby{其}{そ}の
\ruby{光}{ひかり}を
\ruby{見}{み}、
%
\ruby[g]{眞理}{まこと}は
\ruby{智者}{ち|しや}の
\ruby{{\換字{造}}}{つく}れるにもあらねば、
%
\ruby{{\換字{婦}}女}{をん|な}
\ruby{童兒}{こ|ども}の
\ruby{胸}{むね}にも
\ruby{{\換字{浮}}}{うか}みて、
%
\ruby{我}{われ}からとも
\ruby{無}{な}く
\ruby{如是}{か|く}は
\ruby{悟}{さと}れるにや。
%
そも〳〵また
\ruby{佛陀}{ほと|け}の
\ruby{敎法}{をし|へ}に、
%
いつとなく
\ruby{耳}{みゝ}も
\ruby{心}{こゝろ}も
\ruby{染}{そ}まり
\ruby{居}{ゐ}て、
%
それより
\ruby{然}{さ}る
\ruby{事}{こと}をも
\ruby{思}{おも}へるか。

\原本頁{}%
\ruby{其}{そ}の
\ruby{因}{よ}つて
\ruby{出}{い}でしところは
\ruby{兎}{と}まれ
\ruby{角}{かく}かれ、
%
\ruby{{\換字{前}}}{まへ}の
\ruby{世}{よ}
\ruby{有}{あ}りや
\ruby{將}{はた}
\ruby{有}{あ}らずや、
%
\ruby{如何}{い|か}にと
\ruby{問}{と}はれては
\ruby{此}{こ}の
\ruby{我}{われ}もまた、
%
\ruby{少}{すこし}ばかりの
\ruby{智慧}{ち|ゑ}
\ruby{學問}{がく|もん}の、
%
\ruby{果}{はた}して
\ruby{有}{あ}りや
\ruby{{\換字{又}}}{また}
\ruby{無}{な}しやと
\ruby{蜘蛛手}{く|も|で}に
\ruby{働}{はたら}く
\ruby{其}{そ}の
\ruby{下蔭}{した|かげ}に、
%
\ruby{私}{ひそか}に
\ruby{{\換字{前}}}{まへ}の
\ruby{世}{よ}を
\ruby{有}{あ}るもの〻やう
\ruby{思}{おも}ふ
\ruby{心地}{こゝ|ち}も
\ruby{實}{まこと}は
\ruby{爲}{す}るなり。

\原本頁{}%
\ruby{{\換字{迷}}信}{まよ|ひ}なり、
%
\ruby{{\換字{迷}}信}{まよ|ひ}なり、
%
\ruby{{\換字{古}}}{ふる}き
\ruby{{\換字{迷}}信}{まよ|ひ}なり、
%
\ruby{智慧}{ち|ゑ}の
\ruby{光輝}{ひか|り}の
\ruby{及}{およ}ばぬ
\ruby{隈}{くま}には、
%
\ruby{其}{そ}の
\ruby{闇}{くら}さにぞ
\ruby{有}{あ}らぬ
\ruby{現像}{すが|た}の
\ruby{思}{おも}ひ
\ruby{{\換字{遣}}}{や}らる〻、
%
\ruby{其}{それ}を
\ruby{{\換字{前}}}{まへ}の
\ruby{世}{よ}とは
\ruby{云}{い}ひならはしたるならすや。
%
さはあれど、
%
\ruby{彼}{か}の
\ruby{書}{しよ}に、

\原本頁{}%
% \begin{quote}% 原本では引用インデントされていない
『
\ruby[h|]{爾}{なんぢ}
\ruby{見}{み}よ、
%
\ruby{此}{こ}の
\ruby{刹那}{せつ|な}を。
%
\ruby{刹那}{せつ|な}の
\ruby{此}{こ}の
\ruby{關}{せき}より
\ruby[g]{彼方}{かなた}には
\ruby{涯無}{かぎり|な}き
\ruby{路}{みち}の
\ruby{長路}{なが|ぢ}ぞ
\ruby{遙}{はるか}に
\ruby{亘}{わた}れるなる。
%
\ruby{刹那}{せつ|な}の
\ruby{關}{せき}より
\ruby[g]{此方}{こなた}にも
\ruby{涯無}{かぎり|な}き
\ruby{路}{みち}の
\ruby{長路}{なが|ぢ}ぞ
\ruby{遙}{はるか}に
\ruby{亘}{わた}れるなる。

\原本頁{}%
\ruby{思}{おも}へ
\ruby{爾}{なんぢ}、
%
\ruby{起}{おこ}りし
\ruby{事}{こと}のかつて
\ruby{此路}{こ|〻}に
\ruby{起}{おこ}りし
\ruby{事}{こと}ならぬやある?。
%
\ruby{思}{おも}へ
\ruby{爾}{なんぢ}、
%
\ruby{爲}{な}されし
\ruby{事}{こと}のかつて
\ruby{此路}{こ|〻}になされしならぬやある?。
%
\ruby{思}{おも}へ
\ruby{爾}{なんぢ}、
%
\ruby{萬般}{よろ|づ}の
\ruby{事}{こと}、
%
\ruby{萬般}{よろ|づ}の
\ruby{物}{もの}、
%
\ruby{此}{こ}の
\ruby{路}{みち}に
\ruby{上}{のぼ}り、
%
\ruby{此}{こ}の
\ruby{關}{せき}を
\ruby{{\換字{過}}}{す}ぎざりしものやある?。

\原本頁{}%
\ruby{物}{もの}の
\ruby{能}{よ}く
\ruby{此}{こ}の
\ruby{路}{みち}に
\ruby{上}{のぼ}るものは、
%
\ruby{復}{ま}た
\ruby{必}{かなら}ず
\ruby{再度}{ふた|ゝび}
\ruby{此}{こ}の
\ruby{路}{みち}に
\ruby{上}{のぼ}らん。
%
\ruby{事}{こと}の
\ruby{能}{よ}く
\ruby{此}{こ}の
\ruby{關}{せき}を
\ruby{{\換字{過}}}{す}ぐるものは
\ruby{復}{ま}た
\ruby{必}{かなら}ず
\ruby{二度}{ふた|ゝび}
\ruby{此}{こ}の
\ruby{關}{せき}を
\ruby{{\換字{過}}}{す}ぎん!。

\原本頁{}%
やをら〳〵
\ruby{月}{つき}の
\ruby{光}{ひかり}に
\ruby{這}{は}へる
\ruby{此}{こ}の
\ruby{蜘蛛}{く|も}!。
%
\ruby[h|]{爾}{なんぢ}
\ruby{思}{おも}ひ
\ruby{得}{\換字{𛀁}}ずや
\ruby{此}{こ}の
\ruby{蜘蛛}{く|も}の
\ruby{{\換字{過}}去}{む|かし}
\ruby{既}{すで}に
\ruby{一度}{ひと|たび}
\ruby{世}{よ}にありしとは。
%
\ruby{月}{つき}の
\ruby{此}{こ}の
\ruby{光}{ひかり}!、
%
\ruby[h|]{爾}{なんぢ}
\ruby{思}{おも}ひ
\ruby{得}{\換字{𛀁}}ずや
\ruby{月}{つき}の
\ruby{此}{こ}の
\ruby{光}{ひかり}の
\ruby{{\換字{過}}去}{む|かし}
\ruby{既}{すで}に
\ruby{一度}{ひと|たび}
\ruby{世}{よ}に
\ruby{在}{あ}りしとは。

\原本頁{}%
\ruby{此}{こ}の
\ruby{關}{せき}に
\ruby{立}{た}ちて
\ruby{囁}{さゝや}きて、
%
\ruby{共}{とも}に
\ruby{限無}{かぎり|な}く
\ruby{究無}{きはみ|な}きものにつきて
\ruby{囁}{さゝや}ける
\ruby{爾}{なんぢ}よ
\ruby{我}{われ}よ
\ruby{我}{われ}よ
\ruby{爾}{なんぢ}よ、
%
\ruby[h|]{爾}{なんぢ}
\ruby{思}{おも}ひ
\ruby{得}{\換字{𛀁}}ずや
\ruby{我}{われ}も
\ruby{爾}{なんぢ}も
\ruby{{\換字{過}}去}{む|かし}
\ruby{既}{すで}に
\ruby{一度}{ひと|たび}
\ruby{世}{よ}に
\ruby{在}{あ}りしとは。

\原本頁{}%
\ruby{爾}{なんぢ}も
\ruby{我}{われ}も、
%
\ruby{爾}{なんぢ}と
\ruby{我}{われ}との
\ruby{{\換字{前}}}{まへ}なる
\ruby{路}{みち}の、
%
\ruby{長々}{なが|〳〵}しき
\ruby{{\換字{迷}}}{まよひ}の
\ruby{路}{みち}に
\ruby{復}{また}
\ruby{現}{あら}はれて、
%
\ruby{爾}{なんぢ}もふた〻び
\ruby{行}{ゆ}き
\ruby{我}{われ}もふた〻び
\ruby{行}{ゆ}き、
%
さてしも
\ruby{限}{かぎ}り
\ruby{無}{な}く
\ruby{究}{きは}み
\ruby{無}{な}き
\ruby{輪{\換字{廻}}}{りん|ね}の
\ruby{路}{みち}に
\ruby{千度百度往}{ち|たび|もゝ|たび|ゆ}き
\ruby{{\換字{返}}}{かへ}らでは
\ruby{叶}{かな}はぬにはあらずや』
% \end{quote}% 原本では引用インデントされていない

\原本頁{}%
とありしも
\ruby{思}{おも}ひ
\ruby{出}{いだ}されて、
%
\ruby[g]{水野}{みづの}は
\ruby{拭}{ぬぐ}へども
\ruby{拭}{ぬぐ}へども
\ruby{沸}{わ}きあがる
\ruby{蒸氣}{ゆ|げ}に、
%
\ruby{我}{わ}が
\ruby{心}{こゝろ}の
\ruby{鏡}{かゞみ}の
\ruby{曇}{くも}り
\ruby{果}{は}て〻、
%
\ruby{明}{あき}らかなり
\ruby{得}{\換字{𛀁}}ぬやうの
\ruby{心地}{こゝ|ち}したり。

\原本頁{}%
\ruby{今}{いま}こ〻に
\ruby{我}{われ}には
\ruby{{\換字{尊}}}{たふと}き
\ruby{今}{いま}の
\ruby{世}{よ}のあらずや。
%
\ruby{有}{あ}りても
\ruby{可}{よ}く
\ruby{無}{な}くても
\ruby{宜}{よ}きは
\ruby{{\換字{前}}}{まへ}の
\ruby{世}{よ}ならずや。
%
\ruby{輪{\換字{廻}}}{りん|ね}
\ruby{循環}{じゆん|くわん}の
\ruby{談}{だん}は
\ruby{枝葉}{し|\換字{𛀁}ふ}の
\ruby{事}{こと}のみと、
%
\ruby[g]{水野}{みづの}は
\ruby{{\換字{強}}}{し}ひて
\ruby{思}{おも}ひ
\ruby{棄}{す}てんとしけるが、
%
\ruby{生憎}{あい|にく}に% 原文通りルビは「あいにく」
\ruby{{\換字{猶}}}{なほ}
\ruby{物}{もの}の
\ruby{思}{おも}はる〻を
\ruby{如何}{い|かん}とも
\ruby{爲難}{し|がた}くて、
%
\ruby{答}{こた}へもせず
\ruby{獨}{ひと}り
\ruby{{\換字{空}}想}{おも|ひ}に
\ruby{耽}{ふけ}る
\ruby{折}{をり}しも、
%
\ruby{何}{なに}をか
\ruby{吠}{ほ}ゆる
\ruby{彼}{か}の
\ruby{狗}{いぬ}はまた、
%
べう〳〵と
\ruby{同}{おな}じやうに
\ruby{高}{たか}く
\ruby{鳴}{な}けり。

\原本頁{}%
\ruby{狗}{いぬ}の
\ruby{聲}{こゑ}は
\ruby{淋}{さび}しさの
\ruby{中}{うち}より
\ruby{起}{おこ}こりて
\ruby{淋}{さび}しさの
\ruby{中}{うち}に
\ruby{{\換字{消}}}{き}えたり。
%
\ruby[g]{水野}{みづの}は
\ruby{狗}{いぬ}の
\ruby{聲}{こゑ}の
\ruby{{\換字{消}}}{き}え
\ruby{{\換字{終}}}{をは}りし
\ruby{時}{とき}、
%
ふと
\ruby{眼}{め}をあげて
お
\ruby{濱}{はま}を
\ruby{見}{み}れば、
%
お
\ruby{濱}{はま}もまた
\ruby{狗}{いぬ}の
\ruby{聲}{こゑ}の
\ruby{{\換字{消}}}{き}え
\ruby{{\換字{終}}}{をは}りし
\ruby{時}{とき}、
%
\ruby{物}{もの}おもふ
\ruby{眼}{め}をあげて
\ruby[g]{水野}{みづの}を
\ruby{見}{み}たり。

\原本頁{}%
\ruby{生}{うま}れぬ
\ruby{{\換字{前}}}{さき}を
\ruby{思}{おも}ひやれる
\ruby{眼}{め}は、
%
\ruby{生}{うま}れぬ
\ruby{{\換字{前}}}{さき}を
\ruby{思}{おも}へる
\ruby{眼}{まなこ}と、
%
ひたりと
\ruby{相}{あひ}
\ruby{會}{あ}つて、
%
はつと
\ruby{別}{わか}れぬ。
%
\ruby[g]{水野}{みづの}は
\ruby{忽然}{こつ|ぜん}として、
%
\ruby{我}{わ}が
\ruby{{\換字{前}}}{さき}の
\ruby{世}{よ}に、
%
\ruby{我}{われ}は
\ruby{{\換字{猶}}}{なほ}
\ruby{今}{いま}の
\ruby{我}{われ}の
\ruby{如}{ごと}く、
%
お
\ruby{濱}{はま}は
\ruby{{\換字{猶}}}{なほ}
\ruby{今}{いま}の
お
\ruby{濱}{はま}の
\ruby{如}{ごと}くして、
%
しかも
\ruby{我}{わ}が
\ruby[g]{五十子}{いそこ}もまた
\ruby{今}{いま}の
\ruby[g]{五十子}{いそこ}の
\ruby{如}{ごと}く、
%
\ruby{我}{われ}は
\ruby{今}{いま}と
\ruby{同}{おな}じく
\ruby{苦}{くるし}みあくがれて、
%
\ruby{甲{\換字{斐}}}{か|ひ}
\ruby{無}{な}くも
\ruby{長}{とこしな}へに
\ruby{忌}{い}み
\ruby{{\換字{嫌}}}{きら}はれたりし、
%
\ruby{其}{そ}の
\ruby{事}{こと}のまざ〳〵と
\ruby{存}{あ}りしやうに
\ruby{思}{おも}ひて、
%
\ruby{總身}{そう|み}の
\ruby{毛根動}{け|あな|うご}けるが
\ruby{如}{ごと}く、
%
\ruby{慄然}{ぞ|つ}と
\ruby{{\換字{情}}無}{なさけ|な}く
\ruby{堪}{た}へがたき
\ruby{心地}{こゝ|ち}したり。

\原本頁{}%
\ruby[g]{水野}{みづの}の
\ruby{容態}{よう|す}の
\ruby{常}{たゞ}ならぬを
\ruby{見}{み}て、
%
\ruby[g]{吉右衛門}{きちゑもん}は
\ruby{急}{きふ}に
\ruby{言葉}{こと|ば}を
\ruby{出}{いだ}し、

\原本頁{}%
『ハヽヽ、
%
\ruby{{\換字{前}}}{まへ}の
\ruby{世}{よ}は
\ruby{何樣}{ど|う}でも
\ruby{宜}{い}い、
%
\ruby{今夜}{こん|や}を
\ruby{好}{よ}く
\ruby{寢}{ね}さへすりやあ
\ruby{好}{い}いのだ!。
%
\ruby{三歳}{み|つゝ}や
\ruby{四歳}{よ|つゝ}の
\ruby{時}{とき}の
\ruby{事}{こと}を
\ruby{誰}{たれ}が
\ruby{知}{し}つて
\ruby{居}{ゐ}るものか。
%
\ruby{{\換字{前}}}{まへ}の
\ruby{世}{よ}のあるなんぞと
\ruby{思}{おも}ふのは、
%
\ruby{皆}{みんな}ほんとに
\ruby{氣}{き}の
\ruby{{\換字{所}}爲}{せ|ゐ}に
\ruby{定}{きま}つて
\ruby{居}{ゐ}る。
%
もうそんな
\ruby{下}{くだ}らない
\ruby{事}{こと}は
\ruby{止}{や}めて
\ruby{寢}{ね}ると
\ruby{仕}{し}ましやうか。
%
\ruby{寢}{ね}ると
\ruby{私}{わたし}なぞあ
\ruby{{\換字{前}}}{まへ}の
\ruby{世}{よ}が
\ruby{出}{で}て
\ruby{來}{き}て、
%
いつでも
\ruby{{\換字{若}}}{わか}くつて、
%
\ruby{禿}{は}げて
\ruby{居}{ゐ}ないで、
%
いゝ
\ruby{{\換字{若}}衆}{わかい|しゆ}ですからおもしろい。
%
ハヽハヽハ。
』

\原本頁{}%
と
\ruby{高笑}{たか|わら}ひして
\ruby{一座}{いち|ざ}を
\ruby{動}{うご}かしぬ。

