\Entry{其十一}

% メモ 校正終了 2024-04-17 2024-05-30 2024-06-30
\原本頁{61-9}%
\ruby{偶然}{ぐう|ぜん}の
\ruby{事}{こと}
とすれば
それまで
なれども、
%
\ruby{奇}{あや}し
とすれば
\ruby{奇}{あや}しく
もあるかな。
%
かつて
\ruby{我}{わ}が
\ruby{讀}{よ}みし
\ruby{書}{しよ}の
\ruby{中}{うち}に
\ruby[<j>]{『幻と謎と』}{|ヴイジヨン| |リツドル||}% 引用書籍のタイトル名が長いので特殊処理
といへる
\ruby[||j>]{一}{いつ}
\ruby[||j>]{章}{しやう}
% \ruby{一章}{いつ|しやう}
ありて、
%
\ruby{其}{そ}の
\ruby[||j>]{幽}{ゆう}
\ruby[||j>]{怪}{くわい}
% \ruby{幽怪}{ゆう|くわい}
\ruby{神異}{ しん|い}の% ルビ調整(長いルビ対策)「幽怪(ゆうくわい)」のルビが突き出てくる
\ruby{趣味}{おも|むき}は、
%
\ruby{骨身}{ほね|み}に
\ruby{沁}{し}みて
\ruby{忘}{わす}れ
\ruby{{\換字{難}}}{がた}く、
%
\ruby{今}{いま}に
\ruby{鮮明}{あざ|やか}に
\ruby{心頭}{むな|さき}に
\ruby{{\換字{遺}}}{のこ}れる、
%
\ruby{其}{それ}を
お
\ruby{濱}{はま}の
\ruby{知}{し}るべくは
あらねど、
%
\ruby{其}{そ}の
\ruby{言}{い}ふ
\原本頁{62-3}\改行%
ところを
\ruby{聞}{き}けば、
%
\ruby{何}{なん}ぞ
\ruby{彼}{か}の
\ruby{記}{しる}せる
ところと
\ruby{相}{あひ}
\ruby{似}{に}たるや。
%
たゞ% 踊り字調整「〻(二の字点、揺すり点)に濁点に見えるが(ゞ)」
\ruby{彼}{かれ}は
\ruby{考慮}{かん|がへ}に
\ruby{老}{お}いたる
\ruby{人}{ひと}の
\ruby{言葉}{こと|ば}にして、
%
これは
\ruby{何}{なん}の
\ruby{思案}{し|あん}も
\ruby{無}{な}き
\ruby{少女}{こど|も}の
\ruby{言葉}{こと|ば}なり、
%
\ruby{彼}{かれ}は
\ruby{先}{ま}づ
\ruby{思}{おも}ひて
\ruby{後}{のち}に
\ruby{狗}{いぬ}の
\ruby{聲}{こゑ}を
\ruby{聞}{き}き、
%
これは
\ruby{先}{ま}づ
\ruby{狗}{いぬ}
\原本頁{62-6}\改行%
の
\ruby{聲}{こゑ}を
\ruby{聞}{き}いて
\ruby{後}{のち}に
\ruby{思}{おも}ひ
\ruby{起}{おこ}せるの
\ruby{差異}{ちが|ひ}
こそあれ、
%
おのづからに
\ruby{此}{こ}
\改行% 校正作業の簡略化のため
の
\ruby{年}{とし}ゆかぬ
\ruby{娘}{こ}の、
%
\ruby{誰}{たれ}
\ruby{敎}{をし}へぬに
かゝる
\ruby{事}{こと}を
\ruby{想}{おも}ひ
\ruby{出}{いだ}せる
\ruby{不思議}{ふ|し|ぎ}さ!
\改行% 校正作業の簡略化のため
。
%
\原本頁{62-8}\改行%
\ruby{月日}{つき|ひ}は
\ruby{誰}{たれ}の
\ruby{{\換字{所}}有}{も|の}としも
\ruby{無}{な}ければ、
%
\ruby{仰}{あふ}ぐものは
\ruby{皆}{みな}
\ruby{其}{そ}の
\ruby{光}{ひかり}を
\ruby{見}{み}、
%
\ruby{眞理}{ま|こと}は
\ruby{智者}{ち|しや}の
\ruby{{\換字{造}}}{つく}れるにも
あらねば、
%
\ruby{{\換字{婦}}女}{をん|な}
\ruby{童兒}{こど|も}の
\ruby{胸}{むね}にも
\ruby{{\換字{浮}}}{うか}みて、
%
\ruby{我}{われ}
\原本頁{62-10}\改行%
から
とも
\ruby{無}{な}く
\ruby{如是}{か|く}は
\ruby{悟}{さと}れるにや。
%
そも〳〵
また
\ruby{佛陀}{ほと|け}の
\ruby{敎法}{をし|へ}に、
%
いつと
なく
\ruby{耳}{みゝ}も% 踊り字調整「〻(二の字点、揺すり点)に見えるが(ゝ)」
\ruby{心}{こゝろ}も% 踊り字調整「〻(二の字点、揺すり点)に見えるが(ゝ)」
\ruby{染}{そ}まり
\ruby{居}{ゐ}て、
%
それより
\ruby{然}{さ}る
\ruby{事}{こと}をも
\ruby{思}{おも}へるか
\改行% 校正作業の簡略化のため
。
%
\原本頁{63-1}\改行%
\ruby{其}{そ}の
\ruby{因}{よ}つて
\ruby{出}{い}でし
ところは
\ruby{兎}{と}まれ
\ruby{角}{かく}かれ
\footnote{「\ruby{兎}{と}まれ\ruby{角}{かく}まれ」と思われるが原本通り「\ruby{兎}{と}まれ\ruby{角}{かく}かれ」とした
(国会図書館 コマ番号36/160 p-063 l-01)}%
、
%
\ruby{{\換字{前}}}{まへ}の
\ruby{世}{よ}
\ruby{有}{あ}りや
\ruby{將}{はた}
\ruby{有}{あ}らずや、
%
\ruby{如何}{い|か}にと
\ruby{問}{と}はれては
\ruby{此}{こ}の
\ruby{我}{われ}も
また、
%
\ruby{少}{すこし}ばかりの
\ruby{智慧}{ち|ゑ}
\ruby{學問}{がく|もん}の、
%
\ruby{果}{はた}して
\ruby{有}{あ}りや
\ruby{{\換字{又}}}{また}
\ruby{無}{な}しやと
\ruby{蜘蛛手}{く|も|で}に
\ruby{働}{はたら}く
\ruby{其}{そ}の
\ruby{下蔭}{した|かげ}に、
%
\ruby{私}{ひそか}に
\ruby{{\換字{前}}}{まへ}の
\ruby{世}{よ}を
\footnote{「\ruby{私}{ひそか}に\ruby{{\換字{前}}}{まへ}の\ruby{世}{よ}」に続く文字は原本通り「を」とする
(国会図書館 コマ番号36/160 p-063 l-03 l-04)}%
\ruby{有}{あ}るものゝやう% 踊り字調整「〻(二の字点、揺すり点)に見えるが(ゝ)」
\ruby{思}{おも}ふ
\ruby{心地}{こゝ|ち}も% 踊り字調整「〻(二の字点、揺すり点)に見えるが(ゝ)」
\ruby{實}{まこと}は
\ruby{爲}{す}るなり。

\原本頁{63-5}%
\ruby{{\換字{迷}}信}{まよ|ひ}なり、
%
\ruby{{\換字{迷}}信}{まよ|ひ}なり、
%
\ruby{{\換字{古}}}{ふる}き
\ruby{{\換字{迷}}信}{まよ|ひ}なり、
%
\ruby{智慧}{ち|ゑ}の
\ruby{光輝}{ひか|り}の
\ruby{及}{およ}ばぬ
\ruby{隈}{くま}には、
%
\ruby{其}{そ}の
\ruby{闇}{くら}さにぞ
\ruby{有}{あ}らぬ
\ruby{現像}{すが|た}の
\ruby{思}{おも}ひ
\ruby{{\換字{遣}}}{や}らるゝ、% 踊り字調整「〻(二の字点、揺すり点)に見えるが(ゝ)」
%
\ruby{其}{それ}を
\ruby{{\換字{前}}}{まへ}の
\ruby{世}{よ}とは
\ruby{云}{い}ひ
ならはしたる
ならすや。
%
さは
あれど、
%
\ruby{彼}{か}の
\ruby{書}{しよ}に、

\原本頁{63-8}%
% \begin{quote}% 原本では引用インデントされていない
『
\ruby[||j>]{爾}{なんぢ}
\ruby[||j>]{見}{ み }よ、
%
\ruby{此}{こ}の
\ruby{刹那}{せつ|な}を。
%
\ruby{刹那}{せつ|な}の
\ruby{此}{こ}の
\ruby{關}{せき}より
\ruby{彼方}{かな|た}には
\ruby{涯}{かぎり}
\ruby{無}{ な }き
\ruby{路}{みち}の
\ruby{長路}{なが|ぢ}ぞ% デバッグのためにはここに改行を入れたいが「長」までで28文字なのでアキラメ
\ruby{遙}{はるか}に
\ruby{亘}{わた}れるなる。
%
\ruby{刹那}{せつ|な}の
\ruby{關}{せき}より
\ruby{此方}{こな|た}にも% ルビ調整(原本通り)
\ruby{涯}{かぎり}
\ruby{無}{な}き
\ruby{路}{みち}の
\ruby{長路}{なが|ぢ}は
\原本頁{63-10}\改行%
\ruby{遙}{はるか}に
\ruby{亘}{わた}れるなる。
\footnote{「刹那の此の關より彼方には」では『長路ぞ』であるが、「刹那の關より此方にも」では『長路に』となっている
(国会図書館 コマ番号36/160 p-063 l-09)}%

\原本頁{63-11}%
\ruby{思}{おも}へ
\ruby{爾}{なんぢ}、
%
\ruby{起}{おこ}りし
\ruby{事}{こと}の
かつて
\ruby{此路}{こ|ゝ}に% 踊り字調整「〻(二の字点、揺すり点)に見えるが(ゝ)」
\ruby{起}{おこ}りし
\ruby{事}{こと}
ならぬや
ある?。
%
\ruby{思}{おも}へ
\ruby{爾}{なんぢ}、
%
\ruby{爲}{な}されし
こと
\footnote{原本通り(こと)は平仮名とする
(国会図書館 コマ番号37/160 p-064 l-01)}%
の
かつて
\ruby{此路}{こ|ゝ}に% 踊り字調整「〻(二の字点、揺すり点)に見えるが(ゝ)」
なされし
ならぬや
ある?。
%
\ruby{思}{おも}へ
\ruby{爾}{なんぢ}、
%
\ruby{萬般}{よろ|づ}の
\ruby{事}{こと}、
%
\ruby{萬般}{よろ|づ}の
\ruby{物}{もの}、
%
\ruby{此}{こ}の
\ruby{路}{みち}に
\ruby{上}{のぼ}り
\ruby{此}{こ}の
\ruby{關}{せき}を
\ruby{{\換字{過}}}{す}ぎざりしものやある?。

\原本頁{64-4}%
\ruby{物}{もの}の
\ruby{能}{よ}く
\ruby{此}{こ}の
\ruby{路}{みち}に
\ruby{上}{のぼ}るものは、
%
\ruby{復}{また}
\ruby{必}{かなら}ず
\ruby{再度}{ふた|ゝび}% 踊り字調整「〻(二の字点、揺すり点)に見えるが(ゝ)」
\ruby{此}{こ}の
\ruby{路}{みち}に
\ruby{上}{のぼ}らん。
%
\ruby{事}{こと}の
\ruby{能}{よ}く
\ruby{此}{こ}の
\ruby{關}{せき}を
\ruby{{\換字{過}}}{す}ぐる
ものは
\ruby{復}{また}
\ruby{必}{かなら}ず
\ruby{二度}{ふた|ゝび}% 踊り字調整「〻(二の字点、揺すり点)に見えるが(ゝ)」
\ruby{此}{こ}の
\ruby{關}{せき}を
\ruby{{\換字{過}}}{す}ぎん!。

\原本頁{64-6}%
やをら〳〵
\ruby{月}{つき}の
\ruby{光}{ひかり}に
\ruby{這}{は}へる
\ruby{此}{こ}の
\ruby{蜘蛛}{く|も}!。
%
\ruby[||j>]{爾}{なんぢ}
\ruby[||j>]{思}{ おも}ひ
\ruby{得}{{\換字{𛀁}}}ずや
\ruby{此}{こ}の
\ruby{蜘蛛}{く|も}の
\ruby{{\換字{過}}去}{むか|し}
\ruby{既}{すで}に
\ruby{一度}{ひと|たび}
\ruby{世}{よ}に
ありしとは。
%
\ruby{月}{つき}の
\ruby{此}{こ}の
\ruby{光}{ひかり}!、
%
\ruby[||j>]{爾}{なんぢ}
\ruby[||j>]{思}{ おも}ひ
\ruby{得}{{\換字{𛀁}}}ずや
\ruby{月}{つき}の
\ruby{此}{こ}の
\ruby{光}{ひかり}の
\ruby{{\換字{過}}去}{むか|し}
\ruby{既}{すで}に
\ruby{一度}{ひと|たび}
\ruby{世}{よ}に
\ruby{在}{あ}りし
とは。

\原本頁{64-9}%
\ruby{此}{こ}の
\ruby{關}{せき}に
\ruby{立}{た}ちて
\ruby{囁}{さゝや}きて、% 踊り字調整「〻(二の字点、揺すり点)に見えるが(ゝ)」
%
\ruby{共}{とも}に
\ruby{限}{かぎり}
\ruby{無}{な}く
\ruby[||j>]{究}{きはみ}
\ruby[||j>]{無}{ な }き
ものにつきて
\ruby{囁}{さゝや}ける% 踊り字調整「〻(二の字点、揺すり点)に見えるが(ゝ)」
\ruby{爾}{なんぢ}よ
\ruby{我}{われ}よ
\ruby{我}{われ}よ
\ruby{爾}{なんぢ}よ、
%
\ruby[||j>]{爾}{なんぢ}
\ruby[||j>]{思}{ おも}ひ
\ruby{得}{{\換字{𛀁}}}ずや
\ruby{我}{われ}も
\ruby{爾}{なんぢ}も
\ruby{{\換字{過}}去}{むか|し}
\ruby{既}{すで}に
\ruby{一度}{ひと|たび}
\ruby{世}{よ}に
\ruby{在}{あ}りし
とは。

\原本頁{65-1}%
\ruby{爾}{なんぢ}も
\ruby{我}{われ}も、
%
\ruby{爾}{なんぢ}と
\ruby{我}{われ}との
\ruby{{\換字{前}}}{まへ}なる
\ruby{路}{みち}の、
%
\ruby[g]{長々}{なが〳〵}しき
\ruby{{\換字{迷}}}{まよひ}の
\ruby{路}{みち}に
\ruby{復}{また}
\ruby{現}{あら}はれて、
%
\ruby{爾}{なんぢ}も
ふたゝび% 踊り字調整「〻(二の字点、揺すり点)に見えるが(ゝ)」
\ruby{行}{ゆ}き
\ruby{我}{われ}も
ふたゝび% 踊り字調整「〻(二の字点、揺すり点)に見えるが(ゝ)」
\ruby{行}{ゆ}き、
%
さてしも
\ruby{限}{かぎ}り
\ruby{無}{な}く
\ruby{究}{きは}み
\ruby{無}{な}き
\ruby{輪{\換字{廻}}}{りん|ね}
\footnote{原本では、「廴+囘」の「𢌞(U+2231E)」ではなく、「廴+同」の「廻(U+5EFB)→\換字{廻}」を利用している
(国会図書館 コマ番号37/160 p-065 l-03)他}%
の
\ruby{路}{みち}に
\ruby{千度}{ち|たび}
\ruby{百度}{もゝ|たび}% 踊り字調整「〻(二の字点、揺すり点)に見えるが(ゝ)」
\ruby{徃}{ゆ}き
\ruby{{\換字{返}}}{かへ}らでは
\ruby{叶}{かな}はぬには
あらずや。
』
% \end{quote}% 原本では引用インデントされていない

\原本頁{65-4}%
と
ありしも
\ruby{思}{おも}ひ
\ruby{出}{いだ}されて、
%
\ruby{水野}{みづ|の}は
\ruby{拭}{ぬぐ}へども
\ruby{拭}{ぬぐ}へども
\ruby{沸}{わ}きあがる
\ruby{蒸氣}{ゆ|げ}に、
%
\ruby{我}{わ}が
\ruby{心}{こゝろ}の% 踊り字調整「〻(二の字点、揺すり点)に見えるが(ゝ)」
\ruby{鏡}{かゞみ}の% 踊り字調整「〻(二の字点、揺すり点)に濁点に見えるが(ゞ)」
\ruby{曇}{くも}り
\ruby{果}{は}てゝ、% 踊り字調整「〻(二の字点、揺すり点)に見えるが(ゝ)」
%
\ruby{明}{あき}らかなり
\ruby{得}{{\換字{𛀁}}}ぬやうの
\ruby{心地}{こゝ|ち}% 踊り字調整「〻(二の字点、揺すり点)に見えるが(ゝ)」
したり。

\原本頁{65-7}%
\ruby{今}{いま}
こゝに% 踊り字調整「〻(二の字点、揺すり点)に見えるが(ゝ)」
\ruby{我}{われ}には
\ruby{{\換字{尊}}}{たふと}き
\ruby{今}{いま}の
\ruby{世}{よ}の
あらずや。
%
\ruby{有}{あ}りても
\ruby{可}{よ}く
\ruby{無}{な}くても
\ruby{宜}{よ}きは
\ruby{{\換字{前}}}{まへ}の
\ruby{世}{よ}ならずや。
%
\ruby{輪{\換字{廻}}}{りん|ね }
\ruby[<j||]{循}{じゆん}
\ruby[||j>]{環}{くわん}の
\ruby{談}{だん}は
\ruby{枝葉}{し|{\換字{𛀁}}ふ}の
\ruby{事}{こと}のみと、
%
\ruby{水野}{みづ|の}は
\ruby{{\換字{強}}}{し}ひて
\ruby{思}{おも}ひ
\ruby{棄}{す}てんと
しけるが、
%
\ruby{生憎}{あい|にく}に% ルビ調整(原本通り)(あいにく)
\ruby{{\換字{猶}}}{なほ}
\ruby{物}{もの}の
\ruby{思}{おも}はるゝを% 踊り字調整「〻(二の字点、揺すり点)に見えるが(ゝ)」
\ruby{如何}{いか|ん}とも
\ruby{爲{\換字{難}}}{し|がた}くて、
%
\ruby{答}{こた}へも
せず
\ruby{獨}{ひと}り
\ruby{{\換字{空}}想}{おも|ひ}に
\ruby{耽}{ふけ}る
\ruby{折}{をり}しも、
%
\ruby{何}{なに}をか
\ruby{吠}{ほ}ゆる
\ruby{彼}{か}の
\ruby{狗}{いぬ}は
また、
%
べう〳〵と
\ruby{同}{おな}じやうに
\ruby{高}{たか}く
\ruby{鳴}{な}けり。

\原本頁{66-1}%
\ruby{狗}{いぬ}の
\ruby{聲}{こゑ}は
\ruby{淋}{さび}しさの
\ruby{中}{うち}より
\ruby{起}{おこ}りて
\ruby{淋}{さび}しさの
\ruby{中}{うち}に
\ruby{{\換字{消}}}{き}えたり。
%
\ruby{水野}{みづ|の}は
\ruby{狗}{いぬ}の
\ruby{聲}{こゑ}の
\ruby{{\換字{消}}}{き}え
\ruby{{\換字{終}}}{をは}りし
\ruby{時}{とき}、
%
ふと
\ruby{眼}{め}を
あげて
お
\ruby{濱}{はま}を
\ruby{見}{み}れば、
%
お
\ruby{濱}{はま}も
また
\ruby{狗}{いぬ}の
\ruby{聲}{こゑ}の
\ruby{{\換字{消}}}{き}え
\ruby{{\換字{終}}}{をは}りし
\ruby{時}{とき}、
%
\ruby{物}{もの}
おもふ
\ruby{眼}{め}を
あげて
\ruby{水野}{みづ|の}を
\ruby{見}{み}たり
\改行% 校正作業の簡略化のため
。
%
\原本頁{66-4}\改行%
\ruby{生}{うま}れぬ
\ruby{{\換字{前}}}{さき}を
\ruby{思}{おも}ひやれる
\ruby{眼}{め}は、
%
\ruby{生}{うま}れぬ
\ruby{{\換字{前}}}{さき}を
\ruby{思}{おも}へる
\ruby{眼}{まなこ}と、
%
ひたりと
\ruby{相}{あひ}
\ruby{會}{あ}つて、
%
はつと
\ruby{別}{わか}れぬ。
%
\ruby{水野}{みづ|の}は
\ruby{忽然}{こつ|ぜん}として
%
\ruby{我}{わ}が
\ruby{{\換字{前}}}{さき}の
\ruby{世}{よ}に、
%
\ruby{我}{われ}
は
\ruby{{\換字{猶}}}{なほ}
\ruby{今}{いま}の
\ruby{我}{われ}の
\ruby{如}{ごと}く、
%
お
\ruby{濱}{はま}は
\ruby{{\換字{猶}}}{なほ}
\ruby{今}{いま}の
お
\ruby{濱}{はま}の
\ruby{如}{ごと}くして、
%
しかも
\ruby{我}{わ}が
\ruby{五十子}{い|そ|こ}も
また
\ruby{今}{いま}の
\ruby{五十子}{い|そ|こ}の
\ruby{如}{ごと}く、
%
\ruby{我}{われ}は
\ruby{今}{いま}と
\ruby{同}{おな}じく
\ruby{苦}{くるし}み
あくがれて、
%
\ruby{甲{\換字{斐}}}{か|ひ}
\ruby{無}{な}くも
\ruby[<j>]{長}{とこしな}へに
\ruby{忌}{い}み
\ruby{{\換字{嫌}}}{きら}はれたりし、
%
\ruby{其}{そ}の
\ruby{事}{こと}の
まざ〳〵と
\ruby{存}{あ}りしやうに
\ruby{思}{おも}ひて、
%
\ruby{總身}{そう|み}の
\ruby{毛根}{け|あな}
\ruby{動}{うご}けるが
\ruby{如}{ごと}く、
%
\ruby{慄然}{ぞ|つ}と
\ruby[||j>]{{\換字{情}}}{なさけ}
\ruby[||j>]{無}{ な }く
\ruby{堪}{た}へがたき
\ruby{心地}{こゝ|ち}したり。% 踊り字調整「〻(二の字点、揺すり点)に見えるが(ゝ)」

\原本頁{66-11}%
\ruby{水野}{みづ|の}の
\ruby{容態}{よう|す}の
\ruby{常}{たゞ}ならぬを% 踊り字調整「〻(二の字点、揺すり点)に濁点に見えるが(ゞ)」
\ruby{見}{み}て、
%
\ruby{吉右衛門}{きち||ゑ|もん}は
\ruby{急}{きふ}に
\ruby{言葉}{こと|ば}を
\ruby{出}{いだ}し、

\原本頁{67-1}%
『
ハヽヽ、
%
\ruby{{\換字{前}}}{まへ}の
\ruby{世}{よ}は
\ruby{何樣}{ど|う}でも
\ruby{宜}{い}い、
%
\ruby{今夜}{こん|や}を
\ruby{好}{よ}く
\ruby{寢}{ね}さへ
すりやあ
\ruby{好}{い}いのだ!。
%
\ruby{三歳}{みつ|ゝ}や% 踊り字調整「〻(二の字点、揺すり点)に見えるが(ゝ)」
\ruby{四歳}{よつ|ゝ}の% 踊り字調整「〻(二の字点、揺すり点)に見えるが(ゝ)」
\ruby{時}{とき}の
\ruby{事}{こと}を
\ruby{誰}{だれ}が
\ruby{知}{し}つて
\ruby{居}{ゐ}るものか。
%
\ruby{{\換字{前}}}{まへ}の
\ruby{世}{よ}の
あるなんぞと
\ruby{思}{おも}ふのは、
%
\ruby{皆}{みんな}ほんとに
\ruby{氣}{き}の
\ruby{{\換字{所}}爲}{せ|ゐ}に
\ruby{定}{きま}つて
\ruby{居}{ゐ}る。
%
もうそんな
\ruby{下}{くだ}らない
\ruby{事}{こと}は
\ruby{止}{や}めて
\ruby{寢}{ね}ると
\ruby{仕}{し}ましやうか。
%
\ruby{寢}{ね}ると
\ruby{私}{わたし}なぞあ
\ruby{{\換字{前}}}{まへ}の
\ruby{世}{よ}が
\ruby{出}{で}て
\ruby{來}{き}て、
%
いつでも
\ruby{{\換字{若}}}{わか}くつて、
%
\ruby{禿}{はげ}て
\ruby{居}{ゐ}ないで、
%
いゝ% 踊り字調整「〻(二の字点、揺すり点)に見えるが(ゝ)」
\ruby[||j>]{{\換字{若}}}{わかい}
\ruby[||j>]{衆}{ しゆ}
% \ruby{{\換字{若}}衆}{わかい|しゆ}
ですから
おもしろい。
%
ハヽハヽハ。
』

\原本頁{67-7}%
と
\ruby{高}{たか}
\ruby{笑}{わら}ひして
\ruby{一座}{いち|ざ}を
\ruby{動}{うご}かしぬ。
