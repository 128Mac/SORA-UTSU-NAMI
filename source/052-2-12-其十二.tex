\Entry{其十二}

% メモ 校正終了 2024-04-19
\原本頁{67-8}%
\ruby{世界}{せ|かい}は
\ruby{{\換字{紛}}々}{ふん|ぷん}たり、
%
\ruby{萬馬}{ばん|ば}
\ruby{埒}{らち}の
\ruby{内}{うち}を
\ruby{駈}{かけ}り、
%
\ruby{人間}{にん|げん}は
\ruby{擾々}{ぜう|〳〵}たり、
%
\ruby{群蟻}{ぐん|ぎ}
\ruby[||j>]{碓}{からうす}の
\ruby{緣}{ふち}を
\ruby{回}{めぐ}る、% 原本通り「回」
%
と
\ruby{君}{きみ}が
\ruby{此}{こ}の
\ruby{册子}{さう|し}に
\ruby{書}{か}きし
\ruby{言葉}{こと|ば}も
おもしろし、
%
いざや、
%
\ruby{此}{こ}の
\ruby{秋}{あき}の
\ruby{氣}{き}は
\ruby{淸}{す}み
\ruby{風}{かぜ}は
\ruby[||j>]{快}{こ〻ろよ}ければ、% 原本通り「〻(二の字点、揺すり点)」
%
\ruby{家}{いへ}の
\ruby{矮}{ひく}きより
\ruby{出}{い}で〻% 原本通り「〻(二の字点、揺すり点)」
\ruby{山}{やま}の
\ruby{高}{たか}きに
\ruby{登}{のぼ}り、
%
せめては
\ruby{一日}{いち|じつ}を
\ruby{埒}{らち}の
\ruby{内}{うち}より
\ruby{{\換字{逃}}}{のが}れ、
%
\ruby{少時}{しば|し}は
\ruby[||j>]{碓}{からうす}の
\ruby{緣}{ふち}を
\原本頁{67-3}\改行%
\ruby{離}{はな}れて、
%
\ruby{笑}{わら}ひ
\ruby{傲}{おご}らんも
また
\ruby{可}{よ}からずやと、
%
\ruby{{\換字{絕}}}{た}えて
\ruby{久}{ひさ}しき
\ruby{日方}{ひ|かた}
\ruby{八郎}{はち|らう}、
%
\ruby{友{\換字{情}}}{なさ|け}は
\ruby{深}{ふか}き
\ruby{島木}{しま|き}
\ruby{萬五郎}{まん|ご|らう}、
%
\ruby{特}{こと}には
\ruby{懷}{なつ}かしかりし
\ruby{羽{\換字{勝}}}{は|がち}
\ruby{千{\換字{造}}}{せん|ざう}さへ
\ruby{打{\換字{連}}}{うち|つ}れ
\ruby{來}{きた}りて
\ruby{誘}{いざな}ふに、
%
\ruby{日頃}{ひ|ごろ}の
\ruby{崩折}{くづ|を}れきつたる
\ruby{心}{こ〻ろ}も、% 原本通り「〻(二の字点、揺すり点)」
%
\ruby{雨}{あめ}に
\ruby{會}{あ}いたる
\ruby{旱歳}{ひで|り}の
\ruby{草}{くさ}の、
%
\ruby[<j>]{蘇}{よみがへ}り
\ruby{立}{た}つ
\ruby{思}{おもひ}して、
%
\ruby{一議}{いち|ぎ}にも
\ruby{及}{およ}ばず
\ruby{立出}{たち|い}でしが、
%
\原本頁{67-7}\改行%
\ruby{天}{てん}を
\ruby{{\換字{摩}}}{ま}し
\ruby{雲}{くも}に
\ruby{冲}{ひい}る
\ruby{山嶽}{や|ま}の
\ruby{景色}{け|しき}の、
%
\ruby{雄々}{を|〻}しく% 原本通り「〻(二の字点、揺すり点)」
\ruby{崇}{たか}きを
\ruby{打望}{うち|のぞ}みて
\ruby{辿}{たど}りし
\ruby{{\換字{半}}{\換字{途}}}{はん|と}に
\ruby{如何}{い|かゞ}は% TODO 原本の「二の字点、揺すり点」に濁点のグリフが見つからないので「ゞ」
\ruby{仕}{し}けん、
%
\ruby{圖}{はか}らず
\ruby{三人}{さん|にん}とは
\ruby{相}{あひ}
\ruby[||j>]{失}{うしな}ひたり。

\原本頁{67-9}%
『
\ruby{水野}{みづ|の}ーツ、
』

\原本頁{67-10}%
と
\ruby{號令}{がう|れい}
\ruby{聲}{ごゑ}の
\ruby{烈}{はげ}しく
\ruby{叫}{さけ}べるは
\ruby{豪放}{がう|はう}なる
\ruby{我}{わ}が
\ruby{日方}{ひ|かた}の
\ruby{聲}{こゑ}なり。

\原本頁{67-11}%
『
オーイ、
%
\ruby{水野}{みづ|の}、
』

\原本頁{69-1}%
と
\ruby{爽}{さは}やかに
\ruby{喚}{よ}べるは
\ruby[<j||]{快}{くわい}
\ruby[||j>]{活}{くわつ}なる
\ruby{我}{わ}が
\ruby{島木}{しま|き}が
\ruby{聲}{こゑ}なり。
%
\ruby{姿}{すがた}は
\ruby{何處}{いづ|く}とも
\ruby{見}{み}えざれど、
%
\ruby{聲}{こゑ}は
\ruby{{\換字{前}}{\換字{途}}}{ゆく|て}の
\ruby{高}{たか}きにありて、
%
\ruby{後}{おく}れたる
\ruby{我}{われ}を
\ruby{励}{はげ}まし
\ruby{促}{うなが}し、
%
\ruby{來}{きた}れよ、
%
\ruby{上}{のぼ}れよ、
%
\ruby{{\換字{進}}}{す〻}まざるやと、% 原本通り「〻(二の字点、揺すり点)」
%
\ruby{二人}{ふた|り}が
\ruby{心}{こ〻ろ}を% 原本通り「〻(二の字点、揺すり点)」
\ruby{焦立}{いら|だ}て
\ruby{居}{を}れるは、
%
\ruby{其}{その}
\ruby{聲}{こゑ}の
\ruby{色}{いろ}にも
あり〳〵と
\ruby{知}{し}れたり。

\原本頁{69-5}%
\ruby{草萊}{く|さ}も
\ruby{無}{な}ければ、
%
\ruby[<j>]{樸}{ちいさき}
\ruby[||j>]{樕}{ 〻 }も% 原本通り「〻(二の字点、揺すり点)」
% \ruby{樸樕}{ちいさき|〻}も% 原本通り「〻(二の字点、揺すり点)」
% 「樸樕」(ぼくそく) 小さい木。 叢生する小さな木々。
\ruby{無}{な}く、
%
たゞ% TODO 原本の「二の字点、揺すり点」に濁点のグリフが見つからないので「ゞ」
これ
\ruby{圓}{まる}き
\ruby{石塊}{い|し}のみなる
\ruby{荒凉}{くわう|りやう}たる
\ruby{山路}{さん|ろ}の
\ruby{爪端}{つま|さき}
\ruby[||j>]{上}{あがり}は、
%
\ruby{歩}{あゆ}み
\ruby{行}{ゆ}くに
いと
\ruby{歩}{あゆ}み
\ruby{辛}{づら}けれど、
%
\ruby{上}{のぼ}らで
\ruby{止}{や}むべき
\ruby{我}{われ}ならんやと、
%
\ruby{水野}{みづ|の}は
\ruby[<j||]{唇}{くちびる}
\ruby{硬}{かた}く
\ruby{引締}{ひき|し}めて、
%
\ruby{執念}{しふ|ね}くも
\ruby{{\換字{強}}}{し}ひて
\ruby{上}{のぼ}り
\ruby{上}{のぼ}りぬ。

\原本頁{69-9}%
\ruby{歩}{あゆ}めば
\ruby{磧礫}{こ|いし}は
\ruby{我}{わ}が
\ruby{脚}{あし}の
\ruby{下}{した}に
つぶやき
\ruby{言}{ものい}ふ
\ruby{聲}{こゑ}を
なし、
%
\ruby{頑石}{い|し}は
また
\ruby{腹黑}{はら|ぐろ}くも
\ruby{我}{われ}を
\ruby{滑}{すべ}らしむ。
%
されど
\ruby{上}{のぼ}らで
\ruby{止}{や}むべき
\ruby{我}{われ}ならんやと、
%
\ruby{水野}{みづ|の}は
つぶやける
\ruby{磧礫}{こ|いし}の
\ruby{上}{うへ}に
\ruby{冷}{ひや}やかに
\ruby{濶歩}{かわ|つぽ}し、
%
\ruby{滑}{すべ}らしむる
\ruby{頑石}{い|し}の
\ruby{頭}{かしら}を
した〻かに% 原本通り「〻(二の字点、揺すり点)」
\ruby{踏}{ふ}み
\ruby{壓}{おさ}へて、
%
\ruby{{\換字{猶}}}{なほ}
\ruby{執念}{しふ|ね}くも
\ruby{{\換字{強}}}{し}ひて
\ruby{上}{のぼ}り
\ruby{上}{のぼ}りぬ。
%
\ruby{時}{とき}に
\ruby{何處}{いづ|く}より
\ruby{來}{きた}りし
ともなく
\ruby{{\換字{丈}}}{たけ}
\ruby{矮}{ひく}く
\ruby{足}{あし}
\ruby{跛}{な}へたる
\ruby{妖精}{も|の}の、
%
\ruby{其}{そ}の
\ruby[<j||]{狀}{かたち}
\ruby{怪}{あや}しく
して、
%
たゞ% TODO 原本の「二の字点、揺すり点」に濁点のグリフが見つからないので「ゞ」
\ruby{是}{これ}
\ruby{肉}{にく}の
\ruby{團塊}{かた|まり}とも
いふべく
\ruby{氣味}{き|み}
\ruby{惡}{あし}くも
\ruby{重}{おも}きが、
%
\原本頁{70-4}\改行%
\ruby{何時}{い|つ}か
\ruby{我}{わ}が
\ruby{肩}{かた}
\ruby{頭}{さき}に
\ruby{上}{のぼ}り
\ruby{居}{を}りて、
%
\ruby{怖}{おそ}ろしき
\ruby{其}{そ}の
\ruby{力}{ちから}を
もて
\ruby{壓}{お}しに
\ruby{壓}{お}しつ、
%
\ruby{止}{とゞ}まれ、% TODO 原本の「二の字点、揺すり点」に濁点のグリフが見つからないので「ゞ」
%
\ruby{止}{とゞ}まれ、% TODO 原本の「二の字点、揺すり点」に濁点のグリフが見つからないので「ゞ」
%
\ruby{休}{やす}めよ、
%
\ruby{倒}{たふ}れよ、
%
\ruby{地}{ち}に
\ruby{入}{い}れよ、
%
\ruby{奈落}{な|らく}に
\ruby{歿}{ぼつ}せよ
と
\ruby{云}{い}はぬ
ばかりに、
%
\ruby{下方}{し|た}へ
\ruby{下方}{し|た}へと
\ruby{壓}{お}しつけたり。

\原本頁{70-7}%
\ruby[||j>]{汝}{おのれ}、
%
\ruby{我}{わ}が
\ruby{{\換字{魔}}}{ま}、
%
\ruby{我}{わ}が
\ruby{仇敵}{あ|だ}の
\ruby{重力}{おも|さ}の
\ruby{精}{せい}!。
%
\ruby[<j||]{汝}{なんぢ}% 「汝(なんぢ)」の読みは原文のまま
\ruby{千鈞}{せん|きん}の
\ruby{力}{ちから}を
もて
\ruby{我}{われ}を
\ruby{壓}{あつ}さば、
%
\ruby{我}{われ}
また
\ruby{千鈞}{せん|きん}の
\ruby{力}{ちから}を
\ruby{以}{もつ}て
\ruby{汝}{なんぢ}に% 「汝(なんぢ)」の読みは原文のまま
\ruby{當}{あた}らん。
%
\ruby[<j||]{汝}{なんぢ}% 「汝(なんぢ)」の読みは原文のまま
\ruby{萬鈞}{まん|きん}の
\ruby{力}{ちから}を
もて
\ruby{我}{われ}を
\ruby{壓}{あつ}さば、
%
\ruby{我}{われ}
また
\ruby{萬鈞}{まん|きん}の
\ruby{力}{ちから}を
\ruby{以}{もつ}て
\ruby{汝}{なんぢ}に% 「汝(なんぢ)」の読みは原文のまま
\ruby{當}{あた}らん。
%
\ruby{汝}{なんぢ}は% 「汝(なんぢ)」の読みは原文のまま
\ruby{下}{くだ}さんとす、
%
\原本頁{70-10}\改行%
\ruby{我}{われ}は
\ruby{上}{のぼ}らんとす。
%
\ruby{我}{われ}
\ruby{屈}{くつ}せず
\ruby{我}{われ}
\ruby{撓}{たゆ}まず、
%
\ruby{我々}{われ|われ}が% 元本通り非踊り字表記
\ruby{努力}{ど|りよく}を
\ruby{悋}{をし}むこと
\ruby{無}{な}し、
%
\ruby{上}{のぼ}らで
\ruby{止}{や}むべき
\ruby{我}{われ}ならんや、
%
と
\ruby{傲然}{がう|ぜん}として
\ruby{重}{おも}きに
\ruby{堪}{た}へつ〻、% 原本通り「〻(二の字点、揺すり点)」
%
\原本頁{71-1}\改行%
\ruby{言葉}{こと|ば}をも
\ruby{出}{だ}さねば
\ruby{手}{て}をも
\ruby{動}{うご}かさずして、
%
\ruby{水野}{みづ|の}は
\ruby{{\換字{猶}}}{なほ}
\ruby{{\換字{強}}}{し}ひて
\ruby{執念}{しふ|ね}くも
\ruby{上}{のぼ}り
\ruby{上}{のぼ}りぬ。

\原本頁{71-3}%
\ruby{妖精}{も|の}は
\ruby{水野}{みづ|の}が
\ruby{耳}{み〻}に% 原本通り「〻(二の字点、揺すり点)」
\ruby{貼}{つ}きて、
%
\ruby{重々}{おも|〳〵}しき
\ruby{言葉}{こと|ば}の
\ruby{一語}{いち|ご}
\ruby{一語}{いち|ご}に、
%
\ruby{{\換字{鉛}}}{なまり}の
\ruby{雫}{しづく}を
\ruby{頭腦}{かし|ら}の
\ruby{奧}{おく}に
\ruby{{\換字{送}}}{おく}り
\ruby{入}{い}る〻が% 原本通り「〻(二の字点、揺すり点)」
\ruby{如}{ごと}くに
\ruby{囁}{さ〻や}きて% 原本通り「〻(二の字点、揺すり点)」
\ruby{曰}{い}へらく、% 曰く(いわく)の「曰 66f0」、日曜の「日」は 65e5

\原本頁{71-5}%
『
\ruby[||j>]{爾}{なんぢ}、
%
\ruby{水野}{みづ|の}!、
%
おろかにも
\ruby[||j>]{爾}{なんぢ}の
\ruby{思}{おも}ひ
あがれるよ。
%
\ruby[||j>]{爾}{なんぢ}、
%
\ruby{智慧}{ち|ゑ}の
\ruby{石}{いし}!。
%
\ruby[||j>]{爾}{なんぢ}、
%
おのが
\ruby{身}{み}を
\ruby{高}{たか}くも
\ruby{高}{たか}く
\ruby{投}{な}げ
\ruby{上}{あ}げたる
\ruby[||j>]{爾}{なんぢ}。
%
されど、
%
おろかや、
%
\ruby{投}{な}げられし
\ruby{石}{いし}の、
%
\ruby{落}{お}ちて
\ruby{{\換字{返}}}{かへ}らぬ
\ruby{事}{こと}の
いづくにか
ある!。
%
おろかや
\ruby[||j>]{爾}{なんぢ}、
%
\ruby{落}{お}ちて
\ruby{下}{くだ}らん、
%
\ruby{今}{いま}
\ruby{見}{み}よ
\ruby{落}{お}ちて
\ruby{降}{くだ}る
べきなり。

\原本頁{71-9}%
\ruby[||j>]{爾}{なんぢ}、
%
\ruby{水野}{みづ|の}!、
%
\ruby{智慧}{ち|ゑ}の
\ruby{石}{いし}!。
%
\ruby[<j||]{爾}{なんぢ}
\ruby[||j>]{弩}{いしゆみ}より
\ruby{飛}{と}びし
\ruby{石}{いし}、
%
\ruby[<j||]{爾}{なんぢ}
\ruby{天}{あま}つ
\ruby{星}{ぼし}を
\ruby{碎}{くだ}かん
として
\ruby{飛}{と}びし
\ruby{石}{いし}!。
%
\ruby[||j>]{爾}{なんぢ}、
%
おのが
\ruby{身}{み}を
\ruby{高}{たか}くも
\ruby{高}{たか}く
\ruby{投}{な}げ
\ruby{上}{あ}げたる
\ruby[||j>]{爾}{なんぢ}。
%
されど、
%
おろかや
\ruby[||j>]{爾}{なんぢ}、
%
\ruby{投}{な}げられし
\ruby{石}{いし}の
\ruby{落}{お}ちて
\ruby{{\換字{返}}}{かへ}らぬ
\ruby{事}{こと}の
\原本頁{72-1}\改行%
\ruby{那處}{いづ|く}にかある!。
%
おろかや
\ruby[||j>]{爾}{なんぢ}、
%
\ruby{落}{お}ちて
\ruby{降}{くだ}らん、
%
\ruby{今}{いま}
\ruby{見}{み}よ
\ruby{落}{お}ちて
\ruby{降}{くだ}るべきなり。
%
\ruby{聞}{き}け、
%
\ruby{宣告}{のり|ごと}は
かくぞ、
%
\ruby[||j>]{爾}{なんぢ}と
\ruby[||j>]{爾}{なんぢ}の
\ruby{石}{いし}を
\ruby{投}{な}ぐる
\ruby{行爲}{わ|ざ}とよ。
%
\ruby{水野}{みづ|の}、
%
\ruby[<j||]{爾}{なんぢ}
\ruby{高}{たか}くも
\ruby{高}{たか}く
\ruby{石}{いし}を
\ruby{投}{な}げたるよ、
%
されど
\ruby{其}{そ}は
たゞ% TODO 原本の「二の字点、揺すり点」に濁点のグリフが見つからないので「ゞ」
\ruby{爾}{なんぢ}が
\ruby{頭}{かしら}の
\ruby{上}{うへ}に
\ruby{落}{お}ちて
\ruby{{\換字{返}}}{かへ}らんなり。
』

\原本頁{72-5}%
かく
\ruby{云}{い}ひ
\ruby{{\換字{終}}}{おは}りて
\ruby{言}{ことば}を
\ruby{{\換字{絕}}}{た}ちしが、
%
\ruby{妖精}{も|の}は
\ruby{無言}{む|ごん}の
\ruby{恐}{おそろ}しき
\ruby{力}{ちから}を
もて、
%
\ruby{倒}{たふ}れよ
\ruby{地}{つち}に、
%
\ruby{沈}{しづ}めよ
\ruby{奈落}{な|らく}にと、
%
いよ〳〵
\ruby{烈}{はげ}しく
\ruby{壓}{お}しに
\ruby{壓}{お}せば、
%
\ruby{水野}{みづ|の}は
ほと〳〵
\ruby{堪}{た}へ
ざらんと
したり。

\原本頁{72-8}%
されど
\ruby{水野}{みづ|の}は
\ruby{{\換字{更}}}{さら}に
\ruby{屈}{くつ}せず、
%
\ruby[||j>]{盤}{ばん}
\ruby[||j>]{石}{じやく}
% \ruby{盤石}{ばん|じやく}
\ruby[||j>]{虐}{しひた}げ
\ruby{壓}{あつ}すれども
\ruby{幽{\換字{蘭}}}{ゆう|らん}
\ruby{死}{し}せずして、
%
\ruby{{\換字{猶}}}{なほ}
\ruby{能}{よ}く
\ruby{天}{てん}に
\ruby{向}{むか}つて
\ruby{芽}{め}を
\ruby{抽}{ぬき}んづるか
\ruby{如}{ごと}く、
%
\ruby{昻々}{かう|〳〵}
\ruby{然}{ぜん}として
\ruby{頭}{かうべ}を
\ruby{擧}{あ}げて、
%
\ruby{執念}{しふ|ね}くも
\ruby{{\換字{強}}}{し}ひて
\ruby{上}{のぼ}り
\ruby{上}{のぼ}りけるが、
%
\ruby{見}{み}れば
\ruby{路}{みち}の
\ruby{邊}{べ}に
\ruby{病}{や}める
\ruby{女}{をんな}
ありて
\ruby{世}{よ}にも
\ruby{痛}{いた}ましく
\ruby{惱}{なや}み
\ruby{伏}{ふ}したり。
%
\ruby{如何}{い|か}なる
\ruby{人}{ひと}の
\ruby{{\換字{道}}}{みち}
\ruby{行}{ゆ}き
\ruby{患}{わづら}ひて、
%
か〻る% 原本通り「〻(二の字点、揺すり点)」
\ruby{山路}{やま|ぢ}には
あるならんと、
%
いぶかしみて
\ruby{不圖}{ふ|と}
\ruby{眼}{め}を
\ruby{{\換字{留}}}{とゞ}むれば、% TODO 原本の「二の字点、揺すり点」に濁点のグリフが見つからないので「ゞ」
%
\ruby{彼方}{かな|た}も
\ruby{人}{ひと}ありと
\ruby{知}{し}つて
\ruby{此方}{こな|た}を
\ruby{見}{み}かへりたり。
%
\ruby{見}{み}ざりし
\ruby{程}{ほど}こそ
\ruby{心}{こ〻ろ}も% 原本通り「〻(二の字点、揺すり点)」
\ruby{常}{つね}なり
つれ、
%
\ruby{相}{あひ}
\ruby{見}{み}ては
\ruby{互}{たがひ}に
ハツと
\ruby{驚}{おどろ}きて、
%
\ruby{彼方}{かな|た}は
\ruby{面}{おもて}を
\原本頁{77-4}\改行%
\ruby{掩}{おほ}ひ、
%
\ruby{我}{われ}は
\ruby{胸}{むね}を
\ruby{轟}{とゞろ}かす。% TODO 原本の「二の字点、揺すり点」に濁点のグリフが見つからないので「ゞ」
%
\ruby{其}{その}
\ruby{人}{ひと}は
\ruby{少時}{しば|し}も
\ruby{忘}{わす}れぬ
\ruby{我}{わ}が
\ruby{五十子}{い|そ|こ}
なれば、
%
\ruby{何}{なに}として
\ruby{此處}{こ|〻}にはと% 原本通り「〻(二の字点、揺すり点)」
\ruby{先}{ま}づ
\ruby{走}{はし}り
\ruby{寄}{よ}つて、
%
\ruby{慌}{あわ}て〻% 原本通り「〻(二の字点、揺すり点)」
\ruby{扶}{たす}け
\ruby{起}{おこ}さんと
\ruby{其}{その}
\ruby{手}{て}を
\ruby{執}{と}れば、
%
\ruby{我}{わ}が
\ruby{手}{て}に
\ruby{他}{ひと}の
\ruby{手}{て}の
\ruby{觸}{ふ}る〻や% 原本通り「〻(二の字点、揺すり点)」
\ruby{觸}{ふ}れぬに、
%
\ruby{彼}{か}の
\ruby{妖精}{も|の}は
\ruby{異樣}{こと|やう}に
\ruby{高笑}{たか|わら}ひして、

\原本頁{77-8}%
『
\ruby{見}{み}よ、
%
\ruby{地}{つち}より
\ruby{出}{い}でしものよ、
%
\ruby{地}{つち}
\ruby{戀}{こひ}しきかよ。
%
\ruby{我}{われ}
\ruby{見}{み}ん、
%
\ruby{石}{いし}の
\ruby{落}{お}ちて
\ruby{那處}{いづ|く}に
\ruby{至}{いた}るかを。
』

\原本頁{77-10}%
と
\ruby{{\換字{勝}}}{か}ち
\ruby{誇}{ほこ}れるが
\ruby{如}{ごと}く
\ruby{嘲}{あざ}み
\ruby{罵}{の〻し}る% 原本通り「〻(二の字点、揺すり点)」
\ruby{其}{そ}の
\ruby{聲}{こゑ}
\ruby{耳}{み〻}に% 原本通り「〻(二の字点、揺すり点)」
\ruby{徹}{てつ}する
\ruby{{\換字{途}}端}{と|たん}、
%
\ruby{忽地}{たちま|ち}に
\ruby{身}{み}は
\ruby{{\換字{鉛}}}{なまり}よりも
\ruby{重}{おも}くなりて、
%
\ruby{大地}{だい|ち}は
\ruby{{\換字{雪}}}{ゆき}より
\ruby{柔}{やはら}かになり、
%
\ruby{見}{み}る〳〵
\ruby{地}{ち}
\原本頁{74-1}\改行%
\ruby{窪}{くぼ}み
\ruby{身}{み}は
\ruby{陷}{おちゐ}つて、
%
\ruby[<j>]{踝}{くるぶし}
\ruby{沒}{かく}れ、
%
\ruby{脛}{はぎ}
\ruby{沒}{かく}れ、
%
\ruby{膝皿}{ひざ|〻ら}% 原本通り「〻(二の字点、揺すり点)」
\ruby{沒}{かく}れ、
%
\ruby{高腿}{たか|も〻}% 原本通り「〻(二の字点、揺すり点)」
\ruby{沒}{かく}れ、
%
\ruby{腹}{はら}
\ruby{沒}{かく}れ、
%
\ruby{胸}{むね}
\ruby{沒}{かく}れ
%
\ruby{肩}{かた}
\ruby{沒}{かく}れ
%
\ruby{行}{ゆ}きて、
%
\ruby{石人}{せき|じん}の
\ruby{水}{みづ}に
\ruby{沈}{しづ}むが
\ruby{如}{ごと}くに、
%
\ruby{全}{まつた}く
\ruby{自}{みづか}ら
\ruby{支}{さ〻}ふるに% 原本通り「〻(二の字点、揺すり点)」
\ruby[<j||]{力}{ちから}
\ruby{無}{な}く、

\原本頁{74-4}%
『
\ruby{水野}{みづ|の}
ツ』

\原本頁{74-5}%
と
\ruby{呼}{よ}ぶ
\ruby{日方}{ひ|かた}の
\ruby{聲}{こゑ}、

\原本頁{74-6}%
『
オーイ、
オーイ、
』

\原本頁{74-7}%
と
\ruby{喚}{よ}ぶ
\ruby{島木}{しま|き}が
\ruby{聲}{こゑ}を
\ruby{遙}{はるか}に〳〵
\ruby{聞}{き}きながら、
%
\ruby{次第}{し|だい}に
\ruby{現世}{うつし|よ}には
\ruby{{\換字{遠}}}{とほ}ざかりて、
%
\ruby{漸}{やうや}く
\ruby{奈落}{な|らく}の
\ruby{底}{そこ}に
\ruby{沈}{しづ}み
\ruby{行}{ゆ}かんとす。
%
\ruby{今}{いま}は
\ruby{氣}{き}も
\ruby{心}{こ〻ろ}も% 原本通り「〻(二の字点、揺すり点)」
\ruby{{\換字{消}}}{き}え〴〵に
なりて、
%
\ruby{思}{おも}はずも
\ruby{南無}{な|む}と
\ruby{叫}{よ}びか〻る% 原本通り「〻(二の字点、揺すり点)」
\ruby{時}{とき}、
%
\ruby{駈}{か}けつけ
\ruby{吳}{く}れたる
\ruby{我}{わ}が
\ruby{羽{\換字{勝}}}{は|がち}の、
%
ムヅと
\ruby{我}{わ}が
\ruby{頭髮}{か|み}を
\ruby{引攫}{ひつ|〻か}みて、% 原本通り「〻(二の字点、揺すり点)」
%
\ruby{鐵腕}{てつ|わん}の
\ruby{力}{ちから}の
\ruby{恐}{おそ}ろしく
\ruby{凄}{すさま}じくも、
%
ふた〻び% 原本通り「〻(二の字点、揺すり点)」
\ruby{我}{われ}を
\ruby{光}{ひかり}ある
\ruby{世}{よ}に
\ruby{飛礫}{つぶ|て}の
\ruby{如}{ごと}く
\ruby{投}{な}げ
\ruby{上}{あ}げくれたるが、
%
\ruby{投}{な}げられて
\ruby{{\換字{空}}}{そら}を
\ruby{飛}{と}べる
\ruby{身}{み}の
\ruby{呀}{あつ}と
\ruby{驚}{おどろ}きて
\ruby{我}{われ}に
\ruby{{\換字{返}}}{かへ}れば、
%
これは
\原本頁{75-2}\改行%
\ruby{是}{これ}
\ruby{思}{おも}ひ
\ruby{寢}{ね}の
\ruby{惡夢}{あく|む}にして、
%
\ruby{滿身}{まん|しん}の
\ruby{汗}{あせ}
\ruby{絞}{しぼ}るが
\ruby{如}{ごと}く、
%
\ruby{胸}{むね}は
\ruby{今}{いま}
\ruby{{\換字{猶}}}{なほ}
\ruby{浪}{なみ}
\ruby{打}{う}つて
\ruby{騷}{さわ}ぎ、
%
\ruby{枕頭}{ちん|とう}の
\ruby{燈}{ひ}は
\ruby{靑}{あを}くして
\ruby{幽}{かすか}に、
%
\ruby{恰}{あたか}も% 恰も「あ(た)かも」
\ruby{鳴}{な}り
\ruby{出}{いだ}せる
\ruby{茶}{ちや}の
\ruby{室}{ま}の
\ruby{時計}{と|けい}は、
%
\ruby{一}{ひと}ツ、
%
\ruby{二}{ふた}ツ、
%
\ruby{三}{みつ}ツにして
\ruby{復}{また}
\ruby{默}{もく}したり。

\原本頁{75-5}%
\ruby{吉右衛門}{きち||ゑ|もん}は
\ruby{眠}{ねむ}れり、
%
お
\ruby{濱}{はま}は
\ruby{眠}{ねむ}れり、
%
\ruby{日方}{ひ|かた}も
\ruby{島木}{しま|き}も
\ruby{將}{はた}
\ruby{羽{\換字{勝}}}{は|がち}も
\ruby{今}{いま}は
\ruby{思}{おも}ふに
\ruby{睡}{ねむ}れるならん。
%
\ruby{憎}{にく}き
お
\ruby{澤}{さは}
\ruby{婆}{ば〻}も% 「ばゞ」のはずだが、原本通り「〻(二の字点、揺すり点)」
\ruby{睡}{ねむ}れる
ならん。
%
\ruby{可憫}{か|はゆ}き
\ruby{松之助}{まつ|の|すけ}も
\ruby{睡}{ねむ}れる
ならん。
%
\ruby{醫}{い}も
\ruby{睡}{ねむ}れる
ならん。
%
\ruby{看護{\換字{婦}}}{かん|ご|ふ}も
\ruby{睡}{ねむ}れる
ならん。
%
\原本頁{75-8}\改行%
\ruby{覺}{さ}めたるものは
\ruby{我}{われ}のみなるが、
%
たゞ% TODO 原本の「二の字点、揺すり点」に濁点のグリフが見つからないので「ゞ」
\ruby{我}{わ}が
\ruby{病}{やまひ}の
\ruby{蓐}{とこ}に
\ruby{惱}{なや}める
\ruby{五十子}{い|そ|こ}は、
%
\ruby{睡}{ねむ}れりや
\ruby{如何}{い|か}に、
%
\ruby{穩}{おだ}やかに
\ruby{睡}{ねむ}れりや。

\原本頁{75-10}%
\ruby{恐}{おそ}ろしき
\ruby{夢}{ゆめ}に
\ruby{魘}{おそ}はれし
\ruby{水野}{みづ|の}は、
%
\ruby{夢}{ゆめ}の
\ruby{根基}{もと|ひ}と
なりし
\ruby{{\換字{宵}}}{よひ}の
\ruby{談話}{はな|し}を
\ruby{獨}{ひと}り
\ruby{靜}{しづか}に
\ruby{思}{おも}ひ
\ruby{{\換字{返}}}{かへ}して、
%
さま〴〵に
\ruby{思}{おも}ひ
\ruby{亂}{みだ}る〻% 原本通り「〻(二の字点、揺すり点)」
\ruby{折}{をり}しも、
%
いつまで
\ruby{睡}{ねむ}らで
\ruby{吠}{ほ}ゆる
\ruby{彼}{か}の
\ruby{狗}{いぬ}
なるぞや、
%
また
\ruby{彼}{か}の
\ruby{狗}{いぬ}の
\ruby{聲}{こゑ}は
べう〳〵と
\ruby{聞}{きこ}えぬ。
%
か〻る% 原本通り「〻(二の字点、揺すり点)」
\ruby{夜深}{よ|ふか}きに
\ruby{何}{なに}をか
\ruby{見}{み}て
\ruby{吠}{ほ}えし、
%
\ruby{人}{ひと}の
\ruby{魂魄}{た|ま}にても
\ruby{飛}{と}びたるかや、
%
あ〻。% 原本通り「〻(二の字点、揺すり点)」
