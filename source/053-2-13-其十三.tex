\Entry{其十三}

% メモ 校正終了 2024-04-19 2024-05-31 2024-07-01
\原本頁{76-5}%
\ruby{何}{なん}となく
\ruby{五十子}{い|そ|こ}が
\ruby{上}{うへ}の
あやしく
\ruby{氣}{き}に
かゝりて、% 踊り字調整「〻(二の字点、揺すり点)に見えるが(ゝ)」
%
\ruby[g]{水野}{みづの }は
\ruby{睡}{ねむ}らん
としても
また
\ruby{睡}{ねむ}られず、
%
\ruby{{\換字{若}}}{もし}や
\ruby{彼}{か}の
\ruby{人}{ひと}の
\ruby[g]{病狀}{やうす }に
\ruby{變}{へん}などありて、
%
\ruby{今}{いま}を
\ruby[g]{生死}{しやうし}の
\ruby[g]{瀬{\換字{戸}}}{せ と }と
\ruby{苦}{くるし}めるには
あらずや。
%
\ruby{此}{こ}の
\ruby[g]{一日}{いちにち}は
\ruby{快}{よ}きやう
なりしが、
%
\ruby[g]{此際}{こ ゝ }% 踊り字調整「〻(二の字点、揺すり点)に見えるが(ゝ)」
\ruby[g]{數日}{しばらく}を
\ruby{病}{やまひ}の
\ruby{峠}{たふげ}とは、% ルビ調整(原本通り)(たふげ)
%
\ruby[g]{尾竹}{を だけ}も
\ruby{云}{い}ひたる
ところ
なれば、
%
\ruby{取}{と}り
\ruby{別}{わ}けて
\ruby[||j>]{心}{こゝろ}% 踊り字調整「〻(二の字点、揺すり点)に見えるが(ゝ)」
\ruby[||j>]{元}{ もと}
\ruby[||j>]{無}{ な}く
\ruby{思}{おも}はるゝかな。% 踊り字調整「〻(二の字点、揺すり点)に見えるが(ゝ)」
%
\ruby{狗}{いぬ}の
\ruby{長}{なが}
\ruby{吠}{ぼ{\換字{𛀁}}}する
\ruby{時}{とき}は
\ruby{凶}{あし}き
\ruby{事}{こと}あり
といふ
\ruby[||j>]{俗}{いひ}
\ruby[||j>]{說}{つたへ}も、
% \ruby{俗說}{いひ|つたへ}も、
%
\ruby{取}{と}るに
\ruby{足}{た}らぬ
\ruby[g]{{\換字{迷}}信}{まよひ }なり
とは
\ruby{知}{し}りながら、
%
さし
\ruby{當}{あた}つて
\ruby{今}{いま}は
\ruby{忌}{いま}はしくぞ
\ruby{覺}{おぼ}ゆる。
%
\ruby{熱}{ねつ}の
\ruby{高}{たか}きには
\ruby{心}{こゝろ}も% 踊り字調整「〻(二の字点、揺すり点)に見えるが(ゝ)」
\ruby{亂}{みだ}れて、
%
\ruby{夢}{ゆめ}は
\ruby{駈}{か}け
\ruby{回}{めぐ}る% 原本通り「回」
\ruby[g]{曠野}{あらの }の
\ruby{夏}{なつ}に、
%
\ruby[g]{火炎}{ほのほ }と
\ruby{熱}{あつ}き
\ruby{息}{いき}を
\ruby{吐}{つ}きつゝ、% 踊り字調整「〻(二の字点、揺すり点)に見えるが(ゝ)」
%
\ruby[g]{淸水}{し みづ}
\ruby{{\換字{尋}}}{たづ}ね
わづらふ
\ruby{思}{おもひ}に
\ruby{悶}{もだ}え、
%
\ruby[g]{夜具}{や ぐ }に
\ruby{身}{み}を
\ruby{餘}{あま}して
\ruby{我}{われ}
\ruby{知}{し}らず
\ruby{呻}{うめ}く、
%
\ruby{其}{そ}の
\ruby{苦}{くる}しさの
\ruby[g]{經驗}{おぼ{\換字{𛀁}}}は
\ruby{我}{われ}も
\ruby{知}{し}れるが、
%
\ruby{我}{わ}が
\ruby{五十子}{い|そ|こ}は
\ruby{今}{いま}
さる
\ruby{事}{こと}も
なくて、
%
\ruby[<j>]{幸}{さいはひ}に
すやすやと
\ruby{睡}{ねむ}れりや
\ruby[g]{如何}{い か }に。
%
\ruby{安}{やす}らかに
\ruby[||j>]{病}{びやう}
\ruby[||j>]{人}{ にん}の
% \ruby{病人}{びやう|にん}の
\ruby{睡}{ねむ}り
\ruby{居}{を}らば、
%
それより
\原本頁{77-6}\改行%
\ruby{頼}{たの}もしく
\ruby{好}{よ}き
\ruby{事}{こと}は
\ruby{無}{な}けれど、
%
\ruby{或}{あるひ}は
\ruby{{\換字{又}}}{また}
ほそ〴〵と
\ruby{癯}{や}せし
\ruby[g]{手先}{て さき}の、
%
\ruby{物}{もの}
あはれにも
\ruby{枕}{まくら}の
\ruby{端}{はし}
なんどを
\ruby[||j>]{力}{ちから}
\ruby[||j>]{草}{ ぐさ}に
\ruby{執}{と}り
\ruby{絞}{しぼ}りて、
%
\ruby{苦}{くる}しさに
\ruby{堪}{こら}へ
\原本頁{77-8}\改行%
\ruby{堪}{こら}へし
\ruby{果}{は}ては、
%
\ruby{睡}{ねむ}りも
\ruby{睡}{ねむ}り
\ruby{得}{{\換字{𛀁}}}ず、
%
\ruby{醒}{さ}めも
\ruby{醒}{さ}めやらず
なりて、
%
た
だ% ルビ調整(原本通り)非踊り字表記(行末行頭の境界付近)
うつら〳〵と
\ruby[g]{病苦}{びやうく}に
\ruby{責}{せ}められ、
%
\ruby[g]{一{\換字{半}}}{なかば }は
\ruby{現}{うつゝ}、% 踊り字調整「〻(二の字点、揺すり点)に見えるが(ゝ)」
%
\ruby[g]{一{\換字{半}}}{なかば }は
\ruby{夢}{ゆめ}の、
%
\ruby[g]{精神}{こゝろ }は% 踊り字調整「〻(二の字点、揺すり点)に見えるが(ゝ)」
\ruby{幽}{かすか}に
\ruby{{\換字{消}}}{き}え
かゝりて、% 踊り字調整「〻(二の字点、揺すり点)に見えるが(ゝ)」
%
\ruby[g]{現世}{このよ }
\ruby[g]{冥{\換字{途}}}{あのよ }の
\ruby[g]{境界}{さかひ }の
\ruby{上}{うへ}に、
%
\ruby[g]{魂魄}{たましひ}
\ruby{{\換字{迷}}}{まよ}へる
やうに
あらば、
%
あゝ% 踊り字調整「〻(二の字点、揺すり点)に見えるが(ゝ)」
\ruby[g]{如何}{い か }にせん、
%
\ruby[g]{如何}{い か }にせん。
%
おもへば
\ruby{何}{なに}と
なるべき
\ruby{彼}{か}の
\ruby{人}{ひと}の
\ruby{上}{うへ}、
%
\ruby{我}{わ}が
\ruby{上}{うへ}ぞや。
%
\ruby{{\換字{前}}}{さき}の
\ruby{世}{よ}の
\ruby{有}{あ}りや
\ruby{無}{な}しや、
%
それも
\ruby{知}{し}らず、
%
\ruby{後}{のち}の
\ruby{世}{よ}の
\ruby{有}{あ}りや
\ruby{無}{な}しや、
%
それも
\ruby{知}{し}らねど、
%
\ruby{{\換字{若}}}{も}し
\ruby{今}{いま}
\ruby{彼}{か}の
\ruby{人}{ひと}の
\ruby{此}{こ}の
\ruby{世}{よ}を
\ruby{去}{さ}らば、
%
\ruby{我}{わ}が
\ruby{身}{み}は
\ruby[g]{此世}{こ ゝ }にも% 踊り字調整「〻(二の字点、揺すり点)に見えるが(ゝ)」
\ruby{{\換字{遺}}}{のこ}る
べけれど、
%
\ruby{我}{わ}が
\ruby{魂}{たま}は
\原本頁{78-4}\改行%
おのづと
\ruby{他}{ひと}に
\ruby{引}{ひ}かれて、
%
\ruby{必}{かなら}ず
\ruby[g]{冥{\換字{途}}}{あのよ }に
\ruby{去}{さ}る
べければ、
%
\ruby{其}{そ}の
\ruby{後}{のち}の
\ruby{世}{よ}を
\ruby{思}{おも}ふに
つけても、
%
\ruby{此}{こ}の
\ruby{世}{よ}に
\ruby{我}{われ}の
\ruby{現}{あらは}れ
\ruby{出}{い}でしは、
%
\ruby{此}{こ}の
\ruby{世}{よ}に
\ruby{彼}{か}の
\ruby{人}{ひと}の
\ruby{出}{い}でし
がためかと、
%
\ruby{思}{おも}ひやらるゝ% 踊り字調整「〻(二の字点、揺すり点)に見えるが(ゝ)」
\ruby[g]{心地}{こゝち }のして、% 踊り字調整「〻(二の字点、揺すり点)に見えるが(ゝ)」
%
\ruby{{\換字{遠}}}{とほ}く
\ruby[g]{邈焉}{はるか }なる
\ruby{{\換字{前}}}{さき}の
\ruby{世}{よ}にも、
%
\ruby{彼}{か}の
\ruby{人}{ひと}は
\ruby{今}{いま}と
\ruby{同}{おな}じく
\ruby{病}{やまひ}に
\ruby{惱}{なや}み、
%
\ruby{我}{われ}は
\ruby{今}{いま}と
\ruby{同}{おな}じく
\原本頁{78-8}\改行%
\ruby{戀}{こひ}に
\ruby{泣}{な}きし
\ruby{悲}{かな}しき
ありさまの、
%
あり〳〵と
\ruby{此}{こ}の
\ruby{心}{こゝろ}に% 踊り字調整「〻(二の字点、揺すり点)に見えるが(ゝ)」
\ruby{{\換字{浮}}}{うか}び
\ruby{來}{く}るなり。
%
よしや、
%
\ruby{{\換字{前}}}{まへ}の
\ruby{世}{よ}の
\ruby{緣}{{\換字{𛀁}}ん}の
\ruby{悲}{かな}しくも
あれ、
%
\ruby{此}{この}
\ruby{世}{よ}の
\ruby{緣}{{\換字{𛀁}}ん}も
また
\ruby[g]{果敢}{は か }なくも
あれ、
%
\ruby[g]{眞實}{まこと }に
\ruby{{\換字{前}}}{まへ}の
\ruby{世}{よ}の
\ruby{存}{あ}りもせよかし。
%
\ruby{{\換字{前}}}{まへ}の
\ruby{世}{よ}
\ruby[g]{眞實}{まこと }にあらば
\ruby{後}{のち}の
\ruby{世}{よ}もあり、
%
\ruby{其}{そ}の
また
\ruby{後}{のち}の
\ruby{世}{よ}も
あらんに、
%
せめては
\ruby{其}{それ}を
\原本頁{79-1}\改行%
\ruby{頼}{たの}みには
して、
%
\ruby{長}{なが}く
\ruby{盡}{つ}きざる
\ruby{我}{わが}
\ruby{思}{おも}ひの
\ruby[g]{如何}{い か }にか
\ruby{成}{な}り
\ruby{行}{ゆ}く
\ruby{涯}{はて}を
\ruby{見}{み}ん。
%
あゝ% 踊り字調整「〻(二の字点、揺すり点)に見えるが(ゝ)」
\ruby{其}{それ}につけ
\ruby{此}{これ}につけても、
%
\ruby[g]{如何}{い か }なれば
\ruby{此}{こ}の
\ruby{胸}{むね}の
\ruby[g]{如是}{か く }は
\ruby{騷}{さわ}ぎて、
%
\ruby[g]{動悸}{どうき }の
\ruby{浪}{なみ}の
たゞ% 踊り字調整「〻(二の字点、揺すり点)に濁点に見えるが(ゞ)」
ならず
\ruby{打}{う}つ
\ruby{事}{こと}よ。
%
あゝ% 踊り字調整「〻(二の字点、揺すり点)に見えるが(ゝ)」
\ruby{何}{なん}となく
\ruby[||j>]{心}{こゝろ}% 踊り字調整「〻(二の字点、揺すり点)に見えるが(ゝ)」
\ruby[||j>]{悲}{ かな}しく
\ruby{物}{もの}
\ruby{恐}{おそ}ろしき
\ruby{{\換字{感}}}{おもひ}のするかな。
%
\ruby{{\換字{若}}}{もし}くは
\ruby{彼}{か}の
\ruby{人}{ひと}の
\ruby{何}{なに}とか
なせるには
あらずや、
%
\ruby{居}{ゐ}ても
\ruby{立}{た}つても
\ruby{心}{こゝろ}の% 踊り字調整「〻(二の字点、揺すり点)に見えるが(ゝ)」
\ruby{安}{やす}からぬ。
%
あゝ% 踊り字調整「〻(二の字点、揺すり点)に見えるが(ゝ)」
\ruby{何}{なん}とせん、
%
\ruby{何}{なん}となさん。
%
と、
%
とつ
\ruby{置}{おい}つ
\ruby{思}{おも}ひ
\ruby{{\換字{迷}}}{まよ}ひしが、
%
\ruby[g]{此處}{こ ゝ }に% 踊り字調整「〻(二の字点、揺すり点)に見えるが(ゝ)」
ありて
\ruby{{\換字{空}}}{あだ}に
\ruby{悶}{もだ}えん
よりは、
%
\ruby[g]{其處}{そ こ }に
\ruby{至}{いた}りて、
%
\ruby[g]{狀態}{ありさま}を
\ruby{伺}{うかゞ}はんと、% 踊り字調整「〻(二の字点、揺すり点)に濁点に見えるが(ゞ)」
%
\ruby{{\換字{終}}}{つひ}に
\ruby{衣}{い}を
かへて
\ruby{立}{たち}
\ruby{出}{い}でたり。

\原本頁{79-9}%
\ruby{風}{かぜ}も
\ruby{眠}{ねむ}れり、
%
\ruby{石}{いし}も
\ruby{眠}{ねむ}れり。
%
\ruby{誰}{たれ}か
かゝる% 踊り字調整「〻(二の字点、揺すり点)に見えるが(ゝ)」
\ruby{時}{とき}
\ruby[g]{{\換字{戸}}外}{おもて }に
\ruby{在}{あ}らんや。
%
\ruby[g]{{\換字{戸}}締}{と じま}りの
\ruby[g]{如何}{い か }に
すべきも
\ruby[g]{打忘}{うちわす}れて、
%
ふら〳〵と
\ruby[g]{立出}{たちい }でたる
\ruby[g]{水野}{みづの }の
\ruby{{\換字{道}}}{みち}
\ruby{行}{ゆ}く
\ruby[g]{樣子}{やうす }は、
%
たとへば
\ruby{影}{かげ}のみの
\ruby{人}{ひと}の
\ruby{如}{ごと}く、
%
\ruby[g]{現世}{このよ }のものとも
\ruby{思}{おも}はれざりしが、
%
\ruby[g]{水野}{みづの }も
\ruby{既}{すで}に
\ruby[g]{此世}{このよ }を
\ruby{忘}{わす}れて、
%
\ruby{{\換字{空}}}{そら}は
\ruby{今}{いま}
\ruby{曇}{くも}れりや
\ruby{星}{ほし}ありや、
%
\ruby{闇}{やみ}なりや
\ruby{將}{はた}
\ruby{月}{つき}ありやも、
%
\ruby{全}{まつた}く
\ruby{心}{こゝろ}には% 踊り字調整「〻(二の字点、揺すり点)に見えるが(ゝ)」
\ruby{{\換字{留}}}{と}めざるなりけり。
