\Entry{其十三}

\原本頁{}
\ruby{何}{なん}となく
\ruby[g]{五十子}{いそこ}が
\ruby{上}{うへ}のあやしく
\ruby{氣}{き}にかゝりて、
%
\ruby[g]{水野}{みづの}は
\ruby{睡}{ねむ}らんとしてもまた
\ruby{睡}{ねむ}られず、
%
\ruby{{\換字{若}}}{もし}や
\ruby{彼}{か}の
\ruby{人}{ひと}の
\ruby{病狀}{やう|す}に
\ruby{變}{へん}などありて、
%
\ruby{今}{いま}を
\ruby{生死}{しやう|し}の
\ruby{瀬{\換字{戸}}}{せ|と}と
\ruby{苦}{くる}しめるにはあらずや。
%
\ruby{此}{こ}の
\ruby{一日}{いち|にち}は
\ruby{快}{よ}きやうなりしが、
%
\ruby{此際數日}{こ|ゝ|しば|らく}を
\ruby{病}{やまひ}の
\ruby{峠}{たうげ}とは、
%
\ruby[g]{尾竹}{をだけ}も
\ruby{云}{い}ひたるところなれば、
%
\ruby{取}{と}り
\ruby{別}{わ}けて
\ruby{心元無}{こゝろ|もと|な}く
\ruby{思}{おも}はるゝかな。
%
\ruby{狗}{いぬ}の
\ruby{長吠}{なが|ぼ\換字{𛀁}}する
\ruby{時}{とき}は
\ruby{凶}{あし}き
\ruby{事}{こと}ありといふ
\ruby{俗說}{いひ|つたへ}も、
%
\ruby{取}{と}るに
\ruby{足}{た}らぬ
\ruby{{\換字{迷}}信}{まよ|ひ}なりとは
\ruby{知}{し}りながら、
%
さし
\ruby{當}{あ}たつて
\ruby{今}{いま}は
\ruby{忌}{いま}はしくぞ
\ruby{覺}{おぼ}ゆる。
%
\ruby{熱}{ねつ}の
\ruby{高}{たか}きには
\ruby{心}{こゝろ}も
\ruby{亂}{みだ}れて、
%
\ruby{夢}{ゆめ}は
\ruby{駈}{か}け
\ruby{回}{めぐ}る% 原本通り「回」
\ruby{曠野}{あら|の}の
\ruby{夏}{なつ}に、
%
\ruby{火炎}{ほの|ほ}と
\ruby{熱}{あつ}き
\ruby{息}{いき}を
\ruby{吐}{つ}きつゝ、
%
\ruby{淸水}{し|みづ}
\ruby{{\換字{尋}}}{たづ}ねわづらふ
\ruby{思}{おもひ}に
\ruby{悶}{もだ}え、
%
\ruby{夜具}{や|ぐ}に
\ruby{身}{み}を
\ruby{餘}{あま}して
\ruby{我}{われ}
\ruby{知}{し}らず
\ruby{呻}{うめ}く、
%
\ruby{其}{そ}の
\ruby{苦}{くる}しさの
\ruby{經驗}{おぼ|\換字{𛀁}}は
\ruby{我}{われ}も
\ruby{知}{し}れるが、
%
\ruby{我}{わ}が
\ruby[g]{五十子}{いそこ}は
\ruby{今}{いま}さる
\ruby{事}{こと}もなくて、
%
\ruby{幸}{さいはひ}にすやすやと
\ruby{睡}{ねむ}れりや
\ruby{如何}{い|か}に。
%
\ruby{安}{やす}らかに
\ruby{病人}{びやう|にん}の
\ruby{睡}{ねむ}り
\ruby{居}{を}らば、
%
それより
\ruby{頼}{たの}もしき
\ruby{好}{よ}き
\ruby{事}{こと}は
\ruby{無}{な}けれど、
%
\ruby{或}{あるひ}は
\ruby{{\換字{又}}}{また}ほそ〴〵と
\ruby{癯}{や}せし
\ruby{手先}{て|さき}の、
%
\ruby{物}{もの}あはれにも
\ruby{枕}{まくら}の
\ruby{端}{はし}なんどを
\ruby[<h||]{力}{ちから}
\ruby{草}{ぐさ}に
\ruby{執}{と}り
\ruby{絞}{しぼ}りて、
%
\ruby{苦}{くる}しさに
\ruby{堪}{こら}へ
\ruby{堪}{こら}へし
\ruby{果}{は}ては、
%
\ruby{睡}{ねむ}りも
\ruby{睡}{ねむ}り
\ruby{得}{\換字{𛀁}}ず、
%
\ruby{醒}{さ}めも
\ruby{醒}{さ}めやらずなりて、
%
たゞうつら〳〵と
\ruby{病苦}{びやう|く}に
\ruby{責}{せ}められ、
%
\ruby{一{\換字{半}}}{なか|ば}は
\ruby{現}{うつゝ}、
%
\ruby{一{\換字{半}}}{なか|ば}は
\ruby{夢}{ゆめ}の、
%
\ruby{精神}{こゝ|ろ}は
\ruby{幽}{かすか}に
\ruby{{\換字{消}}}{き}えかゝりて、
%
\ruby{現世}{この|よ}
\ruby{冥{\換字{途}}}{あの|よ}の
\ruby{境界}{さ|かひ}の
\ruby{上}{うへ}に、
%
\ruby{魂魄}{たま|しひ}
\ruby{{\換字{迷}}}{まよ}へるやうにあらば、
%
あ〻
\ruby{如何}{い|か}にせん、
%
\ruby{如何}{い|か}にせん。
%
おもへば
\ruby{何}{なに}となるべき
\ruby{彼}{か}の
\ruby{人}{ひと}の
\ruby{上}{うへ}、
%
\ruby{我}{わ}が
\ruby{上}{うへ}ぞや。
%
\ruby{{\換字{前}}}{さき}の
\ruby{世}{よ}の
\ruby{有}{あ}りや
\ruby{無}{な}しや、
%
それも
\ruby{知}{し}らず、
%
\ruby{後}{のち}の
\ruby{世}{よ}の
\ruby{有}{あ}りや
\ruby{無}{な}しや、
%
\ruby{知}{し}らねど、
%
\ruby{{\換字{若}}}{も}し
\ruby{今}{いま}
\ruby{彼}{か}の
\ruby{人}{ひと}の
\ruby{此}{こ}の
\ruby{世}{よ}を
\ruby{去}{さ}らば、
%
\ruby{我}{わ}が
\ruby{身}{み}は
\ruby{此世}{この|よ}にも
\ruby{{\換字{遺}}}{のこ}るべけれど、
%
\ruby{我}{わ}が
\ruby{魂}{たま}はおのづと
\ruby{他}{ひと}に
\ruby{引}{ひ}かれて、
%
\ruby{必}{かなら}ず
\ruby{冥{\換字{途}}}{あの|よ}に
\ruby{去}{さ}るべければ、
%
\ruby{其}{そ}の
\ruby{後}{のち}の
\ruby{世}{よ}を
\ruby{思}{おも}ふにつけても、
%
\ruby{此}{こ}の
\ruby{世}{よ}に
\ruby{我}{われ}の
\ruby{現}{あらは}れ
\ruby{出}{い}でしは、
%
\ruby{此}{こ}の
\ruby{世}{よ}に
\ruby{彼}{か}の
\ruby{人}{ひと}の
\ruby{出}{い}でしがためかと、
%
\ruby{思}{おも}ひやらるゝ
\ruby{心地}{こゝ|ち}のして、
%
\ruby{{\換字{遠}}}{とほ}く
\ruby{邈焉}{はる|か}なる
\ruby{{\換字{前}}}{さき}の
\ruby{世}{よ}にも、
%
\ruby{彼}{か}の
\ruby{人}{ひと}は
\ruby{今}{いま}と
\ruby{同}{おな}じく
\ruby{病}{やまひ}に
\ruby{惱}{なや}み、
%
\ruby{我}{われ}は
\ruby{今}{いま}と
\ruby{同}{おな}じく
\ruby{戀}{こひ}に
\ruby{泣}{な}きし
\ruby{悲}{かな}しきありさまの、
%
あり〳〵と
\ruby{此}{こ}の
\ruby{心}{こゝろ}に
\ruby{{\換字{浮}}}{うか}び
\ruby{來}{く}るなり。
%
よしや、
%
\ruby{{\換字{前}}}{まへ}の
\ruby{世}{よ}の
\ruby{緣}{\換字{𛀁}ん}の
\ruby{悲}{かな}しくもあれ、
%
\ruby{此}{こ}の
\ruby{世}{よ}の
\ruby{緣}{\換字{𛀁}ん}もまた
\ruby{果敢}{は|か}なくもあれ、
%
\ruby[g]{眞實}{まこと}に
\ruby{{\換字{前}}}{まへ}の
\ruby{世}{よ}の
\ruby{存}{あ}りもせよかし。
%
\ruby{{\換字{前}}}{まへ}の
\ruby{世}{よ}
\ruby[g]{眞實}{まこと}にあらば
\ruby{後}{のち}の
\ruby{世}{よ}もあり、
%
\ruby{其}{そ}のまた
\ruby{後}{のち}の
\ruby{世}{よ}もあらんに、
%
せめては
\ruby{其}{それ}を
\ruby{頼}{たの}みにはして、
%
\ruby{長}{なが}く
\ruby{盡}{つ}きざる
\ruby{我思}{わが|おも}ひの
\ruby{如何}{い|か}にか
\ruby{成}{な}り
\ruby{行}{ゆ}く
\ruby{涯}{はて}を
\ruby{見}{み}ん。
%
あ〻
\ruby{其}{それ}につけ
\ruby{此}{これ}につけても、
%
\ruby{如何}{い|か}なれば
\ruby{此}{こ}の
\ruby{胸}{むね}の
\ruby{如是}{か|く}は
\ruby{騷}{さわ}ぎて、
%
\ruby{動悸}{どう|き}の
\ruby{浪}{なみ}の
たゞならず
\ruby{打}{う}つ
\ruby{事}{こと}よ。
%
あ〻
\ruby{何}{なん}となく
\ruby{心悲}{こゝろ|かな}しく
\ruby{物恐}{もの|おそ}ろしき
\ruby{感}{おもひ}のするかな。
%
\ruby{{\換字{若}}}{も}しくは
\ruby{彼}{か}の
\ruby{人}{ひと}の
\ruby{何}{なに}とかなせるにはあらずや、
%
\ruby{居}{ゐ}ても
\ruby{立}{た}つても
\ruby{心}{こゝろ}の
\ruby{安}{やす}からぬ。
%
あ〻
\ruby{何}{なん}とせん、
%
あ〻
\ruby{何}{なん}となさん。
%
と、
%
とつ
\ruby{置}{おい}つ
\ruby{思}{おも}ひ
\ruby{{\換字{迷}}}{まよ}ひしが、
%
\ruby{此處}{こ|ゝ}にありて
\ruby{{\換字{空}}}{あだ}に
\ruby{悶}{もだ}えんよりは、
%
\ruby{其處}{そ|こ}に
\ruby{至}{いた}りて、
%
\ruby{狀態}{あり|さま}を
\ruby{伺}{うかゞ}はんと、
%
\ruby{{\換字{終}}}{つひ}に
\ruby{衣}{い}をかへて
\ruby{立}{た}ち
\ruby{出}{いで}でたり。

\原本頁{}
\ruby{風}{かぜ}も
\ruby{眠}{ねむ}れり、
%
\ruby{石}{いし}も
\ruby{眠}{ねむ}れり。
%
\ruby{誰}{たれ}かかゝる
\ruby{時}{とき}
\ruby{{\換字{戸}}外}{おも|て}に
\ruby{在}{あ}らんや。
%
\ruby{{\換字{戸}}締}{と|じま}りの
\ruby{如何}{い|か}にすべきも
\ruby{打忘}{うち|わす}れて、
%
ふら〳〵と
\ruby{立出}{たち|い}でたる
\ruby[g]{水野}{みづの}の
\ruby{{\換字{道}}行}{みち|ゆ}く
\ruby{樣子}{やう|す}は、
%
たとへば
\ruby{影}{かげ}のみの
\ruby{人}{ひと}の
\ruby{如}{ごと}く、
%
\ruby{現世}{この|よ}のものとも
\ruby{思}{おも}はれざりしが、
%
\ruby[g]{水野}{みづの}も
\ruby{既}{すで}に
\ruby{此世}{この|よ}を
\ruby{忘}{わす}れて、
%
\ruby{{\換字{空}}}{そら}は
\ruby{今}{いま}
\ruby{曇}{くも}れりや
\ruby{星}{ほし}ありや、
%
\ruby{闇}{やみ}なりや、
%
\ruby{將}{はた}
\ruby{月}{つき}ありやも、
%
\ruby{全}{まつた}く
\ruby{心}{こゝろ}には
\ruby{{\換字{留}}}{と}めざるなりけり。
