\Entry{其十四}

% メモ 校正終了 2024-04-20 2024-05-31
\原本頁{80-4}%
\ruby{{\換字{戸}}}{と}に
\ruby[g]{尾栓}{しりざし}せで
\ruby{濟}{す}む
\ruby[g]{村居}{そんきよ}の
\ruby[||j>]{心}{こゝろ}% 踊り字調整「〻(二の字点、揺すり点)に見えるが(ゝ)」
\ruby[||j>]{安}{ やす}さに
\ruby{慣}{な}れたりとは
\ruby{云}{い}へ、
%
\ruby[g]{{\換字{所}}以}{ゆ ゑ }
\ruby{知}{し}らぬ
\makeatletter
\@ifundefined{デバッグ@ビルド}{%
  \ruby[g]{動悸の}{むなさわぎ }
}{%
  \ruby[||j>]{動}{むな}
  \ruby[||j>]{悸}{さわぎ}の
}%
\makeatother
% \ruby{動悸}{むな|さわぎ}の
\ruby{烈}{はげ}しさに
\ruby{其}{そ}の
\ruby{人}{ひと}の
\ruby{氣}{き}づかはしくて
\ruby{堪}{た}え
\ruby{{\換字{兼}}}{かね}たりとは
\ruby{云}{い}へ
\改行% 校正作業の簡略化のため
、
%
\原本頁{80-6}\改行%
\ruby{時}{とき}ならぬに
\ruby[g]{飄然}{ふらり }と
\ruby[g]{立出}{たちい }でし
\ruby[g]{水野}{みづの }の
\ruby[g]{振舞}{ふるまひ}は、
%
\ruby[g]{日頃}{ひ ごろ}にも
\ruby{似}{に}ぬ
\ruby[g]{仕業}{し わざ}
な
\原本頁{80-7}\改行%
りしが、
%
\ruby{老}{おい}の
\ruby[g]{眼敏}{め ざと}き
\ruby{吉右衛門}{きち||ゑ|もん}は、
%
\ruby[g]{先刻}{さ き }より
\ruby[g]{水野}{みづの }が
\ruby{思}{おも}ひ
あまりて
\改行% 校正作業の簡略化のため
、
%
\原本頁{80-8}\改行%
\ruby{我}{われ}
\ruby{知}{し}らず
\ruby{長吁短歎}{ちやう|く|たん|〳〵}する
\ruby{其}{そ}の
\ruby{聲}{こゑ}を
\ruby{聞}{き}きて
\ruby{既}{すで}に
\ruby{覺}{さ}め
\ruby{居}{を}りつ、
%
\ruby[g]{如何}{い か }ばかり
\ruby{戀}{こひ}の
\ruby[g]{山路}{やまぢ }の
\ruby{嶮}{けは}しきに
\ruby{惱}{なや}んで、
%
\ruby{{\換字{若}}}{わか}き
\ruby{人}{ひと}の
\ruby[g]{可惜}{あたら }
\ruby[<j||]{心}{こゝろ}を% 踊り字調整「〻(二の字点、揺すり点)に見えるが(ゝ)」
\ruby{傷}{きづ}つけ
\ruby{血}{ち}を
\原本頁{80-10}\改行%
\ruby{流}{なが}す
\ruby{事}{こと}ぞやと、
%
ひそかに
\ruby{憐}{あわれ}み
\ruby{居}{ゐ}たりしかば、
%
ごと〳〵と
\ruby[g]{雨{\換字{戸}}}{あまど }
\ruby{引}{ひ}き
\ruby{明}{あ}けて
\ruby{外}{そと}に
\ruby{出}{い}づるをも、
%
\ruby[g]{一度}{ひとたび}は
\ruby{咎}{とが}めんと
したれど
\ruby{思}{おも}ひ
\ruby{{\換字{返}}}{かへ}して
\原本頁{81-2}\改行%
\ruby{咎}{とが}めず、
%
\ruby[g]{大方}{おほかた}は
\ruby{病}{や}める
\ruby{人}{ひと}の
\ruby{上}{うへ}の
\ruby{氣}{き}に
かゝりて、% 踊り字調整「〻(二の字点、揺すり点)に見えるが(ゝ)」
%
\ruby{其}{そ}の
\ruby[g]{樣子}{やうす }
\ruby{見}{み}にと
\ruby{行}{ゆ}くならんを、
%
たゞ% 踊り字調整「〻(二の字点、揺すり点)に濁点に見えるが(ゞ)」
\ruby[||j>]{心}{こゝろ}% 踊り字調整「〻(二の字点、揺すり点)に見えるが(ゝ)」
\ruby[||j>]{儘}{ まゝ}ならしめんこそ% 踊り字調整「〻(二の字点、揺すり点)に見えるが(ゝ)」
\ruby[g]{慈悲}{なさけ }なるべけれと、
%
\ruby{睡}{ねむ}れるを
\ruby{裝}{よそほ}ひて
\ruby[g]{咳嗽}{しはぶき}も
せざりけれど、
%
はや
\ruby[g]{水野}{みづの }が
\ruby{四五間}{し|ご|けん}も
\ruby{{\換字{遠}}}{とほ}く
\ruby{去}{さ}りしと
\ruby{覺}{おぼ}しき
\ruby{頃}{ころ}、
%
\ruby{吉右衛門}{きち||ゑ|もん}は
\ruby{別}{べつ}に
\ruby{吉右衛門}{きち||ゑ|もん}の
\ruby{思}{おも}ふ
ところ
ありてや、
%
\ruby{吾}{わ}が
\ruby[<j>]{傍}{かたはら}の
\ruby{床}{とこ}に
\ruby{臥}{ふ}したる
お
\ruby{濱}{はま}の
\ruby[g]{寢顏}{ね がほ}の、
%
\ruby{小}{ちひ}さき
\ruby[g]{洋燈}{らんぷ }の
\ruby{光}{ひかり}に
\ruby{照}{て}らし
\ruby{出}{いだ}されたる、
%
\ruby{罪}{つみ}も
\ruby{無}{な}く
\ruby{美}{うつく}しきを
\ruby{見}{み}て、
%
\ruby{輕}{かろ}く
\ruby{歎}{たん}じたり。

\原本頁{81-8}%
\ruby[g]{水野}{みづの }は
\ruby{覺}{さ}めながら
\ruby[g]{夢路}{ゆめぢ }を
\ruby{辿}{たど}るが
\ruby{如}{ごと}く、
%
\ruby{天}{てん}に
\ruby[||j>]{明}{あかり}
\ruby[||j>]{無}{ な}く
\ruby{地}{ち}に
\ruby{色}{いろ}
\ruby{無}{な}き
\ruby{中}{なか}を、
%
\ruby[g]{何者}{なにもの}にか
\ruby[g]{肝膽}{き も }に
\ruby{糸}{いと}つけて
\ruby{牽}{ひ}かるゝ% 踊り字調整「〻(二の字点、揺すり点)に見えるが(ゝ)」
やうなる
\ruby{云}{い}ひがたき
\ruby{恐}{おそ}ろしさ
\ruby{苦}{くる}しさを
\ruby{覺}{おぼ}えつゝ、% 踊り字調整「〻(二の字点、揺すり点)に見えるが(ゝ)」
%
\ruby{例}{れい}の
お
\ruby{澤}{さは}が
\ruby{家}{いへ}の
\ruby{{\換字{前}}}{まへ}に
さしかゝりたり。% 踊り字調整「〻(二の字点、揺すり点)に見えるが(ゝ)」
%
\原本頁{81-11}\改行%
\ruby[||j>]{心}{こゝろ}に% 踊り字調整「〻(二の字点、揺すり点)に見えるが(ゝ)」
\ruby{眼}{め}
あれば
こそ
\ruby{物}{もの}は
\ruby{見}{み}ゆれど、
%
\ruby{眼}{め}に
\ruby{力}{ちから}は
\ruby{無}{な}くして
\ruby{知}{し}らぬ
\ruby{相}{すがた}は
\原本頁{82-1}\改行%
\ruby{見}{み}えぬ
\ruby{黑}{こく}
\ruby[g]{暗々}{あん〳〵}たる
\ruby{眞}{しん}の
\ruby{闇}{やみ}に、
%
\ruby[g]{水野}{みづの }は
\ruby{歩}{あゆみ}を
とゞめ% 踊り字調整「〻(二の字点、揺すり点)に濁点に見えるが(ゞ)」
\ruby{眼}{め}を
\ruby{凝}{こ}らして
\ruby[<j||]{覗}{うかゞ}へば、% 踊り字調整「〻(二の字点、揺すり点)に濁点に見えるが(ゞ)」
%
\ruby{豫}{かね}て
\ruby{知}{し}れる
\ruby{彼}{か}の
\ruby[g]{{\換字{寒}}竹}{かんちく}の
\ruby[||j>]{藪}{やぶ}
\ruby[||j>]{疊}{だゝみ}の% 踊り字調整「〻(二の字点、揺すり点)に見えるが(ゝ)」
% \ruby{藪疊}{やぶ|だゝみ}の% 踊り字調整「〻(二の字点、揺すり点)に見えるが(ゝ)」
\ruby{開}{ひら}けたる
\ruby{間}{あひだ}より、
%
\ruby{圃}{はたけ}の
\ruby{先}{さき}に
\原本頁{82-3}\改行%
\ruby{當}{あた}りて
\ruby{屋}{や}の
\ruby{棟}{むね}の
\ruby{低}{ひく}きが、
%
\ruby{曇}{くも}れる
\ruby{{\換字{空}}}{そら}に
\makeatletter
\@ifundefined{デバッグ@ビルド}{%
  \ruby[g]{微に}{かすか }
}{%
  \ruby{微}{かすか}に
}%
\makeatother
\ruby{{\換字{透}}}{す}きて
\ruby{立}{た}てり。

\原本頁{82-4}%
\ruby[g]{記臆}{おぼ{{\換字{𛀁}}}}% 原本通り「おぼ𛀁」
あればこそ
\ruby{辛}{から}くも
\ruby{歩}{ある}かるゝなれ、% 踊り字調整「〻(二の字点、揺すり点)に見えるが(ゝ)」
%
\ruby{地}{ち}の
\ruby{底}{そこ}の
\ruby{磐}{いはほ}の
\ruby{内}{うち}にも
\ruby{入}{い}らば
かくも
あらんかと
\ruby{思}{おも}はるゝ% 踊り字調整「〻(二の字点、揺すり点)に見えるが(ゝ)」
\ruby{闇}{やみ}の
\ruby{中}{なか}を、
%
\ruby{心}{こゝろ}% 踊り字調整「〻(二の字点、揺すり点)に見えるが(ゝ)」
あてばかりに
\ruby[g]{此方}{こ ゝ }と% 踊り字調整「〻(二の字点、揺すり点)に見えるが(ゝ)」
\ruby{{\換字{進}}}{すゝ}み% 踊り字調整「〻(二の字点、揺すり点)に見えるが(ゝ)」
\ruby{行}{ゆ}きて、
%
\ruby{漸}{やうや}く
\ruby[g]{草屋}{くさや }の
\ruby{横}{よこ}を
\ruby{{\換字{過}}}{よぎ}らん
とする
\ruby{時}{とき}、
%
\ruby[g]{萬籟}{ばんらい}
\ruby{死}{し}し
\ruby{盡}{つく}せる
\ruby{今}{いま}
\ruby[g]{突然}{とつぜん}
として、

\原本頁{82-8}%
『
ぎりぎりツ、
%
ぎりぎりツ。
』

\原本頁{82-9}%
といふ
\ruby{怪}{あや}しき% 原本通り「怪」
\ruby{響}{ひゞき}したり。% 踊り字調整「〻(二の字点、揺すり点)に濁点に見えるが(ゞ)」

\原本頁{82-10}%
\ruby{鳥}{とり}にあらず
\ruby{鼠}{ねずみ}にあらぬ
\ruby{其}{そ}の
\ruby{音}{おと}の、
%
\ruby{何}{なん}とも
\ruby{云}{い}へず
\ruby{物}{もの}
\ruby{忌}{いま}はしきに、
%
\ruby{思}{おも}はず
\ruby[g]{慄然}{ぞ つ }として
\ruby{耳}{みゝ}を% 踊り字調整「〻(二の字点、揺すり点)に見えるが(ゝ)」
\ruby{立}{た}つれば、
%
\ruby{聲}{こゑ}は
\ruby{我}{われ}
\ruby{{\換字{近}}}{ちか}き
\ruby{荒}{あ}れたる
\ruby{家}{いへ}の
\ruby{中}{うち}よ
\原本頁{83-1}\改行%
り
\ruby{來}{きた}りて、
%
そも〳〵
\ruby{何}{なん}の
\ruby{夢}{ゆめ}にか
\ruby{怒}{いか}れる、
%
\ruby{彼}{か}の
\ruby{鬼}{おに}の
\ruby{如}{ごと}き
お
\ruby{澤}{さは}
\ruby{婆}{ばゝ}の% 踊り字調整「〻(二の字点、揺すり点)に見えるが(ゝ)」
\改行% 校正作業の簡略化のため
、
%
\原本頁{83-2}\改行%
\ruby[g]{笑顏}{ゑ がほ}に
\ruby{見}{み}て
さへも
\ruby{凄}{すさま}じく
\ruby{今}{いま}
\ruby{{\換字{猶}}}{なほ}
\ruby{殘}{のこ}れる
\ruby{彼}{か}の
まばらなる
\ruby{長}{なが}き
\ruby{其}{その}
\ruby{齒}{は}を
\ruby{咬}{か}み
\ruby{鳴}{な}らせるにて、
%
\ruby[g]{其音}{そ れ }に
つゞいて% 踊り字調整「〻(二の字点、揺すり点)に濁点に見えるが(ゞ)」
\ruby{{\換字{又}}}{また}
\ruby{{\換字{更}}}{さら}に、

\原本頁{83-4}%
『
ウーン、
%
ウーン。
』

\原本頁{83-5}%
と
\ruby[g]{寐唸}{ねうなり}する
\ruby{其}{その}
\ruby{聲}{こゑ}は
\ruby{恨}{うら}むが
\ruby{如}{ごと}く
\ruby{詛}{のろ}ふが
\ruby{如}{ごと}く、
%
\ruby[g]{滿腔}{まんこう}の
\ruby[g]{怨毒}{{\換字{𛀁}}んどく}を
\ruby{噴}{ふ}き
\ruby{出}{いだ}して、
%
\ruby{闇}{くら}きに
\ruby[||j>]{{\換字{遊}}}{いう}
\ruby[||j>]{行}{ぎやう}する
% \ruby{{\換字{遊}}行}{いう|ぎやう}する
あらゆる
\ruby[g]{惡鬼}{あくき }を
\ruby{喚}{よ}び
\ruby{集}{つど}へん
とするにも
\ruby{似}{に}たれば、
%
\ruby[g]{水野}{みづの }は
\ruby{彌}{いや}が
\ruby{上}{うへ}にも
\ruby[g]{心地}{こゝち }% 踊り字調整「〻(二の字点、揺すり点)に見えるが(ゝ)」
\ruby{惡}{あし}く
おぼえて、
%
\ruby{止}{や}めよかし、
%
\ruby{止}{や}めよかし、
%
と
\ruby{急}{きふ}に
\ruby{念}{ねん}じたれど、

\原本頁{83-9}%
『
ぎりぎりツ、
%
ぎりぎりツ。
%
ウーン、
%
ウーン。
』

\原本頁{83-10}%
といふ
\ruby{聲}{こゑ}は
\ruby[g]{執念}{しふね }くも
\ruby{起}{おこ}つて、
%
\ruby{我}{わ}が
\ruby[||j>]{腦}{ぼんの}
\ruby[||j>]{後}{ くぼ}に
% \ruby{腦後}{ぼんの|くぼ}に
\ruby{襲}{おそ}ひかゝる% 踊り字調整「〻(二の字点、揺すり点)に見えるが(ゝ)」
がごとく
\ruby{逼}{せま}るに、
%
\ruby{身}{み}も
\ruby{世}{よ}も
あらず
\ruby{厭}{いと}はしく
\ruby{思}{おも}ひて、
%
\ruby{{\換字{追}}}{お}はれ
\ruby[g]{心地}{ごこち }に% ルビ調整(原本通り)非踊り字表記
\ruby{歩}{あゆ}み
\ruby{去}{さ}らんとする
\ruby{折}{をり}しも、
%
\ruby{忽}{たちま}ち
\ruby{我}{わ}が
\ruby{五十子}{い|そ|こ}の
\ruby{家}{いへ}の
\ruby{其}{そ}の
\ruby{方}{かた}より、
%
ひらりと
\ruby{物}{もの}の
\ruby{光}{ひか}りの
\ruby[g]{此方}{こなた }に% ルビ調整(原本通り)
\ruby{射}{さ}し
\ruby{來}{きた}りたり。
