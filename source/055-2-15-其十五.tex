\Entry{其十五}

% メモ 校正終了 2024-04-20
\原本頁{84-4}%
\ruby[||j>]{漆}{うるし}と
\ruby{黑}{くろ}き
\ruby[g]{眼{\換字{前}}}{め さき}の
\ruby{闇}{やみ}に、
%
ぱつと
\ruby{一}{ひ}ト
\ruby[g]{刷毛}{は け }の
\ruby[g]{光線}{ひかり }の
\ruby{散}{ち}つたるを、
%
いづくよりぞと
\ruby[g]{水野}{みづの }は
\ruby{見}{み}れば、
%
\ruby{人}{ひと}の
\ruby{歸}{かへ}るを
\ruby{{\換字{送}}}{おく}り
\ruby{出}{いだ}すと
\ruby{見}{み}えて、
%
\ruby{五十子}{い|そ|こ}が
\ruby{家}{いへ}の
\ruby{{\換字{戸}}}{と}の
\ruby{今}{いま}
\ruby{引}{ひき}
\ruby{開}{あ}けられたる
\ruby[g]{其處}{そ こ }より
\ruby[g]{洋燈}{らんぷ }の
\ruby{光}{ひかり}の
\ruby[g]{晃然}{きらり }と
\ruby{射}{さ}したるなり。

\原本頁{84-8}%
\ruby{問}{と}はでも
\ruby{知}{し}るべし、
%
\ruby[||j>]{病}{びやう}
\ruby[||j>]{者}{ しや}
% \ruby{病者}{びやう|しや}
ある
\ruby{家}{いへ}を、
%
\ruby[g]{如是}{かゝる }% 踊り字調整「〻(二の字点、揺すり点)に見えるが(ゝ)」
\ruby[g]{時刻}{じ こく}に
\ruby{人}{ひと}の
\ruby{出}{で}
\ruby{入}{い}りする
\ruby{事}{こと}、
%
\ruby[g]{必ず}{かなら }
\ruby{凶}{きよう}
ありて
\ruby{吉}{きち}
ある
\ruby{事}{こと}
\ruby{無}{な}し。
%
\ruby{我}{わ}が
\ruby{五十子}{い|そ|こ}は
\ruby{抑}{そも}
\ruby[g]{如何}{い か }にか
したる。
%
\ruby{何}{なに}と
\ruby{無}{な}く
\ruby{堪}{た}へ
\ruby{{\換字{難}}}{がた}き
\ruby[g]{心地}{こゝち }の% 踊り字調整「〻(二の字点、揺すり点)に見えるが(ゝ)」
\ruby{爲}{し}て、
%
\ruby{我}{わ}が
\ruby[g]{此處}{こ ゝ }まで% 踊り字調整「〻(二の字点、揺すり点)に見えるが(ゝ)」
\ruby{獨}{ひと}り
\ruby{{\換字{迷}}}{まよ}ひ
\ruby{出}{い}で
\ruby{來}{き}しも、
%
\ruby{世}{よ}にいふ
\ruby{蟲}{むし}の
\ruby{知}{し}らせし
といふ
\ruby{事}{こと}か、
%
たゞ% 踊り字調整「〻(二の字点、揺すり点)に濁点に見えるが(ゞ)」
ならす
\ruby[g]{動悸}{どうき }の
\原本頁{85-1}\改行%
\ruby{打}{う}ちしも
\ruby{思}{おも}ひ
\ruby{當}{あた}りたりと、
%
\ruby{先}{ま}づ
\ruby{胸}{むね}を
\ruby{轟}{とゞろ}かして% 踊り字調整「〻(二の字点、揺すり点)に濁点に見えるが(ゞ)」
\ruby[g]{彼方}{かなた }を
\ruby{見}{み}るに、
%
やがて
\ruby{{\換字{戸}}}{と}は
また
\ruby{引}{ひき}
\ruby{寄}{よ}せられて、
%
\ruby[g]{{\換字{遠}}目}{とほめ }の
\ruby{定}{さだ}かならねど
\ruby{四ツ目菱}{よ||め|びし}の
\ruby{紋}{もん}つきたる
\ruby[||j>]{提}{ちやう}
\ruby[||j>]{灯}{ ちん}を
% \ruby{提灯}{ちやう|ちん}を
\ruby[g]{片手}{かたて }に、
%
\ruby[g]{片手}{かたて }には
\ruby{小}{ちひさ}き
\ruby[g]{革鞄}{かばん }を
\ruby{持}{も}ちて、
%
ぽく〳〵と
\ruby[g]{此方}{こなた }に% ルビ調整(原本通り)
\ruby{歩}{あゆ}み
\ruby{來}{きた}れるは
\ruby[<j>]{疑}{うたがひ}もなく
\ruby[g]{尾竹}{を だけ}なり。

\原本頁{85-6}%
さては
いよ〳〵
\ruby{五十子}{い|そ|こ}に
\ruby{變}{へん}の
ありて、
%
\ruby[g]{夜{\換字{半}}}{よ は }の
\ruby{{\換字{扉}}}{と}を
たゝき% 踊り字調整「〻(二の字点、揺すり点)に見えるが(ゝ)」
\ruby{招}{よ}び
\ruby{{\換字{迎}}}{むか}へたればこそ、
%
\ruby[g]{尾竹}{を だけ}の
\ruby[g]{先刻}{さ き }に
\ruby{來}{きた}りて
\ruby{今}{いま}
\ruby{歸}{かへ}るなるべけれ。
%
\ruby{歸}{かへ}るは
\ruby{吉}{よ}くてか
\ruby{將}{はた}
\ruby{凶}{あし}くて
\ruby{歟}{か}。
%
\ruby[g]{嗚呼}{あ ゝ }、% 踊り字調整「〻(二の字点、揺すり点)に見えるが(ゝ)」
%
\ruby{五十子}{い|そ|こ}の
\ruby{病}{やまひ}は
\ruby{測}{はか}るべからずして、
%
\原本頁{85-9}\改行%
\ruby[g]{尾竹}{を だけ}の
\ruby[g]{{\換字{技}}倆}{わ ざ }は
\ruby{我}{われ}よく
\ruby{知}{し}れり。
%
\ruby[g]{嗚呼}{あ ゝ }、% 踊り字調整「〻(二の字点、揺すり点)に見えるが(ゝ)」
%
\ruby{人}{ひと}の
\ruby{命}{いのち}!、
%
\ruby{我}{わ}が
\ruby{命}{いのち}!、
%
\ruby{定}{さだ}まりたる
\ruby{天}{てん}の
\ruby{數}{すう}は
\ruby{今}{いま}
\ruby{見}{み}ゆるかや!。
%
\ruby{他}{ひと}をも
\ruby{死}{し}なせじ、
%
\ruby{我}{われ}も
\ruby{死}{し}なじと、
%
\ruby[g]{一念}{いちねん}の
\ruby{火}{ひ}を
\ruby{燃}{も}やしゝも% 踊り字調整「〻(二の字点、揺すり点)に見えるが(ゝ)」
\ruby{{\換字{空}}}{あだ}となつて、
%
\ruby{他}{ひと}も
\ruby{死}{し}に、
%
\ruby{我}{われ}も
\ruby{死}{し}に
\ruby{果}{は}てゝ、% 踊り字調整「〻(二の字点、揺すり点)に見えるが(ゝ)」
%
\ruby{冷}{つめた}き
\ruby{{\換字{灰}}}{はひ}と% ルビ調整(原本遠り)(はい)
なるべき
\ruby{時}{とき}の、
%
\ruby{{\換字{終}}}{つひ}に
\ruby{眼}{め}の
\ruby{{\換字{前}}}{まへ}には
\ruby{來}{きた}りたるかや。
%
\原本頁{86-2}\改行%
\ruby[g]{{\換字{前}}世}{ぜんせ }も
\ruby{知}{し}らず、
%
\ruby[g]{後世}{ご せ }も
\ruby{知}{し}らねど、
%
\ruby{此}{こ}の
\ruby{今}{いま}の
\ruby{世}{よ}は、
%
これまで
なりや、
%
\ruby[g]{嗚呼}{あ ゝ }% 踊り字調整「〻(二の字点、揺すり点)に見えるが(ゝ)」
\ruby{殘}{のこ}り
\ruby{多}{おほ}くも
\ruby[||j>]{恨}{うらみ}
\ruby[||j>]{多}{ おほ}くも、
% \ruby{恨多}{うらみ|おほ}くも、
%
これまでなりや、
%
これまでなりや。
%
と
\ruby{歩}{あゆ}まん
\ruby{意}{こゝろ}も% 踊り字調整「〻(二の字点、揺すり点)に見えるが(ゝ)」
\ruby{無}{な}く
\ruby{言}{ものい}はん
\ruby{意}{こゝろ}も% 踊り字調整「〻(二の字点、揺すり点)に見えるが(ゝ)」
\ruby{無}{な}くなりて、
%
\ruby[g]{水野}{みづの }は
\ruby{地}{つち}の
\ruby{上}{うへ}に
たゞ% 踊り字調整「〻(二の字点、揺すり点)に濁点に見えるが(ゞ)」
\ruby[g]{苟且}{かりそめ}に
\ruby{立}{た}て
\ruby{置}{お}かれたる
\ruby{一}{ひと}つ
\ruby{杭}{ぐひ}の
\ruby{如}{ごと}く、
%
\ruby[g]{少時}{しばらく}
\ruby[g]{茫然}{ばうぜん}として
\ruby{立}{た}ち
\ruby{居}{ゐ}けるが、
%
やがて
ばたりと
\ruby{倒}{たふ}れんと
したり。

\原本頁{86-7}%
されど
\ruby[g]{水野}{みづの }の
\ruby{自}{みづか}ら
\ruby{支}{さゝ}へて、% 踊り字調整「〻(二の字点、揺すり点)に見えるが(ゝ)」
%
\ruby{辛}{から}くも
\ruby{思}{おもひ}を
\ruby{轉}{てん}じたる
\ruby{時}{とき}、
%
\ruby[g]{尾竹}{を だけ}は
\ruby[<j||]{間}{あはひ}% 行末行頭の境界付近なので特例処置を施す
\ruby[||j>]{{\換字{近}}}{ちか}く
% \ruby{間{\換字{近}}}{あはひ|ちか}く
\ruby{{\換字{進}}}{すゝ}み% 踊り字調整「〻(二の字点、揺すり点)に見えるが(ゝ)」
\ruby{來}{きた}りしが、
%
\ruby{思}{おも}ひも
かけぬ
\ruby{闇}{やみ}の
\ruby[g]{眞中}{ま なか}に
\ruby{人}{ひと}の
\ruby{佇}{たゝず}めるを% 踊り字調整「〻(二の字点、揺すり点)に見えるが(ゝ)」
\ruby{認}{みと}めつ
\ruby[g]{愕然}{ぎよつ }として
\ruby[g]{驚き}{おどろ }、
%
\ruby[||j>]{提}{ちやう}
\ruby[||j>]{灯}{ ちん}の
% \ruby{提灯}{ちやう|ちん}の
\ruby{燈}{ひ}に
\ruby[g]{此方}{こなた }を% ルビ調整(原本通り)
すかし
\ruby{見}{み}、

\原本頁{86-10}%
『
み、
%
み、
%
\ruby[g]{水野}{みづの }さん
ですか。
』

\原本頁{86-11}%
と
\ruby{顫}{ふる}へ
\ruby{聲}{ごゑ}に
\ruby{{\換字{尋}}}{たづ}ねたり。

\原本頁{87-1}%
\ruby[g]{凡人}{ぼんじん}の
\ruby{眼}{め}つき、
%
\ruby[g]{凡人}{ぼんじん}の
\ruby{口}{くち}つき、
%
\ruby[g]{凡人}{ぼんじん}の
\ruby{額}{ひたひ}、
%
\ruby[g]{凡人}{ぼんじん}の
\ruby{肩}{かた}、
%
\ruby[g]{身長}{みのたけ}も
\ruby[g]{普{\換字{通}}}{つねなみ}なれば、
%
\ruby[g]{態度}{やうす }も
\ruby[g]{普{\換字{通}}}{つねなみ}にて、
%
\ruby[g]{何處}{ど こ }に
\ruby{一}{ひと}つ
これといふ
ところも
\ruby{無}{な}き
\ruby{其}{そ}の
\ruby[g]{尾竹}{を だけ}の
\ruby{深}{ふか}くも
\ruby[||j>]{忘}{おそろ}
\ruby[||j>]{怖}{ しき}に% 「恐怖」の誤植のように思えるが原本通りにしておく
% \ruby{忘怖}{おそろ|しき}に% 「恐怖」の誤植のように思えるが原本通りにしておく
\ruby{魘}{おそ}はれたるにや、
%
\ruby{眉}{まゆ}を
\ruby{尾}{しり}
\ruby{下}{さが}りにし、
%
\原本頁{87-4}\改行%
\ruby{眼}{め}を
\ruby{壺}{つぼ}
\ruby{深}{ふか}くして、
%
\ruby{頸}{くび}を
\ruby{縮}{ちゞ}めつゝ% 踊り字調整「〻(二の字点、揺すり点)に見えるが(ゝ)」% 踊り字調整「〻(二の字点、揺すり点)に濁点に見えるが(ゞ)」
\ruby[g]{此方}{こなた }を% ルビ調整(原本通り)
\ruby{見}{み}たる
\ruby{其}{そ}の
\ruby{怯}{おそ}れたる
\ruby{狀}{さま}の
いと
\ruby{醜}{みにく}きが、
%
\ruby[||j>]{提}{ちやう}
\ruby[||j>]{灯}{ ちん}の
% \ruby{提灯}{ちやう|ちん}の
\ruby[g]{火影}{ほ かげ}に
ぼつと
\ruby{見}{み}えたるは、
%
\ruby{今}{いま}といふ
\ruby{今}{いま}のみ
\ruby{始}{はじ}めて
\ruby[g]{{\換字{平}}凡}{よのつね}ならず
\ruby[g]{水野}{みづの }が
\ruby{眼}{め}に
\ruby{映}{うつ}りぬ。
