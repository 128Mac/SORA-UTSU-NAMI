\Entry{其十五}

\原本頁{}%
\ruby{漆}{うるし}と
\ruby{黑}{くろ}き
\ruby{眼{\換字{前}}}{め|さき}の
\ruby{闇}{やみ}に、
%
ぱつと
\ruby{一}{ひ}ト
\ruby{刷毛}{は|け}の
\ruby{光線}{こう|せん}の
\ruby{散}{ち}つたるを、
%
いづくよりぞと
\ruby[g]{水野}{みづの}は
\ruby{見}{み}れば、
%
\ruby{人}{ひと}の
\ruby{歸}{かへ}るを
\ruby{{\換字{送}}}{おく}り
\ruby{出}{いだ}すと
\ruby{見}{み}えて、
%
\ruby[g]{五十子}{いそこ}が
\ruby{家}{いへ}の
\ruby{{\換字{戸}}}{と}の
\ruby{今}{いま}
\ruby{引開}{ひき|あ}けられたる
\ruby{其處}{そ|こ}より
\ruby{洋燈}{らん|ぷ}の
\ruby{光}{ひかり}の
\ruby{晃然}{きら|り}と
\ruby{射}{さ}したるなり。

\原本頁{}%
\ruby{問}{と}はでも
\ruby{知}{し}るべし、
%
\ruby{病者}{びやう|しや}ある
\ruby{家}{いへ}を、
%
\ruby{如是時刻}{かゝ|る|じ|こく}に
\ruby{人}{ひと}の
\ruby{出入}{で|い}りする
\ruby{事}{こと}、
%
\ruby{必}{かなら}ず
\ruby{凶}{きよう}ありて
\ruby{吉}{きち}ある
\ruby{事}{こと}
\ruby{無}{な}し。
%
\ruby{我}{わ}が
\ruby[g]{五十子}{いそこ}は
\ruby{抑如何}{そも|い|か}にかしたる。
%
\ruby{何}{なに}と
\ruby{無}{な}く
\ruby{堪}{た}へ
\ruby{難}{がた}き
\ruby{心地}{こゝ|ち}の
\ruby{爲}{し}て、
%
\ruby{我}{わ}が
\ruby{此處}{こ|ゝ}まで
\ruby{獨}{ひと}り
\ruby{{\換字{迷}}}{まよ}ひ
\ruby{出}{い}で
\ruby{來}{き}しも、
%
\ruby{世}{よ}にいふ
\ruby{蟲}{むし}の
\ruby{知}{し}らせしといふ
\ruby{事}{こと}か、
%
たゞならず
\ruby{動悸}{どう|き}の
\ruby{打}{う}ちしも
\ruby{思}{おも}ひ
\ruby{當}{あ}たりたりと、
%
\ruby{先}{ま}づ
\ruby{胸}{むね}を
\ruby{轟}{とゞろ}かして
\ruby{彼方}{かな|た}を
\ruby{見}{み}るに、
%
やがて
\ruby{{\換字{戸}}}{と}はまた
\ruby{引寄}{ひき|よ}せられて、
%
\ruby{{\換字{遠}}目}{とほ|め}の
\ruby{定}{さだ}かならねど
\ruby{四}{よ}ツ
\ruby{目菱}{め|びし}の
\ruby{紋}{もん}つきたる
\ruby{提灯}{ちやう|ちん}を
\ruby{片手}{かた|て}に、
%
\ruby{片手}{かた|て}には
\ruby{小}{ちひさ}き
\ruby{革鞄}{かば|ん}を
\ruby{持}{も}ちて、
%
ぽく〳〵と
\ruby{此方}{こ|なた}に
\ruby{歩}{あゆ}み
\ruby{來}{きた}れるは
\ruby{疑}{うたがひ}もなく
\ruby[g]{尾竹}{をだけ}なり。

\原本頁{}%
さてはいよ〳〵
\ruby[g]{五十子}{いそこ}に
\ruby{變}{へん}のありて、
%
\ruby{夜{\換字{半}}}{よ|は}の
\ruby{{\換字{扉}}}{と}をたゝき
\ruby{招}{よ}び
\ruby{{\換字{迎}}}{むか}へたればこそ、
%
\ruby[g]{尾竹}{をだけ}の
\ruby{先刻}{さ|き}に
\ruby{來}{きた}りて
\ruby{今}{いま}
\ruby{歸}{かへ}るなるべけれ。
%
\ruby{歸}{かへ}るは
\ruby{吉}{よ}くてか
\ruby{將}{はた}
\ruby{凶}{あし}くて
\ruby{歟}{か}。
%
\ruby{嗚呼}{あ|ゝ}、
%
\ruby[g]{五十子}{いそこ}の
\ruby{病}{やまひ}は
\ruby{測}{はか}るべからずして、
%
\ruby[g]{尾竹}{をだけ}の
\ruby{{\換字{技}}倆}{わ|ざ}は
\ruby{我}{われ}よく
\ruby{知}{し}れり。
%
\ruby{嗚呼}{あ|ゝ}、
%
\ruby{人}{ひと}の
\ruby{命}{いのち}!、
%
\ruby{定}{さだ}まりたる
\ruby{天}{てん}の
\ruby{數}{すう}は
\ruby{今}{いま}
\ruby{見}{み}ゆるかや!。
%
\ruby{他}{ひと}をも
\ruby{死}{し}なせし、
%
\ruby{我}{われ}も
\ruby{死}{し}なじと、
%
\ruby{一念}{いち|ねん}の
\ruby{火}{ひ}を
\ruby{燃}{も}やし〻も
\ruby{{\換字{空}}}{あだ}となつて、
%
\ruby{他}{ひと}も
\ruby{死}{し}に、
%
\ruby{我}{われ}も
\ruby{死}{し}に
\ruby{果}{は}て〻、
%
\ruby{冷}{つめ}たき
\ruby{{\換字{灰}}}{はひ}となるべき% ルビは「はい」ではなく原本通り
\ruby{時}{とき}の、
%
\ruby{{\換字{終}}}{つひ}に
\ruby{眼}{め}の
\ruby{{\換字{前}}}{まへ}には
\ruby{來}{きた}りたるかや。
%
\ruby{{\換字{前}}世}{ぜん|せ}も
\ruby{知}{し}らず、
%
\ruby{後世}{ご|せ}も
\ruby{知}{し}らねど、
%
\ruby{此}{こ}の
\ruby{今}{いま}の
\ruby{世}{よ}は、
%
これまでなりや、
%
\ruby{嗚呼}{あ|ゝ}
\ruby{殘}{のこ}り
\ruby{多}{おほ}くも
\ruby{恨多}{うらみ|おほ}くも、これまでなりや、
%
これまでなりや。
%
と
\ruby{歩}{あゆ}まん
\ruby{意}{こゝろ}も
\ruby{無}{な}く
\ruby{言}{ものい}はん
\ruby{意}{こゝろ}も
\ruby{無}{な}くなりて、
%
\ruby[g]{水野}{みづの}は
\ruby{地}{つち}の
\ruby{上}{うへ}にたゞ
\ruby{苟且}{かり|そめ}に
\ruby{立}{た}て
\ruby{置}{お}かれたる
\ruby{一}{ひと}つ
\ruby{杭}{ぐひ}の
\ruby{如}{ごと}く、
%
\ruby{少時}{しば|らく}
\ruby{茫然}{ばう|ぜん}として
\ruby{立}{た}ち
\ruby{居}{ゐ}けるが、
%
やがてばたりと
\ruby{倒}{たふ}れんとしたり。

\原本頁{}%
されど
\ruby[g]{水野}{みづの}の
\ruby{自}{みづか}ら
\ruby{支}{さゝ}へて、
%
\ruby{辛}{から}くも
\ruby{思}{おもひ}を
\ruby{轉}{てん}じたる
\ruby{時}{とき}、
%
\ruby[g]{尾竹}{をだけ}は
\ruby{間{\換字{近}}}{あはひ|ちか}く
\ruby{{\換字{進}}}{すゝ}み
\ruby{來}{きた}りしが、
%
\ruby{思}{おも}ひもかけぬ
\ruby{闇}{やみ}の
\ruby{眞中}{ま|なか}に
\ruby{人}{ひと}の
\ruby{佇}{たゝず}めるを
\ruby{認}{みと}めつ
\ruby{愕然}{ぎよ|つ}として
\ruby{驚}{おどろ}き、
%
\ruby{提灯}{ちやう|ちん}の
\ruby{燈}{ひ}に
\ruby{此方}{こ|なた}をすかし
\ruby{見}{み}、

\原本頁{}%
『み、
%
み、
%
\ruby[g]{水野}{みづの}さんですか。
』

\原本頁{}%
と
\ruby{顫}{ふる}へ
\ruby{聲}{ごゑ}に
\ruby{{\換字{尋}}}{たづ}ねたり。

\原本頁{}%
\ruby{凡人}{ぼん|じん}の
\ruby{眼}{め}つき、
%
\ruby{凡人}{ぼん|じん}の
\ruby{口}{くち}つき、
%
\ruby{凡人}{ぼん|じん}の
\ruby{額}{ひたひ}、
%
\ruby{凡人}{ぼん|じん}の
\ruby{肩}{かた}、
%
\ruby{身長}{みの|たけ}も
\ruby{普{\換字{通}}}{つね|なみ}なれば、
%
\ruby{態度}{やう|す}も
\ruby{普{\換字{通}}}{つね|なみ}にて、
%
\ruby{何處}{ど|こ}に
\ruby{一}{ひと}つこれといふところも
\ruby{無}{な}き
\ruby{其}{そ}の
\ruby[g]{尾竹}{をだけ}の
\ruby{深}{ふか}くも
\ruby{忘怖}{おそ|ろしき}に% 「恐怖」の誤植のように思えるが原本通りにしておく
\ruby{魘}{おそ}はれたるにや、
%
\ruby{眉}{まゆ}を
\ruby{尾下}{しり|さが}りにし、
%
\ruby{眼}{め}を
\ruby{壺深}{つぼ|ふか}くして、
%
\ruby{頸}{くび}を
\ruby{縮}{ちゞ}めつゝ
\ruby{此方}{こ|なた}を
\ruby{見}{み}たる
\ruby{其}{そ}の
\ruby{怯}{おそ}れたる
\ruby{狀}{さま}のいと
\ruby{醜}{みにく}きが、
%
\ruby{提灯}{ちやう|ちん}の
\ruby{火影}{ほ|かげ}にぼつと
\ruby{見}{み}えたるは、
%
\ruby{今}{いま}といふ
\ruby{今}{いま}のみ
\ruby{始}{はじ}めて
\ruby{{\換字{平}}凡}{よの|つね}ならず
\ruby[g]{水野}{みづの}が
\ruby{眼}{め}に
\ruby{映}{うつ}りぬ。
