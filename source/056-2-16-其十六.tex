\Entry{其十六}

% メモ 校正終了 2024-04-20 2024-05-31
\原本頁{87-8}%
\ruby{尾竹}{を|だけ}の
\ruby{聲}{こゑ}は
\ruby{闇}{やみ}の
\ruby{寂寥}{しづ|かさ}に
\ruby{響}{ひゞ}きて、% 踊り字調整「〻(二の字点、揺すり点)に濁点に見えるが(ゞ)」
%
\ruby{愚}{おろか}しくいと
\ruby{大}{おほ}きく
\ruby{聞}{きこ}えたるに、
%
\ruby{水野}{みづ|の}は
\ruby{何}{なん}となく
\ruby{厭}{いと}はしく
\ruby{{\換字{感}}}{かん}じつ、
%
こゝにて% 踊り字調整「〻(二の字点、揺すり点)に見えるが(ゝ)」
\ruby{{\換字{又}}}{また}
\ruby{我}{われ}と
\ruby{此男}{こ|れ}との
\ruby{問}{と}ひつ
\ruby{答}{こた}へつせば、
%
\ruby{其}{そ}の
\ruby{聲}{こゑ}の
\ruby{彼家}{かし|こ}の
\ruby{人々}{ひと|〴〵}にも
\ruby{聞}{きこ}えんことを
\ruby{忌}{いま}はしく
\原本頁{88-1}\改行%
\ruby{思}{おも}ひて、
%
\ruby{言葉}{こと|ば}は
\ruby{無}{な}き
\ruby{擧動}{そぶ|り}
ばかりに
\ruby{尾竹}{を|だけ}を
\ruby{誘}{いざな}ひ、
%
\ruby{突}{つ}と
\ruby{外}{そと}の
\ruby{方}{かた}に
\ruby{去}{さ}らん
とすれば、
%
\ruby{尾竹}{を|だけ}は
\ruby{慌}{あわ}てゝ% 踊り字調整「〻(二の字点、揺すり点)に見えるが(ゝ)」
\ruby{先}{さき}に
\ruby{立}{た}つて、
%
\ruby{手}{て}に
\ruby{持}{も}てる
\ruby[||j>]{提}{ちやう}
\ruby[||j>]{灯}{ ちん}に
% \ruby{提灯}{ちやう|ちん}に
\ruby{足元}{あし|もと}を
\ruby{照}{て}らしたり。

\原本頁{88-4}%
\ruby{共}{とも}に
\ruby{歩}{あゆ}むこと
\ruby{四五歩}{し|ご|ほ}
ならずして、
%
\ruby{彼}{か}の
お
\ruby{澤}{さは}
\ruby{婆}{ばゝ}が% 踊り字調整「〻(二の字点、揺すり点)に見えるが(ゝ)」
\ruby{家}{いへ}の
\ruby{内}{うち}より、

\原本頁{88-5}%
『
ギリギリツ。
』

\原本頁{88-6}%
といふ
\ruby{聲}{こゑ}
\ruby{先}{ま}づ
\ruby{聞}{きこ}えて、
%
\ruby{次}{つ}いで、

\原本頁{88-7}%
『
ウーンウーン。
』

\原本頁{88-8}%
といふ
\ruby{寢唸}{ね|うな}りの
\ruby{聞}{きこ}ゆれば、
%
\ruby{尾竹}{を|だけ}は
\ruby{思}{おも}はずも
ビクリと
\ruby{顫}{ふる}へて、
%
\ruby{手}{て}に
したる
\ruby[||j>]{提}{ちやう}
\ruby[||j>]{灯}{ ちん}に
% \ruby{提灯}{ちやう|ちん}に
\ruby{烈}{はげ}しき
\ruby{浪}{なみ}を
\ruby{打}{う}たせつ、

\原本頁{88-10}%
『
ナ、
%
\ruby{何}{なん}でしやう
\ruby{彼}{あ}の
\ruby{音}{おと}は?。
』

\原本頁{88-11}%
と
\ruby{振}{ふ}り
\ruby{{\換字{返}}}{かへ}つて
\ruby{水野}{みづ|の}に
\ruby{{\換字{尋}}}{たづ}ねたり。

\原本頁{89-1}%
されど
\ruby{水野}{みづ|の}は
\ruby{尾竹}{を|だけ}が
\ruby{此}{こ}の
\ruby{言葉}{こと|ば}を、
%
\ruby{閑事}{むだ|ごと}なりと
\ruby{云}{い}はぬ
ばかりに、
%
たゞ% 踊り字調整「〻(二の字点、揺すり点)に濁点に見えるが(ゞ)」
\ruby{無言}{む|ごん}を
もて
あしらひ
\ruby{棄}{す}て、
%
おのが
\ruby{歩}{あゆ}まん
とする
\ruby{方}{かた}に
\ruby{歩}{あゆ}み
\ruby{去}{さ}りながら、

\原本頁{89-4}%
『
\ruby{岩崎}{いは|さき}は% 原本のこの部分は「いわさき」
\ruby{何樣}{ど|う}で
ございます、
%
よろしいのですか?。
』

\原本頁{89-5}%
と
\ruby{先刻}{さ|き}より
\ruby{此}{こ}の
\ruby{醫}{い}の
\ruby{樣子}{やう|す}に
\ruby{大事}{だい|じ}
\ruby{無}{な}しとは
\ruby{察}{さつ}したれど、
%
\ruby{問}{と}はんとして
\ruby{一刻}{いつ|こく}も
\ruby{忘}{わす}れざりし
\ruby{問}{とひ}を
\ruby{發}{はつ}すれば、
%
\ruby{眞{\換字{情}}}{まこ|と}
\ruby{餘}{あま}りし
\ruby{其}{そ}の
\ruby{言葉}{こと|ば}の
\ruby{自然}{おの|づ}と
\ruby{威}{ゐ}
あるやうなるに
\ruby{尾竹}{を|だけ}は
\ruby{壓}{お}されて、
%
\ruby{今}{いま}
\ruby{我}{わ}が
\ruby{口}{くち}に
\ruby{出}{いだ}したる
\ruby{問}{とひ}
\原本頁{89-8}\改行%
の
\ruby{答}{こたへ}を
\ruby{得}{{\換字{𛀁}}}ざるをも、
%
また
\ruby{水野}{みづ|の}が
\ruby{如何}{い|か}なれば
\ruby{如是}{かゝ|る}% 踊り字調整「〻(二の字点、揺すり点)に見えるが(ゝ)」
\ruby{時{\換字{分}}}{じ|ぶん}に
\ruby{此}{こゝ}% 踊り字調整「〻(二の字点、揺すり点)に見えるが(ゝ)」
\ruby{邊}{ら}には
\原本頁{89-9}\改行%
\ruby{佇}{たゝず}み% 踊り字調整「〻(二の字点、揺すり点)に見えるが(ゝ)」
\ruby{居}{ゐ}たるやと
\ruby{{\換字{尋}}}{たづ}ね
まほしく
\ruby{思}{おも}ひ
\ruby{居}{ゐ}たるをも
\ruby[<j>]{盡}{こと〴〵}く
\ruby{皆}{みな}
\ruby{忘}{わす}れ
\ruby{果}{は}てゝ% 踊り字調整「〻(二の字点、揺すり点)に見えるが(ゝ)」
\改行% 校正作業の簡略化のため
、

\原本頁{89-10}%
『
いや
\ruby{御{\換字{尋}}問}{お|たづ|ね}が
\ruby{無}{な}くても
\ruby{夜}{よ}でも
\ruby{明}{あ}けましたら
\ruby{一寸}{ちよ|つと}
\ruby{上}{あが}つてなりと
\ruby{申上}{まをし|あ}げやうと
\ruby{思}{おも}つて
\ruby{居}{を}りましたが、
%
\ruby{看護{\換字{婦}}}{かん|ご|ふ}の
\ruby{注意}{ちう|い}% 原本通り「ゆ」無しで「ちうい」
からして
\ruby{御使}{お|つかひ}が
あつたので、
%
\ruby{今}{いま}しがた
\ruby{出}{で}て
\ruby{見}{み}ると、
%
\ruby{實}{じつ}は
\ruby{甚}{はなは}だ
\ruby{面白}{おも|しろ}く
\ruby{無}{な}くなつて
\ruby{居}{ゐ}るのです。
%
\ruby{勿論}{もち|ろん}
\ruby{今}{いま}が
\ruby{今}{いま}
といふ
やうな
ことは
ありませんが、
%
\原本頁{90-3}\改行%
\ruby{全體}{ぜん|たい}が
\ruby{{\換字{丈}}夫}{ぢやう|ぶ}づくり
といふ
\ruby{方}{かた}では
\ruby{無}{な}いのです
のに、
%
たゞ% 踊り字調整「〻(二の字点、揺すり点)に濁点に見えるが(ゞ)」
\ruby{氣性}{き|しやう}が
\ruby{確乎}{しつ|かり}として
\ruby{居}{ゐ}らるゝ% 踊り字調整「〻(二の字点、揺すり点)に見えるが(ゝ)」
ばかりで、
%
\ruby{今}{いま}までは
\ruby{病苦}{びやう|く}に
\ruby{負}{ま}けずに
\ruby{居}{ゐ}られた
ところ、
%
\ruby{何樣}{ど|う}して、
%
\ruby{精神}{せい|しん}
\ruby{作用}{さ|よう}だつて
\ruby{限}{かぎり}の
あるものですもの、
%
\ruby{{\換字{連}}日}{れん|じつ}の
\ruby{高度}{かう|ど}の
\ruby{熱}{ねつ}では
\ruby{耐}{たま}りません、
%
とう〳〵
\ruby{堪}{こた}へに
\ruby{堪}{こた}へ
きれなく
なられましたのです。
%
さあ
\ruby{左樣}{さ|う}なると
\ruby{其}{それ}と
\ruby{同時}{どう|じ}に、
%
\ruby{自然}{し|ぜん}と
\ruby{來}{き}て
\ruby{居}{ゐ}た
\ruby[||j>]{衰}{すゐ}
\ruby[||j>]{{\換字{弱}}}{じやく}が、
% \ruby{衰{\換字{弱}}}{すゐ|じやく}が、
%
\ruby{俄然}{が|ぜん}と
\ruby{外}{そと}に
\ruby{現}{あらは}れて
まゐりましたので、
%
\ruby{一體}{いつ|たい}に
\ruby{何處}{ど|こ}も
\原本頁{90-9}\改行%
\ruby{彼處}{かし|こ}も
\ruby{惡}{わる}くなつて
\ruby{來}{き}た
といふやうな
\ruby{譯}{わけ}です。
%
しかし
\ruby[<j>]{幸}{さいはひ}に
\ruby{特}{こと}に
\ruby{肺}{はい}が
\ruby{惡}{わる}くなつたとか
\ruby{心臓}{しん|ざう}が
\ruby{惡}{わる}くなつたとか
\ruby{云}{い}ふ
のでは
ありませんから、
%
まだ〳〵
\ruby{十{\換字{分}}}{じふ|ぶん}
\ruby{有望}{いう|ばう}なので、
%
\ruby{云}{い}はゞ% 踊り字調整「〻(二の字点、揺すり点)に濁点に見えるが(ゞ)」
\ruby{彼樣}{あ|ゝ}% 踊り字調整「〻(二の字点、揺すり点)に見えるが(ゝ)」
いふ
\ruby[||j>]{大}{たい}
\ruby[||j>]{病}{びやう}に
% \ruby{大病}{たい|びやう}に
かゝつた% 踊り字調整「〻(二の字点、揺すり点)に見えるが(ゝ)」
\ruby[||j>]{患}{くわん}
\ruby[||j>]{者}{ じや}の、
% \ruby{患者}{くわん|じや}の、
%
\ruby{何樣}{ど|う}も
\ruby{經{\換字{過}}}{けい|くわ}し
なければ
ならぬ
\ruby{已}{や}むを
\ruby{得}{{\換字{𛀁}}}ざる
\ruby{塲合}{ば|あひ}% 原文通り「塲」
なのです。
』

\原本頁{91-3}%
と
\ruby{一{\換字{半}}}{いつ|ぱん}は
\ruby{水野}{みづ|の}を
\ruby{慰}{なぐさ}め、
%
\ruby{一{\換字{半}}}{いつ|ぱん}は
おのれを
\ruby{辯護}{べん|ご}するが% 弁 瓣 辦 辧 辨 辩 (辯)
\ruby{如}{ごと}く、
%
\ruby{素人}{しろう|と}
\ruby{解}{わか}り
すべきことを
\ruby{條理}{でう|り}
\ruby[||j>]{賢}{かしこ}く
\ruby{{\換字{述}}}{の}べたり。

\原本頁{91-5}%
\ruby{水野}{みづ|の}は
\ruby{五十子}{い|そ|こ}の
\ruby{容態}{よう|だい}
あしゝと% 踊り字調整「〻(二の字点、揺すり点)に見えるが(ゝ)」
\ruby{聞}{き}きて、
%
さて
こそと
\ruby{胸}{むね}を
\ruby{躍}{をど}らせつ
\改行% 校正作業の簡略化のため
、
%
\原本頁{91-6}\改行%
\ruby{先}{ま}ず
\ruby{悲}{かな}しくも
\ruby{腹立}{はら|だ}たしき
おもひして、
%
はや
\ruby{苛々}{いら|〳〵}と
\ruby{心}{こゝろ}は% 踊り字調整「〻(二の字点、揺すり点)に見えるが(ゝ)」
\ruby{烈}{はげ}しくなり、
%
\ruby{此}{こ}の
\ruby{醫者}{い|しや}の
\ruby{{\換字{技}}}{わざ}
\ruby{鈍}{にぶ}きを
\ruby{怒}{いか}る
とにはあらねど、
%
\ruby{其}{そ}の
\ruby{言葉}{こと|ば}
\ruby[||j>]{巧}{たくみ}なるが
\ruby{小憎}{こ|にく}くて、

\原本頁{91-9}%
『
\ruby{已}{や}むを
\ruby{得}{{\換字{𛀁}}}ざる
\ruby{塲合}{ば|あひ}で!。% 原文通り「塲」
%
\ruby{成程}{なる|ほど}
\ruby{御{\換字{道}}理}{ご|もつ|とも}です、
%
\ruby{已}{や}むを
\ruby{得}{{\換字{𛀁}}}ざる
\ruby{塲合}{ば|あひ}で!。% 原文通り「塲」
%
まかり
\ruby{間{\換字{違}}}{ま|ちが}つて
\ruby{何樣}{ど|う}なりましても、
%
\ruby{勿論}{もち|ろん}
みんな
\ruby{已}{や}むを
\ruby{得}{{\換字{𛀁}}}ざる
\ruby{塲合}{ば|あひ}ですナ。% 原文通り「塲」
』

\原本頁{92-1}%
と、
%
\ruby{一}{ひ}ト
\ruby{當}{あて}
\ruby{當}{あ}つれ
\footnote{「\ruby{當}{あ}つ」の活用形の一つとして原本通り「\ruby{當}{あ}つれ」とした
原本通り(国会図書館 コマ番号51/160 p-092 l-01)}%
ば
\ruby{尾竹}{を|だけ}は
\ruby{驚}{おどろ}き、
%
\ruby{{\換字{平}}日}{ひご|ろ}は
\ruby{物}{もの}
\ruby{柔}{やはら}かにして
\ruby{斯樣}{か|う}は
\ruby{無}{な}かりし
\ruby{人}{ひと}の、
%
\ruby{何}{なん}たる
\ruby{氣}{き}の
\ruby{焦}{い}れかたぞやと
\ruby{呆}{あき}れながら、

\原本頁{92-3}%
『
\ruby{左樣}{さ|う}
\ruby{御取}{お|と}りになつては
\ruby{困}{こま}ります。
%
わたくしが
\ruby{責任}{せき|にん}を
\ruby{{\換字{逃}}}{のが}れ
やうとして
\ruby{申}{まを}した
のでは
ござりません。
%
わたくしが
\ruby{其樣}{そ|ん}なもので
\ruby{無}{な}
\原本頁{92-5}\改行%
いことは
\ruby{御}{ご}
\ruby{承知}{しよう|ち}で
ござりましやう。
%
\ruby{小生}{わた|くし}は
\ruby{小生}{わた|くし}の
\ruby{及}{およ}ぶ
\ruby{限}{かぎ}りの
\ruby[<j||]{力}{ちから}を
\ruby{盡}{つく}して
\ruby{居}{を}りますのです。
』

\原本頁{92-7}%
と
\ruby{疾辯}{はや|くち}に% 弁 瓣 辦 辧 辨 辩 (辯)
\ruby{言}{い}ひたる
\ruby{其}{その}
\ruby{聲}{こゑ}は
\ruby{眞}{まこと}に
\ruby{切}{せつ}なげに
\ruby{泣}{な}きさうにも
\ruby{聞}{きこ}えて、
%
\ruby{{\換字{技}}}{わざ}こそ
\ruby{庸常}{よの|つね}にして
\ruby{人}{ひと}に
\ruby{挺}{ぬきん}でも
せざれ、
%
\ruby{心}{こゝろ}は% 踊り字調整「〻(二の字点、揺すり点)に見えるが(ゝ)」
\ruby[||j>]{正}{しやう}
\ruby[||j>]{直}{ ぢき}にして
% \ruby{正直}{しやう|ぢき}にして
\ruby{自}{みづか}ら
\ruby{欺}{あざむ}かざる
\ruby{君子}{くん|し}なるを
\ruby{示}{しめ}せり。

\原本頁{92-10}%
\ruby{水野}{みづ|の}は
\ruby{流石}{さす|が}に
これに
\ruby{氣}{き}の
\ruby{毒}{どく}になりて、

\原本頁{92-11}%
『
ヤ、
%
\ruby{先生}{せん|せい}、
%
\ruby{御氣}{お|き}に
\ruby{御{\換字{留}}}{お|と}め
なすつては
いけません。
%
\ruby{先生}{せん|せい}の
\ruby{御誠實}{ご|せい|じつ}な
\ruby{事}{こと}は
よく
\ruby{存}{ぞん}じて
\ruby{居}{を}ります。
%
\ruby{{\換字{猶}}}{なほ}%%%%%%%% ルビ調整(配置位置補正)
\ruby{此上}{この|うへ}とも%%%%%% ルビ調整(配置位置補正)
\ruby{何{\換字{分}}}{なに|ぶん}% ルビ調整(配置位置補正)
\ruby{御願}{お|ねが}ひ%%%%%%%%%% ルビ調整(配置位置補正)
\ruby{申}{まをし}ます。
』

\原本頁{93-2}%
と
\ruby{和}{やは}らかに
\ruby{云}{い}へば、

\原本頁{93-3}%
『
\ruby{左樣}{さ|う}
\ruby[||j>]{仰}{おつし}あつて
\ruby{下}{くだ}されば
まことに
\ruby{滿足}{まん|ぞく}で
ございます。
%
\ruby{如何}{い|か}
\ruby{樣}{やう}にも
\ruby{此}{この}
\ruby{上}{うへ}
\ruby{{\換字{猶}}}{なほ}
\ruby[||j>]{盡}{じん}
\ruby[||j>]{力}{りよく}を
% \ruby{盡力}{じん|りよく}を
\ruby{辭}{じ}しませぬ。
%
\ruby{併}{しか}し
なか〳〵の
\ruby{重體}{ぢう|たい}の% ルビ調整(原本通り)「重(ぢう)」
\ruby{事}{こと}ですから
\改行% 校正作業の簡略化のため
、
%
\原本頁{93-5}\改行%
\ruby{先日}{せん|じつ}の
\ruby{學士}{がく|し}にも
\ruby{御見}{お|み}せに
なつては?、
』

\原本頁{93-6}%
と
\ruby{腹}{はら}の
\ruby{底}{そこ}に
\ruby{毒}{どく}
\ruby{無}{な}き
\ruby{人}{ひと}の、
%
はや
\ruby{胸}{むな}もとにも
\ruby[<j>]{蟠}{わだかま}りなき
\ruby{挨拶}{あい|さつ}なり。

\原本頁{93-7}%
『
\ruby{非常}{ひ|じやう}に
\ruby{惡}{わる}い
\ruby{方}{はう}へ
\ruby{{\換字{進}}}{すゝ}みまして?。% 踊り字調整「〻(二の字点、揺すり点)に見えるが(ゝ)」
』

\原本頁{93-8}%
『
いや、
%
\ruby{今}{いま}
いけない
といふのでは
\ruby{無}{な}いのですが、
%
\ruby{何樣}{ど|う}も
\ruby{{\換字{前}}}{まへ}
\ruby{申}{まを}した
\ruby{{\換字{通}}}{とほ}り
ですから
\ruby{相良}{さが|ら}さんにも‥‥‥
\ruby{如何}{い|か}にも
\ruby[||j>]{衰}{すゐ}
\ruby[||j>]{{\換字{弱}}}{じやく}が
% \ruby{衰{\換字{弱}}}{すゐ|じやく}が
\ruby{急}{きふ}に
\ruby{甚}{ひど}く
\ruby[<j||]{現}{あらは}れて% 行末行頭の境界付近なので特例処置を施す
\ruby{來}{き}ましたから。
』

\原本頁{93-11}%
と
\ruby{聞}{き}くや
\ruby{否}{いな}や
\ruby{水野}{みづ|の}は
\ruby{心中}{しん|ちう}に
\ruby{疑}{うたが}ひて、
%
\ruby[||j>]{衰}{すゐ}
\ruby[||j>]{{\換字{弱}}}{じやく}は
% \ruby{衰{\換字{弱}}}{すゐ|じやく}は
\ruby{漸々}{やう|やく}に
こそ
\ruby{來}{きた}る
べきなれ、
%
\ruby{急}{きふ}に
\ruby{甚}{ひど}く
\ruby{現}{あらは}るゝ% 踊り字調整「〻(二の字点、揺すり点)に見えるが(ゝ)」
ものにや、
%
\ruby{醫}{い}
ならねば
\ruby{我}{われ}
\ruby{知}{し}らねど、
%
と
\ruby{一度}{ひと|たび}は
\ruby{{\換字{迷}}}{まよ}ひしが、
%
\ruby{惑}{まど}ひて
\ruby{益}{{\換字{𛀁}}き}
\ruby{無}{な}ければ、
%
\ruby[||j>]{一}{いつ}
\ruby[||j>]{瞬}{しゆん}に
% \ruby{一瞬}{いつ|しゆん}に
\ruby{其}{そ}の
\ruby{心}{こゝろ}を% 踊り字調整「〻(二の字点、揺すり点)に見えるが(ゝ)」
\ruby{决}{けつ}して、

\原本頁{94-3}%
『
\ruby{勿論}{もち|ろん}
\ruby{直}{すぐ}に
\ruby{來}{き}て
\ruby{診}{み}て
\ruby{貰}{もら}ひましやう。
』

\原本頁{94-4}%
と
\ruby{云}{い}ひ
\ruby{{\換字{終}}}{をは}つて
\ruby{一禮}{いち|れい}するかと
\ruby{見}{み}えしが、
%
\ruby{忽}{たちま}ち
\ruby{其}{その}
\ruby{姿}{すがた}は
\ruby{闇}{やみ}に
\ruby{隱}{かく}れて
\ruby{眞黑}{まつ|くろ}の
\ruby{中}{うち}に
\ruby{走}{は}せ
\ruby{去}{さ}れば、
%
\ruby{尾竹}{を|だけ}は
\ruby[||j>]{提}{ちやう}
\ruby[||j>]{灯}{ ちん}を
% \ruby{提灯}{ちやう|ちん}を
\ruby{手}{て}
にしたるまゝ、% 踊り字調整「〻(二の字点、揺すり点)に見えるが(ゝ)」
%
うつかりと
\ruby{路央}{みち|なか}に
\ruby{獨}{ひと}り
\ruby{立}{た}つて、
%
\ruby{黑白}{あ|や}なき
\ruby{暗}{くら}さに
\ruby{水野}{みづ|の}の
\ruby{下駄}{げ|た}の
\ruby{音}{おと}の、
%
\ruby{早}{はや}くも
\ruby{隔}{へだ}たり
\ruby{行}{ゆ}く
\ruby{方}{かた}を
\ruby{見}{み}えもせぬに
\ruby{永}{なが}く
\ruby{見}{み}
\ruby{{\換字{送}}}{おく}りたり。
