\Entry{其十六}

% メモ 校正終了 2024-04-20 2024-05-31
\原本頁{87-8}%
\ruby[g]{尾竹}{を だけ}の
\ruby{聲}{こゑ}は
\ruby{闇}{やみ}の
\ruby[g]{寂寥}{しづかさ}に
\ruby{響}{ひゞ}きて、% 踊り字調整「〻(二の字点、揺すり点)に濁点に見えるが(ゞ)」
%
\ruby{愚}{おろか}しくいと
\ruby{大}{おほ}きく
\ruby{聞}{きこ}えたるに、
%
\ruby[g]{水野}{みづの }は
\ruby{何}{なん}となく
\ruby{厭}{いと}はしく
\ruby{{\換字{感}}}{かん}じつ、
%
こゝにて% 踊り字調整「〻(二の字点、揺すり点)に見えるが(ゝ)」
\ruby{{\換字{又}}}{また}
\ruby{我}{われ}と
\ruby[g]{此男}{こ れ }との
\ruby{問}{と}ひつ
\ruby{答}{こた}へつせば、
%
\ruby{其}{そ}の
\ruby{聲}{こゑ}の
\ruby[g]{彼家}{かしこ }の
\ruby[g]{人々}{ひと〴〵}にも
\ruby{聞}{きこ}えんことを
\ruby{忌}{いま}はしく
\原本頁{88-1}\改行%
\ruby{思}{おも}ひて、
%
\ruby[g]{言葉}{ことば }は
\ruby{無}{な}き
\ruby[g]{擧動}{そぶり }
ばかりに
\ruby[g]{尾竹}{を だけ}を
\ruby{誘}{いざな}ひ、
%
\ruby{突}{つ}と
\ruby{外}{そと}の
\ruby{方}{かた}に
\ruby{去}{さ}らん
とすれば、
%
\ruby[g]{尾竹}{を だけ}は
\ruby{慌}{あわ}てゝ% 踊り字調整「〻(二の字点、揺すり点)に見えるが(ゝ)」
\ruby{先}{さき}に
\ruby{立}{た}つて、
%
\ruby{手}{て}に
\ruby{持}{も}てる
\ruby[||j>]{提}{ちやう}
\ruby[||j>]{灯}{ ちん}に
% \ruby{提灯}{ちやう|ちん}に
\ruby[g]{足元}{あしもと}を
\ruby{照}{て}らしたり。

\原本頁{88-4}%
\ruby{共}{とも}に
\ruby{歩}{あゆ}むこと
\ruby{四五歩}{し|ご|ほ}
ならずして、
%
\ruby{彼}{か}の
お
\ruby{澤}{さは}
\ruby{婆}{ばゝ}が% 踊り字調整「〻(二の字点、揺すり点)に見えるが(ゝ)」
\ruby{家}{いへ}の
\ruby{内}{うち}より、

\原本頁{88-5}%
『
ギリギリツ。
』

\原本頁{88-6}%
といふ
\ruby{聲}{こゑ}
\ruby{先}{ま}づ
\ruby{聞}{きこ}えて、
%
\ruby{次}{つ}いで、

\原本頁{88-7}%
『
ウーンウーン。
』

\原本頁{88-8}%
といふ
\ruby[g]{寢唸}{ね うな}りの
\ruby{聞}{きこ}ゆれば、
%
\ruby[g]{尾竹}{を だけ}は
\ruby{思}{おも}はずも
ビクリと
\ruby{顫}{ふる}へて、
%
\ruby{手}{て}に
したる
\ruby[||j>]{提}{ちやう}
\ruby[||j>]{灯}{ ちん}に
% \ruby{提灯}{ちやう|ちん}に
\ruby{烈}{はげ}しき
\ruby{浪}{なみ}を
\ruby{打}{う}たせつ、

\原本頁{88-10}%
『
ナ、
%
\ruby{何}{なん}でしやう
\ruby{彼}{あ}の
\ruby{音}{おと}は?。
』

\原本頁{88-11}%
と
\ruby{振}{ふ}り
\ruby{{\換字{返}}}{かへ}つて
\ruby[g]{水野}{みづの }に
\ruby{{\換字{尋}}}{たづ}ねたり。

\原本頁{89-1}%
されど
\ruby[g]{水野}{みづの }は
\ruby[g]{尾竹}{を だけ}が
\ruby{此}{こ}の
\ruby[g]{言葉}{ことば }を、
%
\ruby[g]{閑事}{むだごと}なりと
\ruby{云}{い}はぬ
ばかりに、
%
たゞ% 踊り字調整「〻(二の字点、揺すり点)に濁点に見えるが(ゞ)」
\ruby[g]{無言}{む ごん}を
もて
あしらひ
\ruby{棄}{す}て、
%
おのが
\ruby{歩}{あゆ}まん
とする
\ruby{方}{かた}に
\ruby{歩}{あゆ}み
\ruby{去}{さ}りながら、

\原本頁{89-4}%
『
\ruby[g]{岩崎}{いはさき}は% 原本のこの部分は「いわさき」
\ruby[g]{何樣}{ど う }で
ございます、
%
よろしいのですか?。
』

\原本頁{89-5}%
と
\ruby[g]{先刻}{さ き }より
\ruby{此}{こ}の
\ruby{醫}{い}の
\ruby[g]{樣子}{やうす }に
\ruby[g]{大事}{だいじ }
\ruby{無}{な}しとは
\ruby{察}{さつ}したれど、
%
\ruby{問}{と}はんとして
\ruby[g]{一刻}{いつこく}も
\ruby{忘}{わす}れざりし
\ruby{問}{とひ}を
\ruby{發}{はつ}すれば、
%
\ruby[g]{眞{\換字{情}}}{まこと }
\ruby{餘}{あま}りし
\ruby{其}{そ}の
\ruby[g]{言葉}{ことば }の
\ruby[g]{自然}{おのづ }と
\ruby{威}{ゐ}
あるやうなるに
\ruby[g]{尾竹}{を だけ}は
\ruby{壓}{お}されて、
%
\ruby{今}{いま}
\ruby{我}{わ}が
\ruby{口}{くち}に
\ruby{出}{いだ}したる
\ruby{問}{とひ}
\原本頁{89-8}\改行%
の
\ruby{答}{こたへ}を
\ruby{得}{{\換字{𛀁}}}ざるをも、
%
また
\ruby[g]{水野}{みづの }が
\ruby[g]{如何}{い か }なれば
\ruby[g]{如是}{かゝる }% 踊り字調整「〻(二の字点、揺すり点)に見えるが(ゝ)」
\ruby[g]{時{\換字{分}}}{じ ぶん}に
\ruby{此}{こゝ}% 踊り字調整「〻(二の字点、揺すり点)に見えるが(ゝ)」
\ruby{邊}{ら}には
\原本頁{89-9}\改行%
\ruby{佇}{たゝず}み% 踊り字調整「〻(二の字点、揺すり点)に見えるが(ゝ)」
\ruby{居}{ゐ}たるやと
\ruby{{\換字{尋}}}{たづ}ね
まほしく
\ruby{思}{おも}ひ
\ruby{居}{ゐ}たるをも
\ruby[<j>]{盡}{こと〴〵}く
\ruby{皆}{みな}
\ruby{忘}{わす}れ
\ruby{果}{は}てゝ% 踊り字調整「〻(二の字点、揺すり点)に見えるが(ゝ)」
\改行% 校正作業の簡略化のため
、

\原本頁{89-10}%
『
いや
\ruby{御{\換字{尋}}問}{お|たづ|ね}が
\ruby{無}{な}くても
\ruby{夜}{よ}でも
\ruby{明}{あ}けましたら
\ruby[g]{一寸}{ちよつと}
\ruby{上}{あが}つてなりと
\ruby[g]{申上}{まをしあ}げやうと
\ruby{思}{おも}つて
\ruby{居}{を}りましたが、
%
\ruby{看護{\換字{婦}}}{かん|ご|ふ}の
\ruby[g]{注意}{ちうい }% 原本通り「ゆ」無しで「ちうい」
からして
\ruby[g]{御使}{おつかひ}が
あつたので、
%
\ruby{今}{いま}しがた
\ruby{出}{で}て
\ruby{見}{み}ると、
%
\ruby{實}{じつ}は
\ruby{甚}{はなは}だ
\ruby[g]{面白}{おもしろ}く
\ruby{無}{な}くなつて
\ruby{居}{ゐ}るのです。
%
\ruby[g]{勿論}{もちろん}
\ruby{今}{いま}が
\ruby{今}{いま}
といふ
やうな
ことは
ありませんが、
%
\原本頁{90-3}\改行%
\ruby[g]{全體}{ぜんたい}が
\ruby[g]{{\換字{丈}}夫}{ぢやうぶ}づくり
といふ
\ruby{方}{かた}では
\ruby{無}{な}いのです
のに、
%
たゞ% 踊り字調整「〻(二の字点、揺すり点)に濁点に見えるが(ゞ)」
\ruby[g]{氣性}{きしやう}が
\ruby[g]{確乎}{しつかり}として
\ruby{居}{ゐ}らるゝ% 踊り字調整「〻(二の字点、揺すり点)に見えるが(ゝ)」
ばかりで、
%
\ruby{今}{いま}までは
\ruby[g]{病苦}{びやうく}に
\ruby{負}{ま}けずに
\ruby{居}{ゐ}られた
ところ、
%
\ruby[g]{何樣}{ど う }して、
%
\ruby[g]{精神}{せいしん}
\ruby[g]{作用}{さ よう}だつて
\ruby{限}{かぎり}の
あるものですもの、
%
\ruby[g]{{\換字{連}}日}{れんじつ}の
\ruby[g]{高度}{かうど }の
\ruby{熱}{ねつ}では
\ruby{耐}{たま}りません、
%
とう〳〵
\ruby{堪}{こた}へに
\ruby{堪}{こた}へ
きれなく
なられましたのです。
%
さあ
\ruby[g]{左樣}{さ う }なると
\ruby{其}{それ}と
\ruby[g]{同時}{どうじ }に、
%
\ruby[g]{自然}{し ぜん}と
\ruby{來}{き}て
\ruby{居}{ゐ}た
\ruby[||j>]{衰}{すゐ}
\ruby[||j>]{{\換字{弱}}}{じやく}が、
% \ruby{衰{\換字{弱}}}{すゐ|じやく}が、
%
\ruby[g]{俄然}{が ぜん}と
\ruby{外}{そと}に
\ruby{現}{あらは}れて
まゐりましたので、
%
\ruby[g]{一體}{いつたい}に
\ruby[g]{何處}{ど こ }も
\原本頁{90-9}\改行%
\ruby[g]{彼處}{かしこ }も
\ruby{惡}{わる}くなつて
\ruby{來}{き}た
といふやうな
\ruby{譯}{わけ}です。
%
しかし
\ruby[<j>]{幸}{さいはひ}に
\ruby{特}{こと}に
\ruby{肺}{はい}が
\ruby{惡}{わる}くなつたとか
\ruby[g]{心臓}{しんざう}が
\ruby{惡}{わる}くなつたとか
\ruby{云}{い}ふ
のでは
ありませんから、
%
まだ〳〵
\ruby[g]{十{\換字{分}}}{じふぶん}
\ruby[g]{有望}{いうばう}なので、
%
\ruby{云}{い}はゞ% 踊り字調整「〻(二の字点、揺すり点)に濁点に見えるが(ゞ)」
\ruby[g]{彼樣}{あ ゝ }% 踊り字調整「〻(二の字点、揺すり点)に見えるが(ゝ)」
いふ
\ruby[||j>]{大}{たい}
\ruby[||j>]{病}{びやう}に
% \ruby{大病}{たい|びやう}に
かゝつた% 踊り字調整「〻(二の字点、揺すり点)に見えるが(ゝ)」
\ruby[||j>]{患}{くわん}
\ruby[||j>]{者}{ じや}の、
% \ruby{患者}{くわん|じや}の、
%
\ruby[g]{何樣}{ど う }も
\ruby[g]{經{\換字{過}}}{けいくわ}し
なければ
ならぬ
\ruby{已}{や}むを
\ruby{得}{{\換字{𛀁}}}ざる
\ruby[g]{塲合}{ば あひ}% 原文通り「塲」
なのです。
』

\原本頁{91-3}%
と
\ruby[g]{一{\換字{半}}}{いつぱん}は
\ruby[g]{水野}{みづの }を
\ruby{慰}{なぐさ}め、
%
\ruby[g]{一{\換字{半}}}{いつぱん}は
おのれを
\ruby[g]{辯護}{べんご }するが% 弁 瓣 辦 辧 辨 辩 (辯)
\ruby{如}{ごと}く、
%
\ruby[g]{素人}{しろうと}
\ruby{解}{わか}り
すべきことを
\ruby[g]{條理}{でうり }
\ruby[||j>]{賢}{かしこ}く
\ruby{{\換字{述}}}{の}べたり。

\原本頁{91-5}%
\ruby[g]{水野}{みづの }は
\ruby{五十子}{い|そ|こ}の
\ruby[g]{容態}{ようだい}
あしゝと% 踊り字調整「〻(二の字点、揺すり点)に見えるが(ゝ)」
\ruby{聞}{き}きて、
%
さて
こそと
\ruby{胸}{むね}を
\ruby{躍}{をど}らせつ
\改行% 校正作業の簡略化のため
、
%
\原本頁{91-6}\改行%
\ruby{先}{ま}ず
\ruby{悲}{かな}しくも
\ruby[g]{腹立}{はらだ }たしき
おもひして、
%
はや
\ruby[g]{苛々}{いら〳〵}と
\ruby{心}{こゝろ}は% 踊り字調整「〻(二の字点、揺すり点)に見えるが(ゝ)」
\ruby{烈}{はげ}しくなり、
%
\ruby{此}{こ}の
\ruby[g]{醫者}{い しや}の
\ruby{{\換字{技}}}{わざ}
\ruby{鈍}{にぶ}きを
\ruby{怒}{いか}る
とにはあらねど、
%
\ruby{其}{そ}の
\ruby[g]{言葉}{ことば }
\ruby[||j>]{巧}{たくみ}なるが
\ruby[g]{小憎}{こ にく}くて、

\原本頁{91-9}%
『
\ruby{已}{や}むを
\ruby{得}{{\換字{𛀁}}}ざる
\ruby[g]{塲合}{ば あひ}で!。% 原文通り「塲」
%
\ruby[g]{成程}{なるほど}
\ruby{御{\換字{道}}理}{ご|もつ|とも}です、
%
\ruby{已}{や}むを
\ruby{得}{{\換字{𛀁}}}ざる
\ruby[g]{塲合}{ば あひ}で!。% 原文通り「塲」
%
まかり
\ruby[g]{間{\換字{違}}}{ま ちが}つて
\ruby[g]{何樣}{ど う }なりましても、
%
\ruby[g]{勿論}{もちろん}
みんな
\ruby{已}{や}むを
\ruby{得}{{\換字{𛀁}}}ざる
\ruby[g]{塲合}{ば あひ}ですナ。% 原文通り「塲」
』

\原本頁{92-1}%
と、
%
\ruby{一}{ひ}ト
\ruby{當}{あて}
\ruby{當}{あ}つれ
\footnote{「\ruby{當}{あ}つ」の活用形の一つとして原本通り「\ruby{當}{あ}つれ」とした
原本通り(国会図書館 コマ番号51/160 p-092 l-01)}%
ば
\ruby[g]{尾竹}{を だけ}は
\ruby{驚}{おどろ}き、
%
\ruby[g]{{\換字{平}}日}{ひごろ }は
\ruby{物}{もの}
\ruby{柔}{やはら}かにして
\ruby[g]{斯樣}{か う }は
\ruby{無}{な}かりし
\ruby{人}{ひと}の、
%
\ruby{何}{なん}たる
\ruby{氣}{き}の
\ruby{焦}{い}れかたぞやと
\ruby{呆}{あき}れながら、

\原本頁{92-3}%
『
\ruby[g]{左樣}{さ う }
\ruby[g]{御取}{お と }りになつては
\ruby{困}{こま}ります。
%
わたくしが
\ruby[g]{責任}{せきにん}を
\ruby{{\換字{逃}}}{のが}れ
やうとして
\ruby{申}{まを}した
のでは
ござりません。
%
わたくしが
\ruby[g]{其樣}{そ ん }なもので
\ruby{無}{な}
\原本頁{92-5}\改行%
いことは
\ruby{御}{ご}
\ruby[g]{承知}{しようち}で
ござりましやう。
%
\ruby[g]{小生}{わたくし}は
\ruby[g]{小生}{わたくし}の
\ruby{及}{およ}ぶ
\ruby{限}{かぎ}りの
\ruby[<j||]{力}{ちから}を
\ruby{盡}{つく}して
\ruby{居}{を}りますのです。
』

\原本頁{92-7}%
と
\ruby[g]{疾辯}{はやくち}に% 弁 瓣 辦 辧 辨 辩 (辯)
\ruby{言}{い}ひたる
\ruby{其}{その}
\ruby{聲}{こゑ}は
\ruby{眞}{まこと}に
\ruby{切}{せつ}なげに
\ruby{泣}{な}きさうにも
\ruby{聞}{きこ}えて、
%
\ruby{{\換字{技}}}{わざ}こそ
\ruby[g]{庸常}{よのつね}にして
\ruby{人}{ひと}に
\ruby{挺}{ぬきん}でも
せざれ、
%
\ruby{心}{こゝろ}は% 踊り字調整「〻(二の字点、揺すり点)に見えるが(ゝ)」
\ruby[||j>]{正}{しやう}
\ruby[||j>]{直}{ ぢき}にして
% \ruby{正直}{しやう|ぢき}にして
\ruby{自}{みづか}ら
\ruby{欺}{あざむ}かざる
\ruby[g]{君子}{くんし }なるを
\ruby{示}{しめ}せり。

\原本頁{92-10}%
\ruby[g]{水野}{みづの }は
\ruby[g]{流石}{さすが }に
これに
\ruby{氣}{き}の
\ruby{毒}{どく}になりて、

\原本頁{92-11}%
『
ヤ、
%
\ruby[g]{先生}{せんせい}、
%
\ruby[g]{御氣}{お き }に
\ruby[g]{御{\換字{留}}}{お と }め
なすつては
いけません。
%
\ruby[g]{先生}{せんせい}の
\ruby{御誠實}{ご|せい|じつ}な
\ruby{事}{こと}は
よく
\ruby{存}{ぞん}じて
\ruby{居}{を}ります。
%
\ruby{{\換字{猶}}}{なほ}%%%%%%%% ルビ調整(配置位置補正)
\ruby[g]{此上}{このうへ}とも%%%%%% ルビ調整(配置位置補正)
\ruby[g]{何{\換字{分}}}{なにぶん}% ルビ調整(配置位置補正)
\ruby[g]{御願}{お ねが}ひ%%%%%%%%%% ルビ調整(配置位置補正)
\ruby{申}{まをし}ます。
』

\原本頁{93-2}%
と
\ruby{和}{やは}らかに
\ruby{云}{い}へば、

\原本頁{93-3}%
『
\ruby[g]{左樣}{さ う }
\ruby[||j>]{仰}{おつし}あつて
\ruby{下}{くだ}されば
まことに
\ruby[g]{滿足}{まんぞく}で
ございます。
%
\ruby[g]{如何}{い か }
\ruby{樣}{やう}にも
\ruby{此}{この}
\ruby{上}{うへ}
\ruby{{\換字{猶}}}{なほ}
\ruby[||j>]{盡}{じん}
\ruby[||j>]{力}{りよく}を
% \ruby{盡力}{じん|りよく}を
\ruby{辭}{じ}しませぬ。
%
\ruby{併}{しか}し
なか〳〵の
\ruby[g]{重體}{ぢうたい}の% ルビ調整(原本通り)「重(ぢう)」
\ruby{事}{こと}ですから
\改行% 校正作業の簡略化のため
、
%
\原本頁{93-5}\改行%
\ruby[g]{先日}{せんじつ}の
\ruby[g]{學士}{がくし }にも
\ruby[g]{御見}{お み }せに
なつては?、
』

\原本頁{93-6}%
と
\ruby{腹}{はら}の
\ruby{底}{そこ}に
\ruby{毒}{どく}
\ruby{無}{な}き
\ruby{人}{ひと}の、
%
はや
\ruby{胸}{むな}もとにも
\ruby[<j>]{蟠}{わだかま}りなき
\ruby[g]{挨拶}{あいさつ}なり。

\原本頁{93-7}%
『
\ruby[g]{非常}{ひじやう}に
\ruby{惡}{わる}い
\ruby{方}{はう}へ
\ruby{{\換字{進}}}{すゝ}みまして?。% 踊り字調整「〻(二の字点、揺すり点)に見えるが(ゝ)」
』

\原本頁{93-8}%
『
いや、
%
\ruby{今}{いま}
いけない
といふのでは
\ruby{無}{な}いのですが、
%
\ruby[g]{何樣}{ど う }も
\ruby{{\換字{前}}}{まへ}
\ruby{申}{まを}した
\ruby{{\換字{通}}}{とほ}り
ですから
\ruby[g]{相良}{さがら }さんにも‥‥‥
\ruby[g]{如何}{い か }にも
\ruby[||j>]{衰}{すゐ}
\ruby[||j>]{{\換字{弱}}}{じやく}が
% \ruby{衰{\換字{弱}}}{すゐ|じやく}が
\ruby{急}{きふ}に
\ruby{甚}{ひど}く
\ruby[<j||]{現}{あらは}れて% 行末行頭の境界付近なので特例処置を施す
\ruby{來}{き}ましたから。
』

\原本頁{93-11}%
と
\ruby{聞}{き}くや
\ruby{否}{いな}や
\ruby[g]{水野}{みづの }は
\ruby[g]{心中}{しんちう}に
\ruby{疑}{うたが}ひて、
%
\ruby[||j>]{衰}{すゐ}
\ruby[||j>]{{\換字{弱}}}{じやく}は
% \ruby{衰{\換字{弱}}}{すゐ|じやく}は
\ruby[g]{漸々}{やうやく}に
こそ
\ruby{來}{きた}る
べきなれ、
%
\ruby{急}{きふ}に
\ruby{甚}{ひど}く
\ruby{現}{あらは}るゝ% 踊り字調整「〻(二の字点、揺すり点)に見えるが(ゝ)」
ものにや、
%
\ruby{醫}{い}
ならねば
\ruby{我}{われ}
\ruby{知}{し}らねど、
%
と
\ruby[g]{一度}{ひとたび}は
\ruby{{\換字{迷}}}{まよ}ひしが、
%
\ruby{惑}{まど}ひて
\ruby{益}{{\換字{𛀁}}き}
\ruby{無}{な}ければ、
%
\ruby[||j>]{一}{いつ}
\ruby[||j>]{瞬}{しゆん}に
% \ruby{一瞬}{いつ|しゆん}に
\ruby{其}{そ}の
\ruby{心}{こゝろ}を% 踊り字調整「〻(二の字点、揺すり点)に見えるが(ゝ)」
\ruby{决}{けつ}して、

\原本頁{94-3}%
『
\ruby[g]{勿論}{もちろん}
\ruby{直}{すぐ}に
\ruby{來}{き}て
\ruby{診}{み}て
\ruby{貰}{もら}ひましやう。
』

\原本頁{94-4}%
と
\ruby{云}{い}ひ
\ruby{{\換字{終}}}{をは}つて
\ruby[g]{一禮}{いちれい}するかと
\ruby{見}{み}えしが、
%
\ruby{忽}{たちま}ち
\ruby{其}{その}
\ruby{姿}{すがた}は
\ruby{闇}{やみ}に
\ruby{隱}{かく}れて
\ruby[g]{眞黑}{まつくろ}の
\ruby{中}{うち}に
\ruby{走}{は}せ
\ruby{去}{さ}れば、
%
\ruby[g]{尾竹}{を だけ}は
\ruby[||j>]{提}{ちやう}
\ruby[||j>]{灯}{ ちん}を
% \ruby{提灯}{ちやう|ちん}を
\ruby{手}{て}
にしたるまゝ、% 踊り字調整「〻(二の字点、揺すり点)に見えるが(ゝ)」
%
うつかりと
\ruby[g]{路央}{みちなか}に
\ruby{獨}{ひと}り
\ruby{立}{た}つて、
%
\ruby[g]{黑白}{あ や }なき
\ruby{暗}{くら}さに
\ruby[g]{水野}{みづの }の
\ruby[g]{下駄}{げ た }の
\ruby{音}{おと}の、
%
\ruby{早}{はや}くも
\ruby{隔}{へだ}たり
\ruby{行}{ゆ}く
\ruby{方}{かた}を
\ruby{見}{み}えもせぬに
\ruby{永}{なが}く
\ruby{見}{み}
\ruby{{\換字{送}}}{おく}りたり。
