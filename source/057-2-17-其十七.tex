\Entry{其十七}

% メモ 校正終了 2024-04-20 2024-05-31 2024-07-01
\原本頁{94-9}%
\ruby{一度}{いち|ど}
あることは
\ruby{二度}{に|ど}
ありといふ
\ruby{世}{よ}の
\ruby[<j>]{諺}{ことわざ}の
\ruby{人}{ひと}を
\ruby{欺}{あざむ}かず、
%
\ruby{水野}{みづ|の}は
ふた〻び% ルビ調整(原本通り)「〻(二の字点、揺すり点)」
\ruby{熬}{い}りつくが
\ruby{如}{ごと}き
\ruby{憂}{うれひ}を
\ruby{抱}{いだ}いて% ルビ調整(原本通り)(いだ)
\ruby{南方}{みな|み}に
\ruby{走}{はし}りけるが、
%
\ruby{闇夜}{やみ|よ}の
\ruby{{\換字{道}}}{みち}の
\ruby{捗取}{はか|ど}らずして、
%
その
\ruby{相良}{さが|ら}が
\ruby{家}{いへ}を
\ruby{訪}{と}ひし
\ruby{時}{とき}は
\ruby{既}{すで}に
\ruby{遲}{おそ}く、
%
\ruby{舎}{いへ}の
\ruby{内}{うち}は
まだ
\ruby{燈火}{とも|しび}
\ruby{無}{な}くてはの
\ruby{頃}{ころ}ながら、
%
\ruby{{\換字{戸}}外}{そ|と}は
\ruby{既}{はや}
\ruby{人顏}{ひと|がほ}
\ruby{定}{さだ}かなる
ほどに
なりて、
%
かつて
\ruby{島木}{しま|き}の
\ruby{寓}{やど}より
\ruby{歸}{かへ}るさに
\ruby{訪}{と}ひし
\ruby{時}{とき}と
\ruby{同}{おな}じ
ほどの
\ruby{明}{あか}るさ
とはなり
\ruby{居}{ゐ}たり。

\原本頁{95-5}%
た〻かれて% ルビ調整(原本通り)「〻(二の字点、揺すり点)」
\ruby{怒}{おこ}らぬものは
\ruby{醫師}{い|し}の
\ruby{家}{いへ}と、
%
\ruby{憚}{はゞか}りも% 「憚 は(ゞ)か」% TODO 原本の「二の字点、揺すり点」に濁点のグリフが見つからないので「ゞ」
\ruby{無}{な}く
\ruby{打}{うち}
\ruby{敲}{た〻}けば、% ルビ調整(原本通り)「〻(二の字点、揺すり点)」
%
\ruby{思}{おも}ひの
ほかに
\ruby{早}{はや}く
\ruby{{\換字{返}}事}{へん|じ}して、
%
\ruby{立}{たち}
\ruby{出}{い}でたるは
\ruby{{\換字{前}}}{さき}の
\ruby{日}{ひ}
\ruby{窘}{くるし}め
やりたる
\ruby{彼}{か}の
\ruby{盤臺面}{ばん|だい|づら}の
\ruby{書生}{しよ|せい}なり。
%
\ruby{我}{われ}を
\ruby{侮}{あなど}りがたき
\ruby{男}{をとこ}と
\ruby{思}{おも}ひ
\ruby{{\換字{込}}}{こ}みてや
\ruby{挨拶}{あい|さつ}も
\原本頁{95-8}\改行%
\ruby{慇懃}{いん|ぎん}に
\ruby{愛想}{あい|さう}
よければ、
%
おのづから
\ruby{物}{もの}も
\ruby{云}{い}ひ
\ruby{易}{やす}くて、
%
わざ〳〵
\ruby{來}{きた}れる
\ruby{{\換字{所}}以}{ゆゑ|ん}を% ルビ調整(原本通り)
\ruby{手短}{て|みじか}に
\ruby{{\換字{述}}}{の}べ、
%
さて
\ruby{先生}{せん|せい}の
\ruby{御來診}{ご|らい|しん}をと
\ruby{乞}{こ}へば、
%
\ruby{書生}{しよ|せい}は
\ruby{困}{こま}りきつたる
\ruby{顏}{かほ}つきして、

\原本頁{95-11}%
『
\ruby{實}{じつ}は
\ruby{先生}{せん|せい}は
たつた
\ruby{今}{いま}
\ruby{出}{で}て
\ruby{行}{ゆ}かれたのです、
%
やはり
\ruby{病家}{びやう|か}の
\ruby{急}{きふ}の
\ruby{{\換字{迎}}}{むか}へを
\ruby{受}{う}けられて。
%
しかし
\ruby{行}{ゆ}かれた
\ruby{先}{さき}が
\ruby{餘計}{よ|けい}
\ruby{{\換字{遠}}}{とほ}い
ところでも
ありませんから、
%
\ruby{二時間}{に|じ|かん}も
\ruby{立}{た}つ
\ruby{中}{うち}には
\ruby{歸}{かへ}らる〻でしやう。% ルビ調整(原本通り)「〻(二の字点、揺すり点)」
%
\ruby{歸}{かへ}られたら
\ruby{必}{かなら}ず
\ruby{左樣}{さ|う}
\ruby{申}{まを}しまして、
%
\ruby{屹度}{きつ|と}
\ruby[||j>]{回}{くわい}
\ruby[||j>]{診}{ しん}になる% 原本通り「回」
% \ruby{回診}{くわい|しん}になる% 原本通り「回」
\ruby{樣}{やう}に
\ruby{致}{いた}しましやう。
』

\原本頁{96-4}%
と
\ruby{云}{い}ひ
\ruby{{\換字{終}}}{をは}りしが、
%
\ruby{水野}{みづ|の}が
\ruby{面}{おもて}に
\ruby[||j>]{{\換字{難}}}{なん}
\ruby[||j>]{色}{しよく}
% \ruby{{\換字{難}}色}{なん|しよく}
あるを
\ruby{見}{み}て、

\原本頁{96-5}%
『
\ruby{勿論}{もち|ろん}
\ruby{先生}{せん|せい}の
\ruby{歸}{かへ}らる〻まで、% ルビ調整(原本通り)「〻(二の字点、揺すり点)」
%
\ruby{此處}{こ|〻}に% ルビ調整(原本通り)「〻(二の字点、揺すり点)」
\ruby{御待}{お|まち}
なすつて
いらしつて、
%
\ruby{御直接}{ご|ぢ|き}に
\ruby{御頼}{お|たの}み
なさるとも
\ruby{其}{それ}は
\ruby{御{\換字{随}}意}{ご|ずゐ|い}です。%「隨」TODO 変更 ⻖左円辶
』

\原本頁{96-6}%
と
\ruby{云}{い}ひ
\ruby{足}{た}したるは、
%
よく〳〵
\ruby{此}{こ}の
\ruby{意地}{い|ぢ}
\ruby{{\換字{強}}}{つよ}き
\ruby{客}{かく}の
\ruby{執念}{しふ|ね}きに
\ruby{凝}{こ}りて
\改行% 校正作業の簡略化のため
、
%
\原本頁{96-8}\改行%
ふた〻び% ルビ調整(原本通り)「〻(二の字点、揺すり点)」
\ruby{{\換字{前}}}{さき}の
\ruby{日}{ひ}の
\ruby{如}{ごと}く
\ruby{其}{そ}の
\ruby{怒}{いか}りを
\ruby{惹}{ひ}く
\ruby{事}{こと}などの
\ruby{無}{な}からん
やうにと、
%
\ruby{勉}{つと}めて
\ruby{意}{こ〻ろ}を% ルビ調整(原本通り)「〻(二の字点、揺すり点)」
\ruby{用}{もち}ゐたりと
\ruby{見}{み}えたり。

\原本頁{96-10}%
\ruby{書生}{しよ|せい}の
\ruby{言}{い}へる
ところは
\ruby[<j>]{全}{まつた}く
\ruby[<j>]{僞}{いつはり}
ならず
\ruby{見}{み}ゆるに、
%
\ruby{世}{よ}に
\ruby{行}{おこな}はる〻% ルビ調整(原本通り)「〻(二の字点、揺すり点)」
\ruby{醫}{い}の
\ruby{忙}{せは}しくして
\ruby[||j>]{暇}{いとま}
\ruby[||j>]{無}{ な}きは
\ruby{如何}{いか|ん}とも
すべから
ざること
ながら
\ruby{差當}{さし|あた}つて
\ruby{今}{いま}を
\ruby{何}{なに}と
せんと、
%
\ruby{水野}{みづ|の}は
\ruby{礑}{はた}と
\ruby{行}{ゆ}き
\ruby{詰}{つま}りて、
%
あたかも
\ruby{帆{\換字{船}}}{ほ|ぶね}に
\ruby{舵}{かぢ}を
\ruby{失}{うしな}ひ、
%
\ruby{奔車}{ほん|しや}に
\ruby{轄}{くさび}を
\ruby{拔}{ぬ}かれたる
ごとく、
%
\ruby{言}{い}はん
かた
\ruby{無}{な}き
\ruby[||j>]{心}{こ〻ろ}% ルビ調整(原本通り)「〻(二の字点、揺すり点)」
\ruby[||j>]{細}{ ぼそ}さを
\ruby{覺}{おぼ}えて、
%
\ruby{憮然}{ぶ|ぜん}として
\ruby{言}{ことば}も
\ruby{無}{な}く
\ruby{物}{もの}を
\ruby{思}{おも}ひたり。

\原本頁{97-4}%
\ruby{書生}{しよ|せい}は
\ruby{水野}{みづ|の}の
\ruby{容子}{よう|す}を
\ruby{見}{み}て
\ruby{氣}{き}の
\ruby{毒}{どく}さに
\ruby{堪}{た}へでや、

\原本頁{97-5}%
『
\ruby{{\換字{遠}}路}{ゑん|ろ}の
ところを
\ruby{御來臨}{お|い|で}に
なつたのに
\ruby{生憎}{あひ|にく}で、
%
\ruby{如何}{い|か}にも
\ruby{御氣}{お|き}の
\ruby{毒}{どく}で
ございますが、
%
\ruby{必}{かなら}ず
\ruby{小生}{わた|くし}は
\ruby{左樣}{さ|う}
\ruby{申}{まを}しまして、
%
\ruby{是非}{ぜ|ひ}とも
\ruby[<j||]{回}{くわい}% 原本通り「回」
\ruby[<j||]{診}{しん}に% 行末行頭の境界付近なので特例処置を施す
% \ruby{回診}{くわい|しん}に% 原本通り「回」
なるやうに
\ruby{致}{いた}しまする。
%
\ruby{時間}{じ|かん}の
ところは
\ruby{兎}{と}に
\ruby{角}{かく}、
%
\ruby{必}{かなら}ず
\ruby{診}{み}てあげますことは
\ruby{診}{み}てあげますやう、
%
これは
\ruby{小生}{わた|くし}が
\ruby{御受合}{お|うけ|あひ}
\ruby{申}{まを}して
\ruby{左樣}{さ|う}いたしますから。
』

\原本頁{97-10}%
と、
%
\ruby{{\換字{前}}}{さき}の
\ruby{日}{ひ}とは
\ruby{打}{う}つて
\ruby{變}{かは}つて
\ruby{親切}{しん|せつ}に
\ruby{言}{い}ひ
\ruby{吳}{く}る〻、% ルビ調整(原本通り)「〻(二の字点、揺すり点)」
%
その
\ruby{言葉}{こと|ば}には
\ruby{力}{ちから}あり、
%
その
\ruby{樣子}{やう|す}には
\ruby[<j>]{勢}{いきほひ}あるに、
%
\ruby{今}{いま}は
\ruby{此}{こ}の
\ruby{男}{をとこ}を
\ruby{頼}{たの}まんより
\ruby{他}{ほか}の
\原本頁{98-1}\改行%
\ruby{{\換字{道}}}{みち}
なければ、
%
\ruby{水野}{みづ|の}は
いと
\ruby{懇切}{ねん|ごろ}に
\ruby{頼}{たの}み
\ruby{聞}{きこ}えて、
%
\ruby{是非}{ぜ|ひ}
\ruby{無}{な}くも
\ruby{元}{もと}
\ruby{來}{き}し
\ruby{{\換字{道}}}{みち}へ
\ruby{引{\換字{返}}}{ひつ|かへ}したり。

\原本頁{98-3}%
\ruby{戀人}{こひ|ゞと}の% TODO 原本の「二の字点、揺すり点」に濁点のグリフが見つからないので「ゞ」
\ruby{病}{やまひ}は
\ruby{{\換字{前}}}{さき}の
\ruby{日}{ひ}より
\ruby{凶}{あし}き
かたへ
\ruby{{\換字{進}}}{す〻}める% ルビ調整(原本通り)「〻(二の字点、揺すり点)」
なり、
%
\ruby{頼}{たの}む
\ruby{醫}{い}は
\ruby{他}{た}に
\ruby{出}{い}で〻% ルビ調整(原本通り)「〻(二の字点、揺すり点)」
\ruby{家}{いへ}に
あらぬなり、
%
\ruby{夢}{ゆめ}
\ruby{見}{み}は
\ruby{忌}{いま}はしかりし
なり、
%
\ruby{胸}{むね}は
\ruby{騷}{さわ}ぎしなり、
%
\ruby{{\換字{若}}}{もし}やと
\ruby{思}{おも}ひし
ことは
\ruby{不思議}{ふ|し|ぎ}にも
\ruby{中}{あた}りしなり、
%
\ruby{{\換字{弱}}}{よわ}り
かへれる
\原本頁{98-6}\改行%
\ruby{五十子}{い|そ|こ}に
\ruby{一應}{いち|おう}の
\ruby{手當}{て|あて}して
\ruby{歸}{かへ}れる
\ruby{尾竹}{を|だけ}よりは
\ruby{心}{こ〻ろ}に% ルビ調整(原本通り)「〻(二の字点、揺すり点)」
かゝる% ルビ調整(原本通り)「〻(二の字点、揺すり点)」
\ruby{言}{ことば}を
\ruby{聞}{き}きしなり、
%
\ruby{氣味}{き|み}
あしき
\ruby{狗}{いぬ}は
\ruby{{\換字{前}}表}{ぜん|ぺう}かと
おぼしく
\ruby{吠}{ほ}えに
\ruby{吠}{ほ}えしなり、
%
\原本頁{98-8}\改行%
\ruby{無心}{む|しん}の
お
\ruby{濱}{はま}は
\ruby{我}{わ}が
\ruby{五十子}{い|そ|こ}の
\ruby{{\換字{遠}}方}{とほ|く}へ
\ruby{行}{ゆ}かんことを
\ruby{無心}{む|しん}に
\ruby{云}{い}へるなり、
%
\ruby{氣}{き}
にか〻ること% ルビ調整(原本通り)「〻(二の字点、揺すり点)」
のみの
\ruby{何}{なん}ぞ
\ruby{多}{おほ}きやと、
%
\ruby{水野}{みづ|の}は
\ruby{此等}{これ|ら}の
\ruby{事}{こと}を
\ruby{思}{おも}ひ
つゞけつ〻、% TODO 原本の「二の字点、揺すり点」に濁点のグリフが見つからないので「ゞ」% ルビ調整(原本通り)「〻(二の字点、揺すり点)」
%
\ruby{恰}{あたか}も% 恰も「あ(た)かも」
\ruby{{\換字{前}}}{さき}の
\ruby{日}{ひ}と
\ruby{同}{おな}じ
\ruby{曉}{あした}の、
%
\ruby{今日}{け|ふ}は
\ruby{風}{かぜ}
\ruby{無}{な}くて
\ruby{曇}{くも}り
\ruby{{\換字{空}}}{ぞら}の
\ruby{少}{すこ}し
\ruby{闇}{くら}き
のみが
\ruby{異}{ことな}れる
\ruby{同}{おな}じ
\ruby{時刻}{ころ|ほひ}に、
%
\ruby{同}{おな}じく
\ruby{人{\換字{通}}}{ひと|ゞほ}り% TODO 原本の「二の字点、揺すり点」に濁点のグリフが見つからないので「ゞ」
\ruby{{\換字{猶}}}{なほ}
\ruby[||j>]{少}{すくな}き
\ruby{並木}{なみ|き}の
\ruby{{\換字{道}}}{みち}を
\ruby{首}{かうべ}を
\ruby{垂}{た}れて
\ruby[||j>]{力}{ちから}
\ruby[||j>]{無}{ な}く
\ruby{行}{ゆ}き
\ruby{盡}{つく}しつ、
%
\ruby{吾妻橋}{あづ|ま|ばし}の% ルビ調整(原本通り)
\ruby{方}{かた}に
\ruby{去}{さ}らんとする
\ruby{時}{とき}、
%
\ruby{突然}{とつ|ぜん}
として
\ruby{人}{ひと}の
\ruby{我}{わ}が
\ruby{手}{て}を
\ruby{執}{と}るありて、
%
しかも
\ruby{執}{と}られし
\ruby{我}{わ}が
\ruby{手首}{て|くび}に、
%
ざらりと
\ruby{物}{もの}の
\ruby{觸}{さは}りたれば、
%
\ruby{何}{なん}ぞと
\ruby{驚}{おどろ}きて
\ruby{顧}{かへり}みるに、
%
\原本頁{99-4}\改行%
\ruby{骨}{ほね}
\ruby[||j>]{露}{あらは}に
\ruby{萎}{しな}び
\ruby{枯}{から}びて
\ruby[||j>]{冷}{つめた}き
\ruby{細}{ほそ}き
\ruby{手}{て}に
\ruby{我}{わ}が
\ruby{手}{て}は
\ruby{捉}{とら}へ
\ruby{居}{を}られて、
%
\ruby{其}{そ}の
\ruby{手首}{て|くび}に
\ruby{掛}{か}けられ
\ruby{居}{ゐ}たる
\ruby{黑}{くろ}き
\ruby{木}{き}の
\ruby{數珠}{ず|ゞ}の% TODO 原本の「二の字点、揺すり点」に濁点のグリフが見つからないので「ゞ」
\ruby{我}{わ}が
\ruby{手}{て}に
\ruby{滑}{すべ}りて
\ruby{落}{お}ち
か〻れるなり。% ルビ調整(原本通り)「〻(二の字点、揺すり点)」
