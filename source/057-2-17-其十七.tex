\Entry{其十七}

% メモ 校正終了 2024-04-20 2024-05-31 2024-07-01
\原本頁{94-9}%
\ruby[g]{一度}{いちど }
あることは
\ruby[g]{二度}{に ど }
ありといふ
\ruby{世}{よ}の
\ruby[<j>]{諺}{ことわざ}の
\ruby{人}{ひと}を
\ruby{欺}{あざむ}かず、
%
\ruby[g]{水野}{みづの }は
ふたゝび% 踊り字調整「〻(二の字点、揺すり点)に見えるが(ゝ)」
\ruby{熬}{い}りつくが
\ruby{如}{ごと}き
\ruby{憂}{うれひ}を
\ruby{抱}{いだ}いて% ルビ調整(原本通り)(いだ)
\ruby[g]{南方}{みなみ }に
\ruby{走}{はし}りけるが、
%
\ruby[g]{闇夜}{やみよ }の
\ruby{{\換字{道}}}{みち}の
\ruby[g]{捗取}{はかど }らずして、
%
その
\ruby[g]{相良}{さがら }が
\ruby{家}{いへ}を
\ruby{訪}{と}ひし
\ruby{時}{とき}は
\ruby{既}{すで}に
\ruby{遲}{おそ}く、
%
\ruby{舎}{いへ}の
\ruby{内}{うち}は
まだ
\ruby[g]{燈火}{ともしび}
\ruby{無}{な}くてはの
\ruby{頃}{ころ}ながら、
%
\ruby[g]{{\換字{戸}}外}{そ と }は
\ruby{既}{はや}
\ruby[g]{人顏}{ひとがほ}
\ruby{定}{さだ}かなる
ほどに
なりて、
%
かつて
\ruby[g]{島木}{しまき }の
\ruby{寓}{やど}より
\ruby{歸}{かへ}るさに
\ruby{訪}{と}ひし
\ruby{時}{とき}と
\ruby{同}{おな}じ
ほどの
\ruby{明}{あか}るさ
とはなり
\ruby{居}{ゐ}たり。

\原本頁{95-5}%
たゝかれて% 踊り字調整「〻(二の字点、揺すり点)に見えるが(ゝ)」
\ruby{怒}{おこ}らぬものは
\ruby[g]{醫師}{い し }の
\ruby{家}{いへ}と、
%
\ruby{憚}{はゞか}りも% 「憚 は(ゞ)か」% 踊り字調整「〻(二の字点、揺すり点)に濁点に見えるが(ゞ)」
\ruby{無}{な}く
\ruby{打}{うち}
\ruby{敲}{たゝ}けば、% 踊り字調整「〻(二の字点、揺すり点)に見えるが(ゝ)」
%
\ruby{思}{おも}ひの
ほかに
\ruby{早}{はや}く
\ruby[g]{{\換字{返}}事}{へんじ }して、
%
\ruby{立}{たち}
\ruby{出}{い}でたるは
\ruby{{\換字{前}}}{さき}の
\ruby{日}{ひ}
\ruby{窘}{くるし}め
やりたる
\ruby{彼}{か}の
\ruby{盤臺面}{ばん|だい|づら}の
\ruby[g]{書生}{しよせい}なり。
%
\ruby{我}{われ}を
\ruby{侮}{あなど}りがたき
\ruby{男}{をとこ}と
\ruby{思}{おも}ひ
\ruby{{\換字{込}}}{こ}みてや
\ruby[g]{挨拶}{あいさつ}も
\原本頁{95-8}\改行%
\ruby[g]{慇懃}{いんぎん}に
\ruby[g]{愛想}{あいさう}
よければ、
%
おのづから
\ruby{物}{もの}も
\ruby{云}{い}ひ
\ruby{易}{やす}くて、
%
わざ〳〵
\ruby{來}{きた}れる
\ruby[g]{{\換字{所}}以}{ゆゑん }を% ルビ調整(原本通り)
\ruby[g]{手短}{てみじか}に
\ruby{{\換字{述}}}{の}べ、
%
さて
\ruby[g]{先生}{せんせい}の
\ruby{御來診}{ご|らい|しん}をと
\ruby{乞}{こ}へば、
%
\ruby[g]{書生}{しよせい}は
\ruby{困}{こま}りきつたる
\ruby{顏}{かほ}つきして、

\原本頁{95-11}%
『
\ruby{實}{じつ}は
\ruby[g]{先生}{せんせい}は
たつた
\ruby{今}{いま}
\ruby{出}{で}て
\ruby{行}{ゆ}かれたのです、
%
やはり
\ruby[g]{病家}{びやうか}の
\ruby{急}{きふ}の
\ruby{{\換字{迎}}}{むか}へを
\ruby{受}{う}けられて。
%
しかし
\ruby{行}{ゆ}かれた
\ruby{先}{さき}が
\ruby[g]{餘計}{よ けい}
\ruby{{\換字{遠}}}{とほ}い
ところでも
ありませんから、
%
\ruby{二時間}{に|じ|かん}も
\ruby{立}{た}つ
\ruby{中}{うち}には
\ruby{歸}{かへ}らるゝでしやう。% 踊り字調整「〻(二の字点、揺すり点)に見えるが(ゝ)」
%
\ruby{歸}{かへ}られたら
\ruby{必}{かなら}ず
\ruby[g]{左樣}{さ う }
\ruby{申}{まを}しまして、
%
\ruby[g]{屹度}{きつと }
\ruby[||j>]{回}{くわい}
\ruby[||j>]{診}{ しん}になる% 原本通り「回」
% \ruby{回診}{くわい|しん}になる% 原本通り「回」
\ruby{樣}{やう}に
\ruby{致}{いた}しましやう。
』

\原本頁{96-4}%
と
\ruby{云}{い}ひ
\ruby{{\換字{終}}}{をは}りしが、
%
\ruby[g]{水野}{みづの }が
\ruby{面}{おもて}に
\ruby[||j>]{{\換字{難}}}{なん}
\ruby[||j>]{色}{しよく}
% \ruby{{\換字{難}}色}{なん|しよく}
あるを
\ruby{見}{み}て、

\原本頁{96-5}%
『
\ruby[g]{勿論}{もちろん}
\ruby[g]{先生}{せんせい}の
\ruby{歸}{かへ}らるゝまで、% 踊り字調整「〻(二の字点、揺すり点)に見えるが(ゝ)」
%
\ruby[g]{此處}{こ ゝ }に% 踊り字調整「〻(二の字点、揺すり点)に見えるが(ゝ)」
\ruby[g]{御待}{お まち}
なすつて
いらしつて、
%
\ruby{御直接}{ご|ぢ|き}に
\ruby[g]{御頼}{お たの}み
なさるとも
\ruby{其}{それ}は
\ruby{御{\換字{随}}意}{ご|ずゐ|い}です。%「隨」グリフ変更 ⻖左円辶
』

\原本頁{96-6}%
と
\ruby{云}{い}ひ
\ruby{足}{た}したるは、
%
よく〳〵
\ruby{此}{こ}の
\ruby[g]{意地}{い ぢ }
\ruby{{\換字{強}}}{つよ}き
\ruby{客}{かく}の
\ruby[g]{執念}{しふね }きに
\ruby{凝}{こ}りて
\改行% 校正作業の簡略化のため
、
%
\原本頁{96-8}\改行%
ふたゝび% 踊り字調整「〻(二の字点、揺すり点)に見えるが(ゝ)」
\ruby{{\換字{前}}}{さき}の
\ruby{日}{ひ}の
\ruby{如}{ごと}く
\ruby{其}{そ}の
\ruby{怒}{いか}りを
\ruby{惹}{ひ}く
\ruby{事}{こと}などの
\ruby{無}{な}からん
やうにと、
%
\ruby{勉}{つと}めて
\ruby{意}{こゝろ}を% 踊り字調整「〻(二の字点、揺すり点)に見えるが(ゝ)」
\ruby{用}{もち}ゐたりと
\ruby{見}{み}えたり。

\原本頁{96-10}%
\ruby[g]{書生}{しよせい}の
\ruby{言}{い}へる
ところは
\ruby[<j>]{全}{まつた}く
\ruby[<j>]{僞}{いつはり}
ならず
\ruby{見}{み}ゆるに、
%
\ruby{世}{よ}に
\ruby{行}{おこな}はるゝ% 踊り字調整「〻(二の字点、揺すり点)に見えるが(ゝ)」
\ruby{醫}{い}の
\ruby{忙}{せは}しくして
\ruby[||j>]{暇}{いとま}
\ruby[||j>]{無}{ な}きは
\ruby[g]{如何}{いかん }とも
すべから
ざること
ながら
\ruby[g]{差當}{さしあた}つて
\ruby{今}{いま}を
\ruby{何}{なに}と
せんと、
%
\ruby[g]{水野}{みづの }は
\ruby{礑}{はた}と
\ruby{行}{ゆ}き
\ruby{詰}{つま}りて、
%
あたかも
\ruby[g]{帆{\換字{船}}}{ほ ぶね}に
\ruby{舵}{かぢ}を
\ruby{失}{うしな}ひ、
%
\ruby[g]{奔車}{ほんしや}に
\ruby{轄}{くさび}を
\ruby{拔}{ぬ}かれたる
ごとく、
%
\ruby{言}{い}はん
かた
\ruby{無}{な}き
\ruby[||j>]{心}{こゝろ}% 踊り字調整「〻(二の字点、揺すり点)に見えるが(ゝ)」
\ruby[||j>]{細}{ ぼそ}さを
\ruby{覺}{おぼ}えて、
%
\ruby[g]{憮然}{ぶ ぜん}として
\ruby{言}{ことば}も
\ruby{無}{な}く
\ruby{物}{もの}を
\ruby{思}{おも}ひたり。

\原本頁{97-4}%
\ruby[g]{書生}{しよせい}は
\ruby[g]{水野}{みづの }の
\ruby[g]{容子}{ようす }を
\ruby{見}{み}て
\ruby{氣}{き}の
\ruby{毒}{どく}さに
\ruby{堪}{た}へでや、

\原本頁{97-5}%
『
\ruby[g]{{\換字{遠}}路}{ゑんろ }の
ところを
\ruby{御來臨}{お|い|で}に
なつたのに
\ruby[g]{生憎}{あひにく}で、
%
\ruby[g]{如何}{い か }にも
\ruby[g]{御氣}{お き }の
\ruby{毒}{どく}で
ございますが、
%
\ruby{必}{かなら}ず
\ruby[g]{小生}{わたくし}は
\ruby[g]{左樣}{さ う }
\ruby{申}{まを}しまして、
%
\ruby[g]{是非}{ぜ ひ }とも
\ruby[<j||]{回}{くわい}% 原本通り「回」
\ruby[<j||]{診}{しん}に% 行末行頭の境界付近なので特例処置を施す
% \ruby{回診}{くわい|しん}に% 原本通り「回」
なるやうに
\ruby{致}{いた}しまする。
%
\ruby[g]{時間}{じ かん}の
ところは
\ruby{兎}{と}に
\ruby{角}{かく}、
%
\ruby{必}{かなら}ず
\ruby{診}{み}てあげますことは
\ruby{診}{み}てあげますやう、
%
これは
\ruby[g]{小生}{わたくし}が
\ruby{御受合}{お|うけ|あひ}
\ruby{申}{まを}して
\ruby[g]{左樣}{さ う }いたしますから。
』

\原本頁{97-10}%
と、
%
\ruby{{\換字{前}}}{さき}の
\ruby{日}{ひ}とは
\ruby{打}{う}つて
\ruby{變}{かは}つて
\ruby[g]{親切}{しんせつ}に
\ruby{言}{い}ひ
\ruby{吳}{く}るゝ、% 踊り字調整「〻(二の字点、揺すり点)に見えるが(ゝ)」
%
その
\ruby[g]{言葉}{ことば }には
\ruby{力}{ちから}あり、
%
その
\ruby[g]{樣子}{やうす }には
\ruby[<j>]{勢}{いきほひ}あるに、
%
\ruby{今}{いま}は
\ruby{此}{こ}の
\ruby{男}{をとこ}を
\ruby{頼}{たの}まんより
\ruby{他}{ほか}の
\原本頁{98-1}\改行%
\ruby{{\換字{道}}}{みち}
なければ、
%
\ruby[g]{水野}{みづの }は
いと
\ruby[g]{懇切}{ねんごろ}に
\ruby{頼}{たの}み
\ruby{聞}{きこ}えて、
%
\ruby[g]{是非}{ぜ ひ }
\ruby{無}{な}くも
\ruby{元}{もと}
\ruby{來}{き}し
\ruby{{\換字{道}}}{みち}へ
\ruby[g]{引{\換字{返}}}{ひつかへ}したり。

\原本頁{98-3}%
\ruby[g]{戀人}{こひゞと}の% 踊り字調整「〻(二の字点、揺すり点)に濁点に見えるが(ゞ)」
\ruby{病}{やまひ}は
\ruby{{\換字{前}}}{さき}の
\ruby{日}{ひ}より
\ruby{凶}{あし}き
かたへ
\ruby{{\換字{進}}}{すゝ}める% 踊り字調整「〻(二の字点、揺すり点)に見えるが(ゝ)」
なり、
%
\ruby{頼}{たの}む
\ruby{醫}{い}は
\ruby{他}{た}に
\ruby{出}{い}でゝ% 踊り字調整「〻(二の字点、揺すり点)に見えるが(ゝ)」
\ruby{家}{いへ}に
あらぬなり、
%
\ruby{夢}{ゆめ}
\ruby{見}{み}は
\ruby{忌}{いま}はしかりし
なり、
%
\ruby{胸}{むね}は
\ruby{騷}{さわ}ぎしなり、
%
\ruby{{\換字{若}}}{もし}やと
\ruby{思}{おも}ひし
ことは
\ruby{不思議}{ふ|し|ぎ}にも
\ruby{中}{あた}りしなり、
%
\ruby{{\換字{弱}}}{よわ}り
かへれる
\原本頁{98-6}\改行%
\ruby{五十子}{い|そ|こ}に
\ruby[g]{一應}{いちおう}の
\ruby[g]{手當}{て あて}して
\ruby{歸}{かへ}れる
\ruby[g]{尾竹}{を だけ}よりは
\ruby{心}{こゝろ}に% 踊り字調整「〻(二の字点、揺すり点)に見えるが(ゝ)」
かゝる% 踊り字調整「〻(二の字点、揺すり点)に見えるが(ゝ)」
\ruby{言}{ことば}を
\ruby{聞}{き}きしなり、
%
\ruby[g]{氣味}{き み }
あしき
\ruby{狗}{いぬ}は
\ruby[g]{{\換字{前}}表}{ぜんぺう}かと
おぼしく
\ruby{吠}{ほ}えに
\ruby{吠}{ほ}えしなり、
%
\原本頁{98-8}\改行%
\ruby[g]{無心}{む しん}の
お
\ruby{濱}{はま}は
\ruby{我}{わ}が
\ruby{五十子}{い|そ|こ}の
\ruby[g]{{\換字{遠}}方}{とほく }へ
\ruby{行}{ゆ}かんことを
\ruby[g]{無心}{む しん}に
\ruby{云}{い}へるなり、
%
\ruby{氣}{き}
にかゝること% 踊り字調整「〻(二の字点、揺すり点)に見えるが(ゝ)」
のみの
\ruby{何}{なん}ぞ
\ruby{多}{おほ}きやと、
%
\ruby[g]{水野}{みづの }は
\ruby[g]{此等}{これら }の
\ruby{事}{こと}を
\ruby{思}{おも}ひ
つゞけつゝ、% 踊り字調整「〻(二の字点、揺すり点)に見えるが(ゝ)」% 踊り字調整「〻(二の字点、揺すり点)に濁点に見えるが(ゞ)」
%
\ruby{恰}{あたか}も% 恰も「あ(た)かも」
\ruby{{\換字{前}}}{さき}の
\ruby{日}{ひ}と
\ruby{同}{おな}じ
\ruby{曉}{あした}の、
%
\ruby[g]{今日}{け ふ }は
\ruby{風}{かぜ}
\ruby{無}{な}くて
\ruby{曇}{くも}り
\ruby{{\換字{空}}}{ぞら}の
\ruby{少}{すこ}し
\ruby{闇}{くら}き
のみが
\ruby{異}{ことな}れる
\ruby{同}{おな}じ
\ruby[g]{時刻}{ころほひ}
\footnote{「刻」のルビは(ほひ)ないし(はひ)と読み取れるが「\ruby{多}{おほ}きや」や
「お\ruby{濱}{はま}」を参考に「\ruby{時刻}{ころ|ほひ}」とした
(国会図書館 コマ番号54/160 p-098 l-15)}%
に、
%
\ruby{同}{おな}じく
\ruby[g]{人{\換字{通}}}{ひとゞほ}り% 踊り字調整「〻(二の字点、揺すり点)に濁点に見えるが(ゞ)」
\ruby{{\換字{猶}}}{なほ}
\ruby[||j>]{少}{すくな}き
\ruby[g]{並木}{なみき }の
\ruby{{\換字{道}}}{みち}を
\ruby{首}{かうべ}を
\ruby{垂}{た}れて
\ruby[||j>]{力}{ちから}
\ruby[||j>]{無}{ な}く
\ruby{行}{ゆ}き
\ruby{盡}{つく}しつ、
%
\ruby{吾妻橋}{あづ|ま|ばし}の% ルビ調整(原本通り)
\ruby{方}{かた}に
\ruby{去}{さ}らんとする
\ruby{時}{とき}、
%
\ruby[g]{突然}{とつぜん}
として
\ruby{人}{ひと}の
\ruby{我}{わ}が
\ruby{手}{て}を
\ruby{執}{と}るありて、
%
しかも
\ruby{執}{と}られし
\ruby{我}{わ}が
\ruby[g]{手首}{て くび}に、
%
ざらりと
\ruby{物}{もの}の
\ruby{觸}{さは}りたれば、
%
\ruby{何}{なん}ぞと
\ruby{驚}{おどろ}きて
\ruby{顧}{かへり}みるに、
%
\原本頁{99-4}\改行%
\ruby{骨}{ほね}
\ruby[||j>]{露}{あらは}に
\ruby{萎}{しな}び
\ruby{枯}{から}びて
\ruby[||j>]{冷}{つめた}き
\ruby{細}{ほそ}き
\ruby{手}{て}に
\ruby{我}{わ}が
\ruby{手}{て}は
\ruby{捉}{とら}へ
\ruby{居}{を}られて、
%
\ruby{其}{そ}の
\ruby[g]{手首}{て くび}に
\ruby{掛}{か}けられ
\ruby{居}{ゐ}たる
\ruby{黑}{くろ}き
\ruby{木}{き}の
\ruby[g]{數珠}{ず ゞ }の% 踊り字調整「〻(二の字点、揺すり点)に濁点に見えるが(ゞ)」
\ruby{我}{わ}が
\ruby{手}{て}に
\ruby{滑}{すべ}りて
\ruby{落}{お}ち
かゝれるなり。% 踊り字調整「〻(二の字点、揺すり点)に見えるが(ゝ)」
