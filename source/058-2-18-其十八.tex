\Entry{其十八}

『あなた!。
いけません、いけません、
\ruby{信}{しん}を
\ruby{御冷}{お|さ}ましなすつては!。
\ruby{此處}{こ|ゝ}を
\ruby{御{\換字{通}}}{お|とほ}りになさりながら、
\ruby{御参詣}{ご|さん|けい}もなさらないんなんて、
\ruby{第一勿體無}{だい|いち|もつ|たい|な}い
\ruby{事}{こと}ではございませんか、さあ、
\ruby{御一緖}{ご|いつ|しよ}に
\ruby{詣}{まゐ}りましやう』

と
\ruby{{\換字{遮}}}{しや}に
\ruby{無}{む}に
\ruby{我}{わ}が
\ruby{手}{て}を
\ruby{牽}{ひ}きに
\ruby{牽}{ひ}くは、
\ruby{{\換字{過}}}{すぎ}し
\ruby{日}{ひ}
\ruby{淺草寺}{せん|さう|じ}の
\ruby{御堂}{み|だう}に
\ruby{普門品}{ふ|もん|ぼん}を
\ruby{誦}{じゆ}して、
\ruby{我}{われ}と
\ruby{共}{とも}に
\ruby{痛}{いた}く
\ruby{書生}{しよ|せい}に
\ruby{罵}{のゝし}られたる、
\ruby{彼}{か}の
\ruby{頭髪薄}{か|み|うす}く
\ruby{額{\換字{脱}}}{ひたひ|ぬ}け
\ruby{上}{あが}がりて
\ruby{鼻細}{はな|ほそ}き
\ruby{貧相}{ひん|さう}の
\ruby{老人}{らう|じん}なり。

\ruby{一樹}{いち|じゆ}の
\ruby{蔭}{かげ}に
\ruby{憩}{いこ}ひ
\ruby{一河}{いち|が}の
\ruby{流}{なが}れを
\ruby{掬}{むす}ぶも
\ruby{他生}{たし|やう}の
\ruby{緣}{えん}といへば、まして
\ruby{一堂}{いち|だう}の
\ruby{内}{うち}に
\ruby{同}{おな}じ
\ruby{御佛}{みほ|とけ}を
\ruby{頼}{たの}み
\ruby{奉}{たてまつ}りて、しかも
\ruby{假初}{かり|そめ}ながら
\ruby{言葉}{こと|ば}をさへ
\ruby{{\換字{交}}}{かは}したる
\ruby{中}{なか}なれば、
\ruby{呼}{よ}びかけられたりとて
\ruby{怪}{け}しうはあらぬながら、
\ruby{手}{て}を
\ruby{執}{と}りて
\ruby{我}{われ}を
\ruby{{\換字{伴}}}{ともな}はんとする
\ruby{擧動}{ふる|まひ}の、
\ruby{馴}{な}れ〳〵しきに
\ruby{{\換字{過}}}{す}ぎたるやうにも
\ruby{思}{おも}はるゝに、
\ruby{水野}{みづ|の}は
\ruby{一度}{ひと|たび}は
\ruby{之}{これ}を
\ruby{異}{あやし}みしが、たゞ〳〵おのが
\ruby{信心}{しん|じん}の
\ruby{同行}{どう|ぎやう}とせんとするほかには、
\ruby{何}{なん}の
\ruby{念}{ねん}も
\ruby{無}{な}かるべき
\ruby{其}{そ}の
\ruby{{\換字{道}}理}{もつ|とも}らしく
\ruby{眞面目}{ま|じ|め}らしき
\ruby{顏}{かほ}の
\ruby{他事無}{た|じ|な}く
\ruby{正直氣}{しやう|ぢき|げ}なる
\ruby{樣子}{やう|す}を
\ruby{見}{み}ては、
\ruby{何}{なん}の
\ruby{故}{ゆゑ}とは
\ruby{無}{な}けれど
\ruby{此}{こ}の
\ruby{老}{お}いたる
\ruby{人}{ひと}の
\ruby{意}{こゝろ}に
\ruby{背}{そむ}かん
\ruby{氣}{き}にはなれずして、
\ruby{引}{ひ}かるゝが
\ruby{儘}{まゝ}に
\ruby{無言}{む|ごん}に
\ruby{従}{したが}ひ
\ruby{行}{ゆ}けり。

『
\ruby{世}{よ}が
\ruby{澆季}{す|ゑ}になつて
\ruby{居}{を}りますのですもの、
\ruby{御同樣}{ご|どう|やう}に
\ruby{鄙}{いや}しい
\ruby{心}{こゝろ}ばかりが
\ruby{先}{さき}に
\ruby{立}{たち}まして、
\ruby{兎角信心}{と|かく|しん|〴〵}の
\ruby{起}{おこ}らないのも
\ruby{是非}{ぜ|ひ}がございませんで、
\ruby{眞}{まこと}に
\ruby{淺}{あさ}ましい
\ruby{口惜}{く|や}しいことでございます。
もう
\ruby{五十六十}{ご|じう|ろく|じう}になりまして、いろ〳〵の
\ruby{經驗}{おぼ|\換字{𛀁}}を
\ruby{積}{つ}んでまゐりました
\ruby{私等}{わたくし|ら}のやうな
\ruby{年齡}{と|し}のものでさへ、
\ruby{何}{なん}ぞにつけても
\ruby{怒}{おこ}つたり
\ruby{泣}{な}いたり
\ruby{致}{いた}しまして、
\ruby{彼奴}{あい|つ}が
\ruby{憎}{にく}いの
\ruby{恨}{うら}めしいのと、
\ruby{詰}{つま}らない
\ruby{修羅}{しゆ|ら}を
\ruby{燃}{も}やしまして、
\ruby{信心氣一方}{しん|〴〵|ぎ|いつ|ぱう}にばかりにはなつて
\ruby{居}{を}られませんのですから、
\ruby{御若}{お|わか}い
\ruby{貴君}{あな|た}
\ruby{方}{がた}ではなか〳〵
\ruby{何樣}{ど|う}いたしまして、
\ruby{幾許}{いく|ら}
\ruby{御發明}{ご|はつ|めい}でいらつしやいましても、
\ruby{何事}{なに|ごと}も
\ruby{佛陀樣}{ほと|け|さま}に
\ruby{御任}{お|まか}せなすつて
\ruby{安心}{あん|しん}して
\ruby{御在}{お|いで}なさるといふ
\ruby{譯}{わけ}にはまいりますまい、
\ruby{御信心}{ご|しん|〴〵}も
\ruby{自然}{し|ぜん}
\ruby{御冷}{お|さめ}になつて、
\ruby{他}{ほか}の
\ruby{方}{はう}へ
\ruby{御{\換字{紛}}}{お|まぎ}れなさるのも
\ruby{御無理}{ご|む|り}はございません!。
\ruby{併}{しか}し
\ruby{貴君}{あな|た}はまあ
\ruby{御頼}{お|たの}もしい
\ruby{方}{かた}で、
\ruby{今}{いま}の
\ruby{御若}{お|わか}い
\ruby{方}{かた}にも
\ruby{御似合}{お|に|あ}ひなさらずに、
\ruby{一心}{いつ|しん}になつて
\ruby{御信心}{ご|しん|〴〵}なすつた
\ruby{{\換字{過}}日}{この|あひだ}の
\ruby{御殊{\換字{勝}}}{ご|しゆ|しよう}さには、つく〴〵
\ruby{私}{わたくし}も
\ruby{感心}{かん|しん}いたしまして、
\ruby{斯樣申}{か|う|まを}しては
\ruby{諛辭}{おせ|じ}のやうでをかしうございますが、
\ruby{宅}{たく}へ
\ruby{歸}{かへ}りましてからも、あ〻
\ruby{未}{ま}だ
\ruby{世}{よ}の
\ruby{中}{なか}は
\ruby{闇}{やみ}にはならない、あ〻いふ
\ruby{若}{わか}い
\ruby{方}{かた}も
\ruby{稀}{まれ}には
\ruby{居}{ゐ}らつしやる!、
\ruby{考}{かんが}へて
\ruby{見}{み}れば
\ruby{自{\換字{分}}}{じ|ぶん}なんぞは
\ruby{罪障}{つ|み}が
\ruby{深}{ふか}くつて
\ruby{昔生}{むかし|うま}れの
\ruby{身}{み}でありながら、
\ruby{何十年}{なん|じう|ねん}といふものを
\ruby{惜}{を}しい
\ruby{欲}{ほ}しいの
\ruby{欲}{よく}ばかりに
\ruby{{\換字{過}}}{すご}して、
\ruby{夢}{ゆめ}のやうにたゞ
\ruby{暮}{くら}した
\ruby{末}{すゑ}、
\ruby{神樣佛樣}{かみ|さま|ほとけ|さま}の
\ruby{有}{あ}り
\ruby{難}{がた}いことを
\ruby{知}{し}つたのも、やつと
\ruby{此}{こ}の
\ruby{四五年}{し|ご|ねん}ばかり
\ruby{以來}{この|かた}の
\ruby{事}{こと}だつたが、
\ruby{御若}{お|わか}いのに
\ruby{彼樣}{あ|ゝ}いふ
\ruby{良}{よ}い
\ruby{方}{かた}もある!。
\ruby{自{\換字{分}}}{じ|ぶん}の
\ruby{彼}{あ}の
\ruby{位}{くらゐ}の
\ruby{齡}{とし}の
\ruby{時}{とき}に
\ruby{比}{くら}べてもよく
\ruby{解}{わか}ること、
\ruby{二十四五}{に|じう|し|ご}や
\ruby{三十{\換字{前}}後}{さん|じう|ぜん|ご}の
\ruby{勢}{いきほひ}では、
\ruby{鬼}{おに}が
\ruby{出}{で}ても
\ruby{攫}{つか}み
\ruby{合}{あ}はうといふ
\ruby{盲元氣}{めくら|げん|き}で、
\ruby{神樣}{かみ|さま}も
\ruby{佛樣}{ほとけ|さま}もありは
\ruby{仕}{し}ないのに、
\ruby{彼}{あ}の
\ruby{方}{かた}は
\ruby{嘘}{うそ}では
\ruby{出}{で}ない
\ruby{{\換字{涙}}}{なみだ}を
\ruby{溢}{こぼ}して、
\ruby{一心}{いつ|しん}になつて
\ruby{祈}{いの}つていらつしやる!。
\ruby{御{\換字{父}}樣}{お|とつ|さま}が
\ruby{御病患}{お|わづ|らひ}でゞもあるか、
\ruby{御母樣}{お|つか|さま}が
\ruby{御惡}{お|わる}いのか、それとも
\ruby{何樣}{ど|う}いふ
\ruby{事}{こと}で
\ruby{思}{おも}ひ
\ruby{餘}{あま}つて、
\ruby{丹精}{たん|せい}を
\ruby{御凝}{お|こ}らしなさるか
\ruby{知}{し}らないけれども、あの
\ruby{御年齡}{お|とし|ば\換字{𛀁}}で
\ruby{既神佛}{もう|かみ|ほとけ}の
\ruby{有難}{あり|がた}い
\ruby{事}{こと}を
\ruby{知}{し}つて
\ruby{居}{ゐ}られるのは、あゝ
\ruby{稀}{めづ}らしい
\ruby{殊{\換字{勝}}}{しゆ|しよう}なかたゞと、
\ruby{眞實}{ほん|と}に
\ruby{貴君}{あな|た}の
\ruby{事}{こと}ばかり
\ruby{思}{おも}つて
\ruby{居}{を}りまして、
\ruby{何}{なん}だか
\ruby{私}{わたくし}は
\ruby{急}{きふ}に
\ruby{一人}{ひと|り}の、
\ruby{私}{わたくし}の
\ruby{味方}{み|かた}が
\ruby{出來}{で|き}たやうな
\ruby{氣}{き}が
\ruby{致}{いた}し、これも
\ruby{觀音樣}{くわん|のん|さま}の
\ruby{御引合}{お|ひき|あは}せ
\ruby{下}{くだ}すつた
\ruby{菩提}{ぼ|だい}の
\ruby{同行}{どう|ぎやう}とでもいふのであらう!、と
\ruby{{\換字{勝}}手}{かつ|て}な
\ruby{考}{かんが}へではございますが
\ruby{思}{おも}ひ
\ruby{詰}{つ}めまして、
\ruby{明{\換字{朝}}御目}{あし|た|お|め}にかゝつたらば、も
\ruby{一度}{いち|ど}
\ruby{御話}{お|はなし}して
\ruby{見}{み}やう、
\ruby{老人}{とし|より}の
\ruby{事}{こと}ゆゑ
\ruby{御{\換字{嫌}}}{お|きら}ひなさるか
\ruby{知}{し}れないが、どうも
\ruby{御話}{お|はなし}を
\ruby{仕}{し}て
\ruby{見}{み}たらば、
\ruby{屹度}{きつ|と}
\ruby{私}{わたくし}の
\ruby{力}{ちから}になつて
\ruby{下}{くだ}さる
\ruby{俠氣}{をと|こぎ}の
\ruby{方}{かた}だらう、といふやうな
\ruby{心持}{こゝろ|もち}が
\ruby{仕}{し}てなりませんでした。
ところが
\ruby{明{\換字{朝}}參}{あし|た|まゐ}つて
\ruby{見}{み}ると
\ruby{御參詣}{お|い|で}はありません、その
\ruby{次}{つぎ}の
\ruby{日}{ひ}も
\ruby{御參詣}{お|まゐ|り}がありません。
ぽろり〳〵と
\ruby{{\換字{涙}}}{なみだ}を
\ruby{落}{おと}として
\ruby{眞}{しん}になつて
\ruby{何事}{なに|ごと}かを
\ruby{願}{ねが}つて
\ruby{居}{ゐ}られた
\ruby{彼}{あ}の
\ruby{方}{かた}が、
\ruby{不信心}{ぶ|しん|じん}になられる
\ruby{理由}{わ|け}は
\ruby{無}{な}いが、あ〻
\ruby{何}{なん}といつても
\ruby{未}{ま}だ
\ruby{御若}{お|わか}い!、
\ruby{下}{くだ}らない
\ruby{惡魔外{\換字{道}}}{あく|ま|げ|だう}の
\ruby{馬鹿書生}{ば|か|しよ|せい}が、
\ruby{愚}{ぐ}につかない
\ruby{事}{こと}を
\ruby{饒舌}{しや|べ}つて
\ruby{居}{ゐ}たが、
\ruby{若}{もし}や
\ruby{彼言}{あ|れ}が
\ruby{毒}{どく}になりは
\ruby{仕}{し}ないか
\ruby{按}{あん}じられる、
\ruby{何}{なん}といつても
\ruby{未}{ま}だ
\ruby{御若}{お|わか}いから!、と
\ruby{大}{おほ}きに
\ruby{彼}{あ}の
\ruby{書生等}{しよ|せい|たち}を
\ruby{憎}{にく}くおもつて
\ruby{居}{を}りました。
』
