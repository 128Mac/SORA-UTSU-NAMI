\Entry{其十八}

% メモ 校正終了 2024-04-21 2024-05-31 2024-07-01
\原本頁{99-8}%
『
あなた!。
%
いけません、
%
いけません、
%
\ruby{信}{しん}を
\ruby[g]{御冷}{お さ }まし
なすつては!。
%
\ruby[g]{此處}{こ ゝ }を% 踊り字調整「〻(二の字点、揺すり点)に見えるが(ゝ)」
\ruby[g]{御{\換字{通}}}{お とほ}り
なさり
ながら、
%
\ruby{御參詣}{ご|さん|けい}も
なさらない
なんて
\改行% 校正作業の簡略化のため
、
%
\原本頁{99-10}\改行%
\ruby[g]{第一}{だいいち}
\ruby{勿體無}{もつ|たい|な}い
\ruby{事}{こと}では
ございませんか、
%
さあ
%
\ruby{御一緖}{ご|いつ|しよ}に
\ruby{詣}{まゐ}り
ましや
\改行% 校正作業の簡略化のため
う!。
』

\原本頁{100-2}%
と
\ruby{{\換字{遮}}}{しや}に
\ruby{無}{む}に
\ruby{我}{わ}が
\ruby{手}{て}を
\ruby{牽}{ひ}きに
\ruby{牽}{ひ}くは、
%
\ruby{{\換字{過}}}{すぎ}し
\ruby{日}{ひ}
\ruby{淺草寺}{せん|さう|じ}の
\ruby[g]{御堂}{み だう}に
\ruby{普門品}{ふ|もん|ぼん}を
\ruby{誦}{じゆ}して、
%
\ruby{我}{われ}と
\ruby{共}{とも}に
\ruby{痛}{いた}く
\ruby[g]{書生}{しよせい}に
\ruby{罵}{のゝし}られたる、% 踊り字調整「〻(二の字点、揺すり点)に見えるが(ゝ)」
%
\ruby{彼}{か}の
\ruby[g]{頭髮}{か み }
\ruby{薄}{うす}く
\ruby[<j||]{額}{ひたひ}% 行末行頭の境界付近なので特例処置を施す
\ruby[||j>]{脫}{ぬ}け
\ruby{上}{あが}りて
\ruby{鼻}{はな}
\ruby{細}{ほそ}き
\ruby[g]{{\換字{貧}}相}{ひんさう}の
\ruby[g]{老人}{らうじん}なり。

\原本頁{100-5}%
\ruby[g]{一樹}{いちじゆ}の
\ruby{蔭}{かげ}に
\ruby{憩}{いこ}ひ
\ruby[g]{一河}{いちが }の
\ruby{流}{なが}れを
\ruby{掬}{むす}ぶも
\ruby[g]{他生}{たしやう}の
\ruby{緣}{{\換字{𛀁}}ん}と
いへば、
%
まして
\ruby[g]{一堂}{いちだう}の
\ruby{内}{うち}に
\ruby{同}{おな}じ
\ruby[g]{御佛}{みほとけ}を
\ruby{頼}{たの}み
\ruby[<j>]{奉}{たてまつ}りて、
%
しかも
\ruby[g]{假初}{かりそめ}
ながら
\ruby[g]{言葉}{ことば }を
さへ
\ruby{{\換字{交}}}{かは}したる
\ruby{中}{なか}なれば、
%
\ruby{呼}{よ}びかけられたりとて
\ruby{怪}{け}しうは
あらぬながら、
%
\ruby{手}{て}を
\ruby{執}{と}りて
\ruby{我}{われ}を
\ruby{{\換字{伴}}}{ともな}はん
とする
\ruby[g]{擧動}{ふるまひ}の、
%
\ruby[g]{馴々}{なれ〳〵}しきに
\ruby{{\換字{過}}}{す}ぎたる
やうにも
\ruby{思}{おも}はるゝに、% 踊り字調整「〻(二の字点、揺すり点)に見えるが(ゝ)」
%
\ruby[g]{水野}{みづの }は
\ruby[g]{一度}{ひとたび}は
\ruby{之}{これ}を
\ruby{異}{あやし}みしが、
%
たゞ〳〵% 踊り字調整「〻(二の字点、揺すり点)に濁点に見えるが(ゞ)」
\原本頁{100-10}\改行%
おのが
\ruby[g]{信心}{しん〴〵}の
\ruby[||j>]{同}{どう}
\ruby[||j>]{行}{ぎやう}と
% \ruby{同行}{どう|ぎやう}と
せんとする
ほかには、
%
\ruby{何}{なん}の
\ruby{念}{ねん}も
\ruby{無}{な}かるべき
%
\ruby{其}{そ}の
\ruby[g]{{\換字{道}}理}{もつとも}らしく
\ruby{眞面目}{ま|じ|め}らしき
\ruby{顏}{かほ}の
\ruby[g]{他事}{た じ }
\ruby{無}{な}く
\ruby{正直氣}{しやう|ぢき|げ}なる
\ruby[g]{樣子}{やうす }を
\ruby{見}{み}ては、
%
\ruby{何}{なん}の
\ruby{故}{ゆゑ}とは
\ruby{無}{な}けれど
\ruby{此}{こ}の
\ruby{老}{お}いたる
\ruby{人}{ひと}の
\ruby{意}{こゝろ}に% 踊り字調整「〻(二の字点、揺すり点)に見えるが(ゝ)」
\ruby{背}{そむ}かん
\ruby{氣}{き}には
なれずして、
%
\ruby{引}{ひ}かるゝが% 踊り字調整「〻(二の字点、揺すり点)に見えるが(ゝ)」
\ruby{儘}{まゝ}に% 踊り字調整「〻(二の字点、揺すり点)に見えるが(ゝ)」
\ruby[g]{無言}{む ごん}に
\ruby{從}{したが}ひ
\ruby{行}{ゆ}けり。

\原本頁{101-3}%
『
\ruby{世}{よ}が
\ruby[g]{澆季}{す ゑ }になつて
\ruby{居}{を}ります
のですもの、
%
\ruby{御同樣}{ご|どう|やう}に
\ruby{鄙}{いや}しい
\ruby{心}{こゝろ}% 踊り字調整「〻(二の字点、揺すり点)に見えるが(ゝ)」
ばかりが
\ruby{先}{さき}に
\ruby{立}{たち}まして、
%
\ruby[g]{兎角}{と かく}
\ruby[g]{信心}{しん〴〵}の
\ruby{起}{おこ}らないのも
\ruby[g]{是非}{ぜ ひ }が
ございませんで、
%
\ruby[g]{眞に}{まこと }
\ruby{淺}{あさ}ましい
\ruby[g]{口惜}{く や }しいことで
ございます!。
%
もう
\ruby{五十六十}{ご|じふ|ろく|じふ}になりまして、% 「いそじ、むそじ」とも読んでいいが
%
\makeatletter
\@ifundefined{デバッグ@ビルド}{%
  \ruby[g]{いろ〳〵}{}の
}{%
  いろ〳〵の
}%
\makeatother
\ruby[g]{經驗}{おぼ{\換字{𛀁}}}を
\ruby{積}{つ}んで
まゐり
ました
\makeatletter
\@ifundefined{デバッグ@ビルド}{%
  \ruby[<g|]{私等の}{わたくしら }
}{%
  \ruby[<j>]{私}{わたくし}% 行末行頭の境界付近なので特例処置を施す
  \ruby{等}{ ら}の
}%
\makeatother
やうな
\ruby[g]{年齡}{と し }のもの
で
さへ、
%
\ruby{何}{なん}ぞに
つけても
\ruby{怒}{おこ}つたり
\ruby{泣}{な}いたり
\原本頁{101-8}\改行%
\ruby{致}{いた}しまして、
%
\ruby[g]{彼奴}{あいつ }が
\ruby{憎}{にく}いの
\ruby{恨}{うら}めしいのと、
%
\ruby{詰}{つま}らない
\ruby[g]{修羅}{しゆら }を
\ruby{燃}{も}やし
まして、
%
\ruby[g]{信心}{しん〴〵}
\ruby{氣}{ぎ}
\ruby[g]{一方}{いつぱう}に
ばかりは
なつて
\ruby{居}{を}られません
のですから、
%
\ruby[g]{御{\換字{若}}}{お わか}い
\ruby[g]{貴君}{あなた }
\ruby{方}{がた}では
なか〳〵
\ruby[g]{何樣}{ど う }
いたしまして、
%
\ruby[g]{幾許}{いくら }
\ruby{御發明}{ご|はつ|めい}で
いらつしやい
ましても、
%
\ruby[g]{何事}{なにごと}も
\ruby{佛陀樣}{ほと|け|さま}に
\ruby[g]{御任}{お まか}せなすつて
\ruby[g]{安心}{あんしん}
\原本頁{102-1}\改行%
して
\ruby[g]{御在}{お いで}なさる
といふ
\ruby{譯}{わけ}には
\ruby{參}{まゐ}り
ますまい、
%
\ruby{御信心}{ご|しん|〴〵}も
\ruby[g]{自然}{し ぜん}
\ruby[g]{御冷}{お さめ}
\原本頁{102-2}\改行%
になつて、
%
\ruby{他}{ほか}の
\ruby{方}{はう}へ
\ruby[g]{御{\換字{紛}}}{お まぎ}れ
なさるのも
\ruby{御無理}{ご|む|り}は
ございません!
\改行% 校正作業の簡略化のため
。
%
\原本頁{102-3}\改行%
\ruby{併}{しか}し
\ruby[g]{貴君}{あなた }は
まあ
\ruby[g]{御頼}{お たの}もしい
\ruby{方}{かた}で、
%
\ruby{今}{いま}の
\ruby[g]{御{\換字{若}}}{お わか}い
\ruby{方}{かた}にも
\ruby{御似合}{お|に|あ}ひ
なさらずに、
%
\ruby[g]{一心}{いつしん}に
なつて
\ruby{御信心}{ご|しん|〴〵}
なすつた
\ruby[||j>]{{\換字{過}}}{この}
\ruby[||j>]{日}{あひだ}の
% \ruby{{\換字{過}}日}{この|あひだ}の
\ruby{御}{ご}
\ruby[||j>]{殊}{しゆ}
\ruby[||j>]{{\換字{勝}}}{しよう}さには、
%
\原本頁{102-5}\改行%
\ruby[g]{つく〴〵}{}
\ruby[<g>]{私も}{わたくし }% ダミールビ文字を一つ追加し若干の空白を確保
\ruby[g]{{\換字{感}}心}{かんしん}
いたしまして、
%
\ruby[g]{斯樣}{か う }
\ruby{申}{まを}しては
\ruby[g]{諛辭}{おせじ }
のやうで
をかしう
ございますが、
%
\ruby{宅}{たく}へ
\ruby{歸}{かへ}り
まして
からも、
%
あゝ% 踊り字調整「〻(二の字点、揺すり点)に見えるが(ゝ)」
\ruby{未}{ま}だ
\ruby{世}{よ}の
\ruby{中}{なか}は
\ruby{闇}{やみ}には
ならない、
%
あゝ% 踊り字調整「〻(二の字点、揺すり点)に見えるが(ゝ)」
いふ
\ruby{{\換字{若}}}{わか}い
\ruby{方}{かた}も
\ruby{稀}{たま}には
\ruby{居}{ゐ}らつしやる!、
%
\原本頁{102-8}\改行%
\ruby[||j>]{考}{かんが}へて
\ruby{見}{み}れば
\ruby[g]{自{\換字{分}}}{じ ぶん}
なんぞは
\ruby[g]{罪障}{つ み }が
\ruby{深}{ふか}くつて、
\ruby[||j>]{昔}{むかし}
\ruby[||j>]{生}{ うま}れの
\ruby{身}{み}で
あり
\原本頁{102-9}\改行%
ながら、
%
\ruby{何十年}{なん|じふ|ねん}
といふものを
\ruby{惜}{を}しい
\ruby{欲}{ほ}しいの
\ruby{欲}{よく}
ばかりに
\ruby{{\換字{過}}}{すご}して
\改行% 校正作業の簡略化のため
、
%
\原本頁{102-10}\改行%
\ruby{夢}{ゆめ}の
やうに
たゞ% 踊り字調整「〻(二の字点、揺すり点)に濁点に見えるが(ゞ)」
\ruby{暮}{くら}した
\ruby{末}{すゑ}、
%
\ruby[g]{神樣}{かみさま}
\ruby[||j>]{佛}{ほとけ}
\ruby[||j>]{樣}{ さま}の
% \ruby{佛樣}{ほとけ|さま}の
\ruby{有}{あ}り
\ruby{{\換字{難}}}{がた}いことを
\ruby{知}{し}つたのも、
%
やつと
\ruby{此}{こ}の
\ruby{四五年}{し|ご|ねん}
ばかり
\ruby[g]{以來}{このかた}の
\ruby{事}{こと}だつたが、
%
\ruby[g]{御{\換字{若}}}{お わか}いのに
\ruby[g]{彼樣}{あ あ }いふ% ルビ調整(原本通り)非踊り字表記(行末行頭の境界付近)
\ruby{良}{い}い
\ruby{方}{かた}もある!。
%
\ruby[g]{自{\換字{分}}}{じ ぶん}の
\ruby{彼}{あ}の
\ruby{位}{くらゐ}の
\ruby{齡}{とし}の
\ruby{時}{とき}に
\ruby{比}{くら}べても
よく
\ruby{解}{わか}る
こと、
%
\ruby{二十四五}{に|じふ|し|ご}や
\ruby{三十{\換字{前}}後}{さん|じふ|ぜん|ご}の
\ruby[<j>]{勢}{いきほひ}では、
%
\ruby{鬼}{おに}が
\ruby{出}{で}ても
\ruby{攫}{つか}み
\ruby{合}{あ}は
\改行% 校正作業の簡略化のため
う
といふ
\ruby{盲元氣}{めくら|げん|き}で、
%
\ruby[g]{神樣}{かみさま}も
\ruby[||j>]{佛}{ほとけ}
\ruby[||j>]{樣}{ さま}も
% \ruby{佛樣}{ほとけ|さま}も
ありは
\ruby{仕}{し}ない
のに、
%
\ruby{彼}{あ}の
\ruby{方}{かた}は
\ruby{嘘}{うそ}では
\ruby{出}{で}ない
\ruby{涙}{なみだ}を
\ruby{溢}{こぼ}して、
%
\ruby[g]{一心}{いつしん}に
なつて
\ruby{祈}{いの}つて
いらつしやる!
\改行% 校正作業の簡略化のため
。
%
\原本頁{103-5}\改行%
\ruby{御{\換字{父}}樣}{お|とつ|さま}が
\ruby{御病患}{お|わづ|らひ}でゞもあるか、% 踊り字調整「〻(二の字点、揺すり点)に濁点に見えるが(ゞ)」
%
\ruby{御母樣}{お|つか|さま}が
\ruby[g]{御惡}{お わる}いのか、
%
それとも
\ruby[g]{何樣}{ど う }いふ
\ruby{事}{こと}で
\ruby{思}{おも}ひ
\ruby{餘}{あま}つて、
%
\ruby[g]{丹精}{たんせい}を
\ruby[g]{御凝}{お こ }らし
なさるか
\ruby{知}{し}らない
けども、
%
あの
\ruby{御年齡}{お|とし|ば{\換字{𛀁}}}で
\ruby{既}{もう}
\ruby[||j>]{神}{かみ}
\ruby[||j>]{佛}{ほとけ}の
% \ruby{神佛}{かみ|ほとけ}の
\ruby[g]{有{\換字{難}}}{ありがた}い
\ruby{事}{こと}を
\ruby{知}{し}つて
\ruby{居}{ゐ}られるのは、
%
% \原本頁{103-8}\改行%
あゝ% 踊り字調整「〻(二の字点、揺すり点)に見えるが(ゝ)」
\ruby{稀}{めづ}らしい
\ruby[||j>]{殊}{しゆ}
\ruby[||j>]{{\換字{勝}}}{しよう}な
% \ruby{殊{\換字{勝}}}{しゆ|しよう}な
かたゞ% 踊り字調整「〻(二の字点、揺すり点)に濁点に見えるが(ゞ)」
と、
%
\ruby[g]{眞實}{ほんと }に
\ruby[g]{貴君}{あなた }の
\ruby{事}{こと}ばかり
\ruby{思}{おも}つて
\ruby{居}{を}りまして、
%
\ruby{何}{なん}だか
\ruby[<j>]{私}{わたくし}は
\ruby{急}{きふ}に
\ruby[g]{一人}{ひとり }の、
%
\makeatletter
\@ifundefined{デバッグ@ビルド}{%
  \ruby[|g|]{私の}{わたくし}
  \ruby[g]{味方}{み かた}が
}{%
  \ruby[<j>]{私}{わたくし}の
  \ruby[g]{味方}{み かた}が
}%
\makeatother
\ruby[g]{出來}{で き }た
やうな
\ruby{氣}{き}が
\ruby{致}{いた}し、
%
これも
\ruby[||j>]{觀}{くわん}% 「觀音」の読みは原本通り「くわん(の)ん」
\ruby[||j>]{音}{ のん}
\ruby[||j>]{樣}{ さま}の
\ruby{御引合}{お|ひき|あは}せ
\ruby{下}{くだ}すつた
\ruby[g]{菩提}{ぼ だい}の
\ruby[||j>]{同}{どう}
\ruby[||j>]{行}{ぎやう}
% \ruby{同行}{どう|ぎやう}
とでも
いふので
あらう!、
%
と
\ruby[g]{{\換字{勝}}手}{かつて }な
\ruby{考}{かんが}へでは
ございますが
\ruby{思}{おも}ひ
\ruby{詰}{つ}めまして、
%
\ruby[g]{明{\換字{朝}}}{あした }
\ruby[g]{御目}{お め }に
かゝつたらば、% 踊り字調整「〻(二の字点、揺すり点)に見えるが(ゝ)」
%
も
\ruby[g]{一度}{いちど }
\ruby[g]{御話}{おはなし}して
\ruby{見}{み}やう、
%
\ruby[g]{老人}{としより}の
\ruby{事}{こと}ゆゑ
\ruby[g]{御{\換字{嫌}}}{お きら}ひ
なさるか
\ruby{知}{し}れないが、
%
どうも
\ruby[g]{御話}{おはなし}を
\ruby{仕}{し}て
\ruby{見}{み}たらば
\改行% 校正作業の簡略化のため
、
%
\原本頁{104-3}\改行%
\ruby[g]{屹度}{きつと }
\ruby[<j>]{私}{わたくし}の
\ruby{力}{ちから}になつて
\ruby{下}{くだ}さる
\ruby[g]{俠氣}{をとこぎ}の
\ruby{方}{かた}だらう、
%
と
いふやうな
\ruby[<j||]{心}{こゝろ}% 踊り字調整「〻(二の字点、揺すり点)に見えるが(ゝ)」
\ruby[<j||]{持}{もち}% 行末行頭の境界付近なので特例処置を施す
\原本頁{104-4}\改行%
が
\ruby{仕}{し}てなりませんでした。
%
ところが
\ruby[g]{明{\換字{朝}}}{あした }
\ruby{參}{まゐ}つて
\ruby{見}{み}ると
\ruby{御參詣}{お|い|で}は
\原本頁{104-5}\改行%
ありません、
%
その
\ruby{次}{つぎ}の
\ruby{日}{ひ}も
\ruby{御參詣}{お|まゐ|り}が
ありません。
%
ぽろり〳〵と
\原本頁{104-6}\改行%
\ruby[||j>]{涙}{なみだ}を
\ruby{落}{おと}して
\ruby{眞}{しん}になつて
\ruby[g]{何事}{なにごと}かを
\ruby{願}{ねが}つて
\ruby{居}{ゐ}られた
\ruby{彼}{あ}の
\ruby{方}{かた}が、
%
\ruby{不信心}{ぶ|しん|じん}に% ルビ調整(原本通り)非踊り字表記(行末行頭の境界付近)
なられる
\ruby[g]{理由}{わ け }は
\ruby{無}{な}いが、
%
あゝ% 踊り字調整「〻(二の字点、揺すり点)に見えるが(ゝ)」
\ruby{何}{なん}と
いつても
\ruby{未}{ま}だ
\ruby[g]{御{\換字{若}}}{お わか}い!、
%
\原本頁{104-8}\改行%
\ruby{下}{くだ}らない
\ruby[g]{惡{\換字{魔}}}{あくま }
\ruby[g]{外{\換字{道}}}{げ だう}の
\ruby[g]{馬鹿}{ば か }
\ruby[g]{書生}{しよせい}が、
%
\ruby{愚}{ぐ}に
つかない
\ruby{事}{こと}を
\ruby[g]{饒舌}{しやべ }つて
\ruby{居}{ゐ}たが、
%
\ruby{{\換字{若}}}{もし}や
\ruby[g]{彼言}{あ れ }が
\ruby{毒}{どく}に
なりは
\ruby{仕}{し}ないか
\ruby{按}{あん}じられる、
%
\ruby{何}{なん}と
いつても
\ruby{未}{ま}だ
\ruby[g]{御{\換字{若}}}{お わか}いから!、
%
と
\ruby{大}{おほ}きに
\ruby{彼}{あ}の
\ruby{書生等}{しよ|せい|たち}を
\ruby{憎}{にく}く
おもつて
\ruby{居}{を}りました。
』
