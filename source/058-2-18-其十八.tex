\Entry{其十八}

% メモ 校正終了 2024-04-21
\原本頁{99-8}%
『
あなた!。
%
いけません、
%
いけません、
%
\ruby{信}{しん}を
\ruby{御冷}{お|さ}まし
なすつては!。
%
\ruby{此處}{こ|〻}を% 原本通り「〻(二の字点、揺すり点)」
\ruby{御{\換字{通}}}{お|とほ}り
なさり
ながら、
%
\ruby{御參詣}{ご|さん|けい}も
なさらない
なんて、
%
%\原本頁{99-10}\改行%
\ruby{第一}{だい|いち}
\ruby{勿體無}{もつ|たい|な}い
\ruby{事}{こと}では
ございませんか、
%
さあ
%
\ruby{御一緖}{ご|いつ|しよ}に
\ruby{詣}{まゐ}り
ましやう!。
』

\原本頁{100-2}%
と
\ruby{{\換字{遮}}}{しや}に
\ruby{無}{む}に
\ruby{我}{わ}が
\ruby{手}{て}を
\ruby{牽}{ひ}きに
\ruby{牽}{ひ}くは、
%
\ruby{{\換字{過}}}{すぎ}し
\ruby{日}{ひ}
\ruby{淺草寺}{せん|さう|じ}の
\ruby{御堂}{み|だう}に
\ruby{普門品}{ふ|もん|ぼん}を
\ruby{誦}{じゆ}して、
%
\ruby{我}{われ}と
\ruby{共}{とも}に
\ruby{痛}{いた}く
\ruby{書生}{しよ|せい}に
\ruby{罵}{の〻し}られたる、% 原本通り「〻(二の字点、揺すり点)」
%
\ruby{彼}{か}の
\ruby{頭髮}{か|み}
\ruby{薄}{うす}く
\ruby{額}{ひたひ}
\ruby{脫}{ぬ}け
\ruby{上}{あが}りて
\ruby{鼻}{はな}
\ruby{細}{ほそ}き
\ruby{{\換字{貧}}相}{ひん|さう}の
\ruby{老人}{らう|じん}なり。

\原本頁{100-5}%
\ruby{一樹}{いち|じゆ}の
\ruby{蔭}{かげ}に
\ruby{憩}{いこ}ひ
\ruby{一河}{いち|が}の
\ruby{流}{なが}れを
\ruby{掬}{むす}ぶも
\ruby{他生}{たし|やう}の
\ruby{緣}{{\換字{𛀁}}ん}と
いへば、
%
まして
\ruby{一堂}{いち|だう}の
\ruby{内}{うち}に
\ruby{同}{おな}じ
\ruby{御佛}{み|ほとけ}を
\ruby{頼}{たの}み
\ruby{奉}{たてまつ}りて、
%
しかも
\ruby{假初}{かり|そめ}
ながら
\ruby{言葉}{こと|ば}を
さへ
\ruby{{\換字{交}}}{かは}したる
\ruby{中}{なか}なれば、
%
\ruby{呼}{よ}びかけられたりとて
\ruby{怪}{け}しうは
あらぬながら、
%
\ruby{手}{て}を
\ruby{執}{と}りて
\ruby{我}{われ}を
\ruby{{\換字{伴}}}{ともな}はん
とする
\ruby{擧動}{ふる|まひ}の、
%
\ruby{馴々}{なれ|〳〵}しきに
\ruby{{\換字{過}}}{す}ぎたる
やうにも
\ruby{思}{おも}はる〻に、% 原本通り「〻(二の字点、揺すり点)」
%
\ruby[g]{水野}{みづの}は
\ruby{一度}{ひと|たび}は
\ruby{之}{これ}を
\ruby{異}{あやし}みしが、
%
たゞ〳〵% TODO 原本の「二の字点、揺すり点」に濁点のグリフが見つからないので「ゞ」
\原本頁{100-10}\改行%
おのか
\ruby{信心}{しん|〴〵}の
\ruby[||g>]{同行}{どうぎやう}と
せんとする
ほかには、
%
\ruby{何}{なん}の
\ruby{念}{ねん}も
\ruby{無}{な}かるべき
%
\ruby{其}{そ}の
\ruby{{\換字{道}}理}{もつ|とも}らしく
\ruby{眞面目}{ま|じ|め}らしき
\ruby{顏}{かほ}の
\ruby{他事}{た|じ}
\ruby{無}{な}く
\ruby{正直氣}{しやう|ぢき|げ}なる
\ruby{樣子}{やう|す}を
\ruby{見}{み}ては、
%
\ruby{何}{なん}の
\ruby{故}{ゆゑ}とは
\ruby{無}{な}けれど
\ruby{此}{こ}の
\ruby{老}{お}いたる
\ruby{人}{ひと}の
\ruby{意}{こ〻ろ}に% 原本通り「〻(二の字点、揺すり点)」
\ruby{背}{そむ}かん
\ruby{氣}{き}には
なれずして、
%
\ruby{引}{ひ}かる〻が% 原本通り「〻(二の字点、揺すり点)」
\ruby{儘}{ま〻}に% 原本通り「〻(二の字点、揺すり点)」
\ruby{無言}{む|ごん}に
\ruby{從}{したが}ひ
\ruby{行}{ゆ}けり。

\原本頁{101-3}%
『
\ruby{世}{よ}が
\ruby{澆季}{す|ゑ}になつて
\ruby{居}{を}ります
のですもの、
%
\ruby{御同樣}{ご|どう|やう}に
\ruby{鄙}{いや}しい
\ruby{心}{こ〻ろ}% 原本通り「〻(二の字点、揺すり点)」
ばかりが
\ruby{先}{さき}に
\ruby{立}{たち}まして、
%
\ruby{兎角}{と|かく}
\ruby{信心}{しん|〴〵}の
\ruby{起}{おこ}らないのも
\ruby{是非}{ぜ|ひ}が
ございませんで、
%
\ruby{眞}{まこと}に
\ruby{淺}{あさ}ましい
\ruby{口惜}{く|や}しいことで
ございます!。
%
もう
\ruby{五十六十}{ご|じう|ろく|じう}になりまして、% 「いそじ、むそじ」とも読んでいいが
%
いろ〳〵の
\ruby{經驗}{おぼ|{\換字{𛀁}}}を
\ruby{積}{つ}んで
まゐり
ました
\ruby{私等}{わたくし|ら}の
やうな
\ruby{年齡}{と|し}のもの
で
さへ、
%
\ruby{何}{なん}ぞに
つけても
\ruby{怒}{おこ}つたり
\ruby{泣}{な}いたり
\原本頁{101-8}\改行%
\ruby{致}{いた}しまして、
%
\ruby{彼奴}{あい|つ}が
\ruby{憎}{にく}いの
\ruby{恨}{うら}めしいのと、
%
\ruby{詰}{つま}らない
\ruby{修羅}{しゆ|ら}を
\ruby{燃}{も}やし
まして、
%
\ruby{信心}{しん|〴〵}
\ruby{氣}{ぎ}
\ruby{一方}{いつ|ぱう}に
ばかりは
なつて
\ruby{居}{を}られません
のですから、
%
\ruby{御{\換字{若}}}{お|わか}い
\ruby{貴君}{あな|た}
\ruby{方}{がた}では
なか〳〵
\ruby{何樣}{ど|う}
いたしまして、
%
\ruby{幾許}{いく|ら}
\ruby{御發明}{ご|はつ|めい}で
いらつしやい
ましても、
%
\ruby{何事}{なに|ごと}も
\ruby{佛陀樣}{ほと|け|さま}に
\ruby{御任}{お|まか}せなすつて
\ruby{安心}{あん|しん}して
\ruby{御在}{お|いで}なさる
といふ
\ruby{譯}{わけ}には
\ruby{參}{まゐ}り
ますまい、
%WADA
\ruby{御信心}{ご|しん|〴〵}も
\ruby{自然}{し|ぜん}
\ruby{御冷}{お|さめ}になつて、
%
\ruby{他}{ほか}の
\ruby{方}{はう}へ
\ruby{御{\換字{紛}}}{お|まぎ}れ
なさるのも
\ruby{御無理}{ご|む|り}は
ございません!。
%
\原本頁{102-3}\改行%
\ruby{併}{しか}し
\ruby{貴君}{あな|た}は
まあ
\ruby{御頼}{お|たの}もしい
\ruby{方}{かた}で、
%
\ruby{今}{いま}の
\ruby{御{\換字{若}}}{お|わか}い
\ruby{方}{かた}にも
\ruby{御似合}{お|に|あ}ひ
なさらずに、
%
\ruby{一心}{いつ|しん}に
なつて
\ruby{御信心}{ご|しん|〴〵}
なすつた
\ruby[||j>]{{\換字{過}}日}{この|あひだ}の
\ruby{御殊{\換字{勝}}}{ご|しゆ|しよう}さには、
%
\原本頁{102-5}\改行%
つく〴〵
\ruby{私}{わたくし}も
\ruby{感心}{かん|しん}
いたしまして、
%
\ruby{斯樣}{か|う}
\ruby{申}{まを}しては
\ruby{諛辭}{おせ|じ}
のやうで
をかしう
ございますが、
%
\ruby{宅}{たく}へ
\ruby{歸}{かへ}り
まして
からも、
%
あ〻% 原本通り「〻(二の字点、揺すり点)」
\ruby{未}{ま}だ
\ruby{世}{よ}の
\ruby{中}{なか}は
\ruby{闇}{やみ}には
ならない、
%
あ〻% 原本通り「〻(二の字点、揺すり点)」
いふ
\ruby{{\換字{若}}}{わか}い
\ruby{方}{かた}も
\ruby{稀}{たま}には
\ruby{居}{ゐ}らつしやる!、
%
\原本頁{102-8}\改行%
\ruby[||j>]{考}{かんが}へて
\ruby{見}{み}れば
\ruby{自{\換字{分}}}{じ|ぶん}
なんぞは
\ruby{罪障}{つ|み}が
\ruby{深}{ふか}くつて、
\ruby[<j||]{昔}{むかし}
\ruby{生}{うま}れの
\ruby{身}{み}で
ありながら、
%
\ruby{何十年}{なん|じう|ねん}
といふものを
\ruby{惜}{を}しい
\ruby{欲}{ほ}しいの
\ruby{欲}{よく}
ばかりに
\ruby{{\換字{過}}}{すご}して、
%
\原本頁{102-10}\改行%
\ruby{夢}{ゆめ}の
やうに
たゞ% TODO 原本の「二の字点、揺すり点」に濁点のグリフが見つからないので「ゞ」
\ruby{暮}{くら}した
\ruby{末}{すゑ}、
%
\ruby{神樣佛樣}{かみ|さま|ほとけ|さま}の
\ruby{有}{あ}り
\ruby{難}{がた}いことを
\ruby{知}{し}つたのも、
%
やつと
\ruby{此}{こ}の
\ruby{四五年}{し|ご|ねん}
ばかり
\ruby{以來}{この|かた}の
\ruby{事}{こと}だつたが、
%
\ruby{御{\換字{若}}}{お|わか}いのに
\ruby{彼樣}{あ|あ}いふ% 原本は行末行頭禁則で非踊り字表記
\ruby{良}{い}い
\ruby{方}{かた}もある!。
%
\ruby{自{\換字{分}}}{じ|ぶん}の
\ruby{彼}{あ}の
\ruby{位}{くらゐ}の
\ruby{齡}{とし}の
\ruby{時}{とき}に
\ruby{比}{くら}べても
よく
\原本頁{103-2}\改行%
\ruby{解}{わか}る
こと、
%
\ruby{二十四五}{に|じう|し|ご}や
\ruby{三十{\換字{前}}後}{さん|じう|ぜん|ご}の
\ruby{勢}{いきほひ}では、
%
\ruby{鬼}{おに}が
\ruby{出}{で}ても
\ruby{攫}{つか}み
\ruby{合}{あ}はう
といふ
\ruby{盲元氣}{めくら|げん|き}で、
%
\ruby{神樣}{かみ|さま}も
\ruby{佛樣}{ほとけ|さま}も
ありは
\ruby{仕}{し}ない
のに、
%
\ruby{彼}{あ}の
\ruby{方}{かた}は
\ruby{嘘}{うそ}では
\ruby{出}{で}ない
\ruby{涙}{なみだ}を
\ruby{溢}{こぼ}して、
%
\ruby{一心}{いつ|しん}に
なつて
\ruby{祈}{いの}つて
いらつしやる!。
%
\原本頁{103-5}\改行%
\ruby{御{\換字{父}}樣}{お|とつ|さま}が
\ruby{御病患}{お|わづ|らひ}でゞもあるか、% TODO 原本の「二の字点、揺すり点」に濁点のグリフが見つからないので「ゞ」
%
\ruby{御母樣}{お|つか|さま}が
\ruby{御惡}{お|わる}いのか、
%
それとも
\ruby{何樣}{ど|う}いふ
\ruby{事}{こと}で
\ruby{思}{おも}ひ
\ruby{餘}{あま}つて、
%
\ruby{丹精}{たん|せい}を
\ruby{御凝}{お|こ}らし
なさるか
\ruby{知}{し}らない
けども、
%
あの
\ruby{御年齡}{お|とし|ば{\換字{𛀁}}}で
\ruby{既}{もう}
\ruby{神佛}{かみ|ほとけ}の
\ruby{有難}{あり|がた}い
\ruby{事}{こと}を
\ruby{知}{し}つて
\ruby{居}{ゐ}られるのは、
%
\原本頁{103-8}\改行%
あ〻% 原本通り「〻(二の字点、揺すり点)」
\ruby{稀}{めづ}らしい
\ruby{殊{\換字{勝}}}{しゆ|しよう}な
かたゞ% TODO 原本の「二の字点、揺すり点」に濁点のグリフが見つからないので「ゞ」
と、
%
\ruby[g]{眞實}{ほんと}に
\ruby{貴君}{あな|た}の
\ruby{事}{こと}ばかり
\ruby{思}{おも}つて
\ruby{居}{を}りまして、
%
\ruby{何}{なん}だか
\ruby{私}{わたくし}は
\ruby{急}{きふ}に
\ruby{一人}{ひと|り}の、
%
\ruby{私}{わたくし}の
\ruby{味方}{み|かた}が
\ruby{出來}{で|き}た
やうな
\ruby{氣}{き}が
\ruby{致}{いた}し、
%
これも
\ruby{觀音樣}{くわん|のん|さま}の% 「觀音」の読みは原本通り「くわん(の)ん」
\ruby{御引合}{お|ひき|あは}せ
\ruby{下}{くだ}すつた
\ruby{菩提}{ぼ|だい}の
\ruby{同行}{どう|ぎやう}
とでも
いふので
あらう!、
%
と
\ruby{{\換字{勝}}手}{かつ|て}な
\ruby{考}{かんが}へでは
ございますが
\ruby{思}{おも}ひ
\ruby{詰}{つ}めまして、
%
\ruby{明{\換字{朝}}}{あし|た}
\ruby{御目}{お|め}に
か〻つたらば、% 原本通り「〻(二の字点、揺すり点)」
%
も
\ruby{一度}{いち|ど}
\ruby{御話}{お|はなし}して
\ruby{見}{み}やう、
%
\ruby{老人}{とし|より}の
\原本頁{104-2}\改行%
\ruby{事}{こと}ゆゑ
\ruby{御{\換字{嫌}}}{お|きら}ひ
なさるか
\ruby{知}{し}れないが、
%
どうも
\ruby{御話}{お|はなし}を
\ruby{仕}{し}て
\ruby{見}{み}たらば、
%
\ruby[<hj>]{屹度}{きつ|と}
\ruby{私}{わたくし}の
\ruby{力}{ちから}になつて
\ruby{下}{くだ}さる
\ruby{俠氣}{をとこ|ぎ}の
\ruby{方}{かた}だらう、
%
と
いふやうな
\ruby{心持}{こ〻ろ|もち}が% 原本通り「〻(二の字点、揺すり点)」
\ruby{仕}{し}てなりませんでした。
%
ところが
\ruby{明{\換字{朝}}}{あし|た}
\ruby{參}{まゐ}つて
\ruby{見}{み}ると
\ruby{御參詣}{お|い|で}は
ありません、
%
その
\ruby{次}{つぎ}の
\ruby{日}{ひ}も
\ruby{御參詣}{お|まゐ|り}が
ありません。
%
ぽろり〳〵と
\原本頁{104-6}\改行%
\ruby[||j>]{涙}{なみだ}を
\ruby{落}{おと}して
\ruby{眞}{しん}になつて
\ruby{何事}{なに|ごと}かを
\ruby{願}{ねが}つて
\ruby{居}{ゐ}られた
\ruby{彼}{あ}の
\ruby{方}{かた}が、
%
\ruby{不信心}{ぶ|しん|じん}に% 原本では行末行頭禁則で非踊り字
なられる
\ruby{理由}{わ|け}は
\ruby{無}{な}いが、
%
あ〻% 原本通り「〻(二の字点、揺すり点)」
\ruby{何}{なん}と
いつても
\ruby{未}{ま}だ
\ruby{御{\換字{若}}}{お|わか}い!、
%
\原本頁{104-8}\改行%
\ruby{下}{くだ}らない
\ruby{惡{\換字{魔}}}{あく|ま}
\ruby{外{\換字{道}}}{げ|だう}の
\ruby{馬鹿}{ば|か}
\ruby{書生}{しよ|せい}が、
%
\ruby{愚}{ぐ}に
つかない
\ruby{事}{こと}を
\ruby{饒舌}{しや|べ}つて
\ruby{居}{ゐ}たが、
%
\ruby{{\換字{若}}}{もし}や
\ruby{彼言}{あ|れ}が
\ruby{毒}{どく}に
なりは
\ruby{仕}{し}ないか
\ruby{按}{あん}じられる、
%
\ruby{何}{なん}と
いつても
\ruby{未}{ま}だ
\ruby{御{\換字{若}}}{お|わか}いから!、
%
と
\ruby{大}{おほ}きに
\ruby{彼}{あ}の
\ruby{書生等}{しよ|せい|たち}を
\ruby{憎}{にく}く
おもつて
\ruby{居}{を}りました。
』
