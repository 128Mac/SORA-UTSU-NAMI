\Entry{其十九}

『
\ruby{何樣致}{ど|う|いた}しまして、
\ruby[g]{貴君}{あなた}、
\ruby{惡{\換字{所}}}{あく|しよ}へ
\ruby{參}{まゐ}りました
\ruby[g]{歸路}{かへり}に
\ruby{{\換字{遠}}慮}{ゑん|りよ}を
\ruby{致}{いた}すことも
\ruby{存}{ぞん}じませんで
\ruby{神社佛閣}{じん|じや|ぶつ|かく}の
\ruby{境内}{けい|だい}へ
\ruby{入}{はい}りますやうな
\ruby{不心得}{ふ|こゝろ|\換字{江}}なものに、
\ruby{何}{なに}が
\ruby{一}{ひと}つ
\ruby{滿足}{まん|ぞく}に
\ruby{世}{よ}の
\ruby{中}{なか}の
\ruby{事}{こと}が
\ruby{解}{わか}りましやう。
みんな
\ruby{彼}{あ}の
\ruby{先日}{せん|じつ}の
\ruby{書生}{しよ|せい}の
\ruby{{\換字{連}}中}{れん|ちう}は、
\ruby{自分}{じ|ぶん}の
\ruby[g]{身體}{からだ}の
\ruby[g]{背後}{うしろ}から
\ruby{親}{おや}や
\ruby{兄}{あに}の
\ruby{氣息}{い|き}が
\ruby{掛}{かゝ}つて
\ruby{居}{ゐ}ればこそ
\ruby{高慢}{こう|まん}な
\ruby{口}{くち}を
\ruby{利}{き}きましても
\ruby{人}{ひと}が
\ruby{赦}{ゆる}して
\ruby{置}{お}いて
\ruby{{\換字{呉}}}{く}れるのだといふ
\ruby{事}{こと}も
\ruby{知}{し}りませんで、
\ruby{定}{きま}りきつた
\ruby{譫語}{たわ|ごと}を
\ruby{申}{まを}しまするが、
\ruby[g]{畢竟}{つまり}
\ruby{彼樣}{あ|ゝ}いふのは、
\ruby{親}{おや}や
\ruby{兄}{あに}の
\ruby{有}{あ}り
\ruby{難}{がた}い
\ruby{事}{こと}さへ
\ruby{解}{わか}つて
\ruby{居}{を}りませんのですもの、
\ruby{中々神佛}{なか|〳〵|かみ|ほとけ}の
\ruby{有}{あ}り
\ruby{難}{がた}い
\ruby{事}{こと}なんぞの
\ruby{解}{わか}らないのも、
\ruby{些}{ちつと}も
\ruby{無理}{む|り}はないのでございます。
それでも
\ruby{當世}{たう|せい}のものゝ
\ruby{事}{こと}でございますから、
\ruby{理屈}{り|くつ}は
\ruby{立}{た}ちさうなやうな
\ruby{理屈臭}{り|くつ|くさ}いことを、
\ruby{曲}{まが}りなりに
\ruby{牽{\換字{強}}}{こじ|つけ}て
\ruby{申}{まを}しますので、
\ruby{一寸聞}{ちよ|つと|き}けば
\ruby{{\換字{道}}理}{もつ|とも}なやうなにも
\ruby{思}{おも}はれます。
そこで
\ruby{穩和}{おと|なし}いものまで
\ruby{{\換字{巻}}}{ま}き
\ruby{{\換字{込}}}{こ}まれまして、やれ
\ruby{神樣}{かみ|さま}を
\ruby{敬}{うやま}ふのは
\ruby[g]{愚{\換字{迷}}}{まよひ}だの、
\ruby{佛樣}{ほとけ|さま}を
\ruby{崇}{あが}めるのは
\ruby{卑劣}{け|ち}だのと、
\ruby{傍}{はた}から
\ruby{始{\換字{終}}}{し|ゞう}
\ruby{云}{い}はれつけますと、
\ruby{矢張}{やつ|ぱり}いつか
\ruby{其氣}{その|き}になつて、
\ruby{其實神樣佛樣}{その|じつ|かみ|さま|ほとけ|さま}を
\ruby{頼}{たの}みたいやうな
\ruby{氣}{き}のすることは
\ruby{有}{あ}つても、
\ruby{神樣佛樣}{かみ|さま|ほとけ|さま}をいぢりまはすのが、
\ruby{何}{なん}だか
\ruby{意氣地}{い|く|ぢ}の
\ruby{無}{な}いやうな
\ruby{羞}{はづか}しいやうな
\ruby{氣}{き}が
\ruby{仕}{し}て、それで
\ruby{神樣}{かみ|さま}にも
\ruby{佛樣}{ほとけ|さま}にも、お
\ruby{縋}{すが}り
\ruby{申}{まを}さないで
\ruby[g]{一人}{ひとり}で
\ruby{下}{くだ}らなく
\ruby{苦}{くるし}みきつて
\ruby{居}{を}ります。
それが
\ruby{當世}{たう|せい}の
\ruby{一體}{いつ|たい}の
\ruby{風}{ふう}でございます。
それにまた
\ruby{何}{なん}とか
\ruby{彼}{か}とか
\ruby{云}{い}はれて
\ruby{居}{ゐ}らつしやる
\ruby{先生方}{せん|せい|がた}でも、
\ruby{正直}{しやう|ぢき}な
\ruby{方}{かた}や
\ruby{良}{よ}い
\ruby{方}{かた}ばかり
\ruby{有}{あ}りは
\ruby{仕}{し}ません。
\ruby{隨{\換字{分}}}{ずゐ|ぶん}わざと
\ruby{若}{わか}いものの
\ruby{氣}{き}に
\ruby{入}{い}るやうな
\ruby{事}{こと}を
\ruby{仰}{おつし}あつたり
\ruby{人}{ひと}を
\ruby{吃驚}{びつ|くり}させるやうな
\ruby{事}{こと}を
\ruby{仰}{おつし}あつたり、
\ruby{中}{なか}には
\ruby{{\換字{評}}{\換字{判}}}{ひやう|ばん}を
\ruby{取}{と}らうの
\ruby{目論見}{もく|ろ|み}やら、
\ruby{面白}{おも|しろ}づくの
\ruby[g]{好奇心}{ものずき}やらから、
\ruby{神}{かみ}も
\ruby{佛}{ほとけ}も
\ruby{耶蘇}{や|そ}もいけない、
\ruby{酒}{さけ}を
\ruby{飮}{の}んで
\ruby{管}{くだ}を
\ruby{{\換字{巻}}}{ま}いているのが
\ruby{一番好}{いち|ばん|い}い、
\ruby{女}{をんな}と
\ruby{戯}{ふざ}けてゐるのが
\ruby{何}{なに}よりだといふやうな
\ruby{大變}{たい|へん}な
\ruby{事}{こと}なんぞを
\ruby{仰}{おつし}ある
\ruby{方}{かた}もあるさうで、
\ruby{左樣}{さ|う}で
\ruby{無}{な}くつてさへ
\ruby{暴}{あば}れたがる
\ruby{若}{わか}いものが、
\ruby{其樣}{そ|ん}な
\ruby{事}{こと}を
\ruby{聞}{き}くのですから
\ruby{堪}{たま}つたものではありません、
\ruby{蝮}{まむし}を
\ruby{食}{く}つた
\ruby[g]{軍鷄}{しやも}のやうに
\ruby{氣}{き}ばかり
\ruby{{\換字{強}}}{つよ}くなつて、
\ruby{世界}{せ|かい}は
\ruby{何}{なん}でも
\ruby{{\換字{勝}}手}{かつ|て}の
\ruby{仕{\換字{勝}}}{し|がち}だと
\ruby{思}{おも}ひまして、
\ruby{相手}{あい|て}さへ
\ruby{見}{み}りやあ
\ruby{雞距}{けづ|め}を
\ruby{打込}{うち|こ}みたがりまする。
\ruby{{\換字{過}}日}{この|あひだ}の
\ruby{書生}{しよ|せい}などが
\ruby{其例}{そ|れ}でござりまして、
\ruby{吾家}{わたく|しども}にも
\ruby[g]{一人}{ひとり}、
\ruby{似}{に}たり
\ruby{寄}{よ}つたりの
\ruby{難物}{なん|ぶつ}がござりまする。
かういふ
\ruby{世間}{せ|けん}でござりまするのに、たま〳〵
\ruby[g]{貴君}{あなた}のやうな
\ruby{方}{かた}をお
\ruby{見受}{み|う}け
\ruby{申}{まを}したのですから、
\ruby{失禮}{しつ|れい}ながら
\ruby{御同年位}{ご|どう|ねん|くらひ}の
\ruby{吾家}{う|ち}の
\ruby{豚兒}{ば|か}めと
\ruby{思}{おも}ひ
\ruby{較}{あは}すにつけ、ほんとに
\ruby{御懷}{おな|つか}しく
\ruby{存}{ぞん}じましたが、
\ruby{其}{そ}の
\ruby[g]{貴君}{あなた}が
\ruby{其限}{それ|ぎ}り
\ruby{御見}{お|み}えになりませんので
\ruby{大變氣}{たい|へん|き}になつてなりませんでした。
\ruby{御若}{お|わか}いから
\ruby{彼}{あ}の
\ruby{書生}{しよ|せい}の
\ruby{云}{い}つた
\ruby{事}{こと}なんぞも
\ruby{御耳}{お|みゝ}に
\ruby{可厭}{い|や}でしたらうが、
\ruby{御{\換字{迷}}}{お|まよ}ひなすつてはいけません。
\ruby{氣}{き}になすつてはいけません。
\ruby{御信心}{ご|しん|〴〵}さへ
\ruby{御續}{お|つゞ}けなされば
\ruby{御利{\換字{益}}}{ご|り|やく}は
\ruby{{\換字{分}}}{わか}つて
\ruby{來}{き}ます。
\ruby{私}{わたくし}なども
\ruby{二三十年}{に|さん|じう|ねん}も
\ruby{前}{まへ}は
\ruby{矢張}{やつ|ぱ}り
\ruby{彼}{あ}の
\ruby{書生}{しよ|せい}でございましたから、
\ruby{彼}{あ}の
\ruby{書生}{しよ|せい}も
\ruby{二十年三十年經}{に|じう|ねん|さん|じう|ねん|た}ちましたら、
\ruby{私}{わたくし}になりまして、
\ruby{御利{\換字{益}}}{ご|り|やく}の
\ruby{力}{ちから}が
\ruby{身}{み}に
\ruby{沁}{し}みるやうになりましやう。
\ruby{一}{ひと}つ
\ruby{家}{や}の
\ruby{婆}{ばあ}さんだつて
\ruby{發起致}{ほつ|き|いた}しますのですもの、
\ruby{何年洋杖}{なん|ねん|すて|つき}を
\ruby{振}{ふ}り
\ruby{{\換字{廻}}}{まは}して
\ruby{威張}{ゐ|ば}つて
\ruby{居}{ゐ}られるものでございましやう?。
\ruby{虛言}{う|そ}や
\ruby{僞言}{いつ|はり}は
\ruby{申}{まを}しません、
\ruby[g]{私等}{わたくしら}は
\ruby{散々世}{さん|〴〵|よ}の
\ruby{中}{なか}の
\ruby{憂}{う}い
\ruby{辛}{つら}いの
\ruby{川}{かは}を
\ruby{越}{こ}して
\ruby{參}{まゐ}つて、
\ruby{此岸}{こち|ら}の
\ruby{信心}{しん|〴〵}の
\ruby{有}{あ}り
\ruby{難}{がた}い
\ruby{事好}{こと|い}い
\ruby{事}{こと}を
\ruby{見}{み}て
\ruby{居}{を}りまするので、
\ruby{彼等}{あれ|ら}は
\ruby{未}{ま}だ
\ruby{川}{かは}の
\ruby{中}{なか}へ
\ruby{入}{はい}り
\ruby{立}{たて}なので、
\ruby{元氣任}{げん|き|まか}せに
\ruby{立泳}{たち|およ}ぎを
\ruby{爲}{し}たり
\ruby{拔手}{ぬき|で}をきつたりしながら、
\ruby{何}{なん}だ
\ruby{對}{むか}ふ
\ruby{岸}{ぎし}に
\ruby{上}{あが}つて
\ruby{居}{ゐ}る
\ruby{奴等}{やつ|ら}の
\ruby{意氣地}{い|く|ぢ}の
\ruby{無}{な}さと
\ruby{申}{まを}して
\ruby{居}{ゐ}るやうなものでございます。
\ruby{疲勞}{くた|び}れたり、こむらが
\ruby{反}{かへ}つたり、
\ruby{流}{なが}れの
\ruby{{\換字{強}}}{つよ}いところへ
\ruby{出}{で}たりしますれば、
\ruby{此方}{こち|ら}の
\ruby{岸}{きし}を
\ruby{見}{み}て
\ruby{泣}{な}かずには
\ruby{居}{を}りません。
\ruby{其時}{その|とき}になつて
\ruby{前}{まへ}に
\ruby{此方}{こち|ら}に
\ruby{居}{ゐ}たものゝ
\ruby{心持}{こゝろ|もち}が
\ruby{解}{わか}ります。
あれ
\ruby{彼}{あ}の
\ruby{銀杏}{ぎん|なん}といふものは
\ruby[g]{公孫樹}{いてふ}の
\ruby{實}{み}です。
\ruby{榧}{かや}の
\ruby{實}{み}でも
\ruby{無}{み}ければ
\ruby{{\換字{又}}}{また}
\ruby{橡}{とち}の
\ruby{實}{み}でも
\ruby{無}{な}く、
\ruby{誰}{だれ}が
\ruby{何}{なん}といつても
\ruby[g]{公孫樹}{いてふ}の
\ruby{實}{み}です。
これに
\ruby{理屈}{り|くつ}が
\ruby{何有}{なに|あ}りましやう、もと〳〵
\ruby[g]{公孫樹}{いてふ}から
\ruby{出}{で}たものですもの!。
\ruby{神樣佛樣}{かみ|さま|ほとけ|さま}に
\ruby{縋}{すが}る
\ruby{私共}{わたく|しども}の
\ruby{此}{こ}の
\ruby{心}{こゝろ}は、
\ruby{何}{なん}の
\ruby{心}{こゝろ}でござりましやう!、
\ruby{人}{ひと}の
\ruby{心}{こゝろ}です。
\ruby{禽}{とり}の
\ruby{心}{こゝろ}でも
\ruby{無}{な}ければ
\ruby{獸}{けもの}の
\ruby{心}{こゝろ}でも
\ruby{無}{な}く、
\ruby{誰}{だれ}が
\ruby{何}{なん}といつても
\ruby{人}{ひと}の
\ruby{心}{こゝろ}です。
これに
\ruby{理屈}{り|くつ}が
\ruby{何有}{なに|あ}りましやう、もと〳〵
\ruby{人}{ひと}が
\ruby{有}{も}つた
\ruby{心}{こゝろ}ですもの!。
\ruby{吾}{わ}が
\ruby{子}{こ}の
\ruby{可愛}{かは|ゆ}いのに
\ruby{理屈}{り|くつ}も
\ruby{無}{な}く、
\ruby{思}{おも}ふ
\ruby{人}{ひと}の
\ruby{大切}{だい|じ}なのに
\ruby{理屈}{り|くつ}も
\ruby{無}{な}ければ、
\ruby{神樣佛樣}{かみ|さま|ほとけ|さま}に
\ruby{御縋}{お|すが}り
\ruby{申}{まを}すのに、何の
\ruby{理屈}{り|くつ}も
\ruby{無}{な}いけれど、それも
\ruby{眞實}{まこ|と}なれば
\ruby{此}{これ}も
\ruby{眞實}{まこ|と}で、
\ruby{理屈}{り|くつ}も
\ruby{要}{い}らないほどの
\ruby{眞實}{まこ|と}です!。
あゝ、いけません
\ruby{御{\換字{迷}}}{お|まよ}ひなすつては!。
いや
\ruby{御{\換字{迷}}}{お|まよ}ひなすつてはいけません
\ruby{貴方}{あな|た}!。
\ruby[g]{公孫樹}{いてふ}の
\ruby{秋}{あき}には
\ruby{銀杏}{ぎん|なん}が
\ruby{生}{な}ります、
\ruby{榧}{かや}の
\ruby{實}{み}も
\ruby{橡}{とち}の
\ruby{實}{み}も
\ruby{生}{な}りは
\ruby{仕}{し}ません、
\ruby{人}{ひと}の
\ruby{胸}{むね}には
\ruby{信心}{しん|〴〵}が
\ruby{生}{な}ります、
\ruby{生}{な}らせまいと
\ruby{思}{おも}つても
\ruby{生}{な}るのが
\ruby{約束}{やく|そく}、
\ruby{信}{しん}を
\ruby{有}{も}たなければ
\ruby{胸}{むね}が
\ruby{騒}{さわ}いで、
\ruby{誰}{だれ}が
\ruby{氣}{き}を
\ruby{安}{やす}くして
\ruby{居}{ゐ}られましやう!。
おゝ
\ruby{貴君}{あな|た}が
\ruby{默}{だま}つて
\ruby{居}{ゐ}らつしやるので
\ruby{私}{わたくし}ばかり
\ruby{饒舌}{しや|べ}りました。
さあ
\ruby{御堂}{お|だう}へ
\ruby{上}{あが}つて
\ruby{拜}{をが}みましやう。

と
\ruby{水野}{みづ|の}を
\ruby{牽}{ひ}きて
\ruby{共}{とも}に
\ruby{堂}{だう}に
\ruby{上}{のぼ}りぬ。

\ruby{老人}{らう|じん}が
\ruby{言}{ことば}を
\ruby{默々}{もく|〳〵}として
\ruby{聞}{き}きながら、
\ruby{水野}{みづ|の}は
\ruby{牽}{ひ}かるゝがまゝに
\ruby{堂}{だう}には
\ruby{上}{のぼ}りしが、
\ruby{{\換字{猶}}}{なほ}
\ruby{今朝}{け|さ}は
\ruby{直}{たゞち}に
\ruby{本{\換字{尊}}}{ほん|ぞん}を
\ruby{拜}{はい}せんともせず、さればとて
\ruby{侮}{あなど}り
\ruby{慢}{あなど}る
\ruby{心}{こゝろ}も
\ruby{無}{な}くて、
\ruby{喪心}{さう|しん}せる
\ruby{人}{ひと}の
\ruby{如}{ごと}く
\ruby{無意味}{む|い|み}に
\ruby{立}{た}ち
\ruby{居}{ゐ}たり。

