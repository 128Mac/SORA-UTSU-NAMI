\Entry{其十九}

% メモ 校正終了 2024-04-21 2024-05-31 2024-07-01
\原本頁{105-2}%
『% この語りは其十九の最後の方まで続く
\ruby[g]{何樣}{ど う }
\ruby{致}{いた}しまして、
%
\ruby[g]{貴君}{あなた }、
%
\ruby[g]{惡{\換字{所}}}{あくしよ}へ
\ruby{參}{まゐ}りました
\ruby[g]{歸路}{かへり }に
\ruby[g]{{\換字{遠}}慮}{ゑんりよ}を
\ruby{致}{いた}すことも
\ruby{存}{ぞん}じませんで
\ruby{神社佛閣}{じん|じや|ぶつ|かく}の
\ruby[g]{境内}{けいだい}へ
\ruby{入}{はい}ります
やうな
\ruby{不心得}{ふ|こゝろ|{\換字{𛀁}}}な% 踊り字調整「〻(二の字点、揺すり点)に見えるが(ゝ)」
ものに、
%
\ruby{何}{なに}が
\ruby{一}{ひと}つ
\ruby[g]{滿足}{まんぞく}に
\ruby{世}{よ}の
\ruby{中}{なか}の
\ruby{事}{こと}が
\ruby{解}{わか}りましやう。
%
みんな
\ruby{彼}{あ}
\原本頁{105-5}\改行%
の
\ruby[g]{先日}{せんじつ}の
\ruby[g]{書生}{しよせい}の
\ruby[g]{{\換字{連}}中}{れんぢう}は、
%
\ruby[g]{自{\換字{分}}}{じ ぶん}の
\ruby[g]{身體}{からだ }の
\ruby[g]{背後}{うしろ }から
\ruby{親}{おや}や
\ruby{兄}{あに}の
\ruby[g]{氣息}{い き }が
\原本頁{105-6}\改行%
\ruby{掛}{かゝ}つて% 踊り字調整「〻(二の字点、揺すり点)に見えるが(ゝ)」
\ruby{居}{ゐ}れば
こそ
\ruby[g]{高慢}{かうまん}な
\ruby{口}{くち}を
\ruby{利}{き}き
ましても
\ruby{人}{ひと}が
\ruby{赦}{ゆる}して
\ruby{置}{お}いて
\ruby{吳}{く}れる
のだ
といふ
\ruby{事}{こと}も
\ruby{知}{し}りませんで、
%
\ruby{定}{きま}り
きつた
\ruby[g]{譫語}{たわごと}を
\ruby{申}{まを}します
\原本頁{105-8}\改行%
るが、
%
\ruby[g]{畢竟}{つまり }
\ruby[g]{彼樣}{あ ゝ }いふのは、% 踊り字調整「〻(二の字点、揺すり点)に見えるが(ゝ)」
%
\ruby{親}{おや}や
\ruby{兄}{あに}の
\ruby{有}{あ}り
\ruby{{\換字{難}}}{がた}い
\ruby{事}{こと}さへ
\ruby{解}{わか}つて
\ruby{居}{を}りません
のですもの、
%
\ruby[g]{中々}{なか〳〵}
\ruby[||j>]{神}{かみ}
\ruby[||j>]{佛}{ほとけ}の
% \ruby{神佛}{かみ|ほとけ}の
\ruby{有}{あ}り
\ruby{{\換字{難}}}{がた}い
\ruby{事}{こと}
なんぞの
\ruby{解}{わか}らないの
\原本頁{105-10}\改行%
も、
%
\ruby{些}{ちつと}も
\ruby[g]{無理}{む り }は
ないので
ございます。
%
それでも
\ruby[g]{當世}{たうせい}の
ものゝ% 踊り字調整「〻(二の字点、揺すり点)に見えるが(ゝ)」
\ruby{事}{こと}で
ございますから、
%
\ruby[g]{理屈}{り くつ}は
\ruby{立}{た}ち
さうな
やうな
\ruby[g]{理屈}{り くつ}
\ruby{臭}{くさ}い
ことを、
%
\原本頁{106-2}\改行%
\ruby{曲}{まが}りなりに
\ruby[g]{牽{\換字{強}}}{こじつけ}て
\ruby{申}{まを}しますので、
%
\ruby[g]{一寸}{ちよつと}
\ruby{聞}{き}けば
\ruby[g]{{\換字{道}}理}{もつとも}な
やうな
にも
\ruby{思}{おも}はれます。
%
そこで
\ruby[g]{穩和}{おとなし}い
ものまでも
\ruby{卷}{ま}き
\ruby{{\換字{込}}}{こ}まれ
まして、
%
やれ
\原本頁{106-4}\改行%
\ruby[g]{神樣}{かみさま}を
\ruby[g]{敬ふ}{うやま }のは
\ruby[g]{愚{\換字{迷}}}{まよひ }だの、
%
\ruby[||j>]{佛}{ほとけ}
\ruby[||j>]{樣}{ さま}を
% \ruby{佛樣}{ほとけ|さま}を
\ruby{崇}{あが}めるのは
\ruby[g]{卑劣}{け ち }だのと、
%
\ruby{傍}{はた}から
\ruby[g]{始{\換字{終}}}{し じう}% ルビ調整(原本通り)「ゆ」無し
\ruby{云}{い}はれ
つけますと、
%
\ruby[g]{矢張}{やつぱり}
いつか
\ruby{其}{その}
\ruby{氣}{き}に
なつて、
%
\ruby{其}{その}
\ruby{實}{じつ}
\ruby[g]{神樣}{かみさま}
\ruby[||j>]{佛}{ほとけ}
\ruby[||j>]{樣}{ さま}を
% \ruby{佛樣}{ほとけ|さま}を
\ruby{頼}{たの}みたい
やうな
\ruby{氣}{き}の
することは
\ruby{有}{あ}つても、
%
\ruby[g]{神樣}{かみさま}
\ruby[||j>]{佛}{ほとけ}
\ruby[||j>]{樣}{ さま}を
% \ruby{佛樣}{ほとけ|さま}を
い
\原本頁{106-7}\改行%
ぢりまはすのが、
%
\ruby{何}{なん}だか
\ruby{意氣地}{い|く|ぢ}の
\ruby{無}{な}いやうな
\ruby{羞}{はづか}しいやうな
\ruby{氣}{き}が
\原本頁{106-8}\改行%
\ruby{仕}{し}て、
%
それで
\ruby[g]{神樣}{かみさま}にも
\ruby[||j>]{佛}{ほとけ}
\ruby[||j>]{樣}{ さま}にも、
% \ruby{佛樣}{ほとけ|さま}にも、
%
\ruby[g]{御縋}{お すが}り
\ruby{申}{まを}さないで、
\ruby[g]{一人}{ひとり }で
\ruby{下}{くだ}
\原本頁{106-9}\改行%
らなく
\ruby{苦}{くるし}み
きつて
\ruby{居}{を}ります。
%
それが
\ruby[g]{當世}{たうせい}の
\ruby[g]{一體}{いつたい}の
\ruby{風}{ふう}で
ございま
\原本頁{106-10}\改行%
す。
%
それに
また
\ruby{何}{なん}とか
\ruby{彼}{か}とか
\ruby{云}{い}はれて
\ruby{居}{ゐ}らつしやる
\ruby{先生方}{せん|せい|がた}でも
\改行% 校正作業の簡略化のため
、
%
\原本頁{106-11}\改行%
\ruby[||j>]{正}{しやう}
\ruby[||j>]{直}{ ぢき}な
% \ruby{正直}{しやう|ぢき}な
\ruby{方}{かた}や
\ruby{良}{い}い
\ruby{方}{かた}
ばかり
\ruby{有}{あ}りは
\ruby{仕}{し}ません。
%
\ruby[g]{{\換字{随}}{\換字{分}}}{ずゐぶん}%「隨」グリフ変更 ⻖左円辶
わざと
\ruby{{\換字{若}}}{わか}い
ものの% ルビ調整(原本通り)非踊り字表記(行末行頭の境界付近)
\ruby{氣}{き}に
\ruby{入}{い}るやうな
\ruby{事}{こと}を
\ruby{仰}{おつし}あつたり
\ruby{人}{ひと}を
\ruby[g]{吃驚}{びつくり}させる
やうな
\ruby{事}{こと}を
\ruby[<j||]{仰}{おつし}
\原本頁{107-2}\改行%
あつたり、
%
\ruby{中}{なか}には
\ruby[||j>]{{\換字{評}}}{ひやう}
\ruby[||j>]{{\換字{判}}}{ ばん}を
% \ruby{{\換字{評}}{\換字{判}}}{ひやう|ばん}を
\ruby{取}{と}らうの
\ruby{目論見}{もく|ろ|み}やら、
%
\ruby[g]{面白}{おもしろ}づくの
\ruby{好奇心}{もの|ず|き}やらから、
%
\ruby{神}{かみ}も
\ruby{佛}{ほとむ}も% ルビ調整(原本通り)国会図書館も(け)ではなく(む)
\ruby[g]{耶蘇}{や そ }も
いけない、
%
\ruby{酒}{さけ}を
\ruby{飮}{の}んで
\ruby{管}{くだ}を
\ruby{卷}{ま}いているのが
\ruby[g]{一番}{いちばん}
\ruby{好}{い}い、
%
\ruby{女}{をんな}と
\ruby{戱}{ふざ}けて
ゐるのが
\ruby{何}{なに}よりだ
といふやうな
\原本頁{107-5}\改行%
\ruby[g]{大變}{たいへん}な
\ruby{事}{こと}
なんぞを
\ruby{仰}{おつし}ある
\ruby{方}{かた}も
あるさうで、
%
\ruby[g]{左樣}{さ う }で
\ruby{無}{な}くつて
さへ
\原本頁{107-6}\改行%
\ruby{暴}{あば}れたがる
\ruby{{\換字{若}}}{わか}い
ものが、
%
\ruby[g]{其樣}{そ ん }な
\ruby{事}{こと}を
\ruby{聞}{き}く
のですから
\ruby{堪}{たま}つた
もの
\原本頁{107-7}\改行%
では
ありません、
%
\ruby{蝮}{まむし}を
\ruby{食}{く}つた
\ruby[g]{軍鷄}{しやも }の
やうに
\ruby{氣}{き}
ばかり
\ruby{{\換字{強}}}{つよ}く
なつて
\改行% 校正作業の簡略化のため
、
%
\原本頁{107-8}\改行%
\ruby[g]{世界}{せ かい}は
\ruby{何}{なん}でも
\ruby[g]{{\換字{勝}}手}{かつて }の
\ruby[g]{仕{\換字{勝}}}{し がち}だと
\ruby{思}{おも}ひまして、
%
\ruby[g]{相手}{あひて }
さへ
\ruby{見}{み}りやあ
\ruby[g]{雞趾}{けづめ }を
\ruby[g]{打{\換字{込}}}{うちこ }み
たがり
まする。
%
\ruby[||j>]{{\換字{過}}}{この}
\ruby[||j>]{日}{あひだ}の
% \ruby{{\換字{過}}日}{この|あひだ}の
\ruby[g]{書生}{しよせい}などが
\ruby[g]{其例}{そ れ }で
ござりまし
\原本頁{107-10}\改行%
て、
%
\ruby[<j>]{吾}{わたくし}
\ruby[||j>]{家}{ ども}にも
% \ruby{吾家}{わたくし|ども}にも
\ruby[g]{一人}{ひとり }、
%
\ruby{似}{に}たり
\ruby{寄}{よ}つたりの
\ruby[g]{{\換字{難}}物}{なんぶつ}が
ござりまする。
%
かういふ
\ruby[g]{世間}{せ けん}で
ござりまするのに、
%
たま〳〵
\ruby[g]{貴君}{あなた }の
やうな
\ruby{方}{かた}を
\ruby{御見受}{お|み|う}け
\ruby{申}{まを}した
のですから、
%
\ruby[g]{失禮}{しつれい}
ながら
\ruby{御}{ご}
\ruby[g]{同年}{どうねん}
\ruby{位}{ぐらゐ}の
\ruby[g]{吾家}{う ち }の
\ruby[g]{豚兒}{ば か }め
\原本頁{108-2}\改行%
と
\ruby{思}{おも}ひ
\ruby{較}{あは}す
につけ、
%
ほんとに
\ruby[g]{御懷}{おなつか}しく
\ruby{存}{ぞん}じましたが、
%
\ruby{其}{そ}の
\ruby[g]{貴君}{あなた }が
\ruby{其}{それ}
\ruby{限}{ぎ}り
\ruby[g]{御見}{お み }えに
なりませんので
\ruby[g]{大變}{たいへん}
\ruby{氣}{き}に
なつて
なりません
で
\原本頁{108-4}\改行%
した。
\ruby[g]{御{\換字{若}}}{お わか}い
から
\ruby{彼}{あ}の
\ruby[g]{書生}{しよせい}の
\ruby{云}{い}つた
\ruby{事}{こと}
なんぞも
\ruby[g]{御耳}{お みゝ}に% 踊り字調整「〻(二の字点、揺すり点)に見えるが(ゝ)」
\ruby[g]{可厭}{い や }でしたらうが、
%
\ruby[g]{御{\換字{迷}}}{お まよ}ひ
なすつては
いけません。
%
\ruby{氣}{き}に
なすつては
いけ
\原本頁{108-6}\改行%
ません。
%
\ruby{御信心}{ご|しん|〴〵}
さへ
\ruby[g]{御續}{お つゞ}け% 踊り字調整「〻(二の字点、揺すり点)に濁点に見えるが(ゞ)」
なされば
\ruby{御利益}{ご|り|やく}は
\ruby{{\換字{分}}}{わか}つて
\ruby{來}{き}ます
%
\ruby[<g>]{。私}{わたくし}% 行末行頭の境界付近なので特例処置を施す
\原本頁{108-7}\改行%
なども
\ruby{二三十年}{に|さん|じふ|ねん}も
\ruby{{\換字{前}}}{まへ}は
\ruby[g]{矢張}{やつぱ }り
\ruby{彼}{あ}の
\ruby[g]{書生}{しよせい}で
ございました
から、
%
\ruby{彼}{あ}
\原本頁{108-8}\改行%
の
\ruby[g]{書生}{しよせい}も
\ruby{二十年三十年經}{に|じふ|ねん|さん|じふ|ねん|た}ち
ましたら、
%
\ruby[<j>]{私}{わたくし}に
なりまして、
%
\ruby{御利益}{ご|り|やく}の
\ruby{力}{ちから}が
\ruby{身}{み}に
\ruby{沁}{し}みる
やうに
なりましやう。
%
\ruby{一ツ家}{ひと||や}の
\ruby{婆}{ばあ}さんだつて
\原本頁{108-10}\改行%
\ruby[g]{發起}{ほつき }
\ruby{致}{いた}しますのですもの、
%
\ruby[g]{何年}{なんねん}
\ruby[g]{洋{\換字{杖}}}{すてつき}を
\ruby{振}{ふ}り
\ruby{{\換字{廻}}}{まは}して
\ruby[g]{威張}{ゐ ば }つて
\ruby{居}{ゐ}られるもので
ございましやう?。
%
\ruby[g]{虛言}{う そ }や
\ruby[g]{僞言}{いつはり}は
\ruby{申}{まを}しません、
%
\ruby[<j>]{私}{わたくし}
\ruby[||j>]{等}{ども}
% \ruby{私等}{わたくし|ども}
\原本頁{109-1}\改行%
は
\ruby[g]{散々}{さん〴〵}
\ruby{世}{よ}の
\ruby{中}{なか}の
\ruby{憂}{う}い
\ruby{辛}{つら}いの
\ruby{川}{かは}を
\ruby{越}{こ}して
\ruby{參}{まゐ}つて、
%
\ruby[g]{此岸}{こちら }の
\ruby[g]{信心}{しん〴〵}の
\ruby{有}{あ}り
\ruby{{\換字{難}}}{がた}い
\ruby{事}{こと}
\ruby{好}{い}い
\ruby{事}{こと}を
\ruby{見}{み}て
\ruby{居}{を}りまする
ので、
%
\ruby[g]{彼等}{あれら }は
\ruby{未}{ま}だ
\ruby{川}{かは}の
\ruby{中}{なか}へ
\ruby{入}{はい}
\原本頁{109-3}\改行%
り
\ruby{立}{たて}なので、
%
\ruby[g]{元氣}{げんき }
\ruby{任}{まか}せに
\ruby[g]{立泳}{たちおよ}ぎを
\ruby{爲}{し}たり
\ruby[g]{拔手}{ぬきで }を
きつたり
しながら、
%
\ruby{何}{なん}だ
\ruby{對}{むか}ふ
\ruby{岸}{ぎし}に
\ruby{上}{あが}つて
\ruby{居}{ゐ}る
\ruby[g]{奴等}{やつら }の
\ruby{意氣地}{い|く|ぢ}の
\ruby{無}{な}さと
\ruby{申}{まを}して
\ruby{居}{ゐ}る
\原本頁{109-5}\改行%
やうな
もので
ございます。
%
\ruby[g]{疲勞}{くたび }れたり、
%
こむらが
\ruby{反}{かへ}つたり、
%
\ruby{流}{なが}れの
\ruby{{\換字{強}}}{つよ}い
ところへ
\ruby{出}{で}たり
しますれば、
%
\ruby[g]{此方}{こちら }の% ルビ調整(原本通り)
\ruby{岸}{きし}を
\ruby{見}{み}て
\ruby{泣}{な}かずに
\原本頁{109-7}\改行%
は
\ruby{居}{を}りません。
%
\ruby{其}{その}
\ruby{時}{とき}に
なつて
\ruby{{\換字{前}}}{さき}に
\ruby[g]{此方}{こちら }に% ルビ調整(原本通り)
\ruby{居}{ゐ}た
ものゝ% 踊り字調整「〻(二の字点、揺すり点)に見えるが(ゝ)」
\ruby[||j>]{心}{こゝろ}% 踊り字調整「〻(二の字点、揺すり点)に見えるが(ゝ)」
\ruby[||j>]{持}{ もち}が
% \ruby{心持}{こゝろ|もち}が% 踊り字調整「〻(二の字点、揺すり点)に見えるが(ゝ)」
\ruby{解}{わか}り
\原本頁{109-8}\改行%
ます。
%
あれ
\ruby{彼}{あ}の
\ruby[g]{銀杏}{ぎんなん}
といふものは
\ruby{公孫樹}{い|て|ふ}の
\ruby{實}{み}です。
%
\ruby{榧}{かや}の
\ruby{實}{み}でも
\原本頁{109-9}\改行%
\ruby{無}{な}ければ
\ruby{{\換字{又}}}{また}
\ruby{橡}{とち}の
\ruby{實}{み}でも
\ruby{無}{な}く、
%
\ruby{誰}{たれ}が
\ruby{何}{なん}
といつても
\ruby{公孫樹}{い|て|ふ}の
\ruby{實}{み}です
\改行% 校正作業の簡略化のため
。
%
\原本頁{109-10}\改行%
これに
\ruby[g]{理屈}{り くつ}が
\ruby{何}{なに}
\ruby{有}{あ}りましやう、
%
もと〳〵
\ruby{公孫樹}{い|て|ふ}から
\ruby{出}{で}た
もので
\原本頁{109-11}\改行%
すもの!。
%
\ruby[g]{神樣}{かみさま}
\ruby[||j>]{佛}{ほとけ}
\ruby[||j>]{樣}{ さま}に
% \ruby{佛樣}{ほとけ|さま}に
\ruby{縋}{すが}る
\ruby[<j>]{私}{わたくし}
\ruby[||j>]{共}{ ども}の
\ruby{此}{こ}の
\ruby{心}{こゝろ}は、% 踊り字調整「〻(二の字点、揺すり点)に見えるが(ゝ)」
%
\ruby{何}{なん}の
\ruby{心}{こゝろ}で% 踊り字調整「〻(二の字点、揺すり点)に見えるが(ゝ)」
ござりま
\原本頁{110-1}\改行%
しやう!、
%
\ruby{人}{ひと}の
\ruby{心}{こゝろ}です。% 踊り字調整「〻(二の字点、揺すり点)に見えるが(ゝ)」
%
\ruby{禽}{とり}の
\ruby{心}{こゝろ}でも% 踊り字調整「〻(二の字点、揺すり点)に見えるが(ゝ)」
\ruby{無}{あ}ければ% ルビ調整(原本通り)
\footnote{他の23箇所では「\ruby{無}{な}ければ」であるがここのみ原本通り「\ruby{無}{あ}ければ」とする
(国会図書館 コマ番号60/160 p-110 l-01)}%
\ruby{獸}{けもの}の
\ruby{心}{こゝろ}でも% 踊り字調整「〻(二の字点、揺すり点)に見えるが(ゝ)」
\ruby{無}{な}く、
%
\ruby{誰}{たれ}が
\ruby{何}{なん}と
いつても
\ruby{人}{ひと}の
\ruby{心}{こゝろ}% 踊り字調整「〻(二の字点、揺すり点)に見えるが(ゝ)」
です。
%
これに
\ruby[g]{理屈}{り くつ}が
\ruby[g]{何有}{なにあ }りましやう、
%
もと〳〵
\ruby{人}{ひと}が
\ruby{有}{も}つた
\ruby{心}{こゝろ}% 踊り字調整「〻(二の字点、揺すり点)に見えるが(ゝ)」
ですもの!。
%
\ruby{吾}{わ}が
\ruby{子}{こ}の
\ruby[g]{可愛}{か はゆ}いのに
\ruby[g]{理屈}{り くつ}も
\原本頁{110-4}\改行%
\ruby{無}{な}く、
%
\ruby{思}{おも}ふ
\ruby{人}{ひと}の
\ruby[g]{大切}{だいじ }なのに
\ruby[g]{理屈}{り くつ}も
\ruby{無}{な}ければ、
%
\ruby[g]{神樣}{かみさま}
\ruby[||j>]{佛}{ほとけ}
\ruby[||j>]{樣}{ さま}に
% \ruby{佛樣}{ほとけ|さま}に
\ruby[g]{御縋}{お すが}り
\ruby{申}{まを}すのに、
%
何の
\ruby[g]{理屈}{り くつ}も
\ruby{無}{な}いけれど、
%
それも
\ruby[g]{眞實}{まこと }なれば
\ruby{此}{これ}も
\ruby[g]{眞實}{まこと }
\原本頁{110-6}\改行%
で、
%
\ruby[g]{理屈}{り くつ}も
\ruby{要}{い}らない
ほどの
\ruby[g]{眞實}{まこと }です!。
%
あゝ、% 踊り字調整「〻(二の字点、揺すり点)に見えるが(ゝ)」
%
いけません
\ruby[g]{御{\換字{迷}}}{お まよ}ひ
なすつては!。
%
いや
\ruby[g]{御{\換字{迷}}}{お まよ}ひ
なすつてはいけません
\ruby[g]{貴方}{あなた }!。
%
\ruby{公孫樹}{い|て|ふ}の
\ruby{秋}{あき}には
\ruby[g]{銀杏}{ぎんなん}が
\ruby{生}{な}ります、
%
\ruby{榧}{かや}の
\ruby{實}{み}も
\ruby{橡}{とち}の
\ruby{實}{み}も
\ruby{生}{な}りは
\ruby{仕}{し}ませ
\原本頁{110-9}\改行%
ん、
%
\ruby{人}{ひと}の
\ruby{胸}{むね}には
\ruby[g]{信心}{しん〴〵}が
\ruby{生}{な}ります、
%
\ruby{生}{な}らせまいと
\ruby{思}{おも}つても
\ruby{生}{な}るの
\原本頁{110-10}\改行%
が
\ruby[g]{約束}{やくそく}、
%
\ruby{信}{しん}を
\ruby{有}{も}たなければ
\ruby{胸}{むね}が
\ruby{騷}{さわ}いで、
%
\ruby{誰}{たれ}が
\ruby{氣}{き}を
\ruby{安}{やす}くして
\ruby{居}{ゐ}ら
\原本頁{110-11}\改行%
れましやう!。
%
おゝ% 踊り字調整「〻(二の字点、揺すり点)に見えるが(ゝ)」
\ruby[g]{貴君}{あなた }が
\ruby{默}{だま}つて
\ruby{居}{ゐ}らつしやるので
\ruby[<j>]{私}{わたくし}
ばかり
\ruby[g]{饒舌}{しやべ }りました。
%
さあ
\ruby[g]{御堂}{お だう}へ
\ruby{上}{あが}つて
\ruby{拜}{をが}みましやう。
』% この其十九の始めから始まった語りの終わり

\原本頁{111-2}%
と
\ruby[g]{水野}{みづの }を
\ruby{牽}{ひ}きて
\ruby{共}{とも}に
\ruby{堂}{だう}に
\ruby{上}{のぼ}りぬ。

\原本頁{111-3}%
\ruby[g]{老人}{らうじん}が
\ruby{言}{ことば}を
\ruby[g]{默々}{もく〳〵}として
\ruby{聞}{き}き
ながら、
%
\ruby[g]{水野}{みづの }は
\ruby{牽}{ひ}かるゝが
まゝに% 踊り字調整「〻(二の字点、揺すり点)に見えるが(ゝ)」
\ruby{堂}{だう}には
\ruby{上}{のぼ}りしが、
%
\ruby{{\換字{猶}}}{なほ}
\ruby[g]{今{\換字{朝}}}{け さ }は
\ruby{直}{たゞち}に% 踊り字調整「〻(二の字点、揺すり点)に濁点に見えるが(ゞ)」
\ruby[g]{本{\換字{尊}}}{ほんぞん}を
\ruby{拜}{はい}せん
ともせず、
%
されば
と
\原本頁{111-4}\改行%
て
\ruby{侮}{あなど}り
\ruby{慢}{あなど}る
\ruby[g]{心も}{こゝろ }% 踊り字調整「〻(二の字点、揺すり点)に見えるが(ゝ)」
\ruby{無}{な}くて、
%
\ruby[g]{喪心}{さうしん}せる
\ruby{人}{ひと}の
\ruby{如}{ごと}く
\ruby{無}{む}
\ruby[g]{意味}{い み }に
\ruby{立}{た}ち
\ruby{居}{ゐ}たり
\改行% 校正作業の簡略化のため
。
