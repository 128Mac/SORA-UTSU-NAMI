\
\Entry{其十九}

% メモ 校正終了 2024-04-21
\原本頁{105-2}%
『% この語りは其十九の最後の方まで続く
\ruby{何樣}{ど|う}
\ruby{致}{いた}しまして、
%
\ruby{貴君}{あな|た}、
%
\ruby{惡{\換字{所}}}{あく|しよ}へ
\ruby{參}{まゐ}りました
\ruby{歸路}{かへ|り}に
\ruby{{\換字{遠}}慮}{ゑん|りよ}を
\ruby{致}{いた}す
\原本頁{105-3}\改行%DEBUG
ことも
\ruby{存}{ぞん}じませんで
\ruby{神社佛閣}{じん|じや|ぶつ|かく}の
\ruby{境内}{けい|だい}へ
\ruby{入}{はい}ります
やうな
\ruby{不心得}{ふ|こ〻ろ|{\換字{𛀁}}}な% 原本通り「〻(二の字点、揺すり点)」
ものに、
%
\ruby{何}{なに}が
\ruby{一}{ひと}つ
\ruby{滿足}{まん|ぞく}に
\ruby{世}{よ}の
\ruby{中}{なか}の
\ruby{事}{こと}が
\ruby{解}{わか}りましやう。
%
みんな
\ruby{彼}{あ}
\原本頁{105-5}\改行%DEBUG
の
\ruby{先日}{せん|じつ}の
\ruby{書生}{しよ|せい}の
\ruby{{\換字{連}}中}{れん|ぢう}は、
%
\ruby{自{\換字{分}}}{じ|ぶん}の
\ruby{身體}{から|だ}の
\ruby{背後}{うし|ろ}から
\ruby{親}{おや}や
\ruby{兄}{あに}の
\ruby{氣息}{い|き}が
\原本頁{105-6}\改行%
\ruby{掛}{か〻}つて% 原本通り「〻(二の字点、揺すり点)」
\ruby{居}{ゐ}れば
こそ
\ruby{高慢}{かう|まん}な
\ruby{口}{くち}を
\ruby{利}{き}き
ましても
\ruby{人}{ひと}が
\ruby{赦}{ゆる}して
\ruby{置}{お}いて
\ruby{吳}{く}れる
のだ
といふ
\ruby{事}{こと}も
\ruby{知}{し}りませんで、
%
\ruby{定}{きま}り
きつた
\ruby{譫語}{たわ|ごと}を
\ruby{申}{まを}します
\原本頁{105-8}\改行%DEBUG
るが、
%
\ruby{畢竟}{つま|り}
\ruby{彼樣}{あ|〻}いふのは、% 原本通り「〻(二の字点、揺すり点)」
%
\ruby{親}{おや}や
\ruby{兄}{あに}の
\ruby{有}{あ}り
\ruby{{\換字{難}}}{がた}い
\ruby{事}{こと}さへ
\ruby{解}{わか}つて
\ruby{居}{を}りません
のですもの、
%
\ruby{中々}{なか|〳〵}
\ruby[||j>]{神}{かみ}
\ruby[||j>]{佛}{ほとけ}の
% \ruby{神佛}{かみ|ほとけ}の
\ruby{有}{あ}り
\ruby{{\換字{難}}}{がた}い
\ruby{事}{こと}
なんぞの
\ruby{解}{わか}らないの
\原本頁{105-10}\改行%
も、
%
\ruby{些}{ちつと}も
\ruby{無理}{む|り}は
ないので
ございます。
%
それでも
\ruby{當世}{たう|せい}の
もの〻% 原本通り「〻(二の字点、揺すり点)」
\ruby{事}{こと}で
ございますから、
%
\ruby{理屈}{り|くつ}は
\ruby{立}{た}ち
さうな
やうな
\ruby{理屈}{り|くつ}
\ruby{臭}{くさ}い
ことを、
%
\原本頁{106-2}\改行%
\ruby{曲}{まが}りなりに
\ruby{牽{\換字{強}}}{こじ|つけ}て
\ruby{申}{まを}しますので、
%
\ruby{一寸}{ちよ|つと}
\ruby{聞}{き}けば
\ruby{{\換字{道}}理}{もつ|とも}な
やうな
にも
\ruby{思}{おも}はれます。
%
そこで
\ruby{穩和}{おと|なし}い
ものまでも
\ruby{卷}{ま}き
\ruby{{\換字{込}}}{こ}まれ
まして、
%
やれ
\原本頁{106-4}\改行%
\ruby{神樣}{かみ|さま}を
\ruby{敬}{うやま}ふのは
\ruby{愚{\換字{迷}}}{まよ|ひ}だの、
%
\ruby[||j>]{佛}{ほとけ}
\ruby[||j>]{樣}{ さま}を
% \ruby{佛樣}{ほとけ|さま}を
\ruby{崇}{あが}めるのは
\ruby{卑劣}{け|ち}だのと、
%
\ruby{傍}{はた}から
\ruby{始{\換字{終}}}{しじ|う}% ルビは原本通り「ゆ」無し
\ruby{云}{い}はれ
つけますと、
%
\ruby{矢張}{やつ|ぱり}
いつか
\ruby{其}{その}
\ruby{氣}{き}に
なつて、
%
\ruby{其}{その}
\ruby{實}{じつ}
\ruby{神樣}{かみ|さま}
\ruby[||j>]{佛}{ほとけ}
\ruby[||j>]{樣}{ さま}を
% \ruby{佛樣}{ほとけ|さま}を
\ruby{頼}{たの}みたい
やうな
\ruby{氣}{き}の
することは
\ruby{有}{あ}つても、
%
\ruby{神樣}{かみ|さま}
\ruby[||j>]{佛}{ほとけ}
\ruby[||j>]{樣}{ さま}を
% \ruby{佛樣}{ほとけ|さま}を
い
\原本頁{106-7}\改行%DEBUG
ぢりまはすのが、
%
\ruby{何}{なん}だか
\ruby{意氣地}{い|く|ぢ}の
\ruby{無}{な}いやうな
\ruby{羞}{はづか}しいやうな
\ruby{氣}{き}が
\原本頁{106-7}\改行%
\ruby{仕}{し}て、
%
それで
\ruby{神樣}{かみ|さま}にも
\ruby[||j>]{佛}{ほとけ}
\ruby[||j>]{樣}{ さま}にも、
% \ruby{佛樣}{ほとけ|さま}にも、
%
\ruby{御縋}{お|すが}り
\ruby{申}{まを}さないで
\ruby{一人}{ひと|り}で
\ruby{下}{くだ}らなく
\ruby{苦}{くるし}み
きつて
\ruby{居}{を}ります。
%
それが
\ruby{當世}{たう|せい}の
\ruby{一體}{いつ|たい}の
\ruby{風}{ふう}で
ございます。
%
それに
また
\ruby{何}{なん}とか
\ruby{彼}{か}とか
\ruby{云}{い}はれて
\ruby{居}{ゐ}らつしやる
\ruby{先生方}{せん|せい|がた}でも、
%
\原本頁{106-11}\改行%
\ruby[||j>]{正}{しやう}
\ruby[||j>]{直}{ ぢき}な
% \ruby{正直}{しやう|ぢき}な
\ruby{方}{かた}や
\ruby{良}{い}い
\ruby{方}{かた}
ばかり
\ruby{有}{あ}りは
\ruby{仕}{し}ません。
%
\ruby{{\換字{随}}{\換字{分}}}{ずゐ|ぶん}%「隨」TODO 変更 ⻖左円辶
わざと
\ruby{{\換字{若}}}{わか}い
ものの% 原本では行末行頭禁則のため非踊り文字
\ruby{氣}{き}に
\ruby{入}{い}るやうな
\ruby{事}{こと}を
\ruby{仰}{おつし}あつたり
\ruby{人}{ひと}を
\ruby{吃驚}{びつ|くり}させる
やうな
\ruby{事}{こと}を
\ruby{仰}{おつし}
\原本頁{107-2}\改行%DEBUG
あつたり、
%
\ruby{中}{なか}には
\ruby[||j>]{{\換字{評}}}{ひやう}
\ruby[||j>]{{\換字{判}}}{ ばん}を
% \ruby{{\換字{評}}{\換字{判}}}{ひやう|ばん}を
\ruby{取}{と}らうの
\ruby{目論見}{もく|ろ|み}やら、
%
\ruby{面白}{おも|しろ}づくの
\ruby{好奇心}{もの|ず|き}やらから、
%
\ruby{神}{かみ}も
\ruby{佛}{ほとけ}も
\ruby{耶蘇}{や|そ}も
いけない、
%
\ruby{酒}{さけ}を
\ruby{飮}{の}んで
\ruby{管}{くだ}を
\ruby{卷}{ま}いているのが
\ruby{一番}{いち|ばん}
\ruby{好}{い}い、
%
\ruby{女}{をんな}と
\ruby{戱}{ふざ}けて
ゐるのが
\ruby{何}{なに}よりだ
といふやうな
\原本頁{107-5}\改行%
\ruby{大變}{たい|へん}な
\ruby{事}{こと}
なんぞを
\ruby{仰}{おつし}ある
\ruby{方}{かた}も
あるさうで、
%
\ruby{左樣}{さ|う}で
\ruby{無}{な}くつて
さへ
\ruby{暴}{あば}れたがる
\ruby{{\換字{若}}}{わか}い
ものが、
%
\ruby{其樣}{そ|ん}な
\ruby{事}{こと}を
\ruby{聞}{き}く
のですから
\ruby{堪}{たま}つた
ものでは
ありません、
%
\ruby{蝮}{まむし}を
\ruby{食}{く}つた
\ruby{軍鷄}{しや|も}の
やうに
\ruby{氣}{き}
ばかり
\ruby{{\換字{強}}}{つよ}く
なつて、
%
\原本頁{107-8}\改行%
\ruby{世界}{せ|かい}は
\ruby{何}{なん}でも
\ruby{{\換字{勝}}手}{かつ|て}の
\ruby{仕{\換字{勝}}}{し|がち}だと
\ruby{思}{おも}ひまして、
%
\ruby{相手}{あひ|て}
さへ
\ruby{見}{み}りやあ
\ruby{雞趾}{け|づめ}を
\ruby{打{\換字{込}}}{うち|こ}み
たがり
まする。
%
\ruby[||j>]{{\換字{過}}}{この}
\ruby[||j>]{日}{あひだ}の
% \ruby{{\換字{過}}日}{この|あひだ}の
\ruby{書生}{しよ|せい}などが
\ruby{其例}{そ|れ}で
ござりまして、
%
\ruby[||j>]{吾}{わたくし}
\ruby[||j>]{家}{  ども}にも
% \ruby{吾家}{わたくし|ども}にも
\ruby{一人}{ひと|り}、
%
\ruby{似}{に}たり
\ruby{寄}{よ}つたりの
\ruby{{\換字{難}}物}{なん|ぶつ}が
ござりまする。
%
か
\原本頁{107-11}\改行%DEBUG
ういふ
\ruby{世間}{せ|けん}で
ござりまするのに、
%
たま〳〵
\ruby{貴君}{あな|た}の
やうな
\ruby{方}{かた}を
\ruby{御見受}{お|み|う}け
\ruby{申}{まを}した
のですから、
%
\ruby{失禮}{しつ|れい}
ながら
\ruby{御同年位}{ご|どう|ねん|ぐらゐ}の
\ruby{吾家}{う|ち}の
\ruby{豚兒}{ば|か}め
\原本頁{108-2}\改行%DEBUG
と
\ruby{思}{おも}ひ
\ruby{較}{をは}す
につけ、
%
ほんとに
\ruby{御懷}{お|なつか}しく
\ruby{存}{ぞん}じましたが、
%
\ruby{其}{そ}の
\ruby{貴君}{あな|た}が
\ruby{其}{それ}
\ruby{限}{ぎ}り
\ruby{御見}{お|み}えに
なりませんので
\ruby{大變}{たい|へん}
\ruby{氣}{き}に
なつて
なりません
で
\原本頁{108-4}\改行%DEBUG
した。
\ruby{御{\換字{若}}}{お|わか}い
から
\ruby{彼}{あ}の
\ruby{書生}{しよ|せい}の
\ruby{云}{い}つた
\ruby{事}{こと}
なんぞも
\ruby{御耳}{お|み〻}に% 原本通り「〻(二の字点、揺すり点)」
\ruby{可厭}{い|や}でしたらうが、
%
\ruby{御{\換字{迷}}}{お|まよ}ひ
なすつては
いけません。
%
\ruby{氣}{き}に
なすつては
いけ
\原本頁{108-6}\改行%DEBUG
ません。
%
\ruby{御信心}{ご|しん|〴〵}
さへ
\ruby{御續}{お|つゞ}け% TODO 原本の「二の字点、揺すり点」に濁点のグリフが見つからないので「ゞ」
なされば
\ruby{御利益}{ご|り|やく}は
\ruby{{\換字{分}}}{わか}つて
\ruby{來}{き}ます。
%
\ruby{私}{わたくし}なども
\ruby{二三十年}{に|さん|じう|ねん}も
\ruby{{\換字{前}}}{まへ}は
\ruby{矢張}{やつ|ぱ}り
\ruby{彼}{あ}の
\ruby{書生}{しよ|せい}で
ございました
から、
%
\ruby{彼}{あ}
\原本頁{108-8}\改行%DEBUG
の
\ruby{書生}{しよ|せい}も
\ruby{二十年三十年經}{に|じう|ねん|さん|じう|ねん|た}ち
ましたら、
%
\ruby{私}{わたくし}に
なりまして、
%
\ruby{御利益}{ご|り|やく}の
\ruby{力}{ちから}が
\ruby{身}{み}に
\ruby{沁}{し}みる
やうに
なりましやう。
%
\ruby{一ッ家}{ひと||や}の
\ruby{婆}{ばあ}さんだつて
\原本頁{108-10}\改行%
\ruby{發起}{ほつ|き}
\ruby{致}{いた}しますのですもの、
%
\ruby{何年}{なん|ねん}
\ruby{洋{\換字{杖}}}{すて|つき}を
\ruby{振}{ふ}り
\ruby{{\換字{廻}}}{まは}して
\ruby{威張}{ゐ|ば}つて
\ruby{居}{ゐ}られるもので
ございましやう?。
%
\ruby{虛言}{う|そ}や
\ruby{僞言}{いつ|はり}は
\ruby{申}{まを}しません、
%
\ruby[||j>]{私}{わたくし}
\ruby[||j>]{等}{  ども}
% \ruby{私等}{わたくし|ども}
\原本頁{109-1}\改行%DEBUG
は
\ruby{散々}{さん|〴〵}
\ruby{世}{よ}の
\ruby{中}{なか}の
\ruby{憂}{う}い
\ruby{辛}{つら}いの
\ruby{川}{かは}を
\ruby{越}{こ}して
\ruby{參}{まゐ}つて、
%
\ruby{此岸}{こち|ら}の
\ruby{信心}{しん|〴〵}の
\ruby{有}{あ}り
\ruby{{\換字{難}}}{がた}い
\ruby{事}{こと}
\ruby{好}{い}い
\ruby{事}{こと}を
\ruby{見}{み}て
\ruby{居}{を}りまする
ので、
%
\ruby{彼等}{あれ|ら}は
\ruby{未}{ま}だ
\ruby{川}{かは}の
\ruby{中}{なか}へ
\ruby{入}{はい}
\原本頁{109-3}\改行%DEBUG
り
\ruby{立}{たて}なので、
%
\ruby{元氣}{げん|き}
\ruby{任}{まか}せに
\ruby{立泳}{たち|およ}ぎを
\ruby{爲}{し}たり
\ruby{拔手}{ぬき|で}を
きつたり
しながら、
%
\ruby{何}{なん}だ
\ruby{對}{むか}ふ
\ruby{岸}{ぎし}に
\ruby{上}{あが}つて
\ruby{居}{ゐ}る
\ruby{奴等}{やつ|ら}の
\ruby{意氣地}{い|く|ぢ}の
\ruby{無}{な}さと
\ruby{申}{まを}して
\ruby{居}{ゐ}る
\原本頁{109-5}\改行%DEBUG
やうな
もので
ございます。
%
\ruby{疲勞}{くた|び}れたり、
%
こむらが
\ruby{反}{かへ}つたり、
%
\ruby{流}{なが}れの
\ruby{{\換字{強}}}{つよ}い
ところへ
\ruby{出}{で}たり
しますれば、
%
\ruby{此方}{こち|ら}の
\ruby{岸}{きし}を
\ruby{見}{み}て
\ruby{泣}{な}かずに
\原本頁{109-7}\改行%DEBUG
は
\ruby{居}{を}りません。
%
\ruby{其}{その}
\ruby{時}{とき}に
なつて
\ruby{{\換字{前}}}{さき}に
\ruby{此方}{こち|ら}に
\ruby{居}{ゐ}た
もの〻% 原本通り「〻(二の字点、揺すり点)」
\ruby[||j>]{心}{こ〻ろ}が% 原本通り「〻(二の字点、揺すり点)」
\ruby[||j>]{持}{ もち}が
% \ruby{心持}{こ〻ろ|もち}が% 原本通り「〻(二の字点、揺すり点)」
\ruby{解}{わか}ります。
%
あれ
\ruby{彼}{あ}の
\ruby{銀杏}{ぎん|なん}
といふものは
\ruby{公孫樹}{い|て|ふ}の
\ruby{實}{み}です。
%
\ruby{榧}{かや}の
\ruby{實}{み}でも
\原本頁{109-9}\改行%
\ruby{無}{な}ければ
\ruby{{\換字{又}}}{また}
\ruby{橡}{とち}の
\ruby{實}{み}でも
\ruby{無}{な}く、
%
\ruby{誰}{たれ}が
\ruby{何}{なん}
といつても
\ruby{公孫樹}{い|て|ふ}の
\ruby{實}{み}です。
%
\原本頁{109-10}\改行%
これに
\ruby{理屈}{り|くつ}が
\ruby{何}{なに}
\ruby{有}{あ}りましやう、
%
もと〳〵
\ruby{公孫樹}{い|て|ふ}から
\ruby{出}{で}た
ものですもの!。
%
\ruby{神樣}{かみ|さま}
\ruby[||j>]{佛}{ほとけ}
\ruby[||j>]{樣}{ さま}に
% \ruby{佛樣}{ほとけ|さま}に
\ruby{縋}{すが}る
\ruby{私共}{わたくし|ども}の
\ruby{此}{こ}の
\ruby{心}{こ〻ろ}は、% 原本通り「〻(二の字点、揺すり点)」
%
\ruby{何}{なん}の
\ruby{心}{こ〻ろ}で
ござりましやう!、% 原本通り「〻(二の字点、揺すり点)」
%
\ruby{人}{ひと}の
\ruby{心}{こ〻ろ}です。% 原本通り「〻(二の字点、揺すり点)」
%
\ruby{禽}{とり}の
\ruby{心}{こ〻ろ}でも% 原本通り「〻(二の字点、揺すり点)」
\ruby{無}{な}ければ
\ruby{獸}{けもの}の
\ruby{心}{こ〻ろ}でも% 原本通り「〻(二の字点、揺すり点)」
\ruby{無}{な}く、
%
\原本頁{110-2}\改行%
\ruby{誰}{たれ}が
\ruby{何}{なん}と
いつても
\ruby{人}{ひと}の
\ruby{心}{こ〻ろ}% 原本通り「〻(二の字点、揺すり点)」
です。
%
これに
\ruby{理屈}{り|くつ}が
\ruby{何有}{なに|あ}りましやう、
%
もと〳〵
\ruby{人}{ひと}が
\ruby{有}{も}つた
\ruby{心}{こ〻ろ}% 原本通り「〻(二の字点、揺すり点)」
ですもの!。
%
\ruby{吾}{わ}が
\ruby{子}{こ}の
\ruby{可愛}{か|はゆ}いのに
\ruby{理屈}{り|くつ}も
\原本頁{110-4}\改行%
\ruby{無}{な}く、
%
\ruby{思}{おも}ふ
\ruby{人}{ひと}の
\ruby{大切}{だい|じ}なのに
\ruby{理屈}{り|くつ}も
\ruby{無}{な}ければ、
%
\ruby{神樣}{かみ|さま}
\ruby[||j>]{佛}{ほとけ}
\ruby[||j>]{樣}{ さま}に
% \ruby{佛樣}{ほとけ|さま}に
\ruby{御縋}{お|すが}り
\ruby{申}{まを}すのに、
%
何の
\ruby{理屈}{り|くつ}も
\ruby{無}{な}いけれど、
%
それも
\ruby{眞實}{まこ|と}なれば
\ruby{此}{これ}も
\ruby{眞實}{まこ|と}
\原本頁{110-6}\改行%DEBUG
で、
%
\ruby{理屈}{り|くつ}も
\ruby{要}{い}らない
ほどの
\ruby{眞實}{まこ|と}です!。
%
あ〻、% 原本通り「〻(二の字点、揺すり点)」
%
いけません
\ruby{御{\換字{迷}}}{お|まよ}ひ
なすつては!。
%
いや
\ruby{御{\換字{迷}}}{お|まよ}ひ
なすつてはいけません
\ruby{貴方}{あな|た}!。
%
\ruby{公孫樹}{い|て|ふ}の
\ruby{秋}{あき}には
\ruby{銀杏}{ぎん|なん}が
\ruby{生}{な}ります、
%
\ruby{榧}{かや}の
\ruby{實}{み}も
\ruby{橡}{とち}の
\ruby{實}{み}も
\ruby{生}{な}りは
\ruby{仕}{し}ませ
\原本頁{110-9}\改行%DEBUG
ん、
%
\ruby{人}{ひと}の
\ruby{胸}{むね}には
\ruby{信心}{しん|〴〵}が
\ruby{生}{な}ります、
%
\ruby{生}{な}らせまいと
\ruby{思}{おも}つても
\ruby{生}{な}るの
\原本頁{110-10}\改行%
が
\ruby{約束}{やく|そく}、
%
\ruby{信}{しん}を
\ruby{有}{も}たなければ
\ruby{胸}{むね}が
\ruby{騷}{さわ}いで、
%
\ruby{誰}{たれ}が
\ruby{氣}{き}を
\ruby{安}{やす}くして
\ruby{居}{ゐ}ら
\原本頁{110-11}\改行%DEBUG
れましやう!。
%
お〻% 原本通り「〻(二の字点、揺すり点)」
\ruby{貴君}{あな|た}が
\ruby{默}{だま}つて
\ruby{居}{ゐ}らつしやるので
\ruby[<j>]{私}{わたくし}
ばかり
\ruby{饒舌}{しや|べ}りました。
%
さあ
\ruby{御堂}{お|だう}へ
\ruby{上}{あが}つて
\ruby{拜}{をが}みましやう。
』% この其十九の始めから始まった語りの終わり

\原本頁{111-2}%
と
\ruby{水野}{みづ|の}を
\ruby{牽}{ひ}きて
\ruby{共}{とも}に
\ruby{堂}{だう}に
\ruby{上}{のぼ}りぬ。

\原本頁{111-3}%
\ruby{老人}{らう|じん}が
\ruby{言}{ことば}を
\ruby{默々}{もく|〳〵}として
\ruby{聞}{き}き
ながら、
%
\ruby{水野}{みづ|の}は
\ruby{牽}{ひ}かる〻が
まゝに% 原本通り「〻(二の字点、揺すり点)」
\ruby{堂}{だう}には
\ruby{上}{のぼ}りしが、
%
\ruby{{\換字{猶}}}{なほ}
\ruby{今{\換字{朝}}}{け|さ}は
\ruby{直}{たゞち}に% TODO 原本の「二の字点、揺すり点」に濁点のグリフが見つからないので「ゞ」
\ruby{本{\換字{尊}}}{ほん|ぞん}を
\ruby{拜}{はい}せん
ともせず、
%
されば
とて
\ruby{侮}{あなど}り
\ruby{慢}{あなど}る
\ruby{心}{こ〻ろ}も% 原本通り「〻(二の字点、揺すり点)」
\ruby{無}{な}くて、
%
\ruby{喪心}{さう|しん}せる
\ruby{人}{ひと}の
\ruby{如}{ごと}く
\ruby{無}{む}
\ruby{意味}{い|み}に
\ruby{立}{た}ち
\ruby{居}{ゐ}たり。
