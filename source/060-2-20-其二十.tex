\Entry{其二十}

% メモ 校正終了 2024-04-22 2024-05-31 2024-07-01
\原本頁{111-7}%
\ruby{既}{すで}に
\ruby{我}{わ}が
\ruby[g]{言葉}{ことば }を
\ruby{戾}{もど}きもせず、
%
また
\ruby{我}{わ}が
\ruby{{\換字{伴}}}{ともな}ふを
\ruby{拒}{こば}みもせねば、
%
\ruby{今}{いま}
\原本頁{111-8}\改行%
\ruby[g]{御堂}{み だう}に
\ruby{上}{のぼ}りて
\ruby[g]{御{\換字{前}}}{おんまへ}に
\ruby{至}{いた}れる
\ruby{上}{うへ}は、
%
\ruby{必}{かな}らず
\ruby{復}{また}
\ruby{{\換字{前}}}{さき}の
\ruby{日}{ひ}の
\ruby{{\換字{朝}}}{あさ}の
\ruby{如}{ごと}くに
\改行% 校正作業の簡略化のため
、
%
\原本頁{111-9}\改行%
たとひ
\ruby[g]{御經は}{おんきやう }
\ruby{誦}{じゆ}せざる
までも、
%
\ruby[<j>]{掌}{たなぞこ}を
\ruby{合}{あは}せ
\ruby{頭}{かうべ}を
\ruby{下}{さ}げて
\ruby[g]{禮拜}{らいはい}するな
\原本頁{111-10}\改行%
らんと、
%
\ruby{獨合點}{ひとり|が|てん}してや
\ruby{彼}{か}の
\ruby[g]{老人}{らうじん}は、
%
\ruby[g]{御堂}{み だう}に
\ruby{上}{のぼ}りて
よりは
\ruby[g]{水野}{みづの }に
\原本頁{112-1}\改行%
\ruby{關}{かま}はず、
%
\ruby{一}{ひと}つは
\ruby[g]{自己}{お の }が
\ruby[g]{信心}{しん〴〵}の
\ruby{誠}{まこと}を
\ruby{致}{いた}さん
とするに
\ruby{忙}{いそが}しきが
\ruby{故}{ゆゑ}
も
\原本頁{112-2}\改行%
あるべし、
%
\ruby{例}{いつも}の
\ruby{如}{ごと}く
\ruby[g]{御{\換字{前}}}{み まへ}に
\ruby[g]{蹲ま}{うづく }りて、
%
\ruby{先}{ま}ず
\ruby[g]{一心}{いつしん}に
\ruby[||j>]{恭}{きやう}
\ruby[||j>]{敬}{ けい}
\ruby[||j>]{禮}{ らい}
\ruby[||j>]{拜}{ はい}
しつ
\改行% 校正作業の簡略化のため
、
%
\原本頁{112-3}\改行%
\ruby[g]{徐々}{しづか}に
% \ruby[g]{妙法蓮華經觀世音菩薩普門品第二十五}{めうはうれんげきやうくわんぜおんぼさつふもんぼんだいにじふご}、と
\ruby{妙法蓮華經}{めう|はふ|れん|げ|きやう}% 「蓮 uf999」(参考「蓮 u84ee」)
\ruby{觀世音}{くわん|ぜ|おん}
\ruby[g]{菩薩}{ぼ さつ}
\ruby{普門品}{ふ|もん|ぼん}
\ruby{第二十五}{だい|に|じふ|ご}、
%
と
\ruby{老}{お}いたる
\ruby{聲}{こゑ}の
\ruby{低}{ひく}く
\ruby{誦}{じゆ}し
\ruby{出}{いだ}したり。

\原本頁{112-5}%
\ruby{{\換字{朝}}}{あさ}の
\ruby{氣}{き}は
\ruby{何}{なん}となく
\ruby{心}{こゝろ}% 踊り字調整「〻(二の字点、揺すり点)に見えるが(ゝ)」
をして
\ruby[||j>]{粛}{しゆく}
\ruby[||j>]{然}{ ぜん}
% \ruby{粛然}{しゆく|ぜん}
たらしめて、
%
\ruby{廣}{ひろ}き
\ruby[g]{御堂}{み だう}の
\ruby{内}{うち}の
\ruby{人}{ひと}
\ruby{無}{な}き
\ruby{物}{もの}
\ruby{靜}{しづ}かさは
\ruby[g]{自然}{おのづ }と
\ruby{胸}{むね}の
\ruby{中}{うち}を
\ruby[g]{淸々}{すが〳〵}し
からしむ。
%
\ruby[g]{今日}{け ふ }は
\ruby[g]{御佛}{みほとけ}を
\原本頁{112-7}\改行%
\ruby{拜}{をが}み
\ruby[<j>]{奉}{たてまつ}り
もせず、
%
さりとて
\ruby{{\換字{又}}}{また}
\ruby[g]{御佛}{みほとけ}より
\ruby{反}{そむ}き
\ruby{去}{さ}りもせず、
%
たゞ% 踊り字調整「〻(二の字点、揺すり点)に濁点に見えるが(ゞ)」
ただ% ルビ調整(原本通り)非踊り字表記(行末行頭の境界付近)
\ruby[g]{從順}{すなほ }なる
\ruby[g]{兒童}{こ ども}の、
%
\ruby{心}{こゝろ}に% 踊り字調整「〻(二の字点、揺すり点)に見えるが(ゝ)」
\ruby{物}{もの}
\ruby{無}{な}きが
\ruby{如}{ごと}く、
%
\ruby{牽}{ひ}かれたる
まゝに% 踊り字調整「〻(二の字点、揺すり点)に見えるが(ゝ)」
\ruby[g]{此處}{こ ゝ }に% 踊り字調整「〻(二の字点、揺すり点)に見えるが(ゝ)」
\ruby{來}{きた}りて、
%
\ruby[g]{此處}{こ ゝ }に% 踊り字調整「〻(二の字点、揺すり点)に見えるが(ゝ)」
\ruby{其}{その}
\ruby{儘}{まゝ}% 踊り字調整「〻(二の字点、揺すり点)に見えるが(ゝ)」
\ruby{止}{とゞま}れる% 踊り字調整「〻(二の字点、揺すり点)に濁点に見えるが(ゞ)」
\ruby[g]{水野}{みづの }は、
%
\ruby[g]{身{\換字{近}}}{み ぢか}なりし
\ruby[||j>]{圓}{まる}
\ruby[||j>]{柱}{ばしら}の
% \ruby{圓柱}{まる|ばしら}の
\ruby{太}{ふと}きに
\原本頁{112-10}\改行%
\ruby{憑}{よ}りて、
%
\ruby{風}{かぜ}
\ruby{吹}{ふ}かぬ
\ruby{間}{ま}を
\ruby[g]{大{\換字{空}}}{おほぞら}に
\ruby{高}{たか}く
\ruby{懸}{かゝ}れる% 踊り字調整「〻(二の字点、揺すり点)に見えるが(ゝ)」
\ruby[||j>]{孤}{ひとつ}
\ruby[||j>]{雲}{ ぐも}の、
% \ruby{孤雲}{ひとつ|ぐも}の、
%
\ruby{何}{なに}に
\ruby{着}{つ}くとも
\ruby{無}{な}き
\ruby{思}{おもひ}に、
%
\ruby[g]{嗒焉}{たう{\換字{𛀁}}ん}として
\ruby{獨}{ひと}り
\ruby{{\換字{空}}}{むな}しく
\ruby{立}{た}てり。

\原本頁{113-1}%
\ruby{老}{お}いたる
\ruby{人}{ひと}の
\ruby{誦}{じゆ}する
\ruby{經}{きやう}の、
%
\ruby{其}{その}
\ruby{意}{こゝろ}は% 踊り字調整「〻(二の字点、揺すり点)に見えるが(ゝ)」
\ruby{曉}{さと}らるゝ% 踊り字調整「〻(二の字点、揺すり点)に見えるが(ゝ)」
\ruby{時}{とき}あり
\ruby{曉}{さと}らえざる
\ruby{時}{とき}あれど、
%
\ruby{其}{その}
\ruby{聲}{こゑ}は
\ruby[g]{波瀾}{な み }
\ruby{無}{な}く
\ruby[g]{山坂}{や ま }
\ruby{無}{な}くして
\ruby[g]{一條}{ひとすぢ}の
\ruby{絲}{いと}を
\ruby{畫}{ひ}ける
にも
\ruby{似}{に}て
\ruby{{\換字{平}}}{たひ}らかなるに、
%
\ruby{聞}{き}き
\ruby{居}{ゐ}る
\ruby{我}{わ}が
\ruby{心}{こゝろ}は% 踊り字調整「〻(二の字点、揺すり点)に見えるが(ゝ)」
\ruby[g]{刻々}{こく〳〵}に
\ruby{安}{やす}まり
\ruby{行}{ゆ}き、
%
\ruby{何}{なん}とは
\原本頁{113-4}\改行%
\ruby{無}{な}けれど
\ruby{引}{ひ}き
\ruby{入}{い}れらるゝ% 踊り字調整「〻(二の字点、揺すり点)に見えるが(ゝ)」
やうに
おぼえて、
%
\ruby{知}{し}らず
\ruby{識}{し}らず
\ruby[g]{無念}{む ねん}
\ruby[g]{無想}{む さう}の
\ruby{境}{さかひ}に
\ruby{入}{い}る
\ruby{折}{をり}しも、
%
\ruby{人}{ひと}の
\ruby[g]{下駄}{げ た }の
\ruby{音}{おと}に
\ruby[g]{不圖}{ふ と }
\ruby{驚}{おどろ}きて、
%
\ruby{見}{み}れば
\ruby[g]{何時}{い つ }の
\ruby{間}{ま}にやら
\ruby[g]{三十}{さんじふ}
ばかり
なる
\ruby[g]{女の}{をんな }
、
%
\ruby[g]{老人}{らうじん}と
\ruby{並}{なら}びて
\ruby[g]{禮拜}{らいはい}
なし
\ruby{居}{を}り、
%
\原本頁{113-7}\改行%
\ruby[g]{老人}{らうじん}の
\ruby[||j>]{誦}{じゆ}
\ruby[||j>]{經}{きやう}は
% \ruby{誦經}{じゆ|きやう}は
\ruby{今}{いま}や
\ruby{{\換字{終}}}{をは}らん
として、
%
\ruby{具一切功德}{ぐ|いつ|さい|く|どく}、
%
\ruby{慈眼視}{じ|げん|じ}
\ruby{衆}{しゆ}
\ruby{生}{じやう}と、
% \ruby{慈眼視衆生}{じ|げん|じ|しゆ|じやう}と、
%
\ruby{偈}{げ}の
\ruby{末}{すゑ}
のところを
\ruby{誦}{よ}み
\ruby{居}{ゐ}たり。
%
\ruby{是}{こ}は
\ruby[g]{不覺}{ふ かく}なりし
\ruby{愚}{おろか}なりし!。
%
\ruby{身}{み}は
こそ
\ruby{動}{うご}かさゞりつれ% 踊り字調整「〻(二の字点、揺すり点)に濁点に見えるが(ゞ)」
\ruby{心}{こゝろ}の% 踊り字調整「〻(二の字点、揺すり点)に見えるが(ゝ)」
\ruby{内}{うち}には、
%
\ruby{吾}{わ}が
\ruby{兒}{こ}の
\ruby[g]{可憐}{かはゆ }いのに
\ruby[g]{理屈}{り くつ}も
\原本頁{113-10}\改行%
\ruby{無}{な}く、
%
\ruby{思}{おも}ふ
\ruby{人}{ひと}の
\ruby[g]{大切}{だいじ }なのに
\ruby[g]{理屈}{り くつ}も
\ruby{無}{な}ければ、
%
\ruby[g]{神樣}{かみさま}
\ruby[||j>]{佛}{ほとけ}
\ruby[||j>]{樣}{ さま}に
% \ruby{佛樣}{ほとけ|さま}に
\ruby[g]{御縋}{お すが}り
\ruby{申}{まを}すのにも、
%
\ruby{何}{なん}の
\ruby[g]{理屈}{り くつ}も
\ruby{無}{な}いなれど、
%
それも
\ruby[g]{眞實}{まこと }
なれば、
%
\ruby{此}{これ}も
\原本頁{114-1}\改行%
\ruby[g]{眞實}{まこと }
で、
%
\ruby[g]{理屈}{り くつ}の
\ruby{要}{い}らない
ほどの
\ruby[g]{眞實}{まこと }
!\inhibitglue{}と
\ruby{云}{い}ひたる
\ruby{此}{こ}の
\ruby[g]{老人}{らうじん}の
\ruby[g]{言葉}{ことば }を
\ruby{味}{あぢ}はひて、
%
\ruby{實}{げ}に
\ruby{云}{い}はるれば
%
\ruby{其}{そ}の
\ruby{如}{ごと}くなり、
%
\ruby{我}{わ}が
\ruby{彼}{か}の
\ruby{人}{ひと}を
\ruby{思}{おも}ひ
\ruby{思}{おも}ふ
\ruby{心}{こゝろ}に、% 踊り字調整「〻(二の字点、揺すり点)に見えるが(ゝ)」
%
\ruby[g]{そも〳〵}{}% 行末行頭の禁則対策
\ruby{何}{なん}の
\ruby[g]{理由}{いはれ }の
ありや、
%
\ruby{何}{なん}の
\ruby[g]{理由}{わ け }とは
\ruby{我}{われ}も
\ruby{知}{し}らず、
%
たゞ% 踊り字調整「〻(二の字点、揺すり点)に濁点に見えるが(ゞ)」
\ruby{我}{われ}と
\ruby{我}{わ}が
\ruby[g]{欺き}{あざむ }
\ruby{{\換字{難}}}{がた}き
\ruby{{\換字{情}}}{こゝろ}の% 踊り字調整「〻(二の字点、揺すり点)に見えるが(ゝ)」
\ruby{萌}{も}えに
\ruby{萌}{も}え
\ruby{出}{い}づるを
\ruby{抑}{おさ}へ
\ruby{得}{{\換字{𛀁}}}ざるぞ
\ruby[g]{眞實}{まこと }なる!。
%
\ruby{思}{おも}ふて
\ruby{思}{おも}はるゝ% 踊り字調整「〻(二の字点、揺すり点)に見えるが(ゝ)」
\ruby{身}{み}ならば
こそ、
%
\ruby[g]{不{\換字{運}}}{ふ うん}にして
\ruby{我}{われ}
\原本頁{114-6}\改行%
\ruby{拙}{つたな}く
\ruby{生}{うま}れ
\ruby{來}{き}て、
%
\ruby{思}{おも}へば
\ruby{思}{おも}ふほど
\ruby{{\換字{嫌}}}{きら}はるゝ% 踊り字調整「〻(二の字点、揺すり点)に見えるが(ゝ)」
\ruby{身}{み}の、
%
\ruby{思}{おも}ふて
\ruby[g]{甲{\換字{斐}}}{か ひ }
\ruby{無}{な}き
\ruby{事}{こと}なれば、
%
\ruby{自}{みづか}ら
\ruby[g]{斷念}{あきら }め
\ruby{思}{おも}ひ
\ruby{切}{き}りて、
%
\ruby{忘}{わす}れ
\ruby{果}{は}てん
こそ
\ruby{人}{ひと}のため
\ruby{身}{み}のため
なれ、
%
\ruby{我}{わ}が
\ruby{爲}{な}す
\ruby{事}{こと}
\ruby{言}{い}う
\ruby{事}{こと}は
\ruby{何}{なに}から
\ruby{何}{なに}まで、
%
\ruby{{\換字{情}}}{なさけ}なくも
\ruby{彼}{か}の
\原本頁{114-9}\改行%
\ruby{人}{ひと}に
\ruby{厭}{いと}はるゝ% 踊り字調整「〻(二の字点、揺すり点)に見えるが(ゝ)」
ながら、
%
\ruby{思}{おも}ひ
\ruby{忘}{わす}るゝ% 踊り字調整「〻(二の字点、揺すり点)に見えるが(ゝ)」
といふ
\ruby[g]{此事}{こ れ }
ばかりは、
%
\ruby{必}{かなら}ず
\ruby{彼}{か}の
\ruby{人}{ひと}に
\ruby{悅}{よろこ}ばるべければ、
%
\ruby[g]{果敢}{は か }なく
\ruby{悲}{かな}しき
\ruby{限}{かぎ}りなれど、
%
とても
かくても
\ruby[g]{味氣}{あぢき }
\ruby{無}{な}き
\ruby{我}{わ}が
\ruby[||j>]{一}{いつ}
\ruby[||j>]{生}{しやう}の
% \ruby{一生}{いつ|しやう}の
\ruby{思}{おも}ひ
\ruby{出}{いで}に、% 原本通りに「(い)で」
%
せめては
\ruby[g]{男兒}{をとこ }らしう
ふつ
\原本頁{115-1}\改行%
つりと% ルビ調整(原本通り)非踊り字表記(行末行頭の境界付近)
\ruby{諦}{あきら}めて、
%
うるさく
\ruby[g]{纏繞}{まつは }る
\ruby[||j>]{蔓}{つた}
\ruby[||j>]{葛}{かつら}の% 原本では「蔦(つた)」でなく「蔓(つる)」
% \ruby{蔓葛}{つた|かつら}の% 原本では「蔦(つた)」でなく「蔓(つる)」
\ruby{離}{はな}れて
\ruby{去}{さ}りし
\ruby{嬉}{うれ}しさよと
\改行% 校正作業の簡略化のため
、
%
\原本頁{115-2}\改行%
\ruby{彼}{か}の
\ruby{人}{ひと}に
\ruby{安}{やす}き
\ruby{思}{おもひ}を
させん、
%
\ruby{人}{ひと}も
\ruby{見}{み}ず
\ruby{人}{ひと}をも
\ruby{見}{み}ざる
\ruby{深}{ふか}き
\ruby{山}{やま}の
\ruby{巖}{いは}の
\ruby[g]{罅隙}{はざま }に
\ruby{我}{われ}
\ruby[g]{一人}{ひとり }
\ruby{入}{い}りて、
%
\ruby{誰}{たれ}
\ruby{憚}{はゞか}らず% 「憚 は(ゞ)か」% 踊り字調整「〻(二の字点、揺すり点)に濁点に見えるが(ゞ)」
\ruby{思}{おも}ふさま
\ruby{泣}{な}きて、
%
\ruby{其}{その}
\ruby[||j>]{淚}{なみだ}の
\ruby{乾}{かは}き
\ruby{聲}{こゑ}の
\ruby{枯}{か}れん
\ruby{時}{とき}
\ruby{我}{われ}
\ruby{{\換字{即}}}{すなは}ち
\ruby{此}{この}
\ruby{世}{よ}を
\ruby{去}{さ}らば
\ruby{濟}{す}むべき
\ruby{事}{こと}なるをや!、
%
と
\ruby[g]{幾度}{いくたび}か〳〵
\ruby{思}{おも}ひしかど、
%
\ruby{諦}{あきら}めても
\ruby{諦}{あきら}めても% ルビ調整(原本通り)非踊り字表記
\ruby{諦}{あきら}め
\ruby{得}{{\換字{𛀁}}}ず、
%
\ruby{彼}{か}の
\ruby{人}{ひと}を
\ruby[g]{背後}{うしろ }にして
\ruby[g]{千里}{せんり }の
\ruby{{\換字{遠}}}{とほ}きに
\ruby{身}{み}を
\ruby{隱}{かく}し
\ruby{棄}{す}てん
とする
\ruby{意}{こゝろ}は% 踊り字調整「〻(二の字点、揺すり点)に見えるが(ゝ)」
ありても、
%
\ruby{彼}{か}の
\原本頁{115-7}\改行%
\ruby{人}{ひと}より
\ruby{距}{へだ}たらん
とすれば
\ruby[g]{一歩}{いつぽ }も
\ruby{去}{さ}り
\ruby{得}{{\換字{𛀁}}}ず、
%
\ruby{我}{わ}が
\ruby{心}{こゝろ}の% 踊り字調整「〻(二の字点、揺すり点)に見えるが(ゝ)」
\ruby{我}{わ}が
\ruby{心}{こゝろ}に% 踊り字調整「〻(二の字点、揺すり点)に見えるが(ゝ)」
\ruby{任}{まか}せずして、
%
あだに
\ruby{苦}{くるし}み
あだに
\ruby{惱}{なや}むは、
%
たゞ% 踊り字調整「〻(二の字点、揺すり点)に濁点に見えるが(ゞ)」
\ruby{我}{われ}と
\ruby{我}{わ}が
\ruby{欺}{あざむ}きがたき
\原本頁{115-9}\改行%
\ruby{{\換字{情}}}{こゝろ}の% 踊り字調整「〻(二の字点、揺すり点)に見えるが(ゝ)」
\ruby{萌}{も}えに
\ruby{萌}{も}ゆればなり。
%
おもへば
\ruby[||j>]{神}{かみ}
\ruby[||j>]{佛}{ほとけ}を
% \ruby{神佛}{かみ|ほとけ}を
\ruby{頼}{たの}み
\ruby[<j>]{奉}{たてまつ}るも
\ruby{實}{げ}に
\ruby{似}{に}たる
\ruby{事}{こと}かな。
%
\ruby{人}{ひと}は
いざ
\ruby{知}{し}らず
\ruby{我}{われ}は
\ruby{我}{わ}が
\ruby{欺}{あざむ}き
\ruby{{\換字{難}}}{がた}き
\ruby{{\換字{情}}}{こゝろ}% 踊り字調整「〻(二の字点、揺すり点)に見えるが(ゝ)」
のありて、
%
\ruby{何}{なん}の
\原本頁{115-11}\改行%
\ruby[g]{理由}{いはれ }とは
\ruby{{\換字{更}}}{さら}に
\ruby{知}{し}らねど、
%
\ruby{神}{かみ}にも
\ruby{憐}{あは}れと
\ruby{思}{おも}はれ
\ruby{佛}{ほとけ}にも
\ruby{憐}{あは}れと
\ruby{思}{おも}はれたき
\ruby[g]{心地}{こゝち }% 踊り字調整「〻(二の字点、揺すり点)に見えるが(ゝ)」
のするなり。
%
\ruby{理}{り}は
\ruby{石}{いし}の
\ruby{如}{ごと}し
\ruby{抂}{ま}ぐ
べからず、
%
\ruby{我}{われ}
これを
\原本頁{116-2}\改行%
\ruby{懷}{いた}きて% ルビ調整(原本通り)(いた)
\ruby{神}{かみ}をも
\ruby{佛}{ほとけ}をも
\ruby{肯}{うけが}はねども、
%
\ruby[g]{{\換字{感}}{\換字{情}}}{こゝろ }は% 踊り字調整「〻(二の字点、揺すり点)に見えるが(ゝ)」
\ruby[<j>]{味}{あぢはひ}の
\ruby{欺}{あざむ}く
べからざるが
\ruby{如}{ごと}く、
%
\ruby{我}{われ}
おのづからに
\ruby{神}{かみ}を
\ruby{戀}{こ}ひ
\ruby{佛}{ほとけ}を
\ruby{慕}{した}はん
とするを
\ruby[g]{如何}{い か }に
すべきや。
%
\ruby{人}{ひと}の
\ruby{戀}{こひ}しき
\ruby{彼}{かれ}も
\ruby[g]{眞實}{まこと }なり、
%
\ruby[||j>]{神}{かみ}
\ruby[||j>]{佛}{ほとけ}の
% \ruby{神佛}{かみ|ほとけ}の
\ruby{頼}{たの}み
\ruby[<j>]{奉}{たてまつ}りたき
\ruby{此}{これ}も
\ruby[g]{眞實}{まこと }なり。
%
\ruby{噫}{あゝ}% 踊り字調整「〻(二の字点、揺すり点)に見えるが(ゝ)」
\ruby{我}{われ}
\ruby[g]{力無}{ちからな}し、
%
\ruby{我}{われ}
\ruby{既}{すで}に
\ruby{我}{わ}が
\ruby{五十子}{い|そ|こ}を
\ruby{思}{おも}ひ
\ruby{棄}{す}て
\ruby{得}{{\換字{𛀁}}}ざるなり、
%
\原本頁{116-6}\改行%
\ruby{我}{われ}
よく
この
\ruby[||j>]{神}{かみ}
\ruby[||j>]{佛}{ほとけ}をば
% \ruby{神佛}{かみ|ほとけ}をば
\ruby{思}{おも}ひ
\ruby{棄}{す}て
\ruby{得}{う}べきや。
%
\ruby{思}{おも}へば
\ruby{我}{われ}ながら
\ruby[g]{覺束}{おぼつか}
\ruby{無}{な}き
\ruby{事}{こと}なるかな!。
%
さはさりながら、
%
さはさりながら。
%
と
\ruby{切}{しきり}に
\ruby[g]{默想}{おもひ }に
\ruby{耽}{ふけ}りし
\ruby{時}{とき}には、
%
\ruby{弘誓深如海}{ぐ|ぜい|しん|によ|かい}、
%
\ruby{歷劫不思議}{れき|がう|ふ|し|ぎ}と
\ruby[g]{老人}{らうじん}の
\ruby{誦}{じゆ}したる
\原本頁{116-9}\改行%
\ruby{聲}{こゑ}を
\ruby{{\換字{猶}}}{なほ}
\ruby{耳}{みゝ}に% 踊り字調整「〻(二の字点、揺すり点)に見えるが(ゝ)」
したりしに、
%
それより
\ruby{兎}{と}せん
\ruby{角}{かく}せんに
\ruby{思}{おも}ひ
\ruby{{\換字{迷}}}{まよ}へる
\ruby{中}{うち}
\改行% 校正作業の簡略化のため
、
%
\原本頁{116-10}\改行%
\ruby[g]{何時}{い つ }の
\ruby{間}{ま}にか
\ruby[g]{瞢然}{うつとり}に% 「瞢然(ぼうぜん)」ぼんやりとしているさま。ぼんやりして愚かなさま。
\footnote{原本通り助詞「に」のまま「瞢然(うつとり)に」とする
(国会図書館 コマ番号63/160 p-116 l-10)}%
\ruby[g]{睡眠}{ねむり }には
\ruby{入}{い}りたるぞや。
%
と
\ruby[g]{水野}{みづの }は
\ruby{自}{みづか}ら
\ruby{私}{ひそか}に
\ruby{慚}{は}ぢたり。
