\Entry{其二十}

% メモ 校正終了 2024-04-22
\原本頁{111-7}%
\ruby{既}{すで}に
\ruby{我}{わ}が
\ruby{言葉}{こと|ば}を
\ruby{戾}{もど}きもせず、
%
また
\ruby{我}{わ}が
\ruby{{\換字{伴}}}{ともな}ふを
\ruby{拒}{こば}みもせねば、
%
\ruby{今}{いま}
\ruby{御堂}{み|だう}に
\ruby{上}{のぼ}りて
\ruby{御{\換字{前}}}{おん|まへ}に
\ruby{至}{いた}れる
\ruby{上}{うへ}は、
%
\ruby{必}{かな}らず
\ruby{復}{また}
\ruby{{\換字{前}}}{さき}の
\ruby{日}{ひ}の
\ruby{{\換字{朝}}}{あさ}の
\ruby{如}{ごと}くに、
%
\原本頁{111-9}\改行%
たとひ
\ruby{御經}{おん|きやう}は
\ruby{誦}{じゆ}せざる
までも、
%
\ruby{掌}{たなぞこ}を
\ruby{合}{あは}せ
\ruby{頭}{かうべ}を
\ruby{下}{さ}げて
\ruby{禮拜}{らい|はい}する
ならんと、
%
\ruby{獨}{ひとり}
\ruby{合點}{が|てん}してや
\ruby{彼}{か}の
\ruby{老人}{らう|じん}は、
%
\ruby{御堂}{み|だう}に
\ruby{上}{のぼ}りて
よりは
\ruby[g]{水野}{みづの}に
\原本頁{112-1}\改行%
\ruby{關}{かま}はず、
%
\ruby{一}{ひと}つは
\ruby{自己}{お|の}が
\ruby{信心}{しん|〴〵}の
\ruby{誠}{まこと}を
\ruby{致}{いた}さん
とするに
\ruby{忙}{いそが}しきが
\ruby{故}{ゆゑ}
もあるべし、
%
\ruby{例}{いつも}の
\ruby{如}{ごと}く
\ruby{御{\換字{前}}}{み|まへ}に
\ruby{蹲}{うづく}まりて、
%
\ruby{先}{ま}ず
\ruby{一心}{いつ|しん}に
\ruby{恭敬禮拜}{きやう|けい|らい|はい}しつ、
%
\原本頁{112-3}\改行%
\ruby{徐々}{しづ|か}に
%\ruby[g]{妙法蓮華經觀世音菩薩普門品第二十五}{めうはうれんげきやうくわんぜおんぼさつふもんぼんだいにじうご}、と
\ruby{妙法蓮華經}{めう|はふ|れん|げ|きやう}% TODO 暫定で「蓮 uf999」とする(参考「蓮 uu84ee」)
\ruby{觀世音菩薩}{くわん|ぜ|おん|ぼ|さつ}
\ruby{普門品}{ふ|もん|ぼん}
\ruby{第二十五}{だい|に|じう|ご}、
%
と
\ruby{老}{お}いたる
\ruby{聲}{こゑ}の
\ruby{低}{ひく}く
\ruby{誦}{じゆ}し
\ruby{出}{いだ}したり。

\原本頁{112-5}%
\ruby{{\換字{朝}}}{あさ}の
\ruby{氣}{き}は
\ruby{何}{なん}となく
\ruby{心}{こ〻ろ}% 原本通り「〻(二の字点、揺すり点)」
をして
\ruby{粛然}{しゆく|ぜん}
たらしめて、
%
\ruby{廣}{ひろ}き
\ruby{御堂}{み|だう}の
\ruby{内}{うち}の
\ruby{人}{ひと}
\ruby{無}{な}き
\ruby{物}{もの}
\ruby{靜}{しづ}かさは
\ruby{自然}{おの|づ}と
\ruby{胸}{むね}の
\ruby{中}{うち}を
\ruby{淸々}{すが|〳〵}し
からしむ。
%
\ruby{今日}{け|ふ}は
\ruby{御佛}{み|ほとけ}を
\原本頁{112-7}\改行%
\ruby{拜}{をが}み
\ruby{奉}{たてまつ}り
もせず、
%
さりとて
\ruby{{\換字{又}}}{また}
\ruby{御佛}{み|ほとけ}より
\ruby{反}{そむ}き
\ruby{去}{さ}りもせず、
%
たゞ% TODO 原本の「二の字点、揺すり点」に濁点のグリフが見つからないので「ゞ」
ただ% 原本は行末行頭禁則により非踊り字表記
\ruby{從順}{すな|ほ}なる
\ruby{兒童}{こ|ども}の、
%
\ruby{心}{こ〻ろ}に% 原本通り「〻(二の字点、揺すり点)」
\ruby{物}{もの}
\ruby{無}{な}きが
\ruby{如}{ごと}く、
%
\ruby{牽}{ひ}かれたる
ま〻に% 原本通り「〻(二の字点、揺すり点)」
\ruby{此處}{こ|〻}に% 原本通り「〻(二の字点、揺すり点)」
\ruby{來}{きた}りて、
%
\ruby{此處}{こ|〻}に% 原本通り「〻(二の字点、揺すり点)」
\ruby{其}{その}
\ruby{儘}{ま〻}% 原本通り「〻(二の字点、揺すり点)」
\ruby{止}{とゞま}れる% TODO 原本の「二の字点、揺すり点」に濁点のグリフが見つからないので「ゞ」
\ruby[g]{水野}{みづの}は、
%
\ruby{身{\換字{近}}}{み|ぢか}なりし
\ruby{圓柱}{まる|ばしら}の
\ruby{太}{ふと}きに
\原本頁{112-10}\改行%
\ruby{憑}{よ}りて、
%
\ruby{風}{かぜ}
\ruby{吹}{ふ}かぬ
\ruby{間}{ま}を
\ruby{大{\換字{空}}}{おほ|ぞら}に
\ruby{高}{たか}く
\ruby{懸}{か〻}れる% 原本通り「〻(二の字点、揺すり点)」
\ruby{孤雲}{ひとつ|ぐも}の、
%
\ruby{何}{なに}に
\ruby{着}{つ}くとも
\ruby{無}{な}き
\ruby{思}{おもひ}に、
%
\ruby{嗒焉}{たう|{\換字{𛀁}}ん}として
\ruby{獨}{ひと}り
\ruby{{\換字{空}}}{むな}しく
\ruby{立}{た}てり。

\原本頁{113-1}%
\ruby{老}{お}いたる
\ruby{人}{ひと}の
\ruby{誦}{じゆ}する
\ruby{經}{きやう}の、
%
\ruby{其}{その}
\ruby{意}{こ〻ろ}は% 原本通り「〻(二の字点、揺すり点)」
\ruby{曉}{さと}らる〻% 原本通り「〻(二の字点、揺すり点)」
\ruby{時}{とき}あり
\ruby{曉}{さと}らえざる
\ruby{時}{とき}あれど、
%
\ruby{其}{その}
\ruby{聲}{こゑ}は
\ruby{波瀾}{な|み}
\ruby{無}{な}く
\ruby{山坂}{や|ま}
\ruby{無}{な}くして
\ruby{一條}{ひと|すぢ}の
\ruby{絲}{いと}を
\ruby{畫}{ひ}ける
にも
\ruby{似}{に}て
\ruby{{\換字{平}}}{たひ}らかなるに、
%
\ruby{聞}{き}き
\ruby{居}{ゐ}る
\ruby{我}{わ}が
\ruby{心}{こ〻ろ}は% 原本通り「〻(二の字点、揺すり点)」
\ruby{刻々}{こく|〳〵}に
\ruby{安}{やす}まり
\ruby{行}{ゆ}き、
%
\ruby{何}{なん}とは
\原本頁{113-4}\改行%
\ruby{無}{な}けれど
\ruby{引}{ひ}き
\ruby{入}{い}れらる〻% 原本通り「〻(二の字点、揺すり点)」
やうに
おぼえて、
%
\ruby{知}{し}らず
\ruby{識}{し}らず
\ruby{無念}{む|ねん}
\ruby{無想}{む|さう}の
\ruby{境}{さかひ}に
\ruby{入}{い}る
\ruby{折}{をり}しも、
%
\ruby{人}{ひと}の
\ruby{下駄}{げ|た}の
\ruby{音}{おと}に
\ruby{不圖}{ふ|と}
\ruby{驚}{おどろ}きて、
%
\ruby{見}{み}れば
\ruby{何時}{い|つ}の
\ruby{間}{ま}にやら
\ruby{三十}{さん|じう}
ばかり
なる
\ruby{女}{をんな}の、
%
\ruby{老人}{らう|じん}と
\ruby{並}{なら}びて
\ruby{禮拜}{らい|はい}
なし
\ruby{居}{を}り、
%
\原本頁{113-7}\改行%
\ruby{老人}{らう|じん}の
\ruby{誦經}{じゆ|きやう}は
\ruby{今}{いま}や
\ruby{{\換字{終}}}{をは}らん
として、
%
\ruby{具一切功德}{ぐ|いつ|さい|く|どく}、
%
\ruby{慈眼視衆生}{じ|げん|じ|しゆ|じやう}と、
%
\ruby{偈}{げ}の
\ruby{末}{すゑ}
のところを
\ruby{誦}{よ}み
\ruby{居}{ゐ}たり。
%
\ruby{是}{こ}は
\ruby{不覺}{ふ|かく}なりし
\ruby{愚}{おろか}なりし!。
%
\ruby{身}{み}は
こそ
\ruby{動}{うご}かさゞりつれ% TODO 原本の「二の字点、揺すり点」に濁点のグリフが見つからないので「ゞ」
\ruby{心}{こ〻ろ}の% 原本通り「〻(二の字点、揺すり点)」
\ruby{内}{うち}には、
%
\ruby{吾}{わ}が
\ruby{兒}{こ}の
\ruby{可憐}{か|はゆ}いのに
\ruby{理屈}{り|くつ}も
\原本頁{113-10}\改行%
\ruby{無}{な}く、
%
\ruby{思}{おも}ふ
\ruby{人}{ひと}の
\ruby{大切}{だい|じ}なのに
\ruby{理屈}{り|くつ}も
\ruby{無}{な}ければ、
%
\ruby{神樣佛樣}{かみ|さま|ほとけ|さま}に
\ruby{御縋}{お|すが}り
\ruby{申}{まを}すのにも、
%
\ruby{何}{なん}の
\ruby{理屈}{り|くつ}も
\ruby{無}{な}いなれど、
%
それも
\ruby[g]{眞實}{まこと}
なれば、
%
\ruby{此}{これ}も
\原本頁{114-1}\改行%
\ruby[g]{眞實}{まこと}
で、
%
\ruby{理屈}{り|くつ}の
\ruby{要}{い}らない
ほどの
\ruby[g]{眞實}{まこと}
!\inhibitglue{}と
\ruby{云}{い}ひたる
\ruby{此}{こ}の
\ruby{老人}{らう|じん}の
\ruby{言葉}{こと|ば}を
\ruby{味}{あぢ}はひて、
%
\ruby{實}{げ}に
\ruby{云}{い}はるれば
%
\ruby{其}{そ}の
\ruby{如}{ごと}くなり、
%
\ruby{我}{わ}が
\ruby{彼}{か}の
\ruby{人}{ひと}を
\ruby{思}{おも}ひ
\ruby{思}{おも}ふ
\ruby{心}{こ〻ろ}に、% 原本通り「〻(二の字点、揺すり点)」
%
そも〳〵
\ruby{何}{なん}の
\ruby{理由}{いは|れ}の
ありや、
%
\ruby{何}{なん}の
\ruby{理由}{わ|け}とは
\ruby{我}{われ}も
\ruby{知}{し}らず、
%
たゞ% TODO 原本の「二の字点、揺すり点」に濁点のグリフが見つからないので「ゞ」
\ruby{我}{われ}と
\ruby{我}{わ}が
\ruby{欺}{あざむ}き
\ruby{{\換字{難}}}{がた}き
\ruby{{\換字{情}}}{こ〻ろ}の% 原本通り「〻(二の字点、揺すり点)」
\ruby{萌}{も}えに
\ruby{萌}{も}え
\ruby{出}{い}づるを
\ruby{抑}{おさ}へ
\ruby{得}{{\換字{𛀁}}}ざるぞ
\ruby[g]{眞實}{まこと}なる!。
%
\ruby{思}{おも}ふて
\ruby{思}{おも}はる〻% 原本通り「〻(二の字点、揺すり点)」
\ruby{身}{み}ならば
こそ、
%
\ruby{不{\換字{運}}}{ふ|うん}にして
\ruby{我}{われ}
\原本頁{114-6}\改行%
\ruby{拙}{つたな}く
\ruby{生}{うま}れ
\ruby{來}{き}て、
%
\ruby{思}{おも}へば
\ruby{思}{おも}ふほど
\ruby{{\換字{嫌}}}{きら}はる〻% 原本通り「〻(二の字点、揺すり点)」
\ruby{身}{み}の、
%
\ruby{思}{おも}ふて
\ruby{甲{\換字{斐}}}{か|ひ}
\ruby{無}{な}き
\ruby{事}{こと}なれば、
%
\ruby{自}{みづか}ら
\ruby{斷念}{あき|ら}め
\ruby{思}{おも}ひ
\ruby{切}{き}りて、
%
\ruby{忘}{わす}れ
\ruby{果}{は}てん
こそ
\ruby{人}{ひと}のため
\ruby{身}{み}のため
なれ、
%
\ruby{我}{わ}が
\ruby{爲}{な}す
\ruby{事}{こと}
\ruby{言}{い}う
\ruby{事}{こと}は
\ruby{何}{なに}から
\ruby{何}{なに}まで、
%
\ruby{{\換字{情}}}{なさけ}なくも
\ruby{彼}{か}の
\原本頁{114-9}\改行%
\ruby{人}{ひと}に
\ruby{厭}{いと}はる〻% 原本通り「〻(二の字点、揺すり点)」
ながら、
%
\ruby{思}{おも}ひ
\ruby{忘}{わす}る〻% 原本通り「〻(二の字点、揺すり点)」
といふ
\ruby{此事}{こ|れ}
ばかりは、
%
\ruby{必}{かなら}ず
\ruby{彼}{か}の
\ruby{人}{ひと}に
\ruby{悅}{よろこ}ばるべければ、
%
\ruby{果敢}{は|か}なく
\ruby{悲}{かな}しき
\ruby{限}{かぎ}りなれど、
%
とても
かくても
\ruby{味氣}{あぢ|き}
\ruby{無}{な}き
\ruby{我}{わ}が
\ruby{一生}{いつ|しやう}の
\ruby{思}{おも}ひ
\ruby{出}{いで}に、% 原本通りに「(い)で」
%
せめては
\ruby{男兒}{をと|こ}らしう
ふつつりと% 原本通り行末行頭禁則により非踊り字表記
\ruby{諦}{あきら}めて、
%
うるさく
\ruby{纏繞}{まつ|は}る
\ruby{蔓葛}{つた|かつら}の% 原本では「蔦(つた)」でなく「蔓(つる)」
\ruby{離}{はな}れて
\ruby{去}{さ}りし
\ruby{嬉}{うれ}しさよと、
%
\原本頁{115-2}\改行%
\ruby{彼}{か}の
\ruby{人}{ひと}に
\ruby{安}{やす}き
\ruby{思}{おもひ}を
させん、
%
\ruby{人}{ひと}も
\ruby{見}{み}ず
\ruby{人}{ひと}をも
\ruby{見}{み}ざる
\ruby{深}{ふか}き
\ruby{山}{やま}の
\ruby{巖}{いは}の
\ruby{罅隙}{はざ|ま}に
\ruby{我}{われ}
\ruby{一人}{ひと|り}
\ruby{入}{い}りて、
%
\ruby{誰}{たれ}
\ruby{憚}{はゞか}らず% 「憚 は(ゞ)か」% TODO 原本の「二の字点、揺すり点」に濁点のグリフが見つからないので「ゞ」
\ruby{思}{おも}ふさま
\ruby{泣}{な}きて、
%
\ruby{其}{その}
\ruby[||j>]{淚}{なみだ}の
\ruby{乾}{かは}き
\ruby{聲}{こゑ}の
\ruby{枯}{か}れん
\ruby{時}{とき}
\ruby{我}{われ}
\ruby{{\換字{即}}}{すなは}ち
\ruby{此}{この}
\ruby{世}{よ}を
\ruby{去}{さ}らば
\ruby{濟}{す}むべき
\ruby{事}{こと}なるをや!、
%
と
\ruby{幾度}{いく|たび}か〳〵
\ruby{思}{おも}ひしかど、
%
\ruby{諦}{あきら}めても
\ruby{諦}{あきら}めても
\ruby{諦}{あきら}め
\ruby{得}{{\換字{𛀁}}}ず、
%
\ruby{彼}{か}の
\ruby{人}{ひと}を
\ruby{背後}{うし|ろ}にして
\ruby{千里}{せん|り}の
\ruby{{\換字{遠}}}{とほ}きに
\ruby{身}{み}を
\ruby{隱}{かく}し
\ruby{棄}{す}てん
とする
\ruby{意}{こ〻ろ}は% 原本通り「〻(二の字点、揺すり点)」
ありても、
%
\ruby{彼}{か}の
\原本頁{115-7}\改行%
\ruby{人}{ひと}より
\ruby{距}{へだ}たらん
とすれば
\ruby{一歩}{いつ|ぽ}も
\ruby{去}{さ}り
\ruby{得}{{\換字{𛀁}}}ず、
%
\ruby{我}{わ}が
\ruby{心}{こ〻ろ}の% 原本通り「〻(二の字点、揺すり点)」
\ruby{我}{わ}が
\ruby{心}{こ〻ろ}に% 原本通り「〻(二の字点、揺すり点)」
\ruby{任}{まか}せずして、
%
あだに
\ruby{苦}{くるし}み
あだに
\ruby{惱}{なや}むは、
%
たゞ% TODO 原本の「二の字点、揺すり点」に濁点のグリフが見つからないので「ゞ」
\ruby{我}{われ}と
\ruby{我}{わ}が
\ruby{欺}{あざむ}きがたき
\原本頁{115-9}\改行%
\ruby{{\換字{情}}}{こ〻ろ}の% 原本通り「〻(二の字点、揺すり点)」
\ruby{萌}{も}えに
\ruby{萌}{も}ゆればなり。
%
おもへば
\ruby{神佛}{かみ|ほとけ}を
\ruby{頼}{たの}み
\ruby{奉}{たてまつ}るも
\ruby{實}{げ}に
\ruby{似}{に}たる
\ruby{事}{こと}かな。
%
\ruby{人}{ひと}は
いざ
\ruby{知}{し}らず
\ruby{我}{われ}は
\ruby{我}{わ}が
\ruby{欺}{あざむ}き
\ruby{{\換字{難}}}{がた}き
\ruby{{\換字{情}}}{こ〻ろ}% 原本通り「〻(二の字点、揺すり点)」
のありて、
%
\ruby{何}{なん}の
\原本頁{115-11}\改行%
\ruby{理由}{いは|れ}とは
\ruby{{\換字{更}}}{さら}に
\ruby{知}{し}らねど、
%
\ruby{神}{かみ}にも
\ruby{憐}{あは}れと
\ruby{思}{おも}はれ
\ruby{佛}{ほとけ}にも
\ruby{憐}{あは}れと
\ruby{思}{おも}はれたき
\ruby{心地}{こ〻|ち}% 原本通り「〻(二の字点、揺すり点)」
のするなり。
%
\ruby{理}{り}は
\ruby{石}{いし}の
\ruby{如}{ごと}し
\ruby{抂}{ま}ぐ
べからず、
%
\ruby{我}{われ}
これを
\原本頁{116-2}\改行%
\ruby{懷}{いだ}きて
\ruby{神}{かみ}をも
\ruby{佛}{ほとけ}をも
\ruby{肯}{うけが}はねども、
%
\ruby{感{\換字{情}}}{こ〻|ろ}は% 原本通り「〻(二の字点、揺すり点)」
\ruby{味}{あぢはひ}の
\ruby{欺}{あざむ}く
べからざるが
\ruby{如}{ごと}く、
%
\ruby{我}{われ}
おのづからに
\ruby{神}{かみ}を
\ruby{戀}{こ}ひ
\ruby{佛}{ほとけ}を
\ruby{慕}{した}はん
とするを
\ruby{如何}{い|か}に
すべきや。
%
\ruby{人}{ひと}の
\ruby{戀}{こひ}しき
\ruby{彼}{かれ}も
\ruby[g]{眞實}{まこと}なり、
%
\ruby{神佛}{かみ|ほとけ}の
\ruby{頼}{たの}み
\ruby{奉}{たてまつ}りたき
\ruby{此}{これ}も
\ruby[g]{眞實}{まこと}なり。
%
\ruby{噫}{あ〻}% 原本通り「〻(二の字点、揺すり点)」
\ruby{我}{われ}
\ruby{力無}{ちから|な}し、
%
\ruby{我}{われ}
\ruby{既}{すで}に
\ruby{我}{わ}が
\ruby[g]{五十子}{いそこ}を
\ruby{思}{おも}ひ
\ruby{棄}{す}て
\ruby{得}{{\換字{𛀁}}}ざるなり、
%
\原本頁{116-6}\改行%
\ruby{我}{われ}
よく
この
\ruby{神佛}{かみ|ほとけ}をば
\ruby{思}{おも}ひ
\ruby{棄}{す}て
\ruby{得}{う}べきや。
%
\ruby{思}{おも}へば
\ruby{我}{われ}ながら
\ruby{覺束}{おぼ|つか}
\ruby{無}{な}き
\ruby{事}{こと}なるかな!。
%
さはさりながら、
%
さはさりながら。
%
と
\ruby{切}{しきり}に
\ruby{默想}{おも|ひ}に
\ruby{耽}{ふけ}りし
\ruby{時}{とき}には、
%
\ruby{弘誓深如海}{ぐ|ぜい|しん|によ|かい}、
%
\ruby{歷劫不思議}{れき|がう|ふ|し|ぎ}と
\ruby{老人}{らう|じん}の
\ruby{誦}{じゆ}したる
\原本頁{116-9}\改行%
\ruby{聲}{こゑ}を
\ruby{{\換字{猶}}}{なほ}
\ruby{耳}{み〻}に% 原本通り「〻(二の字点、揺すり点)」
したりしに、
%
それより
\ruby{兎}{と}せん
\ruby{角}{かく}せんに
\ruby{思}{おも}ひ
\ruby{{\換字{迷}}}{まよ}へる
\ruby{中}{うち}、
%
\ruby{何時}{い|つ}の
\ruby{間}{ま}にか
\ruby{瞢然}{うつ|とり}に% 「瞢然(ぼうぜん)」ぼんやりとしているさま。ぼんやりして愚かなさま。
\ruby{睡眠}{ねむ|り}には
\ruby{入}{い}りたるぞや。
%
と
\ruby[g]{水野}{みづの}は
\ruby{自}{みづか}ら
\ruby{私}{ひそか}に
\ruby{慚}{は}ぢたり。
