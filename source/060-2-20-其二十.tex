\Entry{其二十}

\ruby{既}{すで}に
\ruby{我}{わ}が
\ruby{言葉}{こと|ば}を
\ruby{{\換字{㑦}}}{もど}きもせず、また
\ruby{我}{わ}が
\ruby{{\換字{伴}}}{ともな}ふを
\ruby{拒}{こば}みもせねば、
\ruby{今}{いま}
\ruby{御堂}{み|だう}に
\ruby{上}{のぼ}りて
\ruby{御前}{おん|まへ}に
\ruby{至}{いた}れる
\ruby{上}{うへ}は、
\ruby{必}{かな}らず
\ruby{復前}{また|さき}の
\ruby{日}{ひ}の
\ruby{{\換字{朝}}}{あさ}の
\ruby{如}{ごと}くに、たとひ
\ruby{御經}{おん|きやう}は
\ruby{誦}{じゆ}せざるまでも、
\ruby{掌}{たなぞこ}を
\ruby{合}{あは}せ
\ruby{頭}{かうべ}を
\ruby{下}{さ}げて
\ruby{禮拜}{らい|はい}するならんと、
\ruby{獨合點}{ひとり|が|てん}してや
\ruby{彼}{か}の
\ruby{老人}{らう|じん}は、
\ruby{御堂}{み|だう}に
\ruby{上}{のぼ}りてよりは
\ruby{水野}{みづ|の}に
\ruby{關}{かま}はず、
\ruby{一}{ひと}つは
\ruby{自己}{お|の}が
\ruby{信心}{しん|〴〵}の
\ruby{誠}{まこと}を
\ruby{致}{いた}さんとするに
\ruby{忙}{いそが}しきが
\ruby{故}{ゆえ}もあるべし、
\ruby{例}{いつも}の
\ruby{如}{ごと}く
\ruby{御前}{み|まへ}に
\ruby{蹲}{うづく}まりて、
\ruby{先}{ま}ず
\ruby{一心}{いつ|しん}に
\ruby{恭敬禮拜}{きやう|けい|らい|はい}しつ、
\ruby{徐々}{しづ|か}に
%\ruby[g]{妙法{\換字{蓮}}華經觀世音菩薩普門品第二十五}{めうはうれんげきやうくわんぜおんぼさつふもんぼんだいにじうご}、と
\ruby{妙法{\換字{蓮}}華經}{めう|はう|れん|げき|やう}
\ruby{觀世音菩薩}{くわ|んぜ|おん|ぼさ|つ}
\ruby{普門品}{ふ|もん|ぼん}
\ruby{第二十五}{だい|に|じう|ご}、と
\ruby{老}{お}いたる
\ruby{聲}{こゑ}の
\ruby{低}{ひく}く
\ruby{誦}{じゆ}し
\ruby{出}{いだ}しけり。

\ruby{{\換字{朝}}}{あさ}の
\ruby{氣}{き}は
\ruby{何}{なん}となく
\ruby{心}{こゝろ}をして
\ruby{粛然}{しゆく|ぜん}たらしめて、
\ruby{廣}{ひろ}き
\ruby{御堂}{み|だう}の
\ruby{内}{うち}の
\ruby{人無}{ひと|な}き
\ruby{物靜}{もの|しづ}かさは
\ruby{自然}{おの|づ}と
\ruby{胸}{むね}の
\ruby{中}{うち}を
\ruby{淸々}{すが|〳〵}しからしむ。
\ruby{今日}{け|ふ}は
\ruby{御佛}{みほ|とけ}を
\ruby{拜}{をが}み
\ruby{奉}{たてまつ}りもせず、さりとて
\ruby{{\換字{又}}}{また}
\ruby{御佛}{みほ|とけ}より
\ruby{反}{そむ}き
\ruby{去}{さ}りもせず、たゞたゞ
\ruby{從順}{すな|ほ}なる
\ruby{兒童}{こ|ども}の、
\ruby{心}{こゝろ}に
\ruby{物無}{もの|な}きが
\ruby{如}{ごと}く、
\ruby{牽}{ひ}かれたるまゝに
\ruby{此處}{こ|ゝ}に
\ruby{來}{きた}りて、
\ruby{此處}{こ|ゝ}に
\ruby{其儘止}{その|まゝ|とゞま}れる
\ruby{水野}{みづ|の}は、
\ruby{身{\換字{近}}}{み|ぢか}なりし
\ruby{圓柱}{まる|ばしら}の
\ruby{太}{ふと}きに
\ruby{憑}{よ}りて、
\ruby{風吹}{かぜ|ふ}かぬ
\ruby{間}{ま}を
\ruby{大空}{おほ|ぞら}に
\ruby{高}{たか}く
\ruby{懸}{かゝ}れる
\ruby{孤雲}{ひとつ|ぐも}の、
\ruby{何}{なに}に
\ruby{着}{つ}くとも
\ruby{無}{な}き
\ruby{思}{おもひ}に、
\ruby{嗒焉}{たう|\換字{江}ん}として
\ruby{獨}{ひと}り
\ruby{空}{むな}しく
\ruby{立}{た}てり。

\ruby{老}{お}いたる
\ruby{人}{ひと}の
\ruby{誦}{じゆ}する
\ruby{經}{きやう}の、
\ruby{其意}{その|こゝろ}は
\ruby{曉}{さと}らるゝ
\ruby{時}{とき}あれど、
\ruby{其聲}{その|こゑ}は
\ruby{波瀾無}{な|み|な}く
\ruby{山坂無}{や|ま|な}くして
\ruby{一條}{ひと|すぢ}の
\ruby{絲}{いと}を
\ruby{畫}{ひ}けるにも
\ruby{似}{に}て
\ruby{{\換字{平}}}{たひ}らかなるに、
\ruby{聞}{き}き
\ruby{居}{ゐ}る
\ruby{我}{わ}が
\ruby{心}{こゝろ}は
\ruby{刻々}{こく|〳〵}に
\ruby{安}{やす}まり
\ruby{行}{ゆ}き、
\ruby{何}{なん}とは
\ruby{無}{な}けれど
\ruby{引}{ひ}き
\ruby{入}{い}れらる〻やうにおぼえて、
\ruby{知}{し}らず
\ruby{識}{し}らず
\ruby{無念無想}{む|ねん|む|さう}の
\ruby{境}{さかひ}に
\ruby{入}{い}る
\ruby{折}{をり}しも、
\ruby{人}{ひと}の
\ruby{下駄}{げ|た}の
\ruby{音}{おと}に
\ruby{不圖驚}{ふ|と|おどろ}きて、
\ruby{見}{み}れば
\ruby{何時}{い|つ}の
\ruby{間}{ま}にや
\ruby{三十}{さん|じう}ばかりなる
\ruby{女}{をんな}の、
\ruby{老人}{らう|じん}と
\ruby{並}{なら}びて
\ruby{禮拜}{らい|はい}なし
\ruby{居}{を}り、
\ruby{老人}{らう|じん}の
\ruby{誦經}{じゆ|きやう}は
\ruby{今}{いま}や
\ruby{{\換字{終}}}{をは}らんとして、
\ruby{具一切功德}{ぐ|いつ|さい|く|どく}、
\ruby{慈眼視衆生}{じ|げん|じ|しゆ|じやう}と、
\ruby{偈}{げ}の
\ruby{末}{すゑ}のところを
\ruby{誦}{よ}み
\ruby{居}{い}たり。
\ruby{是}{こ}は
\ruby{不覺}{ふ|かく}なりし
\ruby{愚}{おろか}なりし!。
\ruby{身}{み}はこそ
\ruby{動}{うご}かさゞりつれ
\ruby{心}{こゝろ}の
\ruby{内}{うち}には、
\ruby{吾}{わ}が
\ruby{兒}{こ}の
\ruby{可憐}{かは|ゆ}いのに
\ruby{理屈}{り|くつ}も
\ruby{無}{な}く、
\ruby{思}{おも}ふ
\ruby{人}{ひと}の
\ruby{大切}{だい|じ}なのに
\ruby{理屈}{り|くつ}も
\ruby{無}{な}ければ、
\ruby{神樣佛樣}{かみ|さま|ほとけ|さま}に
\ruby{御縋}{お|すが}り
\ruby{申}{まを}すのにも、
\ruby{何}{なん}の
\ruby{理屈}{り|くつ}も
\ruby{無}{な}いなれど、それも
\ruby{眞實}{ま|こと}なれば、
\ruby{此}{これ}も
\ruby{眞實}{ま|こと}で、
\ruby{理屈}{り|くつ}の
\ruby{要}{い}らないほどの
\ruby{眞實}{ま|こと}!\inhibitglue と
\ruby{云}{い}ひたる
\ruby{此}{こ}の
\ruby{老人}{らう|じん}の
\ruby{言葉}{こと|ば}を
\ruby{味}{あぢ}はひて、
\ruby{實}{げ}に
\ruby{云}{い}はるれば、
\ruby{其}{そ}の
\ruby{如}{ごと}くなり、
\ruby{我}{わ}が
\ruby{彼}{か}の
\ruby{人}{ひと}を
\ruby{思}{おも}ひ
\ruby{思}{おも}ふ
\ruby{心}{こゝろ}に、そも〳〵
\ruby{何}{なん}の
\ruby{理由}{いは|れ}のありや、
\ruby{何}{なん}の
\ruby{理由}{わ|け}とは
\ruby{我}{われ}も
\ruby{知}{し}らず、たゞ
\ruby{我}{われ}と
\ruby{我}{わ}が
\ruby{欺}{あざむ}き
\ruby{難}{がた}き
\ruby{{\換字{情}}}{こゝろ}の
\ruby{萌}{も}えに
\ruby{萌}{も}え
\ruby{出}{い}づるを
\ruby{抑}{おさ}へ
\ruby{得}{\換字{江}}ざるぞ
\ruby{眞實}{ま|こと}なる!。
\ruby{思}{おも}ふて
\ruby{思}{おも}はる〻
\ruby{身}{み}ならばこそ、
\ruby{不{\換字{運}}}{ふ|うん}にして
\ruby{我拙}{われ|つたな}く
\ruby{生}{うま}れ
\ruby{來}{き}て、
\ruby{思}{おも}へば
\ruby{思}{おも}ふほど
\ruby{{\換字{嫌}}}{きら}はる〻
\ruby{身}{み}の、
\ruby{思}{おも}ふて
\ruby{甲{\換字{斐}}無}{か|ひ|な}き
\ruby{事}{こと}なれば、
\ruby{自}{みづか}ら
\ruby{斷念}{あき|ら}め
\ruby{思}{おも}ひ
\ruby{切}{き}りて、
\ruby{忘}{わす}れ
\ruby{果}{は}てんこそ
\ruby{人}{ひと}のため
\ruby{身}{み}のためなれ、
\ruby{我}{わ}が
\ruby{爲}{な}す
\ruby{事言}{こと|い}う
\ruby{事}{こと}は
\ruby{何}{なに}から
\ruby{何}{なに}まで、
\ruby{{\換字{情}}}{なさけ}なくも
\ruby{彼}{か}の
\ruby{人}{ひと}に
\ruby{厭}{いと}はる〻ながら、
\ruby{思}{おも}ひ
\ruby{忘}{わす}る〻といふ
\ruby{此事}{こ|れ}ばかりは、
\ruby{必}{かなら}ず
\ruby{彼}{か}の
\ruby{人}{ひと}に
\ruby{{\換字{悅}}}{よろこ}ばるければ、
\ruby{果敢}{は|か}なく
\ruby{悲}{かな}しき
\ruby{限}{かぎ}りなれど、とてもかくても
\ruby{味氣無}{あぢ|き|な}き
\ruby{我}{わ}が
\ruby{一生}{いつ|しやう}の
\ruby{思}{おも}ひ
\ruby{出}{で}に、せめては
\ruby{男兒}{をと|こ}らしうふつつりと
\ruby{諦}{あきら}めて、うるさく
\ruby{纏繞}{まつ|は}る
\ruby{蔦葛}{つた|かつら}の
\ruby{離}{はな}れて
\ruby{去}{さ}りし
\ruby{嬉}{うれ}しさよと、
\ruby{彼}{か}の
\ruby{人}{ひと}に
\ruby{安}{やす}き
\ruby{思}{おもひ}をさせん、
\ruby{人}{ひと}も
\ruby{見}{み}ず
\ruby{人}{ひと}をも
\ruby{見}{み}ざる
\ruby{深}{ふか}き
\ruby{山}{やま}の
\ruby{巖}{いは}の
\ruby{罅隙}{はざ|ま}に
\ruby{我}{われ}
\ruby{一人}{ひと|り}
\ruby{入}{い}りて、
\ruby{誰憚}{たれ|はゞか}らず
\ruby{思}{おも}ふさま
\ruby{泣}{な}きて、
\ruby{其淚}{その|なみだ}の
\ruby{乾}{かは}き
\ruby{聲}{こゑ}の
\ruby{枯}{か}れん
\ruby{時我即}{とき|われ|すなは}ち
\ruby{此世}{この|よ}を
\ruby{去}{さ}らば
\ruby{濟}{す}むべき
\ruby{事}{こと}なるをや!、と
\ruby{幾度}{いく|たび}か〳〵
\ruby{思}{おも}ひしかど、
\ruby{諦}{あきら}めても
\ruby{諦}{あきら}めても
\ruby{諦}{あきら}め
\ruby{得}{\換字{江}}ず、
\ruby{彼}{か}の
\ruby{人}{ひと}を
\ruby{背後}{うし|ろ}にして
\ruby{千里}{せん|り}の
\ruby{{\換字{遠}}}{とほ}きに
\ruby{身}{み}を
\ruby{隠}{かく}し
\ruby{棄}{す}てんとする
\ruby{意}{こゝろ}はありても、
\ruby{彼}{か}の
\ruby{人}{ひと}より
\ruby{距}{へだ}たらんとすれば
\ruby{一歩}{いつ|ぽ}も
\ruby{去}{さ}り
\ruby{得}{\換字{江}}ず、
\ruby{我}{わ}が
\ruby{心}{こゝろ}の
\ruby{我}{わ}が
\ruby{心}{こゝろ}に
\ruby{任}{まか}せずして、あだに
\ruby{苦}{くるし}みあだに
\ruby{惱}{なや}むは、たゞ
\ruby{我}{われ}と
\ruby{我}{わ}が
\ruby{欺}{あざむ}きがたき
\ruby{{\換字{情}}}{こゝろ}の
\ruby{萌}{も}えに
\ruby{萌}{も}ゆればなり。
おもへば
\ruby{神佛}{かみ|ほとけ}を
\ruby{頼}{たの}み
\ruby{奉}{たてまつ}るも
\ruby{實}{げ}に
\ruby{似}{に}たる
\ruby{事}{こと}かな。
\ruby{人}{ひと}はいざ
\ruby{知}{し}らず
\ruby{我}{われ}は
\ruby{我}{わ}が
\ruby{欺}{あざむ}き
\ruby{難}{がた}き
\ruby{{\換字{情}}}{こゝろ}のありて、
\ruby{何}{なん}の
\ruby{理由}{いは|れ}とは
\ruby{更}{さら}に
\ruby{知}{し}らねど、
\ruby{神}{かみ}にも
\ruby{憐}{あは}れと
\ruby{思}{おも}はれたき
\ruby{心地}{こゝ|ち}のするなり。
\ruby{理}{り}は
\ruby{石}{いし}の
\ruby{如}{ごと}し
\ruby{抂}{ま}ぐべからず、
\ruby{我}{われ}これを
\ruby{懷}{いだ}きて
\ruby{神}{かみ}をも
\ruby{佛}{ほとけ}をも
\ruby{肯}{うけが}はねども、
\ruby{感{\換字{情}}}{こゝ|ろ}は
\ruby{味}{あじはひ}の
\ruby{欺}{あざむ}くべからざるが
\ruby{如}{ごと}く、
\ruby{我}{われ}おのづからに
\ruby{神}{かみ}を
\ruby{戀}{こ}ひ
\ruby{佛}{ほとけ}を
\ruby{慕}{した}はんとするを
\ruby{如何}{い|か}にすべきや。
\ruby{人}{ひと}の
\ruby{戀}{こひ}しき
\ruby{彼}{かれ}も
\ruby{眞實}{まこ|と}なり、
\ruby{神佛}{かみ|ほとけ}の
\ruby{頼}{たの}み
\ruby{奉}{たてまつ}りたき
\ruby{此}{これ}も
\ruby{眞實}{まこ|と}なり。
\ruby{噫我力無}{あゝ|われ|ちから|な}し、
\ruby{我既}{われ|すで}に
\ruby{我}{わ}が
\ruby{五十子}{い|そ|こ}を
\ruby{思}{おも}ひ
\ruby{棄}{す}て
\ruby{得}{\換字{江}}ざるなり、
\ruby{我}{われ}よくこの
\ruby{神佛}{かみ|ほとけ}をば
\ruby{思}{おも}ひ
\ruby{棄}{す}て
\ruby{得}{う}べきや。
\ruby{思}{おも}へば
\ruby{我}{われ}ながら
\ruby{覺束無}{おぼ|つか|な}き
\ruby{事}{こと}なるかな!。
さはさりながら、さはさりながら。
と
\ruby{切}{しきり}に
\ruby{默想}{おも|ひ}に
\ruby{耽}{ふけ}りし
\ruby{時}{とき}には、
\ruby{弘誓深如海}{ぐ|ぜい|しん|によ|かい}、
\ruby{歷劫不思議}{れき|がう|ふ|し|ぎ}と
\ruby{老人}{らう|じん}の
\ruby{誦}{じゆ}したる
\ruby{聲}{こゑ}を
\ruby{{\換字{猶}}}{なほ}
\ruby{耳}{みゝ}にしたりしに、それより
\ruby{兎}{と}せん
\ruby{角}{かく}せんに
\ruby{思}{おも}ひ
\ruby{{\換字{迷}}}{まよ}へる
\ruby{中}{うち}、
\ruby{何時}{い|つ}の
\ruby{間}{ま}にか
\ruby{瞢然}{うつ|とり}と
\ruby{睡眠}{ねむ|り}には
\ruby{入}{い}りたるぞや。
と
\ruby{水野}{みづ|の}は
\ruby{自}{みづか}ら
\ruby{私}{ひそか}に
\ruby{慚}{は}ぢたり。

