\Entry{其二十一}

% メモ 校正終了 2024-04-22 2024-05-31
\原本頁{117-2}%
\ruby{右}{みぎ}せんとすれば
\ruby{左}{ひだり}したき
\ruby{意}{こゝろ}% 踊り字調整「〻(二の字点、揺すり点)に見えるが(ゝ)」
もあり、
%
\ruby{左}{ひだり}せんとすれば
\ruby{右}{みぎ}したき
\ruby[<j||]{意}{こゝろ}% 踊り字調整「〻(二の字点、揺すり点)に見えるが(ゝ)」% 行末行頭の境界付近なので特例処置を施す
もありて、
%
\ruby{廣野}{ひろ|の}の
\ruby{草}{くさ}
\ruby{高}{たか}き
\ruby{中}{なか}の
\ruby{岐路}{わかれ|ぢ}に
さしかゝれる% 踊り字調整「〻(二の字点、揺すり点)に見えるが(ゝ)」
\ruby{身}{み}の、
%
いづれ
\原本頁{117-4}\改行%
と
\ruby{取}{と}り
わづらへば、
%
\ruby{右}{みぎ}にも
\ruby{去}{さ}り
\ruby{得}{{\換字{𛀁}}}ず
%
\ruby{左}{ひだり}にも
\ruby{往}{ゆ}き
\ruby{得}{{\換字{𛀁}}}ざる
\ruby{一時}{いち|じ}
\ruby{二念}{に|ねん}
\原本頁{117-5}\改行%
の
\ruby{心魂}{こゝ|ろ}は% 踊り字調整「〻(二の字点、揺すり点)に見えるが(ゝ)」
\ruby{疲}{つか}れて、
%
\ruby{我}{われ}
\ruby{知}{し}らず
\ruby[||j>]{誦}{じゆ}
\ruby[||j>]{經}{きやう}の
% \ruby{誦經}{じゆ|きやう}の
\ruby{聲}{こゑ}の
\ruby{中}{うち}に
\ruby{攝}{せつ}し
\ruby{去}{さ}られ、
%
\ruby{睡}{ねむ}ると
\原本頁{117-6}\改行%
も
\ruby{無}{な}しに
\ruby{睡}{ねむ}りし
\ruby{歟}{か}、
%
\ruby[||j>]{否}{あらず}
\ruby[||j>]{睡}{ ねむ}りしか
\ruby{睡}{ねむ}らざりし
\ruby{歟}{か}。
%
たゞ% TODO 原本の「二の字点、揺すり点」に濁点のグリフが見つからないので「ゞ」
\ruby{我}{われ}
\ruby{深}{ふか}く〳〵
\原本頁{117-7}\改行%
\ruby{思}{おも}ひ
\ruby{入}{い}りて、
%
いよ〳〵
\ruby{二}{ふた}つの
\ruby{念}{おもひ}の
\ruby[||j>]{力}{ちから}
\ruby[||j>]{相}{ あひ}
\ruby[||j>]{{\換字{均}}}{ ゝと}しく% 踊り字調整「〻(二の字点、揺すり点)に見えるが(ゝ)」
して、
%
\ruby{我}{わ}が
\ruby{心}{こゝろ}の% 踊り字調整「〻(二の字点、揺すり点)に見えるが(ゝ)」
いづれにも
\ruby{動}{うご}かず
なりし
\ruby{其}{そ}の
\ruby{靜}{しづか}さを
\ruby{纔}{わづか}に
おぼえし
\ruby{後}{のち}は、
%
\ruby{聞}{き}くとも
\原本頁{117-9}\改行%
\ruby{無}{な}く
\ruby{聞}{き}ける
\ruby{老人}{らう|じん}の
\ruby{聲}{こゑ}の、
%
いと
\ruby[<j>]{快}{こゝろよ}く% 踊り字調整「〻(二の字点、揺すり点)に見えるが(ゝ)」
\ruby{聞}{きこ}{\換字{𛀁}}しを
\ruby{知}{し}れる
のみなりしが
\改行% 校正作業の簡略化のため
、
%
\原本頁{117-10}\改行%
\ruby{兎}{と}にも
\ruby{角}{かく}にも
\ruby{我}{われ}を
\ruby{忘}{わす}れしは
\ruby{愚}{おろか}なりしと、
%
\ruby{水野}{みづ|の}は
\ruby{繰}{く}り
\ruby{{\換字{返}}}{かへ}して
\ruby{自}{みづか}ら
\原本頁{118-1}\改行%
\ruby{思}{おも}ふ
\ruby{時}{とき}、
%
\ruby{阿耨多羅}{あの|く|た|ら}
\ruby[||j>]{三}{さん}
\ruby[||j>]{藐}{みやく}
% \ruby{三藐}{さん|みやく}
\ruby[||j>]{三}{ さん}
\ruby[||j>]{菩}{ ぼ}
\ruby[||j>]{提}{だい}
\ruby{心}{しん}と、
% 阿耨多羅三藐三菩提あのくたらさんみゃくさんぼだい 仏教用語。
% サンスクリット語の
%   アヌッタラー(無上の)
% ・サムヤク(正しい、完全な)
% ・サンボーディ(悟り)
% anuttarā samyak-sa bodhiの音写。
% 仏の仏たるゆえんである、このうえなく正しい完全なる悟りの智慧(ちえ)のこと
%
\ruby{誦}{じゆ}し
\ruby{{\換字{終}}}{おは}りて
\ruby{一心}{いつ|しん}に
\ruby{禮拜}{らい|はい}せし
\ruby{彼}{か}の
\ruby{老人}{らう|じん}は、
%
\ruby{去}{さ}らず
\ruby{就}{つ}かずに
\ruby{立}{たち}
\ruby{{\換字{迷}}}{まよ}へる
\ruby{水野}{みづ|の}が
\ruby{狀態}{あり|さま}を
\ruby{頭}{かうべ}を
\ruby{反}{かへ}して
\原本頁{118-3}\改行%
\ruby{見}{み}つ、
%
たちまち
\ruby{此方}{こな|た}へ% ルビ調整(原本通り)
すた〳〵と
\ruby{來}{きた}りて、
%
\ruby{眼}{め}の
\ruby{中}{うち}に
\ruby{氣{\換字{遣}}}{き|づか}ふが
\ruby{如}{ごと}く
\ruby{憐}{あはれ}むが
\ruby{如}{ごと}き
\ruby{色}{いろ}を
\ruby{{\換字{浮}}}{うか}めながら、

\原本頁{118-5}%
『
あゝ、% 踊り字調整「〻(二の字点、揺すり点)に見えるが(ゝ)」
\ruby{御{\換字{迷}}}{お|まよ}ひ
なすつては
いけません、
%
\ruby{勿體}{もつ|たい}
\ruby{無}{な}い
\ruby{事}{こと}です!。
%
\ruby{念念}{ねん|ねん}に% ルビ調整(原本通り)非踊り字表記(行末行頭の境界付近)
\ruby[<j>]{疑}{うたがひ}を
\ruby{生}{しやう}ずる
\ruby{勿}{なか}れとは
\ruby{御經}{お|きやう}にも
ございます。
%
\ruby{貴君}{あな|た}
\ruby[||j>]{{\換字{過}}}{この}
\ruby[||j>]{日}{あひだ}は
% \ruby{{\換字{過}}日}{この|あひだ}は
\ruby{泣}{ない}て
\ruby{居}{ゐ}らしつた
ではありませんか、
%
\ruby{貴君}{あな|た}の
やうな
\ruby{良}{い}い
\ruby{方}{かた}が、
%
\ruby{御{\換字{迷}}}{お|まよ}ひ
なさる
なんて
\ruby{飛}{とん}でもない
\ruby{事}{こと}です!。
%
\ruby{信}{しん}を
\ruby{籠}{こ}めて
\ruby{一心}{いつ|しん}に
\ruby{御拜}{お|をが}み
なさらなくつては
いけません、
%
\ruby{善}{ぜん}
\ruby{惡}{あく}
\ruby{共}{とも}に
\ruby{御利益}{ご|り|やく}は
\ruby{屹度}{きつ|と}
あります、
%
\原本頁{118-10}\改行%
さあ
\ruby[<j>]{私}{わたくし}も
\ruby{拜}{をが}みます、
%
\ruby{御一緖}{ご|いつ|しよ}に
\ruby{拜}{をが}みましやう!。
%
さあ、
%
\ruby{貴君}{あな|た}、
%
さ
\原本頁{118-11}\改行%
あ!。
』

\原本頁{119-1}%
と
\ruby{云}{い}ひ〳〵
\ruby{袖}{そで}を
\ruby{引}{ひ}きて
\ruby{御{\換字{前}}}{おん|まへ}へと
\ruby{誘}{いざな}ひ、
%
おのれ
\ruby{先}{ま}づ
\ruby{膝}{ひざ}を
\ruby{折}{を}り
\ruby{身}{み}を
\ruby{屈}{かゞ}めて% TODO 原本の「二の字点、揺すり点」に濁点のグリフが見つからないので「ゞ」
\ruby{禮拜}{らい|はい}し、
%
\ruby{水野}{みづ|の}にも
\ruby{之}{これ}に
\ruby{倣}{なら}はし
めたり。

\原本頁{119-3}%
\ruby{他人}{ひ|と}の
\ruby{胸}{むね}の
\ruby{中}{うち}には
\ruby{何物}{な|に}
ありとも
\ruby{思}{おも}はず、
%
たゞ% TODO 原本の「二の字点、揺すり点」に濁点のグリフが見つからないので「ゞ」
\ruby{我}{わ}が
\ruby{菩提}{ぼ|だい}の
\ruby[||j>]{同}{どう}
\ruby[||j>]{行}{ぎやう}と
% \ruby{同行}{どう|ぎやう}と
\ruby{思}{おも}ふ
ばかりの
\ruby{親切}{しん|せつ}より、
%
\ruby{年}{とし}
\ruby{{\換字{若}}}{わか}き
\ruby{我}{われ}を
あらぬ
\ruby{{\換字{道}}}{みち}へ
\ruby{外}{そ}れ
させじ
との
\原本頁{119-5}\改行%
\ruby{他事}{た|じ}なき
\ruby{願望}{のぞ|み}に、
%
\ruby{人}{ひと}の
\ruby{好}{よ}げなる
\ruby{此}{こ}の
\ruby{老人}{らう|じん}の
\ruby{如是}{か|く}
\ruby[||j>]{心}{こゝろ}を% 踊り字調整「〻(二の字点、揺すり点)に見えるが(ゝ)」
\ruby{使}{つか}ひ
\ruby{身}{み}を
\ruby{使}{つか}ひて
\ruby{老實}{まめ|〳〵}しく
\ruby{振舞}{ふる|ま}ひ
\ruby{吳}{く}るゝを% 踊り字調整「〻(二の字点、揺すり点)に見えるが(ゝ)」
\ruby{見}{み}ては、
%
\ruby[||j>]{心}{こゝろ}% 踊り字調整「〻(二の字点、揺すり点)に見えるが(ゝ)」
\ruby[||j>]{{\換字{弱}}}{ よわ}くも
\ruby{人}{ひと}
\ruby{惡}{あ}しからぬ
\ruby{水野}{みづ|の}は
これを
\ruby{拒}{こば}みかねて、
%
\ruby{牽}{ひ}かるゝが% 踊り字調整「〻(二の字点、揺すり点)に見えるが(ゝ)」
まゝに% 踊り字調整「〻(二の字点、揺すり点)に見えるが(ゝ)」
\ruby{牽}{ひ}かれ、
%
\ruby{屈}{かゞ}ませ% TODO 原本の「二の字点、揺すり点」に濁点のグリフが見つからないので「ゞ」
らる
\原本頁{119-8}\改行%
ゝが% 踊り字調整「〻(二の字点、揺すり点)に見えるが(ゝ)」% ルビ調整踊り字表記(行末行頭の境界付近)
まゝに% 踊り字調整「〻(二の字点、揺すり点)に見えるが(ゝ)」
\ruby{屈}{かゞ}み、% TODO 原本の「二の字点、揺すり点」に濁点のグリフが見つからないので「ゞ」
%
\ruby{{\換字{終}}}{つひ}には
\ruby{御佛}{み|ほとけ}の
\ruby{{\換字{前}}}{まへ}に
\ruby{蹲}{うづく}まりて、
%
\ruby{其}{そ}の
\ruby{老人}{らう|じん}の
\ruby{爲}{な}すが
\ruby{如}{ごと}くに、
%
\ruby{一霎時}{し|ば|し}は
\ruby{頭}{かうべ}を
\ruby{下}{さ}げ
\ruby{眼}{まなこ}を
\ruby{瞑}{ふさ}ぎて、
%
\ruby{一心}{いつ|しん}に
\ruby{大慈}{だい|じ}
\ruby{大悲}{だい|ひ}の
\ruby{我}{わ}が
\ruby{菩薩}{ぼ|さつ}をば、
%
\ruby{我}{われ}を
\ruby{忘}{わす}れて
\ruby{念}{ねん}じ
\ruby[<j>]{奉}{たてまつ}りしが、
%
\ruby{佛力}{ぶつ|りき}
\ruby{甚深}{じん|〳〵}
\ruby{測}{はか}るべからず
\改行% 校正作業の簡略化のため
、
%
\原本頁{119-11}\改行%
\ruby{時}{とき}に
\ruby{不思議}{ふ|し|ぎ}や
\ruby{水野}{みづ|の}は
\ruby{忽}{たちま}ち、
%
\ruby{心}{こゝろ}の% 踊り字調整「〻(二の字点、揺すり点)に見えるが(ゝ)」
\ruby{闇}{やみ}に
\ruby{{\換字{朝}}日}{あさ|ひ}の
\ruby{射}{さ}して、
%
\ruby{胸}{むね}の
\ruby{氷}{こほり}の
\ruby{春風}{はる|かぜ}に
\ruby{逢}{あ}へるが
\ruby{如}{ごと}き
\ruby{思}{おも}ひの
\ruby{仕}{し}つ、
%
\ruby{其}{そ}の
\ruby{故}{ゆゑ}を
\ruby{問}{と}ふ
\ruby{暇}{いとま}も
\ruby{無}{な}く、
%
\ruby{今}{いま}まで
\原本頁{120-2}\改行%
\ruby{知}{し}らざりし
\ruby[||j>]{慰}{やすら}
\ruby[||j>]{安}{ かさ}を
% \ruby{慰安}{やすら|かさ}を
\ruby{得}{{\換字{𛀁}}}て、
%
\ruby{何}{なに}とは
\ruby{無}{な}しの
\ruby[|j>]{忝}{かたじけな}さに、
%
\ruby{涙}{なみだ}は
\ruby{止}{とゞ}めんと% TODO 原本の「二の字点、揺すり点」に濁点のグリフが見つからないので「ゞ」
\原本頁{120-3}\改行%
して
\ruby{止}{とゞ}めあへず、% TODO 原本の「二の字点、揺すり点」に濁点のグリフが見つからないので「ゞ」
%
\ruby[||j>]{水}{すゐ}
\ruby[||j>]{晶}{しやう}の
% \ruby{水晶}{すゐ|しやう}の
\ruby{珠數}{じゆ|ず}
\ruby{俄}{にはか}に
\ruby{斷}{き}れて、
%
\ruby{{\換字{留}}}{と}まらぬ
\ruby{珠}{たま}の
ばらば% ルビ調整(原本通り)非踊り字表記(行末行頭の境界付近)
\原本頁{120-4}\改行%
らと
\ruby{緖}{を}より
\ruby{亂}{みだ}れて
\ruby{落}{お}つるが
\ruby{如}{ごと}く、
%
\ruby{泫然}{げん|ぜん}として
\ruby{泣}{な}きに
\ruby{泣}{な}きたり。
