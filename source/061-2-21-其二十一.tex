\Entry{其二十一}

\原本頁{}%
\ruby{右}{みぎ}せんとすれば
\ruby{左}{ひだり}したき
\ruby{意}{こゝろ}あり、
%
\ruby{左}{ひだり}せんとすれば
\ruby{右}{みぎ}せんとしたき
\ruby{意}{こゝろ}もありて、
%
\ruby{廣野}{ひろ|の}の
\ruby{草高}{くさ|たか}き
\ruby{中}{うち}の
\ruby{岐路}{わかれ|ぢ}にさしか〻れる
\ruby{身}{み}の、
%
いづれと
\ruby{取}{と}りわづらへば、
%
\ruby{右}{みぎ}にも
\ruby{去}{さ}り
\ruby{得}{\換字{𛀁}}ず
\ruby{左}{ひだり}にも
\ruby{往}{ゆ}き
\ruby{得}{\換字{𛀁}}ざる
\ruby{一時}{いち|じ}
\ruby{二念}{に|ねん}の
\ruby{心魂}{こゝ|ろ}は
\ruby{疲}{つか}れて、
%
\ruby{我}{われ}
\ruby{知}{し}らず
\ruby{誦經}{じゆ|きやう}の
\ruby{聲}{こゑ}の
\ruby{中}{うち}に
\ruby{攝}{せつ}し
\ruby{去}{さ}られ、
%
\ruby{睡}{ねむ}るとも
\ruby{無}{な}しに
\ruby{睡}{ねむ}りし
\ruby{歟}{か}、
%
\ruby[<h||]{否}{あらず}
\ruby{睡}{ねむ}りしか
\ruby{睡}{ねむ}らざりし
\ruby{歟}{か}。
%
たゞ
\ruby{我}{われ}
\ruby{深}{ふか}く〳〵
\ruby{思}{おも}ひ
\ruby{入}{い}りて、
%
いよ〳〵
\ruby{二}{ふた}つの
\ruby{念}{おもひ}の
\ruby[<h||]{力}{ちから}
\ruby{相等}{あひ|ゝと}しくして、
%
\ruby{我}{わ}が
\ruby{心}{こゝろ}のいづれにも
\ruby{動}{うご}かずなりし
\ruby{其}{そ}の
\ruby{靜}{しづか}さを
\ruby{纔}{わづか}におぼえし
\ruby{後}{のち}は、
%
\ruby{聞}{き}くとも
\ruby{無}{な}く
\ruby{聞}{き}ける
\ruby{老人}{らう|じん}の
\ruby{聲}{こゑ}の、
%
いと
\ruby{快}{こゝろよ}く
\ruby{聞}{きこ}\換字{𛀁}しを
\ruby{知}{し}れるのみなりしが、
%
\ruby{兎}{と}にも
\ruby{角}{かく}にも
\ruby{我}{われ}を
\ruby{忘}{わす}れしは
\ruby{愚}{おろか}なりしと、
%
\ruby[g]{水野}{みづの}は
\ruby{繰}{く}り
\ruby{{\換字{返}}}{かへ}して
\ruby{自}{みづか}ら
\ruby{思}{おも}ふ
\ruby{時}{とき}、
%
\ruby{阿耨多羅三藐三菩提心}{あの|く|た|ら|さん|みやく|さん|ぼ|だ|しん}と、
%
\ruby{誦}{じゆ}し
\ruby{{\換字{終}}}{おは}りて
\ruby{一心}{いつ|しん}に
\ruby{禮拜}{らい|はい}せし
\ruby{彼}{か}の
\ruby{老人}{らう|じん}は、
%
\ruby{去}{さ}らず
\ruby{就}{つ}かずに
\ruby{立{\換字{迷}}}{たち|まよ}へる
\ruby[g]{水野}{みづの}が
\ruby{狀態}{あり|さま}を
\ruby{頭}{かうべ}を
\ruby{反}{かへ}して
\ruby{見}{み}つ、
%
たちまち
\ruby{此方}{こ|なた}へすた〳〵と
\ruby{來}{きた}りて、
%
\ruby{眼}{め}の
\ruby{中}{うち}に
\ruby{氣{\換字{遣}}}{き|づか}ふが
\ruby{如}{ごと}く
\ruby{憐}{あはれ}むが
\ruby{如}{ごと}き
\ruby{色}{いろ}を
\ruby{{\換字{浮}}}{うか}めながら、

\原本頁{}%
『あ〻
\ruby{御{\換字{迷}}}{お|まよ}ひなすつてはいけません、
%
\ruby{勿體無}{もつ|たい|な}い
\ruby{事}{こと}です!。
%
\ruby{念々}{ねん|〳〵}に
\ruby{疑}{うたがひ}を
\ruby{生}{しやう}ずる
\ruby{勿}{なか}れとは
\ruby{御經}{お|きやう}にもございます。
%
\ruby{貴君}{あな|た}
\ruby{{\換字{過}}日}{この|あひだ}は
\ruby{泣}{な}いて
\ruby{居}{ゐ}らしつたではありませんか、
%
\ruby{貴君}{あな|た}のやうな
\ruby{良}{よ}い
\ruby{方}{かた}が、
%
\ruby{御{\換字{迷}}}{お|まよ}ひなさるなんて
\ruby{飛}{とん}でもない
\ruby{事}{こと}です!。
%
\ruby{信}{しん}を
\ruby{籠}{こ}めて
\ruby{一心}{いつ|しん}に
\ruby{御拜}{お|をが}みなさらなくつてはいけません、
%
\ruby{善惡}{ぜん|あく}
\ruby{共}{とも}に
\ruby{御利益}{ご|り|やく}は
\ruby{屹度}{きつ|と}あります、
%
さあ
\ruby{私}{わたくし}も
\ruby{拜}{をが}みます、
%
\ruby{御一緖}{ご|いつ|しよ}に
\ruby{拜}{をが}みましやう!。
%
さあ、
%
\ruby{貴君}{あな|た}、
%
さあ!。
』

\原本頁{}%
と
\ruby{云}{い}ひ〳〵
\ruby{袖}{そで}を
\ruby{引}{ひ}きて
\ruby{御{\換字{前}}}{おん|まへ}へと
\ruby{誘}{いざな}ひ、
%
おのれ
\ruby{先}{ま}づ
\ruby{膝}{ひざ}を
\ruby{折}{を}り
\ruby{身}{み}を
\ruby{屈}{かゞ}めて
\ruby{禮拜}{らい|はい}し、
%
\ruby[g]{水野}{みづの}にも
\ruby{之}{これ}に
\ruby{倣}{なら}はしめたり。

\原本頁{}%
\ruby{他人}{ひ|と}の
\ruby{胸}{むね}の
\ruby{中}{うち}には
\ruby{何物}{な|に}ありとも
\ruby{思}{おも}はず、
%
たゞ
\ruby{我}{わ}が
\ruby{菩提}{ぼ|だい}の
\ruby{同行}{どう|ぎやう}と
\ruby{思}{おも}ふばかりの
\ruby{親切}{しん|せつ}より、
%
\ruby{年{\換字{若}}}{とし|わか}き
\ruby{我}{われ}をあらぬ
\ruby{{\換字{道}}}{みち}へ
\ruby{外}{そ}れさせじとの
\ruby{他事}{た|じ}なき
\ruby{願望}{のぞ|み}に、
%
\ruby{人}{ひと}の
\ruby{好}{よ}げなる
\ruby{此}{こ}の
\ruby{老人}{らう|じん}の
\ruby{如是}{か|く}
\ruby{心}{こゝろ}を
\ruby{使}{つか}ひ
\ruby{身}{み}を
\ruby{使}{つか}ひて
\ruby{老實}{まめ|〳〵}しく
\ruby{振舞}{ふる|ま}ひ
\ruby{吳}{く}る〻を
\ruby{見}{み}ては、
%
\ruby{心{\換字{弱}}}{こゝろ|よわ}くも
\ruby{人惡}{ひと|あ}しからぬ
\ruby[g]{水野}{みづの}はこれを
\ruby{拒}{こば}みかねて、
%
\ruby{牽}{ひ}かる〻がま〻に
\ruby{牽}{ひ}かれ、
%
\ruby{屈}{かゞ}ませらる〻がま〻に
\ruby{屈}{かゞ}み、
%
\ruby{{\換字{終}}}{つひ}には
\ruby{御佛}{み|ほとけ}の
\ruby{{\換字{前}}}{まへ}に
\ruby{蹲}{うづく}まりて、
%
\ruby{其}{そ}の
\ruby{老人}{らう|じん}の
\ruby{爲}{な}すが
\ruby{如}{ごと}くに、
%
\ruby{一霎時}{し|ば|し}は
\ruby{頭}{かうべ}を
\ruby{下}{さ}げ
\ruby{眼}{まなこ}を
\ruby{瞑}{ふさ}ぎて、
%
\ruby{一心}{いつ|しん}に
\ruby{大慈}{だい|じ}
\ruby{大悲}{だい|ひ}の
\ruby{我}{わ}が
\ruby{菩薩}{ぼ|さつ}をば、
%
\ruby{我}{われ}を
\ruby{忘}{わす}れて
\ruby{念}{ねん}じ
\ruby{奉}{たてまつ}りしが、
%
\ruby{佛力甚深測}{ぶつ|りき|じん|〳〵|はか}るべからず、
%
\ruby{時}{とき}に
\ruby{不思議}{ふ|し|ぎ}や
\ruby[g]{水野}{みづの}は
\ruby{忽}{たちま}ち、
%
\ruby{心}{こゝろ}の
\ruby{闇}{やみ}に
\ruby{{\換字{朝}}日}{あさ|ひ}の
\ruby{射}{さ}して、
%
\ruby{胸}{むね}の
\ruby{氷}{こほり}の
\ruby{春風}{はる|かぜ}に
\ruby{逢}{あ}へるが
\ruby{如}{ごと}き
\ruby{思}{おも}ひの
\ruby{仕}{し}つ、
%
\ruby{其}{そ}の
\ruby{故}{ゆゑ}を
\ruby{問}{と}ふ
\ruby{暇}{いとま}も
\ruby{無}{な}く、
%
\ruby{今}{いま}まで
\ruby{知}{し}らざりし
\ruby{慰安}{やす|らかさ}を
\ruby{得}{\換字{𛀁}}て、
%
\ruby{何}{なん}とは
\ruby{無}{な}しの
\ruby{忝}{かたじけな}さに、
%
\ruby{涙}{なみだ}は
\ruby{止}{と}めんとして
\ruby{止}{と}めあへず、
%
\ruby{水晶}{すゐ|しやう}の
\ruby{珠數}{じゆ|ず}
\ruby{俄}{にはか}に
\ruby{斷}{き}れて、
%
\ruby{{\換字{留}}}{と}まらぬ
\ruby{珠}{たま}のばらばらと
\ruby{緖}{を}より
\ruby{亂}{みだ}れて
\ruby{落}{お}つるが
\ruby{如}{ごと}く、
%
\ruby{泫然}{げん|ぜん}として
\ruby{泣}{な}きに
\ruby{泣}{な}きたり。
