\Entry{其二十二}

% メモ 校正終了 2024-04-23
\原本頁{120-6}%
\ruby{經}{きやう}は
\ruby{誦}{じゆ}したり
といへども
\ruby{老人}{らう|じん}
\ruby{{\換字{迷}}魂}{めい|こん}の
\ruby{{\換字{術}}}{じゆつ}を
\ruby{知}{し}れる
にもあらず、
%
\ruby{心}{こ〻ろ}こそ% 原本通り「〻(二の字点、揺すり点)」
\ruby{惑}{まど}ひたれ
\ruby[g]{水野}{みづの}
\ruby{奪魄}{だつ|ぱく}の
\ruby{法}{はふ}に
\ruby{致}{いた}さる
べくも
あらねど、
%
\ruby[g]{水野}{みづの}が
\ruby{胸中}{きやう|ちう}の
\ruby{{\換字{消}}息}{せう|そく}は
\ruby[g]{水野}{みづの}
ばかりぞ
\ruby{知}{し}る、
%
\ruby{傍觀}{わき|め}より
\ruby{云}{い}へば
たゞ% TODO 原本の「二の字点、揺すり点」に濁点のグリフが見つからないので「ゞ」
\ruby{是}{これ}
\ruby[||j>]{恰}{あたか}も% 恰も「あ(た)かも」
\ruby{神{\換字{文}}}{しん|もん}
\原本頁{120-9}\改行%
\ruby{密呪}{みつ|じゆ}の
\ruby{妖}{あや}しき
\ruby{{\換字{道}}}{みち}に
\ruby{因}{よ}つて
\ruby[g]{縛心{\換字{鎖}}意}{フアツシ{\換字{子}}ート}
されたる
\ruby{人}{ひと}の
\ruby{如}{ごと}く、
%
\ruby{今}{いま}までの
\ruby[g]{水野}{みづの}
\ruby[|j>]{某}{なにがし}は
いづくへやら
\ruby{{\換字{消}}}{き}{\換字{𛀁}}て、
%
\ruby{全}{まつた}く
\ruby{愚痴}{ぐ|ち}
\ruby{{\換字{文}}盲}{もん|もう}の
\ruby{爺}{ぢ〻}% 「ぢゞ」のはずだが、原本通り「〻(二の字点、揺すり点)」
\ruby{婆}{ば〻}% 「ばゞ」のはずだが、原本通り「〻(二の字点、揺すり点)」
のやうになり、
%
\ruby{一心}{いつ|しん}に
\ruby{御佛}{み|ほとけ}を
\ruby{頼}{たの}み
\ruby{奉}{たてまつ}れる
さまの、
%
\ruby{男兒}{をと|こ}らしからず
\ruby{憫然}{あは|れ}にのみ
\ruby{見}{み}{\換字{𛀁}}たり。

\原本頁{121-3}%
\ruby{西}{にし}に
\ruby{對}{むか}ひて
\ruby{放}{はな}ちても
\ruby{東}{ひがし}に
\ruby{對}{むか}ひて
\ruby{放}{はな}ちても、
%
\ruby{滿}{み}つる
\ruby{月}{つき}の
\ruby{形}{かたち}と
\ruby{引絞}{ひき|しぼ}りたる
\ruby{{\換字{強}}弓}{がう|きう}を、
%
きつて
\ruby{放}{はな}つ
\ruby{時}{とき}
おのづからの
\ruby{快}{こ〻ろよ}さ% 原本通り「〻(二の字点、揺すり点)」
あり。
%
\ruby{南}{みなみ}に
むかひて
\ruby{決}{けつ}しても
\ruby{北}{きた}に
むかひて
\ruby{決}{けつ}しても、
%
\ruby{千頃}{せん|けい}の% 「千頃」物事を、ある基準で区分けしたときの一つ一つ。
\ruby{瀦水}{たまり|みづ}の
\ruby{漫々}{まん|〳〵}たるを、
%
\原本頁{121-6}\改行%
\ruby[||j>]{堤}{つ〻み}を% 原本通り「〻(二の字点、揺すり点)」
\ruby{切}{き}つて
\ruby{決}{けつ}する
\ruby{時}{とき}
おのづからの
\ruby{快}{こ〻ろよ}さあり。% 原本通り「〻(二の字点、揺すり点)」
%
そも〳〵
\ruby{心}{こ〻ろ}の% 原本通り「〻(二の字点、揺すり点)」
\ruby{後}{あと}へも
\ruby{先}{さき}へも
\ruby{行}{ゆ}かざるを
\ruby{悶}{もだ{\換字{𛀁}}}とは
\ruby{云}{い}ひ、
%
\ruby{一方}{いつ|ぱう}へ
\ruby{爽}{さわや}かに
\ruby{走}{はし}るを
\ruby{快}{こ〻ろよ}しとは% 原本通り「〻(二の字点、揺すり点)」
\原本頁{121-8}\改行%
\ruby{云}{い}ふなれば、
%
\ruby{佛陀}{ほと|け}の
\ruby{利益}{り|やく}は
\ruby{有}{あ}るにせよ
\ruby{無}{な}きにせよ、
%
\ruby[g]{水野}{みづの}は
\ruby{今}{いま}まさに
\ruby{此}{こ}の
\ruby{快}{こ〻ろよ}さを% 原本通り「〻(二の字点、揺すり点)」
\ruby{味}{あぢは}へる
なるべし。

\原本頁{121-10}%
\ruby{星辰}{せい|しん}
\ruby{上}{かみ}に
か〻り、% 原本通り「〻(二の字点、揺すり点)」
%
\ruby{山河}{さん|が}
\ruby{下}{しも}に
\ruby{布}{し}ける
\ruby{此}{こ}の
\ruby{天地}{てん|ち}の
\ruby{大}{だい}にして
\ruby{大}{だい}なるを
おもひ、
%
\ruby{萬年萬々年}{ばん|ねん|ばん|〳〵|ねん}% 「〴〵」でなく原本通り「〳〵」
の
\ruby{{\換字{前}}}{まへ}に
\ruby{萬年萬々年}{ばん|ねん|ばん|〳〵|ねん}% 「〴〵」でなく原本通り「〳〵」
あり、
%
\ruby{萬年萬々年}{ばん|ねん|ばん|〳〵|ねん}% 「〴〵」でなく原本通り「〳〵」
の
\ruby{後}{のち}に
\原本頁{122-1}%
\ruby{萬年萬々年}{ばん|ねん|ばん|〳〵|ねん}% 「〴〵」でなく原本通り「〳〵」
ある
\ruby{此}{こ}の
\ruby{歳月}{さい|げつ}の
\ruby{久}{ひさ}しくして
\ruby{久}{ひさ}しきを
\ruby{思}{おも}ひ、
%
さて
\ruby{此}{こ}の
\ruby{天地}{てん|ち}の
\ruby{立}{た}てる
\ruby{{\換字{所}}以}{ゆ|{\換字{𛀁}}ん}を
おもひ
\ruby{歳月}{さい|げつ}の
\ruby{經}{ふ}る
\ruby{{\換字{所}}以}{ゆ|{\換字{𛀁}}ん}を
\ruby{思}{おも}ひて、
%
\ruby{此}{こ}の
\ruby{天地}{てん|ち}と
\ruby{歳月}{さい|げつ}との
\ruby{存在}{そん|ざい}を、
%
たゞ〳〵% TODO 原本の「二の字点、揺すり点」に濁点のグリフが見つからないので「ゞ」
\ruby{無}{む}
\ruby{意義}{い|ぎ}なる
\ruby{事實}{こと|がら}のみと
\ruby{認}{みと}めなば、
%
\原本頁{122-4}\改行%
\ruby{誰}{たれ}かは
\ruby{味氣}{あぢ|き}
\ruby{無}{な}き
\ruby{感}{おもひ}に
\ruby{撲}{う}たれて
\ruby{悲}{かなし}み
\ruby{傷}{いた}まざらん。
%
されど
\ruby{此}{こ}の
\ruby{天地}{てん|ち}と
\ruby{歳月}{さい|げつ}との
\ruby{存在}{そん|ざい}の、
%
\ruby{眞}{まこと}は
\ruby{無}{む}
\ruby{意義}{い|ぎ}の
\ruby{事實}{こと|がら}のみ
ならで、
%
\ruby{其}{その}
\ruby{中}{うち}に
\ruby{意義}{い|ぎ}ある
なりと
\ruby{認}{みと}むる
\ruby{時}{とき}は、
%
\ruby{誰}{たれ}かは
\ruby{{\換字{乳}}{\換字{房}}}{ち|ぶさ}を
\ruby{探}{さぐ}り
\ruby{得}{{\換字{𛀁}}}たる
\ruby{嬰兒}{あか|ご}の
\ruby{如}{ごと}く、
%
\原本頁{122-7}\改行%
\ruby{無限}{む|げん}の
\ruby{喜悅}{よろ|こび}に
\ruby{胸}{むね}を
\ruby{躍}{をど}らさゞらん。% TODO 原本の「二の字点、揺すり点」に濁点のグリフが見つからないので「ゞ」
%
\ruby{意義}{い|ぎ}あり、
%
\ruby{意義}{い|ぎ}あり、
%
\ruby{無}{む}
\ruby{意義}{い|ぎ}ならず、
%
\ruby{神}{かみ}の
\ruby{御心}{み|こ〻ろ}% 原本通り「〻(二の字点、揺すり点)」
\ruby{{\換字{即}}}{すなは}ち
\ruby{意義}{い|ぎ}なり、
%
\ruby{佛}{ほとけ}の
\ruby{御心}{み|こ〻ろ}% 原本通り「〻(二の字点、揺すり点)」
\ruby{{\換字{即}}}{すなは}ち
\ruby{意義}{い|ぎ}なり、
%
\ruby{化醇}{くわ|じゆん}の
\ruby[g]{大法}{おきて}は
こ〻にあるなり、% 原本通り「〻(二の字点、揺すり点)」
%
\ruby{歸善}{き|ぜん}の
\ruby{定數}{さだ|まり}
こ〻にあるなり、% 原本通り「〻(二の字点、揺すり点)」
%
\ruby{大慈}{だい|じ}の
\原本頁{122-10}\改行%
\ruby[g]{光明}{ひかり}は
\ruby{柔}{やはら}かに
\ruby{山村}{さん|そん}
\ruby[||j>]{水鄕}{すゐ|きやう}を
\ruby{包}{つ〻}めるなり、% 原本通り「〻(二の字点、揺すり点)」
%
\ruby{大悲}{だい|ひ}の
\ruby{音樂}{おん|がく}は
\ruby{斷}{た}ゆる
\ruby{間}{ま}も
\ruby{無}{な}く
\ruby{{\換字{古}}往}{こ|わう}
\ruby{今來}{こん|らい}に
\ruby{亘}{わた}れるなり、
%
\ruby{我}{われ}は
\ruby{此}{こ}の
\ruby[g]{溫{\換字{暖}}}{あた〻か}き% 原本通り「〻(二の字点、揺すり点)」
\ruby{意義}{い|ぎ}の
\ruby{中}{うち}より
\ruby{生}{うま}れたる
\ruby{子}{こ}なり、
%
\ruby{神}{かみ}の
\ruby{子}{こ}なり
\ruby{佛}{ほとけ}の
\ruby{子}{こ}なり
\ruby[g]{正眞}{まこと}の
\ruby{子}{こ}なり、
%
\ruby{我}{われ}と
\ruby{神佛}{かみ|ほとけ}とは
\ruby{血}{ち}の
\ruby{相}{あひ}
\ruby{{\換字{通}}}{かよ}へる
なり、
%
と
\ruby{如是}{か|く}
\ruby{思}{おも}ふ
\ruby{時}{とき}
おのづと
\ruby{悅}{よろこ}ばしからば、
%
\ruby[g]{水野}{みづの}は
\ruby{今}{いま}きさに
\ruby{此}{こ}の
\ruby{悅}{よろこ}びを
おぼえたる
なるべし。

\原本頁{123-4}%
\ruby[g]{水野}{みづの}の
やうやく
\ruby{念}{ねん}じ
\ruby{{\換字{終}}}{をは}れる
\ruby{時}{とき}、
%
\ruby{老人}{らう|じん}は
また
\ruby[g]{水野}{みづの}に
\ruby{對}{むか}ひて、

\原本頁{1}%
『
あ〻% 原本通り「〻(二の字点、揺すり点)」
\ruby{御信心}{ご|しん|〴〵}なさい
まし〳〵、
%
\ruby{自然}{ひと|りで}に
\ruby{有}{あ}りがたい
ことが
\ruby{能}{よ}く
\ruby{解}{わか}つて
まゐります!。
%
まあ
\ruby{何樣}{ど|ん}な
\ruby{事}{こと}か
\ruby{存}{ぞん}じませんが、
%
\ruby{御樣子}{ご|やう|す}を
\ruby{見}{み}ました
ところでは、
%
よく〳〵の
\ruby{御心配事}{ご|しん|ぱい|ごと}が
\ruby{御有}{お|あ}りなさると
\ruby{御察}{お|さつ}し
\ruby{申}{まをし}ます。
%
\ruby{御籤}{お|みくじ}を
\ruby{御戴}{お|いたゞ}きなさい、% TODO 原本の「二の字点、揺すり点」に濁点のグリフが見つからないので「ゞ」
%
\ruby{御籤}{お|みくじ}を
\ruby{御戴}{お|いたゞ}きなさい。% TODO 原本の「二の字点、揺すり点」に濁点のグリフが見つからないので「ゞ」
%
あ〻% 原本通り「〻(二の字点、揺すり点)」
まだ
\ruby{御戴}{お|いたゞ}きなさつた% TODO 原本の「二の字点、揺すり点」に濁点のグリフが見つからないので「ゞ」
\ruby{事}{こと}が
\ruby{御有}{お|あ}んなさらないので、
%
\ruby{御{\換字{勝}}手}{ご|かつ|て}が
\ruby{知}{し}れないので
ございますネ。
%
\ruby{宜}{よ}うございます
\ruby{私}{わたくし}が
\ruby{戴}{いたゞ}いて% TODO 原本の「二の字点、揺すり点」に濁点のグリフが見つからないので「ゞ」
あげましやう。
』

\原本頁{123-11}%
と、
%
\ruby{世話}{せ|わ}を
\ruby{燒}{や}きて
\ruby[g]{水野}{みづの}が
まだ
\ruby{何}{なに}とも
\ruby{答}{こたへ}を
せざるに、
%
はや
\ruby{御籤}{み|くじ}を
\原本頁{124-1}\改行%
\ruby{管}{つかさど}る
\ruby{僧}{そう}の
\ruby{許}{もと}に
\ruby{至}{いた}りぬ。

\原本頁{124-2}%
やがて
\ruby{僧}{そう}は
\ruby{御籤箱}{お|みくじ|ばこ}を
ふる
なる
べし、
%
かた〳〵
といふ
\ruby{音}{おと}は
\ruby{小暗}{を|ぐら}き
\ruby{其}{そ}の
\ruby{座}{ざ}の
あたりより
\ruby{聞}{きこ}{\換字{𛀁}}ぬ。
