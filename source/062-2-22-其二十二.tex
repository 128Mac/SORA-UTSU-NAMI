\Entry{其二十二}

\ruby{經}{きやう}は
\ruby{誦}{じゆ}したりといへども
\ruby{老人{\換字{迷}}魂}{らう|じん|めい|こん}の
\ruby{{\換字{術}}}{じゆつ}を
\ruby{知}{し}れるにもあらず、
\ruby{心}{こヽろ}こそ
\ruby{惑}{まど}ひたれ
\ruby[g]{水野奪魄}{みづのだつぱく}の
\ruby{法}{はう}に
\ruby{致}{いた}さるべくもあらねど、
\ruby[g]{水野}{みづの}が
\ruby{胸中}{きやう|ちう}の
\ruby{{\換字{消}}息}{せう|そく}は
\ruby[g]{水野}{みづの}ばかりぞ
\ruby{知}{し}る、
\ruby{傍觀}{は|た}より
\ruby{云}{い}へばたヾ
\ruby{是恰}{これ|あたか}も
\ruby{神{\換字{文}}密呪}{しん|もん|みつ|じゆ}の
\ruby{妖}{あや}しき
\ruby{{\GWI{u9053-k}}}{みち}に
\ruby{因}{よ}つて
\ruby[g]{縛心{\GWI{u93bb-g}}意}{ファアツシネーㇳ}されたる
\ruby{人}{ひと}の
\ruby{如}{ごと}く、
\ruby{今}{いま}までの
\ruby[g]{水野某}{みづのなにがし}はいづくへやら
\ruby{{\換字{消}}}{き}\GWI{u1b001}て、
\ruby{全}{まつた}く
\ruby{愚痴{\換字{文}}盲}{ぐ|ち|もん|もう}の
\ruby{爺婆}{ぢヾ|ばヾ}のやうになり、
\ruby{一心}{いつ|しん}に
\ruby{御佛}{み|ほとけ}を
\ruby{頼}{たの}み
\ruby{奉}{たてまつ}れるさまの、
\ruby{男兒}{をと|こ}らしからず
\ruby{憫然}{あは|れ}にのみ
\ruby{見}{み}\GWI{u1b001}たり。

\ruby{西}{にし}に
\ruby{對}{むか}ひて
\ruby{放}{はな}ちても
\ruby{東}{ひがし}に
\ruby{對}{むか}ひて
\ruby{放}{はな}ちても、
\ruby{滿}{み}つる
\ruby{月}{つき}の
\ruby{形}{かたち}と
\ruby{引絞}{ひき|しぼ}りたる
\ruby{{\換字{強}}弓}{がう|きう}を、きつて
\ruby{放}{はな}つ
\ruby{時}{とき}おのづからの
\ruby{快}{こヽろよ}さあり。
\ruby{南}{みなみ}にむかひて
\ruby{決}{けつ}しても
\ruby{北}{きた}にむかひて
\ruby{決}{けつ}しても、
\ruby{千頃}{せん|けい}の
\ruby[g]{瀦水}{たまりみづ}の
\ruby{漫々}{まん|〳〵}たるを、
\ruby{堤}{つヽみ}を
\ruby{切}{き}つて
\ruby{決}{けつ}する
\ruby{時}{とき}おのづからの
\ruby{快}{こヽろよ}さあり。
そも〳〵
\ruby{心}{こヽろ}の
\ruby{後}{あと}へも
\ruby{先}{さき}へも
\ruby{行}{ゆ}かざるを
\ruby{悶}{もだ\GWI{u1b001}}とは
\ruby{云}{い}ひ、
\ruby{一方}{いつ|ぱう}へ
\ruby{爽}{さわや}かに
\ruby{走}{はし}るを
\ruby{快}{こヽろよ}しとは
\ruby{云}{い}ふなれば、
\ruby[g]{佛陀}{ほとけ}の
\ruby{利{\換字{益}}}{り|やく}は
\ruby{有}{あ}るにせよ、
\ruby{無}{な}きにせよ、
\ruby[g]{水野}{みづの}は
\ruby{今}{いま}まさに
\ruby{此}{こ}の
\ruby{快}{こヽろよ}さを
\ruby{味}{あじは}へるなるべし。

\ruby{星辰上}{せい|しん|かみ}にか〻り、
\ruby{山河下}{さん|が|しも}に
\ruby{布}{し}ける
\ruby{此}{こ}の
\ruby{天地}{てん|ち}の
\ruby{大}{だい}にして
\ruby{大}{だい}なるをおもひ、
\ruby{萬年萬々年}{ばん|ねん|ばん|〳〵|ねん}の
\ruby{前}{まへ}に
\ruby{萬年萬々年}{ばん|ねん|ばん|〳〵|ねん}あり、
\ruby{萬年萬々年}{ばん|ねん|ばん|〳〵|ねん}の
\ruby{後}{のち}に
\ruby{萬年萬々年}{ばん|ねん|ばん|〳〵|ねん}ある
\ruby{此}{こ}の
\ruby{歳月}{さい|げつ}の
\ruby{久}{ひさ}しくして
\ruby{久}{ひさ}しきを
\ruby{思}{おも}ひ、さて
\ruby{此}{こ}の
\ruby{天地}{てん|ち}の
\ruby{立}{た}てる
\ruby{{\換字{所}}以}{ゆ|\GWI{u1b001}ん}をおもひ
\ruby{歳月}{さい|げつ}の
\ruby{經}{ふ}る
\ruby{{\換字{所}}以}{ゆ|\GWI{u1b001}ん}を
\ruby{思}{おも}ひて、
\ruby{此}{こ}の
\ruby{天地}{てん|ち}と
\ruby{歳月}{さい|げつ}との
\ruby{存在}{そん|ざい}を、たヾ〳〵
\ruby{無意義}{む|い|ぎ}なる
\ruby{事實}{こと|がら}のみと
\ruby{認}{みと}めなば、
\ruby{誰}{だれ}かは
\ruby{味氣無}{あぢ|き|な}き
\ruby{感}{おもひ}に
\ruby{撲}{う}たれて
\ruby{悲}{かなし}み
\ruby{傷}{いた}まざらん。
されど
\ruby{此}{こ}の
\ruby{天地}{てん|ち}と
\ruby{歳月}{さい|げつ}との
\ruby{存在}{そん|ざい}の、
\ruby{眞}{まこと}は
\ruby{無意義}{む|い|ぎ}の
\ruby{事實}{こと|がら}のみならで、
\ruby{其中}{その|うち}に
\ruby{意義}{い|ぎ}あるなりと
\ruby{認}{みと}むる
\ruby{時}{とき}は、
\ruby{誰}{だれ}かは
\ruby{{\換字{乳}}{\換字{房}}}{ち|ぶさ}を
\ruby{探}{さぐ}り
\ruby{得}{\GWI{u1b001}}たる
\ruby{嬰兒}{あか|ご}の
\ruby{如}{ごと}く、
\ruby{無限}{む|げん}の
\ruby{喜{\換字{悅}}}{よろ|こび}に
\ruby{胸}{むね}を
\ruby{躍}{をど}らさヾらん。
\ruby{意義}{い|ぎ}あり、
\ruby{意義}{い|ぎ}あり、
\ruby{無意義}{む|い|ぎ}ならず、
\ruby{神}{かみ}の
\ruby{御心即}{みこ|ヽろ|すなは}ち
\ruby{意義}{い|ぎ}なり、
\ruby{佛}{ほとけ}の
\ruby{御心即}{みこ|ヽろ|すなは}ち
\ruby{意義}{い|ぎ}なり、
\ruby{化醇}{くわ|じゆん}の
\ruby[g]{大法}{おきて}はこゝにあるなり、
\ruby{歸善}{き|ぜん}の
\ruby{定數}{さだ|まり}こゝにあるなり、
\ruby{大慈}{だい|じ}の
\ruby[g]{光明}{ひかり}は
\ruby{柔}{やはら}かに
\ruby{山村水郷}{さん|そん|すゐ|きやう}を
\ruby{包}{つヽ}めるなり、
\ruby{大悲}{だい|ひ}の
\ruby{音樂}{おん|がく}は
\ruby{斷}{た}ゆる
\ruby{間}{ま}も
\ruby{無}{な}く
\ruby{古往今來}{こ|わう|こん|らい}に
\ruby{亘}{わた}れるなり、
\ruby{我}{われ}は
\ruby{此}{こ}の
\ruby{溫暖}{あた|ヽか}き
\ruby{意義}{い|ぎ}の
\ruby{中}{うち}より
\ruby{生}{うま}れたる
\ruby{子}{こ}なり、
\ruby{神}{かみ}の
\ruby{子}{こ}なり
\ruby{佛}{ほとけ}の
\ruby{子}{こ}なり
\ruby{正眞}{ま|こと}の
\ruby{子}{こ}なり、
\ruby{我}{われ}と
\ruby{神佛}{かみ|ほとけ}とは
\ruby{血}{ち}の
\ruby{相{\換字{通}}}{あひ|かよ}へるなり、と
\ruby{如是思}{か|く|おも}ふ
\ruby{時}{とき}おのづと
\ruby{{\換字{悅}}}{よろこ}ばしからば、
\ruby[g]{水野}{みづの}は
\ruby{今}{いま}まさに
\ruby{此}{こ}の
\ruby{{\換字{悅}}}{よろこ}びをおぼえたるなるべし。

\ruby[g]{水野}{みづの}のやうやく
\ruby{念}{ねん}じ
\ruby{{\換字{終}}}{をは}われる
\ruby{時}{とき}、
\ruby{老人}{らう|じん}はまた
\ruby[g]{水野}{みづの}に
\ruby{對}{むか}ひて、

『あ〻
\ruby{御信心}{ご|しん|〴〵}なさいまし〳〵、
\ruby{自然}{ひと|りで}に
\ruby{有}{あ}りがたいことが
\ruby{能}{よ}く
\ruby{解}{わか}つてまゐります!。
まあ
\ruby{何様}{ど|ん}な
\ruby{事}{こと}か
\ruby{存}{ぞん}じませんが、
\ruby{御様子}{ご|やう|す}を
\ruby{見}{み}ましたところでは、よく〳〵の
\ruby{御心配事}{ご|しん|ぱい|ごと}が
\ruby{御有}{お|あ}りなさると
\ruby{御察}{お|さつ}し
\ruby{申}{まをし}ます。
\ruby{御籤}{おみ|くじ}を
\ruby{御戴}{おい|たヾ}きなさい、
\ruby{御籤}{おみ|くじ}を
\ruby{御戴}{おい|たヾ}きなさい。
あ〻まだ
\ruby{御戴}{おい|たヾ}きなさつた
\ruby{事}{こと}が
\ruby{御有}{お|あ}んなさらないので、
\ruby{御{\換字{勝}}手}{ご|かつ|て}が
\ruby{知}{し}れないのでございますネ。
\ruby{宣}{よ}うございます
\ruby{私}{わたくし}が
\ruby{戴}{いただ}いてあげましやう。
』

と、
\ruby{世話}{せ|わ}を
\ruby{燒}{や}きて
\ruby[g]{水野}{みづの}がまだ
\ruby{何}{なに}とも
\ruby{答}{こたへ}をせざるに、はや
\ruby{御籤}{み|くじ}を
\ruby{管}{つかさど}る
\ruby{僧}{そう}の
\ruby{許}{もと}に
\ruby{至}{いた}りぬ。

やがて
\ruby{僧}{そう}は
\ruby[g]{御籤箱}{おみくじばこ}をふるなるべし、かた〳〵といふ
\ruby{音}{おと}は
\ruby{小暗}{お|ぐら}き
\ruby{其}{そ}の
\ruby{座}{ざ}のあたりより
\ruby{聞}{きこ}\GWI{u1b001}ぬ。

