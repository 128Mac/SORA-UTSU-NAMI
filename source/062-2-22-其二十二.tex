\Entry{其二十二}

% メモ 校正終了 2024-04-23 2024-06-01
\原本頁{120-6}%
\ruby{經}{きやう}は
\ruby{誦}{じゆ}したり
といへども
\ruby[g]{老人}{らうじん}
\ruby[g]{{\換字{迷}}魂}{めいこん}の
\ruby{{\換字{術}}}{じゆつ}を
\ruby{知}{し}れる
にもあらず、
%
\ruby[<j||]{心}{こゝろ}こそ% 踊り字調整「〻(二の字点、揺すり点)に見えるが(ゝ)」% 行末行頭の境界付近なので特例処置を施す
\ruby{惑}{まど}ひたれ
\ruby[g]{水野}{みづの }
\ruby[g]{奪魄}{だつぱく}の
\ruby{法}{はふ}に
\ruby{致}{いた}さる
べくも
あらねど、
%
\ruby[g]{水野}{みづの }が
\ruby[<j||]{胸}{きやう}
\ruby[||j>]{中}{ちう}の
% \ruby{胸中}{きやう|ちう}の
\ruby[g]{{\換字{消}}息}{せうそく}は
\ruby[g]{水野}{みづの }
ばかりぞ
\ruby{知}{し}る、
%
\ruby[g]{傍觀}{わきめ }より
\ruby{云}{い}へば
たゞ% 踊り字調整「〻(二の字点、揺すり点)に濁点に見えるが(ゞ)」
\ruby{是}{これ}
\ruby[||j>]{恰}{あたか}も% 恰も「あ(た)かも」
\ruby[g]{神{\換字{文}}}{しんもん}
\原本頁{120-9}\改行%
\ruby[g]{密呪}{みつじゆ}の
\ruby{妖}{あや}しき
\ruby{{\換字{道}}}{みち}に
\ruby{因}{よ}つて
\ruby{縛心{\換字{鎖}}意}{フア|ツシ|{\換字{子}}ー|ト}
されたる
\ruby{人}{ひと}の
\ruby{如}{ごと}く、
%
\ruby{今}{いま}までの
\ruby[g]{水野}{みづの }
\ruby[<j>]{某}{なにがし}は
いづくへやら
\ruby{{\換字{消}}}{き}{\換字{𛀁}}て、
%
\ruby{全}{まつた}く
\ruby[g]{愚痴}{ぐ ち }
\ruby[g]{{\換字{文}}盲}{もんもう}の
\ruby{爺}{ぢゝ}% 踊り字調整「〻(二の字点、揺すり点)に見えるが(ゝ)」
\ruby{婆}{ばゝ}% 踊り字調整「〻(二の字点、揺すり点)に見えるが(ゝ)」
のやうになり、
%
\ruby[g]{一心}{いつしん}に
\ruby[g]{御佛}{みほとけ}を
\ruby{頼}{たの}み
\ruby[<j>]{奉}{たてまつ}れる
さまの、
%
\ruby[g]{男兒}{をとこ }らしからず
\ruby[g]{憫然}{あはれ }にのみ
\ruby{見}{み}{\換字{𛀁}}たり。

\原本頁{121-3}%
\ruby{西}{にし}に
\ruby{對}{むか}ひて
\ruby{放}{はな}ちても
\ruby{東}{ひがし}に
\ruby{對}{むか}ひて
\ruby{放}{はな}ちても、
%
\ruby{滿}{み}つる
\ruby{月}{つき}の
\ruby{形}{かたち}と
\ruby[g]{引絞}{ひきしぼ}りたる
\ruby[g]{{\換字{強}}弓}{がうきう}を、
%
きつて
\ruby{放}{はな}つ
\ruby{時}{とき}
おのづからの
\ruby[<j>]{快}{こゝろよ}さ% 踊り字調整「〻(二の字点、揺すり点)に見えるが(ゝ)」
あり。
%
\ruby{南}{みなみ}に
むか
\原本頁{121-5}\改行%
ひて
\ruby{决}{けつ}しても
\ruby{北}{きた}に
\ruby{對}{むか}ひて
\ruby{决}{けつ}しても、
%
\ruby[g]{千頃}{せんけい}の% 「千頃」物事を、ある基準で区分けしたときの一つ一つ。
\ruby[||j>]{瀦}{たまり}
\ruby[||j>]{水}{ みづ}の
% \ruby{瀦水}{たまり|みづ}の
\ruby[g]{漫々}{まん〳〵}たるを、
%
\原本頁{121-6}\改行%
\ruby[||j>]{堤}{つゝみ}を% 踊り字調整「〻(二の字点、揺すり点)に見えるが(ゝ)」
\ruby{切}{き}つて
\ruby{决}{けつ}する
\ruby{時}{とき}
おのづからの
\ruby[<j>]{快}{こゝろよ}さあり。% 踊り字調整「〻(二の字点、揺すり点)に見えるが(ゝ)」
%
そも〳〵
\ruby{心}{こゝろ}の% 踊り字調整「〻(二の字点、揺すり点)に見えるが(ゝ)」
\ruby{後}{あと}へも
\ruby{先}{さき}へも
\ruby{行}{ゆ}かざるを
\ruby{悶}{もだ{\換字{𛀁}}}とは
\ruby{云}{い}ひ、
%
\ruby[g]{一方}{いつぱう}へ
\ruby{爽}{さわや}かに
\ruby{走}{はし}るを
\ruby[<j>]{快}{こゝろよ}しとは% 踊り字調整「〻(二の字点、揺すり点)に見えるが(ゝ)」
\原本頁{121-8}\改行%
\ruby{云}{い}ふなれば、
%
\ruby[g]{佛陀}{ほとけ }の
\ruby[g]{利益}{り やく}は
\ruby{有}{あ}るにせよ
\ruby{無}{な}きにせよ、
%
\ruby[g]{水野}{みづの }は
\ruby{今}{いま}まさに
\ruby{此}{こ}の
\ruby[<j>]{快}{こゝろよ}さを% 踊り字調整「〻(二の字点、揺すり点)に見えるが(ゝ)」
\ruby{味}{あぢは}へる
なるべし。

\原本頁{121-10}%
\ruby[g]{星辰}{せいしん}
\ruby{上}{かみ}に
かゝり、% 踊り字調整「〻(二の字点、揺すり点)に見えるが(ゝ)」
%
\ruby[g]{山河}{さんが }
\ruby{下}{しも}に
\ruby{布}{し}ける
\ruby{此}{こ}の
\ruby[g]{天地}{てんち }の
\ruby{大}{だい}にして
\ruby{大}{だい}なるを
おもひ、
%
\ruby{萬年萬々年}{ばん|ねん|ばん|〳〵|ねん}% 「〴〵」でなく原本通り「〳〵」
の
\ruby{{\換字{前}}}{まへ}に
\ruby{萬年萬々年}{ばん|ねん|ばん|〳〵|ねん}% 「〴〵」でなく原本通り「〳〵」
あり、
%
\ruby{萬年萬々年}{ばん|ねん|ばん|〳〵|ねん}% 「〴〵」でなく原本通り「〳〵」
の
\ruby{後}{のち}に
\原本頁{122-1}%
\ruby{萬年萬々年}{ばん|ねん|ばん|〳〵|ねん}% 「〴〵」でなく原本通り「〳〵」
ある
\ruby{此}{こ}の
\ruby[g]{歳月}{さいげつ}の
\ruby{久}{ひさ}しくして
\ruby{久}{ひさ}しきを
\ruby{思}{おも}ひ、
%
さて
\ruby{此}{こ}の
\ruby[g]{天地}{てんち }の
\ruby{立}{た}てる
\ruby[g]{{\換字{所}}以}{ゆ{\換字{𛀁}}ん}を
おもひ
\ruby[g]{歳月}{さいげつ}の
\ruby{經}{ふ}る
\ruby[g]{{\換字{所}}以}{ゆ{\換字{𛀁}}ん}を
\ruby{思}{おも}ひて、
%
\ruby{此}{こ}の
\ruby[g]{天地}{てんち }と
\ruby[g]{歳月}{さいげつ}との
\ruby[g]{存在}{そんざい}を、
%
たゞ〳〵% 踊り字調整「〻(二の字点、揺すり点)に濁点に見えるが(ゞ)」
\ruby{無}{む}
\ruby[g]{意義}{い ぎ }なる
\ruby[g]{事實}{ことがら}のみと
\ruby{認}{みと}めなば、
%
\原本頁{122-4}\改行%
\ruby{誰}{たれ}かは
\ruby[g]{味氣}{あぢき }
\ruby{無}{な}き
\ruby{{\換字{感}}}{おもひ}に
\ruby{撲}{う}たれて
\ruby{悲}{かなし}み
\ruby{傷}{いた}まざらん。
%
されど
\ruby{此}{こ}の
\ruby[g]{天地}{てんち }と
\ruby[g]{歳月}{さいげつ}との
\ruby[g]{存在}{そんざい}の、
%
\ruby{眞}{まこと}は
\ruby{無}{む}
\ruby[g]{意義}{い ぎ }の
\ruby[g]{事實}{ことがら}のみ
ならで、
%
\ruby{其}{その}
\ruby{中}{うち}に
\ruby[g]{意義}{い ぎ }ある
なりと
\ruby{認}{みと}むる
\ruby{時}{とき}は、
%
\ruby{誰}{たれ}かは
\ruby[g]{{\換字{乳}}{\換字{房}}}{ち ぶさ}を
\ruby{探}{さぐ}り
\ruby{得}{{\換字{𛀁}}}たる
\ruby[g]{嬰兒}{あかご }の
\ruby{如}{ごと}く、
%
\原本頁{122-7}\改行%
\ruby[g]{無限}{む げん}の
\ruby[g]{喜悅}{よろこび}に
\ruby{胸}{むね}を
\ruby{躍}{をど}らさゞらん。% 踊り字調整「〻(二の字点、揺すり点)に濁点に見えるが(ゞ)」
%
\ruby[g]{意義}{い ぎ }あり、
%
\ruby[g]{意義}{い ぎ }あり、
%
\ruby{無}{む}
\ruby[g]{意義}{い ぎ }ならず、
%
\ruby{神}{かみ}の
\ruby[g]{御心}{みこゝろ}% 踊り字調整「〻(二の字点、揺すり点)に見えるが(ゝ)」
\ruby{{\換字{即}}}{すなは}ち
\ruby[g]{意義}{い ぎ }なり、
%
\ruby{佛}{ほとけ}の
\ruby[g]{御心}{みこゝろ}% 踊り字調整「〻(二の字点、揺すり点)に見えるが(ゝ)」
\ruby{{\換字{即}}}{すなは}ち
\ruby[g]{意義}{い ぎ }なり、
%
\ruby[<j||]{化}{くわ }% 行末行頭の境界付近なので特例処置を施す
\ruby[<j||]{醇}{じゆん}の
% \ruby{化醇}{くわ|じゆん}の
\ruby[g]{大法}{おきて }は
こゝにあるなり、% 踊り字調整「〻(二の字点、揺すり点)に見えるが(ゝ)」
%
\ruby[g]{歸善}{き ぜん}の
\ruby[g]{定數}{さだまり}
こゝにあるなり、% 踊り字調整「〻(二の字点、揺すり点)に見えるが(ゝ)」
%
\ruby[g]{大慈}{だいじ }の
\原本頁{122-10}\改行%
\ruby[g]{光明}{ひかり }は
\ruby{柔}{やはら}かに
\ruby[g]{山村}{さんそん}
\ruby[||j>]{水}{すゐ}
\ruby[||j>]{鄕}{きやう}を
% \ruby{水鄕}{すゐ|きやう}を
\ruby{包}{つゝ}めるなり、% 踊り字調整「〻(二の字点、揺すり点)に見えるが(ゝ)」
%
\ruby[g]{大悲}{だいひ }の
\ruby[g]{音樂}{おんがく}は
\ruby{斷}{た}ゆる
\ruby{間}{ま}も
\ruby{無}{な}く
\ruby[g]{{\換字{古}}往}{こ わう}
\ruby[g]{今來}{こんらい}に
\ruby{亘}{わた}れるなり、
%
\ruby{我}{われ}は
\ruby{此}{こ}の
\ruby[g]{溫{\換字{暖}}}{あたゝか}き% 踊り字調整「〻(二の字点、揺すり点)に見えるが(ゝ)」
\ruby[g]{意義}{い ぎ }の
\ruby{中}{うし}% ルビ調整(原本通り)% 印刷不明瞭かと思い国会図書館のものもチェックして(し)を確認
\footnote{原本通り「中(うし)」とする(国会図書館 コマ番号66/160 p-122 l-11)}%
より
\ruby{生}{うま}れたる
\ruby{子}{こ}なり、
%
\ruby{神}{かみ}の
\ruby{子}{こ}なり
\ruby{佛}{ほとけ}の
\ruby{子}{こ}なり
\ruby[g]{正眞}{まこと }の
\ruby{子}{こ}なり、
%
\ruby{我}{われ}と
\ruby[||j>]{神}{かみ}
\ruby[||j>]{佛}{ほとけ}とは
% \ruby{神佛}{かみ|ほとけ}とは
\ruby{血}{ち}の
\ruby{相}{あひ}
\ruby{{\換字{通}}}{かよ}へる
なり、
%
と
\ruby[g]{如是}{か く }
\ruby{思}{おも}ふ
\ruby{時}{とき}
おのづと
\ruby{悅}{よろこ}ばしからば、
%
\ruby[g]{水野}{みづの }は
\ruby{今}{いま}
きさ%
\footnote{「きさ【×詭詐】うそをつくこと・偽ること・譎詐(きっさ)」という意味もあることから原本通りとする
(国会図書館 コマ番号66/160 p-123 l-03)}%
% 「きさ」はどれかな?
% きさ【×詭詐】 うそをつくこと。偽ること。譎詐(きっさ)。
%    水野が神佛を信ぜぬと言う自身の考えに偽っている様を示すと思うので原本通り。
% きさ【×橒】 材木の木目(もくめ)の模様。
%     「—の木に、鉄(くろがね)の脚つけたる槽(ふね)」〈宇津保・吹上上〉
% きさ【器差】 測定器が実際に示す値と本来示すべき値との差。測定器の製作過程から生じた誤差。
% きさ【機作】 しくみ。機構。メカニズム。
に%
\ruby{此}{こ}の
\ruby{悅}{よろこ}びを
おぼえたる
なるべし。

\原本頁{123-4}%
\ruby[g]{水野}{みづの }の
やうやく
\ruby{念}{ねん}じ
\ruby{{\換字{終}}}{をは}れる
\ruby{時}{とき}、
%
\ruby[g]{老人}{らうじん}は
また
\ruby[g]{水野}{みづの }に
\ruby{對}{むか}ひて、

\原本頁{123-5}%
『
あゝ% 踊り字調整「〻(二の字点、揺すり点)に見えるが(ゝ)」
\ruby{御信心}{ご|しん|〴〵}なさい
まし〳〵、
%
\ruby[g]{自然}{ひとりで}に
\ruby{有}{あ}りがたい
ことが
\ruby{能}{よ}く
\ruby{解}{わか}つて
まゐります!。
%
まあ
\ruby[g]{何樣}{ど ん }な
\ruby{事}{こと}か
\ruby{存}{ぞん}じませんが、
%
\ruby{御樣子}{ご|やう|す}を
\ruby{見}{み}ました
ところでは、
%
よく〳〵の
\ruby{御心配事}{ご|しん|ぱい|ごと}が
\ruby[g]{御有}{お あ }りなさると
\ruby[g]{御察}{お さつ}し
\ruby{申}{まをし}ます。
%
\ruby[g]{御籤}{おみくじ}を
\ruby[g]{御戴}{おいたゞ}きなさい、% 踊り字調整「〻(二の字点、揺すり点)に濁点に見えるが(ゞ)」
%
\ruby[g]{御籤}{おみくじ}を
\ruby[g]{御戴}{おいたゞ}きなさい。% 踊り字調整「〻(二の字点、揺すり点)に濁点に見えるが(ゞ)」
%
あゝ% 踊り字調整「〻(二の字点、揺すり点)に見えるが(ゝ)」
まだ
\ruby[g]{御戴}{おいたゞ}きなさつた% 踊り字調整「〻(二の字点、揺すり点)に濁点に見えるが(ゞ)」
\ruby{事}{こと}が
\ruby[g]{御有}{お あ }んなさらないので、
%
\ruby{御{\換字{勝}}手}{ご|かつ|て}が
\ruby{知}{し}れな
\原本頁{123-10}\改行%
いのでございますネ、
%
\ruby{宜}{よ}うございます
\ruby[<j>]{私}{わたくし}が
\ruby{戴}{いたゞ}いて% 踊り字調整「〻(二の字点、揺すり点)に濁点に見えるが(ゞ)」
あげましやう
\改行% 校正作業の簡略化のため
。
』

\原本頁{123-11}%
と、
%
\ruby[g]{世話}{せ わ }を
\ruby{燒}{や}きて
\ruby[g]{水野}{みづの }が
まだ
\ruby{何}{なに}とも
\ruby{答}{こたへ}を
せざるに、
%
はや
\ruby[g]{御籤}{み くじ}を
\原本頁{124-1}\改行%
\makeatletter
\@ifundefined{デバッグ@ビルド}{%
  \ruby[<g>]{管}{つかさど}る%
  \ruby{僧}{そう}の%
}{%
  \ruby[||j>]{管る}{つかさど|}% ルビ調整(特殊処理)行頭の長いルビ対策「(る)を含めた」
  \ruby[||j>]{僧}{ そう}の% 「管(つかさど)る」のルビ対策
}%
\makeatother
\ruby{許}{もと}に
\ruby{至}{いた}りぬ。

\原本頁{124-2}%
やがて
\ruby{僧}{そう}は
\ruby{御籤箱}{お|みくじ|ばこ}を
ふる
なる
べし、
%
かた〳〵
といふ
\ruby{音}{おと}は
\ruby[g]{小暗}{を ぐら}き
\ruby{其}{そ}の
\ruby{座}{ざ}の
あたりより
\ruby{聞}{きこ}{\換字{𛀁}}ぬ。
