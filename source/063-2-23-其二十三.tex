\Entry{其二十三}

よしや
\ruby{大吉}{だい|きち}ならぬまでもせめては
\ruby{凶}{きよう}ならぬ
\ruby{御籤}{み|くじ}を
\ruby{得}{え}て、
\ruby{憂}{うれひ}に
\ruby{沈}{しづ}み
\ruby{悲}{かなしみ}に
\ruby{陷}{おちゐ}れる
\ruby{氣}{き}を
\ruby{引立}{ひき|た}て、
\ruby{信心}{しん|じん}の
\ruby{勇}{いさみ}を
\ruby{附}{つ}けて
\ruby{{\換字{呉}}}{く}れんと
\ruby{爲}{し}たるらしき
\ruby{親切}{しん|せつ}の
\ruby{老人}{らう|じん}が、
\ruby{思}{おも}ふこと
\ruby{{\換字{違}}}{たが}ひて
\ruby{甚}{いた}く
\ruby{望}{のぞみ}を
\ruby{失}{うしな}へるは、
\ruby{忽}{たちま}ち
\ruby{先}{ま}づ
\ruby{其}{そ}の
\ruby{色}{いろ}に
\ruby{現}{あらは}れて、
\ruby{僧}{そう}より
\ruby{受取}{うけ|と}りし
\ruby{御籤}{み|くじ}をば、
\ruby{力無}{ちから|な}げに
\ruby{輪}{わ}に
\ruby{{\換字{巻}}}{ま}きながら、
\ruby{鈍}{にぶ}る〳〵
\ruby{此方}{こな|た}へ
\ruby{步}{あゆ}み
\ruby{來}{きた}れるに、
\ruby{水野}{みづ|の}は
\ruby{見}{み}ずして
\ruby{既}{すで}に
\ruby{其}{そ}の
\ruby{{\換字{文}}}{ぶん}の
\ruby{凶}{きよう}なるを
\ruby{知}{し}れり。

\ruby{第何十何番大吉}{だい|なん|じう|なん|ばん|だい|きち}といふならば、
\ruby{如何}{い|か}ばかりか
\ruby{{\換字{悅}}}{よろこ}び
\ruby{勇}{いさ}んで
\ruby{示}{しめ}すべきを、
\ruby{老人}{らう|じん}は
\ruby{{\換字{巻}}}{ま}きたるま〻
\ruby{御籤}{み|くじ}を
\ruby{水野}{みづ|の}の
\ruby{懷中}{ふと|ころ}に
\ruby{輕}{かる}く
\ruby{押入}{おし|い}れて、

『
\ruby{何樣}{ど|う}か
\ruby{吉凶}{よし|あし}にかゝはらず
\ruby{御信心}{ご|しん|〴〵}なさい。
\ruby{大吉}{だい|きち}でも
\ruby{驕}{おご}れば
\ruby{凶}{きよう}に
\ruby{反}{かへ}ります、たとへ
\ruby{凶}{きよう}でも
\ruby{御信心}{ご|しん|〴〵}を
\ruby{{\換字{強}}}{つよ}くなすつて、それからまた
\ruby{改}{あらた}めて
\ruby{御籤}{おみ|くじ}を
\ruby{御戴}{おい|たゞ}きなすつてごらんなさい、
\ruby{吉}{きち}になりますこともございますものです。
\ruby{吉}{よい}につけ
\ruby{凶}{わるい}につけ
\ruby{御信心}{ご|しん|〴〵}が
\ruby{大切}{たい|せつ}です。
\ruby{決}{けつ}して
\ruby{信}{しん}を
\ruby{御冷}{お|さま}しなすつてはいけません。
さてそろ〳〵もう
\ruby{下向}{げ|かう}いたしましやう。
』

と、
\ruby{云}{い}ひ
\ruby{{\換字{終}}}{をは}つて
\ruby{本尊}{ほん|ぞん}をまた
\ruby{一拜}{いつ|ぱい}して、おのれ
\ruby{先}{ま}づ
\ruby{御堂}{み|だう}を
\ruby{去}{さ}らんとしたり。

\ruby{老人}{らう|じん}が
\ruby{樣子}{やう|す}の
\ruby{急}{きふ}にそはつけるは、
\ruby{何}{なん}の
\ruby{意}{こゝろ}も
\ruby{無}{な}かりし
\ruby{我}{われ}に
\ruby{智慧}{ち|ゑ}をつけて
\ruby{御籤}{み|くじ}を
\ruby{取}{と}らせたるに、その
\ruby{御籤}{み|くじ}のことのほか
\ruby{凶}{あし}かりしかば、
\ruby{却}{かへ}つて
\ruby{其}{そ}のために
\ruby{憂}{うれひ}を
\ruby{{\換字{増}}}{ま}し、
\ruby{悲}{かなしみ}を
\ruby{添}{そ}ふることもやと、
\ruby{氣}{き}の
\ruby{毒}{どく}さに
\ruby{堪}{た}へかねて
\ruby{傍}{かたへ}に
\ruby{居}{ゐ}づらく
\ruby{狭}{せま}くして
\ruby{正直}{しやう|ぢき}なる
\ruby{心}{こゝろ}の
\ruby{憫}{あは}れにも
\ruby{落着}{おち|つ}きかぬるが
\ruby{爲}{ため}なるべし。
\ruby{{\換字{平}}生}{ひご|ろ}の
\ruby{我}{われ}を
\ruby{知}{し}らずして、たゞ
\ruby{自己}{お|の}が
\ruby{身}{み}にのみ
\ruby{比較}{ひき|くら}ぶれば、
\ruby{然}{ま}る
\ruby{心{\換字{遣}}}{こゝろ|づかひ}をするも
\ruby{無理}{む|り}ならねど、
\ruby{御佛}{み|ほとけ}の
\ruby{廣大}{くわう|だい}なる
\ruby{御誓願}{おん|ちか|ひ}をこそ
\ruby{頼}{たの}み
\ruby{奉}{たてまつ}りつれ、
\ruby{御鬮}{み|くじ}といふ
\ruby{事}{こと}は
\ruby{御經}{おん|きやう}にも
\ruby{見}{み}えず、
\ruby{賣僧}{まい|す}の
\ruby{仕出}{し|だ}したるなるべき
\ruby{春}{はる}の
\ruby{{\換字{遊}}{\換字{戱}}}{あそ|び}の
\ruby{寶引}{はう|びき}といふにも
\ruby{似}{に}たる
\ruby{埒無}{らち|な}く
\ruby{據無}{よりどころ|な}き
\ruby{御籤}{み|くじ}の
\ruby{{\換字{文}}}{ぶん}なんどに、
\ruby{我}{われ}いかで
\ruby{心}{こゝろ}を
\ruby{動}{うご}かされんや。
それとも
\ruby{知}{し}らずして
\ruby{性質}{ひ|と}の
\ruby{好}{よ}き
\ruby{老人}{らう|じん}の、
\ruby{心}{こゝろ}を
\ruby{{\換字{遣}}}{つか}ふ
\ruby{笑止}{せう|し}さ、と
\ruby{水野}{みづ|の}は
\ruby{却}{かへ}つて
\ruby{老人}{らう|じん}を
\ruby{憐}{あはれ}み、わざと
\ruby{懷中}{くわい|ちう}の
\ruby{御籤}{み|くじ}を
\ruby{其儘}{その|まゝ}にして
\ruby{讀}{よ}まず。
\ruby{共}{とも}に
\ruby{石路}{せき|ろ}の
\ruby{長々}{なが|〳〵}しきを
\ruby{下向}{げ|かう}しけるが、
\ruby{老人}{らう|じん}は
\ruby{懷中}{ふと|ころ}より
\ruby{折本}{をり|ほん}になりたる
\ruby{普門品}{ふ|もん|ほん}の
\ruby{小}{ちひさ}きを
\ruby{取}{と}り
\ruby{出}{いだ}して、

『だいなしになつて
\ruby{居}{を}りまする
\ruby{物}{もの}を、
\ruby{呈}{あ}げると
\ruby{申}{まを}しては
\ruby{失禮}{しつ|れい}ですけれど、まあ
\ruby{如是}{か|う}いふ
\ruby{物}{もの}の
\ruby{事}{こと}ですから
\ruby{御免下}{ご|めん|くだ}さい。
これを
\ruby{貴君}{あな|た}に
\ruby{差上}{さし|あ}げますから、
\ruby{何樣}{ど|う}か
\ruby{御取}{お|と}りなすつて
\ruby{下}{くだ}さいまし。
\ruby{私}{わたくし}はもう
\ruby{無書}{そ|ら}で
\ruby{記}{おぼ}\換字{江}ましたから、
\ruby{此書}{こ|れ}は
\ruby{用}{よう}が
\ruby{明}{あ}いたのでございますが、
\ruby{何樣}{ど|う}か
\ruby{貴君}{あな|た}も
\ruby{御拜}{お|が}みなさるたびに、これを
\ruby{御覧}{ご|らん}になりながら
\ruby{御經}{お|きやう}を
\ruby{御}{お}あげなすつて
\ruby{下}{くだ}されば、
\ruby{私}{わたくし}は
\ruby{大變}{たい|へん}に
\ruby{嬉}{うれ}しいと
\ruby{思}{おも}ふのでございます。
それに
\ruby{此}{こ}の
\ruby{末}{すゑ}の
\ruby{方}{はう}に
\ruby{私}{わたくし}の
\ruby{名住{\換字{所}}}{な|とこ|ろ}が
\ruby{小}{ちひ}さく
\ruby{書}{か}いてございますから、
\ruby{何}{なん}ぞの
\ruby{御序}{おつ|ひで}でも
\ruby{御有}{お|あ}りでしたら
\ruby{御立寄}{お|たち|よ}り
\ruby{下}{くだ}さいまし、いろ〳〵
\ruby{御利生}{ご|り|しやう}の
\ruby{御話}{お|はなし}やなんぞを
\ruby{致}{いた}しましやうから。
ではまた
\ruby{明日御目}{みやう|にち|お|め}にかゝりましやう。
どうか
\ruby{撓}{たゆ}まずに
\ruby{御信心}{ご|しん|〴〵}なすつて!。
』

と
\ruby{云}{い}ひたき
\ruby{事}{こと}のみを
\ruby{云}{い}ひて
\ruby{{\換字{終}}}{つひ}に
\ruby{別}{わか}れたり。

\ruby{冊子}{ほ|ん}は
\ruby{言}{ことば}を
\ruby{費}{つひや}して
\ruby{辭}{いな}むべきほどのものにもあらず、
\ruby{特}{こと}に
\ruby{快}{こゝろよ}く
\ruby{受}{う}け
\ruby{納}{をさ}めて
\ruby{芳志}{こゝろ|ざし}を
\ruby{無}{む}にせざらんは、
\ruby{差}{さ}し
\ruby{當}{あた}つての
\ruby{{\換字{道}}}{みち}なるべしと、
\ruby{水野}{みづ|の}は
\ruby{老人}{らう|じん}に
\ruby{厚意}{かう|い}を
\ruby{謝}{しや}して、
\ruby{袖}{そで}を
\ruby{{\換字{分}}}{わか}つて
\ruby{東方}{ひが|し}へ
\ruby{去}{さ}りつ、
\ruby{先}{ま}づ
\ruby{普門品}{ふ|もん|ぼん}を
\ruby{懷中}{ふと|ころ}に
\ruby{入}{い}るゝに、
\ruby{{\換字{巻}}}{ま}きたる
\ruby{彼}{か}の
\ruby{御籤}{み|くじ}のかさ〳〵と
\ruby{手}{て}に
\ruby{觸}{ふ}れたれば、
\ruby{引{\換字{交}}}{ひき|ちが}へて
\ruby{取}{と}り
\ruby{出}{いだ}して
\ruby{其{\換字{文}}}{その|ぶん}を
\ruby{讀}{よ}むに、\\

\hspace*{1zw}
% 返り点参照情報
% https://www.asahi-net.or.jp/~ax2s-kmtn/ref/unicode/u3190.html
\begin{tblr}{colspec={Q[c] | Q[l,t] Q[l,b]}, stretch=0.5}
  \SetCell[r=4]{c,1em}{第七番凶}&
  \kundoku{登}{ふねにの}{}{㆑}% 「㆑(u3191)レ点」「レ(u30ec)カタカナ」
  \kundoku{舟}{ぼりて }{}{}
  \kundoku{待}{びんぷう}{}{㆓}% 「㆓(u3293)」
  \kundoku{便}{をまてば}{}{}
  \kundoku{風}{   }{}{㆒}。% 「㆒(u3192)」

  & \scriptsize{\noindent
    舟にのりて行かんとす\newline
    ればおひてが無い
  }\\
  %%%
  &
  \kundoku{月}{げつ し}{}{}
  \kundoku{色}{よく く}{}{}
  \kundoku{暗}{らくして}{}{}
  \kundoku{朦}{もう  }{}{}
  \kundoku{朧}{ろう }{}{}。

  & \scriptsize{\noindent
    見れば空もわるくして\\
    月もくらきぞ
  }\\
  %%%
  &
  \kundoku{欲}{かうりん}{}{㆘}% 「㆓(u3298)」
  \kundoku{輾}{をきしら}{}{㆓}% 「㆓(u3293)」
  \kundoku{香}{してさら}{}{}
  \kundoku{輪}{んとほつ}{}{㆒}% 「㆒(u3192)」
  \kundoku{去}{すれば}{}{㆖}。% 「㆒(u3196)」

  & \scriptsize{\noindent
    車にのりておもふとこ\\
    ろへゆかんとすれば
  }\\
  &
  \kundoku{高}{かう  }{}{}
  \kundoku{山}{ざん  }{}{}
  \kundoku{千}{せん  }{}{}
  \kundoku{萬}{ばん  }{}{}
  \kundoku{里}{りなり}{}{}。

  & \scriptsize{\noindent
    つゞける山〻恐ろしく\\ %% 山〻 vs 山々
    高くしてそれも叶はぬ
  }
\end{tblr}
 \\
 \\
とありて、ひし〳〵と
\ruby{我}{わ}が
\ruby{身}{み}の
\ruby{上}{うへ}に
\ruby{巧}{よ}く
\ruby{中}{あた}りたり。

もとより
\ruby{取}{と}るに
\ruby{足}{た}らぬことゝは
\ruby{思}{おも}ひながらも、
\ruby{不思議}{ふ|し|ぎ}に
\ruby{中}{あた}れる
\ruby{此}{こ}の
\ruby{{\換字{文}}}{ぶん}の
\ruby{流石}{さす|が}に
\ruby{胸}{むね}に
\ruby{徹}{こた}へて
\ruby{心}{こゝろ}さびしく、じつと
\ruby{眼}{め}を
\ruby{{\換字{留}}}{と}めて
\ruby{見}{み}れば、
\ruby{末}{すゑ}の
\ruby{方}{かた}に
\ruby[<h||]{女}{をんな}
\ruby{{\換字{文}}字}{も|じ}にて
\ruby{細}{こまか}に
\ruby{注}{ちう}し
\ruby{記}{しる}せる
\ruby{其最先}{その|まつ|さき}に、

\ruby{病事}{やまひ|ごと}は
\ruby{十}{じう}に
\ruby{六七}{ろく|しち}
\ruby{本復無}{ほん|ぷく|な}し、
\ruby{長}{なが}びきたらば
\ruby{後}{のち}は
\ruby{息災}{そく|さい}になる
\ruby{事}{こと}もあるべし、よく
\ruby{信力}{しん|りき}をもて
\ruby{佛神}{ぶつ|しん}を
\ruby{頼}{たの}みて
\ruby{吉}{よし}、

とありたるは、いよ〳〵
\ruby{何}{なに}となく
\ruby{不快}{ふ|くわい}を
\ruby{感}{かん}じて、
\ruby{腹}{はら}の
\ruby{底}{そこ}より
\ruby{{\換字{寒}}}{さむさ}の
\ruby{上}{のぼ}り
\ruby{來}{きた}るやうにおぼえたり。

\ruby{何}{なに}とか
\ruby{思}{おも}ひけん
\ruby{水野}{みづ|の}は
\ruby{引{\換字{返}}}{ひつ|かへ}して、
\ruby{復}{また}
\ruby{相良}{さが|ら}を
\ruby{訪}{と}ひぬ。
\ruby{待}{ま}つ
\ruby{事一時餘}{こと|いち|じ|あま}りにして
\ruby{{\換字{終}}}{つひ}に
\ruby{相良}{さが|ら}に
\ruby{親}{した}しく
\ruby{會}{あ}ひ
\ruby{得}{\換字{江}}て、
\ruby{必}{かなら}ず
\ruby{見舞}{み|ま}はんとの
\ruby{辭}{ことば}を
\ruby{得}{\換字{江}}て
\ruby{歸}{かへ}りしが、
\ruby{幸}{さいはひ}にして
\ruby{今日}{け|ふ}は
\ruby{休校}{やす|み}の
\ruby{日}{ひ}なればこそ
\ruby{宣}{よ}けれ、
\ruby{吾妻橋}{あ|づま|ばし}にかゝれる
\ruby{時}{とき}は
\ruby{既}{すで}に
\ruby{九時}{く|じ}に
\ruby{{\換字{近}}}{ちか}からんとしたり。
