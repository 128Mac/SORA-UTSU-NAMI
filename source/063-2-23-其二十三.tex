\Entry{其二十三}

% メモ 校正終了 2024-04-23
\原本頁{124-5}%
よしや
\ruby{大吉}{だい|きち}
ならぬ
までも
せめては
\ruby{凶}{きよう}
ならぬ
\ruby{御籤}{み|くじ}を
\ruby{得}{え}て、
%
\ruby{憂}{うれひ}に
\ruby{沈}{しづ}み
\ruby{悲}{かなしみ}に
\ruby{陷}{おちゐ}れる
\ruby{氣}{き}を
\ruby{引立}{ひき|た}て、
%
\ruby{信心}{しん|〴〵}の
\ruby{勇}{いさみ}を
\ruby{附}{つ}けて
\ruby{吳}{く}れんと
\ruby{爲}{し}たるらしき
\ruby{親切}{しん|せつ}の
\ruby{老人}{らう|じん}が、
%
\ruby{思}{おも}ふこと
\ruby{{\換字{違}}}{たが}ひて
\ruby{甚}{いた}く
\ruby{望}{のぞみ}を
\ruby{失}{うしな}へるは、
%
\ruby{忽}{たちま}ち
\ruby{先}{ま}づ
\ruby{其}{そ}の
\ruby{色}{いろ}に
\ruby{現}{あらは}れて、
%
\ruby{僧}{そう}より
\ruby{受取}{うけ|と}りし
\ruby{御籤}{み|くじ}をば、
%
\ruby[<j||]{力}{ちから}
\ruby{無}{な}げに
\ruby{輪}{わ}に
\ruby{卷}{ま}き
ながら、
%
\ruby{鈍}{にぶ}る〳〵
\ruby{此方}{こな|た}へ
\ruby{步}{あゆ}み
\ruby{來}{きた}れるに、
%
\ruby{水野}{みづ|の}は
\ruby{見}{み}ずして
\ruby{既}{すで}に
\ruby{其}{そ}の
\ruby{{\換字{文}}}{ぶん}の
\ruby{凶}{きよう}なるを
\ruby{知}{し}れり。

\原本頁{125-1}%
\ruby{第何十何番}{だい|なん|じう|なん|ばん}
\ruby{大吉}{だい|きち}
といふならば、
%
\ruby{如何}{い|か}
ばかりか
\ruby{悅}{よろこ}び
\ruby{勇}{いさ}んで
\ruby{示}{しめ}すべきを、
%
\ruby{老人}{らう|じん}は
\ruby{卷}{ま}きたる
ま〻% 原本通り「〻(二の字点、揺すり点)」
\ruby{御籤}{み|くじ}を
\ruby{水野}{みづ|の}の
\ruby{懷中}{ふと|ころ}に
\ruby{輕}{かる}く
\ruby{押入}{おし|い}れて、

\原本頁{125-3}%
『
\ruby{何樣}{ど|う}か
\ruby{吉}{よし}
\ruby{凶}{あし}に
か〻はらず% 原本通り「〻(二の字点、揺すり点)」
\ruby{御信心}{ご|しん|〴〵}なさい。
%
\ruby{大吉}{だい|きち}でも
\ruby{驕}{おご}れば
\ruby{凶}{きよう}に
\ruby{反}{かへ}ります、
%
たとへ
\ruby{凶}{きよう}でも
\ruby{御信心}{ご|しん|〴〵}を
\ruby{{\換字{強}}}{つよ}く
なすつて、
%
それから
また
\ruby{改}{あらた}めて
\ruby{御籤}{お|みくじ}を
\ruby{御戴}{お|いたゞ}き% TODO 原本の「二の字点、揺すり点」に濁点のグリフが見つからないので「ゞ」
なすつて
ごらんなさい、
%
\ruby{吉}{きち}に
なります
ことも
ございます
ものです。
%
\ruby{吉}{よい}につけ
\ruby{凶}{わるい}につけ
\ruby{御信心}{ご|しん|〴〵}が
\ruby{大切}{たい|せつ}です。
%
\ruby{決}{けつ}して
\ruby{信}{しん}を
\ruby{御冷}{お|さま}し
なすつては
いけません。
%
さて
そろ〳〵
もう
\ruby{下向}{げ|かう}
いたしましやう。
』

\原本頁{125-9}%
と、
%
\ruby{云}{い}ひ
\ruby{{\換字{終}}}{をは}つて
\ruby{本{\換字{尊}}}{ほん|ぞん}を
また
\ruby{一拜}{いつ|ぱい}して、
%
おのれ
\ruby{先}{ま}づ
\ruby{御堂}{み|だう}を
\ruby{去}{さ}らん
としたり。

\原本頁{125-11}%
\ruby{老人}{らう|じん}が
\ruby{樣子}{やう|す}の
\ruby{急}{きふ}に
そはつけるは、
%
\ruby{何}{なん}の
\ruby{意}{こ〻ろ}も% 原本通り「〻(二の字点、揺すり点)」
\ruby{無}{な}かりし
\ruby{我}{われ}に
\ruby{智慧}{ち|ゑ}を
つけて
\ruby{御籤}{み|くじ}を
\ruby{取}{と}らせたるに、
%
その
\ruby{御籤}{み|くじ}の
ことのほか
\ruby{凶}{あし}かりしかば、
%
\ruby{却}{かへ}つて
\ruby{其}{そ}のために
\ruby{憂}{うれひ}を
\ruby{增}{ま}し
%
\ruby{悲}{かなしみ}を
\ruby{添}{そ}ふる
こともやと、
%
\ruby{氣}{き}の
\ruby{毒}{どく}さに
\ruby{堪}{た}へ
かねて
\ruby{傍}{かたへ}に
\ruby{居}{ゐ}づらく、
%
\ruby{狭}{せま}くして
\ruby[<g>]{正直}{しやうぢき}なる
\ruby{心}{こ〻ろ}の% 原本通り「〻(二の字点、揺すり点)」
\ruby{憐}{あは}れにも
\原本頁{126-4}\改行%
\ruby[g]{沈着}{おちつ}き
かぬるが
\ruby{爲}{ため}
なるべし。
%
\ruby{{\換字{平}}生}{ひご|ろ}の
\ruby{我}{われ}を
\ruby{知}{し}らず
して、
%
たゞ% TODO 原本の「二の字点、揺すり点」に濁点のグリフが見つからないので「ゞ」
\ruby{自己}{お|の}が
\ruby{身}{み}にのみ
\ruby{比}{ひき}
\ruby{較}{くら}ぶれば、
%
\ruby{然}{ま}る
\ruby[<j||]{心}{こ〻ろ}% 原本通り「〻(二の字点、揺すり点)」
\ruby[||j>]{{\換字{遣}}}{づかひ}
をするも
\ruby{無理}{む|り}
ならねど、
%
\ruby{御佛}{み|ほとけ}の
\ruby[<j||]{廣}{くわう}
\ruby{大}{だい}なる
\ruby{御誓願}{おん|ちか|ひ}を
こそ
\ruby{頼}{たの}み
\ruby{奉}{たてまつ}りつれ、
%
\ruby{御鬮}{み|くじ}といふ
\ruby{事}{こと}は
\ruby{御}{おん}
\ruby[||j>]{經}{きやう}にも
\ruby{見}{み}えす、
%
\ruby{賣僧}{まい|す}の
\ruby{仕出}{し|だ}したる
なるべき
\ruby{春}{はる}の
\ruby{{\換字{遊}}戱}{あそ|び}の
\ruby{寶引}{ほう|びき}といふにも
\ruby{似}{に}たる
\ruby{埒}{らち}
\ruby{無}{な}く
\ruby{據}{よりどころ}
\ruby{無}{な}き
\ruby{御籤}{み|くじ}の
\ruby{{\換字{文}}}{ぶん}
なんどに、
%
\ruby{我}{われ}
いかで
\ruby{心}{こ〻ろ}を% 原本通り「〻(二の字点、揺すり点)」
\ruby{動}{うご}かされんや。
%
それとも
\ruby{知}{し}らずして
\ruby{性質}{ひ|と}の
\ruby{好}{よ}き
\ruby{老人}{らう|じん}の、
%
\ruby{心}{こ〻ろ}を% 原本通り「〻(二の字点、揺すり点)」
\ruby{{\換字{遣}}}{つか}ふ
\ruby{笑止}{せう|し}さ、
%
と
\ruby{水野}{みづ|の}は
\ruby{却}{かへ}つて
\ruby{老人}{らう|じん}を
\ruby{憐}{あはれ}み、
%
わざと
\ruby[<g>]{懷中}{くわいちう}の% 「懷中(くわいちう)」「ゆ」無し
\ruby{御籤}{み|くじ}を
\ruby{其}{その}
\ruby{儘}{ま〻}にして% 原本通り「〻(二の字点、揺すり点)」
\ruby{讀}{よ}まず。
%
\ruby{共}{とも}に
\ruby{石路}{せき|ろ}の
\ruby{長々}{なが|〳〵}しきを
\ruby{下向}{げ|かう}しけるが、
%
\ruby{老人}{らう|じん}は
\ruby{懷中}{ふと|ころ}より
\ruby{折本}{をり|ほん}に
なりたる
\ruby{普門品}{ふ|もん|ぼん}の
\ruby{小}{ちひさ}きを
\ruby{取}{と}り
\ruby{出}{いだ}して、

\原本頁{127-2}%
『
だいなしに
なつて
\ruby{居}{を}りまする
\ruby{物}{もの}を、
%
\ruby{呈}{あ}げると
\ruby{申}{まを}しては
\ruby{失禮}{しつ|れい}ですけれど、
%
まあ
\ruby{如是}{か|う}いふ
\ruby{物}{もの}の
\ruby{事}{こと}ですから
\ruby{御免下}{ご|めん|くだ}さい。
%
これを
\ruby{貴君}{あな|た}に
\ruby{差上}{さし|あ}げますから、
%
\ruby{何樣}{ど|う}か
\ruby{御取}{お|と}りなすつて
\ruby{下}{くだ}さいまし。
%
\ruby{私}{わたくし}は
もう
\ruby{無書}{そ|ら}で
\ruby{記}{おぼ}{\換字{𛀁}}ましたから、% 送り仮名は原本通り「𛀁」
%
\ruby{此書}{こ|れ}は
\ruby{用}{よう}が
\ruby{明}{あ}いたので
ございますが、
%
\ruby{何樣}{ど|う}か
\ruby{貴君}{あな|た}も
\ruby{御拜}{お|が}みなさる
たびに、
%
これを
\ruby{御覧}{ご|らん}に
なりながら
\ruby{御經}{お|きやう}を
\ruby{御}{お}あげなすつて
\ruby{下}{くだ}されば、
%
\ruby{私}{わたくし}は
\ruby{大變}{たい|へん}に
\ruby{嬉}{うれ}しいと
\ruby{思}{おも}ふので
ございます。
%
それに
\ruby{此}{こ}の
\ruby{末}{すゑ}の
\ruby{方}{はう}に
\ruby{私}{わたくし}の
\ruby{名}{な}
\ruby{住{\換字{所}}}{とこ|ろ}が
\ruby{小}{ちひ}さく
\ruby{書}{か}いて
ございますから、
%
\ruby{何}{なん}ぞの
\ruby{御序}{お|つひで}でも
\ruby{御有}{お|あ}りでしたら
\ruby{御立寄}{お|たち|よ}り
\ruby{下}{くだ}さいまし、
%
いろ〳〵
\ruby{御利生}{ご|り|しやう}の
\ruby{御話}{お|はなし}や
なんぞを
\ruby{致}{いた}しましやうから。
%
\原本頁{127-11}\改行%
では
また
\ruby[<g||]{明日}{みやうにち}
\ruby{御目}{お|め}に
か〻りましやう。% 原本通り「〻(二の字点、揺すり点)」
%
どうか
\ruby{撓}{たゆ}まずに
\ruby{御信心}{ご|しん|〴〵}なすつて!。
』

\原本頁{128-2}%
と
\ruby{云}{い}ひたき
\ruby{事}{こと}のみを
\ruby{云}{い}ひて
\ruby{{\換字{終}}}{つひ}に
\ruby{別}{わか}れたり。

\原本頁{128-3}%
\ruby{册子}{ほ|ん}は
\ruby{言}{ことば}を
\ruby{費}{つひや}して
\ruby{辭}{いな}むべき
ほどの
ものにも
あらず、
%
\ruby{特}{こと}に
\ruby{快}{こ〻ろよ}く% 原本通り「〻(二の字点、揺すり点)」
\ruby{受}{う}け
\ruby{納}{をさ}めて
\ruby{芳志}{こ〻ろ|ざし}を% 原本通り「〻(二の字点、揺すり点)」
\ruby{無}{む}にせざらんは、
%
\ruby{差}{さ}し
\ruby{當}{あた}つての
\ruby{{\換字{道}}}{みち}なるべしと、
%
\原本頁{128-5}\改行%
\ruby{水野}{みづ|の}は
\ruby{老人}{らう|じん}に
\ruby{厚意}{かう|い}を
\ruby{謝}{しや}して、
%
\ruby{袖}{そで}を
\ruby{{\換字{分}}}{わか}つて
\ruby{東方}{ひが|し}へ
\ruby{去}{さ}りつ、
%
\ruby{先}{ま}づ
\ruby{普門品}{ふ|もん|ぼん}を
\ruby{懷中}{ふと|ころ}に
\ruby{入}{い}る〻に、% 原本通り「〻(二の字点、揺すり点)」
%
\ruby{卷}{ま}きたる
\ruby{彼}{か}の
\ruby{御籤}{み|くじ}の
かさ〳〵と
\ruby{手}{て}に
\ruby{觸}{ふ}れたれば、
%
\ruby{引{\換字{交}}}{ひき|ちが}へて
\ruby{取}{と}り
\ruby{出}{いだ}して
\ruby{其}{その}
\ruby{{\換字{文}}}{ぶん}を
\ruby{讀}{よ}むに、

\原本頁{128-8}%
\hspace*{1zw}
% 返り点参照情報
% https://www.asahi-net.or.jp/~ax2s-kmtn/ref/unicode/u3190.html
\begin{tblr}{colspec={Q[c] | Q[l,t] Q[l,b]}, stretch=0.5}
  \SetCell[r=4]{c,1em}{第七番凶}&
  \kundoku{登}{ふねにの}{}{㆑}% 「㆑(u3191)レ点」「レ(u30ec)カタカナ」
  \kundoku{舟}{ぼりて }{}{}
  \kundoku{待}{びんぷう}{}{㆓}% 「㆓(u3293)」
  \kundoku{便}{をまてば}{}{}
  \kundoku{風}{   }{}{㆒}。% 「㆒(u3192)」
  & \scriptsize{\noindent
    舟にのりて行かんとす\newline
    ればおひてが無い
  }\\
  %%%
  &
  \kundoku{月}{げつ し}{}{}
  \kundoku{色}{よく く}{}{}
  \kundoku{暗}{らくして}{}{}
  \kundoku{朦}{もう  }{}{}
  \kundoku{朧}{ろう }{}{}。
  & \scriptsize{\noindent
    見れば空もわるくして\\
    月もくらきぞ
  }\\
  %%%
  &
  \kundoku{欲}{かうりん}{}{㆘}% 「㆓(u3298)」
  \kundoku{輾}{をきしら}{}{㆓}% 「㆓(u3293)」
  \kundoku{香}{してさら}{}{}
  \kundoku{輪}{んとほつ}{}{㆒}% 「㆒(u3192)」
  \kundoku{去}{すれば}{}{㆖}。% 「㆖(u3196)」
  & \scriptsize{\noindent
    車にのりておもふとこ\\
    ろへゆかんとすれば
  }\\
  &
  \kundoku{高}{かう  }{}{}
  \kundoku{山}{ざん  }{}{}
  \kundoku{千}{せん  }{}{}
  \kundoku{萬}{ばん  }{}{}
  \kundoku{里}{りなり}{}{}。

  & \scriptsize{\noindent
    つゞける山〻恐ろしく\\% 山〻 vs 山々% 原本通り「〻(二の字点、揺すり点)」% TODO 原本の「二の字点、揺すり点」に濁点のグリフが見つからないので「ゞ」
    高くしてそれも叶はぬ
  }
\end{tblr}
\hspace*{1zw}

\原本頁{129-1}
とありて、
%
ひし〳〵と
\ruby{我}{わ}が
\ruby{身}{み}の
\ruby{上}{うへ}に
\ruby{巧}{よ}く
\ruby{中}{あた}りたり。

\原本頁{129-2}%
もとより
\ruby{取}{と}るに
\ruby{足}{た}らぬ
こと〻は% 原本通り「〻(二の字点、揺すり点)」
\ruby{思}{おも}ひ
ながらも、
%
\ruby{不思議}{ふ|し|ぎ}に
\ruby{中}{あた}れる
\ruby{此}{こ}の
\ruby{{\換字{文}}}{ぶん}の
\ruby{流石}{さす|が}に
\ruby{胸}{むね}に
\ruby{徹}{こた}へて
\ruby{心}{こ〻ろ}% 原本通り「〻(二の字点、揺すり点)」
さびしく、
%
じつと
\ruby{眼}{め}を
\ruby{{\換字{留}}}{と}めて
\ruby{見}{み}れば、
%
\ruby{末}{すゑ}の
\ruby{方}{かた}に
\ruby[<j||]{女}{をんな}
\ruby{{\換字{文}}字}{も|じ}にて
\ruby{細}{こまか}に
\ruby{注}{ちう}し
\ruby{記}{しる}せる
\ruby{其}{その}
\ruby{最先}{まつ|さき}に、

\原本頁{129-5}%
\ruby[||g>]{病事}{やまひごと}は
\ruby{十}{じう}に
\ruby{六七}{ろく|しち}
\ruby{本}{ほん}
\ruby{復}{ぷく}
\ruby{無}{な}し、
%
\ruby{長}{なが}びき
たらば
\ruby{後}{のち}は
\ruby{息災}{そく|さい}になる
\ruby{事}{こと}も
あるべし、
%
よく
\ruby{信力}{しん|りき}をもて
\ruby{佛神}{ぶつ|しん}を
\ruby{頼}{たの}みて
\ruby{吉}{よし}、

\原本頁{129-7}%
と
ありたるは、
%
いよ〳〵
\ruby{何}{なに}となく
\ruby{不快}{ふ|くわい}を
\ruby{感}{かん}じて、
%
\ruby{腹}{はら}の
\ruby{底}{そこ}より
\ruby{{\換字{寒}}}{さむさ}の
\ruby{上}{のぼ}り
\ruby{來}{きた}る
やうに
おぼえたり。

\原本頁{129-9}%
\ruby{何}{なに}とか
\ruby{思}{おも}ひけん
\ruby{水野}{みづ|の}は
\ruby{引{\換字{返}}}{ひつ|かへ}して、
%
\ruby{復}{また}
\ruby[g]{相良}{さがら}を
\ruby{訪}{と}ひぬ。
%
\ruby{待}{ま}つ
\ruby{事}{こと}
\ruby{一時}{いち|じ}
\ruby{餘}{あま}りにして
\ruby{{\換字{終}}}{つひ}に
\ruby[g]{相良}{さがら}に
\ruby{親}{したし}く
\ruby{會}{あ}ひ
\ruby{得}{{\換字{𛀁}}}て、
%
\ruby{必}{かなら}ず
\ruby{見舞}{み|ま}はんとの
\ruby{辭}{ことば}を
\ruby{得}{{\換字{𛀁}}}て
\ruby{歸}{かへ}りしが、
%
\ruby{幸}{さいはひ}にして
\ruby{今日}{け|ふ}は
\ruby{休校}{やす|み}の
\ruby{日}{ひ}なれば
こそ
\ruby{宜}{よ}けれ、
%
\ruby{吾妻橋}{あ|づま|ばし}に
か〻れる% 原本通り「〻(二の字点、揺すり点)」
\ruby{時}{とき}は
\ruby{既}{すで}に
\ruby{九時}{く|じ}に
\ruby{{\換字{近}}}{ちか}からん
としたり。
