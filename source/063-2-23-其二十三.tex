\Entry{其二十三}

% メモ 校正終了 2024-04-23 2024-06-01 2024-07-02
\原本頁{124-5}%
よしや
\ruby[g]{大吉}{だいきち}
ならぬ
までも
せめては
\ruby{凶}{きよう}
ならぬ
\ruby[g]{御籤}{み くじ}を
\ruby{得}{{\換字{𛀁}}}て、
%
\ruby{憂}{うれひ}に
\ruby{沈}{しづ}み
\ruby[<j>]{悲}{かなしみ}に
\ruby{陷}{おちゐ}れる
\ruby{氣}{き}を
\ruby[g]{引立}{ひきた }て、
%
\ruby[g]{信心}{しん〴〵}の
\ruby{勇}{いさみ}を
\ruby{附}{つ}けて
\ruby{吳}{く}れんと
\ruby{爲}{し}たるらしき
\ruby[g]{親切}{しんせつ}の
\ruby[g]{老人}{らうじん}が、
%
\ruby{思}{おも}ふこと
\ruby{{\換字{違}}}{たが}ひて
\ruby{甚}{いた}く
\ruby{望}{のぞみ}を
\ruby{失}{うしな}へるは、
%
\ruby{忽}{たちま}ち
\ruby{先}{ま}づ
\ruby{其}{そ}の
\ruby{色}{いろ}に
\ruby{現}{あらは}れて、
%
\ruby{僧}{そう}より
\ruby[g]{受取}{うけと }りし
\ruby[g]{御籤}{み くじ}をば、
%
\ruby[||j>]{力}{ちから}
\ruby[||j>]{無}{ な}げに
\ruby{輪}{わ}に
\ruby{卷}{ま}き
ながら、
%
\ruby{鈍}{にぶ}る〳〵
\ruby[g]{此方}{こなた }へ% ルビ調整(原本通り)
\ruby{步}{あゆ}み
\ruby{來}{きた}れるに、
%
\ruby[g]{水野}{みづの }は
\ruby{見}{み}ずして
\ruby{既}{すで}に
\ruby{其}{そ}の
\ruby{{\換字{文}}}{ぶん}の
\ruby{凶}{きよう}なるを
\ruby{知}{し}れり。

\原本頁{125-1}%
\ruby{第何十何番}{だい|なん|じふ|なん|ばん}
\ruby[g]{大吉}{だいきち}
といふならば、
%
\ruby[g]{如何}{い か }
ばかりか
\ruby{悅}{よろこ}び
\ruby{勇}{いさ}んで
\ruby{示}{しめ}すべきを、
%
\ruby[g]{老人}{らうじん}は
\ruby{卷}{ま}きたる
まゝ% 踊り字調整「〻(二の字点、揺すり点)に見えるが(ゝ)」
\ruby[g]{御籤}{み くじ}を
\ruby[g]{水野}{みづの }の
\ruby[g]{懷中}{ふところ}に
\ruby{輕}{かる}く
\ruby[g]{押入}{おしい }れて、

\原本頁{125-3}%
『
\ruby[g]{何樣}{ど う }か
\ruby{吉}{よし}
\ruby{凶}{あし}に
かゝはらず% 踊り字調整「〻(二の字点、揺すり点)に見えるが(ゝ)」
\ruby{御信心}{ご|しん|〴〵}なさい。
%
\ruby[g]{大吉}{だいきち}でも
\ruby{驕}{おご}れば
\ruby{凶}{きよう}に
\ruby{反}{かへ}ります、
%
たとへ
\ruby{凶}{きよう}でも
\ruby{御信心}{ご|しん|〴〵}を
\ruby{{\換字{強}}}{つよ}く
なすつて、
%
それから
また
\ruby{改}{あらた}めて
\ruby[g]{御籤}{おみくじ}を
\ruby[g]{御戴}{おいたゞ}き% 踊り字調整「〻(二の字点、揺すり点)に濁点に見えるが(ゞ)」
なすつて
ごらんなさい、
%
\ruby{吉}{きち}に
なります
ことも
ございます
ものです。
%
\ruby{吉}{よい}につけ
\ruby{凶}{わるい}につけ
\ruby{御信心}{ご|しん|〴〵}が
\ruby[g]{大切}{たいせつ}です。
%
\ruby{决}{けつ}して
\ruby{信}{しん}を
\ruby[g]{御冷}{お さま}し
なすつては
いけません。
%
さて
そろ〳〵
もう
\ruby[g]{下向}{げ かう}
いたしましやう。
』

\原本頁{125-9}%
と、
%
\ruby{云}{い}ひ
\ruby{{\換字{終}}}{をは}つて
\ruby[g]{本{\換字{尊}}}{ほんぞん}を
また
\ruby[g]{一拜}{いつぱい}して、
%
おのれ
\ruby{先}{ま}づ
\ruby[g]{御堂}{み だう}を
\ruby{去}{さ}らん
としたり。

\原本頁{125-11}%
\ruby[g]{老人}{らうじん}が
\ruby[g]{樣子}{やうす }の
\ruby{急}{きふ}に
そはつけるは、
%
\ruby{何}{なん}の
\ruby{意}{こゝろ}も% 踊り字調整「〻(二の字点、揺すり点)に見えるが(ゝ)」
\ruby{無}{な}かりし
\ruby{我}{われ}に
\ruby[g]{智慧}{ち ゑ }を
つけて
\ruby[g]{御籤}{み くじ}を
\ruby{取}{と}らせたるに、
%
その
\ruby[g]{御籤}{み くじ}の
ことのほか
\ruby{凶}{あし}かりしかば、
%
\ruby{却}{かへ}つて
\ruby{其}{そ}のために
\ruby{憂}{うれひ}を
\ruby{增}{ま}し
%
\ruby[<j>]{悲}{かなしみ}を
\ruby{添}{そ}ふる
こともやと、
%
\ruby{氣}{き}の
\ruby{毒}{どく}さに
\ruby{堪}{た}へ
かねて
\ruby{傍}{かたへ}に
\ruby{居}{ゐ}づらく、
%
\ruby{狭}{せま}くして
\ruby[||j>]{正}{しやう}
\ruby[||j>]{直}{ ぢき}なる
% \ruby{正直}{しやう|ぢき}なる
\ruby{心}{こゝろ}の% 踊り字調整「〻(二の字点、揺すり点)に見えるが(ゝ)」
\ruby{憐}{あは}れにも
\原本頁{126-4}\改行%
\ruby[g]{沈着}{おちつ }き
かぬるが
\ruby{爲}{ため}
なるべし。
%
\ruby[g]{{\換字{平}}生}{ひ ごろ}の% ルビ調整(原本通り)
\ruby{我}{われ}を
\ruby{知}{し}らず
して、
%
たゞ% 踊り字調整「〻(二の字点、揺すり点)に濁点に見えるが(ゞ)」
\ruby[g]{自己}{お の }が
\ruby{身}{み}にのみ
\ruby{比}{ひき}
\ruby{較}{くら}ぶれば、
%
\ruby{然}{ま}る
\footnote{「然る」は「さる」が順当だと思うが、「まるごと」の意味もあり
  原文通り「まる」とする(国会図書館 コマ番号68/160 p-126 l-05)}%
\ruby[<j||]{心}{こゝろ}% 踊り字調整「〻(二の字点、揺すり点)に見えるが(ゝ)」
\ruby[||j>]{{\換字{遣}}}{づかひ}
をするも
\ruby[g]{無理}{む り }
ならねど、
%
\ruby[g]{御佛}{みほとけ}の
\ruby[||j>]{廣}{くわう}
\ruby[||j>]{大}{ だい}なる
\ruby{御誓願}{おん|ちか|ひ}を
こそ
\ruby{頼}{たの}み
\ruby[<j>]{奉}{たてまつ}りつれ、
%
\ruby[g]{御鬮}{み くじ}といふ
\ruby{事}{こと}は
\ruby{御}{おん}
\ruby[||j>]{經}{きやう}にも
\ruby{見}{み}えす
\footnote{『詞「みる(見)」の未然形に尊敬の助動詞「す」の付いたものか』
  「 ごらんになる」と解釈し「見えず」ではなく「みえす」と原本通り
  (国会図書館 コマ番号68 / 160 p-126 l-07)}%
、
%
\ruby[g]{賣僧}{まいす }の
\ruby[g]{仕出}{し だ }したる
なるべき
\ruby{春}{はる}の
\ruby[g]{{\換字{遊}}戱}{あそび }の
\ruby[g]{寶引}{ほうびき}といふにも
\ruby{似}{に}たる
\ruby{埒}{らち}
\ruby{無}{な}く
\ruby[|->]{據}{よりどころ}
\ruby[||j>]{無}{ な}き
% \ruby{據無}{よりどころ|な}き
\ruby[g]{御籤}{み くじ}の
\ruby{{\換字{文}}}{ぶん}
なんどに、
%
\ruby{我}{われ}
いかで
\ruby{心}{こゝろ}を% 踊り字調整「〻(二の字点、揺すり点)に見えるが(ゝ)」
\ruby{動}{うご}かされんや。
%
それとも
\ruby{知}{し}らずして
\ruby[g]{性質}{ひ と }の
\ruby{好}{よ}き
\ruby[g]{老人}{らうじん}の、
%
\ruby{心}{こゝろ}を% 踊り字調整「〻(二の字点、揺すり点)に見えるが(ゝ)」
\ruby{{\換字{遣}}}{つか}ふ
\ruby[g]{笑止}{せうし }さ、
%
と
\ruby[g]{水野}{みづの }は
\ruby{却}{かへ}つて
\ruby[g]{老人}{らうじん}を
\ruby{憐}{あはれ}み、
%
わざと
\ruby[||j>]{懷}{くわい}
\ruby[||j>]{中}{ ちう}の% 「懷中(くわいちう)」「ゆ」無し
% \ruby{懷中}{くわい|ちう}の
\ruby[g]{御籤}{み くじ}を
\ruby{其}{その}
\ruby{儘}{まゝ}にして% 踊り字調整「〻(二の字点、揺すり点)に見えるが(ゝ)」
\ruby{讀}{よ}まず。
%
\ruby{共}{とも}に
\ruby[g]{石路}{せきろ }の
\ruby[g]{長々}{なが〳〵}しきを
\ruby[g]{下向}{げ かう}しけるが、
%
\ruby[g]{老人}{らうじん}は
\ruby[g]{懷中}{ふところ}より
\ruby[g]{折本}{をりほん}に
なりたる
\ruby{普門品}{ふ|もん|ぼん}の
\ruby{小}{ちひさ}きを
\ruby{取}{と}り
\ruby{出}{いだ}して、

\原本頁{127-2}%
『
だいなしに
なつて
\ruby{居}{を}りまする
\ruby{物}{もの}を、
%
\ruby{呈}{あ}げると
\ruby{申}{まを}しては
\ruby[g]{失禮}{しつれい}ですけれど、
%
まあ
\ruby[g]{如是}{か う }いふ
\ruby{物}{もの}の
\ruby{事}{こと}ですから
\ruby{御免下}{ご|めん|くだ}さい。
%
これを
\ruby[g]{貴君}{あなた }に
\ruby[g]{差上}{さしあ }げますから、
%
\ruby[g]{何樣}{ど う }か
\ruby[g]{御取}{お と }りなすつて
\ruby{下}{くだ}さいまし。
%
\ruby[<j>]{私}{わたくし}は
もう
\ruby[g]{無書}{そ ら }で
\ruby{記}{おぼ}{\換字{𛀁}}ましたから、% 送り仮名は原本通り「𛀁」
%
\ruby[g]{此書}{こ れ }は
\ruby{用}{よう}が
\ruby{明}{あ}いたので
ございますが、
%
\ruby[g]{何樣}{ど う }か
\ruby[g]{貴君}{あなた }も
\ruby[g]{御拜}{お をが}みなさる
たびに、
%
これを
\ruby[g]{御覧}{ご らん}に
なりながら
\ruby[g]{御經}{おきやう}を
\ruby{御}{お}あげなすつて
\ruby{下}{くだ}されば、
%
\ruby[<j>]{私}{わたくし}は
\ruby[g]{大變}{たいへん}に
\ruby{嬉}{うれ}しいと
\ruby{思}{おも}ふので
ございます。
%
それに
\ruby{此}{こ}の
\ruby{末}{すゑ}の
\ruby{方}{はう}に
\ruby[<j>]{私}{わたくし}の
\ruby{名}{な}
\ruby[g]{住{\換字{所}}}{ところ }が
\ruby{小}{ちひ}さく
\ruby{書}{か}いて
ございますから、
%
\ruby{何}{なん}ぞの
\ruby[g]{御序}{おつひで}でも
\ruby[g]{御有}{お あ }りでしたら
\ruby{御立寄}{お|たち|よ}り
\ruby{下}{くだ}さいまし、
%
いろ〳〵
\ruby{御}{ご}
\ruby[g]{利生}{りしやう}の
\ruby[g]{御話}{おはなし}や
なんぞを
\ruby{致}{いた}しましやうから。
%
\原本頁{127-11}\改行%
では
また
\ruby[||j>]{明}{みやう}
\ruby[||j>]{日}{ にち}
% \ruby{明日}{みやう|にち}
\ruby[g]{御目}{お め }に
かゝりましやう。% 踊り字調整「〻(二の字点、揺すり点)に見えるが(ゝ)」
%
どうか
\ruby{撓}{たゆ}まずに
\ruby{御信心}{ご|しん|〴〵}なすつて!。
』

\原本頁{128-2}%
と
\ruby{云}{い}ひたき
\ruby{事}{こと}のみを
\ruby{云}{い}ひて
\ruby{{\換字{終}}}{つひ}に
\ruby{別}{わか}れたり。

\原本頁{128-3}%
\ruby[g]{册子}{ほ ん }は
\ruby{言}{ことば}を
\ruby{費}{つひや}して
\ruby{辭}{いな}むべき
ほどの
ものにも
あらず、
%
\ruby{特}{こと}に
\ruby[<j>]{快}{こゝろよ}く% 踊り字調整「〻(二の字点、揺すり点)に見えるが(ゝ)」% 行末行頭の境界付近なので特例処置を施す
\ruby{受}{う}け
\ruby{納}{をさ}めて
\ruby[||j>]{芳}{こゝろ}% 踊り字調整「〻(二の字点、揺すり点)に見えるが(ゝ)」
\ruby[||j>]{志}{ ざし}を
\ruby{無}{む}にせざらんは、
%
\ruby{差}{さ}し
\ruby{當}{あた}つての
\ruby{{\換字{道}}}{みち}なるべしと、
%
\原本頁{128-5}\改行%
\ruby[g]{水野}{みづの }は
\ruby[g]{老人}{らうじん}に
\ruby[g]{厚意}{かうい }を
\ruby{謝}{しや}して、
%
\ruby{袖}{そで}を
\ruby{{\換字{分}}}{わか}つて
\ruby[g]{東方}{ひがし }へ
\ruby{去}{さ}りつ、
%
\ruby{先}{ま}づ
\ruby{普門品}{ふ|もん|ぼん}を
\ruby[g]{懷中}{ふところ}に
\ruby{入}{い}るゝに、% 踊り字調整「〻(二の字点、揺すり点)に見えるが(ゝ)」
%
\ruby{卷}{ま}きたる
\ruby{彼}{か}の
\ruby[g]{御籤}{み くじ}の
かさ〳〵と
\ruby{手}{て}に
\ruby{觸}{ふ}れたれば、
%
\ruby[g]{引{\換字{交}}}{ひきちが}へて
\ruby{取}{と}り
\ruby{出}{いだ}して
\ruby{其}{その}
\ruby{{\換字{文}}}{ぶん}を
\ruby{讀}{よ}むに、

\原本頁{128-8}%
\hspace*{1zw}
% 返り点参照情報
% https://www.asahi-net.or.jp/~ax2s-kmtn/ref/unicode/u3190.html
\begin{tblr}{
    colspec  = {Q[c] Q[l,t] Q[l,b]} ,
    stretch  = 0.5                  ,
    vline{2} = {2-Y}{.5pt}          ,
  }
  \SetCell[r=4]{c,1em}{第七番凶}&
  \kundoku{登}{ふねに }{}{㆑}% 「㆑(u3191)レ点」「レ(u30ec)カタカナ」
  \kundoku{舟}{のぼりて}{}{}
  \kundoku{待}{びん  }{}{㆓}% 「㆓(u3293)」
  \kundoku{便}{ぷうを }{}{}
  \kundoku{風}{まてば}{}{㆒}。% 「㆒(u3192)」
  & \scriptsize{\noindent
    舟にのりて行かんとす\newline
    ればおひてが無い
  }\\
  %%%
  &
  \kundoku{月}{げつ  }{}{}
  \kundoku{色}{しよくく}{}{}
  \kundoku{暗}{らくして}{}{}
  \kundoku{朦}{もう  }{}{}
  \kundoku{朧}{ろう }{}{}。
  & \scriptsize{\noindent
    見れば空もわるくして\\
    月もくらきぞ
  }\\
  %%%
  &
  \kundoku{欲}{かうりん}{}{㆘}% 「㆓(u3298)」
  \kundoku{輾}{をきしら}{}{㆓}% 「㆓(u3293)」
  \kundoku{香}{してさら}{}{}
  \kundoku{輪}{んとほつ}{}{㆒}% 「㆒(u3192)」
  \kundoku{去}{すれば}{}{㆖}。% 「㆖(u3196)」
  & \scriptsize{\noindent
    車にのりておもふとこ\\
    ろへゆかんとすれば
  }\\
  &
  \kundoku{高}{かう  }{}{}
  \kundoku{山}{ざん  }{}{}
  \kundoku{千}{せん  }{}{}
  \kundoku{萬}{ばん  }{}{}
  \kundoku{里}{り  }{}{}。

  & \scriptsize{\noindent
    つゞける山々恐ろしく\\% 踊り字調整「〻(二の字点、揺すり点)に濁点に見えるが(ゞ)」
    高くしてそれも叶はぬ
  }
\end{tblr}
\hspace*{1zw}

\原本頁{129-1}
とありて、
%
ひし〳〵と
\ruby{我}{わ}が
\ruby{身}{み}の
\ruby{上}{うへ}に
\ruby{巧}{よ}く
\ruby{中}{あた}りたり。

\原本頁{129-2}%
もとより
\ruby{取}{と}るに
\ruby{足}{た}らぬ
ことゝは% 踊り字調整「〻(二の字点、揺すり点)に見えるが(ゝ)」
\ruby{思}{おも}ひ
ながらも、
%
\ruby{不思議}{ふ|し|ぎ}に
\ruby{中}{あた}れる
\ruby{此}{こ}の
\ruby{{\換字{文}}}{ぶん}の
\ruby[g]{流石}{さすが }に
\ruby{胸}{むね}に
\ruby{徹}{こた}へて
\ruby{心}{こゝわ}% 踊り字調整「〻(二の字点、揺すり点)に見えるが(ゝ)」% ルビ調整((こころ)??)
さびしく、
%
じつと
\ruby{眼}{め}を
\ruby{{\換字{留}}}{と}めて
\ruby{見}{み}れば、
%
\ruby{末}{すゑ}の
\ruby{方}{かた}に
\ruby[||j>]{女}{をんな}
\ruby[||j>]{{\換字{文}}字}{ も|じ}にて
\ruby{細}{こまか}に
\ruby{注}{ちう}し
\ruby{記}{しる}せる
\ruby{其}{その}
\ruby[g]{最先}{まつさき}に、

\原本頁{129-5}%
\ruby[||j>]{病}{やまひ}
\ruby[||j>]{事}{ ごと}は
% \ruby{病事}{やまひ|ごと}は
\ruby{十}{じふ}に
\ruby[g]{六七}{ろくしち}% 原本には漢数字「七」のルビ無し
\ruby{本}{ほん}
\ruby{復}{ぷく}
\ruby{無}{な}し、
%
\ruby{長}{なが}びき
たらば
\ruby{後}{のち}は
\ruby[g]{息災}{そくさい}になる
\ruby{事}{こと}も
あるべし、
%
よく
\ruby[g]{信力}{しんりき}をもて
\ruby[g]{佛神}{ぶつしん}を
\ruby{頼}{たの}みて
\ruby{吉}{よし}、

\原本頁{129-7}%
と
ありたるに、
%
いよ〳〵
\ruby{何}{なに}となく
\ruby[g]{不快}{ふくわい}を
\ruby{{\換字{感}}}{かん}じて、
%
\ruby{腹}{はら}の
\ruby{底}{そこ}より
\ruby[<j||]{{\換字{寒}}}{さむさ}の
\ruby{上}{のぼ}り
\ruby{來}{きた}る
やうに
おぼえたり。

\原本頁{129-9}%
\ruby{何}{なに}とか
\ruby{思}{おも}ひけん
\ruby[g]{水野}{みづの }は
\ruby[g]{引{\換字{返}}}{ひつかへ}して、
%
\ruby{復}{また}
\ruby[g]{相良}{さがら }を
\ruby{訪}{と}ひぬ。
%
\ruby{待}{ま}つ
\ruby{事}{こと}
\ruby[g]{一時}{いちじ }
\ruby{餘}{あま}りにして
\ruby{{\換字{終}}}{つひ}に
\ruby[g]{相良}{さがら }に
\ruby{親}{したし}く
\ruby{會}{あ}ひ
\ruby{得}{{\換字{𛀁}}}て、
%
\ruby{必}{かなら}ず
\ruby[g]{見舞}{み ま }はんとの
\ruby{辭}{ことば}を
\ruby{得}{{\換字{𛀁}}}て
\ruby{歸}{かへ}りしが、
%
\ruby[<j>]{幸}{さいはひ}にして
\ruby[g]{今日}{け ふ }は
\ruby[g]{休校}{やすみ }の
\ruby{日}{ひ}なれば
こそ
\ruby{宜}{よ}けれ、
%
\ruby{吾妻橋}{あづ|ま|ばし}に% ルビ調整(原本通り)
かゝれる% 踊り字調整「〻(二の字点、揺すり点)に見えるが(ゝ)」
\ruby{時}{とき}は
\ruby{既}{すで}に
\ruby[g]{九時}{く じ }に
\ruby{{\換字{近}}}{ちか}からん
としたり。
