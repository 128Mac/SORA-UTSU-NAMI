\Entry{其二十四}

\ruby{雷神門}{かみ|なり|もん}はいつもながら
\ruby{人}{ひと}のぞよつきて
\ruby{目}{め}まぐるしき
\ruby{地}{ところ}なり。
わけて
\ruby{今日}{け|ふ}は
\ruby{日曜}{にち|やう}の
\ruby{事}{こと}とて、
\ruby{掻頭}{かん|ざし}に
\ruby{花}{はな}を
\ruby{{\換字{飾}}}{かざ}らする
\ruby{九歳十歳}{こゝ|のつ|と|を}の
\ruby{女}{をんな}の
\ruby{兒}{こ}、
\ruby{金{\換字{文}}字}{きん|も|じ}かゞやく
\ruby{天鵞絨帽子}{び|ろう|ど|ばう|し}かぶらせたる
\ruby{洋服}{やう|ふく}
\ruby{姿}{すがた}
\ruby{可憐}{か|はゆ}らしき
\ruby{六歳七歳}{むつ|ゝ|なゝ|つ}の
\ruby{男}{をとこ}の
\ruby{兒}{こ}など
\ruby{引{\換字{連}}}{ひき|つ}れて、
\ruby{世}{よ}を
\ruby{樂}{たの}しげに
\ruby{仲見世}{なか|み|せ}に
\ruby{入}{い}る
\ruby{御母樣}{お|つか|さん}もあれば、
\ruby{農家}{ひやく|しやう}には
\ruby{{\換字{違}}無}{ちが|ひな}き
\ruby{乾疥面}{はた|け|がほ}に、
\ruby{白粉}{おし|ろい}の
\ruby{不均}{む|ら}の
\ruby{奇異}{ふし|ぎ}にをかしき、
\ruby{猫}{ねこ}が
\ruby{化}{ば}けたやうな
\ruby{小娘}{こ|むすめ}
\ruby{{\換字{連}}}{れん}の、
\ruby{何憂事}{なに|うき|こと}も
\ruby{知}{し}らで
\ruby{觀音樣}{くわん|のん|さま}に
\ruby{參}{まゐ}るあり。
\ruby{妾}{われ}も
\ruby{人生}{ひと|のよ}の
\ruby{春}{はる}に
\ruby{{\換字{遊}}}{あそ}べる
\ruby{蝶々髷}{てふ|〳〵|まげ}の、まだ
\ruby{何事}{なに|ごと}も
\ruby{知}{し}らざりし
\ruby{頃}{ころ}は、たゞあどけ
\ruby{無}{な}う
\ruby{面白}{おも|しろ}う
\ruby{此地}{こ|ゝ}を
\ruby{極樂}{ごく|らく}のやうに
\ruby{思}{おも}ひし
\ruby{時}{とき}もありしと、
\ruby{遙}{はるか}に
\ruby{山門}{さん|もん}を
\ruby{望}{のぞ}むにも
\ruby{往時}{むか|し}
\ruby{懷}{なつか}しく、
\ruby{{\換字{通}}}{とほ}りすがりなれど
\ruby{御堂}{み|だう}の
\ruby{方}{かた}を
\ruby{一寸}{ちよ|つと}
\ruby{拜}{をが}みて、そのまゝ
\ruby{東}{ひがし}に
\ruby{切}{き}れて
\ruby{行}{ゆ}けば、

『
\ruby{姐樣}{ね\換字{𛀁}|さん}、
\ruby{如何}{い|かゞ}です、
\ruby{御安}{お|やす}くまゐりましやう。
』

『
\ruby{姐樣}{ね\換字{𛀁}|さん}、
\ruby{如何}{い|かゞ}です
\ruby{御安}{お|やす}く
\ruby{如何}{い|かゞ}です。
』

と
\ruby{車夫}{くる|まや}の
\ruby{聲々}{こゑ|〴〵}かしましく
\ruby{煩}{うる}さし。

\ruby{久}{ひさ}しぶりにて
\ruby{渡}{わた}る
\ruby{吾妻橋}{あ|づま|ばし}より
\ruby{川上}{かは|かみ}の
\ruby{方}{かた}を
\ruby{{\換字{遠}}}{とほ}く
\ruby{見}{み}れば、
\ruby{水}{みづ}は
\ruby{昔見}{むかし|み}たりし
\ruby{如}{ごと}く
\ruby{{\換字{緩}}}{ゆる}く
\ruby{流}{なが}れて、
\ruby{右手}{みぎ|て}に
\ruby{長}{なが}き
\ruby{一帶}{いつ|たい}の
\ruby{堤}{つゝみ}の、
\ruby{其}{そ}の
\ruby{狀}{さま}も
\ruby{更}{さら}に
\ruby{記臆}{おぼ|\換字{𛀁}}に
\ruby{異}{かは}らず、
\ruby{岸}{きし}の
\ruby{櫻}{さくら}の
\ruby{葉}{は}も
\ruby{{\換字{透}}}{す}けるながら、その
\ruby{花}{はな}の
\ruby{眺}{なが}めもおもかげに
\ruby{立}{た}つて、あゝ
\ruby{彼}{あ}の
\ruby{花}{はな}の
\ruby{隧{\換字{道}}}{とん|ねる}のやうであつた
\ruby{中}{なか}を、
\ruby{夜}{よる}の
\ruby{風}{かぜ}の
\ruby{些{\換字{寒}}}{やゝ|さむ}かつた
\ruby{時}{とき}、
\ruby{彼人}{ひ|と}に
\ruby{手}{て}を
\ruby{取}{と}られて
\ruby{人目}{ひと|め}の
\ruby{羞}{はづか}しく、
\ruby{暗}{くら}き
\ruby{方}{かた}に
\ruby{身}{み}を
\ruby{寄}{よ}せて
\ruby{歩}{ある}きし
\ruby{春}{はる}の
\ruby{{\換字{宵}}}{よ}もありしが、
\ruby{思}{おも}へば
\ruby{今}{いま}
\ruby{其事}{そ|れ}の
\ruby{思}{おも}ひ
\ruby{出}{だ}さるゝも
\ruby{甲{\換字{斐}}無}{か|ひ|な}く
\ruby{愚}{おろか}かなりと、しきりに
\ruby{路}{みち}を
\ruby{急}{いそ}ぎて
\ruby{橋}{はし}を
\ruby{渡}{わた}り
\ruby{盡}{つく}し、また
\ruby{煩}{うる}さく
\ruby{車夫}{くる|まや}の
\ruby{勸}{すゝ}むる
\ruby{中}{なか}を
\ruby{停車塲}{てい|しや|じよう}へと
\ruby{向}{むか}ひぬ。

\ruby{乘}{の}れと
\ruby{勸}{すゝ}められて
\ruby{乘}{の}らぬを
\ruby{車夫}{くる|まや}の
\ruby{憎}{にく}がりて、

『
\ruby{姐々}{ね\換字{𛀁}|さん}、
\ruby{滊車}{き|しや}なら
\ruby{{\換字{猶}}}{なほ}の
\ruby{事}{こと}、
\ruby{乘}{の}らないと
\ruby{間}{ま}に
\ruby{合}{あ}はないよ、
\ruby{九時四十五{\換字{分}}}{く|じ|よん|じう|ご|ふん}だからもう
\ruby{發車}{で|る}のだよ。
』

『そんなに
\ruby{急}{いそ}いで
\ruby{歩}{ある}くと
\ruby{女振}{をんな|ぶり}が
\ruby{下}{さが}るぜ。
』

『
\ruby{滊車}{き|しや}までなら
\ruby{直}{ぢき}だから、
\ruby{乗}{の}せてつて
\ruby{上}{あ}げようか、
\ruby{無錢}{た|ゞ}でも
\ruby{關}{かま}わないんだ、ハヽヽ。
』

なんど〻
\ruby{口々}{くち|〴〵}に
\ruby{下賤}{げ|す}のものゝ
\ruby{好}{す}きな
\ruby{事}{こと}をいふに、
\ruby{虛言}{う|そ}とは
\ruby{思}{おも}ひながら、おのづと
\ruby{氣}{き}の
\ruby{急}{せ}きて、
\ruby{疾足}{はや|あし}になり、やがて
\ruby{停車塲}{てい|しや|じよう}に
\ruby{到}{いた}り
\ruby{着}{つ}けば、
\ruby{車夫}{しや|ふ}も
\ruby{出鱈目}{で|たら|め}は
\ruby{云}{い}はざりしと
\ruby{見}{み}え、
\ruby{危}{あやふ}くも
\ruby{乘}{の}り
\ruby{後}{おく}れんとするほどのところなりけり。

\ruby{切符}{きつ|ぷ}を
\ruby{買}{か}ふ
\ruby{間}{ま}も
\ruby{疾}{と}しや
\ruby{遲}{おそ}しや、

『
\ruby{早}{はや}く
\ruby{早}{はや}く、』

と
\ruby{驛夫}{\換字{𛀁}き|ふ}の
\ruby{云}{い}ふにいよ〳〵
\ruby{慌}{あわ}てゝ
\ruby{車中}{しや|ちう}に
\ruby{入}{い}れば、どしんといふ
\ruby{恐}{おそ}ろしき
\ruby{音}{おと}して
\ruby{{\換字{扉}}}{と}は
\ruby{烈}{はげ}しく
\ruby{閉}{し}められ、
\ruby{號令笛}{あ|ひ|づ}はピーと
\ruby{鳴}{な}り、
\ruby{車}{くるま}は
\ruby{動}{うご}き
\ruby{出}{だ}しぬ。

\ruby{機關手}{きく|わん|しゆ}の
\ruby{手荒}{て|あら}き
\ruby{男}{をとこ}なればにや、
\ruby{車}{くるま}の
\ruby{俄然}{には|か}に
\ruby{{\換字{強}}}{つよ}く
\ruby{動}{うご}き
\ruby{出}{だ}したるに、
\ruby{人}{ひと}は
\ruby{車室}{しや|しつ}に
\ruby{多}{おほ}からざりしながら、いづくに
\ruby{座}{すわ}らんかと
\ruby{席}{せき}を
\ruby{取}{と}り
\ruby{{\換字{迷}}}{まよ}いて、
\ruby{未}{ま}だ
\ruby{身}{み}を
\ruby{落付}{おち|つ}くるに
\ruby{暇}{いとま}あらざりし
お
\ruby{龍}{りう}は、
\ruby{忽}{たちま}ち
\ruby{危}{あやふ}く
\ruby{倒}{たふ}れんとして、
\ruby{女}{をんな}の
\ruby{意氣地}{い|く|ぢ}
\ruby{無}{な}くよろ〳〵と
\ruby{歩}{あし}の
\ruby{縺}{もつ}るゝ
\ruby{時}{とき}、ハツと
\ruby{思}{おも}ひし
\ruby{折}{をり}は
\ruby{既}{すで}に
\ruby{遲}{おそ}くして、
\ruby{{\換字{猶}}}{なほ}
\ruby{新}{あたら}しき
\ruby{吾妻下駄}{あ|づま|げ|た}の、
\ruby{樫齒}{かし|ば}の
\ruby{角立}{かど|だ}てるを
\ruby{以}{も}てしたゝかに、
\ruby{後方}{うし|ろ}の
\ruby{人}{ひと}の
\ruby{足}{あし}を
\ruby{踏}{ふ}みたり。

