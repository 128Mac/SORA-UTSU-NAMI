\Entry{其二十五}

お
\ruby{龍}{りう}はやうやくにして
\ruby{踏止}{ふみ|とゞ}まりて、
\ruby{驚}{おどろ}き
\ruby{易}{やす}き
\ruby{女氣}{をんな|ぎ}のどつきりと
\ruby{胸}{むね}を
\ruby{躍}{をど}らせつ、
\ruby{何思案}{なに|し|あん}する
\ruby{暇}{ひま}も
\ruby{無}{な}く、

『
\ruby{御免}{ご|めん}なすつて
\ruby{下}{くだ}さいまし、
\ruby{飛}{と}んだ
\ruby{{\換字{過}}失}{そ|そう}を
\ruby{致}{いた}しました。
』

と
\ruby{振}{ふ}り
\ruby{顧}{かへ}りさまに
\ruby{先}{ま}づ
\ruby{謝}{わ}びて、
\ruby{心}{こゝろ}の
\ruby{之}{ゆ}くところを
\ruby{一}{ひ}ㇳ
\ruby{目見}{め|み}れば、
\ruby{是}{こ}は
\ruby{如何}{い|か}に
\ruby{足袋無}{た|び|な}き
\ruby{其}{そ}の
\ruby{人}{ひと}の
\ruby{足}{あし}の
\ruby{小指}{こ|ゆび}は、はや
\ruby{湧}{わ}き
\ruby{出}{い}づる
\ruby{血潮}{ち|しほ}に
\ruby{塗}{ぬ}れて、
\ruby{負傷}{け|が}の
\ruby{樣子}{やう|す}もおぼろげながら、
\ruby{岩根杜鵑花}{いは|ね|つ|ゝ|じ}の
\ruby{花}{はな}の
\ruby{影}{かげ}の
\ruby{流水}{なが|れ}の
\ruby{底}{そこ}に
\ruby{動}{うご}くが
\ruby{如}{ごと}くに
\ruby{紅色流}{くれ|なゐ|なが}れて
\ruby{止}{とゞ}まらず、いまだ
\ruby{{\換字{古}}}{ふる}びぬ
\ruby{薩摩下駄}{さつ|ま|げ|た}の、
\ruby{一}{ひ}ㇳ
\ruby{角}{すみ}は
\ruby{忽}{たちま}ち
\ruby{殷朱}{あ|け}となつたり。

あなやとばかり
\ruby{我}{われ}も
\ruby{驚}{おどろ}けば
\ruby{人}{ひと}も
\ruby{驚}{おどろ}きて、
\ruby{忙}{いそ}がはしく
\ruby{下駄}{げ|た}を
\ruby{{\換字{脱}}}{ぬ}ぎ
\ruby{捨}{す}てつ、
\ruby{男}{をとこ}は
\ruby{急}{きふ}に
\ruby{袂}{たもと}を
\ruby{掻探}{かい|さぐ}りしが、
\ruby{左方}{ひだ|り}にも
\ruby{右方}{み|ぎ}にも
\ruby{片紙}{へん|し}だに
\ruby{紙}{かみ}はあらずして、たゞ
\ruby{小}{ちひさ}き
\ruby{折本}{をり|ほん}のみの
\ruby{取出}{とり|いだ}されたる
\ruby{其間}{その|ま}に、
お
\ruby{龍}{りう}は
\ruby{既}{すで}に
\ruby{我}{わ}が
\ruby{小包}{こ|づゝみ}を
\ruby{傍}{かたへ}の
\ruby{座}{ざ}に
\ruby{置}{お}き、
\ruby{手早}{て|ばや}く
\ruby{帶}{おび}の
\ruby{間}{あひだ}より
\ruby{白紙}{はく|し}を
\ruby{取}{と}り
\ruby{出}{いだ}して、

『まあ
\ruby{何樣}{ど|う}して
\ruby{御謝罪}{お|わ|び}を
\ruby{致}{いた}したら
\ruby{宜}{よろ}しいのでしやう、
\ruby{飛}{と}んでも
\ruby{無}{な}い
\ruby{事}{こと}をいたしました。
どうかまあ
\ruby{貴下}{あな|た}、
\ruby{御腹立}{お|はら|だち}でしやうが
\ruby{何樣}{ど|う}か
\ruby{貴下}{あな|た}、
\ruby{御勘辯}{ご|かん|べん}なすつて
\ruby{下}{くだ}さいまし。
\ruby{定}{さだ}めし
\ruby{御痛}{お|いた}みでございましやう、あ〻
\ruby{濟}{す}みませんことをいたしました。
』

と
\ruby{面}{おもて}を
\ruby{赤}{あか}め
\ruby{淚}{なみだ}を
\ruby{含}{ふく}んで
\ruby{誠意}{まご|ゝろ}に
\ruby{謝罪}{わ|び}ながら、
\ruby{身}{み}を
\ruby{低}{ひく}く
\ruby{屈}{かゞ}めて
\ruby{血汚}{けが|れ}を
\ruby{拭}{ぬぐ}ひつゝ、
\ruby{塵埃}{ほこ|り}に
\ruby{穢}{よご}れたる
\ruby{足}{あし}の
\ruby{赭}{あか}く
\ruby{汚}{きたな}きを、
\ruby{繊々}{ほつ|そり}としたる
\ruby{指}{ゆび}の
\ruby{{\換字{雪}}}{ゆき}と
\ruby{白}{しろ}き
\ruby{手}{て}に
\ruby{執}{と}りて、
\ruby{早}{はや}くも
\ruby{拭}{ぬぐ}ひ
\ruby{捨}{す}つる
\ruby{紙}{かみ}の
\ruby{血}{ち}に
\ruby{染}{し}みて
\ruby{花鮮}{はな|あざ}やかなるを
\ruby{幾枚}{いく|まい}か
\ruby{散}{ち}らせば、
\ruby{男}{をとこ}は
お
\ruby{龍}{りう}の
\ruby{手}{て}を
\ruby{拂}{はら}ひのけ
\ruby{足}{あし}を
\ruby{縮}{ちゞ}めて、

『ナアニ
\ruby{構}{かま}ひません、これんばかりの
\ruby{事}{こと}、
\ruby{痛}{いた}くも
\ruby{何}{なん}ともありはしませんから、
\ruby{勘辯}{かん|べん}も
\ruby{何}{な}にもありや
\ruby{仕}{し}ません、たゞ
\ruby{潮時}{しほ|どき}の
\ruby{{\換字{所}}爲}{せ|ゐ}で
\ruby{血}{ち}がでるのでしやう。
\ruby{紙}{かみ}を
\ruby{少}{すこ}し
\ruby{頂戴}{いた|ゞ}きさへすりやあ
\ruby{宣}{よ}うございます。
』

と
\ruby{云}{い}ひし
\ruby{限}{き}り、ふたゝび
\ruby{手}{て}を
\ruby{觸}{ふ}れしめず、

『でも
\ruby{塵埃}{ご|み}でも
\ruby{入}{はい}りますと
\ruby{惡}{わる}うございますから。
』

と
\ruby{云}{い}ふをも
\ruby{更}{さら}に
\ruby{耳}{みゝ}に
\ruby{入}{い}れて、
\ruby{自}{みづか}ら
\ruby{一應淸潔}{いち|おう|せい|けつ}に
\ruby{拭}{ぬぐ}ひて、
\ruby{幾重}{いく|へ}にか
\ruby{疊}{たゝ}みたる
\ruby{紙}{かみ}に
\ruby{傷處}{き|ず}を
\ruby{包}{つゝ}めば、
お
\ruby{龍}{りう}は
\ruby{袂}{たもと}より
\ruby{絹}{きぬ}の
\ruby{白汗巾兒}{しろ|はん|け|ち}の
\ruby{淸}{きよ}げなるを
\ruby{出}{いだ}して、
\ruby{{\換字{前}}齒}{まへ|ば}に
\ruby{啣}{くは}ふるが
\ruby{早}{はや}きかピリヽと
\ruby{引}{ひ}き
\ruby{裂}{さ}き、
\ruby{男}{をとこ}の
\ruby{辭}{いな}まんとするを
\ruby{辭}{いな}む
\ruby{間}{ま}あらせず、
\ruby{體裁}{さ|ま}よく
\ruby{巧者}{かう|しや}にくる〳〵と
\ruby{{\換字{巻}}}{ま}きて
\ruby{引結}{ひき|むす}びけるが、
\ruby{裂}{さ}きたる
\ruby{時}{とき}に
\ruby{唇}{くち}にや
\ruby{觸}{ふ}れたりけん、その
\ruby{結}{むす}び
\ruby{餘}{あま}りの
\ruby{一端}{いつ|たん}には、
\ruby{血}{のり}ならぬ
\ruby{紅}{あか}きものゝ
\ruby{微}{かすか}に
\ruby{見}{み}えたり。

\ruby{車中}{しや|ちう}のすべての
\ruby{人々}{ひと|〴〵}の
\ruby{眼}{め}は、
\ruby{悉}{こと〴〵}く
\ruby{二人}{ふた|り}が
\ruby{上}{うへ}にのみ
\ruby{注}{そゝ}がれ
\ruby{居}{ゐ}るを、
\ruby{男}{をとこ}は
\ruby{上無}{うへ|な}く
\ruby{不樂}{わ|び}しくおぼえてや、

『
\ruby{紙捻}{こ|より}でも
\ruby{濟}{す}みましたものを
\ruby{御氣}{お|き}の
\ruby{毒}{どく}な!。
いろ〳〵
\ruby{御世話}{お|せ|わ}になつて
\ruby{却}{かへ}つて
\ruby{濟}{す}みませんでした。
』

と、
\ruby{云}{い}ふべきほどの
\ruby{挨拶}{あい|さつ}は
\ruby{眞四角}{まつ|し|かく}に
\ruby{云}{い}ひ
\ruby{仕舞}{し|ま}ひて、
\ruby{一寸}{ちよ|つと}こなたを
\ruby{見}{み}て
\ruby{會釋}{ゑし|やく}せしが、

『
\ruby{何樣}{ど|う}いたしまして、
\ruby{妾}{わたし}こそほんとに
\ruby{濟}{す}まない
\ruby{事}{こと}をいたしました。
\ruby{何卒御免}{なに|とぞ|ご|めん}なすつて
\ruby{下}{くだ}さいまし。
』

と、お
\ruby{龍}{りう}の
\ruby{云}{い}ひし
\ruby{詞}{ことば}は
\ruby{聞}{き}きしや
\ruby{聞}{き}かざりしや、
\ruby{愛想氣無}{あい|そ|げ|な}く
\ruby{後}{うしろ}を
\ruby{見}{み}せて
\ruby{車窓{\換字{近}}}{ま|ど|ちか}く
\ruby{居寄}{ゐ|よ}り、
\ruby{何見}{なに|み}るものあるべくもあらぬ
\ruby{窓外}{そ|と}の
\ruby{方}{かた}を
\ruby{見}{み}たる
\ruby{其}{そ}の
\ruby{横}{よこ}には、
\ruby{先刻}{さ|き}に
\ruby{懷中}{ふと|ころ}より
\ruby{出}{いだ}されたる
\ruby{小}{ちひさ}き
\ruby{折本}{をり|ほん}の
\ruby{置}{お}き
\ruby{棄}{す}てられたり。

\ruby{見}{み}る
\ruby{氣}{き}もなく
\ruby{何}{なん}の
\ruby{本}{ほん}かと
お
\ruby{龍}{りう}の
\ruby{見}{み}たる
\ruby{時}{とき}、
\ruby{其冊子}{その|は|ん}の
\ruby{最初}{さい|しよ}のところは
\ruby{丁度開}{ちよ|うど|あ}き
\ruby{居}{を}りて、
\ruby{配}{ふ}り
\ruby{假名}{が|な}のあるに
\ruby{誰}{たれ}にも
\ruby{解}{わか}りて、
\ruby{観世音菩薩}{くわん|ぜ|おん|ぼ|さつ}
\ruby{普門品}{ふ|もん|ぼん}とは
\ruby{明}{あき}らかに
\ruby{讀}{よ}めたり。

