\Entry{其二十五}

% メモ 校正終了 2024-04-24
\原本頁{134-3}%
お
\ruby{龍}{りう}は
やうやくに
して
\ruby{踏}{ふみ}
\ruby{止}{とゞ}まりて、% TODO 原本の「二の字点、揺すり点」に濁点のグリフが見つからないので「ゞ」
%
\ruby{驚}{おどろ}き
\ruby{易}{やす}き
\ruby{女氣}{をんな|ぎ}の
どつきりと
\ruby{胸}{むね}を
\ruby{躍}{をど}らせつ、
%
\ruby{何}{なに}
\ruby{思案}{し|あん}する
\ruby{暇}{ひま}も
\ruby{無}{な}く、

\原本頁{134-5}%
『
\ruby{御免}{ご|めん}なすつて
\ruby{下}{くだ}さいまし、
%
\ruby{飛}{と}んだ
\ruby{{\換字{過}}失}{そ|そう}を
\ruby{致}{いた}しました。
』

\原本頁{134-6}%
と
\ruby{振}{ふ}り
\ruby{顧}{かへ}り
さまに
\ruby{先}{ま}づ
\ruby{謝}{わ}びて、
%
\ruby{心}{こ〻ろ}の% 原本通り「〻(二の字点、揺すり点)」
\ruby{之}{ゆ}く
ところを
\ruby{一}{ひ}ト
\ruby{目}{め}
\ruby{見}{み}れば、
%
\ruby{是}{こ}は
\ruby{如何}{い|か}に
\ruby{足袋}{た|び}
\ruby{無}{な}き
\ruby{其}{そ}の
\ruby{人}{ひと}の
\ruby{足}{あし}の
\ruby{小指}{こ|ゆび}は、
%
はや
\ruby{湧}{わ}き
\ruby{出}{い}づる
\ruby{血潮}{ち|しほ}に
\ruby{塗}{まみ}れて、
%
\ruby{負傷}{け|が}の
\ruby{樣子}{やう|す}も
おぼろげ
ながら、
%
\ruby{岩根}{いは|ね}% 1 岩の根元。 2 どっしりと根を据えた大きな岩。いわがね。いわお。
\ruby{杜鵑花}{つ|〻|じ}の% 原本通り「〻(二の字点、揺すり点)」
\ruby{花}{はな}の
\ruby{影}{かげ}の
\ruby{流水}{なが|れ}の
\ruby{底}{そこ}に
\ruby{動}{うご}くが
\ruby{如}{ごと}くに
\ruby{紅色}{くれ|なゐ}
\ruby{流}{なが}れて
\ruby{止}{とゞ}まらす、% TODO 原本の「二の字点、揺すり点」に濁点のグリフが見つからないので「ゞ」
%
いまだ
\ruby{{\換字{古}}}{ふる}びぬ
\ruby{薩{\換字{摩}}下駄}{さつ|ま|げ|た}の、
%
\ruby{一}{ひ}ト
\ruby{角}{すみ}は
\ruby{忽}{たちま}ち
\ruby{殷朱}{あ|け}% 朱殷(しゅあん)とは、時間がたった血のような暗い朱色のこと
となつたり。

\原本頁{135-1}%
あなやと
ばかり
\ruby{我}{われ}も
\ruby{驚}{おどろ}けば
\ruby{人}{ひと}も
\ruby{驚}{おどろ}きて、
%
\ruby{忙}{いそ}がはしく
\ruby{下駄}{げ|た}を
\ruby{脫}{ぬ}ぎ
\ruby{捨}{す}てつ、
%
\ruby{男}{をとこ}は
\ruby{急}{きふ}に
\ruby{袂}{たもと}を
\ruby{掻}{かい}
\ruby{探}{さぐ}りしが、
%
\ruby{左方}{ひだ|り}にも
\ruby{右方}{み|ぎ}にも
\ruby{片紙}{へん|し}だに
\原本頁{135-3}\改行%
\ruby{無}{な}きに、
%
いよ〳〵
\ruby{慌}{あわ}て〻% 原本通り「〻(二の字点、揺すり点)」
\ruby{懷中}{ふと|ころ}に
\ruby{手}{て}を
\ruby{入}{い}れしかど、
こ〻にも% 原本通り「〻(二の字点、揺すり点)」
\ruby{生憎}{あひ|にく}
\原本頁{135-4}\改行%
\ruby{紙}{かみ}は
あらずして、
%
たゞ% TODO 原本の「二の字点、揺すり点」に濁点のグリフが見つからないので「ゞ」
\ruby{小}{ちひさ}き
\ruby{折本}{をり|ほん}のみの
\ruby{取出}{とり|いだ}されたる
\ruby{其}{その}
\ruby{間}{ま}に、
%
お
\ruby{龍}{りう}は
\ruby{既}{すで}に
\ruby{我}{わ}が
\ruby{小包}{こ|づ〻み}を% 原本通り「〻(二の字点、揺すり点)」
\ruby{傍}{かたへ}の
\ruby{座}{ざ}に
\ruby{置}{お}き、
%
\ruby{手早}{て|ばや}く
\ruby{帶}{おび}の
\ruby{間}{あひだ}より
\ruby{白紙}{はく|し}を
\ruby{取}{と}り
\ruby{出}{いだ}して、

\原本頁{135-7}%
『
まあ
\ruby{何樣}{ど|う}して
\ruby{御謝罪}{お|わ|び}を
\ruby{致}{いた}したら
\ruby{宜}{よろ}しい
のでしやう、
%
\ruby{飛}{と}んでも
\ruby{無}{な}い
\ruby{事}{こと}を
いたしました。
%
どうか
まあ
\ruby{貴下}{あな|た}、
%
\ruby{御腹立}{お|はら|だち}で
しやうが
\ruby{何樣}{ど|う}か
\ruby{貴下}{あな|た}、
%
\ruby{御勘辨}{ご|かん|べん}% 弁 瓣 辦 辧 (辨) 辩 辯
なすつて
\ruby{下}{くだ}さいまし。
%
\ruby{定}{さだ}めし
\ruby{御痛}{お|いた}みで
ございましやう、
%
あ〻% 原本通り「〻(二の字点、揺すり点)」
\ruby{濟}{す}みません
ことを
いたしました。
』

\原本頁{135-11}%
と
\ruby{面}{おもて}を
\ruby{赤}{あか}め
\ruby{涙}{なみだ}を
\ruby{含}{ふく}んで
\ruby{誠意}{ま|ご〻ろ}に% 原本通り「〻(二の字点、揺すり点)」
\ruby{謝罪}{わ|び}ながら、
%
\ruby{身}{み}を
\ruby{低}{ひく}く
\ruby{屈}{かゞ}めて% TODO 原本の「二の字点、揺すり点」に濁点のグリフが見つからないので「ゞ」
\ruby{血汚}{け|がれ}を
\ruby{拭}{ぬぐ}ひつ〻、% 原本通り「〻(二の字点、揺すり点)」
%
\ruby{塵埃}{ほこ|り}に
\ruby{穢}{よご}れたる
\ruby{足}{あし}の
\ruby{赭}{あか}く
\ruby{汚}{きたな}きを、
%
\ruby{繊々}{ほつ|そり}と
したる
\ruby{指}{ゆび}の
\ruby{{\換字{雪}}}{ゆき}と
\ruby{白}{しろ}き
\ruby{手}{て}に
\ruby{執}{と}りて、
%
\ruby{早}{はや}くも
\ruby{拭}{ぬぐ}ひ
\ruby{捨}{す}つる
\ruby{紙}{かみ}の
\ruby{血}{ち}に
\ruby{染}{し}みて
\ruby{花鮮}{はな|あざ}やか
なるを
\ruby{幾枚}{いく|まい}か
\ruby{散}{ち}らせば、
%
\ruby{男}{をとこ}は
お
\ruby{龍}{りう}の
\ruby{手}{て}を
\ruby{拂}{はら}ひのけ
\ruby{足}{あし}を
\ruby{縮}{ちゞ}めて、% TODO 原本の「二の字点、揺すり点」に濁点のグリフが見つからないので「ゞ」

\原本頁{136-5}%
『
ナアニ
\ruby{構}{かま}ひません、
%
これん
ばかりの
\ruby{事}{こと}、
%
\ruby{痛}{いた}くも
\ruby{何}{なん}とも
ありは
しませんから、
%
\ruby{勘辨}{かん|べん}も% 弁 瓣 辦 辧 (辨) 辩 辯
\ruby{何}{な}にも
ありや
\ruby{仕}{し}ません、
%
たゞ% TODO 原本の「二の字点、揺すり点」に濁点のグリフが見つからないので「ゞ」
\ruby{潮時}{しほ|どき}の
\ruby{{\換字{所}}爲}{せ|ゐ}で
\ruby{血}{ち}が
でる
ので
しやう。
%
\ruby{紙}{かみ}を
\ruby{少}{すこ}し
\ruby{頂戴}{いた|ゞ}き% TODO 原本の「二の字点、揺すり点」に濁点のグリフが見つからないので「ゞ」
さへ
すりやあ
\ruby{宜}{よ}う
ございます。
』

\原本頁{136-9}%
と
\ruby{云}{い}ひし
\ruby{限}{ぎ}り、
%
ふた〻び% 原本通り「〻(二の字点、揺すり点)」
\ruby{手}{て}を
\ruby{觸}{ふ}れ
しめず、

\原本頁{136-10}%
『
でも
\ruby{塵埃}{ご|み}でも
\ruby{入}{はい}りますと
\ruby{惡}{わる}う
ございます
から。
』

\原本頁{136-11}%
と
\ruby{云}{い}ふをも
\ruby{{\換字{更}}}{さら}に
\ruby{耳}{み〻}に% 原本通り「〻(二の字点、揺すり点)」
\ruby{入}{い}れで、
%
\ruby{自}{みづか}ら
\ruby{一應}{いち|おう}
\ruby{淸潔}{せい|けつ}に
\ruby{拭}{ぬぐ}ひて、
%
\ruby{幾重}{いく|へ}にか
\ruby{疊}{た〻}みたる% 原本通り「〻(二の字点、揺すり点)」
\ruby{紙}{かみ}に
\ruby{傷處}{き|ず}を
\ruby{包}{つ〻}めば、% 原本通り「〻(二の字点、揺すり点)」
%
お
\ruby{龍}{りう}は
\ruby{袂}{たもと}より
\ruby{絹}{きぬ}の
\ruby{白}{しろ}
\ruby{汗巾兒}{はん|け|ち}の
\ruby{淸}{きよ}げなるを
\ruby{出}{いだ}して、
%
\ruby{{\換字{前}}}{まへ}
\ruby{齒}{ば}に
\ruby{啣}{くは}ふるが
\ruby{早}{はや}きか
ピリヽと% カタカナ用の踊り字表記と見えないわけではない
\ruby{引}{ひ}き
\ruby{裂}{さ}き、
%
\ruby{男}{をとこ}の
\原本頁{137-3}\改行%
\ruby{辭}{いな}まん
とするを
\ruby{辭}{いな}む
\ruby{間}{ま}
あらせず、
%
\ruby{體裁}{さ|ま}
よく
\ruby{巧者}{かう|しや}に
くる〳〵と
\ruby{卷}{ま}きて
\ruby{引}{ひき}
\ruby{結}{むす}びけるが、
%
\ruby{裂}{さ}きたる
\ruby{時}{とき}に
\ruby{唇}{くち}にや
\ruby{觸}{ふ}れたりけん、
%
その
\ruby{結}{むす}び
\ruby{餘}{あま}りの
\ruby{一端}{いつ|たん}には、
%
\ruby{血}{のり}ならぬ
\ruby{紅}{あか}き
もの〻% 原本通り「〻(二の字点、揺すり点)」
\ruby{微}{かすか}に
\ruby{見}{み}えたり。

\原本頁{137-6}%
\ruby{車中}{しや|ちう}の
すべての
\ruby{人々}{ひと|〴〵}の
\ruby{眼}{め}は、
%
\ruby{悉}{こと〴〵}く
\ruby{二人}{ふた|り}が
\ruby{上}{うへ}に
のみ
\ruby{注}{そ〻}がれ% 原本通り「〻(二の字点、揺すり点)」
\ruby{居}{ゐ}るを、
%
\ruby{男}{をとこ}は
\ruby{上}{うへ}
\ruby{無}{な}く
\ruby{不樂}{わ|び}しく
おぼえてや、

\原本頁{137-8}%
『
\ruby{紙捻}{こ|より}でも
\ruby{濟}{す}みましたものを
\ruby{御氣}{お|き}の
\ruby{毒}{どく}な!。
%
いろ〳〵
\ruby{御世話}{お|せ|わ}に
なつて
\ruby{却}{かへ}つて
\ruby{濟}{す}みませんでした。
』

\原本頁{137-10}%
と、
%
\ruby{云}{い}ふべき
ほどの
\ruby{挨拶}{あい|さつ}は
\ruby{眞四角}{まつ|し|かく}に
\ruby{云}{い}ひ
\ruby{仕舞}{し|ま}ひて、
%
\ruby{一寸}{ちよ|つと}
こなたを
\ruby{見}{み}て
\ruby{會釋}{ゑ|しやく}せしが、

\原本頁{138-1}%
『
\ruby{何樣}{ど|う}
いたしまして、
%
\ruby{妾}{わたし}こそ
ほんとに
\ruby{濟}{す}まない
\ruby{事}{こと}を
いたしました。
%
\ruby{何卒}{どう|ぞ}
\ruby{御免}{ご|めん}なすつて
\ruby{下}{くだ}さいまし。
』

\原本頁{138-3}%
と、
%
お
\ruby{龍}{りう}の
\ruby{云}{い}ひし
\ruby{詞}{ことば}は
\ruby{聞}{き}きしや
\ruby{聞}{き}かざりしや、
%
\ruby{愛想氣}{あい|そ|げ}
\ruby{無}{な}く
\ruby{後}{うしろ}を
\ruby{見}{み}せて
\ruby{車窓{\換字{近}}}{ま|ど|ちか}く
\ruby{居寄}{ゐ|よ}り、
%
\ruby{何}{なに}
\ruby{見}{み}る
もの
ある
べくも
あらぬ
\ruby{窓外}{そ|と}の
\ruby{方}{かた}を
\ruby{見}{み}たる
\ruby{其}{そ}の
\ruby{横}{よこ}には、
%
\ruby{先刻}{さ|き}に
\ruby{懷中}{ふと|ころ}より
\ruby{出}{いだ}されたる
\ruby{小}{ちひさ}き
\ruby{折本}{をり|ほん}の
\ruby{置}{お}き
\ruby{棄}{す}てられたり。

\原本頁{138-7}%
\ruby{見}{み}る
\ruby{氣}{き}も
なく
\ruby{何}{なん}の
\ruby{本}{ほん}かと
お
\ruby{龍}{りう}の
\ruby{見}{み}たる
\ruby{時}{とき}、
%
\ruby{其}{その}
\ruby{册子}{ほ|ん}の
\ruby{最初}{さい|しよ}の
ところは
\ruby{丁度}{ちやう|ど}
\ruby{開}{あ}き
\ruby{居}{を}りて、
%
\ruby{配}{ふ}り
\ruby{假名}{が|な}の
あるに
\ruby{誰}{たれ}にも
\ruby{解}{わか}りて、
%
\ruby{觀世音菩薩}{くわん|ぜ|おん|ぼ|さつ}
\ruby{普門品}{ふ|もん|ぼん}とは
\ruby{明}{あき}らかに
\ruby{讀}{よ}めたり。
