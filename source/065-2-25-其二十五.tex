\Entry{其二十五}

% メモ 校正終了 2024-04-24 2024-06-01 2024-07-02
\原本頁{134-3}%
お
\ruby{龍}{りう}は
やうやくに
して
\ruby{踏}{ふみ}
\ruby{止}{とゞ}まりて、% 踊り字調整「〻(二の字点、揺すり点)に濁点に見えるが(ゞ)」
%
\ruby{驚}{おどろ}き
\ruby{易}{やす}き
\ruby[g]{女氣}{をんなぎ}の
どつきりと
\ruby{胸}{むね}を
\ruby{躍}{をど}らせつ、
%
\ruby{何}{なに}
\ruby[g]{思案}{し あん}する
\ruby{暇}{ひま}も
\ruby{無}{な}く、

\原本頁{134-5}%
『
\ruby[g]{御免}{ご めん}なすつて
\ruby{下}{くだ}さいまし、
%
\ruby{飛}{と}んだ
\ruby[g]{{\換字{過}}失}{そ そう}を
\ruby{致}{いた}しました。
』

\原本頁{134-6}%
と
\ruby{振}{ふ}り
\ruby{顧}{かへ}り
さまに
\ruby{先}{ま}づ
\ruby{謝}{わ}びて、
%
\ruby{心}{こゝろ}の% 踊り字調整「〻(二の字点、揺すり点)に見えるが(ゝ)」
\ruby{之}{ゆ}く
ところを
\ruby{一}{ひ}ト
\ruby{目}{め}
\ruby{見}{み}れば
\改行% 校正作業の簡略化のため
、
%
\原本頁{134-7}\改行%
\ruby{是}{こ}は
\ruby[g]{如何}{い か }に
\ruby[g]{足袋}{た び }
\ruby{無}{な}き
\ruby{其}{そ}の
\ruby{人}{ひと}の
\ruby{足}{あし}の
\ruby[g]{小指}{こ ゆび}は、
%
はや
\ruby{湧}{わ}き
\ruby{出}{い}づる
\ruby[g]{血潮}{ち しほ}に
\ruby{塗}{まみ}れて、
%
\ruby[g]{負傷}{け が }の
\ruby[g]{樣子}{やうす }も
おぼろげ
ながら、
%
\ruby[g]{岩根}{いはね }% 1 岩の根元。 2 どっしりと根を据えた大きな岩。いわがね。いわお。
\ruby{杜鵑花}{つ|ゝ|じ}の% 踊り字調整「〻(二の字点、揺すり点)に見えるが(ゝ)」
\ruby{花}{はな}の
\ruby{影}{かげ}の
\ruby[g]{流水}{ながれ }の
\ruby{底}{そこ}に
\ruby{動}{うご}くが
\ruby{如}{ごと}くに
\ruby[g]{紅色}{くれなゐ}
\ruby{流}{なが}れて
\ruby{止}{とゞ}まらす、% 踊り字調整「〻(二の字点、揺すり点)に濁点に見えるが(ゞ)」
% 『新続古今集』・竜田川いはねのつつじ影みえてなほ水くくる春のくれなゐ<藤原定家
% 『春霞集』  ・岩つつじ岩根の水にうつる火の影とみるまで眺めくらしぬ<毛利元就
%
いまだ
\ruby{{\換字{古}}}{ふる}びぬ
\ruby{薩{\換字{摩}}下駄}{さつ|ま|げ|た}の、
%
\ruby{一}{ひ}ト
\ruby{角}{すみ}は
\ruby[g]{忽ち}{たちま }
\ruby[g]{殷朱}{あ け }% 朱殷(しゅあん)とは、時間がたった血のような暗い朱色のこと
となつたり。

\原本頁{135-1}%
あなやと
ばかり
\ruby{我}{われ}も
\ruby{驚}{おどろ}けば
\ruby{人}{ひと}も
\ruby{驚}{おどろ}きて、
%
\ruby{忙}{いそ}がはしく
\ruby[g]{下駄}{げ た }を
\ruby{脫}{ぬ}ぎ
\ruby{捨}{す}てつ、
%
\ruby{男}{をとこ}は
\ruby{急}{きふ}に
\ruby{袂}{たもと}を
\ruby{掻}{かい}
\ruby{探}{さぐ}りしが、
%
\ruby[g]{左方}{ひだり }にも
\ruby[g]{右方}{み ぎ }にも
\ruby[g]{片紙}{へんし }だに
\原本頁{135-3}\改行%
\ruby{無}{な}きに、
%
いよ〳〵
\ruby{慌}{あわ}てゝ% 踊り字調整「〻(二の字点、揺すり点)に見えるが(ゝ)」
\ruby[g]{懷中}{ふところ}に
\ruby{手}{て}を
\ruby{入}{い}れしかど、
こゝにも% 踊り字調整「〻(二の字点、揺すり点)に見えるが(ゝ)」
\ruby[g]{生憎}{あひにく}
\原本頁{135-4}\改行%
\ruby{紙}{かみ}は
あらずして、
%
たゞ% 踊り字調整「〻(二の字点、揺すり点)に濁点に見えるが(ゞ)」
\ruby{小}{ちひさ}き
\ruby[g]{折本}{をりほん}のみの
\ruby[g]{取出}{とりいだ}されたる
\ruby{其}{その}
\ruby{間}{ま}に、
%
お
\ruby{龍}{りう}は
\ruby{既}{すで}に
\ruby{我}{わ}が
\ruby[g]{小包}{こづゝみ}を% 踊り字調整「〻(二の字点、揺すり点)に見えるが(ゝ)」
\ruby{傍}{かたへ}の
\ruby{座}{ざ}に
\ruby{置}{お}き、
%
\ruby[g]{手早}{て ばや}く
\ruby{帶}{おび}の
\ruby{間}{あひだ}より
\ruby[g]{白紙}{はくし }を
\ruby{取}{と}り
\ruby{出}{いだ}して、

\原本頁{135-7}%
『
まあ
\ruby[g]{何樣}{ど う }して
\ruby{御謝罪}{お|わ|び}を
\ruby{致}{いた}したら
\ruby{宜}{よろ}しい
のでしやう、
%
\ruby{飛}{と}んでも
\ruby{無}{な}い
\ruby{事}{こと}を
いたしました。
%
どうか
まあ
\ruby[g]{貴下}{あなた }、
%
\ruby{御腹立}{お|はら|だち}で
しやうが
\ruby[g]{何樣}{ど う }か
\ruby[g]{貴下}{あなた }、
%
\ruby{御勘辨}{ご|かん|べん}% 弁 瓣 辦 辧 (辨) 辩 辯
なすつて
\ruby{下}{くだ}さいまし。
%
\ruby{定}{さだ}めし
\ruby[g]{御痛}{お いた}みで
ございましやう、
%
あゝ% 踊り字調整「〻(二の字点、揺すり点)に見えるが(ゝ)」
\ruby{濟}{す}みません
ことを
いたしました。
』

\原本頁{135-11}%
と
\ruby{面}{おもて}を
\ruby{赤}{あか}め
\ruby{涙}{なみだ}を
\ruby{含}{ふく}んで
\ruby[g]{誠意}{まごゝろ}に% 踊り字調整「〻(二の字点、揺すり点)に見えるが(ゝ)」
\ruby[g]{謝罪}{わ び }ながら、
%
\ruby{身}{み}を
\ruby{低}{ひく}く
\ruby{屈}{かゞ}めて% 踊り字調整「〻(二の字点、揺すり点)に濁点に見えるが(ゞ)」
\ruby[g]{血汚}{け がれ}を
\ruby{拭}{ぬぐ}ひつゝ、% 踊り字調整「〻(二の字点、揺すり点)に見えるが(ゝ)」
%
\ruby[g]{塵埃}{ほこり }に
\ruby{穢}{よご}れたる
\ruby{足}{あし}の
\ruby{赭}{あか}く
\ruby{汚}{きたな}きを、
%
\ruby[g]{繊々}{ほつそり}と
したる
\ruby{指}{ゆび}の
\ruby{{\換字{雪}}}{ゆき}と
\ruby{白}{しろ}き
\ruby{手}{て}に
\ruby{執}{と}りて、
%
\ruby{早}{はや}くも
\ruby{拭}{ぬぐ}ひ
\ruby{捨}{す}つる
\ruby{紙}{かみ}の
\ruby{血}{ち}に
\ruby{染}{し}みて
\ruby[g]{花鮮}{はなあざ}やか
なるを
\ruby[g]{幾枚}{いくまい}か
\ruby{散}{ち}らせば、
%
\ruby{男}{をとこ}は
お
\ruby{龍}{りう}の
\ruby{手}{て}を
\ruby{拂}{はら}ひのけ
\ruby{足}{あし}を
\ruby{縮}{ちゞ}め% 踊り字調整「〻(二の字点、揺すり点)に濁点に見えるが(ゞ)」
\改行% 校正作業の簡略化のため
て、

\原本頁{136-5}%
『
ナアニ
\ruby{構}{かま}ひません、
%
これん
ばかりの
\ruby{事}{こと}、
%
\ruby{痛}{いた}くも
\ruby{何}{なん}とも
ありは
しませんから、
%
\ruby[g]{勘辨}{かんべん}も% 弁 瓣 辦 辧 (辨) 辩 辯
\ruby{何}{な}にも
ありや
\ruby{仕}{し}ません、
%
たゞ% 踊り字調整「〻(二の字点、揺すり点)に濁点に見えるが(ゞ)」
\ruby[g]{潮時}{しほどき}の
\ruby[g]{{\換字{所}}爲}{せ ゐ }で
\ruby{血}{ち}が
でる
ので
しやう。
%
\ruby{紙}{かみ}を
\ruby{少}{すこ}し
\ruby[g]{頂戴}{いたゞ }き% 踊り字調整「〻(二の字点、揺すり点)に濁点に見えるが(ゞ)」
さへ
すりやあ
\ruby{宜}{よ}う
ございます。
』

\原本頁{136-9}%
と
\ruby{云}{い}ひし
\ruby{限}{ぎ}り、
%
ふたゝび% 踊り字調整「〻(二の字点、揺すり点)に見えるが(ゝ)」
\ruby{手}{て}を
\ruby{觸}{ふ}れ
しめず、

\原本頁{136-10}%
『
でも
\ruby[g]{塵埃}{ご み }でも
\ruby{入}{はい}りますと
\ruby{惡}{わる}う
ございます
から。
』

\原本頁{136-11}%
と
\ruby{云}{い}ふをも
\ruby{{\換字{更}}}{さら}に
\ruby{耳}{みゝ}に% 踊り字調整「〻(二の字点、揺すり点)に見えるが(ゝ)」
\ruby{入}{い}れで、
%
\ruby{自}{みづか}ら
\ruby[g]{一應}{いちおう}
\ruby[g]{淸潔}{せいけつ}に
\ruby{拭}{ぬぐ}ひて、
%
\ruby[g]{幾重}{いくへ }にか
\ruby{疊}{たゝ}みたる% 踊り字調整「〻(二の字点、揺すり点)に見えるが(ゝ)」
\ruby{紙}{かみ}に
\ruby[g]{傷處}{き ず }を
\ruby{包}{つゝ}めば、% 踊り字調整「〻(二の字点、揺すり点)に見えるが(ゝ)」
%
お
\ruby{龍}{りう}は
\ruby{袂}{たもと}より
\ruby{絹}{きぬ}の
\ruby{白}{しろ}
\ruby{汗巾兒}{はん|け|ち}の
\ruby{淸}{きよ}げなるを
\ruby{出}{いだ}して、
%
\ruby{{\換字{前}}}{まへ}
\ruby{齒}{ば}に
\ruby{{\換字{啣}}}{くは}ふるが
\ruby{早}{はや}きか
ピリヽと% カタカナ用の踊り字表記と見えないわけではない
\ruby{引}{ひ}き
\ruby{裂}{さ}き、
%
\ruby{男}{をとこ}の
\原本頁{137-3}\改行%
\ruby{辭}{いな}まん
とするを
\ruby{辭}{いな}む
\ruby{間}{ま}
あらせず、
%
\ruby[g]{體裁}{さ ま }
よく
\ruby[g]{巧者}{かうしや}に
くる〳〵と
\ruby{卷}{ま}きて
\ruby{引}{ひき}
\ruby{結}{むす}びけるが、
%
\ruby{裂}{さ}きたる
\ruby{時}{とき}に
\ruby{唇}{くち}にや
\ruby{觸}{ふ}れたりけん、
%
その
\ruby{結}{むす}び
\ruby{餘}{あま}りの
\ruby[g]{一端}{いつたん}には、
%
\ruby{血}{のり}ならぬ
\ruby{紅}{あか}き
ものゝ% 踊り字調整「〻(二の字点、揺すり点)に見えるが(ゝ)」
\ruby{微}{かすか}に
\ruby{見}{み}えたり。

\原本頁{137-6}%
\ruby[g]{車中}{しやちう}の
すべての
\ruby[g]{人々}{ひと〴〵}の
\ruby{眼}{め}は、
%
\ruby[<j>]{悉}{こと〴〵}く% ルビ調整(配置位置調整)若干ルビの位置を変更
\ruby[g]{二人}{ふたり }が
\ruby{上}{うへ}に
のみ
\ruby{注}{そゝ}がれ% 踊り字調整「〻(二の字点、揺すり点)に見えるが(ゝ)」
\ruby{居}{ゐ}るを
\改行% 校正作業の簡略化のため
、
%
\原本頁{137-7}\改行%
\ruby{男}{をとこ}は
\ruby{上}{うへ}
\ruby{無}{な}く
\ruby[g]{不樂}{わ び }しく
おぼえてや、

\原本頁{137-8}%
『
\ruby[g]{紙捻}{こ より}でも
\ruby{濟}{す}みましたものを
\ruby[g]{御氣}{お き }の
\ruby{毒}{どく}な!。
%
いろ〳〵
\ruby{御世話}{お|せ|わ}に
なつて
\ruby{却}{かへ}つて
\ruby{濟}{す}みませんでした。
』

\原本頁{137-10}%
と、
%
\ruby{云}{い}ふべき
ほどの
\ruby[g]{挨拶}{あいさつ}は
\ruby{眞四角}{まつ|し|かく}に
\ruby{云}{い}ひ
\ruby[g]{仕舞}{し ま }ひて、
%
\ruby[g]{一寸}{ちよつと}
こなたを
\ruby{見}{み}て
\ruby[g]{會釋}{ゑしやく}せしが、

\原本頁{138-1}%
『
\ruby[g]{何樣}{ど う }
いたしまして、
%
\ruby{妾}{わたし}こそ
ほんとに
\ruby{濟}{す}まない
\ruby{事}{こと}を
いたしました。
%
\ruby[g]{何卒}{どうぞ }
\ruby[g]{御免}{ご めん}なすつて
\ruby{下}{くだ}さいまし。
』

\原本頁{138-3}%
と、
%
お
\ruby{龍}{りう}の
\ruby{云}{い}ひし
\ruby{詞}{ことば}は
\ruby{聞}{き}きしや
\ruby{聞}{き}かざりしや、
%
\ruby{愛想氣}{あい|そ|げ}
\ruby{無}{な}く
\ruby{後}{うしろ}を
\ruby{見}{み}せて
\ruby{車窓{\換字{近}}}{ま|ど|ぢか}く
\ruby[g]{居寄}{ゐ よ }り、
%
\ruby{何}{なに}
\ruby{見}{み}る
もの
ある
べくも
あらぬ
\ruby[g]{窓外}{そ と }の
\ruby{方}{かた}を
\ruby{見}{み}たる
\ruby{其}{そ}の
\ruby{横}{よこ}には、
%
\ruby[g]{先刻}{さ き }に
\ruby[g]{懷中}{ふところ}より
\ruby{出}{いだ}されたる
\ruby{小}{ちひさ}き
\ruby[g]{折本}{をりほん}の
\ruby{置}{お}き
\ruby{棄}{す}てられたり。

\原本頁{138-7}%
\ruby{見}{み}る
\ruby{氣}{き}も
なく
\ruby{何}{なん}の
\ruby{本}{ほん}かと
お
\ruby{龍}{りう}の
\ruby{見}{み}たる
\ruby{時}{とき}、
%
\ruby{其}{その}
\ruby[g]{册子}{ほ ん }の
\ruby[g]{最初}{さいしよ}の
とこ
\原本頁{138-8}\改行%
ろは
\ruby[g]{丁度}{ちやうど}
\ruby{開}{あ}き
\ruby{居}{を}りて、
%
\ruby{配}{ふ}り
\ruby[g]{假名}{が な }の
あるに
\ruby{誰}{たれ}にも
\ruby{解}{わか}りて、
%
\ruby{觀世音}{くわん|ぜ|おん}
\ruby[g]{菩薩}{ぼ さつ}
\ruby{普門品}{ふ|もん|ぼん}とは
\ruby{明}{あき}らかに
\ruby{讀}{よ}めたり。
