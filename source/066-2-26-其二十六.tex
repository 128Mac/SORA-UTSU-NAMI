\Entry{其二十六}

\ruby{思}{おも}ひのほかの
\ruby{品}{もの}なりしに、お
\ruby{龍}{りう}は
\ruby{驚}{おどろ}き
\ruby{疑}{うたが}ひて、
\ruby{露照}{つゆ|て}る
\ruby{美}{うつく}しき
\ruby{眼}{まなこ}を
\ruby{睜}{みは}り、あらためて
\ruby{男}{をとこ}を
\ruby{一}{ひ}ㇳ
\ruby{目見}{め|み}しが、
\ruby{男}{をとこ}はそれとも
\ruby{心付}{こヽろ|づ}かず
\ruby{{\換字{猶}}}{なほ}
\ruby{車外}{そ|と}を
\ruby{見居}{み|ゐ}たり。

\ruby{既}{すで}に
\ruby{其人}{その|ひと}の
\ruby{履物}{はき|もの}の
\ruby{汚}{よご}れを
\ruby{{\換字{清}}}{きよ}めたり、
\ruby{落}{お}ち
\ruby{散}{ち}つたる
\ruby{紙}{かみ}の
\ruby{眼}{め}に
\ruby{厭}{いと}はしきをも
\ruby{一}{ひ}ㇳ
\ruby{纏}{まとめ}にして
\ruby{投}{な}げ
\ruby{棄}{す}てたり、
\ruby[g]{謝罪}{あやま}るほどはあやまりて、
\ruby{今}{いま}は
\ruby{何爲}{なに|す}べき
\ruby{事}{こと}も
\ruby{無}{な}きなり。
お
\ruby{龍}{りう}は
\ruby{彼}{か}の
\ruby{男}{をとこ}とは
\ruby[g]{斜線}{すじかひ}に、
\ruby{其}{そ}の
\ruby{反對}{はん|たい}の
\ruby{側}{がは}の
\ruby{車窓近}{ま|ど|ちか}き
\ruby{席}{せき}を
\ruby{取}{と}りて、はじめて
\ruby{身}{み}をも
\ruby{心}{こヽろ}をもおちつけたり。

\ruby{普門品}{ふ|もん|ぼん}!、あの
\ruby{普門品}{ふ|もん|ぼん}!、
\ruby{彼書}{あ|れ}はたしか
\ruby{觀音樣}{くわん|のん|さま}を
\ruby{信}{しん}ずる
\ruby{人}{ひと}の
\ruby{讀}{よ}む
\ruby{御經}{お|きやう}!。
\ruby[g]{五十六十}{ごじうろくじう}の
\ruby{爺婆}{じゝ|ばゝ}ならばいざ
\ruby{知}{し}らず、
\ruby{若}{わか}い
\ruby{盛}{さか}りの
\ruby{當世}{い|ま}の
\ruby{人}{ひと}の、しかも
\ruby[g]{古風}{むかし}を
\ruby{守}{まも}る
\ruby[g]{農夫町人}{ひやくしやうちやうにん}で〻もある
\ruby{事}{こと}か、
\ruby{新}{あたら}しきを
\ruby{{\GWI{u8ffd-k}}}{お}うて
\ruby{學問}{がく|もん}に
\ruby{身}{み}を
\ruby{責}{せ}めれば、まづ
\ruby{神佛}{かみ|ほとけ}とは
\ruby{緣}{\GWI{u1b001}ん}の
\ruby{{\GWI{u9060-k}}}{とほ}さうな
\ruby{書生風}{しょ|せい|ふう}の
\ruby{此様}{こ|う}いふ
\ruby{人}{ひと}の
\ruby{懷中}{ふと|ころ}から、
\ruby{普門品}{ふ|もん|ぼん}とは
\ruby{似合}{に|あ}はしからぬ!。
\ruby{何}{ど}のやうな
\ruby{悲}{かな}しい
\ruby{願}{ねがひ}があつての
\ruby{佛頼}{ほとけ|だの}みか
\ruby{知}{し}らねど、あ〻
\ruby{想}{おも}ひ
\ruby{出}{だ}しても
\ruby{胸}{むね}が
\ruby{痛}{いた}む、
\ruby{妾}{わたし}も
\ruby{一昨年}{を|とヽ|し}の
\ruby{丁度今頃}{ちや|うど|いま|ごろ}、
\ruby{思}{おも}ふ
\ruby{人}{ひと}には
\ruby{{\GWI{u9060-k}}}{とほ}く
\ruby{離}{はな}れて、
\ruby{空}{そら}の
\ruby{色}{いろ}も
\ruby{風}{かぜ}の
\ruby{音}{おと}も
\ruby{{\換字{情}}無}{なさけ|な}い
\ruby{知}{し}らぬ
\ruby{他国}{た|こく}の
\ruby[g]{駿府}{すんぷ}の
\ruby{秋}{あき}、いくら
\ruby{手紙}{て|がみ}を
\ruby{出}{だ}しても
\ruby{問訊}{とひ|たづ}ねしても、
\ruby{{\GWI{u8fd4-k}}事}{へん|じ}さへ
\ruby{來}{こ}ないのが
\ruby{氣}{き}になつて
\ruby{氣}{き}になつて、よもやとは
\ruby{思}{おも}へども
\ruby{心變}{こヽろ|がは}りか、それともまた
\ruby{病患}{やみ|わづらひ}でも
\ruby{仕}{し}てヾは
\ruby{無}{な}いかと、
\ruby{恨}{うら}めしくもあれば
\ruby{心細}{こヽろ|ぼそ}くもあり、はては
\ruby{茫然}{ぼん|やり}と
\ruby{門口}{かど|ぐち}に
\ruby{立}{た}つて、
\ruby{何}{なに}が
\ruby{見}{み}えるでもない
\ruby{東京}{とう|きやう}の
\ruby{方}{ほう}を、
\ruby{空}{くう}に
\ruby{見詰}{み|つ}めてはほろり〳〵と、
\ruby{馬鹿}{ば|か}らしいほど
\ruby{泣}{な}いて
\ruby{泣}{な}いた
\ruby{末}{すへ}、
\ruby{思案}{し|あん}に
\ruby{餘}{あま}つたところから
\ruby{願掛}{がん|か}けして、
\ruby{安東}{あん|どう}の
\ruby{{\換字{清}}水}{きよ|みづ}の
\ruby{觀音樣}{くわん|のん|さま}には
\ruby{御經}{おき|やう}こそ
\ruby{誦}{あ}げなかつたが
\ruby{日參}{ひ|まゐ}りもすれば、
\ruby[g]{足久保}{あしくぼ}の
\ruby{楠木}{くす|のき}の
\ruby{觀音樣}{くわん|のん|さま}の
\ruby{御利生}{ご|りし|やう}の
\ruby{話}{はなし}を
\ruby{聞}{き}いては、
\ruby{二里}{に|り}からの
\ruby{田舎{\GWI{u9053-k}}}{ゐ|なか|みち}を
\ruby{歩}{ある}いた
\ruby{上}{うへ}に、
\ruby{草臥{\GWI{u8fd4-k}}}{くた|びれ|かへ}りながら、
\ruby{御百度}{おひ|やく|ど}まで
\ruby{踏}{ふ}んで、
\ruby{何卒手紙}{どう|ぞ|て|がみ}の
\ruby{{\GWI{u8fd4-k}}事}{へん|じ}の
\ruby{參}{まゐ}りまして
\ruby[g]{彼方}{あちら}の
\ruby{様子}{やう|す}の
\ruby{分}{わか}りまするやう、
\ruby{若}{も}し
\ruby{{\換字{叉}}病氣災難}{また|びや|うき|さい|なん}にでも
\ruby{罹}{かゝ}つて
\ruby{居}{を}りまするなら、
\ruby{御利{\GWI{u76ca-k}}}{ご|り|やく}をもつて
\ruby{助}{たす}かりまするようにと、
\ruby{自分}{じ|ぶん}の
\ruby[g]{身體}{からだ}は
\ruby{一日}{いち|にち}
\ruby{一日}{いち|にち}
\ruby{削}{けづ}るやうに
\ruby{癯}{や}せるのも
\ruby{餘所}{よ|そ}にして、
\ruby{一心}{いつ|しん}になつて
\ruby{信心}{しん|じん}を
\ruby{仕}{し}た
\ruby{苦}{くる}しい
\ruby{切}{せつ}ない
\ruby{經驗}{おぼ|\GWI{u1b001}}もあるが、
\ruby{忘}{わす}れても
\ruby{爲}{す}まいものは
\ruby{戀路}{こひ|ぢ}の
\ruby{{\GWI{u8ff7-k}}}{まよ}ひ、
\ruby{思}{おも}つて
\ruby{思}{おも}ひ
\ruby{止}{や}む
\ruby{日}{ひ}も
\ruby{無}{な}ければ、
\ruby{泣}{な}いて
\ruby{泣}{な}き
\ruby{足}{た}る
\ruby{夜}{よる}も
\ruby{無}{な}く、
\ruby{生}{い}きては
\ruby{居}{ゐ}ても
\ruby{生}{い}きたくも
\ruby{無}{な}く、
\ruby{死}{し}なうとしても
\ruby{死}{し}にきれもせぬ
\ruby{彼}{あ}の
\ruby{厭}{いや}な〳〵な
\ruby{{\換字{情}}無}{なさ|けな}い
\ruby{心持}{こヽろ|もち}!。
\ruby{我身}{わが|み}の
\ruby{痛}{いた}かりし
\ruby{經驗}{おぼ|\GWI{u1b001}}に
\ruby{人}{ひと}の
\ruby{痛}{いた}さも
\ruby{思}{おも}はるゝが、あヽ
\ruby{{\換字{猶}}若}{まだ|わか}い
\ruby{此}{こ}の
\ruby{人}{ひと}の
\ruby{信心}{しん|じん}の、よしや
\ruby{頼}{たの}み
\ruby{無}{な}き
\ruby{老人}{とし|より}の
\ruby{親}{おや}の
\ruby{病氣}{びや|うき}の
\ruby{爲故}{ため|ゆゑ}でもあれ、また
\ruby{何}{ど}の
\ruby{様}{やう}な
\ruby{辛}{つら}い
\ruby{悲}{かな}しい
\ruby{{\GWI{u907a-k}}}{や}る
\ruby{瀬無}{せ|な}い
\ruby{事}{こと}のためでもあれ、たヾ
\ruby{戀故}{こひ|ゆゑ}の
\ruby{信心}{しん|〴〵}で
\ruby{無}{な}かれかし。
\ruby{今妾}{いま|わたし}が
\ruby{仕}{し}たる
\ruby{{\GWI{u904e-k}}失}{あや|まち}は、
\ruby{時}{とき}の
\ruby{拍子}{へう|し}の
\ruby{事}{こと}なれば、
\ruby{誰}{だれ}も
\ruby{容赦}{ゆ|る}しては
\ruby{{\換字{呉}}}{く}れさうな
\ruby{譯}{わけ}ながら、あれほどの
\ruby{血}{ち}の
\ruby{出}{で}た
\ruby{負傷}{け|が}を
\ruby{仕}{し}て、
\ruby{露腹立}{つゆ|はら|だ}たしげな
\ruby{顏色}{かほ|つき}もせず、また
\ruby{恨}{うら}めしき
\ruby{眼色}{め|つき}もせず、
\ruby{毫}{すこし}も
\ruby{變}{かは}つた
\ruby{様子}{やう|す}は
\ruby{無}{な}くて、
\ruby{水}{みづ}の
\ruby{流}{なが}れたやうにさらりと
\ruby{濟}{す}ませて、
\ruby{後}{あと}には
\ruby{物}{もの}も
\ruby{殘}{のこ}さぬ
\ruby{風{\換字{情}}}{ふ|ぜい}の
\ruby{寛大}{おほ|やう}さ!。
\ruby{{\GWI{u7d42-ue0101}}}{しまひ}には
\ruby{反對}{あべ|こべ}に
\ruby{禮}{れい}まで
\ruby{言}{い}ひたるに
\ruby{心}{こヽろ}の
\ruby{優}{やさ}しさは
\ruby{見}{み}えながら、それから
\ruby{知}{し}らぬ
\ruby{顔}{かほ}つくつて、
\ruby{彼方向}{あち|ら|む}いたる
\ruby{振舞}{ふる|まひ}の
\ruby{少}{すこ}し
\ruby{素氣無}{す|げ|な}きに、
\ruby{飼}{か}はれても
\ruby{人}{ひと}の
\ruby{氣}{き}は
\ruby{取}{と}らぬ
\ruby{鷹}{たか}の
\ruby{素振}{そ|ぶり}の、
\ruby{一寸憎}{ちよ|つと|にく}らしいほどな
\ruby{氣位}{きぐ|らゐ}もあらはれて、
\ruby{女}{をんな}さへ
\ruby{見}{み}れば
\ruby{{\換字{嫌}}}{いや}に
\ruby{笑}{わら}ひ
\ruby{掛}{か}ける、
\ruby{世}{よ}に
\ruby{有}{あ}りふれた
\ruby{若}{わか}い
\ruby{人}{ひと}などゝは、
\ruby{其}{そ}の
\ruby{行方}{ゆき|がた}も
\ruby{全}{まる}で
\ruby{異}{かは}れど、さればといつてぎしつきもせず、
\ruby{氣立}{き|だて}ても
\ruby[g]{心持}{こヽろもち}も
\ruby{何}{なに}と
\ruby{無}{な}く
\ruby{{\GWI{u9055-k}}}{ちが}つて、
\ruby{衣服容姿}{な|り|すが|た}は
\ruby{此}{これ}といふことも
\ruby{無}{な}き
\ruby{書生}{しよ|せい}ながら、おのづと
\ruby{普{\GWI{u901a-k}}}{な|み}には
\ruby{思}{おも}へぬところある
\ruby{人}{ひと}!。
\ruby{斯様}{か|う}いふ
\ruby{調子}{てう|し}あひの
\ruby{人}{ひと}なんぞが、
\ruby{若}{も}し
\ruby[g]{萬一}{ひよつと}
\ruby[g]{十年二十年}{じうねんにじうねん}の
\ruby{後}{のち}になつて、
\ruby[g]{立派}{りつぱ}な
\ruby{傑}{すぐ}れた
\ruby{人}{ひと}なんぞになるのではあるまいか?あヽ
\ruby{修行盛}{しゆ|ぎやう|ざか}り
\ruby{出世盛}{しゆ|つせ|ざか}りの
\ruby{此}{こ}の
\ruby{若}{わか}い
\ruby{人}{ひと}!、それに
\ruby{付}{つ}けつても
\ruby{彼}{あ}の
\ruby{普門品}{ふ|もん|ぼん}!。
\ruby{屹度果敢無}{きつ|と|は|か|な}い
\ruby{戀}{こひ}なぞの、
\ruby{其様}{そ|ん}な
\ruby{事}{こと}のためではあるまいなれど、どうか
\ruby{戀}{こひ}ゆえの
\ruby{信心}{しん|じん}で
\ruby{無}{な}かれかし!。
と
\ruby{我身}{わが|み}の
\ruby[g]{往時}{むかし}につまされて、じつと
\ruby{其冊子}{その|ほ|ん}に
\ruby{留}{とど}めし
\ruby{眼}{まなこ}を、
\ruby{今}{いま}しも
\ruby{其人}{その|ひと}の
\ruby{後姿}{すが|た}に
\ruby{移}{うつ}して
\ruby{横顏}{よこ|がほ}をそつと
\ruby{見}{み}やる
\ruby{折}{をり}しも、ふつと
\ruby{男}{をとこ}は
\ruby[g]{此方}{こなた}を
\ruby{見{\GWI{u8fd4-k}}}{み|かへ}し、
\ruby{圖}{はか}らず
\ruby{眼}{め}と
\ruby{眼}{め}と
\ruby{相射}{あひ|い}しが、はやくもお
\ruby{龍}{りう}は
\ruby{男}{をとこ}の
\ruby{睫毛}{まつ|げ}に
\ruby{怪}{あや}しき
\ruby{露}{つゆ}の
\ruby{珠}{たま}あるを
\ruby{見}{み}たり。

