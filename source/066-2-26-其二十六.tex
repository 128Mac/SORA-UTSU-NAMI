\Entry{其二十六}

% メモ 校正終了 2024-04-24 2024-06-01
\原本頁{139-1}%
\ruby{思}{おも}ひの
ほかの
\ruby{品}{もの}
なりしに、
%
お
\ruby{龍}{りう}は
\ruby{驚}{おどろ}き
\ruby{疑}{うたが}ひて、
%
\ruby{露}{つゆ}
\ruby{照}{て}る
\ruby{美}{うつく}しき
\ruby[<j||]{眼}{まなこ}を% 行末行頭の境界付近なので特例処置を施す
\ruby{{\換字{睜}}}{みは}り、
%
あらためて
\ruby{男}{をとこ}を
\ruby{一}{ひ}ト
\ruby{目}{め}
\ruby{見}{み}しが、
%
\ruby{男}{をとこ}は
それとも
\ruby{心付}{こゝろ|づ}かず% 踊り字調整「〻(二の字点、揺すり点)に見えるが(ゝ)」
\ruby{{\換字{猶}}}{なほ}
\ruby{車外}{そ|と}を
\ruby{見}{み}
\ruby{居}{ゐ}たり。

\原本頁{139-4}%
\ruby{既}{すで}に
\ruby{其}{その}
\ruby{人}{ひと}の
\ruby{履物}{はき|もの}の
\ruby{汚}{けが}れを
\ruby{淸}{きよ}めたり、
%
\ruby{落}{お}ち
\ruby{散}{ち}つたる
\ruby{紙}{かみ}の
\ruby{眼}{め}に
\ruby{厭}{いと}は
しきをも
\ruby{一}{ひ}ト
\ruby{纏}{まとめ}にして
\ruby{投}{な}げ
\ruby{棄}{す}てたり、
%
\ruby{謝罪}{あや|ま}る
ほどは
あやまりて
\改行% 校正作業の簡略化のため
、
%
\原本頁{139-6}\改行%
\ruby{今}{いま}は
\ruby{何}{なに}
\ruby{爲}{す}べき
\ruby{事}{こと}も
\ruby{無}{な}きなり。
%
お
\ruby{龍}{りう}は
\ruby{彼}{か}の
\ruby{男}{をとこ}とは
\ruby{斜線}{すぢ|かひ}に、
%
\ruby{其}{そ}の
\ruby{反對}{はん|たい}の
\ruby{側}{がは}の
\ruby{車窓}{ま|ど}
\ruby{{\換字{近}}}{ちか}き
\ruby{席}{せき}を
\ruby{取}{と}りて、
%
はじめて
\ruby{身}{み}をも
\ruby{心}{こゝろ}% 踊り字調整「〻(二の字点、揺すり点)に見えるが(ゝ)」
をもおちつけたり。

\原本頁{139-9}%
\ruby{普門品}{ふ|もん|ぼん}!、
%
あの
\ruby{普門品}{ふ|もん|ぼん}!、
%
\ruby{彼書}{あ|れ}は
たしか
\ruby[||j>]{觀}{くわん}% 「觀音」の読みは原本通り「くわん(の)ん」
\ruby[||j>]{音}{ のん}
\ruby[||j>]{樣}{ さま}を
\ruby{信}{しん}ずる
\ruby{人}{ひと}の
\ruby{讀}{よ}む
\ruby{御經}{お|きやう}!。
%
\ruby{五十六十}{ご|じふ|ろく|じふ}の
\ruby{爺}{ぢゝ}% 踊り字調整「〻(二の字点、揺すり点)に見えるが(ゝ)」
\ruby{婆}{ばゝ}% 踊り字調整「〻(二の字点、揺すり点)に見えるが(ゝ)」
ならばいざ
\ruby{知}{し}らず、
%
\ruby{{\換字{若}}}{わか}い
\ruby{盛}{さか}りの
\ruby{當世}{い|ま}の
\ruby{人}{ひと}の、
%
しかも
\ruby{{\換字{古}}風}{むか|し}を
\ruby{守}{まも}る
\ruby[||-|]{農}{ひやく}
\ruby[||-|]{夫}{しやう}
% \ruby{農夫}{ひやく|しやう}
\ruby[||>]{町}{ちやう}
\ruby[||j>]{人}{ にん}
% \ruby{町人}{ちやう|にん}
でゞもある% TODO 原本の「二の字点、揺すり点」に濁点のグリフが見つからないので「ゞ」
\ruby{事}{こと}か、
%
\ruby{新}{あたら}しきを
\ruby{{\換字{追}}}{お}うて
\ruby{學問}{がく|もん}に
\ruby{身}{み}を
\ruby{責}{せ}めれば、
%
まづ
\ruby[||j>]{神}{かみ}
\ruby[||j>]{佛}{ほとけ}とは
% \ruby{神佛}{かみ|ほとけ}とは
\ruby{緣}{{\換字{𛀁}}ん}の
\ruby{{\換字{遠}}}{とほ}さうな
\ruby{書生}{しよ|せい}
\原本頁{140-2}\改行%
\ruby{風}{ふう}の
\ruby{此樣}{こ|う}いふ
\ruby{人}{ひと}の
\ruby{懷中}{ふと|ころ}から、
%
\ruby{普門品}{ふ|もん|ぼん}とは
\ruby{似合}{に|あ}はしからぬ!。
%
\ruby{何}{ど}のやうな
\ruby{悲}{かな}しい
\ruby{願}{ねがひ}が
あつての
\ruby[||j>]{佛}{ほとけ}
\ruby[||j>]{頼}{ だの}みか
\ruby{知}{し}らねど、
%
あゝ% 踊り字調整「〻(二の字点、揺すり点)に見えるが(ゝ)」
\ruby{想}{おも}ひ
\ruby{出}{だ}しても
\ruby{胸}{むね}が
\ruby{痛}{いた}む、
%
\ruby{妾}{わたし}も
\ruby{一昨年}{をと|ゝ|し}の% 踊り字調整「〻(二の字点、揺すり点)に見えるが(ゝ)」
\ruby{丁度}{ちやう|ど}
\ruby{今頃}{いま|ごろ}、
%
\ruby{思}{おも}ふ
\ruby{人}{ひと}には
\ruby{{\換字{遠}}}{とほ}く
\ruby{離}{はな}れて
\改行% 校正作業の簡略化のため
、
%
\原本頁{140-5}\改行%
\ruby{{\換字{空}}}{そら}の
\ruby{色}{いろ}も
\ruby{風}{かぜ}の
\ruby{音}{おと}も
\ruby[||j>]{{\換字{情}}}{なさけ}
\ruby[||j>]{無}{ な}い、
\ruby{知}{し}らぬ
\ruby{他國}{た|こく}の
\ruby{駿府}{すん|ぷ}の
\ruby{秋}{あき}、
%
いくら
\ruby{手紙}{て|がみ}を
\ruby{出}{だ}して
\ruby{問}{とひ}
\ruby{訊}{たつ}ねしても、
%
\ruby{{\換字{返}}事}{へん|じ}
さへ
\ruby{來}{こ}ないのが
\ruby{氣}{き}になつて
\ruby{氣}{き}になつて、
%
よもやとは
\ruby{思}{おも}へども
\ruby[||j>]{心}{こゝろ}% 踊り字調整「〻(二の字点、揺すり点)に見えるが(ゝ)」
\ruby[||j>]{變}{ がは}りか、
%
それとも
また
\ruby[<j||]{病}{やみ }
\ruby[<j>]{患}{わづらひ}% TODO CHECK
でも
\ruby{仕}{し}てゞは% TODO 原本の「二の字点、揺すり点」に濁点のグリフが見つからないので「ゞ」
\ruby{無}{な}いかと、
%
\ruby{恨}{うら}めしく
もあれば
\ruby[||j>]{心}{こゝろ}% 踊り字調整「〻(二の字点、揺すり点)に見えるが(ゝ)」
\ruby[||j>]{細}{ ぼそ}く
もあり、
%
はては
\ruby{茫然}{ぼん|やり}と
\ruby{門口}{かど|ぐち}に
\ruby{立}{た}つて、
%
\ruby{何}{なに}が
\ruby{見}{み}える
でもない
\ruby[||j>]{東}{とう}
\ruby[||j>]{京}{きやう}の
% \ruby{東京}{とう|きやう}の
\ruby{方}{ほう}を、
%
\ruby{{\換字{空}}}{くう}に
\ruby{見}{み}
\ruby{詰}{つ}めては
ほろり〳〵と、
%
\ruby{馬鹿}{ば|か}らしい
ほど
\ruby{泣}{な}いて
\ruby{泣}{な}いた
\ruby{末}{すゑ}、
%
\ruby{思案}{し|あん}に
\ruby{餘}{あま}つた
ところから
\ruby[||j>]{願}{がわん}
\ruby[||j>]{掛}{ がけ}して、
% \ruby{願掛}{がわん|がけ}して、
%
% * 安東熊野神社 静岡市葵区安東1-6-4
% 駿河七観音・安倍七観音
%   霊山寺 静岡市清水区大内山597
%   徳願寺 静岡市駿河区向敷地689
%   増善寺 静岡市葵区慈悲尾(しいのお)302
%   建穂寺 静岡市葵区建穂2-12-6 山号を「瑞祥山」
% * 法明寺 静岡市葵区足久保奥組1043
%   鉄舟寺 静岡市清水区村松2188
%   平澤寺 静岡市駿河区平沢55
\ruby{安東}{あん|とう}の% ルビ調整(原本通り)「(ど)でなく(と)」
\ruby{淸水}{きよ|みづ}の
\ruby[||j>]{觀}{くわん}% 「觀音」の読みは原本通り「くわん(の)ん」
\ruby[||j>]{音}{ のん}
\ruby[||j>]{樣}{ さま}には
\ruby{御經}{お|きやう}こそ
\ruby{誦}{あ}げなかつたが
\ruby{日}{ひ}
\ruby{參}{まゐ}りもすれば、
%
\ruby{足久保}{あし|く|ぼ}の
\ruby{楠木}{くす|のき}の
\ruby[||j>]{觀}{くわん}% 「觀音」の読みは原本通り「くわん(の)ん」
\ruby[||j>]{音}{ のん}
\ruby[||j>]{樣}{ さま}の
\ruby{御}{ご}
\ruby{利生}{り|しやう}
\原本頁{141-2}\改行%
の
\ruby{話}{はなし}を
\ruby{聞}{き}いては、
%
\ruby{二里}{に|り}からの
\ruby{田舎}{ゐな|か}
\ruby{{\換字{道}}}{みち}を
\ruby{歩}{ある}いた
\ruby{上}{うへ}に、
%
\ruby{草臥}{くた|びれ}
\ruby{{\換字{返}}}{かへ}り
ながら
%
\ruby{御}{お}
\ruby{百度}{ひやく|ど}まで
\ruby{踏}{ふ}んで、
%
\ruby{何卒}{どう|ぞ}
\ruby{手紙}{て|がみ}の
\ruby{{\換字{返}}事}{へん|じ}の
\ruby{參}{まゐ}りまして
\ruby{彼方}{あち|ら}の
\ruby{樣子}{やう|す}の
\ruby{{\換字{分}}}{わか}りまする
やう、
%
\ruby{{\換字{若}}}{も}し
\ruby{{\換字{又}}}{また}
\ruby{病氣}{びやう|き}
\ruby{災{\換字{難}}}{さい|なん}に
でも
\ruby{罹}{かゝ}つて% 踊り字調整「〻(二の字点、揺すり点)に見えるが(ゝ)」
\ruby{居}{を}りまするなら、
%
\ruby{御利益}{ご|り|やく}を
もつて
\ruby{助}{たす}かりまする
ようにと、
%
\ruby{自{\換字{分}}}{じ|ぶん}の
\ruby{身體}{から|だ}は
\ruby{一日}{いち|にち}
\ruby{一日}{いち|にち}
\ruby{{\換字{削}}}{けづ}るやうに
\ruby{癯}{や}せるのも
\ruby{餘{\換字{所}}}{よ|そ}にして、
%
\ruby{一心}{いつ|しん}に
なつて
\ruby{信心}{しん|〴〵}を
\原本頁{141-7}\改行%
\ruby{仕}{し}た
\ruby{苦}{くる}しい
\ruby{切}{せつ}ない
\ruby{經驗}{おぼ|{\換字{𛀁}}}も
あるが、
%
\ruby{忘}{わす}れても
\ruby{爲}{す}まい
ものは
\ruby{戀路}{こひ|ぢ}の
\ruby{{\換字{迷}}}{まよ}ひ、
%
\ruby{思}{おも}つて
\ruby{思}{おも}ひ
\ruby{止}{や}む
\ruby{日}{ひ}も
\ruby{無}{な}ければ、
%
\ruby{泣}{な}いて
\ruby{泣}{な}き
\ruby{足}{た}る
\ruby{夜}{よる}も
\ruby{無}{な}く
\改行% 校正作業の簡略化のため
、
%
\原本頁{141-9}\改行%
\ruby{生}{い}きては
\ruby{居}{ゐ}ても
\ruby{生}{い}きたくも
\ruby{無}{な}く、
%
\ruby{死}{し}なうとしても
\ruby{死}{し}にきれもせぬ
\ruby{彼}{あ}の
\ruby{厭}{いや}な〳〵
\ruby[||j>]{{\換字{情}}}{なさけ}
\ruby[||j>]{無}{ な}い
\ruby[||j>]{心}{こゝろ}% 踊り字調整「〻(二の字点、揺すり点)に見えるが(ゝ)」
\ruby[||j>]{持}{ もち}!。% 「!」は国会図書館ので確認
%
\ruby{我}{わが}
\ruby{身}{み}の
\ruby{痛}{いた}かりし
\ruby{經驗}{おぼ|{\換字{𛀁}}}に
\ruby{人}{ひと}の
\ruby{痛}{いた}さも
\ruby{思}{おも}はるゝが、% 踊り字調整「〻(二の字点、揺すり点)に見えるが(ゝ)」
%
あゝ% 踊り字調整「〻(二の字点、揺すり点)に見えるが(ゝ)」
\ruby{{\換字{猶}}}{まだ}
\ruby{{\換字{若}}}{わか}い
\ruby{此}{こ}の
\ruby{人}{ひと}の
\ruby{信心}{しん|〴〵}の、
%
よしや
\ruby{頼}{たの}み
\ruby{無}{な}き
\ruby{老年}{とし|より}の
\ruby{親}{おや}の
\ruby{病氣}{びやう|き}の
\ruby{爲}{ため}
\ruby{故}{ゆゑ}でも
あれ、
%
また
\ruby{何}{ど}の
\ruby{樣}{やう}な
\ruby{辛}{つら}い
\ruby{悲}{かな}しい
\ruby{{\換字{遺}}}{や}る
\ruby{瀬}{せ}
\ruby{無}{な}い
\ruby{事}{こと}の
ためでもあれ、
%
たゞ% TODO 原本の「二の字点、揺すり点」に濁点のグリフが見つからないので「ゞ」
\ruby{戀}{こひ}
\ruby{故}{ゆゑ}の
\ruby{信心}{しん|〴〵}で
\ruby{無}{な}かれかし。
%
\ruby{今}{いま}
\ruby{妾}{わたし}が
\原本頁{142-3}\改行%
\ruby{仕}{し}たる
\ruby{{\換字{過}}失}{あや|まち}は、
%
\ruby{時}{とき}の
\ruby{拍子}{へう|し}の
\ruby{事}{こと}
なれば、
%
\ruby{誰}{たれ}も
\ruby{容赦}{ゆ|る}しては
\ruby{吳}{く}れさうな
\ruby{譯}{わけ}ながら、
%
あれほどの
\ruby{血}{ち}の
\ruby{出}{で}た
\ruby{負傷}{け|が}を
\ruby{仕}{し}て、
%
\ruby{露}{つゆ}
\ruby{腹立}{はら|だ}たしげな
\ruby{顏色}{かほ|つき}もせず、
%
また
\ruby{恨}{うら}めしき
\ruby{眼色}{め|つき}もせず、
%
\ruby{毫}{すこし}も
\ruby{變}{かは}つた
\ruby{樣子}{やう|す}は
\ruby{無}{な}くて、
%
\ruby{水}{みづ}の
\ruby{流}{なが}れた
やうに
さらりと
\ruby{濟}{す}ませて、
%
\ruby{後}{あと}には
\ruby{物}{もの}も
\ruby{殘}{のこ}さぬ
\ruby{風{\換字{情}}}{ふ|ぜい}の
\ruby{寛大}{おほ|やう}さ!。
%
\ruby{{\換字{終}}}{しまひ}には
\ruby{反對}{あべ|こべ}に
\ruby{禮}{れい}まで
\ruby{言}{い}ひたるに
\ruby{心}{こゝろ}の% 踊り字調整「〻(二の字点、揺すり点)に見えるが(ゝ)」
\ruby{優}{やさ}しさは
\ruby{見}{み}えながら、
%
それから
\ruby{知}{し}らぬ
\ruby{顏}{かほ}
つくつて、
%
\ruby{彼方}{あち|ら}
\ruby{向}{む}いたる
\ruby{振舞}{ふる|まひ}の
\ruby{少}{すこ}し
\ruby{素氣}{す|げ}
\ruby{無}{な}きに、
%
\ruby{飼}{か}はれても
\ruby{人}{ひと}の
\ruby{氣}{き}は
\ruby{取}{と}らぬ
\ruby{鷹}{たか}の
\ruby{素振}{そ|ぶり}の、
%
\ruby{一寸}{ちよ|つと}
\ruby{憎}{にく}らしいほどな
\ruby{氣位}{き|ぐらゐ}も
あらはれて、
%
\ruby{女}{をんな}
さへ
\ruby{見}{み}れば
\ruby{{\換字{嫌}}}{いや}に
\ruby{笑}{わら}ひ
\ruby{掛}{か}ける
\改行% 校正作業の簡略化のため
、
%
\原本頁{142-11}\改行%
\ruby{世}{よ}に
\ruby{有}{あ}りふれた
\ruby{{\換字{若}}}{わか}い
\ruby{人}{ひと}などゝは、% 踊り字調整「〻(二の字点、揺すり点)に見えるが(ゝ)」
%
\ruby{其}{そ}の
\ruby{行方}{ゆき|がた}も
\ruby{全}{まる}で
\ruby{異}{かは}れど、
%
されば
といつて
ぎしつきも
せず、
%
\ruby{氣立}{き|だて}も
\ruby[||j>]{心}{こゝろ}% 踊り字調整「〻(二の字点、揺すり点)に見えるが(ゝ)」
\ruby[||j>]{持}{ もち}も
\ruby{何}{なに}と
\ruby{無}{な}く
\ruby{{\換字{違}}}{ちが}つて、
%
\ruby{衣服}{な|り}
\ruby{容姿}{すが|た}は
\ruby{此}{これ}
といふことも
\ruby{無}{な}き
\ruby{書生}{しよ|せい}ながら、
%
おのづと
\ruby{普{\換字{通}}}{な|み}には
\ruby{思}{おも}へぬ
ところ
ある
\ruby{人}{ひと}!。
%
\ruby{斯樣}{か|う}いふ
\ruby{調子}{てう|し}あひの
\ruby{人}{ひと}なんぞが、
%
\ruby{{\換字{若}}}{も}し
\ruby{萬一}{ひよ|つと}
\ruby{十年}{じふ|ねん}
\ruby{二十年}{に|じふ|ねん}の
\ruby{後}{のち}になつて、
%
\ruby{立派}{りつ|ぱ}な
\ruby{傑}{すぐ}れた
\ruby{人}{ひと}
なんぞに
なるのでは
あるまいか?。
%
あゝ% 踊り字調整「〻(二の字点、揺すり点)に見えるが(ゝ)」
\ruby[||j>]{修}{しゆ}
\ruby[||j>]{行}{ぎやう}
\ruby[||j>]{盛}{ ざか}り
\ruby{出世盛}{しゆつ|せ|ざか}りの
\ruby{此}{こ}の
\ruby{{\換字{若}}}{わか}い
\ruby{人}{ひと}!、
%
それに
\ruby{付}{つ}けても
\ruby{彼}{あ}の
\ruby{普門品}{ふ|もん|ぼん}!。
%
\ruby{屹度}{きつ|と}
\ruby{果敢}{は|か}
\ruby{無}{な}い
\ruby{戀}{こひ}なぞの、
%
\ruby{其樣}{そ|ん}な
\ruby{事}{こと}の
\原本頁{143-7}\改行%
ためでは
あるまい
なれど、
%
どうか
\ruby{戀}{こひ}ゆえの
\ruby{信心}{しん|〴〵}で
\ruby{無}{な}かれかし!
\改行% 校正作業の簡略化のため
。
%
\原本頁{143-8}\改行%
と
\ruby{我身}{わが|み}の
\ruby{往時}{むか|し}に
つまされて、
%
じつと
\ruby{其}{その}
\ruby{册子}{ほ|ん}に
\ruby{{\換字{留}}}{とゞ}めし% TODO 原本の「二の字点、揺すり点」に濁点のグリフが見つからないので「ゞ」
\ruby{眼}{まなこ}を、
%
\ruby{今}{いま}しも
\ruby{其}{その}
\ruby{人}{ひと}の
\ruby{後姿}{すが|た}に
\ruby{移}{うつ}して
\ruby{横顏}{よこ|がほ}を
そつと
\ruby{見}{み}やる
\ruby{折}{をり}しも、
%
ふつと
\ruby{男}{をとこ}は
\ruby{此方}{こな|た}を% ルビ調整(原本通り)
\ruby{見}{み}
\ruby{{\換字{返}}}{かへ}し、
%
\ruby{圖}{はか}らず
\ruby{眼}{め}と
\ruby{眼}{め}と
\ruby{相}{あひ}
\ruby{射}{い}しが、
%
はやくも
お
\ruby{龍}{りう}は
\ruby{男}{をとこ}の
\ruby{睫毛}{まつ|げ}に
\ruby{怪}{あや}しき% TODO CHECK 怪
\ruby{露}{つゆ}の
\ruby{珠}{たま}
あるを
\ruby{見}{み}たり。
