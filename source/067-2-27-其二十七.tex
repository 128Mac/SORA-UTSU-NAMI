\Entry{其二十七}

\原本頁{}%
\ruby{人}{ひと}おの〳〵
\ruby{身}{み}あり、
%
\ruby{身}{み}の
\ruby{居}{を}るところあり。
%
\ruby{{\換字{又}}}{また}おの〳〵
\ruby{心}{こゝろ}あり、
%
\ruby{心}{こゝろ}の
\ruby{思}{おも}ふところあり。
%
されば
\ruby{相}{あひ}
\ruby{知}{し}らぬ
お
\ruby{龍}{りう}と
\ruby{男}{をとこ}との、
%
\ruby{男}{をとこ}は
お
\ruby{龍}{りう}の
\ruby{我}{わ}が
\ruby{爲}{ため}に
\ruby{何}{なに}を
\ruby{思}{おも}へるぞとも
\ruby{知}{し}らねば、
%
お
\ruby{龍}{りう}はまた
\ruby{男}{をとこ}の
\ruby{我}{われ}ゆゑに
\ruby{何}{なに}を
\ruby{悲}{かな}しめるぞとも
\ruby{悟}{さと}らむやう
\ruby{無}{な}きなり。

\原本頁{}%
\ruby{自}{みづか}ら
\ruby{感}{かん}ずる
\ruby{身心}{しん|〳〵}の
\ruby{疲}{つか}れを、
%
せめては
\ruby{滊車}{き|しや}の
\ruby{内}{うち}に
\ruby{休}{やす}めんと、
%
\ruby{少時}{しば|し}を
\ruby{待合}{まち|あ}はせて
\ruby{此}{これ}に
\ruby{乘}{の}りたるに、
%
\ruby{何}{なに}となく
\ruby{氣}{き}の
\ruby{弛}{ゆる}みてうつかりとしたる
\ruby{時}{とき}、
%
\ruby{忽}{たちま}ち
\ruby{足}{あし}を
\ruby{踏}{ふ}まれて
\ruby{驚}{おどろ}きしが、
%
\ruby{眞心}{ま|ごゝろ}を
\ruby{表}{あらは}して
\ruby{謝罪}{わ|び}らるゝに
\ruby{怒}{いか}らんやうは
\ruby{無}{な}く、
%
かゝる
\ruby{事}{こと}は
\ruby{有}{あ}り
\ruby{{\換字{勝}}}{がち}の
\ruby{{\換字{過}}失}{あや|まり}にて
\ruby{珍}{めづ}らしくもあらず、
%
\ruby{且}{かつ}は
\ruby{{\換字{又}}}{また}、
%
\ruby{云}{い}はゞ
\ruby{我}{われ}にも
\ruby{不注意}{ふ|ちう|い}の% 原本通り「ゆ」無しで「ちうい」
\ruby{咎}{とが}の
\ruby{無}{な}きにはあらぬをやと
\ruby{輕}{かる}く
\ruby{思}{おも}ひ
\ruby{棄}{す}てつ、
%
\ruby{却}{かへ}つて
\ruby{他}{ひと}の
\ruby{我}{わ}がためにまめ〳〵しく
\ruby{傷}{きず}を
\ruby{裹}{つゝ}み
\ruby{汚}{けがれ}を
\ruby{拭}{ぬぐ}ひて
\ruby{吳}{く}るゝ
\ruby{氣}{き}の
\ruby{毒}{どく}さに
\ruby{堪}{た}へかねて、
%
よきほどに
\ruby{挨拶}{あい|さつ}して
\ruby{身}{み}を
\ruby{{\換字{退}}}{ひ}きたる
\ruby{男}{をとこ}は、
%
\ruby{何}{なに}を
\ruby{見}{み}るにもあらず
\ruby{窓外}{さう|ぐわい}を% ルビは原本通り(「さ」うぐあい)
\ruby{見}{み}ながら、
%
\ruby{出血}{しゆつ|けつ}を
\ruby{疾}{と}く
\ruby{止}{と}まらしめんがために
\ruby{膝}{ひざ}に
\ruby{載}{の}せたる
\ruby{片足}{かた|あし}の、
%
\ruby{{\換字{猶}}}{なほ}
\ruby[||h>]{聊}{いさゝ}か
\ruby{疼痛}{いた|み}をば
\ruby{覺}{おぼ}ゆるにつけて、
%
\ruby{嗚呼}{あ|ゝ}
\ruby{何}{なん}ぞ
\ruby{人}{ひと}の
\ruby{世}{よ}のことの
\ruby{如是}{か|く}
\ruby{愚}{おろか}しきや!。
%
たま〳〵
\ruby{我}{われ}に
\ruby{{\換字{過}}失}{あや|まち}したる
\ruby{此}{こ}の
\ruby{{\換字{若}}}{わか}き
\ruby{{\換字{婦}}人}{ふ|じん}は、
%
\ruby{我}{わ}が
\ruby{思}{おも}ふ
\ruby{人}{ひと}の
\ruby{如}{ごと}くなる
\ruby{端嚴}{たん|ごん}の
\ruby{相}{さう}こそ
\ruby{無}{な}けれ、
%
\ruby{婀娜}{あ|だ}たる
\ruby{姿}{すがた}、
%
\ruby{野}{の}の
\ruby{花}{はな}のおのづから
\ruby{人}{ひと}の
\ruby{意}{こゝろ}を
\ruby{惹}{ひ}く
\ruby{色香}{いろ|か}あつて、
%
\ruby{一車}{いつ|しや}の
\ruby[h|]{客}{きやく}
\ruby{皆}{みな}
\ruby{眼}{め}をそばだてゝ
\ruby{見}{み}たるほどなれば、
%
\ruby{彼}{か}の
\ruby{人}{ひと}の
\ruby{我}{われ}に
\ruby{思}{おも}はるゝが
\ruby{如}{ごと}くに
\ruby{此}{こ}の
\ruby{女}{ひと}もまた
\ruby{或}{あるひ}は
\ruby{他}{ひと}に
\ruby{思}{おも}はるゝ
\ruby{事}{こと}の
\ruby{無}{な}きには
\ruby{限}{かぎ}らじ。
%
\ruby{我}{わ}が
\ruby{身}{み}の
\ruby{經驗}{おぼ|{\換字{𛀁}}}に
\ruby{我}{われ}よくぞ
\ruby{知}{し}る、
%
\ruby{人}{ひと}を
\ruby{思}{おも}ふものゝ
\ruby{苦}{くる}しさは、
%
\ruby{魂魄}{たま|しひ}を
\ruby{絞木}{しめ|ぎ}にかけられて
\ruby{斷}{た}えず
\ruby{壓}{お}し
\ruby{搾}{しぼ}らるゝに
\ruby{異}{こと}ならず、
%
\ruby{何}{なん}につけ
\ruby{彼}{か}につけ、
%
\ruby{夢}{ゆめ}につけ
\ruby{現}{うつゝ}につけ、
%
\ruby{愁}{うれ}ひ
\ruby{易}{やす}く
\ruby{悲}{かなし}み
\ruby{易}{やす}くなりたる
\ruby{心}{こゝろ}の、
%
\ruby{事}{こと}あるごとに
\ruby{責}{せ}め
\ruby{搾}{しぼ}らるれば、
%
\ruby{誰}{た}が
\ruby{縫}{ぬ}ひてか
\ruby{痊}{いや}すべき
\ruby{胸}{むね}の
\ruby{深創}{ふか|で}より、
%
\ruby{渾々}{こん|〳〵}として
\ruby{流}{なが}るゝ
\ruby{血潮}{ち|しほ}の、
%
\ruby{火}{ひ}とばかり
\ruby{熱}{あつ}きも
\ruby{{\換字{空}}}{むな}しく
\ruby{冷}{ひ}えて、
%
いたづらに
\ruby{地}{ち}に
\ruby{入}{い}つて
\ruby{{\換字{情}}無}{なさけ|な}く
\ruby{廢}{すた}ることの
\ruby{如何}{い|か}ばかりぞや。
%
\ruby{眼}{め}に
\ruby{見}{み}{\換字{𛀁}}えぬ
\ruby{其血}{そ|れ}こそは
\ruby{{\換字{尊}}}{たつと}くも
\ruby{{\換字{尊}}}{たつと}き
\ruby{人}{ひと}の
\ruby[g]{眞誠}{まこと}の
\ruby{生命}{いの|ち}にして、
%
\ruby{眼}{め}に
\ruby{見}{み}ゆる
\ruby{此血}{こ|れ}は
\ruby{言}{い}ふにも
\ruby{足}{た}らぬたゞ
\ruby{鹹}{しほはゆ}き
\ruby{紅}{あか}き
\ruby{水}{みづ}なるを、
%
\ruby{僅}{わづか}に
\ruby{一指}{いつ|し}の
\ruby{端}{はし}を
\ruby{傷}{きず}つけ
\ruby{數滴}{すう|てき}の
\ruby{臙脂}{べ|に}を
\ruby{散}{ち}らせば、
%
\ruby{性{\換字{情}}}{こゝ|ろ}の
\ruby{優}{やさ}しさ
\ruby{見}{み}ゆる
\ruby{此}{こ}の
\ruby{{\換字{婦}}人}{ふ|じん}は、
%
\ruby{僞}{いつは}りならず
\ruby{我}{われ}をいたはしがりて、
%
\ruby{心}{こゝろ}を
\ruby{盡}{つく}し
\ruby{手}{て}を
\ruby{盡}{つく}し
\ruby{慰}{なぐさ}め
\ruby{吳}{く}れしが、
%
\ruby{{\換字{若}}}{も}し
\ruby{此}{こ}の
\ruby{女}{ひと}を
\ruby{思}{おも}ふ
\ruby{男}{をとこ}ありて、
%
\ruby{彼}{か}の
\ruby{人}{ひと}を
\ruby{思}{おも}ふ
\ruby{我}{わ}が
\ruby{如}{ごと}くに、
%
\ruby{果敢無}{は|か|な}き
\ruby{戀}{こひ}の
\ruby{深}{ふか}みに
\ruby{惱}{なや}み、
%
\ruby{日}{ひ}と
\ruby{無}{な}く
\ruby{夜}{よ}と
\ruby{無}{な}く、
%
\ruby{折}{をり}にふれ
\ruby{事}{こと}につけ、
%
\ruby{泉}{いづみ}の
\ruby{如}{ごと}くに
\ruby{止}{と}まらぬ
\ruby{血}{ち}を
\ruby{心窩}{む|ね}の
\ruby{奧底}{おく|そこ}より
\ruby{流}{なが}し
\ruby{溢}{あふ}らさば、
%
それを
\ruby{此}{こ}の
\ruby{女}{ひと}は
\ruby{何}{なに}とか
\ruby{見}{み}るべき?。
%
\ruby{我}{わ}が
\ruby{足}{あし}の
\ruby{指}{ゆび}の
\ruby{一}{ひ}ト
\ruby{{\換字{節}}}{ふし}
\ruby{二}{ふ}タ
\ruby{{\換字{節}}}{ふし}、
%
\ruby{此}{こ}の
\ruby{紅}{あか}き
\ruby{水}{みづ}の
\ruby{幾掬}{いく|むすび}は、
%
よしや
\ruby{{\換字{過}}失}{あや|まち}のために
\ruby{亡失}{うし|な}はれたりとて、
%
\ruby{我}{われ}
\ruby{斯}{かく}ばかり
\ruby{恤}{いたは}られでも
\ruby{可}{よし}、
%
たゞ
\ruby{思}{おも}ふ
\ruby{人}{ひと}に
\ruby{思}{おも}はれぬ
\ruby{辛}{つら}さは
\ruby{身}{み}に
\ruby{徹}{し}みて
\ruby{悲}{かな}しくおぼゆれば、
%
あはれ
\ruby{此}{こ}の
\ruby{優}{やさ}しげなる
\ruby{{\換字{若}}}{わか}き
\ruby{人}{ひと}の、
%
\ruby{{\換字{若}}}{も}し
\ruby{人}{ひと}に
\ruby{思}{おも}はれなば
\ruby{人}{ひと}を
\ruby{思}{おも}へかし、
%
\ruby{思}{おも}ふ
\ruby{人}{ひと}あらば
\ruby{其人}{その|ひと}を
\ruby{思}{おも}ひて
\ruby{{\換字{遣}}}{や}れよかし、
%
\ruby{戀}{こひ}の
\ruby{誠}{まこと}に
\ruby{責}{せ}められて、
%
\ruby{壽命}{いの|ち}を
\ruby{溶}{と}いて
\ruby{涙}{なみだ}と
\ruby{流}{なが}し
\ruby{棄}{す}て、
%
\ruby{男兒}{をと|こ}の
\ruby{智慧}{ち|ゑ}をも
\ruby{保}{たも}ちかねて、
%
\ruby{愚}{ぐ}に
\ruby{甘}{あま}んずるに
\ruby{至}{いた}れる
\ruby{此}{こ}の
\ruby{我}{わ}が
\ruby{如}{ごと}きものにも
\ruby{會}{あ}はゞ、
%
\ruby{假令其}{たと|ひ|そ}の
\ruby{人}{ひと}を
\ruby{蟲}{むし}の
\ruby{{\換字{嫌}}}{きら}はゞとて、
%
せめては
\ruby{可憐}{あは|れ}とも
\ruby{思}{おも}ひて
\ruby{{\換字{遣}}}{や}れかし。
%
\ruby{我}{わ}が
\ruby{身}{み}につまされてつく〴〵と
\ruby{思}{おも}ふ、
%
あゝ
\ruby{人}{ひと}に
\ruby{思}{おも}はれなば
\ruby{人}{ひと}を
\ruby{思}{おも}へかし。
%
こればかりの
\ruby{傷}{きず}にだに
\ruby{痛}{いた}はしとおもふが、
%
\ruby{女性}{をん|な}の
\ruby{欺}{あざむ}かぬ
\ruby{{\換字{情}}}{なさけ}ならば、
%
\ruby{縫}{ぬ}ふべき
\ruby{針}{はり}も
\ruby{糸}{いと}も
\ruby{無}{な}き
\ruby{悲}{かな}しき
\ruby{創口}{きず|ぐち}より
\ruby{流}{なが}れ
\ruby{流}{なが}るゝ
\ruby{火}{ひ}と
\ruby{熱}{あつ}き
\ruby{血}{ち}の、
%
\ruby{花}{はな}と
\ruby{鮮}{あざ}やかなるを
\ruby{見}{み}たる
\ruby{時}{とき}は、
%
\ruby{必}{かな}らずあはれと
\ruby{思}{おも}ひて
\ruby{{\換字{遣}}}{や}れかし。
%
されど
\ruby{測}{はか}り
\ruby{難}{がた}き
\ruby{世}{よ}の
\ruby{{\換字{習}}}{ならひ}、
%
\ruby{此}{こ}の
\ruby{女}{ひと}もまた
\ruby{我}{わ}が
\ruby{思}{おも}ふ
\ruby{人}{ひと}の
\ruby{如}{ごと}く、
%
\ruby{或}{あるひ}はおのれを
\ruby{思}{おも}ふ
\ruby{男}{をとこ}の、
%
\ruby{心血}{しん|けつ}を
\ruby{盡}{つく}して
\ruby{悲}{かなし}み
\ruby{悶}{もだ}ゆるをも、
%
あだに
\ruby{天飛}{そら|と}ぶ
\ruby{雲}{くも}と
\ruby{見{\換字{過}}}{み|すご}して、
%
あはれみてもやらず
\ruby{悲}{かなし}みてもやらず、
%
\ruby{其}{そ}の
\ruby{{\換字{情}}無}{つれ|な}きを
\ruby{喞}{かこ}たれやする?。
%
\ruby{僅}{わづか}なる
\ruby{斯}{か}ばかりの
\ruby{傷}{きず}の
\ruby{如}{ごと}きには、
%
\ruby{心}{こゝろ}をつかひ
\ruby{言葉}{こと|ば}を
\ruby{費}{つひや}すにも
\ruby{當}{あた}らぬながら、
%
\ruby{此}{これ}を
\ruby{大事}{だい|じ}のやうに
\ruby{思}{おも}ふも
\ruby{愚}{おろか}なる
\ruby{世}{よ}の
\ruby{態}{すがた}かな。
%
\ruby{我}{われ}は
\ruby{今}{いま}
\ruby{恐}{おそ}ろしき
\ruby{傷}{きず}を
\ruby{抱}{いだ}きて、
%
\ruby{{\換字{絕}}間無}{た{\換字{𛀁}}|ま|な}く
\ruby{泉}{いづみ}なす
\ruby{血}{ち}を
\ruby{流}{なが}しながら、
%
あはれとも
\ruby{思}{おも}はれぬ
\ruby{悲}{かな}しき
\ruby{身}{み}なるをや。
%
こればかりの
\ruby{血}{ち}の
\ruby{嗚呼}{あ|ゝ}
\ruby{何}{なに}かあらん!。
%
それにつけても
\ruby{此}{こ}の
\ruby{{\換字{若}}}{わか}き
\ruby{女}{ひと}の、
%
\ruby{願}{ねが}はくは
\ruby{人}{ひと}に
\ruby{思}{おも}はれなば
\ruby{人}{ひと}を
\ruby{思}{おも}へかし、
%
と
\ruby{此}{これ}は
\ruby{我}{わ}が
\ruby{身}{み}の
\ruby{現在}{い|ま}につきて
\ruby{思}{おも}ひ
\ruby{入}{い}りながら、
%
\ruby{偶然}{ふ|と}
\ruby{後方}{うし|ろ}をば
\ruby{向}{む}きたるなりしが、
%
\ruby{圖}{はか}らず
\ruby{眼}{め}と
\ruby{眼}{め}と
\ruby{相}{あひ}
\ruby{射}{い}たる
\ruby{時}{とき}、
%
\ruby{男}{をとこ}もまた
お
\ruby{龍}{りう}が
\ruby{何}{なに}を
\ruby{思}{おも}ひてか
\ruby{涙}{なみだ}に
\ruby{其眼}{その|め}の
\ruby{潤}{うる}めるを
\ruby{觀}{み}たり。

\原本頁{}%
\ruby{相}{あひ}
\ruby{會}{あ}つたる
\ruby{眼}{め}は
\ruby{忽}{たちま}ち
\ruby{離}{はな}れぬ。
%
\ruby{男}{をとこ}はまた
\ruby{{\換字{前}}}{まへ}の
\ruby{如}{ごと}くに
\ruby{窓外}{そう|ぐわい}を% ルビは原本通り(「そ」うぐあい)
\ruby{眺}{なが}めたり。
%
\ruby{女}{をんな}は
\ruby{男}{をとこ}の
\ruby{如何}{い|か}なる
\ruby{人}{ひと}なるを
\ruby{知}{し}らず、
%
\ruby{男}{をとこ}もまた
\ruby{女}{をんな}の
\ruby{如何}{い|か}なるものなるを
\ruby{知}{し}らず、
%
\ruby{同}{おな}じ
\ruby{車}{くるま}に
\ruby{乘}{の}りて
\ruby{同}{おな}じ
\ruby{{\換字{道}}}{みち}を、
%
\ruby{同}{おな}じところへ
\ruby{行}{ゆ}く
\ruby{身}{み}ながらも、
%
\ruby{心}{こゝろ}はそれ〳〵のおもむく
\ruby{方}{かた}に
\ruby{馳}{は}するも
\ruby{{\換字{浮}}世}{うき|よ}の
\ruby{態}{すがた}なりや。

\原本頁{}%
やがて
\ruby{滊車}{き|しや}は
\ruby{白鬚}{しら|ひげ}の
\ruby[g]{停車塲}{ていしやぢやう}を% 原文通り「塲」
\ruby{{\換字{過}}}{す}ぎて、
%
はや
\ruby{鐘}{かね}が
\ruby{淵}{ふち}の
\ruby[g]{停車塲}{ていしやぢやう}% 原文通り「塲」
\ruby{{\換字{近}}}{ちか}くなりぬ。

\原本頁{}%
お
\ruby{龍}{りう}は
\ruby{徐}{しづか}に
\ruby{立上}{たち|あが}りて、

\原本頁{}%
『どうも
\ruby{飛}{と}んだ
\ruby{失禮}{しつ|れい}を
\ruby{致}{いた}しました。
%
もう
\ruby{妾}{わたくし}はこの
\ruby{先}{さき}で
\ruby{下車}{お|り}ますのでございますが、
%
\ruby{御痛}{お|いた}みは
\ruby{如何}{い|かゞ}でございます?、
%
\ruby{些}{ちつと}は
お
\ruby{宜}{よろ}しうございますか、
%
まことに
\ruby{濟}{す}みませんことを
\ruby{致}{いた}しました。
%
では
\ruby{{\換字{勝}}手}{かつ|て}ではございますがこれで
\ruby{失禮}{しつ|れい}いたします。
』

\原本頁{}%
と、
%
\ruby{男}{をとこ}も
\ruby{其處}{そ|こ}で
\ruby{下}{お}りるとは
\ruby{知}{し}らで
\ruby{物堅}{もの|がた}く
\ruby{挨拶}{あい|さつ}すれば、
%
\ruby{男}{をとこ}は
\ruby{口數}{くち|かず}
\ruby{少}{すくな}く、

\原本頁{}%
『いやもう
\ruby{何}{なん}ともございません、
%
\ruby{何樣}{ど|う}か
\ruby{御構}{お|かま}ひ
\ruby{無}{な}く。
』

\原本頁{}%
と
\ruby{云}{い}ひたるのみ。

\原本頁{}%
\ruby{滊車}{き|しや}は
\ruby{鐘}{かね}が
\ruby{淵}{ふち}に
\ruby{着}{つ}きて
\ruby[h|]{男}{をとこ}
\ruby{先}{ま}づ
\ruby{降}{お}り、
%
それより
\ruby{人々}{ひと|〴〵}につゞきて
\ruby{女}{をんな}も
\ruby{降}{お}りぬ。
%
\ruby{女}{をんな}の
\ruby[g]{停車塲}{ていしやぢやう}の% 原文通り「塲」
\ruby{構外}{こう|ぐわい}に
\ruby{出}{い}でし
\ruby{時}{とき}は、
%
\ruby{男}{をとこ}の
\ruby{姿}{すがた}ははや
\ruby{見}{み}えざりき。

