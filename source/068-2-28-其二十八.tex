\Entry{其二十八}

% メモ 校正終了 2024-04-24
\原本頁{151-2}%
\ruby{家並}{やな|み}
\ruby{立}{たち}
\ruby{續}{つゞ}ける% TODO 原本の「二の字点、揺すり点」に濁点のグリフが見つからないので「ゞ」
\ruby{都會}{みや|こ}に
\ruby{育}{そだ}ちて、
%
\ruby{賑}{にぎ}やかなる
\ruby{{\換字{道}}路}{み|ち}を
のみ
\ruby{歩}{ある}きつけたる
ものは、
%
\ruby{右}{みぎ}も
\ruby{左}{ひだり}も
\ruby{田甫}{たん|ぼ}
にして、
%
\ruby{{\換字{遠}}}{とほ}
\ruby{見}{み}に
\ruby{榎}{{\換字{𛀁}}のき}やら
\ruby{松}{まつ}やらの
\ruby{樹立}{こ|だち}、
%
\原本頁{151-4}\改行%
その
\ruby{蔭}{かげ}に
\ruby{箱庭}{はこ|には}に
ありさうな
\ruby{藁葺}{わら|ぶき}の
\ruby{家}{いへ}の
\ruby{四}{よ}ツ
\ruby{五}{いつ}ツ
\ruby{並}{なら}ぶ
といふやうなる
\ruby{田舎}{ゐな|か}へ
\ruby{踏出}{ふみ|だ}しては、
%
\ruby{十字路}{よつ|〻|ぢ}に% 本来は一の字点「ゝ」平仮名繰返し記号% 原本通り「〻(二の字点、揺すり点)」
\ruby{問}{と}ふべき
\ruby{店}{みせ}
なきを
\ruby{恨}{うら}み、
%
\ruby{三{\換字{叉}}路}{み|つ|また}に
\ruby{{\換字{尋}}}{たづ}ぬべき
\ruby{人}{ひと}
あらぬを
\ruby{悲}{かなし}み、
%
はては
\ruby{間{\換字{違}}}{ま|ちが}へずとも
\ruby{濟}{す}むべき
\ruby{筈}{はず}の
\ruby{路}{みち}を
\ruby{兎角}{と|かく}に
\ruby{間{\換字{違}}}{ま|ちが}へて、
%
あらぬ
ところに
\ruby{{\換字{迷}}}{まよ}ひ
\ruby{{\換字{込}}}{こ}むが
\ruby{常}{つね}なり。
%
\ruby{鐘が淵}{かね||ふち}の
\ruby[g]{停車塲}{ていしやぢやう}より% 原文通り「塲」
\ruby[g]{四ッ木}{よ ぎ}へは、% TODO 四ツ木
%
\ruby{何}{なん}の
\ruby{譯}{わけ}も
\ruby{無}{な}く
\ruby{知}{し}れ
\ruby{易}{やす}き
\ruby{路}{みち}なるを、
%
\原本頁{151-9}\改行%
お
\ruby{龍}{りう}は
\ruby{如何}{い|か}に
してか
\ruby{{\換字{誤}}}{あやま}りて、
%
\ruby{狐}{きつね}に
\ruby{誑}{ばか}さる〻と% 本来は一の字点「ゝ」平仮名繰返し記号% 原本通り「〻(二の字点、揺すり点)」
\ruby{云}{い}ひし
\ruby{今{\換字{朝}}}{け|さ}の
\ruby[<j||]{戱言}{じやう|だん}も
\ruby{思}{おも}ひ
\ruby{出}{だ}されて
をかしき
\ruby{無益}{む|だ}
\ruby{路}{みち}を
\ruby{歩}{ある}きし
\ruby{末}{すゑ}、
%
やうやくにして
\ruby{目}{め}ざす
\ruby{其}{そ}の
\ruby{村}{むら}へ
\ruby{着}{つ}きたり。

\原本頁{152-2}%
ばつちらけ
\ruby{髮}{がみ}を
\ruby{手拭}{て|ぬぐひ}の
\ruby{鉢卷}{はち|まき}に
\ruby{壓}{おさ}へて、
%
ねん〳〵
ねん〳〵と
\ruby{兒守}{こ|もり}する
\ruby{村}{むら}の
\ruby{娘}{こ}の
\ruby{十三四}{じう|さん|し}なるに。

\原本頁{152-4}%
『
もし、
%
\ruby{山路}{やま|ぢ}さん
といふのは、
』

\原本頁{152-5}%
と
\ruby{{\換字{尋}}}{たづ}ぬれば、

\原本頁{152-6}%
『
\ruby{{\換字{伴}}}{つ}れてつて
\ruby{{\換字{遣}}}{や}るべい。
』

\原本頁{152-7}%
と
\ruby{{\換字{前}}}{さき}に
\ruby{立}{た}つて
\ruby{歩}{ある}きて、

\原本頁{152-8}%
『
\ruby{此處}{こ|〻}だよ。% 原本通り「〻(二の字点、揺すり点)」
』

\原本頁{152-9}%
と
\ruby{敎}{をし}へて
\ruby{吳}{く}れたるは、
%
\ruby{門}{もん}の
\ruby{構}{かま}へも
がつしりと
\ruby{嚴}{いか}めしく
して、
%
\ruby{厚}{あつ}き
\ruby{茅葺}{かや|ぶき}の
\ruby{屋根}{や|ね}も
\ruby{高}{たか}き、
%
\ruby{物持}{もの|もち}らしき
\ruby{立派}{りつ|ぱ}の
\ruby{家}{いへ}なり。
%
これほどの
\ruby{家}{いへ}とは
\ruby{聞}{き}かざりしがと、
%
\ruby{少}{すこ}し
\ruby{訝}{いぶか}りながら
\ruby{音}{おと}なへば、
%
\ruby{丁度}{ちやう|ど}
\ruby{端{\換字{近}}}{はし|ぢか}に
\ruby{居}{ゐ}たる、
%
\ruby{般{\換字{若}}顏}{はん|にや|がほ}の
\ruby{{\換字{丈}}}{たけ}
\ruby{高}{たか}き
\ruby{女}{をんな}の、
%
\ruby{衣服}{な|り}は
\ruby{此家}{こ|〻}の% 原本通り「〻(二の字点、揺すり点)」
\ruby{主人}{ある|じ}の
\ruby{妻}{つま}なるべく
\ruby{見}{み}{\換字{𛀁}}て、
%
\ruby{可笑}{を|か}しき
ほど
\ruby{大}{おほき}なる
\ruby{丸髷}{まる|まげ}に
\ruby{結}{ゆ}びたるが、
%
\ruby{人}{ひと}を
\ruby{媢嫉}{そ|ね}む
やうなる
\ruby{眼}{め}つき
して、
%
しばらくは
\ruby{頭}{かしら}の
\ruby{上}{うへ}より
\ruby{足}{あし}の
\ruby{先}{さき}まで
じろ〳〵と
\ruby{見}{み}たる
\ruby{揚句}{あげ|く}、

\原本頁{153-5}%
『
それは
\ruby{隱居{\換字{所}}}{いん|きよ|じよ}の
\ruby{方}{はう}で
ございましやう、
%
こちらには
\ruby{水野}{みづ|の}
なんていふ
\ruby{人}{ひと}は
\ruby{居}{を}りません。
%
\ruby{其家}{そつ|ち}へ
\ruby{行}{い}つて
お
\ruby{聞}{き}きなさいまし。
』

\原本頁{153-7}%
と、
%
\ruby{可厭}{い|や}に
\ruby{慳貪}{けん|どん}に
\ruby{云}{い}ひ
\ruby{棄}{す}て〻、% 原本通り「〻(二の字点、揺すり点)」
%
\ruby{障子}{しやう|じ}
びつしやり
\ruby{奧}{おく}に
\ruby{入}{い}りたり、
%
\ruby{此家}{こ|〻}と% 原本通り「〻(二の字点、揺すり点)」
\ruby{隱居{\換字{所}}}{いん|きよ|じよ}との
\ruby{間}{あひだ}に
\ruby{何}{ど}のやうな
\ruby{譯}{わけ}の
あるかは
\ruby{知}{し}らず、
%
また
\ruby{何程}{どれ|ほど}
\ruby{大}{たい}した
\ruby{大々盡}{だい|〳〵|じん}の
\ruby{奧樣}{おく|さま}なれば
\ruby{左樣}{さ|う}は
\ruby{勿體}{もつ|たい}ぶるか
\ruby{知}{し}らねど、
%
\ruby{惡}{わる}く
\原本頁{153-10}\改行%
\ruby{人}{ひと}を
\ruby{見下}{み|くだ}した
やうな
\ruby{沒義{\換字{道}}}{も|ぎ|だう}の
\ruby{忌々}{いま|〳〵}しい
\ruby{大顏}{おほ|づら}な
\ruby{田舎}{ゐな|か}
\ruby{{\換字{婦}}}{もの}め
ときかぬ
\ruby{氣}{き}の
お
\ruby{龍}{りう}は
\ruby{打腹立}{うち|はら|だ}ちしが、
%
\ruby{怒}{いか}つて
\ruby{甲{\換字{斐}}}{か|ひ}ある
\ruby{事}{こと}
ならねば、
%
\ruby{其}{その}
\ruby{儘}{ま〻}% 原本通り「〻(二の字点、揺すり点)」
\ruby{突}{つ〻}と% 原本通り「〻(二の字点、揺すり点)」
\ruby{外}{そと}に
\ruby{出}{い}でたり。

\原本頁{154-1}%
\ruby{見}{み}れば
\ruby{{\換字{前}}}{さき}の
\ruby{兒}{こ}は
\ruby{{\換字{猶}}}{なほ}
\ruby{其處}{そ|こ}に
\ruby{居}{を}りて、
%
ねん〳〵
ねん〳〵と
\ruby{負}{お}へる
\ruby{子}{こ}を
\ruby{賺}{すか}しながら、
%
かな
\ruby{絲}{いと}を
もて
\ruby{手鞠}{て|まり}を
\ruby{{\換字{造}}}{つく}り
\ruby{居}{ゐ}たるに、
%
お
\ruby{龍}{りう}は
また
\ruby{其}{そ}の
\ruby{娘}{こ}を
\ruby{呼}{よ}びかけて、

\原本頁{154-5}%
『
\ruby{折角}{せつ|かく}
\ruby{汝}{おまへ}さんに
\ruby{敎}{をし}へて
\ruby{貰}{もら}つた
けれど、
%
\ruby{此家}{こ|〻}は% 原本通り「〻(二の字点、揺すり点)」
\ruby{妾}{わたし}の
\ruby{{\換字{尋}}}{たづ}ね
やうといふ
\ruby{家}{うち}と
\ruby{異}{ちが}つて
\ruby{居}{ゐ}たの!。
%
\ruby{隱居{\換字{所}}}{いん|きよ|じよ}の
\ruby{方}{はう}
といふのを
\ruby{知}{し}つて
おいでなら、
%
\ruby{其}{その}
\ruby{方}{はう}を
\ruby{一寸}{ちよ|つと}
\ruby{敎}{をし}へて
\ruby{下}{くだ}さいな。
』

\原本頁{154-8}%
と、
%
\ruby{笑}{ゑみ}を
つくつて
\ruby{云}{い}へば、
%
\ruby{子守}{こ|もり}も
\ruby{莞爾}{に|こ}つき、

\原本頁{154-9}%
『
あ〻% 原本通り「〻(二の字点、揺すり点)」
お
\ruby{濱}{はま}ちやんの
\ruby{家}{うち}の
\ruby{事}{こと}かい、
%
そんなら
\ruby{{\換字{猶}}}{なほ}の
\ruby{事}{こと}だ、
%
\ruby{{\換字{伴}}}{つ}れてつて
\ruby{{\換字{遣}}}{や}るべい。
』

\原本頁{154-11}%
と
\ruby{云}{い}ひながら
ずん〳〵
\ruby{先}{さき}に
\ruby{立}{た}ちて、
%
\ruby{卷}{ま}き
かけたる
\ruby{鞠}{まり}を
\ruby{袂}{たもと}に
して
\ruby{導}{みちび}き
くれたり。

\原本頁{153-2}%
『
ヘーエ、
%
お
\ruby{濱}{はま}ちやん
といふ
\ruby{女}{こ}が
\ruby{其}{そ}
\ruby{家}{こ}には
\ruby{居}{ゐ}るの?。
』

\原本頁{155-3}%
『
アレ
\ruby{隱居}{いん|きよ}の
\ruby{方}{はう}へ
\ruby{行}{い}く
\ruby{人}{ひと}で
\ruby{居}{ゐ}て、
%
それで
お
\ruby{濱}{はま}ちやんを
\ruby{知}{し}らない
だかエ\換字{!?}。
』

\原本頁{155-5}%
『
だつて
\ruby{妾}{わたし}は
\ruby{初}{はじめ}て
\ruby{來}{き}たもんで、
%
\ruby{水野}{みづ|の}さん
ていふ
\ruby{方}{かた}を
\ruby{{\換字{尋}}}{たづ}ねるんだもの!。
』

\原本頁{155-7}%
『
\ruby{水野}{みづ|の}さん
ところへ
\ruby{{\換字{尋}}}{たづ}ねて
\ruby{來}{き}たつて!、
%
アノ
\ruby{先生}{せん|せい}の
\ruby{水野}{みづ|の}さん
ところへ\換字{!?}。
』

\原本頁{155-9}%
\ruby{振{\換字{返}}}{ふり|かへ}つて
\ruby{子守}{こ|もり}は
\ruby{新}{あらた}に
お
\ruby{龍}{りう}を
\ruby{見}{み}しが、
%
\ruby{其}{そ}の
\ruby{都}{みやこ}びて
\ruby{淸潔}{きれ|い}に
\ruby{美}{うつく}しきは、
%
\ruby{何}{なに}
\ruby{知}{し}らぬ
\ruby{眼}{め}にも
\ruby{明}{あき}らかに
\ruby{映}{うつ}りたり。

\原本頁{155-11}%
『
\ruby{姐}{ね{\換字{𛀁}}}さんは
\ruby{水野}{みづ|の}さんの
\ruby{妹}{いもうと}ツ
\ruby{子}{こ}かエ。
』

\原本頁{156-1}%
お
\ruby{龍}{りう}は
\ruby{其}{そ}の
\ruby[||j>]{頓狂}{とん|きやう}なる
\ruby{考}{かんが}へと
\ruby{唐突}{だし|ぬけ}なる
\ruby{問}{とひ}とに
\ruby{自然}{おの|づ}と
\ruby{笑}{わらひ}を
\ruby{催}{もよほ}さしめられたり。

\原本頁{156-3}%
『
\ruby{何故}{な|ぜ}?。
』

\原本頁{156-4}%
『
\ruby{何故}{な|ぜ}つて、
%
\ruby{東京}{とう|きやう}から
わざ〳〵
\ruby{來}{き}たので
\ruby{無}{な}いかエ。
』

\原本頁{156-5}%
『ホヽヽ。
%
おもしろい
\ruby{事}{こと}を
お
\ruby{云}{い}ひだ\換字{子}、
%
そりやあ
\ruby{東京}{とう|きやう}から
\ruby{來}{き}た
のだけれども、
%
\ruby{東京}{とう|きやう}から
\ruby{來}{き}た
とつて
\ruby{妹}{いもうと}たあ
\ruby{定}{きま}りやあ
\ruby{仕}{し}ません。
』

\原本頁{156-7}%
『
アヽ
\ruby{解}{わか}つた。
%
ぢやあ
\ruby{姐}{ね{\換字{𛀁}}}さんは
\ruby{水野}{みづ|の}
\ruby{樣}{さん}の
\ruby{内君}{おかみ|さん}になる
\ruby{人}{ひと}だベエ。
』

\原本頁{156-8}%
『
いやだよ、
%
そんな
\ruby{飛}{と}んでもない
\ruby{事}{こと}を!。
%
ホヽヽ、
%
\ruby{妾}{わたし}あ
まだ
\ruby{水野}{みづ|の}さん
ていふ
\ruby{方}{かた}にも、
%
お目に
か〻つた% 原本通り「〻(二の字点、揺すり点)」
\ruby{事}{こと}さへ
\ruby{有}{あ}りや
しないのだよ。
』

\原本頁{156-11}%
『
\ruby{隱}{かく}しても
いかないだ!。
%
\ruby{姐}{ね{\換字{𛀁}}}さん
\ruby{今}{いま}
\ruby[j]{些}{ちつと}
\ruby{紅}{あか}い
\ruby{顏}{かほ}
したゞ!。% TODO 原本の「二の字点、揺すり点」に濁点のグリフが見つからないので「ゞ」
%
ホレ
もう
\ruby{此家}{こ|〻}が% 原本通り「〻(二の字点、揺すり点)」
\ruby{御亭主}{ご|てい|しゆ}の
\ruby{家}{うち}だ。
%
\ruby{植{\換字{込}}}{うゑ|こみ}
いぢつて
\ruby{居}{ゐ}るのが
お
\ruby{濱}{はま}ちやんの
お
\ruby{爺}{ぢい}さんだよ。
』

\原本頁{157-3}%
\ruby{子守}{こ|もり}は
\ruby{五六歩}{ご|ろつ|ぽ}
いきなりに
\ruby{駈}{か}け
\ruby{拔}{ぬ}けて、
%
\ruby{植{\換字{込}}}{うゑ|こみ}の
\ruby{不揃}{ふ|ぞろひ}に
なりたるを
\ruby{鋏}{はさみ}
もて
\ruby{剪}{き}り
\ruby{居}{ゐ}たる
\ruby[||j>]{禿頭}{はげ|あたま}の
\ruby{澤}{つや}やかに
\ruby{人}{ひと}の
\ruby{好}{よ}さ〻うなる% 原本通り「〻(二の字点、揺すり点)」
\ruby{老人}{とし|より}に
\ruby{對}{むか}つて、
%
\ruby{一二間}{いち|に|けん}
こなたより
\ruby{左}{さ}らでも
\ruby{徹}{とほ}る
\ruby{聲}{こゑ}を
\ruby{{\換字{遠}}慮}{ゑん|りよ}
\ruby{無}{な}く
\ruby{大}{おほき}くして、

\原本頁{157-6}%
『
\ruby{爺々}{おぢ|さん}!、
%
\ruby{妾}{おら}
\ruby{東京}{とう|きやう}の
お
\ruby{客樣}{きやく|さま}を
\ruby{案内}{あん|ない}して
\ruby{來}{き}たゞよ。% TODO 原本の「二の字点、揺すり点」に濁点のグリフが見つからないので「ゞ」
』

\原本頁{157-7}%
と、
%
\ruby{確實}{たし|か}に
\ruby{然樣}{さ|う}と
\ruby{思}{おも}ひ
\ruby{{\換字{込}}}{こ}んで、
%
\ruby{獨}{ひと}り
\ruby{承知}{しよう|ち}して
\ruby{叫}{さけ}び
\ruby{知}{し}らすれば、
%
\ruby[g]{吉右衛門}{きちゑもん}は
\ruby{不意}{ふ|い}なるに
\ruby{驚}{おどろ}きて
\ruby{答}{こた}へ
もせぬ
\ruby{時}{とき}、
%
\ruby{此方}{こな|た}に
\ruby{向}{むか}へる
\ruby{入口}{いり|くち}の
\ruby{障子}{しやう|じ}は
さつと
\ruby{開}{あ}けられて、
%
\ruby{繪}{ゑ}に
\ruby{見}{み}るやうなる
\ruby{色白}{いろ|じろ}の
\ruby{容貌}{きり|やう}
\ruby{好}{よ}き
\原本頁{157-11}\改行%
\ruby{娘}{こ}の、
%
\ruby{星}{ほし}の
やうなる
\ruby{美}{うつく}しき
\ruby{眼}{め}を
\ruby{光}{ひか}らせたると、
%
\ruby{水野}{みづ|の}とは
\ruby{此}{この}
\ruby{人}{ひと}なるべき
\ruby{{\換字{若}}}{わか}き
\ruby{男}{をとこ}との
\ruby{現}{あら}はれしが、
%
\ruby{是}{こ}は
\ruby{如何}{い|か}に
\ruby{其}{その}
\ruby[||j>]{男}{をとこ}は
\ruby{今}{いま}し
\ruby{方}{がた}
\ruby{滊車}{き|しや}の
\原本頁{158-2}\改行%
\ruby{内}{うち}にて
\ruby{妾}{わ}が
\ruby{傷}{きず}つけし
\ruby{其}{そ}の
\ruby{人}{ひと}
なれば、
%
\ruby{互}{たがひ}に
\ruby{顏}{かほ}を
\ruby{見合}{み|あ}はす
\ruby{{\換字{途}}端}{と|たん}、
%
\ruby{言葉}{こと|ば}こそ
\ruby{出}{いだ}さぬ
\ruby{彼}{かれ}
\ruby{此}{これ}
\ruby{共}{とも}に、
%
これはと
\ruby{驚}{おどろ}きたる
\ruby{其}{そ}の
\ruby{樣子}{やう|す}は、
%
\ruby{明}{あき}らかに
\ruby{傍}{かたへ}の
\ruby{人々}{ひと|〴〵}にも
\ruby{見}{み}{\換字{𛀁}}たり。

\原本頁{158-5}%
『
ホーラ、
%
\ruby{隱}{かく}しても
いかないだよ!。
%
\ruby{兩方}{りやう|はう}で
\ruby{知}{し}つて
たゞ!。% TODO 原本の「二の字点、揺すり点」に濁点のグリフが見つからないので「ゞ」
』

\原本頁{158-6}%
\ruby{子守}{こ|もり}は
\ruby{獨}{ひと}り
\ruby{定}{ぎめ}に
\ruby{凱歌}{とち|どき}を% TODO 「かちどき」と思われるが原本通りのルビ
\ruby{擧}{あ}げて、
%
\ruby{得意}{とく|い}の
\ruby{顏}{かほ}つきして
\ruby{彼方}{かな|た}に
\ruby{走}{はし}り
\ruby{去}{さ}り、
%
お
\ruby{濱}{はま}と
\ruby[g]{吉右衛門}{きちゑもん}とは
\ruby{無言}{む|ごん}に
お
\ruby{龍}{りう}
\ruby{水野}{みづ|の}を
\ruby{見}{み}れば、
%
お
\ruby{龍}{りう}は
たゞ% TODO 原本の「二の字点、揺すり点」に濁点のグリフが見つからないので「ゞ」
\ruby{何}{なに}とも
\ruby{{\換字{分}}}{わか}らぬ
\ruby{心地}{こ〻|ち}に、% 原本通り「〻(二の字点、揺すり点)」
%
かつと
\ruby{紅色}{くれ|なゐ}
\ruby{潮}{さ}したる
\ruby{面}{おもて}、
%
\ruby{一朶}{いち|だ}の
\ruby{芙蓉}{ふ|よう}
\ruby{十{\換字{分}}}{じう|ぶん}に
\ruby{醉}{よ}ひたり。% 「醉」は原本通り「よ」で調整
