\Entry{其二十九}

% メモ 校正終了 2024-04-24 2024-06-02
\原本頁{159-1}%
\ruby{既}{すで}に
\ruby{我}{わ}が
\ruby{{\換字{尋}}}{たづ}ぬる
\ruby{水野}{みづ|の}とは
\ruby{其}{その}
\ruby{人}{ひと}
なるべく、
%
\ruby{{\換字{又}}}{また}
\ruby{今}{いま}
\ruby{聞}{き}ける
お
\ruby{濱}{はま}とは
\ruby{其}{その}
\ruby{娘}{こ}
なるべしと
\ruby{猜}{すゐ}し
\ruby{知}{し}りたれど、
%
\ruby{飛}{と}んでも
\ruby{無}{な}き
\ruby{子守女}{こ|も|り}の
\ruby{言葉}{こと|ば}に
\ruby{度}{ど}を
\ruby{失}{うしな}ひたる
お
\ruby{龍}{りう}は、
%
\ruby{思}{おも}はざる
\ruby{横風}{よこ|かぜ}に
\ruby{{\換字{扇}}}{あふ}られて
\ruby{目}{め}ざす
ところに
\ruby{{\換字{船}}首}{み|よし}を
\ruby{向}{む}け
はぐりたる
\ruby{{\換字{船}}}{ふね}の、
%
\ruby{先}{ま}づ
\ruby{取}{と}りあへず
\ruby{間{\換字{近}}}{ま|ぢか}なる
\ruby[||j>]{纜}{もやひ}
\ruby[||j>]{杭}{ ぐひ}に
\ruby{取}{と}り
つけたるが
\ruby{如}{ごと}く、
%
\ruby{吉右衛門}{きち||ゑ|もん}に
\ruby{向}{むか}ひて
\ruby{小腰}{こ|ゞし}を% TODO 原本の「二の字点、揺すり点」に濁点のグリフが見つからないので「ゞ」
\ruby{屈}{かゞ}めつ、% TODO 原本の「二の字点、揺すり点」に濁点のグリフが見つからないので「ゞ」

\原本頁{159-5}%
『
\ruby{妾}{わたし}はあの、
%
\ruby{岩崎}{いは|ざき}の
\ruby{母}{は〻}の% ルビ調整(原本通り)「〻(二の字点、揺すり点)」
\ruby{許}{ところ}から
\ruby{參}{まゐ}つたもので
ございますが、
%
\ruby{水野}{みづ|の}さんが
おいでに
なりますなら
\ruby{何卒}{どう|か}‥‥‥、
』

\原本頁{159-8}%
と、
%
\ruby{辛}{から}くも
これだけを
\ruby{言}{い}ひて
ホツと
\ruby{息}{いき}
\ruby{吐}{つ}きたり。

\原本頁{159-9}%
\ruby{合點}{が|てん}の
\ruby{惡}{わか}からぬ
\ruby{吉右衛門}{きち||ゑ|もん}は、
%
\ruby{例}{れい}の
\ruby{眼鏡}{め|がね}
\ruby{越}{ご}しに
お
\ruby{龍}{りう}を
\ruby{見}{み}しが、
%
\ruby{手}{て}に
せし
\ruby{剪刀}{はさ|み}を
\ruby{樹}{き}の
\ruby{枝}{{\換字{𛀁}}だ}に
\ruby{一寸}{ちよ|つと}
\ruby{掛}{か}け
すて〻、% ルビ調整(原本通り)「〻(二の字点、揺すり点)」
%
\ruby{淸}{きよ}らなる
\ruby{赤}{あか}ら
\ruby{顏}{がほ}に
\ruby{笑}{ゑみ}を
さへ
\ruby{含}{ふく}み、

\原本頁{160-1}%
『
ハア
\ruby{左樣}{さ|う}ですか、
%
さあ
\ruby{御上}{お|あが}んなさい。
%
\ruby{丁度}{ちやう|ど}
\ruby{今}{いま}しがた
\ruby{御歸宅}{お|かへ|り}でした。
%
お
\ruby{濱}{はま}や、
%
\ruby{先生}{せん|せい}の
ところへ
\ruby{御客樣}{お|きやく|さま}だよ。
%
ハヽヽ、
%
お
\ruby{蝶}{てふ}ツ
\ruby{子}{こ}が
\ruby{何}{なに}を
\ruby{下}{くだ}らない!。
』

\原本頁{160-3}%
と
\ruby{末}{すゑ}は
\ruby[||j>]{獨}{ひとり}
\ruby[||j>]{語}{ ごと}
のやうに
\ruby{云}{い}ふ
ところへ、
%
\ruby{生々}{いき|〳〵}
として
\ruby{美}{うつく}しき
\ruby{娘}{こ}は
\ruby{下}{お}り
\ruby{來}{きた}りて、
%
たゞ% TODO 原本の「二の字点、揺すり点」に濁点のグリフが見つからないので「ゞ」
\ruby{纔}{わづか}に
\ruby{頭}{かしら}を
\ruby{下}{さ}げたる
ばかりに
\ruby{愛度氣無}{あ|ど|け|な}く
\ruby{會釋}{ゑ|しやく}し、

\原本頁{160-6}%
『
どうか、
%
\ruby{此方}{こち|ら}から% ルビ調整(原本通り)
\ruby{御上}{お|あが}んなすつて、
』

\原本頁{160-7}%
と
\ruby{先}{さき}に
\ruby{立}{た}つて
ずつと
\ruby{庭}{には}を
\ruby{貫}{とほ}して
\ruby{導}{みちび}くは、
%
\ruby{入口}{いり|ぐち}より
\ruby{{\換字{通}}}{とほ}さば
\ruby{今}{いま}は
\ruby{其處}{そ|こ}に
\ruby{取}{と}り
\ruby{亂}{みだ}したる
\ruby{室}{へや}の
\ruby{他人}{ひ|と}には
\ruby{見}{み}せたからぬ
\ruby{狀}{さま}なるが
\ruby{有}{あ}れば
なるべし。

\原本頁{160-10}%
\ruby{云}{い}はる〻がまゝに% ルビ調整(原本通り)「〻(二の字点、揺すり点)」
お
\ruby{龍}{りう}は
\ruby{庭{\換字{前}}}{には|さき}より
\ruby{上}{あが}りて、
%
\ruby{{\換字{通}}}{とほ}されたる
\ruby{室}{へや}に
ちまぢまと
\ruby{座}{すわ}れば、

\原本頁{161-1}%
『
\ruby{一寸}{ちよ|いと}
\ruby{御待}{お|ま}ちなすつて。
%
たゞ% TODO 原本の「二の字点、揺すり点」に濁点のグリフが見つからないので「ゞ」
\ruby{今}{いま}
\ruby{直}{すぐ}、
』

\原本頁{161-2}%
と
\ruby{云}{い}ひ
\ruby{置}{お}きて
\ruby{娘}{むすめ}は
\ruby{彼方}{かな|た}に
\ruby{去}{さ}りぬ。

\原本頁{161-3}%
\ruby{入口{\換字{近}}}{いり|ぐち|〻か}き% ルビ調整(原本通り)「〻(二の字点、揺すり点)」
\ruby{茶}{ちや}の
\ruby{室}{ま}と
おぼしき
\ruby{方}{かた}に、
%
\ruby{其}{そ}の
\ruby{人}{ひと}も
\ruby{娘}{むすめ}も
\ruby{在}{あ}る
\ruby{樣子}{やう|す}
ながら
\改行% 校正作業の簡略化のため
、
\原本頁{161-4}\改行%
%
\ruby{何}{なに}を
\ruby{爲}{な}し
\ruby{居}{を}ればにや
\ruby{{\換字{猶}}}{なほ}
\ruby{出}{い}で
\ruby{來}{きた}らず、
%
\ruby{我}{われ}
たゞ% TODO 原本の「二の字点、揺すり点」に濁点のグリフが見つからないので「ゞ」
\ruby{一人}{ひと|り}
\ruby{兀然}{つく|ねん}として
\ruby{室}{へや}の
\原本頁{161-5}\改行%
\ruby{内}{うち}を
\ruby{見}{み}れば、
%
\ruby{二本立}{に|ほん|たち}の
\ruby{書箱}{ほん|ばこ}
\ruby{一}{ひと}ツ
\ruby[<j||]{机}{つくゑ}% ルビ調整(配置位置補正)漢数字のルビ化による影響
\ruby[||j>]{一}{いつ}
\ruby[||j>]{脚}{きやく}、
% \ruby{一脚}{いつ|きやく}、
%
\ruby{本箱}{ほん|ばこ}に
\ruby{餘}{あま}れる
\ruby{本}{ほん}の
\ruby{幾十册}{いく|じふ|さつ}か
\ruby{壁}{かべ}に
\ruby{添}{そ}ひて
\ruby{積}{つ}まれたると、
%
\ruby{奧行}{おく|ゆき}の
\ruby{淺}{あさ}き
\ruby{床}{とこ}の
\ruby{間}{ま}に
\ruby{西洋本}{せい|やう|ぼん}の
\ruby[<j||]{少}{すくな}からず% 行末行頭の境界付近なので特例処置を施す
\ruby{置}{お}かれたる
\ruby{其}{その}
\ruby{他}{ほか}には
\ruby{何}{なん}の
\ruby{{\換字{道}}具}{だう|ぐ}も
\ruby{無}{な}く
\ruby{裝{\換字{飾}}}{かざ|り}も
\ruby{無}{な}く、
%
\ruby{味}{あぢ}も
\ruby{無}{な}く
\ruby{素氣}{そつ|け}も
\ruby{無}{な}き
\ruby{其}{そ}の
\ruby{態}{さま}は、
%
\ruby{惡口}{わる|くち}を
\ruby{云}{い}はゞ% TODO 原本の「二の字点、揺すり点」に濁点のグリフが見つからないので「ゞ」
\ruby{{\換字{巡}}査}{じゆん|さ}の
\ruby{{\換字{交}}番{\換字{所}}}{かう|ばん|しよ}に
\ruby{少}{すこ}し
ばかり
\ruby{書籍}{ほ|ん}の
あるやうな
ものなり。

\原本頁{161-10}%
お
\ruby{龍}{りう}は
\ruby{生}{うま}れてより
\ruby{未}{いま}だ
かつて
\ruby{見}{み}ぬ
\ruby{室}{へや}の
\ruby{狀態}{やう|す}に、
%
\ruby{荒野}{あら|の}に
\ruby{立}{た}つたる
やうの
\ruby[||j>]{心}{こ〻ろ}% ルビ調整(原本通り)「〻(二の字点、揺すり点)」
\ruby[||j>]{淋}{ さび}しさを
\ruby{覺}{おぼ}えて、
%
\ruby{何}{なに}を
\ruby{書}{か}いたものか
\ruby{知}{し}れぬ
\ruby{西洋本}{せい|やう|ぼん}の、
%
\原本頁{162-1}\改行%
\ruby{表紙}{へう|し}の
\ruby{金字}{きん|じ}の
\ruby{燦々}{きら|〳〵}と
\ruby{輝}{かゞや}けるに
のみ
たゞ% TODO 原本の「二の字点、揺すり点」に濁点のグリフが見つからないので「ゞ」
\ruby{{\換字{所}}在}{しよ|ざい}
\ruby{無}{な}さの
\ruby{眼}{め}を
\ruby{{\換字{留}}}{と}めて
\ruby{見}{み}つめ
\ruby{居}{ゐ}れば、
%
\ruby{物}{もの}
\ruby{靜}{しづ}かなる
\ruby{田舎}{ゐな|か}の
\ruby{晝間}{ひ|る}も
\ruby{寂}{しん}として、
%
\ruby{彼}{か}の
\ruby{老人}{とし|より}が
\ruby{使}{つか}ふ
\ruby{剪刀}{はさ|み}の
\ruby{音}{おと}は
\ruby{時々}{とき|〴〵}
ちよつきり
ちよつきりと% ルビ調整(原本通り)非踊り字表記
\ruby{聞}{きこ}{\換字{𛀁}}
\ruby{來}{く}るなり。

\原本頁{162-4}%
\ruby[||j>]{心}{こ〻ろ}
おのづから% ルビ調整(原本通り)「〻(二の字点、揺すり点)」
\ruby{靜}{しづ}まれば
\ruby{耳}{み〻}
おのづから% ルビ調整(原本通り)「〻(二の字点、揺すり点)」
\ruby{聰}{さと}く
なりて、
%
\ruby{小聲}{こ|ゞゑ}に% TODO 原本の「二の字点、揺すり点」に濁点のグリフが見つからないので「ゞ」
\ruby{相}{あひ}
\ruby{語}{かた}る
\ruby{彼方}{かな|た}の
\ruby{室}{ま}の
\ruby{話}{はなし}は、
%
\ruby{幽微}{かす|か}には
あれど
\ruby{今}{いま}は
\ruby{聞}{きこ}ゆ。

\原本頁{162-6}%
『
\ruby{左樣}{さ|う}!、
%
それで
\ruby{知}{し}つて
\ruby{居}{ゐ}らしつたの!、
%
あの
\ruby{人}{ひと}が
\ruby{先生}{せん|せい}の
\ruby{足}{あし}を
\ruby{踏}{ふ}んだ
\ruby{人}{ひと}なの!。
%
あら
\ruby{可厭}{い|や}な
\ruby{人}{ひと}だこと、
%
\ruby[||j>]{妾}{わたし}
\ruby[||j>]{{\換字{嫌}}}{ きら}ひだは!。
』

\原本頁{162-8}%
『
だつて
\ruby{{\換字{過}}失}{そ|さう}だもの
\ruby{仕方}{し|かた}が
\ruby{無}{な}い!。
%
\ruby{大變}{たい|へん}
\ruby{氣}{き}の
\ruby{毒}{どく}がつて
\ruby{叮嚀}{てい|ねい}に
\ruby[<j||]{謝}{あやま}つた
のだもの。
\ruby{却}{かへ}つて
\ruby{優}{やさ}しい
\ruby{人}{ひと}だと
\ruby{私}{わたし}は
\ruby{思}{おも}つて
\ruby{居}{ゐ}るよ。
』

\原本頁{162-10}%
『
\ruby{左樣}{さ|う}ねえ!。
%
\ruby{左樣}{さ|う}いへば
\ruby{汗巾}{はん|けち}を
\ruby{破}{やぶ}つて
\ruby{傷}{きず}を
\ruby{卷}{ま}いたつて。
%
アヽ
\ruby{矢張}{やつ|ぱ}り
\ruby{眞實}{ほん|と}は
\ruby{好}{い}い
\ruby{人}{ひと}なのね{\換字{𛀁}}!。
%
ぢやあ
\ruby{妾}{わたし}は
\ruby{{\換字{嫌}}}{きら}ひぢや
\ruby{無}{な}くつて
\原本頁{163-1}\改行%
\ruby{好}{すき}なのよ。
%
\ruby{何}{なん}だか
\ruby{最初}{さい|しよ}
\ruby{見}{み}た
\ruby{時}{とき}から
\ruby{妾}{わたし}は
\ruby{好}{すき}だつたのよ。
%
だけども
\ruby{先生}{せん|せい}の
\ruby{足}{あし}を
\ruby{踏}{ふ}んだつて
\ruby{云}{い}ふので
\ruby{{\換字{嫌}}}{いや}だと
\ruby{思}{おも}つたの!。
%
そしたら
\ruby{矢張}{やつ|ぱり}
また
\ruby{好}{すき}に
なつて
\ruby{仕舞}{し|ま}つたのよ。
%
\ruby{眞箇}{ほん|と}に
\ruby{奇麗}{き|れい}な
\ruby{好}{い}い
\ruby{人}{ひと}ねえ!。
』

\原本頁{163-4}%
『
ハヽヽ、
%
\ruby{好}{すき}だの
\ruby{{\換字{嫌}}}{きらひ}だの
つて、
%
お
\ruby{濱}{はま}ちやん
\ruby{位}{ぐらゐ}
いろ〳〵な
\ruby{事}{こと}を
いふ
\ruby{人}{ひと}は
ありや
\ruby{仕}{し}ない。
%
そりやあ
\ruby{宜}{い}いけれども
お
\ruby{茶}{ちや}でも
\ruby{與}{や}つて
おくれ、
%
\ruby{置}{おき}つぱなしぢやあ
\ruby{可}{か}
\ruby[<j>]{憫}{あいさう}ぢやあ
\ruby{無}{な}いか。
』

\原本頁{163-7}%
『
あら
\ruby{左樣}{さ|う}ぢや
\ruby{無}{な}くつてよ、
%
\ruby{今}{いま}
\ruby{御給仕}{お|きふ|じ}が
\ruby{濟}{す}んでから
\ruby{御茶}{お|ちや}を
\ruby{入}{い}れやうと% ここは原本通り「やふ」でなく「やう」
\ruby{思}{おも}つて
\ruby{居}{ゐ}たのよ。
』

\原本頁{163-9}%
『
い〻{\換字{𛀁}}% ルビ調整(原本通り)「〻(二の字点、揺すり点)」
\ruby{私}{わたし}には
\ruby{關}{かま}は
なくつても
い〻よ。% ルビ調整(原本通り)「〻(二の字点、揺すり点)」
%
さあ〳〵
もう
\ruby{御{\換字{終}}}{お|しまひ}
だか
\改行% 校正作業の簡略化のため
ら!。
』

\原本頁{163-11}%
\ruby{言}{い}ふもの
\ruby{恐}{おそ}らくは
\ruby{何}{なん}の
\ruby[||j>]{意}{こ〻ろ}% ルビ調整(原本通り)「〻(二の字点、揺すり点)」
\ruby[||j>]{無}{ な}からん、
%
\ruby{聞}{き}くもの
\ruby{未}{いま}だ
\ruby{必}{かなら}ずしも
\ruby{{\換字{感}}}{かん}
\ruby{無}{な}くば
あらざる
べし。
