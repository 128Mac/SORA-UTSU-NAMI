\Entry{其二十九}

\ruby{既}{すで}に
\ruby{我}{わ}が
\ruby{{\換字{尋}}}{たづ}ぬる
\ruby{水野}{みづ|の}とは
\ruby{其人}{その|ひと}なるべく、
\ruby{{\換字{又}}}{また}
\ruby{今}{いま}
\ruby{聞}{き}ける
お
\ruby{濱}{はま}とは
\ruby{其娘}{その|こ}なるべしと
\ruby{猜}{すひ}し
\ruby{知}{し}りたれど、
\ruby{飛}{と}んでも
\ruby{無}{な}き
\ruby{子守女}{こ|も|り}の
\ruby{言葉}{こと|ば}に
\ruby{度}{ど}を
\ruby{失}{うしな}ひたる
お
\ruby{龍}{りう}は、
\ruby{思}{おも}はざる
\ruby{横風}{よこ|かぜ}に
\ruby{{\換字{扇}}}{あふ}られて
\ruby{目}{め}ざすところに
\ruby{{\換字{船}}首}{み|よし}を
\ruby{向}{む}けはぐりたる
\ruby{{\換字{船}}}{ふね}の、
\ruby{先}{ま}づ
\ruby{取}{と}りあへず
\ruby{間{\換字{近}}}{ま|ぢか}なる
\ruby{纜杭}{もやひ|ぐひ}に
\ruby{取}{と}りつけたるが
\ruby{如}{ごと}く、
\ruby{吉右衛門}{き|ち|ゑ|もん}に
\ruby{向}{むか}ひて
\ruby{小腰}{こ|ゞし}を
\ruby{屈}{かゞ}めつ、

『
\ruby{妾}{わたし}はあの、
\ruby{岩崎}{いは|ざき}の
\ruby{母}{はゝ}の
\ruby{許}{ところ}から
\ruby{參}{まゐ}つたものでございますが、
\ruby{水野}{みづ|の}さんがおいでになりますなら
\ruby{何卒}{どう|か}…………、』

と、
\ruby{辛}{から}くもこれだけを
\ruby{言}{い}ひてホツと
\ruby{息吐}{いき|つ}きたり。

\ruby{合點}{が|てん}の
\ruby{惡}{わか}からぬ
\ruby{吉右衛門}{き|ち|ゑ|もん}は、
\ruby{例}{れい}の
\ruby{眼鏡越}{め|がね|ご}しに
お
\ruby{龍}{りう}を
\ruby{見}{み}しが、
\ruby{手}{て}にせし
\ruby{剪刀}{はさ|み}を
\ruby{樹}{き}の
\ruby{枝}{\換字{江}だ}に
\ruby{一寸}{ちよ|つと}
\ruby{掛}{か}けすてゝ、
\ruby{淸}{きよ}らなる
\ruby{赤}{あか}ら
\ruby{顏}{がほ}に
\ruby{笑}{ゑみ}をさへ
\ruby{含}{ふく}み、

『ハア
\ruby{左樣}{さ|う}ですか、さあ
\ruby{御上}{お|あが}んなさい。
\ruby{丁度今}{ちやう|ど|いま}しがた
\ruby{御歸宅}{お|かへ|り}でした。
お
\ruby{濱}{はま}や、
\ruby{先生}{せん|せい}のところへ
\ruby{御客樣}{おき|やく|さま}だよ。
ハヽヽ、お
\ruby{蝶}{てふ}ツ
\ruby{子}{こ}が
\ruby{何}{なに}を
\ruby{下}{くだ}らない!。
』

と
\ruby{末}{すゑ}は
\ruby{獨語}{ひとり|ごと}のやうに
\ruby{云}{い}ふところへ、
\ruby{生々}{いき|〳〵}として
\ruby{美}{うつく}しき
\ruby{娘}{こ}は
\ruby{下}{お}り
\ruby{來}{きた}りて、たゞ
\ruby{纔}{わづか}に
\ruby{頭}{かしら}を
\ruby{下}{さ}げたるばかりに
\ruby{愛度氣無}{あ|ど|け|な}く
\ruby{會釋}{ゑし|やく}し、

『どうか、
\ruby{此方}{こち|ら}から
\ruby{御上}{お|あが}んなすつて、』

と
\ruby{先}{さき}に
\ruby{立}{た}つてずつと
\ruby{庭}{には}を
\ruby{貫}{とほ}して
\ruby{導}{みちび}くは、
\ruby{入口}{いり|ぐち}より
\ruby{{\換字{通}}}{とほ}さば
\ruby{今}{いま}は
\ruby{其處}{そ|こ}に
\ruby{取}{と}り
\ruby{亂}{みだ}したる
\ruby{室}{へや}の
\ruby{他人}{ひ|と}には
\ruby{見}{み}せたからぬ
\ruby{狀}{さま}なるが
\ruby{有}{あ}ればなるべし。

\ruby{云}{い}はるゝがまゝに
お
\ruby{龍}{りう}は
\ruby{庭前}{には|さき}より
\ruby{上}{あが}りて、
\ruby{{\換字{通}}}{とほ}されたる
\ruby{室}{へや}にちまぢまと
\ruby{座}{すわ}れば、

『
\ruby{一寸}{ちよ|いと}
\ruby{御待}{おま|ち}ちなすつて。
たゞ
\ruby{今}{いま}
\ruby{直}{すぐ}、』

と
\ruby{云}{い}ひ
\ruby{置}{お}きて
\ruby{娘}{むすめ}は
\ruby{彼方}{かな|た}に
\ruby{去}{さ}りぬ。
\ruby{入口{\換字{近}}}{いり|ぐち|ゝか}き
\ruby{茶}{ちや}の
\ruby{室}{ま}とおぼしき
\ruby{方}{かた}に、
\ruby{其}{そ}の
\ruby{人}{ひと}も
\ruby{娘}{むすめ}も
\ruby{在}{あ}る
\ruby{樣子}{やう|す}ながら、
\ruby{何}{なに}を
\ruby{爲}{な}し
\ruby{居}{を}ればにや
\ruby{{\換字{猶}}}{なほ}
\ruby{出}{い}で
\ruby{來}{きた}らず、
\ruby{我}{われ}たゞ
\ruby{一人}{ひと|り}
\ruby{兀然}{つく|ねん}として
\ruby{室}{へや}の
\ruby{内}{うち}を
\ruby{見}{み}れば、
\ruby{二本立}{に|ほん|だち}の
\ruby{書箱}{ほん|ばこ}
\ruby{一}{ひと}ツ
\ruby{机一脚}{つくゑ|いつ|きやく}、
\ruby{本箱}{ほん|ばこ}に
\ruby{餘}{あま}れる
\ruby{本}{ほん}の
\ruby{幾十冊}{いく|じう|さつ}か
\ruby{壁}{かべ}に
\ruby{添}{そ}ひて
\ruby{積}{つ}まれたると、
\ruby{奥行}{おく|ゆ}きの
\ruby{淺}{あさ}き
\ruby{床}{とこ}の
\ruby{間}{ま}に
\ruby{西洋本}{せい|やう|ぼん}の
\ruby{少}{すくな}からず
\ruby{置}{お}かれたる
\ruby{其他}{その|ほか}には
\ruby{何}{なん}の
\ruby{{\換字{道}}具}{だう|ぐ}も
\ruby{無}{な}く
\ruby{裝{\換字{飾}}}{かざ|り}も
\ruby{無}{な}く、
\ruby{味}{あじ}も
\ruby{無}{な}く
\ruby{素氣}{そつ|け}も
\ruby{無}{な}き
\ruby{其}{そ}の
\ruby{態}{さま}は、
\ruby{惡口}{わる|くち}を
\ruby{云}{い}はゞ
\ruby{{\換字{巡}}査}{じゆ|んさ}の
\ruby{{\換字{交}}番{\換字{所}}}{かう|ばん|しよ}に
\ruby{少}{すこ}しばかり
\ruby{書籍}{ほ|ん}のあるやうなものなり。

お
\ruby{龍}{りう}は
\ruby{生}{う}まれてより
\ruby{未}{いま}だかつて
\ruby{見}{み}ぬ
\ruby{室}{へや}の
\ruby{狀態}{やう|す}に、
\ruby{荒野}{あら|の}に
\ruby{立}{た}つたるやうの
\ruby{心淋}{こゝろ|さび}しさを
\ruby{覺}{おぼ}えて、
\ruby{何}{なに}を
\ruby{書}{か}いたものか
\ruby{知}{し}れぬ
\ruby{西洋本}{せい|やう|ぼん}の、
\ruby{表紙}{へう|し}の
\ruby{金字}{きん|じ}
\ruby{燦々}{きら|〳〵}と
\ruby{輝}{かゞや}けるにのみたゞ
\ruby{{\換字{所}}在無}{しよ|ざい|な}さの
\ruby{眼}{め}を
\ruby{{\換字{留}}}{と}めて
\ruby{見}{み}つめ
\ruby{居}{ゐ}れば、
\ruby{物靜}{もの|しづ}かなる
\ruby{田舎}{ゐな|か}の
\ruby{晝間}{ひ|る}も
\ruby{寂}{しん}として、
\ruby{彼}{か}の
\ruby{老人}{とし|より}が
\ruby{使}{つか}ふ
\ruby{剪刀}{はさ|み}の
\ruby{音}{おと}は
\ruby{時々}{とき|〴〵}ちよつきりちよつきりと
\ruby{聞}{きこ}\換字{江}
\ruby{來}{く}るなり。

\ruby{心}{こゝろ}おのづから
\ruby{靜}{しづ}まれば
\ruby{耳}{みゝ}おのづから
\ruby{{\換字{聡}}}{さと}くなりて、
\ruby{小聲}{こ|ゞ\換字{江}}に
\ruby{相語}{あひ|かた}る
\ruby{彼方}{かな|た}の
\ruby{室}{ま}の
\ruby{話}{はなし}は、
\ruby[g]{幽微}{かすか}にはあれど
\ruby{今}{いま}は
\ruby{聞}{きこ}ゆ。

『
\ruby{左樣}{さ|う}!、それで
\ruby{知}{し}つて
\ruby{居}{ゐ}らしつたの!、あの
\ruby{人}{ひと}が
\ruby{先生}{せん|せい}の
\ruby{足}{あし}を
\ruby{踏}{ふ}んだ
\ruby{人}{ひと}なの!。
あら
\ruby{可厭}{い|や}な
\ruby{人}{ひと}だこと、
\ruby{妾{\換字{嫌}}}{わたし|きら}ひだは!。
』

『だつて
\ruby{{\換字{過}}失}{そ|さう}だもの
\ruby{仕方}{し|かた}が
\ruby{無}{な}い!。
\ruby{大變氣}{たい|へん|き}の
\ruby{毒}{どく}がつて
\ruby{叮嚀}{てい|ねい}に
\ruby{謝}{あやま}つたのだもの
\ruby{却}{かへ}つて
\ruby{優}{やさ}しい
\ruby{人}{ひと}だと
\ruby{私}{わたし}は
\ruby{思}{おも}つて
\ruby{居}{ゐ}るよ。
』

『
\ruby{左樣}{さ|う}ねえ!。
\ruby{左樣}{さ|う}いへば
\ruby{汗巾}{はん|けち}を
\ruby{破}{やぶ}つて
\ruby{傷}{きず}を
\ruby{{\換字{巻}}}{ま}いたつて。
アヽ
\ruby{矢張}{やつ|ぱ}り
\ruby{眞實}{ほん|と}は
\ruby{好}{い}い
\ruby{人}{ひと}なのね\換字{江}!。
ぢやあ
\ruby{妾}{わたし}は
\ruby{{\換字{嫌}}}{きら}ひぢや
\ruby{無}{な}くつて
\ruby{好}{すき}なのよ。
\ruby{何}{なん}だか
\ruby{最初見}{さい|しよ|み}た
\ruby{時}{とき}から
\ruby{妾}{わたし}は
\ruby{好}{すき}だつたのよ。
だけども
\ruby{先生}{せん|せい}の
\ruby{足}{あし}を
\ruby{踏}{ふ}んだつて
\ruby{云}{い}ふので
\ruby{{\換字{嫌}}}{いや}だと
\ruby{思}{おも}つたの!。
\ruby{眞箇}{ほん|と}に
\ruby{奇麗}{き|れい}な
\ruby{好}{い}い
\ruby{人}{ひと}ねえ!。
』

『ハヽヽ、
\ruby{好}{すき}だの
\ruby{{\換字{嫌}}}{きらひ}だのつて、
お
\ruby{濱}{はま}ちやん
\ruby{位}{ぐらゐ}いろ〳〵な
\ruby{事}{こと}をいふ
\ruby{人}{ひと}はありや
\ruby{仕}{し}ない。
そりやあ
\ruby{宣}{い}いけれども
お
\ruby{茶}{ちや}でも
\ruby{與}{や}つておくれ、
\ruby{置}{おき}つぱなしぢやあ
\ruby{可憫}{かあ|いさう}ぢやあ
\ruby{無}{な}いか。
』

『あら
\ruby{左樣}{さ|う}ぢや
\ruby{無}{な}くつてよ、
\ruby{今}{いま}
\ruby{御給仕}{お|きふ|じ}が
\ruby{濟}{す}んでから
\ruby{御茶}{お|ちや}を
\ruby{入}{い}れやふと
\ruby{思}{おも}つて
\ruby{居}{ゐ}たのよ。
』

『い〻\換字{江}
\ruby{私}{わたし}には
\ruby{關}{かま}はなくつてもい〻よ。
さあ〳〵もう
\ruby{御{\換字{終}}}{おし|まひ}だから!。
』

\ruby{言}{い}ふもの
\ruby{恐}{おそ}らくは
\ruby{何}{なん}の
\ruby{意無}{こゝろ|な}からん、
\ruby{聞}{き}くもの
\ruby{未}{いま}だ
\ruby{必}{かなら}ずしも
\ruby{感無}{かん|な}くばあらざるべし。

