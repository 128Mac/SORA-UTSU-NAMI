\Entry{其三十}

\原本頁{}%
お
\ruby{龍}{りう}は
\ruby{抑如何}{そも|い|か}なる
\ruby{人}{ひと}ぞや。

\原本頁{}%
お
\ruby{孃樣}{ぢやう|さま}
お
\ruby{孃樣}{ぢやう|さま}で
\ruby{育}{そだ}てられたる
\ruby{身}{み}にはあらねど、
%
\ruby{生}{うま}れついての
\ruby{心{\換字{情}}}{こゝろ|もち}に
\ruby{人}{ひと}とは
\ruby{異}{かは}つたるところあつて、
%
\ruby{駿府}{すん|ぷ}の
\ruby{叔母}{を|ば}のところへ
\ruby{引取}{ひき|と}られたる
\ruby{其夜}{その|よ}、
%
はじめて
\ruby{何}{なに}も
\ruby{無}{な}き
\ruby{座敷}{ざ|しき}に
\ruby{寐}{ね}かされて、
%
\ruby{吾家}{う|ち}では
\ruby{如是}{か|う}は
\ruby{無}{な}かつたものをと
\ruby{物足}{もの|た}らぬ
\ruby{心地}{こゝ|ち}し、
%
\ruby{{\換字{翌}}日}{あく|るひ}
\ruby{我}{わ}が
\ruby{荷物}{に|もつ}の
\ruby{行李}{こう|り}を
\ruby{解}{と}きし
\ruby{次}{ついで}に、
%
\ruby{我}{わ}が
\ruby{好}{す}きなもの〻
\ruby{數多}{かず|おほ}き
\ruby{中}{なか}より
\ruby{{\換字{平}}生}{ひご|ろ}
\ruby{氣}{き}に
\ruby{入}{い}りの
\ruby{永徳齋}{{\換字{𛀁}}い|とく|さい}の
\ruby{小人形}{こ|にん|ぎやう}を
\ruby{取}{と}り
\ruby{出}{いだ}して、
%
そつと
\ruby{小棚}{こ|だな}に
\ruby{{\換字{飾}}}{かざ}り
\ruby{置}{お}きしに、
%
それを
\ruby{固}{かた}い
\ruby{自慢}{じ|まん}の
\ruby{叔母}{を|ば}の
\ruby{帝釋樣}{たい|しやく|さま}のやうな
\ruby{三角}{さん|かく}の
\ruby{眼}{め}に
\ruby{睨}{にら}まれて、
%
\ruby{其樣}{そ|ん}な
\ruby{大}{おほ}きい
\ruby{形體}{な|り}をして
\ruby{人形}{にん|ぎやう}なんぞを
\ruby{捏}{こ}ね
\ruby{{\換字{廻}}}{まは}して
\ruby{{\換字{遊}}}{あそ}ぶと
\ruby{云}{い}ふ
\ruby{事}{こと}がありますか、
%
\ruby{藏}{しま}つて
\ruby{御置}{お|お}きなさい、
%
と
\ruby{唯}{たゞ}
\ruby{一言}{ひと|こと}に
\ruby{叱}{しか}りつけられ、
%
あ〻あんまりつまらない
\ruby{{\換字{情}}無}{なさけ|な}い
\ruby{叔母樣}{を|ば|さん}、
%
\ruby{何樣}{ど|う}すれば
\ruby{其樣}{そ|ん}な
\ruby{乾魚}{ひ|もの}のやうな
\ruby{氣}{き}になつて
\ruby{居}{ゐ}らる〻
\ruby{事}{こと}かと、
%
\ruby{恨}{うら}み
\ruby{疑}{うたが}ひながらも
\ruby{爭}{あらそ}ひかねて、
%
\ruby{其時}{その|とき}よりやうやく『わたしの
\ruby{好}{すき}な
\ruby{物}{もの}』を
\ruby{身}{み}の
\ruby{傍}{ほとり}に
\ruby{置}{お}かずして
\ruby{日}{ひ}を
\ruby{{\換字{送}}}{おく}るに
\ruby{慣}{な}る〻に
\ruby{至}{いた}りたるなり。

\原本頁{}%
されば
\ruby{頼}{たの}もしからぬ
\ruby{男}{をとこ}に
\ruby{一生}{いつ|しやう}を
\ruby{{\換字{過}}}{あやま}られて、
%
\ruby{涙}{なみだ}の
\ruby{淵瀬}{ふち|せ}に
\ruby{{\換字{浮}}}{う}き
\ruby{沈}{しづ}みしたる
\ruby{後}{のち}、
%
\ruby{今}{いま}は
\ruby{他人}{ひ|と}の
\ruby{家}{いへ}に
\ruby{寄食客}{かゝ|り|びと}の
\ruby{身}{み}の
\ruby{長閑}{のど|か}らしく
\ruby[g]{玩弄品}{おもちや}
\ruby{三昧}{ざん|まい}をするとにはあらねど、
%
\ruby{傳}{でん}といひ、
%
\ruby{淸}{せい}といひ、
%
\ruby{{\換字{勝}}}{かつ}といひ、
%
\ruby{彦}{ひこ}といひ、
%
\ruby{出入}{で|はい}る
\ruby{{\換字{若}}}{わか}き
\ruby[<j|]{男}{をとこ}
\ruby{共}{ども}の
\ruby{爭}{あらそ}つて
\ruby{氣}{き}を
\ruby{取}{と}らんとて、
%
\ruby{折々}{をり|〳〵}
\ruby{吳}{く}れたる
\ruby{種々}{いろ|〳〵}の
\ruby{物品}{も|の}の
\ruby{中}{うち}、
%
\ruby{傳}{でん}が
\ruby{持}{も}て
\ruby{來}{きた}れる
\ruby{薄色}{うす|いろ}の
\ruby{瑪瑙}{め|なう}の
\ruby{細工}{さい|く}の
\ruby{小}{ちひさ}き
\ruby{兎}{うさぎ}の、
%
\ruby{姿}{すがた}しほらしくふつくりとして、
%
ぽつちりと
\ruby{紅}{あか}き
\ruby{眼}{め}のいと
\ruby{可憐}{か|はゆ}く
\ruby{出來}{で|き}たるが
\ruby{甚}{いた}く
\ruby{氣}{き}に
\ruby{入}{い}り、あれかこれかと、
%
アナ
\ruby{絲}{いと}の
\ruby{色}{いろ}を
\ruby{擇}{{\換字{𛀁}}ら}みに
\ruby{擇}{{\換字{𛀁}}ら}んで、
%
\ruby{其}{そ}のために
\ruby{敷}{し}くべき
\ruby{蒲團}{ふ|とん}の
\ruby{花}{はな}やかに
\ruby{美}{うつく}しきを
\ruby{{\換字{編}}}{あ}みて
\ruby{{\換字{遣}}}{や}りつ、
%
はじめて
\ruby{其}{それ}に
\ruby{載}{の}せて
\ruby{見}{み}たる
\ruby{時}{とき}、
%
\ruby{色}{いろ}の
\ruby{映}{うつ}り
\ruby{合}{あ}ひていよいよ
\ruby{好}{この}ましく
\ruby{愛}{あい}らしく
\ruby{見}{み}えたる
\ruby{嬉}{うれ}しさの
\ruby{餘}{あま}りの
\ruby{戱}{たはむ}れに、
%
\ruby{此兎}{こ|れ}は
\ruby{妾}{わたし}の
\ruby{大切}{だい|じ}な
\ruby{人}{ひと}なの!\inhibitglue{}と
\ruby{獨語}{ひとり|ごと}したりしが、
%
\ruby{其語}{そ|れ}を
\ruby{人}{ひと}より
\ruby{聞}{き}きて
\ruby{勘{\換字{違}}}{かん|ちが}ひしてか、
%
\ruby{其頃}{その|ころ}より
\ruby{傳}{でん}の
\ruby{煩}{うるさ}く
\ruby{付}{つ}き
\ruby{纏}{まと}ふ、
%
\ruby{其}{それ}は
\ruby{何}{なに}よりの
\ruby{{\換字{迷}}惑}{めい|わく}ながら、
%
\ruby{今}{いま}だに
\ruby{兎}{うさぎ}の
\ruby{可愛}{か|はゆ}さは
\ruby{冷}{さ}めず、
%
\ruby{何}{なん}ぞの
\ruby{折}{をり}には『
\ruby{兎之}{う|の}さん』と
\ruby{喚}{よ}びかけて、
%
\ruby{心}{こゝろ}の
\ruby{淋}{さび}しさ
\ruby{{\換字{遣}}}{や}る
\ruby{方無}{かた|な}き
\ruby{時}{とき}の、
%
\ruby{語}{かた}らう
\ruby{友無}{とも|な}き
\ruby{孤獨}{ひと|りみ}の
\ruby{憂}{う}さを、
%
\ruby{苟且}{かり|そめ}に
\ruby{一寸}{ちよ|つと}
\ruby{慰}{なぐさ}め
\ruby{忘}{わす}る〻なり。

\原本頁{}%
\ruby{是}{かく}のごとき
お
\ruby{龍}{りう}は
\ruby{今}{いま}
\ruby{一室}{いつ|しつ}の
\ruby{中}{うち}に、
%
\ruby{眼}{め}を
\ruby{慰}{なぐさ}め
\ruby{心}{こゝろ}を
\ruby{寄}{よ}せて
\ruby{{\換字{情}}懷}{おも|ひ}の
\ruby{{\換字{遣}}}{や}りどころとすべき
\ruby{物}{もの}の
\ruby{一}{ひと}つも
\ruby{無}{な}くて、
%
\ruby{床}{とこ}に
\ruby{插花瓶}{さし|ばな|いけ}は
\ruby{有}{あ}りながら
\ruby{末枯}{す|が}れたる
\ruby{花}{はな}も
\ruby{無}{な}く、
%
\ruby{机上}{つく|ゑ}に
\ruby{筆架水滴}{ふで|かけ|みづ|いれ}の
\ruby{影}{かげ}もあらで
\ruby{裸硯}{はだか|すゞり}の
\ruby{淋}{さび}しく
\ruby{置}{お}かれたるものなるを
\ruby{見}{み}て、
%
\ruby{成程}{なる|ほど}
\ruby{書生}{しよ|せい}さんは
\ruby{如是}{か|う}したものか
\ruby{知}{し}らねど、
%
\ruby{餘}{あま}りといへば
\ruby{曲}{きよく}の
\ruby{無}{な}い
\ruby{何}{なん}といふ
\ruby{此室}{この|ま}の
\ruby{狀態}{さ|ま}と、
%
ひそかに
\ruby{室主}{ある|じ}を
\ruby{疎}{うと}ましく
\ruby{思}{おも}へる
\ruby{折}{をり}しも、
%
\ruby{此家}{こ|ゝ}の
\ruby{娘}{むすめ}が
\ruby{我}{われ}を
\ruby{可厭}{い|や}な
\ruby{人}{ひと}と
\ruby{云}{い}ひしに
\ruby{對}{むか}ひて、
%
\ruby{我}{われ}を
\ruby{優}{やさ}しき
\ruby{人}{ひと}と
\ruby{云}{い}ひなし
\ruby{吳}{く}れたるを
\ruby{聞}{き}きて
\ruby{憎}{にく}く
\ruby{思}{おも}はんやうは
\ruby{無}{な}く、
%
あ〻まだ
\ruby{知}{し}りもせぬ
\ruby{人}{ひと}を
\ruby{惡}{わる}くばかり
\ruby{量}{つも}つたる
\ruby{事}{こと}と
\ruby{思}{おも}ひ
\ruby{{\換字{返}}}{かへ}す
\ruby{時}{とき}、
%
\ruby{無{\換字{造}}作}{む|ざう|さ}にすらりと
\ruby{間}{あひ}の
\ruby{襖}{ふすま}を
\ruby{開}{あ}けて、
%
\ruby{次}{つぎ}の
\ruby{室}{ま}より
\ruby{立出}{たち|い}でたる
\ruby{男}{をとこ}は
\ruby{我}{わ}が
\ruby{{\換字{前}}}{まへ}に
\ruby{座}{すは}れり。
