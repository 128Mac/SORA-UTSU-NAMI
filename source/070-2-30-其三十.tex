\Entry{其三十}

% メモ 校正終了 2024-04-25 2024-06-02
\原本頁{164-3}%
お
\ruby{龍}{りう}は
\ruby{抑}{そも}
\ruby{如何}{い|か}なる
\ruby{人}{ひと}ぞや。

\原本頁{164-4}%
お
\ruby[||j>]{孃}{ぢやう}
\ruby[||j>]{樣}{ さま}
お
\ruby[||j>]{孃}{ぢやう}
\ruby[||j>]{樣}{ さま}
で
\ruby{育}{そだ}てられたる
\ruby{身}{み}には
あらねど、
%
\ruby{生}{うま}れついての
\ruby[<j||]{心}{こゝろ}% 踊り字調整「〻(二の字点、揺すり点)に見えるが(ゝ)」
\ruby{{\換字{情}}}{もち}に% 踊り字調整「〻(二の字点、揺すり点)に見えるが(ゝ)」
\ruby{人}{ひと}とは
\ruby{異}{かは}つたる
ところ
あつて、
%
\ruby{駿府}{すん|ぷ}の
\ruby{叔母}{を|ば}の
ところへ
\ruby{引取}{ひき|と}られたる
\ruby{其}{その}
\ruby{夜}{よ}、
%
はじめて
\ruby{何}{なに}も
\ruby{無}{な}き
\ruby{座敷}{ざ|しき}に
\ruby{寐}{ね}かされて、
%
\ruby{吾家}{う|ち}では
\原本頁{164-7}\改行%
\ruby{如是}{か|う}は
\ruby{無}{な}かつた
ものをと
\ruby{物}{もの}
\ruby{足}{た}らぬ
\ruby{心地}{こゝ|ち}し、% 踊り字調整「〻(二の字点、揺すり点)に見えるが(ゝ)」
%
\ruby{{\換字{翌}}日}{あくる|ひ}
\ruby{我}{わ}が
\ruby{荷物}{に|もつ}の
\ruby{行李}{こう|り}を
\ruby{解}{と}きし
\ruby{次}{ついで}に、
%
\ruby{我}{わ}が
\ruby{好}{す}きな
ものゝ% 踊り字調整「〻(二の字点、揺すり点)に見えるが(ゝ)」
\ruby{數}{かず}
\ruby{多}{おほ}き
\ruby{中}{なか}より
\ruby{{\換字{平}}生}{ひ|ごろ}% ルビ調整(原本通り)
\ruby{氣}{き}に
\ruby{入}{い}りの
\原本頁{164-9}\改行%
\ruby{永徳齋}{{\換字{𛀁}}い|とく|さい}の
\ruby{小}{こ}
\ruby[||j>]{人}{にん}
\ruby[||j>]{形}{ぎやう}を
\ruby{取}{と}り
\ruby{出}{いだ}して、
%
そつと
\ruby{小棚}{こ|だな}に
\ruby{{\換字{飾}}}{かざ}り
\ruby{置}{お}きしに、
%
それを
\ruby{固}{かた}い
\ruby{自慢}{じ|まん}の
\ruby{叔母}{を|ば}の
\ruby[||j>]{帝}{たい}
\ruby[||j>]{釋}{しやく}
\ruby[||j>]{樣}{ さま}
のやうな
\ruby{三角}{さん|かく}の
\ruby{眼}{め}に
\ruby{睨}{にら}まれて、
%
\ruby{其}{そ}
\原本頁{165-1}\改行%
\ruby{樣}{ん}な
\ruby{大}{おほ}きい
\ruby{形體}{な|り}を
して
\ruby[||j>]{人}{にん}
\ruby[||j>]{形}{ぎやう}
% \ruby{人形}{にん|ぎやう}
なんぞを
\ruby{捏}{こ}ね
\ruby{{\換字{廻}}}{まは}して
\ruby{{\換字{遊}}}{あそ}ぶと
\ruby{云}{い}ふ
\ruby{事}{こと}が
ありますか、
%
\ruby{藏}{しま}つて
\ruby{御置}{お|お}きなさい、
%
\ruby{見}{み}つとも
\ruby{無}{な}い、
%
と
\ruby{唯}{たゞ}% 踊り字調整「〻(二の字点、揺すり点)に濁点に見えるが(ゞ)」
\ruby{一言}{ひと|こと}に
\原本頁{165-3}\改行%
\ruby{叱}{しか}り
つけられ、
%
あゝ% 踊り字調整「〻(二の字点、揺すり点)に見えるが(ゝ)」
あんまり
つまらない
\ruby[||j>]{{\換字{情}}}{なさけ}
\ruby[||j>]{無}{ な}い
\ruby{叔母樣}{を|ば|さん}、
%
\ruby{何樣}{ど|う}すれば
\ruby{其樣}{そ|ん}な
\ruby{乾魚}{ひ|もの}の
やうな
\ruby{氣}{き}になつて
\ruby{居}{ゐ}らるゝ% 踊り字調整「〻(二の字点、揺すり点)に見えるが(ゝ)」
\ruby{事}{こと}かと、
%
\ruby{恨}{うら}み
\ruby{疑}{うたが}ひ
ながらも
\ruby{爭}{あらそ}ひ
かねて、
%
\ruby{其}{その}
\ruby{時}{とき}
より
やうやく
『
わたしの
\ruby{好}{すき}な
\ruby{物}{もの}
』
を
\ruby{身}{み}の% 「『...』」の影響か、「身」まで29文字あり
\ruby{傍}{ほとり}に
\ruby{置}{お}かずして
\ruby{日}{ひ}を
\ruby{{\換字{送}}}{おく}るに
\ruby{慣}{な}るゝに% 踊り字調整「〻(二の字点、揺すり点)に見えるが(ゝ)」
\ruby{至}{いた}り
たるなり。

\原本頁{165-7}%
されば
\ruby{頼}{たの}もし
からぬ
\ruby{男}{をとこ}に
\ruby[||j>]{一}{いつ}
\ruby[||j>]{生}{しやう}を
% \ruby{一生}{いつ|しやう}を
\ruby{{\換字{過}}}{あやま}られて、
%
\ruby{涙}{なみだ}の
\ruby{淵瀬}{ふち|せ}に
\ruby{{\換字{浮}}}{う}き
\ruby{沈}{しづ}み
したる
\ruby{後}{のち}、
%
\ruby{今}{いま}は
\ruby{他人}{ひ|と}の
\ruby{家}{いへ}に
\ruby{寄食客}{かゝ|り|びと}の% 踊り字調整「〻(二の字点、揺すり点)に見えるが(ゝ)」
\ruby{身}{み}の
\ruby{長閑}{のど|か}
らしく
\ruby{玩弄品}{おも|ち|や}
\ruby{三昧}{ざん|まい}を
するとには
あらねど、
%
\ruby{傳}{でん}といひ、
%
\ruby{淸}{せい}といひ、
%
\ruby{{\換字{勝}}}{かつ}といひ、
%
\ruby{彦}{ひこ}といひ、
%
\ruby{出入}{で|はい}る
\ruby{{\換字{若}}}{わか}き
\ruby[||j>]{男}{をとこ}
\ruby[||j>]{共}{ ども}の
\ruby{爭}{あらそ}つて
\ruby{氣}{き}を
\ruby{取}{と}らん
とて、
%
\ruby{折々}{をり|〳〵}
\ruby{吳}{く}れたる
\原本頁{165-11}\改行%
\ruby{種々}{いろ|〳〵}の
\ruby{物品}{も|の}の
\ruby{中}{うち}、
%
\ruby{傳}{でん}が
\ruby{持}{も}て
\ruby{來}{きた}れる
\ruby{薄色}{うす|いろ}の
\ruby{瑪瑙}{め|なう}の
\ruby{細工}{さい|く}の
\ruby{小}{ちひさ}き
\ruby{兎}{うさぎ}の
\改行% 校正作業の簡略化のため
、
%
\原本頁{166-1}\改行%
\ruby[||j>]{姿}{すがた}
しほらしく
ふつくり
として、
%
ぽつちりと
\ruby{紅}{あか}き
\ruby{眼}{め}の
いと
\ruby{可憐}{かは|ゆ}く
\ruby{出來}{で|き}たるが
\ruby{甚}{いた}く
\ruby{氣}{き}に
\ruby{入}{い}り、
あれか
これか
と、
%
アナ
\ruby{絲}{いと}の
\ruby{色}{いろ}を
\ruby{擇}{{\換字{𛀁}}ら}みに
\ruby{擇}{{\換字{𛀁}}ら}んで、
%
\ruby{其}{そ}のために
\ruby{敷}{し}くべき
\ruby{蒲團}{ふ|とん}の
\ruby{花}{はな}やかに
\ruby{美}{うつく}しきを
\ruby{{\換字{編}}}{あ}みて
\原本頁{166-4}\改行%
\ruby{{\換字{遣}}}{や}りつ、
%
はじめて
\ruby{其}{それ}に
\ruby{載}{の}せて
\ruby{見}{み}たる
\ruby{時}{とき}、
%
\ruby{色}{いろ}の
\ruby{映}{うつ}り
\ruby{合}{あ}ひて
いよいよ% ルビ調整(原本通り)非踊り字表記(行末行頭の境界付近)
\ruby{好}{この}ましく
\ruby{愛}{あい}らしく
\ruby{見}{み}えたる
\ruby{嬉}{うれ}しさの
\ruby{餘}{あま}りの
\ruby{戱}{たはむ}れに、
%
\ruby{此兎}{こ|れ}は
\ruby[<j||]{妾}{わたし}の
\ruby{大切}{だい|じ}な
\ruby{人}{ひと}なの!\inhibitglue{}%
と
\ruby[||j>]{獨}{ひとり}
\ruby[||j>]{語}{ ごと}
したりしが、
%
\ruby{其語}{そ|れ}を
\ruby{人}{ひと}より
\ruby{聞}{き}きて
\ruby{勘{\換字{違}}}{かん|ちが}ひしてか、
%
\ruby{其}{その}
\ruby{頃}{ころ}より
\ruby{傳}{でん}の
\ruby{煩}{うるさ}く
\ruby{付}{つ}き
\ruby{纏}{まと}ふ、
%
\ruby{其}{それ}は
\ruby{何}{なに}よりの
\ruby{{\換字{迷}}惑}{めい|わく}
ながら、
%
\ruby{今}{いま}だに
\ruby{兎}{うさぎ}の
\ruby{可愛}{かは|ゆ}さは
\ruby{冷}{さ}めず、
%
\ruby{何}{なん}ぞの
\ruby{折}{をり}には
『
\ruby{兎之}{う|の}さん
』と
\ruby{喚}{よ}び% 途中に「『...』」があるため文字の高さがう通常と異なる
かけて、
%
\ruby{心}{こゝろ}の% 踊り字調整「〻(二の字点、揺すり点)に見えるが(ゝ)」
\ruby{淋}{さび}しさ
\ruby{{\換字{遣}}}{や}る
\ruby{方}{かた}
\ruby{無}{な}き
\ruby{時}{とき}の、
%
\ruby{語}{かた}らう
\ruby{友}{とも}
\ruby{無}{な}き
\ruby{孤獨}{ひと|りみ}の
\ruby{憂}{う}さを、
%
\ruby{苟且}{かり|そめ}に
\ruby{一寸}{ちよ|つと}
\ruby[||j>]{慰}{なぐさ}め
\ruby{忘}{わす}るゝなり。% 踊り字調整「〻(二の字点、揺すり点)に見えるが(ゝ)」

\原本頁{166-11}%
\ruby{是}{かく}の
\ruby{如}{ごと}き
お
\ruby{龍}{りう}は
\ruby{今}{いま}
\ruby{一室}{いつ|しつ}の
\ruby{中}{うち}に、
%
\ruby{眼}{め}を
\ruby{慰}{なぐさ}め
\ruby{心}{こゝろ}を% 踊り字調整「〻(二の字点、揺すり点)に見えるが(ゝ)」
\ruby{寄}{よ}せて
\ruby{{\換字{情}}懷}{おも|ひ}の
\ruby{{\換字{遣}}}{や}り
どころと
すべき
\ruby{物}{もの}の
\ruby{一}{ひと}つも
\ruby{無}{な}くて、
%
\ruby{床}{とこ}に
\ruby{插花瓶}{さし|ばな|いけ}は
\ruby{有}{あ}りながら
\ruby{末枯}{す|が}れたる
\ruby{花}{はな}も
\ruby{無}{な}く、
%
\ruby{机上}{つく|ゑ}に
\ruby{筆架}{ふで|かけ}
\ruby{水滴}{みづ|いれ}の
\ruby{影}{かげ}も
あらで
\ruby[<j||]{裸}{はだか}% 踊り字調整「〻(二の字点、揺すり点)に濁点に見えるが(ゞ)」
\ruby[||j>]{硯}{すゞり}の% 踊り字調整「〻(二の字点、揺すり点)に濁点に見えるが(ゞ)」
\ruby{淋}{さび}しく
\原本頁{167-3}\改行%
\ruby{置}{お}かれたる
ものなるを
\ruby{見}{み}て、
%
\ruby{成程}{なる|ほど}
\ruby{書生}{しよ|せい}さんは
\ruby{如是}{か|う}
したものか
\ruby{知}{し}らねど、
%
\ruby{餘}{あま}り
といへば
\ruby{曲}{きよく}の
\ruby{無}{な}い
\ruby{何}{なん}といふ
\ruby{此室}{この|ま}の
\ruby{狀態}{さ|ま}と、
%
ひそかに
\ruby{室主}{ある|じ}を
\ruby{疎}{うと}ましく
\ruby{思}{おも}へる
\ruby{折}{をり}しも、
%
\ruby{此家}{こ|ゝ}の% 踊り字調整「〻(二の字点、揺すり点)に見えるが(ゝ)」
\ruby{娘}{むすめ}が
\ruby{我}{われ}を
\ruby{可厭}{い|や}な
\ruby{人}{ひと}と
\ruby{云}{い}ひしに
\ruby{對}{むか}ひて、
%
\ruby{我}{われ}を
\ruby{優}{やさ}しき
\ruby{人}{ひと}と
\ruby{云}{い}ひなし
\ruby{吳}{く}れたるを
\ruby{聞}{き}きて
\ruby{憎}{にく}く
\原本頁{167-7}\改行%
\ruby{思}{おも}はん
やうは
\ruby{無}{な}く、
%
あゝ% 踊り字調整「〻(二の字点、揺すり点)に見えるが(ゝ)」
まだ
\ruby{知}{し}りも
せぬ
\ruby{人}{ひと}を
\ruby{惡}{わる}く
ばかり
\ruby{量}{つも}つたる
\ruby{事}{こと}と
\ruby{思}{おも}ひ
\ruby{{\換字{返}}}{かへ}す
\ruby{時}{とき}、
%
\ruby{無{\換字{造}}作}{む|ざう|さ}に
すらりと
\ruby{間}{あひ}の
\ruby{襖}{ふすま}を
\ruby{開}{あ}けて
%
\ruby{次}{つぎ}の
\ruby{室}{ま}より
\ruby{立出}{たち|い}でたる
\ruby{男}{をとこ}は
\ruby{我}{わ}が
\ruby{{\換字{前}}}{まへ}に
\ruby{座}{すは}れり。
