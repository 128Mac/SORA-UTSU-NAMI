\Entry{其三十一}

% メモ 校正終了 2024-04-26 2024-06-03
\原本頁{168-1}%
お
\ruby{龍}{りう}が
\ruby{頭}{かうべ}を
\ruby{下}{さ}げて
\ruby{禮}{れい}を
なしつ、
%
やがて
\ruby{言}{ものい}ひ
\ruby{出}{い}でん
とする
\ruby{間}{ま}も
あらせす、

\原本頁{168-3}%
『
イヤ
お
\ruby{待}{また}せ
\ruby{申}{まをし}ました、
%
\ruby[g]{小生}{わたくし}は
\ruby[g]{水野}{みづの }です。
』

\原本頁{168-4}%
と
\ruby{云}{い}ひたる、
%
\ruby[g]{言語}{げんご }
\ruby[g]{明晰}{はつきり}として
\ruby[g]{冗處}{む だ }も
\ruby{無}{な}く
\ruby[g]{餘裕}{ゆとり }も
\ruby{無}{な}く、
%
\ruby{石}{いし}
\ruby[||j>]{甃}{だゝみ}を% 踊り字調整「〻(二の字点、揺すり点)に見えるが(ゝ)」
\ruby{見}{み}るやうに
\ruby{角}{かく}ばつたる
\ruby{云}{い}ひ
ざまの、
%
\ruby{聲}{こゑ}つき
\ruby[||j>]{自}{おのづ}
\ruby[||j>]{然}{ から}
\ruby[||j>]{威}{ いき}
\ruby[||j>]{勢}{ ほひ}あるに
お
\ruby{龍}{りう}は
\ruby{吞}{の}まれて、
%
\ruby[g]{釣{\換字{込}}}{つりこ }まれ
\ruby[g]{氣味}{ぎ み }に
\ruby[g]{此方}{こなた }も% ルビ調整(原本通り)
\ruby{堅}{かた}くなり、

\原本頁{168-7}%
『
あの
\ruby[<j>]{妾}{わたくし}は
\ruby[g]{岩崎}{いはざき}の
\ruby{母}{はゝ}の% 踊り字調整「〻(二の字点、揺すり点)に見えるが(ゝ)」
ところから
\ruby{出}{で}ましたもので、
』

\原本頁{168-8}%
と、
%
\ruby{先}{ま}づ
\ruby[g]{一句}{いつく }
\ruby{明}{あき}らかに
\ruby[g]{那處}{いづく }より
\ruby{來}{きた}れるかを
\ruby{{\換字{更}}}{さら}に
\ruby{告}{つ}げたり。

\原本頁{168-9}%
『
ハア。
%
\ruby[g]{左樣}{さ う }して
\ruby[g]{貴下}{あなた }は
\ruby{御{\換字{近}}{\換字{所}}}{ご|きん|じよ}の
\ruby{方}{かた}でゝも% 踊り字調整「〻(二の字点、揺すり点)に見えるが(ゝ)」
お
\ruby{有}{あ}りですか。
』

\原本頁{168-10}%
『
ハイ、
%
イエ、
%
\ruby{御}{ご}
\ruby[g]{承知}{しようち}は
ございますまいが
\ruby[<j>]{妾}{わたくし}は
あの、
%
\ruby[g]{彼方}{あちら }に
\ruby{御厄介}{ご|やく|かい}に
なつて
\ruby{居}{を}る
もので
ございまして、
%
\ruby{舊}{もと}は
\ruby[g]{彼方}{あちら }で
お
\ruby[g]{稽{\換字{古}}}{けいこ }を
\ruby{願}{ねが}つた
もので
ございます。
』

\原本頁{169-2}%
『
アヽ
\ruby[g]{左樣}{さ う }ですか、
%
して
お
\ruby[g]{師匠}{し よ }さんは
お
\ruby{變}{かは}り
もありませんか。
』

\原本頁{169-3}%
\ruby[g]{師匠}{しゝやう}は% 踊り字調整「〻(二の字点、揺すり点)に見えるが(ゝ)」
\ruby{打}{うち}
\ruby{擲}{たゝ}いても% 踊り字調整「〻(二の字点、揺すり点)に見えるが(ゝ)」
\ruby{死}{し}なざるべく
\ruby[g]{壯健}{じやうぶ}にして、
%
\ruby{酒}{さけ}を
\ruby{飮}{の}み
\ruby[g]{{\換字{情}}夫}{をとこ }と
\ruby{{\換字{連}}}{つ}れ
\ruby{立}{だ}ちて
\ruby{{\換字{遊}}}{あそ}び
\ruby{歩}{ある}けるものを、
%
かゝる% 踊り字調整「〻(二の字点、揺すり点)に見えるが(ゝ)」
\ruby{生眞面目}{き|ま|じ|め}なる
\ruby{人}{ひと}に
\ruby[g]{虛言}{う そ }を
\ruby{云}{い}ふことの
\ruby[<j||]{心}{こゝろ}% 踊り字調整「〻(二の字点、揺すり点)に見えるが(ゝ)」
\ruby[g]{咎せ}{とがめ }られぬ
には
あらざれど、

\原本頁{169-6}%
『
ハイ
\ruby{有}{あ}りがたう
ございます。
%
まあ
\ruby[g]{別條}{べつでう}は
\ruby{無}{な}い
やうな
もので
ございますが、
%
\ruby{先}{この}
\ruby[||j>]{般}{あひだ}から
\ruby[g]{一寸}{ちよつと}
\ruby[g]{時候}{じ こう}
あたりを
\ruby{致}{いた}して
\ruby{{\換字{弱}}}{よわ}つて
\ruby{居}{を}りますので。
』

\原本頁{169-9}%
と
\ruby{已}{やむ}を
\ruby{得}{{\換字{𛀁}}}ず
\ruby{豫}{かね}ての
\ruby[g]{命令}{いひつけ}を
\ruby{{\換字{終}}}{つひ}に
\ruby{果}{はた}したり。

\原本頁{169-10}%
『
それは
\ruby[g]{何樣}{ど う }も
いけませんナ、
%
たゞの% 踊り字調整「〻(二の字点、揺すり点)に濁点に見えるが(ゞ)」
\ruby[g]{風邪}{か ぜ }ですか。
』

\原本頁{169-11}%
『
イエ
もう、
%
\ruby{眞}{ほん}の
\ruby[g]{一寸}{ちよつと}した
\ruby{事}{こと}で
ございまして、
%
しかも
\ruby{治}{なほ}り
\ruby[g]{加減}{か げん}で
ございますから、
%
お
\ruby{案}{あん}じ
\ruby{下}{くだ}さい
ますな。
%
それに
\ruby{就}{つ}きまし
\ruby[<g||]{て妾}{わたくし}が% 行末行頭の境界付近なので特例処置を施す「(て)を親文字に入れた」
\ruby{出}{で}ました
やうな
\ruby{譯}{わけ}で
ございますが、
%
\ruby[g]{師匠}{しゝやう}が% 踊り字調整「〻(二の字点、揺すり点)に見えるが(ゝ)」
\ruby{申}{まを}しますには、
%
\ruby{{\換字{過}}}{この}
\ruby[||j>]{般}{あひだ}からは
また
\ruby[g]{度々}{たび〳〵}の
お
\ruby[g]{手紙}{て がみ}で、
%
\ruby[g]{五十}{い そ }の
\ruby[g]{病氣}{びやうき}を
\ruby[g]{一々}{いち〳〵}
お
\ruby{知}{し}らせ
\ruby{下}{くだ}さつたり、
%
\ruby{其}{その}
\ruby{上}{うへ}
また
いろ〳〵
お
\ruby[g]{世話}{せ わ }を
\ruby{戴}{いたゞ}いたり% 踊り字調整「〻(二の字点、揺すり点)に濁点に見えるが(ゞ)」
しまして、
%
お
\ruby{禮}{れい}を
\原本頁{170-5}\改行%
\ruby{申}{まを}さうやうも
\ruby{無}{な}く
\ruby{有}{あ}り
\ruby{{\換字{難}}}{がた}く
\ruby{存}{ぞん}じて
\ruby{居}{を}りまする。
%
\ruby[g]{早{\換字{速}}}{さつそく}にも
\ruby[g]{自{\換字{分}}}{じ ぶん}で
\ruby{出}{で}て
お
\ruby{禮}{れい}を
\ruby[||j>]{申}{まをし}
\ruby[||j>]{上}{ あ}げ、
%
\ruby[g]{五十}{い そ }の
\ruby[g]{見舞}{み まひ}も
\ruby{看}{かん}
\ruby[||j>]{病}{びやう}も
\ruby{致}{いた}さなくつては
ならないので
ございますが、
%
\ruby[g]{生憎}{あいにく}と% ルビ調整(原本通り)(あいにく)
\ruby[g]{自{\換字{分}}}{じ ぶん}も
\ruby{患}{わづら}つて
\ruby{居}{を}りまするので、
%
\ruby{存}{ぞん}じながら
\ruby{思}{おも}ふ
やうにも
\ruby{參}{まゐ}りません。
%
\ruby[g]{水野}{みづの }さんが
\ruby{在}{い}らしつて
\ruby{下}{くだ}さるから
\ruby{好}{い}いはで
もつて
\ruby[g]{打棄}{うつちや}つて
\ruby{居}{を}るやうで、
%
\ruby[g]{大變}{たいへん}
\ruby[||j>]{心}{こゝろ}% 踊り字調整「〻(二の字点、揺すり点)に見えるが(ゝ)」
\ruby[||j>]{苦}{ ぐる}しう
\ruby{存}{ぞん}じて
\ruby{居}{を}るので
ございますが、
%
\ruby{全}{まつた}く
\ruby[g]{左樣}{さ う }いふ
\ruby{譯}{わけ}では
ございません。
%
\原本頁{170-11}\改行%
\ruby{御承知}{ご|しよう|ち}の
\ruby{{\換字{通}}}{とほ}りの
\ruby[||j>]{女}{をんな}
\ruby[||j>]{暮}{ ぐら}しで、
%
\ruby[g]{手{\換字{前}}}{て まへ}に
ばかり
かまけて
\ruby{居}{を}りまするので、
%
\ruby[g]{彼樣}{あ ゝ }も% 踊り字調整「〻(二の字点、揺すり点)に見えるが(ゝ)」
\ruby{仕}{し}たい、
%
\ruby[g]{此樣}{こ う }も
\ruby{仕}{し}たい
と
\ruby[g]{色々}{いろ〳〵}に、
%
\ruby{心}{こゝろ}では% 踊り字調整「〻(二の字点、揺すり点)に見えるが(ゝ)」
\ruby{思}{おも}つて
\ruby{居}{を}りましても
\ruby{手}{て}が
\ruby{屆}{とゞ}きませんから、% 「屆」「届」 原本通り「屆」% 踊り字調整「〻(二の字点、揺すり点)に濁点に見えるが(ゞ)」
%
たゞ% 踊り字調整「〻(二の字点、揺すり点)に濁点に見えるが(ゞ)」
\ruby{蔭}{かげ}で
もつて
\ruby{神信心}{かみ|しん|〴〵}
ばかり
\ruby{致}{いた}して
\ruby{居}{を}るやうな
\ruby{譯}{わけ}で
ございます!。
%
と
\ruby[g]{如是}{か う }
\ruby[g]{申上}{まをしあ}げて、
%
\ruby[g]{何樣}{ど う }か
\ruby[g]{何{\換字{分}}}{なにぶん}にも
\ruby{惡}{あ}しからず
\ruby[||j>]{思}{おぼし}
\ruby[||j>]{召}{ めし}になるやうに、
% \ruby{思召}{おぼし|めし}になるやうに、
%
\ruby{善}{よ}く
\ruby{汝}{おまへ}から
\ruby[g]{有體}{ありてい}の
ところ
\原本頁{171-5}\改行%
を
\ruby{細}{こまか}に
お
\ruby[g]{話仕}{はなしゝ}て% 踊り字調整「〻(二の字点、揺すり点)に見えるが(ゝ)」
お
\ruby{吳}{く}れとの
\ruby{事}{こと}に
ございまする。
%
\ruby{{\換字{又}}}{また}、
%
どうか
\ruby{此}{この}
\ruby{上}{うへ}とも
お
\ruby[g]{世話}{せ わ }を
\ruby{下}{くだ}さいますように、
%
\ruby[g]{老母}{ばゝあ }は% 踊り字調整「〻(二の字点、揺すり点)に見えるが(ゝ)」
\ruby[g]{{\換字{勝}}手}{かつて }な
\ruby{奴}{やつ}だ
\ruby{顏}{かほ}も
\ruby{出}{だ}さないと、
%
お
\ruby{愛想盡}{あい|そ|づか}し
になりましても、
%
\ruby[||j>]{病}{びやう}
\ruby[||j>]{人}{ にん}は
\ruby{何}{なに}も
\ruby{知}{し}らない
\ruby{事}{こと}で
ございますから、
%
お
\ruby{愛想盡}{あい|そ|づか}し
を
なさらない
やうに。
%
\ruby[g]{五十}{い そ }の
\ruby{事}{こと}は
\ruby{實}{じつ}は
\ruby[g]{我儘}{わがまゝ}な% 踊り字調整「〻(二の字点、揺すり点)に見えるが(ゝ)」
\ruby{申}{まを}し
\ruby{樣}{やう}ですが、
%
\ruby{疾}{とう}から
\ruby[g]{貴下}{あなた }に
お
\ruby{任}{まか}せ
\ruby{申}{まを}した
つもりで
\ruby{居}{を}りまする
ので
ございます
から、
%
\ruby{何}{ど}のやう
にでも
お
\ruby[||j>]{心}{こゝろ}% 踊り字調整「〻(二の字点、揺すり点)に見えるが(ゝ)」
\ruby[||j>]{持}{ もち}
\ruby[||j>]{次第}{ し|だい}
になすつて
\ruby{戴}{いたゞ}きたい% 踊り字調整「〻(二の字点、揺すり点)に濁点に見えるが(ゞ)」
ので、
%
\ruby{御親切}{ご|しん|せつ}の
\ruby[g]{貴下}{あなた }の
お
\ruby[g]{世話}{せ わ }を
\ruby{戴}{いたゞ}いて、% 踊り字調整「〻(二の字点、揺すり点)に濁点に見えるが(ゞ)」
%
\ruby{其}{それ}
でいけなければ
\ruby{殘}{のこ}り
\ruby{惜}{をし}い
\ruby{事}{こと}は
ございません、
%
\ruby{全}{まつた}く
\ruby[g]{當人}{たうにん}の
\ruby{{\換字{運}}}{うん}の
\ruby{無}{な}いのだと
\ruby{諦}{あき}らめます。
%
いづれ
\ruby{其}{その}
\ruby{中}{うち}には
\ruby[g]{是非}{ぜ ひ }とも
\ruby{伺}{うかゞ}つて% 踊り字調整「〻(二の字点、揺すり点)に濁点に見えるが(ゞ)」
お
\ruby{禮}{れい}を
\ruby{申}{まを}す
つもりで
ございます。
%
\ruby[||j>]{汝}{おまへ}
\ruby[||j>]{彼}{ あ}
\ruby[||j>]{方}{ちら}
\ruby[||j>]{樣}{さま}へ
\ruby{上}{あが}つたら、
%
\ruby[g]{何樣}{ど う }か
\ruby{妾}{わたし}が
\ruby[g]{如是}{か う }いふ
\原本頁{172-4}\改行%
\ruby[||j>]{心}{こゝろ}% 踊り字調整「〻(二の字点、揺すり点)に見えるが(ゝ)」
\ruby{持}{ もち}を%
\ruby{有}{も}つて
\ruby{居}{を}ります
といふ
\ruby{事}{こと}を
\ruby{云}{い}つて、
%
\ruby[g]{十{\換字{分}}}{じふぶん}に
お
\ruby{禮}{れい}を
\ruby[g]{申上}{まをしあ}げて、
%
\ruby{而}{そ}して
\ruby[g]{五十}{い そ }の
\ruby[g]{病氣}{びやうき}の
\ruby[g]{樣子}{やうす }も
\ruby{伺}{うかゞ}つて% 踊り字調整「〻(二の字点、揺すり点)に濁点に見えるが(ゞ)」
\ruby{來}{き}て
お
\ruby{吳}{く}れ、
%
と
\ruby[g]{斯樣}{か やう}に
\ruby{申}{まを}す
ので
ございます。
%
それで
お
\ruby[g]{馴染}{なじみ }も% 「{馴染}{なじみ}」だと思うが原本通り
\ruby{無}{な}い
\ruby[<j>]{妾}{わたくし}
では
ございますが、
%
\原本頁{172-7}\改行%
\ruby{他}{ほか}に
\ruby{參}{まゐ}るものも
\ruby{無}{な}い
ので
ございます
から、
%
\ruby[g]{一寸}{ちよつと}
\ruby{上}{あが}つた
ので
ございます。
』

\原本頁{172-9}%
お
\ruby{龍}{りう}は
\ruby{果}{はた}さでは
\ruby{叶}{かな}はぬ
\ruby[g]{使者}{つかひ }の
\ruby[g]{役目}{やくめ }を
\ruby{務}{つと}め
\ruby{果}{おほ}せん
\ruby[g]{一心}{いつしん}に、
%
\ruby[<j||]{一}{いつ }% 行末行頭の境界付近なので特例処置を施す
\ruby[<j||]{生}{しやう}
\ruby[g]{懸命}{けんめい}になりて
\ruby[g]{如是}{か く }
\ruby{{\換字{述}}}{の}べ
\ruby{{\換字{終}}}{をは}りしが、
%
\ruby{辛}{から}くも
\ruby[g]{吩咐}{いひつ }け% 吩咐 ... 言いつける、指図する
られし
だけは
\ruby{云}{い}ひ
\原本頁{172-11}\改行%
\ruby{得}{{\換字{𛀁}}}たるに
ホツと
\ruby[g]{氣息}{い き }
\ruby{吐}{つ}きて、
%
\ruby{男}{をとこ}の
\ruby[g]{樣子}{やうす }を
\ruby[g]{如何}{い か }にと
\ruby{見}{み}れば、
%
\ruby{男}{をとこ}は
\原本頁{173-1}\改行%
\ruby[g]{律義}{りちぎ }
\ruby[||j>]{眞}{まつ}
\ruby[||j>]{正}{しやう}
\ruby[||j>]{直}{ ぢき}に
\ruby{物}{もの}
\ruby{堅}{がた}く
\ruby{愼}{つゝし}みて% 踊り字調整「〻(二の字点、揺すり点)に見えるが(ゝ)」
\ruby{耳}{みゝ}を% 踊り字調整「〻(二の字点、揺すり点)に見えるが(ゝ)」
\ruby{傾}{かたむ}け、
%
\ruby{見}{み}す〳〵の
\ruby{我}{わ}が
\ruby[g]{虛言}{う そ }を
\ruby{實}{げ}に
\ruby[g]{{\換字{道}}理}{もつとも}と
\ruby{聞}{き}ける
やうなるに、
%
\ruby{此}{こ}のやうなる
\ruby{人}{ひと}を
\ruby[g]{口頭}{くちさき}に
\ruby{操}{あやつ}るはと
\改行% 校正作業の簡略化のため
、
%
\原本頁{173-3}\改行%
\ruby{我}{われ}
\ruby[||j>]{羞}{はづか}しき
\ruby[g]{心地}{こゝち }の% 踊り字調整「〻(二の字点、揺すり点)に見えるが(ゝ)」
\ruby{爲}{し}たり。

\原本頁{173-4}%
『
ハイ、
%
\ruby[g]{一々}{いち〳〵}
\ruby{精}{よ}く
\ruby{解}{わか}りました、
%
\ruby[g]{承知}{しようち}
\ruby{致}{いた}しました。
%
お
\ruby[g]{言葉}{ことば }が
\ruby{無}{な}くて
さへ
いろ〳〵に
\ruby[g]{心配}{しんぱい}は
\ruby{致}{いた}して
\ruby{居}{を}りました
のですから、
%
\ruby[g]{其樣}{そ う }いふ
お
\ruby[g]{言葉}{ことば }を
\ruby{伺}{うかゞ}ひます% 踊り字調整「〻(二の字点、揺すり点)に濁点に見えるが(ゞ)」
\ruby{上}{うへ}は
\ruby{{\換字{猶}}}{なほ}の
\ruby{事}{こと}で
ございます。
%
\ruby[g]{水野}{みづの }が
\ruby[g]{出來}{で き }まする
だけの
\ruby{事}{こと}は
\ruby{致}{いた}しますから、
%
\ruby{五十子}{い|そ|こ}さんの
\ruby{事}{こと}は
お
\ruby[<j||]{心}{こゝろ}% 踊り字調整「〻(二の字点、揺すり点)に見えるが(ゝ)」
\ruby[g]{{\換字{遣}}無}{づかひな}く、
%
よく
\ruby{御}{ご}
\ruby{養}{やう}
\ruby[||j>]{生}{じやう}をなすつて
\ruby{早}{はや}く
\ruby{御}{ご}
\ruby{全}{ぜん}
\ruby[||j>]{快}{くわい}なさる
やうにと
\ruby{仰}{おつし}あつて
\ruby{下}{くだ}さいまし。
%
\ruby{五十子}{い|そ|こ}さんは
\ruby[<j||]{必}{かなら}ず
\ruby[<j>]{私}{わたくし}が
\ruby{癒}{なほ}らせます。
%
\ruby[g]{何樣}{ど う }しても
\ruby[g]{一度}{いちど }は
\ruby[g]{屹度}{きつと }
\ruby{癒}{なほ}らせますと
\ruby[g]{小生}{わたくし}が
\ruby{申}{まを}したと
\ruby{仰}{おつし}あつて
\ruby{下}{くだ}さいまし。
』

\原本頁{173-11}%
\ruby{人}{ひと}の
\ruby{命}{いのち}は
\ruby{知}{し}る
べからざるを、
%
あゝ% 踊り字調整「〻(二の字点、揺すり点)に見えるが(ゝ)」
\ruby{何}{なん}ぞ
\ruby{其}{その}
\ruby[g]{言葉}{ことば }の
\ruby[g]{男兒}{をとこ }らしく
\ruby{頼}{たの}も
\原本頁{174-1}\改行%
しきや。
%
\ruby{聲}{こゑ}の
\ruby{大}{おほき}く
なりたるも
\ruby{思}{おも}はず
\ruby[g]{誠意}{まこと }の
\ruby{籠}{こも}り
たれば
なるべし
\改行% 校正作業の簡略化のため
。
%
\原本頁{174-2}\改行%
\ruby[g]{如斯}{か く }
\ruby{云}{い}へる
\ruby{其}{そ}の
\ruby[g]{言葉}{ことば }の
\ruby{力}{ちから}
あるに
\ruby{驚}{おどろ}かされて、
%
お
\ruby{龍}{りう}は
\ruby{今}{いま}
\ruby{{\換字{又}}}{また}
\ruby[||j>]{{\換字{更}}}{あらた}めて
\ruby{竊}{そつ}と
\ruby{其}{その}
\ruby{人}{ひと}を
\ruby{伺}{うかゞ}へば、% 踊り字調整「〻(二の字点、揺すり点)に濁点に見えるが(ゞ)」
%
\ruby{聊}{いさゝ}か% 踊り字調整「〻(二の字点、揺すり点)に見えるが(ゝ)」
\ruby{窶}{やつ}れたる
\ruby[g]{淺黑}{あさぐろ}き
\ruby{面}{おもて}の、
%
\ruby[g]{鼻筋}{はなすぢ}
\ruby{{\換字{通}}}{とほ}り
\ruby{口}{くち}
\ruby{締}{しま}りて、
%
\ruby{巖}{いは}も
\ruby[g]{黑鐵}{くろがね}も
\ruby{貫}{つらぬ}き
\ruby{徹}{とほ}すべき
\ruby[g]{精神}{きあひ }は、
%
\ruby{切}{き}れの
\ruby{長}{なが}き
\ruby[g]{尾上}{しりあが}りの
\ruby{眼}{め}の
\原本頁{174-5}\改行%
\ruby{中}{うち}の
\ruby{光}{ひかり}に
\ruby{現}{あらは}れたるに、
%
\ruby{生}{うま}れて
\ruby{初}{はじ}めて
かゝる% 踊り字調整「〻(二の字点、揺すり点)に見えるが(ゝ)」
\ruby[g]{意氣}{い き }
\ruby{組}{ぐみ}の
\ruby{{\換字{銳}}}{するど}くして
\ruby{烈}{はげ}しき、
%
\ruby[<j||]{{\換字{古}}}{むかし}
\ruby{物}{もの}
\ruby[||j>]{語}{がたり}の
\ruby{中}{うち}の
\ruby[g]{勇士}{ゆうし }の
やうなる
\ruby{人}{ひと}を
\ruby{眼}{め}の
\ruby{{\換字{前}}}{まへ}に
\ruby{見}{み}て、
%
あゝ% 踊り字調整「〻(二の字点、揺すり点)に見えるが(ゝ)」
\ruby{何}{なん}といふ
\ruby[g]{氣味}{き み }の
よい
\ruby{人}{ひと}と、
%
\ruby{深}{ふか}きに
\ruby{望}{のぞ}む
\ruby{千}{せん}
\ruby[||j>]{尺}{じやく}の
\ruby{崖}{がけ}に
\ruby{立}{た}つて
\ruby{吹}{ふ}き
\ruby{來}{く}る
\原本頁{174-8}\改行%
\ruby[g]{秋風}{あきかぜ}に
\ruby{袂}{たもと}を
\ruby{{\換字{扇}}}{あふ}らせ
たるが
\ruby{如}{ごと}く、
%
\ruby{凄}{すさま}じきが
\ruby{中}{なか}に
\ruby[g]{爽快を}{いさぎよき }
\ruby{覺}{おぼ}えて、
%
\ruby{怖}{こは}らしくは
\ruby{思}{おも}ひながら
\ruby{好}{この}ましくも
\ruby{思}{おも}ひたり。
