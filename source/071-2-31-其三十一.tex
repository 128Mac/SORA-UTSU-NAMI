\Entry{其三十一}

お
\ruby{龍}{りう}が
\ruby{頭}{かうべ}を
\ruby{下}{さ}げて
\ruby{禮}{れい}をなしつ、やがて
\ruby{言}{い}ひ
\ruby{出}{いだ}でんとする
\ruby{間}{ま}もあらせず、

『イヤお
\ruby{待}{また}せ
\ruby{申}{まをし}ました、
\ruby{小生}{わた|くし}は
\ruby{水野}{みづ|の}です。
』

と
\ruby{云}{い}ひたる、
\ruby{言語明晰}{げん|ご|はつ|きり}として
\ruby{冗處}{む|だ}も
\ruby{無}{な}く
\ruby{餘裕}{ゆ|とり}も
\ruby{無}{な}く、
\ruby{石甃}{いし|だゝみ}を
\ruby{見}{み}るやうに
\ruby{角}{かく}ばつたる
\ruby{云}{い}ひざまの、
\ruby{聲}{こゑ}つき
\ruby{自然}{おのづ|から}
\ruby{威勢}{いき|ほひ}あるに
お
\ruby{龍}{りう}は
\ruby{吞}{の}まれて、
\ruby{釣込}{つり|こ}まれ
\ruby{氣味}{ぎ|み}に
\ruby{此方}{こな|た}も
\ruby{堅}{かた}くなり、

『あの
\ruby{妾}{わたくし}は
\ruby{岩崎}{いわ|ざき}の
\ruby{母}{はゝ}のところから
\ruby{出}{で}ましたもので、』

と、
\ruby{先}{ま}づ
\ruby{一句明}{いつ|く|あき}らかに
\ruby{那處}{いづ|く}より
\ruby{來}{きた}れるかを
\ruby{{\換字{更}}}{さら}に
\ruby{告}{つ}げたり。

『ハア。
\ruby{左樣}{さ|う}して
\ruby{貴下}{あな|た}は
\ruby{御{\換字{近}}{\換字{所}}}{ご|きん|じよ}の
\ruby{方}{かた}で〻も
お
\ruby{有}{あ}りですか。
』

『ハイ、イエ、
\ruby{御承知}{ご|しよう|ち}はございますまいが
\ruby{妾}{わたくし}はあの、
\ruby{彼方}{あち|ら}に
\ruby{御厄介}{ご|やく|かい}になつて
\ruby{居}{を}るものでございまして、
\ruby{舊}{もと}は
\ruby{彼方}{あち|ら}で
お
\ruby{稽古}{けい|こ}を
\ruby{願}{ねが}つたものでございます。
』

『アヽ
\ruby{左樣}{さ|う}ですか、して
お
\ruby{師匠}{し|よ}さんは
お
\ruby{變}{かは}りありませんか。
』

\ruby{師匠}{しゝ|やう}は
\ruby{打擲}{うち|たゝ}いても
\ruby{死}{し}なざるべく
\ruby{壯健}{じや|うぶ}にして、
\ruby{酒}{さけ}を
\ruby{飮}{の}み
\ruby{{\換字{情}}夫}{をと|こ}と
\ruby{{\換字{連}}}{つ}れ
\ruby{立}{だ}ちて
\ruby{{\換字{遊}}}{あそ}び
\ruby{歩}{ある}けるものを、か〻る
\ruby{生眞面目}{き|ま|じ|め}なる
\ruby{人}{ひと}に
\ruby{虛言}{う|そ}を
\ruby{云}{い}ふことの
\ruby{心咎}{こゝろ|とがめ}せられぬにはあらざれど、

『ハイ
\ruby{有}{あ}りがたうございます。
まあ
\ruby{別條}{べつ|でう}は
\ruby{無}{な}いやうなものでございますが、
\ruby{先般}{この|あひだ}から
\ruby{一寸}{ちよ|つと}
\ruby{時候}{じ|こう}あたりを
\ruby{致}{いた}して
\ruby{{\換字{弱}}}{よわ}つて
\ruby{居}{を}りますので。
』

と
\ruby{已}{やむ}を
\ruby{得}{\換字{江}}ず
\ruby{豫}{かね}ての
\ruby{命令}{いひ|つけ}を
\ruby{{\換字{終}}}{つひ}に
\ruby{果}{はた}したり。

『それは
\ruby{何樣}{ど|う}もいけませんなナ、たゞの
\ruby{風邪}{か|ぜ}ですか。
』

『イエもう、
\ruby{眞}{ほん}の
\ruby{一寸}{ちよ|つと}した
\ruby{事}{こと}でございまして、しかも
\ruby{治}{なほ}り
\ruby{加減}{か|げん}でございますから、
お
\ruby{案}{あん}じ
\ruby{下}{くだ}さいますな。
それに
\ruby{就}{つ}きまして
\ruby{妾}{わたくし}が
\ruby{出}{で}ましたやうな
\ruby{譯}{わけ}でございますが、
\ruby{師匠}{しゝ|やう}が
\ruby{申}{まう}しますには、
\ruby{{\換字{過}}般}{この|あひだ}からはまた
\ruby{度々}{たび|〳〵}の
お
\ruby{手紙}{て|がみ}で、
\ruby{五十}{い|そ}の
\ruby{病氣}{びや|うき}を
\ruby{一々}{いち|〳〵}
お
\ruby{知}{し}らせ
\ruby{下}{くだ}さつたり、
\ruby{其上}{その|うへ}またいろ〳〵
お
\ruby{世話}{せ|わ}を
\ruby{戴}{いたゞ}いたりしまして、
お
\ruby{禮}{れい}を
\ruby{申}{まを}さうやうも
\ruby{無}{な}く
\ruby{有}{あ}り
\ruby{難}{がた}く
\ruby{存}{ぞん}じて
\ruby{居}{を}りまする。
\ruby{早{\換字{速}}}{さつ|そく}にも
\ruby{自{\換字{分}}}{じ|ぶん}で
\ruby{出}{で}て
お
\ruby{禮}{れい}を
\ruby{申上}{まを|しあ}げ、
\ruby{五十}{い|そ}の
\ruby{見舞}{み|まひ}も
\ruby{看病}{かん|びやう}も
\ruby{致}{いた}さなくつてはならないのではございますが、
\ruby{生憎}{あい|にく}と
\ruby{自{\換字{分}}}{じ|ぶん}も
\ruby{患}{わづら}つて
\ruby{居}{を}りまするので、
\ruby{存}{ぞん}じながら
\ruby{思}{おも}ふやうにも
\ruby{參}{まゐ}りません。
\ruby{水野}{みづ|の}さんが
\ruby{在}{い}らしつて
\ruby{下}{くだ}さるから
\ruby{好}{い}いはでもつて
\ruby{打棄}{うつ|ちや}つて
\ruby{居}{を}るやうで、
\ruby{大變}{たい|へん}
\ruby{心苦}{こゝろ|ぐる}しう
\ruby{存}{ぞん}じて
\ruby{居}{を}るのでございますが、
\ruby{全}{まつた}く
\ruby{左樣}{さ|う}いふ
\ruby{譯}{わけ}ではございません。
\ruby{御承知}{ご|しよ|うち}の
\ruby{{\換字{通}}}{とほ}りの
\ruby{女暮}{をんな|ぐら}しで、
\ruby{手前}{て|まへ}にばかりかまけて
\ruby{居}{を}りまするので、
\ruby{彼樣}{あ|ゝ}も
\ruby{仕}{し}たい、
\ruby{此樣}{こ|う}も
\ruby{仕}{し}たいと
\ruby{色々}{いろ|〳〵}に、
\ruby{心}{こゝろ}では
\ruby{思}{おも}つて
\ruby{居}{を}りましても
\ruby{手}{て}が
\ruby{届}{とゞ}きませんから、たゞ
\ruby{陰}{かげ}でもつて
\ruby{神信心}{かみ|しん|〴〵}ばかり
\ruby{致}{いた}して
\ruby{居}{を}るやうな
\ruby{譯}{わけ}でございます!。
と
\ruby{如是申上}{か|う|まを|しあ}げて、
\ruby{何樣}{ど|う}か
\ruby{何{\換字{分}}}{なに|ぶん}にも
\ruby{惡}{あ}しからず
\ruby{思召}{おぼし|めし}になるやうに、
\ruby{善}{よ}く
\ruby{汝}{おまへ}から
\ruby{有體}{あり|てい}のところを
\ruby{細}{こまか}に
お
\ruby{話仕}{はな|しゝ}て
お
\ruby{{\換字{呉}}}{く}れとの
\ruby{事}{こと}にございまする。
\ruby{{\換字{又}}}{また}、どうか
\ruby{此上}{この|うへ}とも
お
\ruby{世話}{せ|わ}を
\ruby{下}{くだ}さいますように、
\ruby{老母}{ばゞ|あ}は
\ruby{{\換字{勝}}手}{かつ|て}な
\ruby{奴}{やつ}だ
\ruby{顏}{かほ}も
\ruby{出}{だ}さないと、
お
\ruby{愛想盡}{あい|そ|づか}しになりましても、
\ruby{病人}{びやう|にん}は
\ruby{何}{なに}も
\ruby{知}{し}らない
\ruby{事}{こと}でございますから、
お
\ruby{愛想盡}{あい|そ|づか}しをなさらないやうに。
\ruby{五十}{い|そ}の
\ruby{事}{こと}は
\ruby{實}{じつ}は
\ruby{我儘}{わが|まゝ}な
\ruby{申}{まを}し
\ruby{樣}{やう}ですが、
\ruby{疾}{とう}から
\ruby{貴下}{あな|た}に
お
\ruby{任}{まか}せ
\ruby{申}{まを}したつもりで
\ruby{居}{を}りまするのでございますから、
\ruby{何}{ど}のやうにでも
お
\ruby{心持次第}{こゝろ|もち|し|だい}になすつて
\ruby{戴}{いたゞ}きたいので、
\ruby{御親切}{ご|しん|せつ}の
\ruby{貴下}{あな|た}の
お
\ruby{世話}{せ|わ}を
\ruby{戴}{いたゞ}いて、
\ruby{其}{それ}でいけなければ
\ruby{殘}{のこ}り
\ruby{惜}{をし}い
\ruby{事}{こと}はございません、
\ruby{全}{まつた}く
\ruby{當人}{たう|にん}の
\ruby{{\換字{運}}}{うん}の
\ruby{無}{な}いのだと
\ruby{諦}{あき}らめます。
いづれ
\ruby{其中}{その|うち}には
\ruby{是非}{ぜ|ひ}とも
\ruby{伺}{うかゞ}つて
お
\ruby{禮}{れい}を
\ruby{申}{まを}すつもりでございます。
\ruby{汝}{おまへ}
\ruby{彼方樣}{あち|ら|さま}へ
\ruby{上}{あが}つたら、
\ruby{何樣}{ど|う}か
\ruby{妾}{わたし}が
\ruby{如是}{か|う}いふ
\ruby{心持}{こゝろ|もち}を
\ruby{有}{も}つて
\ruby{居}{を}りますといふ
\ruby{事}{こと}を
\ruby{云}{い}つて、
\ruby{十{\換字{分}}}{じう|ぶん}に
お
\ruby{禮}{れい}を
\ruby{申上}{まを|しあ}げて、
\ruby{而}{そ}して
\ruby{五十}{い|そ}の
\ruby{病氣}{びや|うき}の
\ruby{樣子}{やう|す}も
\ruby{伺}{うかゞ}つて
\ruby{來}{き}て
お
\ruby{{\換字{呉}}}{く}れ、と
\ruby{斯樣}{か|やう}に
\ruby{申}{まを}すのでございます。
それでお
\ruby{馴染}{な|じ}みも
\ruby{無}{な}い
\ruby{妾}{わたくし}ではございますが、
\ruby{他}{ほか}に
\ruby{參}{まゐ}るものも
\ruby{無}{な}いのでございますから、
\ruby{一寸}{ちよ|つと}
\ruby{上}{あが}つたのでございます。
』

お
\ruby{龍}{りう}は
\ruby{果}{はた}さでは
\ruby{叶}{かな}はぬ
\ruby[g]{使者}{つかひ}の
\ruby{役目}{やく|め}を
\ruby{務}{つと}め
\ruby{果}{おほ}せん
\ruby{一心}{いつ|しん}に、
\ruby{一生懸命}{いつ|しやう|けん|めい}になりて
\ruby{如是{\換字{述}}}{か|く|の}べ
\ruby{{\換字{終}}}{をは}りしが、
\ruby{辛}{から}くも
\ruby{{\換字{吩}}附}{いひ|つ}けられしだけは
\ruby{云}{い}ひ
\ruby{得}{\換字{江}}たるにホツと
\ruby{氣息吐}{い|き|つ}きて、
\ruby{男}{をとこ}の
\ruby{樣子}{やう|す}を
\ruby{如何}{い|か}にと
\ruby{見}{み}れば、
\ruby{男}{をとこ}は
\ruby{律義眞正直}{りち|ぎ|まつ|しやう|ぢき}に
\ruby{物堅}{もの|がた}く
\ruby{愼}{つゝし}みて
\ruby{耳}{みゝ}を
\ruby{傾}{かたむ}け、
\ruby{見}{み}す〳〵の
\ruby{我}{わ}が
\ruby{虛言}{う|そ}を
\ruby{實}{げ}に
\ruby{{\換字{道}}理}{もつ|とも}と
\ruby{聞}{き}けるやうなるに、
\ruby{此}{こ}のやうなる
\ruby{人}{ひと}を
\ruby{口頭}{くち|さき}に
\ruby{操}{あやつ}るはと、
\ruby{我羞}{われ|はづか}しき
\ruby{心地}{こゝ|ち}の
\ruby{爲}{し}たり。

『ハイ、
\ruby{一々精}{いち|〳〵|よ}く
\ruby{解}{わか}りました、
\ruby{承知致}{しや|うち|いた}しました。
お
\ruby{言葉}{こと|ば}が
\ruby{無}{な}くてさへいろ〳〵に
\ruby{心配}{しん|ぱい}は
\ruby{致}{いた}して
\ruby{居}{を}りましたのですから、
\ruby{其樣}{そ|う}いふ
お
\ruby{言葉}{こと|ば}を
\ruby{伺}{うかゞ}ひます
\ruby{上}{うへ}は
\ruby{{\換字{猶}}}{なほ}の
\ruby{事}{こと}でございます。
\ruby{水野}{みづ|の}が
\ruby{出來}{で|き}まする
\ruby{事}{こと}は
\ruby{致}{いた}しますから、
\ruby{五十子}{い|そ|こ}さんの
\ruby{事}{こと}は
お
\ruby{心{\換字{遣}}無}{こゝろ|づか|ひな}く、よく
\ruby{御養生}{ご|やう|じやう}をなすつて
\ruby{早}{はや}く
\ruby{御全快}{ご|ぜん|くわい}なさるやうにと
\ruby{仰}{おつし}あつて
\ruby{下}{くだ}さいまし。
\ruby{五十子}{い|そ|こ}さんは
\ruby{必}{かなら}ず
\ruby{私}{わたくし}が
\ruby{癒}{なほ}らせます。
\ruby{何樣}{ど|う}しても
\ruby{一度}{いち|ど}は
\ruby{屹度}{きつ|と}
\ruby{癒}{なほ}らせますと
\ruby{小生}{わた|くし}が
\ruby{申}{まを}したと
\ruby{仰}{おつし}あつて
\ruby{下}{くだ}さいまし。
』

\ruby{人}{ひと}の
\ruby{命}{いのち}は
\ruby{知}{し}るべからざるを、あ〻
\ruby{何}{なん}ぞ
\ruby{其}{その}
\ruby{言葉}{こと|ば}の
\ruby{男兒}{をと|こ}らしく
\ruby{頼}{たの}もしきや。
\ruby{聲}{こゑ}の
\ruby{大}{おほき}くなりたるも
\ruby{思}{おも}はず
\ruby{誠意}{まこ|と}の
\ruby{籠}{こも}りたればなるべし。
\ruby{如斯云}{か|く|い}へる
\ruby{其}{そ}の
\ruby{言葉}{こと|ば}の
\ruby{力}{ちから}あるに
\ruby{驚}{おどろ}かされて、
お
\ruby{龍}{りう}は
\ruby{今}{いま}
\ruby{{\換字{又}}改}{また|あらた}めて
\ruby{窃}{そつ}と
\ruby{其人}{その|ひと}を
\ruby{伺}{うかゞ}へば、
\ruby{聊}{いさゝ}か
\ruby{窶}{やつ}れたる
\ruby{淺黑}{あさ|ぐろ}き
\ruby{面}{おもて}の、
\ruby{鼻筋{\換字{通}}}{はな|すじ|とほ}り
\ruby{口締}{くち|しま}りて、
\ruby{巖}{いは}も
\ruby{黑鐵}{くろ|がね}も
\ruby{貫}{つらぬ}き
\ruby{徹}{とほ}すべき
\ruby{精神}{きあ|ひ}は、
\ruby{切}{き}れの
\ruby{長}{なが}き
\ruby{尾上}{しり|あが}りの
\ruby{眼}{め}の
\ruby{中}{うち}の
\ruby{光}{ひかり}に
\ruby{現}{あらは}れたるに、
\ruby{生}{うま}れて
\ruby{初}{はじ}めてか〻る
\ruby{意氣組}{い|き|ぐみ}の
\ruby{{\換字{銳}}}{するど}くして
\ruby{烈}{はげ}しき、
\ruby{昔物語}{むかし|もの|がたり}の
\ruby{中}{うち}の
\ruby{勇士}{ゆう|し}のやうなる
\ruby{人}{ひと}を
\ruby{眼}{め}の
\ruby{前}{まへ}に
\ruby{見}{み}て、あ〻
\ruby{何}{なん}といふ
\ruby{氣味}{き|み}のよい
\ruby{人}{ひと}と、
\ruby{深}{ふか}きに
\ruby{望}{のぞ}む
\ruby{千尺}{せん|じやく}の
\ruby{崖}{がけ}に
\ruby{立}{た}つて
\ruby{吹}{ふ}き
\ruby{來}{く}る
\ruby{秋風}{あき|かぜ}に
\ruby{袂}{たもと}を
\ruby{{\換字{扇}}}{あふ}らせたるが
\ruby{如}{ごと}く、
\ruby{凄}{すさま}じきが
\ruby{中}{なか}に
\ruby[g]{爽快}{いさぎよき}を
\ruby{覺}{おぼ}えて、
\ruby{怖}{こは}らしくは
\ruby{思}{おも}ひながら
\ruby{好}{この}ましくも
\ruby{思}{おも}ひたり。

