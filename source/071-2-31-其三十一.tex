\Entry{其三十一}

% メモ 校正終了 2024-04-26
\原本頁{168-1}%
お
\ruby{龍}{りう}が
\ruby{頭}{かうべ}を
\ruby{下}{さ}げて
\ruby{禮}{れい}を
なしつ、
%
やがて
\ruby{言}{ものい}ひ
\ruby{出}{いだ}でん
とする
\ruby{間}{ま}も
あらせす、

\原本頁{168-3}%
『
イヤ
お
\ruby{待}{また}せ
\ruby{申}{まをし}ました、
%
\ruby{小生}{わた|くし}は
\ruby{水野}{みづ|の}です。
』

\原本頁{168-4}%
と
\ruby{云}{い}ひたる、
%
\ruby{言語}{げん|ご}
\ruby{明晰}{はつ|きり}として
\ruby{冗處}{む|だ}も
\ruby{無}{な}く
\ruby{餘裕}{ゆ|とり}も
\ruby{無}{な}く、
%
\ruby{石}{いし}
\ruby[||j>]{甃}{だ〻み}を% 原本通り「〻(二の字点、揺すり点)」
\ruby{見}{み}るやうに
\ruby{角}{かく}ばつたる
\ruby{云}{い}ひ
ざまの、
%
\ruby{聲}{こゑ}つき
\ruby[<j||]{自}{おのづ}
\ruby{然}{から}
\ruby{威勢}{いき|ほひ}あるに
お
\ruby{龍}{りう}は
\ruby{吞}{の}まれて、
%
\ruby{釣}{つり}
\ruby{{\換字{込}}}{こ}まれ
\ruby{氣味}{ぎ|み}に
\ruby{此方}{こな|た}も
\ruby{堅}{かた}くなり、

\原本頁{168-7}%
『
あの
\ruby{妾}{わたくし}は
\ruby{岩崎}{いは|ざき}の
\ruby{母}{は〻}の% 原本通り「〻(二の字点、揺すり点)」
ところから
\ruby{出}{で}ましたもので、
』

\原本頁{168-8}%
と、
%
\ruby{先}{ま}づ
\ruby{一句}{いつ|く}
\ruby{明}{あき}らかに
\ruby{那處}{いづ|く}より
\ruby{來}{きた}れるかを
\ruby{{\換字{更}}}{さら}に
\ruby{告}{つ}げたり。

\原本頁{168-9}%
『
ハア。
%
\ruby{左樣}{さ|う}して
\ruby{貴下}{あな|た}は
\ruby{御{\換字{近}}{\換字{所}}}{ご|きん|じよ}の
\ruby{方}{かた}で〻も% 原本通り「〻(二の字点、揺すり点)」
お
\ruby{有}{あ}りですか。
』

\原本頁{168-10}%
『
ハイ、
%
イエ、
%
\ruby{御承知}{ご|しよう|ち}は
ございますまいが
\ruby{妾}{わたくし}は
あの、
%
\ruby{彼方}{あち|ら}に
\ruby{御厄介}{ご|やく|かい}に
なつて
\ruby{居}{を}る
もので
ございまして、
%
\ruby{舊}{もと}は
\ruby{彼方}{あち|ら}で
お
\ruby{稽{\換字{古}}}{けい|こ}を
\ruby{願}{ねが}つた
もので
ございます。
』

\原本頁{169-2}%
『
アヽ
\ruby{左樣}{さ|う}ですか、
%
して
お
\ruby{師匠}{し|よ}さんは
お
\ruby{變}{かは}り
もありませんか。
』

\原本頁{169-3}%
\ruby{師匠}{し|〻やう}は% 原本通り「〻(二の字点、揺すり点)」
\ruby{打}{うち}
\ruby{擲}{た〻}いても% 原本通り「〻(二の字点、揺すり点)」
\ruby{死}{し}なざるべく
\ruby{壯健}{じやう|ぶ}にして、
%
\ruby{酒}{さけ}を
\ruby{飮}{の}み
\ruby{{\換字{情}}夫}{をと|こ}と
\ruby{{\換字{連}}}{つ}れ
\ruby{立}{だ}ちて
\ruby{{\換字{遊}}}{あそ}び
\ruby{歩}{ある}けるものを、
%
か〻る% 原本通り「〻(二の字点、揺すり点)」
\ruby{生眞面目}{き|ま|じ|め}なる
\ruby{人}{ひと}に
\ruby{虛言}{う|そ}を
\ruby{云}{い}ふことの
\ruby[<j||]{心}{こ〻ろ}% 原本通り「〻(二の字点、揺すり点)」
\ruby{咎}{とがめ}せられぬ
には
あらざれど、

\原本頁{169-6}%
『
ハイ
\ruby{有}{あ}りがたう
ございます。
%
まあ
\ruby{別條}{べつ|でう}は
\ruby{無}{な}い
やうな
もので
ございますが、
%
\ruby{先}{この}
\ruby[||j>]{般}{あひだ}から
\ruby{一寸}{ちよ|つと}
\ruby{時候}{じ|こう}
あたりを
\ruby{致}{いた}して
\ruby{{\換字{弱}}}{よわ}つて
\ruby{居}{を}りますので。
』

\原本頁{169-9}%
と
\ruby{已}{やむ}を
\ruby{得}{{\換字{𛀁}}}ず
\ruby{豫}{かね}ての
\ruby{命令}{いひ|つけ}を
\ruby{{\換字{終}}}{つひ}に
\ruby{果}{はた}したり。

\原本頁{169-10}%
『
それは
\ruby{何樣}{ど|う}も
いけませんナ、
%
たゞの% TODO 原本の「二の字点、揺すり点」に濁点のグリフが見つからないので「ゞ」
\ruby{風邪}{か|ぜ}ですか。
』

\原本頁{169-11}%
『
イエ
もう、
%
\ruby{眞}{ほん}の
\ruby{一寸}{ちよ|つと}した
\ruby{事}{こと}で
ございまして、
%
しかも
\ruby{治}{なほ}り
\ruby{加減}{か|げん}で
ございますから、
%
お
\ruby{案}{あん}じ
\ruby{下}{くだ}さい
ますな。
%
それに
\ruby{就}{つ}きまして
\ruby{妾}{わたくし}が
\ruby{出}{で}ました
やうな
\ruby{譯}{わけ}で
ございますが、
%
\ruby{師匠}{し|〻やう}が% 原本通り「〻(二の字点、揺すり点)」
\ruby{申}{まを}しますには、
%
\ruby{{\換字{過}}}{この}
\ruby[||j>]{般}{あひだ}からは
また
\ruby{度々}{たび|〳〵}の
お
\ruby{手紙}{て|がみ}で、
%
\ruby{五十}{い|そ}の
\ruby{病氣}{びやう|き}を
\ruby{一々}{いち|〳〵}
お
\ruby{知}{し}らせ
\ruby{下}{くだ}さつたり、
%
\ruby{其}{その}
\ruby{上}{うへ}
また
いろ〳〵
お
\ruby{世話}{せ|わ}を
\ruby{戴}{いたゞ}いたり% TODO 原本の「二の字点、揺すり点」に濁点のグリフが見つからないので「ゞ」
しまして、
%
お
\ruby{禮}{れい}を
\原本頁{170-5}\改行%
\ruby{申}{まを}さうやうも
\ruby{無}{な}く
\ruby{有}{あ}り
\ruby{{\換字{難}}}{がた}く
\ruby{存}{ぞん}じて
\ruby{居}{を}りまする。
%
\ruby{早{\換字{速}}}{さつ|そく}にも
\ruby{自{\換字{分}}}{じ|ぶん}で
\ruby{出}{で}て
お
\ruby{禮}{れい}を
\ruby[<j||]{申}{まをし}
\ruby{上}{あ}げ、
%
\ruby{五十}{い|そ}の
\ruby{見舞}{み|まひ}も
\ruby{看}{かん}
\ruby[||j>]{病}{びやう}も
\ruby{致}{いた}さなくつては
ならないので
ございますが、
%
\ruby{生憎}{あい|にく}と% 原文通りルビは「あいにく」
\ruby{自{\換字{分}}}{じ|ぶん}も
\ruby{患}{わづら}つて
\ruby{居}{を}りまするので、
%
\ruby{存}{ぞん}じながら
\ruby{思}{おも}ふ
やうにも
\ruby{參}{まゐ}りません。
%
\ruby{水野}{みづ|の}さんが
\ruby{在}{い}らしつて
\ruby{下}{くだ}さるから
\ruby{好}{い}いはで
もつて
\ruby{打棄}{うつ|ちや}つて
\ruby{居}{を}るやうで、
%
\ruby{大變}{たい|へん}
\ruby[<j||]{心}{こ〻ろ}% 原本通り「〻(二の字点、揺すり点)」
\ruby{苦}{ぐる}しう
\ruby{存}{ぞん}じて
\ruby{居}{を}るので
ございますが、
%
\ruby{全}{まつた}く
\ruby{左樣}{さ|う}いふ
\ruby{譯}{わけ}では
ございません。
%
\原本頁{170-11}\改行%
\ruby{御承知}{ご|しよう|ち}の
\ruby{{\換字{通}}}{とほ}りの
\ruby[<j||]{女}{をんな}
\ruby{暮}{ぐら}しで、
%
\ruby{手{\換字{前}}}{て|まへ}に
ばかり
かまけて
\ruby{居}{を}りまするので、
%
\ruby{彼樣}{あ|〻}も% 原本通り「〻(二の字点、揺すり点)」
\ruby{仕}{し}たい、
%
\ruby{此樣}{こ|う}も
\ruby{仕}{し}たい
と
\ruby{色々}{いろ|〳〵}に、
%
\ruby{心}{こ〻ろ}では% 原本通り「〻(二の字点、揺すり点)」
\ruby{思}{おも}つて
\ruby{居}{を}りましても
\ruby{手}{て}が
\ruby{屆}{とゞ}きませんから、% 「屆」「届」 原本通り「屆」% TODO 原本の「二の字点、揺すり点」に濁点のグリフが見つからないので「ゞ」
%
たゞ% TODO 原本の「二の字点、揺すり点」に濁点のグリフが見つからないので「ゞ」
\ruby{蔭}{かげ}で
もつて
\ruby{神信心}{かみ|しん|〴〵}
ばかり
\ruby{致}{いた}して
\ruby{居}{を}るやうな
\ruby{譯}{わけ}で
ございます!。
%
と
\ruby{如是}{か|う}
\ruby{申上}{まをし|あ}げて、
%
\ruby{何樣}{ど|う}か
\ruby{何{\換字{分}}}{なに|ぶん}にも
\ruby{惡}{あ}しからず
\ruby[||j>]{思}{おぼし}
\ruby[||j>]{召}{ めし}になるやうに、
% \ruby{思召}{おぼし|めし}になるやうに、
%
\ruby{善}{よ}く
\ruby{汝}{おまへ}から
\ruby{有體}{あり|てい}の
ところ
\原本頁{171-5}\改行%
を
\ruby{細}{こまか}に
お
\ruby{話仕}{はなし|〻}て% 原本通り「〻(二の字点、揺すり点)」
お
\ruby{吳}{く}れとの
\ruby{事}{こと}に
ございまする。
%
\ruby{{\換字{又}}}{また}、
%
どうか
\ruby{此}{この}
\ruby{上}{うへ}とも
お
\ruby{世話}{せ|わ}を
\ruby{下}{くだ}さいますように、
%
\ruby{老母}{ば〻|あ}は% 「ばゞ」のはずだが、原本通り「〻(二の字点、揺すり点)」
\ruby{{\換字{勝}}手}{かつ|て}な
\ruby{奴}{やつ}だ
\ruby{顏}{かほ}も
\ruby{出}{だ}さないと、
%
お
\ruby{愛想盡}{あい|そ|づか}し
になりましても、
%
\ruby[<j||]{病}{びやう}
\ruby{人}{にん}は
\ruby{何}{なに}も
\ruby{知}{し}らない
\ruby{事}{こと}で
ございますから、
%
お
\ruby{愛想盡}{あい|そ|づか}し
を
なさらない
やうに。
%
\ruby{五十}{い|そ}の
\ruby{事}{こと}は
\ruby{實}{じつ}は
\ruby{我儘}{わが|ま〻}な% 原本通り「〻(二の字点、揺すり点)」
\ruby{申}{まを}し
\ruby{樣}{やう}ですが、
%
\ruby{疾}{とう}から
\ruby{貴下}{あな|た}に
お
\ruby{任}{まか}せ
\ruby{申}{まを}した
つもりで
\ruby{居}{を}りまする
ので
ございます
から、
%
\ruby{何}{ど}のやう
にでも
お
\ruby[<j||]{心}{こ〻ろ}% 原本通り「〻(二の字点、揺すり点)」
\ruby{持}{もち}
\ruby{次第}{し|だい}
になすつて
\ruby{戴}{いたゞ}きたい% TODO 原本の「二の字点、揺すり点」に濁点のグリフが見つからないので「ゞ」
ので、
%
\ruby{御親切}{ご|しん|せつ}の
\ruby{貴下}{あな|た}の
お
\ruby{世話}{せ|わ}を
\ruby{戴}{いたゞ}いて、% TODO 原本の「二の字点、揺すり点」に濁点のグリフが見つからないので「ゞ」
%
\ruby{其}{それ}
でいけなければ
\ruby{殘}{のこ}り
\ruby{惜}{をし}い
\ruby{事}{こと}は
ございません、
%
\ruby{全}{まつた}く
\ruby{當人}{たう|にん}の
\ruby{{\換字{運}}}{うん}の
\ruby{無}{な}いのだと
\ruby{諦}{あき}らめます。
%
いづれ
\ruby{其}{その}
\ruby{中}{うち}には
\ruby{是非}{ぜ|ひ}とも
\ruby{伺}{うかゞ}つて% TODO 原本の「二の字点、揺すり点」に濁点のグリフが見つからないので「ゞ」
お
\ruby{禮}{れい}を
\ruby{申}{まを}す
つもりで
ございます。
%
\ruby[<j||]{汝}{おまへ}
\ruby{彼方樣}{あち|ら|さま}へ
\ruby{上}{あが}つたら、
%
\ruby{何樣}{ど|う}か
\ruby{妾}{わたし}が
\ruby{如是}{か|う}いふ
\原本頁{172-4}\改行%
\ruby[||j>]{心}{こ〻ろ}% 原本通り「〻(二の字点、揺すり点)」
\ruby{持}{ もち}を%
\ruby{有}{も}つて
\ruby{居}{を}ります
といふ
\ruby{事}{こと}を
\ruby{云}{い}つて、
%
\ruby{十{\換字{分}}}{じう|ぶん}に
お
\ruby{禮}{れい}を
\ruby{申上}{まをし|あ}げて、
%
\ruby{而}{そ}して
\ruby{五十}{い|そ}の
\ruby{病氣}{びやう|き}の
\ruby{樣子}{やう|す}も
\ruby{伺}{うかゞ}つて% TODO 原本の「二の字点、揺すり点」に濁点のグリフが見つからないので「ゞ」
\ruby{來}{き}て
お
\ruby{吳}{く}れ、
%
と
\ruby{斯樣}{か|やう}に
\ruby{申}{まを}す
ので
ございます。
%
それで
お
\ruby{馴染}{な|じ}みも
\ruby{無}{な}い
\ruby{妾}{わたくし}
では
ございますが、
%
\ruby{他}{ほか}に
\ruby{參}{まゐ}るものも
\ruby{無}{な}い
ので
ございます
から、
%
\ruby{一寸}{ちよ|つと}
\ruby{上}{あが}つた
ので
ございます。
』

\原本頁{172-9}%
お
\ruby{龍}{りう}は
\ruby{果}{はた}さでは
\ruby{叶}{かな}はぬ
\ruby{使者}{つか|ひ}の
\ruby{役目}{やく|め}を
\ruby{務}{つと}め
\ruby{果}{おほ}せん
\ruby{一心}{いつ|しん}に、
%
\ruby{一生懸命}{いつ|しやう|けん|めい}になりて
\ruby{如是}{か|く}
\ruby{{\換字{述}}}{の}べ
\ruby{{\換字{終}}}{をは}りしが、
%
\ruby{辛}{から}くも
\ruby{吩咐}{いひ|つ}け% 吩咐 ... 言いつける、指図する
られし
だけは
\ruby{云}{い}ひ
\原本頁{172-11}\改行%
\ruby{得}{{\換字{𛀁}}}たるに
ホツと
\ruby{氣息}{い|き}
\ruby{吐}{つ}きて、
%
\ruby{男}{をとこ}の
\ruby{樣子}{やう|す}を
\ruby{如何}{い|か}にと
\ruby{見}{み}れば、
%
\ruby{男}{をとこ}は
\原本頁{173-1}\改行%
\ruby{律義}{りち|ぎ}
\ruby{眞正直}{まつ|しやう|ぢき}に
\ruby{物}{もの}
\ruby{堅}{がた}く
\ruby{愼}{つ〻し}みて% 原本通り「〻(二の字点、揺すり点)」
\ruby{耳}{み〻}を% 原本通り「〻(二の字点、揺すり点)」
\ruby{傾}{かたむ}け、
%
\ruby{見}{み}す〳〵の
\ruby{我}{わ}が
\ruby{虛言}{う|そ}を
\ruby{實}{げ}に
\ruby{{\換字{道}}理}{もつ|とも}と
\ruby{聞}{き}ける
やうなるに、
%
\ruby{此}{こ}のやうなる
\ruby{人}{ひと}を
\ruby{口頭}{くち|さき}に
\ruby{操}{あやつ}るはと、
%
\ruby{我}{われ}
\ruby[||j>]{羞}{はづか}しき
\ruby{心地}{こ〻|ち}の% 原本通り「〻(二の字点、揺すり点)」
\ruby{爲}{し}たり。

\原本頁{173-4}%
『
ハイ、
%
\ruby{一々}{いち|〳〵}
\ruby{精}{よ}く
\ruby{解}{わか}りました、
%
\ruby{承知}{しよう|ち}
\ruby{致}{いた}しました。
%
お
\ruby{言葉}{こと|ば}が
\ruby{無}{な}くて
さへ
いろ〳〵に
\ruby{心配}{しん|ぱい}は
\ruby{致}{いた}して
\ruby{居}{を}りました
のですから、
%
\ruby{其樣}{そ|う}いふ
お
\ruby{言葉}{こと|ば}を
\ruby{伺}{うかゞ}ひます% TODO 原本の「二の字点、揺すり点」に濁点のグリフが見つからないので「ゞ」
\ruby{上}{うへ}は
\ruby{{\換字{猶}}}{なほ}の
\ruby{事}{こと}で
ございます。
%
\ruby{水野}{みづ|の}が
\ruby{出來}{で|き}まする
だけの
\ruby{事}{こと}は
\ruby{致}{いた}しますから、
%
\ruby{五十子}{い|そ|こ}さんの
\ruby{事}{こと}は
お
\ruby[<j||]{心}{こ〻ろ}% 原本通り「〻(二の字点、揺すり点)」
\ruby{{\換字{遣}}無}{づかひ|な}く、
%
よく
\ruby{御}{ご}
\ruby{養}{やう}
\ruby[||j>]{生}{じやう}をなすつて
\ruby{早}{はや}く
\ruby{御}{ご}
\ruby{全}{ぜん}
\ruby[||j>]{快}{くわい}なさる
やうにと
\ruby{仰}{おつし}あつて
\ruby{下}{くだ}さいまし。
%
\ruby{五十子}{い|そ|こ}さんは
\ruby{必}{かなら}ず
\ruby{私}{わたくし}が
\ruby{癒}{なほ}らせます。
%
\ruby{何樣}{ど|う}しても
\ruby{一度}{いち|ど}は
\ruby{屹度}{きつ|と}
\ruby{癒}{なほ}らせますと
\ruby{小生}{わたく|し}が
\ruby{申}{まを}したと
\ruby{仰}{おつし}あつて
\ruby{下}{くだ}さいまし。
』

\原本頁{173-11}%
\ruby{人}{ひと}の
\ruby{命}{いのち}は
\ruby{知}{し}る
べからざるを、
%
あ〻% 原本通り「〻(二の字点、揺すり点)」
\ruby{何}{なん}ぞ
\ruby{其}{その}
\ruby{言葉}{こと|ば}の
\ruby{男兒}{をと|こ}らしく
\ruby{頼}{たの}もしきや。
%
\ruby{聲}{こゑ}の
\ruby{大}{おほき}く
なりたるも
\ruby{思}{おも}はず
\ruby{誠意}{まこ|と}の
\ruby{籠}{こも}り
たれば
なるべし。
%
\原本頁{174-2}\改行%
\ruby{如斯}{か|く}
\ruby{云}{い}へる
\ruby{其}{そ}の
\ruby{言葉}{こと|ば}の
\ruby{力}{ちから}
あるに
\ruby{驚}{おどろ}かされて、
%
お
\ruby{龍}{りう}は
\ruby{今}{いま}
\ruby{{\換字{又}}}{また}
\ruby[||j>]{{\換字{更}}}{あらた}めて
\ruby{竊}{そつ}と
\ruby{其}{その}
\ruby{人}{ひと}を
\ruby{伺}{うかゞ}へば、% TODO 原本の「二の字点、揺すり点」に濁点のグリフが見つからないので「ゞ」
%
\ruby{聊}{いさ〻}か% 原本通り「〻(二の字点、揺すり点)」
\ruby{窶}{やつ}れたる
\ruby{淺黑}{あさ|ぐろ}き
\ruby{面}{おもて}の、
%
\ruby{鼻筋}{はな|すぢ}
\ruby{{\換字{通}}}{とほ}り
\ruby{口}{くち}
\ruby{締}{しま}りて、
%
\ruby{巖}{いは}も
\ruby{黑鐵}{くろ|がね}も
\ruby{貫}{つらぬ}き
\ruby{徹}{とほ}すべき
\ruby{精神}{きあ|ひ}は、
%
\ruby{切}{き}れの
\ruby{長}{なが}き
\ruby{尾上}{しり|あが}りの
\ruby{眼}{め}の
\原本頁{174-5}\改行%
\ruby{中}{うち}の
\ruby{光}{ひかり}に
\ruby{現}{あらは}れたるに、
%
\ruby{生}{うま}れて
\ruby{初}{はじ}めて
か〻る% 原本通り「〻(二の字点、揺すり点)」
\ruby{意氣}{い|き}
\ruby{組}{ぐみ}の
\ruby{{\換字{銳}}}{するど}くして
\ruby{烈}{はげ}しき、
%
\ruby[<j||]{{\換字{古}}}{むかし}
\ruby{物}{もの}
\ruby[||j>]{語}{がたり}の
\ruby{中}{うち}の
\ruby{勇士}{ゆう|し}の
やうなる
\ruby{人}{ひと}を
\ruby{眼}{め}の
\ruby{{\換字{前}}}{まへ}に
\ruby{見}{み}て、
%
あ〻% 原本通り「〻(二の字点、揺すり点)」
\ruby{何}{なん}といふ
\ruby{氣味}{き|み}の
よい
\ruby{人}{ひと}と、
%
\ruby{深}{ふか}きに
\ruby{望}{のぞ}む
\ruby{千}{せん}
\ruby[||j>]{尺}{じやく}の
\ruby{崖}{がけ}に
\ruby{立}{た}つて
\ruby{吹}{ふ}き
\ruby{來}{く}る
\原本頁{174-8}\改行%
\ruby{秋風}{あき|かぜ}に
\ruby{袂}{たもと}を
\ruby{{\換字{扇}}}{あふ}らせ
たるが
\ruby{如}{ごと}く、
%
\ruby{凄}{すさま}じきが
\ruby{中}{なか}に
\ruby[<j||]{爽}{いさぎ}
\ruby{快}{よき}を
\ruby{覺}{おぼ}えて、
%
\ruby{怖}{こは}らしくは
\ruby{思}{おも}ひながら
\ruby{好}{この}ましくも
\ruby{思}{おも}ひたり。
