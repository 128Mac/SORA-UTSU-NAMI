\Entry{其三十二}

か\ninojiten{}るところへ
\ruby{新}{あらた}に
\ruby{茶}{ちや}をいれて
\ruby{持來}{もち|きた}りしお
\ruby{濱}{はま}に、はつきりと
\ruby{美}{うつく}しき
\ruby{眼}{め}に
\ruby{優}{やさ}しくお
\ruby{龍}{りう}を
\ruby{見}{み}て、しとやかに
\ruby{其}{そ}の
\ruby{一盞}{いつ|さん}を
\ruby{取}{と}りて
\ruby{薦}{す〻}むれば、
\ruby{水野}{みづ|の}を
\ruby{見}{み}たる
\ruby{目}{め}を
\ruby{此人}{この|ひと}に
\ruby{移}{うつ}しては、
\ruby{懷暗}{ふところ|くら}き
\ruby[g]{常綠樹}{ときはぎ}の
\ruby{高}{たか}く
\ruby{聳}{そび}\換字{江}たるを
\ruby{見}{み}たる
\ruby{目}{め}に、しほらしく
\ruby{{\換字{咲}}}{さ}く
\ruby{初櫻}{はつ|ざくら}の、ぱつと
\ruby{明}{あか}るき
\ruby{花}{はな}の
\ruby{枝}{えだ}を
\ruby{忽}{たちま}ち
\ruby{見}{み}たる
\ruby{心地}{こゝ|ち}して、おのづから
\ruby{胸}{むね}も
\ruby{開}{ひら}くるやうするに、お
\ruby{龍}{りう}は、

『どうもはゞかりさま、
\ruby{恐}{おそ}れ
\ruby{入}{い}ります。
』

と
\ruby{身}{み}を
\ruby{{\換字{謙}}{\換字{退}}}{へり|くだ}りて
\ruby{會釋}{ゑし|やく}しつ、
\ruby{互}{たがひ}に
\ruby{顏}{かほ}を
\ruby{見合}{み|あ}はせしが、
\ruby{笑}{わら}ふとも
\ruby{無}{な}く
\ruby{媽然}{につ|こり}としたる
\ruby{彼此一時}{かれ|これ|いち|じ}の
\ruby{笑容}{ゑ|み}の
\ruby{中}{うち}に、
\ruby{語}{かた}らで
\ruby{語}{かた}り
\ruby{聞}{き}かで
\ruby{聞}{き}く
\ruby{心}{こゝろ}と
\ruby{心}{こゝろ}と
\ruby{働}{はたら}きて、
\ruby{思}{おも}へば
\ruby{思}{おも}ひ
\ruby{好}{す}けば
\ruby{好}{す}く
\ruby{性}{しやう}の
\ruby{合}{あ}ふ
\ruby{同士}{どう|し}
\ruby{女}{をんな}
\ruby{同士}{どう|し}、
\ruby{何}{なに}の
\ruby{故}{ゆゑ}とは
\ruby{無}{な}けれども
\ruby{相}{あひ}なつかしみ
\ruby{相{\換字{悅}}}{あひ|よろこ}びたり。

されどお
\ruby{濱}{はま}は
\ruby{何時}{い|つ}まで
\ruby{此處}{こ|\ninojiten}にあるべきならねば、お
\ruby{龍}{りう}と
\ruby{物語}{もの|がた}りして
\ruby{{\換字{遊}}}{あそ}びたきやうの
\ruby{思}{おもひ}は
\ruby{仕}{し}ながら、
\ruby{一盞}{いつ|さん}を
\ruby{取}{と}りて
\ruby{水野}{みづ|の}に
\ruby{與}{あた}へて、
\ruby{好}{よ}きほどのところに
\ruby{茶具}{ちや|ぐ}を
\ruby{置}{お}き
\ruby{捨}{す}て、おのれは
\ruby{茶}{ちや}の
\ruby{間}{ま}に
\ruby{{\換字{退}}}{しりぞ}きて
\ruby{二人}{ふた|り}の
\ruby{話}{はなし}を
\ruby{聞}{き}けり。

お
\ruby{龍}{りう}は
\ruby{{\換字{猶}}}{なほ}
\ruby{五十子}{い|そ|こ}の
\ruby{容態}{よう|だい}を
\ruby{聞}{き}かでは
\ruby{叶}{かな}はざるなり。

『ほんとうに
\ruby{段々}{だん|〳〵}との
\ruby{深}{ふか}い
\ruby{御親切}{ご|しん|せつ}さまで、まことに
\ruby{有}{あ}り
\ruby{難}{がた}う
\ruby{存}{ぞん}じます。
\ruby{歸}{かへ}つて
\ruby{御言葉}{お|こと|ば}
\ruby{{\換字{通}}}{どほ}りに
\ruby{左樣申}{さ|う|まを}し
\ruby{傳}{つた}へましたら、
\ruby{何樣}{ど|ん}なにか
\ruby{師匠}{し〻|やう}も
\ruby{{\換字{悅}}}{よろこ}ぶことでございましやう。
\ruby{左樣}{さ|う}いたしまして
\ruby{只今}{たゞ|いま}は、
\ruby{病人}{びやう|にん}は
\ruby{何樣}{ど|ん}な
\ruby{樣子}{やう|す}でございますの?』

『いや
\ruby{何樣}{ど|う}も
\ruby{中々良}{なか|〳〵|よ}くないのです。
それで
\ruby{大}{おほ}きに
\ruby{心配致}{しん|ぱい|いた}しましたが、
\ruby{淺草}{あさ|くさ}の
\ruby{醫者}{い|しや}を
\ruby{招}{よ}びに
\ruby{行}{ゆ}きました
\ruby[g]{歸路}{かへり}に、たつた
\ruby{今此村}{いま|こ|〻}の
\ruby{醫者}{い|しや}に
\ruby{容態}{よう|だい}を
\ruby{聞}{き}きましたら、
\ruby{大}{おほ}きに
\ruby{見直}{み|なほ}したやうな
\ruby{具合}{ぐ|あひ}でして、
\ruby{重病}{ぢう|びやう}だから
\ruby{何}{なん}とも
\ruby{云}{い}へないが、
\ruby{此儘}{この|ま〻}で
\ruby{日}{ひ}さへ
\ruby{經}{へ}て
\ruby{{\換字{呉}}}{く}れ〻ばまあ
\ruby{宣}{よ}いといふので…………』

『では
\ruby{食事}{しよ|くじ}なんどは?。
』

『なか〳〵まだ
\ruby{食事}{しよ|くじ}なんぞといふ
\ruby{段}{だん}では
\ruby{無}{な}いので。
やつと
\ruby{流動物}{りう|どう|ぶつ}が
\ruby[g]{小量許入}{すこしばかりはい}る
\ruby{位}{くらゐ}です。
しかし
\ruby{變}{へん}さへ
\ruby{無}{な}ければ、
\ruby{大抵}{たい|てい}は
\ruby{經{\換字{過}}日數}{けい|くわ|につ|すう}が
\ruby{定}{きま}つて
\ruby{居}{ゐ}るものださうですから。
』

『
\ruby{案}{あん}じるやうな
\ruby{事}{こと}はまあ
\ruby{無}{な}いのでございますか。
』

『
\ruby{左樣}{さ|う}ばかりにもいきますまいが。
』

『
\ruby{變}{へん}の
\ruby{無}{な}いやうに
\ruby{致}{いた}しかたは
\ruby{無}{な}いものでございましやうか。
』

『そりやあ
\ruby{左樣}{さ|う}したいのは
\ruby{山々}{やま|〳〵}ですが、
\ruby{{\換字{情}}無}{なさ|けな}い
\ruby{事}{こと}には
\ruby{醫者}{い|しや}の
\ruby{力}{ちから}でも
\ruby{其處}{そ|こ}までは
\ruby{何樣}{ど|う}もなりません。
』

『それぢやあ
\ruby{神樣}{かみ|さま}にでも
\ruby{御願申}{おね|がひ|まを}すよりほかには!。
』

『
\ruby{然樣}{さ|う}です。
とてもまあ
\ruby{其樣}{そ|ん}な
\ruby{事}{こと}よりほかには!。
』

\ruby{男}{をとこ}の
\ruby{聲}{こゑ}はこ\ninojiten{}に
\ruby{至}{いた}つて
\ruby{甚}{ひど}く
\ruby{沈}{しづ}めり。
お
\ruby{龍}{りう}は
\ruby{忽然}{こつ|ぜん}として
\ruby{思}{おも}ひ
\ruby{{\換字{浮}}}{う}かぶるところあり。
\ruby{我}{われ}に
\ruby{對}{むか}へる
\ruby{此人}{この|ひと}は
\ruby{誰}{たれ}ぞ。
この
\ruby{人}{ひと}は
\ruby{是彼}{これ|か}の
\ruby{普門品}{ふ|もん|ぼん}の
\ruby{主}{ぬし}ならずや。
\ruby{何}{なに}をか
\ruby{獨}{ひと}り
\ruby{物思}{もの|おも}ひして
\ruby{睫毛}{まつ|げ}に
\ruby{露}{つゆ}を
\ruby{湛}{た\ninojiten}へし
\ruby{人}{ひと}ならずや。
あはれ
\ruby{戀故}{こひ|ゆゑ}の
\ruby{信心}{しん|〴〵}で
\ruby{無}{な}かれかしと、よそながら
\ruby{我}{わ}が
\ruby{念}{ねん}じ
\ruby{{\換字{遣}}}{や}りし
\ruby{其人}{その|ひと}ならずや。
\ruby{何}{なに}をか
\ruby{獨}{ひと}り
\ruby{物思}{もの|おも}ひして
\ruby{睫毛}{まつ|げ}に
\ruby{露}{つゆ}を
\ruby{湛}{た\ninojiten}へし
\ruby{人}{ひと}ならずや。
\ruby{{\換字{滊}}車}{き|しや}の
\ruby{中}{うち}の
\ruby{素振}{そ|ぶり}、
\ruby[g]{先刻}{さつき}よりの
\ruby{應對}{おう|たい}、
\ruby{今}{いま}の
\ruby{此}{こ}の
\ruby{樣子}{やう|す}に、
\ruby[g]{一切}{すべて}は
\ruby{解}{わか}りたり。
\ruby{師匠}{し〻|やう}は
\ruby{碌}{ろく}にも
\ruby{我}{われ}に
\ruby{語}{かた}らざりしが、
\ruby{此人}{この|ひと}は
\ruby{是五十子}{これ|い|そ|こ}といへるに
\ruby{深}{ふか}く
\ruby{思}{おもひ}を
\ruby{懸}{か}けて
\ruby{戀}{こひ}せるなるべし。
\ruby{似合}{に|あ}はしからぬ
\ruby{佛頼}{ほとけ|だの}みにも
\ruby{其胸}{その|むね}の
\ruby{中}{うち}の
\ruby{苦}{くるし}さぞ
\ruby{知}{し}らる\ninojiten{}!。
\ruby{嗚呼一昨年}{あ|\ninojiten|をと|と|し}の
\ruby{我}{われ}を
\ruby[g]{男子}{をとこ}にして
\ruby{見}{み}る、
\ruby{其}{そ}の
\ruby{顏}{かほ}の
\ruby{愁}{うれひ}に
\ruby{痩}{や}せて
\ruby{{\換字{情}}無}{なさ|けな}い
\ruby{有樣}{あり|さま}!、
\ruby{其}{そ}の
\ruby{眼}{め}の
\ruby{戀}{こひ}に
\ruby{疲}{つか}れきつて
\ruby{和}{なご}やかなるところの
\ruby{彼}{あ}の
\ruby{乏}{とぼ}しさ!。
\ruby{血属}{ち|すぢ}や
\ruby{見寄}{み|より}の
\ruby{有}{あ}りは
\ruby{有}{あ}つても、まことに
\ruby{戀}{こひ}に
\ruby{惱}{なや}む
\ruby{時}{とき}は、いつか
\ruby[g]{孤獨}{ひとり}の
\ruby{身}{み}となり
\ruby{果}{は}て\ninojiten{}、
\ruby{誰一人}{たれ|ひと|り}
\ruby{味方}{み|かた}になつて
\ruby{泣}{な}いて
\ruby{{\換字{呉}}}{く}れるものも
\ruby{無}{な}いのが
\ruby{世}{よ}の
\ruby{{\換字{習}}}{ならひ}!。
あ\ninojiten{}
\ruby{憫然}{かはゆ|さう}な〳〵
\ruby{人}{ひと}!。
と
\ruby[g]{經驗}{おぼ{\換字{江}}}ある
\ruby{身}{み}の
\ruby{思}{おも}ひ
\ruby{{\換字{遣}}}{や}り
\ruby{深}{ふか}く、

『あ\ninojiten{}、
\ruby{眞實}{ほん|と}に
\ruby{左樣}{さ|う}でございます!。
\ruby{神樣佛樣}{かみ|さま|ほとけ|さま}よりほかには
\ruby{左樣}{さ|う}いふ
\ruby{時}{とき}には、
\ruby{御賴}{お|たの}み
\ruby{申}{まを}すところもございません。
\ruby[g]{歸路}{かへり}には
\ruby{淺草}{あさ|くさ}の
\ruby{觀音樣}{くわん|おん|さま}で、
\ruby{妾}{わたし}も
\ruby{御百度}{お|ひやく|ど}でも
\ruby{踏}{ふ}みまして、
\ruby{何樣}{ど|う}か
\ruby{快}{よ}く
\ruby{御}{お}なりなさるやうに
\ruby{願}{ねが}ひませう。
』

と
\ruby{云}{い}はれて
\ruby{水野}{みづ|の}も
\ruby{心嬉}{こゝろ|うれ}しく、

『そりやあ、
\ruby{有}{あ}り
\ruby{難}{がた}い
\ruby{御親切}{ご|しん|せつ}の
\ruby{事}{こと}です。
\ruby{何樣}{ど|う}か
\ruby{病人}{びやう|にん}の
\ruby{快}{い}いやうに
\ruby{祈}{いの}つて
\ruby{下}{くだ}さい。
』

と、
\ruby{全}{まつた}く
\ruby{{\換字{平}}凡}{た|ゞ}の
\ruby{人}{ひと}の
\ruby{如}{ごと}き
\ruby{挨拶}{あい|さつ}をすれば、

『アラ、
\ruby{何樣}{ど|う}したのだらう?
\ruby{先生}{せん|せい}が!。
\ruby{觀音樣}{くわん|おん|さま}なんかに
\ruby{祈}{いの}つて
\ruby{{\換字{呉}}}{く}れなんて!。
ホヽヽ、
\ruby{古}{ふる}ぼけた
\ruby[g]{老婆}{おばあさん}かなんか
\ruby{見}{み}たやうに。
』

と
\ruby{何知}{なに|し}らぬお
\ruby{濱}{はま}は
\ruby{之}{これ}を
\ruby{蔭}{かげ}にて
\ruby{聞}{き}きて、
\ruby{聞}{きこ}えぬほどに
\ruby{獨語}{ひと|りご}ちて
\ruby{笑}{わら}へり。

\ruby{命令}{いひ|つけ}られたる
\ruby{事}{こと}は
\ruby{大槪果}{おほ|よそ|はた}したれば、ここにお
\ruby{龍}{りう}ははじめて
\ruby{隙}{ひま}を
\ruby{得}{\換字{江}}て、

『つい
\ruby{申}{まを}しそびれて
\ruby{居}{を}りましたが
\ruby{先刻}{さき|ほど}は
\ruby{何樣}{ど|う}も、とんだ
\ruby{{\換字{過}}失}{そ|さう}を
\ruby{致}{いた}しました。
\ruby{此方}{こち|ら}へ
\ruby{上}{あが}つてお
\ruby{目}{め}にかかると、
\ruby{貴下}{あな|た}が
\ruby{其方}{その|かた}だつたのでまた
\ruby{吃驚致}{びつ|くり|いた}しましたのでございます。
お
\ruby{怪我}{け|が}をさせまして
\ruby{眞}{まこと}に
\ruby{濟}{す}みません、どうか
\ruby{御免}{ご|めん}なさつてくださいまし。
』

と
\ruby{改}{あらた}めて
\ruby{謝罪}{わ|び}れば
\ruby{水野}{みづ|の}は
\ruby{慨然}{がい|ぜん}として、

『ナアニ
\ruby{貴女}{あな|た}に
\ruby{踏}{ふ}まれて
\ruby{流}{なが}れた
\ruby{彼樣}{あ|ん}な
\ruby{紅}{あか}い
\ruby{水}{みづ}、
\ruby[g]{少許}{ちつと}や
\ruby[g]{若干量流}{そつとなが}れたつて
\ruby{何}{なに}が
\ruby{何}{なん}でしやう!。
ハヽハヽハヽ。
』

と
\ruby{裏枯}{うら|が}れたる
\ruby{聲}{こゑ}して
\ruby{自}{みづか}ら
\ruby{嘲}{あざけ}るやうに
\ruby{淋}{さび}しく
\ruby{笑}{わら}へり。
\ruby{其意}{その|こゝろ}を
\ruby{解}{と}きて
\ruby{知}{し}るよしも
\ruby{無}{な}けれど、
\ruby{其}{そ}の
\ruby{言葉}{こと|ば}の
\ruby{異樣}{こと|やう}にして
\ruby{其}{そ}の
\ruby[g]{調子}{てうし}の
\ruby{悲哀}{かな|しみ}を
\ruby{含}{ふく}めるに、
\ruby{感}{かん}じ
\ruby{易}{やす}きお
\ruby{龍}{りう}は
\ruby{一種}{いつ|しゆ}の
\ruby{感}{かん}に
\ruby{打}{う}たれて、
\ruby{頓}{とみ}には
\ruby{答}{こたへ}をさへ
\ruby{出}{いだ}しかねたり。

