\Entry{其三十二}

% メモ 校正終了 2024-04-27 2024-06-03
\原本頁{175-1}%
か〻る% 本来は一の字点「ゝ」平仮名繰返し記号% 原本通り「〻(二の字点、揺すり点)」
ところへ
\ruby{新}{あらた}に
\ruby{茶}{ちや}を
いれて
\ruby{持}{もち}
\ruby{來}{きた}りし
お
\ruby{濱}{はま}は、
%
はつきりと
\ruby[<j||]{美}{うつく}しき
\ruby{眼}{め}に
\ruby{優}{やさ}しく
お
\ruby{龍}{りう}を
\ruby{見}{み}て、
%
しとやかに
\ruby{其}{そ}の
\ruby{一盞}{いつ|さん}を
\ruby{取}{と}りて
\ruby{薦}{す〻}むれば、% 原本通り「〻(二の字点、揺すり点)」
%
\ruby{水野}{みづ|の}を
\ruby{見}{み}たる
\ruby{目}{め}を
\ruby{此}{この}
\ruby{人}{ひと}に
\ruby{移}{うつ}しては、
%
\ruby[<j||]{懷}{ふとこ}% 文字間を詰めるため親文字とルビ及び突き出しを調整している
\ruby[||j>]{暗}{ろくら}き
\ruby{常綠樹}{とき|は|ぎ}の
\ruby{高}{たか}く
%\原本頁{175-4}\改行%
\ruby{聳}{そび}{\換字{𛀁}}たるを
\ruby{見}{み}たる
\ruby{目}{め}に、
%
しほらしく
\ruby{{\換字{咲}}}{さ}く
\ruby{初}{はつ}
\ruby[||j>]{櫻}{ざくら}の、
%
ぱつと
\ruby{明}{あか}るき
\ruby{花}{はな}の
\ruby{枝}{{\換字{𛀁}}だ}を
\ruby{忽}{たちま}ち
\ruby{見}{み}たる
\ruby{心地}{こ〻|ち}して、% 原本通り「〻(二の字点、揺すり点)」
%
おのづから
\ruby{胸}{むね}も
\ruby{開}{ひら}くる
やうするに、
%
お
\ruby{龍}{りう}は、

\原本頁{175-7}%
『
どうも
はゞかり% TODO 原本の「二の字点、揺すり点」に濁点のグリフが見つからないので「ゞ」
さま、
%
\ruby{恐}{おそ}れ
\ruby{入}{い}ります。
』

\原本頁{175-8}%
と
\ruby{身}{み}を
\ruby{{\換字{謙}}{\換字{退}}}{へり|くだ}りて
\ruby{會釋}{ゑ|しやく}しつ、
%
\ruby{互}{たがひ}に
\ruby{顏}{かほ}を
\ruby{見合}{み|あ}はせしが、
%
\ruby{笑}{わら}ふ
とも
\ruby{無}{な}く
\ruby{嫣然}{につ|こり}と
したる
\ruby{彼}{かれ}
\ruby{此}{これ}
\ruby{一時}{いち|じ}の
\ruby{笑容}{ゑ|み}の
\ruby{中}{なか}に、
%
\ruby{語}{かた}らで
\ruby{語}{かた}り
\ruby{聞}{き}かで
\ruby{聞}{き}く
\ruby[||j>]{心}{こ〻ろ}と% 原本通り「〻(二の字点、揺すり点)」
\ruby[||j>]{心}{こ〻ろ}と% 原本通り「〻(二の字点、揺すり点)」
\ruby[||j>]{働}{はたら}きて、
%
\ruby{思}{おも}へば
\ruby{思}{おも}ひ
\ruby{好}{す}けば
\ruby{好}{す}く
\ruby{性}{しやう}の
\ruby{合}{あ}ふ
\ruby{同士}{ど|し}% 「同士」のルビは原本通り
\ruby{女同士}{をんな|ど|し}、% 「同士」のルビは原本通り
%
\ruby{何}{なに}の
\ruby{故}{ゆゑ}とは
\ruby{無}{な}けれども
\ruby{相}{あひ}
なつかしみ
\ruby{相}{あひ}
\ruby[||j>]{悅}{よろこ}びたり。

\原本頁{176-1}%
されど
お
\ruby{濱}{はま}は
\ruby{何時}{い|つ}まで
\ruby{此處}{こ|〻}に% 原本通り「〻(二の字点、揺すり点)」
あるべき
ならねば、
%
お
\ruby{龍}{りう}と
\ruby{物語}{もの|がた}りして
\ruby{{\換字{遊}}}{あそ}び
たき
やうの
\ruby{思}{おもひ}は
\ruby{仕}{し}ながら、
%
\ruby{一盞}{いつ|さん}を
\ruby{取}{と}りて
\ruby{水野}{みづ|の}に
\ruby{與}{あた}へて
\改行% 校正作業の簡略化のため
、
%
\原本頁{176-3}\改行%
\ruby{好}{よ}き
ほどの
ところに
\ruby{茶具}{ちや|ぐ}を
\ruby{置}{お}き
\ruby{捨}{す}て、
%
おのれは
\ruby{茶}{ちや}の
\ruby{間}{ま}に
\ruby{{\換字{退}}}{しりぞ}きて
\ruby{二人}{ふた|り}の
\ruby{話}{はなし}を
\ruby{聞}{き}けり。

\原本頁{176-5}%
お
\ruby{龍}{りう}は
\ruby{{\換字{猶}}}{なほ}
\ruby{五十子}{い|そ|こ}の
\ruby{容態}{よう|だい}を
\ruby{聞}{き}かでは
\ruby{叶}{かな}はざる
なり。

\原本頁{176-6}%
『
ほんとうに
\ruby{段々}{だん|〴〵}との
\ruby{深}{ふか}い
\ruby{御親切}{ご|しん|せつ}さ
まで、
%
まことに
\ruby{有}{あ}り
\ruby{{\換字{難}}}{がた}う
\ruby{存}{ぞん}じます。
%
\ruby{歸}{かへ}つて
\ruby{御言葉}{お|こと|ば}
\ruby{{\換字{通}}}{どほ}りに
\ruby{左樣}{さ|う}
\ruby{申}{まを}し
\ruby{傳}{つた}へ
ましたら、
%
\ruby{何樣}{ど|ん}なにか
\ruby{師匠}{し|〻やう}も% 原本通り「〻(二の字点、揺すり点)」
\ruby{悅}{よろこ}ぶ
ことで
ございましやう。
%
\ruby{左樣}{さ|う}
いたしまして
\ruby{只}{たゞ}% TODO 原本の「二の字点、揺すり点」に濁点のグリフが見つからないので「ゞ」
\ruby{今}{いま}は
\改行% 校正作業の簡略化のため
、
\原本頁{176-9}\改行%
%
\ruby[||j>]{病}{びやう}
\ruby[||j>]{人}{ にん}は
% \ruby{病人}{びやう|にん}は
\ruby{何樣}{ど|ん}な
\ruby{樣子}{やう|す}で
ございますの?
』

\原本頁{176-10}%
『
いや
\ruby{何樣}{ど|う}も
\ruby{中々}{なか|〳〵}
\ruby{良}{よ}く
ないのです。
%
それで
\ruby{大}{おほ}きに
\ruby{心配}{しん|ぱい}
\ruby{致}{いた}しましたが、
%
\ruby{淺草}{あさ|くさ}の
\ruby{醫者}{い|しや}を
\ruby{招}{よ}びに
\ruby{行}{ゆ}きました
\ruby{歸路}{かへ|り}に、
%
たつた
\ruby{今}{いま}
\ruby{此村}{こ|〻}の% 原本通り「〻(二の字点、揺すり点)」
%\原本頁{177-1}\改行%
\ruby{醫者}{い|しや}に
\ruby{容態}{よう|だい}を
\ruby{聞}{き}きましたら、
%
\ruby{大}{おほ}きに
\ruby{見直}{み|なほ}した
やうな
\ruby{工合}{ぐ|あひ}で
して
\改行% 校正作業の簡略化のため
、
%
\原本頁{177-2}\改行%
\ruby[||j>]{重}{ぢう}
\ruby[||j>]{病}{びやう}% 原本通り「重(ぢう)」
だから
\ruby{何}{なん}とも
\ruby{云}{い}へないが、
%
\ruby{此}{この}
\ruby{儘}{ま〻}で% 原本通り「〻(二の字点、揺すり点)」
\ruby{日}{ひ}さへ
\ruby{經}{へ}て
\ruby{吳}{く}れ〻ば% 原本通り「〻(二の字点、揺すり点)」
まあ
\ruby{宜}{よ}い
といふので‥‥‥
』

\原本頁{177-4}%
『
では
\ruby{食事}{しよく|じ}
なんどは?。
』

\原本頁{177-5}%
『
なか〳〵
まだ
\ruby{食事}{しよく|じ}
なんぞといふ
\ruby{段}{だん}では
\ruby{無}{な}いので。
%
やつと
\ruby{流動物}{りう|どう|ぶつ}が
\ruby{小量}{すこ|し }% 「 (全角空白)」は「許(ばかり)」の前突出対策
\ruby[<j||]{許}{ばかり}
\ruby{入}{はい}る
\ruby{位}{くらゐ}です。
%
しかし
\ruby{變}{へん}さへ
\ruby{無}{な}ければ、
%
\ruby{大抵}{たい|てい}は
\ruby{經{\換字{過}}}{けい|くわ}
\ruby{日數}{につ|すう}が
\ruby{定}{きま}つて
\ruby{居}{ゐ}る
ものださう
ですから。
』

\原本頁{177-8}%
『
\ruby{案}{あん}じる
やうな
\ruby{事}{こと}は
まあ
\ruby{無}{な}い
のでございますか。
』

\原本頁{177-9}%
『
\ruby{左樣}{さ|う}
ばかりにも
いきますまいが。
』

\原本頁{177-10}%
『
\ruby{變}{へん}の
\ruby{無}{な}いやうに
\ruby{致}{いた}しかたは
\ruby{無}{な}い
もので
ございましやうか。
』

\原本頁{177-11}%
『
そりやあ
\ruby{左樣}{さ|う}
したいのは
\ruby{山々}{やま|〳〵}ですが、
%
\ruby[||j>]{{\換字{情}}}{なさけ}
\ruby[||j>]{無}{ な}い
\ruby{事}{こと}には
\ruby{醫者}{い|しや}の
\ruby[<j||]{力}{ちから}でも% 行末行頭の境界付近なので特例処置を施す
\ruby{其處}{そ|こ}までは
\ruby{何樣}{ど|う}も
なりません。
』

\原本頁{178-2}%
『
それ
ぢやあ
\ruby{神樣}{かみ|さま}に
でも
\ruby{御願}{お|ねがひ}
\ruby{申}{まを}す
より
ほかには?。
』

\原本頁{178-3}%
『
\ruby{然樣}{さ|う}です。
%
とても
まあ
\ruby{其樣}{そ|ん}な
\ruby{事}{こと}より
ほかには!。
』

\原本頁{178-4}%
\ruby[||j>]{男}{をとこ}の
\ruby{聲}{こゑ}は
こ〻に% 本来は一の字点「ゝ」平仮名繰返し記号% 原本通り「〻(二の字点、揺すり点)」
\ruby{至}{いた}つて
\ruby{甚}{いた}く
\ruby{沈}{しづ}めり。
%
お
\ruby{龍}{りう}は
\ruby{忽然}{こつ|ぜん}として
\ruby{思}{おも}ひ
\ruby{{\換字{浮}}}{うか}ぶる
ところ
あり。
%
\ruby{我}{われ}に
\ruby{對}{むか}へる
\ruby{此}{この}
\ruby{人}{ひと}は
\ruby{誰}{たれ}ぞ。
%
この
\ruby{人}{ひと}は
\ruby{是}{これ}
\ruby{彼}{か}の
\ruby{普門品}{ふ|もん|ぼん}の
\ruby{主}{ぬし}
ならずや。
%
\ruby{何}{なに}をか
\ruby{獨}{ひと}り
\ruby{物思}{もの|おも}ひして
\ruby{睫毛}{まつ|げ}に
\ruby{露}{つゆ}を
\ruby{湛}{たゞ}へし% TODO 原本の「二の字点、揺すり点」に濁点のグリフが見つからないので「ゞ」
\ruby{人}{ひと}
ならずや。
%
あはれ
\ruby{戀}{こひ}
\ruby{故}{ゆゑ}の
\ruby{信心}{しん|〴〵}で
\ruby{無}{な}かれかしと、
%
よそながら
\ruby{我}{わ}が
\ruby{念}{ねん}じ
\ruby{{\換字{遣}}}{や}りし
\ruby{其}{その}
\ruby{人}{ひと}
ならずや。
%
\ruby{滊車}{き|しや}の
\ruby{中}{うち}の
\ruby{素振}{そ|ぶり}、
%
\ruby{先刻}{さつ|き}よりの
\ruby{應對}{おう|たい}、
%
\ruby{今}{いま}の
\原本頁{178-9}\改行%
\ruby{此}{こ}の
\ruby{樣子}{やう|す}に、
%
\ruby{一切}{すべ|て}は
\ruby{解}{わか}りたり。
%
\ruby{師匠}{し|〻やう}は% 原本通り「〻(二の字点、揺すり点)」
\ruby{碌}{ろく}にも
\ruby{我}{われ}に
\ruby{語}{かた}らざりしが
\改行% 校正作業の簡略化のため
、
%
\原本頁{178-10}\改行%
\ruby{此}{この}
\ruby{人}{ひと}は
\ruby{是}{これ}
\ruby{五十子}{い|そ|こ}と
いへるに
\ruby{深}{ふか}く
\ruby{思}{おもひ}を
\ruby{懸}{か}けて
\ruby{戀}{こひ}せる
なるべし。
%
\ruby{似合}{に|あ}はし
からぬ
\ruby[||j>]{佛}{ほとけ}
\ruby[||j>]{頼}{ だの}みにも
\ruby{其}{その}
\ruby{胸}{むね}の
\ruby{中}{うち}の
\ruby{苦}{くるし}さぞ
\ruby{知}{し}らる〻!。% 本来は一の字点「ゝ」平仮名繰返し記号% 原本通り「〻(二の字点、揺すり点)」
%
\ruby{嗚呼}{あ|〻}% 原本通り「〻(二の字点、揺すり点)」
\ruby{一昨年}{をと|と|し}の% 原本通りのルビ% 行末行頭禁則につき非踊り字表記
\ruby{我}{われ}を
\ruby{男子}{をと|こ}にして
\ruby{見}{み}る、
%
\ruby{其}{そ}の
\ruby{顏}{かほ}の
\ruby{愁}{うれひ}に
\ruby{痩}{や}せて
\ruby{{\換字{情}}無}{なさけ|な}い
\ruby{有樣}{あり|さま}!
\改行% 校正作業の簡略化のため
、
%
\原本頁{179-2}\改行%
\ruby{其}{そ}の
\ruby{眼}{め}の
\ruby{戀}{こひ}に
\ruby{疲}{つか}れ
きつて
\ruby{和}{なご}やか
なるところの
\ruby{彼}{あ}の
\ruby{乏}{とぼ}しさ!。
%
\ruby{血属}{ち|すぢ}や
\ruby{見寄}{み|より}の
\ruby{有}{あ}りは
\ruby{有}{あ}つても、
%
まことに
\ruby{戀}{こひ}に
\ruby{惱}{なや}む
\ruby{時}{とき}は、
%
いつか
\ruby{孤獨}{ひと|り}の
\ruby{身}{み}となり
\ruby{果}{は}て〻、% 本来は一の字点「ゝ」平仮名繰返し記号% 原本通り「〻(二の字点、揺すり点)」
%
\ruby{誰}{たれ}
\ruby{一人}{ひと|り}
\ruby{味方}{み|かた}になつて
\ruby{泣}{な}いて
\ruby{吳}{く}れる
ものも
\原本頁{179-5}\改行%
\ruby{無}{な}いのが
\ruby{世}{よ}の
\ruby{{\換字{習}}}{ならひ}!。
%
あ〻% 本来は一の字点「ゝ」平仮名繰返し記号% 原本通り「〻(二の字点、揺すり点)」
\ruby[||j>]{憫}{かは}
\ruby[||j>]{然}{ゆさう}な〳〵% 「憫然 か(は)ゆさう」
% \ruby{憫然}{かは|ゆさう}な〳〵% 「憫然 か(は)ゆさう」
\ruby{人}{ひと}!。
%
と
\ruby{經驗}{おぼ|{\換字{𛀁}}}
ある
\ruby{身}{み}の
\ruby{思}{おも}ひ
\ruby{{\換字{遣}}}{や}り
\ruby{深}{ふか}く、

\原本頁{179-7}%
『
あ〻、% 本来は一の字点「ゝ」平仮名繰返し記号% 原本通り「〻(二の字点、揺すり点)」
%
\ruby{眞實}{ほん|と}に
\ruby{左樣}{さ|う}で
ございます!。
%
\ruby{神樣}{かみ|さま}
\ruby[||j>]{佛}{ほとけ}
\ruby[||j>]{樣}{ さま}より% 「 (全角空白)」は「佛(ほとけ)」の後突出対策
ほかには
\ruby{左樣}{さ|う}いふ
\ruby{時}{とき}には、
%
\ruby{御賴}{お|たの}み
\ruby{申}{まを}す
ところも
ございません。
%
\ruby{歸路}{かへ|り}には
\ruby{淺草}{あさ|くさ}の
\ruby[||j>]{觀}{くわん}% 「觀音」の読みは原本通り「くわん(お)ん」
\ruby[||j>]{音}{ おん}
\ruby[||j>]{樣}{ さま}で、
%
\ruby{妾}{わたし}も
\ruby{御}{お}
\ruby{百度}{ひやく|ど}でも
\ruby{踏}{ふ}みまして、
%
\ruby{何樣}{ど|う}か
\ruby{快}{よ}く
\ruby{御}{お}なり
なさるやうに
\ruby{願}{ねが}ひませう。
』

\原本頁{179-11}%
と
\ruby{云}{い}はれて
\ruby{水野}{みづ|の}も
\ruby[||j>]{心}{こ〻ろ}% 原本通り「〻(二の字点、揺すり点)」
\ruby[||j>]{嬉}{ うれ}しく、

\原本頁{180-1}%
『
そりやあ
%
\ruby{有}{あ}り
\ruby{{\換字{難}}}{がた}い
\ruby{御親切}{ご|しん|せつ}の
\ruby{事}{こと}です。
%
\ruby{何樣}{ど|う}か
\ruby[||j>]{病}{びやう}
\ruby[||j>]{人}{ にん}の
\ruby{快}{い}いやうに
\ruby{祈}{いの}つて
\ruby{下}{くだ}さい。
』

\原本頁{180-3}%
と、
%
\ruby{全}{まつた}く
\ruby{{\換字{平}}凡}{た|ゞ}の% TODO 原本の「二の字点、揺すり点」に濁点のグリフが見つからないので「ゞ」
\ruby{人}{ひと}の
\ruby{如}{ごと}き
\ruby{挨拶}{あい|さつ}をすれば、

\原本頁{180-4}%
『
アラ
%
\ruby{何樣}{ど|う}した
のだらう?\inhibitglue{}%
\ruby{先生}{せん|せい}が!。
%
\ruby[||j>]{觀}{くわん}% 「觀音」の読みは原本通り「くわん(の)ん」
\ruby[||j>]{音}{ のん}
\ruby[||j>]{樣}{ さま}
なんかに
\ruby{祈}{いの}つて
\ruby{吳}{く}れなんて!。
%
ホヽヽ、
%
\ruby{{\換字{古}}}{ふる}ぼけた
\ruby[||j>]{老}{おばあ}
\ruby[||j>]{婆}{ さん}かなんか
\ruby{見}{み}たやうに。
』

\原本頁{180-6}%
と
\ruby{何}{なに}
\ruby{知}{し}らぬ
お
\ruby{濱}{はま}は
\ruby{之}{これ}を
\ruby{蔭}{かげ}にて
\ruby{聞}{き}きて、
%
\ruby{聞}{きこ}えぬ
ほどに
\ruby[||j>]{獨}{ひとり}
\ruby[||j>]{語}{ ご }ちて
\ruby{笑}{わら}へり。

\原本頁{180-8}%
\ruby{命令}{いひ|つけ}られたる
\ruby{事}{こと}は
\ruby{大槪}{おほ|よそ}
\ruby{果}{はた}したれば、
%
ここに
お
\ruby{龍}{りう}は
はじめて
\ruby{隙}{ひま}を
\ruby{得}{{\換字{𛀁}}}て、

\原本頁{180-10}%
『
つい
\ruby{申}{まを}し
そびれて
\ruby{居}{を}りましたが
\ruby{先刻}{さき|ほど}は
\ruby{何樣}{ど|う}も、
%
とんだ
\ruby{{\換字{過}}失}{そ|さう}を
\ruby{致}{いた}しました。
%
\ruby{此方}{こち|ら}へ
\ruby{上}{あが}つて
お
\ruby{目}{め}に
かかると、
%
\ruby{貴下}{あな|た}が
\ruby{其}{その}
\ruby{方}{かた}
だつたので
また
\ruby{吃驚}{びつ|くり}
\ruby{致}{いた}しました
ので
ございます。
%
お
\ruby{怪我}{け|が}を
させまして
\ruby{眞}{まこと}に
\ruby{濟}{す}みません、
%
どうか
\ruby{御免}{ご|めん}
なさつて
\ruby{下}{くだ}さいまし。
』

\原本頁{181-3}%
と
\ruby{改}{あらた}めて
\ruby{謝罪}{わ|び}れば
\ruby{水野}{みづ|の}は
\ruby{慨然}{がい|ぜん}として、

\原本頁{181-4}%
『
ナアニ
\ruby{貴女}{あな|た}に
\ruby{踏}{ふ}まれて
\ruby{流}{なが}れた
\ruby{彼樣}{あ|ん}な
\ruby{紅}{あか}い
\ruby{水}{みづ}、
%
\ruby{少許}{ちつ|と}や
\ruby{{\換字{若}}干量}{そ|つ|と}
\ruby{流}{なが}れ
たつて
\ruby{何}{なに}が
\ruby{何}{なん}でしやう!。
%
ハヽハヽハヽ。
』

\原本頁{181-6}%
と
\ruby{裏枯}{うら|が}れたる
\ruby{聲}{こゑ}して
\ruby{自}{みづか}ら
\ruby{嘲}{あざけ}るやうに
\ruby{淋}{さび}しく
\ruby{笑}{わら}へり。
%
\ruby{其}{その}
\ruby[||j>]{意}{こ〻ろ}を% 原本通り「〻(二の字点、揺すり点)」
\ruby{解}{と}きて
\ruby{知}{し}るよしも
\ruby{無}{な}けれど、
%
\ruby{其}{そ}の
\ruby{言葉}{こと|ば}の
\ruby{異樣}{こと|やう}にして
\ruby{其}{そ}の
\ruby{調子}{てう|し}の
\ruby{悲哀}{かな|しみ}を
\ruby{含}{ふく}めるに、
%
\ruby{{\換字{感}}}{かん}じ
\ruby{易}{やす}き
お
\ruby{龍}{りう}は
\ruby{一種}{いつ|しゆ}の
\ruby{{\換字{感}}}{かん}に
\ruby{打}{う}たれて、
%
\ruby{頓}{とみ}には
\ruby{答}{こたへ}を
さへ
\ruby{出}{いだ}し
かねたり。
