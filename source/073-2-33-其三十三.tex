\Entry{其三十三}

% メモ 校正終了 2024-04-28 2024-06-04
\原本頁{182-1}%
お
\ruby{龍}{りう}は
\ruby{徐}{しづか}に
\ruby[g]{三絃}{さみせん}の
\ruby{絃}{いと}を
\ruby{弛}{ゆる}めて
\ruby[g]{三絃}{さみせん}
\ruby{掛}{かけ}へ
\ruby{掛}{か}け
\ruby{納}{をさ}むれば、
%
\ruby[g]{今日}{け ふ }
\ruby{目}{め}
\ruby{見}{み}
\ruby{得}{{\換字{𛀁}}}に
\ruby{來}{きた}りし
\ruby[g]{小婢}{こをんな}
お
\ruby{熊}{くま}は
\ruby[g]{高麗}{こ ま }
\ruby{鼠}{ねずみ}
のやうに
くる〳〵と
\ruby{働}{はたら}きて、
%
しきりに
\ruby[g]{其邊}{そこら }を
\ruby{取}{と}り
\ruby[g]{片付}{かたづ }けしが、
%
\ruby[g]{{\換字{煙}}草}{たばこ }
\ruby{{\換字{盆}}}{ぼん}の
\ruby[<j>]{傍}{かたはら}より
\ruby{玉}{ぎよく}の
\ruby[g]{{\換字{煙}}管}{パイプ}の
いと
\ruby[<j||]{小}{ちひさ}なるを
\ruby{拾}{ひろ}ひ
あげて
\ruby[g]{洋燈}{らんぷ }
\ruby{{\換字{近}}}{ちか}く
さし
\ruby{出}{いだ}し、

\原本頁{182-5}%
『
これ
\ruby[g]{此樣}{こ ん }な
\ruby{物}{もの}が
\ruby{{\換字{遺}}}{お}ちて
\ruby{居}{を}りました、
』

\原本頁{182-6}%
といふ。

\原本頁{182-7}%
\ruby{一}{ひ}ト
\ruby{目}{め}
\ruby{見}{み}て
お
\ruby{龍}{りう}は
それを
\ruby[g]{師匠}{しゝやう}に% 踊り字調整「〻(二の字点、揺すり点)に見えるが(ゝ)」
\ruby[g]{遞與}{わ た }し、

\原本頁{182-8}%
『
こりやあ
\ruby{傳}{でん}さんが
\ruby{{\換字{遺}}}{わす}れて
\ruby{行}{い}つた
のでしやう。
%
あの
\ruby{人}{ひと}で
\ruby{無}{な}け
りやあ
\ruby[g]{此樣}{こ ん }なものを
\ruby{持}{も}ち
さうな
\ruby{人}{ひと}は
ありませんから。
』

\原本頁{182-10}%
と
\ruby{云}{い}へば、
%
お
\ruby{關}{せき}は
\ruby[g]{受取}{うけと }つて
\ruby[g]{指頭}{ゆびさき}に
\ruby[<j>]{弄}{もてあそ}び、

\原本頁{182-11}%
『
あゝ% 踊り字調整「〻(二の字点、揺すり点)に見えるが(ゝ)」
\ruby[g]{然樣}{さ う }だよ、
%
\ruby[g]{屹度}{きつと }
\ruby{彼}{あ}の
\ruby{男}{をとこ}のだよ。
%
\ruby[g]{今日}{け ふ }は
\ruby{妾}{わたし}も
\ruby[g]{大變}{たいへん}
\ruby{夙}{はや}
\ruby{起}{おき}を
\ruby{仕}{し}たし、
%
\ruby{汝}{おまへ}も
\ruby{{\換字{遠}}}{とほ}い
ところへ
\ruby{行}{い}つて
\ruby{來}{き}たので
\ruby[g]{草臥}{くたびれ}て
\ruby{居}{ゐ}る
からつて
いふので
\ruby{{\換字{逐}}}{お}ひ
\ruby{立}{た}てゝ% 踊り字調整「〻(二の字点、揺すり点)に見えるが(ゝ)」
やつたもんだから、
%
\ruby{慌}{あわ}てゝ% 踊り字調整「〻(二の字点、揺すり点)に見えるが(ゝ)」
\ruby{歸}{かへ}つて
\ruby{行}{い}つて
\ruby{{\換字{遺}}}{わす}れたん
だらう。
%
\ruby{取}{と}り
\ruby{上}{あ}げて
\ruby[g]{仕舞}{し ま }つて
\ruby{{\換字{遣}}}{や}らうか
\ruby{知}{し}らん。
%
ハヽヽ、
%
マア
\ruby[g]{堪{\換字{忍}}}{かんにん}して% 原文通り「堪忍」
\ruby{{\換字{遣}}}{や}ると
\ruby{仕}{し}やう。
%
\ruby{何}{なん}でも
\ruby{彼}{あ}の
\ruby{男}{をとこ}は
\ruby[g]{親類}{しんるゐ}
\ruby{内}{うち}か
なんぞに、
%
\原本頁{183-5}\改行%
\ruby{玉}{たま}や
\ruby{石}{いし}の
\ruby[g]{細工}{さいく }
をする
\ruby{家}{うち}か
なんぞを
\ruby{有}{も}つて
\ruby{居}{ゐ}るんだよ。
%
\ruby[g]{御覧}{ご らん}よ、
%
\ruby[||j>]{小}{ちひさ}い
けれども
\ruby[g]{此品}{こ れ }だつて
\ruby{買}{か}つたら
\ruby{{\換字{廉}}}{やす}くは
なさゝうなものだ\換字{子}。% 踊り字調整「〻(二の字点、揺すり点)に見えるが(ゝ)」
』

\原本頁{183-7}%
と、
%
\ruby[g]{一度}{ひとたび}は
お
\ruby{龍}{りう}に
\ruby{示}{しめ}して、
%
さて
\ruby[g]{火鉢}{ひ ばち}の
\ruby[g]{抽斗}{ひきだし}に
\ruby{無}{む}
\ruby[g]{{\換字{造}}作}{ざうさ }に
\ruby{藏}{しま}ひたり
\改行% 校正作業の簡略化のため
。

\原本頁{183-8}%
『
ハア
\ruby[g]{左樣}{さ う }
なんで
しやうよ。
%
\ruby{兎}{うさぎ}を
\ruby{吳}{く}れたんでも
\ruby{{\換字{分}}}{わか}つて
\ruby{居}{ゐ}ますよ。
%
\ruby[g]{屹度}{きつと }
\ruby[g]{叔{\換字{父}}}{を ぢ }さんか
\ruby{何}{なに}かゞ% 踊り字調整「〻(二の字点、揺すり点)に濁点に見えるが(ゞ)」
\ruby[g]{玉屋}{たまや }さん
なんです\換字{子}。
』

\原本頁{183-10}%
『
\ruby[g]{何樣}{ど う }も
\ruby[g]{左樣}{さ う }らしいよ。
%
\ruby{妾}{わたし}も
\ruby[g]{往日}{いつか }
\ruby[g]{瑪瑙}{め なう}の
\ruby{好}{い}い
\ruby{色}{いろ}の
\ruby[<j>]{簪}{かんざし}
\ruby{珠}{ だま}を% 「 (全角空白)」は「簪(かんざし)」の後突出対策
\ruby{貰}{もら}つたがね、
%
\ruby{汝}{おまへ}、
%
\ruby{兎}{うさぎ}
なんぞぢや
\ruby[g]{仕樣}{し やう}が
\ruby{無}{な}いぢや
\ruby{無}{な}いか。
%
\ruby[g]{今度}{こんど }は
\ruby[g]{寶石}{い し }
\ruby{入}{い}りの
\ruby[g]{指輪}{ゆびわ }か
なんか
\ruby[g]{{\換字{強}}{\換字{請}}}{ね だ }つて
\ruby[g]{御{\換字{遣}}}{お や }りナ。
%
\ruby{金剛石}{ダ|イ|ヤ}とでも
\ruby{云}{いつ}たら
\ruby{二}{に}の
\ruby{足}{あし}を
\ruby{踏}{ふ}むか
\ruby{知}{し}らないが、
%
サフイヤや
\ruby[g]{眞珠}{しんじゆ}の
\ruby{位}{ぐらゐ}なら
\ruby[g]{屹度}{きつと }
\ruby{二}{ふた}ツ
\ruby[g]{{\換字{返}}事}{へんじ }で
\ruby[||j>]{悅}{よろこ}んで
\ruby{持}{も}つて
\ruby{來}{く}るよ。
%
\ruby{物}{もの}を
\ruby{取}{と}つて
\ruby{{\換字{遣}}}{や}るのも
\ruby[g]{功徳}{く どく}に
なるの
だから
\ruby{關}{かま}やあ
\ruby{仕}{し}ない
\ruby{吹}{ふつ}かけて
\ruby[g]{御覧}{ご らん}、
%
\ruby[g]{相槌}{あひづち}は
\ruby{妾}{わたし}が
\ruby{巧}{うま}く
\ruby{打}{う}つて
\ruby{上}{あ}げるか
\改行% 校正作業の簡略化のため
ら。
』

\原本頁{184-6}%
『
あら
\ruby{{\換字{嫌}}}{いや}な
\ruby{御師匠}{お|し|よ}さん!。
%
\ruby{妾}{わたし}あ
\ruby[g]{指輪}{ゆびわ }
なんか
\ruby{欲}{ほ}しかあ
\ruby{無}{な}いんですよ。
%
しかも
\ruby{傳}{でん}さんに
なんか
\ruby{貰}{もら}ひたかあ
\ruby{有}{あ}りません。
』

\原本頁{184-8}%
『
\ruby[g]{然樣}{さ う }かネエ。
%
\ruby{汝}{おまへ}は
ほんとに
\ruby{慾}{よく}に
\ruby{掛}{か}けちやあ
\ruby{氣}{き}が
\ruby{{\換字{弱}}}{よわ}いよ。
%
だが
\ruby{取}{と}つて
\ruby{{\換字{遣}}}{や}る
\ruby{方}{はう}が
\ruby{可}{いゝ}ぢやあ% 踊り字調整「〻(二の字点、揺すり点)に見えるが(ゝ)」
\ruby{無}{な}いか。
%
あの
\ruby{兎}{うさぎ}でも
\ruby{知}{し}れてるは\換字{子}、
%
\ruby[<j||]{汝}{おまへ}の% 行末行頭の境界付近なので特例処置を施す
\ruby{氣}{き}に
\ruby{入}{い}つたのを
\ruby{見}{み}て
\ruby[g]{何樣}{ど ん }なに
\ruby{嬉}{うれし}がつてるか
\ruby{知}{し}れや
\ruby{仕}{し}ないよ。
』

\原本頁{184-11}%
『
だから
\ruby{妾}{わたし}あ
\ruby{厭}{いや}なんですよ。
%
その
\ruby{嬉}{うれ}し
がられるのが
\ruby[g]{氣障}{き ざ }ぢや
\ruby{有}{あ}りませんか。
』

\原本頁{185-2}%
『
ホイ
\ruby{大失敗}{おほ|しく|じり}だネ、
%
ハヽハヽハヽ。
%
\ruby[g]{指輪}{ゆびわ }の
\ruby{談}{はなし}で
\ruby{想}{おも}ひ
\ruby{出}{だ}したが、
%
\ruby{先}{せん}に
\ruby{汝}{おまへ}が
あの
\ruby{何}{なん}に
(
\ruby{源}{げん}を
\ruby{指}{さ}す
)% 原本では、この注釈の「(」「)」の影響で、この行は29文字になっているが
%   調整できないのでこの行の調整は行わない
\ruby[g]{御貰}{お もら}ひのは
\ruby[||j>]{汝}{おまへ}
\ruby[||j>]{有}{ も }つて
おいでゝ% 踊り字調整「〻(二の字点、揺すり点)に見えるが(ゝ)」
\ruby{無}{な}いネエ。
%
\ruby{妾}{わたし}が
\ruby[g]{見立}{み た }てゝ% 踊り字調整「〻(二の字点、揺すり点)に見えるが(ゝ)」
\ruby{買}{か}はせたん
だから
まだ
\ruby{記}{おぼ}えて% 送り仮名は原本通り「え」
\ruby{居}{ゐ}るが、
%
\ruby[||j>]{汝}{おまへ}
\ruby[g]{彼品}{あ れ }は
\ruby[g]{何樣}{ど う }か
\ruby{仕}{し}て
お
\ruby[g]{仕舞}{し まひ}かエ。
』

\原本頁{185-6}%
『
だつて
\ruby{御師匠}{お|し|よ}さん、
%
まだ
\ruby{妾}{わたし}が
\ruby[g]{彼品}{あ れ }を
\ruby{持}{も}つて
\ruby{居}{ゐ}やう
\ruby{譯}{わけ}は
\ruby{無}{な}からうぢや
\ruby{有}{あ}りませんか。
%
いよ〳〵
\ruby[g]{不實}{ふ じつ}な
\ruby{人}{ひと}だと
\ruby{思}{おも}ひつめた
\ruby{時}{とき}は、
%
\ruby[g]{口惜}{く やし}くつて
\ruby[g]{口惜}{く やし}くつて
\ruby[g]{仕方}{し かた}が
\ruby{無}{な}かつた
んですもの!。
%
\ruby{宿}{と}めて
\ruby{貰}{もら}つて
\ruby{居}{ゐ}た
\ruby{藥研堀}{や|げん|ぼり}の
おとうさん%
{---}{---}%
\ruby{御師匠}{お|し|よ}さんは
\ruby[g]{御知}{お し }ん
なさらないが
\ruby{妾}{わたし}の
\ruby[g]{仲好}{なかよ }しの
\ruby{其}{そ}の
\ruby{家}{うち}を
\ruby{出}{で}て、
%
をかアしな
\ruby{氣}{き}になつて
ふらふらと% ルビ調整(原本通り)非踊り字表記(行末行頭の境界付近)
\ruby[||j>]{兩}{りやう}
\ruby[||j>]{國}{ ごく}
\ruby[||j>]{橋}{ ばし}の
% \ruby{兩國橋}{りやう|ごく|ばし}の
\ruby{上}{うへ}を
\ruby{往}{い}つたり
\ruby{復}{かへ}つたりした
\ruby{其}{そ}の
\ruby[g]{擧句}{あげく }でした、
%
ふいと
\ruby[||j>]{意}{こゝろ}% 踊り字調整「〻(二の字点、揺すり点)に見えるが(ゝ)」
\ruby[||j>]{持}{ もち}が
\ruby{變}{かは}つたんで
\ruby{指}{ゆび}から
\ruby{脫}{はづ}して、
%
\ruby[g]{大川}{おほかは}の
\ruby{流}{なが}れの
\ruby{中}{なか}へ
\ruby{抛}{はふ}り
\ruby{{\換字{込}}}{こ}んで
\ruby[g]{仕舞}{し ま }つたんですよ。
』

\原本頁{186-3}%
『
ヘーエ、
%
\ruby[g]{勿體}{もつたい}
\ruby{無}{な}い
\ruby{事}{こと}を
\ruby[g]{御仕}{お し }だつた\換字{子}エ、
%
マア
\ruby{妾}{わたし}なら
\ruby{同}{おな}じ
\ruby{棄}{す}てるにも
お
\ruby{金}{かね}に
\ruby{仕}{し}て
\ruby{棄}{す}てるものを。
%
だが
\ruby{鑄掛松}{ゐ|かけ|まつ}を
\ruby[g]{色氣}{いろけ }で
\ruby{行}{い}つたのは、
%
\ruby[g]{一寸}{ちよつと}
\ruby{覗}{のぞ}いて
\ruby{見}{み}たい
やうな
\ruby{幕}{まく}だつた\換字{子}。
』

\原本頁{186-6}%
『
ホヽヽ、
%
\ruby{厭}{いや}ですよ。
%
たんと
\ruby[g]{御嬲}{お なぶ}り
なさい、
%
\ruby{人}{ひと}の
\ruby{惡}{わる}い!。
%
\ruby{今}{いま}なら
\ruby{妾}{わたし}だつて‥‥‥。
』

\原本頁{186-8}%
『
\ruby[g]{何樣}{ど う }
\ruby[g]{御仕}{お し }だエ?、
』

\原本頁{186-9}%
『
\ruby[g]{御魚}{おさかな}にやあ
\ruby{與}{や}らないで
\ruby[g]{瞽女}{ご ぜ }にでも
\ruby{與}{や}ります。
』

\原本頁{186-10}%
『
\ruby[g]{{\換字{分}}別}{ふんべつ}らしい
けれども
\ruby[g]{{\換字{猶}}且}{やつぱり}
\ruby{{\換字{若}}}{わか}い\換字{子}エ。
%
ハヽヽ、
%
\ruby[g]{瞽女}{ご ぜ }が
\ruby[||j>]{汝}{おまへ}
\ruby[||j>]{狂}{ くる}ひ
\ruby{浪}{なみ}の
\ruby{彫}{ほり}に
\ruby{小}{ちひさ}な
\ruby[g]{寶石}{い し }の
\ruby{散}{ち}らばつて
\ruby{居}{ゐ}る
\ruby[g]{彼樣}{あ ん }な
\ruby[g]{華麗}{はでやか}な
\ruby{物}{もの}を
\ruby{指}{ゆび}に
\ruby{嵌}{は}めて
\ruby[g]{何樣}{ど う }なるものかネ。
』

\原本頁{187-2}%
『
ぢやあ
\ruby{御師匠}{お|し|よ}さんが
\ruby{妾}{わたし}だつたら
\ruby[g]{何樣}{ど う }なさるの?。
』

\原本頁{187-3}%
お
\ruby{關}{せき}は
\ruby{我}{わ}が
\ruby{鼻}{はな}を
\ruby{指}{ゆび}さしながら、

\原本頁{187-4}%
『
\ruby[g]{此處}{こ ゝ }に% 踊り字調整「〻(二の字点、揺すり点)に見えるが(ゝ)」
\ruby{居}{ゐ}る
\ruby[g]{美麗}{き れい}な
\ruby[g]{可憐}{かはゆ }らしい
\ruby[g]{新{\換字{造}}}{しんぞ }に
\ruby{與}{や}つて
\ruby{悅}{よろこ}ばせるはネ。
』

\原本頁{187-5}%
と
\ruby{云}{い}ひさして、
%
ハヽハヽハヽと
\ruby{打}{うち}
\ruby{笑}{わら}へば、
%
お
\ruby{龍}{りう}も
ホヽと
\ruby{笑}{わら}ひ
\ruby{出}{だ}し、
%
\ruby{臺}{だい}
\ruby{{\換字{所}}}{どころ}の
\ruby{方}{かた}に
\ruby{{\換字{退}}}{しりぞ}きたる
お
\ruby{熊}{くま}さへ
\ruby{貰}{もら}ひ
\ruby{笑}{わら}ひしたり。

\原本頁{187-7}%
『
あゝ、% 踊り字調整「〻(二の字点、揺すり点)に見えるが(ゝ)」
%
\ruby{笑}{わら}つたんで
\ruby[||j>]{心}{こゝろ}% 踊り字調整「〻(二の字点、揺すり点)に見えるが(ゝ)」
\ruby[||j>]{持}{ もち}が
\ruby{佳}{い}い。
%
さあ
お
\ruby{熊}{くま}や
\ruby[g]{方々}{はう〴〵}
\ruby[g]{{\換字{戸}}締}{と じま}りを
\ruby{仕}{し}て
お
\ruby[g]{仕舞}{おしま }ひ。
%
お
\ruby{龍}{りう}ちやんも
\ruby[g]{歸路}{かへり }に
\ruby{御}{お}
\ruby[g]{百度}{ひやくど}まで
\ruby{踏}{ふ}んで
\ruby{御}{お}
\ruby{吳}{く}れぢやあ
\改行% 校正作業の簡略化のため
、
%
\原本頁{187-9}\改行%
ほんとに
\ruby[g]{隨{\換字{分}}}{ずゐぶん}
おくたびれ
だらう。
』

\原本頁{187-10}%
\ruby[||j|]{隨}{こゝろ}% 踊り字調整「〻(二の字点、揺すり点)に見えるが(ゝ)」
\ruby[||j|]{意}{まかせ}に
\ruby{休}{やす}めといふ
\ruby{意}{こゝろ}は% 踊り字調整「〻(二の字点、揺すり点)に見えるが(ゝ)」
\ruby{明}{あき}らかなれど、
%
お
\ruby{龍}{りう}は
\ruby{眠}{ねむ}りたくも
\ruby{思}{おも}はぬ
\ruby{眼}{め}つきなり。

\原本頁{188-1}%
『
\ruby{足}{あし}は
\ruby{些}{ちつと}
ばかり
\ruby[g]{草臥}{くたびれ}ました
けれど、
%
\ruby[g]{先刻}{さつき }
お
\ruby{湯}{ゆ}に
\ruby{入}{はい}つたので
もう
\ruby{治}{なほ}りましたし、
%
\ruby{氣}{き}は
\ruby[g]{疲勞}{くたびれ}も
\ruby{何}{なに}も
\ruby{仕}{し}やあ
\ruby{仕}{し}ません。
』

\原本頁{188-3}%
『
いゝねえ% 踊り字調整「〻(二の字点、揺すり点)に見えるが(ゝ)」
\ruby{{\換字{若}}}{わか}い
\ruby{人}{ひと}は!。
%
\ruby{戀}{こひ}も
いさくさも
\ruby{其}{そ}の
\ruby[g]{威勢}{ゐ せい}の
ある
\ruby{中}{うち}の
\ruby{花}{はな}
なんだよ。
%
\ruby{妾}{わたし}なんざあ
\ruby{四ツ木}{よ| |ぎ}へ
\ruby{行}{い}かうもんなら
\ruby[g]{二日}{ふつか }
\ruby[||j>]{位}{ぐらゐ}は
\ruby{腰}{こし}が
\ruby{痛}{いた}いので、
%
しよぼけて
\ruby{居}{ゐ}なくちやあ
ならないんだよ。
』

\原本頁{188-6}%
『
ホヽヽ
\ruby[g]{虛言}{う そ }ばつかり!。
%
まだ
\ruby{御師匠}{お|し|よ}さんは
お
\ruby{{\換字{若}}}{わか}いは。
%
そんな
\ruby{事}{こと}を
\ruby{仰}{おつし}あつても
\ruby[g]{水々}{みづ〳〵}として
\ruby{在}{い}らつしやるぢありませんか。
』

\原本頁{188-8}%
『
オヤ
\ruby{汝}{おまへ}こそ
\ruby{人}{ひと}が
\ruby{惡}{わる}いよ、
%
\ruby{御調戱}{お|から|かひ}で
\ruby{無}{な}い。
%
いゝよ、% 踊り字調整「〻(二の字点、揺すり点)に見えるが(ゝ)」
%
\ruby[g]{何樣}{ど う }せ
\ruby{奢}{おご}らないから、
%
ハヽハヽハヽ。
』

\原本頁{188-10}%
『
でも
ほんたう
ですよ。
』

\原本頁{188-11}%
\ruby{渴}{かは}き% ルビ調整(原本通り)「か(は)」
\ruby[g]{氣味}{ぎ み }にや
\ruby{身}{み}を
\ruby{伸}{の}ばして
\ruby[||j>]{及}{および}
\ruby[||j>]{腰}{ ごし}に
\ruby[g]{火鉢}{ひ ばち}の
\ruby[g]{横手}{よこて }の
\ruby[g]{茶棚}{ちやだな}より
\ruby{小}{ちひさ}き
\ruby[g]{湯呑}{ゆ のみ}を
\ruby{取}{と}り、
%
\ruby[g]{鐵瓶}{てつびん}の
\ruby{湯}{ゆ}を
\ruby{注}{つ}ぎて
\ruby{心}{こゝろ}% 踊り字調整「〻(二の字点、揺すり点)に見えるが(ゝ)」
ゆたかに
\ruby{其}{それ}を
\ruby{冷}{さ}まして
\ruby{飮}{の}める
お
\ruby{龍}{りう}を
\ruby{見}{み}れば、
%
\ruby{女}{をんな}には
\ruby{先}{ま}づ
\ruby{目}{め}に
つく
\ruby{髮}{かみ}の
\ruby{毛}{け}の
\ruby{漆}{うるし}と
\ruby{黑}{くろ}くて
\ruby[g]{加之}{しかも }
\ruby{膨}{ふつ}くりと
したる
\ruby{鬢}{びん}に、
%
\ruby{櫛}{くし}の
\ruby{齒}{は}の
\ruby{痕}{あと}
あざやかに
\ruby{殘}{のこ}りて、
%
\ruby[g]{肌理}{き め }
\ruby{密}{こま}かに
\ruby[g]{色白}{いろじろ}なる
\ruby{顏}{かほ}の
ほんのりと
\ruby{紅}{あか}きは、
%
たゞ% 踊り字調整「〻(二の字点、揺すり点)に濁点に見えるが(ゞ)」
\ruby{是}{これ}
\ruby{淸}{きよ}き
\ruby{芳野紙}{よし|の|がみ}の
\ruby[g]{珊瑚}{さんご }を
\ruby{包}{つゝ}めるに% 踊り字調整「〻(二の字点、揺すり点)に見えるが(ゝ)」
\ruby{異}{こと}ならず。
%
ざつに
\ruby{座}{すわ}つたる
\ruby{身}{み}の
\ruby{稍}{やゝ}% 踊り字調整「〻(二の字点、揺すり点)に見えるが(ゝ)」
\ruby{歪}{ゆが}みて
\ruby{少}{すこ}し
\ruby{俯}{うつむ}いたるに、
%
\ruby[||j>]{細}{ほつそ}りと
したる
\ruby[g]{領頸}{{\換字{𛀁}}りくび}の
いとゞ% 踊り字調整「〻(二の字点、揺すり点)に濁点に見えるが(ゞ)」
しほらしく
\ruby[g]{柔和}{にうわ }に
\ruby{見}{み}えて、
%
\ruby{物}{もの}ごし
\ruby[g]{恰好}{かつかう}
\ruby{冴}{さ}え〳〵と
\ruby{艶}{{\換字{𛀁}}ん}なり。% 原本通り「𛀁ん」

\原本頁{189-8}%
お
\ruby{關}{せき}は
\ruby[g]{見惚}{み と }れしやうに
\ruby{良}{やゝ}% 踊り字調整「〻(二の字点、揺すり点)に見えるが(ゝ)」
\ruby{久}{ひさ}しく
\ruby[g]{見居}{み ゐ }つ。

\原本頁{189-9}%
『
そりや
まあ
\ruby[g]{何樣}{ど う }
でも
\ruby{可}{いゝ}% 踊り字調整「〻(二の字点、揺すり点)に見えるが(ゝ)」
としたところで、
%
\ruby[g]{矢張}{やつぱ }り
お
\ruby{{\換字{前}}}{まへ}にやあ
\ruby[g]{此頃}{このごろ}に
\ruby{御馳走}{ご|ち|そう}を
\ruby[g]{仕無}{し な }くちやあ
ならない。
%
ほんとに
\ruby{汝}{おまへ}の
\ruby[g]{氣合}{き あひ}の
\ruby{好}{い}いのには
\ruby[g]{{\換字{感}}心}{かんしん}しちまふよ。
%
\ruby[g]{歸路}{かへり }には
\ruby[g]{馴染}{なじみ }も% 「{馴染}{なじみ}」だと思うが原本通り
\ruby{無}{な}い
お
\ruby[g]{五十}{い そ }のために
お
\原本頁{190-1}\改行%
\ruby[g]{百度}{ひやくど}まで
\ruby{踏}{ふ}んで
\ruby{吳}{く}れる
なんて、
%
\ruby[g]{何樣}{ど う }すれば
\ruby[g]{其樣}{そ ん }なに
\ruby{優}{やさ}しい
\ruby{氣}{き}に
なつて、
%
しかも
\ruby[g]{俠氣}{をとこぎ}な
\ruby{事}{こと}が
\ruby[g]{出來}{で き }るだらう。
%
\ruby{妾}{わたし}や
\ruby[g]{全然}{すつかり}
お
\ruby{{\換字{前}}}{まへ}にやあ
\ruby{惚}{ほ}れつ
\ruby[g]{仕舞}{ち ま }つたよ。
%
お
\ruby{{\換字{前}}}{まへ}さへ
\ruby[g]{吾家}{う ち }に
\ruby{居}{ゐ}て
お
\ruby{吳}{く}れなら、
%
あんな
お
\ruby[g]{五十}{い そ }なんか
\ruby[g]{何樣}{ど う }なつた
からつて
\ruby{關}{かま}やあ
\ruby[g]{仕無}{し な }いよ。
』

\原本頁{190-5}%
『
あら
マア
\ruby{飛}{と}んでも
\ruby{無}{な}い
\ruby{酷}{ひど}い
\ruby{事}{こと}を!。
%
お
\ruby[g]{師匠}{し よ }さんの
\ruby[g]{左樣}{さ う }
\ruby{仰}{おつし}やるのを
\ruby[g]{本當}{ほんたう}に
した
ところで、
%
\ruby{五十子}{い|そ|こ}さんが
お
\ruby{惡}{わる}く
\ruby{御}{お}なんなさらうもんなら
\ruby[g]{水野}{みづの }さん
ていふ
\ruby{方}{かた}が、
%
\ruby[g]{何樣}{ど ん }なに
\ruby[g]{御騷}{お さわ}ぎなさるか
\ruby{知}{し}れやしません!。
』

\原本頁{190-9}%
『
\ruby{騷}{さわ}いだつて
\ruby{可}{いゝ}やね、% 踊り字調整「〻(二の字点、揺すり点)に見えるが(ゝ)」
%
\ruby{騷}{さわ}がして
\ruby{置}{おき}やあ。
』

\原本頁{190-10}%
『
まだ
\ruby{詳}{くは}しい
\ruby[g]{御話}{おはなし}を
\ruby{伺}{うかゞ}ひませんが、% 踊り字調整「〻(二の字点、揺すり点)に濁点に見えるが(ゞ)」
%
\ruby[g]{一體}{いつたい}
\ruby[g]{水野}{みづの }さん
ていふ
\ruby{方}{かた}は
\ruby[g]{何樣}{ど う }いふ
\ruby{方}{かた}なの?。
』

\原本頁{191-1}%
『
オヤ〳〵
をかしいよお
\ruby{龍}{りう}ちやんは。
%
\ruby[g]{今日}{け ふ }
お
\ruby[g]{晝{\換字{過}}}{ひるすぎ}に
\ruby{家}{うち}へ
\ruby{歸}{かへ}つて
\ruby{來}{き}てから、
%
これで
\ruby[g]{丁度}{ちやうど}
\ruby[g]{水野}{みづの }の
\ruby{事}{こと}を
\ruby[g]{三度}{さんど }
\ruby[g]{御聞}{お きゝ}だよ。% 踊り字調整「〻(二の字点、揺すり点)に見えるが(ゝ)」
%
ハヽヽ
まさか
\ruby[||j>]{汝}{おまへ}の
やうに
\ruby{{\換字{分}}}{わか}つた
\ruby{人}{ひと}が、
%
\ruby[g]{彼樣}{あ ん }な
\ruby{唐變木}{たう|へん|ぼく}に
\ruby[g]{何樣}{ど う }か
\ruby[g]{御爲}{お し }だとも
\ruby{思}{おも}やあ
\ruby{仕}{し}ないが\換字{子}。
%
よつぽど
\ruby{氣}{き}に
なるやうな
\ruby{變}{へん}な
\ruby{顏}{かほ}でも
\ruby{仕}{し}て
\ruby{居}{ゐ}たのかエ。
%
\ruby{彼}{あり}や
\ruby{何}{なん}でも
\ruby{有}{あ}りや
\ruby{仕}{し}ないのさ。
%
たゞ% 踊り字調整「〻(二の字点、揺すり点)に濁点に見えるが(ゞ)」
\ruby[g]{彼村}{あすこ }の
\ruby[g]{學校}{がくかう}の
\ruby[g]{敎師}{けうし }で
もつて、
%
\ruby{{\換字{平}}}{ひら}つたく
\ruby{云}{い}やあ
お
\ruby[g]{五十}{い そ }に
\ruby{惚}{ほ}れてる
といふだけの
\ruby{鈍痴氣}{とん|ち|き}なんだよ。
』

\原本頁{191-8}%
『
だつて
\ruby{其}{そん}なら
\ruby{妾}{わたし}が
\ruby{御師匠}{お|し|よ}さんの
\ruby[g]{御使}{おつかひ}に、
%
わざ〳〵
\ruby{彼}{あ}の
\ruby{人}{ひと}の
ところへ
\ruby{行}{い}かなくつてもぢや
\ruby{有}{あ}りませんか。
』

\原本頁{191-10}%
『
そりや
お
\ruby[g]{五十}{い そ }の
\ruby{事}{こと}の
\ruby[g]{關係}{つゞき }から\換字{子}、% 踊り字調整「〻(二の字点、揺すり点)に濁点に見えるが(ゞ)」
%
\ruby{妾}{わたし}も
\ruby[g]{困究}{こ ま }つた
\ruby{時}{とき}に
\ruby{彼}{あの}
\ruby[||j>]{男}{をとこ}に
\ruby[g]{融{\換字{通}}}{ゆうづう}を
\ruby{頼}{たの}んだ
\ruby{事}{こと}も
あるし、
%
\ruby[g]{今度}{こんど }も
\ruby[g]{全然}{すつかり}
お
\ruby[g]{五十}{い そ }が
\ruby[g]{世話}{せ わ }に
なつて
\ruby{居}{ゐ}るからさ。
』

\原本頁{192-2}%
『
ぢやあ
\ruby[g]{矢張}{やつぱ }り
\ruby[g]{畢竟}{つまり }は
\ruby{五十子}{い|そ|こ}さんと
\ruby[g]{一{\換字{所}}}{いつしよ}になる
\ruby{譯}{わけ}の
\ruby{方}{かた}ぢや
ありませんか。
%
\ruby[g]{{\換字{道}}理}{だうり }で
\ruby{心}{しん}から
\ruby{底}{そこ}から
\ruby{御}{ご }
\ruby[<j||]{病}{びやう}
\ruby{人}{にん}を
\ruby[g]{大切}{たいせつ}に
\ruby{思}{おも}つて
\ruby{居}{ゐ}らつしやる
やうに
\ruby{見}{み}えましたよ。
%
ほんとに
\ruby{五十子}{い|そ|こ}さんは
\ruby{御幸福}{お|しあ|はせ}な%「幸福」ここは「は」
\ruby{事}{こと}!
\改行% 校正作業の簡略化のため
、
%
\原本頁{192-5}\改行%
あんな
\ruby{頼}{たの}もしさうな
\ruby{方}{かた}に
\ruby[g]{御思}{お おも}はれ
なすつて!。
』

\原本頁{192-6}%
『
ところが
お
\ruby{{\換字{前}}}{まへ}、
%
いくら
\ruby{彼}{あの}
\ruby[||j>]{男}{をとこ}が
\ruby{思}{おも}つても、
%
\ruby{妾}{わたし}の
\ruby{云}{い}ふ
\ruby{事}{こと}さへ
\ruby{聽}{き}かない
やうな、
%
ヘチ
\ruby[g]{頑固}{ぐわんこ}の
お
\ruby[g]{五十}{い そ }の
\ruby{事}{こと}だから、
%
\ruby{{\換字{嫌}}}{きら}つて
\ruby{{\換字{嫌}}}{きら}ひぬいて
\ruby{關}{かま}はないのだよ。
%
\ruby{彼}{あ}の
\ruby{男}{をとこ}の
\ruby{思}{おもひ}なんぞは
\ruby[g]{玻瓈}{がらす }に
\ruby{書}{か}く
\ruby{字}{じ}で、
%
\ruby[g]{以上}{いじやう}
\ruby{經}{たつ}ても
\ruby{{\換字{通}}}{とほ}りつこは
\ruby{無}{な}いのさ。
』

\原本頁{192-10}%
『
でも
\ruby{御師匠}{お|し|よ}さんは
\ruby{{\換字{終}}}{しまひ}にやあ
\ruby{彼}{あ}の
\ruby{人}{ひと}を
\ruby[g]{御婿}{お むこ}さんにと% (婿 5a7f) 聟 805f
\ruby{思}{おも}つて
らつしやる
でしやう。
』

\原本頁{193-1}%
『
だつて
お
\ruby[g]{五十}{い そ }が
\ruby{妾}{わたし}の
\ruby{云}{い}ふ
\ruby{事}{こと}なんか
\ruby{聽}{き}くんぢや
\ruby{無}{な}いから
\ruby[g]{仕方}{し かた}が
\ruby{無}{な}いやね。
%
\ruby{妾}{わたし}あ
\ruby{打}{うつ}
\ruby{棄}{ちや}つて
\ruby{置}{お}いて
\ruby{關}{かま}やあ
\ruby[g]{仕無}{し な }いのさ。
』

\原本頁{193-3}%
『
あら
\ruby{憫}{かは}% 憫然
\ruby[||j>]{然}{いさう}に、
%
それぢやあ% 「憫然 か(は)いさう」
\ruby{彼}{あ}の
\ruby{人}{ひと}の
\ruby[g]{立塲}{たちば }が% 原文通り「塲」
\ruby{無}{な}いぢやあ
\ruby{有}{あ}りませんか。
』

\原本頁{193-5}%
『
だから
\ruby{唐變木}{たう|へん|ぼく}で
\ruby{鈍痴氣}{どん|ち|き}
だといふんだア\換字{子}。
』

\原本頁{193-6}%
『
なんですつて\換字{?!}、
%
マア!。
』

\原本頁{193-7}%
\ruby{優}{やさ}しき
\ruby{姿}{すがた}は
\ruby[g]{其儘}{そのまゝ}に、% 踊り字調整「〻(二の字点、揺すり点)に見えるが(ゝ)」
%
\ruby[g]{身動}{み じろ}きは
\ruby[g]{一寸}{いつすん}も
せざりしが、
%
\ruby[g]{愛嬌}{あいけう}
こぼるゝ% 踊り字調整「〻(二の字点、揺すり点)に見えるが(ゝ)」
\ruby{面}{おもて}
ながら、
%
じろりと
\ruby{斜}{なゝめ}に% 踊り字調整「〻(二の字点、揺すり点)に見えるが(ゝ)」
\ruby{上}{うは}
\ruby{睨}{にら}みして、
%
お
\ruby{關}{せき}を
\ruby{見}{み}やりたる
お
\ruby{龍}{りう}の
\ruby{眼}{め}には、
%
\ruby{瞋}{いか}るか
\ruby{恨}{うら}むか
\ruby[g]{蔑視}{さげす }むか、
%
\ruby{怪}{あや}しき
\ruby[g]{一種}{いつしゆ}の
\ruby[g]{氣味}{き み }
\ruby[g]{合籠}{あひこも}りて、
%
\ruby{花}{はな}の
\ruby[g]{樹蔭}{こ かげ}に
\ruby{蛇}{へび}の
\ruby{出}{い}でたる
\ruby[g]{其狀}{そ れ }にも
\ruby{似}{に}たる
\ruby[g]{風{\換字{情}}}{ふ ぜい}を
\ruby{見}{み}せたり。
