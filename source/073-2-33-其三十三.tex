\Entry{其三十三}

% メモ 校正終了 2024-04-28
\原本頁{182-1}%
お
\ruby{龍}{りう}は
\ruby{徐}{しづか}に
\ruby{三絃}{さみ|せん}の
\ruby{絃}{いと}を
\ruby{弛}{ゆる}めて
\ruby{三絃}{さみ|せん}
\ruby{掛}{かけ}へ
\ruby{掛}{か}け
\ruby{納}{をさ}むれば、
%
\ruby{今日}{け|ふ}
\ruby{目}{め}
\ruby{見}{み}
\ruby{得}{{\換字{𛀁}}}に
\ruby{來}{きた}りし
\ruby{小婢}{こ|をんな}
お
\ruby{熊}{くま}は
\ruby{高麗鼠}{こ|ま|ねずみ}
のやうに
くる〳〵と
\ruby{働}{はたら}きて、
%
しきりに
\ruby{其邊}{そこ|ら}を
\ruby{取}{と}り
\ruby{片付}{かた|づ}けしが、
%
\ruby{{\換字{煙}}草}{たば|こ}
\ruby{{\換字{盆}}}{ぼん}の
\ruby{傍}{かたはら}より
\ruby{玉}{ぎよく}の
\ruby{{\換字{煙}}管}{パイ|プ}の
いと
\ruby[||j>]{小}{ちひさ}なるを
\ruby{拾}{ひろ}ひ
あげて
\ruby{洋燈}{らん|ぷ}
\ruby{{\換字{近}}}{ちか}く
さし
\ruby{出}{いだ}し、

\原本頁{182-5}%
『
これ
\ruby{此樣}{こ|ん}な
\ruby{物}{もの}が
\ruby{{\換字{遺}}}{お}ちて
\ruby{居}{を}りました、
』

\原本頁{182-6}%
といふ。

\原本頁{182-7}%
\ruby{一}{ひ}ト
\ruby{目}{め}
\ruby{見}{み}て
お
\ruby{龍}{りう}は
それを
\ruby{師匠}{し|〻やう}に% 原本通り「〻(二の字点、揺すり点)」
\ruby{遞與}{わ|た}し、

\原本頁{182-8}%
『
こりやあ
\ruby{傳}{でん}さんが
\ruby{{\換字{遺}}}{わす}れて
\ruby{行}{い}つた
のでしやう。
%
あの
\ruby{人}{ひと}で
\ruby{無}{な}け
りやあ
\ruby{此樣}{こ|ん}なものを
\ruby{持}{も}ち
さうな
\ruby{人}{ひと}は
ありませんから。
』

\原本頁{182-10}%
と
\ruby{云}{い}へば、
%
お
\ruby{關}{せき}は
\ruby{受取}{うけ|と}つて
\ruby{指頭}{ゆび|さき}に
\ruby{弄}{もてあそ}び、

\原本頁{182-11}%
『
あ〻% 本来は一の字点「ゝ」平仮名繰返し記号% 原本通り「〻(二の字点、揺すり点)」
\ruby{然樣}{さ|う}だよ、
%
\ruby{屹度}{きつ|と}
\ruby{彼}{あ}の
\ruby{男}{をとこ}のだよ。
%
\ruby{今日}{け|ふ}は
\ruby{妾}{わたし}も
\ruby{大變}{たい|へん}
\ruby{夙}{はや}
\ruby{起}{おき}を
\ruby{仕}{し}たし、
%
\ruby{汝}{おまへ}も
\ruby{{\換字{遠}}}{とほ}い
ところへ
\ruby{行}{い}つて
\ruby{來}{き}たので
\ruby{草臥}{くた|びれ}て
\ruby{居}{ゐ}る
からつて
いふので
\ruby{{\換字{逐}}}{お}ひ
\ruby{立}{た}て〻% 本来は一の字点「ゝ」平仮名繰返し記号% 原本通り「〻(二の字点、揺すり点)」
やつたもんだから、
%
\ruby{慌}{あわ}て〻% 本来は一の字点「ゝ」平仮名繰返し記号% 原本通り「〻(二の字点、揺すり点)」
\ruby{歸}{かへ}つて
\ruby{行}{い}つて
\ruby{{\換字{遺}}}{わす}れたん
だらう。
%
\ruby{取}{と}り
\ruby{上}{あ}げて
\ruby{仕舞}{し|ま}つて
\ruby{{\換字{遣}}}{や}らうか
\ruby{知}{し}らん。
%
ハヽヽ、
%
マア
\ruby{堪{\換字{忍}}}{かん|にん}して% 原文通り「堪忍」
\ruby{{\換字{遣}}}{や}ると
\ruby{仕}{し}やう。
%
\ruby{何}{なん}でも
\ruby{彼}{あ}の
\ruby{男}{をとこ}は
\ruby{親類}{しん|るゐ}
\ruby{内}{うち}か
なんぞに、
%
\原本頁{183-5}\改行%
\ruby{玉}{たま}や
\ruby{石}{いし}の
\ruby{細工}{さい|く}
をする
\ruby{家}{うち}か
なんぞを
\ruby{有}{も}つて
\ruby{居}{ゐ}るんだよ。
%
\ruby{御覧}{ご|らん}よ、
%
\ruby[||j>]{小}{ちひさ}い
けれども
\ruby{此品}{こ|れ}だつて
\ruby{買}{か}つたら
\ruby{{\換字{廉}}}{やす}くは
なさ〻うなものだ\換字{子}。% 本来は一の字点「ゝ」平仮名繰返し記号% 原本通り「〻(二の字点、揺すり点)」
』

\原本頁{183-7}%
と、
%
\ruby{一度}{ひと|たび}は
お
\ruby{龍}{りう}に
\ruby{示}{しめ}して、
%
さて
\ruby{火鉢}{ひ|ばち}の
\ruby{抽斗}{ひき|だし}に
\ruby{無}{む}
\ruby{{\換字{造}}作}{ざう|さ}に
\ruby{藏}{しま}ひたり。

\原本頁{183-8}%
『
ハア
\ruby{左樣}{さ|う}
なんで
しやうよ。
%
\ruby{兎}{うさぎ}を
\ruby{吳}{く}れたんでも
\ruby{{\換字{分}}}{わか}つて
\ruby{居}{ゐ}ますよ。
%
\ruby{屹度}{きつ|と}
\ruby{叔{\換字{父}}}{を|ぢ}さんか
\ruby{何}{なに}かゞ% TODO 原本の「二の字点、揺すり点」に濁点のグリフが見つからないので「ゞ」
\ruby{玉屋}{たま|や}さん
なんです\換字{子}。
』

\原本頁{183-10}%
『
\ruby{何樣}{ど|う}も
\ruby{左樣}{さ|う}らしいよ。
%
\ruby{妾}{わたし}も
\ruby{往日}{いつ|か}
\ruby{瑪瑙}{め|なう}の
\ruby{好}{い}い
\ruby{色}{いろ}の
\ruby{簪}{かんざし}
\ruby{珠}{  だま}を% 「 (全角空白)」は「簪(かんざし)」の後突出対策
\ruby{貰}{もら}つたがね、
%
\ruby{汝}{おまへ}、
%
\ruby{兎}{うさぎ}
なんぞぢや
\ruby{仕樣}{し|やう}が
\ruby{無}{な}いぢや
\ruby{無}{な}いか。
%
\ruby{今度}{こん|ど}は
\ruby{寶石}{い|し}
\ruby{入}{い}りの
\ruby{指輪}{ゆび|わ}か
なんか
\ruby{{\換字{強}}{\換字{請}}}{ね|だ}つて
\ruby{御{\換字{遣}}}{お|や}りナ。
%
\ruby[g]{金剛石}{ダイヤ}とでも
\ruby{云}{いつ}たら
\ruby{二}{に}の
\ruby{足}{あし}を
\ruby{踏}{ふ}むか
\ruby{知}{し}らないが、
%
サフイヤや
\ruby{眞珠}{しん|じゆ}の
\ruby{位}{ぐらゐ}なら
\ruby{屹度}{きつ|と}
\ruby{二}{ふた}ッ
\ruby{{\換字{返}}事}{へん|じ}で
\ruby[||j>]{悅}{よろこ}んで
\ruby{持}{も}つて
\ruby{來}{く}るよ。
%
\ruby{物}{もの}を
\ruby{取}{と}つて
\ruby{{\換字{遣}}}{や}るのも
\ruby{功徳}{く|どく}に
なるの
だから
\ruby{關}{かま}やあ
\ruby{仕}{し}ない
\ruby{吹}{ふつ}かけて
\ruby{御覧}{ご|らん}、
%
\ruby{相槌}{あひ|づち}は
\ruby{妾}{わたし}が
\ruby{巧}{うま}く
\ruby{打}{う}つて
\ruby{上}{あ}げるから。
』

\原本頁{184-6}%
『あら
\ruby{{\換字{嫌}}}{いや}な
\ruby{御師匠}{お|し|よ}さん!。
%
\ruby{妾}{わたし}あ
\ruby{指輪}{ゆび|わ}
なんか
\ruby{欲}{ほ}しかあ
\ruby{無}{な}いんですよ。
%
しかも
\ruby{傳}{でん}さんに
なんか
\ruby{貰}{もら}ひたたかあ
\ruby{有}{あ}りません。
』

\原本頁{184-8}%
『
\ruby{然樣}{さ|う}かネエ。
%
\ruby{汝}{おまへ}は
ほんとに
\ruby{慾}{よく}に
\ruby{掛}{か}けちやあ
\ruby{氣}{き}が
\ruby{{\換字{弱}}}{よわ}いよ。
%
だが
\ruby{取}{と}つて
\ruby{{\換字{遣}}}{や}る
\ruby{方}{はう}が
\ruby{可}{い〻}ぢやあ% 原本通り「〻(二の字点、揺すり点)」
\ruby{無}{な}いか。
%
あの
\ruby{兎}{うさぎ}でも
\ruby{知}{し}れてるは\換字{子}、
%
\ruby{汝}{おまへ}の
\ruby{氣}{き}に
\ruby{入}{い}つたのを
\ruby{見}{み}て
\ruby{何樣}{どん|な}なに
\ruby{嬉}{うれし}がつてるか
\ruby{知}{し}れや
\ruby{仕}{し}ないよ。
』

\原本頁{184-11}%
『
だから
\ruby{妾}{わたし}あ
\ruby{厭}{いや}なんですよ。
%
その
\ruby{嬉}{うれ}し
がられるのが
\ruby{氣障}{き|ざ}ぢや
\ruby{有}{あ}りませんか。
』

\原本頁{185-2}%
『
ホイ
\ruby{大失敗}{おほ|しく|じり}だネ、
%
ハヽハヽハヽ。
%
\ruby{指輪}{ゆび|わ}の
\ruby{談}{はなし}で
\ruby{想}{おも}ひ
\ruby{出}{だ}したが、
%
\ruby{先}{せん}に
\ruby{汝}{おまへ}が
あの
\ruby{何}{なん}に
(
\ruby{源}{げん}を
\ruby{指}{さ}す
)
\ruby{御貰}{お|もら}ひのは
\ruby[<j||]{汝}{おまへ}
\ruby{有}{も}つておいで〻% 本来は一の字点「ゝ」平仮名繰返し記号% 原本通り「〻(二の字点、揺すり点)」
\ruby{無}{な}いネエ。
%
\ruby{妾}{わたし}が
\ruby{見立}{み|た}て〻% 本来は一の字点「ゝ」平仮名繰返し記号% 原本通り「〻(二の字点、揺すり点)」
\ruby{買}{か}はせたん
だから
まだ
\ruby{記}{おぼ}えて% 送り仮名は原本通り「え」
\ruby{居}{ゐ}るが、
%
\ruby[<j||]{汝}{おまへ}
\ruby{彼品}{あ|れ}は
\ruby{何樣}{ど|う}か
\ruby{仕}{し}て
お
\ruby{仕舞}{し|まひ}かエ。
』

\原本頁{185-6}%
『
だつて
\ruby{御師匠}{お|し|よ}さん、
%
まだ
\ruby{妾}{わたし}が
\ruby{彼品}{あ|れ}を
\ruby{持}{も}つて
\ruby{居}{ゐ}やう
\ruby{譯}{わけ}は
\ruby{無}{な}からうぢや
\ruby{有}{あ}りませんか。
%
いよ〳〵
\ruby{不實}{ふ|じつ}な
\ruby{人}{ひと}だと
\ruby{思}{おも}ひつめた
\ruby{時}{とき}は、
%
\ruby{口惜}{く|やし}くつて
\ruby{口惜}{く|やし}くつて
\ruby{仕方}{し|かた}が
\ruby{無}{な}かつた
んですもの!。
%
\ruby{宿}{と}めて
\ruby{貰}{もら}つて
\ruby{居}{ゐ}た
\ruby{藥研堀}{や|げん|ぼり}の
おとうさん
{---}{---}
\ruby{御師匠}{お|し|よ}さんは
\ruby{御知}{お|し}ん
なさらないが
\ruby{妾}{わたし}の
\ruby{仲好}{なか|よ}しの
\ruby{其}{そ}の
\ruby{家}{うち}を
\ruby{出}{で}て、
%
をかアしな
\ruby{氣}{き}になつて
ふらふらと% 原本では行末行頭禁則のため非踊り字表記
\ruby[<j||]{兩}{りやう}% 兩國橋
\ruby{國橋}{ごく|ばし}の
\ruby{上}{うへ}を
\ruby{往}{い}つたり
\ruby{復}{かへ}つたりした
\ruby{其}{そ}の
\ruby{擧句}{あげ|く}でした、
%
ふいと
\ruby[<j||]{意}{こ〻ろ}% 原本通り「〻(二の字点、揺すり点)」
\ruby{持}{もち}が
\ruby{變}{かは}つたんで
\ruby{指}{ゆび}から
\ruby{脫}{はづ}して、
%
\ruby{大川}{おほ|かは}の
\ruby{流}{なが}れの
\ruby{中}{なか}へ
\ruby{抛}{はふ}り
\ruby{{\換字{込}}}{こ}んで
\ruby{仕舞}{し|ま}つたんですよ。
』

\原本頁{186-3}%
『
ヘーエ、
%
\ruby{勿體}{もつ|たい}
\ruby{無}{な}い
\ruby{事}{こと}を
\ruby{御仕}{お|し}だつた\換字{子}エ、
%
マア
\ruby{妾}{わたし}なら
\ruby{同}{おな}じ
\ruby{棄}{す}てるにも
お
\ruby{金}{かね}に
\ruby{仕}{し}て
\ruby{棄}{す}てるものを。
%
だが
\ruby{鑄掛松}{ゐ|かけ|まつ}を
\ruby{色氣}{いろ|け}で
\ruby{行}{い}つたのは、
%
\ruby{一寸}{ちよ|つと}
\ruby{覗}{のぞ}いて
\ruby{見}{み}たい
やうな
\ruby{幕}{まく}だつた\換字{子}。
』

\原本頁{186-6}%
『
ホヽヽ、
%
\ruby{厭}{いや}ですよ。
%
たんと
\ruby{御嬲}{お|なぶ}り
なさい、
%
\ruby{人}{ひと}の
\ruby{惡}{わる}い!。
%
\ruby{今}{いま}なら
\ruby{妾}{わたし}だつて‥‥‥。
』

\原本頁{186-8}%
『
\ruby{何樣}{ど|う}
\ruby{御仕}{お|し}だエ?、
』

\原本頁{186-9}%
『
\ruby{御魚}{お|さかな}にやあ
\ruby{與}{や}らないで
\ruby{瞽女}{ご|ぜ}にでも
\ruby{與}{や}ります。
』

\原本頁{186-10}%
『
\ruby{{\換字{分}}別}{ふん|べつ}らしい
けれども
\ruby{{\換字{猶}}且}{やつ|ぱり}
\ruby{{\換字{若}}}{わか}い\換字{子}エ。
%
ハヽヽ、
%
\ruby{瞽女}{ご|ぜ}が
\ruby[<j||]{汝}{おまへ}
\ruby{狂}{くる}ひ
\ruby{浪}{なみ}の
\ruby{彫}{ほり}に
\ruby{小}{ちひさ}な
\ruby{寶石}{い|し}の
\ruby{散}{ち}らばつて
\ruby{居}{ゐ}る
\ruby{彼樣}{あ|ん}な
\ruby{華麗}{はで|やか}な
\ruby{物}{もの}を
\ruby{指}{ゆび}に
\ruby{嵌}{は}めて
\ruby{何樣}{ど|う}なるものかネ。
』

\原本頁{187-2}%
『
ぢやあ
\ruby{御師匠}{お|し|よ}さんが
\ruby{妾}{わたし}だつたら
\ruby{何樣}{ど|う}なさるの?。
』

\原本頁{187-3}%
お
\ruby{關}{せき}は
\ruby{我}{わ}が
\ruby{鼻}{はな}を
\ruby{指}{ゆび}さしながら、

\原本頁{187-4}%
『
\ruby{此處}{こ|〻}に% 原本通り「〻(二の字点、揺すり点)」
\ruby{居}{ゐ}る
\ruby{美麗}{き|れい}な
\ruby{可憐}{か|はゆ}らしい
\ruby{新{\換字{造}}}{しん|ぞ}に
\ruby{與}{や}つて
\ruby{悅}{よろこ}ばせるはネ。
』

\原本頁{187-5}%
と
\ruby{云}{い}ひさして、
%
ハヽハヽハヽと
\ruby{打}{うち}
\ruby{笑}{わら}へば、
%
お
\ruby{龍}{りう}も
ホヽと
\ruby{笑}{わら}ひ
\ruby{出}{だ}し、
%
\ruby{臺{\換字{所}}}{だい|どころ}の
\ruby{方}{かた}に
\ruby{{\換字{退}}}{しりぞ}きたる
お
\ruby{熊}{くま}さへ
\ruby{貰}{もら}ひ
\ruby{笑}{わら}ひしたり。

\原本頁{187-7}%
『
あ〻、% 本来は一の字点「ゝ」平仮名繰返し記号% 原本通り「〻(二の字点、揺すり点)」
%
\ruby{笑}{わら}つたんで
\ruby[<j||]{心}{こ〻ろ}% 原本通り「〻(二の字点、揺すり点)」
\ruby{持}{もち}が
\ruby{佳}{い}い。
%
さあ
お
\ruby{熊}{くま}や
\ruby{方々}{はう|〴〵}
\ruby{{\換字{戸}}締}{と|じま}りを
\ruby{仕}{し}て
お
\ruby{仕舞}{おし|ま}ひ。
%
お
\ruby{龍}{りう}ちやんも
\ruby{歸路}{かへ|り}に
\ruby{御}{お}
\ruby{百度}{ひやく|ど}まで
\ruby{踏}{ふ}んで
\ruby{御}{お}
\ruby{吳}{く}れぢやあ、
%
ほんとに
\ruby{隨{\換字{分}}}{ずゐ|ぶん}
おくたびれ
だらう。
』

\原本頁{187-10}%
\ruby{隨意}{こ〻ろ|まかせ}に% 原本通り「〻(二の字点、揺すり点)」
\ruby{休}{やす}めといふ
\ruby{意}{こ〻ろ}は% 原本通り「〻(二の字点、揺すり点)」
\ruby{明}{あき}らかなれど、
%
お
\ruby{龍}{りう}は
\ruby{眠}{ねむ}りたくも
\ruby{思}{おも}はぬ
\ruby{眼}{め}つきなり。

\原本頁{188-1}%
『
\ruby{足}{あし}は
\ruby{些}{ちつと}
ばかり
\ruby{草臥}{くた|びれ}ました
けれど、
%
\ruby{先刻}{さつ|き}
お
\ruby{湯}{ゆ}に
\ruby{入}{はい}つたので
もう
\ruby{治}{なほ}りましたし、
%
\ruby{氣}{き}は
\ruby{疲勞}{くた|びれ}も
\ruby{何}{なに}も
\ruby{仕}{し}やあ
\ruby{仕}{し}ません。
』

\原本頁{188-3}%
『
い〻ねえ% 本来は一の字点「ゝ」平仮名繰返し記号% 原本通り「〻(二の字点、揺すり点)」
\ruby{{\換字{若}}}{わか}い
\ruby{人}{ひと}は!。
%
\ruby{戀}{こひ}も
いさくさも
\ruby{其}{そ}の
\ruby{威勢}{ゐ|せい}の
ある
\ruby{中}{うち}の
\ruby{花}{はな}
なんだよ。
%
\ruby{妾}{わたし}なんざあ
\ruby[g]{四ッ木}{よ ぎ}へ% TODO 四ツ木
\ruby{行}{い}かうもんなら
\ruby{二日}{ふつ|か}
\ruby[||j>]{位}{ぐらゐ}は
\ruby{腰}{こし}が
\ruby{痛}{いた}いので、
%
しよぼけて
\ruby{居}{ゐ}なくちやあ
ならないんだよ。
』

\原本頁{188-6}%
『
ホヽヽ
\ruby{虛言}{う|そ}ばつかり!。
%
まだ
\ruby{御師匠}{お|し|よ}さんは
お
\ruby{{\換字{若}}}{わか}いは。
%
そんな
\ruby{事}{こと}を
\ruby{仰}{おつし}あつても
\ruby{水々}{みづ|〳〵}として
\ruby{在}{い}らつしやるぢあ
ありませんか。
』

\原本頁{188-8}%
『
オヤ
\ruby{汝}{おまへ}こそ
\ruby{人}{ひと}が
\ruby{惡}{わる}いよ、
%
\ruby{御調戱}{お|から|かひ}で
\ruby{無}{な}い。
%
い〻よ、% 本来は一の字点「ゝ」平仮名繰返し記号% 原本通り「〻(二の字点、揺すり点)」
%
\ruby{何樣}{ど|う}せ
\ruby{奢}{おご}らないから、
%
ハヽハヽハヽ。
』

\原本頁{188-10}%
『
でも
ほんたう
ですよ。
』

\原本頁{188-11}%
\ruby{渴}{かは}き% 原本通りルビは「か(は)」
\ruby{氣味}{ぎ|み}にや
\ruby{身}{み}を
\ruby{伸}{の}ばして
\ruby[<j||]{及}{および}
\ruby{腰}{ごし}に
\ruby{火鉢}{ひ|ばち}の
\ruby{横手}{よこ|て}の
\ruby{茶棚}{ちや|だな}より
\ruby{小}{ちひさ}き
\ruby{湯呑}{ゆ|のみ}を
\ruby{取}{と}り、
%
\ruby{鐵瓶}{てつ|びん}の
\ruby{湯}{ゆ}を
\ruby{注}{つ}ぎて
\ruby{心}{こ〻ろ}% 原本通り「〻(二の字点、揺すり点)」
ゆたかに
\ruby{其}{それ}を
\ruby{冷}{さ}まして
\ruby{飮}{の}める
お
\ruby{龍}{りう}を
\ruby{見}{み}れば、
%
\ruby{女}{をんな}には
\ruby{先}{ま}づ
\ruby{目}{め}に
つく
\ruby{髮}{かみ}の
\ruby{毛}{け}の
\ruby{漆}{うるし}と
\ruby{黑}{くろ}くて
\ruby{加之}{しか|も}
\ruby{膨}{ふつ}くりと
したる
\ruby{鬢}{びん}に、
%
\ruby{櫛}{くし}の
\ruby{齒}{は}の
\ruby{痕}{あと}
あざやかに
\ruby{殘}{のこ}りて、
%
\ruby{肌理}{き|め}
\ruby{密}{こま}かに
\ruby{色白}{いろ|じろ}なる
\ruby{顏}{かほ}の
ほんのりと
\ruby{紅}{あか}きは、
%
たゞ% TODO 原本の「二の字点、揺すり点」に濁点のグリフが見つからないので「ゞ」
\ruby{是}{これ}
\ruby{淸}{きよ}き
\ruby{芳野紙}{よし|の|がみ}の
\ruby{珊瑚}{さん|ご}を
\ruby{包}{つ〻}めるに% 原本通り「〻(二の字点、揺すり点)」
\ruby{異}{こと}ならず。
%
ざつに
\ruby{座}{すわ}つたる
\ruby{身}{み}の
\ruby{稍}{や〻}% 原本通り「〻(二の字点、揺すり点)」
\ruby{歪}{ゆが}みて
\ruby{少}{すこ}し
\ruby{俯}{うつむ}いたるに、
%
\ruby[||j>]{細}{ほつそ}りと
したる
\ruby{領頸}{{\換字{𛀁}}り|くび}の
いとゞ% TODO 原本の「二の字点、揺すり点」に濁点のグリフが見つからないので「ゞ」
しほらしく
\ruby{柔和}{にう|わ}に
\ruby{見}{み}えて、
%
\ruby{物}{もの}ごし
\ruby{恰好}{かつ|かう}
\ruby{冴}{さ}え〳〵と
\ruby{艶}{{\換字{𛀁}}ん}なり。% 原本通り「𛀁ん」

\原本頁{189-8}%
お
\ruby{關}{せき}は
\ruby{見惚}{み|と}れたやうに
\ruby{良}{や〻}% 原本通り「〻(二の字点、揺すり点)」
\ruby{久}{ひさ}しく
\ruby{見居}{み|ゐ}つ。

\原本頁{189-9}%
『
そりや
まあ
\ruby{何樣}{ど|う}
でも
\ruby{可}{い〻}% 原本通り「〻(二の字点、揺すり点)」
としたところで、
%
\ruby{矢張}{やつ|ぱ}り
お
\ruby{{\換字{前}}}{まへ}にやあ
\ruby{此頃}{この|ごろ}に
\ruby{御馳走}{ご|ち|そう}を
\ruby{仕無}{し|な}くちやあ
ならない。
%
ほんとに
\ruby{汝}{おまへ}の
\ruby{氣合}{き|あひ}の
\ruby{好}{い}いのには
\ruby{感心}{かん|しん}しちまふよ。
%
\ruby{歸路}{かへ|り}には
\ruby{馴染}{な|じみ}も
\ruby{無}{な}い
お
\ruby{五十}{い|そ}のために
お
\原本頁{190-1}\改行%
\ruby{百度}{ひやく|ど}まで
\ruby{踏}{ふ}んで
\ruby{吳}{く}れる
なんて、
%
\ruby{何樣}{ど|う}すれば
\ruby{其樣}{そ|ん}なに
\ruby{優}{やさ}しい
\ruby{氣}{き}に
なつて、
%
しかも
\ruby{俠氣}{をとこ|ぎ}な
\ruby{事}{こと}が
\ruby{出來}{で|き}るだらう。
%
\ruby{妾}{わたし}や
\ruby{全然}{すつ|かり}
お
\ruby{{\換字{前}}}{まへ}にやあ
\ruby{惚}{ほ}れつ
\ruby{仕舞}{ち|ま}つたよ。
%
お
\ruby{{\換字{前}}}{まへ}さへ
\ruby{吾家}{う|ち}に
\ruby{居}{ゐ}て
お
\ruby{吳}{く}れなら、
%
あんな
お
\ruby{五十}{い|そ}なんか
\ruby{何樣}{ど|う}なつた
からつて
\ruby{關}{かま}やあ
\ruby{仕無}{し|な}いよ。
』

\原本頁{190-5}%
『
あら
マア
\ruby{飛}{と}んでも
\ruby{無}{な}い
\ruby{酷}{ひど}い
\ruby{事}{こと}を!。
%
お
\ruby{師匠}{し|よ}さんの
\ruby{左樣}{さ|う}
\ruby{仰}{おつし}やるのを
\ruby{本當}{ほん|たう}に
した
ところで、
%
\ruby[g]{五十子}{いそこ}さんが
お
\ruby{惡}{わる}く
\ruby{御}{お}なんなさらうもんなら
\ruby{水野}{みづ|の}さん
ていふ
\ruby{方}{かた}が、
%
\ruby{何樣}{ど|ん}なに
\ruby{御騷}{お|さわ}ぎなさるか
\ruby{知}{し}れやしません!。
』

\原本頁{190-9}%
『
\ruby{騷}{さわ}いだつて
\ruby{可}{い〻}やね、% 原本通り「〻(二の字点、揺すり点)」
%
\ruby{騷}{さわ}がして
\ruby{置}{おき}やあ。
』

\原本頁{190-10}%
『
まだ
\ruby{詳}{くは}しい
\ruby{御話}{お|はなし}を
\ruby{伺}{うかゞ}ひませんが、% TODO 原本の「二の字点、揺すり点」に濁点のグリフが見つからないので「ゞ」
%
\ruby{一體}{いつ|たい}
\ruby{水野}{みづ|の}さん
ていふ
\ruby{方}{かた}は
\ruby{何樣}{ど|う}いふ
\ruby{方}{かた}なの?。
』

\原本頁{191-1}%
『
オヤ〳〵
をかしいよお
\ruby{龍}{りう}ちやんは。
%
\ruby{今日}{け|ふ}
お
\ruby{晝{\換字{過}}}{ひる|すぎ}に
\ruby{家}{うち}へ
\ruby{歸}{かへ}つて
\ruby{來}{き}てから、
%
これで
\ruby{丁度}{ちやう|ど}
\ruby{水野}{みづ|の}の
\ruby{事}{こと}を
\ruby{三度}{さん|ど}
\ruby{御聞}{お|き〻}だよ。% 原本通り「〻(二の字点、揺すり点)」
%
ハヽヽ
まさか
\ruby[||j>]{汝}{おまへ}の
やうに
\ruby{{\換字{分}}}{わか}つた
\ruby{人}{ひと}が、
%
\ruby{彼樣}{あ|ん}な
\ruby{唐變木}{たう|へん|ぼく}に
\ruby{何樣}{ど|う}か
\ruby{御爲}{お|し}だとも
\ruby{思}{おも}やあ
\ruby{仕}{し}ないが\換字{子}。
%
よつぽど
\ruby{氣}{き}に
なるやうな
\ruby{變}{へん}な
\ruby{顏}{かほ}でも
\ruby{仕}{し}て
\ruby{居}{ゐ}たのかエ。
%
\ruby{彼}{あり}や
\ruby{何}{なん}でも
\ruby{有}{あ}りや
\ruby{仕}{し}ないのさ。
%
たゞ% TODO 原本の「二の字点、揺すり点」に濁点のグリフが見つからないので「ゞ」
\ruby{彼村}{あす|こ}の
\ruby{學校}{がく|かう}の
\ruby{敎師}{けう|し}で
もつて、
%
\ruby{{\換字{平}}}{ひら}つたく
\ruby{云}{い}やあ
お
\ruby{五十}{い|そ}に
\ruby{惚}{ほ}れてる
といふだけの
\ruby{鈍痴氣}{とん|ち|き}なんだよ。
』

\原本頁{191-8}%
『
だつて
\ruby{其}{そん}なら
\ruby{妾}{わたし}が
\ruby{御師匠}{お|し|よ}さんの
\ruby{御使}{お|つかひ}に、
%
わざ〳〵
\ruby{彼}{あ}の
\ruby{人}{ひと}の
ところへ
\ruby{行}{い}かなくつてもぢや
\ruby{有}{あ}りませんか。
』

\原本頁{191-10}%
『
そりや
お
\ruby{五十}{い|そ}の
\ruby{事}{こと}の
\ruby{關係}{つゞ|き}から\換字{子}、% TODO 原本の「二の字点、揺すり点」に濁点のグリフが見つからないので「ゞ」
%
\ruby{妾}{わたし}も
\ruby{困究}{こ|ま}つた
\ruby{時}{とき}に
\ruby{彼}{あの}
\ruby[||j>]{男}{をとこ}に
\ruby{融{\換字{通}}}{ゆう|づう}を
\ruby{頼}{たの}んだ
\ruby{事}{こと}も
あるし、
%
\ruby{今度}{こん|ど}も
\ruby{全然}{すつ|かり}
お
\ruby{五十}{い|そ}が
\ruby{世話}{せ|わ}に
なつて
\ruby{居}{ゐ}るからさ。
』

\原本頁{192-2}%
『
ぢやあ
\ruby{矢張}{やつ|ぱ}り
\ruby{畢竟}{つま|り}は
\ruby[g]{五十子}{いそこ}さんと
\ruby{一{\換字{所}}}{いつ|しよ}になる
\ruby{譯}{わけ}の
\ruby{方}{かた}ぢや
ありませんか。
%
\ruby{{\換字{道}}理}{だう|り}で
\ruby{心}{しん}から
\ruby{底}{そこ}から
\ruby{御}{ご }
\ruby[<j||]{病}{びやう}
\ruby{人}{にん}を
\ruby{大切}{たい|せつ}に
\ruby{思}{おも}つて
\ruby{居}{ゐ}らつしやる
やうに
\ruby{見}{み}えましたよ。
%
ほんとに
\ruby[g]{五十子}{いそこ}さんは
\ruby{御幸福}{お|しあ|はせ}な%「幸福」ここは「は」
\ruby{事}{こと}!、
%
あんな
\ruby{頼}{たの}もしさうな
\ruby{方}{かた}に
\ruby{御思}{お|おも}はれ
なすつて!。
』

\原本頁{192-6}%
『
ところが
お
\ruby{{\換字{前}}}{まへ}、
%
いくら
\ruby{彼}{あの}
\ruby[||j>]{男}{をとこ}が
\ruby{思}{おも}つても、
%
\ruby{妾}{わたし}の
\ruby{云}{い}ふ
\ruby{事}{こと}さへ
\ruby{聽}{き}かない
やうな、
%
ヘチ
\ruby{頑固}{ぐわん|こ}の
お
\ruby{五十}{い|そ}の
\ruby{事}{こと}だから、
%
\ruby{{\換字{嫌}}}{きら}つて
\ruby{{\換字{嫌}}}{きら}ひぬいて
\ruby{關}{かま}はないのだよ。
%
\ruby{彼}{あ}の
\ruby{男}{をとこ}の
\ruby{思}{おもひ}なんぞは
\ruby[g]{玻瓈}{がらす}に
\ruby{書}{か}く
\ruby{字}{じ}で、
%
\ruby{以上}{い|じやう}
\ruby{經}{たつ}ても
\ruby{{\換字{通}}}{とほ}りつこは
\ruby{無}{な}いのさ。
』

\原本頁{192-10}%
『
でも
\ruby{御師匠}{お|し|よ}さんは
\ruby{{\換字{終}}}{しまひ}にやあ
\ruby{彼}{あ}の
\ruby{人}{ひと}を
\ruby{御婿}{お|むこ}さんにと% (婿 5a7f) 聟 805f
\ruby{思}{おも}つて
らつしやる
でしやう。
』

\原本頁{193-1}%
『
だつて
お
\ruby{五十}{い|そ}が
\ruby{妾}{わたし}の
\ruby{云}{い}ふ
\ruby{事}{こと}なんか
\ruby{聽}{き}くんぢや
\ruby{無}{な}いから
\ruby{仕方}{し|かた}が
\ruby{無}{な}いやね。
%
\ruby{妾}{わたし}あ
\ruby{打}{うつ}
\ruby{棄}{ちや}つて
\ruby{置}{お}いて
\ruby{關}{かま}やあ
\ruby{仕無}{し|な}いのさ。
』

\原本頁{193-3}%
『
あら
\ruby{憫}{かは}% 憫然
\ruby[||j>]{然}{いさう}に、
%
それぢやあ% 「憫然 か(は)いさう」
\ruby{彼}{あ}の
\ruby{人}{ひと}の
\ruby{立塲}{たち|ば}が% 原文通り「塲」
\ruby{無}{な}いぢやあ
\ruby{有}{あ}りませんか。
』

\原本頁{193-5}%
『
だから
\ruby{唐變木}{たう|へん|ぼく}で
\ruby{鈍痴氣}{どん|ち|き}
だといふんだア\換字{子}。
』

\原本頁{193-6}%
『
なんですつて\換字{?!}、
%
マア!。
』

\原本頁{193-7}%
\ruby{優}{やさ}しき
\ruby{姿}{すがた}は
\ruby{其儘}{その|ま〻}に、% 本来は一の字点「ゝ」平仮名繰返し記号% 原本通り「〻(二の字点、揺すり点)」
%
\ruby{身動}{み|じろ}きは
\ruby{一寸}{いつ|すん}も
せざりしが、
%
\ruby{愛嬌}{あい|けう}
こぼる〻% 本来は一の字点「ゝ」平仮名繰返し記号% 原本通り「〻(二の字点、揺すり点)」
\ruby{面}{おもて}
ながら、
%
じろりと
\ruby{斜}{な〻め}に% 原本通り「〻(二の字点、揺すり点)」
\ruby{上}{うは}
\ruby{睨}{にら}みして、
%
お
\ruby{關}{せき}を
\ruby{見}{み}やりたる
お
\ruby{龍}{りう}の
\ruby{眼}{め}には、
%
\ruby{瞋}{いか}るか
\ruby{恨}{うら}むか
\ruby{蔑視}{さげ|す}むか、
%
\ruby{怪}{あや}しき% TODO CHECK 怪
\ruby{一種}{いつ|しゆ}の
\ruby{氣味}{き|み}
\ruby{合籠}{あひ|こも}りて、
%
\ruby{花}{はな}の
\ruby{樹蔭}{こ|かげ}に
\ruby{蛇}{へび}の
\ruby{出}{い}でたる
\ruby{其狀}{そ|れ}にも
\ruby{似}{に}たる
\ruby{風{\換字{情}}}{ふ|ぜい}を
\ruby{見}{み}せたり。
