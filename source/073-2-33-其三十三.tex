\Entry{其三十三}

お
\ruby{龍}{りう}は
\ruby{徐}{しづか}に
\ruby{三絃}{さみ|せん}の
\ruby{糸}{いと}を
\ruby{弛}{ゆる}めて
\ruby{三絃掛}{さみ|せん|かけ}へ
\ruby{掛}{か}け
\ruby{納}{をさ}むれば、
\ruby{今日}{け|ふ}
\ruby{目見得}{め|み|\換字{江}}に
\ruby{來}{きた}りし
\ruby{小婢}{こを|んな}お
\ruby{熊}{くま}は
\ruby{高麗鼠}{こ|ま|ねずみ}のやうにくる〳〵と
\ruby{働}{はたら}きて、しきりに
\ruby{其邊}{そこ|ら}を
\ruby{取}{と}り
\ruby{片付}{かた|づ}けしが、
\ruby{煙草{\換字{盆}}}{たば|こ|ぼん}の
\ruby{傍}{かたはら}より
\ruby{玉}{ぎよく}の
\ruby{煙管}{パイ|プ}のいと
\ruby{小}{ちいさ}なるを
\ruby{拾}{ひろ}ひあげて
\ruby{洋燈近}{らん|ぷ|ちか}くさし
\ruby{出}{いだ}し、

『これ
\ruby{此樣}{こ|ん}な
\ruby{物}{もの}が
\ruby{{\換字{遺}}}{お}ちて
\ruby{居}{を}りました、』

といふ。

\ruby{一}{ひ}ㇳ
\ruby{目見}{め|み}てお
\ruby{龍}{りう}はそれを
\ruby{師匠}{し|しやう}に
\ruby{遞與}{わ|た}し、

『こりやあ
\ruby{傳}{でん}さんが
\ruby{{\換字{遺}}}{わす}れて
\ruby{行}{い}つたのでしやう。
あの
\ruby{人}{ひと}で
\ruby{無}{な}けりやあ
\ruby{此樣}{こ|ん}なものを
\ruby{持}{も}ちさうな
\ruby{人}{ひと}はありませんから。
』

と
\ruby{云}{い}へば、お
\ruby{關}{せき}は
\ruby{受取}{うけ|と}つて
\ruby{指頭}{ゆび|さき}に
\ruby{弄}{もてあそ}び、

『あ\ninojiten{}
\ruby{然樣}{さ|う}だよ、
\ruby{屹度彼}{きつ|と|あ}の
\ruby{男}{をとこ}のだよ。
\ruby{今日}{け|ふ}は
\ruby{妾}{わたし}も
\ruby{大變夙起}{たい|へん|はや|おき}を
\ruby{仕}{し}たし、
\ruby{汝}{おまへ}も
\ruby{{\換字{遠}}}{とほ}いところへ
\ruby{行}{い}つて
\ruby{來}{き}たので
\ruby{草臥}{くた|びれ}て
\ruby{居}{ゐ}るからつていふので
\ruby{{\換字{逐}}}{お}ひ
\ruby{立}{た}て\ninojiten{}やつたもんだから、
\ruby{慌}{あわ}て\ninojiten{}
\ruby{歸}{かへ}つて
\ruby{行}{い}つて
\ruby{{\換字{遺}}}{わす}れたんだらう。
\ruby{取}{と}り
\ruby{上}{あ}げて
\ruby{仕舞}{し|ま}つて
\ruby{{\換字{遣}}}{や}らうか
\ruby{知}{し}らん。
ハヽヽ、マア
\ruby{堪忍}{かん|にん}して
\ruby{{\換字{遣}}}{や}ると
\ruby{仕}{し}やう。
\ruby{何}{なん}でも
\ruby{彼}{あ}の
\ruby{男}{をとこ}は
\ruby{親類内}{しん|るい|うち}かなんぞに、
\ruby{玉}{たま}や
\ruby{石}{いし}の
\ruby{細工}{さい|く}をする
\ruby{家}{うち}かなんぞを
\ruby{有}{も}つて
\ruby{居}{ゐ}るんだよ。
\ruby{御覧}{ご|らん}よ、
\ruby{小}{ちひさ}いけれども
\ruby{此品}{こ|れ}だつて
\ruby{買}{か}つたら
\ruby{廉}{やす}くはなさ\ninojiten{}うなものだ\換字{子}。
』

と、
\ruby{一度}{ひと|たび}はお
\ruby{龍}{りう}に
\ruby{示}{しめ}して、さて
\ruby{火鉢}{ひ|ばち}の
\ruby{抽斗}{ひき|だし}に
\ruby{無{\換字{造}}作}{む|ざう|さ}に
\ruby{藏}{しま}ひたり。

『ハア
\ruby{左樣}{さ|う}なんでしやうよ。
\ruby{兎}{うさぎ}を
\ruby{{\換字{呉}}}{く}れたんでも
\ruby{{\換字{分}}}{わ}かつて
\ruby{居}{ゐ}ますよ。
\ruby{屹度叔父}{きつ|と|を|ぢ}さんか
\ruby{何}{なに}かゞ
\ruby{玉屋}{たま|や}さんなんです\換字{子}。
』

『
\ruby{何樣}{ど|う}も
\ruby{左樣}{さ|う}らしいよ。
\ruby{妾}{わたし}も
\ruby{往日瑪瑙}{いつ|か|め|なう}の
\ruby{好}{い}い
\ruby{色}{いろ}の
\ruby{簪珠}{かんざし|だま}を
\ruby{貰}{もら}つたがね、
\ruby{汝}{おまへ}、
\ruby{兎}{うさぎ}なんぞぢや
\ruby{仕樣}{し|やう}が
\ruby{無}{な}いぢや
\ruby{無}{な}いか。
\ruby{今度}{こん|ど}は
\ruby{寶石入}{い|し|い}りの
\ruby{指輪}{ゆび|わ}かなんか
\ruby{{\換字{強}}{\換字{請}}}{ね|だ}つて
\ruby{御}{お}
\ruby{{\換字{遣}}}{や}りナ。
\ruby{金剛石}{ダ|イ|ヤ}とでも
\ruby{云}{い}つたら
\ruby{二}{に}の
\ruby{足}{あし}を
\ruby{踏}{ふ}むか
\ruby{知}{し}らないが、サファイヤや
\ruby{真珠}{しん|じゆ}の
\ruby{位}{ぐらゐ}なら
\ruby{屹度二}{きつ|と|ふた}つ
\ruby{{\換字{返}}事}{へん|じ}で
\ruby{{\換字{悅}}}{よろこ}んで
\ruby{持}{も}つて
\ruby{來}{く}るよ。
\ruby{物}{もの}を
\ruby{取}{と}つて
\ruby{{\換字{遣}}}{や}るのも
\ruby{功徳}{く|どく}になるのだから
\ruby{關}{かま}やあ
\ruby{仕}{し}ない
\ruby{吹}{ふつ}かけて
\ruby{御覧}{ご|らん}、
\ruby{相槌}{あひ|づち}は
\ruby{妾}{わたし}が
\ruby{巧}{うま}く
\ruby{打}{う}つて
\ruby{上}{あ}げるから。
』

『あら
\ruby{{\換字{嫌}}}{いや}な
\ruby{御師匠}{お|し|よ}さん!。
\ruby{妾}{わたし}あ
\ruby{指輪}{ゆび|わ}なんか
\ruby{欲}{ほ}しかあ
\ruby{無}{な}いんですよ。
しかも
\ruby{傳}{でん}さんになんかあ
\ruby{貰}{もら}ひたたかあ
\ruby{有}{あ}りません。
』

『
\ruby{然樣}{さ|う}かネエ。
\ruby{汝}{おまへ}はほんとに
\ruby{慾}{よく}に
\ruby{掛}{か}けちやあ
\ruby{氣}{き}が
\ruby{{\換字{弱}}}{よわ}いよ。
だが
\ruby{取}{と}つて
\ruby{{\換字{遣}}}{や}る
\ruby{方}{はう}が
\ruby{可}{い\ninojiten}ぢやあ
\ruby{無}{な}いか。
あの
\ruby{兎}{うさぎ}でも
\ruby{知}{し}れてるは\換字{子}、
\ruby{汝}{おまへ}の
\ruby{氣}{き}に
\ruby{入}{い}つたのを
\ruby{見}{み}て
\ruby[g]{何樣}{どんな}なに
\ruby{嬉}{うれし}がつてるか
\ruby{知}{し}れや
\ruby{仕}{し}ないよ。
』

『だから
\ruby{妾}{わたし}あ
\ruby{厭}{いや}なんですよ。
その
\ruby{嬉}{うれし}がられるのが
\ruby{気障}{き|ざ}ぢや
\ruby{有}{あ}りませんか。
』

『ホイ
\ruby{大失敗}{おほ|しく|じり}だネ、ハヽハヽハヽ。
\ruby{指輪}{ゆび|わ}の
\ruby{談}{はなし}で
\ruby{想}{おも}ひ
\ruby{出}{だ}したが、
\ruby{先}{せん}に
\ruby{汝}{おまへ}があの
\ruby{何}{なん}に(
\ruby{源}{げん}を
\ruby{指}{さ}す)
\ruby{御貰}{お|もら}ひのは
\ruby{汝有}{おまへ|も}つておいで\ninojiten{}
\ruby{無}{な}いネエ。
\ruby{妾}{わたし}が
\ruby{見立}{み|た}て\ninojiten{}
\ruby{買}{か}はせたんだからまだ
\ruby{記}{おぼ}えて
\ruby{居}{ゐ}るが、
\ruby{汝彼品}{おまへ|あ|れ}は
\ruby{何樣}{ど|う}か
\ruby{仕}{し}てお
\ruby{仕舞}{し|まひ}かエ。
』

『だって
\ruby{御師匠}{お|し|よ}さん、まだ
\ruby{妾}{わたし}が
\ruby{彼品}{あ|れ}を
\ruby{持}{も}つて
\ruby{居}{ゐ}やう
\ruby{譯}{わけ}は
\ruby{無}{な}からうぢや
\ruby{有}{あ}りませんか。
いよ〳〵
\ruby{不實}{ふ|じつ}な
\ruby{人}{ひと}だと
\ruby{思}{おも}ひつめた
\ruby{時}{とき}は、
\ruby[g]{口惜}{くやし}くつて
\ruby[g]{口惜}{くやし}くつて
\ruby{仕方}{し|かた}が
\ruby{無}{な}かつたんですもの!。
\ruby{宿}{と}めて
\ruby{貰}{もら}つて
\ruby{居}{ゐ}た
\ruby{薬研堀}{や|げん|ぼり}のおとうさん \------
\ruby{御師匠}{お|し|よ}さんは
\ruby{御知}{お|し}んなさらないが
\ruby{妾}{わたし}の
\ruby{仲好}{なか|よ}しの
\ruby{其}{そ}の
\ruby{家}{うち}を
\ruby{出}{で}て、をかアしな
\ruby{氣}{き}になつてふらふらと
\ruby{兩國橋}{りやう|ごく|ばし}の
\ruby{上}{うへ}を
\ruby{往}{い}つたり
\ruby{復}{かへ}つたりした
\ruby{其}{そ}の
\ruby[g]{擧句}{あげく}でした、ふいと
\ruby{意持}{こゝろ|もち}が
\ruby{變}{かは}つたんで
\ruby{指}{ゆび}から
\ruby{{\換字{脱}}}{はづ}して、
\ruby{大川}{おほ|かわ}の
\ruby{流}{なが}れの
\ruby{中}{なか}へ
\ruby{抛}{はふ}り
\ruby{込}{こ}んで
\ruby{仕舞}{し|ま}つたんですよ。
』

『ヘーエ、
\ruby{勿体無}{もつ|たい|な}い
\ruby{事}{こと}を
\ruby{御仕}{お|し}だつた\換字{子}ェ、マァ
\ruby{妾}{わたし}なら
\ruby{同}{おな}じ
\ruby{棄}{す}てるにもお
\ruby{金}{かね}に
\ruby{仕}{し}て
\ruby{棄}{す}てるものを。
だが
\ruby{鑄掛松}{ゐ|かけ|まつ}を
\ruby{色氣}{いろ|け}で
\ruby{行}{い}つたのは、
\ruby{一寸覗}{ちよ|つと|のぞ}いて
\ruby{見}{み}たいやうな
\ruby{幕}{まく}だつた\換字{子}。
』

『ホヽヽ、
\ruby{厭}{いや}ですよ。
たんと
\ruby{御嬲}{お|なぶ}りなさい、
\ruby{人}{ひと}の
\ruby{惡}{わる}い!。
\ruby{今}{いま}なら
\ruby{妾}{わたし}だつて \------。
』

『
\ruby{何樣}{ど|う}
\ruby{御仕}{お|し}だェ、』

『
\ruby{御魚}{お|さかな}にやあ
\ruby{與}{や}らないで
\ruby{瞽女}{ご|ぜ}にでも
\ruby{與}{や}ります。
』

『
\ruby{{\換字{分}}別}{ふん|べつ}らしいけれど
\ruby{{\換字{猶}}且若}{やつ|ぱり|わか}い\換字{子}ェ。
ハヽヽ、
\ruby{瞽女}{ご|ぜ}が
\ruby{汝}{おまへ\ }
\ruby{狂}{くる}ひ
\ruby{浪}{なみ}の
\ruby{彫}{ほり}に
\ruby{小}{ちひさ}な
\ruby{寶石}{い|し}の
\ruby{散}{ち}らばつて
\ruby{居}{ゐ}る
\ruby{彼樣}{あ|ん}な
\ruby{華麗}{はな|やか}な
\ruby{物}{もの}を
\ruby{指}{ゆび}に
\ruby{嵌}{は}めて
\ruby{何樣}{ど|う}なるものかネ。
』

『ぢやあ
\ruby{御師匠}{お|し|よ}さんが
\ruby{妾}{わたし}だつたら
\ruby{何樣}{ど|う}なさるの?。
』

お
\ruby{關}{せき}は
\ruby{我}{わ}が
\ruby{鼻}{はな}を
\ruby{指}{ゆび}さしながら、

『
\ruby{此處}{こ|ゝ}に
\ruby{居}{ゐ}る
\ruby{美麗}{き|れい}な
\ruby{可憐}{か|はゆ}らしい
\ruby{新{\換字{造}}}{しん|ぞ}に
\ruby{與}{や}つて
\ruby{{\換字{悅}}}{よろこ}ばせるはネ。
』

と
\ruby{云}{い}ひさして、ハヽハヽハヽと
\ruby{打笑}{うち|わら}へば、お
\ruby{龍}{りう}もホヽと
\ruby{笑}{わら}ひ
\ruby{出}{だ}し、
\ruby{臺所}{だい|どころ}の
\ruby{方}{かた}に
\ruby{{\換字{退}}}{しりぞ}きたるお
\ruby{熊}{くま}さへ
\ruby{貰}{もら}ひ
\ruby{笑}{わら}ひしたり。

『あ\ninojiten{}、
\ruby{笑}{わら}つたんで
\ruby{心持}{こゝろ|もち}が
\ruby{佳}{い}い。
さあお
\ruby{熊}{くま}や
\ruby{方方戸締}{はう|〴〵|と|じま}りを
\ruby{仕}{し}てお
\ruby{仕舞}{お|しま}ひ。
お
\ruby{龍}{りう}ちやんも
\ruby[g]{歸路}{かへり}に
\ruby{御百度}{お|ひや|くど}まで
\ruby{踏}{ふ}んで
\ruby{御}{お}
\ruby{{\換字{呉}}}{く}れぢやあ、ほんとに
\ruby{隨{\換字{分}}}{ずい|ぶん}おくたびれだらう。
』

\ruby{随意}{こゝろ|まかせ}に
\ruby{休}{やす}めといふ
\ruby{意}{こゝろ}は
\ruby{明}{あき}らかなれど、お
\ruby{龍}{りう}は
\ruby{眠}{ねむ}りたくも
\ruby{思}{おも}はぬ
\ruby{眼}{め}つきなり。

『
\ruby{足}{あし}は
\ruby{些}{ちつと}ばかり
\ruby{草臥}{くた|びれ}ましたけれど、
\ruby[g]{先刻}{さつき}お
\ruby{湯}{ゆ}に
\ruby{入}{はい}つたのでもう
\ruby{治}{なほ}りましたし、
\ruby{氣}{き}は
\ruby{疲勞}{くた|びれ}も
\ruby{何}{なに}も
\ruby{仕}{し}やあ
\ruby{仕}{し}ません。
』

『い\ninojiten{}ねえ
\ruby{若}{わか}い
\ruby{人}{ひと}は!。
\ruby{戀}{こひ}もいさくさも
\ruby{其}{そ}の
\ruby{威勢}{ゐ|せい}のある
\ruby{中}{うち}の
\ruby{花}{はな}なんだよ。
\ruby{妾}{わたし}なんざあ
\ruby{四}{よ}つ
\ruby{木}{ぎ}へ
\ruby{行}{い}かうもんなら
\ruby{二日位}{ふつ|か|ぐらゐ}は
\ruby{腰}{こし}が
\ruby{痛}{いた}いので、しよぼけて
\ruby{居}{ゐ}なくちやあならないんだよ。
』

『ホヽヽ
\ruby{虛言}{う|そ}ばつかり!。
まだ
\ruby{御師匠}{お|し|よ}さんはお
\ruby{若}{わか}いは。
そんな
\ruby{事}{こと}を
\ruby{仰}{おつし}あつても
\ruby{水々}{みづ|〳〵}として
\ruby{在}{い}らつしやるぢぁありませんか。
』

『オヤ
\ruby{汝}{おまへ}こそ
\ruby{人}{ひと}が
\ruby{惡}{わる}いよ、
\ruby{御調戯}{お|から|かひ}で
\ruby{無}{な}い。
い\ninojiten{}よ、
\ruby{何樣}{ど|う}せ
\ruby{奢}{おご}らないから、ハヽハヽハヽ。
』

『でもほんたうですよ。
』

\ruby{渴}{かは}き
\ruby{氣味}{ぎ|み}にや
\ruby{身}{み}を
\ruby{伸}{の}ばして
\ruby{及腰}{および|ごし}に
\ruby{火鉢}{ひ|ばち}の
\ruby{横手}{よこ|て}の
\ruby{茶棚}{ちゃ|だな}より
\ruby{小}{ちひさ}き
\ruby{湯呑}{ゆ|のみ}を
\ruby{取}{と}り、
\ruby{鐵瓶}{てつ|びん}の
\ruby{湯}{ゆ}を
\ruby{注}{つ}ぎて
\ruby{心}{こゝろ}ゆたかに
\ruby{其}{それ}を
\ruby{冷}{さ}まして
\ruby{飮}{の}めるお
\ruby{龍}{りう}を
\ruby{見}{み}れば、
\ruby{女}{をんな}には
\ruby{先}{ま}づ
\ruby{目}{め}につく
\ruby{髮}{かみ}の
\ruby{毛}{け}の
\ruby{漆}{うるし}と
\ruby{黑}{くろ}くて
\ruby{加之膨}{しか|も|ふつ}くりとしたる
\ruby{鬢}{びん}に、
\ruby{櫛}{くし}の
\ruby{齒}{は}の
\ruby{痕}{あと}あざやかに
\ruby{殘}{のこ}りて、
\ruby{肌理密}{き|め|こま}かに
\ruby{色白}{いろ|じろ}なる
\ruby{顏}{かほ}のほんのりと
\ruby{紅}{あか}きは、たゞ
\ruby{是}{これ}
\ruby{淸}{きよ}き
\ruby{芳野紙}{よし|の|がみ}の
\ruby{珊瑚}{さん|ご}を
\ruby{包}{つ\ninojiten}めるに
\ruby{異}{こと}ならず。
ざつに
\ruby{座}{すわ}つたる
\ruby{身}{み}の
\ruby{稍歪}{やゝ|ゆが}みて
\ruby{少}{すこ}し
\ruby{俯}{うつむ}いたるに、
\ruby{細}{ほつそ}りとしたる
\ruby{領頸}{\換字{江}り|くび}のいとゞしほらしく
\ruby{柔和}{にう|わ}に
\ruby{見}{み}えて、
\ruby{物}{もの}ごし
\ruby{恰好冴}{かつ|かう|さ}え〳〵と
\ruby{艶}{\換字{江}ん}なり。

お
\ruby{關}{せき}は
\ruby{見惚}{み|と}れたやうに
\ruby{良久}{やゝ|ひさ}しく
\ruby{見居}{み|ゐ}つ。

『そりやまあ
\ruby{何樣}{ど|う}でも
\ruby{可}{い\ninojiten{}}としたところで、
\ruby{矢張}{やつ|ぱ}りお
\ruby{前}{まへ}にやあ
\ruby{此頃}{この|ごろ}に
\ruby{御馳走}{ご|ち|そう}を
\ruby{仕無}{し|な}くちやあならない。
ほんとに
\ruby{汝}{おまへ}の
\ruby{氣合}{き|あひ}の
\ruby{好}{い}いのには
\ruby{感心}{かん|しん}しちまふよ。
\ruby[g]{歸路}{かへり}には
\ruby[g]{馴染}{なじみ}も
\ruby{無}{な}いお
\ruby{五十}{い|そ}のためにお
\ruby{百度}{ひやく|ど}まで
\ruby{踏}{ふ}んで
\ruby{{\換字{呉}}}{く}れるなんて、
\ruby{何樣}{ど|う}すれば
\ruby{其樣}{そ|ん}なに
\ruby{優}{やさ}しい
\ruby{氣}{き}になつて、しかも
\ruby{俠氣}{をとこ|ぎ}な
\ruby{事}{こと}が
\ruby{出來}{で|き}るだらう。
\ruby{妾}{わたし}や
\ruby{全然}{すつ|かり}お
\ruby{前}{まへ}にやあ
\ruby{惚}{ほ}れつ
\ruby{仕舞}{ち|ま}つたよ。
お
\ruby{前}{まへ}さへ
\ruby{吾家}{う|ち}に
\ruby{居}{ゐ}てお
\ruby{{\換字{呉}}}{く}れなら、あんなお
\ruby{五十}{い|そ}なんか
\ruby{何樣}{ど|う}なつたからつて
\ruby{關}{かま}やあ
\ruby{仕無}{し|な}いよ。
』

『あらマア
\ruby{飛}{と}んでも
\ruby{無}{な}い
\ruby{酷}{ひど}い
\ruby{事}{こと}を!。
お
\ruby{師匠}{し|よ}さんの
\ruby{左樣仰}{さ|う|おつし}やるのを
\ruby{本當}{ほん|たう}にしたところで、
\ruby{五十子}{い|そ|こ}さんがお
\ruby{惡}{わる}く
\ruby{御}{お}なんなさらうもんなら
\ruby{水野}{みづ|の}さんていふ
\ruby{方}{かた}が、
\ruby{何樣}{どん|な}に
\ruby{御騒}{お|さわ}ぎなさるか
\ruby{知}{し}れやしません!。
』

『
\ruby{騒}{さわ}いだつて
\ruby{可}{い\ninojiten{}}やね、
\ruby{騒}{さわ}がして
\ruby{置}{おき}やあ。
』

『まだ
\ruby{詳}{くは}しい
\ruby{御話}{お|はなし}を
\ruby{伺}{うかゞ}ひませんが、
\ruby{一體}{いつ|たい}
\ruby{水野}{みづ|の}さんていふ
\ruby{方}{かた}は
\ruby{何樣}{ど|う}いふ
\ruby{方}{かた}なの?。
』

『オヤ〳〵をかしいよお
\ruby{龍}{りう}ちやんは。
\ruby{今日}{け|ふ}お
\ruby{晝{\換字{過}}}{ひる|すぎ}に
\ruby{家}{いへ}へ
\ruby{歸}{かへ}つて
\ruby{來}{き}てから、これで
\ruby{丁度}{ちや|うど}
\ruby{水野}{みづ|の}の
\ruby{事}{こと}を
\ruby{三度御聞}{さん|ど|お|きゝ}だよ。
ハヽヽまさか
\ruby{汝}{おまへ}のやうに
\ruby{{\換字{分}}}{わか}つた
\ruby{人}{ひと}が、
\ruby{彼樣}{あ|ん}な
\ruby{唐變木}{たう|へん|ぼく}に
\ruby{何樣}{ど|う}か
\ruby{御爲}{お|し}だとも
\ruby{思}{おも}やあ
\ruby{仕}{し}ないが\換字{子}。
よつぽど
\ruby{氣}{き}になるやうな
\ruby{變}{へん}な
\ruby{顏}{かほ}でも
\ruby{仕}{し}て
\ruby{居}{ゐ}たのかェ。
\ruby{彼}{あり}や
\ruby{何}{なん}でも
\ruby{有}{あ}りや
\ruby{仕}{し}ないのさ。
たゞ
\ruby[g]{彼村}{あすこ}の
\ruby{學校}{がく|かう}の
\ruby{{\換字{教}}師}{けう|し}でもつて、
\ruby{{\換字{平}}}{ひら}つたく
\ruby{云}{い}やあお
\ruby{五十}{い|そ}に
\ruby{惚}{ほ}れてるといふだけの
\ruby{鈍痴氣}{どん|ち|き}なんだよ。
』

『だつて
\ruby{其}{そん}なら
\ruby{妾}{わたし}が
\ruby{御師匠}{お|し|よ}さんの
\ruby{御使}{おつ|かひ}に、わざ〳〵
\ruby{彼}{あ}の
\ruby{人}{ひと}のところへ
\ruby{行}{い}かなくつてもぢや
\ruby{有}{あ}りませんか。
』

『そりやお
\ruby{五十}{い|そ}の
\ruby{事}{こと}の
\ruby[g]{關係}{つゞき}から\換字{子}、
\ruby{妾}{わたし}も
\ruby{困究}{こ|ま}つた
\ruby{時}{とき}に
\ruby{彼男}{あの|をとこ}に
\ruby{融{\換字{通}}}{ゆう|づう}を
\ruby{頼}{たの}んだ
\ruby{事}{こと}もあるし、
\ruby[g]{今度}{こんど}も
\ruby{全然}{すつ|かり}お
\ruby{五十}{い|そ}が
\ruby{世話}{せ|わ}になつて
\ruby{居}{ゐ}るからさ。
』

『ぢやあ
\ruby{矢張}{やつ|ぱ}り
\ruby[g]{畢竟}{つまり}は
\ruby{五十子}{い|そ|こ}さんと
\ruby{一{\換字{所}}}{いつ|しよ}になる
\ruby{譯}{わけ}の
\ruby{方}{かた}ぢやありませんか。
\ruby{{\換字{道}}理}{だう|り}で
\ruby{心}{しん}から
\ruby{底}{そこ}から
\ruby{御病人}{ご|びやう|にん}を
\ruby{大切}{たい|せつ}に
\ruby{思}{おも}つて
\ruby{居}{ゐ}らつしやるやうに
\ruby{見}{み}えましたよ。
ほんとに
\ruby{五十子}{い|そ|こ}さんは
\ruby{御幸福}{お|しあ|はせ}な
\ruby{事}{こと}!、あんな
\ruby{頼}{たの}もしさうな
\ruby{方}{かた}に
\ruby{御思}{お|おも}はれなすつて!。
』

『ところがお
\ruby{前}{まへ}、いくら
\ruby{彼}{あ}の
\ruby{男}{をとこ}が
\ruby{思}{おも}つても、
\ruby{妾}{わたし}の
\ruby{云}{い}ふ
\ruby{事}{こと}さへ
\ruby{聽}{き}かないやうな、ヘチ
\ruby{頑固}{ぐわん|こ}のお
\ruby{五十}{い|そ}の
\ruby{事}{こと}だから、
\ruby{{\換字{嫌}}}{きら}つて
\ruby{{\換字{嫌}}}{きら}ひぬいて
\ruby{關}{かま}はないのだよ。
\ruby{彼}{あ}の
\ruby{男}{をとこ}の
\ruby{思}{おもひ}なんぞは
\ruby[g]{玻瓈}{がらす}に
\ruby{書}{か}く
\ruby{字}{じ}で、
\ruby{以上經}{い|じやう|たつ}ても
\ruby{{\換字{通}}}{とほ}りつこは
\ruby{無}{な}いのさ。
』

『でも
\ruby{御師匠}{お|し|よ}さんは
\ruby{{\換字{終}}}{しまひ}には
\ruby{彼}{あ}の
\ruby{人}{ひと}を
\ruby{御婿}{お|むこ}さんにと
\ruby{思}{おも}つてらつしやるでしやう。
』

『だってお
\ruby{五十}{い|そ}が
\ruby{妾}{わたし}の
\ruby{云}{い}ふ
\ruby{事}{こと}なんか
\ruby{聽}{き}くんぢ
\ruby{無}{な}いから
\ruby{仕方}{し|かた}が
\ruby{無}{な}いやね。
\ruby{妾}{わたし}あ
\ruby{打棄}{うつ|ちや}つて
\ruby{置}{お}いて
\ruby{關}{かま}やあ
\ruby{仕無}{し|な}いのさ。
』

『あら
\ruby{憫然}{かはい|さう}に、それぢやあ
\ruby{彼}{あ}の
\ruby{人}{ひと}の
\ruby{立場}{たち|ば}が
\ruby{無}{な}いぢやあ
\ruby{有}{あ}りませんか。
』

『だから
\ruby{唐變木}{たう|へん|ぼく}で
\ruby{鈍痴氣}{どん|ち|き}だといふんだア\換字{子}。
』

『なんですつて\換字{?!}、マア!。
』

\ruby{優}{やさ}しき
\ruby{姿}{すがた}は
\ruby{其儘}{その|ま\ninojiten{}}に、
\ruby{身動}{み|じろ}きは
\ruby{一寸}{いつ|すん}もせざりしが、
\ruby{愛嬌}{あい|けう}こぼる\ninojiten{}
\ruby{面}{おもて}ながら、じろりと
\ruby{斜}{な\ninojiten{}め}に
\ruby{上睨}{う\換字{江}|にら}みして、お
\ruby{關}{せき}を
\ruby{見}{み}やりたるお
\ruby{龍}{りう}の
\ruby{眼}{め}には、
\ruby{瞋}{いか}るか
\ruby{恨}{うら}むか
\ruby{蔑視}{さげ|す}むか、
\ruby{怪}{あや}しき
\ruby{一種}{いつ|しゆ}の
\ruby{氣味合籠}{き|み|あひ|こも}りて、
\ruby{花}{はな}の
\ruby{樹蔭}{こ|かげ}に
\ruby{蛇}{へび}の
\ruby{出}{い}でたる
\ruby{其狀}{そ|れ}にも
\ruby{似}{に}たる
\ruby{風{\換字{情}}}{ふ|ぜい}を
\ruby{見}{み}せたり。

