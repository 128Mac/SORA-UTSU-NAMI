\Entry{其三十四}

% メモ 校正終了 2024-04-28 2024-06-04
\原本頁{194-2}%
\ruby[g]{下物}{さかな }は
\ruby{論}{ろん}
\ruby{無}{な}し、
%
たゞ% 踊り字調整「〻(二の字点、揺すり点)に濁点に見えるが(ゞ)」
\ruby{鮮}{あざら}けきを
\ruby{用}{もち}ゐ、
%
\ruby{酒}{さけ}は
\ruby[g]{定例}{さだめ }あつて、
%
\ruby{必}{かなら}ず
\ruby{醇}{じゆん}なるを
\ruby{{\換字{酌}}}{く}む、
%
\ruby[g]{島木}{しまき }が
\ruby[g]{性{\換字{情}}}{こゝろ }% 踊り字調整「〻(二の字点、揺すり点)に見えるが(ゝ)」
\ruby{見}{み}ゆる
\ruby[g]{待{\換字{遇}}}{もてなし}に、
%
\ruby[g]{日方}{ひ かた}は
\ruby{既}{はや}
\ruby{醉}{よ}ひて% 「醉」は原本通り「よ」で調整
\ruby{面}{おもて}を
\ruby{染}{そ}め
\改行% 校正作業の簡略化のため
、
%
\原本頁{194-4}\改行%
\ruby{大胡座}{おほ|あぐ|ら}
かいて
\ruby{座}{すわ}れる、
%
\ruby[g]{軍服}{ぐんぷく}の
\ruby{怒}{いか}れる
\ruby{肩}{かた}、
%
\ruby{五{\換字{分}}刈}{ご|ぶ|がり}の
\ruby{大}{おほい}なる
\ruby{頭}{あたま}、
%
\ruby[g]{姿勢}{すがた }は
まだ
\ruby{崩}{くづ}さず
\ruby[g]{傲然}{がうぜん}として、
%
\ruby{葡萄酒}{ぶ|だう|しゆ}の
\ruby{盞}{さん}を
\ruby{手}{て}に
しながら、
%
\ruby{親}{した}しきが
\ruby{中}{なか}の
\ruby[g]{打解}{うちと }け
\ruby{話}{ばなし}に
おのづから
\ruby{催}{もよほ}さるゝ% 踊り字調整「〻(二の字点、揺すり点)に見えるが(ゝ)」
\ruby{歡}{よろこ}びの
\ruby{色}{いろ}を
\ruby{{\換字{浮}}}{うか}べて、

\原本頁{194-7}%
『
アヽ
\ruby{快}{い}い
\ruby[||j>]{心}{こゝろ}% 踊り字調整「〻(二の字点、揺すり点)に見えるが(ゝ)」
\ruby[||j>]{持}{ もち}だ、
%
\ruby{佳}{い}い
\ruby{酒}{さけ}だ。
%
いつも
\ruby{葡萄酒}{ぶ|だう|しゆ}とは
\ruby[g]{贅澤}{ぜいたく}な
\ruby{奴}{やつ}だ。
%
\原本頁{194-8}\改行%
\ruby[g]{羽{\換字{勝}}}{は がち}が
\ruby{斷}{ことわ}つて
\ruby{來}{き}たのは
\ruby[g]{殘念}{ざんねん}だが、
%
\ruby{酒}{さけ}は
\ruby{好}{よ}し、
%
\ruby[g]{主人}{しゆじん}の
\ruby{汝}{きさま}も
\ruby{好}{い}い
\ruby[g]{男兒}{をとこ }だし、
%
\ruby{客}{きやく}の
\ruby[g]{乃公}{お れ }も
\ruby{大{\換字{丈}}夫}{だい|ぢやう|ぶ}だし、
%
\ruby[g]{談話}{はなし }が
\ruby[g]{面白}{おもしろ}いので
\ruby{小氣味}{こ|き|み}よく
\ruby{醉}{よ}つた。% 「醉」は原本通り「よ」で調整
』

\原本頁{195-1}%
と
\ruby{云}{い}ひさして
\ruby[g]{滿足}{まんぞく}げに
\ruby[g]{仰飮}{あ ふ }ぎ
\ruby{盡}{つく}せば、
%
\ruby[g]{島木}{しまき }は
\ruby{例}{れい}の
\ruby{布袋顏}{ほ|てい|がほ}して
\ruby{笑}{わら}
\改行% 校正作業の簡略化のため
ひ、

\原本頁{195-3}%
『
ハヽヽ、
\ruby{好}{い}い
\ruby[g]{男兒}{をとこ }たあ
\ruby{有}{あ}り
\ruby{{\換字{難}}}{がて}えナ。
%
だが
\ruby[g]{乃公}{お ら }ア
\ruby{汝}{きさま}にやあ
\ruby[g]{卑劣}{ひ れつ}
\ruby{{\換字{漢}}}{かん}だと
\ruby{罵}{のゝし}られて、% 踊り字調整「〻(二の字点、揺すり点)に見えるが(ゝ)」
%
\ruby{撲}{なぐ}られた
\ruby{事}{こと}が
あつたじやあ
\ruby{無}{ね}えか。
%
ハヽヽ
\ruby[<j||]{汝}{きさま}の% 行末行頭の境界付近なので特例処置を施す
\ruby{云}{い}ふ
ことも
\ruby{當}{あて}にやあ
ならねえ、
%
やつぱり
\ruby[g]{相塲}{さうば }% 原文通り「塲」
\ruby[g]{同樣}{どうやう}で
\ruby{上}{あ}げ
\ruby{下}{さ}げ
があるナ。
』

\原本頁{195-7}%
と
\ruby{打}{うち}
\ruby{戲}{たはむ}れたり。

\原本頁{195-8}%
『
ハヽヽ、
%
\ruby{直}{すぐ}と
\ruby{何}{なん}でも
\ruby{彼}{かん}でも
\ruby[g]{自{\換字{分}}}{じ ぶん}の
\ruby{{\換字{道}}}{みち}に
\ruby[g]{牽{\換字{強}}}{こじつ }けるナ。
%
イヤ
\ruby{時}{とき}の
\ruby[g]{相塲}{さうば }ぢやあ% 原文通り「塲」
\ruby{無}{な}い、
%
\ruby{全}{まつた}くの
\ruby{事}{こと}だ。
%
\ruby{全}{まつた}く
\ruby{汝}{きさま}は
\ruby{好}{い}い
\ruby[g]{男兒}{をとこ }だ、
%
\ruby[g]{{\換字{所}}謂}{いはゆる}
\ruby[g]{好{\換字{漢}}}{かうかん}だナ、
%
\ruby{快男兒}{くわい|だん|じ}だナ。
』

\原本頁{195-11}%
『
ハヽ、
%
\ruby[g]{大層}{たいそう}
\ruby[g]{風向}{かざむき}が
\ruby{好}{い}いが
\ruby{奢}{おご}らねえぜ。
%
\ruby{何}{なん}で
また
\ruby[g]{其樣}{そ う }
\ruby{急}{きふ}に
\ruby{價}{ね}が
\ruby{上}{あが}つたのだ。
』

\原本頁{196-1}%
『
\ruby[g]{羽{\換字{勝}}}{は がち}から
\ruby{聞}{き}いて
\ruby[||j>]{皆}{みんな}
\ruby[||j>]{知}{ し}つたぞ。
%
\ruby{能}{よ}く
\ruby{汝}{きさま}ア
\ruby{彼}{あ}の
\ruby{馬鹿野郎}{ば|か|や|らう}の
\ruby[g]{水野}{みづの }を、
%
\ruby[g]{自{\換字{分}}}{じ ぶん}の
\ruby{危}{あぶな}かつた
\ruby[g]{間際}{ま ぎは}で
\ruby[g]{世話}{せ わ }を
\ruby{仕}{し}て
\ruby{{\換字{遣}}}{や}つたナア。
%
\ruby[g]{流石}{さすが }に
\ruby[g]{島木}{しまき }は
\ruby[g]{島木}{しまき }だ、
%
\ruby{好}{い}い
\ruby[g]{氣象}{きしやう}だ、
%
と
\ruby{眞面目}{ま|じ|め}に
\ruby[g]{{\換字{感}}激}{かんげき}して
\ruby[g]{羽{\換字{勝}}}{は がち}が
\ruby{話}{はな}したぞ。
』

\原本頁{196-5}%
『
ハヽヽ、
%
それで
\ruby{汝}{きさま}ア
\ruby{萬五郎}{まん|ご|らう}に
\ruby{惚}{ほ}れたか。
』

\原本頁{196-6}%
『
ン、
%
\ruby{惚}{ほ}れたナア、
%
ハヽヽ。
%
\ruby[g]{日方}{ひ かた}
\ruby[g]{八郎}{はちらう}も
\ruby{大}{おほき}に
\ruby{惚}{ほ}れ
\ruby{{\換字{込}}}{こ}んだぞ。
』

\原本頁{196-7}%
『
\ruby{{\換字{嫌}}}{いや}な
\ruby[g]{野郎}{や らう}だナア、
%
\ruby{好}{す}かねえ
\ruby{奴}{やつ}だ。
%
\ruby[g]{何程}{いくら }
\ruby{惚}{ほ}れやがつても
\ruby{振}{ふ}りつけて
\ruby{{\換字{遣}}}{や}るぞ。
』

\原本頁{196-9}%
『
\ruby[g]{何故}{な ぜ }?。
』

\原本頁{196-10}%
『
\ruby{惚}{ほ}れやうが
\ruby[g]{一體}{いつたい}
\ruby{氣}{き}に
\ruby{食}{く}はねえ
から。
』

\原本頁{196-11}%
『
フーン、
%
そりやあ
\ruby{{\換字{又}}}{また}
\ruby{何}{なん}で。
』

\原本頁{197-1}%
『
それが
\ruby{{\換字{分}}}{わか}らねえかえ、
%
\ruby[g]{仕方}{し かた}が
\ruby{無}{ね}えナア。
%
\ruby[g]{後學}{こうがく}の
ために
\ruby{記}{おぼ}えて% 送り仮名は原本通り「え」
\ruby{置}{お}きねえ、
%
\ruby{惚}{ほ}れるのに
\ruby[g]{理由}{いはれ }が
あるやうぢやあ
\ruby[g]{眞物}{ほんもの}ぢやあ
\ruby{無}{ね}えんだ。
%
\ruby{同}{おな}じ
\ruby{此}{こ}の
\ruby{萬五郎}{まん|ご|らう}に
\ruby{惚}{ほ}れるならナア‥‥‥。
』

\原本頁{197-4}%
『
ウン。
』

\原本頁{197-5}%
『
\ruby[g]{乃公}{お れ }が
\ruby{惡}{わる}い
\ruby{事}{こと}を
\ruby[g]{爲盡}{し つく}して、
%
\ruby{誰}{たれ}にも
\ruby{彼}{かれ}にも
\ruby[g]{見放}{み はな}されてナ、
%
\ruby{溝}{どぶ}ん
\ruby{中}{なか}へ
でも
\ruby[g]{蹴{\換字{込}}}{け こ }まれた
やうな
\ruby{時}{とき}、
%
\ruby{萬}{まん}ちやん
\ruby{萬}{まん}ちやんツて
\ruby{云}{い}つて
\ruby{吳}{く}れろヤイ。
%
\ruby[g]{左樣}{さ う }したら
\ruby{其}{その}
\ruby{時}{とき}ア
\ruby{此}{こ}の
\ruby{萬}{まん}ちやんも、
%
\ruby[g]{些少}{ちつた }ア
\ruby{惚}{ほ}れ
\ruby{{\換字{返}}}{かへ}して
\ruby{{\換字{遣}}}{や}るめえ
もんでも
\ruby{無}{ね}えんだ。
』

\原本頁{197-9}%
『
アツハヽハヽ、
%
\ruby{甚}{ひど}い
\ruby[g]{氣{\換字{㷔}}}{き{\換字{𛀁}}ん}だナ、
%
\ruby[||j>]{怪}{くわい}
\ruby[||j>]{人}{ じん}の
\ruby[g]{怪語}{くわいご}だ。
%
\ruby[g]{皮肉}{ひ にく}も
\ruby{其}{それ}までに
なると
\ruby[g]{愛嬌}{あいけう}が
\ruby{出}{で}て
\ruby[g]{面白}{おもしろ}い。
%
アヽ
\ruby[g]{{\換字{愉}}快}{ゆくわい}だ
\ruby[g]{大笑}{おほわら}ひに
\ruby{笑}{わら}つたので
\ruby[g]{馬鹿}{ば か }に
\ruby{醉}{よ}つた。% 「醉」は原本通り「よ」で調整
%
\ruby{久}{ひさ}しぶりで
\ruby{一}{ひと}ツ
\ruby[g]{朗吟}{ろうぎん}を
やるぞ。
』

\原本頁{198-1}%
『
\ruby{宜}{よ}からう。
%
\ruby{長}{なが}い
\ruby{事}{こと}
\ruby[||j>]{汝}{きさま}の
\ruby[g]{怒鳴}{ど な }るのも
\ruby{聞}{き}かなかつたナア。
』

\原本頁{198-2}%
『
\ruby[g]{蒲海}{ほ かい}の{---}{---}%
\ruby[<j>]{曉}{あかつき}の{---}{---}%
\ruby{霜}{しも}は{---}{---}、
%
\ruby{馬}{うま}の{---}{---}%
\ruby{尾}{を}に{---}{---}%
\ruby{凝}{こ}り{---}{---}、
%
\ruby[g]{葱山}{そうざん}の{---}{---}%
\ruby{夜}{よる}の{---}{---}%
\ruby{{\換字{雪}}}{ゆき}は{---}{---}、
%
\ruby{旌}{はた}の{---}{---}%
\ruby{竿}{さを}を{---}{---}%
\ruby{撲}{う}つ{---}{---}。
%
ヱースト。
』

\原本頁{198-5}%
『
\ruby{鯨}{くじら}が
\ruby{鳴}{な}くやうな
\ruby{馬鹿聲}{ば|か|ごゑ}だナア、
%
\ruby[g]{障子}{しやうじ}が
\ruby{破}{やぶ}ける
から
もう
\ruby[g]{堪{\換字{忍}}}{か に }して% 原文通り「堪忍」
\ruby{吳}{く}れ、
%
\ruby[g]{此邊}{こゝいら}の% 踊り字調整「〻(二の字点、揺すり点)に見えるが(ゝ)」
\ruby{奴}{やつ}あ
\ruby{目}{め}を
\ruby{{\換字{廻}}}{まは}さあ。
%
しかも
\ruby[g]{唐人}{たうじん}の
\ruby[g]{囈語}{ね ごと}で
\ruby[g]{毫末}{ちつと }も
\ruby{{\換字{分}}}{わか}ら
\ruby{無}{ね}え。
%
\ruby{戰}{いくさ}の
\ruby{詩}{し}の
\ruby{句}{く}かえ。
』

\原本頁{198-8}%
『
ウン
\ruby[g]{其樣}{そ ん }なもんだ。
』

\原本頁{198-9}%
『
\ruby{有}{あ}るかい?\inhibitglue{}%
いよ〳〵、
%
\ruby[||j>]{戰}{どん}
\ruby[||j>]{爭}{ちやん}は。
% \ruby{戰爭}{どん|ちやん}は。
』

\原本頁{198-10}%
『
そんな
\ruby{事}{こと}は
\ruby{乃公{\換字{達}}}{お|れ|たち}よりは
\ruby[g]{汝等}{きさまら}
\ruby{相塲師}{さう|ば|し}% 原文通り「塲」
なんぞの
\ruby{方}{はう}が
\ruby{却}{かへ}つて
\ruby{知}{し}つて
\ruby{居}{ゐ}る
といふ
ことだぞ。
』

\原本頁{199-1}%
\ruby[g]{如是}{か く }
\ruby{云}{い}ひ
\ruby{{\換字{終}}}{をは}りし
\ruby{時}{とき}
\ruby[g]{日方}{ひ かた}は
\ruby{忽}{たちま}ち
\ruby[g]{嚴然}{げんぜん}たる
\ruby[g]{面色}{おもて }に
なりて、

\原本頁{199-2}%
『
いかんナア、
%
\ruby[g]{此樣}{こ ん }な
\ruby[g]{世態}{せ たい}では!。
%
\ruby{實}{じつ}に
\ruby[g]{慨歎}{がいたん}に% 「慨歎」気が高ぶるほど嘆いて心配すること
\ruby{堪}{た}へん。
』

\原本頁{199-3}%
と
\ruby{正}{まさ}しく
\ruby[g]{島木}{しまき }には
\ruby{語}{かた}るならで
\ruby{獨}{ひと}り
\ruby{歎}{たん}ぜしが、
%
\ruby[g]{忽地}{たちまち}にして
\ruby{氣}{き}をかへて、

\原本頁{199-5}%
『
\ruby[g]{{\換字{丈}}夫}{ぢやうぶ}{---}{---}%
\ruby{誓}{ちか}つて
\ruby{國}{くに}に
\ruby{許}{ゆる}す、
%
\ruby[g]{憤惋}{ふん{\換字{𛀁}}ん}{---}{---}%
\ruby{復}{また}
\ruby{何}{なに}か
\ruby{有}{あ}らん、
%
だ。
%
\ruby[g]{少尉}{しようゐ}や
そこらで
\ruby{物}{もの}を
\ruby{思}{おも}ふナア
\ruby{生意氣}{なま|い|き}
なんなのだ。
』

\原本頁{199-7}%
と
\ruby{自}{みづか}ら
\ruby{寛}{ゆる}くして
\ruby{打}{うち}
\ruby{笑}{わら}ひたり。

\原本頁{199-8}%
『
\ruby{時}{とき}に
\ruby[g]{島木}{しまき }!。
%
\ruby[g]{何樣}{ど う }だ
\ruby{今}{いま}から
\ruby[g]{一緖}{いつしよ}に
\ruby[g]{水野}{みづの }を
\ruby{訪}{と}はんか。
%
\ruby{實}{じつ}は
\ruby[g]{羽{\換字{勝}}}{は がち}が
\ruby{來}{き}たら
\ruby{君}{きみ}を
\ruby{誘}{さそ}つて、
%
\ruby[g]{三人}{さんにん}で
\ruby{{\換字{尋}}}{たづ}ねて
\ruby{{\換字{遣}}}{や}らうと
\ruby{思}{おも}つて
\ruby{居}{ゐ}たんだが
\改行% 校正作業の簡略化のため
。
』

\原本頁{199-10}%
『
フーム、
%
\ruby[g]{萬一}{ひよつと}すると
\ruby[||j>]{汝}{きさま}
\ruby[||j>]{出征}{ で|かけ}るのかナ。
』

\原本頁{199-11}%
『
イヤ
まだ
\ruby{其}{それ}は
\ruby[g]{實際}{じつさい}
\ruby{{\換字{分}}}{わか}らんが、
%
\ruby{出}{で}る
やうに
なるに
しても
\ruby{出}{で}ないに
しても、
%
\ruby[g]{此頃}{このごろ}の
\ruby[g]{水野}{みづの }の
\ruby[g]{面色}{かほつき}も
\ruby{見}{み}て
\ruby{{\換字{遣}}}{や}りたいし、
%
\ruby{少}{すこ}し
\ruby{話}{はなし}を
\ruby[g]{仕度}{し た }いと
\ruby{思}{おも}ふ
\ruby{事}{こと}も
\ruby{有}{あ}るから。
』

\原本頁{200-3}%
『
ぢやあ
\ruby{汝}{きさま}の
\ruby{剛}{がう}
\ruby[||j>]{直}{ちよく}な
\ruby{其}{そ}の
\ruby{氣}{き}に
\ruby{任}{まか}せて
\ruby[g]{手{\換字{強}}}{て ごは}い
\ruby[g]{意見}{い けん}を
\ruby{仕}{し}やうと
\ruby{云}{い}ふんだナ。
』

\原本頁{200-5}%
『
\ruby[g]{勿論}{もちろん}だ。
%
\ruby[g]{戀愛}{れんあい}だ
なんぞ
といふ
\ruby{下}{くだ}らない
\ruby{事}{こと}に、
%
\ruby[g]{可惜}{あたら }
\ruby[g]{水野}{みづの }を
\ruby{沈}{しづ}ませて
\ruby{置}{お}いて、
%
\ruby{知}{し}らん
\ruby{顏}{かほ}を
\ruby{仕}{し}て
\ruby{居}{ゐ}ては
\ruby[g]{友{\換字{道}}}{み ち }が
\ruby{立}{た}たんと
\ruby{思}{おも}ふ。
%
\ruby{諫}{いさ}めて
\ruby{諫}{いさ}めて
\ruby{彼}{あ}の
\ruby[g]{水野}{みづの }を、
%
\ruby{舊}{もと}の
\ruby[g]{水野}{みづの }に
\ruby{復}{かへ}らせる
つもりだ。
』

\原本頁{200-8}%
『
そりやあ
\ruby{汝}{きさま}、
%
\ruby[g]{人{\換字{情}}}{じやう }は
\ruby{厚}{あつ}い
\ruby[g]{行爲}{しうち }だが、
%
\ruby[g]{智慧}{ち ゑ }は
\ruby{足}{た}らねえ
\ruby{事}{こと}だぜ!。
』

\原本頁{200-9}%
『
ナニ?。
』

\原本頁{200-10}%
『
マア
\ruby{下}{くだ}ら
\ruby{無}{ね}えから
\ruby{止}{や}めたら
\ruby{宜}{よ}からう!。
』

\原本頁{200-11}%
『
なんだと。
』
