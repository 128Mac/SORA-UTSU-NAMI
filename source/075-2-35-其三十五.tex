\Entry{其三十五}

\ruby{島木}{しま|き}は
\ruby{莞爾}{にこ|り}と
\ruby{笑}{わら}ひながら
\ruby{酒}{さけ}を
\ruby{注}{つ}ぎやりつ、

『また
\ruby{直}{ぢき}に
\ruby{左樣}{さ|う}ムキになつて
\ruby{突掛}{つ〻|か〻}つて% 本来は一の字点「ゝ」平仮名繰返し記号
\ruby{來}{く}るよ。
いくら
\ruby{酒}{さけ}の
\ruby{氣}{き}が
あるからといつて
\ruby{野暮}{や|ぼ}な
\ruby{男}{をとこ}だナ。
』

『
\ruby{何}{なに}も
\ruby{决}{けつ}して
\ruby{怒}{おこ}るのぢやあ
\ruby{無}{な}い。
しかし
\ruby{乃公}{お|れ}が
\ruby{爲}{し}やうと
\ruby{思}{おも}ふことを
\ruby{下}{くだ}らないとは
\ruby{何}{なん}だ。
\ruby{智慧}{ち|ゑ}が
\ruby{足}{た}りても
\ruby{足}{た}らなくつても
\ruby{其}{それ}は
\ruby{仕方}{し|かた}が
\ruby{無}{な}い。
\ruby{默}{だま}つて
\ruby{知}{し}らん
\ruby{顏}{かほ}を
\ruby{仕}{し}ては
\ruby{居}{を}られんから
\ruby{{\換字{尋}}}{たづ}ねやうといふのだ。
\ruby{其}{それ}をたゞ
\ruby{一槪}{いち|がい}に
\ruby{止}{や}めたら
\ruby{宜}{よ}からうと
\ruby{云}{い}はれては
\ruby{面白}{おも|しろ}く
\ruby{無}{な}い。
\ruby{何}{なに}が
\ruby{下}{くだ}らない?、
\ruby{何故}{な|ぜ}
\ruby{智慧}{ち|ゑ}が
\ruby{足}{た}らん?。
』

『
\ruby{何故}{な|ぜ}と
\ruby{云}{いつ}て、
\ruby{考}{かんが}へて
\ruby{見}{み}りやあ
\ruby{{\換字{分}}}{わか}る
\ruby{事}{こと}だ。
』

『いや
\ruby{{\換字{分}}}{わか}らん
\ruby{{\換字{分}}}{わか}らん、
\ruby{考}{かんが}へて
\ruby{見}{み}ても
\ruby{{\換字{分}}}{わか}らんに
\ruby{定}{きま}つて
\ruby{居}{ゐ}る。
よし
\ruby{乃公}{お|れ}の
\ruby{爲}{す}ることが
\ruby{智慧}{ち|ゑ}が
\ruby{足}{た}らんにしろ、
\ruby{智慧}{ち|ゑ}が
\ruby{足}{た}らんために
\ruby{其効}{その|かう}が
\ruby{無}{な}いのならば、
\ruby{汝}{きさま}が
\ruby{智慧}{ち|ゑ}を
\ruby{添}{そ}へて
\ruby{効}{かう}があるやうにして
\ruby{吳}{く}れても
\ruby{宜}{い}い
\ruby{譯}{わけ}では
\ruby{無}{な}いか。
\ruby{水野}{みづ|の}は
\ruby{乃公}{お|れ}ばかりの
\ruby{朋友}{はう|いう}では
\ruby{無}{な}い、
\ruby{汝}{きさま}にも
\ruby{矢張}{や|はり}
\ruby{朋友}{はう|いう}では
\ruby{無}{な}いか。
\ruby{朋友}{はう|いう}の
\ruby{{\換字{道}}}{みち}は
\ruby{何樣}{ど|う}するのが
\ruby{正當}{ほん|たう}だ。
\ruby{互}{たがひ}に
\ruby{氣}{き}に
\ruby{入}{い}るやうにばかり
\ruby{仕}{し}て
\ruby{居}{ゐ}ればそれで
\ruby{可}{い〻}といふのか、そんな% 本来は一の字点「ゝ」平仮名繰返し記号
\ruby{理窟}{り|くつ}がどこにあるものだ。% ここは「理(屈)」ではない
\ruby{勿論}{もち|ろん}
\ruby{朋友}{はう|いう}の
\ruby{幇}{たす}け
\ruby{合}{あ}ふのは
\ruby{知}{し}れた
\ruby{事}{こと}だが、
\ruby{劍{\換字{術}}}{けん|じゆつ}を
\ruby{{\換字{習}}}{なら}へば
\ruby{竹刀}{しな|ひ}に
\ruby{會釋無}{ゑし|やく|な}く
\ruby{引撲}{ひつ|ぱた}き
\ruby{合}{あ}ふのが
\ruby{朋友}{はう|いう}の
\ruby{眞實}{ま|こと}だ、
\ruby{碁}{ご}の
\ruby{一目}{いち|もく}、
\ruby{競射}{きよう|しや}の
\ruby{一點}{いつ|てん}に
\ruby{齒咬}{は|が}みを
\ruby{仕}{し}て
\ruby{爭}{あらそ}ひ
\ruby{合}{あ}ふのも
\ruby{朋友}{とも|だち}の
\ruby{面白味}{おも|しろ|み}だ。
だから
\ruby{欺}{あざむ}かぬ
\ruby{心}{こゝろ}も
\ruby{無}{な}くちやならん。
\ruby{競}{せ}り
\ruby{合}{あ}ふ
\ruby{氣}{き}も
\ruby{無}{な}くちやならん。
まして
\ruby{眼}{め}に
\ruby{餘}{あま}つたり
\ruby{腑}{ふ}に
\ruby{落}{お}ち
\ruby{無}{な}かつたりする
\ruby{事}{こと}があれば、
\ruby{忠告}{ちう|こく}も
\ruby{爲}{し}やうし、
\ruby{爭}{あらそ}ひも
\ruby{爲}{し}やうし、
\ruby{齒}{は}に
\ruby{衣被}{きぬ|き}せず
\ruby{詈}{ののし}り
\ruby{詈}{ののし}らうとも、
\ruby{互}{たがひ}に
\ruby{他人}{ひ|と}の
\ruby{物笑}{もの|わら}ひには、させぬやうに、
\ruby{{\換字{又}}}{また}ならぬやうにと、
\ruby{男兒}{をと|こ}を
\ruby{磨}{みが}きあふのが
\ruby{朋友}{とも|だち}の
\ruby{甲{\換字{斐}}}{か|ひ}では
\ruby{無}{な}いか。
それを
\ruby{何}{なん}だ
\ruby{汝}{きさま}の
\ruby{此頃}{この|ごろ}の
\ruby{仕方}{し|かた}は。
たゞ
\ruby{水野}{みづ|の}の
\ruby{云}{い}ふ
\ruby{{\換字{通}}}{とほ}りにばかり
\ruby{仕}{し}て
\ruby{與}{や}つて
\ruby{居}{ゐ}る。
そりやあ
\ruby{汝}{きさま}の
\ruby{俠氣}{をとこ|ぎ}の
\ruby{振舞}{ふる|まひ}は
\ruby{乃公}{お|れ}も
\ruby{感謝}{かん|しや}して
\ruby{居}{ゐ}るが、それほどに
\ruby{水野}{みづ|の}の
\ruby{爲}{ため}を
\ruby{思}{おも}ふなら、
\ruby{何故}{な|ぜ}
\ruby{一歩}{いつ|ぽ}
\ruby{{\換字{進}}}{すゝ}んで
\ruby{諫}{いさ}めては
\ruby{{\換字{遣}}}{や}らんか、
\ruby{彼}{あ}の
\ruby{男}{をとこ}の
\ruby{{\換字{迷}}}{まよひ}を
\ruby{解}{と}いては
\ruby{{\換字{遣}}}{や}らんか、
\ruby{諫}{いさ}めても
\ruby{聽}{き}かずば
\ruby{何故}{な|ぜ}
\ruby{爭}{あらそ}つては
\ruby{{\換字{遣}}}{や}らん。
\ruby{士爭友}{し|さう|いう}あれば
\ruby{令名}{れい|めい}に
\ruby{離}{はな}れずといふ
\ruby{孝經}{かう|きやう}の
\ruby{語}{ご}を、たとひ
\ruby{其語}{その|ことば}を
\ruby{知}{し}らんでも
\ruby{其}{そ}の
\ruby{理合}{り|あひ}に
\ruby{眜}{くら}いやうな
\ruby{汝}{きさま}では
\ruby{無}{な}いが、
\ruby{何故}{な|ぜ}
\ruby{汝}{きさま}は
\ruby{水野}{みづ|の}の
\ruby{爭友}{さう|いう}にはなつてやらんのだ。
\ruby{云}{い}はゞ
\ruby{汝}{きさま}は
\ruby{水野}{みづ|の}を
\ruby{愛}{あい}して、
\ruby{贔負}{ひゐ|き}に
\ruby{仕{\換字{過}}}{し|す}ぎて
\ruby{間無}{ま|ちが}つた
\ruby{事}{こと}をさせて
\ruby{居}{ゐ}るのだ。
いや
\ruby{頭}{かしら}を
\ruby{振}{ふ}つても
\ruby{左樣}{さ|う}で
\ruby{無}{な}いとは
\ruby{言}{い}はさん、
\ruby{見晴}{み|はら}しでの
\ruby{汝}{きさま}の
\ruby{言葉}{こと|ば}といひ、
\ruby{羽{\換字{勝}}}{は|がち}から
\ruby{聞}{き}いた
\ruby{事實}{じ|ゞつ}といひ、
\ruby{先刻}{さつ|き}からの
\ruby{汝}{きさま}の
\ruby{話}{はな}し
\ruby{工合}{ぐ|あひ}といひ、
\ruby{汝}{きさま}は
\ruby{水野}{みづ|の}の
\ruby{爭友}{さう|いう}となつて、
\ruby{彼}{あ}の
\ruby{男}{をとこ}に
\ruby{{\換字{過}}失無}{くわ|しつ|な}からしめてやら
うといふ
\ruby{考}{かんがへ}は
\ruby{有}{も}たんで、
\ruby{却}{かへ}つて
\ruby{庇護}{か|ば}ひ
\ruby{立}{だて}をする
\ruby{氣味}{き|み}がある。
\ruby{其樣}{そ|ん}な
\ruby{下}{くだ}らんことが
\ruby{何處}{ど|こ}にあるものか。
』

『オイ、
\ruby{大上段}{だい|じやう|だん}に
\ruby{振}{ふ}り
\ruby{被}{かぶ}つて
\ruby{睨}{にら}み
\ruby{{\換字{廻}}}{まは}すなあ
\ruby{其邊}{そこ|いら}で
\ruby{措}{お}いて
\ruby{吳}{く}れ。
\ruby{下}{くだ}らなくつても
\ruby{乃公}{お|れ}は
\ruby{搆}{かま}はねえ。
\ruby{汝}{きさま}の
\ruby{云}{い}ふ
\ruby{事位}{こと|ぐらゐ}は
\ruby{乃公}{お|れ}だつて
\ruby{知}{し}つてゐるが、
\ruby{諫}{いさ}めたつて
\ruby{爭}{あらそ}つたつて
\ruby{役}{やく}に
\ruby{立}{た}たねえ
\ruby{事}{こと}だから、
\ruby{乃公}{お|ら}あ
\ruby{意見}{い|けん}も
\ruby{云}{い}はずに
\ruby{打棄}{うつ|ちや}つて
\ruby{置}{お}くんだ。
\ruby{{\換字{迷}}}{まよ}ふな〳〵
\ruby{思}{おも}ひ
\ruby{切}{き}れつて
\ruby{云}{い}つたつて、
\ruby{料簡}{れう|けん}
\ruby{方}{かた}が
\ruby{{\換字{煙}}管}{きせ|る}の
\ruby{羅宇}{ら|う}のやうにすげかへが
\ruby{出來}{で|き}るものぢやあ
\ruby{無}{な}し、
\ruby{川柳}{せん|りう}が
\ruby{巧}{うめ}え
\ruby{事}{こと}を
\ruby{云}{い}つて
\ruby{居}{ゐ}らあナ、「
\ruby{極無理}{ごく|む|り}な
\ruby{意見}{い|けん}
\ruby{魂魄}{たま|しひ}
\ruby{入}{い}れ
\ruby{換}{かへ}ろ」つて。
よく
\ruby{有}{あ}る
\ruby{奴}{やつ}だが、いくら
\ruby{魂魄}{たま|しひ}を
\ruby{入}{い}れ
\ruby{換}{かへ}ろつて
\ruby{云}{い}つたつて
\ruby{出來}{で|き}る
\ruby{相談}{さう|だん}じやあ
\ruby{無}{ね}え。
しかし
\ruby{水野}{みづ|の}に
\ruby{意見}{い|けん}をするなあ
\ruby{汝}{きさま}の
\ruby{{\換字{勝}}手}{かつ|て}だ。
\ruby{止}{よ}せと
\ruby{云}{い}つたなあ
\ruby{大}{おほき}に
\ruby{御世話}{お|せ|わ}だつた。
\ruby{芝}{しば}で
\ruby{會}{あ}つた
\ruby{時}{とき}
\ruby{云}{い}つた
\ruby{{\換字{通}}}{とほ}りだ。
\ruby{乃公}{お|れ}は
\ruby{乃公}{お|れ}だから
\ruby{乃公}{お|れ}は
\ruby{行}{い}かねえ。

\ruby{汝}{きさま}は
\ruby{汝}{きさま}だから
\ruby{行}{い}くなら
\ruby{行}{い}くがい〻。
』

『よしツ、
\ruby{汝}{きさま}が
\ruby{行}{い}かんでも
\ruby{乃公}{お|れ}は
\ruby{行}{い}かなくつて!。
\ruby{是}{これ}から
\ruby{直}{すぐ}に
\ruby{行}{い}つて
\ruby{諫}{いさ}めて
\ruby{{\換字{遣}}}{や}る。
\ruby{熱誠}{ねつ|せい}を
\ruby{以}{もつ}て
\ruby{大}{おほ}に
\ruby{爭}{あらそ}つて
\ruby{{\換字{遣}}}{や}る。
\ruby{憫然}{かはい|さう}に、
\ruby{可惜}{あつ|たら}
\ruby{好漢}{かう|かん}の
\ruby{水野}{みづ|の}を
\ruby{區々}{く|〻}たる% 本来は一の字点「ゝ」平仮名繰返し記号
\ruby{戀愛}{れん|あい}に
\ruby{悶死}{もん|し}させて
\ruby{堪}{たま}るもんか。
\ruby{日方}{ひ|かた}は
\ruby{彼}{かれ}のために
\ruby{爭友}{さう|いう}を
\ruby{以}{もつ}て
\ruby{任}{にん}じて
\ruby{{\換字{遣}}}{や}る。
\ruby{智慧}{ち|ゑ}の
\ruby{足}{た}らん
\ruby{男}{をとこ}がするの
\ruby{結果}{けつ|か}を
\ruby{見}{み}ろ。
』

『ハヽヽ、
\ruby{乃公}{お|れ}の
\ruby{云}{いつ}た
\ruby{事}{こと}が
\ruby{氣}{き}に
\ruby{入}{い}らなかつたからつて
\ruby{激}{げき}しちやあいけねえ。
\ruby{出}{で
}かけるるなあ
\ruby{可}{い}いが
\ruby{其猛勢}{その|いき|ほい}で
\ruby{行}{い}つて、
\ruby{水野}{みづ|の}と
\ruby{喧嘩}{けん|くわ}をしちやあ
\ruby{汝}{きさま}いけねえぜ。
\ruby{彼}{あ}の
\ruby{男}{をとこ}もおとなしいけれど
\ruby{蟲持}{むし|もち}だから。
』

『ハヽヽ、しかし
\ruby{乃公}{お|れ}の
\ruby{言}{い}ふ
\ruby{事}{こと}を
\ruby{聽}{き}かなかつたら
\ruby{攫}{つか}み
\ruby{挫}{ひし}ぐかも
\ruby{知}{し}れんぞ。
』

『
\ruby{戯談}{じよう|だん}ぢやあ
\ruby{無}{ね}えぜ、
\ruby{人}{ひと}が
\ruby{眞面目}{ま|じ|め}で
\ruby{云}{い}つて
\ruby{居}{ゐ}るのに。
』

『
\ruby{大{\換字{丈}}夫}{だい|ぢやう|ぶ}だ、
\ruby{日方}{ひ|かた}は
\ruby{粗暴}{そ|ばう}でもまさか
\ruby{喧嘩}{けん|くわ}はせん。
』

『い〻かい
\ruby{大將}{たい|しやう}、
\ruby{屹度}{きつ|と}だぜ、
\ruby{釘}{くぎ}をさしたぜ。
』

『ウン、よしツ。
\ruby{時}{とき}に
\ruby{島木}{しま|き}、』

『
\ruby{何}{なん}だ。
』

『
\ruby{汝}{きさま}が
\ruby{{\換字{平}}生}{いつ|も}
\ruby{飮}{の}んで
\ruby{居}{ゐ}る
\ruby{此}{こ}の
\ruby{葡萄酒}{ぶ|だう|しゆ}は
\ruby{中々佳}{なか|〳〵|い}いナ。
』

『それほどぢやあ
\ruby{無}{な}いがマア
\ruby{飮}{の}めるよ。
』

『
\ruby{手土産}{て|みや|げ}に
\ruby{仕}{し}て
\ruby{持}{も}つて
\ruby{行}{い}つて、
\ruby{久}{ひさ}しぶりで
\ruby{水野}{みづ|の}と
\ruby{談}{はな}しながら
\ruby{飮}{の}むのだ。
\ruby{些細}{さ|さい}な
\ruby{御用}{ご|よう}だ、
\ruby{二本}{に|ほん}ばかり
\ruby{徴發}{ちよう|はつ}するぞ。
』

『ハヽヽ、
\ruby{他}{ひと}の
\ruby{物}{もの}を
\ruby{徴發}{ちよう|はつ}して
\ruby{土産}{みや|げ}にするたあ
\ruby{此奴}{こい|つ}あ
\ruby{蟲}{むし}がい〻。
\ruby{可}{い}い〳〵。
\ruby{持}{も}つて
\ruby{行}{い}け、
\ruby{今}{いま}
\ruby{縛}{く〻}らせやう。% 本来は一の字点「ゝ」平仮名繰返し記号
』
