\Entry{其三十五}

% メモ 校正終了 2024-04-29 2024-06-04
\原本頁{201-2}%
\ruby[g]{島木}{しまき }は
\ruby[g]{莞爾}{にこり }と
\ruby{笑}{わら}ひ
ながら
\ruby{酒}{さけ}を
\ruby{注}{つ}ぎやりつ、

\原本頁{201-3}%
『
また
\ruby{直}{ぢき}に
\ruby[g]{左樣}{さ う }
ムキに
なつて
\ruby{突}{つゝ}% 踊り字調整「〻(二の字点、揺すり点)に見えるが(ゝ)」
\ruby{掛}{かゝ}つて% 踊り字調整「〻(二の字点、揺すり点)に見えるが(ゝ)」
\ruby{來}{く}るよ。
%
いくら
\ruby{酒}{さけ}の
\ruby{氣}{き}が
あるから
といつて
\ruby[g]{野暮}{や ぼ }な
\ruby{男}{をとこ}だナ。
』

\原本頁{201-5}%
『
\ruby{何}{なに}も
\ruby{决}{けつ}して
\ruby{怒}{おこ}る
のぢやあ
\ruby{無}{な}い。
%
しかし
\ruby[g]{乃公}{お れ }が
\ruby{爲}{し}やうと
\ruby{思}{おも}ふことを
\ruby{下}{くだ}らないとは
\ruby{何}{なん}だ。
%
\ruby[g]{智慧}{ち ゑ }が
\ruby{足}{た}りても
\ruby{足}{た}らなくつても
\ruby{其}{それ}は
\ruby[g]{仕方}{し かた}が
\ruby{無}{な}い。
%
\ruby{默}{だま}つて
\ruby{知}{し}らん
\ruby{顏}{かほ}を
\ruby{仕}{し}ては
\ruby{居}{を}られん
から
\ruby{{\換字{尋}}}{たづ}ねやう
といふのだ。
%
\ruby{其}{それ}を
たゞ% 踊り字調整「〻(二の字点、揺すり点)に濁点に見えるが(ゞ)」
\ruby[g]{一槪}{いちがい}に
\ruby{止}{や}めたら
\ruby{宜}{よ}からうと
\ruby{云}{い}はれては
\ruby[g]{面白}{おもしろ}く
\ruby{無}{な}い。
%
\ruby{何}{なに}が
\ruby{下}{くだ}らない?、
%
\ruby[g]{何故}{な ぜ }
\ruby[g]{智慧}{ち ゑ }が
\ruby{足}{た}らん?。
』

\原本頁{201-10}%
『
\ruby[g]{何故}{な ぜ }と
\ruby{云}{いつ}て、
%
\ruby{考}{かんが}へて
\ruby{見}{み}りやあ
\ruby{{\換字{分}}}{わか}る
\ruby{事}{こと}だ。
』

\原本頁{202-1}%
『
いや
\ruby{{\換字{分}}}{わか}らん
\ruby{{\換字{分}}}{わか}らん、
%
\ruby{考}{かんが}へて
\ruby{見}{み}ても
\ruby{{\換字{分}}}{わか}らんに
\ruby{定}{きま}つて
\ruby{居}{ゐ}る。
%
よし
\原本頁{202-2}\改行%
\ruby[g]{乃公}{お れ }の
\ruby{爲}{す}ることが
\ruby[g]{智慧}{ち ゑ }が
\ruby{足}{た}らん
にしろ、
%
\ruby[g]{智慧}{ち ゑ }が
\ruby{足}{た}らん
ために
\ruby{其}{その}
\原本頁{202-3}\改行%
\ruby{効}{かう}が
\ruby{無}{な}い
のならば、
%
\ruby{汝}{きさま}が
\ruby[g]{智慧}{ち ゑ }を
\ruby{添}{そ}へて
\ruby{効}{かう}が
あるやうにして
\ruby{吳}{く}れても
\ruby{宜}{い}い
\ruby{譯}{わけ}では
\ruby{無}{な}いか。
%
\ruby[g]{水野}{みづの }は
\ruby[g]{乃公}{お れ }
ばかりの
\ruby[g]{朋友}{ほういう}では
\ruby{無}{な}い、
%
\ruby[<j||]{汝}{きさま}% 行末行頭の境界付近なので特例処置を施す
にも
\ruby[g]{矢張}{や はり}
\ruby[g]{朋友}{ほういう}では
\ruby{無}{な}いか。
%
\ruby[g]{朋友}{ほういう}の
\ruby{{\換字{道}}}{みち}は
\ruby[g]{何樣}{ど う }するのが
\ruby[g]{正當}{ほんたう}だ。
%
\ruby[<j||]{互}{たがひ}% 行末行頭の境界付近なので特例処置を施す
に
\ruby{氣}{き}に
\ruby{入}{い}る
やうに
ばかり
\ruby{仕}{し}て
\ruby{居}{ゐ}れば
それで
\ruby{可}{いゝ}% 踊り字調整「〻(二の字点、揺すり点)に見えるが(ゝ)」
といふのか、
%
そんな
\ruby[g]{理窟}{り くつ}が% ここは「理(屈)」ではない
どこに
あるものだ。
%
\ruby[g]{勿論}{もちろん}
\ruby[g]{朋友}{ほういう}の
\ruby{幇}{たす}け
\ruby{合}{あ}ふのは
\ruby{知}{し}れた
\ruby{事}{こと}だが、
%
\ruby[||j>]{劍}{けん}
\ruby[||j>]{{\換字{術}}}{じゆつ}を
% \ruby{劍{\換字{術}}}{けん|じゆつ}を
\ruby{{\換字{習}}}{なら}へば
\ruby[g]{竹刀}{しなひ }に
\ruby[g]{會釋}{ゑしやく}
\ruby{無}{な}く
\ruby[g]{引撲}{ひつぱた}き
\ruby{合}{あ}ふのが
\ruby[g]{朋友}{ともだち}の
\ruby[g]{眞實}{まこと }
だ、
%
\ruby{碁}{ご}の
\ruby[g]{一目}{いちもく}、
%
\ruby[||j>]{競}{きよう}
\ruby[||j>]{射}{ しや}の
% \ruby{競射}{きよう|しや}の
\ruby[g]{一點}{いつてん}に
\ruby[g]{齒咬}{は が }みを
\ruby{仕}{し}て
\ruby{爭}{あらそ}ひ
\ruby{合}{あ}ふのも
\ruby[g]{朋友}{ともだち}の
\原本頁{202-10}\改行%
\ruby{面白味}{おも|しろ|み}だ。
%
だから
\ruby{欺}{あざむ}かぬ
\ruby{心}{こゝろ}も% 踊り字調整「〻(二の字点、揺すり点)に見えるが(ゝ)」
\ruby{無}{な}くちや
ならん。
%
\ruby{競}{せ}り
\ruby{合}{あ}ふ
\ruby{氣}{き}も
\ruby{無}{な}くちや
ならん。
%
まして
\ruby{眼}{め}に
\ruby{餘}{あま}つたり
\ruby{腑}{ふ}に
\ruby{落}{お}ち
\ruby{無}{な}かつたり
する
\ruby{事}{こと}
\原本頁{203-1}\改行%
が
あれば、
%
\ruby[g]{忠告}{ちうこく}も% 原本通り(ちう)(国会図書館 コマ番号 106/160 p203 l1)
\ruby{爲}{し}やうし、
%
\ruby{爭}{あらそ}ひも
\ruby{爲}{し}やうし、
%
\ruby{齒}{は}に
\ruby{衣}{きぬ}
\ruby{被}{き}せず
\ruby[<j||]{罵}{のゝし}り% 踊り字調整「〻(二の字点、揺すり点)に見えるが(ゝ)」% 行末行頭の境界付近なので特例処置を施す
\ruby{詈}{のゝし}らう% 踊り字調整「〻(二の字点、揺すり点)に見えるが(ゝ)」
とも、
%
\ruby{互}{たがひ}に
\ruby[g]{他人}{ひ と }の
\ruby{物}{もの}
\ruby{笑}{わら}ひには、
%
させぬやうに、
%
\ruby{{\換字{又}}}{また}
なら
\原本頁{203-3}\改行%
ぬやうに
と、
%
\ruby[g]{男兒}{をとこ }を
\ruby{磨}{みが}き
あふのが
\ruby[g]{朋友}{ともだち}の
\ruby[g]{甲{\換字{斐}}}{か ひ }では
\ruby{無}{な}いか。
%
それを
\ruby{何}{なん}だ
\ruby{汝}{きさま}の
\ruby[g]{此頃}{このごろ}の
\ruby[g]{仕方}{し かた}は。
%
たゞ% 踊り字調整「〻(二の字点、揺すり点)に濁点に見えるが(ゞ)」
\ruby[g]{水野}{みづの }の
\ruby{云}{い}ふ
\ruby{{\換字{通}}}{とほ}りに
ばかり
\ruby{仕}{し}て
\ruby{與}{や}
\原本頁{203-5}\改行%
つて
\ruby{居}{ゐ}る。
%
そりやあ
\ruby{汝}{きさま}の
\ruby[g]{俠氣}{をとこぎ}の
\ruby[g]{振舞}{ふるまひ}は
\ruby[g]{乃公}{お れ }も
\ruby[g]{{\換字{感}}謝}{かんしや}して
\ruby{居}{ゐ}るが、
%
\原本頁{203-6}\改行%
それほどに
\ruby[g]{水野}{みづの }の
\ruby{爲}{ため}を
\ruby{思}{おも}ふなら、
%
\ruby[g]{何故}{な ぜ }
\ruby[g]{一歩}{いつぽ }
\ruby{{\換字{進}}}{すゝ}んで% 踊り字調整「〻(二の字点、揺すり点)に見えるが(ゝ)」
\ruby{諫}{いさ}めては
\ruby{{\換字{遣}}}{や}らんか、
%
\ruby{彼}{あ}の
\ruby{男}{をとこ}の
\ruby{{\換字{迷}}}{まよひ}を
\ruby{解}{と}いては
\ruby{{\換字{遣}}}{や}らんか、
%
\ruby{諫}{いさ}めても
\ruby{聽}{き}かずば
\ruby[g]{何故}{な ぜ }
\原本頁{203-8}\改行%
\ruby[||j>]{爭}{あらそ}つては
\ruby{{\換字{遣}}}{や}らん。
%
% [原文](孝経 諌諍) 士有争友、則身不離於令名。
% [書き下し文]      士に争友(そうゆう)有らば、則ち身は令名(れいめい)を離れず。
% [原文の語訳]      士に厳しい諌言をしてくれる友がいれば、名声を失うことはない。
\ruby{士}{し}
\ruby[g]{爭友}{さういう}
あれば
\ruby[g]{令名}{れいめい}に
\ruby{離}{はな}れず
といふ
\ruby{孝}{かう}
\ruby[||j>]{經}{きやう}の
\ruby{語}{ご}を
\改行% 校正作業の簡略化のため
、
%
\原本頁{203-9}\改行%
たとひ
\ruby{其}{その}
\ruby[||j>]{語}{ことば}を
\ruby{知}{し}らん
でも
\ruby{其}{そ}の
\ruby[g]{理合}{り あひ}に
\ruby{眜}{くら}い
やうな
\ruby{汝}{きさま}では
\ruby{無}{な}いが、
%
\原本頁{203-10}\改行%
\ruby[g]{何故}{な ぜ }
\ruby[||j>]{汝}{きさま}は
\ruby[g]{水野}{みづの }の
\ruby[g]{爭友}{さういう}には
なつて
やらんのだ。
%
\ruby{云}{い}はゞ% 踊り字調整「〻(二の字点、揺すり点)に濁点に見えるが(ゞ)」
\ruby{汝}{きさま}は
\ruby[g]{水野}{みづの }を
\原本頁{203-11}\改行%
\ruby{愛}{あい}して、
%
\ruby[g]{贔負}{ひゐき }に
\ruby{仕}{し}
\ruby{{\換字{過}}}{す}ぎて
\ruby[g]{間無}{まちが }つた
\ruby{事}{こと}を
させて
\ruby{居}{ゐ}るのだ。
%
いや
\ruby[<j||]{頭}{かしら}
\原本頁{204-1}\改行%
を
\ruby{振}{ふ}つても
\ruby[g]{左樣}{さ う }で
\ruby{無}{な}いとは
\ruby{言}{い}はさん、
%
\ruby[g]{見晴}{み はら}しでの
\ruby{汝}{きさま}の
\ruby[g]{言葉}{ことば }
とい
\原本頁{204-2}\改行%
ひ、
%
\ruby[g]{羽{\換字{勝}}}{は がち}から
\ruby{聞}{き}いた
\ruby[g]{事實}{じ ゞつ}% 踊り字調整「〻(二の字点、揺すり点)に濁点に見えるが(ゞ)」
といひ、
%
\ruby[g]{先刻}{さつき }からの
\ruby{汝}{きさま}の
\ruby{話}{はな}し
\ruby[g]{工合}{ぐ あひ}
とい
\原本頁{204-3}\改行%
ひ、
%
\ruby{汝}{きさま}は
\ruby[g]{水野}{みづの }の
\ruby[g]{爭友}{さういう}と
なつて、
%
\ruby{彼}{あ}の
\ruby{男}{をとこ}に
\ruby[g]{{\換字{過}}失}{くわしつ}
\ruby{無}{な}からしめて
やら
\原本頁{2}\改行%
うといふ
\ruby[<j>]{考}{かんがへ}は
\ruby{有}{も}たんで、
%
\ruby{却}{かへ}つて
\ruby[g]{庇護}{か ば }ひ
\ruby{立}{だて}をする
\ruby[g]{氣味}{き み }がある。
%
\ruby[g]{其樣}{そ ん }な
\ruby{下}{くだ}らんことが
\ruby[g]{何處}{ど こ }に
あるものか。
』

\原本頁{204-6}%
『
オイ、
%
\ruby{大}{おほ }% 「上」の前後に突出部分の調整に全角スペースを挿入
\ruby{上}{じやう}
\ruby{段}{ だん}に
\ruby{振}{ふ}り
\ruby{被}{かぶ}つて
\ruby{睨}{にら}み
\ruby{{\換字{廻}}}{まは}すなあ
\ruby[g]{其邊}{そこいら}で
\ruby{措}{お}いて
\ruby{吳}{く}れ。
%
\原本頁{204-7}\改行%
\ruby{下}{くだ}らなくつても
\ruby[g]{乃公}{お れ }は
\ruby{搆}{かま}はねえ。
%
\ruby{汝}{きさま}の
\ruby{云}{い}ふ
\ruby{事}{こと}
\ruby[||j>]{位}{ぐらゐ}は
\ruby[g]{乃公}{お れ }だつて
\ruby{知}{し}
\原本頁{204-8}\改行%
つてゐるが、
%
\ruby{諫}{いさ}めたつて
\ruby{爭}{あらそ}つたつて
\ruby{役}{やく}に
\ruby{立}{た}たねえ
\ruby{事}{こと}だから、
%
\ruby[g]{乃公}{お ら }あ
\ruby[g]{意見}{い けん}も
\ruby{云}{い}はずに
\ruby[g]{打棄}{うつちや}つて
\ruby{置}{お}くんだ。
%
\ruby{{\換字{迷}}}{まよ}ふな〳〵
\ruby{思}{おも}ひ
\ruby{切}{き}れつ
\原本頁{204-10}\改行%
て
\ruby{云}{い}つたつて、
%
\ruby[g]{料簡}{れうけん}
\ruby{方}{かた}が
\ruby[g]{{\換字{煙}}管}{きせる }の
\ruby[g]{羅宇}{ら う }の
やうに
すげかへが
\ruby[g]{出來}{で き }る
\原本頁{204-11}\改行%
もの
ぢやあ
\ruby{無}{な}し、
%
\ruby[g]{川柳}{せんりう}が
\ruby{巧}{うめ}え
\ruby{事}{こと}を
\ruby{云}{い}つて
\ruby{居}{ゐ}らあナ、
%
「
\ruby{極}{ごく}
\ruby[g]{無理}{む り }な
\ruby[g]{意見}{い けん}
\ruby[g]{魂魄}{たましひ}
\ruby{入}{い}れ
\ruby{換}{かへ}ろ
」つて。
%
よく
\ruby{有}{あ}る
\ruby{奴}{やつ}だが、
%
いくら
\ruby[g]{魂魄}{たましひ}を
\ruby{入}{い}れ
\ruby{換}{かへ}
% 川柳引用符の「と」の取り扱いにより行溢れが生じ、改行の制御が困難
\原本頁{205-2}\改行%
ろつて
\ruby{云}{い}つたつて
\ruby[g]{出來}{で き }る
\ruby[g]{相談}{さうだん}
じやあ
\ruby{無}{ね}え。
%
しかし
\ruby[g]{水野}{みづの }に
\ruby[g]{意見}{い けん}を
するなあ
\ruby{汝}{きさま}の
\ruby[g]{{\換字{勝}}手}{かつて }だ。
%
\ruby{止}{よ}せと
\ruby{云}{い}つたなあ
\ruby{大}{おほき}に
\ruby{御世話}{お|せ|わ}だつた。
%
\ruby{芝}{しば}で
\ruby{會}{あ}つた
\ruby{時}{とき}
\ruby{云}{い}つた
\ruby{{\換字{通}}}{とほ}りだ。
%
\ruby[g]{乃公}{お れ }は
\ruby[g]{乃公}{お れ }だから
\ruby[g]{乃公}{お れ }は
\ruby{行}{い}かねえ。
%
\ruby{汝}{きさま}は
\ruby{汝}{きさま}だから
\ruby{行}{い}くなら
\ruby{行}{い}くが
いゝ。% 踊り字調整「〻(二の字点、揺すり点)に見えるが(ゝ)」
』

\原本頁{205-6}%
『
よしツ、
%
\ruby{汝}{きさま}が
\ruby{行}{い}かんでも
\ruby[g]{乃公}{お れ }は
\ruby{行}{い}かなくつて!。
%
\ruby{是}{これ}から
\ruby{直}{すぐ}に
\原本頁{205-7}\改行%
\ruby{行}{い}つて
\ruby{諫}{いさ}めて
\ruby{{\換字{遣}}}{や}る。
%
\ruby[g]{熱誠}{ねつせい}を
\ruby{以}{もつ}て
\ruby{大}{おほい}に
\ruby{爭}{あらそ}つて
\ruby{{\換字{遣}}}{や}る。
%
\ruby{憫}{かは}% 「憫然 か(は)いさう」
\ruby[||j>]{然}{いさう}に、
%
\ruby[g]{可惜}{あつたら}
\原本頁{205-8}\改行%
\ruby[g]{好{\換字{漢}}}{かうかん}の
\ruby[g]{水野}{みづの }を
\ruby[g]{區々}{く ゝ }たる% 踊り字調整「〻(二の字点、揺すり点)に見えるが(ゝ)」
\ruby[g]{戀愛}{れんあい}に
\ruby[g]{悶死}{もんし }させて
\ruby{堪}{たま}るもんか。
%
\ruby[g]{日方}{ひ かた}は
\ruby{彼}{かれ}
\原本頁{205-9}\改行%
のために
\ruby[g]{爭友}{さういう}を
\ruby{以}{もつ}て
\ruby{任}{にん}じて
\ruby{{\換字{遣}}}{や}る。
%
\ruby[g]{智慧}{ち ゑ }の
\ruby{足}{た}らん
\ruby{男}{をとこ}がする
\ruby{事}{こと}の
\ruby[g]{結果}{けつか }を
\ruby{見}{み}ろ。
』

\原本頁{205-11}%
『
ハヽヽ、
%
\ruby[g]{乃公}{お れ }の
\ruby{云}{いつ}た
\ruby{事}{こと}が
\ruby{氣}{き}に
\ruby{入}{い}らなかつた
からつて
\ruby{激}{げき}しちや
\原本頁{206-1}\改行%
あ
いけねえ。
%
\ruby{出}{で}かけるなあ
\ruby{可}{よ}いが
\ruby{其}{その}
\ruby[g]{猛勢}{いきほい}で
\ruby{行}{い}つて、
%
\ruby[g]{水野}{みづの }と
\ruby[g]{喧嘩}{けんくわ}を
しちやあ
\ruby{汝}{きさま}
いけねえぜ。
%
\ruby{彼}{あ}の
\ruby{男}{をとこ}も
おとなしいけれど
\ruby{蟲}{むし}
\ruby{持}{もち}だか
\原本頁{206-3}\改行%
ら。
』

\原本頁{206-4}%
『
ハヽヽ、
%
しかし
\ruby[g]{乃公}{お れ }の
\ruby{言}{い}ふ
\ruby{事}{こと}を
\ruby{聽}{き}かなかつたら
\ruby{攫}{つか}み
\ruby{挫}{ひし}ぐかも
\ruby{知}{し}れんぞ。
』

\原本頁{206-6}%
『
\ruby[||j>]{戱}{じよう}
\ruby[||j>]{談}{ だん}ぢやあ
% \ruby{戱談}{じよう|だん}ぢやあ
\ruby{無}{ね}えぜ、
%
\ruby{人}{ひと}が
\ruby{眞面目}{ま|じ|め}で
\ruby{云}{い}つて
\ruby{居}{ゐ}るのに。
』

\原本頁{206-7}%
『
\ruby{大{\換字{丈}}夫}{だい|ぢやう|ぶ}だ、
%
\ruby[g]{日方}{ひ かた}は
\ruby[g]{粗暴}{そ ばう}でも
まさか
\ruby[g]{喧嘩}{けんくわ}は
せん。
』

\原本頁{206-8}%
『
いゝかい% 踊り字調整「〻(二の字点、揺すり点)に見えるが(ゝ)」
\ruby[||j>]{大}{たい}
\ruby[||j>]{將}{しやう}、
% \ruby{大將}{たい|しやう}、
%
\ruby[g]{屹度}{きつと }だぜ、
%
\ruby{釘}{くぎ}を
さしたぜ。
』

\原本頁{206-9}%
『
ウン、
%
よしツ。
%
\ruby{時}{とき}に
\ruby[g]{島木}{しまき }、
』

\原本頁{206-10}%
『
\ruby{何}{なん}だ。
』

\原本頁{206-11}%
『
\ruby{汝}{きさま}が
\ruby[g]{{\換字{平}}生}{いつも }% ルビ調整(原本通り)
\ruby{飮}{の}んで
\ruby{居}{ゐ}る
\ruby{此}{こ}の
\ruby{葡萄酒}{ぶ|だう|しゆ}は
\ruby[g]{中々}{なか〳〵}
\ruby{佳}{い}いナ。
』

\原本頁{207-1}%
『
それほど
ぢやあ
\ruby{無}{な}いが
マア
\ruby{飮}{の}めるよ。
』

\原本頁{207-2}%
『
\ruby{手土產}{て|みや|げ}に
\ruby{仕}{し}て
\ruby{持}{も}つて
\ruby{行}{い}つて、
%
\ruby{久}{ひさ}しぶりで
\ruby[g]{水野}{みづの }と
\ruby{談}{はな}し
ながら
\ruby{飮}{の}むのだ。
%
\ruby[g]{些細}{さ さい}な
\ruby[g]{御用}{ご よう}だ、
%
\ruby[g]{二本}{に ほん}
ばかり
\ruby[||j>]{徴}{ちよう}
\ruby[||j>]{發}{ はつ}
% \ruby{徴發}{ちよう|はつ}
するぞ。
』

\原本頁{207-4}%
『
ハヽヽ、
%
\ruby{他}{ひと}の
\ruby{物}{もの}を
\ruby[||j>]{徴}{ちよう}
\ruby[||j>]{發}{ はつ}して
% \ruby{徴發}{ちよう|はつ}して
\ruby[g]{土產}{みやげ }に
するたあ
\ruby[g]{此奴}{こいつ }あ
\ruby{蟲}{むし}が
いゝ。% 踊り字調整「〻(二の字点、揺すり点)に見えるが(ゝ)」
%
\ruby{可}{い}い〳〵。
%
\ruby{持}{も}つて
\ruby{行}{い}け、
%
\ruby{今}{いま}
\ruby{縛}{くゝ}らせやう。% 踊り字調整「〻(二の字点、揺すり点)に見えるが(ゝ)」
』
