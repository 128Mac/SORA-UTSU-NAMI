\Entry{其三十五}

% メモ 校正終了 2024-04-29
\原本頁{201-2}%
\ruby{島木}{しま|き}は
\ruby{莞爾}{にこ|り}と
\ruby{笑}{わら}ひ
ながら
\ruby{酒}{さけ}を
\ruby{注}{つ}ぎやりつ、

\原本頁{201-3}%
『
また
\ruby{直}{ぢき}に
\ruby{左樣}{さ|う}
ムキに
なつて
\ruby{突}{つ〻}% 本来は一の字点「ゝ」平仮名繰返し記号% 原本通り「〻(二の字点、揺すり点)」
\ruby{掛}{か〻}つて% 本来は一の字点「ゝ」平仮名繰返し記号% 原本通り「〻(二の字点、揺すり点)」
\ruby{來}{く}るよ。
%
いくら
\ruby{酒}{さけ}の
\ruby{氣}{き}が
あるから
といつて
\ruby{野暮}{や|ぼ}な
\ruby{男}{をとこ}だナ。
』

\原本頁{201-5}%
『
\ruby{何}{なに}も
\ruby{决}{けつ}して
\ruby{怒}{おこ}る
のぢやあ
\ruby{無}{な}い。
%
しかし
\ruby{乃公}{お|れ}が
\ruby{爲}{し}やうと
\ruby{思}{おも}ふこ
\原本頁{201-6}\改行%
とを
\ruby{下}{くだ}らないとは
\ruby{何}{なん}だ。
%
\ruby{智慧}{ち|ゑ}が
\ruby{足}{た}りても
\ruby{足}{た}らなくつても
\ruby{其}{それ}は
\ruby{仕方}{し|かた}が
\ruby{無}{な}い。
%
\ruby{默}{だま}つて
\ruby{知}{し}らん
\ruby{顏}{かほ}を
\ruby{仕}{し}ては
\ruby{居}{を}られん
から
\ruby{{\換字{尋}}}{たづ}ねやう
とい
\原本頁{201-8}\改行%
ふのだ。
%
\ruby{其}{それ}を
たゞ% TODO 原本の「二の字点、揺すり点」に濁点のグリフが見つからないので「ゞ」
\ruby{一槪}{いち|がい}に
\ruby{止}{や}めたら
\ruby{宜}{よ}からうと
\ruby{云}{い}はれては
\ruby{面白}{おも|しろ}く
\ruby{無}{な}い。
%
\ruby{何}{なに}が
\ruby{下}{くだ}らない?、
%
\ruby{何故}{な|ぜ}
\ruby{智慧}{ち|ゑ}が
\ruby{足}{た}らん?。
』

\原本頁{201-10}%
『
\ruby{何故}{な|ぜ}と
\ruby{云}{いつ}て、
%
\ruby{考}{かんが}へて
\ruby{見}{み}りやあ
\ruby{{\換字{分}}}{わか}る
\ruby{事}{こと}だ。
』

\原本頁{202-1}%
『
いや
\ruby{{\換字{分}}}{わか}らん
\ruby{{\換字{分}}}{わか}らん、
%
\ruby{考}{かんが}へて
\ruby{見}{み}ても
\ruby{{\換字{分}}}{わか}らんに
\ruby{定}{きま}つて
\ruby{居}{ゐ}る。
%
よし
\原本頁{202-2}\改行%
\ruby{乃公}{お|れ}の
\ruby{爲}{す}ることが
\ruby{智慧}{ち|ゑ}が
\ruby{足}{た}らん
にしろ、
%
\ruby{智慧}{ち|ゑ}が
\ruby{足}{た}らん
ために
\ruby{其}{その}
\原本頁{202-3}\改行%
\ruby{効}{かう}が
\ruby{無}{な}い
のならば、
%
\ruby{汝}{きさま}が
\ruby{智慧}{ち|ゑ}を
\ruby{添}{そ}へて
\ruby{効}{かう}が
あるやうにして
\ruby{吳}{く}れ
\原本頁{202-4}\改行%
ても
\ruby{宜}{い}い
\ruby{譯}{わけ}では
\ruby{無}{な}いか。
%
\ruby{水野}{みづ|の}は
\ruby{乃公}{お|れ}
ばかりの
\ruby{朋友}{ほう|いう}では
\ruby{無}{な}い、
%
\ruby{汝}{きさま}
\原本頁{202-5}\改行%
にも
\ruby{矢張}{や|はり}
\ruby{朋友}{ほう|いう}では
\ruby{無}{な}いか。
%
\ruby{朋友}{ほう|いう}の
\ruby{{\換字{道}}}{みち}は
\ruby{何樣}{ど|う}するのが
\ruby{正當}{ほん|たう}だ。
%
\ruby[<j||]{互}{たがひ}
\原本頁{202-6}\改行%
に
\ruby{氣}{き}に
\ruby{入}{い}る
やうに
ばかり
\ruby{仕}{し}て
\ruby{居}{ゐ}れば
それで
\ruby{可}{い〻}% 原本通り「〻(二の字点、揺すり点)」
といふのか、
%
そんな
\ruby{理窟}{り|くつ}が% ここは「理(屈)」ではない
どこに
あるものだ。
%
\ruby{勿論}{もち|ろん}
\ruby{朋友}{ほう|いう}の
\ruby{幇}{たす}け
\ruby{合}{あ}ふのは
\ruby{知}{し}れた
\ruby{事}{こと}
\原本頁{202-8}\改行%
だが、
%
\ruby{劍{\換字{術}}}{けん|じゆつ}を
\ruby{{\換字{習}}}{なら}へば
\ruby{竹刀}{しな|ひ}に
\ruby{會釋}{ゑ|しやく}
\ruby{無}{な}く
\ruby{引撲}{ひつ|ぱた}き
\ruby{合}{あ}ふのが
\ruby{朋友}{とも|だち}の
\ruby{眞實}{まこ|と}
\原本頁{202-9}\改行%
だ、
%
\ruby{碁}{ご}の
\ruby{一目}{いち|もく}、
%
\ruby{競射}{きよう|しや}の
\ruby{一點}{いつ|てん}に
\ruby{齒咬}{は|が}みを
\ruby{仕}{し}て
\ruby{爭}{あらそ}ひ
\ruby{合}{あ}ふのも
\ruby{朋友}{とも|だち}の
\原本頁{202-10}\改行%
\ruby{面白味}{おも|しろ|み}だ。
%
だから
\ruby{欺}{あざむ}かぬ
\ruby{心}{こ〻ろ}も% 原本通り「〻(二の字点、揺すり点)」
\ruby{無}{な}くちや
ならん。
%
\ruby{競}{せ}り
\ruby{合}{あ}ふ
\ruby{氣}{き}も
\ruby{無}{な}くちや
ならん。
%
まして
\ruby{眼}{め}に
\ruby{餘}{あま}つたり
\ruby{腑}{ふ}に
\ruby{落}{お}ち
\ruby{無}{な}かつたり
する
\ruby{事}{こと}
\原本頁{203-1}\改行%
が
あれば、
%
\ruby{忠告}{ちう|こく}も
\ruby{爲}{し}やうし、
%
\ruby{爭}{あらそ}ひも
\ruby{爲}{し}やうし、
%
\ruby{齒}{は}に
\ruby{衣}{きぬ}
\ruby{被}{き}せず
\ruby{罵}{の〻し}り% 原本通り「〻(二の字点、揺すり点)」
\ruby{詈}{の〻し}らう% 原本通り「〻(二の字点、揺すり点)」
とも、
%
\ruby{互}{たがひ}に
\ruby{他人}{ひ|と}の
\ruby{物}{もの}
\ruby{笑}{わら}ひには、
%
させぬやうに、
%
\ruby{{\換字{又}}}{また}
なら
\原本頁{203-3}\改行%
ぬやうに
と、
%
\ruby{男兒}{をと|こ}を
\ruby{磨}{みが}き
あふのが
\ruby{朋友}{とも|だち}の
\ruby{甲{\換字{斐}}}{か|ひ}では
\ruby{無}{な}いか。
%
それを
\ruby{何}{なん}だ
\ruby{汝}{きさま}の
\ruby{此頃}{この|ごろ}の
\ruby{仕方}{し|かた}は。
%
たゞ% TODO 原本の「二の字点、揺すり点」に濁点のグリフが見つからないので「ゞ」
\ruby{水野}{みづ|の}の
\ruby{云}{い}ふ
\ruby{{\換字{通}}}{とほ}りに
ばかり
\ruby{仕}{し}て
\ruby{與}{や}
\原本頁{203-5}\改行%
つて
\ruby{居}{ゐ}る。
%
そりやあ
\ruby{汝}{きさま}の
\ruby{俠氣}{をとこ|ぎ}の
\ruby{振舞}{ふる|まひ}は
\ruby{乃公}{お|れ}も
\ruby{感謝}{かん|しや}して
\ruby{居}{ゐ}るが、
%
\原本頁{203-6}\改行%
それほどに
\ruby{水野}{みづ|の}の
\ruby{爲}{ため}を
\ruby{思}{おも}ふなら、
%
\ruby{何故}{な|ぜ}
\ruby{一歩}{いつ|ぽ}
\ruby{{\換字{進}}}{す〻}んで% 原本通り「〻(二の字点、揺すり点)」
\ruby{諫}{いさ}めては
\ruby{{\換字{遣}}}{や}らんか、
%
\ruby{彼}{あ}の
\ruby{男}{をとこ}の
\ruby{{\換字{迷}}}{まよひ}を
\ruby{解}{と}いては
\ruby{{\換字{遣}}}{や}らんか、
%
\ruby{諫}{いさ}めても
\ruby{聽}{き}かずば
\ruby{何故}{な|ぜ}
\原本頁{203-8}\改行%
\ruby[||j>]{爭}{あらそ}つては
\ruby{{\換字{遣}}}{や}らん。
%
% [原文](孝経 諌諍) 士有争友、則身不離於令名。
% [書き下し文]      士に争友(そうゆう)有らば、則ち身は令名(れいめい)を離れず。
% [原文の語訳]      士に厳しい諌言をしてくれる友がいれば、名声を失うことはない。
\ruby{士}{し}
\ruby{爭友}{さう|いう}
あれば
\ruby{令名}{れい|めい}に
\ruby{離}{はな}れず
といふ
\ruby{孝}{かう}
\ruby[||j>]{經}{きやう}の
\ruby{語}{ご}を、
%
\原本頁{203-9}\改行%
たとひ
\ruby{其}{その}
\ruby[||j>]{語}{ことば}を
\ruby{知}{し}らん
でも
\ruby{其}{そ}の
\ruby{理合}{り|あひ}に
\ruby{眜}{くら}い
やうな
\ruby{汝}{きさま}では
\ruby{無}{な}いが、
%
\原本頁{203-10}\改行%
\ruby{何故}{な|ぜ}
\ruby[||j>]{汝}{きさま}は
\ruby{水野}{みづ|の}の
\ruby{爭友}{さう|いう}には
なつて
やらんのだ。
%
\ruby{云}{い}はゞ% TODO 原本の「二の字点、揺すり点」に濁点のグリフが見つからないので「ゞ」
\ruby{汝}{きさま}は
\ruby{水野}{みづ|の}を
\原本頁{203-11}\改行%
\ruby{愛}{あい}して、
%
\ruby{贔負}{ひゐ|き}に
\ruby{仕}{し}
\ruby{{\換字{過}}}{す}ぎて
\ruby[g]{間無}{まちが}つた
\ruby{事}{こと}を
させて
\ruby{居}{ゐ}るのだ。
%
いや
\ruby[<j||]{頭}{かしら}
\原本頁{204-1}\改行%
を
\ruby{振}{ふ}つても
\ruby{左樣}{さ|う}で
\ruby{無}{な}いとは
\ruby{言}{い}はさん、
%
\ruby{見晴}{み|はら}しでの
\ruby{汝}{きさま}の
\ruby{言葉}{こと|ば}
とい
\原本頁{204-2}\改行%
ひ、
%
\ruby{羽{\換字{勝}}}{は|がち}から
\ruby{聞}{き}いた
\ruby{事實}{じ|ゞつ}% TODO 原本の「二の字点、揺すり点」に濁点のグリフが見つからないので「ゞ」
といひ、
%
\ruby{先刻}{さつ|き}からの
\ruby{汝}{きさま}の
\ruby{話}{はな}し
\ruby{工合}{ぐ|あひ}
とい
\原本頁{204-3}\改行%
ひ、
%
\ruby{汝}{きさま}は
\ruby{水野}{みづ|の}の
\ruby{爭友}{さう|いう}と
なつて、
%
\ruby{彼}{あ}の
\ruby{男}{をとこ}に
\ruby{{\換字{過}}失}{くわ|しつ}
\ruby{無}{な}からしめて
やら
\原本頁{2}\改行%
うといふ
\ruby{考}{かんがへ}は
\ruby{有}{も}たんで、
%
\ruby{却}{かへ}つて
\ruby{庇護}{か|ば}ひ
\ruby{立}{だて}をする
\ruby{氣味}{き|み}がある。
%
\ruby{其樣}{そ|ん}な
\ruby{下}{くだ}らんことが
\ruby{何處}{ど|こ}に
あるものか。
』

\原本頁{204-6}%
『
オイ、
%
\ruby{大}{おほ }% 「上」の前後に突出部分の調整に全角スペースを挿入
\ruby{上}{じやう}
\ruby{段}{ だん}に
\ruby{振}{ふ}り
\ruby{被}{かぶ}つて
\ruby{睨}{にら}み
\ruby{{\換字{廻}}}{まは}すなあ
\ruby{其邊}{そこ|いら}で
\ruby{措}{お}いて
\ruby{吳}{く}れ。
%
\原本頁{204-7}\改行%
\ruby{下}{くだ}らなくつても
\ruby{乃公}{お|れ}は
\ruby{搆}{かま}はねえ。
%
\ruby{汝}{きさま}の
\ruby{云}{い}ふ
\ruby{事}{こと}
\ruby[||j>]{位}{ぐらゐ}は
\ruby{乃公}{お|れ}だつて
\ruby{知}{し}
\原本頁{204-8}\改行%
つてゐるが、
%
\ruby{諫}{いさ}めたつて
\ruby{爭}{あらそ}つたつて
\ruby{役}{やく}に
\ruby{立}{た}たねえ
\ruby{事}{こと}だから、
%
\ruby{乃公}{お|ら}あ
\ruby{意見}{い|けん}も
\ruby{云}{い}はずに
\ruby{打棄}{うつ|ちや}つて
\ruby{置}{お}くんだ。
%
\ruby{{\換字{迷}}}{まよ}ふな〳〵
\ruby{思}{おも}ひ
\ruby{切}{き}れつ
\原本頁{204-10}\改行%
て
\ruby{云}{い}つたつて、
%
\ruby{料簡}{れう|けん}
\ruby{方}{かた}が
\ruby{{\換字{煙}}管}{きせ|る}の
\ruby{羅宇}{ら|う}の
やうに
すげかへが
\ruby{出來}{で|き}る
\原本頁{204-11}\改行%
もの
ぢやあ
\ruby{無}{な}し、
%
\ruby{川柳}{せん|りう}が
\ruby{巧}{うめ}え
\ruby{事}{こと}を
\ruby{云}{い}つて
\ruby{居}{ゐ}らあナ、
%
「
\ruby{極}{ごく}
\ruby{無理}{む|り}な
\ruby{意見}{い|けん}
\ruby{魂魄}{たま|しひ}
\ruby{入}{い}れ
\ruby{換}{かへ}ろ
」つて。
%
よく
\ruby{有}{あ}る
\ruby{奴}{やつ}だが、
%
いくら
\ruby{魂魄}{たま|しひ}を
\ruby{入}{い}れ
\ruby{換}{かへ}
\原本頁{205-2}\改行%
ろつて
\ruby{云}{い}つたつて
\ruby{出來}{で|き}る
\ruby{相談}{さう|だん}
じやあ
\ruby{無}{ね}え。
%
しかし
\ruby{水野}{みづ|の}に
\ruby{意見}{い|けん}を
\原本頁{205-3}\改行%
するなあ
\ruby{汝}{きさま}の
\ruby{{\換字{勝}}手}{かつ|て}だ。
%
\ruby{止}{よ}せと
\ruby{云}{い}つたなあ
\ruby{大}{おほき}に
\ruby{御世話}{お|せ|わ}だつた。
%
\ruby{芝}{しば}
\原本頁{205-4}\改行%
で
\ruby{會}{あ}つた
\ruby{時}{とき}
\ruby{云}{い}つた
\ruby{{\換字{通}}}{とほ}りだ。
%
\ruby{乃公}{お|れ}は
\ruby{乃公}{お|れ}だから
\ruby{乃公}{お|れ}は
\ruby{行}{い}かねえ。
%
\原本頁{205-5}\改行%
\ruby{汝}{きさま}は
\ruby{汝}{きさま}だから
\ruby{行}{い}くなら
\ruby{行}{い}くが
い〻。% 原本通り「〻(二の字点、揺すり点)」
』

\原本頁{205-6}%
『
よしツ、
%
\ruby{汝}{きさま}が
\ruby{行}{い}かんでも
\ruby{乃公}{お|れ}は
\ruby{行}{い}かなくつて!。
%
\ruby{是}{これ}から
\ruby{直}{すぐ}に
\原本頁{205-7}\改行%
\ruby{行}{い}つて
\ruby{諫}{いさ}めて
\ruby{{\換字{遣}}}{や}る。
%
\ruby{熱誠}{ねつ|せい}を
\ruby{以}{もつ}て
\ruby{大}{おほ}に
\ruby{爭}{あらそ}つて
\ruby{{\換字{遣}}}{や}る。
%
\ruby{憫}{かは}% 「憫然 か(は)いさう」
\ruby[||j>]{然}{いさう}に、
%
\ruby{可惜}{あつ|たら}
\原本頁{205-8}\改行%
\ruby{好{\換字{漢}}}{かう|かん}の
\ruby{水野}{みづ|の}を
\ruby{區々}{く|〻}たる% 原本通り「〻(二の字点、揺すり点)」
\ruby{戀愛}{れん|あい}に
\ruby{悶死}{もん|し}させて
\ruby{堪}{たま}るもんか。
%
\ruby{日方}{ひ|かた}は
\ruby{彼}{かれ}
\原本頁{205-9}\改行%
のために
\ruby{爭友}{さう|いう}を
\ruby{以}{もつ}て
\ruby{任}{にん}じて
\ruby{{\換字{遣}}}{や}る。
%
\ruby{智慧}{ち|ゑ}の
\ruby{足}{た}らん
\ruby{男}{をとこ}がする
\ruby{事}{こと}の
\ruby{結果}{けつ|か}を
\ruby{見}{み}ろ。
』

\原本頁{205-11}%
『
ハヽヽ、
%
\ruby{乃公}{お|れ}の
\ruby{云}{いつ}た
\ruby{事}{こと}が
\ruby{氣}{き}に
\ruby{入}{い}らなかつた
からつて
\ruby{激}{げき}しちや
\原本頁{206-1}\改行%
あ
いけねえ。
%
\ruby{出}{で}かけるなあ
\ruby{可}{よ}いが
\ruby{其}{その}
\ruby{猛勢}{いき|ほい}で
\ruby{行}{い}つて、
%
\ruby{水野}{みづ|の}と
\ruby{喧嘩}{けん|くわ}を
しちやあ
\ruby{汝}{きさま}
いけねえぜ。
%
\ruby{彼}{あ}の
\ruby{男}{をとこ}も
おとなしいけれど
\ruby{蟲}{むし}
\ruby{持}{もち}だから。
』

\原本頁{206-4}%
『
ハヽヽ、
%
しかし
\ruby{乃公}{お|れ}の
\ruby{言}{い}ふ
\ruby{事}{こと}を
\ruby{聽}{き}かなかつたら
\ruby{攫}{つか}み
\ruby{挫}{ひし}ぐかも
\ruby{知}{し}れんぞ。
』

\原本頁{206-6}%
『
\ruby{戱談}{じよう|だん}ぢやあ
\ruby{無}{ね}えぜ、
%
\ruby{人}{ひと}が
\ruby{眞面目}{ま|じ|め}で
\ruby{云}{い}つて
\ruby{居}{ゐ}るのに。
』

\原本頁{206-7}%
『
\ruby{大{\換字{丈}}夫}{だい|ぢやう|ぶ}だ、
%
\ruby{日方}{ひ|かた}は
\ruby{粗暴}{そ|ばう}でも
まさか
\ruby{喧嘩}{けん|くわ}は
せん。
』

\原本頁{206-8}%
『
い〻かい% 原本通り「〻(二の字点、揺すり点)」
\ruby{大將}{たい|しやう}、
%
\ruby{屹度}{きつ|と}だぜ、
%
\ruby{釘}{くぎ}を
さしたぜ。
』

\原本頁{206-9}%
『
ウン、
%
よしツ。
%
\ruby{時}{とき}に
\ruby{島木}{しま|き}、
』

\原本頁{206-10}%
『
\ruby{何}{なん}だ。
』

\原本頁{206-11}%
『
\ruby{汝}{きさま}が
\ruby{{\換字{平}}生}{いつ|も}
\ruby{飮}{の}んで
\ruby{居}{ゐ}る
\ruby{此}{こ}の
\ruby{葡萄酒}{ぶ|だう|しゆ}は
\ruby{中々}{なか|〳〵}
\ruby{佳}{い}いナ。
』

\原本頁{207-1}%
『
それほど
ぢやあ
\ruby{無}{な}いが
マア
\ruby{飮}{の}めるよ。
』

\原本頁{207-2}%
『
\ruby{手土產}{て|みや|げ}に
\ruby{仕}{し}て
\ruby{持}{も}つて
\ruby{行}{い}つて、
%
\ruby{久}{ひさ}しぶりで
\ruby{水野}{みづ|の}と
\ruby{談}{はな}し
ながら
\ruby{飮}{の}むのだ。
%
\ruby{些細}{さ|さい}な
\ruby{御用}{ご|よう}だ、
%
\ruby{二本}{に|ほん}
ばかり
\ruby{徴發}{ちよう|はつ}
するぞ。
』

\原本頁{207-4}%
『
ハヽヽ、
%
\ruby{他}{ひと}の
\ruby{物}{もの}を
\ruby{徴發}{ちよう|はつ}して
\ruby{土產}{みや|げ}に
するたあ
\ruby{此奴}{こい|つ}あ
\ruby{蟲}{むし}が
い〻。% 原本通り「〻(二の字点、揺すり点)」
%
\ruby{可}{い}い〳〵。
%
\ruby{持}{も}つて
\ruby{行}{い}け、
%
\ruby{今}{いま}
\ruby{縛}{く〻}らせやう。% 本来は一の字点「ゝ」平仮名繰返し記号% 原本通り「〻(二の字点、揺すり点)」
』
