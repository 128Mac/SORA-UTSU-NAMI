\Entry{其三十六}

% メモ 校正終了 2024-04-29 2024-06-04
\原本頁{207-7}%
\ruby{牽牛花}{あ|さ|がほ}の
\ruby{花}{はな}の
\ruby{色}{いろ}は
\ruby{去年}{こ|ぞ}と
\ruby{今年}{こ|とし}と
\ruby{同}{おな}じく
\ruby{{\換字{咲}}}{さ}かず、
%
\ruby{人}{ひと}の
\ruby{心}{こ〻ろ}の% ルビ調整(原本通り)「〻(二の字点、揺すり点)」
\ruby{傾}{かたむ}きは
\ruby{昨日}{きの|ふ}に
\ruby{今日}{け|ふ}の
\ruby{變}{かは}るが
\ruby{常}{つね}ながら、
%
\ruby{水野}{みづ|の}は
\ruby{{\換字{過}}}{す}ぎし
\ruby{日}{ひ}の
\ruby{日曜}{にち|{\換字{𛀁}}う}
より、
%
\ruby{如何}{い|か}にか
しけん
\ruby{今}{いま}までの
\ruby{水野}{みづ|の}には
あらず
なりて、
%
たゞ% TODO 原本の「二の字点、揺すり点」に濁点のグリフが見つからないので「ゞ」
\ruby{世}{よ}に
ありふれたる
\ruby{爺}{ぢ〻}% 「ぢゞ」のはずだが、原本通り「〻(二の字点、揺すり点)」
\ruby{婆}{ば〻}% 「ばゞ」のはずだが、原本通り「〻(二の字点、揺すり点)」
の
\ruby{無智}{む|ち}
\ruby{無學}{む|がく}なる
もの〻% ルビ調整(原本通り)「〻(二の字点、揺すり点)」
\ruby{如}{ごと}くなりつ、
%
ひたすらに
\ruby{御佛}{み|ほとけ}を
\ruby{頼}{たの}み
\ruby[<j>]{奉}{たてまつ}り、
%
\ruby{日}{ひ}に〳〵
\ruby{我}{わ}が
\ruby{{\換字{勤}}務}{つ|とめ}を
\ruby{{\換字{終}}}{をは}るや
\ruby{否}{いな}や、
%
\ruby{直}{たゞち}に% TODO 原本の「二の字点、揺すり点」に濁点のグリフが見つからないので「ゞ」
\ruby{淺草}{あさ|くさ}に
\ruby{走}{はし}り
\ruby{行}{ゆ}きて、
%
\ruby{本{\換字{尊}}}{ほん|ぞん}の
\ruby{御{\換字{前}}}{おん|まへ}に
\ruby{祈念}{き|ねん}を
\ruby{凝}{こ}らし、
%
いつはり
\ruby{無}{な}き
\ruby{心}{こ〻ろ}の% ルビ調整(原本通り)「〻(二の字点、揺すり点)」
\ruby{誠}{まこと}を
\ruby{獻}{さ〻}げ% ルビ調整(原本通り)「〻(二の字点、揺すり点)」
つくして、
%
さて% 本来は一の字点「〻」平仮名繰返し記号% ルビ調整(原本通り)「〻(二の字点、揺すり点)」
\ruby{後}{のち}
やうやく
\ruby{寓}{やど}に
\ruby{歸}{かへ}るを
\ruby{常{\換字{習}}}{なら|ひ}
とするに
\ruby{至}{いた}りたり
\改行% 校正作業の簡略化のため
。
%
\原本頁{208-4}\改行%
\ruby{今日}{け|ふ}は
\ruby{日曜}{にち|{\換字{𛀁}}う}に
\ruby{當}{あた}りて
\ruby{身}{み}に
\ruby{閑暇}{いと|ま}
あれば、
%
お
\ruby{濱}{はま}の
\ruby{何時}{いつ|も}も
ながらに
\ruby[<j||]{訝}{いぶか}り
\ruby{怪}{あやし}みて
\ruby{其}{そ}の
\ruby{美}{うつく}しき
\ruby{眉}{まゆ}を
\ruby{顰}{ひそ}むるをば
\ruby{背後}{うし|ろ}に
\ruby{見}{み}
\ruby{棄}{す}てつ、
%
\ruby{水野}{みづ|の}は
\ruby{正午}{ひ|る}
\ruby{{\換字{過}}}{す}ぐる
\ruby{頃}{ころ}に
\ruby{家}{いへ}を
\ruby{立}{たち}
\ruby{出}{い}でたり。

\原本頁{208-7}%
\ruby{吉右衛門}{きち||ゑ|もん}は
\ruby{本家}{ほん|け}に
\ruby{相談事}{さう|だん|ごと}
ありとて
\ruby{招}{まね}かれて
\ruby{去}{さ}り、
%
お
\ruby{濱}{はま}
\ruby{一人}{ひと|り}
\ruby{餘念}{よ|ねん}
\ruby{無}{な}く
\ruby{新刊}{しん|かん}の
\ruby{雜誌}{ざつ|し}を
\ruby{讀}{よ}み
ながら、
%
お
\ruby{鍋}{なべ}を
\ruby{相手}{あひ|て}に
\ruby{{\換字{留}}守}{る|す}し
\ruby{居}{を}る
ところ
\原本頁{208-9}\改行%
へ、

\原本頁{208-10}%
『
\ruby{山路}{やま|ぢ}。
%
ウン
\ruby{此家}{こ|〻}だナ。% ルビ調整(原本通り)「〻(二の字点、揺すり点)」
』

\原本頁{208-11}%
と
\ruby{名札}{な|ふだ}を
\ruby{讀}{よ}んで
\ruby{獨語}{ひとり|ご}つ
やがてに、
%
\ruby{胴{\換字{魔}}聲}{どう|ま|ごゑ}の
\ruby{人}{ひと}を
\ruby{驚}{おどろ}かす
ほど
\ruby{恐}{おそ}ろしく
\ruby{大}{おほき}く、

\原本頁{209-2}%
『
\ruby{頼}{たの}む。
』

\原本頁{209-3}%
と
\ruby{一}{ひ}ト
\ruby{聲}{こゑ}
\ruby{呼}{よ}ばはれる
ものあり。

\原本頁{209-4}%
『
\ruby{誰}{だれ}か
\ruby{呼}{よ}ばはつた
でがす。
』

\原本頁{209-5}%
『
さうだネ、
%
お
\ruby{{\換字{前}}}{まへ}
\ruby{出}{で}て
\ruby{御覽}{ご|らん}ナ。
』

\原本頁{209-6}%
お
\ruby{濱}{はま}は
\ruby{{\換字{猶}}}{なほ}
\ruby{雜誌}{ざつ|し}をば
\ruby{讀}{よ}み
つゞけ% TODO 原本の「二の字点、揺すり点」に濁点のグリフが見つからないので「ゞ」
\ruby{居}{ゐ}しが、
%
\ruby{應對}{おう|たい}の
\ruby{模樣}{も|やう}は
\ruby{明}{あき}らかに
\ruby{聞}{きこ}
\改行% 校正作業の簡略化のため
ゆ。

\原本頁{209-8}%
『
\ruby{水野}{みづ|の}は
\ruby{居}{を}るか。
』

\原本頁{209-9}%
『
\ruby{今}{いま}ア
\ruby{居}{ゐ}ねえでがす。
』

\原本頁{209-10}%
『
\ruby{何處}{ど|こ}へ
\ruby{行}{い}つた。
』

\原本頁{209-11}%
『
\ruby{知}{し}りましねえ。
』

\原本頁{210-1}%
『
しかし
\ruby{出}{で}た
ものなら
いづれ
\ruby{歸}{かへ}る
だらう。
』

\原本頁{210-2}%
『
どうでがすかサ。
』

\原本頁{210-3}%
『
\ruby{{\換字{遠}}方}{ゑん|ぱう}
わざ〳〵
\ruby{來}{き}たもの
だから
\ruby{上}{あが}つて
\ruby{待}{ま}つて
\ruby{居}{ゐ}やう。
』

\原本頁{210-4}%
『
いかねえでがす。
%
\ruby{待}{また}つせえ
お
\ruby{{\換字{前}}樣}{め{\換字{𛀁}}|さま}。
』

\原本頁{210-5}%
お
\ruby{鍋}{なべ}は
\ruby{慌}{あわ}て〻% ルビ調整(原本通り)「〻(二の字点、揺すり点)」
\ruby{入}{い}り
\ruby{來}{きた}りて、

\原本頁{210-6}%
『
いやに
\ruby{身體}{から|だ}の
\ruby{魁偉}{い|か}い
\ruby{{\換字{尊}}大}{おほ|ふう}の
\ruby{野郎}{や|らう}でがす。
%
\ruby{水野}{みづ|の}さんの
\ruby{事}{こと}
\ruby{聞}{き}くから
\ruby{不在}{る|す}
だつて
\ruby{云}{い}つたら、
%
\ruby{上}{あが}つて
\ruby{待}{ま}たうと
\ruby{吐}{ぬか}します。
%
どうして
\ruby{吳}{く}れます
べい。
%
イヤな
\ruby{奴}{やつ}でがす。
』

\原本頁{210-9}%
と
\ruby{云}{い}へば、
%
お
\ruby{濱}{はま}は、
%
\ruby{辛}{から}く
\ruby{雜誌}{ざつ|し}より
\ruby{目}{め}を
\ruby{離}{はな}して
\ruby{笑}{わら}ひ
\ruby{出}{いだ}し、

\原本頁{210-10}%
『
\ruby{{\換字{分}}}{わか}らないねえ
お
\ruby{{\換字{前}}}{まへ}は、
%
\ruby{言葉}{こと|ば}の
\ruby{樣子}{やう|す}ぢやあ
\ruby{水野}{みづ|の}さんと
\ruby{仲}{なか}の
\ruby{好}{い}い
\ruby{御朋友}{お|とも|だち}
らしいぢや
\ruby{無}{な}いか。
%
どれ
\ruby{妾}{わたし}が
\ruby{行}{い}つて
\ruby{見}{み}やう。
』

\原本頁{211-1}%
と
\ruby{立出}{たち|い}でたり。

\原本頁{211-2}%
\ruby{見}{み}れば
\ruby{客}{きやく}は
\ruby{血氣}{けつ|き}
\ruby{壯盛}{さか|ん}の
\ruby{陸軍}{りく|ぐん}
\ruby{士官}{し|くわん}にして、
%
\ruby{頭顱}{かし|ら}
\ruby[||j>]{大}{おほき}く
\ruby{肩}{かた}
\ruby{厚}{あつ}き
さまは
\原本頁{211-3}\改行%
\ruby{素人}{しろう|と}
づくねの
\ruby[||j>]{土}{つち}
\ruby[||j>]{人}{にん}
\ruby[||j>]{形}{ぎやう}
などの
\ruby{如}{ごと}く、
%
\ruby{無骨}{ぶ|こつ}
\ruby{一{\換字{遍}}}{いつ|ぺん}の
\ruby[||j>]{正}{しやう}
\ruby[||j>]{直}{ ぢき}さうな
% \ruby{正直}{しやう|ぢき}さうな
\ruby{人}{ひと}なり
\改行% 校正作業の簡略化のため
。

\原本頁{211-4}%
『
\ruby{水野}{みづ|の}さんは
\ruby{今}{いま}
\ruby{御不在}{お|る|す}ですが
\ruby{誰樣}{どな|た}で
いらつしやいます?。
』

\原本頁{211-5}%
\ruby{言葉}{こと|ば}
\ruby{無}{な}く
\ruby{名刺}{めい|し}を
\ruby{出}{いだ}して
\ruby{客}{きやく}の
\ruby{渡}{わた}すを、
%
お
\ruby{濱}{はま}は
\ruby{手}{て}に
\ruby{取}{と}りて
\ruby{讀}{よ}みて
\ruby{急}{きふ}に
\ruby{笑顏}{ゑ|がほ}に
なりぬ。
%
\ruby{未}{いま}だ
\ruby{面}{おもて}を
こそ
\ruby{對}{あは}せざり
つれ、
%
\ruby{水野}{みづ|の}の
\ruby{友}{とも}に
\ruby{其}{その}
\ruby{人}{ひと}ある
よしの
\ruby{日方}{ひ|かた}
\ruby{八郎}{はち|らう}
といふ
\ruby{名}{な}は、
%
かねて
\ruby{聞}{き}き
\ruby{馴}{な}れて
\ruby{何時}{い|つ}と
\ruby{無}{な}く
\ruby{疏}{うと}からず
\ruby{覺}{おぼ}え
\ruby{居}{ゐ}たればなり。

\原本頁{211-8}%
『
たしか
\ruby{島木}{しま|き}さん
や
なんぞと
\ruby{御一緖}{ご|いつ|しよ}の、
%
\ruby{御同國}{ご|どう|こく}の
\ruby{方}{かた}で
いらつしやい
ましたね。
』

\原本頁{211-11}%
\ruby{一應}{いち|おう}
\ruby{念}{ねん}を
\ruby{推}{お}す
お
\ruby{濱}{はま}をば、
%
\ruby{日方}{ひ|かた}は
\ruby{眼}{め}を
\ruby{正}{たゞ}しくして% TODO 原本の「二の字点、揺すり点」に濁点のグリフが見つからないので「ゞ」
\ruby{一寸}{ちよ|つと}
\ruby{見}{み}しが、
%
\ruby{何}{なに}
\ruby{訝}{いぶ}かるべくも
\ruby{無}{な}き
\ruby{處女}{き|むすめ}の、
%
たゞ% TODO 原本の「二の字点、揺すり点」に濁点のグリフが見つからないので「ゞ」
\ruby{怜悧}{り|こう}なるべく
\ruby{見}{み}ゆるのみの
\ruby{淸}{きよ}らなる
\ruby{娘}{むすめ}なれば、

\原本頁{212-3}%
『
\ruby{其}{その}
\ruby{{\換字{通}}}{とほ}り。
』

\原本頁{212-4}%
と
\ruby{甚}{いと}
\ruby{明}{あき}らかに
\ruby{答}{こた}へたり。

\原本頁{212-5}%
『
\ruby{水野}{みづ|の}さんは
\ruby{淺草}{あさ|くさ}まで
\ruby{御}{お}いでに
なつたのですから、
%
\ruby{御{\換字{退}}屈}{ご|たい|くつ}でも
\ruby{御待}{お|ま}ちなさる
ならば、
%
\ruby{此方}{こち|ら}へ% ルビ調整(原本通り)
\ruby{御{\換字{通}}}{お|とほ}り
なすつて。
』

\原本頁{212-7}%
\ruby{何時}{い|つ}か
お
\ruby{濱}{はま}の
\ruby{背後}{うし|ろ}に
\ruby{出}{い}で
\ruby{來}{きた}り
\ruby{居}{ゐ}し
お
\ruby{鍋}{なべ}は
そつと
\ruby{袖}{そで}を
\ruby{引}{ひ}きて

\原本頁{212-8}%
『
\ruby{宜}{い}いでがすかエ
\ruby{其樣}{そ|ん}な
\ruby{事}{こと}を
\ruby{仕}{し}て、
%
\ruby{何}{なん}だか
\ruby{蟲}{むし}の
\ruby{好}{す}かねえ
\ruby{厭}{いや}な
\ruby{奴}{やつ}でがすよ。
』

\原本頁{212-9}%
と
\ruby{心配}{しん|ぱい}し
\ruby{{\換字{過}}}{すご}して
\ruby{小聲}{こ|ゞゑ}に% TODO 原本の「二の字点、揺すり点」に濁点のグリフが見つからないので「ゞ」
\ruby{止}{とゞ}むるを、% TODO 原本の「二の字点、揺すり点」に濁点のグリフが見つからないので「ゞ」
%
お
\ruby{濱}{はま}は
\ruby{顧}{かへり}みず
\ruby{日方}{ひ|かた}を
\ruby{案内}{あん|ない}して
\改行% 校正作業の簡略化のため
、
%
\原本頁{212-11}\改行%
\ruby{水野}{みづ|の}の
\ruby{室}{へや}に
\ruby{{\換字{通}}}{とほ}したり。

\原本頁{213-1}%
\ruby{日方}{ひ|かた}は
\ruby{水野}{みづ|の}が
\ruby{机}{つくゑ}の
\ruby{横}{よこ}に
どつかりと
\ruby{座}{すわ}りて、

\原本頁{213-2}%
『
ハヽア
\ruby{何}{なに}も
\ruby[||j>]{裝}{さう}
\ruby[||j>]{{\換字{飾}}}{しよく}は
% \ruby{裝{\換字{飾}}}{さう|しよく}は
\ruby{無}{な}いが
\ruby{惡}{わる}くない
\ruby{部屋}{へ|や}だナ。
%
\ruby{相}{あひ}
\ruby{變}{かは}らず
\ruby{有}{あ}るものは
\ruby{書籍}{ほ|ん}ばかりで、
%
\ruby[||j>]{長}{ちやう}
\ruby[||j>]{物}{ ぶつ}の
\ruby{無}{な}いところは
\ruby{流石}{さす|が}に
\ruby{{\換字{感}}心}{かん|しん}だ。
』% 原本では植字ミスと思われるので「』」を追加

\原本頁{213-4}%
と
\ruby{先}{ま}づ
\ruby{{\換字{評}}}{ひやう}する
\ruby{時}{とき}、
%
お
\ruby{濱}{はま}は
お
\ruby{鍋}{なべ}が
\ruby{汲}{く}み
\ruby{來}{きた}りし
\ruby{茶}{ちや}を
\ruby{薦}{す〻}むれば、% ルビ調整(原本通り)「〻(二の字点、揺すり点)」

\原本頁{213-5}%
『
\ruby{君}{きみ}は
\ruby{此家}{こ|〻}の% ルビ調整(原本通り)「〻(二の字点、揺すり点)」
\ruby{娘}{むすめ}さんかナ。
%
どうだ
\ruby{水野}{みづ|の}は。
%
\ruby{此頃}{この|ごろ}も
\ruby{相}{あひ}
\ruby{變}{かは}らず
\ruby[<j||]{勉}{べん }% 行末行頭の境界付近なので特例処置を施す
\ruby[<j||]{{\換字{強}}}{きやう}
% \ruby{勉{\換字{強}}}{べん|きやう}
\原本頁{213-6}\改行%
か。
』

\原本頁{213-7}%
と
\ruby{話}{はな}し
\ruby{仕度}{し|た}さに
\ruby{打}{うち}
\ruby{解}{と}けて
\ruby{問}{と}ふを、
%
\ruby{水野}{みづ|の}〳〵と
\ruby{呼}{よ}びつけに
するが
\ruby{小面}{こ|づら}
\ruby{憎}{にく}くてか、

\原本頁{213-9}%
『
ハイ。
』

\原本頁{213-10}%
と
\ruby{僅々}{わづ|か}
\ruby{一句}{いつ|く}に
\ruby{答}{こたへ}を
\ruby{切}{き}りて、

\原本頁{213-11}%
『
\ruby{御自由}{ご|じ|ゆう}に
おいで
なすつて。
』

\原本頁{214-1}%
と
\ruby{言}{い}ひ
\ruby{棄}{す}てし
ま〻、% ルビ調整(原本通り)「〻(二の字点、揺すり点)」
%
\ruby{突}{つ}と
\ruby{次}{つぎ}の
\ruby{間}{ま}に
\ruby{出}{い}で〻% ルビ調整(原本通り)「〻(二の字点、揺すり点)」
\ruby{唐紙}{から|かみ}
ぴつしやり、
%
お
\ruby{鍋}{なべ}の
\ruby{後}{あと}を
\ruby{{\換字{追}}}{お}ふて
\ruby{茶}{ちや}の
\ruby{室}{ま}に
\ruby{{\換字{退}}}{しりぞ}けば、
%
お
\ruby{鍋}{なべ}は、
%
\ruby{手}{て}の
\ruby{甲}{かふ}を
\ruby{口}{くち}に
あて〻% ルビ調整(原本通り)「〻(二の字点、揺すり点)」
\ruby{笑}{わら}ひ
ながら
\改行% 校正作業の簡略化のため
、

\原本頁{214-4}%
『
\ruby{女}{をんな}を
\ruby{呼}{よ}ばるのに
\ruby{君}{きみ}だ
なんて、
%
ホヽヽハヽヽ。
』

\原本頁{214-5}%
と、
%
げらつきて
\ruby{已}{や}まず。
%
お
\ruby{濱}{はま}も
\ruby{睨}{にら}む
\ruby{眞似}{ま|ね}して
\ruby{叱}{しか}りは
\ruby{叱}{しか}り
ながら
\改行% 校正作業の簡略化のため
、
%
\原本頁{214-6}\改行%
おのれも
\ruby{口}{くち}の
あたりに
\ruby{笑}{わらひ}を
\ruby{{\換字{浮}}}{う}かめぬ。

\原本頁{214-7}%
\ruby[||j>]{話}{はなし}
\ruby[||j>]{敵}{がたき}
\ruby[||j>]{無}{ な }き
\ruby{{\換字{所}}在}{しよ|ざい}
\ruby{無}{な}さの
\ruby{餘}{あま}り、
%
\ruby{日方}{ひ|かた}は
\ruby{其邊}{そこ|ら}を
\ruby{見{\換字{廻}}}{み|まは}しつ、
%
\ruby{机}{つくゑ}の
\ruby{上}{うへ}に
\ruby{在}{あ}りし
\ruby{折本}{をり|ほん}に
\ruby{偶然}{ふ|と}
\ruby{目}{め}を
\ruby{着}{つ}けて、
%
\ruby{手}{て}に
\ruby{取}{と}りて
\ruby{何}{なに}
\ruby[||j>]{心}{ご〻ろ}なく% ルビ調整(原本通り)「〻(二の字点、揺すり点)」
\ruby{披}{ひら}き
\ruby{見}{み}しが、
%
\ruby{忽}{たちま}ち
\ruby{其{\換字{所}}}{そ|こ}に
\ruby{抛}{はふ}り
\ruby{出}{いだ}し。

\原本頁{214-10}%
『
\ruby{何}{なん}だ、
%
\ruby{普門品}{ふ|もん|ぼん}!。
%
\ruby{何}{なん}だ
\ruby{是}{これ}あ
\ruby{何}{なん}だ!。
%
\ruby{御有{\換字{難}}{\換字{連}}}{お|あり|がた|れん}の
\ruby{誦}{よ}む
ものでは
ないか。
%
まさか
\ruby{水野}{みづ|の}が
\ruby{信心}{しん|〴〵}する
のでは
あるまいが、
%
\ruby{如是}{こ|ん}な
ものが
\ruby{机}{つくゑ}に
\ruby{載}{の}つて
\ruby{居}{ゐ}るのは
\ruby{何樣}{ど|う}した
\ruby{馬鹿}{ば|か}な
\ruby{事}{こつ}た。
』

\原本頁{215-2}%
と
\ruby{其處}{そ|こ}に
\ruby{罵}{の〻し}るべき% 本来は一の字点「ゝ」平仮名繰返し記号% ルビ調整(原本通り)「〻(二の字点、揺すり点)」
\ruby{人}{ひと}にても
あるが
\ruby{如}{ごと}くに
\ruby{罵}{の〻し}つたり。% 本来は一の字点「ゝ」平仮名繰返し記号% ルビ調整(原本通り)「〻(二の字点、揺すり点)」
