\Entry{其三十六}

\ruby{牽牛花}{あさ|が|ほ}の
\ruby{花}{はな}の
\ruby{色}{いろ}は
\ruby{去年}{こ|ぞ}と
\ruby{今年}{こ|とし}と
\ruby{同}{おな}じく
\ruby{{\換字{咲}}}{さ}かず、
\ruby{人}{ひと}の
\ruby{心}{こゝろ}の
\ruby{傾}{かたむ}きは
\ruby{昨日}{きの|ふ}に
\ruby{今日}{け|ふ}の
\ruby{變}{かは}るが
\ruby{常}{つね}ながら、
\ruby{水野}{みづ|の}は
\ruby{{\換字{過}}}{す}ぎし
\ruby{日}{ひ}の
\ruby{日曜}{にち|\換字{江}う}より、
\ruby{如何}{い|か}にかしけん
\ruby{今}{いま}までの
\ruby{水野}{みづ|の}にはあらずなりて、たゞ
\ruby{世}{よ}にありふれたる
\ruby{爺婆}{ぢ\ninojiten{}|ば\ninojiten{}}の
\ruby{無智無學}{む|ち|む|がく}なるもの〻
\ruby{如}{ごと}くなりつ、ひたすらに
\ruby{御佛}{みほ|とけ}を
\ruby{頼}{たの}み
\ruby{奉}{たてまつ}り、
\ruby{日}{ひ}に〳〵
\ruby{我}{わ}が
\ruby{勤務}{つ|とめ}を
\ruby{{\換字{終}}}{をは}るや
\ruby{否}{いな}や、
\ruby{直}{たゞち}に
\ruby{淺草}{あさ|くさ}に
\ruby{走}{はし}り
\ruby{行}{ゆ}きて、
\ruby{本尊}{ほん|ぞん}の
\ruby{御前}{おん|まへ}に
\ruby{祈念}{き|ねん}を
\ruby{凝}{こ}らし、いつはり
\ruby{無}{な}き
\ruby{心}{こゝろ}の
\ruby{誠}{まこと}を
\ruby{獻}{さ\ninojiten{}}げつくして、さて
\ruby{後}{のち}やうやく
\ruby{寓}{やど}に
\ruby{歸}{かへ}るを
\ruby{常{\換字{習}}}{なら|ひ}とするに
\ruby{至}{いた}りたり。

\ruby{今日}{け|ふ}は
\ruby{日曜}{にち|\換字{江}う}に
\ruby{當}{あた}りて
\ruby{身}{み}に
\ruby{閑暇}{いと|ま}あれば、
お
\ruby{濱}{はま}の
\ruby{何時}{いつ|も}もながらに
\ruby{訝}{いぶか}り
\ruby{怪}{あやし}みて
\ruby{其}{そ}の
\ruby{美}{うつく}しき
\ruby{眉}{まゆ}を
\ruby{顰}{ひそ}むるをば
\ruby{背後}{うし|ろ}に
\ruby{見棄}{み|す}てつ、
\ruby{水野}{みづ|の}は
\ruby{正午{\換字{過}}}{ひ|る|す}ぐる
\ruby{頃}{ころ}に
\ruby{家}{いへ}を
\ruby{立出}{たち|い}でたり。

\ruby{吉右衛門}{き|ち|ゑ|もん}は
\ruby{本家}{ほん|け}に
\ruby{相談事}{さう|だん|ごと}ありとて
\ruby{招}{まね}かれて
\ruby{去}{さ}り、
お
\ruby{濱}{はま}
\ruby{一人}{ひと|り}
\ruby{餘令}{よ|ねん}
\ruby{無}{な}く
\ruby{新刊}{しん|かん}の
\ruby{雜誌}{ざつ|し}を
\ruby{讀}{よ}みながら、
お
\ruby{鍋}{なべ}を
\ruby{相手}{あい|て}に
\ruby{{\換字{留}}守}{る|す}し
\ruby{居}{を}るところへ、

『
\ruby{山路}{やま|ぢ}。
ウン
\ruby{此家}{こ|\ninojiten{}}だナ。
』

と
\ruby{名札}{な|ふだ}を
\ruby{讀}{よ}んで
\ruby{獨語}{ひとり|ご}つやがてに、
\ruby{胴魔聲}{どう|ま|ごゑ}の
\ruby{人}{ひと}を
\ruby{驚}{おどろ}かすほど
\ruby{恐}{おそ}ろしく
\ruby{大}{おほき}く、

『
\ruby{頼}{たの}む。
』

と
\ruby{一}{ひ}ト
\ruby{聲呼}{こゑ|よ}ばはれるものあり。

『
\ruby{誰}{たれ}か
\ruby{呼}{よ}ばはつたでがす。
』

『さうだネ、お
\ruby{前出}{まへ|で}て
\ruby{御覽}{ご|らん}ナ。
』

お
\ruby{濱}{はま}は
\ruby{猶雜誌}{なほ|ざつ|し}をば
\ruby{讀}{よ}みつゞけ
\ruby{居}{ゐ}しが、
\ruby{應對}{おう|たい}の
\ruby{模樣}{も|やう}は
\ruby{明}{あき}らかに
\ruby{聞}{きこ}ゆ。

『
\ruby{水野}{みづ|の}は
\ruby{居}{を}るか。
』

「
\ruby{今}{いま}ア
\ruby{居}{ゐ}ねえでがす。
』

『
\ruby{何處}{ど|こ}へ
\ruby{行}{い}つた。
』

『
\ruby{知}{し}りましねえ。
』

『しかし
\ruby{出}{で}たものならいづれ
\ruby{歸}{かへ}るだらう。
』

『どうでがすかサ。
』

『
\ruby{遠方}{ゑん|ぱう}わざ〳〵
\ruby{來}{き}たものだから
\ruby{上}{あが}つて
\ruby{待}{ま}つて
\ruby{居}{ゐ}やう。
』

『いかねえでがす。
\ruby{待}{また}つせえ
お
\ruby{前樣}{め\換字{江}|さま}。
』

お
\ruby{鍋}{なべ}は
\ruby{慌}{あわ}て〻
\ruby{入}{い}り
\ruby{來}{きた}りて、

『いやに
\ruby{身體}{から|だ}の
\ruby{魁偉}{い|か}い
\ruby{尊大}{おほ|ふう}の
\ruby{野郎}{や|らう}でがす。
\ruby{水野}{みづ|の}さんの
\ruby{事聞}{こと|き}くか
ら
\ruby{不在}{る|す}だつて
\ruby{云}{い}つたら、
\ruby{上}{あが}つて
\ruby{待}{ま}たうと
\ruby{吐}{ぬか}します。
どうして
\ruby{{\換字{呉}}}{く}れますべい。
イヤな
\ruby{奴}{やつ}でがす。
』

と
\ruby{云}{い}へば、
お
\ruby{濱}{はま}は、
\ruby{辛}{から}く
\ruby{雑誌}{ざつ|し}より
\ruby{目}{め}を
\ruby{離}{はな}して
\ruby{笑}{わら}ひ
\ruby{出}{いだ}し、

『
\ruby{{\換字{分}}}{わか}らないねえ
お
\ruby{前}{まへ}は、
\ruby{言葉}{こと|ば}の
\ruby{樣子}{やう|す}ぢやあ
\ruby{水野}{みづ|の}さんと
\ruby{仲}{なか}の
\ruby{好}{い}い
\ruby{御朋友}{お|とも|だち}らしいぢや
\ruby{無}{な}いか。
どれ
\ruby{妾}{わたし}が
\ruby{行}{い}つて
\ruby{見}{み}やう。

と
\ruby{立出}{たち|いで}でたり。

\ruby{見}{み}れば
\ruby{客}{きやく}は
\ruby[g]{血氣壯盛}{けつきさかん}の
\ruby{陸軍士官}{りく|ぐん|しく|わん}にして、
\ruby{頭顱}{かし|ら}
\ruby{大}{おほき}く
\ruby{肩厚}{かた|あつ}きさまは
\ruby{素人}{しろう|と}づくねの
\ruby{土人形}{つち|にん|ぎやう}などの
\ruby{如}{ごと}く、
\ruby{無骨一遍}{ぶ|こつ|いつ|ぺん}の
\ruby{正直}{しやう|ぢき}さうな
\ruby{人}{ひと}なり。

『
\ruby{水野}{みづ|の}さんは
\ruby{今御不在}{いま|お|る|す}ですが
\ruby{誰樣}{どな|た}でいらつしやいます?。
』

\ruby{言葉}{こと|ば}
\ruby{無}{な}く
\ruby{名刺}{めい|し}を
\ruby{出}{いだ}して
\ruby{客}{きやく}の
\ruby{渡}{わた}すを、
お
\ruby{濱}{はま}は
\ruby{手}{て}に
\ruby{取}{と}りて
\ruby{讀}{よ}みて
\ruby{急}{きふ}に
\ruby{笑顏}{ゑ|がほ}になりぬ。
\ruby{未}{ま}だ
\ruby{面}{おもて}をこそ
\ruby{對}{あは}せざりつれ、
\ruby{水野}{みづ|の}の
\ruby{友}{とも}に
\ruby{其人}{その|ひと}あるよしの
\ruby{日方八郎}{ひ|かた|はち|らう}といふ
\ruby{名}{な}は、かねて
\ruby{聞}{き}き
\ruby{馴}{な}れて
\ruby{何時}{い|つ}と
\ruby{無}{な}く
\ruby{疏}{うと}からず
\ruby{覺}{おぼ}え
\ruby{居}{ゐ}たればなり。

『たしか
\ruby{島木}{しま|き}さんやなんぞと
\ruby{御一緖}{ご|いつ|しよ}の、
\ruby{御同國}{ご|どう|こく}の
\ruby{方}{かた}でいらつしやいましたね。
』

\ruby{一應念}{いち|おう|ねん}を
\ruby{推}{お}す
お
\ruby{濱}{はま}をば、
\ruby{日方}{ひ|かた}は
\ruby{眼}{め}を
\ruby{正}{たゞ}しくして
\ruby{一寸}{ちよ|つと}
\ruby{見}{み}しが、
\ruby{何訝}{なに|いぶ}かるべくも
\ruby{無}{な}き
\ruby{處女}{き|むすめ}の、たゞ
\ruby{怜悧}{り|かう}なるべく
\ruby{見}{み}ゆるのみの
\ruby{淸}{きよ}らなる
\ruby{娘}{むすめ}なれば、

『
\ruby{其通}{その|とほ}り。
』

と
\ruby{甚明}{いと|あき}らかに
\ruby{答}{こた}へたり。

『
\ruby{水野}{みづ|の}さんは
\ruby{淺草}{あさ|くさ}まで
\ruby{御}{お}いでになつたのですから、
\ruby{御{\換字{退}}屈}{ご|たい|くつ}でも
\ruby{御待}{お|ま}ちなさるならば、
\ruby{此方}{こち|ら}へ
\ruby{御{\換字{通}}}{お|とほ}りなすつて。
』

\ruby{何時}{い|つ}か
お
\ruby{濱}{はま}の
\ruby{背後}{うし|ろ}に
\ruby{出}{い}で
\ruby{來}{きた}り
\ruby{居}{ゐ}し
お
\ruby{鍋}{なべ}はそつと
\ruby{袖}{そで}を
\ruby{引}{ひ}きて

『
\ruby{宜}{い}いでがすかエ
\ruby{其樣}{そ|ん}な
\ruby{事}{こと}を
\ruby{仕}{し}て、
\ruby{何}{なん}だか
\ruby{蟲}{むし}の
\ruby{好}{す}かねえ
\ruby{厭}{いや}な
\ruby{奴}{やつ}でがすよ。
』

と
\ruby{心配}{しん|ぱい}し
\ruby{{\換字{過}}}{すご}して
\ruby{小聲}{こ|ゞゑ}に
\ruby{止}{とゞ}むるを、
お
\ruby{濱}{はま}は
\ruby{顧}{かへり}みず
\ruby{日方}{ひ|かた}を
\ruby{案内}{あん|ない}して、
\ruby{水野}{みづ|の}の
\ruby{室}{へや}に
\ruby{{\換字{通}}}{とほ}したり。

\ruby{日方}{ひ|かた}は
\ruby{水野}{みづ|の}が
\ruby{机}{つくゑ}の
\ruby{横}{よこ}にどつかりと
\ruby{座}{すわ}りて、

『ハヽア
\ruby{何}{なに}も
\ruby{裝飾}{さう|しよく}は
\ruby{無}{な}いが
\ruby{惡}{わる}くない
\ruby{部屋}{へ|や}だナ。
\ruby{相變}{あひ|かは}らず
\ruby{有}{あ}るものは
\ruby{書籍}{ほ|ん}ばかりで、
\ruby{長物}{ちやう|ぶつ}の
\ruby{無}{な}いところは
\ruby{流石}{さす|が}に
\ruby{感心}{かん|しん}だ。

と
\ruby{先}{ま}づ
\ruby{{\換字{評}}}{ひやう}する
\ruby{時}{とき}、
お
\ruby{濱}{はま}は
お
\ruby{鍋}{なべ}が
\ruby{汲}{く}み
\ruby{來}{きた}りし
\ruby{茶}{ちや}を
\ruby{鷹}{すゝ}むれば、

「
\ruby{君}{きみ}は
\ruby{此家}{こ|\ninojiten{}}の
\ruby{娘}{むすめ}さんかナ。
どうだ
\ruby{水野}{みづ|の}は。
\ruby{此頃}{この|ごろ}も
\ruby{相變}{あひ|かは}らず
\ruby{勉強}{べん|きやう}か。
』

と
\ruby{話}{はな}し
\ruby{仕度}{し|た}さに
\ruby{打解}{うち|と}けて
\ruby{問}{と}ふを、
\ruby{水野}{みづ|の}〳〵と
\ruby{呼}{よ}びつけにするが
\ruby{小面憎}{こ|づら|にく}くてか、

『ハイ。
』

と
\ruby{僅々一句}{わづ|か|いつ|く}に
\ruby{答}{こたへ}を
\ruby{切}{き}りて、

『
\ruby{御自由}{ご|じ|ゆう}においでなすつて。
』

と
\ruby{言}{い}ひ
\ruby{棄}{す}てしま〻、
\ruby{突}{つ}と
\ruby{次}{つぎ}の
\ruby{間}{ま}に
\ruby{出}{い}で〻
\ruby{唐紙}{から|かみ}ぴつしやり、
お
\ruby{鍋}{なべ}の
\ruby{後}{あと}を
\ruby{{\換字{追}}}{お}ふて
\ruby{茶}{ちや}の
\ruby{室}{ま}に
\ruby{{\換字{退}}}{しりぞ}けば、
お
\ruby{鍋}{なべ}は、
\ruby{手}{て}の
\ruby{甲}{かふ}を
\ruby{口}{くち}にあて〻
\ruby{笑}{わら}ひながら、

『
\ruby{女}{をんな}を
\ruby{呼}{よ}ばるのに
\ruby{君}{きみ}だなんて、ホヽヽハヽヽ。
』

と、げらつきて
\ruby{已}{や}まず。
お
\ruby{濱}{はま}も
\ruby{睨}{にら}む
\ruby{眞似}{ま|ね}して
\ruby{叱}{しか}りは
\ruby{叱}{しか}りながら、おのれも
\ruby{口}{くち}のあたりに
\ruby{笑}{わらひ}を
\ruby{{\換字{浮}}}{う}かめぬ。

\ruby{話敵無}{はなし|がたき|な}き
\ruby{{\換字{所}}在無}{しよ|ざい|な}さの
\ruby{餘}{あま}り、
\ruby{日方}{ひ|かた}は
\ruby{其邊}{そこ|ら}を
\ruby{見{\換字{廻}}}{み|まは}しつ、
\ruby{机}{つくゑ}の
\ruby{上}{うへ}に
\ruby{在}{あ}りし
\ruby{折本}{をり|ほん}に
\ruby{偶然目}{ふ|と|め}を
\ruby{着}{つ}けて、
\ruby{手}{て}に
\ruby{取}{と}りて
\ruby{何心}{なに|ごゝろ}なく
\ruby{披}{ひら}き
\ruby{見}{み}しが、
\ruby{忽}{たちま}ち
\ruby{其{\換字{所}}}{そ|こ}に
\ruby{抛}{はふ}り
\ruby{出}{いだ}し、

「
\ruby{何}{なん}だ、
\ruby{普門品}{ふもん|ぼん|}!。
\ruby{何}{なん}だ
\ruby{是}{これ}あ
\ruby{何}{なん}だ!。
\ruby{御有難{\換字{連}}}{お|あり|がた|れん}の
\ruby{誦}{よ}むものではないか。
まさか
\ruby{水野}{みづ|の}が
\ruby{信心}{しん|じん}するのではあるまいが、
\ruby{如是}{こ|ん}なものが
\ruby{机}{つくゑ}に
\ruby{載}{の}つて
\ruby{居}{ゐ}るのは
\ruby{何樣}{ど|う}した
\ruby{馬鹿}{ば|か}な
\ruby{事}{こつ}た。

と
\ruby{其處}{そ|こ}に
\ruby{罵}{の\ninojiten{}し}るべき
\ruby{人}{ひと}にてもあるが
\ruby{如}{ごと}くに
\ruby{罵}{の\ninojiten{}し}つたり。

