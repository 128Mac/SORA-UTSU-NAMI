\Entry{其三十七}

\ruby{待}{ま}てども〳〵
\ruby{水野}{みづ|の}は
\ruby{歸}{かへ}らぬなり、
\ruby{此家}{この|や}の
\ruby{者}{もの}は
\ruby{彼方}{かな|た}に
\ruby{{\換字{退}}}{しりぞ}きて
\ruby{音}{おと}もさせぬなり、
\ruby{日方}{ひ|かた}はほと〳〵
\ruby{身}{み}を
\ruby{持餘}{もて|あま}して、
\ruby{四圍}{あた|り}の
\ruby{書}{ほん}などを
\ruby{手}{て}あたりまかせに
\ruby{抽}{ひ}き
\ruby{出}{いだ}しては
\ruby{讀}{よ}み
\ruby{散}{ち}らし
\ruby{居}{ゐ}しが、それにも
\ruby{忽}{たちま}ち
\ruby{倦}{あ}きて
\ruby{無聊}{ぶ|りよう}に
\ruby{堪}{た}へかね、
\ruby{小齋}{せう|さい}の
\ruby{靜坐}{せい|ざ}には
\ruby{更}{さら}に
\ruby{慣}{なら}はぬ
\ruby{身}{み}の、
\ruby{何}{なに}をがな
\ruby{{\換字{消}}閑}{せう|かん}の
\ruby{具}{ぐ}にと
\ruby{見回}{み|まは}す
\ruby{折}{をり}しも、
\ruby{携}{たづさ}へ
\ruby{來}{き}し
\ruby{二罎}{ふた|びん}の
\ruby{酒}{さけ}に
\ruby{眼}{め}の
\ruby{止}{と}まれば
\ruby{先}{ま}づ
\ruby{微笑}{び|せう}を
\ruby{{\換字{浮}}}{うか}め、

『
\ruby{仕方}{し|かた}が
\ruby{無}{な}い、これでも
\ruby{飮}{の}んで
\ruby{待}{ま}つて
\ruby{居}{ゐ}て
\ruby{{\換字{遣}}}{や}らう。
』

と
\ruby{口}{くち}にこそ
\ruby{言}{い}はぬ
\ruby{心}{こゝろ}に
\ruby{思}{おも}ひて、

『オイ、
\ruby{君}{きみ}!。
オイオイ、
\ruby{君}{きみ}!。
』

と
\ruby{呼}{よ}び
\ruby{立}{た}てたり。

『ハヽヽ、また
\ruby{君}{きみ}イ
\ruby{君}{きみ}イつて
\ruby{呼}{よ}ばつて
\ruby{居}{ゐ}るでがす、
\ruby{妾}{わたし}が
\ruby{君}{きみ}イになつて
\ruby{出}{で}て
\ruby{行}{い}きますべいか。
』

『ホヽヽ、いゝよ、
\ruby{妾}{わたし}が
\ruby{行}{い}つて
\ruby{見}{み}るから。
』

お
\ruby{濱}{はま}は
\ruby{立}{た}つて
\ruby{客}{きやく}の
\ruby{{\換字{前}}}{まへ}に
\ruby{到}{いた}れば、

『
\ruby{此酒}{こ|れ}を
\ruby{飮}{や}つて
\ruby{居}{ゐ}ながら
\ruby{待}{ま}たうと
\ruby{思}{おも}ふのだ。
\ruby{栓拔}{せん|ぬ}きと
\ruby{洋盞}{コツ|プ}とを
\ruby{假}{か}して
\ruby{吳}{く}れたまへ。
』

と
\ruby{酒罎}{び|ん}を
\ruby{指}{ゆび}さしながらの
\ruby{無邪氣}{む|じや|き}の
\ruby{言}{ことば}なり。

『ハイ、
\ruby{洋盞}{コツ|プ}はありましたが、
\ruby{栓拔}{せん|ぬ}きが…………。
』

とお
\ruby{濱}{はま}の
\ruby{一寸}{ちよ|つと}
\ruby{行詰}{ゆき|つま}りしも
\ruby{無理}{む|り}ならず、
\ruby{誰}{たれ}も
\ruby{洋酒}{やう|しゆ}など
\ruby{用}{もち}ゐるもの
\ruby{無}{な}き
\ruby{温厚者}{おと|なし|や}
\ruby{揃}{そろ}ひの、
\ruby{此家}{こ|ゝ}は
\ruby{特}{こと}に
\ruby{隱居處}{いん|きよ|じよ}の
\ruby{事}{こと}とて
\ruby{當世}{たう|せい}の
\ruby{人}{ひと}の
\ruby{出入}{で|いり}もおのづから
\ruby{少}{すくな}きより、
\ruby{事少}{こと|すくな}き
\ruby{村住居}{むら|ずま|ゐ}の
\ruby{簡素}{てが|るさ}に
\ruby{馴}{な}れて、
\ruby{今日}{け|ふ}の
\ruby{今}{いま}まで
\ruby{栓拔}{せん|ぬ}きに
\ruby{用}{よう}も
\ruby{無}{な}かりしほどなれば、
\ruby{貸}{か}さんと
\ruby{欲}{ほつ}して
\ruby{其物無}{その|もの|な}きに
\ruby{困}{こう}じ
\ruby{躊躇}{たゆ|た}へるなり。

お
\ruby{鍋}{なべ}を
\ruby{隣家}{とな|り}に
\ruby{走}{はし}らしめんか、
\ruby{隣家}{とな|り}はたゞの
\ruby{小{\換字{前}}}{こ|まへ}なれば、
\ruby{{\換字{猶}}}{なほ}さら
\ruby{栓拔}{せん|ぬき}などの
\ruby{有}{あ}るべくもあらず、さらば
\ruby{本家}{ほん|け}に
\ruby{至}{いた}らしめんか、
\ruby{本家}{ほん|け}と
\ruby{此家}{こ|ゝ}との
\ruby{餘}{あま}り
\ruby{隔}{へだゝ}りたり、
\ruby{如何}{い|かゞ}せん、と
お
\ruby{濱}{はま}は
\ruby{少時}{しば|し}
\ruby{{\換字{迷}}}{まよ}ひたりしが、ふと
\ruby{水野}{みづ|の}が
\ruby{洋小刀}{ナ|イ|フ}に
\ruby{栓拔}{せん|ぬ}きの
\ruby{添}{そ}ひ
\ruby{居}{ゐ}しを
\ruby{思}{おも}ひ
\ruby{出}{いだ}し、
\ruby{先}{ま}づ
お
\ruby{鍋}{なべ}を
\ruby{呼}{よ}びて
\ruby{小}{ちひさ}き
\ruby{{\換字{盆}}}{ぼん}に
\ruby{洋盞}{コツ|プ}を
\ruby{載}{の}せて
\ruby{持來}{もち|き}たらしめ、おのれは
\ruby{机}{つくゑ}の
\ruby{周圍}{まは|り}、
\ruby{本箱}{ほん|ばこ}の
\ruby{上}{うへ}などを
\ruby{見}{み}つ、
\ruby{彼}{か}の
\ruby{心當}{こゝろ|あて}の
\ruby{小刀}{ナイ|フ}をと
\ruby{{\換字{尋}}}{たづ}ね
\ruby{捜}{さが}したり。

されど
\ruby{小刀}{ナイ|フ}は
\ruby{外}{そと}に
\ruby{出}{い}で
\ruby{居}{を}
らずして、
\ruby{{\換字{終}}}{つひ}に
\ruby{見當}{み|あた}る
\ruby{事無}{こと|な}かりしかば、
\ruby{若}{もし}や
\ruby{此内}{この|うち}にと、
\ruby{机}{つくゑ}の
\ruby{下}{した}なる
\ruby{手箱}{て|ばこ}を
\ruby{引出}{ひき|いだ}して、
\ruby{日頃}{ひ|ごろ}の
\ruby{心易立}{こゝろ|やす|だて}に
\ruby{何}{なに}の
\ruby{氣}{き}も
\ruby{無}{な}く
\ruby{掻}{かい}
\ruby{撈}{さぐ}れば、
\ruby{書簡}{て|がみ}、
\ruby{雜記帳}{ざつ|き|ちやう}、
\ruby{物書}{もの|か}きさしたる
\ruby{反故}{ほ|ご}なんどの
\ruby{底}{そこ}の
\ruby{方}{かた}より
\ruby{洋小刀}{ナ|イ|フ}は
\ruby{出}{い}でたり。

『ヤ
\ruby{栓拔}{せん|ぬ}きは
\ruby{此品}{こ|れ}で
\ruby{澤山}{たく|さん}だ。
\ruby{何}{なん}だか
\ruby{面白}{おも|しろ}いものが
\ruby{出}{で}さうな
\ruby{匣}{はこ}だナ。
どれ
\ruby{{\換字{退}}屈{\換字{紛}}}{たい|くつ|まぎ}らしに
\ruby{見}{み}てやらうか。
』

\ruby{日方}{ひ|かた}は
\ruby{眼快}{め|ばや}く
\ruby{既}{すで}に
\ruby{彼}{か}の
\ruby{小刀}{ナイ|フ}を
\ruby{取}{と}りて、
\ruby{{\換字{猶}}}{なほ}また
\ruby{其匣}{その|はこ}の
\ruby{内}{うち}の
\ruby{物}{もの}を
\ruby{見}{み}んとすれば、

『およしなさいよ、
\ruby{他人}{ひ|と}さんの
\ruby{物}{もの}を。
\ruby{貴下}{あな|た}は
\ruby{亂暴}{らん|ぼう}\換字{子}。
』

と
\ruby{窘}{たしな}むるが
\ruby{如}{ごと}き
\ruby{口氣}{こう|き}に
\ruby{{\換字{強}}}{つよ}く
\ruby{云}{い}ひ
\ruby{懲}{こら}して、
お
\ruby{濱}{はま}は
\ruby{直}{たゞち}に
\ruby{匣}{はこ}の
\ruby{蓋}{ふた}を
\ruby{閉}{と}ぢ、
\ruby{机}{つくゑ}の
\ruby{下深}{した|ふか}く
\ruby{押入}{おし|い}れつ、
\ruby{無{\換字{遠}}慮}{ぶ|ゑん|りよ}も
\ruby{程度}{ほ|ど}のあるものをと
\ruby{腹立}{はら|だ}ちて、
あどけ
\ruby{無}{な}き
\ruby{顏}{かほ}にも
\ruby{瞋}{いかり}を
\ruby{含}{ふく}んで
\ruby{其處}{そ|こ}を
\ruby{{\換字{退}}}{しりぞ}きたり。

もとより
\ruby{年}{とし}もゆかぬ
お
\ruby{濱}{はま}などには
\ruby{眼}{め}も
\ruby{吳}{く}れざる
\ruby{日方}{ひ|かた}は、
\ruby{手酌}{て|じやく}の
\ruby{無興氣}{ぶ|きよ|うげ}に
\ruby{一盃}{いつ|ぱい}
\ruby{一盃}{いつ|ぱい}を
\ruby{重}{かさ}ねしが、
\ruby{飮}{の}んではいよ〳〵
\ruby{相手}{あひ|て}
\ruby{欲}{ほ}しさに
\ruby{獨居}{ひとり|ゐ}の
\ruby{淋}{さみ}しく、
\ruby{{\換字{所}}在無}{しよ|ざい|な}さの
\ruby{餘}{あま}りのわざくれに、
\ruby{{\換字{前}}}{さき}に
\ruby{見}{み}し
\ruby{手匝}{て|ばこ}を
\ruby{我}{わ}が
\ruby{{\換字{前}}{\換字{近}}}{まへ|ちか}く
\ruby{引寄}{ひき|よ}せ、
\ruby{内}{うち}なる
\ruby{雜記帳樣}{ざつ|き|ちやう|やう}のものを
\ruby{取出}{とり|いだ}して、
\ruby{此頃}{この|ごろ}
\ruby{水野}{みづ|の}が
\ruby{如何}{い|か}なる
\ruby{事}{こと}をか
\ruby{書}{か}けると、
\ruby{其}{それ}を
\ruby{知}{し}りたきばかりの
\ruby{好奇心}{かう|き|しん}に
\ruby{隔無}{へだ|てな}き
\ruby{中}{なか}とて
\ruby{無{\換字{遠}}慮}{ぶ|ゑん|りよ}にも、
\ruby{一盃}{いつ|ぱい}
\ruby{仰}{あふ}いでは
\ruby{一葉}{いち|\換字{𛀁}ふ}
\ruby{飜}{ひろがへ}し、
\ruby{一枚}{いち|まい}
\ruby{讀}{よ}みては
\ruby{一杯}{いつ|ぱい}
\ruby{仰}{あふ}いで、
\ruby{{\換字{終}}}{つひ}に
\ruby{我知}{われ|し}らず
\ruby{醉}{よひ}に
\ruby{入}{い}りぬ。

\ruby{冊子}{さう|し}は
\ruby{何}{なに}くれと
\ruby{無}{な}く
\ruby{水野}{みづ|の}が
\ruby{讀}{よ}み
\ruby{{\換字{過}}}{す}ごしたる
\ruby{或}{あるひ}は
\ruby{國書}{こく|しよ}
\ruby{或}{あるひ}は
\ruby{漢籍}{かん|せき}、
\ruby{或}{あるひ}は
\ruby{洋書}{やう|しよ}の
\ruby{其中}{その|うち}より、
\ruby{我}{わ}が
\ruby{意}{こゝろ}に
\ruby{{\換字{適}}}{てき}したる
\ruby{語}{ご}、
\ruby{詩句}{しの|く}、
\ruby{事實}{じ|ゞつ}なんゞを、
\ruby{或}{あるひ}は
\ruby{原}{もと}のまゝに、
\ruby{或}{あるひ}は
\ruby{引直}{ひき|なほ}して、
\ruby{筆任}{ふで|まか}せに
\ruby{記}{しる}したる
\ruby{眞實}{ま|こと}の
\ruby{雜抄}{ざつ|せう}にて、
\ruby{恰}{あたか}も
\ruby{人}{ひと}の
\ruby{摘}{つ}み
\ruby{集}{あつ}めし
\ruby{花}{はな}のいろ〳〵の
\ruby{線}{せん}に
\ruby{貫}{つらぬ}かれたるを
\ruby{見}{み}るが
\ruby{如}{ごと}く
\ruby{趣味}{おも|むき}あるものなれば、
\ruby{日方}{ひ|かた}は
\ruby{心竊}{こゝろ|ひそか}に
\ruby{水野}{みづ|の}が
\ruby{苦學}{く|がく}を
\ruby{怠}{おこた}らぬを
\ruby{悅}{よろこ}びながら
\ruby{讀}{よ}み
\ruby{居}{ゐ}しが、
\ruby{讀}{よ}む
\ruby{事{\換字{半}}途}{こと|なか|ば}にして
\ruby{間}{なか}に
\ruby{介}{はさ}まり
\ruby{居}{ゐ}し
\ruby{一片}{いつ|ぺん}の
\ruby{紙}{かみ}の
\ruby{偶然飛}{ふ|と|ゝ}び
\ruby{出}{い}でたれば、
\ruby{何}{なん}ならんと
\ruby{急}{きふ}に
\ruby{手}{て}に
\ruby{取}{と}りて
\ruby{見}{み}るに、
\ruby{第七番凶}{だい|なな|ばん|きよう}といふ
\ruby{觀音}{くわん|のん}の
\ruby{御籤}{み|くじ}なり。

『いかんナ。
\ruby{何樣}{ど|う}も
\ruby{怪}{をか}しいナ、
\ruby{此樣}{こ|ん}なものが
\ruby{出}{で}るとは。
\ruby{机}{つくゑ}の
\ruby{上}{うへ}には
\ruby{普門品}{ふ|もん|ぼん}がある、こゝには
\ruby{此樣}{こ|ん}なものが
\ruby{介}{はさま}つてゐる。

\ruby{何樣}{ど|う}したのだらう、
\ruby{何}{なん}だが
\ruby{怪}{おか}しいナ。
』

されど
\ruby{怪}{おか}しき
\ruby{事}{こと}は
\ruby{介}{はさ}まり
\ruby{居}{ゐ}し
\ruby{其}{それ}のみにして、
\ruby{冊子}{さう|し}の
\ruby{三{\換字{分}}}{さん|ぶ}の
\ruby{一}{いち}ほどは
\ruby{{\換字{猶}}}{なほ}
\ruby{白紙}{しら|かみ}の
\ruby{物}{もの}も
\ruby{書}{か}かれず
\ruby{殘}{のこ}れるなり。
これまでと
\ruby{日方}{ひ|かた}は
\ruby{其}{そ}の
\ruby{冊子}{さう|し}を
\ruby{伏}{ふ}せ
\ruby{棄}{す}てゝ、
\ruby{盃}{はい}を
\ruby{啣}{ふく}みて
\ruby{物}{もの}を
\ruby{案}{あん}じ
\ruby{居}{ゐ}しが、
\ruby{見}{み}るとも
\ruby{無}{な}しに
\ruby{見}{み}れば
\ruby{册子}{さう|し}の
\ruby{後}{うしろ}の
\ruby{表紙}{へう|し}には、
\ruby{反故染}{ほ|ご|ぞめ}といふものゝ
\ruby{如}{ごと}くに、
\ruby{落書}{らく|がき}の
\ruby{上}{うへ}に
\ruby{落書重}{らく|がき|かさ}なりて、
\ruby{縱横斜角}{たて|よこ|すぢ|かひ}に
\ruby{何}{なに}か
\ruby{書}{しる}されたり。
\ruby{何事}{なに|ごと}を
\ruby{加是}{か|く}は
\ruby{落書}{らく|がき}したりしやと、
\ruby{讀}{よ}み
\ruby{易}{やす}きを
\ruby{辿}{たど}りて
\ruby{一}{ひ}トつゞきを
\ruby{讀}{よ}めば、
\ruby{此}{こ}は
\ruby{是}{これ}
\ruby{一首}{いつ|しゆ}の
\ruby{歌}{うた}にして、

  %全角空白
\ruby{立}{た}ちて
\ruby{居}{ゐ}る
\ruby{方便}{たづ|き}も
\ruby{知}{し}らに
\ruby{我}{わ}が
\ruby{心天}{こゝろ|あま}つ
\ruby{{\換字{空}}}{そら}なり
\ruby{地}{つち}は
\ruby{踏}{ふ}めども

とありたり。

『フヽーン、
\ruby{精}{よ}くは
\ruby{{\換字{分}}}{わか}らんが
\ruby{戀}{こひ}の
\ruby{歌}{うた}だナ。
\ruby{水野}{みづ|の}が
\ruby{詠}{よ}んだのか
\ruby{知}{し}らん。
ウン
\ruby{彼}{あれ}のだらう。
も
\ruby{一}{ひと}ツは
\ruby{何}{なん}だ、ン、
\ruby{是}{これ}も
\ruby{歌}{うた}かナ。
ナニ。

  %全角空白
\ruby{天地}{あめ|つち}に
\ruby{少}{すこ}し
\ruby{至}{いた}らぬ
\ruby{大{\換字{丈}}夫}{ます|ら|を}と
\ruby{思}{おも}ひし
\ruby{我}{われ}や
\ruby{雄心}{をご|ゝろ}も
\ruby{無}{な}き

ハヽア、
\ruby{舊}{もと}は
\ruby{絶大}{ぜつ|だい}な
\ruby{抱負}{はう|ふ}も
\ruby{有}{あ}つた
\ruby{身}{み}だがと、
\ruby{戀}{こひ}に
\ruby{{\換字{迷}}}{まよ}つた
\ruby{今}{いま}を
\ruby{自}{みづか}ら
\ruby{悲}{かなし}む
\ruby{歌}{うた}だナ。
アヽ
\ruby{佳}{い}い
\ruby{歌}{うた}だ、
\ruby{乃公}{お|れ}にも
\ruby{解}{わか}
る。
\ruby{天地}{てん|ち}にも
\ruby{多}{おほ}くは
\ruby{劣}{おと}るまいと
\ruby{思}{おも}つて
\ruby{居}{ゐ}た
\ruby{此}{こ}の
\ruby{我身}{わが|み}だがなあと、
\ruby{戀}{こひ}の
\ruby{苦}{くる}しさに
\ruby{萎}{いほ}たれて、
\ruby{呻}{うめ}き
\ruby{出}{だ}した
\ruby{此}{こ}の
\ruby{歌}{うた}の
\ruby{主}{ぬし}の
\ruby{腹}{はら}ん
\ruby{中}{なか}が
\ruby{憫然}{かはい|さう}

\ruby{憫然}{かはい|さう}でならん。
\ruby{此方}{こつ|ち}に
\ruby{書}{か}いてあるのは
\ruby{何}{なん}だ。
\ruby{何}{なん}だと。

  %全角空白
\ruby{大{\換字{丈}}夫}{ます|ら|を}のさとき
\ruby{心}{こゝろ}も
\ruby{今}{いま}は
\ruby{無}{な}し
\ruby{戀}{こひ}の
\ruby{奴}{やつこ}と
\ruby{我}{われ}
は
\ruby{死}{し}ぬべし

アヽいかん〳〵、
\ruby{怪}{け}しからん
\ruby{事}{こつ}た、
\ruby{馬鹿}{ば|か}
\g詰めruby{々々}{〳〵}しい。
\ruby{散}{ち}らして
\ruby{書}{か}いてある
\ruby{此}{こ}の
\ruby{讀}{よ}
みにくいのは
\ruby{何}{なん}だ。

  %全角空白
\ruby{久堅}{ひさ|かた}のあまみづ

エート、

  %全角空白
\ruby{久堅}{ひさ|かた}の
\ruby{天}{あま}みつ
\ruby{{\換字{空}}}{そら}に
\ruby{照}{て}れる
\ruby{日}{ひ}の
\ruby{失}{う}せなん
\ruby{日}{ひ}こそ
\ruby{我}{わ}が
\ruby{戀止}{こひ|や}まめ

いかんナ、いかんナ、
\ruby{斯樣}{か|う}
\ruby{恐}{をそ}ろしく
\ruby{思}{おも}ひ
\ruby{{\換字{込}}}{こ}んでは
\ruby{始末}{し|まつ}が
\ruby{着}{つ}かん、
\ruby{斯樣}{か|う}
\ruby{滅茶}{め|ちや}
\ruby{苦茶}{く|ちや}になつては
\ruby{實}{じつ}にいかん、
\ruby{大馬鹿野郎}{おほ|ば|か|や|らう}だ、
\ruby{戀愛狂}{れん|あい|きやう}だ。
』

\ruby{此方}{こな|た}にては
\ruby{日方}{ひ|かた}が
\ruby{夢中}{む|ちう}になつて
\ruby{醉}{よひ}に
\ruby{乘}{じよう}じて
\ruby{如是}{か|く}
\ruby{罵}{のゝし}れる
\ruby{時}{とき}、
\ruby{彼方}{かな|た}にては
お
\ruby{濱}{はま}が
\ruby{悅}{よろこ}びに
\ruby{冴}{さ}ゆる
\ruby{聲}{こゑ}して、

『マア
\ruby{遲}{おそ}かつたのネエ、
\ruby{大變}{たい|へん}に
\ruby{待}{ま}つてたは。
それにアノ
\ruby{日方}{ひ|かた}さんといふ
\ruby{人}{ひと}が
\ruby{來}{き}て
\ruby{待}{ま}つてゝよ。
』

と
\ruby{忙}{せは}しげに
\ruby{言}{ものい}へば、
\ruby{同}{おな}じく
\ruby{聊}{いさゝ}か
\ruby{疾辯}{はや|くち}に、

『
\ruby{左樣}{さ|う}かエ、
\ruby{觀音樣}{くわん|のん|さま}であの
お
\ruby{龍}{りう}つていふ
\ruby{人}{ひと}にひよつくり
\ruby{逢}{あ}つて、あの
\ruby{人}{ひと}
の
\ruby{朋友}{とも|だち}だとか
\ruby{云}{い}ふ
\ruby{立派}{りつ|ぱ}な
\ruby{{\換字{婦}}人}{おく|さん}と
\ruby{二人}{ふた|り}に
\ruby{無理}{む|り}に
\ruby{{\換字{強}}}{し}ひられて
\ruby{御馳走}{ご|ち|そう}になつたりなんぞ
\ruby{仕}{し}たものだから、
\ruby{大}{おほき}に
\ruby{歸}{かへ}りが
\ruby{遲}{おそ}くなつて
\ruby{仕舞}{し|ま}つた。
\ruby{日方}{ひ|かた}は
\ruby{一時間}{いち|じ|かん}も
\ruby{{\換字{前}}}{まへ}から
\ruby{待}{ま}つて
\ruby{居}{ゐ}てかエ。
』

と、
\ruby{水野}{みづ|の}が
\ruby{語}{かた}る
\ruby{聲}{こゑ}の
\ruby{爲}{し}たり。
