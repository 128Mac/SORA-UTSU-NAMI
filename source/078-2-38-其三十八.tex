\Entry{其三十八}

% メモ 校正終了 2024-04-29
\原本頁{224-2}%
\ruby{好}{この}まぬ
\ruby{酒}{さけ}を
\ruby{人}{ひと}に
\ruby{{\換字{強}}}{し}ひられて、
%
\ruby[g]{水野}{みづの}は
\ruby{既}{すで}に
\ruby{醒}{さ}めたれども
\ruby{三{\換字{分}}}{さん|ぶ}の
\ruby{醉}{よひ}あり、% 「醉」は原本通り「よ」で調整
%
\ruby{好}{この}める
\ruby{酒}{さけ}を
\ruby{一人}{ひと|り}
\ruby{汲}{く}みて、
%
\ruby[g]{日方}{ひかた}は
\ruby{{\換字{猶}}}{なほ}
\ruby{足}{た}らずと
すれども
\ruby{七{\換字{分}}}{しち|ぶ}の
\ruby{醉}{よひ}あり。% 「醉」は原本通り「よ」で調整
%
たゞさへ% TODO 原本の「二の字点、揺すり点」に濁点のグリフが見つからないので「ゞ」
\ruby{醉}{よ}へる% 「醉」は原本通り「よ」で調整
\ruby{同士}{どう|し}は
\ruby{打解}{うち|と}け
\ruby{易}{やす}きに、
%
まして
\ruby{是}{これ}は
\ruby{一}{ひ}ト
\ruby{方}{かた}
ならぬ
\ruby{中}{なか}の
\ruby{舊友}{きう|いう}の、
%
たまさかに
\ruby{相}{あひ}
\ruby{逢}{あ}へる
なれば、
%
\ruby{笑顏}{ゑ|がほ}に
\ruby{云}{い}ひ
\ruby{出}{だ}されし、

\原本頁{224-7}%
『ヤ』

\原本頁{224-8}%
『ヤ』

\原本頁{224-9}%
の
\ruby{一}{ひ}ト
\ruby{聲}{こゑ}より
\ruby{先}{ま}づ
\ruby{碎}{くだ}け
\ruby{合}{あ}ひて、

\原本頁{224-10}%
『
\ruby{其}{その}
\ruby{後}{〻ち}% 原本通り「〻(二の字点、揺すり点)」
\ruby{久}{ひさ}しく
\ruby{會}{あ}はなかつたナア。
』

\原本頁{225-1}%
『
ほんとに
\ruby{長}{なが}い
\ruby{事}{こと}
\ruby{會}{あ}はなかつたナ。
』

\原本頁{225-2}%
『
ウン、
%
\ruby{乃公}{お|れ}が
\ruby{候補生}{こう|ほ|せい}に
なつた
\ruby{時}{とき}
\ruby[||j>]{祝}{しゆく}して
\ruby{吳}{く}れた
\ruby{會}{くわい}で
\ruby{會}{あ}つた
\ruby{限}{き}り
だつたナア。
』

\原本頁{225-4}%
『アヽ
\ruby{左樣}{さ|う}だつた。
%
\ruby{早}{はや}いもので
もう
\ruby{大{\換字{分}}}{だい|ぶ}
\ruby{{\換字{過}}去}{あ|と}に
なつた。
』

\原本頁{225-5}%
と
\ruby{互}{たがひ}に
\ruby{懷}{なつ}かしげに
\ruby{凝然}{じ|つ}と
\ruby{面}{かほ}を
\ruby{見}{み}
\ruby{合}{あ}ひしが、
%
\ruby[g]{水野}{みづの}が
\ruby{目}{め}には
\ruby[g]{日方}{ひかた}が
\ruby{肥}{こ}え
\ruby{肉}{にく}づきて
いよ〳〵
\ruby{男兒}{をと|こ}らしく
\ruby{立派}{りつ|ぱ}になれるが、
%
\ruby{羨}{うらや}ましくも
また
\ruby{好}{この}ましく
\ruby{見}{み}え、
%
\ruby[g]{日方}{ひかた}が
\ruby{眼}{め}には
\ruby[g]{水野}{みづの}が
\ruby{痩}{や}せ
\ruby{窶}{やつ}れて
\ruby{往時}{むか|し}の
\ruby{生々}{いき|〳〵}と
したる
\ruby{氣合}{き|あひ}の
\ruby{失}{う}せたるが、
%
\ruby{{\換字{情}}無}{なさけ|な}くも
また
\ruby{口惜}{くち|をし}く
\ruby{見}{み}えたり。

\原本頁{225-9}%
『
\ruby[g]{日方}{ひかた}!。
%
\ruby{久}{ひさ}しいと
\ruby{云}{い}つても
\ruby[<j||]{僅}{わづか}
\ruby{見}{み}
\ruby{無}{な}い
\ruby{中}{うち}に、
%
\ruby{君}{きみ}は
まあ
\ruby{實}{じつ}に
\ruby{立派}{りつ|ぱ}な
\ruby{好}{い}い
\ruby{身體}{から|だ}に
なつたナア。
』

\原本頁{225-11}%
『
\ruby{乃公}{お|れ}は
\ruby{其樣}{そ|ん}なに
\ruby{云}{い}はれる
ほどでも
ないが、
%
\ruby[g]{水野}{みづの}、
%
\ruby{汝}{きさま}は
また、
%
\ruby{大層}{たい|そう}
\ruby{痩}{や}せ
\ruby{枯}{から}びて
\ruby{年}{とし}を
\ruby{取}{と}つたナア。
』

\原本頁{226-2}%
\ruby{主人}{ある|じ}も
\ruby{客}{きやく}も
\ruby{共}{とも}に
\ruby{一種}{いつ|しゆ}の
\ruby{言}{い}ひ
\ruby{{\換字{難}}}{がた}き
\ruby{感}{かん}に
\ruby{打}{う}たれしが、
%
\ruby[g]{日方}{ひかた}は
\ruby{猿臂}{ゑん|ぴ}を
\ruby{伸}{の}ばして
\ruby[g]{水野}{みづの}の
\ruby{手}{て}を
\ruby{執}{と}り、

\原本頁{226-4}%
『
この
\ruby{骨}{ほね}つぽい
\ruby{痩}{や}せ
\ruby{切}{き}つた
\ruby{此手}{こ|れ}が、
%
\ruby{相撲}{すま|ふ}
\ruby{取}{と}りを
\ruby{仕}{し}ては
\ruby{{\換字{随}}{\換字{分}}}{ずゐ|ぶん}%「隨」TODO 変更 ⻖左円辶
\ruby{手}{て}
ひどく
\ruby{乃公}{お|れ}を
\ruby{投}{な}げつけた
\ruby{事}{こと}もある
\ruby{脅力}{ちか|ら}の
あつた
\ruby{手}{て}だらうか。
%
\ruby{此}{こ}の
\ruby{樣子}{やう|す}では
\ruby{今}{いま}では
\ruby{乃公}{お|れ}には、
%
\ruby{中々}{なか|〳〵}
\ruby{敵}{かな}ふ
どころでは
ありは
\ruby{仕}{し}まいが。
』

\原本頁{226-7}%
と
\ruby{云}{い}へば
\ruby{云}{い}はれたる
\ruby[g]{水野}{みづの}は
\ruby{歎}{たん}じて、

\原本頁{226-8}%
『
アヽ、
%
\ruby{今}{いま}ぢやあ
\ruby{一}{ひ}ト
\ruby{堪}{たま}りも
\ruby{無}{な}く
\ruby{負}{ま}かされて
\ruby{仕舞}{し|ま}はう。
%
これほど
\ruby[||j>]{衰}{おとろ}へて
\ruby{居}{ゐ}るとは
\ruby{自{\換字{分}}}{じ|ぶん}でも
\ruby{思}{おも}は
\ruby{無}{な}かつたが、
%
\ruby{君}{きみ}の
がつしりと
\ruby{仕}{し}た
\ruby{手}{て}と
\ruby{斯樣}{か|う}
\ruby{比}{くら}べては、
%
\ruby{羞}{はづ}かしい
やうな
\ruby[<j||]{心}{こ〻ろ}% 原本通り「〻(二の字点、揺すり点)」
\ruby{持}{もち}が
\ruby{仕}{し}て、
%
\ruby{物}{もの}
\ruby{悲}{がな}しい
\ruby{淋}{さび}しい
\ruby{感}{かん}じ
がする。
』

\原本頁{227-1}%
と
\ruby{蔽}{かく}す
ところも
\ruby{無}{な}く
\ruby{思}{おも}ふ
ま〻を% 原本通り「〻(二の字点、揺すり点)」
\ruby{打}{うち}
\ruby{出}{いだ}したり。
%
お
\ruby{濱}{はま}は
\ruby{小娘}{こ|むすめ}の
\ruby{智慧}{ち|ゑ}の
\ruby{乏}{とぼ}しけれど
\ruby{心}{こ〻ろ}ばかりの% 原本通り「〻(二の字点、揺すり点)」
\ruby{饗應}{もて|なし}に、
%
お
\ruby{鍋}{なべ}と
\ruby{相談}{さう|だん}して、
%
\ruby{干魚}{ひ|ざかな}を
\ruby{燒}{や}きて
\ruby{裂}{さ}きたると
\ruby{漬物}{つけ|もの}とを、
%
\ruby{酒}{さけ}の
\ruby{下物}{さか|な}にと
\ruby{案}{あん}じ
\ruby{出}{いだ}して
\ruby{持}{もち}
\ruby{來}{きた}りて
\ruby{歸}{かへ}りしが、
%
\ruby[g]{日方}{ひかた}は
それにも
\ruby{心}{こ〻ろ}づかぬ% 原本通り「〻(二の字点、揺すり点)」
\ruby{如}{ごと}く、

\原本頁{227-5}%
『
\ruby{左樣}{さ|う}だらう、
%
\ruby{定}{さだ}めし
\ruby{左樣}{さ|う}いふ
\ruby{感}{かん}じが
\ruby{仕}{し}やう。
%
\ruby{舊}{もと}と
\ruby{異}{ちが}つたとは
\ruby{乃公}{お|れ}
ばかりでも
\ruby{無}{な}い。
%
\ruby{汝}{きさま}は
\ruby{羽{\換字{勝}}}{は|がち}にも
まだ
\ruby{會}{あ}ふまいが、
%
\ruby{彼}{あれ}も
\ruby{鐵}{てつ}のやうな
\ruby{男兒}{をと|こ}に
\ruby{自{\換字{分}}}{じ|ぶん}を
\ruby{鍛}{きた}ひ
\ruby{上}{あ}げて、
%
\ruby{料簡}{れう|けん}にも
\ruby{言語}{もの|いひ}にも
\ruby{身體}{から|だ}つきにも、
%
\ruby{弛{\換字{緩}}}{だ|ら}けた
ところの
\ruby{無}{な}い
\ruby{確固}{しつ|かり}
\ruby{{\換字{漢}}}{もの}になつて
\ruby{來}{き}たぞ。
%
もとから
\ruby{一}{ひ}ト
\ruby{風}{ふう}ある
\ruby{男}{をとこ}だつたが、
%
いよ〳〵
\ruby{實}{み}が
\ruby{入}{い}つて
\ruby{物}{もの}に
なつた。
%
\ruby{今}{いま}に
\ruby{見}{み}ろ、
%
\ruby{何}{なに}か
\ruby{{\換字{遣}}}{や}り
\ruby{始}{はじ}めて、
%
\ruby{生命}{いの|ち}さへ
\ruby{有}{あ}りやあ
\ruby{屹度}{きつ|と}
\ruby{{\換字{遣}}}{や}り
\ruby{{\換字{遂}}}{と}げるは。
%
\ruby{島木}{しま|き}が
\ruby{金}{かね}を
\ruby{出}{だ}して
\ruby{{\換字{船}}}{ふね}を
\ruby{買}{か}つて、
%
\ruby{{\換字{遠}}洋漁業}{ゑん|やう|ぎよ|げふ}を
\ruby{爲}{や}るとか
\ruby{何}{なん}とか
\ruby{云}{い}つて
\原本頁{228-1}\改行%
\ruby{居}{ゐ}るから、
%
いづれ
\ruby{着々}{ちやく|〳〵}と
\ruby{歩}{ほ}を
\ruby{{\換字{進}}}{す〻}めて% 原本通り「〻(二の字点、揺すり点)」
\ruby{居}{ゐ}る
のだらう。
%
\ruby{今日}{け|ふ}も
\ruby{實}{じつ}は
\ruby{島木}{しま|き}の
ところで
\ruby{羽{\換字{勝}}}{は|がち}と
\ruby{乃公}{お|れ}と、
%
\ruby{三人}{さん|にん}
\ruby{落}{おち}
\ruby{合}{あ}つて
\ruby{此家}{こ|〻}へ% 原本通り「〻(二の字点、揺すり点)」
\ruby{來}{く}る
\ruby{筈}{はず}だつたが、
%
\ruby{羽{\換字{勝}}}{は|がち}に
\ruby{差支}{さし|つかへ}が
あつて
\ruby{斷}{ことわ}つて
\ruby{來}{き}たので、
%
\ruby{島木}{しま|き}は
\ruby{出}{で}ないと
\ruby{云}{い}ふし
\ruby{仕方}{し|かた}が
\ruby{無}{な}いから、
%
そこで
\ruby{乃公}{お|れ}
\ruby{一人}{ひと|り}で
\ruby{出}{で}て
\ruby{來}{き}たのだが‥‥‥、
%
\原本頁{228-5}\改行%
\ruby[g]{水野}{みづの}ツ!、
%
\ruby{久}{ひさ}しぶりで
\ruby{會}{あ}つて
\ruby{顏}{かほ}を
\ruby{見}{み}ると
\ruby{直}{すぐ}と、
%
\ruby{面白}{おも|しろ}く
\ruby{無}{な}い
\ruby{事}{こと}を
\ruby{云}{い}ひ
\ruby{出}{だ}す
やうだけれど、
%
\ruby{猿}{さる}が
\ruby{物}{もの}を
\ruby{含}{ふく}んで
\ruby{溜}{た}めて
\ruby{居}{ゐ}る
やうに、
%
\ruby{思}{おも}つた
\ruby{事}{こと}を
\ruby{口}{くち}の
\ruby{内}{うち}に
まごつかせては
\ruby{居}{ゐ}られない
\ruby{乃公}{お|れ}だ。
%
\ruby{汝}{きさま}の
\ruby[<j||]{俊}{しゆん}
\ruby[||j>]{寛}{くわん}
くさい
\ruby{血}{ち}の
\ruby{氣}{け}の
\ruby{足}{た}らん
\ruby{其}{その}
\ruby{面}{つら}つきを
\ruby{見}{み}、
%
\ruby[g]{狗骨樹}{ひ〻らぎ}の% 原本通り「〻(二の字点、揺すり点)」
\ruby{皮}{かは}を% 原本通り「皮 か(は)」
\ruby{剝}{む}いた
やうに
\ruby{瘠}{や}せつこけ
\ruby{切}{き}つた
\ruby{此樣}{こ|ん}な
\ruby{手}{て}を
\ruby{見}{み}ては、
%
\ruby{云}{い}はずには
\ruby{居}{ゐ}
られん、
%
\原本頁{228-10}\改行%
\ruby{堪{\換字{忍}}}{がま|ん}が% 原文通り「堪忍」
\ruby{出來}{で|き}ん、
\ruby{汝}{きさま}の
ために
\ruby{云}{い}ひ
\ruby{出}{だ}さずには
\ruby{居}{ゐ}られん。
%
\ruby{厭}{いや}でも
\ruby{應}{おう}でも
\ruby{聞}{き}いて
\ruby{貰}{もら}はねば
ならん。
%
\ruby[g]{日方}{ひかた}は
\ruby{汝}{きさま}に
\ruby{苦}{にが}いことを
\ruby{云}{い}はう
ために
わざ〳〵
\ruby{此家}{こ|〻}へ% 原本通り「〻(二の字点、揺すり点)」
\ruby{來}{き}たのだ。
%
さあ
\ruby{確乎}{しつ|かり}として
\ruby{熟}{よ}く
\ruby{聞}{き}いて
\ruby{吳}{く}れ
\ruby[g]{水野}{みづの}!。
』

\原本頁{229-3}%
と
\ruby{居{\換字{丈}}高}{ゐ|たけ|だか}となつて
\ruby{聲色}{せい|しよく}
\ruby{激}{はげ}しく
\ruby{說}{と}き
\ruby{出}{いだ}したり。
