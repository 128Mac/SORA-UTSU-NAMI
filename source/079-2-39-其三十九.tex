\Entry{其三十九}

% メモ 校正終了 2024-04-29 2024-06-04
\原本頁{229-5}%
かつて
\ruby[g]{島木}{しまき }が
\ruby{我}{われ}に
\ruby{告}{つ}げし
\ruby{言}{ことば}に
よりて、
%
\ruby[g]{日方}{ひ かた}が
\ruby{今}{いま}
\ruby{何}{なに}を
\ruby{云}{い}はんと
するかを
\ruby[g]{水野}{みづの }は
\ruby{猜}{すゐ}し
\ruby{知}{し}れる
なり。

\原本頁{229-7}%
\ruby{我}{われ}を
\ruby{思}{おも}ひ
\ruby{吳}{く}るゝ% 踊り字調整「〻(二の字点、揺すり点)に見えるが(ゝ)」
\ruby[g]{朋友}{ほういう}の
\ruby[g]{眞{\換字{情}}}{まごゝろ}% 踊り字調整「〻(二の字点、揺すり点)に見えるが(ゝ)」
より、
%
\ruby{我}{わ}が
\ruby{戀}{こひ}に
\ruby{惱}{なや}めるをば
\ruby{愚}{おろか}なり
として、
%
\ruby{說}{と}き
\ruby{醒}{さ}まし
\ruby{吳}{く}れん
とする
\ruby{其}{その}
\ruby{人}{ひと}に
\ruby{對}{むか}ひては、
%
そも〳〵
\ruby[g]{如何}{い か }なる
\ruby[g]{言葉}{ことば }
をもて
\ruby{應}{こた}ふ
べきぞや。
%
\ruby[g]{辯解}{いひわけ}% 弁 瓣 辦 辧 辨 辩 (辯)
すべき
\ruby{事}{こと}にも
あらず、
%
また
\ruby{本}{もと}より
\ruby{云}{い}ひ
\ruby{戾}{もど}くべき
\ruby{事}{こと}にも
あらねば、
%
\ruby{愼}{つゝし}みて% 踊り字調整「〻(二の字点、揺すり点)に見えるが(ゝ)」
\ruby{聞}{き}くより
ほかの
\ruby{事}{こと}は
\ruby{無}{な}かる
べし。
%
されど
\ruby{人}{ひと}の
\ruby[g]{言葉}{ことば }を
\ruby{聞}{き}きて
\ruby{思}{おも}ひ
\ruby{止}{と}まる
ことの
\ruby{叶}{かな}ふ
\原本頁{230-2}\改行%
ほど
ならば、
%
\ruby{世}{よ}に
\ruby{戀}{こひ}に
\ruby{悶}{もだ}ゆる
ものは
\ruby[g]{一人}{ひとり }も
\ruby{無}{な}くて、
%
\ruby[g]{他人}{ひ と }に
\ruby{云}{い}はるゝまでも% 踊り字調整「〻(二の字点、揺すり点)に見えるが(ゝ)」
あらず
\ruby{先}{ま}づ
\ruby{我}{われ}と
\ruby{吾}{わ}が
\ruby[g]{{\換字{分}}別}{ふんべつ}に、
%
よしなき
\ruby{惑}{まどひ}は
\ruby{思}{おも}ひ
\ruby{斷}{き}る
べきを、
%
\ruby{諦}{あきら}めても
\ruby{諦}{あきら}めても
\ruby{諦}{あきら}められぬ
にこそ
\ruby[g]{生命}{いのち }の
\ruby{縮}{ちゞ}む% 踊り字調整「〻(二の字点、揺すり点)に濁点に見えるが(ゞ)」
をも
\ruby{忘}{わす}れ
\ruby{人}{ひと}の
\ruby{謗}{そしり}
をも
\ruby{顧}{かへり}みで
\ruby{惱}{なや}み
\ruby{苦}{くるし}み
はするなれ。
%
それを
\ruby[g]{如何}{い か }に
\ruby[g]{朋友}{ほういう}の
% \原本頁{230-6}\改行%
\ruby[||j>]{眞}{まこと}の
\ruby{{\換字{情}}}{じやう}より
\ruby[g]{{\換字{道}}理}{ことわり}
せめて
\ruby{云}{い}ひ
\ruby{諭}{さと}されたり
とて、
%
\ruby{口}{くち}には
\ruby{思}{おも}ひ
\ruby{斷}{た}えたり
とも
\ruby{云}{い}はゞ% 踊り字調整「〻(二の字点、揺すり点)に濁点に見えるが(ゞ)」
\ruby{云}{い}ふ
べし、
%
\ruby{心}{こゝろ}より% 踊り字調整「〻(二の字点、揺すり点)に見えるが(ゝ)」
\ruby{全}{まつた}く
\ruby{改}{あらた}むる
\ruby{事}{こと}の
\ruby{何}{なに}として
\ruby{成}{な}るべき。
%
たゞ% 踊り字調整「〻(二の字点、揺すり点)に濁点に見えるが(ゞ)」
\ruby[g]{他人}{ひ と }の
\ruby[g]{親切}{しんせつ}にて
\ruby{言}{い}ひ
\ruby{吳}{く}るゝ% 踊り字調整「〻(二の字点、揺すり点)に見えるが(ゝ)」
\ruby{事}{こと}は、
%
よしや
\ruby{少}{すこ}しは
\ruby[g]{無理}{む り }なる
\ruby{{\換字{廉}}}{かど}
ありとも
\ruby{受}{う}くべきが
\ruby{{\換字{道}}}{みち}
なれば、
%
\ruby[g]{水野}{みづの }は
\ruby{頭}{かうべ}を
\ruby{垂}{た}れ
\ruby{肩}{かた}を
\ruby{窄}{すぼ}めて
\原本頁{230-10}\改行%
\ruby[g]{默々}{もく〳〵}と、
%
\ruby{雨}{あめ}に
\ruby{濕}{ぬ}れたる
\ruby{鷄}{とり}の
\ruby{如}{ごと}く
\ruby[||j>]{力}{ちから}
\ruby[||j>]{無}{ な}げに、
%
\ruby[g]{悄然}{せうぜん}と
\ruby[g]{日方}{ひ かた}の
\ruby{云}{い}ふ
ところをば
\ruby{聞}{き}かん
と
なしたり。

\原本頁{231-1}%
\ruby[g]{日方}{ひ かた}は
\ruby[g]{水野}{みづの }が
しほらしき
\ruby{此}{この}
\ruby{態}{てい}を
\ruby{見}{み}て
あはれを
\ruby{催}{もよほ}し、
%
\ruby{新}{あらた}に
また
\ruby{葡萄酒}{ぶ|だう|しゆ}の
\ruby{栓}{せん}を
\ruby{拔}{ぬ}きて、
%
\ruby[g]{水野}{みづの }が
\ruby{座}{ざ}の
\ruby{横}{よこ}に
\ruby[g]{何時}{い つ }か
\ruby{置}{お}かれたる
\ruby[g]{酒盞}{さかづき}に
\ruby{注}{つ}ぎ
\ruby{與}{や}りつ。

\原本頁{231-4}%
『
しかし
まあ
\ruby[g]{其樣}{そ ん }なに
\ruby{堅}{かた}く
ならん
でも
\ruby{宜}{い}いは
\ruby[g]{水野}{みづの }。
%
\ruby[g]{一杯}{いつぱい}
\ruby{飮}{や}つて
\原本頁{231-5}\改行%
\ruby{吳}{く}れ、
%
わざ〳〵
\ruby{持}{も}つて
\ruby{來}{き}たのだ。
%
\ruby{久}{ひさ}しぶりで
\ruby{汝}{きさま}と
\ruby[g]{一緖}{いつしよ}に
\ruby{飮}{や}らうと
\ruby{思}{おも}つて、
%
\ruby[g]{島木}{しまき }の
ところから
\ruby[||j>]{徴}{ちよう}
\ruby[||j>]{發}{ はつ}して
\ruby{來}{き}たのだ。
%
\ruby{何}{なに}も
\ruby[g]{左樣}{さ う }
\ruby[g]{危坐}{かしこま}つて
\ruby{貰}{もら}はん
でも
\ruby{宜}{い}い、
%
\ruby{汝}{きさま}と
\ruby[g]{乃公}{お れ }との
\ruby{中}{なか}ぢや
\ruby{無}{な}いか。
\ruby[g]{乃公}{お れ }は
サーベル
\ruby[g]{三昧}{ざんまい}、
%
\ruby{汝}{きさま}は
\ruby[g]{書籍}{ほ ん }
\ruby[g]{三昧}{ざんまい}、
%
たづさはる
\ruby{{\換字{道}}}{みち}が
\ruby{異}{ちが}ふので
\ruby{姑}{しばら}く
\ruby{{\換字{遠}}}{とほざ}かつたが、
%
\ruby[g]{幾年}{いくねん}か
\ruby{{\換字{前}}}{まへ}は
\ruby{一}{ひと}ツに
\ruby{居}{ゐ}て、
%
\ruby{醉眠秋被}{すゐ|みん|あき|ひ}
\footnote{
「李十二白と同に活十の隠居を尋ぬ」詩の冒頭六句の一句の
「酔眠秋共被 酔いて眠れば秋に被(掛け布団)を共にし」
からの引用と思われ原本通り「秋」は(あき)とする
(国会図書館 コマ番号120/160 p-231 l-09)}%
を
\ruby{共}{とも}にし、
%
\ruby{手}{て}を
\ruby{携}{たづさ}へて
\原本頁{231-10}\改行%
\ruby{日}{ひ}に
\ruby[g]{同行}{どうかう}す
といふ
\ruby{{\換字{古}}}{ふる}い
\ruby{詩}{し}の
\ruby{句}{く}の
\ruby{{\換字{通}}}{とほ}りを
\ruby{其}{その}
\ruby{儘}{まゝ}の% 踊り字調整「〻(二の字点、揺すり点)に見えるが(ゝ)」
\ruby[||j>]{境}{きやう}
\ruby[||j>]{界}{ かい}だナアと、
%
ソレ
\ruby{笑}{わら}ひ
\ruby{合}{あ}つた
\ruby{事}{こと}も
\ruby{有}{あ}つた
\ruby{中}{なか}
だもの、
%
\ruby[g]{{\換字{遠}}慮}{ゑんりよ}も
\ruby{斟}{しん}
\ruby[||j>]{{\換字{酌}}}{しやく}も
\ruby{有}{あ}らう
\ruby{筈}{はず}は
\ruby{無}{な}
\原本頁{232-1}\改行%
い。
%
さあ
\ruby[g]{左樣}{さ う }いふ
\ruby{中}{なか}
だによつて
\ruby{默}{だま}つては
\ruby{居}{を}られんで、
%
\ruby[g]{言語}{ことば }に
\ruby{艶}{つや}も
\ruby{付}{つ}けず
\ruby[g]{露骨}{むきだし}に
\ruby{云}{い}ふが、
%
\ruby[g]{水野}{みづの }!\inhibitglue{}%
\ruby{汝}{きさま}は
\ruby{何}{なん}で
\ruby[g]{{\換字{情}}無}{なさけな}い
\ruby{{\換字{魔}}}{ま}に
\ruby{憑}{つ}かれた!
\改行% 校正作業の簡略化のため
。
%
\原本頁{232-3}\改行%
\ruby[g]{我々}{われ〳〵}の
\ruby{中}{うち}で
\ruby{年}{とし}は
\ruby{{\換字{若}}}{わか}いが、
%
\ruby[g]{聰明}{そうめい}で
\ruby{慾}{よく}が
\ruby{寡}{すくな}くて
\ruby[g]{學問}{がくもん}が
\ruby{好}{すき}で、
%
\ruby[g]{立派}{りつぱ }な
\ruby[g]{學者}{がくしや}か
\ruby[g]{詩仙}{し せん}か
に
ならうより
ほかには
\ruby{爲}{な}りやうも
\ruby{無}{な}いと
\ruby{思}{おも}つて
\ruby{居}{ゐ}た
\ruby{汝}{きさま}が、
%
\ruby{此}{この}
\ruby{頃}{ごろ}の
\ruby[g]{墮落}{だ らく}の
\ruby[g]{仕方}{し かた}は
\ruby{何}{なん}といふ
\ruby[g]{{\換字{情}}無}{なさけな}い
\ruby{態}{てい}だ。
%
\ruby{隱}{かく}しても
いかん
\ruby[g]{悉皆}{みんな }
\ruby{知}{し}つて
\ruby{居}{ゐ}る。
%
\ruby{其}{そ}の
\ruby{顏}{かほ}の
\ruby[g]{樵悴}{やつれ }は
\ruby{何}{なに}からの
\ruby{事}{こと}だ!。
%
\ruby{其}{そ}の
\ruby[g]{身體}{からだ }の
\ruby[g]{枯稿}{や せ }は
\ruby{何}{なに}
\ruby{故}{ゆゑ}の
\ruby[g]{枯稿}{や せ }だ。
%
\ruby{憫}{かあ}% 「憫然 か(あ)いさう」
\ruby[||j>]{然}{いさう}に
\ruby[g]{其樣}{そ ん }な
ひがいすな
\ruby[g]{身體}{からだ }に
なつて
\ruby{何}{なに}が
\ruby[g]{出來}{で き }やう?。
%
\ruby{眼}{め}に
\ruby{見}{み}える
ところ
さへ
\ruby{其}{その}
\ruby{{\換字{通}}}{とほ}り
だもの、
%
まし
\原本頁{232-9}\改行%
て
\ruby{心}{こゝろ}の% 踊り字調整「〻(二の字点、揺すり点)に見えるが(ゝ)」
\ruby{{\換字{弱}}}{よわ}りは
\ruby[g]{何程}{どれほど}
だらうと
\ruby{思}{おも}ひ
\ruby{{\換字{遣}}}{や}られて、
%
\ruby{汝}{きさま}の
ために
\ruby{涙}{なみだ}が
\ruby{出}{で}る
\改行% 校正作業の簡略化のため
、
%
\原本頁{232-10}\改行%
\ruby[g]{口惜}{くちをし}くなる、
%
\ruby{腹}{はら}が
\ruby{立}{た}
つ!。
%
それも
\ruby{此}{これ}も
\ruby{時}{とき}の
\ruby[<j>]{災}{わざはひ}
\ruby{人}{ ゝと}の% 踊り字調整「〻(二の字点、揺すり点)に見えるが(ゝ)」% 「災(わざはひ)」の後突出対策
\ruby{爲}{しわざ}の
\ruby{故}{せい}% ルビ調整(原本通り)「せ(い)」
でもあればこそ、
%
\ruby{汝}{きさま}の
\ruby[g]{一心}{いつしん}の
\ruby{据}{す}ゑやうが
\ruby{惡}{わる}くて、
%
\ruby{高}{たか}の
\ruby{知}{し}れた
\ruby{一}{いつ}
\ruby[g]{{\換字{婦}}人}{ぷ じん}に
\原本頁{233-1}\改行%
\ruby{氣}{き}を
\ruby{取}{と}られた
からとは、
%
\ruby[g]{{\換字{平}}生}{ひ ごろ}の% ルビ調整(原本通り)
\ruby{汝}{きさま}にも
\ruby[g]{似合}{に あ }はん
\ruby{愚}{ぐ}な
\ruby{事}{こと}では
\ruby{無}{な}いか。
%
\ruby[g]{{\換字{婦}}女}{をんな }が
\ruby{何}{なん}だ!。
%
\ruby{戀}{こひ}が
\ruby{何}{なん}だ!。
%
たとひ
\ruby[g]{美女}{び ぢよ}
だらうが
\ruby[g]{賢女}{けんぢよ}
だらうが、
%
\ruby{我}{われ}を
\ruby{{\換字{迷}}}{まよ}は
せりやあ
\ruby{我}{われ}の
\ruby[g]{仇敵}{かたき }だ。
%
\ruby[g]{男兒}{をとこ }の
\ruby[g]{正氣}{ほんき }に
なつて
\ruby{働}{はたら}かう
といふ
\ruby[g]{事業}{し ごと}の、% ルビ調整(原本通り)
%
\ruby[||j>]{障}{しやう}
\ruby[||j>]{礙}{ がい}になる
\ruby{奴}{やつ}あ
\ruby[g]{悉皆}{みんな }
\ruby[g]{仇敵}{かたき }だ。
%
\ruby{戀}{こひ}たあ
\ruby[g]{料簡}{れうけん}の
\ruby{弛}{ゆる}みへ
\ruby{出}{で}る
\ruby{黴}{かび}だ、
%
\ruby[g]{閑暇}{ひ ま }な
\ruby[g]{馬鹿}{ば か }
\ruby[g]{野郎}{や らう}の
\ruby{掌}{て}の
\ruby{中}{なか}の
\ruby{玩弄物}{おも|ち|や}だ。
%
\ruby[g]{世間}{せ けん}
\ruby[g]{一體}{いつたい}の
\ruby{風}{ふう}とは
\ruby{云}{い}ひ
ながら、
%
\ruby[g]{新聞}{しんぶん}を
\ruby{見}{み}ても
\ruby[g]{書籍}{ほ ん }を
\ruby{見}{み}ても、
%
\ruby{戀}{こひ}だ
\ruby{董}{すみれ}だ
\ruby{蝶}{てふ}だ
\ruby[g]{百合}{ゆ り }だ
と、
%
\ruby{女}{をんな}
\ruby{臭}{くさ}い
ことばかり
\ruby[g]{流行}{は や }つて
\ruby{居}{ゐ}て、
%
まるで
\ruby[g]{明治}{めいぢ }の
\ruby{{\換字{若}}}{わか}い
\ruby{奴}{やつ}は、
%
\ruby{戀}{こひ}を
するために
\ruby{此}{こ}の
\ruby{世}{よ}の
\ruby{中}{なか}へ
\ruby{生}{うま}れて
\ruby{來}{き}たので、
%
\ruby[g]{希望}{の ぞみ}も
\原本頁{233-9}\改行%
\ruby[g]{事業}{し ごと}も% ルビ調整(原本通り)
\ruby{無}{な}い
ものゝやうだが、% 踊り字調整「〻(二の字点、揺すり点)に見えるが(ゝ)」
%
\ruby[g]{水野}{みづの }!\inhibitglue{}%
\ruby{汝}{きさま}まで
\ruby{其}{その}
\ruby{風}{ふう}に
\ruby[g]{{\換字{感}}染}{か ぶ }れた
とは
\ruby{何}{なん}たる
\ruby{事}{こつ}た!。
%
\ruby[g]{南風}{みなみ }が
\ruby{吹}{ふ}きやあ
\ruby{北}{きた}へ
\ruby[g]{貼然}{べつたり}、
%
\ruby{{\換字{又}}}{また}
%
\ruby[g]{北風}{き た }が
\ruby{吹}{ふ}きやあ
\ruby[<j||]{南}{みなみ}へ
\ruby[g]{貼然}{べつたり}する、
%
\ruby{{\換字{平}}々凡々}{へい|〳〵|ぼん|〴〵}の
\ruby{草}{くさ}のやうに、
%
\ruby{自}{みづか}ら
\ruby{立}{た}つて
\ruby{居}{ゐ}る
\ruby{事}{こと}が
\ruby[g]{出來}{で き }ないとは
\ruby[g]{見下}{み さ }げた
\ruby{奴}{やつ}だナ。
%
\ruby[g]{其樣}{そ ん }な
\ruby{腰}{こし}の
\ruby{無}{な}い
\ruby{奴}{やつ}では
\ruby{無}{な}かつたが、
%
\原本頁{234-2}\改行%
\ruby{汝}{きさま}も
\ruby[g]{一世}{いつせ }の
\ruby[g]{風潮}{ふうてう}には
\ruby{捲}{ま}き
\ruby{倒}{たふ}されない
\ruby[g]{男兒}{をとこ }らしい
\ruby[g]{男兒}{をとこ }
には
なりかねて、
%
\ruby{波}{なみ}に
\ruby{{\換字{随}}}{したが}ひ%「隨」グリフ変更 ⻖左円辶
\ruby{浪}{なみ}を
\ruby{{\換字{逐}}}{お}ふ
\ruby{意氣地}{い|く|ぢ}
\ruby{無}{な}し
なつたか!。
』
