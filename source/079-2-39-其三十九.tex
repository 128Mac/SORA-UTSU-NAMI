\Entry{其三十九}

かつて
\ruby[g]{島木}{しまき}が
\ruby{我}{われ}に
\ruby{告}{つ}げし
\ruby{言}{ことば}によりて、
\ruby[g]{日方}{ひかた}が
\ruby{今何}{いま|なに}を
\ruby{云}{い}はんとするかを
\ruby[g]{水野}{みづの}は
\ruby{猜}{すひ}し
\ruby{知}{し}れるなり。

\ruby{我}{われ}を
\ruby{思}{おも}ひ
\ruby{{\換字{呉}}}{く}るヽ
\ruby{朋友}{はう|いう}の
\ruby{眞{\換字{情}}}{まご|ころ}より、
\ruby{我}{わ}が
\ruby{戀}{こひ}に
\ruby{惱}{なや}めるをは
\ruby{愚}{おろか}なりとして、
\ruby{{\換字{説}}}{と}き
\ruby{醒}{さ}まし
\ruby{{\換字{呉}}}{く}れんとする
\ruby{其人}{その|ひと}に
\ruby{對}{むか}ひては、そも〳〵
\ruby{如何}{い|か}なる
\ruby[g]{言葉}{ことば}をもて
\ruby{應}{こた}ふべきぞや。
\ruby{辯解}{いひ|わけ}すべき
\ruby{事}{こと}にもあらず、また
\ruby{本}{もと}より
\ruby{云}{い}ひ
\ruby{戾}{もど}くべき
\ruby{事}{こと}にもあらねば、
\ruby{愼}{つヽし}みて
\ruby{聞}{き}くよりほかの
\ruby{事}{こと}は
\ruby{無}{な}かるべし。

されど
\ruby{人}{ひと}の
\ruby{言葉}{こと|ば}を
\ruby{聞}{き}きて
\ruby{思}{おも}
ひ
\ruby{止}{と}まることの
\ruby{叶}{かな}ふほどならば、
\ruby{世}{よ}に
\ruby{戀}{こひ}に
\ruby{悶}{もだ}ゆるものは
\ruby[g]{一人}{ひとり}も
\ruby{無}{な}くて、
\ruby{他人}{ひ|と}に
\ruby{云}{い}はるヽまでもあらず
\ruby{先}{ま}づ
\ruby{我}{われ}と
\ruby{吾}{わ}が
\ruby{分別}{ふん|べつ}に、よしなき
\ruby{惑}{まどひ}は
\ruby{思}{おも}ひ
\ruby{斷}{き}るべきを、
\ruby{諦}{あきら}めても
\ruby{諦}{あきら}めても
\ruby{諦}{あきら}められぬにこそ
\ruby[g]{生命}{いのち}の
\ruby{縮}{ちヾ}
むをも
\ruby{忘}{わす}れ
\ruby{人}{ひと}の
\ruby{謗}{そしり}をも
\ruby{顧}{かへり}みで
\ruby{惱}{なや}み
\ruby{苦}{くるし}みはするなれ。
それを
\ruby{如何}{い|か}に
\ruby{朋友}{ほう|いう}の
\ruby{眞}{まこと}の
\ruby{{\換字{情}}}{じやう}より
\ruby{道理}{こと|はり}せめて
\ruby{云}{い}ひ
\ruby{諭}{さと}されたりとて、
\ruby{口}{くち}には
\ruby{思}{おも}ひ
\ruby{斷}{た}えたりとも
\ruby{云}{い}ふべし、
\ruby{心}{こヽろ}より
\ruby{全}{まつた}く
\ruby{改}{あらた}むる
\ruby{事}{こと}の
\ruby{何}{なん}として
\ruby{成}{な}るべき。
たヾ
\ruby{他人}{ひ|と}の
\ruby{親切}{しん|せつ}にて
\ruby{言}{い}ひ
\ruby{{\換字{呉}}}{く}るヽ
\ruby{事}{こと}は、よしや
\ruby{少}{すこ}しは
\ruby{無理}{む|り}なる
\ruby{廉}{かど}ありとも
\ruby{受}{う}くべきが
\ruby{{\GWI{u9053-k}}}{みち}なれば、
\ruby[g]{水野}{みづの}は
\ruby{頭}{かうべ}を
\ruby{垂}{た}れ
\ruby{肩}{かた}を
\ruby{窄}{すぼ}めて
\ruby{默々}{もく|〳〵}と、
\ruby{雨}{あめ}に
\ruby{濕}{ぬ}れたる
\ruby{鷄}{とり}の
\ruby{如}{ごと}く
\ruby{力無}{ちから|な}げに、
\ruby{悄然}{せう|ぜん}と
\ruby[g]{日方}{ひかた}の
\ruby{云}{い}ふところをば
\ruby{聞}{き}かんとなしたり。

\ruby[g]{日方}{ひかた}は
\ruby[g]{水野}{みづの}がしほらしき
\ruby{此態}{この|てい}を
\ruby{見}{み}てあはれを
\ruby{催}{もよほ}し、
\ruby{新}{あらた}にまた
\ruby{葡萄酒}{ぶ|だう|しゆ}の
\ruby{栓}{せん}を
\ruby{拔}{ぬ}きて、
\ruby[g]{水野}{みづの}が
\ruby{座}{ざ}の
\ruby{横}{よこ}に
\ruby{何時}{い|つ}か
\ruby{置}{お}かれたる
\ruby{酒盞}{さか|づき}に
\ruby{注}{つ}ぎ
\ruby{與}{や}りつ。

『しかしまあ
\ruby{其樣}{そ|ん}なに
\ruby{堅}{かた}くならんでも
\ruby{宜}{よ}いは
\ruby[g]{水野}{みづの}。
\ruby[g]{一杯飮}{いつぱいや}つて
\ruby{{\換字{呉}}}{く}れ、わざ〳〵
\ruby{持}{も}つて
\ruby{來}{き}たのだ。
\ruby{久}{ひさ}しぶりで
\ruby{汝}{きさま}と
\ruby{一緒}{いつ|しよ}に
\ruby{飮}{や}らうと
\ruby{思}{おも}つて、
\ruby{島木}{しま|き}のところから
\ruby{徴發}{ちよう|はつ}して
\ruby{來}{き}たのだ。
\ruby{何}{なに}も
\ruby{左樣危坐}{さ|う|かし|こま}つて
\ruby{貰}{もら}はんでも
\ruby{宜}{い}い、
\ruby{汝}{きさま}と
\ruby{乃公}{お|れ}との
\ruby{中}{なか}ぢや
\ruby{無}{な}いか。

\ruby{乃公}{お|れ}はサーベル
\ruby{三昧}{ざん|まい}、
\ruby{汝}{きさま}は
\ruby{書籍三昧}{ほ|ん|ざん|まい}、たづさはる
\ruby{{\GWI{u9053-k}}}{みち}が
\ruby{異}{ちが}ふので
\ruby{姑}{しばら}く
\ruby{{\GWI{u9060-k}}}{とほざ}かつたが、
\ruby{幾年}{いく|ねん}か
\ruby{前}{まへ}は
\ruby{一}{ひと}ツに
\ruby{居}{ゐ}て、
\ruby[g]{醉眠秋被}{すゐみんあきひ}を
\ruby{共}{とも}にし、
\ruby{手}{て}を
\ruby{携}{たづさ}へて
\ruby{日}{ひ}に
\ruby{同行}{どう|かう}すといふ
\ruby{古}{ふる}い
\ruby{詩}{し}の
\ruby{句}{く}の
\ruby{通}{とほ}りを
\ruby{其儘}{その|まヽ}の
\ruby[g]{境界}{きやうかい}だナアと、ソレ
\ruby{笑}{わら}ひ
\ruby{合}{あ}つた
\ruby{事}{こと}も
\ruby{有}{あ}つた
\ruby{中}{なか}だもの、
\ruby{{\GWI{u9060-k}}慮}{ゑん|りよ}も
\ruby{斟酌}{しん|しやく}も
\ruby{有}{あ}らう
\ruby{筈}{はず}は
\ruby{無}{な}い。
さあ
\ruby{左樣}{さ|う}いふ
\ruby{中}{なか}だによつて
\ruby{默}{だま}つては
\ruby{居}{を}られんで、
\ruby{言語}{こと|ば}に
\ruby{艶}{つや}も
\ruby{付}{つ}けず
\ruby{露骨}{むき|だし}に
\ruby{云}{い}ふが、
\ruby[g]{水野}{みづの}!
\ruby{汝}{きさま}は
\ruby{何}{なん}で
\ruby{{\換字{情}}無}{なさ|けな}い
\ruby{魔}{ま}に
\ruby{憑}{つ}かれた!。

\ruby[g]{我々}{われ〳〵}の
\ruby{中}{うち}で
\ruby{年}{とし}は
\ruby{若}{わか}いが、
\ruby{聰明}{そう|めい}で
\ruby{慾}{よく}が
\ruby{寡}{すくな}くて
\ruby{學問}{がく|もん}が
\ruby{好}{すき}で、
\ruby[g]{立派}{りつぱ}な
\ruby{學者}{がく|しや}か
\ruby{詩仙}{し|せん}かにならうよりほかには
\ruby{爲}{な}りやうも
\ruby{無}{な}いと
\ruby{思}{おも}つて
\ruby{居}{ゐ}た
\ruby{汝}{きさま}が、
\ruby{此頃}{この|ごろ}の
\ruby{墮落}{だ|らく}の
\ruby{仕方}{し|かた}は
\ruby{何}{なん}といふ
\ruby{{\換字{情}}無}{なさ|けな}い
\ruby{態}{てい}だ。
\ruby{隱}{かく}してもいかん
\ruby[g]{悉皆知}{みんなし}つて
\ruby{居}{ゐ}る。
\ruby{其}{そ}の
\ruby{顏}{かほ}の
\ruby{樵悴}{やつ|れ}は
\ruby{何}{なに}からの
\ruby{事}{こと}だ!。
\ruby{其}{そ}の
\ruby{身體}{から|だ}の
\ruby[g]{枯稿}{やせ}は
\ruby{何故}{なに|ゆゑ}の
\ruby{枯稿}{や|せ}だ。

\ruby{憫然}{かあい|さう}に
\ruby{其樣}{そ|ん}なひがいすな
\ruby[g]{身體}{からだ}になつて
\ruby{何}{なに}が
\ruby{出來}{で|き}やう?。
\ruby{眼}{め}に
\ruby{見}{み}えるところさへ
\ruby{其{\換字{通}}}{その|とほ}りだもの、まして
\ruby{心}{こヽろ}の
\ruby{{\換字{弱}}}{よわ}りは
\ruby{何程}{どれ|ほど}だらうと
\ruby{思}{おも}ひ
\ruby{{\GWI{u9063-k}}}{や}られて、
\ruby{汝}{きさま}のために
\ruby{淚}{なみだ}が
\ruby{出}{で}る、
\ruby{口惜}{くち|をし}くなる、
\ruby{腹}{はら}が
\ruby{立}{た}
つ!。
それも
\ruby{此}{これ}も
\ruby{時}{とき}の
\ruby[g]{災人}{わざはひヽと}の
\ruby{爲}{しわざ}の
\ruby{故}{せい}でもあればこそ、
\ruby{汝}{きさま}の
\ruby{一心}{いつ|しん}の
\ruby{据}{す}ゑやうが
\ruby{惡}{わる}くて、
\ruby{高}{たか}の
\ruby{知}{し}れた
\ruby[g]{一婦人}{いちふじん}に
\ruby{氣}{き}を
\ruby{取}{と}られたからとは、
\ruby[g]{{\換字{平}}生}{ひごろ}の
\ruby{汝}{きさま}にも
\ruby{似合}{に|あ}はん
\ruby{愚}{ぐ}な
\ruby{事}{こと}では
\ruby{無}{な}いか。
\ruby[g]{婦女}{をんな}が
\ruby{何}{なん}だ!。
\ruby{戀}{こひ}が
\ruby{何}{なん}だ!。
たとひ
\ruby[g]{美女}{びぢよ}だらうが
\ruby{賢女}{けん|じよ}だらうが、
\ruby{我}{われ}を
\ruby{{\換字{迷}}}{まよ}はせりやあ
\ruby{我}{われ}の
\ruby[g]{仇敵}{かたき}だ。
\ruby[g]{男兒}{をとこ}の
\ruby{正氣}{ほん|き}になつて
\ruby{働}{はたら}かうといふ
\ruby{事業}{し|ごと}の、
\ruby[g]{障礙}{しやうがい}になる
\ruby{奴}{やつ}あ
\ruby[g]{悉皆仇敵}{みんなかたき}だ。
\ruby{戀}{こひ}たあ
\ruby{料簡}{れう|けん}の
\ruby{弛}{ゆる}みへ
\ruby{出}{で}る
\ruby{黴}{かび}だ、
\ruby{閑暇}{ひ|ま}な
\ruby{馬鹿野郎}{ば|か|や|らう}の
\ruby{掌}{て}の
\ruby{中}{なか}の
\ruby[g]{玩弄物}{おもちや}だ。
\ruby{世間一體}{せ|けん|いつ|たい}の
\ruby{風}{ふう}とは
\ruby{云}{い}ひながら、
\ruby{新聞}{しん|ぶん}を
\ruby{見}{み}ても
\ruby{書籍}{ほ|ん}を
\ruby{見}{み}ても、
\ruby{戀}{こひ}だ
\ruby{董}{すみれ}だ
\ruby{蝶}{てふ}た
\ruby{百合}{ゆ|り}だと、
\ruby{女臭}{をんな|くさ}いことばかり
\ruby{流行}{は|や}つて
\ruby{居}{ゐ}て、まるで
\ruby{明治}{めい|じ}の
\ruby{若}{わか}い
\ruby{奴}{やつ}は、
\ruby{戀}{こひ}をするために
\ruby{此}{こ}の
\ruby{世}{よ}の
\ruby{中}{なか}へ
\ruby{生}{うま}れて
\ruby{來}{き}たので、
\ruby{希望}{の|ぞみ}も
\ruby{事業}{し|ごと}も
\ruby{無}{な}いものヽやうだが、
\ruby[g]{水野}{みづの}!
\ruby{汝}{きさま}まで
\ruby{其風}{その|ふう}に
\ruby{感染}{か|ぶ}れたとは
\ruby{何}{なん}たる
\ruby{事}{こつ}た!。
\ruby[g]{南風}{みなみ}が
\ruby{吹}{ふ}きやあ
\ruby{北}{きた}へ
\ruby{貼然}{べつ|たり}、
\ruby[g]{{\換字{叉}}北風}{またきた}が
\ruby{吹}{ふ}きやあ
\ruby{南}{みなみ}へ
\ruby{貼然}{べつ|たり}する、
\ruby{平々凡々}{へい|〳〵|ぼん|〳〵}の
\ruby{草}{くさ}のやうに、
\ruby{自}{みづか}ら
\ruby{立}{た}つて
\ruby{居}{ゐ}る
\ruby{事}{こと}が
\ruby{出來}{で|き}ないとは
\ruby{見下}{み|さ}げた
\ruby{奴}{やつ}だナ。
\ruby{其樣}{そ|ん}な
\ruby{腰}{こし}の
\ruby{無}{な}い
\ruby{奴}{やつ}では
\ruby{無}{な}かつたが、
\ruby{汝}{きさま}も
\ruby{一世}{いつ|せ}の
\ruby{風潮}{ふう|てう}には
\ruby{捲}{ま}き
\ruby{倒}{たふ}されない
\ruby[g]{男兒}{をとこ}らしい
\ruby[g]{男兒}{をとこ}になりかねて、
\ruby{波}{なみ}に
\ruby{隨}{したが}ひ
\ruby{浪}{なみ}を
\ruby{逐}{お}ふ
\ruby{意氣地無}{い|く|じ|な}しなつたか!。
』

