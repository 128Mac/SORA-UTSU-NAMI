\Entry{其三十九}

\原本頁{229-5}%
かつて
\ruby[g]{島木}{しまき}が
\ruby{我}{われ}に
\ruby{告}{つ}げし
\ruby{言}{ことば}によりて、
%
\ruby[g]{日方}{ひかた}が
\ruby{今}{いま}
\ruby{何}{なに}を
\ruby{云}{い}はんとするかを
\ruby[g]{水野}{みづの}は
\ruby{猜}{すゐ}し
\ruby{知}{し}れるなり。

\原本頁{229-7}%
\ruby{我}{われ}を
\ruby{思}{おも}ひ
\ruby{吳}{く}るゝ
\ruby{朋友}{ほう|いう}の
\ruby{眞{\換字{情}}}{ま|ごゝろ}より、
%
\ruby{我}{わ}が
\ruby{戀}{こひ}に
\ruby{惱}{なや}めるをは
\ruby{愚}{おろか}なりとして、
%
\ruby{說}{と}き
\ruby{醒}{さ}まし
\ruby{吳}{く}れんとする
\ruby{其人}{その|ひと}に
\ruby{對}{むか}ひては、
%
そも〳〵
\ruby{如何}{い|か}なる
\ruby{言葉}{こと|ば}をもて
\ruby{應}{こた}ふべきぞや。
%
\ruby{辯解}{いひ|わけ}すべき% 弁 瓣 辦 辧 辨 辩 (辯)
\ruby{事}{こと}にもあらず、
%
また
\ruby{本}{もと}より
\ruby{云}{い}ひ
\ruby{戾}{もど}くべき
\ruby{事}{こと}にもあらねば、
%
\ruby{愼}{つゝし}みて
\ruby{聞}{き}くよりほかの
\ruby{事}{こと}は
\ruby{無}{な}かるべし。
%
\原本頁{230-1}%
されど
\ruby{人}{ひと}の
\ruby{言葉}{こと|ば}を
\ruby{聞}{き}きて
\ruby{思}{おも}
ひ
\ruby{止}{と}まることの
\ruby{叶}{かな}ふほどならば、
%
\ruby{世}{よ}に
\ruby{戀}{こひ}に
\ruby{悶}{もだ}ゆるものは
\ruby{一人}{ひと|り}も
\ruby{無}{な}くて、
%
\ruby{他人}{ひ|と}に
\ruby{云}{い}はるゝまでもあらず
\ruby{先}{ま}づ
\ruby{我}{われ}と
\ruby{吾}{わ}が
\ruby{{\換字{分}}別}{ふん|べつ}に、
%
よしなき
\ruby{惑}{まどひ}は
\ruby{思}{おも}ひ
\ruby{斷}{き}るべきを、
%
\ruby{諦}{あきら}めても
\ruby{諦}{あきら}めても
\ruby{諦}{あきら}められぬにこそ
\ruby{生命}{いの|ち}の
\ruby{縮}{ちゞ}
むをも
\ruby{忘}{わす}れ
\ruby{人}{ひと}の
\ruby{謗}{そしり}をも
\ruby{顧}{かへり}みで
\ruby{惱}{なや}み
\ruby{苦}{くるし}みはするなれ。
%
それを
\ruby{如何}{い|か}に
\ruby{朋友}{ほう|いう}の
\ruby{眞}{まこと}の
\ruby{{\換字{情}}}{じやう}より
\ruby{{\換字{道}}理}{こと|わり}せめて
\ruby{云}{い}ひ
\ruby{諭}{さと}されたりとて、
%
\ruby{口}{くち}には
\ruby{思}{おも}ひ
\ruby{斷}{た}えたりとも
\ruby{云}{い}ふべし、
%
\ruby{心}{こゝろ}より
\ruby{全}{まつた}く
\ruby{改}{あらた}むる
\ruby{事}{こと}の
\ruby{何}{なん}として
\ruby{成}{な}るべき。
%
たゞ
\ruby{他人}{ひ|と}の
\ruby{親切}{しん|せつ}にて
\ruby{言}{い}ひ
\ruby{吳}{く}るゝ
\ruby{事}{こと}は、
%
よしや
\ruby{少}{すこ}しは
\ruby{無理}{む|り}なる
\ruby{{\換字{廉}}}{かど}ありとも
\ruby{受}{う}くべきが
\ruby{{\換字{道}}}{みち}なれば、
%
\ruby[g]{水野}{みづの}は
\ruby{頭}{かうべ}を
\ruby{垂}{た}れ
\ruby{肩}{かた}を
\ruby{窄}{すぼ}めて
\ruby{默々}{もく|〳〵}と、
%
\ruby{雨}{あめ}に
\ruby{濕}{ぬ}れたる
\ruby{鷄}{とり}の
\ruby{如}{ごと}く
\ruby[<h||]{力}{ちから}
\ruby{無}{な}げに、
%
\ruby{悄然}{せう|ぜん}と
\ruby[g]{日方}{ひかた}の
\ruby{云}{い}ふところをば
\ruby{聞}{き}かんとなしたり。

\原本頁{}%
\ruby[g]{日方}{ひかた}は
\ruby[g]{水野}{みづの}がしほらしき
\ruby{此態}{この|てい}を
\ruby{見}{み}てあはれを
\ruby{催}{もよほ}し、
%
\ruby{新}{あらた}にまた
\ruby{葡萄酒}{ぶ|だう|しゆ}の
\ruby{栓}{せん}を
\ruby{拔}{ぬ}きて、
%
\ruby[g]{水野}{みづの}が
\ruby{座}{ざ}の
\ruby{横}{よこ}に
\ruby{何時}{い|つ}か
\ruby{置}{お}かれたる
\ruby{酒盞}{さか|づき}に
\ruby{注}{つ}ぎ
\ruby{與}{や}りつ。

\原本頁{231-4}%
『しかしまあ
\ruby{其樣}{そ|ん}なに
\ruby{堅}{かた}くならんでも
\ruby{宜}{い}いは
\ruby[g]{水野}{みづの}。
%
\ruby{一杯}{いつ|ぱい}
\ruby{飮}{や}つて
\ruby{吳}{く}れ、
%
わざ〳〵
\ruby{持}{も}つて
\ruby{來}{き}たのだ。
%
\ruby{久}{ひさ}しぶりで
\ruby{汝}{きさま}と
\ruby{一緖}{いつ|しよ}に
\ruby{飮}{や}らうと
\ruby{思}{おも}つて、
%
\ruby[g]{島木}{しまき}のところから
\ruby{徴發}{ちよう|はつ}して
\ruby{來}{き}たのだ。
%
\ruby{何}{なに}も
\ruby{左樣}{さ|う}
\ruby{危坐}{かし|こま}つて
\ruby{貰}{もら}はんでも
\ruby{宜}{い}い、
%
\ruby{汝}{きさま}と
\ruby{乃公}{お|れ}との
\ruby{中}{なか}ぢや
\ruby{無}{な}いか。
\ruby{乃公}{お|れ}はサーベル
\ruby{三昧}{ざん|まい}、
%
\ruby{汝}{きさま}は
\ruby{書籍三昧}{ほ|ん|ざん|まい}、
%
たづさはる
\ruby{{\換字{道}}}{みち}が
\ruby{異}{ちが}ふので
\ruby{姑}{しばら}く
\ruby{{\換字{遠}}}{とほざ}かつたが、
%
\ruby{幾年}{いく|ねん}か
\ruby{{\換字{前}}}{まへ}は
\ruby{一}{ひと}ツに
\ruby{居}{ゐ}て、
%
\ruby{醉眠秋被}{すゐ|みん|あき|ひ}を
\ruby{共}{とも}にし、
%
\ruby{手}{て}を
\ruby{携}{たづさ}へて
\ruby{日}{ひ}に
\ruby{同行}{どう|かう}すといふ
\ruby{{\換字{古}}}{ふる}い
\ruby{詩}{し}の
\ruby{句}{く}の
\ruby{{\換字{通}}}{とほ}りを
\ruby{其儘}{その|まゝ}の
\ruby{境界}{きやう|かい}だナアと、
%
ソレ
\ruby{笑}{わら}ひ
\ruby{合}{あ}つた
\ruby{事}{こと}も
\ruby{有}{あ}つた
\ruby{中}{なか}だもの、
%
\ruby{{\換字{遠}}慮}{ゑん|りよ}も
\ruby{斟{\換字{酌}}}{しん|しやく}も
\ruby{有}{あ}らう
\ruby{筈}{はず}は
\ruby{無}{な}い。
%
さあ
\ruby{左樣}{さ|う}いふ
\ruby{中}{なか}だによつて
\ruby{默}{だま}つては
\ruby{居}{を}られんで、
%
\ruby{言語}{こと|ば}に
\ruby{艶}{つや}も
\ruby{付}{つ}けず
\ruby{露骨}{むき|だし}に
\ruby{云}{い}ふが、
%
\ruby[g]{水野}{みづの}!
\ruby{汝}{きさま}は
\ruby{何}{なん}で
\ruby{{\換字{情}}無}{なさけ|な}い
\ruby{{\換字{魔}}}{ま}に
\ruby{憑}{つ}かれた!。
%
\ruby{我々}{われ|〳〵}の
\ruby{中}{うち}で
\ruby{年}{とし}は
\ruby{{\換字{若}}}{わか}いが、
%
\ruby{聰明}{そう|めい}で
\ruby{慾}{よく}が
\ruby{寡}{すくな}くて
\ruby{學問}{がく|もん}が
\ruby{好}{すき}で、
%
\ruby{立派}{りつ|ぱ}な
\ruby{學者}{がく|しや}か
\ruby{詩仙}{し|せん}かにならうよりほかには
\ruby{爲}{な}りやうも
\ruby{無}{な}いと
\ruby{思}{おも}つて
\ruby{居}{ゐ}た
\ruby{汝}{きさま}が、
%
\ruby{此頃}{この|ごろ}の
\ruby{墮落}{だ|らく}の
\ruby{仕方}{し|かた}は
\ruby{何}{なん}といふ
\ruby{{\換字{情}}無}{なさけ|な}い
\ruby{態}{てい}だ。
%
\ruby{隱}{かく}してもいかん
\ruby{悉皆}{みん|な}
\ruby{知}{し}つて
\ruby{居}{ゐ}る。
%
\ruby{其}{そ}の
\ruby{顏}{かほ}の
\ruby{樵悴}{やつ|れ}は
\ruby{何}{なに}からの
\ruby{事}{こと}だ!。
%
\ruby{其}{そ}の
\ruby{身體}{から|だ}の
\ruby{枯稿}{や|せ}は
\ruby{何故}{なに|ゆゑ}の
\ruby{枯稿}{や|せ}だ。
%
\ruby{憫然}{かあ|いさう}に% 「憫然 か(あ)いさう」
\ruby{其樣}{そ|ん}なひがいすな
\ruby{身體}{から|だ}になつて
\ruby{何}{なに}が
\ruby{出來}{で|き}やう?。
%
\ruby{眼}{め}に
\ruby{見}{み}えるところさへ
\ruby{其{\換字{通}}}{その|とほ}りだもの、
%
まして
\ruby{心}{こゝろ}の
\ruby{{\換字{弱}}}{よわ}りは
\ruby{何程}{どれ|ほど}だらうと
\ruby{思}{おも}ひ
\ruby{{\換字{遣}}}{や}られて、
%
\ruby{汝}{きさま}のために
\ruby{涙}{なみだ}が
\ruby{出}{で}る、
%
\ruby{口惜}{くち|をし}くなる、
%
\ruby{腹}{はら}が
\ruby{立}{た}
つ!。
%
それも
\ruby{此}{これ}も
\ruby{時}{とき}の
\ruby{災人}{わざはひ|ゝと}の
\ruby{爲}{しわざ}の
\ruby{故}{せい}でもあればこそ、% 原本通り「せ(い)」
%
\ruby{汝}{きさま}の
\ruby{一心}{いつ|しん}の
\ruby{据}{す}ゑやうが
\ruby{惡}{わる}くて、
%
\ruby{高}{たか}の
\ruby{知}{し}れた
\ruby{一{\換字{婦}}人}{いち|ふ|じん}に
\ruby{氣}{き}を
\ruby{取}{と}られたからとは、
%
\ruby{{\換字{平}}生}{ひご|ろ}の
\ruby{汝}{きさま}にも
\ruby{似合}{に|あ}はん
\ruby{愚}{ぐ}な
\ruby{事}{こと}では
\ruby{無}{な}いか。
%
\ruby{{\換字{婦}}女}{をん|な}が
\ruby{何}{なん}だ!。
%
\ruby{戀}{こひ}が
\ruby{何}{なん}だ!。
%
たとひ
\ruby{美女}{び|ぢよ}だらうが
\ruby{賢女}{けん|ぢよ}だらうが、
%
\ruby{我}{われ}を
\ruby{{\換字{迷}}}{まよ}はせりやあ
\ruby{我}{われ}の
\ruby{仇敵}{かた|き}だ。
%
\ruby{男兒}{をと|こ}の
\ruby{正氣}{ほん|き}になつて
\ruby{働}{はたら}かうといふ
\ruby{事業}{し|ごと}の、
%
\ruby{障礙}{しやう|がい}になる
\ruby{奴}{やつ}あ
\ruby{悉皆}{みん|な}
\ruby{仇敵}{かた|き}だ。
%
\ruby{戀}{こひ}たあ
\ruby{料簡}{れう|けん}の
\ruby{弛}{ゆる}みへ
\ruby{出}{で}る
\ruby{黴}{かび}だ、
%
\ruby{閑暇}{ひ|ま}な
\ruby{馬鹿野郎}{ば|か|や|らう}の
\ruby{掌}{て}の
\ruby{中}{なか}の
\ruby[g]{玩弄物}{おもちや}だ。
%
\ruby{世間}{せ|けん}
\ruby{一體}{いつ|たい}の
\ruby{風}{ふう}とは
\ruby{云}{い}ひながら、
%
\ruby{新聞}{しん|ぶん}を
\ruby{見}{み}ても
\ruby{書籍}{ほ|ん}を
\ruby{見}{み}ても、
%
\ruby{戀}{こひ}だ
\ruby{董}{すみれ}だ
\ruby{蝶}{てふ}だ
\ruby{百合}{ゆ|り}だと、
%
\ruby{女臭}{をんな|くさ}いことばかり
\ruby{流行}{は|や}つて
\ruby{居}{ゐ}て、
%
まるで
\ruby{明治}{めい|じ}の
\ruby{{\換字{若}}}{わか}い
\ruby{奴}{やつ}は、
%
\ruby{戀}{こひ}をするために
\ruby{此}{こ}の
\ruby{世}{よ}の
\ruby{中}{なか}へ
\ruby{生}{うま}れて
\ruby{來}{き}たので、
%
\ruby{希望}{の|ぞみ}も
\ruby{事業}{し|ごと}も
\ruby{無}{な}いものゝやうだが、
%
\ruby[g]{水野}{みづの}!
\ruby{汝}{きさま}まで
\ruby{其風}{その|ふう}に
\ruby{感染}{か|ぶ}れたとは
\ruby{何}{なん}たる
\ruby{事}{こつ}た!。
%
\ruby{南風}{みな|み}が
\ruby{吹}{ふ}きやあ
\ruby{北}{きた}へ
\ruby{貼然}{べつ|たり}、
%
\ruby{{\換字{又}}}{また}
\ruby{北風}{き|た}が
\ruby{吹}{ふ}きやあ
\ruby{南}{みなみ}へ
\ruby{貼然}{べつ|たり}する、
%
\ruby{{\換字{平}}々凡々}{へい|〳〵|ぼん|〴〵}の
\ruby{草}{くさ}のやうに、
%
\ruby{自}{みづか}ら
\ruby{立}{た}つて
\ruby{居}{ゐ}る
\ruby{事}{こと}が
\ruby{出來}{で|き}ないとは
\ruby{見下}{み|さ}げた
\ruby{奴}{やつ}だナ。
%
\ruby{其樣}{そ|ん}な
\ruby{腰}{こし}の
\ruby{無}{な}い
\ruby{奴}{やつ}では
\ruby{無}{な}かつたが、
%
\ruby{汝}{きさま}も
\ruby{一世}{いつ|せ}の
\ruby{風潮}{ふう|てう}には
\ruby{捲}{ま}き
\ruby{倒}{たふ}されない
\ruby{男兒}{をと|こ}らしい
\ruby{男兒}{をと|こ}になりかねて、
%
\ruby{波}{なみ}に
\ruby{{\換字{随}}}{したが}ひ%「隨」TODO 変更 ⻖左円辶
\ruby{浪}{なみ}を
\ruby{{\換字{逐}}}{お}ふ
\ruby{意氣地}{い|く|ぢ}
\ruby{無}{な}しなつたか!。
』
