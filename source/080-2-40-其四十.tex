\Entry{其四十}

% メモ 校正終了 2024-04-30
\原本頁{234-5}%
『
\ruby{水野}{みづ|の}、
%
よもや
\ruby{汝}{きさま}は
まだ
\ruby{自{\換字{分}}}{じ|ぶん}で
\ruby{云}{い}つた
\ruby{事}{こと}を
\ruby{忘}{わす}れる
ほどに
\ruby{耄碌}{まう|ろく}は
\原本頁{234-6}\改行%
\ruby{爲}{し}まい。
%
\ruby{數年{\換字{前}}}{す|ねん|ぜん}に% 他では「すう..」となっているが原文通り「す」
\ruby{我々}{われ|〳〵}が
\ruby{寄}{よ}り
\ruby{合}{あ}つて、
%
\ruby{互}{たがひ}に
\ruby{抱負}{はう|ふ}を
\ruby{{\換字{述}}}{の}べて
\ruby{談笑}{だん|せふ}した% 原本通り「だんせ(ふ)」
\ruby{時}{とき}、
%
\ruby{大{\換字{丈}}夫}{だい|ぢやう|ぶ}の
\ruby{身}{み}をもつて
\ruby{詩{\換字{文}}}{し|ぶん}の
\ruby{小{\換字{技}}}{せう|ぎ}に
\ruby{身}{み}を
\ruby{委}{ゆだ}ね
やうとは
\ruby{何}{なん}の
\ruby{事}{こと}だ、
%
\ruby{雛蟲篆刻}{てう|ちう|てん|こく}% 詩文を作るのに、虫を彫り、
%%%%%%%%%%%%%%%%%%%%%%%%%%%%%%%%%%%%% 篆字を刻みつけるように、
%%%%%%%%%%%%%%%%%%%%%%%%%%%%%%%%%%%%% 細部まで技巧で飾りたてること。
%%%%%%%%%%%%%%%%%%%%%%%%%%%%%%%%%%%%% また、そのような技巧に走った内容のない文章。
%%%%%%%%%%%%%%%%%%%%%%%%%%%%%%%%%%%%% 転じて、取るに足らないつまらない小細工。
\ruby{壯夫}{さう|ふ}は%%%%%%%%%%%%%%% 壮年の男性。また、勇壮な男性。
\ruby{爲}{な}さずと、
%
\ruby{楊雄}{やう|ゆう}
づれで
さへ
\ruby{云}{い}つて
\ruby{居}{ゐ}る
のに、
%
\ruby{歌}{うた}の
ポエムの
と
\ruby{捏}{こ}ぬ
\ruby{{\換字{返}}}{かへ}して、
%
\ruby{食}{く}へ
もせず
\ruby{衣}{き}られ
もせぬものに
\ruby{苦勞}{く|らう}
しやうとは、
%
\ruby{{\換字{道}}樂}{だう|らく}
\ruby{{\換字{過}}}{す}ぎて
\ruby{餘}{あま}り
\ruby{詰}{つま}らぬ
と、
%
\ruby{乃公}{お|れ}が
\ruby{口}{くち}を
\ruby{極}{きは}めて
\ruby{非{\換字{難}}}{ひ|なん}したらば、
%
\ruby{今}{いま}と
\ruby{異}{ちが}つて
\ruby{元氣}{げん|き}の
あつた
\ruby{其}{その}
\ruby{頃}{ころ}の
\ruby{汝}{きさま}は、
%
\ruby{眉}{まゆ}を
\ruby{昻}{あ}げ
\ruby{面}{おもて}を
\ruby{正}{たゞし}くして% TODO 原本の「二の字点、揺すり点」に濁点のグリフが見つからないので「ゞ」
\ruby{凛然}{りん|ぜん}と
\ruby{答}{こた}へた
\ruby{其}{そ}の
\ruby{挨拶}{あい|さつ}に
\ruby{何}{なん}と
\ruby{云}{い}つた!。
%
\ruby{食}{しよく}は
\ruby{身}{み}
の
\ruby{糧}{かて}、
%
\ruby{詩}{し}は
\ruby{心}{こ〻ろ}% 原本通り「〻(二の字点、揺すり点)」
の
\ruby{糧}{かて}、
%
\ruby{衣}{きもの}は
\ruby{暑}{あつ}さ
\ruby{{\換字{寒}}}{さむ}さに
\ruby{對}{たい}して
\ruby{人}{ひと}の
\ruby{身}{み}を
\ruby{護}{まも}り、
%
\ruby{詩}{し}は
\原本頁{235-4}\改行%
\ruby[||j>]{悲}{かなし}みにも
\ruby{怒}{いか}りにも
\ruby{對}{むか}つて
\ruby{人}{ひと}の
\ruby{心}{こ〻ろ}を% 原本通り「〻(二の字点、揺すり点)」
\ruby{調}{と〻の}へる、% 原本通り「〻(二の字点、揺すり点)」
%
それを
\ruby{益}{{\換字{𛀁}}き}の
\ruby{無}{な}い
もののやうに
\ruby{云}{い}ふは
\ruby{淺}{あさ}ましい
\ruby{{\換字{誤}}謬}{あや|まり}。
%
\ruby{貝}{かひ}に
\ruby{眞珠}{しん|じゆ}あり、
%
\ruby{人}{ひと}に
\ruby{詩}{し}あり、
%
\ruby{詩歌}{し|か}を
\ruby{除}{のぞ}きて
\ruby{人}{ひと}の
\ruby{作}{つく}れる
ものに、
%
\ruby{野菊}{の|ぎく}の
\ruby{花}{はな}の
\ruby{一輪}{いち|りん}
だけの
\ruby{美}{うつく}しさ
の
\原本頁{235-7}\改行%
あるものも
\ruby{無}{な}く、
%
\ruby{阿{\換字{房}}}{あ|ぼう}
\ruby{威陽}{かん|やう}
% 阿房宮 秦の始皇帝が現在の阿房宮村に建設した宮殿
%     秦帝国の首都であった咸陽からは渭水をはさんで南側に位置
% 威陽  戦国時代の前350年、秦の孝公が渭水流域の関中に築いた都
は
\ruby{羞}{はづか}しく
\ruby{醜}{みにく}い。
%
\ruby{美}{うつく}しき
\ruby{胸}{むね}の
\ruby{働}{はたら}きの
\ruby{目}{め}にも
\ruby{見}{み}えぬが、
%
\ruby{凝}{こ}つて
\ruby{詩}{し}と
なつて
\ruby{{\換字{文}}字}{もん|じ}に
\ruby{現}{あらは}る
れば、
%
\ruby{讀}{よ}むもの
\ruby[<j||]{恍}{くわう}
\ruby{惚}{こつ}として
\ruby{我}{われ}を
\ruby{忘}{わす}れて、
%
\ruby{作}{つく}る
\ruby{人}{ひと}が
\ruby{泣}{な}けば
\ruby{泣}{な}き、
%
\ruby{憤}{いか}れば
\ruby{憤}{いか}る。
%
されば
\ruby{人間}{ひ|と}の
\ruby{性}{せい}
\ruby[||j>]{{\換字{情}}}{じやう}を
\ruby{敦}{あつ}く
し、
%
\ruby{世}{よ}の
\ruby{氣風}{き|ふう}を
\ruby{嘉}{よ}く
するもの、
%
\ruby{詩}{し}に
\ruby{越}{こ}すものは
\ruby{無}{な}い。
%
\ruby{大言}{たい|げん}の
やうだが
\ruby{此}{こ}の
\ruby{水野}{みづ|の}は、
%
たゞ% TODO 原本の「二の字点、揺すり点」に濁点のグリフが見つからないので「ゞ」
\ruby{蝶花}{てふ|はな}の
おもしろさや
\原本頁{236-1}\改行%
\ruby{月露}{げつ|ろ}の
あはれさを
\ruby{歌}{うた}つて
のみ
\ruby{我}{わ}が
\ruby[||j>]{一}{いつ}
\ruby[||j>]{生}{しやう}を
% \ruby{一生}{いつ|しやう}を
\ruby{{\換字{過}}}{すご}さん
とは
\ruby{仕}{し}ない。
%
\ruby{百年千年}{ひやく|ねん|せん|ねん}
にして
\ruby{一}{ひ}ト
\ruby{度}{たび}
\ruby{出}{い}づる
\ruby{大詩人}{だい|し|じん}の、
%
\ruby{一代}{いち|だい}の
\ruby{人心}{じん|〳〵}を
\ruby{新}{あらた}にして、
%
\原本頁{236-3}\改行%
\ruby{萬世}{ばん|せい}に
\ruby{天意}{てん|い}の
\ruby{眞}{まこと}を
\ruby{傳}{つた}へん
とする、
%
\ruby{其}{それ}は
\ruby{及}{およ}ばざる
\ruby{願}{ねがひ}
にもせよ、
%
\ruby{時勢}{じ|せい}の
\ruby{幇間}{ほう|かん}と
なつて
\ruby{徳}{とく}を
\ruby{頌}{しよう}する
やうな
\ruby{賤}{いや}しい
\ruby{意}{こ〻ろ}は% 原本通り「〻(二の字点、揺すり点)」
\ruby{微塵}{み|ぢん}も
\ruby{有}{も}たない。
%
\原本頁{236-5}\改行%
\ruby{長}{なが}い
\ruby{眼}{め}で
\ruby{見}{み}て
\ruby{居}{ゐ}て
\ruby{吳}{く}れ
たまへ、
%
\ruby{此}{こ}の
\ruby{水野}{みづ|の}は
たとひ
\ruby{世}{よ}に
\ruby{背}{そむ}いても、
\原本頁{236-6}\改行%
\ruby{世}{よ}と
\ruby{爭}{あらそ}つても、
%
\ruby{屹度}{きつ|と}
\ruby{血}{ち}も
ある
\ruby{涙}{なみだ}も
ある
\ruby{詩}{し}を
\ruby{作}{つく}つて、
%
\ruby{聖代}{せい|だい}に
\ruby{生}{うま}れ
\ruby{合}{あ}はせた
\ruby{男兒}{をと|こ}
\ruby{一人}{ひと|り}
だけの、
%
\ruby{任務}{つと|め}は
\ruby{其}{それ}で
\ruby{果}{はた}す
つもり
だと、
%
さも
\ruby{潔}{いさぎ}よく
\ruby{言}{い}つたでは
\ruby{無}{な}いか。
%
%
\ruby{其}{そ}の
\ruby{意氣}{い|き}は
\ruby{今}{いま}
\ruby{何處}{ど|こ}へ
\ruby{無}{な}く
した?。
%
\ruby{其}{そ}の
\ruby{言葉}{こと|ば}は
\ruby{既}{もう}
\ruby{忘}{わす}れ
\ruby{果}{は}て
たか。
%
ヤイ
\ruby{水野}{みづ|の}!。
%
\ruby{詩}{し}の
\ruby{一篇}{いつ|ぺん}も
\ruby{作}{つく}らう
といふものが、
%
\ruby{現在}{げん|ざい}の
\ruby{人{\換字{情}}世態}{にん|じやう|せ|たい}に
\ruby{眼}{め}は
\ruby{離}{はな}す
まいが、
%
\ruby{今}{いま}の
\ruby{日本}{に|ほん}の
\ruby{狀態}{あり|さま}を
\原本頁{236-11}\改行%
\ruby{何樣}{ど|う}
\ruby{思}{おも}ふ?\inhibitglue{}%
%
\ruby{汝}{きさま}!。
%
\ruby{今}{いま}の
\ruby{世界}{せ|かい}の
\ruby{狀態}{あり|さま}を
\ruby{何樣}{ど|う}
おもふ?\inhibitglue{}%
%
\ruby{汝}{きさま}!。
%
\ruby{浪}{なみ}の
\ruby{立}{た}たない
\ruby{海}{うみ}も
\ruby{無}{な}ければ、
%
\ruby{風}{かぜ}の
\ruby{荒}{あ}れない
\ruby{{\換字{空}}}{そら}も
\ruby{無}{な}くつて、
%
\ruby{國}{くに}は
\ruby{國}{くに}と
\ruby{競}{せ}り
\ruby{合}{あ}ひ、
%
\ruby{人種}{じん|しゆ}は
\ruby{人種}{じん|しゆ}と
\ruby{鬪}{た〻か}ふ、% 原本通り「〻(二の字点、揺すり点)」
%
\ruby{世界}{せ|かい}の
\ruby{浪}{なみ}
\ruby{風}{かぜ}は
\ruby{轟々}{がう|〳〵}として、
%
\ruby{我}{わ}が
\ruby{國}{くに}の
\ruby{濱}{はま}へも
\ruby{磯}{いそ}へも
\ruby{寄}{よ}せて
\ruby{來}{き}て
\ruby{居}{ゐ}るでは
\ruby{無}{な}いか。
%
それだのに
\ruby{國内}{こく|ない}の
\原本頁{237-4}\改行%
\ruby{狀態}{あり|さま}は
\ruby{何樣}{ど|う}だ。
%
\ruby{武士{\換字{道}}}{ぶ|し|だう}は
\ruby{廢}{すた}り
\ruby{儒敎}{じゆ|けう}は
\ruby{棄}{す}てられ、
%
\ruby{舊}{ふる}い
\ruby{敎}{をしへ}は
\ruby{壞}{こは}れ
\ruby{果}{は}てたが、
%
\ruby{眞面目}{ま|じ|め}に
\ruby{受}{う}け
\ruby{入}{い}れ
られた
\ruby{新}{あたら}しい
\ruby{敎}{をしへ}も
\ruby{無}{な}く、
%
\ruby{{\換字{過}}去帳}{か|こ|ちやう}を
\ruby{讀}{よ}むやうに
\ruby{哲人}{てつ|じん}の
\ruby{名}{な}
ばかりは
\ruby{忙}{せは}しく
\ruby{呼}{よび}
\ruby{立}{た}てられて、
%
やがて
\ruby{直}{すぐ}
\ruby[||j>]{片}{かた}
\ruby[||j>]{端}{つぱし}
\原本頁{237-7}\改行%
から
\ruby{忘}{わす}れて
\ruby{行}{ゆ}かれる!。
%
\ruby{社}{しや}
\ruby[||j>]{會}{くわい}に
\ruby{善惡}{ぜん|あく}の
\ruby{目安}{め|やす}が
\ruby{無}{な}いから、
%
\ruby{{\換字{勝}}手}{かつ|て}
\ruby{次第}{し|だい}の
\ruby{{\換字{強}}}{つよ}い
もの
\ruby{{\換字{勝}}}{がち}、
%
\ruby{智慧}{ち|ゑ}で
\ruby[<j||]{爭}{あらそ}ふ、
%
\ruby{言說}{く|ち}で
\ruby[<j||]{爭}{あらそ}ふ、
%
\ruby{筆}{ふで}で
\ruby[<j||]{爭}{あらそ}ふ、
%
\ruby{金}{かね}で
\ruby[<j||]{爭}{あらそ}ふ、
%
しかし
\ruby{{\換字{道}}理}{だう|り}で
\ruby[<j||]{爭}{あらそ}つた
のを
\ruby{聞}{き}いた
\ruby{事}{こと}が
\ruby{無}{な}い。
%
\ruby{金}{かね}を
\ruby{欲}{ほ}しがる、
%
\原本頁{237-10}\改行%
\ruby{權威}{けん|ゐ}を
\ruby{欲}{ほ}しがる、
%
\ruby{名}{な}を
\ruby{欲}{ほ}しがる、
%
\ruby{肉慾}{にく|よく}の
\ruby{滿足}{まん|ぞく}を
\ruby{欲}{ほ}しがる、
%
しかし
\ruby{徳}{とく}を
\ruby{欲}{ほ}しがる
ものは
\ruby{藥}{くすり}に
\ruby{仕度}{し|たく}も
\ruby{無}{な}い。
%
\ruby{坊主}{ばう|ず}が
\ruby{役}{やく}
\ruby{立}{た}たん、
%
\ruby{新聞}{しん|ぶん}
\原本頁{238-1}\改行%
\ruby{記者}{き|しや}が
\ruby{頼}{たの}もしく
\ruby{無}{な}い、
%
\ruby{敎育家}{けう|いく|か}が
\ruby{下}{くだ}らん、
%
\ruby{學者}{がく|しや}は
\ruby{學說}{がく|せつ}の
\ruby{桂庵}{けい|あん}
% 1 縁談や訴訟の仲立ちをする人。
%  また、雇い人・奉公人の 斡旋 あっせん を職業とする人。口入れ屋。
% 2 お世辞。 追従 ついしょう 。また、それを言う人。
ばかりで、
%
\ruby{{\換字{文}}學者}{ぶん|がく|しや}は
\ruby{春枝}{はる|{\換字{𛀁}}}さん
\ruby{靜枝}{しづ|{\換字{𛀁}}}さんの
\ruby{御機{\換字{嫌}}}{ご|き|げん}
\ruby{取}{と}りに
\ruby{{\換字{過}}}{す}ぎん。
%
\ruby{世間}{せ|けん}
\原本頁{238-3}\改行%
\ruby{一體}{いつ|たい}は
\ruby{全}{まる}で
\ruby{不調子}{ふ|てう|し}で、
%
\ruby{錢}{ぜに}の
ある
\ruby{時}{とき}は
ハイカラになり、
%
\ruby{錢}{ぜに}の
\ruby{無}{な}い
\ruby{時}{とき}は
\ruby{蠻}{ばん}カラ、
%
\ruby{忰}{せがれ}は
\ruby{戀愛}{れん|あい}
\ruby{論}{ろん}、
%
\ruby{親{\換字{父}}}{おや|ぢ}は
\ruby{料理}{れう|り}
\ruby{談}{だん}、
%
\ruby{滔々}{たう|〳〵}% 物事が一つの方向へよどみなく流れ向かうさま。
として
\ruby{一般}{いつ|ぱん}の
\ruby{趣味}{しゆ|み}は
\ruby{日}{ひ}に
\ruby{墮落}{だ|らく}して
\ruby{居}{ゐ}る。
%
\ruby{想}{おも}つても
\ruby{恐}{おそ}ろしい
\ruby{世界}{せ|かい}の
ありさま、
%
\ruby{見}{み}る
さへ
\ruby{{\換字{嫌}}}{いや}な
\ruby{人}{にん}
\ruby[||j>]{{\換字{情}}}{じやう}の
\ruby{調子}{てう|し}、
%
\ruby{彼}{あれ}と
\ruby{此}{これ}とを
\ruby{思}{おも}ひ
\ruby{合}{あ}はせれば、
%
\ruby{此}{こ}の
\ruby{無骨}{ぶ|こつ}
\原本頁{238-7}\改行%
\ruby{不風流}{ぶ|ふう|りう}の
\ruby{乃公}{お|れ}
でさへも、
%
\ruby{無限}{む|げん}の
\ruby{{\換字{感}}慨}{かん|がい}に
\ruby{打}{う}たれて、
%
\ruby{詩}{し}の
やうなものが
\ruby{呻}{うめ}き
\ruby{出}{だ}したく
なる、
%
まして
\ruby{汝}{きさま}が
\ruby{{\換字{感}}慨}{かん|がい}の
\ruby{無}{な}いわ
けは
\ruby{有}{あ}る
まいに
\ruby{何故}{な|ぜ}
\ruby{一片}{いつ|ぺん}
\ruby{耿々}{かう|〳〵}% 1 光が明るく輝くさま。 2 気にかかることがあって、心が安らかでないさま
たる
\ruby{神州}{しん|しう}
\ruby{男兒}{だん|じ}の
\ruby{丹心}{たん|しん}% まごころ。赤心。丹情。丹誠。丹地。
から、
%
\ruby{國}{くに}を
\ruby{愛}{あい}し
\ruby{世}{よ}を
\ruby{憂}{うれ}ふる
の
\ruby{誠}{まこと}を
\ruby{披瀝}{ひ|れき}
して、
%
\ruby{詩}{し}
でも
\ruby[||j>]{{\換字{文}}}{ぶん}
\ruby[||j>]{章}{しやう}
% \ruby{{\換字{文}}章}{ぶん|しやう}
でも
\ruby{作}{つく}り
\ruby{出}{だ}して
\ruby{吳}{く}れぬ?。
%
\ruby{手}{て}
\ruby{{\換字{緩}}}{ぬる}い
\原本頁{238-11}\改行%
\ruby{事}{こと}では
\ruby{無}{な}い、
%
\ruby{今}{いま}の
\ruby{今}{いま}でも
\ruby{國{\換字{運}}}{こく|うん}を
\ruby{賭}{と}して
\ruby{戰爭}{た〻|かひ}を% 原本通り「〻(二の字点、揺すり点)」
\ruby{始}{はじ}めれば、
さしづめ
\ruby{乃公}{お|れ}たちは
\ruby{水火}{すゐ|くわ}の
\ruby{中}{なか}にも
\ruby{飛}{と}びこまねば
ならぬ
\ruby{時}{とき}に
\ruby{逼}{せま}つて
\ruby{居}{ゐ}る
\原本頁{239-2}\改行%
\ruby{塲合}{ば|あひ}だ。% 原文通り「塲」
%
しかし
\ruby{詩}{し}は
\ruby{興}{きよう}が
\ruby{發}{はつ}しない
と
\ruby{云}{い}へば
それまでの
\ruby{事}{こと}、
%
\ruby{出來}{で|き}んなら
\ruby{出來}{で|き}んで
\ruby{是非}{ぜ|ひ}は
\ruby{無}{な}いが、
%
\ruby{汝}{きさま}
までが
\ruby{世}{よ}の
\ruby{風}{ふう}に
\ruby{負}{ま}けて
\ruby{戀愛}{れん|あい}
\ruby{騷}{さわ}ぎ
を
するとは
\ruby{何}{なに}
\ruby{事}{ごと}
だ。
%
そんな
\ruby{柔}{にう}
\ruby[||j>]{{\換字{弱}}}{じやく}
な、
%
\ruby{性根}{しやう|ね}の
\ruby{拔}{ぬ}けた
\ruby{事}{こと}で、
%
\ruby{何}{なん}の
\原本頁{239-5}\改行%
\ruby{詩}{し}も
\ruby{歌}{うた}も
あつたものか。
%
\ruby{時勢}{じ|せい}の
\ruby{幇間}{ほう|かん}% 宴席などで客の機嫌をとり、酒宴の興を助けるのを職業とする男。 太鼓持ち。 男芸者。
とならぬと
\ruby{云}{い}つた
\ruby{其}{そ}の
\ruby{意氣}{い|き}は
\ruby{今}{いま}
どこに
\ruby{在}{あ}る?。
%
\ruby{正}{まさ}しく
\ruby{汝}{きさま}は
\ruby{時勢}{じ|せい}の
%
\ruby{幇間}{ほう|かん}となつた、
%
\ruby{奴隷}{ど|れい}となつた、
%
\ruby{狗}{いぬ}となつた!。
%
\ruby{男子}{だん|し}の
\ruby{眞}{まこと}の
\ruby{心}{こ〻ろ}を% 原本通り「〻(二の字点、揺すり点)」
\ruby{失}{うしな}つた。
%
\ruby{男心}{を|ご〻ろ}も% 原本通り「〻(二の字点、揺すり点)」
\ruby{無}{な}い
\ruby{白痴}{たは|け}
に
なつたナ。
%
\ruby{戀}{こひ}の
\ruby{奴}{やつこ}と
\ruby{我}{われ}は
\ruby{死}{し}ぬ
べし
とは
\ruby{何}{なん}たる
\ruby{事}{こと}だ。
%
\ruby{此}{こ}の
\ruby{普門品}{ふ|もん|ぼん}は
\ruby{誰}{だれ}が
\ruby{誦}{よ}んで、
%
\ruby{其}{そ}の
\ruby{下}{くだ}らん
\ruby{御籤}{み|くじ}
と
いふものは
\ruby{誰}{だれ}が
\ruby{抽}{と}つた?。
%
\原本頁{239-10}\改行%
ちらりと
\ruby{聞}{き}けば
\ruby[<j||]{觀}{くわん}
\ruby{音}{のん}% 「觀音」の読みは原本通り「くわん(の)ん」
\ruby[||j>]{詣}{まうで}
して、
%
\ruby{而}{さう}して
\ruby{纔}{やつ}と
\ruby{今}{いま}
\ruby{歸}{かへ}つて
\ruby{來}{き}たのだナ。
%
\ruby[||j>]{汝}{きさま}が
\ruby{思}{おも}つて
\ruby{居}{ゐ}る
\ruby{女}{をんな}が
\ruby{大}{たい}
\ruby[||j>]{病}{びやう}だ
とか
いふ
\ruby{島木}{しま|き}の
\ruby{談話}{はな|し}も
\ruby{思}{おも}ひ
\ruby{合}{あ}はせて、
%
\原本頁{240-1}\改行%
すつかり
\ruby{汝}{きさま}の
\ruby{{\換字{所}}業}{し|わざ}は
\ruby{{\換字{分}}}{わか}つたが、
%
\ruby{女}{をんな}の
ために
\ruby{經}{きやう}を
\ruby{誦}{よ}んだり、
%
\ruby{御籤}{み|くじ}を
\ruby{取}{と}つたり、
%
わざ〳〵
\ruby{淺草}{あさ|くさ}まで
\ruby{歩}{あゆみ}を
\ruby{{\換字{運}}}{はこ}んだり
して
\ruby{居}{ゐ}るのだナ。
%
\原本頁{240-3}\改行%
エーツ
\ruby{{\換字{情}}無}{なさけ|な}くも
\ruby{衰}{おとろ}へに
\ruby{衰}{おとろ}へた
\ruby{奴}{やつ}だ。
%
\ruby{書}{しよ}も
\ruby{讀}{よ}み
\ruby{理}{り}にも
\ruby{眛}{くら}からぬ
\ruby{水野}{みづ|の}
とも
ある
もの
が、
%
\ruby{如何}{い|か}に
\ruby{{\換字{迷}}}{まよ}へば
とて
\ruby{一}{いつ}
\ruby{{\換字{婦}}人}{ぷ|じん}
の
ために、
%
それほども
\ruby{愚}{ぐ}に
なつて、
%
\ruby{成}{な}りきつたか。
%
\ruby{{\換字{魔}}}{ま}に
\ruby{憑}{つ}かれたか
\ruby{何}{なに}に
\ruby{憑}{つ}かれたか、
%
\ruby{全然}{まる|で}
\ruby{正氣}{しやう|き}の
\ruby{沙汰}{さ|た}
では
\ruby{無}{な}いが、
%
\ruby{男兒}{をと|こ}の
\ruby{魂魄}{たま|しひ}が
\ruby{少許}{すこ|し}
でも
あれば、
%
\ruby{正氣}{しやう|き}に
\ruby{{\換字{返}}}{かへ}れ、
%
\ruby{正氣}{しやう|き}に
\ruby{仕}{し}て
やらう。
%
\ruby{目}{め}を
\ruby{覺}{さ}ませ
\ruby{水野}{みづ|の}。
』

\原本頁{240-8}%
と
\ruby{云}{い}ひ
さまに、
%
\ruby{普門品}{ふ|もん|ぼん}を
\ruby{右手}{みぎ|て}に
\ruby{鷲握}{わし|づか}み
にして、
%
\ruby{左手}{ひだり|て}に
\ruby{水野}{みづ|の}を
\ruby{取}{と}つて
\ruby{引伏}{ひき|ふ}せ、

\原本頁{240-10}%
『
\ruby{{\換字{情}}無}{なさけ|な}い
\ruby{奴}{やつ}だ!。
%
\ruby{正氣}{しやう|き}に
\ruby{{\換字{返}}}{かへ}らんか、
%
\ruby{朋友}{とも|だち}の
\ruby{{\換字{情}}誼}{なさ|け}だ、
%
\ruby{身}{み}に
\ruby{染}{し}みて
\ruby{受}{う}けろ。
』

\原本頁{241-1}%
と
ピシリ〳〵
と
\ruby{續}{つゞ}け% TODO 原本の「二の字点、揺すり点」に濁点のグリフが見つからないので「ゞ」
さまに
\ruby{打}{う}つたり。
