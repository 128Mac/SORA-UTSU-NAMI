\Entry{其四十}

% メモ 校正終了 2024-04-30 2024-06-04
\原本頁{234-5}%
『
\ruby[g]{水野}{みづの }、
%
よもや
\ruby{汝}{きさま}は
まだ
\ruby[g]{自{\換字{分}}}{じ ぶん}で
\ruby{云}{い}つた
\ruby{事}{こと}を
\ruby{忘}{わす}れる
ほどに
\ruby[g]{耄碌}{まうろく}は
\原本頁{234-6}\改行%
\ruby{爲}{し}まい。
%
\ruby{數年{\換字{前}}}{す|ねん|ぜん}に% 他では「すう..」となっているが原文通り「す」
\ruby[g]{我々}{われ〳〵}が
\ruby{寄}{よ}り
\ruby{合}{あ}つて、
%
\ruby{互}{たがひ}に
\ruby[g]{抱負}{はうふ }を
\ruby{{\換字{述}}}{の}べて
\ruby[g]{談笑}{だんせふ}した% ルビ調整(原本通り)「だんせ(ふ)」
\ruby{時}{とき}、
%
\ruby{大{\換字{丈}}夫}{だい|ぢやう|ぶ}の
\ruby{身}{み}をもつて
\ruby[g]{詩{\換字{文}}}{し ぶん}の
\ruby[g]{小{\換字{技}}}{せうぎ }に
\ruby{身}{み}を
\ruby{委}{ゆだ}ね
やうとは
\ruby{何}{なん}の
\ruby{事}{こと}だ、
%
\ruby{雛蟲篆刻}{てう|ちう|てん|こく}% 詩文を作るのに、虫を彫り、
%%%%%%%%%%%%%%%%%%%%%%%%%%%%%%%%%%%%% 篆字を刻みつけるように、
%%%%%%%%%%%%%%%%%%%%%%%%%%%%%%%%%%%%% 細部まで技巧で飾りたてること。
%%%%%%%%%%%%%%%%%%%%%%%%%%%%%%%%%%%%% また、そのような技巧に走った内容のない文章。
%%%%%%%%%%%%%%%%%%%%%%%%%%%%%%%%%%%%% 転じて、取るに足らないつまらない小細工。
\ruby[g]{壯夫}{さうふ }は%%%%%%%%%%%%%%% 壮年の男性。また、勇壮な男性。
\ruby{爲}{な}さずと、
%
\ruby[g]{楊雄}{やうゆう}% 楊雄(ようゆう)は、中国の小説で四大奇書の一つである『水滸伝』の登場人物。
づれで
さへ
\ruby{云}{い}つて
\ruby{居}{ゐ}る
のに、
%
\ruby{歌}{うた}の
ポエムの
と
\ruby{捏}{こ}ぬ
\ruby{{\換字{返}}}{かへ}して、
%
\ruby{食}{く}へ
もせず
\ruby{衣}{き}られ
もせぬものに
\ruby[g]{苦勞}{く らう}
しやうとは、
%
\ruby[g]{{\換字{道}}樂}{だうらく}
\ruby{{\換字{過}}}{す}ぎて
\ruby{餘}{あま}り
\ruby{詰}{つま}らぬ
と、
%
\ruby[g]{乃公}{お れ }が
\ruby{口}{くち}を
\ruby{極}{きは}めて
\ruby[g]{非{\換字{難}}}{ひ なん}したらば、
%
\ruby{今}{いま}と
\ruby{異}{ちが}つて
\ruby[g]{元氣}{げんき }の
あつた
\ruby{其}{その}
\ruby{頃}{ころ}の
\ruby{汝}{きさま}は、
%
\ruby{眉}{まゆ}を
\ruby{昻}{あ}げ
\ruby{面}{おもて}を
\ruby{正}{たゞし}くして% 踊り字調整「〻(二の字点、揺すり点)に濁点に見えるが(ゞ)」
\ruby[g]{凛然}{りんぜん}と
\ruby{答}{こた}へた
\ruby{其}{そ}の
\ruby[g]{挨拶}{あいさつ}に
\ruby{何}{なん}と
\ruby{云}{い}つた!。
%
\ruby{食}{しよく}は
\ruby{身}{み}
の
\ruby{糧}{かて}、
%
\ruby{詩}{し}は
\ruby{心}{こゝろ}% 踊り字調整「〻(二の字点、揺すり点)に見えるが(ゝ)」
の
\ruby{糧}{かて}、
%
\ruby{衣}{きもの}は
\ruby{暑}{あつ}さ
\ruby{{\換字{寒}}}{さむ}さに
\ruby{對}{たい}して
\ruby{人}{ひと}の
\ruby{身}{み}を
\ruby{護}{まも}り、
%
\ruby{詩}{し}は
\原本頁{235-4}\改行%
\ruby[||j>]{悲}{かなし}みにも
\ruby{怒}{いか}りにも
\ruby{對}{むか}つて
\ruby{人}{ひと}の
\ruby{心}{こゝろ}を% 踊り字調整「〻(二の字点、揺すり点)に見えるが(ゝ)」
\ruby{調}{とゝの}へる、% 踊り字調整「〻(二の字点、揺すり点)に見えるが(ゝ)」
%
それを
\ruby{益}{{\換字{𛀁}}き}の
\ruby{無}{な}い
もののやうに
\ruby{云}{い}ふは
\ruby{淺}{あさ}ましい
\ruby[g]{{\換字{誤}}謬}{あやまり}。
%
\ruby{貝}{かひ}に
\ruby[g]{眞珠}{しんじゆ}あり、
%
\ruby{人}{ひと}に
\ruby{詩}{し}あり、
%
\ruby[g]{詩歌}{し か }を
\ruby{除}{のぞ}きて
\ruby{人}{ひと}の
\ruby{作}{つく}れる
ものに、
%
\ruby[g]{野菊}{の ぎく}の
\ruby{花}{はな}の
\ruby[g]{一輪}{いちりん}
だけの
\ruby{美}{うつく}しさ
の
\原本頁{235-7}\改行%
あるものも
\ruby{無}{な}く、
%
\ruby[g]{阿{\換字{房}}}{あ ぼう}
\ruby[g]{威陽}{かんやう}
% 阿房宮 秦の始皇帝が現在の阿房宮村に建設した宮殿
%     秦帝国の首都であった咸陽からは渭水をはさんで南側に位置
% 威陽  戦国時代の前350年、秦の孝公が渭水流域の関中に築いた都
は
\ruby{羞}{はづか}しく
\ruby{醜}{みにく}い。
%
\ruby{美}{うつく}しき
\ruby{胸}{むね}の
\ruby{働}{はたら}きの
\ruby{目}{め}にも
\ruby{見}{み}えぬが、
%
\ruby{凝}{こ}つて
\ruby{詩}{し}と
なつて
\ruby[g]{{\換字{文}}字}{もんじ }に
\ruby{現}{あらは}る
れば、
%
\ruby{讀}{よ}むもの
\ruby[<j||]{恍}{くわう}
\ruby{惚}{こつ}として
\ruby{我}{われ}を
\ruby{忘}{わす}れて、
%
\ruby{作}{つく}る
\ruby{人}{ひと}が
\ruby{泣}{な}けば
\ruby{泣}{な}き、
%
\ruby{憤}{いか}れば
\ruby{憤}{いか}る。
%
されば
\ruby[g]{人間}{ひ と }の
\ruby{性}{せい}
\ruby[||j>]{{\換字{情}}}{じやう}を
\ruby{敦}{あつ}く
し、
%
\ruby{世}{よ}の
\ruby[g]{氣風}{き ふう}を
\ruby{嘉}{よ}く
するもの、
%
\ruby{詩}{し}に
\ruby{越}{こ}すものは
\ruby{無}{な}い。
%
\ruby[g]{大言}{たいげん}の
やうだが
\ruby{此}{こ}の
\ruby[g]{水野}{みづの }は、
%
たゞ% 踊り字調整「〻(二の字点、揺すり点)に濁点に見えるが(ゞ)」
\ruby[g]{蝶花}{てふはな}の
おもしろさや% この行 29 文字あり。句読点の高さが違うからだと思う
\原本頁{236-1}\改行%
\ruby[g]{月露}{げつろ }の
あはれさを
\ruby{歌}{うた}つて
のみ
\ruby{我}{わ}が
\ruby[g]{一生を}{いつしやう }
% \ruby{一生}{いつ|しやう}を
\ruby{{\換字{過}}}{すご}さん
とは
\ruby{仕}{し}ない。
%
\makeatletter
\@ifundefined{デバッグ@ビルド}{%
  \ruby[||j>]{百}{ひやく}
  \ruby[||j>]{年}{ ねん}
  \ruby[||j>]{千}{ せん}
  \ruby[||j>]{年}{ ねん}
}{%
  \ruby[<j||]{百}{ひやく}
  \ruby{年}{ねん}
  \ruby[g]{千年}{せんねん}
}%
\makeatother
にして
\ruby{一}{ひ}ト
\ruby{度}{たび}
\ruby{出}{い}づる
\ruby{大詩人}{だい|し|じん}の、
%
\ruby[g]{一代}{いちだい}の
\ruby[g]{人心}{じん〳〵}を
\ruby{新}{あらた}にして、
%
\原本頁{236-3}\改行%
\ruby[g]{萬世}{ばんせい}に
\ruby[g]{天意}{てんい }の
\ruby{眞}{まこと}を
\ruby{傳}{つた}へん
とする、
%
\ruby{其}{それ}は
\ruby{及}{およ}ばざる
\ruby{願}{ねがひ}
にもせよ、
%
\ruby{時}{じ}
\原本頁{236-4}\改行%
\ruby{勢}{せい}の
\ruby[g]{幇間}{ほうかん}と
なつて
\ruby{徳}{とく}を
\ruby{頌}{しよう}する
やうな
\ruby{賤}{いや}しい
\ruby{意}{こゝろ}は% 踊り字調整「〻(二の字点、揺すり点)に見えるが(ゝ)」
\ruby[g]{微塵}{み ぢん}も
\ruby{有}{も}たない
\改行% 校正作業の簡略化のため
。
%
\原本頁{236-5}\改行%
\ruby{長}{なが}い
\ruby{眼}{め}で
\ruby{見}{み}て
\ruby{居}{ゐ}て
\ruby{吳}{く}れ
たまへ、
%
\ruby{此}{こ}の
\ruby[g]{水野}{みづの }は
たとひ
\ruby{世}{よ}に
\ruby{背}{そむ}いても
\改行% 校正作業の簡略化のため
、
%
\原本頁{236-6}\改行%
\ruby{世}{よ}と
\ruby{爭}{あらそ}つても、
%
\ruby[g]{屹度}{きつと }
\ruby{血}{ち}も
ある
\ruby{涙}{なみだ}も
ある
\ruby{詩}{し}を
\ruby{作}{つく}つて、
%
\ruby[g]{聖代}{せいだい}に
\ruby{生}{うま}れ
\ruby{合}{あ}はせた
\ruby[g]{男兒}{をとこ }
\ruby[g]{一人}{ひとり }
だけの、
%
\ruby[g]{任務}{つとめ }は
\ruby{其}{それ}で
\ruby{果}{はた}す
つもり
だと、
%
さも
\ruby[<j||]{潔}{いさぎ}よく
\ruby{言}{い}つたでは
\ruby{無}{な}いか。
%
%
\ruby{其}{そ}の
\ruby[g]{意氣}{い き }は
\ruby{今}{いま}
\ruby[g]{何處}{ど こ }へ
\ruby{無}{な}く
した?。
%
\ruby{其}{そ}の
\ruby[g]{言葉}{ことば }は
\ruby{既}{もう}
\ruby{忘}{わす}れ
\ruby{果}{は}て
たか。
%
ヤイ
\ruby[g]{水野}{みづの }!。
%
\ruby{詩}{し}の
\ruby[g]{一篇}{いつぺん}も
\ruby{作}{つく}らう
といふ
\原本頁{236-10}\改行%
ものが、
%
\ruby[g]{現在}{げんざい}の
\ruby{人{\換字{情}}世態}{にん|じやう|せ|たい}に
\ruby{眼}{め}は
\ruby{離}{はな}す
まいが、
%
\ruby{今}{いま}の
\ruby[g]{日本}{に ほん}の
\ruby[g]{狀態}{ありさま}を
\原本頁{236-11}\改行%
\ruby[g]{何樣}{ど う }
\ruby{思}{おも}ふ?\inhibitglue{}%
%
\ruby{汝}{きさま}!。
%
\ruby{今}{いま}の
\ruby[g]{世界}{せ かい}の
\ruby[g]{狀態}{ありさま}を
\ruby[g]{何樣}{ど う }
おもふ?\inhibitglue{}%
%
\ruby{汝}{きさま}!。
%
\ruby{浪}{なみ}の
\ruby{立}{た}たない
\ruby{海}{うみ}も
\ruby{無}{な}ければ、
%
\ruby{風}{かぜ}の
\ruby{荒}{あ}れない
\ruby{{\換字{空}}}{そら}も
\ruby{無}{な}くつて、
%
\ruby{國}{くに}は
\ruby{國}{くに}と
\ruby{競}{せ}り
\ruby{合}{あ}ひ、
%
\ruby[g]{人種}{じんしゆ}は
\ruby[g]{人種}{じんしゆ}と
\ruby{鬪}{たゝか}ふ、% 踊り字調整「〻(二の字点、揺すり点)に見えるが(ゝ)」
%
\ruby[g]{世界}{せ かい}の
\ruby{浪}{なみ}
\ruby{風}{かぜ}は
\ruby[g]{轟々}{がう〳〵}として、
%
\ruby{我}{わ}が
\ruby{國}{くに}の
\ruby{濱}{はま}へも
\ruby{磯}{いそ}へも
\ruby{寄}{よ}せて
\ruby{來}{き}て
\ruby{居}{ゐ}るでは
\ruby{無}{な}いか。
%
それだのに
\ruby[g]{國内}{こくない}の
\原本頁{237-4}\改行%
\ruby[g]{狀態}{ありさま}は
\ruby[g]{何樣}{ど う }だ。
%
\ruby{武士{\換字{道}}}{ぶ|し|だう}は
\ruby{廢}{すた}り
\ruby[g]{儒敎}{じゆけう}は
\ruby{棄}{す}てられ、
%
\ruby{舊}{ふる}い
\ruby{敎}{をしへ}は
\ruby{壞}{こは}れ
\ruby{果}{は}てたが、
%
\ruby{眞面目}{ま|じ|め}に
\ruby{受}{う}け
\ruby{入}{い}れ
られた
\ruby{新}{あたら}しい
\ruby{敎}{をしへ}も
\ruby{無}{な}く、
%
\ruby{{\換字{過}}去帳}{か|こ|ちやう}を
\ruby{讀}{よ}むやうに
\ruby[g]{哲人}{てつじん}の
\ruby{名}{な}
ばかりは
\ruby{忙}{せは}しく
\ruby{呼}{よび}
\ruby{立}{た}てられて、
%
やがて
\ruby[<j||]{直}{すぐ }
\ruby[<j||]{片}{かた }
\ruby[<j||]{端}{つぱし}
\原本頁{237-7}\改行%
から
\ruby{忘}{わす}れて
\ruby{行}{ゆ}かれる!。
%
\ruby{社}{しや}
\ruby[||j>]{會}{くわい}に
\ruby[g]{善惡}{ぜんあく}の
\ruby[g]{目安}{め やす}が
\ruby{無}{な}いから、
%
\ruby[g]{{\換字{勝}}手}{かつて }
\ruby[g]{次第}{し だい}の
\ruby{{\換字{強}}}{つよ}い
もの
\ruby{{\換字{勝}}}{がち}、
%
\ruby[g]{智慧}{ち ゑ }で
\ruby[||j>]{爭}{あらそ}ふ、
%
\ruby[g]{言說}{く ち }で
\ruby[||j>]{爭}{あらそ}ふ、
%
\ruby{筆}{ふで}で
\ruby[||j>]{爭}{あらそ}ふ、
%
\ruby{金}{かね}で
\ruby[<j||]{爭}{あらそ}ふ、
%
しかし
\ruby[g]{{\換字{道}}理}{だうり }で
\ruby[||j>]{爭}{あらそ}つた
のを
\ruby{聞}{き}いた
\ruby{事}{こと}が
\ruby{無}{な}い。
%
\ruby{金}{かね}を
\ruby{欲}{ほ}しがる、
%
\原本頁{237-10}\改行%
\ruby[g]{權威}{けんゐ }を
\ruby{欲}{ほ}しがる、
%
\ruby{名}{な}を
\ruby{欲}{ほ}しがる、
%
\ruby[g]{肉慾}{にくよく}の
\ruby[g]{滿足}{まんぞく}を
\ruby{欲}{ほ}しがる、
%
しかし
\ruby{徳}{とく}を
\ruby{欲}{ほ}しがる
ものは
\ruby{藥}{くすり}に
\ruby[g]{仕度}{し たく}も
\ruby{無}{な}い。
%
\ruby[g]{坊主}{ばうず }が
\ruby{役}{やく}
\ruby{立}{た}たん、
%
\ruby[g]{新聞}{しんぶん}
\原本頁{238-1}\改行%
\ruby[g]{記者}{き しや}が
\ruby{頼}{たの}もしく
\ruby{無}{な}い、
%
\ruby{敎育家}{けう|いく|か}が
\ruby{下}{くだ}らん、
%
\ruby[g]{學者}{がくしや}は
\ruby[g]{學說}{がくせつ}の
\ruby[g]{桂庵}{けいあん}
% 1 縁談や訴訟の仲立ちをする人。
%  また、雇い人・奉公人の 斡旋 あっせん を職業とする人。口入れ屋。
% 2 お世辞。 追従 ついしょう 。また、それを言う人。
ばかりで、
%
\ruby{{\換字{文}}學者}{ぶん|がく|しや}は
\ruby[g]{春枝}{はる{\換字{𛀁}}}さん
\ruby[g]{靜枝}{しづ{\換字{𛀁}}}さんの
\ruby{御機{\換字{嫌}}}{ご|き|げん}
\ruby{取}{と}りに
\ruby{{\換字{過}}}{す}ぎん。
%
\ruby[g]{世間}{せ けん}
\原本頁{238-3}\改行%
\ruby[g]{一體}{いつたい}は
\ruby{全}{まる}で
\ruby{不調子}{ふ|てう|し}で、
%
\ruby{錢}{ぜに}の
ある
\ruby{時}{とき}は
ハイカラになり、
%
\ruby{錢}{ぜに}の
\ruby{無}{な}い
\ruby{時}{とき}は
\ruby{蠻}{ばん}カラ、
%
\ruby{忰}{せがれ}は
\ruby[g]{戀愛}{れんあい}
\ruby{論}{ろん}、
%
\ruby[g]{親{\換字{父}}}{おやぢ }は
\ruby[g]{料理}{れうり }
\ruby{談}{だん}、
%
\ruby[g]{滔々}{たう〳〵}% 物事が一つの方向へよどみなく流れ向かうさま。
として
\ruby[g]{一般}{いつぱん}の
\ruby[g]{趣味}{しゆみ }は
\ruby{日}{ひ}に
\ruby[g]{墮落}{だ らく}して
\ruby{居}{ゐ}る。
%
\ruby{想}{おも}つても
\ruby{恐}{おそ}ろしい
\ruby[g]{世界}{せ かい}の
ありさま、
%
\ruby{見}{み}る
さへ
\ruby{{\換字{嫌}}}{いや}な
\ruby{人}{にん}
\ruby[||j>]{{\換字{情}}}{じやう}の
\ruby[g]{調子}{てうし }、
%
\ruby{彼}{あれ}と
\ruby{此}{これ}とを
\ruby{思}{おも}ひ
\ruby{合}{あ}はせれば、
%
\ruby{此}{こ}の
\ruby[g]{無骨}{ぶ こつ}
\原本頁{238-7}\改行%
\ruby{不風流}{ぶ|ふう|りう}の
\ruby[g]{乃公}{お れ }
でさへも、
%
\ruby[g]{無限}{む げん}の
\ruby[g]{{\換字{感}}慨}{かんがい}に
\ruby{打}{う}たれて、
%
\ruby{詩}{し}の
やうなものが
\ruby{呻}{うめ}き
\ruby{出}{だ}したく
なる、
%
まして
\ruby{汝}{きさま}が
\ruby[g]{{\換字{感}}慨}{かんがい}の
\ruby{無}{な}いわ
けは
\ruby{有}{あ}る
まいに
\ruby[g]{何故}{な ぜ }
\ruby[g]{一片}{いつぺん}
\ruby[g]{耿々}{かう〳〵}% 1 光が明るく輝くさま。 2 気にかかることがあって、心が安らかでないさま
たる
\ruby[g]{神州}{しんしう}
\ruby[g]{男兒}{だんじ }の
\ruby[g]{丹心}{たんしん}% まごころ。赤心。丹情。丹誠。丹地。
から、
%
\ruby{國}{くに}を
\ruby{愛}{あい}し
\ruby{世}{よ}を
\ruby{憂}{うれ}ふる
の
\ruby{誠}{まこと}を
\ruby[g]{披瀝}{ひ れき}
して、
%
\ruby{詩}{し}
でも
\ruby[||j>]{{\換字{文}}}{ぶん}
\ruby[||j>]{章}{しやう}
% \ruby{{\換字{文}}章}{ぶん|しやう}
でも
\ruby{作}{つく}り
\ruby{出}{だ}して
\ruby{吳}{く}れぬ?。
%
\ruby{手}{て}
\ruby{{\換字{緩}}}{ぬる}い
\原本頁{238-11}\改行%
\ruby{事}{こと}では
\ruby{無}{な}い、
%
\ruby{今}{いま}の
\ruby{今}{いま}でも
\ruby[g]{國{\換字{運}}}{こくうん}を
\ruby{賭}{と}して
\ruby[g]{戰爭}{たゝかひ}を% 踊り字調整「〻(二の字点、揺すり点)に見えるが(ゝ)」
\ruby{始}{はじ}めれば、
さしづめ
\ruby[g]{乃公}{お れ }たちは
\ruby[g]{水火}{すゐくわ}の
\ruby{中}{なか}にも
\ruby{飛}{と}びこまねば
ならぬ
\ruby{時}{とき}に
\ruby{逼}{せま}つて
\ruby{居}{ゐ}る
\原本頁{239-2}\改行%
\ruby[g]{塲合}{ば あひ}だ。% 原文通り「塲」
%
しかし
\ruby{詩}{し}は
\ruby{興}{きよう}が
\ruby{發}{はつ}しない
と
\ruby{云}{い}へば
それまでの
\ruby{事}{こと}、
%
\ruby[g]{出來}{で き }んなら
\ruby[g]{出來}{で き }んで
\ruby[g]{是非}{ぜ ひ }は
\ruby{無}{な}いが、
%
\ruby{汝}{きさま}
までが
\ruby{世}{よ}の
\ruby{風}{ふう}に
\ruby{負}{ま}けて
\ruby[g]{戀愛}{れんあい}
\ruby{騷}{さわ}ぎ
を
するとは
\ruby{何}{なに}
\ruby{事}{ごと}
だ。
%
そんな
\ruby{柔}{にう}
\ruby[||j>]{{\換字{弱}}}{じやく}
な、
%
\ruby[g]{性根}{しやうね}の
\ruby{拔}{ぬ}けた
\ruby{事}{こと}で、
%
\ruby{何}{なん}の
\原本頁{239-5}\改行%
\ruby{詩}{し}も
\ruby{歌}{うた}も
あつたものか。
%
\ruby[g]{時勢}{じ せい}の
\ruby[g]{幇間}{ほうかん}% 宴席などで客の機嫌をとり、酒宴の興を助けるのを職業とする男。 太鼓持ち。 男芸者。
とならぬと
\ruby{云}{い}つた
\ruby{其}{そ}の
\ruby[g]{意氣}{い き }は
\ruby{今}{いま}
どこに
\ruby{在}{あ}る?。
%
\ruby{正}{まさ}しく
\ruby{汝}{きさま}は
\ruby[g]{時勢}{じ せい}の
%
\ruby[g]{幇間}{ほうかん}となつた、
%
\ruby[g]{奴隷}{ど れい}となつた、
%
\ruby{狗}{いぬ}となつた!。
%
\ruby[g]{男子}{だんし }の
\ruby{眞}{まこと}の
\ruby{心}{こゝろ}を% 踊り字調整「〻(二の字点、揺すり点)に見えるが(ゝ)」
\ruby{失}{うしな}つた。
%
\ruby[g]{男心}{をごゝろ}も% 踊り字調整「〻(二の字点、揺すり点)に見えるが(ゝ)」
\ruby{無}{な}い
\ruby[g]{白痴}{たはけ }
に
なつたナ。
%
\ruby{戀}{こひ}の
\ruby{奴}{やつこ}と
\ruby{我}{われ}は
\ruby{死}{し}ぬ
べし
とは
\ruby{何}{なん}たる
\ruby{事}{こと}だ。
%
\ruby{此}{こ}の
\ruby{普門品}{ふ|もん|ぼん}は
\ruby{誰}{だれ}が
\ruby{誦}{よ}んで、
%
\ruby{其}{そ}の
\ruby{下}{くだ}らん
\ruby[g]{御籤}{み くじ}
と
いふものは
\ruby{誰}{だれ}が
\ruby{抽}{と}つた?。
%
\原本頁{239-10}\改行%
ちらりと
\ruby{聞}{き}けば
\ruby[<j||]{觀}{くわん}% 「觀音」の読みは原本通り「くわん(の)ん」
\ruby[||j>]{音}{のん}
\ruby[||j>]{詣}{まうで}
して、
%
\ruby{而}{さう}して
\ruby{纔}{やつ}と
\ruby{今}{いま}
\ruby{歸}{かへ}つて
\ruby{來}{き}たのだナ。
%
\ruby[||j>]{汝}{きさま}が
\ruby{思}{おも}つて
\ruby{居}{ゐ}る
\ruby{女}{をんな}が
\ruby{大}{たい}
\ruby[||j>]{病}{びやう}だ
とか
いふ
\ruby[g]{島木}{しまき }の
\ruby[g]{談話}{はなし }も
\ruby{思}{おも}ひ
\ruby{合}{あ}はせて
\改行% 校正作業の簡略化のため
、
%
\原本頁{240-1}\改行%
すつかり
\ruby{汝}{きさま}の
\ruby[g]{{\換字{所}}業}{し わざ}は
\ruby{{\換字{分}}}{わか}つたが、
%
\ruby{女}{をんな}の
ために
\ruby{經}{きやう}を
\ruby{誦}{よ}んだり、
%
\ruby[g]{御籤}{み くじ}を
\ruby{取}{と}つたり、
%
わざ〳〵
\ruby[g]{淺草}{あさくさ}まで
\ruby{歩}{あゆみ}を
\ruby{{\換字{運}}}{はこ}んだり
して
\ruby{居}{ゐ}るのだナ。
%
\原本頁{240-3}\改行%
エーツ
\ruby[g]{{\換字{情}}無}{なさけな}くも
\ruby{衰}{おとろ}へに
\ruby{衰}{おとろ}へた
\ruby{奴}{やつ}だ。
%
\ruby{書}{しよ}も
\ruby{讀}{よ}み
\ruby{理}{り}にも
\ruby{眛}{くら}からぬ
\ruby[g]{水野}{みづの }
とも
ある
もの
が、
%
\ruby[g]{如何}{い か }に
\ruby{{\換字{迷}}}{まよ}へば
とて
\ruby{一}{いつ}
\ruby[g]{{\換字{婦}}人}{ぷ じん}
の
ために、
%
それほども
\ruby{愚}{ぐ}に
なつて、
%
\ruby{成}{な}りきつたか。
%
\ruby{{\換字{魔}}}{ま}に
\ruby{憑}{つ}かれたか
\ruby{何}{なに}に
\ruby{憑}{つ}かれたか、
%
\ruby[g]{全然}{まるで }
\ruby[g]{正氣}{しやうき}の
\ruby[g]{沙汰}{さ た }
では
\ruby{無}{な}いが、
%
\ruby[g]{男兒}{をとこ }の
\ruby[g]{魂魄}{たましひ}が
\ruby[g]{少許}{すこし }
でも
あれば
\改行% 校正作業の簡略化のため
、
%
\原本頁{240-7}\改行%
\ruby[g]{正氣}{しやうき}に
\ruby{{\換字{返}}}{かへ}れ、
%
\ruby[g]{正氣}{しやうき}に
\ruby{仕}{し}て
やらう。
%
\ruby{目}{め}を
\ruby{覺}{さ}ませ
\ruby[g]{水野}{みづの }。
』

\原本頁{240-8}%
と
\ruby{云}{い}ひ
さまに、
%
\ruby{普門品}{ふ|もん|ぼん}を
\ruby[g]{右手}{みぎて }に
\ruby[g]{鷲握}{わしづか}み
にして、
%
\ruby[g]{左手}{ひだりて}に
\ruby[g]{水野}{みづの }を
\ruby{取}{と}つて
\ruby[g]{引伏}{ひきふ }せ、

\原本頁{240-10}%
『
\ruby[g]{{\換字{情}}無}{なさけな}い
\ruby{奴}{やつ}だ!。
%
\ruby[g]{正氣}{しやうき}に
\ruby{{\換字{返}}}{かへ}らんか、
%
\ruby[g]{朋友}{ともだち}の
\ruby[g]{{\換字{情}}誼}{なさけ }だ、
%
\ruby{身}{み}に
\ruby{染}{し}みて
\ruby{受}{う}けろ。
』

\原本頁{241-1}%
と
ピシリ〳〵
と
\ruby{續}{つゞ}け% 踊り字調整「〻(二の字点、揺すり点)に濁点に見えるが(ゞ)」
さまに
\ruby{打}{う}つたり。
