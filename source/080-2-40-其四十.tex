\Entry{其四十}

『
\ruby{水野}{みづ|の}、よもや
\ruby{汝}{きさま}はまだ
\ruby{自{\換字{分}}}{じ|ぶん}で
\ruby{云}{い}つた
\ruby{事}{こと}を
\ruby{忘}{わす}れるほどに
\ruby{耄碌}{まう|ろく}は
\ruby{爲}{し}まい。
\ruby[g]{數年前}{すねんぜん}に
\ruby{我々}{われ|〳〵}が
\ruby{寄}{よ}り
\ruby{合}{あ}つて、
\ruby{互}{たがひ}に
\ruby{抱負}{はう|ふ}を
\ruby{{\換字{述}}}{の}べて
\ruby{談笑}{だん|せう}した
\ruby{時}{とき}、
\ruby{大丈夫}{だい|ぢやう|ぶ}の
\ruby{身}{み}をもつて
\ruby{詩文}{し|ぶん}の
\ruby{小技}{せう|ぎ}に
\ruby{身}{み}を
\ruby{委}{ゆだ}ねやうとは
\ruby{何}{なん}の
\ruby{事}{こと}だ、
\ruby[g]{雛蟲篆刻壯夫}{てうちうてんこくさうふ}は
\ruby{爲}{な}さずと、
\ruby{楊雄}{やう|ゆう}づれでさへ
\ruby{云}{い}つて
\ruby{居}{ゐ}るのに、
\ruby{歌}{うた}のポエムのと
\ruby{捏}{こ}ぬ
\ruby{{\換字{返}}}{かへ}して、
\ruby{食}{く}へもせず
\ruby{衣}{き}られもせぬものに
\ruby{苦勞}{く|らう}しやうとは、
\ruby{道樂{\換字{過}}}{だう|らく|す}ぎて
\ruby{餘}{あま}り
\ruby{詰}{つま}らぬと、
\ruby{乃公}{お|れ}が
\ruby{口}{くち}を
\ruby{極}{きは}めて
\ruby[g]{非難}{ひなん}したらば、
\ruby{今}{いま}と
\ruby{異}{ちが}つて
\ruby{元氣}{げん|き}のあつた
\ruby{其頃}{その|ころ}の
\ruby{汝}{きさま}は、
\ruby{眉}{まゆ}を
\ruby{昻}{あ}げ
\ruby{面}{おもて}を
\ruby{正}{たゞし}くして
\ruby{凛然}{りん|ぜん}と
\ruby{答}{こた}へた
\ruby{其}{そ}の
\ruby{挨拶}{あい|さつ}に
\ruby{何}{なん}と
\ruby{云}{い}つた!。
\ruby{食}{しよく}は
\ruby{身}{み}の
\ruby{糧}{かて}、
\ruby{詩}{し}は
\ruby{心}{こゝろ}の
\ruby{糧}{かて}、
\ruby{衣}{きもの}は
\ruby{暑}{あつ}さ
\ruby{寒}{さむ}さに
\ruby{對}{たい}して
\ruby{人}{ひと}の
\ruby{身}{み}を
\ruby{護}{まも}り、
\ruby{詩}{し}は
\ruby{悲}{かなし}みにも
\ruby{怒}{いか}りにも
\ruby{對}{むか}つて
\ruby{人}{ひと}の
\ruby{心}{こゝろ}を
\ruby{調}{とゝの}へる、それを
\ruby{{\換字{益}}}{\換字{江}き}の
\ruby{無}{な}いもののやうに
\ruby{云}{い}ふは
\ruby{淺}{あさ}ましい
\ruby{誤謬}{あや|まり}。
\ruby{貝}{かひ}に
\ruby{眞珠}{しん|じゆ}あり、
\ruby{人}{ひと}に
\ruby{詩}{し}あり、
\ruby{詩歌}{し|か}を
\ruby{除}{のぞ}きて
\ruby{人}{ひと}の
\ruby{作}{つく}れるものに、
\ruby[g]{野菊}{のぎく}の
\ruby{花}{はな}の
\ruby{一輪}{いち|りん}だけの
\ruby{美}{うつく}しさのあるものも
\ruby{無}{な}く、
\ruby[g]{阿{\換字{房}}威陽}{あぼうかんやう}は
\ruby{羞}{はづか}しく
\ruby{醜}{みにく}い。

\ruby{美}{うつく}しき
\ruby{胸}{むね}の
\ruby{働}{はたら}きの
\ruby{目}{め}にも
\ruby{見}{み}えぬが、
\ruby{凝}{こ}つて
\ruby{詩}{し}となつて
\ruby{文字}{もん|じ}に
\ruby{現}{あらは}るれば、
\ruby{讀}{よ}むもの
\ruby{恍惚}{くわう|こつ}として
\ruby{我}{われ}を
\ruby{忘}{わす}れて、
\ruby{作}{つく}る
\ruby{人}{ひと}が
\ruby{泣}{な}けば
\ruby{泣}{な}き、
\ruby{憤}{いか}れば
\ruby{憤}{いか}る。
されは
\ruby{人間}{ひ|と}の
\ruby{性{\換字{情}}}{せい|じやう}を
\ruby{敦}{あつ}くし、
\ruby{世}{よ}の
\ruby{氣風}{き|ふう}を
\ruby{嘉}{よ}くするもの、
\ruby{詩}{し}に
\ruby{越}{こ}すまのは
\ruby{無}{な}い。
\ruby{大言}{たい|げん}のやうだが
\ruby{此}{こ}の
\ruby{水野}{みづ|の}は、たゞ
\ruby{蝶花}{てふ|はな}のおもしろさや
\ruby{月露}{げつ|ろ}のあはれさを
\ruby{歌}{うた}つてのみ
\ruby{我}{わ}が
\ruby{一生}{いつ|しやう}を
\ruby{{\換字{過}}}{すご}さんとは
\ruby{仕}{し}ない。
\ruby{百年千年}{ひやく|ねん|せん|ねん}にして
\ruby{一}{ひ}ト
\ruby{度出}{たび|い}づる
\ruby{大詩人}{だい|し|じん}の、
\ruby{一代}{いち|だい}の
\ruby{人心}{じん|〳〵}を
\ruby{新}{あらた}にして、
\ruby{萬世}{ばん|せい}に
\ruby{天意}{てん|い}の
\ruby{眞}{まこと}を
\ruby{傳}{つた}へんとする、
\ruby{其}{それ}は
\ruby{及}{およ}ばざる
\ruby{願}{ねがひ}にもせよ、
\ruby[g]{時勢}{じせい}の
\ruby{幇間}{ほう|かん}となつて
\ruby{徳}{とく}を
\ruby{頌}{しよう}するやうな
\ruby{賤}{いや}しい
\ruby{意}{こゝろ}は
\ruby{微塵}{み|じん}も
\ruby{有}{も}たない。
\ruby{長}{なが}い
\ruby{眼}{め}で
\ruby{見}{み}て
\ruby{居}{ゐ}て
\ruby{呉}{く}れたまへ、
\ruby{此}{こ}の
\ruby{水野}{みづ|の}はたとひ
\ruby{世}{よ}に
\ruby{背}{そむ}いても
\ruby{世}{よ}と
\ruby{爭}{あらそ}つても、
\ruby{屹度}{きつ|と}
\ruby{血}{ち}もある
\ruby{淚}{なみだ}もある
\ruby{詩}{し}を
\ruby{作}{つく}つて、
\ruby{聖代}{せい|だい}に
\ruby{生}{うま}れ
\ruby{合}{あ}はせた
\ruby{男兒}{をと|こ}
\ruby{一人}{ひと|り}だけの、
\ruby[g]{任務}{つとめ}は
\ruby{其}{それ}で
\ruby{果}{はた}すつもりだと、さも
\ruby{潔}{いさぎ}よく
\ruby{言}{い}つたでは
\ruby{無}{な}いか。
ヤイ
\ruby{水野}{みづ|の}!。
\ruby{詩}{し}の
\ruby{一篇}{いつ|ぺん}も
\ruby{作}{つく}らうといふものが、
\ruby{現在}{げん|ざい}の
\ruby[g]{人{\換字{情}}世態}{にんじやうせたい}に
\ruby{眼}{め}は
\ruby{離}{はな}すまいが、
\ruby{今}{いま}の
\ruby{日本}{に|ほん}の
\ruby{狀態}{あり|さま}を
\ruby{何樣思}{ど|う|おも}ふ?。
\ruby{汝}{きさま}!。
\ruby{今}{いま}の
\ruby{世界}{せ|かい}の
\ruby{狀態}{あり|さま}を
\ruby{何樣}{ど|う}おもふ?。
\ruby{汝}{きさま}!。
\ruby{浪}{なみ}の
\ruby{立}{た}たない
\ruby{海}{うみ}も
\ruby{無}{な}ければ、
\ruby{風}{かぜ}の
\ruby{荒}{あ}れない
\ruby{空}{そら}も
\ruby{無}{な}くつて、
\ruby{國}{くに}は
\ruby{國}{くに}と
\ruby{競}{せ}り
\ruby{合}{あ}ひ、
\ruby{人種}{じん|しゆ}は
\ruby{人種}{じん|しゆ}と
\ruby{鬪}{たゝか}ふ、
\ruby{世界}{せ|かい}の
\ruby{浪風}{なみ|かぜ}は
\ruby{轟々}{がう|〳〵}として、
\ruby{我}{わ}が
\ruby{國}{くに}の
\ruby{濱}{はま}へも
\ruby{磯}{いそ}へも
\ruby{寄}{よ}せて
\ruby{來}{き}て
\ruby{居}{ゐ}るでは
\ruby{無}{な}いか。
それだのに
\ruby{國内}{こく|ない}の
\ruby{狀態}{あり|さま}は
\ruby{何樣}{ど|う}だ。
\ruby[g]{武士{\換字{道}}}{ぶしだう}は
\ruby{廢}{すた}り
\ruby{儒教}{じゆ|けう}は
\ruby{棄}{す}てられ、
\ruby{舊}{ふる}い
\ruby{教}{をしへ}は
\ruby{壞}{こは}れ
\ruby{果}{は}てたが、
\ruby{眞面目}{ま|じ|め}に
\ruby{受}{う}け
\ruby{入}{い}れられた
\ruby{新}{あたら}しい
\ruby{教}{をしへ}も
\ruby{無}{な}く、
\ruby[g]{{\換字{過}}去帳}{かこちやう}を
\ruby{讀}{よ}むやうに
\ruby{哲人}{てつ|じん}の
\ruby{名}{な}ばかりは
\ruby{忙}{せは}しく
\ruby{呼立}{よび|た}てられて、やがて
\ruby[g]{直片端}{すぐかたつぱし}から
\ruby{忘}{わす}れて
\ruby{行}{ゆ}かれる!。
\ruby{社會}{しや|くわい}に
\ruby{善惡}{ぜん|あく}の
\ruby[g]{目安}{めやす}が
\ruby{無}{な}いから、
\ruby{{\換字{勝}}手}{かつ|て}
\ruby{次第}{し|だい}の
\ruby{{\換字{強}}}{つよ}いもの
\ruby{{\換字{勝}}}{がち}、
\ruby{智慧}{ち|ゑ}で
\ruby{爭}{あらそ}ふ、
\ruby{言{\換字{説}}}{く|ち}で
\ruby{爭}{あらそ}ふ、
\ruby{筆}{ふで}で
\ruby{爭}{あらそ}ふ、
\ruby{金}{かね}で
\ruby{爭}{あらそ}ふ、しかし
\ruby{{\換字{道}}理}{だう|り}で
\ruby{爭}{あらそ}つたのを
\ruby{聞}{き}いた
\ruby{事}{こと}が
\ruby{無}{な}い。
\ruby{金}{かね}を
\ruby{欲}{ほ}しがる、
\ruby{權威}{けん|ゐ}を
\ruby{欲}{ほ}しがる、
\ruby{名}{な}を
\ruby{欲}{ほ}しがる、
\ruby{肉慾}{にく|よく}の
\ruby{滿足}{まん|ぞく}を
\ruby{欲}{ほ}しがる、しかし
\ruby{徳}{とく}を
\ruby{欲}{ほ}しがるものは
\ruby{藥}{くすり}に
\ruby{仕度}{し|たく}も
\ruby{無}{な}い。
\ruby{坊主}{ばう|ず}が
\ruby{役立}{やく|た}たん、
\ruby{新開記者}{しん|ぶん|き|しや}が
\ruby{頼}{たの}もしく
\ruby{無}{な}い、
\ruby{教育家}{けう|いく|か}が
\ruby{下}{くだ}らん、
\ruby{學者}{がく|しや}は
\ruby{學{\換字{説}}}{がく|せつ}の
\ruby{桂庵}{けい|あん}ばかりで、
\ruby{文學者}{ぶん|がく|しや}は
\ruby{春枝}{はる|\換字{江}}さん
\ruby{靜枝}{しづ|\換字{江}}さんの
\ruby{御機嫌取}{ご|き|げん|と}りに
\ruby{{\換字{過}}}{す}ぎん。
\ruby{世間一體}{せ|けん|いつ|たい}は
\ruby{全}{まる}で
\ruby[g]{不調子}{ふてうし}で、
\ruby{錢}{ぜに}のある
\ruby{時}{とき}はハイカラになり、
\ruby{錢}{ぜに}の
\ruby{無}{な}い
\ruby{時}{とき}は
\ruby{{\換字{蛮}}}{ばん}カラ、
\ruby{忰}{せがれ}は
\ruby{戀愛論}{れん|あい|ろん}、
\ruby{親父}{おや|ぢ}は
\ruby{料理談}{れう|り|だん}、
\ruby{滔々}{たう|〳〵}として
\ruby{一般}{いつ|ぱん}の
\ruby{趣味}{しゆ|み}は
\ruby{日}{ひ}に
\ruby{墮落}{だ|らく}して
\ruby{居}{ゐ}る。
\ruby{想}{おも}つても
\ruby{恐}{おそ}ろしい
\ruby{世界}{せ|かい}のありさま、
\ruby{見}{み}るさへ
\ruby{{\換字{嫌}}}{いや}な
\ruby{人{\換字{情}}}{にん|じやう}の
\ruby{調子}{てう|し}、
\ruby{彼}{あれ}と
\ruby{此}{これ}とを
\ruby{思}{おも}ひ
\ruby{合}{あ}はせれば、
\ruby{此}{こ}の
\ruby{無骨不風流}{ぶ|こつ|ぶ|ふう|りう}の
\ruby{乃公}{お|れ}でさへも、
\ruby{無限}{む|げん}の
\ruby{感{\換字{慨}}}{かん|がい}に
\ruby{打}{う}たれて、
\ruby{詩}{し}のやうなものが
\ruby{呻}{うめ}き
\ruby{出}{だ}したくなる、まして
\ruby{汝}{きさま}が
\ruby{感{\換字{慨}}}{かん|がい}の
\ruby{無}{な}いわけは
\ruby{有}{あ}るまいに
\ruby[g]{何故一片耿々}{なぜいつぺんかう〳〵}たる
\ruby{神州}{しん|しう}
\ruby{男兒}{だん|じ}の
\ruby{丹心}{たん|しん}から、
\ruby{國}{くに}を
\ruby{愛}{あい}し
\ruby{世}{よ}を
\ruby{憂}{うれ}ふるの
\ruby{誠}{まこと}を
\ruby[g]{披瀝}{ひれき}して、
\ruby{詩}{し}でも
\ruby{文章}{ぶん|しやう}でも
\ruby{作}{つく}り
\ruby{出}{だ}して
\ruby{{\換字{呉}}}{く}れぬ?。
\ruby{手緩}{て|ぬる}い
\ruby{事}{こと}では
\ruby{無}{な}い、
\ruby{今}{いま}の
\ruby{今}{いま}でも
\ruby{國{\換字{運}}}{こく|うん}を
\ruby{賭}{と}して
\ruby{戰爭}{たゝ|かひ}を
\ruby{始}{はじ}めればさしずめ
\ruby{乃公}{お|れ}たちは
\ruby{水火}{すゐ|くわ}の
\ruby{中}{なか}にも
\ruby{飛}{と}びこまねばならぬ
\ruby{時}{とき}に
\ruby{逼}{せま}つて
\ruby{居}{ゐ}る
\ruby{塲合}{ば|あひ}だ。
しかし
\ruby{詩}{し}は
\ruby{興}{きよう}が
\ruby{發}{はつ}しないと
\ruby{云}{い}へばそれまでの
\ruby{事}{こと}、
\ruby{出來}{で|き}んなら
\ruby{出來}{で|き}んで
\ruby{是非}{ぜ|ひ}は
\ruby{無}{な}いが、
\ruby{汝}{きさま}までが
\ruby{世}{よ}の
\ruby{風}{ふう}に
\ruby{負}{ま}けて
\ruby{戀愛騒}{れん|あい|さわ}ぎをするとは
\ruby{何事}{なに|ごと}だ。
そんな
\ruby{柔{\換字{弱}}}{にう|じやく}な、
\ruby[g]{性根}{しやうね}の
\ruby{拔}{ぬ}けた
\ruby{事}{こと}で、
\ruby{何}{なん}の
\ruby{詩}{し}も
\ruby{歌}{うた}もあつたものか。
\ruby{時勢}{じ|せい}の
\ruby{幇間}{ほう|かん}とならぬと
\ruby{云}{い}つた
\ruby{其}{そ}の
\ruby{意氣}{い|き}は
\ruby{今}{いま}どこに
\ruby{在}{あ}る?。

\ruby{正}{まさ}しく
\ruby{汝}{きさま}は
\ruby{時勢}{じ|せい}の
\ruby{幇間}{ほう|かん}となつた、
\ruby{奴隷}{ど|れい}となつた、
\ruby{狗}{いぬ}となつた!。
\ruby{男子}{だん|し}の
\ruby{眞}{まこと}の
\ruby{心}{こゝろ}を
\ruby{失}{うしな}つた。
\ruby{男心}{をご|ゝろ}も
\ruby{無}{な}い
\ruby{白痴}{た|はけ}になつたナ。
\ruby{戀}{こひ}の
\ruby{奴}{やつこ}と
\ruby{我}{われ}は
\ruby{死}{し}ぬべしとは
\ruby{何}{なん}たる
\ruby{事}{こと}だ。
\ruby{此}{こ}の
\ruby{普門品}{ふ|もん|ぼん}は
\ruby{誰}{たれ}が
\ruby{誦}{よ}んで、
\ruby{其}{そ}の
\ruby{下}{くだ}らん
\ruby{御籤}{み|くじ}といふものは
\ruby{誰}{たれ}が
\ruby{抽}{と}つた?。
ちらりと
\ruby{聞}{き}けば
\ruby[g]{觀音詣}{くわんのんまうで}して、
\ruby{而}{さう}して
\ruby{纔}{やつ}と
\ruby{今歸}{いま|かへ}つて
\ruby{來}{き}たのだナ。
\ruby{汝}{きさま}が
\ruby{思}{おも}つて
\ruby{居}{ゐ}る
\ruby{女}{をんな}が
\ruby{大病}{たい|びやう}だとかいふ
\ruby{島木}{しま|き}の
\ruby{談話}{はな|し}も
\ruby{思}{おも}ひ
\ruby{合}{あ}はせて、すつかり
\ruby{汝}{きさま}の
\ruby{{\換字{所}}業}{し|わざ}は
\ruby{{\換字{分}}}{わか}つたが、
\ruby{女}{をんな}のために
\ruby{經}{きやう}を
\ruby{誦}{よ}んだり、
\ruby{御籤}{み|くじ}を
\ruby{取}{と}つたり、わざ〳〵
\ruby{淺草}{あさ|くさ}まで
\ruby{歩}{あゆみ}を
\ruby{{\換字{運}}}{はこ}んだりして
\ruby{居}{ゐ}るのだナ。
エーツ
\ruby[g]{情無}{なさけな}くも
\ruby{衰}{おとろ}へに
\ruby{衰}{おとろ}へた
\ruby{奴}{やつ}だ。

\ruby{書}{しよ}も
\ruby{讀}{よ}み
\ruby{理}{り}にも
\ruby{眛}{くら}からぬ
\ruby{水野}{みづ|の}ともあるものが、
\ruby{如何}{い|か}に
\ruby{{\換字{迷}}}{まよ}へばとて
\ruby{一婦人}{いち|ふ|じん}のために、それほども
\ruby{愚}{ぐ}になつて、
\ruby{成}{な}りきつたか。
\ruby{魔}{ま}に
\ruby{憑}{つ}かれたか
\ruby{何}{なに}に
\ruby{憑}{つ}かれたか、
\ruby{全然}{まる|で}
\ruby{正氣}{しやう|き}の
\ruby{沙汰}{さ|た}では
\ruby{無}{な}いが、
\ruby{男兒}{をと|こ}の
\ruby{魂魄}{たま|しひ}が
\ruby[g]{少許}{すこし}でもあれば、
\ruby{正氣}{しやう|き}に
\ruby{{\換字{返}}}{かへ}れ、
\ruby{正氣}{しやう|き}に
\ruby{仕}{し}てやらう。
\ruby{目}{め}を
\ruby{覺}{さ}
ませ
\ruby{水野}{みづ|の}。
』

と
\ruby{云}{い}ひさまに、
\ruby{普門品}{ふ|もん|ぼん}を
\ruby{右手}{みぎ|て}に
\ruby{鷲握}{わし|づか}みにして、
\ruby{左手}{ひだり|て}に
\ruby{水野}{みづ|の}を
\ruby{取}{と}つて
\ruby{引伏}{ひき|ふ}せ、

『
\ruby[g]{情無}{なさけな}い
\ruby{奴}{やつ}だ!。
\ruby{正氣}{しやう|き}に
\ruby{返}{かへ}らんか、
\ruby{朋友}{とも|だち}の
\ruby{情誼}{なさ|け}だ、
\ruby{身}{み}に
\ruby{染}{し}みて
\ruby{受}{う}けろ。
』

とビシリ〳〵と
\ruby{續}{つゞ}けさまに
\ruby{打}{う}つたり。

