\Entry{其四十一}

% メモ 校正終了 2024-04-30 2024-06-05
\原本頁{241-3}%
\改行%
\ruby[||j>]{苟}{いやし}くも
\ruby[g]{男兒}{をとこ }なり、
%
\ruby{辱}{はづか}しめられて
\ruby{怒}{いかり}を
\ruby{發}{おこ}さゞるは% 踊り字調整「〻(二の字点、揺すり点)に濁点に見えるが(ゞ)」
あらず、
%
\ruby{特}{こと}に
\ruby[g]{表面}{うはべ }
こそ
\ruby[g]{柔和}{にうわ }なれ、
%
\ruby{心}{こゝろ}の% 踊り字調整「〻(二の字点、揺すり点)に見えるが(ゝ)」
\ruby{底}{そこ}には
\ruby[g]{王侯}{わうこう}
\ruby[g]{貴人}{き にん}をも
\ruby{重}{おも}くは
\ruby{視}{み}ぬ
ほどの
\ruby[g]{水野}{みづの }の、
%
\ruby[g]{如何}{い か }に
\ruby[g]{朋友}{ほういう}の
\ruby[g]{好意}{よしみ }よりの
\ruby[g]{振舞}{ふるまひ}とは
\ruby{云}{い}へ、
%
\ruby{物}{もの}も
\ruby{云}{い}はさずに
\ruby{手}{て}
\ruby{荒}{あら}く
\ruby{打}{うち}
\ruby{擲}{たゝ}かれては、% 踊り字調整「〻(二の字点、揺すり点)に見えるが(ゝ)」
%
\ruby[g]{勃然}{む つ }
として
\ruby{胸}{むね}に
\ruby{衝}{つ}き
\ruby{上}{あが}る
ものゝ% 踊り字調整「〻(二の字点、揺すり点)に見えるが(ゝ)」
\ruby{無}{な}き
ならねば、
%
\ruby{我}{わ}が
\ruby{襟}{{\換字{𛀁}}り}を
\ruby{捉}{とら}へし
\ruby[g]{日方}{ひ かた}の
\ruby{手}{て}を、
%
\ruby{急}{きふ}に
\ruby{{\換字{捩}}}{ね}ぢ
\ruby{放}{はな}して
\ruby{身}{み}を
\ruby{{\換字{退}}}{ひ}きつ
\改行% 校正作業の簡略化のため
、
%
\原本頁{241-9}\改行%
\ruby[g]{嚴然}{きつと }
\ruby{居}{ゐ}ずまひを
\ruby{正}{たゞ}して% 踊り字調整「〻(二の字点、揺すり点)に濁点に見えるが(ゞ)」
\ruby{眼}{め}つき
\ruby{嶮}{けは}しく
\ruby[g]{無言}{む ごん}に
\ruby{見}{み}
\ruby{{\換字{返}}}{かへ}しゝが、% 踊り字調整「〻(二の字点、揺すり点)に見えるが(ゝ)」
%
あゝ% 踊り字調整「〻(二の字点、揺すり点)に見えるが(ゝ)」
\ruby{思}{おも}へば
\ruby{今}{いま}
\ruby{我}{われ}
こゝに% 踊り字調整「〻(二の字点、揺すり点)に見えるが(ゝ)」
\ruby{何}{なに}をか
\ruby{言}{い}はん、
%
まことや
\ruby{我}{われ}は
\ruby[g]{往時}{むかし }の
\ruby{我}{われ}ならず、
%
% \原本頁{241-11}\改行%
\ruby{比}{くら}べて
\ruby{明}{あき}らかに
\ruby{知}{し}るゝ% 踊り字調整「〻(二の字点、揺すり点)に見えるが(ゝ)」
\ruby[g]{身體}{からだ }の
\ruby{衰}{おとろ}へに、
%
\ruby{心}{こゝろ}の% 踊り字調整「〻(二の字点、揺すり点)に見えるが(ゝ)」
\ruby{衰}{おとろ}へも
\ruby{自}{みづか}ら
\ruby{知}{し}る!。
%
% \原本頁{242-1}\改行%
まさかに
\ruby[g]{一旦}{いつたん}
\ruby{懷}{いだ}きし
\ruby[g]{本來}{ほんらい}の
\ruby[|j>]{志}{こゝろざし}を、% 踊り字調整「〻(二の字点、揺すり点)に見えるが(ゝ)」
%
\ruby{忘}{わす}れ
\ruby{果}{は}てゝ% 踊り字調整「〻(二の字点、揺すり点)に見えるが(ゝ)」
\ruby{好}{い}いと
\ruby{思}{おも}ふ
やうな
\ruby{氣}{き}は
\ruby{持}{も}たねども、
%
\ruby[||j>]{正}{しやう}
\ruby[||j>]{直}{ ぢき}を
\ruby{云}{い}へば
\ruby[g]{何時}{い つ }の
\ruby{間}{ま}にか、
%
\ruby{{\換字{空}}}{あだ}に
\ruby{物}{もの}を
のみ
% \原本頁{242-3}\改行%
\ruby{思}{おも}ふ
\ruby{癖}{くせ}の
つきて、
%
\ruby[g]{自{\換字{分}}}{じ ぶん}の
\ruby{心}{こゝろ}にも% 踊り字調整「〻(二の字点、揺すり点)に見えるが(ゝ)」
\ruby[g]{自{\換字{分}}}{じ ぶん}の
\ruby{心}{こゝろ}が% 踊り字調整「〻(二の字点、揺すり点)に見えるが(ゝ)」
\ruby[g]{何樣}{ど う }も
ならぬ
といふ
\ruby[g]{{\換字{情}}無}{なさけな}い
\ruby{身}{み}の
\ruby{上}{うへ}、
%
これでは
ならぬと
\ruby{思}{おも}ひ
\ruby{{\換字{返}}}{かへ}しても、
%
\ruby{思}{おも}ひ
\ruby{{\換字{返}}}{かへ}す
\ruby{其}{その}
\ruby{下}{した}より
\ruby{其}{その}
\ruby{人}{ひと}の
\ruby{事}{こと}
ばかりが
\ruby{思}{おも}はれて、
%
\ruby[g]{茫然}{ばうぜん}
として
\ruby{日}{ひ}を
\ruby{暮}{く}らして
\ruby[g]{仕舞}{し ま }ふ
\ruby{羞}{はづ}かしい
\ruby[||j>]{境}{きよう}
\ruby[||j>]{界}{ がい}。
% \ruby{境界}{きよう|がい}。
%
むかしは
\ruby{{\換字{若}}}{わか}い
\ruby[g]{氣勢}{いきほひ}に
\ruby{神}{かみ}も
\ruby{佛}{ほとけ}も
\ruby{頼}{たの}まざりしが、
%
% \原本頁{242-7}\改行%
\ruby{信}{しん}ぜず
には
\ruby{居}{ゐ}られなく
なつて
\ruby{今}{いま}は
\ruby{信}{しん}ずる
\ruby{此}{こ}の
\ruby{我}{わ}が
\ruby[g]{擧動}{ふるまひ}を、
%
\ruby{他}{ひと}より
\ruby{見}{み}たらば、
%
\ruby[g]{成程}{なるほど}
\ruby{意氣地}{い|く|ぢ}の
\ruby{無}{な}い
\ruby{愚夫愚{\換字{婦}}}{ぐ|ふ|ぐ|ふ}の% ルビ調整(原本通り)非踊り字表記
\ruby[g]{{\換字{所}}爲}{しわざ }と、% 「所爲」(せい)(しよい)その人の行為から起こった結果、そのことによる結果。
%
\ruby{譏}{そし}られても
\ruby[||j>]{罵}{のゝし}られても% 踊り字調整「〻(二の字点、揺すり点)に見えるが(ゝ)」
\ruby[g]{仕方}{し かた}は
\ruby{無}{な}く、
%
\ruby{云}{い}ひ
\ruby{解}{と}かう
に
\ruby{云}{い}ひ
\ruby{解}{と}かう
ところも
\ruby{無}{な}し
\改行% 校正作業の簡略化のため
。
%
\原本頁{242-10}\改行%
されば
\ruby{打}{ぶ}たれても
\ruby{擲}{たゝ}かれても% 踊り字調整「〻(二の字点、揺すり点)に見えるが(ゝ)」
\ruby{罵}{のゝし}られても、% 踊り字調整「〻(二の字点、揺すり点)に見えるが(ゝ)」
%
\ruby[g]{男兒}{をとこ }
らしく
\ruby{顏}{かほ}を
\ruby{擡}{あ}げて
\ruby{云}{い}ひ
\ruby{爭}{あらそ}はう
には、
%
\ruby{餘}{あま}りに
\ruby{云}{い}ひ
\ruby[g]{甲{\換字{斐}}}{が ひ }
\ruby{無}{な}くも
\ruby{思}{おも}ひに
\ruby{{\換字{弱}}}{よわ}れる
\ruby{我}{われ}かな
\改行% 校正作業の簡略化のため
。
%
\原本頁{243-1}\改行%
あゝ% 踊り字調整「〻(二の字点、揺すり点)に見えるが(ゝ)」
\ruby{我}{われ}
ながら
\ruby[g]{{\換字{情}}無}{なさけな}くも
\ruby[g]{{\換字{情}}無}{なさけな}し。
%
せめて
\ruby{他}{ひと}に
\ruby{打}{うち}
\ruby{擲}{たゝ}かれて% 踊り字調整「〻(二の字点、揺すり点)に見えるが(ゝ)」
\ruby{憤}{いかり}を
\ruby{發}{おこ}して、
%
\ruby{思}{おも}ひ
\ruby{切}{き}る
\ruby{事}{こと}の
\ruby{叶}{かな}ふ
ほどの
\ruby{淺}{あさ}き
\ruby{戀}{こひ}
ならば、
%
\ruby{此}{こ}の
\ruby{頼}{たの}もしき
\ruby{我}{わ}が
\ruby{友}{とも}の
\ruby[g]{{\換字{情}}誼}{なさけ }に、
%
\ruby{打}{う}つて〳〵
\ruby[g]{脊骨}{せ ぼね}
\ruby[g]{首骨}{くびほね}の
\ruby{碎}{くだ}くる
ほど
\ruby{打}{う}つて
\ruby{貰}{もら}はんを
\改行% 校正作業の簡略化のため
、
%
\原本頁{243-4}\改行%
\ruby{打}{う}たれても
\ruby{擲}{たゝ}かれても% 踊り字調整「〻(二の字点、揺すり点)に見えるが(ゝ)」
\ruby{我}{わ}が
\ruby{心}{こゝろ}の、% 踊り字調整「〻(二の字点、揺すり点)に見えるが(ゝ)」
%
\ruby{死}{し}に
\ruby{{\換字{近}}}{ちか}き
\ruby{馬}{うま}の
やうに
\ruby{動}{うご}かぬが
\ruby[g]{{\換字{情}}無}{なさけな}い!。
%
\ruby{打}{う}たれ
\ruby{辱}{はづか}しめ
られたが
\ruby{悲}{かな}しくも
\ruby{無}{な}く、
%
\ruby{打}{う}たれて
\ruby{云}{い}ひ
% \原本頁{243-6}\改行%
\ruby[||j>]{抗}{あらそ}ふ
ことの
\ruby[g]{出來}{で き }ない
のも
\ruby{悲}{かな}しくは
\ruby{無}{な}いが、
%
たゞ% 踊り字調整「〻(二の字点、揺すり点)に濁点に見えるが(ゞ)」
\ruby{物}{もの}も
\ruby{云}{い}はず
\ruby{怒}{いか}り
もせずに
\ruby[g]{凝然}{じ つ }と
\ruby{仕}{し}て
\ruby{居}{ゐ}て
\ruby{人}{ひと}に
\ruby{打}{う}たれた
ぎりで、
%
\ruby{吾}{わ}が
\ruby{迷}{まよひ}を
\ruby{棄}{す}てやう
\ruby{思}{おもひ}を
\ruby{忘}{わす}れやう
といふ
\ruby{意}{こゝろ}が、% 踊り字調整「〻(二の字点、揺すり点)に見えるが(ゝ)」
%
\ruby[g]{何處}{ど こ }からも
\ruby{出}{で}て
\ruby{來}{こ}ぬ
ほどに
\ruby{愚}{おろか}にも
\原本頁{243-9}\改行%
\ruby{思}{おも}ひこんだ
\ruby[g]{自{\換字{分}}}{じ ぶん}が
\ruby{悲}{かな}しい
\ruby[g]{{\換字{情}}無}{なさけな}い!。
%
と
\ruby{擡}{あ}げし
\ruby{頭}{かしら}を
\ruby[g]{何時}{い つ }か
また
\ruby{下}{さ}げ、
%
\ruby[g]{一度}{ひとたび}
\ruby{肩}{かた}を
\ruby{聳}{そびや}かしたる
\ruby{身}{み}の
\ruby{復}{また}
\ruby{崩}{くづ}
\ruby{折}{を}るれば、
%
\ruby{其}{そ}の
\ruby[g]{樣子}{やうす }を
\ruby{見}{み}て
\ruby{取}{と}りて
\ruby[g]{日方}{ひ かた}は
いよ〳〵
\ruby[g]{齒痒}{は がゆ}がり。
%』% この閉じカッコ』の対は無いので削除
\footnote{原本は「齒痒がり。』」であるが、対の「『」がないので「』」を削除
(国会図書館 コマ番号126/160 p-243 l-15)}%

\原本頁{244-1}\改行%
『
エヽ
\ruby[g]{男兒}{をとこ }
らしくも
\ruby{無}{な}い、
%
\ruby{其}{その}
\ruby{面}{つら}は
\ruby{何}{なん}だ!。
%
\ruby{身}{み}を
\ruby{{\換字{退}}}{ひ}いて
\ruby{眼}{め}を
\ruby{{\換字{睜}}}{みは}つて
\ruby[g]{乃公}{お れ }を
\ruby{見}{み}た
\ruby{時}{とき}は、
%
\ruby[g]{水野}{みづの }
\ruby{汝}{きさま}も
まだ
\ruby{話}{はな}せると
\ruby{思}{おも}つたが、
%
やがて
\ruby{直}{すぐ}
\原本頁{244-3}\改行%
に
\ruby{力}{ちから}の
\ruby{脫}{ぬ}けた
\ruby{泣}{な}きつ
\ruby{面}{つら}に
なつて、
%
\ruby{涙}{なみだ}
ぐんで
\ruby{俯}{うつむ}いた
のあ、
%
アヽ
\ruby{見}{み}ぐるしいは。
%
なるほど
\ruby{大{\換字{丈}}夫}{ます|ら|を}
のさとき
\ruby{心}{こゝろ}も% 踊り字調整「〻(二の字点、揺すり点)に見えるが(ゝ)」
\ruby{今}{いま}は
\ruby{無}{な}い
だらう、
%
\ruby{其}{そ}の
\ruby[g]{狀態}{やうす }
ぢやあ
\ruby{戀}{こひ}の
\ruby[g]{奴と}{やつこ }
\ruby{死}{し}ぬのも
\ruby{{\換字{遠}}}{とほ}く
も
あるまい。
%
\ruby{汝}{きさま}は
\ruby{戀}{こひ}の
\ruby{奴}{やつこ}
と
\原本頁{244-6}\改行%
なつて
\ruby{死}{し}ぬのが
\ruby[g]{本望}{ほんまう}か
\ruby{知}{し}らんが、
%
\ruby{氣}{き}の
\ruby{毒}{どく}だが
\ruby[g]{左樣}{さ う }は
\ruby[g]{乃公}{お れ }が
\ruby{死}{し}なさん。
%
ヤイ
\ruby[g]{水野}{みづの }、
%
\ruby[g]{日方}{ひ かた}は
いたづらに
\ruby[g]{怒罵}{ど ば }
\ruby[g]{暴行}{ばうかう}は
せん、
%
たゞ% 踊り字調整「〻(二の字点、揺すり点)に濁点に見えるが(ゞ)」
\ruby[g]{大切}{たいせつ}の
\ruby[g]{一人}{ひとり }の
\ruby[g]{朋友}{と も }の
\ruby{爲}{ため}にナ。
%
\ruby{才}{さい}を
\ruby{惜}{をし}み
\ruby{名}{な}を
\ruby{惜}{をし}んで
\ruby{{\換字{遣}}}{や}れば
こそ
\ruby{爭}{あらそ}ふ
のだ。
%
\ruby[g]{乃公}{お れ }の
\ruby[g]{大切}{たいせつ}の
\ruby[g]{朋友}{ともだち}の
\ruby[g]{水野}{みづの }
\ruby{何}{なに}
\ruby{某}{がし}を、
%
\ruby{一}{いつ}
\ruby[g]{{\換字{婦}}人}{ぷ じん}に
\ruby{{\換字{迷}}}{まよ}つて
\ruby{戀}{こひ}に
\ruby{死}{し}んだ
とは
\ruby{笑}{わら}は
さん。
%
とても
\ruby{汝}{きさま}が
\ruby{戀}{こひ}に
\ruby{死}{し}ぬ
ほど
ならば、
%
\ruby{此}{こ}の
\ruby[g]{日方}{ひ かた}
\ruby[g]{八郎}{はちらう}が
\ruby{打}{ぶち}
\ruby{殺}{ころ}して
\ruby{{\換字{遣}}}{や}る。
%
\ruby{汝}{きさま}は
\ruby[g]{羽{\換字{勝}}}{は がち}の
\ruby{會}{くわい}へも
\ruby{出}{で}て
\ruby{來}{こ}なかつた
ほど、
%
\ruby[g]{朋友}{ともだち}には
\ruby{薄}{うす}く
\ruby{戀}{こひ}に
\ruby{厚}{あつ}くつても、
%
\ruby[g]{乃公}{お れ }は
\ruby[g]{朋友}{ともだち}には
\ruby{厚}{あつ}くする、
%
\ruby{戀}{こひ}には
\原本頁{245-2}\改行%
\ruby{關}{かま}はん。
%
\ruby[g]{{\換字{父}}母}{ふ ぼ }の
\ruby{名}{な}も
\ruby{顯}{あらは}さんで
\ruby{戀}{こひ}に
\ruby{死}{し}なう
とは
\ruby[g]{不孝}{ふ かう}な
\ruby{奴}{やつ}だ、
%
\ruby[g]{國民}{こくみん}の
\ruby[g]{義務}{ぎ む }も
\ruby{碌}{ろく}に
\ruby{果}{はた}さんで
\ruby{戀}{こひ}に
\ruby{死}{し}なうとは
\ruby[g]{不義}{ふ ぎ }な
\ruby{奴}{やつ}だ、
%
\ruby{生}{せい}を
\ruby{此}{この}
\ruby{世}{よ}に
\ruby{受}{う}けた
\ruby[g]{甲{\換字{斐}}}{か ひ }も
\ruby{殘}{のこ}さんで
\ruby{{\換字{空}}}{むな}しく
\ruby{死}{し}なう
とは
\ruby[g]{卑劣}{ひ れつ}
きはまる!。
%
\ruby{身}{み}
\ruby[g]{{\換字{勝}}手}{がつて }
ばかりの
\ruby[g]{穀潰}{ごくつぶ}し
とは
\ruby{戀}{こひ}に
\ruby{死}{し}ぬ
やうな
\ruby[g]{白痴}{たはけ }た
\ruby{奴}{やつ}の
\ruby{事}{こと}だ。
%
\ruby{才}{さい}を
\ruby{惜}{をし}んで
\ruby{及}{およ}ばん
\ruby[g]{以上}{いじやう}は
\ruby{名}{な}を
\ruby{惜}{をし}んでやる!。
%
\ruby{汝}{きさま}を
\ruby[g]{不孝}{ふ かう}
\ruby[g]{不義}{ふ ぎ }
\ruby[g]{卑劣}{ひ れつ}な
\ruby[g]{穀潰}{ごくつぶ}し
とは
\ruby{呼}{よ}ばさん、
%
\ruby{戀}{こひ}には
\ruby{死}{し}なせん、
%
\ruby{打}{ぶち}
\ruby{殺}{ころ}すが
\ruby[g]{何樣}{ど う }だ。
』

\原本頁{245-8}%
と
\ruby[g]{激語}{げきご }は
\ruby{口}{くち}より
\ruby{出}{い}づるに
\ruby{任}{まか}せて、
%
ふたゝび% 踊り字調整「〻(二の字点、揺すり点)に見えるが(ゝ)」
\ruby[g]{水野}{みづの }を
\ruby{引}{ひき}
\ruby{据}{す}ゑて
\ruby{打}{う}たん
とする
\ruby{時}{とき}、
%
\ruby{隔}{へだて}の
\ruby{襖}{ふすま}は
すらりと
\ruby{明}{あ}きて、
%
\ruby{春}{はる}の
\ruby{燕}{つばめ}と
\ruby{身}{み}も
\ruby{輕}{かろ}く、
%
ひらりと
\ruby{躍}{をど}り
\ruby{入}{い}つたる
お
\ruby{濱}{はま}は、
%
\ruby[g]{突然}{いきなり}に
\ruby[g]{日方}{ひ かた}の
\ruby{{\換字{拳}}}{こぶし}に
\ruby{取}{と}りつきて、
%
\ruby{是}{これ}はと
\ruby{{\換字{迷}}}{まよ}ひ
\ruby{疑}{うたが}ふ
\ruby{間}{あひだ}に、
%
\ruby{早}{はや}くも
\ruby{其}{その}
\ruby{手}{て}より
\ruby{普門品}{ふ|もん|ぼん}を
\ruby{奪}{うば}つて、
%
\ruby[g]{口惜}{く や }しさ
\原本頁{246-1}\改行%
\ruby{憎}{にく}さ
\ruby{取}{と}り
\ruby{{\換字{交}}}{ま}ぜて
\ruby{籠}{こ}むる
\ruby{力}{ちから}の
\ruby{有}{あ}らん
\ruby{限}{かぎ}りに、
%
\ruby[g]{日方}{ひ かた}の
\ruby{五{\換字{分}}苅}{ご|ぶ|がり}
\ruby{頭}{あたま}を
ぴしや〳〵と% 原本通り踊り字表記(行末行頭の境界付近)
\ruby{打}{う}つたり。
