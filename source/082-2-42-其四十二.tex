\Entry{其四十二}

% メモ 校正終了 2024-05-07 2024-06-05
\原本頁{246-4}%
\ruby{犬坊丸}{いぬ|ばう|まる}に% 蘇我物語犬吠丸の伝説
\ruby[g]{鞭撻}{むちう }たれたる
\ruby[g]{曾我}{そ が }の
\ruby[g]{五郎}{ご らう}を
\ruby{今}{いま}
\ruby{樣}{やう}にして
\ruby{見}{み}るごとき
\ruby[g]{日方}{ひ かた}は
\ruby{且}{かつ}
\ruby{驚}{おどろ}き
\ruby{且}{かつ}
\ruby{呆}{あき}れて、
%
\ruby{眼}{め}を
\ruby{圓}{まる}くして
\ruby{我}{われ}を
\ruby{打}{う}つものを
\ruby[g]{何者}{なにもの}と
\ruby{屹}{きつ}と
\ruby{睨}{にら}めば、
%
\ruby[g]{夕日}{ゆふひ }
かゞやく% 踊り字調整「〻(二の字点、揺すり点)に濁点に見えるが(ゞ)」
\ruby[g]{緋櫻}{ひざくら}と
\ruby{燃}{も}{\換字{𛀁}}
\ruby{立}{た}つ
\ruby{顏}{かほ}して、
%
\ruby{匂}{にほ}やかなる
\ruby{眉}{まゆ}を
\ruby{昻}{あ}げ
\ruby{美}{うつく}しき
\ruby{眼}{め}を
\ruby{瞋}{いか}らせたる
お
\ruby{濱}{はま}は、
%
\ruby{其}{その}
\ruby{時}{とき}
\ruby[g]{日方}{ひ かた}の
\ruby[||j>]{面}{めん}
\ruby[||j>]{上}{じやう}を
% \ruby{面上}{めん|じやう}を
\ruby{望}{のぞ}んで
\ruby{普門品}{ふ|もん|ぼん}を
\ruby{抛}{なげう}ち
\ruby{棄}{す}て、
%
\ruby[g]{物言}{ものい }ふも
\ruby[g]{可厭}{い や }と
\ruby{云}{い}はぬばかりに
\ruby{突}{つ}と
\ruby[g]{後向}{うしろむ}き、
%
\ruby{身}{み}を
\ruby{飜}{ひるが}へして
\ruby{倒}{たふ}るゝが% 踊り字調整「〻(二の字点、揺すり点)に見えるが(ゝ)」
\ruby{如}{ごと}く
\ruby[g]{水野}{みづの }の
\ruby{膝}{ひざ}に
\ruby[g]{突伏}{つゝぷ }し、% 踊り字調整「〻(二の字点、揺すり点)に見えるが(ゝ)」
%
\ruby{忽}{たちま}ち
\ruby{堰}{せ}き
\ruby{上}{あ}げくる
\ruby{涙}{なみだ}の
\ruby{聲}{こゑ}になつて、

\原本頁{246-10}%
『
エヽ
\ruby[g]{口惜}{く や }しい〳〵、
%
あんまり
\ruby[g]{口惜}{く や }しい!。
%
こんな
\ruby[||j>]{醉}{よつ}
\ruby[||j>]{{\換字{漢}}}{ぱらひ}の% 「醉」は原本通り「よ」で調整
% \ruby{醉{\換字{漢}}}{よつ|ぱらひ}の% 「醉」は原本通り「よ」で調整
\ruby{亂暴人}{らん|ばう|にん}に、
%
\ruby[g]{何故}{な ぜ }
\ruby{默}{だま}つて
\ruby{打}{ぶ}たれて
\ruby{居}{ゐ}
\ruby{無}{な}くては
いけないの?。
%
\ruby[g]{何故}{な ぜ }
\ruby[g]{打{\換字{返}}}{ぶちかへ}して
やらないの?。
%
だから
\ruby[||j>]{觀}{くわん}% 「觀音」の読みは原本通り「くわん(の)ん」
\ruby[||j>]{音}{ のん}
% \ruby{觀音樣}{くわん|のん}ぞ% 「觀音」の読みは原本通り「くわん(の)ん」
\ruby[||j>]{樣}{ さま}
なんぞ
\ruby[g]{信心}{しん〴〵}するのは
をかしいと
\ruby{云}{い}つて
\ruby{妾}{わたし}が
\ruby{止}{と}めたのに、
%
\ruby[g]{先生}{せんせい}が
\ruby{餘}{あんま}り
\ruby[g]{夢中}{む ちう}に
なるもんだから、
%
\ruby{人}{ひと}に
\ruby[g]{馬鹿}{ば か }にされて
\ruby[g]{此樣}{こ ん }な
\ruby{目}{め}に
\ruby{會}{あ}ふ
やうに
なつたのよ。
%
それも
みんな
\ruby{五十子}{い|そ|こ}さんが
\ruby{惡}{わる}い
お
\ruby{蔭}{かげ}よ、
%
あゝ% 踊り字調整「〻(二の字点、揺すり点)に見えるが(ゝ)」
\ruby[g]{口惜}{く や }しい!。
%
\ruby{妾}{わたし}が
\ruby[g]{口借}{く や }しくつて
\ruby[g]{仕方}{し かた}が
\ruby{無}{な}いから、
%
こんな
\ruby[||j>]{醉}{よつ}
\ruby[||j>]{{\換字{漢}}}{ぱらひ}の% 「醉」は原本通り「よ」で調整
% \ruby{醉{\換字{漢}}}{よつ|ぱらひ}の% 「醉」は原本通り「よ」で調整
\ruby[g]{無茶}{む ちや}な
\ruby{人}{ひと}なんか、
%
\ruby{早}{はや}く
\ruby{妾}{わたし}の
\ruby{家}{うち}から
\ruby{{\換字{逐}}}{お}ひ
\ruby{出}{だ}して
\ruby{{\換字{遣}}}{や}つてよ
\ruby[g]{先生}{せんせい}!。
%
ほんとに
\ruby{憎}{にく}らしい
\ruby{厭}{いや}な
\ruby{奴}{やつ}だつちや
\ruby{無}{な}い。
%
エヽ
\ruby[g]{何故}{な ぜ }
\ruby[g]{先生}{せんせい}は
\ruby{默}{だま}つて
ばかり
\ruby{居}{ゐ}るの!、
%
\ruby{默}{だま}つてちやあ
\ruby[||j>]{妾}{わたし}
\ruby[||j>]{厭}{ いや}よ、
%
\ruby{怒}{おこ}つてよ、
%
\ruby{怒}{おこ}つてよ、
%
\ruby{怒}{おこ}り
\ruby{出}{だ}して
\makeatletter
\@ifundefined{デバッグ@ビルド}{%
  \ruby[g]{頂戴よ}{ちやうだい }、
}{%
  \ruby[||j>]{頂}{ちやう}
  \ruby[||j>]{戴}{ だい}よ、
}%
\makeatother
% \ruby{頂戴}{ちやう|だい}よ、
%
エヽ
\ruby[g]{口惜}{く や }しい。
』

\原本頁{248-1}%
と
\ruby{身}{み}を
\ruby{揉}{も}んで
\ruby{悶}{もだ}ゆる
\ruby{其}{そ}の
\ruby{八}{や}ツ
\ruby{口}{くち}より
\ruby[g]{襦袢}{じゆばん}の
\ruby{袖}{そで}の
\ruby[g]{紅色}{くれなゐ}
こぼれて、
%
\ruby{低}{ひく}く
\ruby{伏}{ふ}したる
\ruby[g]{背中}{せ なか}つきの
すらりと
\ruby{優}{やさ}しきも
いと
しほらしく、
%
それを
\ruby{中}{なか}にして
\ruby{對}{むか}ひ
\ruby{坐}{ざ}せる
\ruby[g]{痩軀}{やせじゝ}の% 踊り字調整「〻(二の字点、揺すり点)に見えるが(ゝ)」
\ruby[g]{水野}{みづの }、
%
\ruby{肥}{こ}{\換字{𛀁}}たる
\ruby[g]{日方}{ひ かた}、
%
\ruby{揉}{も}みくちやに
されて
\ruby{捨}{す}てられたる
\ruby{普門品}{ふ|もん|ぼん}、
%
\ruby{倒}{たふ}されたる
\ruby{葡萄酒}{ぶ|だう|しゆ}の
\ruby{{\換字{空}}洋盞}{から|こつ|ぷ}、
%
\原本頁{248-5}\改行%
すべて
\ruby{是}{これ}
\ruby{亂}{みだ}れたる
\ruby[||j>]{一}{いち}
\ruby[||j>]{塲}{ぢやう}の% 原文通り「塲」
% \ruby{一塲}{いち|ぢやう}の% 原文通り「塲」
\ruby[g]{景色}{け しき}ながら、
%
\ruby{描}{ゑが}かば
\ruby{描}{ゑが}くべき
\ruby[g]{風{\換字{情}}}{ふ ぜい}あり
\改行% 校正作業の簡略化のため
。

\原本頁{248-6}%
\ruby[g]{水野}{みづの }は
\ruby{默}{もく}して
\ruby{石}{いし}の
\ruby{如}{ごと}く
\ruby{語}{かた}らず、
%
\ruby{思}{おも}はぬ
ものに
\ruby{出}{で}られて
\ruby[g]{日方}{ひ かた}は
\ruby{困}{こう}じたる
\ruby{時}{とき}、
%
お
\ruby{鍋}{なべ}は
\ruby[g]{先刻}{さつき }より
\ruby[g]{彼方}{かなた }にて
\ruby{人}{ひと}と
\ruby[g]{應接}{おうせつ}し
\ruby{居}{ゐ}たりしが、
%
\ruby{{\換字{終}}}{つひ}に
\ruby[g]{此處}{こ ゝ }へと% 踊り字調整「〻(二の字点、揺すり点)に見えるが(ゝ)」
\ruby[g]{一人}{いちにん}
\footnote{%
この文脈では、「一人」のルビは(ひとり)と思われるが、
原本では「\ruby[g]{一人}{  にん}」となっていたため、
漢数字のルビ化に伴い(いちにん)とした。
(国会図書館 コマ番号129/134 p-248 l-08)
}%
の
\ruby{男}{をとこ}を
\ruby{導}{みちび}き
\ruby{來}{きた}れり。

\原本頁{248-9}%
『
オヽ
\ruby[g]{羽{\換字{勝}}}{は がち}
か。
』

\原本頁{248-10}%
『
ア、
%
\ruby[g]{羽{\換字{勝}}}{は がち}
\ruby{君}{くん}か。
』

\原本頁{248-11}%
\ruby[g]{日方}{ひ かた}と
\ruby[g]{水野}{みづの }とが
\ruby[g]{同時}{どうじ }に
\ruby{聲}{こゑ}かくるを、
%
\ruby{眞面目}{ま|じ|め}に
\ruby{受}{う}けながら、
%
いつも
\ruby{變}{かは}らぬ
\ruby[g]{洋服}{やうふく}
\ruby{姿}{すがた}の
\ruby[g]{羽{\換字{勝}}}{は がち}は
\ruby{靜}{しづか}に
\ruby{坐}{ざ}して、

\原本頁{249-2}%
『
\ruby[g]{日方}{ひ かた}!、
%
\ruby{君}{きみ}はいかんぞ。
%
\ruby{今}{いま}
\ruby[g]{此家}{こ ゝ }の% 踊り字調整「〻(二の字点、揺すり点)に見えるが(ゝ)」
\ruby{婢}{をんな}に
\ruby[g]{仔細}{し さい}を
\ruby{聞}{き}いたは。
%
\ruby[g]{島木}{しまき }に
\ruby{釘}{くぎ}を
さゝれて% 踊り字調整「〻(二の字点、揺すり点)に見えるが(ゝ)」
\ruby{居}{ゐ}ながら、
%
\ruby{何}{なに}を
するのだ、
%
いかんぞ
\ruby[g]{何樣}{ど う }も!。
%
\原本頁{249-4}\改行%
\ruby[g]{水野}{みづの }!、
%
\ruby{久}{ひさ}しく
\ruby{逢}{あ}はなかつた
ナア。
%
しかし
\ruby{君}{きみ}も
\ruby[g]{無事}{ぶ じ }、
%
\ruby{僕}{ぼく}も
\ruby[g]{無事}{ぶ じ }で、
%
お
\ruby{互}{たがひ}に
\ruby[g]{滿足}{まんぞく}だ。
%
\ruby{實}{じつ}は
\ruby[g]{今日}{け ふ }
\ruby[g]{日方}{ひ かた}と
\ruby[g]{約束}{やくそく}して、
%
\ruby[g]{島木}{しまき }と
\ruby[g]{三人}{さんにん}で
\ruby{君}{きみ}を
\ruby{{\換字{尋}}}{たづ}ねる
\ruby{筈}{はず}だつたが、
%
\ruby{僕}{ぼく}は
\ruby[g]{身體}{からだ }が
\ruby{忙}{いそ}がしかつたので
\ruby{斷}{ことわ}りを
\ruby{出}{だ}したところが、
%
\ruby{思}{おも}ひのほか
\ruby{早}{はや}く
\ruby[g]{身體}{からだ }が
\ruby{明}{あ}いたので、
%
\ruby[g]{島木}{しまき }の
ところへ
\ruby{行}{い}つて
\ruby{見}{み}ると、
%
\ruby[g]{日方}{ひ かた}は
\ruby[g]{一人}{ひとり }で
\ruby[g]{此方}{こつち }へとの% ルビ調整(原本通り)
\ruby{事}{こと}だ。
%
\ruby[g]{島木}{しまき }は
\ruby{何}{なに}か
\ruby[<j||]{商}{しやう}
\ruby{業}{げふ}
\ruby{上}{じやう}
の
% \ruby{商業上}{しやう|げふ|じやう}の
\ruby[g]{推算}{すゐさん}に
\ruby{身}{み}を
\ruby{入}{い}れて
\ruby{居}{ゐ}る
\ruby[g]{樣子}{やうす }で、
%
\ruby{誘}{さそ}つても
\ruby{氣}{き}の
\ruby{無}{な}い
\ruby[g]{{\換字{返}}辭}{へんじ }を
するやうになつて
\ruby{居}{ゐ}るし、
%
そこで
\ruby[g]{一人}{ひとり }で
\ruby{後}{あと}を
\ruby{{\換字{追}}}{お}つて
\ruby{{\換字{遣}}}{や}つて
\ruby{來}{き}たが、
%
ひよつと
すると
\ruby[g]{日方}{ひ かた}が
\ruby[g]{言葉}{ことば }に
\ruby{募}{つの}つて
\ruby{暴}{ばう}な
\ruby{事}{こと}でも
\ruby{仕}{し}はせぬか
\原本頁{250-1}\改行%
と
\ruby{思}{おも}つた
\ruby{{\換字{通}}}{とほ}りに、
%
\ruby{來}{き}て
\ruby{見}{み}ると
\ruby{果}{はた}して
\ruby[g]{亂暴}{らんばう}の
\ruby[g]{{\換字{所}}爲}{しわざ }だ。% 「所爲」(せい)(しよい)その人の行為から起こった結果、そのことによる結果。
%
\ruby{然}{しか}し
まあ
\ruby{僕}{ぼく}に
\ruby{免}{めん}じて
\ruby{赦}{ゆる}して
\ruby{吳}{く}れたまへ、
%
\ruby{何}{なに}も
\ruby[g]{惡氣}{わるぎ }では
\ruby{爲}{せ}ん
\ruby[g]{日方}{ひ かた}
だから。
%
もう
\ruby{僕}{ぼく}が
\ruby{來}{き}た
\ruby{上}{うへ}は
\ruby{暴}{ばう}は
させん、
%
\ruby[g]{三人}{さんにん}で
\ruby[<j>]{快}{こゝろよ}く% 踊り字調整「〻(二の字点、揺すり点)に見えるが(ゝ)」
\ruby{靜}{しづか}に
\ruby{話}{はな}さう。
%
\ruby[g]{水野}{みづの }、
%
\ruby{君}{きみ}は
\ruby{今}{いま}でも
\ruby{甘}{あま}い
\ruby{黨}{たう}の
\ruby{方}{はう}だらう。
%
\ruby{小兒欺}{こ|ども|だま}しだが
\ruby{舶來菓子}{はく|らい|ぐわ|し}を
\ruby{少}{すこ}し
\ruby{持}{も}つて
\ruby{來}{き}た。
%
\ruby[g]{此邊}{こゝら }には% 踊り字調整「〻(二の字点、揺すり点)に見えるが(ゝ)」
\ruby{珍}{めづ}しからうと
\ruby{思}{おも}つて、
%
\ruby{枕絹}{サイデ\換字{子}|キツシエン}
とか
バタカツプ
とか
いふ
\ruby{奴}{やつ}を
\ruby{持}{も}つて
\ruby{來}{き}たが、
%
\ruby[g]{舟人}{ふなのり}の
\ruby{酒}{さけ}を
\ruby{{\換字{強}}}{つよ}く
\ruby{好}{す}かん
\ruby{奴}{やつ}は
\ruby[g]{菓子}{くわし }に
\ruby[g]{趣味}{たのしみ}を
\ruby{有}{も}つ
\ruby{癖}{くせ}が
\ruby{出}{で}るのも
をかしい
ことだ。
%
さあ
\ruby[g]{日方}{ひ かた}は
\ruby{飮}{の}むなら
\ruby{飮}{の}め、
%
\ruby[g]{此方}{こつち }は% ルビ調整(原本通り)
\ruby{茶}{ちや}で
\ruby{談}{はな}さう。
』

\原本頁{250-9}%
と
\ruby{常}{つね}には
\ruby{似}{に}ず
\ruby{勉}{つと}めて
\ruby[g]{口數}{くちかず}
きゝて、% 踊り字調整「〻(二の字点、揺すり点)に見えるが(ゝ)」
%
\ruby{白}{しら}け
きつたる
\ruby{此}{この}
\ruby{坐}{ざ}を
\ruby{黑}{くろ}めん
とすれば、
%
お
\ruby{濱}{はま}は
\ruby{竊}{そつ}と
\ruby{其}{その}
\ruby{人}{ひと}を
\ruby{覗}{うかゞ}ひ% 踊り字調整「〻(二の字点、揺すり点)に濁点に見えるが(ゞ)」
\ruby{見}{み}て、
%
\ruby{正}{たゞ}しげなる% 踊り字調整「〻(二の字点、揺すり点)に濁点に見えるが(ゞ)」
\ruby{此}{こ}の
\ruby[g]{新來}{しんらい}の
\ruby[<j||]{客}{きやく}に、% 行末行頭の境界付近なので特例処置を施す
%
\ruby[g]{泣顏}{なきがほ}
\ruby{見}{み}せん
\ruby{事}{こと}を
\ruby{憂}{う}く
おもひてや、
%
\ruby{面}{おもて}を
\ruby{蔽}{かく}して
\ruby{{\換字{逃}}}{に}ぐるが
\ruby{如}{ごと}くに
\ruby[g]{此處}{こ ゝ }を% 踊り字調整「〻(二の字点、揺すり点)に見えるが(ゝ)」
\ruby{去}{さ}つたり。
