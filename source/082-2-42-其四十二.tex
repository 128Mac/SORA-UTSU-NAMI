\Entry{其四十二}

\原本頁{}%
\ruby{犬坊丸}{いぬ|ばう|まる}に
\ruby{鞭撻}{むち|う}たれたる
\ruby{曾我}{そ|が}の
\ruby{五郎}{ご|らう}を
\ruby{今}{いま}
\ruby{樣}{やう}にして
\ruby{見}{み}るごとき
\ruby{日方}{ひ|かた}は
\ruby{且}{かつ}
\ruby{驚}{おどろ}き
\ruby{且}{かつ}
\ruby{呆}{あき}れて、
%
\ruby{眼}{め}を
\ruby{圓}{まる}くして
\ruby{我}{われ}を
\ruby{打}{う}つものを
\ruby{何者}{なに|もの}と
\ruby{屹}{きつ}と
\ruby{睨}{にら}めば、
%
\ruby{夕日}{ゆふ|ひ}かゞやく
\ruby{緋櫻}{ひ|ざくら}と
\ruby{燃}{も}{\換字{𛀁}}
\ruby{立}{た}つ
\ruby{顏}{かほ}して、
%
\ruby{匂}{にほ}やかなる
\ruby{眉}{まゆ}を
\ruby{昻}{あ}げ
\ruby{美}{うつく}しき
\ruby{眼}{め}を
\ruby{瞋}{いか}らせたる
お
\ruby{濱}{はま}は、
%
\ruby{其}{その}
\ruby{時}{とき}
\ruby{日方}{ひ|かた}の
\ruby[||j>]{面}{めん}
\ruby[||j>]{上}{じやう}を
% \ruby{面上}{めん|じやう}を
\ruby{望}{のぞ}んで
\ruby{普門品}{ふ|もん|ぼん}を
\ruby{抛}{なげう}ち
\ruby{棄}{す}て、
%
\ruby{物言}{もの|い}ふも
\ruby{可厭}{い|や}と
\ruby{云}{い}はぬばかりに
\ruby{突}{つ}と
\ruby{後向}{うし|ろむ}き、
%
\ruby{身}{み}を
\ruby{飜}{ひるが}へして
\ruby{倒}{たふ}るゝが
\ruby{如}{ごと}く
\ruby{水野}{みづ|の}の
\ruby{膝}{ひざ}に
\ruby{突伏}{つゝ|ぷ}し、
%
\ruby{忽}{たちま}ち
\ruby{堰}{せ}き
\ruby{上}{あ}げくる
\ruby{涙}{なみだ}の
\ruby{聲}{こゑ}になつて、

\原本頁{}%
『エヽ
\ruby{口惜}{く|や}しい〳〵、
%
あんまり
\ruby{口惜}{く|や}しい!。
%
こんな
\ruby[||j>]{醉}{よつ}
\ruby[||j>]{{\換字{漢}}}{ぱらひ}の% 「醉」は原本通り「よ」で調整
% \ruby{醉{\換字{漢}}}{よつ|ぱらひ}の% 「醉」は原本通り「よ」で調整
\ruby{亂暴人}{らん|ばう|にん}に、
%
\ruby{何故}{な|ぜ}
\ruby{默}{だま}つて
\ruby{打}{ぶ}たれて
\ruby{居無}{ゐ|な}くてはいけないの?。
%
\ruby{何故}{な|ぜ}
\ruby{打{\換字{返}}}{ぶち|かへ}してやらないの?。
%
だから
\ruby{觀音樣}{くわん|のん|さま}なんぞ% 「觀音」の読みは原本通り「くわん(の)ん」
\ruby{信心}{しん|〴〵}するのはをかしいと
\ruby{云}{い}つて
\ruby{妾}{わたし}が
\ruby{止}{と}めたのに、
%
\ruby{先生}{せん|せい}が
\ruby{餘}{あんま}り
\ruby{夢中}{む|ちう}になるもんだから、
%
\ruby{人}{ひと}に
\ruby{馬鹿}{ば|か}にされて
\ruby{此樣}{こ|ん}な
\ruby{目}{め}に
\ruby{會}{あ}ふやうになつたのよ。
%
それもみんな
\ruby{五十子}{い|そ|こ}さんが
\ruby{惡}{わる}い
お
\ruby{蔭}{かげ}よ、
%
あゝ
\ruby{口惜}{く|や}しい!。
%
\ruby{妾}{わたし}が
\ruby{口借}{く|や}しくつて
\ruby{仕方}{し|かた}が
\ruby{無}{な}いから、
%
こんな
\ruby[||j>]{醉}{よつ}
\ruby[||j>]{{\換字{漢}}}{ぱらひ}の% 「醉」は原本通り「よ」で調整
% \ruby{醉{\換字{漢}}}{よつ|ぱらひ}の% 「醉」は原本通り「よ」で調整
\ruby{無茶}{む|ちや}な
\ruby{人}{ひと}なんか、
%
\ruby{早}{はや}く
\ruby{妾}{わたし}の
\ruby{家}{うち}か
\ruby{{\換字{逐}}}{お}ひ
\ruby{出}{だ}して
\ruby{{\換字{遣}}}{や}つてよ
\ruby{先生}{せん|せい}!。
%
ほんとに
\ruby{憎}{にく}らしい
\ruby{厭}{いや}な
\ruby{奴}{やつ}だつちや
\ruby{無}{な}い。
%
エヽ
\ruby{何故}{な|ぜ}
\ruby{先生}{せん|せい}は
\ruby{默}{だま}つてばかり
\ruby{居}{ゐ}るの!、
%
\ruby{默}{だま}つてちやあ
\ruby[<j||]{妾}{わたし}
\ruby{厭}{いや}よ、
%
\ruby{怒}{おこ}つてよ、
%
\ruby{怒}{おこ}つてよ、
%
\ruby{怒}{おこ}り
\ruby{出}{だ}して
\ruby[||j>]{頂}{ちやう}
\ruby[||j>]{戴}{ だい}よ、
% \ruby{頂戴}{ちやう|だい}よ、
%
エヾ
\ruby{口惜}{く|や}しい。
』

\原本頁{}%
と
\ruby{身}{み}を
\ruby{揉}{も}んで
\ruby{悶}{もだ}ゆる
\ruby{其}{そ}の
\ruby{八}{や}ツ
\ruby{口}{くち}より
\ruby{襦袢}{じゆ|ばん}の
\ruby{袖}{そで}の
\ruby{紅色}{くれ|なゐ}こぼれて、
%
\ruby{低}{ひく}く
\ruby{伏}{ふ}したる
\ruby{背中}{せ|なか}つきのすらりと
\ruby{優}{やさ}しきもいとしほらしく、
%
それを
\ruby{中}{なか}にして
\ruby{對}{むか}ひ
\ruby{坐}{ざ}せる
\ruby{痩軀}{やせ|じゝ}の
\ruby{水野}{みづ|の}、
%
\ruby{肥}{こ}{\換字{𛀁}}たる
\ruby{日方}{ひ|かた}、
%
\ruby{揉}{も}みくちやにされて
\ruby{捨}{す}てられたる
\ruby{普門品}{ふ|もん|ぼん}、
%
\ruby{倒}{たふ}されたる
\ruby{葡萄酒}{ぶ|だう|しゆ}の
\ruby{{\換字{空}}洋盞}{から|こつ|ぷ}、
%
すべで
\ruby{是}{これ}
\ruby{亂}{みだ}れたる
\ruby[||j>]{一}{いち}
\ruby[||j>]{塲}{ぢやう}の% 原文通り「塲」
% \ruby{一塲}{いち|ぢやう}の% 原文通り「塲」
\ruby{景色}{け|しき}ながら、
%
\ruby{描}{ゑが}かば
\ruby{描}{ゑが}くべき
\ruby{風{\換字{情}}}{ふ|ぜい}あり。

\原本頁{}%
\ruby{水野}{みづ|の}は
\ruby{默}{もく}して
\ruby{石}{いし}の
\ruby{如}{ごと}く
\ruby{語}{かた}らず、
%
\ruby{思}{おも}はぬものに
\ruby{出}{で}られて
\ruby{日方}{ひ|かた}は
\ruby{困}{こう}じたる
\ruby{時}{とき}、
%
お
\ruby{鍋}{なべ}は
\ruby{先刻}{さつ|き}より
\ruby{彼方}{かな|た}にて
\ruby{人}{ひと}と
\ruby{應接}{おう|せつ}し
\ruby{居}{ゐ}たりしが、
%
\ruby{{\換字{終}}}{つひ}に
\ruby{此處}{こ|ゝ}へと
\ruby{一人}{いち|にん}の
\ruby{男}{をとこ}を
\ruby{導}{みちび}き
\ruby{來}{きた}れり。

\原本頁{}%
『オヽ
\ruby{羽{\換字{勝}}}{は|がち}か。
』

\原本頁{}%
『ア、
%
\ruby{羽{\換字{勝}}}{は|がち}
\ruby{君}{くん}か。
』

\原本頁{}%
\ruby{日方}{ひ|かた}と
\ruby{水野}{みづ|の}とが
\ruby{同時}{どう|じ}に
\ruby{聲}{こゑ}かくるを、
%
\ruby{眞面目}{ま|じ|め}に
\ruby{受}{う}けながら、
%
いつも
\ruby{變}{かは}らぬ
\ruby{洋服}{やう|ふく}
\ruby{姿}{すがた}の
\ruby{羽{\換字{勝}}}{は|がち}は
\ruby{靜}{しづか}に
\ruby{坐}{ざ}して、

\原本頁{}%
『
\ruby{日方}{ひ|かた}!、
%
\ruby{君}{きみ}はいかんぞ。
%
\ruby{今}{いま}
\ruby{此家}{こ|ゝ}の
\ruby{婢}{をんな}に
\ruby{仔細}{し|さい}を
\ruby{聞}{き}いたは。
%
\ruby{島木}{しま|き}に
\ruby{釘}{くぎ}をさゝれて
\ruby{居}{ゐ}ながら、
%
\ruby{何}{なに}をするのだ、
%
いかんぞ
\ruby{何樣}{ど|う}も!。
%
\ruby{水野}{みづ|の}!、
%
\ruby{久}{ひさ}しく
\ruby{逢}{あ}はなかつたナア。
%
しかし
\ruby{君}{きみ}も
\ruby{無事}{ぶ|じ}、
%
\ruby{僕}{ぼく}も
\ruby{無事}{ぶ|じ}で、
%
お
\ruby{互}{たがひ}に
\ruby{滿足}{まん|ぞく}だ。
%
\ruby{實}{じつ}は
\ruby{今日}{け|ふ}
\ruby{日方}{ひ|かた}と
\ruby{約束}{やく|そく}して、
%
\ruby{島木}{しま|き}と
\ruby{三人}{さん|にん}で
\ruby{君}{きみ}を
\ruby{{\換字{尋}}}{たづ}ねる
\ruby{筈}{はず}だつたが、
%
\ruby{僕}{ぼく}は
\ruby{身體}{から|だ}が
\ruby{忙}{いそ}がしかつたので
\ruby{斷}{ことわ}りを
\ruby{出}{だ}したところが、
%
\ruby{思}{おも}ひのほか
\ruby{早}{はや}く
\ruby{身體}{から|だ}が
\ruby{明}{あ}いたので、
%
\ruby{島木}{しま|き}のところへ
\ruby{行}{い}つて
\ruby{見}{み}ると、
%
\ruby{日方}{ひ|かた}は
\ruby{一人}{ひと|り}で
\ruby{此方}{こつ|ち}へとの
\ruby{事}{こと}だ。
%
\ruby{島木}{しま|き}は
\ruby{何}{なに}か
\ruby{商業上}{しやう|げふ|じやう}の
\ruby{推算}{すゐ|さん}に
\ruby{身}{み}を
\ruby{入}{い}れて
\ruby{居}{ゐ}る
\ruby{樣子}{やう|す}で、
%
\ruby{誘}{さそ}つても
\ruby{氣}{き}の
\ruby{無}{な}い
\ruby{{\換字{返}}辭}{へん|じ}をするやうになつて
\ruby{居}{ゐ}るし、
%
そこで
\ruby{一人}{ひと|り}で
\ruby{後}{あと}を
\ruby{{\換字{追}}}{お}つて
\ruby{{\換字{遣}}}{や}つて
\ruby{來}{き}たが、
%
ひよつとすると
\ruby{日方}{ひ|かた}が
\ruby{言葉}{こと|ば}に
\ruby{募}{つの}つて
\ruby{暴}{ばう}な
\ruby{事}{こと}でも
\ruby{仕}{し}はせぬかと
\ruby{思}{おも}つた
\ruby{{\換字{通}}}{とほ}りに、
%
\ruby{來}{き}て
\ruby{見}{み}ると
\ruby{果}{はた}して
\ruby{亂暴}{らん|ばう}の
\ruby{{\換字{所}}爲}{し|わざ}だ。
%
\ruby{然}{しか}しまあ
\ruby{僕}{ぼく}に
\ruby{免}{めん}じて
\ruby{赦}{ゆる}して
\ruby{吳}{く}れたまへ、
%
\ruby{何}{なに}も
\ruby{惡氣}{わる|ぎ}では
\ruby{爲}{せ}ん
\ruby{日方}{ひ|かた}だから。
%
もう
\ruby{僕}{ぼく}が
\ruby{來}{き}た
\ruby{上}{うへ}は
\ruby{暴}{ばう}はさせん、
%
\ruby{三人}{さん|にん}で
\ruby{快}{こゝろよ}く
\ruby{靜}{しづか}に
\ruby{話}{はな}さう。
%
\ruby{水野}{みづ|の}、
%
\ruby{君}{きみ}は
\ruby{今}{いま}でも
\ruby{甘}{あま}い
\ruby{黨}{たう}の
\ruby{方}{はう}だらう。
%
\ruby{小兒欺}{こ|ども|だま}しだが
\ruby{舶來菓子}{はく|らい|ぐわ|し}を
\ruby{少}{すこ}
\ruby{持}{も}つて
\ruby{來}{き}た。
%
\ruby{此邊}{こゝ|ら}には
\ruby{珍}{めづ}しからうと
\ruby{思}{おも}つて、
%
\ruby{枕絹}{サイデ\換字{子}|キツシエン}とかバタカツプとかいふ
\ruby{奴}{やつ}を
\ruby{持}{も}つて
\ruby{來}{き}たが、
%
\ruby{舟人}{ふな|のり}の
\ruby{酒}{さけ}を
\ruby{{\換字{強}}}{つよ}く
\ruby{好}{す}かん
\ruby{奴}{やつ}は
\ruby{菓子}{くわ|し}に
\ruby{趣味}{たの|しみ}を
\ruby{有}{も}つ
\ruby{癖}{くせ}が
\ruby{出}{で}るのもをかしいことだ。
%
さあ
\ruby{日方}{ひ|かた}は
\ruby{飮}{の}むなら
\ruby{飮}{の}め、
%
\ruby{此方}{こつ|ち}は
\ruby{茶}{ちや}で
\ruby{談}{はな}さう。
』

\原本頁{}%
と
\ruby{常}{つね}には
\ruby{似}{に}ず
\ruby{勉}{つと}めて
\ruby{口數}{くち|かず}きゝて、
%
\ruby{白}{しら}けきつたる
\ruby{此坐}{この|ざ}を
\ruby{黑}{くろ}めんとすれば、
%
お
\ruby{濱}{はま}は
\ruby{竊}{そつ}と
\ruby{其}{その}
\ruby{人}{ひと}を
\ruby{覗}{うかゞ}ひ
\ruby{見}{み}て、
%
\ruby{正}{たゞ}しげなる
\ruby{此}{こ}の
\ruby{新來}{しん|らい}の
\ruby{客}{きやく}に、
%
\ruby{泣顏}{なき|がほ}
\ruby{見}{み}せん
\ruby{事}{こと}を
\ruby{憂}{う}くおもひてや、
%
\ruby{面}{おもて}を
\ruby{蔽}{かく}して
\ruby{{\換字{逃}}}{に}ぐるが
\ruby{如}{ごと}くに
\ruby{此處}{こ|ゝ}を
\ruby{去}{さ}つたり。
