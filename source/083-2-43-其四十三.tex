\Entry{其四十三}

% メモ 校正終了 2024-05-07 2024-06-05
\原本頁{251-3}%
\ruby[g]{島木}{しまき }の
\ruby{胸}{むね}
\ruby{濶}{ひろ}くして
\ruby{能}{よ}く
\ruby[||j>]{人}{にん}
\ruby[||j>]{{\換字{情}}}{じやう}に
% \ruby{人{\換字{情}}}{にん|じやう}に
\ruby{{\換字{通}}}{つう}ぜる
といひ、
%
\ruby[g]{日方}{ひ かた}の
\ruby[||j>]{心}{こゝろ}% 踊り字調整「〻(二の字点、揺すり点)に見えるが(ゝ)」
\ruby[||j>]{剛}{ がう}にして
% \ruby{心剛}{こゝろ|がう}にして% 踊り字調整「〻(二の字点、揺すり点)に見えるが(ゝ)」
\ruby{{\換字{飽}}}{あく}まで
\ruby[g]{義理}{ぎ り }に
\ruby{仗}{よ}らん
とする
といひ、
%
\ruby{其}{その}
\ruby{他}{た}
\ruby[g]{山瀬}{やませ }といひ
\ruby[g]{楢井}{ならい }といひ
\改行% 校正作業の簡略化のため
、
%
\原本頁{251-5}\改行%
いづれも
\ruby{我}{われ}に
\ruby{取}{と}りては
おろか
ならぬ
\ruby{友}{とも}なるが、
%
わけて
\ruby{誰}{たれ}にも
\ruby{彼}{かれ}にも
\ruby{優}{まさ}りて
\ruby{我}{わ}が
\ruby{親}{した}しく
\ruby{語}{かた}らひて、
%
\ruby{眞}{まこと}の
\ruby{兄}{あに}とも
\ruby{頼}{たの}み
\ruby{思}{おも}へるは
\ruby{此}{こ}の
\原本頁{251-7}\改行%
\ruby[g]{羽{\換字{勝}}}{は がち}なり。
%
\ruby{其}{その}
\ruby[g]{性質}{せいしつ}の
\ruby{我}{われ}に
\ruby[g]{似{\換字{通}}}{に かよ}ひ
たる
ところの
あるが
\ruby{爲}{ため}にや、
%
\ruby{世}{よ}にいふ
\ruby[||j>]{合}{あひ}
\ruby[||j>]{性}{しやう}といふ
% \ruby{合性}{あひ|しやう}といふ
\ruby{事}{こと}の
\ruby{爲}{ため}にや、
%
たゞしは% 踊り字調整「〻(二の字点、揺すり点)に濁点に見えるが(ゞ)」
\ruby[g]{眞實}{まこと }
\ruby{{\換字{前}}}{まへ}の
\ruby{世}{よ}に
\ruby[g]{如何}{い か }なる
\原本頁{251-9}\改行%
\ruby[g]{因緣}{いん{\換字{𛀁}}ん}の
ありての
\ruby{事}{こと}か、
%
\ruby{他}{ひと}に
\ruby{超}{こ}えて
\ruby[g]{世話}{せ わ }に
なり
なられつ
したる
\ruby[g]{恩義}{おんぎ }の
\ruby[||j>]{關}{くわん}
\ruby[||j>]{係}{ けい}は
% \ruby{關係}{くわん|けい}は
\ruby[g]{島木}{しまき }に
\ruby{及}{およ}ばず、
%
\ruby{一}{ひと}ツ
\ruby{窓}{まど}の
\ruby{光}{ひかり}を
\ruby[g]{各自}{めい〳〵}の
\ruby{机}{つくゑ}に
\ruby{{\換字{分}}}{わか}つて、
%
\ruby[g]{奇{\換字{文}}}{き ぶん}を
\ruby{共}{とも}に
\ruby{賞}{しやう}し
\ruby[g]{疑義}{ぎ ゝ }を% 踊り字調整「〻(二の字点、揺すり点)に見えるが(ゝ)」
\ruby{相}{あひ}
\ruby{質}{たゞ}す% 踊り字調整「〻(二の字点、揺すり点)に濁点に見えるが(ゞ)」
\ruby[g]{學問}{がくもん}の
\ruby{{\換字{交}}}{まじは}りは
\ruby[g]{山瀬}{やませ }に
\ruby{如}{し}かざりしかども、
%
たゞ% 踊り字調整「〻(二の字点、揺すり点)に濁点に見えるが(ゞ)」
\ruby{何}{なん}と
\ruby{無}{な}く
\ruby{我}{われ}
\ruby{彼}{かれ}を
\ruby{他}{ほか}ならず
\ruby{懷}{なつか}しめば、
%
\ruby{彼}{かれ}も
また
\ruby{我}{われ}を
\ruby{他}{ほか}ならず
\ruby{愛}{あい}して、
%
\ruby[g]{{\換字{分}}桃}{ぶんたう}の% 「桃を分ける」=「分桃(フェンタオ)」が「男性同士の愛」
\ruby{痴}{し}れたる
\ruby{{\換字{情}}}{じやう}
こそは
\ruby{有}{あ}らざりけれ、
%
\ruby[g]{斷金}{だんきん}の% 「断金」金をも切断するほど強固に結ばれた友情。
まことの
\ruby{契}{ちぎり}は
\ruby{淺}{あさ}からざりし
なり。

\原本頁{252-5}%
されど
\ruby{人}{ひと}
おの〳〵
\ruby{望}{のぞ}む
\ruby{處}{ところ}を
\ruby{異}{こと}にすれば、
%
\ruby{彼}{かれ}は
\ruby[g]{一帆}{いつぱん}の
\ruby{風}{かぜ}に
\ruby[g]{萬里}{ばんり }の
\ruby{海}{うみ}を
\ruby{渡}{わた}つて
\ruby{波瀾淘湧}{は|らん|きよう|ゆう}の% 類似熟語「紫瀾洶湧」
\ruby{中}{うち}に
\ruby{身}{み}を
\ruby{托}{たく}するの
\ruby[g]{{\換字{船}}人}{ふなびと}となり、
%
\ruby{我}{われ}は
\ruby[g]{{\換字{半}}夜}{はんや }の
\ruby[<j>]{燈}{ともしび}に% ルビ調整(特殊処理)ルビがくっついてしまうので[||j>] でなく [<j>] とする
\ruby[||j>]{幾}{いく}
\ruby[||j>]{卷}{くわん}の
% \ruby{幾卷}{いく|くわん}の
\ruby{書}{しよ}と
\ruby{對}{たい}して
\ruby[||j>]{寂}{じやく}
\ruby[||j>]{寞}{ まく}たる
% \ruby{寂寞}{じやく|まく}たる
\ruby[g]{小齋}{せうさい}の
\ruby{裏}{うち}に
\ruby{思}{おもひ}を
\ruby{錬}{ね}るの
\ruby[g]{學究}{がくきう}たるを
\ruby{甘}{あま}んぜるより、
%
\ruby{相}{あひ}
\ruby{見}{み}ざる
\ruby[g]{月日}{つきひ }は
おのづと
\ruby{多}{おほ}くなり
\ruby{行}{ゆ}きしが
\改行% 校正作業の簡略化のため
、
%
\原本頁{252-9}\改行%
しかも
\ruby{相}{あひ}
\ruby{思}{おも}ふ
\ruby{心}{こゝろ}は% 踊り字調整「〻(二の字点、揺すり点)に見えるが(ゝ)」
\ruby{{\換字{更}}}{さら}に
\ruby{變}{かは}らず、
%
\ruby{彼}{かれ}
\ruby[||j>]{海}{かい}
\ruby[||j>]{上}{じやう}に
% \ruby{海上}{かい|じやう}に
ありと
\ruby{知}{し}る
\ruby{時}{とき}は、
%
\ruby{風}{かぜ}の
\ruby{曉}{あした}、
%
\ruby{{\換字{雪}}}{ゆき}の
\ruby{夕}{ゆふべ}、
%
あゝ% 踊り字調整「〻(二の字点、揺すり点)に見えるが(ゝ)」
\ruby[g]{羽{\換字{勝}}}{は がち}はと
\ruby[g]{此方}{こなた }に% ルビ調整(原本通り)
\ruby{思}{おも}はぬ
\ruby{折}{をり}も
\ruby{無}{な}ければ、
%
\ruby[g]{富士}{ふ じ }の
\原本頁{252-11}\改行%
\ruby[g]{高根}{たかね }も
\ruby{浪}{なみ}に
\ruby{{\換字{消}}}{き}{\換字{𛀁}}て
\ruby{夢}{ゆめ}ならでは
\ruby[g]{日本}{に ほん}の
\ruby{見}{み}えぬ
\ruby[g]{異鄕}{いきやう}の
\ruby{津}{つ}に
\ruby{在}{あ}りても
\改行% 校正作業の簡略化のため
、
%
\原本頁{253-1}\改行%
\ruby[g]{彼方}{かなた }も
\ruby{我}{われ}を
\ruby{{\換字{猶}}}{なほ}
\ruby{思}{おも}ひ
\ruby{吳}{く}れて、
%
\ruby[g]{他邦}{よ そ }の
\ruby{港}{みなと}を
\ruby{目}{め}の
\ruby{{\換字{前}}}{まへ}に
\ruby{見}{み}る
\ruby{繪葉書}{ゑ|は|がき}の、
%
\ruby[||j>]{此}{この}
\ruby[||j>]{岬}{みさき}の
% \ruby{此岬}{この|みさき}の
\ruby{下}{した}
\ruby{此}{こ}の
\ruby{水}{みづ}の
\ruby{上}{うへ}に
\ruby{汝}{なんぢ}の% 「汝(なんぢ)」の読みは原文のまま
\ruby{友}{とも}の
\ruby[g]{羽{\換字{勝}}}{は がち}
\ruby{在}{あ}りと、
%
\ruby[g]{村居}{そんきよ}の
\ruby{閑}{しづか}なる
\ruby{机}{つくゑ}の
\ruby{上}{うへ}に、
%
\ruby{天}{てん}の
\ruby[g]{一方}{いつぱう}より
\ruby[<j>]{溫}{あたゝか}き% 踊り字調整「〻(二の字点、揺すり点)に見えるが(ゝ)」
\ruby{{\換字{情}}}{こゝろ}を% 踊り字調整「〻(二の字点、揺すり点)に見えるが(ゝ)」
\ruby{寄}{よ}せ
\ruby{吳}{く}るゝこと% 踊り字調整「〻(二の字点、揺すり点)に見えるが(ゝ)」
\ruby[g]{數々}{しば〳〵}なりき。

\原本頁{253-4}%
\ruby{我}{われ}とは
かくの
\ruby{如}{ごと}き
\ruby{中}{なか}なる
\ruby[g]{羽{\換字{勝}}}{は がち}が
\ruby{久}{ひさ}しぶりにて
\ruby{歸}{かへ}りしを
\ruby{{\換字{迎}}}{むこ}ふるの
\ruby{會}{くわい}に、
%
\ruby[g]{一篇}{いつぺん}の
\ruby{歌}{うた}をも
\ruby{寄}{よ}すること
\ruby{無}{な}く、
%
\ruby[g]{數句}{すうく }の
\ruby{語}{ことば}をも
\ruby{{\換字{交}}}{まじ}ふること
\ruby{無}{な}くして、
%
\ruby[g]{全く}{まつた }
\ruby{面}{おもて}を
\ruby{出}{いだ}さゞりしは、% 踊り字調整「〻(二の字点、揺すり点)に濁点に見えるが(ゞ)」
%
\ruby[g]{水野}{みづの }の
\ruby{胸}{むね}
\ruby{濟}{す}まず
\ruby{思}{おも}へる
ところ
なりしが、
%
\ruby{其}{そ}の
\ruby{事}{こと}
\ruby{彼}{か}の
\ruby{事}{こと}の
\ruby[g]{煩累}{わづらひ}に
\ruby{心}{こゝろ}を% 踊り字調整「〻(二の字点、揺すり点)に見えるが(ゝ)」
\ruby{取}{と}られて、
%
\ruby{其}{その}
\ruby{後}{ゝち}も% 踊り字調整「〻(二の字点、揺すり点)に見えるが(ゝ)」
\ruby{思}{おも}ひながら
\ruby{{\換字{尋}}}{たづ}ね
さへ
せざりし
\ruby{其}{そ}の
\ruby[g]{羽{\換字{勝}}}{は がち}に、
%
\ruby[g]{忽然}{こつぜん}として
\ruby{{\換字{尋}}}{たづ}ね
\ruby{寄}{よ}られては、
%
あゝ% 踊り字調整「〻(二の字点、揺すり点)に見えるが(ゝ)」
\ruby{此}{この}
\ruby{人}{ひと}を
\ruby{{\換字{尋}}}{たづ}ねでは
\ruby{濟}{す}まざりしものを、
%
\ruby[g]{差當}{さしあた}りての
\ruby{苦}{くるし}き
おもひに
のみ
\ruby{惹}{ひ}かされて、
%
\ruby{我}{われ}に
\ruby{疎}{うと}き
\ruby[g]{意の}{こゝろ }% 踊り字調整「〻(二の字点、揺すり点)に見えるが(ゝ)」
\ruby{露}{つゆ}
ありてには
あらねど、
%
おのづから
\ruby{人}{ひと}の
\ruby{{\換字{情}}}{なさけ}を
\ruby{{\換字{空}}}{あだ}に
したる
やうに
なりし
\ruby{悲}{かな}しさ、
%
と
\ruby[||j>]{其}{その}
\ruby[||j>]{懷}{なつか}しき
% \ruby{其懷}{その|なつか}しき
\ruby{顏}{かほ}を
\ruby{一}{ひ}ト
\ruby[g]{目見}{め み }るより
\ruby{早}{はや}く、
%
\ruby{何}{なに}より
\ruby{先}{さき}に
\ruby{我}{わ}が
\ruby[g]{振舞}{ふるまひ}の
\ruby[g]{{\換字{勝}}手}{かつて }
\ruby{{\換字{過}}}{す}ぎたるが
\ruby{羞}{はづか}しく
なりて、
%
\ruby{正}{まさ}しくは
\ruby{對}{むか}ひ
\ruby{見}{み}る
\ruby{事}{こと}も
\ruby{叶}{かな}はぬ
やうの
\ruby[g]{心地}{こゝち }しつ、% 踊り字調整「〻(二の字点、揺すり点)に見えるが(ゝ)」
%
\ruby[g]{滔々}{たう〳〵}として
\ruby[g]{日方}{ひ かた}の
\ruby{我}{われ}を
\ruby{諫}{いさ}め
くれたる
\ruby{其}{そ}の
\ruby{幾千言}{いく|せん|げん}を
\ruby{聞}{き}ける
よりも、
%
\ruby{我}{われ}と
\ruby{我}{わ}が
\ruby{果敢無}{は|か|な}き
\ruby{戀}{こひ}に
\ruby{{\換字{迷}}}{まよ}ひて、
%
\ruby{此}{こ}の
\ruby{{\換字{情}}}{じやう}の
\ruby{篤}{あつ}く
\ruby{義}{ぎ}の
\ruby{{\換字{強}}}{つよ}き
\ruby[<j||]{{\換字{尊}}}{たつと}むべき% 行末行頭の境界付近なので特例処置を施す
\ruby{友}{とも}に
\ruby{負}{そむ}きたる
\ruby{罪}{つみ}の
\ruby{輕}{かろ}からぬを
おぼえ、
%
よし
\ruby{無}{な}き
\ruby{想}{おもひ}に
のみ
\原本頁{254-6}\改行%
\ruby{沈}{しづ}める
\ruby[g]{昨日}{きのふ }
\ruby[g]{今日}{け ふ }の
\ruby{我}{わ}が
\ruby{愚}{おろか}しき
をば
\ruby{自}{みづか}ら
\ruby{慚}{は}ぢ
\ruby{自}{みづか}ら
\ruby{責}{せ}むるの
\ruby{{\換字{情}}}{じやう}は
\ruby{燬}{や}くが
\ruby{如}{ごと}くに
\ruby{起}{おこ}りて、
%
\ruby[g]{嗚呼}{あ ゝ }% 踊り字調整「〻(二の字点、揺すり点)に見えるが(ゝ)」
\ruby{我}{われ}
\ruby[g]{心裏}{こゝろ }に% 踊り字調整「〻(二の字点、揺すり点)に見えるが(ゝ)」
\ruby{物}{もの}
\ruby{無}{な}くして
\ruby{懷}{なつか}しき
\ruby{此}{こ}の
\ruby{友}{とも}と
\ruby{今}{いま}
こゝに% 踊り字調整「〻(二の字点、揺すり点)に見えるが(ゝ)」
\ruby[g]{相語}{あひかた}らば、
%
\ruby[g]{如何}{い か }ばかり
\ruby[g]{今日}{け ふ }の
\ruby[g]{團欒}{まどゐ }の
\ruby{嬉}{うれ}しく
\ruby{樂}{たの}しからんを
\改行% 校正作業の簡略化のため
、
%
\原本頁{254-9}\改行%
\ruby[g]{彼方}{かなた }は
\ruby{相}{あひ}も
\ruby{變}{かは}らず
\ruby{胸}{むね}を
\ruby{開}{ひら}きて
\ruby[g]{物語}{ものがた}れど、
%
\ruby{我}{われ}は
\ruby{人}{ひと}には
\ruby{告}{つ}け
\ruby{{\換字{難}}}{がた}き
\ruby[g]{私{\換字{情}}}{わたくし}を
\ruby{胸}{むね}に
\ruby{抱}{いだ}き% ルビ調整(原本通り)(いだ)
\ruby{居}{を}りて、
%
\ruby[g]{往時}{むかし }の
\ruby{無邪氣}{む|じや|き}の
\ruby{我}{われ}ならねば、
%
\ruby{隔}{へだ}つる
\ruby{氣}{き}の
\ruby{{\換字{更}}}{さら}に
あるには
あらねど、
%
\ruby{水}{みづ}と
\ruby{油}{あぶら}との
\ruby{一}{ひと}つに
なりがたき
やうに、
%
\原本頁{255-1}\改行%
\ruby[g]{何處}{ど こ }と
\ruby{無}{な}く
\ruby{奧}{おく}
\ruby{底}{そこ}
なくは
\ruby{打}{うち}
\ruby{解}{と}け
\ruby{{\換字{難}}}{がた}き
\ruby[g]{心地}{こゝち }して、% 踊り字調整「〻(二の字点、揺すり点)に見えるが(ゝ)」
%
\ruby[g]{言葉}{ことば }に
\ruby{餘}{あま}る
\ruby{思}{おもひ}は
ありながらも、
%
\ruby[g]{{\換字{所}}以}{ゆ ゑ }
\ruby{知}{し}らず
\ruby[g]{自然}{おのつ }と
\ruby{我}{わ}が
\ruby{口}{くち}の
\ruby{結}{むす}ばるゝを% 踊り字調整「〻(二の字点、揺すり点)に見えるが(ゝ)」
\ruby{何}{なん}と
せんと
\改行% 校正作業の簡略化のため
、
%
\原本頁{255-3}\改行%
\ruby[g]{水野}{みづの }は
\ruby{私}{ひそか}に
\ruby{自}{みづか}ら
\ruby{苦}{くるし}めり。

\原本頁{255-4}%
\ruby{見}{み}れば
\ruby[g]{日方}{ひ かた}の
\ruby{言}{い}ひしに
\ruby{露}{つゆ}
\ruby{差}{たが}はず、
%
\ruby[g]{生來}{せいらい}の
\ruby[g]{沈毅}{ちんき }の
\ruby[g]{氣性}{きしやう}は
\ruby[g]{{\換字{浮}}世}{うきよ }に
\ruby{鍛}{きた}はれて、
%
いよ〳〵
\ruby{萎}{ひる}まず
\ruby{怯}{おく}れぬ
\ruby{大{\換字{丈}}夫}{だい|ぢやう|ふ}
となりたるは、
\ruby{其}{そ}の
\ruby{額}{ひたひ}には
\ruby{曇}{くもり}の
\ruby{{\換字{絕}}}{た}えて
\ruby{無}{な}くて、
%
\ruby{眼}{め}には
\ruby{{\換字{銳}}}{するど}さの
\ruby{加}{くは}はりたるにも
\ruby{知}{し}られ、
\ruby[g]{眞窣}{しんそつ}なれども
\ruby[g]{擧動}{きよどう}に
\ruby{威}{ゐ}
あり
おちつき
あり、
%
\ruby[g]{{\換字{平}}易}{へいゝ }% 踊り字調整「〻(二の字点、揺すり点)に見えるが(ゝ)」
なれども
\ruby[g]{言葉}{ことば }に
\ruby[g]{思慮}{し りよ}あり
\ruby[||j>]{斟}{しん}
\ruby[||j>]{{\換字{酌}}}{しやく}あるに、
% \ruby{斟{\換字{酌}}}{しん|しやく}あるに、
%
あだには
\ruby[g]{月日}{つきひ }を
\ruby{經}{へ}ざりしを
\ruby{示}{しめ}したり。

\原本頁{255-9}%
\ruby[g]{水野}{みづの }に
\ruby[g]{水野}{みづの }の
\ruby[g]{{\換字{所}}思}{おもひ }あれば、
%
\ruby[g]{羽{\換字{勝}}}{は がち}にも
\ruby[g]{羽{\換字{勝}}}{は がち}の
\ruby[g]{{\換字{所}}思}{おもひ }ありて、
%
\ruby[g]{累々}{るゐ〳〵}として
\ruby[g]{喪家}{さうか }の
\ruby{狗}{いぬ}の
ごとく
\ruby{衰}{おとろ}へ
\ruby{果}{は}てたる
\ruby{我}{わ}が
\ruby{友}{とも}の
\ruby[g]{容態}{ようす }をば、
%
しばし
\ruby[g]{無言}{む ごん}にして
\ruby[g]{羽{\換字{勝}}}{は がち}は
\ruby{眺}{なが}めしが、
%
たゞ% 踊り字調整「〻(二の字点、揺すり点)に濁点に見えるが(ゞ)」
\ruby[g]{日方}{ひ かた}
のみは
\ruby{思}{おも}つては
\ruby{言}{い}はずに
\原本頁{256-1}\改行%
\ruby{居}{ゐ}ず、
%
\ruby[g]{一旦}{いつたん}は
\ruby[g]{羽{\換字{勝}}}{は がち}を
\ruby{憚}{はゞか}りて% 「憚 は(ゞ)か」% 踊り字調整「〻(二の字点、揺すり点)に濁点に見えるが(ゞ)」
\ruby{默}{もく}せしが、
%
\ruby{堪}{こら}へ
\ruby{{\換字{兼}}}{か}ねてか
\ruby{忽}{たちま}ち
また、

\原本頁{256-2}%
『
\ruby[g]{水野}{みづの }、
』

\原本頁{256-3}%
と
\ruby{一}{ひ}ト
\ruby{聲}{こゑ}
\ruby{呼}{よ}び
かけたり。
