\Entry{其四十四}
% メモ 校正終了 2024-03-16

\原本頁{256-5}
お
\ruby{濱}{はま}は
\ruby{何處}{いづ|く}にか
\ruby{去}{さ}つて
\ruby{復}{また}
\ruby{現}{あら}れず、
%
むくつけき
\ruby{田舎女}{ゐな|か|もの}の
お
\ruby{鍋}{なべ}は
\ruby{茶}{ちや}を
もて
\ruby{來}{きた}りしが、
%
\ruby{先}{まづ}づ
\ruby{無作法}{ぶ|さ|はふ}に
\ruby{人々}{ひと|〴〵}の
\ruby{顏}{かほ}を
\ruby{見渡}{み|わた}して、
%
\ruby{初}{はじめ}に
\ruby[g]{羽{\換字{勝}}}{はがち}が
\ruby{{\換字{前}}}{まへ}に
\ruby{一盞}{いつ|さん}を
\ruby{薦}{すゝ}め、
%
\ruby{次}{つぎ}に
\ruby[g]{水野}{みづの}が
\ruby{{\換字{前}}}{まへ}に
また
\ruby{一盞}{いつ|さん}を
\ruby{置}{お}き、
%
\ruby{茶}{ちや}は
\ruby{注}{つ}ぎて
\ruby{其}{そ}の
\ruby{盞}{さん}を
\ruby{滿}{み}たしながら
\ruby[g]{日方}{ひかた}が
\ruby{{\換字{前}}}{まへ}には
\ruby{取}{と}りても
\ruby{與}{や}らず、

\原本頁{256-9}
『
\ruby{汝樣}{おめへ|さま}は
\ruby{{\換字{勝}}手}{かつ|て}に
\ruby{取}{と}つて
\ruby{飮}{の}まつせえ。
』

\原本頁{256-10}
と
\ruby{云}{い}はぬばかりの
\ruby{顏}{かほ}つきしつ、
%
\ruby[g]{其邊}{あたり}の
\ruby{亂}{みだ}れたるを
\ruby{取片付}{とり|かた|づ}けて、
%
\ruby{默}{だま}つて
\ruby{{\換字{退}}}{しりぞ}き
\ruby{去}{さ}れば、
%
\ruby[g]{水野}{みづの}は
\ruby{氣}{き}の
\ruby{毒}{どく}さに
\ruby{堪}{た}へずして、
%
\ruby{自}{みづか}ら
\ruby{茶盞}{ちや|さん}を
\ruby{取}{と}つて
\ruby[g]{日方}{ひかた}に
\ruby{與}{あた}へたり。

\原本頁{257-3}
\ruby[g]{日方}{ひかた}は
\ruby{此等}{これ|ら}の
\ruby{瑣事}{さ|じ}には
\ruby{頓着}{とん|ちやく}もせず、
%
\ruby{感慨}{かん|がい}に
\ruby{堪}{た}へぬ
\ruby{面}{おもて}の
\ruby{色}{いろ}、
%
\ruby{{\換字{睜}}開}{み|は}れる
\ruby{眼}{め}には
\ruby{露}{つゆ}をさへ
\ruby{宿}{やど}して、

\原本頁{257-5}
『
\ruby[g]{水野}{みづの}!。
%
もう
\ruby{乃公}{お|れ}は
\ruby{一}{ひ}ト
\ruby{{\換字{通}}}{とほ}り
\ruby{云}{い}ひ
\ruby{盡}{つく}したから
\ruby{繰}{く}り
\ruby{{\換字{返}}}{かへ}して
また
\ruby{言}{い}ふのでは
\ruby{無}{な}いが、
%
\ruby{如何}{い|か}に
\ruby{心}{こゝろ}が
\ruby{{\換字{弱}}}{よわ}つたればとて、
%
\ruby{何}{なん}といふ
\ruby{汝}{きさま}の
\ruby{衰}{おとろ}へ
\ruby{方}{かた}だ!。
%
\ruby{{\換字{迷}}}{まよ}ふなら
\ruby{{\換字{迷}}}{まよ}ふで
\ruby{仕方}{し|かた}は
\ruby{無}{な}いやうなものゝ、
%
\ruby{同}{おな}じ
\ruby{{\換字{迷}}}{まよ}ひにも
それ〴〵があらう。
%
\ruby{何故}{な|ぜ}
\ruby{{\換字{迷}}}{まよ}ふにしても
\ruby{男兒}{をと|こ}らしくは
\ruby{{\換字{迷}}}{まよ}はぬ?。
%
\ruby{汝}{きさま}の
\ruby{衰}{おとろ}へに
\ruby{衰}{おとろ}へ
\ruby{果}{はて}てゝ
\ruby{女}{をんな}の
\ruby{腐}{くさ}つた
のゝやうに
\ruby{成}{な}り
\ruby{果}{はて}てたのが、
%
\ruby{何}{なに}より
\ruby{彼}{か}より
\ruby{{\換字{情}}無}{なさけ|な}いは。
%
\ruby{汝}{きさま}は
\ruby{本}{もと}より
\ruby{剛{\換字{強}}}{がう|きやう}な
\ruby{鐵石}{てつ|せき}の
\ruby{男}{をとこ}といふのでは
\ruby{無}{な}かつたが、
%
\ruby{外面}{うは|べ}は
\ruby{柔}{やはら}かでも
\ruby{事}{こと}によつては、
%
\ruby{人}{ひと}と
\ruby{爭}{あらそ}つて
\原本頁{258}
\ruby{後}{あと}へは
\ruby{决}{けつ}して
\ruby{{\換字{退}}}{ひ}かぬ、
%
\ruby{怖}{おそろ}しい
\ruby{氣合}{き|あひ}を
\ruby{含}{ふく}んだ
\ruby{奴}{やつ}で、
%
\ruby{釅}{きぶ}い
\ruby{醋}{す}のやうなところがあると、
%
\ruby{{\換字{平}}生}{つね|〴〵}
\ruby{乃公}{お|れ}が
\ruby{{\換字{評}}}{ひやう}したほどの
\ruby{男兒}{をと|こ}であつたが
\ruby{今}{いま}は
\ruby{何樣}{ど|う}だ。
%
\ruby{醋}{す}なら
\ruby{醋}{す}は
\ruby{腐}{くさ}つて
\ruby{仕舞}{し|ま}つたのか
\ruby{黴}{か}びて
\ruby{仕舞}{し|ま}つたのか、
%
\ruby{乃公}{お|れ}に
\ruby{打}{う}たれて
\ruby{抵抗}{てむ|かひ}も
せぬやうになつたとは
\ruby{嗚呼}{あ|ゝ}
\ruby{{\換字{情}}無}{なさけ|な}い!。
%
これ
\ruby{眼}{め}を
\ruby{開}{あ}いて
\ruby{天地}{てん|ち}を
\ruby{見}{み}ろ!。
%
\ruby{畫工}{ゑ|かき}には
\ruby{畫}{ゑ}を
\ruby{敎}{をし}へぬ
\ruby{草木}{くさ|き}も
\ruby{無}{な}い、
%
\ruby{男兒}{をと|こ}を
\ruby{磨}{みが}かうといふものには
\ruby{我}{わ}が
\ruby{精神}{こゝ|ろ}を
\ruby{奮}{ふる}はせて
\ruby{歩}{あゆみ}を
\ruby{{\換字{進}}}{すゝ}ます
\ruby{鞭}{むち}や
\ruby{刺馬輪}{し|ば|りん}で
\ruby{無}{な}いものは
\ruby{無}{な}い!。
%
\ruby{見}{み}なかつたか
\ruby{盲目漢}{め|く|ら}!、
%
\ruby{氣}{き}が
\ruby{注}{つ}かんか
\ruby{放心漢}{う|つ|け}!、
%
\ruby{此家}{こ|ゝ}の
\ruby{小娘}{こ|むすめ}が
\ruby{何}{なに}を
\ruby{仕}{し}たぞ。
%
\ruby{齡}{とし}はたつた
\ruby{十五}{じう|ご}か
\ruby{十六}{じう|ろく}かで、
%
\ruby{乃公}{お|れ}の
\ruby{一}{ひ}ト
\ruby{攫}{つかみ}にも
\ruby{足}{た}らぬ
\ruby{優}{やさ}しい
\ruby{身體}{から|だ}、
%
それでも
\ruby{流石}{さす|が}に
\ruby{日本}{に|ほん}の
\ruby{女}{をんな}だ、
%
\ruby{{\換字{平}}生}{へい|せい}
\ruby{一}{ひと}ツ
\ruby{家}{いへ}に
\ruby{居}{ゐ}る
\ruby{汝}{きさま}が
\ruby{乃公}{お|れ}に
\ruby{撲}{う}たれ
\ruby{辱}{はづかし}められるのを
\ruby{見}{み}ては
\ruby{慨然}{がい|ぜん}として、
%
\ruby{身}{み}を
\ruby{挺}{ぬき}んでゝ
\ruby{汝}{きさま}を
\ruby{護}{かば}つて
\ruby{乃公}{お|れ}に
\ruby{當}{あた}り
\原本頁{259}
あの
\ruby{愛}{あい}らしい
\ruby{美}{うつく}しい
\ruby{眼}{め}から、
%
\ruby{寶石}{ほう|せき}のやうな
\ruby{光}{ひかり}を
\ruby{輝}{かゞや}かして、
%
\ruby{眞紅}{まつ|か}な
\ruby{顏}{かほ}に
\ruby{血}{ち}を
\ruby{沸}{にや}して
\ruby{打}{う}つて
かゝつたでは
\ruby{無}{な}いか!。
%
\ruby{女性}{をん|な}だ、
%
\ruby{小兒}{こ|ども}だ、
%
\ruby{孱{\換字{弱}}}{か|よわ}い
\ruby{娘}{むすめ}だ。
%
それでさへ
\ruby{一旦}{いつ|たん}
\ruby{激動}{げき|どう}すれば、
%
\ruby{此}{こ}の
\ruby[g]{日方}{ひかた}にも
\ruby{取}{と}つて
かゝる、
%
それが
\ruby{貴}{たつと}い
\ruby{人間}{ひ|と}の
\ruby{勇氣}{ゆう|き}だ、
%
\ruby{人}{ひと}の
\ruby{人}{ひと}たる
\ruby{{\換字{所}}以}{ゆ|ゑん}を
\ruby{支}{さゝ}へるものだ。
%
それだのに
\ruby{何}{なん}だ
\ruby{汝}{きさま}の
\ruby{其}{そ}の
\ruby{態}{てい}は!。
%
\ruby{一少女}{いち|せう|ぢよ}にも
\ruby{及}{およ}ばなくなつて、
%
たゞ
\ruby{崩折}{くづ|を}れて
\ruby{萎}{しを}れきつて
\ruby{居}{ゐ}る!。
%
よく
\ruby{彼}{あ}の
\ruby{娘}{むすめ}に
\ruby{對}{たい}しても
\ruby{慚死}{ざん|し}せぬナ。
%
\ruby[g]{水野}{みづの}!、
%
\ruby{汝}{きさま}は
\ruby{决}{けつ}して
\ruby{决}{けつ}して
\ruby{本心}{ほん|しん}を
\ruby{失}{うしな}ふやうな、
%
\ruby{其樣}{そ|ん}な
\ruby{腑甲{\換字{斐}}無}{ふ|が|い|な}い
\ruby{奴}{やつ}では
\ruby{無}{な}いが、
%
\ruby{何樣}{ど|う}すれば
\ruby{此樣}{こ|ん}なに
\ruby{意氣地}{い|く|ぢ}が
\ruby{無}{な}くなつた。
%
こゝの
\ruby{娘}{むすめ}の
\ruby{擧動}{ふる|まひ}を
\ruby{眼}{め}の
\ruby{{\換字{前}}}{まへ}に
\ruby{見}{み}て、
%
よく
\ruby{汝}{きさま}は
\ruby{自{\換字{分}}}{じ|ぶん}が
\ruby{羞}{はづか}しくないナ。
%
\ruby{一少女}{いち|せう|ぢよ}でさへ
\ruby{彼}{あ}の
\ruby{{\換字{通}}}{とほ}りだ、
%
\ruby{汝}{きさま}は
\ruby{堂々}{だう|〴〵}たる
\ruby{男兒}{だん|じ}で
\ruby{無}{な}いか、
%
\ruby{乃公}{お|れ}は
\ruby{彼}{あ}の
\ruby{娘}{むすめ}に
\ruby{頭}{あたま}を
\ruby{撲}{う}たれたが、
%
\ruby{汝}{きさま}は
\ruby{精神}{こゝ|ろ}に
\ruby{鞭}{むち}を
\ruby{受}{う}けなかつたか。
%
\原本頁{260}
\ruby{苟}{いやし}くも
\ruby{舊}{もと}の
\ruby[g]{水野}{みづの}であるならば、
%
\ruby{人一倍}{ひと|いち|ばい}
\ruby{物}{もの}を
\ruby{思}{おも}ふ
\ruby{汝}{きさま}の
\ruby{事}{こと}だもの、
%
\ruby{必}{かなら}ず
\ruby{感奮}{かん|ぷん}せずには
\ruby{居}{を}らぬ
\ruby{筈}{はず}だが、
%
\ruby{衰}{おとろ}へ
\ruby{果}{は}て
\ruby{{\換字{弱}}}{よわ}り
\ruby{果}{は}てた
\ruby{今}{いま}の
\ruby{汝}{きさま}は、
%
\ruby{矢張}{やつ|ぱ}り
\ruby{首}{くび}を
\ruby{俛}{た}るゝばかりか。
%
\ruby{此家}{こ|ゝ}の
\ruby{娘}{むすめ}の
\ruby{健氣}{けな|げ}な
\ruby{振舞}{ふる|まひ}と、
%
\ruby{汝}{きさま}の
\ruby{其}{そ}の
\ruby{萎}{しを}れきつた
\ruby{狀態}{あり|さま}とを、
%
\ruby{見比}{み|くら}べ
\ruby{思}{おも}ひ
\ruby{比}{くら}べると
\ruby{此}{こ}の
\ruby[g]{日方}{ひかた}は、
%
これほどまでに
\ruby{汝}{きさま}は
\ruby{衰}{おとろ}へたかと、
%
\ruby{汝}{きさま}の
\ruby{衰}{おとろ}へ
\ruby{果}{は}てたのが
\ruby{悲}{かな}しくて
\ruby{涙}{なみだ}が
\ruby{出}{で}る!。
%
\ruby{女}{をんな}にも
\ruby{劣}{おと}る
やうになつたとは
\ruby{餘}{あま}り
\ruby{{\換字{情}}無}{なさけ|な}い!。
%
\ruby{何故}{な|ぜ}
\ruby{{\換字{迷}}}{まよ}ふにしても
\ruby{男兒}{をと|こ}らしく
\ruby{{\換字{迷}}}{まよ}つて
\ruby{吳}{く}れぬ?。
』
