\Entry{其四十四}

% メモ 校正終了 2024-05-08 2024-06-05
\原本頁{256-5}%
お
\ruby{濱}{はま}は
\ruby[g]{何處}{いづく }にか
\ruby{去}{さ}つて
\ruby{復}{また}
\ruby{現}{あら}れず、
%
むくつけき
\ruby{田舎女}{ゐな|か|もの}の
お
\ruby{鍋}{なべ}は
\ruby{茶}{ちや}を
もて
\ruby{來}{きた}りしが、
%
\ruby{先}{ま}づ
\ruby{無作法}{ぶ|さ|はふ}に
\ruby[g]{人々}{ひと〴〵}の
\ruby{顏}{かほ}を
\ruby[g]{見渡}{み わた}して、
%
\ruby{初}{はじめ}に
\ruby[g]{羽{\換字{勝}}}{は がち}が
\ruby{{\換字{前}}}{まへ}に
\ruby[g]{一盞}{いつさん}を
\ruby{薦}{すゝ}め、% 踊り字調整「〻(二の字点、揺すり点)に見えるが(ゝ)」
%
\ruby{次}{つぎ}に
\ruby[g]{水野}{みづの }が
\ruby{{\換字{前}}}{まへ}に
また
\ruby[g]{一盞}{いつさん}を
\ruby{置}{お}き、
%
\ruby{茶}{ちや}は
\ruby{注}{つ}ぎて
\ruby{其}{そ}の
\ruby{盞}{さん}を
\ruby{滿}{み}たしながら
\ruby[g]{日方}{ひ かた}が
\ruby{{\換字{前}}}{まへ}には
\ruby{取}{と}りても
\ruby{與}{や}らず、

\原本頁{256-9}%
『
\ruby[||j>]{汝}{おめへ}
\ruby[||j>]{樣}{ さま}は
% \ruby{汝樣}{おめへ|さま}は
\ruby[g]{{\換字{勝}}手}{かつて }に
\ruby{取}{と}つて
\ruby{飮}{の}まつせえ。
』

\原本頁{256-10}%
と
\ruby{云}{い}はぬばかりの
\ruby{顏}{かほ}つきしつ、
%
\ruby[g]{其邊}{あたり }の
\ruby{亂}{みだ}れたるを
\ruby{取片付}{とり|かた|づ}けて、
%
\ruby{默}{だま}つて
\ruby{{\換字{退}}}{しりぞ}き
\ruby{去}{さ}れば、
%
\ruby[g]{水野}{みづの }は
\ruby{氣}{き}の
\ruby{毒}{どく}さに
\ruby{堪}{た}へずして、
%
\ruby{自}{みづか}ら
\ruby[g]{茶盞}{ちやさん}を
\ruby{取}{と}つて
\ruby[g]{日方}{ひ かた}に
\ruby{與}{あた}へたり。

\原本頁{257-3}%
\ruby[g]{日方}{ひ かた}は
\ruby[g]{此等}{これら }の
\ruby[g]{瑣事}{さ じ }には
\ruby[||j>]{頓}{とん}
\ruby[||j>]{着}{ちやく}もせず、
% \ruby{頓着}{とん|ちやく}もせず、
%
\ruby[g]{{\換字{感}}慨}{かんがい}に
\ruby{堪}{た}へぬ
\ruby{面}{おもて}の
\ruby{色}{いろ}、
%
\ruby[g]{{\換字{睜}}開}{み は }れる
\ruby{眼}{め}には
\ruby{露}{つゆ}をさへ
\ruby{宿}{やど}して、

\原本頁{257-5}%
『
\ruby[g]{水野}{みづの }!。
%
もう
\ruby[g]{乃公}{お れ }は
\ruby{一}{ひ}ト
\ruby{{\換字{通}}}{とほ}り
\ruby{云}{い}ひ
\ruby{盡}{つく}したから
\ruby{繰}{く}り
\ruby{{\換字{返}}}{かへ}して
また
\原本頁{257-6}\改行%
\ruby{言}{い}ふのでは
\ruby{無}{な}いが、
%
\ruby[g]{如何}{い か }に
\ruby{心}{こゝろ}が% 踊り字調整「〻(二の字点、揺すり点)に見えるが(ゝ)」
\ruby{{\換字{弱}}}{よわ}つた
ればとて、
%
\ruby{何}{なん}といふ
\ruby{汝}{きさま}の
\原本頁{257-7}\改行%
\ruby{衰}{おとろ}へ
\ruby{方}{かた}だ!。
%
\ruby{{\換字{迷}}}{まよ}ふなら
\ruby{{\換字{迷}}}{まよ}ふで
\ruby[g]{仕方}{し かた}は
\ruby{無}{な}いやうなものゝ、% 踊り字調整「〻(二の字点、揺すり点)に見えるが(ゝ)」
%
\ruby{同}{おな}じ
\ruby{{\換字{迷}}}{まよ}ひにも
それ〴〵があらう。
%
\ruby[g]{何故}{な ぜ }
\ruby{{\換字{迷}}}{まよ}ふにしても
\ruby[g]{男兒}{をとこ }らしくは
\ruby{{\換字{迷}}}{まよ}はぬ?。
%
\ruby{汝}{きさま}の
\ruby{衰}{おとろ}へに
\ruby{衰}{おとろ}へ
\ruby{果}{は}てゝ% 踊り字調整「〻(二の字点、揺すり点)に見えるが(ゝ)」
\ruby{女}{をんな}の
\ruby{腐}{くさ}つた
のゝやうに% 踊り字調整「〻(二の字点、揺すり点)に見えるが(ゝ)」
\ruby{成}{な}り
\ruby{果}{は}てたのが、
%
\ruby{何}{なに}より
\ruby{彼}{か}より
\ruby[g]{{\換字{情}}無}{なさけな}いは。
%
\ruby[g]{汝は}{きさま }
\ruby{本}{もと}より
\ruby[||j>]{剛}{がう}
\ruby[||j>]{{\換字{強}}}{きやう}な
% \ruby{剛{\換字{強}}}{がう|きやう}な
\ruby[g]{鐵石}{てつせき}の
\ruby{男}{をとこ}といふのでは
\ruby{無}{な}かつたが、
%
\ruby[g]{外面}{うはべ }は
\ruby{柔}{やはら}かでも
\ruby{事}{こと}によつては、
%
\ruby{人}{ひと}と
\ruby{爭}{あらそ}つ
\原本頁{258-1}\改行%
て% 「て」以降「やう」で29文字ある。二箇所の「、」部分を詰めたと思われる
\ruby{後}{あと}へは
\ruby{决}{けつ}して
\ruby{{\換字{退}}}{ひ}かぬ、
%
\ruby{怖}{おそろ}しい
\ruby[g]{氣合}{き あひ}を
\ruby{含}{ふく}んだ
\ruby{奴}{やつ}で、
%
\ruby{釅}{きぶ}い
\ruby{醋}{す}の
やう% 「て」以降「やう」で29文字ある。二箇所の「、」部分を詰めたと思われる
\原本頁{258-2}\改行%
な
ところが
あると、
%
\ruby[g]{{\換字{平}}生}{つね〴〵}% ルビ調整(原本通り)
\ruby[g]{乃公}{お れ }が
\ruby{{\換字{評}}}{ひやう}した
ほどの
\ruby[g]{男兒}{をとこ }であつたが
\ruby{今}{いま}は
\ruby[g]{何樣}{ど う }だ。
%
\ruby{醋}{す}なら
\ruby{醋}{す}は
\ruby{腐}{くさ}つて
\ruby[g]{仕舞}{し ま }つたのか
\ruby{黴}{か}びて
\ruby[g]{仕舞}{し ま }つたのか
\改行% 校正作業の簡略化のため
、
%
\原本頁{258-4}\改行%
\ruby[g]{乃公}{お れ }に
\ruby{打}{う}たれて
\ruby[g]{抵抗}{てむかひ}も
せぬやうになつたとは
\ruby[g]{嗚呼}{あ ゝ }% 踊り字調整「〻(二の字点、揺すり点)に見えるが(ゝ)」
\ruby[g]{{\換字{情}}無}{なさけな}い!。
%
これ
\ruby{眼}{め}を
\ruby{開}{あ}いて
\ruby[g]{天地}{てんち }を
\ruby{見}{み}ろ!。
%
\ruby[g]{畫工}{ゑ かき}には
\ruby{畫}{ゑ}を
\ruby{敎}{をし}へぬ
\ruby[g]{草木}{くさき }も
\ruby{無}{な}い、
%
\原本頁{258-6}\改行%
\ruby[g]{男兒}{をとこ }を
\ruby{磨}{みが}かう
といふ
ものには
\ruby{我}{わ}が
\ruby[g]{精神}{こゝろ }を% 踊り字調整「〻(二の字点、揺すり点)に見えるが(ゝ)」
\ruby{奮}{ふる}はせて
\ruby{歩}{あゆみ}を
\ruby{{\換字{進}}}{すゝ}ます% 踊り字調整「〻(二の字点、揺すり点)に見えるが(ゝ)」
\ruby{鞭}{むち}や
\ruby{刺馬輪}{し|ば|りん}
% 刺馬輪 は 拍車 のようである
% 江戸時代に入ってきた当時は拍車(spur)を表す和名はなく、仮名で「スポール」と記していた
% 本格的に拍車が利用されるようになるのは明治時代以降
% 明治初期になると和訳した名称「刺馬輪」が登場
% 現在の拍車は明治中期に登場した単語
で
\ruby{無}{な}いものは
\ruby{無}{な}い!。
%
\ruby{見}{み}なかつたか
\ruby{盲目{\換字{漢}}}{め|く|ら}!、
%
\ruby{氣}{き}が
\ruby{注}{つ}かんか
\ruby{放心{\換字{漢}}}{う|つ|け}!、
%
\ruby[g]{此家}{こ ゝ }の% 踊り字調整「〻(二の字点、揺すり点)に見えるが(ゝ)」
\ruby[g]{小娘}{こむすめ}が
\ruby{何}{なに}を
\ruby{仕}{し}たぞ。
%
\ruby{齡}{とし}はたつた
\ruby[g]{十五}{じふご }か
\原本頁{258-9}\改行%
\ruby[g]{十六}{じふろく}かで、
%
\ruby[g]{乃公}{お れ }の
\ruby{一}{ひ}ト
\ruby[g]{攫に}{つかみ }
も
\ruby{足}{た}らぬ
\ruby{優}{やさ}しい
\ruby[g]{身體}{からだ }、
%
それでも
\ruby[g]{流石}{さすが }に
\ruby[g]{日本}{に ほん}の
\ruby{女}{をんな}だ、
%
\ruby[g]{{\換字{平}}生}{へいぜい}% ルビ調整(原本通り)
\ruby{一}{ひと}ツ
\ruby{家}{いへ}に
\ruby{居}{ゐ}る
\ruby[g]{汝が}{きさま }、
\ruby[g]{乃公}{お れ }に
\ruby{撲}{う}たれ
\ruby[<j>]{辱}{はづかし}め
られるのを
\ruby{見}{み}ては
\ruby[g]{慨然}{がいぜん}
として、
%
\ruby{身}{み}を
\ruby{挺}{ぬき}んでゝ% 踊り字調整「〻(二の字点、揺すり点)に見えるが(ゝ)」
\ruby{汝}{きさま}を
\ruby{護}{かば}つて
\ruby[g]{乃公}{お れ }に
\ruby{當}{あた}り
\原本頁{259-1}\改行%
あの
\ruby{愛}{あい}らしい
\ruby{美}{うつく}しい
\ruby{眼}{め}から、
%
\ruby[g]{寶石}{ほうせき}の
やうな
\ruby{光}{ひかり}を
\ruby{輝}{かゞや}かして、% 踊り字調整「〻(二の字点、揺すり点)に濁点に見えるが(ゞ)」
%
\ruby[g]{眞紅}{まつか }な
\ruby{顏}{かほ}に
\ruby{血}{ち}を
\ruby{沸}{にや}して
\ruby{打}{う}つて
かゝつたでは% 踊り字調整「〻(二の字点、揺すり点)に見えるが(ゝ)」
\ruby{無}{な}いか!。
%
\ruby[g]{女性}{をんな }だ、
%
\ruby[g]{小兒}{こ ども}だ、
%
\ruby[g]{孱{\換字{弱}}}{か よわ}い
\ruby{娘}{むすめ}だ。
%
それでさへ
\ruby[g]{一旦}{いつたん}
\ruby[g]{激動}{げきどう}すれば、
%
\ruby{此}{こ}の
\ruby[g]{日方}{ひ かた}にも
\ruby{取}{と}つて
かゝる、% 踊り字調整「〻(二の字点、揺すり点)に見えるが(ゝ)」
%
それが
\ruby{貴}{たつと}い
\ruby[g]{人間}{ひ と }の
\ruby[g]{勇氣}{ゆうき }だ、
%
\ruby{人}{ひと}の
\ruby{人}{ひと}たる
\ruby[g]{{\換字{所}}以}{ゆ ゑん}を
\ruby{支}{さゝ}へるものだ。% 踊り字調整「〻(二の字点、揺すり点)に見えるが(ゝ)」
%
それだのに
\ruby{何}{なん}だ
\ruby{汝}{きさま}の
\ruby{其}{そ}の
\ruby{態}{てい}は!。
%
\ruby{一}{いち}
\ruby[g]{少女}{せうぢよ}にも
\ruby{及}{およ}ばなく
なつて、
%
たゞ% 踊り字調整「〻(二の字点、揺すり点)に濁点に見えるが(ゞ)」
\ruby{崩}{くづ}
\ruby{折}{を}れて
\ruby{萎}{しを}れ
きつて
\ruby{居}{ゐ}る!。
%
よく
\ruby{彼}{あ}の
\ruby{娘}{むすめ}に
\ruby{對}{たい}しても
\ruby[g]{慚死}{ざんし }せぬナ。
%
\ruby[g]{水野}{みづの }!、
%
\ruby{汝}{きさま}は
\ruby{决}{けつ}して
\ruby{决}{けつ}して
\ruby[g]{本心}{ほんしん}を
\ruby{失}{うしな}ふやうな
\改行% 校正作業の簡略化のため
、
%
\原本頁{259-8}\改行%
\ruby[g]{其樣}{そ ん }な
\ruby{腑甲{\換字{斐}}}{ふ|が|い}
\ruby{無}{な}い
\ruby{奴}{やつ}では
\ruby{無}{な}いが、
%
\ruby[g]{何樣}{ど う }すれば
\ruby[g]{此樣}{こ ん }なに
\ruby{意氣地}{い|く|ぢ}が
\ruby{無}{な}くなつた。
%
こゝの% 踊り字調整「〻(二の字点、揺すり点)に見えるが(ゝ)」
\ruby{娘}{むすめ}の
\ruby[g]{擧動}{ふるまひ}を
\ruby{眼}{め}の
\ruby{{\換字{前}}}{まへ}に
\ruby{見}{み}て、
%
よく
\ruby{汝}{きさま}は
\ruby[g]{自{\換字{分}}}{じ ぶん}が
\原本頁{259-10}\改行%
\ruby{羞}{はづか}しくないナ。
%
\ruby{一}{いち}
\ruby[g]{少女}{せうぢよ}でさへ
\ruby{彼}{あ}の
\ruby{{\換字{通}}}{とほ}りだ、
%
\ruby{汝}{きさま}は
\ruby[g]{堂々}{だう〳〵}たる% ルビ調整(原本通り)濁点なし
\ruby[g]{男兒}{だんじ }で
\ruby{無}{な}いか、
%
\ruby[g]{乃公}{お れ }は
\ruby{彼}{あ}の
\ruby{娘}{むすめ}に
\ruby{頭}{あたま}を
\ruby{撲}{う}たれたが、
%
\ruby{汝}{きさま}は
\ruby[g]{精神}{こゝろ }に% 踊り字調整「〻(二の字点、揺すり点)に見えるが(ゝ)」
\ruby{鞭}{むち}を
\ruby{受}{う}けなかつたか。
%
\ruby{苟}{いやし}くも
\ruby{舊}{もと}の
\ruby[g]{水野}{みづの }
である
ならば、
%
\ruby{人}{ひと}
\ruby[g]{一倍}{いちばい}
\ruby{物}{もの}を
\ruby{思}{おも}ふ
\ruby[<j||]{汝}{きさま}の% 行末行頭の境界付近なので特例処置を施す
\ruby{事}{こと}だもの、
%
\ruby{必}{かなら}ず
\ruby[g]{{\換字{感}}奮}{かんぷん}
せずには
\ruby{居}{を}らぬ
\ruby{筈}{はず}だが、
%
\ruby[g]{衰へ}{おとろ }
\ruby{果}{は}て
\ruby{{\換字{弱}}}{よわ}り
\ruby{果}{は}てた
\ruby{今}{いま}の
\ruby{汝}{きさま}は、
%
\ruby[g]{矢張}{やつぱ }り
\ruby{首}{くび}を
\ruby{俛}{た}るゝばかりか。% 踊り字調整「〻(二の字点、揺すり点)に見えるが(ゝ)」
%
\ruby[g]{此家}{こ ゝ }の% 踊り字調整「〻(二の字点、揺すり点)に見えるが(ゝ)」
\ruby{娘}{むすめ}の
\ruby[g]{健氣}{けなげ }な
\原本頁{260-4}\改行%
\ruby[g]{振舞}{ふるまひ}と、
%
\ruby{汝}{きさま}の
\ruby{其}{そ}の
\ruby{萎}{しを}れ
きつた
\ruby[g]{狀態}{ありさま}
とを、
%
\ruby[g]{見比}{み くら}べ
\ruby{思}{おも}ひ
\ruby{比}{くら}べると
\ruby{此}{こ}の
\ruby[g]{日方}{ひ かた}は、
%
これほど
までに
\ruby{汝}{きさま}は
\ruby{衰}{おとろ}へた
かと、
%
\ruby{汝}{きさま}の
\ruby{衰}{おとろ}へ
\ruby{果}{は}てたのが
\ruby{悲}{かな}しくて
\ruby{涙}{なみだ}が
\ruby{出}{で}る!。
%
\ruby{女}{をんな}にも
\ruby{劣}{おと}る
やうになつたとは
\ruby{餘}{あま}り
\ruby[g]{{\換字{情}}無}{なさけな}い!。
%
\ruby[g]{何故}{な ぜ }
\ruby{{\換字{迷}}}{まよ}ふ
にしても
\ruby[g]{男兒}{をとこ }
らしく
\ruby{{\換字{迷}}}{まよ}つて
\ruby{吳}{く}れぬ?。
』
