\Entry{其四十五}

% メモ 校正終了 2024-05-08 2024-06-05
\原本頁{260-9}%
『
\ruby{心}{こゝろ}を% 踊り字調整「〻(二の字点、揺すり点)に見えるが(ゝ)」
\ruby{一}{いつ}
\ruby[g]{{\換字{婦}}人}{ぷ じん}に% ルビ調整(原本通り)「婦」のルビは原本では「ぷ」に見える
\ruby{苦}{くるし}むる
\ruby{汝}{きさま}を
\ruby{見}{み}るのも
\ruby[g]{忌々}{いま〳〵}しいが、
%
\ruby{勇}{ゆう}を
\ruby{一}{いち}
\ruby[g]{少女}{せうぢよ}に
\ruby{遜}{ゆづ}る
\ruby[g]{汝の}{きさま }
\ruby{腑甲{\換字{斐}}}{ふ|が|ひ}
なさを
\ruby{見}{み}ては、
%
あゝ% 踊り字調整「〻(二の字点、揺すり点)に見えるが(ゝ)」
\ruby[g]{凡骨}{ぼんこつ}では
\ruby{無}{な}かつた
\ruby[g]{水野}{みづの }
\ruby[<j>]{某}{なにがし}が
\改行% 校正作業の簡略化のため
、
%
\原本頁{261-1}\改行%
\ruby[g]{如是}{か う }も
\ruby{衰}{おとろ}へた
ものかと
\ruby[g]{口惜}{くちをし}くなる!。
%
\ruby[g]{島木}{しまき }の
\ruby{言}{い}つたことが
\ruby[g]{眞實}{まこと }ならば、
%
\ruby{此}{こ}の
\ruby[g]{日方}{ひ かた}は
\ruby[g]{全然}{ぜん〴〵}
\ruby[g]{否認}{ひ にん}する
けれど、
%
そりやあ
\ruby{或}{あるひ}は
\ruby[g]{戀愛}{れんあい}に
\原本頁{261-3}\改行%
\ruby{陷}{おちい}るのも
\ruby{已}{や}むを
\ruby{得}{え}ん
ことか
\ruby{知}{し}らんが、
%
\ruby[g]{何故}{な ぜ }
\ruby[g]{戀愛}{れんあい}に
\ruby{陷}{おちい}つたら
\ruby{陷}{おちい}つたで
\ruby[g]{男兒}{をとこ }らしく
はせん?。
%
\ruby{同}{おな}じ
\ruby[g]{{\換字{迷}}に}{まよひ }
\ruby{陷}{おちい}つても、
%
\ruby{人}{ひと}にも
\ruby{告}{つ}げず
\ruby{物}{もの}を
\ruby{思}{おも}つて
\ruby{{\換字{空}}}{むな}しく
\ruby{泣}{な}き
\ruby{悶}{もだ}{\換字{𛀁}}て
\ruby{居}{ゐ}るばかりが
\ruby{{\換字{道}}}{みち}
でも
ある
まい。
%
いたづらに
\ruby[g]{遲疑}{ち ぎ }
\ruby[g]{躊躇}{ちうちよ}して、
%
\ruby[g]{何等}{なんら }の
\ruby[g]{措置}{そ ち }をも
\ruby{取}{と}る
ことを
\ruby{敢}{あへ}て
せぬのは
\改行% 校正作業の簡略化のため
、
%
\原本頁{261-7}\改行%
\ruby{大{\換字{丈}}夫}{だい|ぢやう|ぶ}の
\ruby{最}{もつと}も
\ruby{慚}{は}づる
ところだ。
%
たとひ
\ruby[g]{少々}{せう〳〵}は
\ruby{其}{そ}の
\ruby[g]{{\換字{所}}爲}{しよゐ }% ルビ調整(原本通り)「所爲」のルビは(せゐ)ではなく(しよゐ)
\ruby{宜}{よろし}きを
\ruby[<j||]{失}{うしな}つても、% 行末行頭の境界付近なので特例処置を施す
%
\ruby{慮}{はか}つて、
%
\ruby{斷}{だん}じて、
%
\ruby{行}{おこな}つて、
%
\ruby{着々}{ちやく|〳〵}と
\ruby[g]{事{\換字{情}}}{じじやう}の
\ruby[g]{展開}{てんかい}に
\ruby{應}{おう}じて
\原本頁{261-9}\改行%
\ruby{行}{ゆ}くのが、
%
\ruby[g]{男子}{だんし }の
\ruby{敢}{あへ}て
すべき
\ruby{{\換字{道}}}{みち}では
\ruby{無}{な}いか。
%
\ruby[g]{{\換字{猶}}豫}{ゆうよ }して
\ruby{决}{けつ}せざるは、
%
\ruby[g]{軍務}{ぐんむ }では
\ruby{何}{なに}よりも
\ruby[<j>]{甚}{はなはだ}しく
\ruby{惡}{にく}むところだが、
%
\ruby{獨}{ひと}り
\ruby[g]{軍人}{ぐんじん}
のみが
\原本頁{261-11}\改行%
\ruby[g]{左樣}{さ う }
\ruby[g]{覺悟}{かくご }
すべき
では
\ruby{無}{な}い、
%
\ruby[g]{何人}{なんびと}に
\ruby{取}{と}つても
\ruby[g]{遲疑}{ち ぎ }
\ruby[g]{躊躇}{ちうちよ}ほど、
%
\ruby{其}{その}
\ruby{人}{ひと}を
\ruby{{\換字{害}}}{がい}する
ものは
あるまい。
%
\ruby{同}{おな}じ
\ruby[g]{{\換字{婦}}人}{ふ じん}に
\ruby[||j>]{愛}{あい}
\ruby[||j>]{着}{ちやく}
% \ruby{愛着}{あい|ちやく}
するなら、
%
\ruby[g]{水野}{みづの }
\ruby{汝}{きさま}も
\原本頁{262-2}\改行%
\ruby[g]{男兒}{をとこ }では
\ruby{無}{な}いか、
%
\ruby[g]{何故}{な ぜ }
\ruby{男}{をとこ}
らしく
\ruby[g]{行動}{かうどう}
せぬ?。
%
ビスマークは
\ruby[g]{何樣}{ど う }して
\ruby{其}{そ}の
\ruby{妻}{つま}を
\ruby{得}{た}た?。% ルビ調整(原本通り)「得」のルビは(た)で(𛀁)に見えない
%
\ruby{烈}{はげ}しく
\ruby{思}{おも}つた、
%
\ruby{明}{あき}らかに
\ruby{求}{もと}めた、
%
\ruby{而}{そ}して
\原本頁{262-4}\改行%
\ruby{{\換字{終}}}{つひ}に
\ruby{得}{{\換字{𛀁}}}た
といふに
\ruby{{\換字{過}}}{す}ぎん
\ruby{事}{こと}
ではないか。
%
\ruby{今}{いま}は
\ruby{其}{そ}の
\ruby[g]{夫人}{ふ じん}も
\ruby{世}{よ}を
\ruby{去}{さ}られたが、
%
\ruby{我}{わ}が
\ruby[g]{陸軍}{りくぐん}
\ruby[||j>]{大}{たい}
\ruby[||j>]{將}{しやう}の
% \ruby{大將}{たい|しやう}の
\ruby[g]{某侯}{ぼうこう}が、
%
\ruby{年}{とし}も
\ruby{{\換字{若}}}{わか}く
\ruby{身}{み}も
\ruby{鄙}{いやし}かつた
\ruby{時}{とき}の
\原本頁{262-6}\改行%
\ruby{戀}{こひ}の
\ruby[||j>]{物}{もの}
\ruby[||j>]{語}{がたり}は、
% \ruby{物語}{もの|がたり}は、
%
\ruby[g]{虛實}{きよじつ}は
\ruby{知}{し}らぬが
\ruby{汝}{きさま}も
\ruby{知}{し}つて
\ruby{居}{ゐ}やう。
%
\ruby[g]{徒然}{と ぜん}を
\ruby{慰}{なぐさ}める
ばかりに
\ruby{讀}{よ}んだ
\ruby[g]{雜書}{ざつしよ}に、
%
\ruby[g]{{\換字{文}}覺}{もんがく}の
\ruby{事}{こと}を
\ruby{記}{しる}して
あつたが、
%
\ruby{彼}{あれ}を
\ruby{見}{み}て
\原本頁{262-8}\改行%
\ruby[g]{先夜}{せんや }も
\ruby{汝}{きさま}の
\ruby{上}{うへ}を、
%
\ruby[g]{自然}{おのづ }と
\ruby{胸}{むね}に
\ruby{思}{おも}ひ
\ruby{{\換字{浮}}}{うか}めた。
%
\ruby[g]{{\換字{文}}覺}{もんがく}は
% 文覚(もんがく、生没年不詳)
% 平安時代末期から鎌倉時代初期にかけての武士・真言宗の僧。
% 父は左近将監茂遠(もちとお)。
% 俗名は遠藤盛遠(えんどうもりとお)。
% ここは水野のことではなく
% 同僚の源渡 (みなもとのわたる)の妻袈裟(けさ)に恋慕し、
% 誤って彼女を殺したのが動機で出家し、諸国の霊場を遍歴、修行した武士・真言宗の僧らしい。
\ruby{全}{まつた}く
\ruby[g]{失敗}{しつぱい}し、
%
\原本頁{262-9}\改行%
ビスマークや
\ruby{我}{わ}が
\ruby[||j>]{大}{たい}
\ruby[||j>]{將}{しやう}は
% \ruby{大將}{たい|しやう}は
\ruby{思}{おも}ひを
\ruby{{\換字{遂}}}{と}げたが、
%
\ruby{其}{そ}の
\ruby[g]{遲疑}{ち ぎ }
\ruby[g]{躊躇}{ちうちよ}して
\ruby{{\換字{空}}}{あだ}に
\ruby{物}{もの}を
\ruby{思}{おも}はぬは
\ruby{同}{おな}じ
\ruby{事}{こと}だ、
%
\ruby{{\換字{飽}}}{あく}まで
\ruby[g]{男兒}{をとこ }らしく
\ruby{戀}{こひ}をしたのは
\ruby{同}{おな}じ
\原本頁{262-11}\改行%
\ruby{事}{こと}だ、
%
\ruby{世}{よ}の
\ruby[g]{小說}{せうせつ}に
あるやうに
\ruby[g]{女々}{め ゝ }しく% 踊り字調整「〻(二の字点、揺すり点)に見えるが(ゝ)」
\ruby[g]{月日}{つきひ }を
\ruby{經}{へ}ぬ
のは
\ruby{同}{おな}じ
\ruby{事}{こと}だ
\改行% 校正作業の簡略化のため
。
%%%%%%%%%%
\原本頁{263-1}\改行%
\ruby{彼}{あ}の
\ruby[g]{{\換字{文}}覺}{もんがく}が
\ruby{云}{い}つた
\ruby{言}{ことば}に、
%
\ruby{戀}{こひ}には
\ruby{人}{ひと}の
\ruby{死}{し}なぬ
ものかは、
%
と
\ruby{苦}{くる}しい
\ruby{思}{おもひ}を
\ruby[||j>]{白}{はく}
\ruby[||j>]{狀}{じやう}
% \ruby{白狀}{はく|じやう}
してゐるが、
%
\ruby[g]{水野}{みづの }、
%
\ruby{汝}{きさま}も
\ruby{其}{そ}の
\ruby{衰}{おとろ}へかた
\ruby{其}{そ}の
\ruby{窶}{やつ}れかたでは、
%
\ruby[g]{成程}{なるほど}
\ruby{汝}{きさま}も
\ruby[g]{死{\換字{兼}}}{しにか }ねない
\ruby[g]{樣子}{やうす }だ。
%
とても
\ruby{其}{それ}
\ruby{程}{ほど}に
\ruby{{\換字{迷}}}{まよ}つた
ならば、
%
\原本頁{263-4}\改行%
\ruby[g]{何故}{な ぜ }
\ruby[g]{男兒}{をとこ }
らしく
\ruby{{\換字{進}}}{すゝ}んでは% 踊り字調整「〻(二の字点、揺すり点)に見えるが(ゝ)」
\ruby[g]{振舞}{ふるま }はぬ?、
%
\ruby{默}{だま}つて
\ruby{物}{もの}を
\ruby{思}{おも}つても
\ruby{死}{し}ぬなら、
%
\ruby[g]{何故}{な ぜ }
\ruby[g]{成敗}{せいばい}
\ruby[g]{生死}{しやうし}
\ruby{此}{こ}の
\ruby[g]{一擲}{いつてき}と、
%
\ruby[g]{男兒}{をとこ }
らしく
\ruby[g]{{\換字{運}}命}{うんめい}の
\ruby{何}{なに}を
\ruby{與}{あた}ふる
かを
\ruby{見}{み}ぬ?。
%
\ruby[g]{{\換字{文}}覺}{もんがく}は
たゞ% 踊り字調整「〻(二の字点、揺すり点)に濁点に見えるが(ゞ)」
\ruby[g]{我慢}{が まん}
ばかりの
\ruby{男}{をとこ}では
\ruby{無}{な}い、
%
\ruby[g]{袈裟}{け さ }を
\ruby{殺}{ころ}した
\ruby{其}{そ}の
\ruby{後}{あと}では、
%
\ruby{辰}{たつ}の
\ruby{刻}{こく}より
\ruby{未}{ひつじ}の
\ruby{刻}{こく}まで、
%
\ruby[g]{四時}{よ とき}と
\ruby{云}{い}へば
\ruby{八時間}{はち|じ|かん}だ
\改行% 校正作業の簡略化のため
、
%
\原本頁{263-8}\改行%
\ruby{其}{そ}の
\ruby{八時間}{はち|じ|かん}を
\ruby[g]{大聲}{おほごゑ}
\ruby{揚}{あ}げて、
%
\ruby{荒}{あら}くれた
\ruby{眼}{め}から
\ruby{霰}{あられ}のやうな
\ruby[g]{涙を}{なみだ }
\ruby{落}{おと}しながら
\ruby{泣}{な}き
\ruby{{\換字{通}}}{とほ}した
とある、
%
\ruby{恐}{おそろ}しい
\ruby{{\換字{情}}}{じやう}の
\ruby{深}{ふか}い
\ruby[g]{熱烈}{ねつれつ}な
\ruby{奴}{やつ}だ。
%
\ruby[<j||]{汝}{おまへ}の% 行末行頭の境界付近なので特例処置を施す
\makeatletter
\@ifundefined{デバッグ@ビルド}{%
  \ruby[g]{其位の}{そのくらゐ }
}{%
  \ruby[||j>]{其}{その}
  \ruby[||j>]{位}{くらゐ}の
}%
\makeatother
% \ruby{其位}{その|くらゐ}の
\原本頁{263-10}\改行%
\ruby{奴}{やつ}が
\ruby[g]{手荒}{て あら}い
\ruby{事}{こと}を
するまでには、
%
\ruby{一}{ひ}ト
\ruby{{\換字{通}}}{とほ}りや
\ruby{二}{ふ}タ
\ruby{{\換字{通}}}{とほ}りで
\ruby{無}{な}く
\ruby{物}{もの}を
\ruby{思}{おも}つた
らうが、
%
\ruby{歸}{き}する
ところ
\ruby{暴}{ぼう}でも
\ruby{何}{なん}でも
\ruby[g]{男兒}{をとこ }
らしく
\ruby{思}{おも}ふ
まゝに% 踊り字調整「〻(二の字点、揺すり点)に見えるが(ゝ)」
\ruby[g]{振舞}{ふるま }つた
のは
また
\ruby{已}{や}むを
\ruby{得}{{\換字{𛀁}}}ん。
%
とても
かくても
\ruby{物}{もの}を
\ruby{思}{おも}つて、
\原本頁{264-2}\改行%
\ruby{戀}{こひ}に
\ruby[g]{死{\換字{兼}}}{しにか }ね
もすまい
ならば、
%
\ruby[g]{何故}{な ぜ }
\ruby[g]{男兒}{をとこ }
らしくは
\ruby[g]{振舞}{ふるま }はぬ?。
%
\ruby{當}{あた}つて
\ruby{碎}{くだ}くか
\ruby{碎}{くだ}けろかだ、
%
\ruby[g]{突貫し}{とつくわん }て
\ruby{倒}{たふ}さるゝか% 踊り字調整「〻(二の字点、揺すり点)に見えるが(ゝ)」
\ruby{倒}{たふ}すかの
\ruby{事}{こと}だ、
%
\ruby[<j||]{首}{かうべ}% 行末行頭の境界付近なので特例処置を施す
\ruby{離}{はな}ると
% \ruby{首離}{かうべ|はな}ると
\ruby{雖}{いへど}も
\ruby{身}{み}
\ruby{懲}{こ}りず、
%
といふ
\ruby[<j>]{勢}{いきほひ}で
\ruby[||j>]{突}{とつ}
\ruby[||j>]{貫}{くわん}して
% \ruby{突貫}{とつ|くわん}して
\ruby[g]{仕舞}{し ま }へ。
%
\ruby{汝}{きさま}が
\ruby{良}{い}い
\ruby[g]{{\換字{婦}}人}{ふ じん}を
\ruby{得}{{\換字{𛀁}}}て
\ruby[||j>]{大}{たい}
\ruby[||j>]{將}{しやう}になるか、
% \ruby{大將}{たい|しやう}になるか、
%
たゞし% 踊り字調整「〻(二の字点、揺すり点)に濁点に見えるが(ゞ)」
\ruby[g]{{\換字{文}}覺}{もんがく}の
やうな
\ruby[|j|]{狂}{きちがひ}
\ruby{僧}{ばうず}に
なるか、
%
\原本頁{264-6}\改行%
それは
\ruby[g]{何方}{どちら }に
なつても
\ruby[g]{乃公}{お れ }は
\ruby{關}{かま}はんが、
%
\ruby[g]{何樣}{ど う }せ
\ruby[g]{汝は}{きさま }
\ruby{欲}{よく}が
\ruby{薄}{うす}くて
\ruby[g]{高慢}{かうまん}が
\ruby{{\換字{強}}}{つよ}い、
%
\ruby[g]{變挺}{へんてこ}な
\ruby{男}{をとこ}に
\ruby{生}{うま}れて
\ruby{居}{ゐ}るのだから、
%
\ruby[g]{坊主}{ばうず }に
なつて
\ruby[g]{仕舞}{し ま }ふ
のも
\ruby[||j>]{寧}{いつそ}
\ruby[||j>]{宜}{ よか}らう、
%
\ruby[g]{日方}{ひ かた}は
\ruby[g]{{\換字{貧}}乏}{びんばう}でも
\ruby{汝}{きさま}が
\ruby[g]{左樣}{さ う }
なつたら、
%
\ruby{{\換字{麻}}}{あさ}の
\ruby[<j||]{衣}{ころも}
\ruby[g]{位は}{ぐらゐ }
\ruby[g]{寄{\換字{進}}}{き しん}して
\ruby{立}{たて}
\ruby{{\換字{過}}}{すご}して
\ruby{{\換字{遣}}}{や}る!。
%
\ruby{汝}{きさま}が
\ruby{衰}{おとろ}へに
\ruby{衰}{おとろ}へて、
%
\ruby{一}{いち}
\ruby[g]{少女}{せうぢよ}にも
\原本頁{264-10}\改行%
\ruby{其}{そ}の
\ruby[g]{勇氣}{ゆうき }が
\ruby{及}{およ}ばん
やうに
なつて
\ruby{戀}{こひ}に
\ruby{死}{し}ぬのを、
%
\ruby[g]{見殺}{み ごろ}し
にするのは
\ruby[g]{乃公}{お れ }には
\ruby[g]{出來}{で き }ぬ。
%
\ruby[g]{男兒}{をとこ }
らしく
\ruby[g]{振舞}{ふるま }へ、
%
\ruby{女}{をんな}
では
あるまい。
%
\ruby{高}{たか}が
\原本頁{265-1}\改行%
\ruby{一}{いつ}
\ruby[g]{{\換字{婦}}人}{ぷ じん}を
\ruby[g]{對敵}{あひて }にして、
%
\ruby{{\換字{遠}}距離}{ゑん|きよ|り}で
\ruby[g]{彈藥}{だんやく}を
\ruby{使}{つか}ひ
\ruby{盡}{つく}すのは
\ruby{愚}{おろか}な
\ruby{事}{こと}だ。
%
いつそ
\ruby{一}{ひ}と
\ruby{思}{おもひ}に
\ruby[g]{突貫し}{とつくわん }て
\ruby[g]{仕舞}{し ま }へ。
%
\ruby{{\換字{勝}}}{か}つか
\ruby{負}{ま}けるかの
\ruby{他}{ほか}には
\ruby{物}{もの}は
\ruby{有}{あ}りは
\ruby[g]{仕無}{し な }い。
%
\ruby[g]{{\換字{遠}}地}{とほく }から
\ruby{敵}{てき}に
\ruby{{\換字{勝}}}{か}たう
といふのは
\ruby[g]{贅澤}{ぜいたく}な
\ruby[g]{詮義}{せんぎ }だ。
%
\原本頁{265-4}\改行%
\ruby[g]{羽{\換字{勝}}}{は がち}も
\ruby[g]{乃公}{お れ }の
\ruby{言}{い}ふことを
\ruby[g]{無理}{む り }とは
\ruby{思}{おも}ふまい、
%
\ruby[g]{何樣}{ど う }だ
\ruby[g]{水野}{みづの }
\ruby{汝}{きさま}は
\ruby{何}{なん}と
\ruby{思}{おも}ふ?。
%
\ruby[g]{女々}{め ゝ }しい% 踊り字調整「〻(二の字点、揺すり点)に見えるが(ゝ)」
\ruby{事}{こと}は
\ruby{宜}{い}い
\ruby[g]{加減}{か げん}に
\ruby{止}{や}めろ。
%
もう
\ruby[g]{乃公}{お れ }
は
\ruby{此}{これ}
\ruby{限}{き}り
\ruby{物}{もの}は
\ruby{言}{い}はぬ、
%
これだけ
\ruby{言}{い}つても
\ruby[g]{乃公}{お れ }の
\ruby{云}{い}ふ
\ruby{事}{こと}を
\ruby{用}{もち}ゐん
ならば、
%
\ruby{舊}{もと}の
\ruby[g]{水野}{みづの }に
なり
\ruby{{\換字{返}}}{かへ}る
までは、
%
\ruby{汝}{きさま}には
\ruby{會}{あ}はん。
』
