\Entry{其四十五}

『
\ruby{心}{こゝろ}を
\ruby{一{\換字{婦}}人}{いつ|ぷ|じん}に% 「婦」のルビは原本では「ぷ」に見えるのでそのまま
\ruby{苦}{くるし}むる
\ruby{汝}{きさま}を
\ruby{見}{み}るのも
\ruby{忌々}{いま|〳〵}しいが、
\ruby{勇}{ゆう}を
\ruby{一少女}{いち|せう|ぢよ}に
\ruby{遜}{ゆづ}る
\ruby[<h||]{汝}{きさま}
\ruby{腑甲斐}{ふ|が|ひ}なさを
\ruby{見}{み}ては、あゝ
\ruby{凡骨}{ぼん|こつ}では
\ruby{無}{な}かつた
\ruby{水野某}{みづ|の|なにがし}が、
\ruby{如是}{か|う}も
\ruby{衰}{おとろ}へたものかと
\ruby{口惜}{くち|をし}くなる!。

\ruby{島木}{しま|き}の
\ruby{言}{い}つたことが
\ruby{眞實}{まこ|と}ならば、
\ruby{此}{こ}の
\ruby{日方}{ひ|かた}は
\ruby{全然}{ぜん|〴〵}
\ruby{否認}{ひ|にん}するけれど、そりやあ
\ruby{或}{あるひ}は
\ruby{戀愛}{れん|あい}に
\ruby{陷}{おちい}るのも
\ruby{已}{や}むを
\ruby{得}{え}んことか
\ruby{知}{し}らんが、
\ruby{何故}{な|ぜ}
\ruby{戀愛}{れん|あい}に
\ruby{陷}{おちい}つたで
\ruby{男兒}{をと|こ}らしくはせん?。
\ruby{同}{おな}じ
\ruby{{\換字{迷}}}{まよひ}に
\ruby{陷}{おちい}つても、
\ruby{人}{ひと}にも
\ruby{告}{つ}げず
\ruby{物}{もの}を
\ruby{思}{おも}つて
\ruby{空}{むな}しく
\ruby{泣}{な}き
\ruby{悶}{もだ}\換字{𛀁}て
\ruby{居}{ゐ}るばかりが
\ruby{{\換字{道}}}{みち}でもあるまい。
いたづらに
\ruby{遲疑躊躇}{ち|ぎ|ちう|ちよ}して、
\ruby{何等}{なん|ら}の
\ruby{措置}{そ|ち}をも
\ruby{取}{と}ることを
\ruby{敢}{あへ}てせぬのは
\ruby{大{\換字{丈}}夫}{だい|ぢやう|ぶ}の
\ruby{最}{もつと}も
\ruby{慚}{は}づるところだ。
たとひ
\ruby{少々}{せう|〳〵}は
\ruby{其}{そ}の
\ruby{{\換字{所}}爲宜}{せ|ゐ|よろし}きを
\ruby{失}{うしな}つても、
\ruby{慮}{はか}つて、
\ruby{斷}{だん}じて、
\ruby{行}{おこな}つて、
\ruby{着々}{ちやく|〳〵}と
\ruby{事{\換字{情}}}{じ|じやう}の
\ruby{展開}{てん|かい}に
\ruby{應}{おう}じて
\ruby{行}{ゆ}くのが、
\ruby{男子}{だん|し}の
\ruby{敢}{あへ}てすべき
\ruby{{\換字{道}}}{みち}では
\ruby{無}{な}いか。
\ruby{{\換字{猶}}豫}{ゆう|よ}して
\ruby{决}{けつ}せざるは、
\ruby{軍務}{ぐん|む}では
\ruby{何}{なに}よりも
\ruby{甚}{はなはだ}しく
\ruby{惡}{にく}むところだが、
\ruby{獨}{ひと}り
\ruby{軍人}{ぐん|じん}のみが
\ruby{左樣}{さ|う}
\ruby{覺悟}{かく|ご}すべきでは
\ruby{無}{な}い、
\ruby{何人}{なん|びと}に
\ruby{取}{と}つても
\ruby{遲疑躊躇}{ち|ぎ|ちう|ちよ}ほど、
\ruby{其人}{その|ひと}を
\ruby{{\換字{害}}}{がい}するものはあるまい。
\ruby{同}{おな}じ
\ruby{{\換字{婦}}人}{ふ|じん}に
\ruby{愛着}{あい|ちやく}するなら、
\ruby{水野}{みづ|の}
\ruby{汝}{きさま}も
\ruby{男兒}{をと|こ}では
\ruby{無}{な}いか、
\ruby{何故}{な|ぜ}
\ruby{男兒}{をと|こ}らしく
\ruby{行動}{かう|どう}せぬ?。
ビスマークは
\ruby{何樣}{ど|う}して
\ruby{其}{そ}の
\ruby{妻}{つま}を
\ruby{得}{\換字{𛀁}}た!。% 原文の「得」のルビは「た」だった
\ruby{烈}{はげ}しく
\ruby{思}{おも}つた、
\ruby{明}{あき}らかに
\ruby{求}{もと}めた、
\ruby{而}{そ}して
\ruby{{\換字{終}}}{つひ}に
\ruby{得}{\換字{𛀁}}たといふに
\ruby{{\換字{過}}}{す}ぎん
\ruby{事}{こと}ではないか。
\ruby{今}{いま}は
\ruby{其}{そ}の
\ruby{夫人}{ふ|じん}も
\ruby{世}{よ}を
\ruby{去}{さ}られたが、
\ruby{我}{わ}が
\ruby{陸軍大將}{りく|ぐん|たい|しやう}の
\ruby{某侯}{ぼう|こう}が、
\ruby{年}{とし}も
\ruby{若}{わか}く
\ruby{身}{み}も
\ruby{鄙}{いやし}かつた
\ruby{時}{とき}の
\ruby{戀}{こひ}の
\ruby{物語}{もの|がたり}は、
\ruby{虛實}{きよ|じつ}は
\ruby{知}{し}らぬが
\ruby{汝}{きさま}も
\ruby{知}{し}つて
\ruby{居}{ゐ}やう。
\ruby{徒然}{と|ぜん}を
\ruby{慰}{なぐさ}めるばかりに
\ruby{讀}{よ}んだ
\ruby{雜書}{ざつ|しよ}に、
\ruby{{\換字{文}}覺}{もん|がく}の
\ruby{事}{こと}を
\ruby{記}{しる}してあつたが、
\ruby{彼}{あれ}を
\ruby{見}{み}て
\ruby{先夜}{せん|や}も
\ruby{汝}{きさま}の
\ruby{上}{うへ}を、
\ruby{自然}{おの|づ}と
\ruby{胸}{むね}に
\ruby{思}{おも}ひ
\ruby{{\換字{浮}}}{うか}めた。
\ruby{{\換字{文}}覺}{もん|がく}は
\ruby{全}{まつた}く
\ruby{失敗}{しつ|ぱい}し、ピスマークや
\ruby{我}{わ}が
\ruby{大將}{たい|しやう}は
\ruby{思}{おも}ひを
\ruby{{\換字{遂}}}{と}げたが、
\ruby{其}{そ}の
\ruby{遲疑躊躇}{ち|ぎ|ちう|ちよ}して
\ruby{空}{あだ}に
\ruby{物}{もの}を
\ruby{思}{おも}はぬは
\ruby{同}{おな}じ
\ruby{事}{こと}だ、
\ruby{{\換字{飽}}}{あく}まで
\ruby{男兒}{をと|こ}らしく
\ruby{戀}{こひ}をしたのは
\ruby{同}{おな}じ
\ruby{事}{こと}だ。
\ruby{彼}{あ}の
\ruby{{\換字{文}}覺}{もん|がく}が
\ruby{云}{い}つた
\ruby{言}{ことば}に、
\ruby{戀}{こひ}には
\ruby{人}{ひと}の
\ruby{死}{し}なぬものかは、と
\ruby{苦}{くる}しい
\ruby{思}{おもひ}を
\ruby{白狀}{はく|じやう}してゐるが、
\ruby{水野}{みづ|の}、
\ruby{汝}{きさま}も
\ruby{其}{そ}の
\ruby{衰}{おとろ}へかた
\ruby{其}{そ}の
\ruby{窶}{やつ}れかたでは、
\ruby{成程}{なる|ほど}
\ruby{汝}{きさま}も
\ruby{死{\換字{兼}}}{しに|か}ねない
\ruby{樣子}{やう|す}だ。
とても
\ruby{其程}{それ|ほど}に
\ruby{{\換字{迷}}}{まよ}つたならば、
\ruby{{\換字{進}}}{すゝ}んでは
\ruby{振舞}{ふる|ま}はぬ?、
\ruby{默}{だま}つて
\ruby{物}{もの}を
\ruby{思}{おも}つても
\ruby{死}{し}ぬなら、
\ruby{何故}{な|ぜ}
\ruby{成敗}{せい|ばい}
\ruby{生死}{しやう|し}
\ruby{此}{こ}の
\ruby{一擲}{いつ|てき}と、
\ruby{男兒}{をと|こ}らしく
\ruby{{\換字{運}}命}{うん|めい}の
\ruby{何}{なに}を
\ruby{與}{あた}ふるかを
\ruby{見}{み}ぬ?。
\ruby{{\換字{文}}覺}{もん|がく}はたゞ
\ruby{我慢}{が|まん}ばかりの
\ruby{男}{をとこ}では
\ruby{無}{な}い、
\ruby{袈裟}{け|さ}を
\ruby{殺}{ころ}した
\ruby{其}{そ}の
\ruby{後}{あと}では、
\ruby{辰}{たつ}の
\ruby{刻}{こく}より
\ruby{未}{ひつじ}の
\ruby{刻}{こく}まで、
\ruby{四時}{よ|とき}と
\ruby{云}{い}へば
\ruby{八時間}{はち|じ|かん}だ、
\ruby{其}{そ}の
\ruby{八時間}{はち|じ|かん}を
\ruby{大聲}{おほ|ごゑ}
\ruby{揚}{あ}げて、
\ruby{荒}{あら}くれた
\ruby{眼}{め}から
\ruby{霰}{あられ}のやうな
\ruby{淚}{なみだ}を
\ruby{落}{おと}しながら
\ruby{泣}{な}き
\ruby{通}{とほ}したとある、
\ruby{恐}{おそろ}しい
\ruby{{\換字{情}}}{じやう}の
\ruby{深}{ふか}い
\ruby{熱烈}{ねつ|れつ}な
\ruby{奴}{やつ}だ。
\ruby{其位}{その|くらゐ}の
\ruby{奴}{やつ}が
\ruby{手荒}{て|あら}い
\ruby{事}{こと}をするまでには、
\ruby{一}{ひ}ト
\ruby{通}{とほ}りや
\ruby{二}{ふ}タ
\ruby{通}{とほ}りで
\ruby{無}{な}く
\ruby{物}{もの}を
\ruby{思}{おも}つたらうが、
\ruby{歸}{き}するところ
\ruby{暴}{ぼう}でも
\ruby{何}{なん}でも
\ruby{男兒}{をと|こ}らしく
\ruby{思}{おも}ふまゝに
\ruby{振舞}{ふる|ま}つたのはまた
\ruby{已}{や}むを
\ruby{得}{\換字{𛀁}}ん。
とてもかくても
\ruby{物}{もの}を
\ruby{思}{おも}つて
\ruby{戀}{こひ}に
\ruby{死{\換字{兼}}}{しに|か}ねもすまいならば、
\ruby{何故}{な|ぜ}
\ruby{男兒}{をと|こ}らしくは
\ruby{振舞}{ふる|ま}はぬ?。
\ruby{當}{あた}つて
\ruby{碎}{くだ}くか
\ruby{碎}{くだ}けろかだ、
\ruby{{\換字{突}}貫}{とつ|くわん}してして
\ruby{倒}{たふ}さるゝか
\ruby{倒}{たふ}すかの
\ruby{事}{こと}だ、
\ruby{首離}{かうべ|はな}ると
\ruby{雖}{いへど}も
\ruby{身懲}{み|こ}りず、といふ
\ruby{勢}{いきほひ}で
\ruby{{\換字{突}}貫}{とつ|くわん}して
\ruby{仕舞}{し|ま}へ。
\ruby{汝}{きさま}が
\ruby{良}{よ}い
\ruby{{\換字{婦}}人}{ふ|じん}を
\ruby{得}{\換字{𛀁}}て
\ruby{大將}{たい|しやう}になるか、たゞし
\ruby{{\換字{文}}覺}{もん|がく}のやうな
\ruby{狂僧}{きちがひ|ばうず}になるかそれは
\ruby{何方}{どち|ら}になつても
\ruby{乃公}{お|れ}は
\ruby{關}{かま}はんが、
\ruby{何樣}{ど|う}せ
\ruby{汝}{きさま}は
\ruby{欲}{よく}が
\ruby{薄}{うす}くて
\ruby{高慢}{かう|まん}が
\ruby{{\換字{強}}}{つよ}い、
\ruby{變挺}{へん|てこ}な
\ruby{男}{をとこ}に
\ruby{生}{うま}れて
\ruby{居}{ゐ}るのだから、
\ruby{坊主}{ばう|ず}になつて
\ruby{仕舞}{し|ま}ふのも
\ruby{寧}{いつそ}
\ruby{宜}{よか}らう、
\ruby{日方}{ひ|かた}は
\ruby{貧乏}{びん|ばう}でも
\ruby{汝}{きさま}が
\ruby{左樣}{さ|う}なつたら、
\ruby{{\換字{麻}}}{あさ}の
\ruby{衣位}{ころも|ぐらゐ}は
\ruby{寄{\換字{進}}}{き|しん}して
\ruby{立{\換字{過}}}{たて|すご}して
\ruby{{\換字{遣}}}{や}る!。
\ruby{汝}{きさま}が
\ruby{衰}{おとろ}へに
\ruby{衰}{おとろ}へて、
\ruby{一少女}{いち|せう|ぢよ}にも
\ruby{其}{そ}の
\ruby{勇氣}{ゆう|き}が
\ruby{及}{およ}ばんやうになつて
\ruby{戀}{こひ}に
\ruby{死}{し}ぬのを、
\ruby{見殺}{み|ごろ}しにするのは
\ruby{乃公}{お|れ}には
\ruby{出來}{で|き}ぬ。
\ruby{男兒}{をと|こ}らしく
\ruby{振舞}{ふる|ま}へ、
\ruby{女}{をんな}ではあるまい。
\ruby{高}{たか}が
\ruby{一{\換字{婦}}人}{いち|ふ|じん}を
\ruby{對敵}{あひ|て}にして、
\ruby{{\換字{遠}}距離}{ゑん|きよ|り}で
\ruby{彈藥}{だん|やく}を
\ruby{使}{つか}ひ
\ruby{盡}{つく}すのは
\ruby{愚}{おろか}な
\ruby{事}{こと}だ。
いつそ
\ruby{一}{ひ}と
\ruby{思}{おもひ}に
\ruby{突貫}{とつ|くわん}して
\ruby{仕舞}{し|ま}へ。

\ruby{{\換字{勝}}}{か}つか
\ruby{負}{ま}けるかの
\ruby{他}{ほか}には
\ruby{物}{もの}は
\ruby{有}{あ}りは
\ruby{仕無}{し|な}い。
\ruby{{\換字{遠}}地}{とほ|く}から
\ruby{敵}{てき}に
\ruby{{\換字{勝}}}{か}たうといふのは
\ruby{贅澤}{ぜい|たく}な
\ruby{詮義}{せん|ぎ}だ。
\ruby{羽{\換字{勝}}}{は|がち}
\ruby{乃公}{お|れ}の
\ruby{言}{い}ふことを
\ruby{無理}{む|り}とは
\ruby{思}{おも}ふまい、
\ruby{何樣}{ど|う}だ
\ruby{水野}{みづ|の}
\ruby{汝}{きさま}は
\ruby{何}{なん}と
\ruby{思}{おも}ふ?。
\ruby{女々}{め|ゝ}しい
\ruby{事}{こと}は
\ruby{宜}{い}い
\ruby{加減}{か|げん}に
\ruby{止}{や}めろ。
もう
\ruby{乃公}{お|れ}
は
\ruby{此限}{これ|ぎ}り
\ruby{物}{もの}は
\ruby{言}{い}はぬ、これだけ
\ruby{言}{い}つても
\ruby{乃公}{お|れ}の
\ruby{云}{い}ふ
\ruby{事}{こと}を
\ruby{用}{もち}ゐんならば、
\ruby{舊}{もと}の
\ruby{水野}{みづ|の}になり
\ruby{{\換字{返}}}{かへ}るまでは、
\ruby{汝}{きさま}には
\ruby{會}{あ}はん。
』

