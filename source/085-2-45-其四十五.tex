\Entry{其四十五}

% メモ 校正終了 2024-05-08 2024-06-05
\原本頁{260-9}%
『
\ruby{心}{こゝろ}を% 踊り字調整「〻(二の字点、揺すり点)に見えるが(ゝ)」
\ruby{一}{いつ}
\ruby{{\換字{婦}}人}{ぷ|じん}に% ルビ調整(原本通り)「婦」のルビは原本では「ぷ」に見える
\ruby{苦}{くるし}むる
\ruby{汝}{きさま}を
\ruby{見}{み}るのも
\ruby{忌々}{いま|〳〵}しいが、
%
\ruby{勇}{ゆう}を
\ruby{一}{いち}
\ruby{少女}{せう|ぢよ}に
\ruby{遜}{ゆづ}る
\ruby{汝}{きさま}の
\ruby{腑甲{\換字{斐}}}{ふ|が|ひ}
なさを
\ruby{見}{み}ては、
%
あゝ% 踊り字調整「〻(二の字点、揺すり点)に見えるが(ゝ)」
\ruby{凡骨}{ぼん|こつ}では
\ruby{無}{な}かつた
\ruby{水野}{みづ|の}
\ruby[<j>]{某}{なにがし}が
\改行% 校正作業の簡略化のため
、
%
\原本頁{261-1}\改行%
\ruby{如是}{か|う}も
\ruby{衰}{おとろ}へた
ものかと
\ruby{口惜}{くち|をし}くなる!。
%
\ruby{島木}{しま|き}の
\ruby{言}{い}つたことが
\ruby{眞實}{まこ|と}ならば、
%
\ruby{此}{こ}の
\ruby{日方}{ひ|かた}は
\ruby{全然}{ぜん|〴〵}
\ruby{否認}{ひ|にん}する
けれど、
%
そりやあ
\ruby{或}{あるひ}は
\ruby{戀愛}{れん|あい}に
\原本頁{261-3}\改行%
\ruby{陷}{おちい}るのも
\ruby{已}{や}むを
\ruby{得}{え}ん
ことか
\ruby{知}{し}らんが、
%
\ruby{何故}{な|ぜ}
\ruby{戀愛}{れん|あい}に
\ruby{陷}{おちい}つたら
\ruby{陷}{おちい}つたで
\ruby{男兒}{をと|こ}らしく
はせん?。
%
\ruby{同}{おな}じ
\ruby{{\換字{迷}}}{まよひ}に
\ruby{陷}{おちい}つても、
%
\ruby{人}{ひと}にも
\ruby{告}{つ}げず
\ruby{物}{もの}を
\ruby{思}{おも}つて
\ruby{{\換字{空}}}{むな}しく
\ruby{泣}{な}き
\ruby{悶}{もだ}{\換字{𛀁}}て
\ruby{居}{ゐ}るばかりが
\ruby{{\換字{道}}}{みち}
でも
ある
まい。
%
いたづらに
\ruby{遲疑}{ち|ぎ}
\ruby{躊躇}{ちう|ちよ}して、
%
\ruby{何等}{なん|ら}の
\ruby{措置}{そ|ち}をも
\ruby{取}{と}る
ことを
\ruby{敢}{あへ}て
せぬのは
\改行% 校正作業の簡略化のため
、
%
\原本頁{261-7}\改行%
\ruby{大{\換字{丈}}夫}{だい|ぢやう|ぶ}の
\ruby{最}{もつと}も
\ruby{慚}{は}づる
ところだ。
%
たとひ
\ruby{少々}{せう|〳〵}は
\ruby{其}{そ}の
\ruby{{\換字{所}}爲}{しよ|ゐ}% ルビ調整(原本通り)「所爲」のルビは(せゐ)ではなく(しよゐ)
\ruby{宜}{よろし}きを
\ruby[<j||]{失}{うしな}つても、% 行末行頭の境界付近なので特例処置を施す
%
\ruby{慮}{はか}つて、
%
\ruby{斷}{だん}じて、
%
\ruby{行}{おこな}つて、
%
\ruby{着々}{ちやく|〳〵}と
\ruby{事{\換字{情}}}{じ|じやう}の
\ruby{展開}{てん|かい}に
\ruby{應}{おう}じて
\原本頁{261-9}\改行%
\ruby{行}{ゆ}くのが、
%
\ruby{男子}{だん|し}の
\ruby{敢}{あへ}て
すべき
\ruby{{\換字{道}}}{みち}では
\ruby{無}{な}いか。
%
\ruby{{\換字{猶}}豫}{ゆう|よ}して
\ruby{决}{けつ}せざるは、
%
\ruby{軍務}{ぐん|む}では
\ruby{何}{なに}よりも
\ruby[<j>]{甚}{はなはだ}しく
\ruby{惡}{にく}むところだが、
%
\ruby{獨}{ひと}り
\ruby{軍人}{ぐん|じん}
のみが
\原本頁{261-11}\改行%
\ruby{左樣}{さ|う}
\ruby{覺悟}{かく|ご}
すべき
では
\ruby{無}{な}い、
%
\ruby{何人}{なん|びと}に
\ruby{取}{と}つても
\ruby{遲疑}{ち|ぎ}
\ruby{躊躇}{ちう|ちよ}ほど、
%
\ruby{其}{その}
\ruby{人}{ひと}を
\ruby{{\換字{害}}}{がい}する
ものは
あるまい。
%
\ruby{同}{おな}じ
\ruby{{\換字{婦}}人}{ふ|じん}に
\ruby[||j>]{愛}{あい}
\ruby[||j>]{着}{ちやく}
% \ruby{愛着}{あい|ちやく}
するなら、
%
\ruby{水野}{みづ|の}
\ruby{汝}{きさま}も
\原本頁{262-2}\改行%
\ruby{男兒}{をと|こ}では
\ruby{無}{な}いか、
%
\ruby{何故}{な|ぜ}
\ruby{男}{をとこ}
らしく
\ruby{行動}{かう|どう}
せぬ?。
%
ビスマークは
\ruby{何樣}{ど|う}して
\ruby{其}{そ}の
\ruby{妻}{つま}を
\ruby{得}{た}た?。% ルビ調整(原本通り)「得」のルビは(た)で(𛀁)に見えない
%
\ruby{烈}{はげ}しく
\ruby{思}{おも}つた、
%
\ruby{明}{あき}らかに
\ruby{求}{もと}めた、
%
\ruby{而}{そ}して
\原本頁{262-4}\改行%
\ruby{{\換字{終}}}{つひ}に
\ruby{得}{{\換字{𛀁}}}た
といふに
\ruby{{\換字{過}}}{す}ぎん
\ruby{事}{こと}
ではないか。
%
\ruby{今}{いま}は
\ruby{其}{そ}の
\ruby{夫人}{ふ|じん}も
\ruby{世}{よ}を
\ruby{去}{さ}られたが、
%
\ruby{我}{わ}が
\ruby{陸軍}{りく|ぐん}
\ruby[||j>]{大}{たい}
\ruby[||j>]{將}{しやう}の
% \ruby{大將}{たい|しやう}の
\ruby{某侯}{ぼう|こう}が、
%
\ruby{年}{とし}も
\ruby{{\換字{若}}}{わか}く
\ruby{身}{み}も
\ruby{鄙}{いやし}かつた
\ruby{時}{とき}の
\原本頁{262-6}\改行%
\ruby{戀}{こひ}の
\ruby[||j>]{物}{もの}
\ruby[||j>]{語}{がたり}は、
% \ruby{物語}{もの|がたり}は、
%
\ruby{虛實}{きよ|じつ}は
\ruby{知}{し}らぬが
\ruby{汝}{きさま}も
\ruby{知}{し}つて
\ruby{居}{ゐ}やう。
%
\ruby{徒然}{と|ぜん}を
\ruby{慰}{なぐさ}める
ばかりに
\ruby{讀}{よ}んだ
\ruby{雜書}{ざつ|しよ}に、
%
\ruby{{\換字{文}}覺}{もん|がく}の
\ruby{事}{こと}を
\ruby{記}{しる}して
あつたが、
%
\ruby{彼}{あれ}を
\ruby{見}{み}て
\原本頁{262-8}\改行%
\ruby{先夜}{せん|や}も
\ruby{汝}{きさま}の
\ruby{上}{うへ}を、
%
\ruby{自然}{おの|づ}と
\ruby{胸}{むね}に
\ruby{思}{おも}ひ
\ruby{{\換字{浮}}}{うか}めた。
%
\ruby{{\換字{文}}覺}{もん|がく}は
% 文覚(もんがく、生没年不詳)
% 平安時代末期から鎌倉時代初期にかけての武士・真言宗の僧。
% 父は左近将監茂遠(もちとお)。
% 俗名は遠藤盛遠(えんどうもりとお)。
% ここは水野のことではなく
% 同僚の源渡 (みなもとのわたる)の妻袈裟(けさ)に恋慕し、
% 誤って彼女を殺したのが動機で出家し、諸国の霊場を遍歴、修行した武士・真言宗の僧らしい。
\ruby{全}{まつた}く
\ruby{失敗}{しつ|ぱい}し、
%
\原本頁{262-9}\改行%
ビスマークや
\ruby{我}{わ}が
\ruby[||j>]{大}{たい}
\ruby[||j>]{將}{しやう}は
% \ruby{大將}{たい|しやう}は
\ruby{思}{おも}ひを
\ruby{{\換字{遂}}}{と}げたが、
%
\ruby{其}{そ}の
\ruby{遲疑}{ち|ぎ}
\ruby{躊躇}{ちう|ちよ}して
\ruby{{\換字{空}}}{あだ}に
\ruby{物}{もの}を
\ruby{思}{おも}はぬは
\ruby{同}{おな}じ
\ruby{事}{こと}だ、
%
\ruby{{\換字{飽}}}{あく}まで
\ruby{男兒}{をと|こ}らしく
\ruby{戀}{こひ}をしたのは
\ruby{同}{おな}じ
\原本頁{262-11}\改行%
\ruby{事}{こと}だ、
%
\ruby{世}{よ}の
\ruby{小說}{せう|せつ}に
あるやうに
\ruby{女々}{め|ゝ}しく% 踊り字調整「〻(二の字点、揺すり点)に見えるが(ゝ)」
\ruby{月日}{つき|ひ}を
\ruby{經}{へ}ぬ
のは
\ruby{同}{おな}じ
\ruby{事}{こと}だ
\改行% 校正作業の簡略化のため
。
%%%%%%%%%%
\原本頁{263-1}\改行%
\ruby{彼}{あ}の
\ruby{{\換字{文}}覺}{もん|がく}が
\ruby{云}{い}つた
\ruby{言}{ことば}に、
%
\ruby{戀}{こひ}には
\ruby{人}{ひと}の
\ruby{死}{し}なぬ
ものかは、
%
と
\ruby{苦}{くる}しい
\ruby{思}{おもひ}を
\ruby[||j>]{白}{はく}
\ruby[||j>]{狀}{じやう}
% \ruby{白狀}{はく|じやう}
してゐるが、
%
\ruby{水野}{みづ|の}、
%
\ruby{汝}{きさま}も
\ruby{其}{そ}の
\ruby{衰}{おとろ}へかた
\ruby{其}{そ}の
\ruby{窶}{やつ}れかたでは、
%
\ruby{成程}{なる|ほど}
\ruby{汝}{きさま}も
\ruby{死{\換字{兼}}}{しに|か}ねない
\ruby{樣子}{やう|す}だ。
%
とても
\ruby{其}{それ}
\ruby{程}{ほど}に
\ruby{{\換字{迷}}}{まよ}つた
ならば、
%
\原本頁{263-4}\改行%
\ruby{何故}{な|ぜ}
\ruby{男兒}{をと|こ}
らしく
\ruby{{\換字{進}}}{すゝ}んでは% 踊り字調整「〻(二の字点、揺すり点)に見えるが(ゝ)」
\ruby{振舞}{ふる|ま}はぬ?、
%
\ruby{默}{だま}つて
\ruby{物}{もの}を
\ruby{思}{おも}つても
\ruby{死}{し}ぬなら、
%
\ruby{何故}{な|ぜ}
\ruby{成敗}{せい|ばい}
\ruby{生死}{しやう|し}
\ruby{此}{こ}の
\ruby{一擲}{いつ|てき}と、
%
\ruby{男兒}{をと|こ}
らしく
\ruby{{\換字{運}}命}{うん|めい}の
\ruby{何}{なに}を
\ruby{與}{あた}ふる
かを
\ruby{見}{み}ぬ?。
%
\ruby{{\換字{文}}覺}{もん|がく}は
たゞ% TODO 原本の「二の字点、揺すり点」に濁点のグリフが見つからないので「ゞ」
\ruby{我慢}{が|まん}
ばかりの
\ruby{男}{をとこ}では
\ruby{無}{な}い、
%
\ruby{袈裟}{け|さ}を
\ruby{殺}{ころ}した
\ruby{其}{そ}の
\ruby{後}{あと}では、
%
\ruby{辰}{たつ}の
\ruby{刻}{こく}より
\ruby{未}{ひつじ}の
\ruby{刻}{こく}まで、
%
\ruby{四時}{よ|とき}と
\ruby{云}{い}へば
\ruby{八時間}{はち|じ|かん}だ
\改行% 校正作業の簡略化のため
、
%
\原本頁{263-8}\改行%
\ruby{其}{そ}の
\ruby{八時間}{はち|じ|かん}を
\ruby{大聲}{おほ|ごゑ}
\ruby{揚}{あ}げて、
%
\ruby{荒}{あら}くれた
\ruby{眼}{め}から
\ruby{霰}{あられ}のやうな
\ruby{涙}{なみだ}を
\ruby{落}{おと}しながら
\ruby{泣}{な}き
\ruby{{\換字{通}}}{とほ}した
とある、
%
\ruby{恐}{おそろ}しい
\ruby{{\換字{情}}}{じやう}の
\ruby{深}{ふか}い
\ruby{熱烈}{ねつ|れつ}な
\ruby{奴}{やつ}だ。
%
\ruby[||j>]{其}{その}
\ruby[||j>]{位}{くらゐ}の
% \ruby{其位}{その|くらゐ}の
\原本頁{263-10}\改行%
\ruby{奴}{やつ}が
\ruby{手荒}{て|あら}い
\ruby{事}{こと}を
するまでには、
%
\ruby{一}{ひ}ト
\ruby{{\換字{通}}}{とほ}りや
\ruby{二}{ふ}タ
\ruby{{\換字{通}}}{とほ}りで
\ruby{無}{な}く
\ruby{物}{もの}を
\ruby{思}{おも}つた
らうが、
%
\ruby{歸}{き}する
ところ
\ruby{暴}{ぼう}でも
\ruby{何}{なん}でも
\ruby{男兒}{をと|こ}
らしく
\ruby{思}{おも}ふ
まゝに% 踊り字調整「〻(二の字点、揺すり点)に見えるが(ゝ)」
\ruby{振舞}{ふる|ま}つた
のは
また
\ruby{已}{や}むを
\ruby{得}{{\換字{𛀁}}}ん。
%
とても
かくても
\ruby{物}{もの}を
\ruby{思}{おも}つて、
\原本頁{264-2}\改行%
\ruby{戀}{こひ}に
\ruby{死{\換字{兼}}}{しに|か}ね
もすまい
ならば、
%
\ruby{何故}{な|ぜ}
\ruby{男兒}{をと|こ}
らしくは
\ruby{振舞}{ふる|ま}はぬ?。
%
\ruby{當}{あた}つて
\ruby{碎}{くだ}くか
\ruby{碎}{くだ}けろかだ、
%
\ruby[||j>]{突}{とつ}
\ruby[||j>]{貫}{くわん}して
% \ruby{突貫}{とつ|くわん}して
\ruby{倒}{たふ}さるゝか% 踊り字調整「〻(二の字点、揺すり点)に見えるが(ゝ)」
\ruby{倒}{たふ}すかの
\ruby{事}{こと}だ、
%
\ruby[<j||]{首}{かうべ}% 行末行頭の境界付近なので特例処置を施す
\ruby{離}{はな}ると
% \ruby{首離}{かうべ|はな}ると
\ruby{雖}{いへど}も
\ruby{身}{み}
\ruby{懲}{こ}りず、
%
といふ
\ruby[<j>]{勢}{いきほひ}で
\ruby[||j>]{突}{とつ}
\ruby[||j>]{貫}{くわん}して
% \ruby{突貫}{とつ|くわん}して
\ruby{仕舞}{し|ま}へ。
%
\ruby{汝}{きさま}が
\ruby{良}{い}い
\ruby{{\換字{婦}}人}{ふ|じん}を
\ruby{得}{{\換字{𛀁}}}て
\ruby[||j>]{大}{たい}
\ruby[||j>]{將}{しやう}になるか、
% \ruby{大將}{たい|しやう}になるか、
%
たゞし% TODO 原本の「二の字点、揺すり点」に濁点のグリフが見つからないので「ゞ」
\ruby{{\換字{文}}覺}{もん|がく}の
やうな
\ruby[|j|]{狂}{きちがひ}
\ruby{僧}{ばうず}に
なるか、
%
\原本頁{264-6}\改行%
それは
\ruby{何方}{どち|ら}に
なつても
\ruby{乃公}{お|れ}は
\ruby{關}{かま}はんが、
%
\ruby{何樣}{ど|う}せ
\ruby{汝}{きさま}は
\ruby{欲}{よく}が
\ruby{薄}{うす}くて
\ruby{高慢}{かう|まん}が
\ruby{{\換字{強}}}{つよ}い、
%
\ruby{變挺}{へん|てこ}な
\ruby{男}{をとこ}に
\ruby{生}{うま}れて
\ruby{居}{ゐ}るのだから、
%
\ruby{坊主}{ばう|ず}に
なつて
\ruby{仕舞}{し|ま}ふ
のも
\ruby[||j>]{寧}{いつそ}
\ruby[||j>]{宜}{ よか}らう、
%
\ruby{日方}{ひ|かた}は
\ruby{{\換字{貧}}乏}{びん|ばう}でも
\ruby{汝}{きさま}が
\ruby{左樣}{さ|う}
なつたら、
%
\ruby{{\換字{麻}}}{あさ}の
\ruby[<j||]{衣}{ころも}
\ruby[||j>]{位}{ぐらゐ}は
\ruby{寄{\換字{進}}}{き|しん}して
\ruby{立}{たて}
\ruby{{\換字{過}}}{すご}して
\ruby{{\換字{遣}}}{や}る!。
%
\ruby{汝}{きさま}が
\ruby{衰}{おとろ}へに
\ruby{衰}{おとろ}へて、
%
\ruby{一}{いち}
\ruby{少女}{せう|ぢよ}にも
\原本頁{264-10}\改行%
\ruby{其}{そ}の
\ruby{勇氣}{ゆう|き}が
\ruby{及}{およ}ばん
やうに
なつて
\ruby{戀}{こひ}に
\ruby{死}{し}ぬのを、
%
\ruby{見殺}{み|ごろ}し
にするのは
\ruby{乃公}{お|れ}には
\ruby{出來}{で|き}ぬ。
%
\ruby{男兒}{をと|こ}
らしく
\ruby{振舞}{ふる|ま}へ、
%
\ruby{女}{をんな}
では
あるまい。
%
\ruby{高}{たか}が
\原本頁{265-1}\改行%
\ruby{一}{いつ}
\ruby{{\換字{婦}}人}{ぷ|じん}を
\ruby{對敵}{あひ|て}にして、
%
\ruby{{\換字{遠}}距離}{ゑん|きよ|り}で
\ruby{彈藥}{だん|やく}を
\ruby{使}{つか}ひ
\ruby{盡}{つく}すのは
\ruby{愚}{おろか}な
\ruby{事}{こと}だ。
%
いつそ
\ruby{一}{ひ}と
\ruby{思}{おもひ}に
\ruby[||j>]{突}{とつ}
\ruby[||j>]{貫}{くわん}して
% \ruby{突貫}{とつ|くわん}して
\ruby{仕舞}{し|ま}へ。
%
\ruby{{\換字{勝}}}{か}つか
\ruby{負}{ま}けるかの
\ruby{他}{ほか}には
\ruby{物}{もの}は
\ruby{有}{あ}りは
\ruby{仕無}{し|な}い。
%
\ruby{{\換字{遠}}地}{とほ|く}から
\ruby{敵}{てき}に
\ruby{{\換字{勝}}}{か}たう
といふのは
\ruby{贅澤}{ぜい|たく}な
\ruby{詮義}{せん|ぎ}だ。
%
\原本頁{265-4}\改行%
\ruby{羽{\換字{勝}}}{は|がち}も
\ruby{乃公}{お|れ}の
\ruby{言}{い}ふことを
\ruby{無理}{む|り}とは
\ruby{思}{おも}ふまい、
%
\ruby{何樣}{ど|う}だ
\ruby{水野}{みづ|の}
\ruby{汝}{きさま}は
\ruby{何}{なん}と
\ruby{思}{おも}ふ?。
%
\ruby{女々}{め|ゝ}しい% 踊り字調整「〻(二の字点、揺すり点)に見えるが(ゝ)」
\ruby{事}{こと}は
\ruby{宜}{い}い
\ruby{加減}{か|げん}に
\ruby{止}{や}めろ。
%
もう
\ruby{乃公}{お|れ}
は
\ruby{此}{これ}
\ruby{限}{き}り
\ruby{物}{もの}は
\ruby{言}{い}はぬ、
%
これだけ
\ruby{言}{い}つても
\ruby{乃公}{お|れ}の
\ruby{云}{い}ふ
\ruby{事}{こと}を
\ruby{用}{もち}ゐん
ならば、
%
\ruby{舊}{もと}の
\ruby{水野}{みづ|の}に
なり
\ruby{{\換字{返}}}{かへ}る
までは、
%
\ruby{汝}{きさま}には
\ruby{會}{あ}はん。
』
