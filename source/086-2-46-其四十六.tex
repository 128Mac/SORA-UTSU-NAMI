\Entry{其四十六}

\原本頁{}%
\ruby[g]{日方}{ひかた}の
\ruby{言}{い}ふところも
\ruby{無理}{む|り}ばかりにはあらず、
%
\ruby{思}{おも}ふて
\ruby{言}{い}はざる
\ruby{苦}{くるし}さに
\ruby{堪}{た}へかねては、
%
\ruby{兎}{と}せん
\ruby{角}{かく}せんと
\ruby{意}{こゝろ}を
\ruby{動}{うご}かしたる
\ruby{折}{をり}も
\ruby{無}{な}きにあらねど、
%
おのづからに
\ruby{思}{おも}ひ
\ruby{切}{き}つたる
\ruby{事}{こと}を
\ruby{何故}{なに|ゆゑ}とも
\ruby{無}{な}く
\ruby{做}{な}し
\ruby{出}{いだ}しかねて、
%
\ruby{女々}{め|ゝ}しと
\ruby{云}{い}はゞ
\ruby{女々}{め|ゝ}しと
\ruby{云}{い}はるべく
\ruby{今日}{け|ふ}までは
\ruby{{\換字{過}}}{すご}せるなり。
%
されど
\ruby{差當}{さし|あた}つて
\ruby{今}{いま}
\ruby[g]{日方}{ひかた}に
\ruby{對}{むか}つて、
%
\ruby{其}{そ}の
\ruby{言葉}{こと|ば}に
\ruby{從}{したが}ふべし
\ruby{意見}{い|けん}に
\ruby{就}{つ}くべしとも
\ruby{云}{い}ひかねて、
%
\ruby[g]{水野}{みづの}は
\ruby{何}{なん}とも
\ruby{言}{ものい}はねば、
%
\ruby[g]{羽{\換字{勝}}}{はがち}は
\ruby{徐々}{おも|むろ}に
\ruby{口}{くち}を
\ruby{開}{ひら}きて、
%
\ruby{言葉}{こと|ば}づかひも
\ruby{重々}{おも|〳〵}しく、

\原本頁{}%
『
\ruby[g]{水野}{みづの}、
%
\ruby{默}{もく}して
\ruby{仕舞}{し|ま}つてはいかん。
%
\ruby[g]{日方}{ひかた}の
\ruby{言}{ことば}は
\ruby{或}{あるひ}は
\ruby{不當}{ふ|たう}だ、
%
しかし
\ruby[g]{日方}{ひかた}の
\ruby{意}{こゝろ}は
\ruby{親切}{しん|せつ}に
\ruby{他}{ほか}ならんのだ。
%
\ruby{其言}{その|ことば}を
\ruby{{\換字{採}}}{と}ると
\ruby{{\換字{採}}}{と}らんとは
\ruby{別}{べつ}として、
%
\ruby{其親切}{その|しん|せつ}は
\ruby{十{\換字{分}}}{じう|ぶん}に
\ruby{受}{う}け
\ruby{納}{い}れねばならん。
%
\ruby{無論}{む|ろん}
\ruby{君}{きみ}は
\ruby[g]{日方}{ひかた}の
\ruby{好意}{かう|い}に
\ruby{對}{たい}して
\ruby{感謝}{かん|しや}して
\ruby{居}{を}るだらうナ。
』

\原本頁{}%
と
\ruby{優}{やさし}しく
\ruby[g]{水野}{みづの}を
\ruby{誘}{いざな}ひて
\ruby{言}{ものい}はせんとすれど、
%
\ruby[g]{水野}{みづの}はたゞ
\ruby{僞}{いつはり}ならぬ
\ruby{眼色}{め|つき}して
\ruby{打點頭}{うち|うな|づ}きて、
%
\ruby{然}{しか}り、
%
と
\ruby{答}{こた}へたるばかりなり。

\原本頁{}%
『
\ruby{人}{ひと}は
\ruby{人各々}{ひと|おの|〳〵}の
\ruby{性質}{せい|しつ}がある、
%
\ruby{境{\換字{遇}}}{きやう|ぐう}がある。
%
\ruby{深}{ふか}く
\ruby{他人}{た|にん}の
\ruby{事}{こと}に
\ruby{立入}{たち|い}るのは
\ruby{僕}{ぼく}は
\ruby{取}{と}らん。
%
\ruby[g]{日方}{ひかた}の
\ruby{親切}{しん|せつ}は
\ruby{僕}{ぼく}も
\ruby{有}{も}つて
\ruby{居}{ゐ}る。
%
たゞし
\ruby[g]{日方}{ひかた}の
\ruby{如}{ごと}く
\ruby{自{\換字{分}}}{じ|ぶん}の
\ruby{意思}{い|し}
\ruby{感{\換字{情}}}{かん|じやう}を、
%
\ruby{君}{きみ}の
\ruby{上}{うへ}に
\ruby{押}{お}し
\ruby{被}{かぶ}せやうとは
\ruby{僕}{ぼく}は
\ruby{能}{よく}せん。
%
\ruby[g]{水野}{みづの}!
\ruby{歸}{かへ}つて
\ruby{來}{き}てから
\ruby{君}{きみ}の
\ruby{{\換字{評}}{\換字{判}}}{ひやう|ばん}を
いろ〳〵
\ruby{聞}{き}いた。
%
\ruby{僕}{ぼく}は
\ruby{考}{かんが}へた。
%
\ruby{考慮}{かん|がへ}を
\ruby{錬}{ね}つた。
%
\ruby{而}{そ}して
\ruby{君}{きみ}に
\ruby{對}{たい}して
\ruby{贈}{おく}るべき
\ruby{或物}{ある|もの}を
\ruby{得}{{\換字{𛀁}}}た。
%
しかし
\ruby{今}{いま}の
\ruby{君}{きみ}に
\ruby{對}{たい}して
\ruby{何}{なに}を
\ruby{贈}{おく}つても
\ruby{無益}{む|{\換字{𛀁}}き}に
\ruby{{\換字{終}}}{をは}るべきを
\ruby{知}{し}つた。
%
よつて
\ruby{君}{きみ}に
\ruby{對}{たい}して
\ruby{何}{なに}をも
\ruby{言}{い}ふまいと
\ruby{思}{おも}つた。
%
しかし
\ruby{今}{いま}
\ruby[g]{日方}{ひかた}の
\ruby{言}{い}つたところは
\ruby{不幸}{ふ|かう}にして、
%
\ruby{僕}{ぼく}が
\ruby{考}{かんが}へて
\ruby{云}{い}はうと
\ruby{思}{おも}つたところと
\ruby{正反對}{せい|はん|たい}の
\ruby{言}{げん}であるので、
%
\ruby{已}{や}むを
\ruby{得}{{\換字{𛀁}}}ず
\ruby{誘}{さそ}ひ
\ruby{出}{だ}されて
\ruby{一言}{ひと|こと}いふ。
%
\ruby[g]{日方}{ひかた}の
\ruby{言}{げん}を
\ruby{駁}{ばく}するのでは
\ruby{無}{な}い。
%
もとより
\ruby{僕}{ぼく}が
\ruby{言}{い}はんと
\ruby{欲}{ほつ}して
\ruby{居}{ゐ}たところなのだ。
%
\ruby[g]{水野}{みづの}、
%
\ruby{君}{きみ}は
\ruby{聰明}{そう|めい}の
\ruby{人}{ひと}だ、
%
\ruby{僕等}{ぼく|ら}は
\ruby{及}{およ}ばん。
%
たゞ、
%
\原本頁{268-1}%
\ruby{此}{こ}の
\ruby{世}{よ}の
\ruby{中}{なか}に
\ruby{立{\換字{交}}}{たち|まじ}つて、
%
\ruby{人}{ひと}に
\ruby{接}{せつ}し
\ruby{事}{こと}に
\ruby{應}{おう}ずるに
\ruby{於}{おい}ては
\ruby{齡}{とし}の
\ruby{多}{おほ}いだけに、
%
\ruby{僕}{ぼく}は
\ruby{私}{ひそか}に
\ruby{思}{おも}ふに
\ruby{君}{きみ}に
\ruby{對}{たい}しても、
%
\ruby{必}{かなら}ず
\ruby{一日}{いち|じつ}の
\ruby{長}{ちやう}があると
\ruby{信}{しん}ずる。
%
\ruby{僕}{ぼく}は
\ruby{書}{しよ}を
\ruby{讀}{よ}んで
\ruby{理}{り}を
\ruby{{\換字{尋}}}{たづ}ねたで
\ruby{無}{な}い、
%
\ruby{事}{こと}に
\ruby{當}{あた}つて
\ruby{自}{みづか}ら
\ruby{知}{し}つたのだ。
%
\ruby{僕}{ぼく}は
\ruby{人}{ひと}に
\ruby{使}{つか}はれた。
%
\ruby{人}{ひと}を
\ruby{使}{つか}つた。
%
\ruby{而}{そ}して
\ruby{人}{ひと}と
\ruby{人}{ひと}との
\ruby{間}{あひだ}の
\ruby{感{\換字{情}}}{かん|じやう}といふものが、
%
\ruby{如何}{い|か}に
\ruby{大切}{たい|せつ}なものであるかといふこを
\ruby{身}{み}に
\ruby{染}{し}みて
\ruby{覺}{おぼ}えた。
%
\ruby{而}{そ}して
\ruby{我}{わ}が
\ruby{感{\換字{情}}}{かん|じやう}に
\ruby{任}{まか}すことの
\ruby{危{\換字{害}}}{き|がい}を
\ruby{實驗}{じつ|けん}した。
%
\ruby{僕}{ぼく}は
\ruby{愚}{ぐ}であつたから
\ruby{同}{おな}じ
\ruby{{\換字{過}}失}{あや|まち}を
\ruby{二度}{ふた|ゝび}した。
%
\ruby{三度}{み|たび}した。
%
\ruby{四度}{よ|たび}した
\ruby{五度}{ご|たび}した。
%
\ruby{幾十度}{いく|じう|ど}と
\ruby{無}{な}く
\ruby{實驗}{じつ|けん}した。
%
\ruby{而}{そ}して
\ruby{後纔}{のち|わづか}に
\ruby{我}{わ}が
\ruby{感{\換字{情}}}{かん|じやう}を
\ruby{調御}{てう|ぎよ}することの
\ruby{如何}{い|か}に
\ruby{大切}{たい|せつ}なものであるかといふ
\ruby{事}{こと}を
\ruby{知}{し}つた。
%
\ruby{罵}{のゝし}らるれば
\ruby{怒}{いか}る、
%
\ruby{氣}{き}に
\ruby{入}{い}れば
\ruby{愛}{あい}する。
%
それは
\ruby{欺}{あざむ}かぬ
\ruby{感{\換字{情}}}{かん|じやう}である。
%
\ruby{其}{そ}の
\ruby{感{\換字{情}}}{かん|じやう}に
\ruby{任}{まか}せて
\ruby{喜怒}{き|ど}するを
\ruby{天眞爛{\換字{熳}}}{てん|しん|らん|まん}だなんぞといふ。
%
\ruby{一{\換字{船}}}{いつ|せん}の
\ruby{中}{うち}で
\ruby{事端}{じ|たん}を
\ruby{生}{しやう}ずるのは、
%
\ruby{何時}{い|つ}でも
\ruby{天眞爛{\換字{熳}}}{てん|しん|らん|まん}の
\ruby{人}{ひと}だ。
%
\ruby{怒}{いか}るには
\ruby{怒}{いか}る
\ruby{理由}{わ|け}がある。
%
\ruby{愛}{あい}するには
\ruby{愛}{あい}する
\ruby{理由}{わ|け}がある。
%
しかし
\ruby{感{\換字{情}}}{かん|じやう}ばかりが
\ruby{最上}{さい|じやう}なものでは
\ruby{無}{な}い。
%
\ruby{感{\換字{情}}}{かん|じやう}に
\ruby{任}{まか}すのを
\ruby{是}{ぜ}とする
\ruby{人}{ひと}は
\ruby{{\換字{船}}員}{せん|いん}の
\ruby{中}{うち}の
\ruby{最}{もつと}も
\ruby{危險}{き|けん}な
\ruby{人}{ひと}だ。
%
\ruby{自{\換字{分}}}{じ|ぶん}の
\ruby{感{\換字{情}}}{かん|じやう}を
\ruby{調御}{てう|ぎよ}しなければ、
%
\ruby{自{\換字{分}}}{じ|ぶん}は
\ruby{人}{ひと}に
\ruby{使}{つか}はれることが
\ruby{出來}{で|き}ぬ。
%
\ruby{自{\換字{分}}}{じ|ぶん}の
\ruby{感{\換字{情}}}{かん|じやう}を
\ruby{調御}{てう|ぎよ}しなければ
\ruby{自{\換字{分}}}{じ|ぶん}は
\ruby{人}{ひと}を
\ruby{使}{つか}ふことが
\ruby{出來}{で|き}ぬ。
%
\ruby{自{\換字{分}}}{じ|ぶん}の
\ruby{感{\換字{情}}}{かん|じやう}を
\ruby{調御}{てう|ぎよ}しなければ、
%
\ruby{自{\換字{分}}}{じ|ぶん}は
\ruby{人}{ひと}に
\ruby{{\換字{交}}}{まじは}ることが
\ruby{出來}{で|き}ぬ。
%
\ruby{人}{ひと}に
\ruby{使}{つか}はれず、
%
\ruby{人}{ひと}を
\ruby{使}{つか}はず、
%
\ruby{人}{ひと}に
\ruby{{\換字{交}}}{まじは}らずに
\ruby{濟}{す}む
\ruby{世間}{せ|けん}は
\ruby{無}{な}い。
%
\ruby{僕}{ぼく}は
\ruby{僕}{ぼく}だけの
\ruby{小}{ちひさ}な
\ruby{經驗}{けい|けん}だが、
%
しかし
\ruby{確實堅固}{くわく|じつ|けん|ご}な
\ruby{經驗}{けい|けん}から、
%
\ruby{非常}{ひ|じやう}に
\ruby{{\換字{強}}}{つよ}く
\ruby{深}{ふか}く
\ruby{感{\換字{情}}}{かん|じやう}の
\ruby{調御}{てう|ぎよ}が
\ruby{人世}{じん|せい}の
\ruby{最大}{さい|だい}
\ruby{必要}{ひつ|{\換字{𛀁}}う}のものであるといふことを
\ruby{確信}{くわく|しん}して
\ruby{居}{ゐ}る。
%
\ruby{君}{きみ}は
\ruby{聰明絶倫}{そう|めい|ぜつ|りん}な
\ruby{人}{ひと}だが、
%
\ruby{此}{こ}の
\ruby{點}{てん}の
\ruby{經驗}{けい|けん}は
\ruby{或}{あるひ}は
\ruby{薄}{うす}からう。
%
\ruby{戀愛}{れん|あい}も
\ruby{是非}{ぜ|ひ}がない。
%
\ruby{苦悶}{く|もん}も
\ruby{已}{や}むを
\ruby{得}{{\換字{𛀁}}}ぬ。
%
\ruby{一切}{いつ|さい}の
\ruby{事}{こと}は
\ruby{謝}{しや}せんとして
\ruby{謝}{しや}せぬが
\ruby{天命}{てん|めい}だ。
%
\ruby{風}{かぜ}の
\ruby{{\換字{前}}面}{ま|へ}から
\ruby{吹}{ふ}く
\ruby{日}{ひ}もある。
%
\ruby{潮流}{し|ほ}の
\ruby{横}{よこ}へと
\ruby{行}{ゆ}く
\ruby{夜}{よ}もある。
%
\ruby{颶風}{つむ|じ}も
\ruby{龍卷}{たつ|まき}も
\ruby{起}{おこ}る
\ruby{日}{ひ}は
\ruby{起}{おこ}る。
%
しかし
\ruby{其間}{その|あひだ}に
\ruby{立}{た}つて
\ruby{屹然}{きつ|ぜん}として、
%
\ruby{我}{わ}が
\ruby{正當}{せい|たう}の
\ruby{處置}{しよ|ち}を
\ruby{取}{と}つて
\ruby{行}{ゆ}けば
\ruby{死}{し}して
\ruby{餘}{あま}りあるのだ。
%
\ruby[g]{水野}{みづの}!。
%
\ruby{君}{きみ}が
\ruby{君}{きみ}の
\ruby{欺}{あざむ}かぬ
\ruby{感{\換字{情}}}{かん|じやう}のために
\ruby{死}{し}にたくば
\ruby{其迄}{それ|まで}の
\ruby{事}{こと}だ。
%
しかし
\ruby{君}{きみ}が
\ruby{君}{きみ}として
\ruby{世}{よ}に
\ruby{立}{た}たうとした
\ruby{大{\換字{丈}}夫}{だい|ぢやう|ぶ}の
\ruby{志}{こゝろざし}を
\ruby{忘}{わす}れぬ
\ruby{限}{かぎ}りは、
%
\ruby{君}{きみ}は
\ruby{君}{きみ}の
\ruby{感{\換字{情}}}{かん|じやう}を
\ruby{調御}{てう|ぎよ}することを
\ruby{忘}{わす}れてはならぬ。
%
\ruby{必}{かなら}ず
\ruby{感{\換字{情}}}{かん|じやう}の
\ruby{調御}{てう|ぎよ}といふことを
\ruby{忘}{わす}れずに
\ruby{居}{ゐ}て
\ruby{欲}{ほし}しい。
%
\ruby{君}{きみ}が
\ruby{{\換字{文}}覺}{もん|がく}の
\ruby{如}{ごと}き
\ruby{人}{ひと}とならんことは、
%
\ruby{僕}{ぼく}の
\ruby{最}{もつと}も
\ruby{恐}{おそ}れて
\ruby{居}{ゐ}るところだ。
%
\ruby{{\換字{文}}覺}{もん|がく}の
\ruby{如}{ごと}きは
\ruby{僕}{ぼく}の
\ruby{蛇蝎視}{だ|かつ|し}する
\ruby{人}{ひと}だ。
%
しかし
\ruby{僕}{ぼく}と
\ruby[g]{日方}{ひかた}とは
\ruby{言}{ことば}は
\ruby{異}{こと}にして
\ruby{意}{こゝろ}は
\ruby{同}{おな}じだ。
%
たまたま
\ruby[g]{日方}{ひかた}の
\ruby{言}{ことば}に
\ruby{僕}{ぼく}の
\ruby{胸裏}{む|ね}に
\ruby{觸}{ふ}れたところが
\ruby{一寸}{ちよ|つと}あつたので、
%
\ruby{言}{い}はずとものことを
\ruby{饒舌}{しや|べ}つたが、
%
\ruby{二人}{ふた|り}の
\ruby{言}{ことば}の
\ruby{異}{ことな}るところを
\ruby{忘}{わす}れて、
%
\ruby{其}{そ}の
\ruby{意}{こゝろ}の
\ruby{同}{おな}じところをさへ
\ruby{取}{と}つて
\ruby{吳}{く}れゝば、
%
\ruby[g]{日方}{ひかた}も
\ruby{僕}{ぼく}も
\ruby{何程}{どれ|ほど}
\ruby{悅}{よろこ}ばう!。
』
