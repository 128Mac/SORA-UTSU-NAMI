\Entry{其四十六}

% メモ 校正終了 2024-05-08 2024-06-05
\原本頁{265-9}%
\ruby[g]{日方}{ひ かた}の
\ruby{言}{い}ふ
ところも
\ruby[g]{無理}{む り }
ばかりには
あらず、
%
\ruby{思}{おも}ふて
\ruby{言}{い}はざる
\ruby[<j||]{苦}{くるし}さに
\ruby{堪}{た}へ
かねては、
%
\ruby{兎}{と}せん
\ruby{角}{かく}せんと
\ruby{意}{こゝろ}を% 踊り字調整「〻(二の字点、揺すり点)に見えるが(ゝ)」
\ruby{動}{うご}かしたる
\ruby{折}{をり}も
\ruby{無}{な}きに
\原本頁{266-1}\改行%
あらねど、
%
おのづ
からに
\ruby{思}{おも}ひ
\ruby{切}{き}つたる
\ruby{事}{こと}を
\ruby[g]{何故}{なにゆゑ}
とも
\ruby{無}{な}く
\ruby{做}{な}し
\ruby{出}{いだ}し
かねて、
%
\ruby[g]{女々}{め ゝ }しと% 踊り字調整「〻(二の字点、揺すり点)に見えるが(ゝ)」
\ruby{云}{い}はゞ% 踊り字調整「〻(二の字点、揺すり点)に濁点に見えるが(ゞ)」
\ruby[g]{女々}{め ゝ }しと% 踊り字調整「〻(二の字点、揺すり点)に見えるが(ゝ)」
\ruby{云}{い}はるべく
\ruby[g]{今日}{け ふ }
までは
\ruby{{\換字{過}}}{すご}せる
なり。
%
されど
\ruby[g]{差當}{さしあた}つて
\ruby{今}{いま}
\ruby[g]{日方}{ひ かた}に
\ruby{對}{むか}つて、
%
\ruby{其}{そ}の
\ruby[g]{言葉}{ことば }に
\ruby{從}{したが}ふべし
\原本頁{266-4}\改行%
\ruby[g]{意見}{い けん}に
\ruby{就}{つ}く
べしとも
\ruby{云}{い}ひ
かねて、
%
\ruby[g]{水野}{みづの }は
\ruby{何}{なん}とも
\ruby{言}{ものい}はねば、
%
\ruby[g]{羽{\換字{勝}}}{は がち}は
\ruby[g]{徐々}{おもむろ}に
\ruby{口}{くち}を
\ruby{開}{ひら}きて、
%
\ruby[g]{言葉}{ことば }づかひも
\ruby[g]{重々}{おも〳〵}しく、

\原本頁{266-6}%
『
\ruby[g]{水野}{みづの }、
%
\ruby{默}{もく}して
\ruby[g]{仕舞}{し ま }つては
いかん。
%
\ruby[g]{日方}{ひ かた}の
\ruby{言}{ことば}は
\ruby{或}{あるひ}は
\ruby[g]{不當}{ふ たう}だ、
%
しかし
\ruby[g]{日方}{ひ かた}の
\ruby{意}{こゝろ}は% 踊り字調整「〻(二の字点、揺すり点)に見えるが(ゝ)」
\ruby[g]{親切}{しんせつ}に
\ruby{他}{ほか}ならん
のだ。
%
\ruby[||j>]{其}{その}
\ruby[||j>]{言}{ことば}を
% \ruby{其言}{その|ことば}を
\ruby{{\換字{採}}}{と}ると
\ruby{{\換字{採}}}{と}らんとは
\ruby{別}{べつ}として、
%
\ruby{其}{その}
\ruby[g]{親切}{しんせつ}は
\ruby[g]{十{\換字{分}}}{じふぶん}に
\ruby{受}{う}け
\ruby{納}{い}れねば
ならん。
%
\ruby[g]{無論}{む ろん}
\ruby{君}{きみ}は
\ruby[g]{日方}{ひ かた}の
\ruby[g]{好意}{かうい }に
\ruby{對}{たい}して
\ruby[g]{{\換字{感}}謝}{かんしや}して
\ruby{居}{を}る
だらうナ。
』

\原本頁{266-10}%
と
\ruby{優}{やさ}しく
\ruby[g]{水野}{みづの }を
\ruby{誘}{いざな}ひて
\ruby{言}{ものい}はせん
とすれど、
%
\ruby[g]{水野}{みづの }は
たゞ% 踊り字調整「〻(二の字点、揺すり点)に濁点に見えるが(ゞ)」
\ruby[<j>]{僞}{いつはり}ならぬ
\ruby[g]{眼色}{め いろ}して
\ruby{打}{うち}
\ruby[g]{點頭}{うなづ }きて、
%
\ruby{然}{しか}り、
%
と
\ruby{答}{こた}へたる
ばかり
なり。

\原本頁{267-1}%
『
\ruby{人}{ひと}は
\ruby{人}{ひと}
\ruby[g]{各々}{おの〳〵}の
\ruby[g]{性質}{せいしつ}がある、
%
\ruby[||j>]{境}{きやう}
\ruby[||j>]{{\換字{遇}}}{ ぐう}がある。
% \ruby{境{\換字{遇}}}{きやう|ぐう}がある。
%
\ruby{深}{ふか}く
\ruby[g]{他人}{た にん}の
\ruby{事}{こと}に
\ruby{立}{たち}
\ruby{入}{い}るのは
\ruby{僕}{ぼく}は
\ruby{取}{と}らん。
%
\ruby[g]{日方}{ひ かた}の
\ruby[g]{親切}{しんせつ}は
\ruby{僕}{ぼく}も
\ruby{有}{も}つて
\ruby{居}{ゐ}る。
%
たゞし% 踊り字調整「〻(二の字点、揺すり点)に濁点に見えるが(ゞ)」
\ruby[g]{日方}{ひ かた}の
\ruby{如}{ごと}く
\ruby[g]{自{\換字{分}}}{じ ぶん}の
\ruby[g]{意思}{い し }
\ruby[||j>]{{\換字{感}}}{かん}
\ruby[||j>]{{\換字{情}}}{じやう}を、
% \ruby{{\換字{感}}{\換字{情}}}{かん|じやう}を、
%
\ruby{君}{きみ}の
\ruby{上}{うへ}に
\ruby{押}{お}し
\ruby{被}{かぶ}せやう
とは
\ruby{僕}{ぼく}は
\ruby{能}{よく}せん。
%
\ruby[g]{水野}{みづの }!\inhibitglue{}%
\ruby{歸}{かへ}つて
\ruby{來}{き}てから
\ruby{君}{きみ}の
\ruby[||j>]{{\換字{評}}}{ひやう}
\ruby[||j>]{{\換字{判}}}{ ばん}を
% \ruby{{\換字{評}}{\換字{判}}}{ひやう|ばん}を
いろ〳〵
\ruby{聞}{き}いた。
%
\ruby{僕}{ぼく}は
\ruby[<j||]{考}{かんが}へた。% 行末行頭の境界付近なので特例処置を施す
%
\ruby[g]{考慮}{かんがへ}を
\ruby{錬}{ね}つた。
%
\ruby{而}{そ}して
\ruby{君}{きみ}に
\ruby{對}{たい}して
\ruby{贈}{おく}るべき
\ruby[g]{或物}{あるもの}を
\ruby{得}{{\換字{𛀁}}}た。
%
\原本頁{267-6}\改行%
しかし
\ruby{今}{いま}の
\ruby{君}{きみ}に
\ruby{對}{たい}して
\ruby{何}{なに}を
\ruby{贈}{おく}つても
\ruby[g]{無益}{む{\換字{𛀁}}き}に
\ruby{{\換字{終}}}{をは}る
べきを
\ruby{知}{し}つた。
%
よつて
\ruby{君}{きみ}に
\ruby{對}{たい}して
\ruby{何}{なに}をも
\ruby{言}{い}ふまいと
\ruby{思}{おも}つた。
%
しかし
\ruby{今}{いま}
\ruby[g]{日方}{ひ かた}の
\ruby{言}{い}つた
ところは
\ruby[g]{不幸}{ふ かう}に
して、
%
\ruby{僕}{ぼく}が
\ruby{考}{かんが}へて
\ruby{云}{い}はうと
\ruby{思}{おも}つた
ところと
\原本頁{267-9}\改行%
\ruby{正反對}{せい|はん|たい}の
\ruby{言}{げん}であるので、
%
\ruby{已}{や}むを
\ruby{得}{{\換字{𛀁}}}ず
\ruby{誘}{さそ}ひ
\ruby{出}{だ}されて
\ruby[g]{一言}{いちごん}
いふ。
%
\ruby[g]{日方}{ひ かた}の
\ruby{言}{げん}を
\ruby{駁}{ばく}するのでは
\ruby{無}{な}い。
%
もとより
\ruby{僕}{ぼく}が
\ruby{言}{い}はんと
\ruby{欲}{ほつ}して
\ruby{居}{ゐ}た
ところ
なのだ。
%
\ruby[g]{水野}{みづの }、
%
\ruby{君}{きみ}は
\ruby[g]{聰明}{そうめい}の
\ruby{人}{ひと}だ、
%
\ruby[g]{僕等}{ぼくら }は
\ruby{及}{およ}ばん。
%
たゞ、% 踊り字調整「〻(二の字点、揺すり点)に濁点に見えるが(ゞ)」
%
\原本頁{268-1}\改行%
\ruby{此}{こ}の
\ruby{世}{よ}の
\ruby{中}{なか}に
\ruby{立}{たち}
\ruby{{\換字{交}}}{まじ}つて、
%
\ruby{人}{ひと}に
\ruby{接}{せつ}し
\ruby{事}{こと}に
\ruby{應}{おう}ずるに
\ruby{於}{おい}ては
\ruby{齡}{とし}の
\ruby{多}{おほ}い
だけに、
%
\ruby{僕}{ぼく}は
\ruby{私}{ひそか}に
\ruby{思}{おも}ふに
\ruby{君}{きみ}に
\ruby{對}{たい}しても、
%
\ruby{必}{かなら}ず
\ruby[g]{一日}{いちじつ}の
\ruby{長}{ちやう}が
あると
\原本頁{268-3}\改行%
\ruby{信}{しん}ずる。
%
\ruby{僕}{ぼく}は
\ruby{書}{しよ}を
\ruby{讀}{よ}んで
\ruby{理}{り}を
\ruby{{\換字{尋}}}{たづ}ねた
で
\ruby{無}{な}い、
%
\ruby{事}{こと}に
\ruby{當}{あた}つて
\ruby{自}{みづか}ら
\ruby{知}{し}つたのだ。
%
\ruby{僕}{ぼく}は
\ruby{人}{ひと}に
\ruby{使}{つか}はれた。
%
\ruby{人}{ひと}を
\ruby{使}{つか}つた。
%
\ruby{而}{そ}して
\ruby{人}{ひと}と
\ruby{人}{ひと}との
\ruby{間}{あひだ}の
\ruby[||j>]{{\換字{感}}}{かん}
\ruby[||j>]{{\換字{情}}}{じやう}
% \ruby{{\換字{感}}{\換字{情}}}{かん|じやう}
といふものが、
%
\ruby[g]{如何}{い か }に
\ruby[g]{大切}{たいせつ}な
もので
あるか
といふ
ことを
\ruby{身}{み}に
\ruby{染}{し}みて
\ruby{覺}{おぼ}えた。
%
\ruby{而}{そ}して
\ruby{我}{わ}が
\ruby[||j>]{{\換字{感}}}{かん}
\ruby[||j>]{{\換字{情}}}{じやう}
% \ruby{{\換字{感}}{\換字{情}}}{かん|じやう}
に
\ruby{任}{まか}す
ことの
\ruby[g]{危{\換字{害}}}{き がい}を
\ruby[g]{實驗}{じつけん}した。
%
\ruby{僕}{ぼく}は
\ruby{愚}{ぐ}で
あつたから
\ruby{同}{おな}じ
\ruby[g]{{\換字{過}}失}{あやまち}を
\ruby[g]{二度}{ふたゝび}した。% 踊り字調整「〻(二の字点、揺すり点)に見えるが(ゝ)」
%
\ruby[g]{三度}{み たび}した。
%
\ruby[g]{四度}{よ たび}した
\ruby[g]{五度}{ご たび}した。
%
\ruby{幾十度}{いく|じふ|たび}と
\ruby{無}{な}く
\ruby[g]{實驗}{じつけん}した。
%
\ruby{而}{そ}して
\ruby[||j>]{後}{のち}
\ruby[||j>]{纔}{わづか}に
% \ruby{後纔}{のち|わづか}に
\ruby{我}{わ}が
\ruby[||j>]{{\換字{感}}}{かん}
\ruby[||j>]{{\換字{情}}}{じやう}
% \ruby{{\換字{感}}{\換字{情}}}{かん|じやう}
を
\ruby[g]{調御}{てうぎよ}する
ことの
\ruby[g]{如何}{い か }に
\ruby[g]{大切}{たいせつ}な
もので
あるか
といふ
\ruby{事}{こと}を
\ruby{知}{し}つた。
%
\ruby{罵}{のゝし}らるれば% 踊り字調整「〻(二の字点、揺すり点)に見えるが(ゝ)」
\ruby{怒}{いか}る、
%
\ruby{氣}{き}に
\ruby{入}{い}れば
\ruby{愛}{あい}する。
%
それは
\ruby{欺}{あざむ}かぬ
\ruby[||j>]{{\換字{感}}}{かん}
\ruby[||j>]{{\換字{情}}}{じやう}
% \ruby{{\換字{感}}{\換字{情}}}{かん|じやう}
である。
%
\ruby{其}{そ}の
\ruby[||j>]{{\換字{感}}}{かん}
\ruby[||j>]{{\換字{情}}}{じやう}
% \ruby{{\換字{感}}{\換字{情}}}{かん|じやう}
に
\ruby{任}{まか}せて
\ruby[g]{喜怒}{き ど }するを
\ruby{天眞爛{\換字{熳}}}{てん|しん|らん|まん}
だ
なんぞ
といふ。
%
\原本頁{269-1}\改行%
\ruby[g]{一{\換字{船}}}{いつせん}の
\ruby{中}{うち}で
\ruby[g]{事端}{じ たん}を
\ruby{生}{しやう}ずるのは、
%
\ruby[g]{何時}{い つ }でも
\ruby{天眞爛{\換字{熳}}}{てん|しん|らん|まん}
の
\ruby{人}{ひと}だ。
%
\ruby{怒}{いか}るには
\ruby{怒}{いか}る
\ruby[g]{理由}{わ け }がある。
%
\ruby{愛}{あい}するには
\ruby{愛}{あい}する
\ruby[g]{理由}{わ け }がある。
%
しかし
\ruby[||j>]{{\換字{感}}}{かん}
\ruby[||j>]{{\換字{情}}}{じやう}
% \ruby{{\換字{感}}{\換字{情}}}{かん|じやう}
ばかりが
\ruby[||j>]{最}{さい}
\ruby[||j>]{上}{じやう}
% \ruby{最上}{さい|じやう}
なものでは
\ruby{無}{な}い。
%
\ruby[||j>]{{\換字{感}}}{かん}
\ruby[||j>]{{\換字{情}}}{じやう}
% \ruby{{\換字{感}}{\換字{情}}}{かん|じやう}
に
\ruby{任}{まか}すのを
\ruby{是}{ぜ}とする
\ruby{人}{ひと}は
\改行% 校正作業の簡略化のため
、
%
\原本頁{269-4}\改行%
\ruby[g]{{\換字{船}}員}{せんゐん}の
\ruby{中}{うち}の
\ruby{最}{もつと}も
\ruby[g]{危險}{き けん}な
\ruby{人}{ひと}だ。
%
\ruby[g]{自{\換字{分}}}{じ ぶん}の
\ruby[||j>]{{\換字{感}}}{かん}
\ruby[||j>]{{\換字{情}}}{じやう}
% \ruby{{\換字{感}}{\換字{情}}}{かん|じやう}
を
\ruby[g]{調御}{てうぎよ}
しなければ、
%
\ruby[g]{自{\換字{分}}}{じ ぶん}は
\ruby{人}{ひと}に
\ruby{使}{つか}はれる
ことが
\ruby[g]{出來}{で き }ぬ。
%
\ruby[g]{自{\換字{分}}}{じ ぶん}の
\ruby[||j>]{{\換字{感}}}{かん}
\ruby[||j>]{{\換字{情}}}{じやう}
% \ruby{{\換字{感}}{\換字{情}}}{かん|じやう}
を
\ruby[g]{調御}{てうぎよ}
しなければ
\原本頁{269-6}\改行%
\ruby[g]{自{\換字{分}}}{じ ぶん}は
\ruby{人}{ひと}を
\ruby{使}{つか}ふことが
\ruby[g]{出來}{で き }ぬ。
%
\ruby[g]{自{\換字{分}}}{じ ぶん}の
\ruby[||j>]{{\換字{感}}}{かん}
\ruby[||j>]{{\換字{情}}}{じやう}を
% \ruby{{\換字{感}}{\換字{情}}}{かん|じやう}を
\ruby[g]{調御}{てうぎよ}しなければ、
%
\原本頁{269-7}\改行%
\ruby[g]{自{\換字{分}}}{じ ぶん}は
\ruby{人}{ひと}に
\ruby{{\換字{交}}}{まじは}ることが
\ruby[g]{出來}{で き }ぬ。
%
\ruby{人}{ひと}に
\ruby{使}{つか}はれず、
%
\ruby{人}{ひと}を
\ruby{使}{つか}はず、
%
\ruby{人}{ひと}に
\ruby{{\換字{交}}}{まじは}らずに
\ruby{濟}{す}む
\ruby[g]{世間}{せ けん}は
\ruby{無}{な}い。
%
\ruby{僕}{ぼく}は
\ruby{僕}{ぼく}だけの
\ruby{小}{ちひさ}な
\ruby[g]{經驗}{けいけん}だが、
%
しかし
\ruby[||j>]{確}{くわく}
\ruby[||j>]{實}{ じつ}
% \ruby{確實}{くわく|じつ}
\ruby[||j>]{堅}{ けん}% ルビ調整(特殊処理)ルビが重なるので
\ruby[||j>]{固}{ ご}な% ルビ調整(特殊処理)ルビが重なるので
\ruby[g]{經驗}{けいけん}から、
%
\ruby[g]{非常}{ひじやう}に
\ruby{{\換字{強}}}{つよ}く
\ruby{深}{ふか}く
\ruby[||j>]{{\換字{感}}}{かん}
\ruby[||j>]{{\換字{情}}}{じやう}
% \ruby{{\換字{感}}{\換字{情}}}{かん|じやう}
の
\ruby[g]{調御}{てうぎよ}が
\ruby[g]{人世}{じんせい}の
\ruby[g]{最大}{さいだい}
\ruby[g]{必要}{ひつ{\換字{𛀁}}う}
のもの
である
といふ
ことを
\ruby[||j>]{確}{くわく}
\ruby[||j>]{信}{ しん}して
% \ruby{確信}{くわく|しん}して
\ruby{居}{ゐ}る。
%
\ruby{君}{きみ}は
\ruby[g]{聰明}{そうめい}
\ruby[g]{絶倫}{ぜつりん}な
\ruby{人}{ひと}だが、
%
\ruby{此}{こ}の
\ruby{點}{てん}の
\ruby[g]{經驗}{けいけん}は
\ruby{或}{あるひ}は
\ruby{薄}{うす}からう。
%
\ruby[g]{戀愛}{れんあい}も
\ruby[g]{是非}{ぜ ひ }がない。
%
\原本頁{270-1}\改行%
\ruby[g]{苦悶}{く もん}も
\ruby{已}{や}むを
\ruby{得}{{\換字{𛀁}}}ぬ。
%
\ruby[g]{一切}{いつさい}の
\ruby{事}{こと}は
\ruby{謝}{しや}せん
として
\ruby{謝}{しや}せぬが
\ruby[g]{天命}{てんめい}だ。
%
\ruby{風}{かぜ}の
\ruby[g]{{\換字{前}}面}{ま へ }から
\ruby{吹}{ふ}く
\ruby{日}{ひ}もある。
%
\ruby[g]{潮流}{し ほ }の
\ruby{横}{よこ}へと
\ruby{行}{ゆ}く
\ruby{夜}{よ}もある。
%
\ruby[g]{颶風}{つむじ }も
\ruby[g]{龍卷}{たつまき}も
\ruby{起}{おこ}る
\ruby{日}{ひ}は
\ruby{起}{おこ}る。
%
しかし
\ruby[||j>]{其}{その}
\ruby[||j>]{間}{あひだ}に
% \ruby{其間}{その|あひだ}に
\ruby{立}{た}つて
\ruby[g]{屹然}{きつぜん}として、
%
\ruby{我}{わ}が
\原本頁{270-4}\改行%
\ruby[g]{正當}{せいたう}の
\ruby[g]{處置}{しよち }を
\ruby{取}{と}つて
\ruby{行}{ゆ}けば
\ruby{死}{し}して
\ruby{餘}{あま}り
あるのだ。
%
\ruby[g]{水野}{みづの }!。
%
\ruby{君}{きみ}が
\ruby{君}{きみ}の
\ruby{欺}{あざむ}かぬ
\ruby[||j>]{{\換字{感}}}{かん}
\ruby[||j>]{{\換字{情}}}{じやう}
% \ruby{{\換字{感}}{\換字{情}}}{かん|じやう}
のために
\ruby{死}{し}にたくば
\ruby{其}{それ}
\ruby{迄}{まで}の
\ruby{事}{こと}だ。
%
しかし
\ruby{君}{きみ}が
\ruby{君}{きみ}として
\ruby{世}{よ}に
\ruby{立}{た}たう
とした
\ruby{大{\換字{丈}}夫}{だい|ぢやう|ぶ}の
\ruby[<j>]{志}{こゝろざし}を% 踊り字調整「〻(二の字点、揺すり点)に見えるが(ゝ)」
\ruby{忘}{わす}れぬ
\ruby{限}{かぎ}りは、
%
\ruby{君}{きみ}は
\ruby{君}{きみ}の
\ruby[||j>]{{\換字{感}}}{かん}
\ruby[||j>]{{\換字{情}}}{じやう}
% \ruby{{\換字{感}}{\換字{情}}}{かん|じやう}
を
\ruby[g]{調御}{てうぎよ}
する
ことを
\ruby{忘}{わす}れては
ならぬ。
%
\ruby{必}{かなら}ず
\ruby[||j>]{{\換字{感}}}{かん}
\ruby[||j>]{{\換字{情}}}{じやう}
% \ruby{{\換字{感}}{\換字{情}}}{かん|じやう}
の
\ruby[g]{調御}{てうぎよ}
といふ
ことを
\ruby{忘}{わす}れずに
\ruby{居}{ゐ}て
\ruby{欲}{ほ}しい。
%
\ruby{君}{きみ}が
\ruby[g]{{\換字{文}}覺}{もんがく}の
\ruby{如}{ごと}き
\ruby{人}{ひと}と
ならん
ことは、
%
\ruby{僕}{ぼく}の
\ruby{最}{もつと}も
\ruby{恐}{おそ}れて
\ruby{居}{ゐ}る
ところだ。
%
\ruby[g]{{\換字{文}}覺}{もんがく}の
\ruby{如}{ごと}きは
\ruby{僕}{ぼく}の
\ruby{蛇蝎視}{だ|かつ|し}する
\ruby{人}{ひと}だ。
%
しかし
\ruby{僕}{ぼく}と
\ruby[g]{日方}{ひ かた}とは
\ruby{言}{ことば}は
\ruby{異}{こと}にして
\ruby{意}{こゝろ}は% 踊り字調整「〻(二の字点、揺すり点)に見えるが(ゝ)」
\ruby{同}{おな}じだ。
%
たまたま% ルビ調整(原本通り)非踊り字表記(行末行頭の境界付近)
\ruby[g]{日方}{ひ かた}の
\ruby{言}{ことば}に
\ruby{僕}{ぼく}の
\ruby[g]{胸裏}{む ね }に
\ruby{觸}{ふ}れた
ところが
\ruby[g]{一寸}{ちよつと}
あつたので、
%
\ruby{言}{い}はずとも
のことを
\ruby[g]{饒舌}{しやべ }つたが、
%
\ruby[g]{二人}{ふたり }の
\ruby{言}{ことば}の
\ruby{異}{ことな}る
ところを
\ruby{忘}{わす}れて、
%
\ruby{其}{そ}の
\ruby{意}{こゝろ}の% 踊り字調整「〻(二の字点、揺すり点)に見えるが(ゝ)」
\ruby{同}{おな}じ
ところ
をさへ
\ruby{取}{と}つて
\ruby{吳}{く}れゝば、% 踊り字調整「〻(二の字点、揺すり点)に見えるが(ゝ)」
%
\ruby[g]{日方}{ひ かた}も
\ruby{僕}{ぼく}も
\ruby[<j||]{何}{どれ }% 行末行頭の境界付近なので特例処置を施す
\ruby[<j||]{程}{ほど }
\ruby[<j||]{悅}{よろこ}ばう!。
』
