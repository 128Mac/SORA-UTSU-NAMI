\Entry{其四十七}

\ruby[g]{{\換字{羽}\換字{勝}}}{はがち}が
\ruby[g]{同情}{おもひやり}のいと
\ruby{厚}{あつ}くして、
\ruby{而}{しか}も
\ruby{道理}{だう|り}の
\ruby{正}{ただし}きに
\ruby{據}{よ}れる、
\ruby{其}{そ}の
\ruby{言}{ことば}には
\ruby{力}{ちから}あり、
\ruby{其}{そ}の
\ruby{意}{こヽろ}には
\ruby[g]{仁有}{なさけあ}るに、
\ruby{分}{わ}けて
\ruby{此頃}{この|ごろ}は
\ruby{感}{かん}じ
\ruby{易}{やす}くなれる
\ruby[g]{水野}{みづの}の、
\ruby{心}{こヽろ}の
\ruby{中}{うち}に
\ruby{深}{ふか}く
\ruby{恩}{おん}を
\ruby{謝}{しや}しながら、
\ruby{言}{い}はれしことの
\ruby{本末}{もと|すゑ}を
\ruby{思}{おも}ひ
\ruby{味}{あじは}ふ
\ruby{時}{とき}、
\ruby[g]{{\換字{羽}\換字{勝}}}{はがち}は
\ruby{復}{ふたヽ}び
\ruby{口}{くち}を
\ruby{開}{ひら}きて、

『
\ruby{僕}{ぼく}の
\ruby{言}{げん}は
\ruby{或}{あるひ}は
\ruby{漠然}{ばく|ぜん}として、
\ruby{捉}{とら}へどころの
\ruby{無}{な}いやうにも
\ruby{思}{おも}へやう。
しかし
\ruby{僕}{ぼく}は
\ruby{漠然}{ばく|ぜん}たることは
\ruby{决}{けつ}して
\ruby{云}{い}はぬ。
\ruby{手}{て}を
\ruby{下}{くだ}すところの
\ruby{知}{し}れぬ
\ruby[g]{教訓}{をしへ}は
\ruby{僕}{ぼく}は
\ruby{嫌}{きら}ふ。
\ruby{着手}{ちやく|しゆ}するところが
\ruby{分明}{ぶん|みやう}で
\ruby{無}{な}ければ
\ruby[g]{實務}{じつむ}は
\ruby{擧}{あが}らぬ。
\ruby{收穫}{とり|いれ}の
\ruby{算用}{さん|よう}を
\ruby{播種}{たね|まき}の
\ruby{前}{まへ}に
\ruby{爲}{す}るのは
\ruby{最}{もつと}も
\ruby{忌}{い}むところだ。
たヾ
\ruby{感{\換字{情}}}{かん|じやう}の
\ruby{訓練}{くん|れん}と
\ruby{云}{い}つても、
\ruby{着手}{ちやく|しゆ}のところを
\ruby{云}{い}はねば
\ruby{空言}{くう|げん}になる。
\ruby{煩}{うるさ}いか
\ruby{知}{し}らんが
\ruby{空言}{くう|げん}にならぬやうに、
\ruby{適切}{てき|せつ}に
\ruby{敢}{あへ}て
\ruby{君}{きみ}のために
\ruby{云}{い}はう。
\ruby{云}{い}ひ
\ruby{{\GWI{u904e-k}}}{す}ぎて
\ruby{無禮}{ぶ|れい}であつても
\ruby{免}{ゆる}し
\ruby{玉}{たま}へ。
たとへば
\ruby{人}{ひと}を
\ruby{思}{おも}ふとすれば、
\ruby{其}{そ}の
\ruby{{\換字{情}}}{じやう}は
\ruby{胸中}{きやう|ちう}に
\ruby{鬱滯}{うつ|たい}して
\ruby{結}{むす}ぼれる。
また
\ruby{例}{たと}へば
\ruby{人}{ひと}を
\ruby{怒}{いか}るとすれば、
\ruby{其}{そ}の
\ruby{{\換字{情}}}{じやう}は
\ruby{心頭}{しん|とう}に
\ruby{狂}{くる}ひ
\ruby{立}{た}つて
\ruby{已}{や}まぬ。
それを
\ruby{其儘}{その|まヽ}に
\ruby{任}{まか}せて
\ruby{置}{お}けば、
\ruby{我}{わ}が
\ruby{本分}{ほん|ぶん}の
\ruby{事}{こと}は
\ruby{其}{そ}れがために
\ruby{{\GWI{u8aa4-k}}}{あやま}られる。
\ruby{夫}{ふなのり}が
\ruby{思}{おも}ひも
\ruby{寄}{よ}らぬ
\ruby{{\GWI{u904e-k}}失}{くわ|しつ}をして、
\ruby{不測}{ふ|そく}の
\ruby{禍{\換字{害}}}{わざ|はひ}を
\ruby{得}{う}る
\ruby{其}{そ}の
\ruby{多}{おほ}くは、
\ruby{胸中}{きやう|ちう}に
\ruby{職務以外}{しよ|くむ|いぐ|わい}の
\ruby{何物}{なに|もの}かヾ
\ruby{蟠}{わだか}まつて、
\ruby[g]{職務}{しよくむ}に
\ruby{放心}{うつ|かり}して
\ruby{居}{ゐ}る
\ruby{時}{とき}に
\ruby{起}{おこ}る。
\ruby{{\換字{叉}}一{\換字{船}}}{また|いつ|せん}の
\ruby[g]{{\換字{平}}和}{へいわ}の
\ruby{破壞}{やぶ|れ}は
\ruby{激烈}{げき|れつ}の
\ruby{感{\換字{情}}}{かん|じやう}の
\ruby{暴發}{ぼう|はつ}に
\ruby{基}{もとづ}く。
そこで
\ruby{自分}{じ|ぶん}が
\ruby{自分}{じ|ぶん}の
\ruby[g]{當直時間}{たうちよくじかん}だけ、
\ruby[g]{甲板}{デツキ}に
\ruby{在}{あ}つて
\ruby[g]{執務}{しつむ}する
\ruby{間}{あひだ}は、
\ruby[g]{何等}{なんら}の
\ruby{私情}{しヾ|やう}が
\ruby{胸中}{きやう|ちう}に
\ruby{在}{あ}らうとも、それを
\ruby{壓}{おさ}へつけて
\ruby[g]{放肆}{はうし}ならしめぬやうに
\ruby{敢}{あへ}てせねばならぬ。
\ruby{親}{おや}を
\ruby{思}{おも}ふは
\ruby[g]{孝子}{かうし}の
\ruby{眞{\換字{情}}}{しん|じやう}だ。
しかし
\ruby{病}{や}んで
\ruby{居}{ゐ}る
\ruby{親}{おや}を
\ruby{思}{おも}つて
\ruby{茫然}{ばう|ぜん}としたヽめ、
\ruby{{\換字{船}}}{ふね}の
\ruby[g]{進路}{しんろ}を
\ruby{{\GWI{u904e-k}}}{あやま}つて
\ruby{洲}{す}へ
\ruby{上}{あ}げたでは
\ruby{濟}{す}まぬ。
\ruby[g]{職務}{しよくむ}を
\ruby{執}{と}つて
\ruby{居}{ゐ}る
\ruby{其間}{その|あひだ}だけは、
\ruby{如何}{い|か}に
\ruby{孝子}{かう|し}でも
\ruby{自}{みづか}ら
\ruby{忍}{しの}んで、
\ruby{親}{おや}を
\ruby{思}{おも}ふ
\ruby{{\換字{情}}}{こヽろ}に
\ruby{氣}{き}を
\ruby{取}{と}られぬやうに、
\ruby{嚴然}{げん|ぜん}と
\ruby{胸中}{きやう|ちう}を
\ruby{{\換字{清}}潔}{せい|けつ}にせねばならぬ。
\ruby{湧}{わ}き
\ruby{上}{あが}り
\ruby{起}{おこ}り
\ruby{立}{た}つ
\ruby{感{\換字{情}}}{かん|じやう}を
\ruby{抑制}{よく|せい}せばならん。
\ruby{訓{\換字{練}}}{くん|れん}して
\ruby{我}{わ}が
\ruby{命令}{めい|れい}に
\ruby{服}{ふく}させねばならん。
これは
\ruby[g]{實務}{じつむ}に
\ruby{身}{み}を
\ruby{{\換字{練}}}{ね}るものヽ
\ruby{必}{かなら}ず
\ruby{知}{し}つて
\ruby{居}{ゐ}るところだ。
\ruby[g]{日方}{ひかた}なども
\ruby{必}{かなら}ず
\ruby[g]{經驗}{けいけん}して
\ruby{居}{ゐ}るところだ。
たヾ
\ruby{世}{よ}に
\ruby{一種}{いつ|しゆ}の
\ruby{人}{ひと}があつて、おのづから
\ruby{感{\換字{情}}}{かん|じやう}の
\ruby{訓{\換字{練}}}{くん|れん}を
\ruby{敢}{あへ}てせぬ
\ruby{履{\換字{歴}}}{り|れき}を
\ruby{有}{いう}して
\ruby{居}{ゐ}る。
\ruby{僕}{ぼく}に
\ruby{云}{い}はせれば
\ruby{其人}{その|ひと}は
\ruby{最}{もつと}も
\ruby{不幸}{ふ|かう}な
\ruby{人}{ひと}だ。
\ruby{直言}{ちよく|げん}すれば、
\ruby[g]{水野}{みづの}、
\ruby{君}{きみ}が
\ruby{其人}{その|ひと}だ。
\ruby{君}{きみ}は
\ruby{美}{うるは}しい
\ruby{感{\換字{情}}}{かん|じやう}を
\ruby{有}{ゆう}して
\ruby{居}{ゐ}て、
\ruby{今}{いま}までは
\ruby{訓{\換字{練}}}{くん|れん}を
\ruby{要}{\GWI{u1b001}う}する
\ruby{事}{こと}がなかつた、それほど
\ruby{美}{うるは}しい
\ruby{感{\換字{情}}}{かん|じやう}を
\ruby{有}{いう}して
\ruby{居}{ゐ}たのだ。
その
\ruby{上}{うへ}、
\ruby{感{\換字{情}}}{かん|じやう}の
\ruby{訓{\換字{練}}}{くん|れん}の
\ruby{必要}{ひつ|\GWI{u1b001}う}を
\ruby{感}{かん}ずる
\ruby{如}{ごと}き
\ruby[g]{職務}{しよくむ}に
\ruby{身}{み}を
\ruby{置}{お}かなかつたのだ。
そこで
\ruby{感{\換字{情}}}{かん|じやう}の
\ruby{訓練}{くん|れん}の
\ruby[g]{履歷}{りれき}を
\ruby{有}{いう}して
\ruby{居}{ゐ}ぬ、それは
\ruby{慥}{たしか}に
\ruby{大}{おほい}に
\ruby{君}{きみ}を
\ruby{苦}{くるし}めるのだ。
\ruby{感{\換字{情}}}{かん|じやう}は
\ruby{馬}{うま}だ。
\ruby{{\GWI{u92b3-k}}}{するど}い
\ruby{感{\換字{情}}}{かん|じやう}を
\ruby{有}{いう}して
\ruby{居}{ゐ}る
\ruby{人}{ひと}は
\ruby{駿馬}{しゆ|んめ}に
\ruby{乘}{の}つて
\ruby{居}{ゐ}る
\ruby{人}{ひと}だ。
\ruby{駿馬}{しゆ|んめ}は
\ruby{愈訓練}{いよ〳〵|くん|れん}せねばならん。
\ruby{然}{さ}も
\ruby{無}{な}けれは、
\ruby{乘}{の}つて
\ruby{居}{ゐ}るものは
\ruby{危}{あぶな}い
\ruby{目}{め}にあふ。
\ruby[g]{水野}{みづの}、
\ruby{君}{きみ}は
\ruby{生來駿馬}{せい|らい|しゆ|んめ}に
\ruby{乘}{の}つて
\ruby{居}{ゐ}る
\ruby{人}{ひと}だ。
\ruby{而}{そ}して
\ruby{今其}{いま|そ}の
\ruby{駿馬}{しゆ|んめ}は
\ruby{無法}{む|はふ}に
\ruby{走}{はし}り
\ruby{出}{だ}して
\ruby{居}{ゐ}るのでは
\ruby{無}{な}いか。
\ruby{谷}{たに}に
\ruby{陷}{おちい}るか
\ruby{崖}{がけ}から
\ruby{墜}{お}つるか、
\ruby{淵}{ふち}へ
\ruby{躍}{をど}り
\ruby{{\GWI{u8fbc-k}}}{こ}むか
\ruby{前{\GWI{u9014-k}}}{さ|き}が
\ruby{知}{し}れぬ。
\ruby{僕等}{ぼく|ら}は
\ruby{傍}{はた}から
\ruby{見}{み}て
\ruby{冷汗}{ひや|あせ}を
\ruby{流}{なが}して、
\ruby{非常}{ひ|じやう}に
\ruby{{\換字{寒}}心}{かん|しん}して
\ruby{居}{ゐ}るのだ。
\ruby{善}{よ}く
\ruby{御}{ぎよ}さなけれは
\ruby[g]{危險}{きけん}は
\ruby{目}{め}の
\ruby{前}{まへ}だ。
どうか
\ruby{訓練}{くん|れん}を
\ruby{敢}{あへ}て
\ruby{爲}{し}て
\ruby{{\換字{呉}}}{く}れたまへ。
\ruby{馬}{うま}のための
\ruby{人}{ひと}では
\ruby{無}{な}い、
\ruby{人}{ひと}のための
\ruby{馬}{うま}だ。
\ruby{馬}{うま}は
\ruby{人}{ひと}の
\ruby{命令}{めい|れい}に
\ruby{服}{ふく}させて、
\ruby{而}{そ}して
\ruby{其}{そ}の
\ruby{能力}{のう|りよく}を
\ruby{盡}{つく}させた
\ruby{時}{とき}、はじめて
\ruby{駿馬}{しゆ|んめ}の
\ruby{貴}{たつと}ぶべきが
\ruby{知}{し}れるのだ。
\ruby{{\換字{文}}覺}{もん|がく}の
\ruby{如}{ごと}きは
\ruby{馬術}{ばじ|ゆつ}をも
\ruby[g]{心掛}{こヽろが}けずして、
\ruby{一生荒馬}{いつ|しやう|あら|うま}に
\ruby{乘}{の}つて
\ruby{無法}{む|はふ}に
\ruby{驅}{か}けて、
\ruby{{\換字{終}}}{しまひ}には
\ruby{撥}{は}ね
\ruby{落}{おと}されて
\ruby{死}{し}んだのに
\ruby{{\GWI{u904e-k}}}{す}ぎん。
\ruby{僕等}{ぼく|ら}は
\ruby{駑馬}{ど|ば}に
\ruby{乘}{の}つて
\ruby{居}{ゐ}るものだ。
\ruby{君}{きみ}は
\ruby{幸}{さいはひ}に
\ruby{駿馬}{しゆ|んめ}に
\ruby{乘}{の}つて
\ruby{居}{ゐ}る
\ruby{人}{ひと}だ。
くれ〴〵も
\ruby{云}{い}ふ
\ruby{人}{ひと}のための
\ruby{馬}{うま}だ、
\ruby{馬}{うま}のための
\ruby{人}{ひと}で
\ruby{無}{な}い。
どうか
\ruby{善}{よ}く
\ruby{{\GWI{u92b3-k}}}{するど}い
\ruby{感{\換字{情}}}{かん|じやう}を
\ruby{御}{ぎよ}して、
\ruby{而}{さう}して
\ruby{君}{きみ}の
\ruby[g]{千萬里}{せんばんり}を
\ruby{馳騁}{ち|へい}するところを
\ruby{見}{み}せて
\ruby{{\換字{呉}}}{く}れたまへ。
\ruby{駿馬}{しゆ|んめ}のために
\ruby{谷}{たに}に
\ruby{陷}{おちい}り
\ruby{淵}{ふち}に
\ruby{落}{お}つる
\ruby{不幸}{ふ|かう}を
\ruby{見}{み}せて
\ruby{{\換字{呉}}}{く}れたまふな。
』

と
\ruby{諄々}{じゆん|〳〵}として
\ruby{徐}{しづか}に
\ruby{{\GWI{u8aaa-jv}}}{と}く
\ruby{時}{とき}、
\ruby[g]{日方}{ひかた}は
\ruby{膝}{ひざ}を
\ruby{打}{う}つて
\ruby{嗟嘆}{さ|たん}して、

『
\ruby{可矣}{い|ヽ}。
\ruby{確言動}{くわく|げん|うご}かすべからずだ。
\ruby[g]{{\換字{羽}\換字{勝}}}{はがち}の
\ruby{言}{げん}だけある!。
\ruby{此馬陣}{この|うま|ぢん}に
\ruby{臨}{のぞ}んで
\ruby{久}{ひさ}しく
\ruby{敵無}{てき|な}し、
\ruby{人}{ひと}と
\ruby{一心}{いつ|しん}にして
\ruby{大功}{たい|こう}を
\ruby{成}{な}すといふ、
\ruby{句}{く}の、
\ruby{彼}{あ}の
\ruby{人}{ひと}と
\ruby{一心}{いつ|しん}といふ
\ruby{四字}{よ|じ}が
\ruby{響}{ひヾ}き
\ruby{渡}{わた}つて、
\ruby{今更{\換字{強}}}{いま|さら|つよ}く
\ruby{面白}{おも|しろ}く
\ruby{感}{かん}じられる!。
\ruby[g]{水野}{みづの}、
\ruby{馬}{うま}をして
\ruby{我}{わ}が
\ruby{意}{こヽろ}に
\ruby{從}{したが}はしめなければならんぞ。
』

と
\ruby{傍}{かたはら}よりまた
\ruby{言葉}{こと|ば}を
\ruby{添}{そ}へたり。

