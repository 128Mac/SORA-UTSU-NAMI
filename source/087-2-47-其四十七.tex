\Entry{其四十七}

% メモ 校正終了 2024-05-08 2024-06-05
\原本頁{271-5}%
\ruby{羽{\換字{勝}}}{は|がち}が
\ruby[||j>]{同}{おも}
\ruby[||j>]{{\換字{情}}}{ひやり}の
% \ruby{同{\換字{情}}}{おも|ひやり}の
いと
\ruby{厚}{あつ}くして、
%
\ruby{而}{しか}も
\ruby{{\換字{道}}理}{だう|り}の
\ruby{正}{たゞし}きに% 踊り字調整「〻(二の字点、揺すり点)に濁点に見えるが(ゞ)」
\ruby{據}{よ}れる、
%
\ruby{其}{そ}の
\ruby[<j||]{言}{ことば}には% 行末行頭の境界付近なので特例処置を施す
\ruby{力}{ちから}あり、
%
\ruby{其}{そ}の
\ruby{意}{こゝろ}には% 踊り字調整「〻(二の字点、揺すり点)に見えるが(ゝ)」
\ruby{仁有}{なさけ|あ}るに、
%
\ruby{{\換字{分}}}{わ}けて
\ruby{此}{この}
\ruby{頃}{ごろ}は
\ruby{{\換字{感}}}{かん}じ
\ruby{易}{やす}くなれる
\ruby{水野}{みづ|の}の、
%
\ruby{心}{むね}の
\ruby{中}{うち}に
\ruby{深}{ふか}くも
\ruby{恩}{おん}を
\ruby{謝}{しや}し
ながら、
%
\ruby{言}{い}はれし
ことの
\ruby{本}{もと}
\ruby{末}{すゑ}を
\ruby{思}{おも}ひ
\ruby{味}{あぢは}ふ
\ruby{時}{とき}、
%
\ruby{羽{\換字{勝}}}{は|がち}は
\ruby{復}{ふたゝ}び% 踊り字調整「〻(二の字点、揺すり点)に見えるが(ゝ)」
\ruby{口}{くち}を
\ruby{開}{ひら}きて、

\原本頁{271-9}%
『
\ruby{僕}{ぼく}の
\ruby{言}{げん}は
\ruby{或}{あるひ}は
\ruby{漠然}{ばく|ぜん}
として、
%
\ruby{捉}{とら}へ
どころの
\ruby{無}{な}い
やうにも
\ruby{思}{おも}へやう。
%
しかし
\ruby{僕}{ぼく}は
\ruby{漠然}{ばく|ぜん}
たることは
\ruby{决}{けつ}して
\ruby{云}{い}はぬ。
%
\ruby{手}{て}を
\ruby{下}{くだ}す
ところの
\ruby{知}{し}れぬ
\ruby{敎訓}{をし|へ}は
\ruby{僕}{ぼく}は
\ruby{{\換字{嫌}}}{きら}ふ。
%
\ruby[||j>]{着}{ちやく}
\ruby[||j>]{手}{ しゆ}
% \ruby{着手}{ちやく|しゆ}
するところが
\ruby[||j>]{{\換字{分}}}{ぶん}
\ruby[||j>]{明}{みやう}で
% \ruby{{\換字{分}}明}{ぶん|みやう}で
\ruby{無}{な}ければ
\ruby{實務}{じつ|む}は
\ruby{擧}{あが}らぬ。
%
\ruby{收穫}{とり|いれ}の
\ruby{算用}{さん|よう}を
\ruby{播種}{たね|まき}の
\ruby{{\換字{前}}}{まへ}に
\ruby{爲}{す}るのは
\ruby{最}{もつと}も
\ruby{忌}{い}む
ところだ。
%
たゞ% 踊り字調整「〻(二の字点、揺すり点)に濁点に見えるが(ゞ)」
\ruby[||j>]{{\換字{感}}}{かん}
\ruby[||j>]{{\換字{情}}}{じやう}の
% \ruby{{\換字{感}}{\換字{情}}}{かん|じやう}の
\ruby{訓練}{くん|れん}と
\ruby{云}{い}つても、
%
\ruby[||j>]{着}{ちやく}
\ruby[||j>]{手}{ しゆ}の
% \ruby{着手}{ちやく|しゆ}の
ところを
\ruby{云}{い}はねば
\ruby{{\換字{空}}言}{くう|げん}に
なる。
%
\ruby{煩}{うるさ}いか
\ruby{知}{し}らんが
\ruby{{\換字{空}}言}{くう|げん}に
ならぬ
やうに、
%
\ruby{{\換字{適}}切}{てき|せつ}に
\ruby{敢}{あへ}て
\ruby{君}{きみ}の
ために
\ruby{云}{い}はう。
%
\ruby{云}{い}ひ
\ruby{{\換字{過}}}{す}ぎて
\ruby{無禮}{ぶ|れい}で
あつても
\ruby{免}{ゆる}し
\ruby{玉}{たま}へ。
%
たとへば
\原本頁{272-6}\改行%
\ruby{人}{ひと}を
\ruby{思}{おも}ふ
とすれば、
%
\ruby{其}{そ}の
\ruby{{\換字{情}}}{じやう}は
\ruby[||j>]{胸}{きやう}
\ruby[||j>]{中}{ ちう}に
% \ruby{胸中}{きやう|ちう}に
\ruby{鬱滯}{うつ|たい}して
\ruby{結}{むす}ぼれる。
%
また
\ruby{例}{たと}へば
\ruby{人}{ひと}を
\ruby{怒}{いか}る
とすれば、
%
\ruby{其}{そ}の
\ruby{{\換字{情}}}{じやう}は
\ruby{心頭}{しん|とう}に
\ruby{狂}{くる}ひ
\ruby{立}{た}つて
\ruby{已}{や}まぬ。
%
それを
\ruby{其}{その}
\ruby{儘}{まゝ}に% 踊り字調整「〻(二の字点、揺すり点)に見えるが(ゝ)」
\ruby{任}{まか}せて
\ruby{置}{お}けば、
%
\ruby{我}{わ}が
\ruby{本{\換字{分}}}{ほん|ぶん}の
\ruby{事}{こと}は
\ruby{其}{そ}れ
がために
\ruby{{\換字{誤}}}{あやま}られる。
%
\ruby{{\換字{船}}夫}{ふな|のり}が
\ruby{思}{おも}ひも
\ruby{寄}{よ}らぬ
\ruby{{\換字{過}}失}{くわ|しつ}を
して、
%
\ruby{不測}{ふ|そく}の
\ruby{禍{\換字{害}}}{わざ|はひ}を
\ruby{得}{う}る
\ruby{其}{そ}の
\ruby{多}{おほ}くは、
%
\ruby[||j>]{胸}{きやう}
\ruby[||j>]{中}{ ちう}に
% \ruby{胸中}{きやう|ちう}に
\ruby{職務}{しよく|む}
\ruby{以外}{い|ぐわい}の
\ruby{何物}{なに|もの}かゞ% 踊り字調整「〻(二の字点、揺すり点)に濁点に見えるが(ゞ)」
\ruby{蟠}{わだか}まつて、
%
\ruby{職務}{しよく|む}に
\ruby{放心}{うつ|かり}して
\ruby{居}{ゐ}る
\ruby{時}{とき}に
\ruby{起}{おこ}る。
%
\ruby{{\換字{又}}}{また}
\ruby{一{\換字{船}}}{いつ|せん}の
\ruby{{\換字{平}}和}{へい|わ}の
\ruby{破壞}{やぶ|れ}は
\ruby{激烈}{げき|れつ}の
\ruby[||j>]{{\換字{感}}}{かん}
\ruby[||j>]{{\換字{情}}}{じやう}の
% \ruby{{\換字{感}}{\換字{情}}}{かん|じやう}の
\ruby{暴發}{ぼう|はつ}に
\ruby{基}{もとづ}く。
%
\原本頁{273-1}\改行%
そこで
\ruby{自{\換字{分}}}{じ|ぶん}が
\ruby{自{\換字{分}}}{じ|ぶん}の
\ruby{當直時間}{たう|ちよく|じ|かん}だけ、
%
\ruby{甲板}{デツ|キ}に
\ruby{在}{あ}つて
\ruby{執務}{しつ|む}する
\ruby{間}{あひだ}は% 行末行頭の境界付近なので特例処置を施す必要があるが微妙なところ
\改行% 校正作業の簡略化のため
、
%
\原本頁{273-2}\改行%
\ruby{何等}{なん|ら}の
\ruby{私{\換字{情}}}{しゞ|やう}が% 踊り字調整「〻(二の字点、揺すり点)に濁点に見えるが(ゞ)」
\ruby[||j>]{胸}{きやう}
\ruby[||j>]{中}{ ちう}に
% \ruby{胸中}{きやう|ちう}に
\ruby{在}{あ}らうとも、
%
それを
\ruby{壓}{おさ}へ
つけて
\ruby{放肆}{はう|し}
ならしめぬ
やうに
\ruby{敢}{あへ}て
せねば
ならぬ。
%
\ruby{親}{おや}を
\ruby{思}{おも}ふは
\ruby{孝子}{かう|し}の
\ruby[||j>]{眞}{しん}
\ruby[||j>]{{\換字{情}}}{じやう}だ。
% \ruby{眞{\換字{情}}}{しん|じやう}だ。
%
しかし
\ruby{病}{や}んで
\ruby{居}{ゐ}る
\ruby{親}{おや}を
\ruby{思}{おも}つて
\ruby{茫然}{ばう|ぜん}
としたゝめ、% 踊り字調整「〻(二の字点、揺すり点)に見えるが(ゝ)」
%
\ruby{{\換字{船}}}{ふね}の
\ruby{{\換字{進}}路}{しん|ろ}を
\ruby{{\換字{過}}}{あやま}つて
\ruby{洲}{す}へ
\ruby{上}{あ}げたでは
\ruby{濟}{す}まぬ。
%
\ruby{職務}{しよく|む}を
\ruby{執}{と}つて
\ruby{居}{ゐ}る
\ruby[||j>]{其}{その}
\ruby[||j>]{間}{あひだ}だけは、
% \ruby{其間}{その|あひだ}だけは、
%
\ruby{如何}{い|か}に
\ruby{孝子}{かう|し}でも
\ruby{自}{みづか}ら
\ruby{{\換字{忍}}}{しの}んで、
%
\ruby{親}{おや}を
\ruby{思}{おも}ふ
\ruby{{\換字{情}}}{こゝろ}に% 踊り字調整「〻(二の字点、揺すり点)に見えるが(ゝ)」
\ruby{氣}{き}を
\ruby{取}{と}られぬ
やうに、
%
\ruby{嚴然}{げん|ぜん}と
\原本頁{273-7}\改行%
\ruby[||j>]{胸}{きやう}
\ruby[||j>]{中}{ ちう}を
% \ruby{胸中}{きやう|ちう}を
\ruby{淸潔}{せい|けつ}に
せねば
ならぬ。
%
\ruby{湧}{わ}き
\ruby{上}{あが}り
\ruby{起}{おこ}り
\ruby{立}{た}つ
\ruby[||j>]{{\換字{感}}}{かん}
\ruby[||j>]{{\換字{情}}}{じやう}を
% \ruby{{\換字{感}}{\換字{情}}}{かん|じやう}を
\ruby{抑制}{よく|せい}
せねば
ならん。
%
\ruby{訓練}{くん|れん}して
\ruby{我}{わ}が
\ruby{命令}{めい|れい}に
\ruby{服}{ふく}させねば
ならん。
%
これは
\ruby{實務}{じつ|む}に
\ruby{身}{み}を
\ruby{練}{ね}るものゝ% 踊り字調整「〻(二の字点、揺すり点)に見えるが(ゝ)」
\ruby{必}{かなら}ず
\ruby{知}{し}つて
\ruby{居}{ゐ}る
ところだ。
%
\ruby{日方}{ひ|かた}
なども
\ruby{必}{かなら}ず
\ruby{經驗}{けい|けん}して
\ruby{居}{ゐ}る
ところだ。
%
たゞ% 踊り字調整「〻(二の字点、揺すり点)に濁点に見えるが(ゞ)」
\ruby{世}{よ}に
\ruby{一種}{いつ|しゆ}の
\ruby{人}{ひと}が
あつて、
%
おのづから
\ruby[||j>]{{\換字{感}}}{かん}
\ruby[||j>]{{\換字{情}}}{じやう}の
% \ruby{{\換字{感}}{\換字{情}}}{かん|じやう}の
\ruby{訓練}{くん|れん}を
\ruby{敢}{あへ}てせぬ
\ruby{履歷}{り|れき}を
\ruby{有}{いう}して
\ruby{居}{ゐ}る。
%
\ruby{僕}{ぼく}に
\ruby{云}{い}はせれば
\ruby{其}{その}
\ruby{人}{ひと}は
\ruby{最}{もつと}も
\ruby{不幸}{ふ|かう}な
\ruby{人}{ひと}だ。
%
\ruby[||j>]{直}{ちよく}
\ruby[||j>]{言}{ げん}すれば、
% \ruby{直言}{ちよく|げん}すれば、
%
\ruby{水野}{みづ|の}、
%
\ruby{君}{きみ}が
\ruby{其}{その}
\ruby{人}{ひと}だ。
%
\ruby{君}{きみ}は
\ruby{美}{うるは}しい
\ruby[||j>]{{\換字{感}}}{かん}
\ruby[||j>]{{\換字{情}}}{じやう}を
% \ruby{{\換字{感}}{\換字{情}}}{かん|じやう}を
\ruby{有}{いう}して
\ruby{居}{ゐ}て、
%
\ruby{今}{いま}までは
\ruby{訓練}{くん|れん}を
\ruby{要}{{\換字{𛀁}}う}する% ルビ調整(原本通り)「𛀁う」
\ruby{事}{こと}が
なかつた、
%
それほど
\ruby{美}{うるは}しい
\ruby[||j>]{{\換字{感}}}{かん}
\ruby[||j>]{{\換字{情}}}{じやう}を
% \ruby{{\換字{感}}{\換字{情}}}{かん|じやう}を
\ruby{有}{いう}して
\ruby{居}{ゐ}たのだ。
%
その
\ruby{上}{うへ}、
%
\ruby[||j>]{{\換字{感}}}{かん}
\ruby[||j>]{{\換字{情}}}{じやう}の
% \ruby{{\換字{感}}{\換字{情}}}{かん|じやう}の
\ruby{訓練}{くん|れん}の
\ruby{必要}{ひつ|{\換字{𛀁}}う}を
\ruby{{\換字{感}}}{かん}ずる
\ruby{如}{ごと}き
\ruby{職務}{しよく|む}に
\ruby{身}{み}を
\ruby{置}{お}かなかつた
のだ。
%
そこで
\ruby[||j>]{{\換字{感}}}{かん}
\ruby[||j>]{{\換字{情}}}{じやう}の
% \ruby{{\換字{感}}{\換字{情}}}{かん|じやう}の
\ruby{訓練}{くん|れん}の
\ruby{履歷}{り|れき}を
\ruby{有}{いう}して
\ruby{居}{ゐ}ぬ、
%
それは
\ruby{慥}{たしか}に
\ruby{大}{おほい}に
\ruby{君}{きみ}を
\ruby{苦}{くるし}めるのだ。
%
\ruby[<j||]{{\換字{感}}}{かん }% 行末行頭の境界付近なので特例処置を施す
\ruby[<j||]{{\換字{情}}}{じやう}は
% \ruby{{\換字{感}}{\換字{情}}}{かん|じやう}は
\ruby{馬}{うま}だ。
%
\ruby{{\換字{銳}}}{するど}い
\ruby[||j>]{{\換字{感}}}{かん}
\ruby[||j>]{{\換字{情}}}{じやう}を
% \ruby{{\換字{感}}{\換字{情}}}{かん|じやう}を
\ruby{有}{いう}して
\ruby{居}{ゐ}る
\ruby{人}{ひと}は
\ruby{駿馬}{しゆ|んめ}に
\ruby{乘}{の}つて
\ruby{居}{ゐ}る
\ruby{人}{ひと}だ。
%
\ruby[<j||]{駿}{しゆん}% 行末行頭の境界付近なので特例処置を施す
\ruby{馬}{め}は
\ruby[<j>]{愈}{いよ〳〵}% ルビ調整(特殊処理)「愈」のルビが4文字
\ruby[||j>]{訓}{ くん}
\ruby[||j>]{練}{ れん}
せねば
ならん。
%
\ruby{然}{さ}も
\ruby{無}{な}けれは、
%
\ruby{乘}{の}つて
\ruby{居}{ゐ}るものは
\ruby[<j||]{危}{あぶな}い% 行末行頭の境界付近なので特例処置を施す
\ruby{目}{め}にあふ。
%
\ruby{水野}{みづ|の}、
%
\ruby{君}{きみ}は
\ruby{生來}{せい|らい}
\ruby{駿馬}{しゆ|んめ}に
\ruby{乘}{の}つて
\ruby{居}{ゐ}る
\ruby{人}{ひと}だ。
%
\ruby{而}{そ}して
\ruby{今}{いま}
\原本頁{274-9}\改行%
\ruby{其}{そ}の
\ruby{駿馬}{しゆ|んめ}は
\ruby{無法}{む|はふ}に
\ruby{走}{はし}り
\ruby{出}{だ}して
\ruby{居}{ゐ}るのでは
\ruby{無}{な}いか。
%
\ruby{谷}{たに}に
\ruby{陷}{おちい}るか、
\ruby{崖}{がけ}から
\ruby{墜}{お}つるか、
%
\ruby{淵}{ふち}へ
\ruby{躍}{をど}り
\ruby{{\換字{込}}}{こ}むか
\ruby{{\換字{前}}{\換字{途}}}{さ|き}が
\ruby{知}{し}れぬ。
%
\ruby{僕等}{ぼく|ら}は
\ruby{傍}{はた}から
\原本頁{274-11}\改行%
\ruby{見}{み}て
\ruby{冷}{ひや}
\ruby{汗}{あせ}を
\ruby{流}{なが}して、
%
\ruby{非常}{ひ|じやう}に
\ruby{{\換字{寒}}心}{かん|しん}して
\ruby{居}{ゐ}るのだ。
%
\ruby{善}{よ}く
\ruby{御}{ぎよ}さなければ
\ruby{危險}{き|けん}は
\ruby{目}{め}の
\ruby{{\換字{前}}}{まへ}だ。
%
どうか
\ruby{訓練}{くん|れん}を
\ruby{敢}{あへ}て
\ruby{爲}{し}て
\ruby{吳}{く}れたまへ。
%
\ruby{馬}{うま}のための
\ruby{人}{ひと}では
\ruby{無}{な}い、
%
\ruby{人}{ひと}のための
\ruby{馬}{うま}だ。
%
\ruby{馬}{うま}は
\ruby{人}{ひと}の
\ruby{命令}{めい|れい}に
\ruby{服}{ふく}させて、
%
\原本頁{275-3}\改行%
\ruby{而}{そ}して
\ruby{其}{そ}の
\ruby[||j>]{能}{のう}
\ruby[||j>]{力}{りよく}を
% \ruby{能力}{のう|りよく}を
\ruby{盡}{つく}させた
\ruby{時}{とき}、
%
はじめて
\ruby{駿馬}{しゆ|んめ}の
\ruby{貴}{たつと}ぶ
べきが
\ruby{知}{し}れるのだ。
%
\ruby{{\換字{文}}覺}{もん|がく}の
\ruby{如}{ごと}きは
\ruby{馬{\換字{術}}}{ば|じゆつ}をも
\ruby{心掛}{こゝろ|が}けずして、% 踊り字調整「〻(二の字点、揺すり点)に見えるが(ゝ)」
%
\ruby[<j||]{一}{いつ }% ルビ調整(特殊処理)ルビが重なるので調整
\ruby[<j||]{生}{しやう}
% \ruby{一生}{いつ|しやう}
\ruby{荒馬}{あら|うま}に
\ruby{乘}{の}つて
\ruby{無法}{む|はふ}に
\ruby{驅}{か}けて、
%
\ruby{{\換字{終}}}{しまひ}には
\ruby{撥}{は}ね
\ruby{落}{おと}されて
\ruby{死}{し}んだのに
\ruby{{\換字{過}}}{す}ぎん。
%
\ruby{僕等}{ぼく|ら}は
\ruby{駑馬}{ど|ば}に
\ruby{乘}{の}つて
\ruby{居}{ゐ}るものだ。
%
\ruby{君}{きみ}は
\ruby[<j>]{幸}{さいはひ}に
\ruby{駿馬}{しゆ|んめ}に
\ruby{乘}{の}つて
\ruby{居}{ゐ}る
\ruby{人}{ひと}だ。
%
\原本頁{275-7}\改行%
くれ〴〵も
\ruby{云}{い}ふ
\ruby{人}{ひと}のための
\ruby{馬}{うま}だ、
%
\ruby{馬}{うま}のための
\ruby{人}{ひと}で
\ruby{無}{な}い。
%
どうか
\ruby{善}{よ}く
\ruby{{\換字{銳}}}{するど}い
\ruby[||j>]{{\換字{感}}}{かん}
\ruby[||j>]{{\換字{情}}}{じやう}を
% \ruby{{\換字{感}}{\換字{情}}}{かん|じやう}を
\ruby{御}{ぎよ}して、
%
\ruby{而}{さう}して
\ruby{君}{きみ}の
\ruby{千萬里}{せん|ばん|り}を
\ruby{馳騁}{ち|へい}
する
ところを
\ruby{見}{み}せて
\ruby{吳}{く}れ
たまへ。
%
\ruby{駿馬}{しゆ|んめ}の
ために
\ruby{谷}{たに}に
\ruby{陷}{おちい}り
\ruby{淵}{ふち}に
\ruby{落}{お}つる
\ruby{不幸}{ふ|かう}を
\ruby{見}{み}せて
\ruby{吳}{く}れたまふな。
』

\原本頁{275-11}%
と
\ruby[||j>]{諄}{じゆん}
\ruby[||j>]{々}{ 〴〵}として
% \ruby{諄々}{じゆん|〳〵}として
\ruby{徐}{しづか}に
\ruby{說}{と}く
\ruby{時}{とき}、
%
\ruby{日方}{ひ|かた}は
\ruby{膝}{ひざ}を
\ruby{打}{う}つて
\ruby{嗟嘆}{さ|たん}して、

\原本頁{276-1}%
『
\ruby{可矣}{い|ゝ}。% 踊り字調整「〻(二の字点、揺すり点)に見えるが(ゝ)」
%
\ruby[||j>]{確}{くわく}
\ruby[||j>]{言}{ げん}
% \ruby{確言}{くわく|げん}
\ruby[||j>]{動}{ うご}かす
べからずだ。
%
\ruby{羽{\換字{勝}}}{は|がち}の
\ruby{言}{げん}だけある!。
%
% 高都護驄馬行 杜甫
%
% 安西都護胡青驄 ,聲價欻然來向東。
%     安西都護高仙芝の乗馬される西方産の葦毛の駿馬、
%     その評判ともども東の方長安の方へむかって来た。
% 此馬臨陣久無敵,與人一心成大功。
%     この馬は戦陣に臨んでは久しい、
%     前から敵するものがなく、
%     のり手と心を同一にして大功を成した馬である
% 功成惠養隨所致,飄飄遠自流沙至。
%     この馬が諷々とかけて遠く流抄の地方からやってきた、
%     すでに大功を成した馬だから
%           どんな手あつい飼養の方法でも
%           馬がしたいとおもうままにしている。
% 雄姿未受伏櫪恩,猛氣猶思戰場利。
%     この馬はまだ雄々しい姿をしていて
%           老馬が受ける様なへたばって物を貰うようなことはしない、
%     その猛烈な元気はいまだに戦場の勝利の事をかんがえているのである。
% 腕促蹄高如踣鐵,交河幾蹴會冰裂。
%     この馬は腕の長さがつまり蹄はあつく之をふみとどろかすときは堅くて鉄をふむようである、
%     この蹄でいく度か交河のあたりで、かさなった泳を蹴ってくだいた。
% 五花散作雲滿身,萬裡方看汗流血。
%     からだは五色の梅の花がたが散らばって雲が一ぱいにひろがっているようだ。
%     我々は眼前この馬が万里の道中をして来て血の汗を流すのをみるのである。
% 長安壯兒不敢騎,走過掣電傾城知。
%     長安の若者もこの馬にのりこなすことはできない。
%     この馬が走りさるときは
%           電光をひくようにはやいことは長安城中のものだれも知らぬものはない。
% 青絲絡頭為君老,何由卻出橫門道。
%     この馬が主君の意のまま丁重に飼われ、
%     頭には青糸をまきつけて飾られて為す事なくしてそのまま老いてゆく、
%     馬の心ではどうしたら今の無為の状況を脱して横門の道から外へでられるだろうかとかんがえている。
\ruby{此}{この}
\ruby{馬}{うま}
\ruby{陣}{ぢん}に
\ruby{臨}{のぞ}んで
\ruby{久}{ひさ}しく
\ruby{敵}{てき}
\ruby{無}{な}し、
%
\ruby{人}{ひと}と
\ruby{一心}{いつ|しん}にして
\ruby{大功}{たい|こう}を
\ruby{成}{な}すといふ、
%
\ruby{句}{く}の、
%
\ruby{彼}{あ}の
\ruby{人}{ひと}と
\ruby{一心}{いつ|しん}といふ
\ruby{四字}{よ|じ}が
\ruby{響}{ひゞ}き% 踊り字調整「〻(二の字点、揺すり点)に濁点に見えるが(ゞ)」
\ruby{渡}{わた}つて、
%
\ruby{今}{いま}
\ruby{{\換字{更}}}{さら}
\ruby{{\換字{強}}}{つよ}く
\ruby{面白}{おも|しろ}く
\ruby{{\換字{感}}}{かん}じられる!。
%
\ruby{水野}{みづ|の}、
%
\ruby{馬}{うま}をして
\ruby{我}{わ}が
\ruby{意}{こゝろ}に% 踊り字調整「〻(二の字点、揺すり点)に見えるが(ゝ)」
\ruby{從}{したが}はしめ
なければ
ならんぞ。
』

\原本頁{276-6}%
と
\ruby[<j>]{傍}{かたはら}より
また
\ruby{言葉}{こと|ば}を
\ruby{添}{そ}へたり。
