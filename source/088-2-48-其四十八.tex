\Entry{其四十八}

\ruby[g]{日方}{ひかた}が
\ruby{手荒}{て|あら}き
\ruby{擧動}{ふる|まひ}といひ、
\ruby[g]{{\GWI{u7fbd-k}\換字{勝}}}{はがち}が
\ruby{物固}{もの|がた}き
\ruby{言葉}{こと|ば}といひ、
\ruby{皆}{みな}これ
\ruby{淺}{あさ}からず
\ruby{我}{われ}を
\ruby{思}{おも}ひ
\ruby{{\GWI{u5449-itaiji-002}}}{く}るヽ
\ruby{朋友}{と|も}の
\ruby{{\GWI{u60c5-k}}}{なさけ}の
\ruby[g]{眞實}{まこと}なりとおもふに、
\ruby[g]{水野}{みづの}は
\ruby{泣}{な}かぬばかりの
\ruby{面}{かほ}つきとなつて、
\ruby{血}{ち}の
\ruby{氣}{け}も
\ruby{失}{う}せたるやうの
\ruby{兩}{りやう}の
\ruby{頬}{ほう}には、
\ruby[g]{勢無}{いきほひな}き
\ruby{心}{こヽろ}の
\ruby{淋}{さび}しさを
\ruby{現}{あら}はし、
\ruby{露}{つゆ}ばかりも
\ruby{動}{うご}かざる
\ruby{眼}{め}の
\ruby{中}{うち}は
\ruby{一念}{いち|ねん}の
\ruby{沈}{しづ}みきつて
\ruby{一}{ひ}ㇳ
\ruby{處}{ところ}に
\ruby{凝}{こ}れる
\ruby{狀態}{あり|さま}を
\ruby{示}{しめ}す
\ruby{如}{ごと}く、やヽ
\ruby[g]{少時}{しばし}は
\ruby{物}{もの}をさへ
\ruby{云}{い}ひ
\ruby{{\換字{兼}}}{か}ねたりしが、やがて
\ruby{感激}{かん|げき}に
\ruby{堪}{た}へ
\ruby{得}{\GWI{u1b001}}ずしてや、さしぐむ
\ruby{淚}{なみだ}に
\ruby{聲}{こゑ}も
\ruby{{\GWI{u5f31-k}}々}{よわ|〳〵}と、

『あヽ
\ruby{有難}{あり|がた}い!、
\ruby{實}{じつ}に
\ruby{謝}{しや}する!、
\ruby{二君}{に|くん}の
\ruby{厚意}{こう|い}は
\ruby{决}{けつ}して
\ruby{忘}{わす}れぬ。
\ruby{特}{こと}に
\ruby[g]{{\GWI{u7fbd-k}\換字{勝}}君}{はがちくん}の
\ruby{教}{をしへ}は
\ruby[g]{心魂}{しんこん}に
\ruby{徹}{てつ}して、
\ruby[g]{愚鈍}{ぐどん}の
\ruby{僕}{ぼく}にもよく
\ruby{解}{わか}つた。
\ruby{君等}{きみ|たち}の
\ruby{親切}{しん|せつ}に
\ruby{激勵}{は|げ}まされて、
\ruby{出來}{で|き}ないまでも
\ruby{僕}{ぼく}は
\ruby{自}{みづか}ら
\ruby{勉}{つと}めて
\ruby{{\GWI{u904e-k}}}{あやま}たぬやうにする。
\ruby{感{\GWI{u60c5-k}}}{かん|じやう}の
\ruby{訓練}{くん|れん}といふ
\ruby{事}{こと}も
\ruby[g]{屹度敢}{きつとあへ}てする。
\ruby{不幸}{ふ|かう}にして
\ruby{力}{ちから}が
\ruby{足}{た}らなくつて、
\ruby{轉}{ころ}んでも
\ruby{倒}{たふ}れても
\ruby{溪}{たに}に
\ruby{落}{お}ちても、
\ruby{轉}{ころ}べば
\ruby{起上}{おき|あが}る、
\ruby{倒}{たふ}るれば
\ruby{立}{た}つ、
\ruby{溪}{たに}に
\ruby{落}{お}ちても
\ruby[g]{屹度這}{きつとは}ひ
\ruby{上}{あが}つて、
\ruby{目}{め}ざところまで
\ruby{必}{かなら}ず
\ruby{行}{ゆ}かうといふ
\ruby{氣}{き}ばかりは、
\ruby{何樣}{ど|う}あつても
\ruby[g]{屹度忘}{きつとわす}れぬつもりだ。
\ruby{僕}{ぼく}に
\ruby[g]{生命}{いのち}の
\ruby{有}{あ}らん
\ruby{限}{かぎ}りは、
\ruby{一日}{いち|にち}に
\ruby{一日}{いち|にち}だけ
\ruby{此}{こ}の
\ruby{心}{こヽろ}を
\ruby{懷}{いだ}いて、
\ruby{苦}{くるし}んでも
\ruby{悶}{もだ}えても
\ruby[g]{生存}{ながら}へやうと
\ruby{思}{おも}ふ
\ruby{此}{こ}の
\ruby{僕}{ぼく}の
\ruby{眞}{しん}の
\ruby{意}{こヽろ}を
\ruby{汲}{く}んで
\ruby{{\GWI{u5449-itaiji-002}}}{く}れて、
\ruby{何樣}{ど|う}か
\ruby{僕}{ぼく}を
\ruby{見放}{み|はな}さずに
\ruby{居}{ゐ}て
\ruby{{\GWI{u5449-itaiji-002}}}{く}れたまへ。
\ruby{長}{なが}く
\ruby{此}{こ}の
\ruby{僕}{ぼく}に
\ruby{君等}{きみ|たち}の
\ruby{友}{とも}たる
\ruby{幸福}{さい|はひ}を
\ruby{得}{\GWI{u1b001}}させて
\ruby{置}{おい}て
\ruby{{\GWI{u5449-itaiji-002}}}{く}れたまへ。
\ruby{君等}{きみ|たち}は
\ruby{皆優}{みな|やさ}しく
\ruby{教}{おし}へて
\ruby{{\GWI{u5449-itaiji-002}}}{く}れるし、
\ruby{自分}{じ|ぶん}でも
\ruby{氣}{き}が
\ruby{付}{つ}いて
\ruby{居}{ゐ}るし、
\ruby{自}{みづか}ら
\ruby{克}{か}たうとしたり
\ruby{自}{みづか}ら
\ruby{憤}{いきどほ}つたり、
\ruby{自}{みづか}ら
\ruby{爭}{あらそ}つたり
\ruby{自}{みづか}ら
\ruby{鬪}{たヽか}つたり、
\ruby{心}{こヽろ}の
\ruby{中}{うち}の
\ruby{揉}{も}めぬ
\ruby{日}{ひ}も
\ruby{無}{な}く、
\ruby{力}{ちから}も
\ruby{根}{こん}も
\ruby{使}{つか}ひ
\ruby{盡}{つく}して
\ruby{今日}{け|ふ}まで
\ruby{來}{き}たが、
\ruby{何}{なん}と
\ruby{無}{な}く
\ruby{行末}{ゆく|すゑ}が
\ruby[g]{物怖}{ものおそろ}しくて、
\ruby{知}{し}りつヽ
\ruby{高}{たか}い
\ruby{崖}{がけ}から
\ruby{深}{ふか}い
\ruby{淵}{ふち}に
\ruby{陷}{おちい}るやうな
\ruby{時}{とき}が
\ruby{有}{あ}りはせぬかと
\ruby{思}{おも}ふ。
\ruby{必}{かなら}ず〳〵
\ruby{其樣}{そ|ん}なことにはならい
\ruby{樣}{やう}に、
\ruby{君等}{きみ|たち}の
\ruby{厚意}{こう|い}を
\ruby{空}{むな}しくせぬやうにと、
\ruby{一生懸命}{いつ|しやう|けん|めい}に
\ruby{思}{おも}つては
\ruby{居}{ゐ}るが、
\ruby[g]{萬一萬々一左樣}{まんいちまん〳〵いちさふ}いふ
\ruby{目}{め}にあつても、
\ruby[g]{屹度}{きつと}それきりにはならぬつもりの、
\ruby{其點}{そ|こ}を
\ruby[g]{水野}{みづの}だと
\ruby{見}{み}て
\ruby{{\GWI{u5449-itaiji-002}}}{く}れて、あれほど
\ruby{諭}{さと}したのに
\ruby{云}{い}ひ
\ruby{甲斐}{が|い}の
\ruby{無}{な}い、とう〳〵
\ruby{深}{ふか}みへ
\ruby{落}{お}ちた
\ruby{馬鹿}{ば|か}な
\ruby{奴}{やつ}だと、
\ruby{爪彈}{つま|はじ}きして
\ruby{棄}{す}てるやうなことを
\ruby{爲}{し}て
\ruby{{\GWI{u5449-itaiji-002}}}{く}れたまふな。
\ruby{餘}{あま}り
\ruby{愚}{ぐ}な
\ruby{事}{こと}をいふやうだが、たヾ
\ruby{何}{なん}と
\ruby{無}{な}く
\ruby{僕}{ぼく}の
\ruby[g]{前途}{ぜんと}に
\ruby{恐}{おそ}ろしい
\ruby{不幸}{ふ|かう}が
\ruby{手}{て}を
\ruby{擴}{ひろ}げて、
\ruby{僕}{ぼく}の
\ruby{行}{ゆ}くのを
\ruby{待}{ま}つて
\ruby{居}{ゐ}るやうに
\ruby{思}{おも}へる。
\ruby{何樣}{ど|う}も
\ruby{左樣思}{さ|う|おも}へてならんので、それで
\ruby{如是}{こ|ん}なことも
\ruby{言}{い}ひ
\ruby{出}{いだ}すのだが、
\ruby{何樣罷}{ど|う|まか}り
\ruby[g]{問違}{まちが}つても
\ruby{本來}{ほん|らい}の
\ruby{一心}{いつ|しん}は、
\ruby{君等}{きみ|たち}に
\ruby{對}{たい}しても
\ruby{决}{けつ}して
\ruby{忘}{わす}れぬ、
\ruby{其處}{そ|こ}をたヾ
\ruby[g]{水野}{みづの}だと
\ruby{思}{おも}つて
\ruby[g]{交際}{つきあ}つて
\ruby{{\GWI{u5449-itaiji-002}}}{く}れたまへ。

\ruby{人}{ひと}の
\ruby{運命}{うん|めい}の
\ruby[g]{明日}{あした}は
\ruby{分}{わか}らぬが、
\ruby{君等}{きみ|たち}の
\ruby{厚意}{こう|い}は
\ruby{夢}{ゆめ}の
\ruby{間}{ま}も
\ruby{忘}{わす}れぬ。
\ruby{君等}{きみ|たち}に
\ruby{負}{そむ}かぬやうにとは
\ruby[g]{屹度努力}{きつとどりよく}する。
』

と、
\ruby{心}{こヽろ}に
\ruby{張}{は}りのあるさまは
\ruby{{\GWI{u7336-k}}見}{なほ|み}えながら、
\ruby{意氣}{い|き}は
\ruby{振}{ふる}はずして
\ruby{龍鍾}{しを|〳〵}と
\ruby{言}{い}ふ
\ruby{其}{そ}の
\ruby{哀}{あは}れなる
\ruby[g]{樣子}{やうす}を
\ruby[g]{日方}{ひかた}は
\ruby[g]{見{\GWI{u904e-k}}}{みすご}しかね、

『なに!、
\ruby{何}{なん}と
\ruby{無}{な}く
\ruby{行末}{ゆく|すゑ}が
\ruby{怖}{おそ}ろしくつて、
\ruby{不幸}{ふ|かう}の
\ruby{運命}{うん|めい}が
\ruby{待}{まつ}て
\ruby{居}{ゐ}るやうに
\ruby{思}{おも}へるつて?。
\ruby{何其樣}{なに|そ|ん}なことが
\ruby{有}{あ}つて
\ruby{堪}{たま}るものか。
\ruby[g]{我々}{われ〳〵}の
\ruby{行末}{ゆく|すゑ}は
\ruby[g]{皆輝}{みなかがや}いて
\ruby{居}{ゐ}る!。
\ruby{我々七人}{われ|〳〵|しち|にん}の
\ruby{行末}{ゆく|すゑ}に
\ruby{暗黒}{や|み}は
\ruby{無}{な}いのた!。
\ruby{燃}{も}える
\ruby{火}{ひ}の
\ruby{前}{まへ}に
\ruby{暗黒}{や|み}が
\ruby{有}{あ}るかい!。
\ruby{暗黒}{や|み}はたヾ
\ruby{{\GWI{u904e-k}}}{す}ぎた
\ruby{昨日}{きの|ふ}の
\ruby{事}{こと}!。

\ruby{生}{い}きて
\ruby{居}{ゐ}る
\ruby{人間}{にん|げん}、
\ruby{燃}{も}えて
\ruby{居}{ゐ}る
\ruby{火}{ひ}の、
\ruby{其前}{その|まへ}に
\ruby{暗黒}{や|み}が
\ruby{有}{あ}るとは
\ruby{誰}{だれ}が
\ruby{言}{い}ふ?。
そんな
\ruby{事}{こと}を
\ruby{思}{おも}ふのは
\ruby{氣}{き}の
\ruby{{\GWI{u8ff7-k}}}{まよ}ひだ。
\ruby[g]{悉皆汝}{みんなきさま}の
\ruby{衰{\GWI{u5f31-k}}}{おと|ろへ}からだ!。
しつかり
\ruby{爲}{し}なくてはいかんぞ
\ruby[g]{水野}{みづの}!。
\ruby[g]{喇叭}{らつぱ}が
\ruby{{\GWI{u9032-k}}}{すヽ}めと
\ruby{鳴}{な}りやあ
\ruby{敵}{てき}はもう
\ruby{無}{な}いんだ。
\ruby{大丈夫}{だい|ぢやう|ぶ}の
\ruby{向}{むか}つて
\ruby{行}{ゆ}くところには
\ruby{不幸}{ふ|かう}も
\ruby{何}{なに}も
\ruby{無}{な}い。
\ruby{下}{くだ}らんことをいつてまだ
\ruby{撲}{なぐ}られたいか。
\ruby[g]{{\GWI{u7fbd-k}\換字{勝}}}{はがち}
\ruby{言}{げん}に
\ruby{從}{したが}つて
\ruby{努力}{どり|よく}して
\ruby{日}{ひ}を
\ruby{{\GWI{u9001-k}}}{おく}れ。
\ruby{汝}{きさま}の
\ruby[g]{前{\GWI{u9014-k}}}{ぜんと}の
\ruby[g]{多幸}{たかう}なのは
\ruby{乃公}{お|れ}が
\ruby{受合}{うけ|あ}ふ。
』

と
\ruby[g]{壯語}{さうご}の
\ruby{有}{あ}る
\ruby{限}{かぎ}りを
\ruby{盡}{つく}して
\ruby{氣}{き}を
\ruby{引立}{ひき|た}てたる
\ruby{其時室外}{その|とき|しつ|ぐわい}に
\ruby{人}{ひと}の
\ruby{氣色}{け|はひ}して、
\ruby{忽}{たちま}ち
\ruby{間}{あひ}の
\ruby{襖}{ふすま}は
\ruby[g]{右左}{みぎひだり}に
\ruby{大}{おほき}く
\ruby{開}{ひら}かれたり。

