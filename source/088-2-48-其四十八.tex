\Entry{其四十八}

% メモ 校正終了 2024-05-08 2024-06-05
\原本頁{276-8}%
\ruby[g]{日方}{ひ かた}が
\ruby{手}{て}
\ruby{荒}{あら}き
\ruby[g]{擧動}{ふるまひ}といひ、
%
\ruby[g]{羽{\換字{勝}}}{は がち}が
\ruby{物}{もの}
\ruby{固}{がた}き
\ruby[g]{言葉}{ことば }といひ、
%
\ruby{皆}{みな}
これ
\ruby{淺}{あさ}からず
\ruby{我}{われ}を
\ruby{思}{おも}ひ
\ruby{吳}{く}るゝ% 踊り字調整「〻(二の字点、揺すり点)に見えるが(ゝ)」
\ruby[g]{朋友}{と も }の
\ruby{{\換字{情}}}{なさけ}の
\ruby[g]{眞實}{まこと }なり
とおもふに、
%
\ruby[g]{水野}{みづの }は
\ruby{泣}{な}かぬ
ばかりの
\ruby{面}{かほ}
つき
となつて、
%
\ruby{血}{ち}の
\ruby{氣}{け}も
\ruby{失}{う}せ
たる
やうの
\ruby{兩}{りやう}の
\原本頁{277-1}\改行%
\ruby{頰}{ほう}には、
%
\ruby[<j>]{勢}{いきほひ}
\ruby[||j>]{無}{ な }き
\ruby{心}{こゝろ}の% 踊り字調整「〻(二の字点、揺すり点)に見えるが(ゝ)」
\ruby{淋}{さび}しさを
\ruby{現}{あら}はし、
%
\ruby{露}{つゆ}
ばかりも
\ruby{動}{うご}かざる
\ruby{眼}{め}の
\ruby{中}{うち}は、
\ruby[g]{一念}{いちねん}の
\ruby{沈}{しづ}みきつて
\ruby{一}{ひ}ト
\ruby{處}{ところ}に
\ruby{凝}{こ}れる
\ruby[g]{狀態}{ありさま}を
\ruby{示}{しめ}す
\ruby{如}{ごと}く、
%
やゝ% 踊り字調整「〻(二の字点、揺すり点)に見えるが(ゝ)」
\ruby[g]{少時}{しばし }は
\ruby{物}{もの}
をさへ
\ruby{云}{い}ひ
\ruby{{\換字{兼}}}{か}ねたりしが、
%
やがて
\ruby[g]{{\換字{感}}激}{かんげき}に
\ruby{堪}{た}へ
\ruby{得}{{\換字{𛀁}}}ずしてや、
%
さしぐむ
\ruby{涙}{なみだ}に
\ruby{聲}{こゑ}も
\ruby[g]{{\換字{弱}}々}{よわ〳〵}と、

\原本頁{277-5}%
『
あゝ% 踊り字調整「〻(二の字点、揺すり点)に見えるが(ゝ)」
\ruby[g]{有{\換字{難}}}{ありがた}い!、
%
\ruby{實}{じつ}に
\ruby{謝}{しや}する!、
%
\ruby[g]{二君}{に くん}の
\ruby[g]{厚意}{かうい }は
\ruby{决}{けつ}して
\ruby{忘}{わす}れぬ。
%
\原本頁{277-6}\改行%
\ruby{特}{こと}に
\ruby[g]{羽{\換字{勝}}}{は がち}
\ruby{君}{くん}の
\ruby{敎}{をしへ}は
\ruby[g]{心魂}{しんこん}に
\ruby{徹}{てつ}して、
%
\ruby[g]{愚鈍}{ぐ どん}の
\ruby{僕}{ぼく}にも
よく
\ruby{解}{わか}つた。
%
\ruby[g]{君等}{きみたち}の
\ruby[g]{親切}{しんせつ}に
\ruby[g]{激勵}{は げ }まされて、
%
\ruby[g]{出來}{で き }ないまでも
\ruby{僕}{ぼく}は
\ruby{自}{みづか}ら
\ruby{勉}{つと}めて
\ruby{{\換字{過}}}{あやま}たぬ
やうにする。
%
\ruby[||j>]{{\換字{感}}}{かん}
\ruby[||j>]{{\換字{情}}}{じやう}の
% \ruby{{\換字{感}}{\換字{情}}}{かん|じやう}の
\ruby[g]{訓練}{くんれん}
といふ
\ruby{事}{こと}も
\ruby[g]{屹度}{きつと }
\ruby{敢}{あへ}てする。
%
\ruby[g]{不幸}{ふ かう}にして
\ruby{力}{ちから}が
\ruby{足}{た}らなくつて、
%
\ruby{轉}{ころ}んでも
\ruby{倒}{たふ}れても
\ruby{溪}{たに}に
\ruby{落}{お}ちても、
%
\ruby{轉}{ころ}べば
\原本頁{277-10}\改行%
\ruby{起}{おき}
\ruby{上}{あが}る、
%
\ruby{倒}{たふ}るれば
\ruby{立}{た}つ、
%
\ruby{溪}{たに}に
\ruby{落}{お}ちても
\ruby[g]{屹度}{きつと }
\ruby{這}{は}ひ
\ruby{上}{あが}つて、
%
\ruby{目}{め}ざす
ところまで
\ruby{必}{かなら}ず
\ruby{行}{ゆ}かう
といふ
\ruby{氣}{き}
ばかりは、
%
\ruby[g]{何樣}{ど う }
あつても
\ruby[g]{屹度}{きつと }
\ruby{忘}{わす}れぬ
つもりだ。
%
\ruby{僕}{ぼく}に
\ruby[g]{生命}{いのち }の
\ruby{有}{あ}らん
\ruby{限}{かぎ}りは、
%
\ruby[g]{一日}{いちにち}に
\ruby[g]{一日}{いちにち}だけ
\ruby{此}{こ}の
\ruby{心}{こゝろ}を% 踊り字調整「〻(二の字点、揺すり点)に見えるが(ゝ)」
\ruby{懷}{いだ}いて、
%
\ruby{苦}{くるし}んでも
\ruby{悶}{もだ}えても
\ruby[g]{生存}{ながら }へやうと
\ruby{思}{おも}ふ、
\ruby{此}{こ}の
\ruby{僕}{ぼく}の
\ruby{眞}{しん}の
\ruby{意}{こゝろ}を% 踊り字調整「〻(二の字点、揺すり点)に見えるが(ゝ)」
\ruby{汲}{く}んで
\ruby{吳}{く}れて、
%
\ruby[g]{何樣}{ど う }か
\ruby{僕}{ぼく}を
\ruby{見}{み}
\ruby{放}{はな}さずに
\ruby{居}{ゐ}て
\ruby{吳}{く}れたまへ。
%
\原本頁{278-4}\改行%
\ruby{長}{なが}く
\ruby{此}{こ}の
\ruby{僕}{ぼく}に
\ruby[g]{君等}{きみたち}の
\ruby{友}{とも}たる
\ruby[g]{幸福}{さいはひ}を
\ruby{得}{{\換字{𛀁}}}させて
\ruby{置}{おい}て
\ruby{吳}{く}れたまへ。
%
\ruby[g]{君等}{きみたち}は
\ruby{皆}{みな}
\ruby{優}{やさ}しく
\ruby{敎}{をし}へて
\ruby{吳}{く}れるし、
%
\ruby[g]{自{\換字{分}}}{じ ぶん}でも
\ruby{氣}{き}が
\ruby{付}{つ}いて
\ruby{居}{ゐ}るし、
%
\ruby[<j||]{自}{みづか}ら% 行末行頭の境界付近なので特例処置を施す
\ruby{克}{か}たう
としたり
\ruby[<j||]{自}{みづか}ら% ルビ調整(特殊処理)次の「憤」のルビと離す
\ruby[<j>]{憤}{いきどほ}つたり、
%
\ruby{自}{みづか}ら
\ruby{爭}{あらそ}つたり
\ruby{自}{みづか}ら
\ruby{鬪}{たゝか}つたり、% 踊り字調整「〻(二の字点、揺すり点)に見えるが(ゝ)」
%
\原本頁{278-7}\改行%
\ruby{心}{こゝろ}の% 踊り字調整「〻(二の字点、揺すり点)に見えるが(ゝ)」
\ruby{中}{うち}の
\ruby{揉}{も}めぬ
\ruby{日}{ひ}も
\ruby{無}{な}く、
%
\ruby{力}{ちから}も
\ruby{根}{こん}も
\ruby{使}{つか}ひ
\ruby{盡}{つく}して
\ruby[g]{今日}{け ふ }まで
\ruby{來}{き}たが
\改行% 校正作業の簡略化のため
、
%
\原本頁{278-8}\改行%
\ruby{何}{なん}と
\ruby{無}{な}く
\ruby[g]{行末}{ゆくすゑ}が
\ruby[||j>]{物}{もの}
\ruby[||j>]{怖}{おそろ}しくて、
% \ruby{物怖}{もの|おそろ}しくて、
%
\ruby{知}{し}りつゝ% 踊り字調整「〻(二の字点、揺すり点)に見えるが(ゝ)」
\ruby{高}{たか}い
\ruby{崖}{がけ}から
\ruby{深}{ふか}い
\ruby{淵}{ふち}に
\ruby{陷}{おちい}るやうな
\ruby{時}{とき}が
\ruby{有}{あり}は
せぬかと
\ruby{思}{おも}ふ。
%
\ruby{必}{かなら}ず〳〵
\ruby[g]{其樣}{そ ん }な
ことには
ならぬ
\ruby{樣}{やう}に、
%
\ruby[g]{君等}{きみたち}の
\ruby[g]{厚意}{かうい }を
\ruby{{\換字{空}}}{むな}しく
せぬ
やうにと、
%
\ruby[||j>]{一}{いつ}
\ruby[||j>]{生}{しやう}
% \ruby{一生}{いつ|しやう}
\ruby[||j>]{懸}{ けん}
\ruby[||j>]{命}{ めい}に
\ruby{思}{おも}つては
\原本頁{278-11}\改行%
\ruby{居}{ゐ}るが、
%
\ruby[g]{萬一}{まんいち}
\ruby{萬々一}{まん|〳〵|いち}
\ruby[g]{左樣}{さ う }いふ
\ruby{目}{め}に
あつても、
%
\ruby[g]{屹度}{きつと }
それきり
にはならぬ
つもりの、
%
\ruby[g]{其點}{そ こ }を
\ruby[g]{水野}{みづの }だと
\ruby{見}{み}て
\ruby{吳}{く}れて、
%
あれほど
\ruby{諭}{さと}したのに
\ruby{云}{い}ひ
\ruby[g]{甲{\換字{斐}}}{が ひ }の
\ruby{無}{な}い、
%
とう〳〵
\ruby{深}{ふか}みへ
\ruby{落}{お}ちた
\ruby[g]{馬鹿}{ば か }な
\ruby{奴}{やつ}だと、
%
\ruby{爪}{つま}
\ruby{彈}{はじ}きして
\ruby{棄}{す}てる
やうなことを
\ruby{爲}{し}て
\ruby{吳}{く}れたまふな。
%
\ruby{餘}{あま}り
\ruby{愚}{ぐ}な
\ruby{事}{こと}を
\原本頁{279-4}\改行%
いふやうだが、
%
たゞ% 踊り字調整「〻(二の字点、揺すり点)に濁点に見えるが(ゞ)」
\ruby{何}{なん}と
\ruby{無}{な}く
\ruby{僕}{ぼく}の
\ruby[g]{{\換字{前}}{\換字{途}}}{ぜんと }に
\ruby{恐}{おそ}ろしい
\ruby[g]{不幸}{ふ かう}が
\ruby{手}{て}を
\ruby{擴}{ひろ}げて、
%
\ruby{僕}{ぼく}の
\ruby{行}{ゆ}くのを
\ruby{待}{ま}つて
\ruby{居}{ゐ}るやうに
\ruby{思}{おも}へる。
%
\ruby[g]{何樣}{ど う }も
\ruby[g]{左樣}{さ う }
\ruby{思}{おも}へて
ならんので、
%
それで
\ruby[g]{如是}{こ ん }な
ことも
\ruby{言}{い}ひ
\ruby{出}{いだ}す
のだが、
%
\ruby[g]{何樣}{ど う }
\ruby{罷}{まか}り
\原本頁{279-7}\改行%
\ruby[g]{問{\換字{違}}}{ま ちが}つても
\ruby[g]{本來}{ほんらい}の
\ruby[g]{一心}{いつしん}は、
%
\ruby[g]{君等}{きみたち}に
\ruby{對}{たい}しても
\ruby{决}{けつ}して
\ruby{忘}{わす}れぬ、
%
\ruby[g]{其處}{そ こ }を
たゞ% 踊り字調整「〻(二の字点、揺すり点)に濁点に見えるが(ゞ)」
\ruby[g]{水野}{みづの }だと
\ruby{思}{おも}つて
\ruby[g]{{\換字{交}}際}{つきあ }つて
\ruby{吳}{く}れたまへ。
%
\ruby{人}{ひと}の
\ruby[g]{{\換字{運}}命}{うんめい}の
\ruby[g]{明日}{あした }は
\ruby{{\換字{分}}}{わか}らぬが、
%
\ruby[g]{君等}{きみたち}の
\ruby[g]{厚意}{かうい }は
\ruby{夢}{ゆめ}の
\ruby{間}{ま}も
\ruby{忘}{わす}れぬ。
%
\ruby[g]{君等}{きみたち}に
\ruby{負}{そむ}かぬ
やうにとは
\ruby[g]{屹度}{きつと }
\ruby[g]{努力}{どりよく}する。
』

\原本頁{279-11}%
と、
%
\ruby{心}{こゝろ}に% 踊り字調整「〻(二の字点、揺すり点)に見えるが(ゝ)」
\ruby{張}{は}り
のある
さまは
\ruby{{\換字{猶}}}{なほ}
\ruby{見}{み}え
ながら、
%
\ruby[g]{意氣}{い き }は
\ruby{振}{ふる}はずして
\ruby[g]{龍鍾}{しをしを}と% ルビ調整(原本通り)非踊り字表記(行末行頭の境界付近)
% 「龍鍾(りゅうしょう)」老いて疲れ病むこと。また、うちしおれていること。
% 「しおしお」落ちがして力がぬけた様子。しおれて元気がないさま。
\ruby{言}{い}ふ
\ruby{其}{そ}の
\ruby{哀}{あは}れなる
\ruby[g]{樣子}{やうす }を
\ruby[g]{日方}{ひ かた}は
\ruby[g]{見{\換字{過}}}{み すご}しかね、

\原本頁{280-2}%
『
なに!、
%
\ruby{何}{なん}と
\ruby{無}{な}く
\ruby[g]{行末}{ゆくすゑ}が
\ruby{怖}{おそろ}しくつて、
%
\ruby[g]{不幸}{ふ かう}の
\ruby[g]{{\換字{運}}命}{うんめい}が
\ruby{待}{まつ}て
\ruby{居}{ゐ}る
やうに
\ruby{思}{おも}へるつて?。
%
\ruby{何}{なに}
\ruby[g]{其樣}{そ ん }なことが
\ruby{有}{あ}つて
\ruby{堪}{たま}るものか。
%
\ruby[g]{我々}{われ〳〵}
\原本頁{280-4}\改行%
の
\ruby[g]{行末}{ゆくすゑ}は
\ruby{皆}{みな}
\ruby{輝}{かゞや}いて% 踊り字調整「〻(二の字点、揺すり点)に濁点に見えるが(ゞ)」
\ruby{居}{ゐ}る!。
%
\ruby[g]{我々}{われ〳〵}
\ruby[g]{七人}{しちにん}の% 原本には漢数字「七」のルビ無し
\ruby[g]{行末}{ゆくすゑ}に
\ruby[g]{暗黒}{や み }は
\ruby{無}{な}いのだ!
\改行% 校正作業の簡略化のため
。
%
\原本頁{280-5}\改行%
\ruby{我}{わ}が
\ruby[g]{日本}{に ほん}
\ruby[g]{國民}{こくみん}の
\ruby[g]{{\換字{前}}{\換字{途}}}{ぜんと }には
\ruby[g]{暗黒}{や み }は
\ruby{無}{な}い
のだ!。
%
\ruby{燃}{も}える
\ruby{火}{ひ}の
\ruby{{\換字{前}}}{まへ}に
\ruby[g]{暗黒}{や み }が
\ruby{有}{あ}るかい!。
%
\ruby[g]{暗黒}{や み }は
たゞ% 踊り字調整「〻(二の字点、揺すり点)に濁点に見えるが(ゞ)」
\ruby{{\換字{過}}}{す}ぎた
\ruby[g]{昨日}{きのふ }の
\ruby{事}{こと}!。
%
\ruby{生}{い}きて
\ruby{居}{ゐ}る
\ruby[g]{人間}{にんげん}、
%
\ruby{燃}{も}えて
\ruby{居}{ゐ}る
\ruby{火}{ひ}の、
%
\ruby{其}{その}
\ruby{{\換字{前}}}{まへ}に
\ruby[g]{暗黒}{や み }が
\ruby{有}{あ}るとは
\ruby{誰}{だれ}が
\ruby{言}{い}ふ?。
%
そんな
\ruby{事}{こと}を
\ruby{思}{おも}ふのは
\ruby{氣}{き}の
\ruby{{\換字{迷}}}{まよ}ひだ。
%
\ruby[g]{悉皆}{みんな }
\ruby{汝}{きさま}の
\ruby[g]{衰{\換字{弱}}}{おとろへ}からだ!。
%
しつかり
\原本頁{280-9}\改行%
\ruby{爲}{し}なくては
いかんぞ
\ruby[g]{水野}{みづの }!。
%
\ruby[g]{喇叭}{らつぱ }が
\ruby{{\換字{進}}}{すゝ}めと% 踊り字調整「〻(二の字点、揺すり点)に見えるが(ゝ)」
\ruby{鳴}{な}りやあ
\ruby{敵}{てき}は
もう
\ruby{無}{な}いんだ。
%
\ruby{大{\換字{丈}}夫}{だい|ぢやう|ぶ}の
\ruby{向}{むか}つて
\ruby{行}{ゆ}く
ところには
\ruby[g]{不幸}{ふ かう}も
\ruby{何}{なに}も
\ruby{無}{な}い。
%
\ruby{下}{くだ}らん
ことを
いつて
まだ
\ruby{撲}{なぐ}られたいか。
%
\ruby[g]{羽{\換字{勝}}}{は がち}の
\ruby{言}{げん}に
\ruby{從}{したが}つて
\ruby[g]{努力}{どりよく}して
\原本頁{281-1}\改行%
\ruby{日}{ひ}を
\ruby{{\換字{送}}}{おく}れ。
%
\ruby{汝}{きさま}の
\ruby[g]{{\換字{前}}{\換字{途}}}{ぜんと }の
\ruby[g]{多幸}{た かう}なのは
\ruby[g]{乃公}{お れ }が
\ruby{受}{うけ}
\ruby{合}{あ}ふ。
』

\原本頁{281-2}%
と
\ruby[g]{壯語}{さうご }の
\ruby{有}{あ}る
\ruby{限}{かぎ}りを
\ruby{盡}{つく}して
\ruby{氣}{き}を
\ruby[g]{引立}{ひきた }てたる
\ruby{其}{その}
\ruby{時}{とき}
\ruby[||j>]{室}{しつ}
\ruby[||j>]{外}{ぐわい}に
% \ruby{室外}{しつ|ぐわい}に
\ruby{人}{ひと}の
\ruby[g]{氣色}{け はひ}して、
%
\ruby{忽}{たちま}ち
\ruby{間}{あひ}の
\ruby{襖}{ふすま}は
\ruby[||j>]{右}{みぎ}
\ruby[||j>]{左}{ひだり}に
% \ruby{右左}{みぎ|ひだり}に
\ruby{大}{おほき}く
\ruby{開}{ひら}かれたり。
