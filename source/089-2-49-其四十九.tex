\Entry{其四十九}

\ruby{壽長}{いのち|なが}ければ
\ruby{智慧}{ち|ゑ}
\ruby{多}{おほ}し。
\ruby[g]{吉右衛門}{きちゑもん}は
\ruby{眼}{め}に
\ruby{世}{よ}の
\ruby{人}{ひと}のそれ〴〵を
\ruby{見覺}{み|おぼ}えて、
\ruby{水野}{みづ|の}を
\ruby{今}{いま}に
\ruby{稀}{まれ}なる
\ruby{{\換字{若}}者}{わか|もの}と
\ruby{悅}{よろこ}び、
\ruby{初}{はじめ}はたゞ
\ruby{高田}{たか|た}の
\ruby{依頼}{たの|み}によりて
\ruby{寄寓}{き|ぐう}を
\ruby{許}{ゆる}したるに
\ruby{{\換字{過}}}{す}ぎざりしが、
\ruby{後}{のち}
\ruby{後}{おく}
には
\ruby{其}{そ}の
\ruby{品行}{おこ|なひ}を
\ruby{見}{み}、
\ruby{其}{そ}の
\ruby{人}{ひと}となりを
\ruby{知}{し}つて、
\ruby{之}{これ}を
\ruby{重}{おも}んずることは
\ruby{主}{しゆ}の
\ruby{如}{ごと}く、
\ruby{之}{これ}を
\ruby{思}{おも}ふことは
\ruby{子}{こ}の
\ruby{如}{ごと}く、
\ruby{他人}{た|にん}あしらひにはせずして
\ruby{月日}{つき|ひ}を
\ruby{{\換字{過}}}{すご}し
\ruby{來}{きた}れる
\ruby{程}{ほど}なれば、
\ruby{今}{いま}
\ruby{本家}{ほん|け}より
\ruby{歸}{かへ}り
\ruby{來}{きた}りて、
\ruby{水野}{みづ|の}が
\ruby{許}{もと}に
\ruby{訪}{と}ひ
\ruby{寄}{よ}れる
\ruby{人々}{ひと|〴〵}の、いづれも
\ruby{表面}{うは|べ}ばかりの
\ruby{友}{とも}にはあらずして、
\ruby{水野}{みづ|の}のために
\ruby{或}{あるひ}は
\ruby{諫}{いさ}め
\ruby{或}{あるひ}は
\ruby{諭}{さと}す
\ruby{其}{そ}の
\ruby{一片}{かた|はし}を、ちら〳〵と
\ruby{耳}{みゝ}に
\ruby{入}{い}るゝにつけ、
\ruby{特}{こと}には
\ruby{日方}{ひ|かた}といへるが
\ruby{如何}{い|か}に
\ruby{振舞}{ふる|ま}ひて、また
\ruby{我}{わ}が
\ruby{孫}{まご}の
お
\ruby{濱}{はま}が
\ruby{日方}{ひ|かた}に
\ruby{對}{たい}して
\ruby{如何}{い|か}に
\ruby{振舞}{ふる|まひ}ひしかをも
\ruby{聞}{き}きて
\ruby{知}{し}るにつけ、たゞ
\ruby{其}{そ}のままにはあり
\ruby{得}{\換字{𛀁}}ぬ
\ruby{心地}{こゝ|ち}して、
\ruby{不自由}{ふ|じ|ゆう}なる
\ruby{田舎}{ゐな|か}の
\ruby{心}{こゝろ}には
\ruby{任}{まか}せねど、
お
\ruby{濱}{はま}
お
\ruby{鍋}{なべ}に
\ruby{指揮}{さし|ず}して
\ruby{酒肴}{しゆ|かう}を
\ruby{調}{とゝの}へしめ、
\ruby{水野}{みづ|の}が
\ruby{命令}{いひ|つけ}の
\ruby{無}{な}きにも
\ruby{關}{かゝは}らず、
\ruby{其座}{その|ざ}に
\ruby{其}{それ}を
\ruby{持出}{もち|いだ}さしめたり。
\ruby{老人}{らう|じん}の
\ruby{親切}{しん|せつ}なる
\ruby{心}{こゝろ}より、
\ruby{此頃}{この|ごろ}の
\ruby{水野}{みづ|の}の
\ruby{擧動}{ふる|まひ}を
\ruby{憂}{うれ}ひ
\ruby{居}{ゐ}し
\ruby{矢先}{や|さき}に、
\ruby{我}{わ}が
\ruby{心}{こゝろ}を
\ruby{得}{\換字{𛀁}}たる
\ruby{二人}{ふた|り}の
\ruby{客}{きやく}の
\ruby{物語}{もの|がたり}をば、
\ruby{一}{ひ}ト
\ruby{方}{かた}ならず
\ruby{嬉}{うれ}しく
\ruby{思}{おも}へる
\ruby{餘}{あま}りなるべし。

\ruby{何}{なん}の
\ruby{馳走}{ち|そう}も
\ruby{無}{な}き
\ruby{饗應}{もて|なし}なれど、
\ruby{膳}{ぜん}を
\ruby{配}{くば}らせながら
\ruby[g]{吉右衛門}{きちゑもん}は
\ruby{笑}{ゑ}みつ、

『どなだも
\ruby{邊鄙}{へん|ぴ}のところへ
\ruby{好}{よ}く
\ruby{御來臨}{お|い|で}なさいました、
\ruby{私}{わたくし}は
\ruby{此家}{こ|ゝ}の
\ruby{老夫}{おや|ぢ}でございますが、
\ruby{此}{こ}の
\ruby{兀}{は}げたところをでも
\ruby{今後御覺}{これ|から|お|おぼ}え
\ruby{願}{ねが}ひます。
\ruby{島木}{しま|き}さんには
\ruby{御心易}{お|こゝろ|やす}く
\ruby{願}{ねが}つて
\ruby{居}{を}ります、
\ruby{折角}{せつ|かく}
\ruby{諸君}{みな|さん}が
\ruby{來臨下}{おい|で|くだ}すつたのですから、』

と
\ruby{云}{い}ひかけて
\ruby{一寸}{ちよ|つと}
\ruby{水野}{みづ|の}を
\ruby{見}{み}て、

『お
\ruby{差圖}{さし|づ}も
\ruby{伺}{うかゞ}ひませんでしたが、
\ruby{御談話}{お|はな|し}の
\ruby{繋}{つな}ぎのためばかりに、
\ruby{一獻}{ひと|つ}あげるやうに
\ruby{致}{いた}しました。
\ruby{田舎}{ゐな|か}の
\ruby{事}{こと}ですから
\ruby{何}{なに}もございません。
おまけに
\ruby{飮酒家}{や|り|て}の
\ruby{無}{な}い
\ruby{家}{うち}の
\ruby{事}{こと}でございますから、
\ruby{御惣菜}{お|そう|ざい}みたやうなものばかりで、
\ruby{氣取}{き|どり}も
\ruby{何}{なに}もございませんが、まあ
\ruby{何}{なに}も
\ruby{御笑}{お|わら}ひ
\ruby{草}{ぐさ}になすつて
\ruby{飮}{あが}つて
\ruby{下}{くだ}さいまし。
\ruby{日方}{ひ|かた}さんへは
\ruby{御謝罪}{お|わ|び}の
\ruby{印}{しるし}と
\ruby{申}{まを}しましても
\ruby{宜}{よ}いので、
\ruby{孫}{まご}めが
\ruby{飛}{と}んだ
\ruby{失禮}{しつ|れい}を
\ruby{致}{いた}しましが、
\ruby{何樣}{ど|う}か
\ruby{御勘辯}{ご|かん|べん}
\ruby{下}{くだ}さいまし、
\ruby{其代}{その|かわ}り
\ruby{澤山}{たん|と}
\ruby{御酌}{お|しやく}をさせますから、ハヽヽ。
これお
\ruby{濱}{はま}こゝへ
\ruby{來}{き}て
\ruby{御謝罪}{お|わ|び}を
\ruby{仕}{し}ろ。
』

と
\ruby{云}{い}へば、
\ruby{其}{そ}の
\ruby{背後}{うし|ろ}に
\ruby{小}{ちひさ}くなり
\ruby{居}{ゐ}し
お
\ruby{濱}{はま}は、
\ruby{面}{おもて}を
\ruby{染}{そ}めて
\ruby{是非無}{ぜ|ひ|な}く
\ruby{頭}{かうべ}を
\ruby{下}{さ}げんとす。
\ruby{日方}{ひ|かた}は
\ruby{老{\換字{父}}}{ぢ|ゞ}の
\ruby{言}{ことば}を
\ruby{心地}{こゝ|ち}
\ruby{快}{よ}げに
\ruby{聞}{き}き
\ruby{居}{ゐ}しが

『ハヽヽ。
\ruby{君}{きみ}、なに、
\ruby{謝罪}{あや|ま}らんでも
\ruby{可}{い}いさ。
お
\ruby{濱}{はま}さんといふかね、
\ruby{好}{い}い
\ruby{氣象}{きし|やう}の
\ruby{娘}{むすめ}さんだ。
\ruby{日方八郎}{ひ|かた|はち|らう}
\ruby{生}{ゝま}れて
\ruby{初}{はじ}めて
\ruby{頭}{あたま}へ
\ruby{手}{て}を
\ruby{上}{あ}げられたが、
\ruby{打}{ぶ}たれて
\ruby{怒}{おこ}るどころではない、
\ruby{全然}{すつ|かり}
\ruby{感心}{かん|しん}した。
\ruby{日本}{につ|ぽん}の
\ruby{{\換字{婦}}女}{をん|な}は
\ruby{誰}{たれ}も
\ruby{彼}{かれ}も、
お
\ruby{濱}{はま}さんのやうな
\ruby{氣合}{き|あひ}で
\ruby{居}{ゐ}て
\ruby{欲}{ほ}しい。
\ruby{偉}{\換字{𛀁}ら}い
\ruby{娘}{むすめ}さんだ、
\ruby{好}{い}い
\ruby{氣象}{きし|やう}だ。
\ruby{祖{\換字{父}}}{お|ぢい}さんに
\ruby{何}{なん}か
\ruby{云}{い}はれたつて
\ruby{頭}{あたま}なんか
\ruby{下}{さ}げてはいかん。
\ruby{其代}{その|かわ}り
\ruby{御酌}{お|しやく}は
\ruby{御{\換字{遠}}慮無}{ご|ゑん|りよ|な}しに
\ruby{願}{ねが}はう。
ハヽヽ。
』

と
\ruby{無邪氣}{む|じや|き}に
\ruby{制}{せい}し
\ruby{止}{とゞ}めたり。

『
\ruby{左樣}{さ|う}
\ruby{仰}{おつし}あつて
\ruby{下}{くだ}されば
\ruby{先}{ま}づ
\ruby{老夫}{ぢゞ|い}も
\ruby{助}{たす}かります。
\ruby{何樣}{ど|う}か
\ruby{御機{\換字{嫌}}}{ご|き|げん}
\ruby{好}{よ}く
\ruby{御談}{お|はな}しなすつて。
\ruby{兀頭}{はげ|あたま}は
\ruby{{\換字{古}}風}{むか|し}
\ruby{物}{もの}で
\ruby{時代{\換字{違}}}{じ|だい|ちが}ひですから、
\ruby{御{\換字{若}}}{お|わか}い
\ruby{方}{かた}の
\ruby{中}{なか}では
\ruby{氣}{き}が
\ruby{退}{ひ}けてなりません。
\ruby{御免蒙}{ご|めん|かうむ}りますから
\ruby{御寛}{ご|ゆる}りと。
』

『イヤ
\ruby{左樣}{さ|う}で
\ruby{無}{な}い。
\ruby{君}{きみ}は
\ruby{中々話}{なか|〳〵|はな}せる。
いゝぢや
\ruby{無}{な}いか
\ruby{老翁}{おぢい|さん}、ここに
\ruby{居}{ゐ}たまヘナ。
』

『ハヽヽ、
\ruby{有}{あ}り
\ruby{難}{がた}うございますが
\ruby{萬一}{ひよ|つと}
\ruby{何樣}{ゞ|ん}な
\ruby{事}{こと}でか
\ruby{叱}{しか}られまして、
\ruby{{\換字{若}}}{も}し
\ruby{御卷骨}{お|げん|こつ}を
\ruby{頂戴}{ちやう|だい}しますと、
\ruby{兀頭}{は|げ}は
\ruby{特別}{とく|べつ}に
\ruby{利}{き}きますからナ。
まあ
\ruby{引{\換字{退}}}{ひき|さが}つて
\ruby{居}{ゐ}る
\ruby{方}{はう}が
\ruby{無難}{ぶ|なん}でございます。
ハヽヽ、イヤこれは
\ruby{冗談}{じやう|だん}を、
\ruby{失禮}{しつ|れい}いたしました。
』

\ruby[g]{吉右衛門}{きちゑもん}は
\ruby{{\換字{終}}}{つひ}に
\ruby{彼方}{かな|た}へ
\ruby{去}{さ}れば、
\ruby{日方}{ひ|かた}は
\ruby{羽{\換字{勝}}}{は|がち}と
\ruby{相見}{あひ|み}て
\ruby{笑}{わら}つて、

『
\ruby{好}{い}い
\ruby{老夫}{おぢい|さん}だナア。
\ruby{如何}{い|か}にも
\ruby{奇麗}{き|れい}な
\ruby{輕}{かる}い
\ruby{調子}{てう|し}で、そして
\ruby{親切}{しん|せつ}に
\ruby{滿}{み}ちて
\ruby{居}{ゐ}る、
\ruby{{\換字{透}}徹}{すき|とほ}るやうな
\ruby{人}{ひと}だナ。
』

『
\ruby{左樣}{さ|う}だ。
まだ
\ruby{我々}{われ|〳〵}の
\ruby{及}{およ}ばんところがある。
』

と
\ruby{{\換字{評}}}{ひやう}し
\ruby{合}{あ}つて
\ruby{樂}{たの}しげに
\ruby{酒盞}{さか|づき}を
\ruby{擧}{あ}げたり。

『ハヽヽ、
\ruby{乃公}{お|れ}ぐらゐ
\ruby{能}{よ}く
\ruby{飮}{の}む
\ruby{奴}{やつ}はあるまい。
\ruby{何}{なん}だか
\ruby{老人}{おぢい|さん}が
\ruby{出}{で}て
\ruby{來}{き}たので
\ruby{甚}{ひど}く
\ruby{氣}{き}が
\ruby{和}{やはら}いで、
\ruby{何程}{いく|ら}でも
\ruby{悠然}{ゆつ|くり}と
\ruby{飮}{の}めさうなやうな
\ruby{心持}{こゝろ|もち}になつて
\ruby{來}{き}た。
』

