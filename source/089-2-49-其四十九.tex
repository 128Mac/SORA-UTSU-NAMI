\Entry{其四十九}

% メモ 校正終了 2024-05-08 2024-06-05
\原本頁{281-5}%
\ruby[||j>]{壽}{いのち}
\ruby[||j>]{長}{ なが}ければ
% \ruby{壽長}{いのち|なが}ければ
\ruby[g]{智慧}{ち ゑ }
\ruby{多}{おほ}し。
%
\ruby{吉右衛門}{きち||ゑ|もん}は
\ruby{眼}{め}に
\ruby{世}{よ}の
\ruby{人}{ひと}の
それ〴〵を
\ruby{見}{み}
\ruby{覺}{おぼ}えて、
%
\ruby[g]{水野}{みづの }を
\ruby{今}{いま}に
\ruby{稀}{まれ}なる
\ruby[g]{{\換字{若}}者}{わかもの}と
\ruby{悅}{よろこ}び、
%
\ruby{初}{はじめ}は
たゞ% 踊り字調整「〻(二の字点、揺すり点)に濁点に見えるが(ゞ)」
\ruby[g]{高田}{たかた }の
\ruby[g]{依頼}{たのみ }に
よりて
\ruby[g]{寄寓}{き ぐう}を
\ruby{許}{ゆる}したるに
\ruby{{\換字{過}}}{す}ぎ
ざりしが、
%
\ruby{後}{のち}
には
\ruby{其}{そ}の
\ruby[g]{品行}{おこなひ}を
\ruby{見}{み}、
%
\ruby{其}{そ}の
\ruby{人}{ひと}と
なりを
\ruby{知}{し}つて、
%
\ruby{之}{これ}を
\ruby{重}{おも}んずる
ことは
\ruby{主}{しゆ}の
\ruby{如}{ごと}く、
%
\ruby{之}{これ}を
\ruby{思}{おも}ふ
ことは
\ruby{子}{こ}の
\ruby{如}{ごと}く、
%
\ruby[g]{他人}{た にん}
あしらひ
には
せずして
\ruby[g]{月日}{つきひ }を
\ruby{{\換字{過}}}{すご}し
\ruby{來}{きた}れる
\原本頁{281-10}\改行%
\ruby{程}{ほど}
なれば、
%
\ruby{今}{いま}
\ruby[g]{本家}{ほんけ }より
\ruby{歸}{かへ}り
\ruby{來}{きた}りて、
%
\ruby[g]{水野}{みづの }が
\ruby{許}{もと}に
\ruby{訪}{と}ひ
\ruby{寄}{よ}れる
\ruby[g]{人々}{ひと〴〵}の、
%
いづれも
\ruby[g]{表面}{うはべ }
ばかりの
\ruby{友}{とも}には
あらずして、
%
\ruby[g]{水野}{みづの }の
ために
\ruby[<j||]{或}{あるひ}は% 行末行頭の境界付近なので特例処置を施す
\ruby{諫}{いさ}め
\ruby{或}{あるひ}は
\ruby{諭}{さと}す
\ruby{其}{そ}の
\ruby[g]{一片}{かたはし}を、
%
ちら〳〵と
\ruby{耳}{みゝ}に% 踊り字調整「〻(二の字点、揺すり点)に見えるが(ゝ)」
\ruby{入}{い}るゝにつけ、% 踊り字調整「〻(二の字点、揺すり点)に見えるが(ゝ)」
%
\ruby{特}{こと}には
\ruby[g]{日方}{ひ かた}と
いへるが
\ruby[g]{如何}{い か }に
\ruby{振}{ふる}
\ruby{舞}{ま}ひて、
%
また
\ruby{我}{わ}が
\ruby{孫}{まご}の
お
\ruby{濱}{はま}が
\ruby[g]{日方}{ひ かた}に
\ruby{對}{たい}して
\ruby[g]{如何}{い か }に
\ruby{振}{ふる}
\ruby{舞}{まひ}ひし
かをも
\ruby{聞}{き}きて
\ruby{知}{し}るにつけ、
%
たゞ% 踊り字調整「〻(二の字点、揺すり点)に濁点に見えるが(ゞ)」
\ruby{其}{そ}のまま
には
あり
\ruby{得}{{\換字{𛀁}}}ぬ
\ruby[g]{心地}{こゝち }して、% 踊り字調整「〻(二の字点、揺すり点)に見えるが(ゝ)」
%
\ruby{不自由}{ふ|じ|ゆう}なる
\ruby[g]{田舎}{ゐなか }の
\ruby{心}{こゝろ}には% 踊り字調整「〻(二の字点、揺すり点)に見えるが(ゝ)」
\ruby{任}{まか}せねど、
%
\原本頁{282-6}\改行%
お
\ruby{濱}{はま}
お
\ruby{鍋}{なべ}に
\ruby[g]{指揮}{さしづ }して
\ruby[g]{酒肴}{しゆかう}を
\ruby{調}{とゝの}へ% 踊り字調整「〻(二の字点、揺すり点)に見えるが(ゝ)」
しめ、
%
\ruby[g]{水野}{みづの }が
\ruby[g]{命令}{いひつけ}の
\ruby{無}{な}きにも
\ruby[<j||]{關}{かゝは}らず、% 踊り字調整「〻(二の字点、揺すり点)に見えるが(ゝ)」% 行末行頭の境界付近なので特例処置を施す
%
\ruby{其}{その}
\ruby{座}{ざ}に
\ruby{其}{それ}を
\ruby{持}{もち}
\ruby{出}{いだ}さし
めたり。
%
\ruby[g]{老人}{らうじん}の
\ruby[g]{親切}{しんせつ}なる
\ruby{心}{こゝろ}より、% 踊り字調整「〻(二の字点、揺すり点)に見えるが(ゝ)」
%
\ruby{此}{この}
\ruby{頃}{ごろ}の
\ruby[g]{水野}{みづの }の
\ruby[g]{擧動}{ふるまひ}を
\ruby{憂}{うれ}ひ
\ruby{居}{ゐ}し
\ruby[g]{矢先}{や さき}に、
%
\ruby{我}{わ}が
\ruby{心}{こゝろ}を% 踊り字調整「〻(二の字点、揺すり点)に見えるが(ゝ)」
\ruby{得}{{\換字{𛀁}}}たる
\ruby[g]{二人}{ふたり }の
\ruby{客}{きやく}の
\ruby[||j>]{物}{もの}
\ruby[||j>]{語}{がたり}をば、
% \ruby{物語}{もの|がたり}をば、
%
\ruby{一}{ひ}
ト
\ruby{方}{かた}
ならず
\ruby{嬉}{うれ}しく
\ruby{思}{おも}へる
\ruby{餘}{あま}りなるべし。

\原本頁{282-10}%
\ruby{何}{なん}の
\ruby[g]{馳走}{ち そう}も
\ruby{無}{な}き
\ruby[g]{饗應}{もてなし}なれど、
%
\ruby{膳}{ぜん}を
\ruby{配}{くば}らせ
ながら
\ruby{吉右衛門}{きち||ゑ|もん}は
\ruby{笑}{ゑ}み
\改行% 校正作業の簡略化のため
つ、
%

\原本頁{283-1}%
『
どなたも
\ruby[g]{邊鄙}{へんぴ }の
ところへ
\ruby{好}{よ}く
\ruby{御來臨}{お|い|で}なさいました、
%
\ruby[<j>]{私}{わたくし}は
\ruby[g]{此家}{こ ゝ }の% 踊り字調整「〻(二の字点、揺すり点)に見えるが(ゝ)」
\ruby[g]{老夫}{おやぢ }で
ございますが、
%
\ruby{此}{こ}の
\ruby{兀}{は}げた
ところ
をでも
\ruby[g]{今後}{これから}
\ruby[g]{御覺}{お おぼ}え
\ruby{願}{ねが}ひます。
%
\ruby[g]{島木}{しまき }さん
には
\ruby{御心易}{お|こゝろ|やす}く% 踊り字調整「〻(二の字点、揺すり点)に見えるが(ゝ)」
\ruby{願}{ねが}つて
\ruby{居}{を}ります、
%
\ruby[g]{折角}{せつかく}
\ruby[g]{諸君}{みなさん}が
\ruby{御來臨}{お|い|で}
\ruby{下}{くだ}すつた
のですから、
』

\原本頁{283-5}%
と
\ruby{云}{い}ひかけて
\ruby[g]{一寸}{ちよつと}
\ruby[g]{水野}{みづの }を
\ruby{見}{み}て、

\原本頁{283-6}%
『
お
\ruby[g]{差圖}{さしづ }も
\ruby{伺}{うかゞ}ひません% 踊り字調整「〻(二の字点、揺すり点)に濁点に見えるが(ゞ)」
でしたが、
%
\ruby{御談話}{お|はな|し}の
\ruby{繋}{つな}ぎ
のため
ばかりに、
%
\ruby[g]{一獻}{ひとつ }
あげるやうに
\ruby{致}{いた}しました。
%
\ruby[g]{田舎}{ゐなか }の
\ruby{事}{こと}ですから
\ruby{何}{なに}も
ございません。
%
おまけに
\ruby{飮酒家}{や|り|て}の
\ruby{無}{な}い
\ruby{家}{うち}の
\ruby{事}{こと}で
ございますから、
%
\ruby{御惣{\換字{菜}}}{お|そう|ざい}
みたやうな
ものばかりで、
%
\ruby[g]{氣取}{き どり}も
\ruby{何}{なに}も
ございませんが、
%
まあ
\ruby{何}{なに}も
\ruby[g]{御笑}{お わら}ひ
\ruby{草}{ぐさ}に
なすつて
\ruby{飮}{あが}つて
\ruby{下}{くだ}さいまし。
%
\ruby[g]{日方}{ひ かた}さんへは
\ruby{御謝罪}{お|わ|び}の
\ruby{印}{しるし}と
\ruby{申}{まを}しましても
\ruby{宜}{よ}いので、
%
\ruby{孫}{まご}めが
\ruby{飛}{と}んだ
\ruby[g]{失禮}{しつれい}を
\ruby{致}{いた}しましたが、
%
\ruby[g]{何樣}{ど う }か
\ruby{御勘辨}{ご|かん|べん}% 弁 瓣 辦 辧 (辨) 辩 辯
\ruby{下}{くだ}さいまし、
%
\ruby{其}{その}
\ruby{代}{かは}り
\ruby[g]{澤山}{たんと }
\ruby[g]{御{\換字{酌}}}{おしやく}を
させますから
\改行% 校正作業の簡略化のため
、
%
\原本頁{284-2}\改行%
ハヽヽ。
%
これ
お
\ruby{濱}{はま}
こゝへ% 踊り字調整「〻(二の字点、揺すり点)に見えるが(ゝ)」
\ruby{來}{き}て
\ruby{御謝罪}{お|わ|び}を
\ruby{仕}{し}ろ。
』

\原本頁{284-3}%
と
\ruby{云}{い}へば、
%
\ruby{其}{そ}の
\ruby[g]{背後}{うしろ }に
\ruby{小}{ちひさ}くなり
\ruby{居}{ゐ}し
お
\ruby{濱}{はま}は、
%
\ruby{面}{おもて}を
\ruby{染}{そ}めて
\ruby[g]{是非}{ぜ ひ }
\ruby{無}{な}く
\ruby{頭}{かうべ}を
\ruby{下}{さ}げんとす。
%
\ruby[g]{日方}{ひ かた}は
\ruby[g]{老{\換字{父}}}{ぢ ゞ }の% 踊り字調整「〻(二の字点、揺すり点)に濁点に見えるが(ゞ)」
\ruby{言}{ことば}を
\ruby[g]{心地}{こゝち }% 踊り字調整「〻(二の字点、揺すり点)に見えるが(ゝ)」
\ruby{快}{よ}げに
\ruby{聞}{き}き
\ruby{居}{ゐ}しが、

\原本頁{284-5}%
『
ハヽヽ。
%
\ruby{君}{きみ}、
%
なに、
%
\ruby[g]{謝罪}{あやま }らんでも
\ruby{可}{い}いさ。
%
お
\ruby{濱}{はま}さん
といふかね、
%
\ruby{好}{い}い
\ruby[g]{氣象}{きしやう}の
\ruby{娘}{むすめ}さんだ。
%
\ruby[g]{日方}{ひ かた}
\ruby[g]{八郎}{はちらう}
\ruby{生}{ゝま}れて% 踊り字調整「〻(二の字点、揺すり点)に見えるが(ゝ)」
\ruby{初}{はじ}めて
\ruby{頭}{あたま}へ
\ruby{手}{て}を
\ruby{上}{あ}げられたが、
%
\ruby{打}{ぶ}たれて
\ruby{怒}{おこ}る
どころ
ではない、
%
\ruby[g]{全然}{すつかり}
\ruby[g]{{\換字{感}}心}{かんしん}した。
%
\ruby[g]{日本}{につぽん}の
\ruby[g]{{\換字{婦}}女}{をんな }は
\ruby{誰}{たれ}も
\ruby{彼}{かれ}も、
%
お
\ruby{濱}{はま}さん
のやうな
\ruby[g]{氣合}{き あひ}で
\ruby{居}{ゐ}て
\ruby{欲}{ほ}しい。
%
\ruby{偉}{{\換字{𛀁}}ら}い
\原本頁{284-9}\改行%
\ruby{娘}{むすめ}さんだ、
%
\ruby{好}{い}い
\ruby[g]{氣象}{きしやう}だ。
%
\ruby[g]{祖{\換字{父}}}{お ぢい}さんに
\ruby{何}{なん}か
\ruby{云}{い}はれたつて
\ruby{頭}{あたま}なんか
\ruby{下}{さ}げては
いかん。
%
\ruby{其}{その}
\ruby{代}{かは}り
\ruby[g]{御{\換字{酌}}}{おしやく}は
\ruby{御{\換字{遠}}慮}{ご|ゑん|りよ}
\ruby{無}{な}しに
\ruby{願}{ねが}はう。
%
ハヽヽ。
』

\原本頁{284-11}%
と
\ruby{無邪氣}{む|じや|き}に
\ruby{制}{せい}し
\ruby{止}{とゞ}めたり。% 踊り字調整「〻(二の字点、揺すり点)に濁点に見えるが(ゞ)」

\原本頁{285-1}%
『
\ruby[g]{左樣}{さ う }
\ruby{仰}{おつし}あつて
\ruby{下}{くだ}されば
\ruby{先}{ま}づ
\ruby[g]{老夫}{ぢゞい }も% 踊り字調整「〻(二の字点、揺すり点)に濁点に見えるが(ゞ)」
\ruby{助}{たす}かります。
%
\ruby[g]{何樣}{ど う }か
\ruby{御機{\換字{嫌}}}{ご|き|げん}
\ruby{好}{よ}く
\ruby[g]{御談}{お はな}し
なすつて。
%
\ruby[||j>]{兀}{はげ}
\ruby[||j>]{頭}{あたま}は
% \ruby{兀頭}{はげ|あたま}は
\ruby[g]{{\換字{古}}風}{むかし }
\ruby{物}{もの}で
\ruby[g]{時代}{じ だい}
\ruby{{\換字{違}}}{ちが}ひ
ですから、
%
\ruby[g]{御{\換字{若}}}{お わか}い
\ruby{方}{かた}の
\ruby{中}{なか}では
\ruby{氣}{き}が
\ruby{{\換字{退}}}{ひ}けて
なりません。
%
\ruby[g]{御免}{ご めん}
\ruby{蒙}{かうむ}りますから
\ruby[g]{御寛}{ご ゆる}り
\改行% 校正作業の簡略化のため
と。
』

\原本頁{285-5}%
『
イヤ
\ruby[g]{左樣}{さ う }で
\ruby{無}{な}い。
%
\ruby{君}{きみ}は
\ruby[g]{中々}{なか〳〵}
\ruby{話}{はな}せる。
%
いゝぢや% 踊り字調整「〻(二の字点、揺すり点)に見えるが(ゝ)」
\ruby{無}{な}いか
\ruby[||j>]{老}{おぢい}
\ruby[||j>]{{\換字{翁}}}{ さん}、
% \ruby{老{\換字{翁}}}{おぢい|さん}、
%
ここに
\ruby{居}{ゐ}たまヘナ。
』

\原本頁{285-7}%
『
ハヽヽ、
%
\ruby{有}{あ}り
\ruby{{\換字{難}}}{がた}う
ございますが
\ruby[g]{萬一}{ひよつと}
\ruby[g]{何樣}{ゞ ん }な% 踊り字調整「〻(二の字点、揺すり点)に濁点に見えるが(ゞ)」
\ruby{事}{こと}でか
\ruby{叱}{しか}られまして、
%
\ruby{{\換字{若}}}{も}し
\ruby{御卷骨}{お|げん|こつ}を
\ruby[||j>]{頂}{ちやう}
\ruby[||j>]{戴}{ だい}
% \ruby{頂戴}{ちやう|だい}
しますと、
%
\ruby[g]{兀頭}{は げ }は
\ruby[g]{特別}{とくべつ}に
\ruby{利}{き}きますからナ
\改行% 校正作業の簡略化のため
。
%
\原本頁{285-9}\改行%
まあ
\ruby{引}{ひき}
\ruby{{\換字{退}}}{さが}つて
\ruby{居}{ゐ}る
\ruby{方}{はう}が
\ruby[g]{無{\換字{難}}}{ぶ なん}で
ございます。
%
ハヽヽ、
%
イヤ
これは
\ruby[||j>]{冗}{じやう}
\ruby[||j>]{談}{ だん}を、
% \ruby{冗談}{じやう|だん}を、
%
\ruby[g]{失禮}{しつれい}
いたしました。
』

\原本頁{285-11}%
\ruby{吉右衛門}{きち||ゑ|もん}は
\ruby{{\換字{終}}}{つひ}に
\ruby[g]{彼方}{かなた }へ
\ruby{去}{さ}れば、
%
\ruby[g]{日方}{ひ かた}は
\ruby[g]{羽{\換字{勝}}}{は がち}と
\ruby{相}{あひ}
\ruby{見}{み}て
\ruby{笑}{わら}つて、

\原本頁{286-1}%
『
\ruby{好}{い}い
\ruby[||j>]{老}{おぢい}
\ruby[||j>]{夫}{ さん}だナア。
% \ruby{老夫}{おぢい|さん}だナア。
%
\ruby[g]{如何}{い か }にも
\ruby[g]{奇麗}{き れい}な
\ruby{輕}{かる}い
\ruby[g]{調子}{てうし }で、
%
そして
\ruby[g]{親切}{しんせつ}に
\ruby{滿}{み}ちて
\ruby{居}{ゐ}る、
%
\ruby[g]{{\換字{透}}徹}{すきとほ}る
やうな
\ruby{人}{ひと}だナ。
』

\原本頁{286-3}%
『
\ruby[g]{左樣}{さ う }だ。
%
まだ
\ruby[g]{我々}{われ〳〵}の
\ruby{及}{およ}ばん
ところが
ある。
』

\原本頁{286-4}%
と
\ruby{{\換字{評}}}{ひやう}し
\ruby{合}{あ}つて
\ruby{樂}{たの}しげに
\ruby[g]{酒盞}{さかづき}を
\ruby{擧}{あ}げたり。

\原本頁{286-5}%
『
ハヽヽ、
%
\ruby[g]{乃公}{お れ }
ぐらゐ
\ruby{能}{よ}く
\ruby{飮}{の}む
\ruby{奴}{やつ}は
あるまい。
%
\ruby{何}{なん}だか
\ruby[||j>]{老}{おぢい}
\ruby[||j>]{人}{ さん}が
% \ruby{老人}{おぢい|さん}が
\ruby{出}{で}て
\ruby{來}{き}たので
\ruby{甚}{ひど}く
\ruby{氣}{き}が
\ruby{和}{やはら}いで、
%
\ruby[g]{何程}{いくら }でも
\ruby[g]{悠然}{ゆつくり}と
\ruby{飮}{の}め
さうな
やうな
\ruby[||j>]{心}{こゝろ}% 踊り字調整「〻(二の字点、揺すり点)に見えるが(ゝ)」
\ruby[||j>]{持}{ もち}
% \ruby{心持}{こゝろ|もち}% 踊り字調整「〻(二の字点、揺すり点)に見えるが(ゝ)」
になつて
\ruby{來}{き}た。
』
