\Entry{其五十}

\ruby{言}{ものい}はねども
\ruby{花}{はな}あれば
\ruby{野}{の}は
\ruby{自}{おのづ}から
\ruby{春}{はる}なり。
あどけ
\ruby{無}{な}きお
\ruby{濱一人}{はま|ひ|とり}の
\ruby{交}{まじ}りたるに
\ruby{一座}{いち|ざ}は
\ruby{和}{やはら}ぎて
\ruby{理屈}{り|くつ}を
\ruby{離}{はな}るれば
\ruby[g]{談話}{はなし}に
\ruby[g]{角無}{かどな}く、
\ruby{笑聲漸}{せう|せい|やうや}く
\ruby{起}{おこ}れば
\ruby{酒}{さけ}の
\ruby{味饒}{あぢは|ひおほ}く、
\ruby{謹嚴}{きん|ごん}の
\ruby[g]{{\換字{羽}\換字{勝}}}{はがち}、
\ruby{沈鬱}{ちん|うつ}せる
\ruby[g]{水野}{みづの}さへ、
\ruby{何時}{い|つ}か
\ruby{六七年}{ろく|しち|ねん}の
\ruby[g]{往時}{むかし}に
\ruby{復}{かへ}りて、
\ruby{心}{こヽろ}は
\ruby{若}{わか}く
\ruby{氣}{き}は
\ruby{易}{やす}く
\ruby{語}{かた}らへば、まして
\ruby[g]{日方}{ひかた}は
\ruby{興}{きよう}に
\ruby{入}{い}りて、
\ruby[g]{{\換字{羽}\換字{勝}}}{はがち}の
\ruby{斥}{しりぞ}けたる
\ruby{天眞爛漫}{てん|しん|らん|まん}、
\ruby[g]{醉態淋漓}{すいたいりんり}として
\ruby{受}{う}けては
\ruby{飮}{の}み
\ruby{受}{う}けては
\ruby{飮}{の}み、

『
\ruby{島木}{しま|き}、
\ruby{馬鹿野郎}{ば|か|や|らう}、
\ruby{一緒}{いつ|しよ}に
\ruby{來}{く}れば
\ruby{宜}{い}いのに。
\ruby{金儲}{かね|まうけ}に
\ruby{忙}{いそが}しがつたつて
\ruby{何}{なん}になるものか。
』

と
\ruby{幾度}{いく|たび}か
\ruby{繰}{く}り
\ruby{返}{かへ}して
\ruby{罵}{のヽし}つては、
\ruby{{\換字{叉}}餘念}{また|よ|ねん}も
\ruby{無}{な}く
\ruby[g]{二人}{ふたり}を
\ruby{相手}{あひ|て}に
\ruby{談笑}{だん|せう}して
\ruby{盃}{さかづき}を
\ruby{手}{て}にしたり。

『お
\ruby{濱}{はま}さん、その
\ruby{色}{いろ}の
\ruby{黑}{くろ}い
\ruby[g]{眞面目老夫}{まじめおやぢ}の
\ruby[g]{{\換字{羽}\換字{勝}}}{はがち}に
\ruby{飮}{の}ませて
\ruby{{\GWI{u9063-k}}}{や}つて
\ruby{{\換字{呉}}}{く}れたまへ。
コラ
\ruby[g]{{\換字{羽}\換字{勝}}}{はがち}!、
\ruby{飮}{の}まんかい、
\ruby[g]{水野}{みづの}の
\ruby{妹}{いもうと}の
\ruby{酌}{しやく}だ。
ハヽハ
\ruby{{\換字{船}}}{ふね}では
\ruby{成}{な}るべく
\ruby{酒}{さけ}を
\ruby{用}{もち}ゐん
\ruby{習慣}{く|せ}を
\ruby{付}{つ}けて
\ruby{居}{ゐ}るから
\ruby{飮}{の}めんなぞといふのは
\ruby{虛言}{う|そ}だらう。
\ruby{{\換字{船}}員}{ふな|のり}は
\ruby{大抵善}{たい|てい|よ}く
\ruby{飮}{の}むといふぞ。
』

『イヤもういかん。
\ruby{虛言}{う|そ}では
\ruby{無}{な}い、
\ruby{{\換字{船}}}{ふね}では
\ruby{成}{な}るべく
\ruby{用}{もち}ゐんやうにして
\ruby{居}{ゐ}るのだ。
\ruby[g]{執務}{しつむ}の
\ruby[g]{不確實}{ふかくじつ}になる
\ruby{基}{もとゐ}だから
\ruby{飮酒}{いん|しゆ}は
\ruby{忌}{い}む。
これは
\ruby{海員}{かい|ゐん}の
\ruby{精神}{せい|しん}の
\ruby[g]{{\GWI{u9032-k}}歩}{しんぽ}した
\ruby{趨勢}{すう|せい}で、
\ruby[g]{古來}{こらい}の
\ruby{海員}{かい|ゐん}の
\ruby{飮酒}{いん|しゆ}に
\ruby{耽}{ふけ}つた
\ruby{惡{\換字{習}}}{あく|しふ}を
\ruby{洗}{あら}ふ
\ruby{任}{にん}は
\ruby[g]{我々}{われ〳〵}の
\ruby{肩}{かた}にあるのだ。
だから
\ruby{實際僕}{じつ|さい|ぼく}なぞは
\ruby{餘}{あま}り
\ruby{用}{もち}ゐん。
しかし
\ruby{非常}{ひ|じやう}な
\ruby{暴風雨}{ぼう|ふう|ヽ}の
\ruby{時}{とき}、
\ruby[g]{襯衣}{シヤツ}まで
\ruby{濡}{ぬ}れ
\ruby{浸}{ひた}りながら
\ruby{困苦極}{こん|く|きは}まる
\ruby{勞働}{らう|どう}を
\ruby{仕}{し}た
\ruby{後}{あと}などでは、
\ruby{水夫等}{すい|ふ|ら}にも
\ruby{少量}{せう|りやう}の
\ruby{酒類}{しゆ|るゐ}を
\ruby{與}{あた}へ、
\ruby{自分等}{じ|ぶん|ら}もまた
\ruby{聊}{いさヽ}か
\ruby{用}{もち}ゐる。
その
\ruby{味}{あじ}はまた
\ruby{君等}{きみ|ら}の
\ruby{知}{し}らんところだ。
\ruby{烈}{はげ}しい
\ruby{怖}{おそ}ろしい
\ruby{風}{かぜ}、
\ruby{酷}{むご}い
\ruby{痛}{いた}い
\ruby{雨}{あめ}、
\ruby{眞黑}{まつ|くろ}な
\ruby{天}{そら}、
\ruby{荒}{あれ}れ
\ruby{立}{た}つ
\ruby{水}{みづ}、
\ruby[g]{{\GWI{u9020-k}}物主}{ざうぶつしゆ}が
\ruby{其}{そ}の
\ruby{偉大}{ゐ|だい}な
\ruby{働}{はたら}きを
\ruby{見}{み}せる
\ruby{大洋}{たい|やう}の
\ruby{上}{うへ}で、
\ruby{木}{き}の
\ruby{葉}{は}にも
\ruby{等}{ひと}しい
\ruby[g]{孤舟}{こしふ}に
\ruby{立}{た}つて、たヾ
\ruby{我}{わ}が
\ruby{堅確}{けん|かく}な
\ruby{意志}{い|し}と
\ruby{智識}{ち|しき}の
\ruby{{\換字{判}}斷}{はん|だん}とのみを
\ruby{我}{わ}が
\ruby[g]{味方}{みかた}にして、あらゆる
\ruby{試}{こヽろ}みに
\ruby{耐}{た}へて
\ruby{奮{\GWI{u9032-k}}}{ふん|しん}して
\ruby{行}{い}つて、
\ruby{終}{つひ}に
\ruby{其}{そ}の
\ruby{試}{こヽろ}みに
\ruby{打{\換字{勝}}}{うち|か}ち
\ruby{果}{おほ}せた
\ruby{時}{とき}、ラムでもジンでも
\ruby[g]{日本酒}{にほんしゆ}でもの、
\ruby{一小杯}{いち|せう|はい}を
\ruby{手}{て}にして
\ruby{自}{みづか}ら
\ruby{犒}{ねぎら}ふ
\ruby{其}{そ}の
\ruby{一種}{いつ|しゆ}の
\ruby{言}{い}ふべからざる
\ruby{感}{かん}じは
\ruby{海員}{かい|ゐん}で
\ruby{無}{な}くては
\ruby{解}{わか}らん。
\ruby{陸上}{を|か}の
\ruby[g]{料理屋}{れうりや}やなんぞで
\ruby{飮}{の}むのとは
\ruby{全然異}{まる|で|ちが}ふ
\ruby{味}{あぢ}がする。
\ruby{僕}{ぼく}はたヾ
\ruby{其樣}{そ|う}いふ
\ruby{怖}{おそ}ろしい
\ruby[g]{暴風雨}{しけ}の
\ruby{後}{あと}なんぞに、
\ruby[g]{濕氣拂}{しつけばら}ひのため、
\ruby{疲勞}{ひ|らう}の
\ruby{回復}{くわい|ふく}のために、
\ruby{飮}{の}む
\ruby{時}{とき}ばかりは
\ruby{眞}{しん}に
\ruby{酒}{さけ}を
\ruby{賞}{しやう}するが、
\ruby{其}{そ}の
\ruby{他}{た}の
\ruby{時}{とき}に
\ruby{左程好}{さ|ほど|この}まん。
もう
\ruby{澤山}{たく|さん}だ。
\ruby{大分醉}{だい|ぶ|よ}つた。
』

『
\ruby[g]{然樣固}{さうかた}くばかりいふな、さあ
\ruby[g]{一盃}{ひとつ}
\ruby{{\GWI{u9063-k}}}{や}る。
\ruby{見}{み}ろ、お
\ruby{濱}{はま}さんが
\ruby{眼}{め}を
\ruby{丸}{まる}くして、
\ruby{一心}{いつ|しん}に
\ruby{君}{きみ}の
\ruby{暴雨風}{あ|ら|し}の
\ruby[g]{談話}{はなし}に
\ruby{聞}{き}き
\ruby{惚}{ほ}れて
\ruby{居}{ゐ}る、
\ruby{其}{そ}の
\ruby{罪}{つみ}の
\ruby{無}{な}い
\ruby[g]{純潔}{きれい}な
\ruby[g]{樣子}{やうす}を
\ruby{見}{み}ろ。
\ruby{此}{こ}の
\ruby{人}{ひと}が
\ruby{勸}{すヽ}める
\ruby{酒}{さけ}を
\ruby{飮}{の}まんといふ
\ruby{事}{こと}があるか。
』

\ruby[g]{水野}{みづの}はこヽに
\ruby{至}{いた}つて
\ruby{自}{おのづ}から
\ruby{微笑}{び|せう}を
\ruby{催}{もよほ}し、

『
\ruby[g]{{\換字{羽}\換字{勝}}}{はがち}
\ruby{君}{くん}、まあ
\ruby{一}{ひと}つ
\ruby{{\GWI{u904e-k}}}{すご}して
\ruby{{\換字{呉}}}{く}れたまへ。
\ruby[g]{魯敏孫漂流記}{ろびんそんへうりうき}を
\ruby{讀}{よ}んで
\ruby{非常}{ひ|じやう}に
\ruby{感}{かん}じて、
\ruby[g]{魯敏孫}{ろびんそん}と
\ruby{一處}{いつ|しよ}に
\ruby{棲}{す}みたいといつたほどの
\ruby{崇拜者}{すう|はい|しや}となつて
\ruby{居}{ゐ}る、
\ruby[g]{航海者好}{かうかいしやずき}の
\ruby{其人}{その|ひと}の
\ruby{御酌}{お|しやく}だから。
』

と
\ruby{前}{さき}の
\ruby{夜}{よ}の
\ruby{事}{こと}を
\ruby{思}{おも}ひ
\ruby{起}{おこ}して
\ruby{語}{かた}り
\ruby{出}{い}づれば、

『あら、よくつてよ
\ruby{先生}{せん|せい}、
\ruby{餘計}{よ|けい}な
\ruby{事}{こと}を。
』

とお
\ruby{濱}{はま}の
\ruby{打消}{うち|け}さんとするが
\ruby{如}{ごと}く
\ruby{言}{い}へると
\ruby{同時}{どう|じ}に、
\ruby[g]{日方}{ひかた}は
\ruby{笑}{ゑ}ましげに、

『
\ruby{何}{なん}だ、
\ruby[g]{魯敏孫}{ろびんそん}の
\ruby{崇拜者}{すう|はい|しや}だ!、こりやあ
\ruby{面白}{おも|しろ}い。
\ruby{偉}{えら}い!。
\ruby{然樣來}{さ|う|こ}なくちやならん、
\ruby{其}{それ}で
\ruby{無}{な}くちやいかん。
\ruby{實}{じつ}に
\ruby{愉快}{ゆ|くわい}な
\ruby{人}{ひと}だ、
\ruby{頼}{たの}もしい!。
\ruby[g]{成程日方}{なるほどひかた}が
\ruby{頭}{あたま}を
\ruby{撲}{は}られたのも
\ruby{無理}{む|り}は
\ruby{無}{な}いは。
ハヽヽ、
\ruby{君}{きみ}のやうな
\ruby{人}{ひと}になら、もう
\ruby[g]{少々打撲}{せう〳〵ぶんなぐ}られても
\ruby{關}{かま}はんは、あヽ
\ruby{面白}{おも|しろ}い。
\ruby[g]{水野猪口}{みづのちよく}を
\ruby{與}{よこ}せ、さあ
\ruby[g]{魯敏孫夫人御酌}{ろびんそんふじんおしやく}を
\ruby{願}{ねが}ふ。
』

と
\ruby{打興}{うち|きよう}じたり。
されど
\ruby[g]{{\換字{羽}\換字{勝}}}{はがち}は
\ruby{冷然}{れい|ぜん}として、たヾお
\ruby{濱}{はま}をば
\ruby{一瞥}{いち|べつ}せしのみ、
\ruby[g]{水野}{みづの}に
\ruby{對}{むか}つて
\ruby{物靜}{もの|しづ}かに、

『
\ruby{海國}{かい|こく}の
\ruby{日本}{に|ほん}の
\ruby{事}{こと}だもの、
\ruby[g]{魯敏孫漂流記}{ろびんそんへうりうき}に
\ruby{興味}{きよ|うみ}を
\ruby{感}{かん}ずるやうな
\ruby{女子}{ぢよ|し}の
\ruby{出}{で}て
\ruby{來}{き}て
\ruby{{\換字{呉}}}{く}れるのは
\ruby[g]{當然}{たうぜん}の
\ruby{事}{こと}だ。
\ruby{僕}{ぼく}は
\ruby{此席}{この|せき}にさへ
\ruby{此樣}{こ|う}いふ
\ruby{婦人}{ふ|じん}を
\ruby{見}{み}る
\ruby{世}{よ}に、まだ
\ruby{海國}{かい|こく}の
\ruby{日本}{に|ほん}の
\ruby{詩}{し}にも
\ruby{小{\GWI{u8aaa-jv}}}{せう|せつ}にも、
\ruby{海}{うみ}に
\ruby{關}{くわん}したものヽ
\ruby{甚}{はなは}だ
\ruby{少}{すくな}いのを
\ruby{{\GWI{u907a-k}}憾}{ゐ|かん}に
\ruby{思}{おも}ふ。
\ruby[g]{水野}{みづの}!。
\ruby[g]{今年中}{ことしちう}には
\ruby[g]{島木}{しまき}の
\ruby{{\換字{船}}}{ふね}を
\ruby{何樣}{ど|う}しても
\ruby{出}{だ}す。
\ruby{僕}{ぼく}は
\ruby[g]{無論全權}{むろんぜんけん}を
\ruby{有}{も}つて
\ruby{出掛}{で|かけ}けるのだ。
\ruby{何樣}{ど|う}だ、
\ruby{君}{きみ}
\ruby{一}{ひと}つ
\ruby{奮發}{ふん|ぱつ}して
\ruby{海上}{かい|じやう}に
\ruby{出}{で}んか。
\ruby{决}{けつ}して
\ruby{危險}{き|けん}なんぞは
\ruby{有}{あ}るもので
\ruby{無}{な}い。
\ruby{好}{い}い
\ruby{機會}{き|くわい}だ、
\ruby{大洋}{たい|やう}の
\ruby[g]{美觀壯觀}{びくわんさうくわん}を
\ruby{君}{きみ}の
\ruby{眼}{め}に
\ruby{入}{い}れんか。
\ruby{茫々}{ばう|〳〵}たる
\ruby{大洋}{たい|やう}の
\ruby{大}{おほき}な
\ruby[g]{景氣}{けしき}の
\ruby{中}{なか}へ
\ruby{出}{で}て、
\ruby{人間}{にん|げん}の
\ruby{紛々}{ふん|ぷん}たる
\ruby{葛藤}{かつ|とう}を
\ruby{逃}{のが}れて、
\ruby{直接}{ちよく|せつ}に
\ruby{{\GWI{u9020-k}}化}{ざう|くわ}の
\ruby{懷中}{ふと|ころ}に
\ruby{寢}{ね}て
\ruby{見}{み}んか
\ruby[g]{水野}{みづの}。
たしかに
\ruby{君}{きみ}の
\ruby{知}{し}らん
\ruby[g]{心持}{こヽろもち}が
\ruby{爲}{し}やうぜ。
』

と
\ruby{豫}{かね}て
\ruby{考}{かんが}へ
\ruby{來}{きた}りしことにやあらん、
\ruby{思}{おも}ひのほかなる
\ruby{點}{てん}を
\ruby[g]{沈着}{おちつ}いて
\ruby{云}{い}ひ
\ruby{出}{だ}しぬ。

