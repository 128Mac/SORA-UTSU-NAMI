\Entry{其五十}

% メモ 校正終了 2024-05-09 2024-06-05
\原本頁{286-9}%
\ruby{言}{ものい}はねども
\ruby{花}{はな}
あれば
\ruby{野}{の}は
\ruby{自}{おのづ}から
\ruby{春}{はる}なり。
%
あどけ
\ruby{無}{な}き
お
\ruby{濱}{はま}
\ruby[g]{一人}{ひとり }の
\ruby{{\換字{交}}}{まじ}りたるに
\ruby[g]{一座}{いちざ }は
\ruby{和}{やはら}ぎて、
\ruby[g]{理屈}{り くつ}を
\ruby{離}{はな}るれば
\ruby[g]{談話}{はなし }に
\ruby{角}{かど}
\ruby{無}{な}く、
%
\ruby[g]{笑聲}{せうせい}
\ruby[||j>]{漸}{やうや}く
\ruby{起}{おこ}れば
\ruby{酒}{さけ}の
\ruby[<j>]{味}{あぢはひ}
\ruby[||j>]{饒}{ おほ}く、
%
\ruby[g]{謹嚴}{きんごん}の
\ruby[g]{羽{\換字{勝}}}{は がち}、
%
\ruby[g]{沈鬱}{ちんうつ}せる
\ruby[g]{水野}{みづの }さへ、
%
\ruby[g]{何時}{い つ }か
\ruby{六七年}{ろく|しち|ねん}の% 原本には漢数字「七」のルビ無し
\ruby[g]{往時}{むかし }に
\ruby{復}{かへ}りて、
%
\ruby{心}{こゝろ}は% 踊り字調整「〻(二の字点、揺すり点)に見えるが(ゝ)」
\ruby{{\換字{若}}}{わか}く
\ruby{氣}{き}は
\ruby{易}{やす}く
\ruby{語}{かた}らへば、
%
まして
\ruby[g]{日方}{ひ かた}は
\ruby{興}{きよう}に
\ruby{入}{い}りて、
%
\ruby[g]{羽{\換字{勝}}}{は がち}の
\ruby{{\換字{斥}}}{しりぞ}けたる
\ruby{天眞爛{\換字{熳}}}{てん|しん|らん|まん}、
%
\ruby{醉態淋漓}{すゐ|たい|りん|り}として
\ruby{受}{う}けては
\ruby{飮}{の}み
\ruby{受}{う}けては
\ruby{飮}{の}み、

\原本頁{287-5}%
『
\ruby[g]{島木}{しまき }、
%
\ruby{馬鹿野郎}{ば|か|や|らう}、
%
\ruby[g]{一緖}{いつしよ}に
\ruby{來}{く}れば
\ruby{宜}{い}いのに。
%
\ruby[||j>]{金}{かね}
\ruby[||j>]{儲}{まうけ}に
% \ruby{金儲}{かね|まうけ}に
\ruby{忙}{いそが}しがつたつて
\ruby{何}{なん}に
なるものか。
』

\原本頁{287-7}%
と
\ruby[g]{幾度}{いくたび}か
\ruby{繰}{く}り
\ruby{{\換字{返}}}{かへ}して
\ruby{罵}{のゝし}つては、% 踊り字調整「〻(二の字点、揺すり点)に見えるが(ゝ)」
%
\ruby{{\換字{又}}}{また}
\ruby[g]{餘念}{よ ねん}も
\ruby{無}{な}く
\ruby[g]{二人}{ふたり }を
\ruby[g]{相手}{あひて }に
\ruby[g]{談笑}{だんせう}して% ルビ調整(原本通り)「だんせ(う)」
\ruby[<j>]{盃}{さかづき}を
\ruby{手}{て}にしたり。

\原本頁{287-9}%
『
お
\ruby{濱}{はま}さん、
%
その
\ruby{色}{いろ}の
\ruby{黑}{くろ}い
\ruby{眞面目}{ま|じ|め}
\ruby[g]{老夫}{おやぢ }の
\ruby[g]{羽{\換字{勝}}}{は がち}に
\ruby{飮}{の}ませて
\ruby{{\換字{遣}}}{や}つて
\原本頁{287-10}\改行%
\ruby{吳}{く}れたまへ。
%
コラ
\ruby[g]{羽{\換字{勝}}}{は がち}!、
%
\ruby{飮}{の}まんかい、
%
\ruby[g]{水野}{みづの }の
\ruby[<j>]{妹}{いもうと}の
\ruby{{\換字{酌}}}{しやく}だ。
%
ハヽハ
\ruby{{\換字{船}}}{ふね}では
\ruby{成}{な}るべく
\ruby{酒}{さけ}を
\ruby{用}{もち}ゐん
\ruby[g]{{\換字{習}}慣}{く せ }を
\ruby{付}{つ}けて
\ruby{居}{ゐ}るから
\ruby{飮}{の}めん
なぞ
といふのは
\ruby[g]{虛言}{う そ }だらう。
%
\ruby[g]{{\換字{船}}員}{ふなのり}は
\ruby[g]{大抵}{たいてい}
\ruby{善}{よ}く
\ruby{飮}{の}む
といふぞ。
』

\原本頁{288-2}%
『
イヤ
もう
いかん。
%
\ruby[g]{虛言}{う そ }では
\ruby{無}{な}い、
%
\ruby{{\換字{船}}}{ふね}では
\ruby{成}{な}るべく
\ruby{用}{もち}ゐんやうにして
\ruby{居}{ゐ}るのだ。
%
\ruby[g]{執務}{しつむ }の
\ruby{不確實}{ふ|かく|じつ}になる
\ruby{基}{もとゐ}だから
\ruby[g]{飮酒}{いんしゆ}は
\ruby{忌}{い}む。
%
これは
\ruby[g]{海員}{かいゐん}の
\ruby[g]{精神}{せいしん}の
\ruby[g]{{\換字{進}}歩}{しんぽ }した
\ruby[g]{趨勢}{すうせい}で、
%
\ruby[g]{{\換字{古}}來}{こ らい}の
\ruby[g]{海員}{かいゐん}の
\ruby[g]{飮酒}{いんしゆ}に
\ruby{耽}{ふけ}つた
\原本頁{288-5}\改行%
\ruby[g]{惡{\換字{習}}}{あくしふ}を
\ruby{洗}{あら}ふ
\ruby{任}{にん}は
\ruby[g]{我々}{われ〳〵}の
\ruby{肩}{かた}に
あるのだ。
%
だから
\ruby[g]{實際}{じつさい}
\ruby{僕}{ぼく}なぞは
\ruby{餘}{あま}り
\原本頁{288-6}\改行%
\ruby{用}{もち}ゐん。
%
しかし
\ruby[g]{非常}{ひじやう}な
\ruby{暴風雨}{ぼう|ふう|ゝ}の% 踊り字調整「〻(二の字点、揺すり点)に見えるが(ゝ)」
\ruby{時}{とき}、
%
\ruby[g]{襯衣}{シヤツ}まで
\ruby{濡}{ぬ}れ
\ruby{浸}{ひた}りながら
\ruby[g]{困苦}{こんく }
\ruby{極}{きは}まる
\ruby[g]{勞働}{らうどう}を
\ruby{仕}{し}た
\ruby{後}{あと}
など
では、
%
\ruby{水夫等}{すゐ|ふ|ら}にも
\ruby[||j>]{少}{せう}
\ruby[||j>]{量}{りやう}の
% \ruby{少量}{せう|りやう}の
\ruby[g]{酒類}{しゆるゐ}を
\ruby{與}{あた}へ
\改行% 校正作業の簡略化のため
、
%
\原本頁{288-8}\改行%
\ruby{自{\換字{分}}等}{じ|ぶん|ら}も
また
\ruby{聊}{いさゝ}か% 踊り字調整「〻(二の字点、揺すり点)に見えるが(ゝ)」
\ruby{用}{もち}ゐる。
%
その
\ruby{味}{あぢ}は
また
\ruby[g]{君等}{きみら }の
\ruby{知}{し}らん
ところだ
\改行% 校正作業の簡略化のため
。
%
\原本頁{288-9}\改行%
\ruby{烈}{はげ}しい
\ruby{怖}{おそ}ろしい
\ruby{風}{かぜ}、
%
\ruby{酷}{むご}い
\ruby{痛}{いた}い
\ruby{雨}{あめ}、
%
\ruby[g]{眞黑}{まつくろ}な
\ruby{天}{そら}、
%
\ruby{荒}{あ}れ
\ruby{立}{た}つ
\ruby{水}{みづ}、
%
\ruby{{\換字{造}}物主}{ざう|ぶつ|しゆ}が
\ruby{其}{そ}の
\ruby[g]{偉大}{ゐ だい}な
\ruby{働}{はたら}きを
\ruby{見}{み}せる
\ruby[g]{大洋}{たいやう}の
\ruby{上}{うへ}で、
%
\ruby{木}{き}の
\ruby{葉}{は}にも
\ruby{等}{ひと}しい
\ruby[g]{孤舟}{こ しふ}に
\ruby{立}{た}つて、
%
たゞ% 踊り字調整「〻(二の字点、揺すり点)に濁点に見えるが(ゞ)」
\ruby{我}{わ}が
\ruby[g]{堅確}{けんかく}な
\ruby[g]{意志}{い し }と
\ruby[g]{智識}{ち しき}の
\ruby[g]{{\換字{判}}斷}{はんだん}と
のみを
\ruby{我}{わ}が
\ruby[g]{味方}{み かた}にして、
%
あらゆる
\ruby{試}{こゝろ}みに% 踊り字調整「〻(二の字点、揺すり点)に見えるが(ゝ)」
\ruby{耐}{た}へて
\ruby[g]{奮{\換字{進}}}{ふんしん}して
\ruby{行}{い}つて、
%
\ruby{{\換字{終}}}{つひ}に
\ruby{其}{そ}の
\ruby[<j||]{試}{こゝろ}みに% 踊り字調整「〻(二の字点、揺すり点)に見えるが(ゝ)」% 行末行頭の境界付近なので特例処置を施す
\ruby{打}{うち}
\ruby{{\換字{勝}}}{か}ち
\ruby{果}{おほ}せた
\ruby{時}{とき}、
%
ラムでも
ジンでも
\ruby{日本酒}{に|ほん|しゆ}でもの、
%
\ruby{一小杯}{いち|せう|はい}を% 小杯 ... こさかずき/コップ
\ruby{手}{て}にして
\ruby{自}{みづか}ら
\ruby{犒}{ねぎら}ふ
\ruby{其}{そ}の
\ruby[g]{一種}{いつしゆ}の
\ruby{言}{い}ふ
べからざる
\ruby{{\換字{感}}}{かん}じは
\ruby[g]{海員}{かいゐん}で
\ruby{無}{な}くては
\ruby{解}{わか}らん。
%
\ruby[g]{陸上}{を か }の
\ruby{料理屋}{れう|り|や}
やなんぞで
\ruby{飮}{の}むのとは
\ruby[g]{全然}{まるで }
\ruby{異}{ちが}ふ
\ruby{味}{あぢ}がする。
%
\ruby{僕}{ぼく}は
たゞ% 踊り字調整「〻(二の字点、揺すり点)に濁点に見えるが(ゞ)」
\ruby[g]{其樣}{そ う }いふ
\ruby{怖}{おそ}ろしい
\ruby{暴風雨}{ し||け}の
\ruby{後}{あと}なんぞに、
%
\ruby[g]{濕氣}{しつけ }
\原本頁{289-6}\改行%
\ruby{拂}{ばら}ひ
のため、
%
\ruby[g]{疲勞}{ひ らう}の
\ruby[||j>]{回}{くわい}% 原本通り「回」
\ruby[||j>]{復}{ ふく}のために、
% \ruby{回復}{くわい|ふく}のために、% 原本通り「回」
%
\ruby{飮}{の}む
\ruby{時}{とき}
ばかりは
\ruby{眞}{しん}に
\ruby{酒}{さけ}を
\ruby[<j||]{賞}{しやう}するが、% 行末行頭の境界付近なので特例処置を施す
%
\ruby{其}{そ}の
\ruby{他}{た}の
\ruby{時}{とき}には
\ruby[g]{左程}{さ ほど}
\ruby{好}{この}まん。
%
もう
\ruby[g]{澤山}{たくさん}だ。
%
\ruby[g]{大{\換字{分}}}{だいぶ }
\ruby{醉}{よ}つた% 「醉」は原本通り「よ」で調整
\改行% 校正作業の簡略化のため
。
』

\原本頁{289-8}%
『
\ruby[g]{然樣}{さ う }
\ruby{固}{かた}く
ばかり
いふな、
%
さあ
\ruby[g]{一盃}{ひとつ }
\ruby{{\換字{遣}}}{や}る。
%
\ruby{見}{み}ろ、
%
お
\ruby{濱}{はま}さんが
\ruby{眼}{め}を
\ruby{丸}{まる}くして、
%
\ruby[g]{一心}{いつしん}に
\ruby{君}{きみ}の
\ruby{暴雨風}{あ|ら|し}の% ここは「暴風雨」でなく「暴雨風」
\ruby[g]{談話}{はなし }に
\ruby{聞}{き}き
\ruby{惚}{ほ}れて
\ruby{居}{ゐ}る、
%
\ruby{其}{そ}の
\原本頁{289-10}\改行%
\ruby{罪}{つみ}の
\ruby{無}{な}い
\ruby[g]{純潔}{き れい}な
\ruby[g]{樣子}{やうす }を
\ruby{見}{み}ろ。
%
\ruby{此}{こ}の
\ruby{人}{ひと}が
\ruby{勸}{すゝ}める% 踊り字調整「〻(二の字点、揺すり点)に見えるが(ゝ)」
\ruby{酒}{さけ}を
\ruby{飮}{の}まん
といふ
\ruby{事}{こと}が
あるか。
』

\原本頁{290-1}%
\ruby[g]{水野}{みづの }は
こゝに% 踊り字調整「〻(二の字点、揺すり点)に見えるが(ゝ)」
\ruby{至}{いた}つて
\ruby{自}{おのづ}から
\ruby[g]{微笑}{び せう}を
\ruby{催}{もよほ}し、

\原本頁{290-2}%
『
\ruby[g]{羽{\換字{勝}}}{は がち}
\ruby{君}{くん}、
%
まあ
\ruby{一}{ひと}つ
\ruby{{\換字{過}}}{すご}して
\ruby{吳}{く}れたまへ。
%
\ruby{魯敏孫}{ろ|びん|そん}
\ruby{漂流記}{へう|りう|き}を
\ruby{讀}{よ}んで
\原本頁{290-3}\改行%
\ruby[g]{非常}{ひじやう}に
\ruby{{\換字{感}}}{かん}じて、
%
\ruby{魯敏孫}{ろ|びん|そん}と
\ruby[g]{一處}{いつしよ}に
\ruby{棲}{す}みたい
といつたほどの
\ruby{崇拜者}{すう|はい|しや}
となつて
\ruby{居}{ゐ}る、
%
\ruby{航海者好}{かう|かい|しや|ずき}の
\ruby{其}{その}
\ruby{人}{ひと}の
\ruby[g]{御{\換字{酌}}}{おしやく}だから。
』

\原本頁{290-5}%
と
\ruby{{\換字{前}}}{さき}の
\ruby{夜}{よ}の
\ruby{事}{こと}を
\ruby{思}{おも}ひ
\ruby{起}{おこ}して
\ruby{語}{かた}り
\ruby{出}{い}づれば、

\原本頁{290-6}%
『
あら、
%
よくつてよ
\ruby[g]{先生}{せんせい}、
%
\ruby[g]{餘計}{よ けい}な
\ruby{事}{こと}を。
』

\原本頁{290-7}%
と
お
\ruby{濱}{はま}の
\ruby{打}{うち}
\ruby{{\換字{消}}}{け}さん
とするが
\ruby{如}{ごと}く
\ruby{言}{い}へると
\ruby[g]{同時}{どうじ }に、
%
\ruby[g]{日方}{ひ かた}は
\ruby{笑}{ゑ}ましげに、

\原本頁{290-9}%
『
\ruby{何}{なん}だ、
%
\ruby{魯敏孫}{ろ|びん|そん}の
\ruby{崇拜者}{すう|はい|しや}だ!、
%
こりやあ
\ruby[g]{面白}{おもしろ}い。
%
\ruby{偉}{{\換字{𛀁}}ら}い!。
%
\ruby[g]{然樣}{さ う }
\原本頁{290-10}\改行%
\ruby{來}{こ}なくちや
ならん、
%
\ruby{其}{それ}で
\ruby{無}{な}くちや
いかん。
%
\ruby{實}{じつ}に
\ruby[g]{{\換字{愉}}快}{ゆくわい}な
\ruby{人}{ひと}だ、
%
\ruby{頼}{たの}もしい!。
%
\ruby[g]{成程}{なるほど}
\ruby[g]{日方}{ひ かた}が
\ruby{頭}{あたま}を
\ruby{撲}{は}られたのも
\ruby[g]{無理}{む り }は
\ruby{無}{な}いは。
%
ハヽヽ
\改行% 校正作業の簡略化のため
、
%
\原本頁{291-1}\改行%
\ruby{君}{きみ}のやうな
\ruby{人}{ひと}になら、
%
もう
\ruby[g]{少々}{せう〳〵}
\ruby[g]{打撲}{ぶんなぐ}られても
\ruby{關}{かま}はんは、
%
あゝ% 踊り字調整「〻(二の字点、揺すり点)に見えるが(ゝ)」
\ruby[g]{面白}{おもしろ}い。
%
\ruby[g]{水野}{みづの }
\ruby[g]{猪口}{ちよく }を
\ruby{與}{よこ}せ、
%
さあ
\ruby{魯敏孫}{ろ|びん|そん}
\ruby[g]{夫人}{ふ じん}
\ruby[g]{御{\換字{酌}}}{おしやく}を
\ruby{願}{ねが}ふ。
』

\原本頁{291-3}%
と
\ruby[||j>]{打}{うち}
\ruby[||j>]{興}{きよう}じたり。
% \ruby{打興}{うち|きよう}じたり。

\原本頁{291-4}%
されど
\ruby[g]{羽{\換字{勝}}}{は がち}は
\ruby[g]{冷然}{れいぜん}として、
%
たゞ% 踊り字調整「〻(二の字点、揺すり点)に濁点に見えるが(ゞ)」
お
\ruby{濱}{はま}をば
\ruby[g]{一瞥}{いちべつ}せしのみ、
%
\ruby[g]{水野}{みづの }に
\ruby{對}{むか}つて
\ruby{物}{もの}
\ruby{靜}{しづ}かに、

\原本頁{291-6}%
『
\ruby[g]{海國}{かいこく}の
\ruby[g]{日本}{に ほん}の
\ruby{事}{こと}だもの、
%
\ruby{魯敏孫}{ろ|びん|そん}
\ruby{漂流記}{へう|りう|き}に
\ruby[g]{興味}{きようみ}を
\ruby{{\換字{感}}}{かん}ずるやうな
\原本頁{291-7}\改行%
\ruby[g]{女子}{ぢよし }の
\ruby{出}{で}て
\ruby{來}{き}て
\ruby{吳}{く}れるのは
\ruby[g]{當然}{たうぜん}の
\ruby{事}{こと}だ。
%
\ruby{僕}{ぼく}は
\ruby{此}{この}
\ruby{席}{せき}にさへ
\ruby[g]{此樣}{こ う }いふ
\ruby[g]{{\換字{婦}}人}{ふ じん}を
\ruby{見}{み}る
\ruby{世}{よ}に、
%
まだ
\ruby[g]{海國}{かいこく}の
\ruby[g]{日本}{に ほん}の
\ruby{詩}{し}にも
\ruby[g]{小說}{せうせつ}にも、
%
\ruby{海}{うみ}に
\ruby[<j||]{關}{くわん}した
ものゝ% 踊り字調整「〻(二の字点、揺すり点)に見えるが(ゝ)」
\ruby{甚}{はなは}だ
\ruby{少}{すくな}いのを
\ruby[g]{{\換字{遺}}憾}{ゐ かん}に
\ruby{思}{おも}ふ。
%
\ruby[g]{水野}{みづの }!。
%
\ruby{今年中}{こ|とし|ぢう}には
\ruby[g]{島木}{しまき }の
\ruby{{\換字{船}}}{ふね}を
\ruby[g]{何樣}{ど う }しても
\ruby{出}{だ}す。
%
\ruby{僕}{ぼく}は
\ruby[g]{無論}{む ろん}
\ruby[g]{全權}{ぜんけん}を
\ruby{有}{も}つて
\ruby[g]{出掛}{で か }けるのだ。
%
\原本頁{291-11}\改行%
\ruby[g]{何樣}{ど う }だ、
%
\ruby{君}{きみ}
\ruby{一}{ひと}つ
\ruby[g]{奮發}{ふんぱつ}して
\ruby[||j>]{海}{かい}
\ruby[||j>]{上}{じやう}に
% \ruby{海上}{かい|じやう}に
\ruby{出}{で}んか。
%
\ruby{决}{けつ}して
\ruby[g]{危險}{き けん}
なんぞは
\ruby{有}{あ}るもので
\ruby{無}{な}い。
%
\ruby{好}{い}い
\ruby[g]{機會}{きくわい}だ、
%
\ruby[g]{大洋}{たいやう}の
\ruby[g]{美觀}{びくわん}
\ruby[||j>]{壯}{さう }
\ruby[||j>]{觀}{くわん}
% \ruby{壯觀}{さう|くわん}
を
\ruby{君}{きみ}の
\ruby{眼}{め}に
\ruby{入}{い}れんか。
%
\ruby[g]{茫々}{ばう〳〵}たる
\ruby[g]{大洋}{たいやう}の
\ruby{大}{おほき}な
\ruby[g]{景氣}{け しき}の
\ruby{中}{なか}へ
\ruby{出}{で}て、
%
\ruby[g]{人間}{にんげん}の
\ruby[g]{{\換字{紛}}々}{ふんぷん}たる
\ruby[g]{葛藤}{かつとう}を
\ruby{{\換字{逃}}}{のが}れて、
%
\ruby[||j>]{直}{ちよく}
\ruby[||j>]{接}{ せつ}に
% \ruby{直接}{ちよく|せつ}に
\ruby[g]{{\換字{造}}化}{ざうくわ}の
\ruby[g]{懷中}{ふところ}に
\ruby{寢}{ね}て
\ruby{見}{み}んか
\ruby[g]{水野}{みづの }。
%
たしかに
\ruby{君}{きみ}の
\ruby{知}{し}らん
\ruby[||j>]{心}{こゝろ}% 踊り字調整「〻(二の字点、揺すり点)に見えるが(ゝ)」
\ruby[||j>]{持}{ もち}が
% \ruby{心持}{こゝろ|もち}が% 踊り字調整「〻(二の字点、揺すり点)に見えるが(ゝ)」
\ruby{爲}{し}やうぜ。
』

\原本頁{292-5}%
と
\ruby{豫}{かね}て
\ruby{考}{かんが}へ
\ruby{來}{きた}りし
ことにや
あらん、
%
\ruby{思}{おも}ひ
のほかなる
\ruby{點}{てん}を
\ruby[g]{沈着}{おちつ }いて
\ruby{云}{い}ひ
\ruby{出}{だ}しぬ。
