\Entry{其五十一}

% メモ 校正終了 2024-05-09 2024-06-06
\原本頁{292-8}%
\ruby[g]{水野}{みづの }の
\ruby{答}{こた}へに
\ruby{答}{こた}へ
かぬる
\ruby{時}{とき}、
%
\ruby[g]{羽{\換字{勝}}}{は がち}
はふたゝび% 踊り字調整「〻(二の字点、揺すり点)に見えるが(ゝ)」
\ruby[g]{言葉}{ことば }
をつぎて、

\原本頁{292-9}%
『
\ruby{實}{じつ}は
\ruby[g]{{\換字{遠}}洋}{ゑんやう}へ
\ruby{出}{で}る
\ruby[g]{漁{\換字{船}}}{ぎよせん}
など
では、
%
\ruby[||j>]{便}{びん}
\ruby[||j>]{乘}{じよう}
\ruby[||j>]{者}{ しや}を
% \ruby{便乘者}{びん|じよう|しや}を
\ruby{特}{こと}のほかに
\ruby[g]{{\換字{迷}}惑}{めいわく}がる
のだ。
%
しかし
\ruby{君}{きみ}
が
\ruby{好}{この}む
ならば
\ruby{僕}{ぼく}は
\ruby{勸}{すゝ}めても% 踊り字調整「〻(二の字点、揺すり点)に見えるが(ゝ)」
\ruby{乘}{の}せたい。
%
\ruby{君}{きみ}を
\ruby[g]{大洋}{たいやう}の
\ruby{中}{なか}へ
\ruby{引}{ひき}
\ruby{出}{だ}したい。
%
いろ〳〵の
\ruby[g]{人爲}{じんゐ }の
\ruby[g]{複雜}{ふくざつ}な
\ruby[g]{組織}{そ しき}で、
%
\ruby[g]{自然}{し ぜん}の
\ruby[g]{眞趣}{おもむき}を
\ruby{蔽}{おほ}ひ
\ruby{盡}{つく}して
ゐる
\ruby[||j>]{陸}{りく}
\ruby[||j>]{上}{じやう}から
% \ruby{陸上}{りく|じやう}から
\ruby{君}{きみ}を
\ruby{離}{はな}れ
させたい。
%
\ruby[g]{直接}{たゞち }に% 踊り字調整「〻(二の字点、揺すり点)に濁点に見えるが(ゞ)」
\ruby[g]{自然}{し ぜん}の
\原本頁{293-3}\改行%
\ruby{{\換字{前}}}{まへ}に
\ruby{出}{で}て
\ruby{貰}{もら}ひたい。
%
\ruby[g]{直接}{たゞち }に% 踊り字調整「〻(二の字点、揺すり点)に濁点に見えるが(ゞ)」
\ruby[g]{自然}{し ぜん}の
\ruby[g]{詩卷}{しくわん}を
\ruby{讀}{よ}んで
\ruby{見}{み}て
\ruby{貰}{もら}ひたい。
%
\ruby{僕}{ぼく}は
よくは
\ruby{詩}{し}を
\ruby{知}{し}らん。
%
しかし
\ruby{僕}{ぼく}が
\ruby{知}{し}つて
\ruby{居}{ゐ}る
\ruby[g]{自然}{し ぜん}は、
%
\ruby{僕}{ぼく}の
\ruby{知}{し}つて
\ruby{居}{ゐ}る
\ruby[g]{一切}{いつさい}の
\ruby{詩}{し}とは
\ruby{甚}{はなは}だ
\ruby{{\換字{遠}}}{とほ}い
ものだ。
%
\ruby{僕}{ぼく}は
\ruby[g]{自然}{し ぜん}の
\ruby[g]{或者}{あるもの}を
\ruby{解}{かい}して
\ruby{居}{ゐ}る
\ruby{點}{てん}に
\ruby{於}{おい}て
\ruby[g]{詩人}{し じん}に
\ruby{{\換字{勝}}}{まさ}つて
\ruby{居}{ゐ}るとは
\ruby{信}{しん}せぬ。
%
たゞし% 踊り字調整「〻(二の字点、揺すり点)に濁点に見えるが(ゞ)」
\ruby[<j||]{海}{かい }% ルビ調整(特殊処理)「海上に」続く「關する」のルビを分離
\ruby[<j||]{上}{じやう}に
% \ruby{海上}{かい|じやう}に
\ruby[<j||]{關}{くわん}する% 行末行頭の境界付近なので特例処置を施す
\ruby{詩}{し}の
\ruby{甚}{はなは}だ
\ruby[g]{淺薄}{せんぱく}なのは
\ruby{{\換字{感}}}{かん}じて
\ruby{居}{ゐ}る。
%
\ruby{{\換字{若}}}{も}し
\ruby[g]{詩想}{し さう}
のある
\ruby{人}{ひと}が
\ruby[g]{大洋}{たいやう}
\原本頁{293-8}\改行%
に
\ruby{{\換字{浮}}}{うか}んで、
%
\ruby[g]{自然}{し ぜん}の
\ruby[||j>]{廣}{くわう}
\ruby[||j>]{大}{ だい}な
% \ruby{廣大}{くわう|だい}な
\ruby[g]{背景}{はいけい}の
\ruby{{\換字{前}}}{まへ}で、
%
\ruby[g]{人間}{にんげん}の
\ruby{自}{おのづ}から
\ruby{抱}{いだ}く% ルビ調整(原本通り)(いだ)
\ruby{{\換字{感}}}{かん}じを
\ruby{味}{あぢは}つたら、
%
\ruby[g]{在來}{ざいらい}の
\ruby{詩}{し}
のやうな
もの
ばかりは
\ruby[g]{出來}{で き }て
\ruby{居}{ゐ}まいと
\ruby{思}{おも}ふ
\改行% 校正作業の簡略化のため
。
%
\原本頁{293-10}\改行%
まあ
\ruby{想}{おも}つても
\ruby{見}{み}たまへ。
%
\ruby[g]{彼方}{あつち }から
\ruby[g]{此方}{こつち }へ% ルビ調整(原本通り)
\ruby{歸}{かへ}る
\ruby{路}{みち}の、
%
\ruby{太{\換字{平}}洋}{たい|へい|やう}の
\ruby[g]{眞中}{まんなか}
あたりで、
%
\ruby{僕}{ぼく}が
たゞ% 踊り字調整「〻(二の字点、揺すり点)に濁点に見えるが(ゞ)」
\ruby[g]{一人}{ひとり }
\ruby[g]{舷頭}{げんとう}に
\ruby{立}{た}つて
\ruby{居}{ゐ}た
ことが
ある。
%
\ruby[g]{丁度}{ちやうど}
\原本頁{294-1}\改行%
\ruby{月}{つき}は
\ruby[g]{眞珠}{しんじゆ}を
\ruby{溶}{と}かした
やうな
\ruby{光}{ひかり}を
\ruby{投}{な}げて
\ruby[g]{一切}{いつさい}を
\ruby{包}{つゝ}んで% 踊り字調整「〻(二の字点、揺すり点)に見えるが(ゝ)」
\ruby{居}{ゐ}る。
%
\ruby{其}{そ}の
\ruby{中}{なか}を
\ruby{走}{はし}つて
\ruby{居}{ゐ}る
\ruby[g]{自{\換字{分}}}{じ ぶん}の
\ruby{{\換字{船}}}{ふね}は
\ruby[g]{何處}{ど こ }へ
\ruby{行}{ゆ}く
のだらう。
%
\ruby{行}{ゆ}く
\ruby{先}{さき}も
\ruby{見}{み}えん、
%
\ruby{來}{き}た
ところも
\ruby{見}{み}えん。
%
たゞ% 踊り字調整「〻(二の字点、揺すり点)に濁点に見えるが(ゞ)」
\ruby{淡}{あは}い
\ruby{光}{ひかり}の
\ruby{滿}{み}ちて
\ruby{居}{ゐ}る
\ruby{天}{そら}
\ruby{水}{みづ}の
\ruby{中}{なか}を
\原本頁{294-4}\改行%
\ruby{歩}{ある}いて
\ruby{居}{ゐ}る。
%
\ruby{海}{うみ}は
\ruby{絹毛氈}{きぬ|まう|せん}のやうに
\ruby{滑}{なめ}らかで
\ruby{美}{うつく}しく
\ruby{廣}{ひろ}がつて
\ruby{居}{ゐ}る
\改行% 校正作業の簡略化のため
。
%
\原本頁{294-5}\改行%
\ruby{柔}{やはら}かい〳〵
しかも
\ruby{心}{こゝろ}の% 踊り字調整「〻(二の字点、揺すり点)に見えるが(ゝ)」
\ruby{正}{たゞ}しい% 踊り字調整「〻(二の字点、揺すり点)に濁点に見えるが(ゞ)」
\ruby{貿易風}{ぼう|{\換字{𛀁}}き|ふう}は、
%
\ruby[g]{恩愛}{おんあい}の
\ruby{溢}{あふ}るゝ% 踊り字調整「〻(二の字点、揺すり点)に見えるが(ゝ)」
ばかりの
\原本頁{294-6}\改行%
\ruby[g]{慈母}{は ゝ }の% 踊り字調整「〻(二の字点、揺すり点)に見えるが(ゝ)」
\ruby{手}{て}から
\ruby{出}{で}る
\ruby[g]{團{\換字{扇}}}{うちは }の
\ruby{風}{かぜ}が、
%
\ruby{睡}{ね}て
\ruby{居}{ゐ}る
\ruby[g]{嬰兒}{あかご }の
\ruby{顏}{かほ}へ
\ruby{當}{あた}るやうに
\改行% 校正作業の簡略化のため
、
%
\原本頁{294-7}\改行%
そより〳〵と
\ruby{後}{うしろ}から
\ruby{吹}{ふ}いて
\ruby{居}{ゐ}る。
%
\ruby{帆}{ほ}は
\ruby{一}{いつ}ぱいに
\ruby{張}{は}られた
まゝで% 踊り字調整「〻(二の字点、揺すり点)に見えるが(ゝ)」
パタリとも
\ruby{動}{うご}かぬ。
%
\ruby[g]{休番}{やすみ }
のものは
\ruby{皆}{みな}
\ruby[||j>]{熟}{じゆく}
\ruby[||j>]{睡}{ すゐ}して
% \ruby{熟睡}{じゆく|すゐ}して
\ruby{居}{ゐ}る。
%
\ruby[g]{當番}{たうばん}
のものも、
%
こくり〳〵と
\ruby{{\換字{遣}}}{や}つて
\ruby{居}{ゐ}る。
%
\ruby[g]{一切}{いつさい}の
\ruby[g]{用事}{ようじ }は
\ruby{皆}{みな}
\ruby{忘}{わす}れられて
\ruby{居}{ゐ}て
\改行% 校正作業の簡略化のため
、
%
\原本頁{294-10}\改行%
\ruby{胸}{むね}の
\ruby{中}{なか}にも
\ruby{頭}{あたま}の
\ruby{中}{なか}にも
\ruby{何}{なんに}も
\ruby{無}{な}い。
%
\ruby{何}{なに}
\ruby{一}{ひと}つ
\ruby{耳}{みゝ}に% 踊り字調整「〻(二の字点、揺すり点)に見えるが(ゝ)」
\ruby{立}{た}つ
\ruby{音}{おと}も
\ruby[g]{爲無}{し な }い。
%
\ruby{何}{なに}も
\ruby{見}{み}えん
\ruby{天}{そら}と
\ruby{水}{みづ}との
\ruby{間}{あひだ}を
\ruby[g]{茫然}{ぼ つ }として
\ruby{見}{み}て
\ruby{居}{ゐ}ると、
%
\ruby[g]{何時}{い つ }か
もう
\原本頁{295-1}\改行%
\ruby[g]{自{\換字{分}}}{じ ぶん}の
\ruby[g]{身體}{からだ }も
\ruby{{\換字{消}}}{き}えて
\ruby[g]{仕舞}{し ま }つて、
%
\ruby[g]{矢張}{や はり}
\ruby[g]{眞珠}{しんじゆ}の
\ruby{溶}{と}けた
やうな
\ruby{月}{つき}の
\ruby[<j||]{光}{ひかり}と% 行末行頭の境界付近なので特例処置を施す
\ruby[g]{一緖}{いつしよ}になつて、
%
\ruby[g]{大{\換字{空}}}{おほぞら}の
\ruby{中}{なか}に
\ruby{流}{なが}れ
\ruby{瀰}{わた}つて
\ruby{居}{ゐ}るやうな
\ruby{氣}{き}がする。
%
\原本頁{295-3}\改行%
\ruby[g]{左樣}{さ う }いふ
\ruby[||j>]{心}{こゝろ}% 踊り字調整「〻(二の字点、揺すり点)に見えるが(ゝ)」
\ruby[||j>]{持}{ もち}の
% \ruby{心持}{こゝろ|もち}の% 踊り字調整「〻(二の字点、揺すり点)に見えるが(ゝ)」
\ruby{仕}{し}た
ことが
ある。
%
\ruby{其}{そ}の
\ruby{時}{とき}の
\ruby{僕}{ぼく}の
\ruby{心}{こゝろ}の% 踊り字調整「〻(二の字点、揺すり点)に見えるが(ゝ)」
\ruby{中}{なか}の
\ruby[<j>]{味}{あぢはい}
といふものは、
%
とても
\ruby{僕}{ぼく}の
\ruby{口}{くち}では
\ruby{云}{い}ふ
\ruby{事}{こと}が
\ruby[g]{出來}{で き }んが、
%
あゝ% 踊り字調整「〻(二の字点、揺すり点)に見えるが(ゝ)」
\ruby{{\換字{若}}}{も}し
\ruby[g]{自{\換字{分}}}{じ ぶん}が
\ruby[g]{水野}{みづの }
であつたらば、
%
\ruby[g]{屹度}{きつと }
\ruby{此}{こ}の
\ruby[g]{美し}{うるは }
い
\ruby{何}{なん}とも
\ruby{云}{い}へぬ
\ruby{{\換字{感}}}{かん}じを、
%
\ruby[g]{{\換字{文}}字}{もんじ }に
\ruby{現}{あらは}して
\ruby{人}{ひと}に
\ruby{示}{しめ}す
\ruby{事}{こと}が
\ruby[g]{出來}{で き }る
であらう
もの
をと、
%
\ruby{深}{ふか}く
\ruby{其}{そ}の
\ruby{時}{とき}に
\ruby{僕}{ぼく}は
\ruby{思}{おも}つた。
%
\ruby[g]{何樣}{ど う }だ
\ruby{君}{きみ}
\ruby{一}{ひと}つ
\ruby[||j>]{海}{かい}
\ruby[||j>]{上}{じやう}に
% \ruby{海上}{かい|じやう}に
\ruby{出}{で}て
\ruby[g]{自然}{し ぜん}が
\ruby{君}{きみ}に
\ruby{何}{なに}を
\ruby{與}{あた}へる
かを
\ruby{試}{こゝろ}みては% 踊り字調整「〻(二の字点、揺すり点)に見えるが(ゝ)」
\ruby{見}{み}ないか。
%
\ruby{必}{かな}らず
\ruby{君}{きみ}を
\ruby{益}{{\換字{𛀁}}き}する
\ruby{事}{こと}は
\ruby{少}{すくな}く
\ruby{無}{な}からう。
%
\原本頁{295-9}\改行%
\ruby{凪}{なぎ}は
\ruby{凪}{なぎ}で
\ruby[g]{面白}{おもしろ}い、
%
\ruby{暴風雨}{あ|ら|し}は
\ruby{暴風雨}{あ|ら|し}で
\ruby[g]{面白}{おもしろ}い。
%
\ruby[||j>]{海}{かい}
\ruby[||j>]{上}{じやう}の
% \ruby{海上}{かい|じやう}の
\ruby[||j>]{生}{せい}
\ruby[||j>]{活}{くわつ}も
% \ruby{生活}{せい|くわつ}も
\ruby[g]{{\換字{半}}歳}{はんとし}
\ruby[||j>]{位}{ぐらゐ}は
\ruby{宜}{よ}からう。
%
\ruby{小}{ちひさ}な
\ruby[g]{屋根}{や ね }の
\ruby{下}{した}から
\ruby{飛}{と}び
\ruby{出}{だ}して
\ruby{見}{み}ないか。
%
\ruby{大熊星}{たい|ゆう|せい}の
\ruby{光}{ひかり}は
\ruby{北}{きた}で
\ruby{待}{ま}つて
\ruby{居}{ゐ}る、
%
\ruby{十字星}{じふ|じ|せい}の
\ruby{光}{ひかり}は
\ruby{南}{みなみ}で
\ruby[g]{莞爾}{に こ }ついて
\ruby{居}{ゐ}る。
%
\ruby[<j||]{大}{おほき}い〳〵% 行末行頭の境界付近なので特例処置を施す
\ruby{此}{こ}の
\ruby[g]{天地}{てんち }では
\ruby{無}{な}いか。
%
\ruby{米}{こめ}
\ruby{粒}{つぶ}に
\ruby[g]{{\換字{文}}字}{も じ }を
\ruby{書}{か}くやうに、
%
\ruby{細}{こまか}い
\ruby{事}{こと}
\原本頁{296-2}\改行%
ばかり
\ruby{考}{かんが}へ
\ruby{{\換字{込}}}{こ}まずとも、
%
\ruby{其}{そ}の
\ruby{米}{こめ}
\ruby{粒}{つぶ}は
\ruby{姑}{しばら}く
\ruby{傍}{わき}へ
\ruby{置}{お}いて、
%
\ruby[g]{自然}{し ぜん}の
\ruby[<j||]{大}{おほき}な% 行末行頭の境界付近なので特例処置を施す
\ruby[g]{景色}{け しき}に
\ruby{親}{した}しんで
\ruby{見}{み}ないか。
%
\ruby[g]{何樣}{ど う }だい
\ruby[g]{水野}{みづの }、
%
\ruby{何}{なん}と
\ruby{思}{おも}ふ?。
%
\ruby{君}{きみ}が
\原本頁{296-4}\改行%
\ruby{{\換字{嫌}}}{いや}なら
\ruby[g]{仕方}{し かた}は
\ruby{無}{な}いが、
%
\ruby[g]{學校}{がくかう}の
\ruby[g]{敎師}{けうし }も
\ruby{既}{もう}
よからう、
%
\ruby{一}{ひと}つ
\ruby{{\換字{遊}}}{あそ}んで
\ruby{見}{み}ては
\ruby[g]{何樣}{ど ん }なものだ?。
』

\原本頁{296-6}%
と、
%
\ruby{勉}{つと}めて
\ruby[g]{水野}{みづの }の
\ruby{意}{こゝろ}を% 踊り字調整「〻(二の字点、揺すり点)に見えるが(ゝ)」
\ruby{動}{うご}かさんと
\ruby[||j>]{心}{こゝろ}% 踊り字調整「〻(二の字点、揺すり点)に見えるが(ゝ)」
\ruby[||j>]{長}{ なが}く
% \ruby{心長}{こゝろ|なが}く% 踊り字調整「〻(二の字点、揺すり点)に見えるが(ゝ)」
\ruby{說}{と}きたるは、
%
\ruby[<j||]{全}{まつた}く% ルビ調整(原本通り)連なるルビを離す
\ruby[<j||]{心}{こゝろ}% 踊り字調整「〻(二の字点、揺すり点)に見えるが(ゝ)」
\ruby{搆}{がま}へして
% \ruby{心搆}{こゝろ|がま}へして% 踊り字調整「〻(二の字点、揺すり点)に見えるが(ゝ)」
\ruby{來}{きた}りし
なる
べし。

\原本頁{296-8}%
\ruby[g]{羽{\換字{勝}}}{は がち}の
\ruby{意}{こゝろ}の% 踊り字調整「〻(二の字点、揺すり点)に見えるが(ゝ)」
\ruby{解}{げ}せぬ
\ruby[g]{水野}{みづの }
ならねば、
%
\ruby{少}{すくな}からず
\ruby{其}{そ}の
\ruby{話}{はなし}に
\ruby{{\換字{情}}}{こゝろ}を% 踊り字調整「〻(二の字点、揺すり点)に見えるが(ゝ)」
\ruby{動}{うご}かして、
%
まことに
\ruby[g]{趣味}{おもむき}
\ruby{多}{おほ}かるべき
\ruby[||j>]{海}{かい}
\ruby[||j>]{上}{じやう}の
% \ruby{海上}{かい|じやう}の
\ruby[||j>]{生}{せい}
\ruby[||j>]{活}{くわつ}を
% \ruby{生活}{せい|くわつ}を
\ruby{試}{こゝろ}み% 踊り字調整「〻(二の字点、揺すり点)に見えるが(ゝ)」
たき
やうの
\ruby{念}{おもひ}も
\原本頁{296-10}\改行%
\ruby{起}{おこ}る
\ruby[<j>]{傍}{かたはら}、
%
\ruby[g]{羽{\換字{勝}}}{は がち}が
\ruby{我}{わ}が
ために
\ruby{思}{おもひ}を
\ruby{費}{つひや}して、
%
かゝる% 踊り字調整「〻(二の字点、揺すり点)に見えるが(ゝ)」
\ruby{事}{こと}を
\ruby{勸}{すゝ}めくるゝ% 踊り字調整「〻(二の字点、揺すり点)に見えるが(ゝ)」
\ruby{其}{その}
\ruby{意}{い}を
\ruby{{\換字{感}}}{かん}じて、
%
\ruby{嬉}{うれ}しとも
\ruby[<j>]{忝}{かたじけな}しとも
\ruby{胸}{むね}の
\ruby{中}{うち}には、
%
\ruby[g]{幾度}{いくたび}か
\ruby[g]{{\換字{感}}謝}{かんしや}して
また
\ruby[g]{{\換字{感}}謝}{かんしや}しぬ。

\原本頁{297-2}%
されど
\ruby[g]{水野}{みづの }は
\ruby{今}{いま}
こゝに% 踊り字調整「〻(二の字点、揺すり点)に見えるが(ゝ)」
\ruby[||j>]{其}{その}
\ruby[||j>]{言}{ことば}に
% \ruby{其言}{その|ことば}に
\ruby{{\換字{随}}}{したが}はんとも%「隨」グリフ変更 ⻖左円辶
\ruby{云}{い}ひ
\ruby{{\換字{兼}}}{か}ねて、
%
\ruby{何}{なに}と
\ruby{應}{こた}へんと
\ruby{思}{おも}ひ
めぐらすを、
%
\ruby{見}{み}
\ruby{取}{と}りて
\ruby[g]{羽{\換字{勝}}}{は がち}は
\ruby[g]{言葉}{ことば }
\ruby{{\換字{緩}}}{ゆる}く、

\原本頁{297-4}%
『
\ruby{何}{なに}も
\ruby{今}{いま}
\ruby{君}{きみ}の
\ruby[g]{{\換字{返}}辭}{へんじ }を
\ruby{求}{もと}めるのでは
\ruby{無}{な}い。
%
\ruby{{\換字{船}}}{ふね}は
\ruby{凡}{およ}そ
\ruby{十二月}{じふ|に|ぐわつ}に
\ruby{出}{だ}す
\原本頁{297-5}\改行%
\ruby[g]{心算}{つもり }
なのだから、
%
それまでは
\ruby{間}{あひだ}もある、
%
ゆつくり
\ruby{考}{かんが}へたまへ。
%
\ruby{{\換字{若}}}{も}し
\ruby{其}{それ}までに
\ruby[g]{何樣}{ど ん }な
\ruby{事}{こと}
でもあつて、
%
\ruby{海}{うみ}へ
\ruby{出}{で}たいと
\ruby{思}{おも}ふやうな
\ruby{事}{こと}
でもあつたら、
%
いつでも
\ruby[g]{相談}{さうだん}に
\ruby{乘}{の}る、
%
\ruby{悅}{よろこ}んで
\ruby{應}{おう}じる。
%
\ruby[g]{大洋}{たいやう}を
\ruby{見}{み}るのも
\ruby{宜}{よ}からうと
\ruby{思}{おも}ふよ。
』

\原本頁{297-9}%
と
\ruby{少}{すこ}しも
\ruby[g]{無理}{む り }
\ruby{{\換字{強}}}{じひ}の
\ruby[g]{氣味}{き み }
\ruby{無}{な}く
\ruby{云}{い}へば、

\原本頁{297-10}%
『
\ruby[g]{賛成}{さんせい}だ、
%
\ruby{大賛成}{だい|さん|せい}だ。
%
\ruby[g]{大洋}{たいやう}
\ruby[||j>]{生}{せい}
\ruby[||j>]{活}{くわつ}を
% \ruby{生活}{せい|くわつ}を
\ruby{{\換字{遣}}}{や}つて
\ruby{見}{み}ろ、
%
\ruby[g]{水野}{みづの }。
%
\ruby{女}{をんな}の
\ruby{傍}{そば}
なんぞに
へばり
\ruby{着}{つ}いて
\ruby{居}{ゐ}ないで、
%
\ruby{飛}{と}び
\ruby{出}{だ}せ、
%
\ruby[g]{羽{\換字{勝}}}{は がち}と
\ruby[g]{一緖}{いつしよ}に
\ruby{行}{ゆ}け。
%
お
\ruby{濱}{はま}さん
でさへ
\ruby{魯敏孫}{ろ|びん|そん}と
\ruby[g]{同棲}{どうせい}
しやうといふ
\ruby[g]{氣槪}{き がい}が
\ruby{有}{あ}るぢや
\ruby{無}{な}い
\改行% 校正作業の簡略化のため
か。
』

\原本頁{298-3}%
と
\ruby[g]{日方}{ひ かた}は
\ruby{却}{かへ}つて
\ruby{{\換字{強}}}{し}ひ
\ruby{立}{た}てたり。

\makeatletter
\@ifundefined{全三巻@一括ビルド}{%
\vspace{2zw}
{\Large{天うつ浪 {\normalsize 第二{\換字{終}}}}}
}
\makeatother
