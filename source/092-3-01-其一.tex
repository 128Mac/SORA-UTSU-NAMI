\makeatletter
\@ifundefined{全三巻@一括ビルド}{%
{\huge
\ruby{天}{そら} う つ %空白有り
\ruby{浪}{なみ}}  {\normalsize 第三}
\vspace*{4zw}
\Entry{其一}
}
\makeatother

% メモ 校正終了 2024-05-09 2024-06-06
\原本頁{1-3}%
\ruby{親}{した}しきが
\ruby{中}{なか}の
\ruby{{\換字{絕}}}{た}えて
\ruby{久}{ひさ}しくして
\ruby{相}{あひ}
\ruby{會}{あ}ひたるに、
%
\ruby{痛飮}{つう|いん}
\ruby[||j>]{快}{くわい}
\ruby[||j>]{談}{ だん}して
% \ruby{快談}{くわい|だん}して
\ruby{歸}{かへ}るを
\ruby{忘}{わす}るゝ
\ruby{日方}{ひ|かた}を、
%
\ruby{幾度}{いく|たび}か
\ruby{羽{\換字{勝}}}{は|がち}の
\ruby{促}{うなが}し
\ruby{立}{た}てゝ、
%
\ruby{漸}{やうや}くに
\ruby[|g|]{二人}{ふたり}の
\ruby[<j||]{暇}{いとま}を% 行末行頭の境界付近なので特例処置を施す
\ruby{告}{つ}げし
\ruby{時}{とき}は、
%
\ruby{日}{ひ}は
\ruby{既}{すで}に
\ruby{暮}{く}れ
\ruby{果}{は}てゝ
\ruby{一時間}{いち|じ|かん}
\ruby{餘}{よ}も
\ruby{經}{へ}たり。

\原本頁{1-6}%
お
\ruby{濱}{はま}
お
\ruby{鍋}{なべ}は
\ruby{後片付}{あと|かた|づけ}に
\ruby{忙}{せは}しく、
%
\ruby{水野}{みづ|の}は
\ruby{獨}{ひと}り
\ruby{机}{つくゑ}に
\ruby{憑}{よ}つて
\ruby{醉}{よひ}を% 「醉」は原本通り「よ」で調整
\ruby{吐}{は}きつつ、
%
\ruby{飮}{の}み
\ruby{慣}{な}れぬ
\ruby{酒}{さけ}に
\ruby{聊}{いさゝ}か
\ruby{苦}{くるし}みて、
%
\ruby{頻}{しきり}に
\ruby{微溫}{ぬ|る}き
\ruby{茶}{ちや}に
\ruby{渴}{かわき}を% 原本通りルビは「「か(わ)き」
\ruby{癒}{いや}し
ながら、
%
\ruby{羽{\換字{勝}}}{は|がち}が
\ruby{言}{い}ひたる
\ruby[||j>]{海}{かい}
\ruby[||j>]{上}{じやう}の
% \ruby{海上}{かい|じやう}の
\ruby[||j>]{生}{せい}
\ruby[||j>]{活}{くわつ}の
% \ruby{生活}{せい|くわつ}の
\ruby{如何}{い|か}に
\ruby{趣味}{おも|むき}
ある
べき
かを
\ruby{想}{おも}ひ
\原本頁{1-9}\改行%
\ruby{{\換字{遣}}}{や}り、
%
\ruby{或}{あるひ}は
\ruby{{\換字{又}}}{また}
\ruby[|-|]{飜}{ひるがへ}つて
\ruby{日方}{ひ|かた}が
\ruby{我}{われ}を
\ruby{撲}{う}ちたる
\ruby{時}{とき}の
\ruby[<j>]{勢}{いきほひ}の
\ruby{烈}{はげ}し
かりし
こと
などを
\ruby{思}{おも}ひ
\ruby{{\換字{廻}}}{めぐ}らす
\ruby{折}{をり}しも、
%
\ruby{日方}{ひ|かた}が
\ruby{引}{ひ}き
\ruby{出}{いだ}し
\ruby{散}{ち}らしたる
\ruby{我}{わが}
\ruby{雜記}{ざつ|き}は
\ruby{我}{わ}が
\ruby{膝}{ひざ}
\ruby{{\換字{近}}}{ちか}く
ありて、
%
\ruby{其}{そ}の
\ruby{裏面}{う|ら}に
\ruby{我}{わ}が
\ruby{落書}{らく|がき}
したる
\ruby{萬葉}{まん|えふ}の
\ruby{幾首}{いく|しゆ}の
\ruby{歌}{うた}の、
%
\ruby{横}{よこ}に、
%
\ruby{縦}{たて}に、
%
\ruby{{\換字{逆}}}{さか}さまに
なりて
\ruby{我}{われ}を
\ruby{慰}{なぐさ}むるが
\ruby{如}{ごと}きが
\ruby{偶然}{ふ|と}
\ruby{眼}{め}に
\ruby{入}{い}りたり。

\原本頁{2-5}%
\ruby{唐詩}{たう|し}は
\ruby{好}{この}みて
\ruby{誦}{しよう}
すれども
\ruby{和歌}{わ|か}には
\ruby{疎}{うと}き
\ruby{日方}{ひ|かた}の、
%
いづれも
\ruby{此}{これ}は
\ruby{{\換字{古}}}{ふる}き
\ruby{歌}{うた}なるを
\ruby{知}{し}らで、
%
\ruby{我}{わ}が
\ruby{詠}{えい}じたる
ものゝ
やうに
\ruby{思}{おも}ひ
\ruby{{\換字{込}}}{こ}みて
\ruby{我}{われ}を
\ruby{罵}{のゝし}りしが、
%
\ruby{言}{い}ひ
\ruby{解}{と}かんも
\ruby{煩}{うるさ}ければ
\ruby{其}{その}
\ruby{儘}{まゝ}に
\ruby{寃罪}{つ|み}を
\ruby{負}{お}いたる、
%
\ruby{其事}{そ|れ}も
\ruby{思}{おも}へば
\ruby{何}{なに}か
\ruby{厭}{いと}
はしかる
べきや、
%
\ruby{歌}{うた}は
\ruby{皆}{みな}
\ruby{他}{ひと}の
\ruby{歌}{うた}
ながら、
%
\ruby{詠}{よ}まれたる
\ruby{思}{おもひ}は
\ruby{我}{わ}が
\ruby{思}{おもひ}なる
をと、
%
\ruby{凝然}{じ|つ}と
\ruby{見入}{み|い}り
つゝ、
%
\ruby{{\換字{文}}字}{も|じ}を
\ruby{辿}{たど}りて、
%
\ruby{久方}{ひさ|かた}の
\ruby{天}{あま}つみ
\ruby{{\換字{空}}}{そら}に
\ruby{照}{て}れる
\ruby{日}{ひ}の
\ruby{亡}{う}せなん
\ruby{日}{ひ}こそ
\ruby{我}{わ}が
\ruby{戀止}{こひ|や}まめ、
%
と
\ruby{心}{こゝろ}の
\ruby{中}{うち}に
\ruby{自}{みづか}ら
\ruby{讀}{よ}みたり。

\原本頁{3-1}%
\ruby{醉}{よひ}に% 「醉」は原本通り「よ」で調整
\ruby{我}{わ}が
\ruby{心}{こゝろ}は
\ruby{蒸}{む}さるゝが
\ruby{如}{ごと}くにして、
%
\ruby{身}{み}の
\ruby{筋}{すぢ}は
\ruby{弛}{ゆる}み
\ruby{骨{\換字{節}}}{ほ|ね}は
\ruby{和}{やはら}いで
\ruby[<j>]{快}{こゝろよ}く
\ruby{懈}{だる}き
やうなるに、
%
\ruby{精神}{たま|しひ}は
\ruby{何}{なに}にか
\ruby{憧}{あくが}るゝ、
%
\ruby{{\換字{空}}}{あだ}に
\ruby{{\換字{浮}}}{う}きて
\ruby{止}{や}まず、
%
たゞ〳〵
\ruby{我}{われ}を
\ruby{笑}{ゑ}ますに
\ruby{足}{た}るものを
\ruby{得}{え}て、
%
\ruby{面白}{おも|しろ}く
\ruby{破顏}{は|がん}して
\ruby{笑}{ゑ}みたき
やうの
\ruby{氣}{き}のする
\ruby{水野}{みづ|の}は、
%
\ruby{明}{あき}らかに
\ruby{此}{これ}を
\ruby{酒}{さけ}の
さする
\ruby{事}{わざ}と
\ruby{知}{し}り
ながら、
%
\ruby{{\換字{猶}}}{なほ}
\ruby{我}{わ}が
\ruby{心}{こゝろ}の
\ruby[|g|]{自然}{おのづ}と
\ruby{動}{うご}くに
\ruby{任}{まか}せて、
%
\ruby{何}{なに}と
せん
\ruby{念慮}{おも|ひ}も
\ruby{無}{な}く
\ruby{恍然}{うつ|とり}となり
\ruby{居}{ゐ}たり。

\原本頁{3-7}%
\ruby{珍}{めづ}らしくも
\ruby{水野}{みづ|の}の
\ruby{面}{おもて}は
\ruby[|-|]{{\換字{暖}}}{あたゝか}げに
\ruby{微紅色}{うす|くれ|なゐ}に、
%
\ruby{其}{その}
\ruby{眼}{め}は
\ruby{優}{やさ}しき
\ruby{光}{ひかり}を
\ruby{湛}{たゝ}へたれど、
%
\ruby{例}{いつも}の
\ruby{癖}{くせ}の
\ruby[||j>]{物}{もの}
\ruby[||j>]{思}{おもひ}に
% \ruby{物思}{もの|おもひ}に
\ruby{耽}{ふけ}れるかと
\ruby{見}{み}えて
\ruby{身動}{み|うご}き
もせざるに
\改行% 校正作業の簡略化のため
、
%
% \原本頁{3-9}\改行% ここはコメントにしておく
\ruby[|g|]{此方}{こなた}に
\ruby{入}{い}り
\ruby{來}{きた}れる
\ruby{吉右衛門}{きち||ゑ|もん}は、

\原本頁{3-10}%
『
\ruby{御酒}{ご|しゆ}の
\ruby{後}{あと}ですから
\ruby{御考}{お|かんが}へ
\ruby{事}{ごと}は
\ruby{毒}{どく}です。
%
\ruby[<j||]{些}{ちつと}
\ruby{御話}{お|はなし}
でも
なさいませんか。
%
\ruby{日方}{ひ|かた}さんと
\ruby{仰}{おつし}ある
\ruby{方}{かた}は
\ruby{結構}{けつ|こう}な
\ruby{方}{かた}ですが、
%
\ruby{軍人}{ぐん|じん}で
\ruby{在}{い}らつしやる
だけに
\ruby{荒}{あら}い
\ruby{方}{かた}ですネ。
%
\換字{志}かし
\ruby{羽{\換字{勝}}}{は|がち}さんと
\ruby{仰}{おつし}ある
\ruby{方}{かた}でも
\ruby{彼}{あ}の
\原本頁{4-2}\改行%
\ruby{方}{かた}でも、
%
\ruby[||j>]{皆}{みんな}
\ruby[||j>]{心}{ しん}
\ruby[||j>]{底}{ そこ}
から
\ruby[|g|]{貴下}{あなた}を
\ruby{思}{おも}つて
\ruby{居}{ゐ}らつしやる、
%
\ruby[|g|]{眞實}{ほんと}に
\ruby{結構}{けつ|こう}な
\ruby{好}{い}い
\ruby{方々}{かた|〴〵}です。
%
\ruby{御氣}{お|き}に
\ruby{入}{い}らない
\ruby{事}{こと}も
\ruby{仰}{おつし}あつてゞしやうが、
%
\ruby{何}{なに}も
\ruby{彼}{か}も
\ruby[||j>]{皆}{みんな}
\ruby[||j>]{御親切}{ ご|しん|せつ}
から
\ruby{出}{で}た
\ruby{事}{こと}ですから、
%
\ruby{御氣}{お|き}に
\ruby{御止}{お|と}め
なすつて
\ruby{惡}{わる}く
なんぞ
\ruby{御考}{お|かんが}へ
なさらないが
\ruby{宜}{よ}うございます。
』

\原本頁{4-6}%
と
\ruby{言}{い}ひたり。

\原本頁{4-7}%
\ruby{吉右衛門}{きち||ゑ|もん}は
\ruby{水野}{みづ|の}が
\ruby{身動}{み|うご}きもせで
\ruby{物}{もの}を
\ruby{思}{おも}へるを、
%
\ruby{胸}{むね}の
\ruby{中}{うち}に
\ruby{羽{\換字{勝}}}{は|がち}
\ruby{日方}{ひ|かた}
が
\ruby{振舞}{ふる|まひ}
\ruby{言語}{もの|いひ}を
\ruby{忘}{わす}れ
\ruby{{\換字{兼}}}{か}ねて
\ruby{繰}{く}り
\ruby{{\換字{返}}}{かへ}し
\ruby{繰}{く}り
\ruby{{\換字{返}}}{かへ}せると
\ruby{猜}{すゐ}したる
なるが、
%
かく
\ruby{云}{い}はれて
\ruby{水野}{みづ|の}は
\ruby{我}{われ}に
\ruby{復}{かへ}りて
ハツと
\ruby{驚}{おどろ}きぬ。
%
\ruby{實}{げ}に
\ruby{我}{われ}は
\ruby{今}{いま}
\原本頁{4-10}\改行%
\ruby{此}{この}
\ruby{老人}{とし|より}が
\ruby{言}{い}へるが
\ruby{如}{ごと}くに、
%
\ruby{羽{\換字{勝}}}{は|がち}
\ruby{日方}{ひ|かた}
の
\ruby{我}{われ}に
\ruby{與}{あた}へたる
\ruby{數々}{かず|〴〵}の
\ruby{言葉}{こと|ば}に
\ruby{就}{つ}いて
\ruby{物}{もの}
をこそ
\ruby{思}{おも}ふべき
\ruby{筈}{はず}なるに、
%
\ruby{我}{われ}は
\ruby{今}{いま}
\ruby{抑}{そも}
\ruby{何}{なに}をか
\ruby{思}{おも}ひ
\ruby{居}{ゐ}し
\改行% 校正作業の簡略化のため
。
%
\原本頁{5-1}\改行%
\ruby{羽{\換字{勝}}}{は|がち}が
\ruby{言}{い}ひし
\ruby{海}{うみ}の
\ruby{上}{うへ}の
\ruby[||j>]{生}{せい}
\ruby[||j>]{活}{くわつ}に
% \ruby{生活}{せい|くわつ}に
\ruby{就}{つ}いて
\ruby{歟}{か}。
%
あらず、
%
\ruby{海}{うみ}の
\ruby{上}{うへ}などの
\ruby{事}{こと}は
\ruby{既}{すで}に
\ruby{思}{おも}はざりき。
%
\ruby{日方}{ひ|かた}が
\ruby{我}{われ}に
\ruby{加}{くは}へし
\ruby{鐵{\換字{拳}}}{てつ|けん}に
\ruby{就}{つ}いてか。
%
あらず、
%
\ruby{日方}{ひ|かた}が
\ruby{事}{こと}などは
\ruby{既}{すで}に
\ruby{忘}{わす}れ
\ruby{居}{ゐ}たりき。
%
\ruby{我}{われ}は
\ruby{我}{わ}が
\ruby{胸}{むね}の
\ruby{中}{うち}に
\ruby{何}{なに}を
\ruby{思}{おも}ひ
\ruby{居}{ゐ}たりしや。
%
\ruby{我}{われ}は
\ruby{今日}{け|ふ}
\ruby{日方}{ひ|かた}に
\ruby{逢}{あ}はず
\ruby{羽{\換字{勝}}}{は|がち}に
\ruby{逢}{あ}はざりし
\ruby{{\換字{前}}}{まへ}、
%
\ruby{大士堂}{だい|し|だう}
\ruby{{\換字{前}}}{ぜん}に
\ruby{圖}{はか}らず
\ruby{相}{あひ}
\ruby{會}{あ}ひたる
\ruby{彼}{か}の
\ruby{物}{もの}
\ruby{優}{やさ}しき
お
\ruby{龍}{りう}を
\ruby{思}{おも}ひ
\ruby{居}{ゐ}たりしなり。
%
\ruby{如何}{い|か}なる
\ruby{人}{ひと}の
\ruby{憐}{あはれ}みをも
\ruby{惹}{ひ}かんとも
\ruby{思}{おも}はざりし
\ruby{愚}{おろか}なる
\ruby{此}{こ}の
\ruby{我}{わ}が
ために、
%
\ruby{我}{わ}が
\ruby{思}{おも}へる
\ruby{五十子}{い|そ|こ}の
\ruby{病}{やまひ}の
\ruby{疾}{と}く
\ruby{癒}{なほ}れかしと、
%
\ruby{日々}{ひ|ゞ}に
\原本頁{5-8}\改行%
\ruby{歩}{あゆみ}を
\ruby{{\換字{運}}}{はこ}びて
\ruby{祈}{いの}りて
\ruby{吳}{く}れし
といふ
\ruby{優}{やさ}しくも
\ruby{優}{やさ}しき
\ruby{彼}{か}の
お
\ruby{龍}{りう}をば
\ruby{思}{おも}ひ
\ruby{居}{ゐ}たりしなり。
%
\ruby{其}{そ}の
\ruby{親}{した}しき
\ruby{友}{とも}なり
といふ
\ruby{驚}{おどろ}くべき
\ruby{美人}{び|じん}%
{---}{---}%
\ruby{年}{とし}は
\ruby{既}{すで}に
\ruby{三十}{さん|じふ}に
\ruby{{\換字{近}}}{ちか}かる
べき
ながら
\ruby{人}{ひと}を
\ruby{驚}{おどろ}かす
\ruby{美人}{び|じん}の、
%
\ruby{扮装}{いで|たち}も
\ruby{極}{きは}めて
\ruby{立派}{りつ|ぱ}なりし
おとうと
やら
いへる
より、
%
お
\ruby{龍}{りう}が
\ruby{悲}{かな}しき
\ruby{身}{み}の
\ruby{上}{うへ}を
\原本頁{6-1}\改行%
\ruby{朧氣}{おぼろ|げ}に
\ruby{聞}{き}きて、
%
\ruby{{\換字{終}}}{つひ}に
\ruby{堪}{こら}へ
\ruby{得}{え}て、
%
\ruby{我}{われ}は
\ruby{涙}{なみだ}を
\ruby{濺}{そゝ}ぎて
\ruby{泣}{な}きたり
しが、
%
\ruby{其}{その}
\ruby{憐}{あは}れなる
お
\ruby{龍}{りう}を
のみ
\ruby{思}{おも}ひ
\ruby{居}{ゐ}たり
しなり。
%
\ruby{美}{うる}はしく
\ruby{淸}{きよ}かりし
\ruby{戀}{こひ}の
\ruby{誠}{まこと}の、
%
人の
\ruby{僞}{いつは}りに
\ruby{{\換字{情}}無}{なさけ|な}く
\ruby{廢}{すた}りて、
%
\ruby{狂}{くる}ひに
\ruby{狂}{くる}ひ、
%
\ruby{悲}{かなし}みに
\ruby{悲}{かなし}みたる
\ruby{末}{すゑ}の
\ruby{其}{そ}の
\ruby{女}{ひと}の、
%
\ruby{苦}{くる}しき
\ruby{思}{おも}ひに
\ruby{疲}{つか}るゝ
\ruby{我}{われ}を
\ruby{憐}{あは}れと
\ruby{見}{み}て、
%
\ruby{{\換字{猶}}}{なほ}
\ruby{有}{あ}り
\原本頁{6-5}\改行%
\ruby{餘}{あま}る
\ruby{優}{やさ}しき
\ruby{{\換字{情}}}{こゝろ}を
\ruby{傾}{かたむ}けて
\ruby{我}{われ}に
\ruby{寄}{よ}せ
くるゝ
\ruby{其}{そ}の
\ruby{行爲}{ふる|まひ}
はかりに% (ばかり)でなく原本通り(はかり)とする
\ruby[<j>]{樂}{たのしみ}% 行末行頭の境界付近なので特例処置を施す
\ruby[||j>]{無}{ な }き% 行末行頭の境界付近なので特例処置を施す
\ruby{今}{いま}の
\ruby[|g|]{自己}{おのれ}を
\ruby{自}{みづか}ら
\ruby{慰}{なぐさ}むる
といふ
\ruby{薄命}{はく|めい}の
お
\ruby{龍}{りう}を
のみ
\ruby{思}{おも}ひ
\ruby{居}{ゐ}たり
しなり。
%
\ruby{我}{われ}は
\ruby{我}{わ}が
\ruby{{\換字{迷}}}{まよ}ひて
\ruby{泣}{な}き、
%
\ruby{苦}{くるし}みて
\ruby{悶}{もだ}えたる
\ruby{心}{こゝろ}の
\ruby{闇}{やみ}に、
%
\ruby{優}{やさ}しき
\ruby[<j||]{光}{ひかり}の
\ruby[<j>]{線}{いとすぢ}を
\ruby{投}{な}げ
\ruby{吳}{く}るゝ
\ruby{星}{ほし}を
\ruby{認}{みと}めし
\ruby{心地}{こゝ|ち}して、
%
\ruby{我}{わ}が
\ruby{其}{その}
\ruby{人}{ひと}に
\ruby{會}{あ}ひし
をば
\原本頁{6-9}\改行%
\ruby{滿身}{まん|しん}に
\ruby{悅}{よろこ}び
\ruby{{\換字{愉}}}{よろこ}びつ、
%
\ruby{我}{わ}が
\ruby{懷}{なつか}しき
お
\ruby{龍}{りう}をのみ
\ruby{思}{おも}ひ
\ruby{居}{ゐ}たり
しなり。
