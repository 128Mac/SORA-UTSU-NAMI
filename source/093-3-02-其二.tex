\Entry{其二}

% メモ 校正終了 2024-05-10 2024-06-06
\原本頁{7-1}%
\ruby{思}{おも}はん
とも
せずして
\ruby{思}{おも}ひ
\ruby{居}{ゐ}たるは、
%
\ruby{心}{こゝろ}の
\ruby{其}{それ}に
\ruby{染}{そ}み
たれば
なるべし。
%
されども
\ruby{吉右衛門}{きち||ゑ|もん}に
\ruby{話}{はな}し
\ruby{掛}{か}けられて、
%
\ruby[g]{水野}{みづの }は
\ruby{忽}{たちま}ち
\ruby{覺}{さ}めたる
\ruby{如}{ごと}く、

\原本頁{7-4}%
『
\ruby{惡}{わる}く
\ruby{思}{おも}ふ
なんぞ
といふ
\ruby[<j>]{考}{かんがへ}が
\ruby[g]{何樣}{ど う }して
\ruby{私}{わたし}に
‥‥。
%
\ruby[g]{羽{\換字{勝}}}{は がち}だつて
\ruby[g]{日方}{ひ かた}だつて
\ruby[<j||]{皆}{みんな}
\ruby[||j>]{私}{わたし}の
\ruby{兄}{あに}
\ruby[g]{同樣}{どうよう}
なのだもの!、
%
\ruby{何}{なに}を
\ruby{言}{い}はれたつて
\ruby{惡}{わる}く
\ruby{取}{と}つたり
\ruby{氣}{き}に
\ruby{仕}{し}たり
するやうな
\ruby{事}{こと}は
\ruby{有}{あ}りは
\ruby{仕}{し}ない
ので。
%
\ruby{私}{わたし}は
\ruby{今}{いま}
ただ% ルビ調整(原本通り)非踊り字表記(行末行頭の境界付近)
\ruby[g]{恍然}{うつかり}
\footnote{(うっかり)には「心惹かれ他に注意の向かない様=うっとり」の意味もあるので、原本通りとする %
(国会図書館 コマ番号 7/146 p-007 l-07)}%
として
\ruby{居}{ゐ}た
ところ
でした。
%
いや
\ruby[g]{今日}{け ふ }は
\ruby[g]{大層}{たいそう}
\ruby{御世話}{お|せ|わ}でした
\改行% 校正作業の簡略化のため
。
%
\原本頁{7-8}\改行%
お
\ruby{蔭}{かげ}で
\ruby[g]{一同}{みんな }
\ruby{悅}{よろこ}んで
\ruby{歸}{かへ}りましたが、
%
あれを
\ruby{殘}{のこ}らず
\ruby{御厄介}{ご|やく|かい}に
なる
\ruby[g]{理由}{いはれ }は
ありません
から、
%
せめて
\ruby[g]{御酒}{ご しゆ}だけも
\ruby{私}{わたし}の
\ruby{{\換字{分}}}{ぶん}にして、
』

\原本頁{7-10}%
と
\ruby{云}{い}ひ
\ruby{掛}{か}くるを
\ruby[g]{主人}{あるじ }は% ルビ調整(原本通り)非グループルビ
\ruby{悅}{よろこ}ばぬ
\ruby{氣}{げ}なる
\ruby{顏}{かほ}して、

\原本頁{7-11}%
『
また
\ruby[g]{水野}{みづの }さんの
\ruby[g]{他人}{た にん}
\ruby[g]{行儀}{ぎやうぎ}が
はじまつた。
%
\ruby{几帳面}{きち|やう|めん}
\ruby{{\換字{過}}}{す}ぎて
\ruby[g]{厭氣}{いやき }が
さします。
%
\ruby{宜}{い}いぢやあ
\ruby{有}{あ}りませんか
\ruby[g]{些細}{わづか }の
\ruby{事}{こと}
ですもの。
』

\原本頁{8-2}%
と
\ruby[g]{打{\換字{消}}}{うちけ }しつ、

\原本頁{8-3}%
『
それは
\ruby[g]{左樣}{さ う }と
\ruby[|g|]{先刻}{さつき}
\ruby[g]{老夫}{わたくし}が
\ruby[g]{高田}{たかた }さんに
\ruby{逢}{あ}ひましたら、
%
\ruby[g]{水野}{みづの }さん
% \原本頁{8-4}\改行%
に
\ruby[g]{一寸}{ちよつと}
\ruby{來}{き}て
\ruby{貰}{もら}ひたい
ことが
あるから
\ruby[g]{然樣}{さ う }
\ruby{云}{い}つて
\ruby{吳}{く}れ、
%
\ruby[g]{他人}{ひ と }の
\ruby{居}{ゐ}ない
\ruby{時}{とき}
\ruby{會}{あ}いたい
から
\ruby{成}{な}るべくば
\ruby[g]{今夜}{こんや }
あたり、
%
といふ
\ruby[g]{御談}{おはなし}で
ございました。
%
\ruby{御酒氣}{ご|しゆ|き}は
\ruby[g]{大{\換字{分}}}{だいぶ }
\ruby[g]{御有}{お あ }んなさる
けれども、
%
\ruby[|g|]{貴下}{あなた}の
\ruby{事}{こと}ですから
\ruby{宜}{よ}う
ございましやう。
%
\ruby{{\換字{更}}}{ふ}けない
\ruby{中}{うち}
\ruby[g]{一寸}{ちよつと}
\ruby{行}{い}つて
\ruby{居}{ゐ}らつしやい
ませんか。
』

\原本頁{8-9}%
と
\ruby{云}{い}ひ
\ruby{出}{いだ}したり。

\原本頁{8-10}%
\ruby[g]{高田}{たかた }は
\ruby{我}{わ}が
\ruby{職}{しよく}を
\ruby{奉}{ほう}ずる
\ruby[g]{學校}{がくかう}の
\ruby{長}{ちやう}にして、
%
\ruby{吉右衛門}{きち||ゑ|もん}とも
\ruby[<j||]{心}{こゝろ}% ルビ調整(配置位置調整)若干ルビ配置が異なるけど
\ruby[||j>]{易}{やす}き
% \ruby{心易}{こゝろ|やす}き
\ruby[<j||]{男}{をとこ}% 行末行頭の境界付近なので特例処置を施す
なれば、
%
\ruby[g]{水野}{みづの }は
\ruby{{\換字{更}}}{さら}に
\ruby{考}{かんが}ふるまでも
\ruby{無}{な}くして、

\原本頁{9-1}%
『
\ruby{何}{なん}だか
さつぱり
\ruby{{\換字{分}}}{わか}らない
けれども、
%
\ruby[g]{其樣}{そ ん }なら
\ruby[g]{一寸}{ちよいと}
\ruby{行}{い}つて
\ruby{來}{き}ましやう。
』

\原本頁{9-3}%
と
\ruby{答}{こた}へつ、
%
\ruby{吉右衛門}{きち||ゑ|もん}が
お
\ruby{濱}{はま}を
\ruby{呼}{よ}び
\ruby{立}{た}てゝ、
%
\ruby[||j>]{提}{ちやう}
\ruby[||j>]{灯}{ ちん}を
% \ruby{提灯}{ちやう|ちん}を
と
\ruby{云}{い}ふを、
%
それにも
\ruby{及}{およ}ばずと
\ruby{制}{とゞ}め、
%
たゞ
\ruby{纔}{わづか}に
\ruby{帶}{おび}
\ruby{締}{し}め
\ruby{直}{なほ}しゝ
のみにて
\ruby[g]{立出}{たちい }でた
\原本頁{9-5}\改行%
り。

\原本頁{9-6}%
\ruby[g]{高田}{たかた }が
\ruby{家}{いへ}は
\ruby[g]{學校}{がくかう}の
\ruby{直}{すぐ}
\ruby[g]{後面}{うしろ }にて、
%
\ruby[g]{農家}{のうか }
\ruby{{\換字{造}}}{づく}りにて
こそは
あらね、
%
\ruby[g]{趣味}{おもむき}も
\ruby{無}{な}き
\ruby{{\換字{平}}々凡々}{へい|〳〵|ぼん|〴〵}の
\ruby[g]{住居}{すまゐ }なるが、% ルビ調整(原本通り)非グループルビ
%
\ruby[|g|]{主人}{あるじ}も
\ruby{其}{その}
\ruby{家}{いへ}に
\ruby[|g|]{相應}{ふさは}しき
\ruby{{\換字{平}}々凡々}{へい|〳〵|ぼん|〴〵}の、
%
\ruby{何}{なん}の
\ruby[g]{奇處}{き しよ}も
\ruby{無}{な}き
\ruby[g]{五十}{い そ }
\ruby{男}{をとこ}にて、
%
\ruby[<j||]{農}{ひやく}
\ruby[||j>]{夫}{しやう}にて
% \ruby{農夫}{ひやく|しやう}にて
こそは
あらね、
%
\ruby{面白味}{おも|しろ|み}も
\ruby{無}{な}き
\ruby{氣}{き}の
\ruby{小}{ちひさ}なる
\ruby[g]{謹直}{まじめ }
\ruby[g]{三昧}{ざんまい}の
\ruby{人}{ひと}なり。

\原本頁{9-10}%
\ruby[g]{{\換字{半}}白}{はんぱく}の
\ruby{髮}{かみ}の
\ruby{毛}{け}は
\ruby[g]{割合}{わりあひ}に
\ruby{多}{おほ}かれども、
%
\ruby[g]{光澤}{つ や }
\ruby{無}{な}く
\ruby[g]{黃色}{き いろ}に
\ruby{痩}{や}せきつたる
\ruby{顏}{かほ}の、
%
\ruby{口}{くち}の
\ruby{傍}{はた}の
\ruby[g]{條{\換字{文}}}{す ぢ }、
%
\ruby{額}{ひたひ}の
\ruby{皺}{しわ}など
\ruby[g]{目立}{め だ }つて
\ruby{深}{ふか}く、
%
\ruby[g]{光無}{ひかりな}き
\ruby{小}{ちひさ}なる
\ruby{眼}{め}、
%
\ruby[g]{骨立}{ほねだ }つて
\ruby{高}{たか}き
\ruby{鼻}{はな}、
%
おちつきの
\ruby{無}{な}き
\ruby[g]{起居}{たちゐ }
\ruby[g]{動作}{ふるまひ}、
%
\ruby[g]{活氣}{いきほひ}の
\ruby{無}{な}き
\原本頁{10-2}\改行%
\ruby{物}{もの}の
\ruby{言}{い}ひぶり、
%
すべての
\ruby[|g|]{乾燥}{ひから}びたる
\ruby[g]{狀態}{ありさま}は、
%
\ruby[g]{如何}{い か }にも
\ruby{能}{よ}く
\ruby{此}{この}
\ruby{人}{ひと}の、% 「、」を詰め込んでいるようだ、「の」以下で 30文字あり
    『
    \ruby{人}{ひと}の
    \ruby{子}{こ}を
    \ruby{{\換字{誤}}}{あやま}るが
    \ruby{如}{ごと}き
    \ruby{{\換字{強}}}{つよ}き
    \ruby{人}{ひと}
    』
ならで、% 「、」を詰め込んでいるようだ
%
    『
    \ruby{決}{けつ}して
    \ruby{人}{ひと}の
    \ruby{子}{こ}を
    \ruby[<j||]{{\換字{害}}}{そこな}はぬ% 行末行頭の境界付近なので特例処置を施す
    \ruby{{\換字{古}}}{ふ}りたる
    \ruby{敎育家}{けう|いく|か}
    』
たる
\ruby{事}{こと}をば
\ruby{現}{あらは}し
\ruby{示}{しめ}せり。

\原本頁{10-5}%
\ruby[g]{高田}{たかた }は
\ruby{今}{いま}
\ruby[g]{水野}{みづの }の
\ruby{來}{きた}り
\ruby{訪}{と}ふに
\ruby{會}{あ}ひて、
%
\ruby[|g|]{一昨日}{をとゝひ}も
\ruby[|g|]{昨日}{きのふ}も
\ruby{會}{あ}ひたる
\ruby[g]{同士}{どうし }
なるに、
%
\ruby[g]{三年}{さんねん}
\ruby[g]{四年}{よ ねん}も
\ruby{隔}{へだ}てゝ
\ruby{面}{おもて}を
\ruby{見}{み}たるものゝ
\ruby{如}{ごと}く、
%
\ruby[g]{慇懃}{いんぎん}に
\ruby[g]{時候}{じ こう}の
\ruby[g]{挨拶}{あいさつ}など
\ruby[g]{管々}{くだ〴〵}しく
\ruby{仕}{し}て、
%
\ruby{三十匁}{さん|じふ|め}
ばかりの
\ruby{{\換字{廉}}價茶}{や|す|ぢや}を
\ruby[g]{事々}{こと〴〵}しく
\原本頁{10-8}\改行%
\ruby{湯}{ゆ}を
\ruby{冷}{さ}まし
などして
\ruby{入}{い}れ、
%
\ruby{隱}{かく}れ
\ruby{蓑}{みの}、
%
\ruby{隱}{かく}れ
\ruby{笠}{がさ}、
%
\ruby[g]{打出}{うちで }の
\ruby{槌}{つち}
なんどの
\ruby[||j>]{寶}{たから}
\ruby[||j>]{盡}{ づく}しを
% \ruby{寶盡}{たから|づく}しを
\ruby{描}{ゑが}きたる
\ruby[g]{水金}{みづきん}の
\ruby{光}{ひか}り
\ruby[g]{爛々}{きら〳〵}とする
\ruby{菓子鉢}{くわ|し|ばち}に、
%
\ruby[g]{三月}{み つき}も
\ruby{{\換字{前}}}{まへ}より
\ruby{盛}{も}られし
\ruby{儘}{まゝ}かと
\ruby{想}{おも}はるゝ
やうなる
\ruby[g]{最中}{も なか}の
\ruby{月}{つき}の
\ruby{淋}{さび}しげに
\ruby[g]{干縮}{ひ すば}りたるを、

\原本頁{11-1}%
『
\ruby[g]{何樣}{ど う }ぞ
\ruby{詰}{つま}らんものですが
\ruby[g]{御摘}{お つま}みなすつて。
』

\原本頁{11-2}%
と
\ruby[g]{叮嚀}{ていねい}に
\ruby{薦}{すゝ}め、
%
\ruby[g]{何時}{い つ }
\ruby[g]{用事}{ようじ }を
\ruby{云}{い}ひ
\ruby{出}{いだ}すべき
\ruby[g]{氣色}{け はひ}も
\ruby{無}{な}く、
%
\ruby{興}{きよう}も
\ruby{無}{な}き
\ruby{世}{よ}の
\ruby{噂}{うはさ}、
%
\ruby[g]{他{\換字{所}}}{よ そ }の
\ruby{事}{こと}をのみ、
%
\ruby[g]{熱心}{ねつしん}も
\ruby{無}{な}く
\ruby[g]{氣燄}{いきほひ}も
\ruby{無}{な}く、
%
\ruby[g]{溫和}{をんわ }に
\ruby[g]{冷靜}{れいせい}に
\ruby[g]{打語}{うちかた}りたり。

\原本頁{11-5}%
\ruby[g]{水野}{みづの }も
\ruby{初}{はじめ}は
\ruby{謹}{つゝし}み
\ruby{居}{ゐ}しが、
%
\ruby{{\換字{終}}}{つひ}に
\ruby{堪}{こら}へ
\ruby{得}{え}ずして
\ruby{口}{くち}を
\ruby{開}{ひら}き、

\原本頁{11-6}%
『
\ruby[g]{山路}{やまぢ }の
\ruby[g]{老人}{らうじん}に
\ruby{御言傳}{お|こと|づけ}
でしたので
\ruby{出}{で}ました
のですが、
%
\ruby[g]{御用}{ご よう}を
\ruby[g]{何樣}{ど う }か
\ruby{伺}{うかゞ}ひたい
もので。
』

\原本頁{11-8}%
と
\ruby{促}{うなが}すが
\ruby{如}{ごと}くに
\ruby{云}{い}ひ
\ruby{出}{い}づれば、

\原本頁{11-9}%
『
イヤー、
%
\ruby[g]{何樣}{ど う }もハヤ
\ruby{詰}{つま}らん
\ruby{事}{こと}で、
』

\原本頁{11-10}%
と
\ruby[g]{磊落}{らいらく}らしく
\ruby{右}{みぎ}の
\ruby{手}{て}を
\ruby{上}{あ}げて
\ruby[g]{頭髮}{あたま }を% ルビ調整(原本通り)非グループルビ
\ruby{撫}{な}でしが、
%
やがて
\ruby{然}{さ}も〳〵
\ruby[g]{決心}{けつしん}
したり
といふ
やうに
\ruby{眞面目}{ま|じ|め}に
なつて
\ruby[g]{自己}{お の }が
\ruby{膝}{ひざ}を
\ruby[g]{見詰}{み つ }め、

\原本頁{12-1}%
『
\ruby[g]{水野}{みづの }さん
\ruby{決}{けつ}して
\ruby[g]{御怒}{お おこ}り
なすつては
いけませんよ。
%
\ruby{萬}{ばん}
\ruby{已}{や}むを
\ruby{得}{え}んから
\ruby[g]{是非}{ぜ ひ }
\ruby{無}{な}く
\ruby[g]{御話}{お はな}しを
\ruby{致}{いた}しますがネ。
%
これも
\ruby[g]{小生}{わたくし}の
\ruby[g]{地位}{ち ゐ }から
\原本頁{12-3}\改行%
\ruby{致}{いた}しまして
\ruby[g]{詮方}{せんかた}が
\ruby{無}{な}いので、
%
\ruby[g]{何樣}{ど う }か
\ruby{惡}{あし}からず
\ruby{御}{ご}
\ruby[||j>]{解}{かい}
\ruby[||j>]{釋}{しやく}
を
\ruby{願}{ねが}ひ
ますのです。
%
\ruby{實}{じつ}は
\ruby[|g|]{貴下}{あなた}の
\ruby{御}{ご}
\ruby[<j||]{{\換字{評}}}{ひやう}
\ruby[<j||]{{\換字{判}}}{ばん}
% \ruby{{\換字{評}}{\換字{判}}}{ひやう|ばん}
が
\ruby{甚}{はなは}だ
\ruby{思}{おも}はしく
ないので。
%
イヤ
\ruby[g]{小生}{わたくし}は
\ruby[g]{何{\換字{所}}}{ど こ }までも
\ruby[|g|]{貴下}{あなた}を
\ruby{信}{しん}じて
\ruby{居}{を}りまするから、
%
\ruby{他}{ひと}が
\ruby{何}{なん}と
\ruby{申}{まを}しても
\原本頁{12-6}\改行%
\ruby{關}{かま}ひませんが、
%
\ruby[g]{何樣}{ど う }も
\ruby[g]{種々}{いろ〳〵}の
\ruby{事}{こと}を
\ruby{申}{まを}しまするので。
%
ハヽヽ、
%
\ruby[g]{世間}{せ けん}
といふものは
\ruby{煩}{うるさ}い
ものでしてナア、
%
\ruby[g]{信仰}{しんかう}の
\ruby[g]{自由}{じ ゆう}といふ
\ruby{事}{こと}は
\ruby[g]{嚴然}{ちやん }と
\ruby{許}{ゆる}されて
\ruby{居}{を}りまするのに、
%
\ruby[|g|]{貴下}{あなた}の
\ruby{事}{こと}を
\ruby[g]{妄信}{まうしん}に
\ruby{陷}{おちい}つたの
\ruby{何}{なん}のと
\ruby{申}{まを}しましてナ、
%
\ruby{其}{それ}は
\ruby{{\換字{又}}}{また}
\ruby[g]{斯樣}{か う }いふ
\ruby[g]{理由}{わ け }からだの
\ruby[g]{彼樣}{あ ゝ }いふ
\ruby[g]{仔細}{し さい}からだのと
\ruby{下}{くだ}らん
\ruby{事}{こと}を
\ruby{云}{い}ひましてナ、
%
それで
\ruby[g]{何樣}{ど う }も
\ruby[g]{兎角}{と かく}
\ruby[g]{小生}{わたくし}の
\ruby{耳}{みゝ}へ
\ruby{煩}{うるさ}い
\ruby{事}{こと}が
\ruby{入}{はい}ります。
%
\ruby{就}{つ}きましては
\ruby[g]{小生}{わたくし}の
\ruby{考}{かんが}へまするには、
%
\ruby[g]{貴下}{あなた }も% ルビ調整(原本通り)非グループルビ
\ruby{其}{それ}では
\ruby[g]{生徒}{せいと }の
\ruby[g]{{\換字{父}}兄}{ふ けい}の
\ruby[g]{手{\換字{前}}}{て まへ}や
\ruby{何}{なん}ぞ、
%
どうも
\ruby[||j>]{敎}{けう}
\ruby[||j>]{職}{しよく}を
% \ruby{敎職}{けう|しよく}を
お
\ruby{執}{と}り
なさり
\ruby{{\換字{難}}}{にく}い
やうな
\ruby{譯}{わけ}
ですから、
%
\ruby[g]{一應}{いちおう}
\ruby[g]{此村}{こ ゝ }の
\ruby{校}{かう}の
\ruby{方}{はう}を
\ruby[g]{御{\換字{退}}}{お ひ }き
なすつて
\原本頁{13-3}\改行%
\ruby{頂}{いたゞ}いて、
%
\ruby{他}{た}の
\ruby{校}{かう}へ
\ruby{行}{い}つて
\ruby{頂}{いたゞ}いた
\ruby{方}{はう}が
\ruby[|g|]{貴下}{あなた}の
\ruby{御利益}{ご|り|えき}で、
%
\ruby{{\換字{又}}}{また}
\ruby{{\換字{延}}}{ひ}いては
\ruby{校}{かう}の
\ruby{爲}{ため}にも
\ruby{聊}{いさゝ}か
\ruby[g]{利益}{り えき}かと
\ruby[g]{勘考}{かんかう}
\ruby{致}{いた}しましたです。
%
\ruby{御轉校}{ご|てん|かう}の
\ruby{事}{こと}は
\原本頁{13-5}\改行%
\ruby[|g|]{貴下}{あなた}の
\ruby{御不都合}{ご|ふ|つ|がふ}にならんように、
%
\ruby{必}{かなら}ず
\ruby[g]{小生}{わたくし}が
\ruby[g]{取計}{とりはか}らひ
まするか
\原本頁{13-6}\改行%
ら。
』

\原本頁{13-7}%
と、
%
\ruby{辛}{から}くして
\ruby{云}{い}ひ
\ruby{出}{いだ}したる
\ruby{其}{そ}の
\ruby[g]{眞意}{しんい }は、
%
\ruby{我}{われ}をして
\ruby{職}{しよく}を
\ruby{辭}{じ}さしめん
といふ
ことなりけり。

\原本頁{13-9}%
\ruby[g]{高田}{たかた }は
\ruby[||j>]{重}{ぢゆう}
\ruby[||j>]{大}{ だい}の
% \ruby{重大}{ぢゆう|だい}の
\ruby{事}{こと}と
\ruby{思}{おも}へる
なるべし、
%
\ruby[g]{水野}{みづの }は
\ruby{斯}{か}ばかりの
\ruby{事}{こと}かと
\ruby{毛}{け}より
\ruby{輕}{かろ}く
\ruby{思}{おも}ひて、

\原本頁{13-10}%
『
\ruby{解}{わか}りました。
%
\ruby[g]{早{\換字{速}}}{さつそく}
\ruby[g]{御諭}{お さと}しの
\ruby{{\換字{通}}}{とほ}りに
\ruby{致}{いた}しましやう。
』

\原本頁{14-1}%
と
\ruby[||j>]{心}{こゝろ}
\ruby[||j>]{易}{ やす}く
% \ruby{心易}{こゝろ|やす}く
\ruby{答}{こた}ふれば、
%
\ruby[g]{高田}{たかた }は
ホツト
\ruby{息}{いき}を
つける
\ruby{樣}{さま}なり。
