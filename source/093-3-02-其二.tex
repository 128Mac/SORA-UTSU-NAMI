\Entry{其二}

\原本頁{}
\ruby{思}{おも}はんともせずして
\ruby{思}{おも}ひ
\ruby{居}{ゐ}たるは、
%
\ruby{心}{こゝろ}の
\ruby{其}{それ}に
\ruby{染}{そ}みたればなるべし。
%
されども
\ruby[g]{吉右衛門}{きちゑもん}に
\ruby{話}{はな}し
\ruby{掛}{か}けられて、
%
\ruby[g]{水野}{みづの}は
\ruby{忽}{たちま}ち
\ruby{覺}{さ}めたる
\ruby{如}{ごと}く、

\原本頁{}
『
\ruby{惡}{わる}く
\ruby{思}{おも}ふなんぞといふ
\ruby{考}{かんがへ}が
\ruby{何樣}{ど|う}して
\ruby{私}{わたし}に………… 。
%
\ruby[g]{羽{\換字{勝}}}{はがち}だつて
\ruby[g]{日方}{ひかた}だつて
\ruby[<h||]{皆}{みんな}
\ruby[||h>]{私}{わたし}の
\ruby{兄}{あに}
\ruby{同樣}{どう|よう}なのだもの!、
%
\ruby{何}{なに}を
\ruby{言}{い}はれたつて
\ruby{惡}{わる}く
\ruby{取}{と}つたり
\ruby{氣}{き}に
\ruby{仕}{し}たりするやうな
\ruby{事}{こと}は
\ruby{有}{あ}りは
\ruby{仕}{し}ないので。
%
\ruby{私}{わたし}は
\ruby{今}{いま}たゞ
\ruby{恍然}{うつ|かり}として
\ruby{居}{ゐ}たところでした。
%
いや
\ruby{今日}{け|ふ}は
\ruby{大層}{たい|そう}
\ruby{御世話}{お|せ|わ}でした。
%
お
\ruby{蔭}{かげ}で
\ruby{一同}{みん|な}
\ruby{悅}{よろこ}んで
\ruby{歸}{かへ}りましたが、
%
あれを
\ruby{殘}{のこ}らず
\ruby{御厄介}{ご|やく|かい}になる
\ruby{理由}{いは|れ}はありませんから、
%
せめて
\ruby{御酒}{ご|しゆ}だけでも
\ruby{私}{わたし}の
\ruby{{\換字{分}}}{ぶん}にして、
』

\原本頁{}
と
\ruby{云}{い}ひ
\ruby{掛}{か}くるを
\ruby{主人}{ある|じ}は
\ruby{悅}{よろこ}ばぬ
\ruby{氣}{げ}なる
\ruby{顏}{かほ}して、

\原本頁{}
『また
\ruby[g]{水野}{みづの}さんの
\ruby{他人行儀}{た|にん|ぎやう|ぎ}がはじまつた。
%
\ruby{几帳面}{きち|やう|めん}
\ruby{{\換字{過}}}{す}ぎて
\ruby{厭氣}{いや|き}がさします。
%
\ruby{宜}{い}いぢやあ
\ruby{有}{あ}りませんか
\ruby{些細}{わづ|か}の
\ruby{事}{こと}ですもの。
』

\原本頁{}
と
\ruby{打{\換字{消}}}{うち|け}しつ、

\原本頁{}
『それは
\ruby{左樣}{さ|う}と
\ruby{先刻}{さつ|き}
\ruby{老夫}{わた|くし}が
\ruby{高田}{たか|た}さんに
\ruby{逢}{あ}ひましたら、
%
\ruby[g]{水野}{みづの}さんに
\ruby{一寸}{ちよ|つと}
\ruby{來}{き}て
\ruby{貰}{もら}ひたいことがあるから
\ruby{然樣}{さ|う}
\ruby{云}{い}つてくれ、
%
\ruby{他人}{ひ|と}の
\ruby{居}{ゐ}ない
\ruby{時}{とき}
\ruby{會}{あ}いたいから
\ruby{成}{な}るべくば
\ruby{今夜}{こん|や}あたり、
%
といふ
\ruby{御談}{お|はなし}でございました。
%
\ruby{御酒氣}{ご|しゆ|き}は
\ruby{大{\換字{分}}}{だい|ぶ}
\ruby{御有}{お|あ}んなさるけれども、
%
\ruby{貴下}{あな|た}の
\ruby{事}{こと}ですから
\ruby{宜}{よ}うございましやう。
%
\ruby{{\換字{更}}}{ふ}けない
\ruby{中}{うち}
\ruby{一寸}{ちよ|つと}
\ruby{行}{い}つて
\ruby{居}{ゐ}らつしやいませんか。
』

\原本頁{}
と
\ruby{云}{い}ひ
\ruby{出}{いだ}したり。

\原本頁{}
\ruby{高田}{たか|た}は
\ruby{我}{わ}が
\ruby{職}{しよく}を
\ruby{奉}{ほう}ずる
\ruby{學校}{がく|かう}の
\ruby{長}{ちやう}にして、
%
\ruby[g]{吉右衛門}{きちゑもん}とも
\ruby{心易}{こゝろ|やす}き
\ruby{男}{をとこ}なれば、
%
\ruby[g]{水野}{みづの}は
\ruby{{\換字{更}}}{さら}に
\ruby{考}{かんが}ふるまでも
\ruby{無}{な}くして、

\原本頁{}
『
\ruby{何}{なん}だかさつぱり
\ruby{{\換字{分}}}{わか}からないけれども、
%
\ruby{其樣}{そ|ん}なら
\ruby{一寸}{ちよ|いと}
\ruby{行}{い}つて
\ruby{來}{き}ましやう。
』

\原本頁{}
と
\ruby{答}{こた}へつ、
%
\ruby[g]{吉右衛門}{きちゑもん}が
お
\ruby{濱}{はま}を
\ruby{呼}{よ}び
\ruby{立}{た}てゝ、
%
\ruby{提灯}{ちやう|ちん}をと
\ruby{云}{い}ふを、
%
それにも
\ruby{及}{およ}ばずと
\ruby{制}{とゞ}め、
%
たゞ
\ruby{纔}{わづか}に
\ruby{帶}{おび}
\ruby{締}{し}め
\ruby{直}{なほ}しゝのみにて
\ruby{立出}{たち|い}でたり。

\原本頁{}
\ruby{高田}{たか|た}が
\ruby{家}{いへ}は
\ruby{學校}{がく|かう}の
\ruby{直後面}{すぐ|うし|ろ}にて、
%
\ruby{農家{\換字{造}}}{のう|か|づく}りにてこそはあらね、
%
\ruby{趣味}{おも|むき}も
\ruby{無}{な}き
\ruby{{\換字{平}}々凡々}{へい|〳〵|ぼん|〴〵}の
\ruby{住居}{すま|ゐ}なるが、
%
\ruby{主人}{ある|じ}も
\ruby{其家}{その|いへ}に
\ruby{相應}{ふさ|は}しき
\ruby{{\換字{平}}々凡々}{へい|〳〵|ぼん|〴〵}の、
%
\ruby{何}{なん}の
\ruby{奇處}{き|しよ}も
\ruby{無}{な}き
\ruby{五十男}{い|そ|を}にて、
%
\ruby{農夫}{ひやく|しやう}にてこそはあらね、
%
\ruby{面白味}{おも|しろ|み}も
\ruby{無}{な}き
\ruby{氣}{き}の
\ruby{小}{ちひさ}なる
\ruby{謹直}{まじ|め}
\ruby{三昧}{ざん|まい}の
\ruby{人}{ひと}なり。

\原本頁{}
\ruby{{\換字{半}}白}{はん|ぱく}の
\ruby{髮}{かみ}の
\ruby{毛}{け}は
\ruby{割合}{わり|あひ}に
\ruby{多}{おほ}かれども、
%
\ruby{光澤無}{つ|や|な}く
\ruby{黄色}{き|いろ}に
\ruby{痩}{や}せきつたる
\ruby{顏}{かほ}の、
%
\ruby{口}{くち}の
\ruby{傍}{はた}の
\ruby{條{\換字{文}}}{す|ぢ}、
%
\ruby{額}{ひたひ}の
\ruby{皺}{しわ}など
\ruby{目立}{め|だ}つて
\ruby{深}{ふか}く、
%
\ruby{光無}{ひかり|な}き
\ruby{小}{ちひさ}なる
\ruby{眼}{め}、
%
\ruby{骨立}{ほね|だ}つて
\ruby{高}{たか}き
\ruby{鼻}{はな}、
%
おちつきの
\ruby{無}{な}き
\ruby{起居動作}{たち|ゐ|ふる|まひ}、
%
\ruby{活氣}{いき|ほひ}の
\ruby{無}{な}き
\ruby{物}{もの}の
\ruby{言}{い}ひぶり、
%
すべての
\ruby{乾燥}{ひか|ら}びたる
\ruby{狀態}{あり|さま}は、
%
\ruby{如何}{い|か}にも
\ruby{能}{よ}く
\ruby{此人}{この|ひと}の『
\ruby{人}{ひと}の
\ruby{子}{こ}を
\ruby{{\換字{誤}}}{あやま}るが
\ruby{如}{ごと}き
\ruby{{\換字{強}}}{つよ}き
\ruby{人}{ひと}』ならで、
%
『
\ruby{決}{けつ}して
\ruby{人}{ひと}の
\ruby{子}{こ}を
\ruby{{\換字{害}}}{そこな}はぬ
\ruby{{\換字{古}}}{ふ}りたる
\ruby{敎育家}{けう|いく|か}』たる
\ruby{事}{こと}をば
\ruby{現}{あらは}し
\ruby{示}{しめ}せり。

\原本頁{}
\ruby{高田}{たか|た}は
\ruby{今}{いま}
\ruby[g]{水野}{みづの}の
\ruby{來}{きた}り
\ruby{訪}{と}ふに
\ruby{會}{あ}ひて、
%
\ruby{一昨日}{を|とゝ|ひ}も
\ruby{昨日}{きの|ふ}も
\ruby{會}{あ}ひたる
\ruby{同士}{どう|し}なるに、
%
\ruby{三年}{さん|ねん}
\ruby{四年}{よ|ねん}も
\ruby{隔}{へだ}てゝ
\ruby{面}{おもて}を
\ruby{見}{み}たるものゝ
\ruby{如}{ごと}く、
%
\ruby{慇懃}{いん|ぎん}に
\ruby{時候}{じ|こう}の
\ruby{挨拶}{あい|さつ}など
\ruby{管々}{くだ|〴〵}しく
\ruby{仕}{し}て、
%
\ruby{三十匁}{さん|じう|め}ばかりの
\ruby{{\換字{廉}}價茶}{やす|ぢ|や}を
\ruby{事々}{こと|〴〵}しく
\ruby{湯}{ゆ}を
\ruby{冷}{さ}ましなどして
\ruby{入}{い}れ、
%
\ruby{隱}{かく}れ
\ruby{蓑}{みの}、
%
\ruby{隱}{かく}れ
\ruby{笠}{がさ}、
%
\ruby{打出}{うち|で}の
\ruby{槌}{つち}なんどの
\ruby{寶盡}{たから|づく}しを
\ruby{描}{ゑが}きたる
\ruby{水金}{みづ|きん}の
\ruby{光}{ひか}り
\ruby{爛々}{きら|〳〵}とする
\ruby{菓子鉢}{くわ|し|ばち}に、
%
\ruby{三月}{み|つき}も
\ruby{{\換字{前}}}{まへ}より
\ruby{盛}{も}られし
\ruby{儘}{まゝ}かと
\ruby{想}{おも}はるゝやうなる
\ruby{最中}{も|なか}の
\ruby{月}{つき}の
\ruby{淋}{さび}しげに
\ruby{干縮}{ひ|すば}りたるを、

\原本頁{}
『
\ruby{何樣}{ど|う}ぞ
\ruby{詰}{つま}らんものですが
\ruby{御摘}{お|つま}みなすつて。
』

\原本頁{}
と
\ruby{叮嚀}{てい|ねい}に
\ruby{薦}{すゝ}め、
%
\ruby{何時}{い|つ}
\ruby{用事}{よう|じ}を
\ruby{云}{い}ひ
\ruby{出}{いだ}すべき
\ruby{氣色}{け|はひ}も
\ruby{無}{な}く、
%
\ruby{興}{きよう}も
\ruby{無}{な}き
\ruby{世}{よ}の
\ruby{噂}{うはさ}、
%
\ruby{他{\換字{所}}}{よ|そ}の
\ruby{事}{こと}をのみ、
%
\ruby{熱心}{ねつ|しん}も
\ruby{無}{な}く
\ruby{氣燄}{いき|ほひ}も
\ruby{無}{な}く、
%
\ruby{溫和}{をん|わ}に
\ruby{冷靜}{れい|せい}に
\ruby{打語}{うち|かた}りたり。

\原本頁{}
\ruby[g]{水野}{みづの}も
\ruby{初}{はじめ}は
\ruby{謹}{つゝし}み
\ruby{居}{ゐ}しが、
%
\ruby{{\換字{終}}}{つひ}に
\ruby{堪}{こら}へ
\ruby{得}{え}ずして
\ruby{口}{くち}を
\ruby{開}{ひら}き、

\原本頁{}
『
\ruby{山路}{やま|ぢ}の
\ruby{老人}{らう|じん}に
\ruby{御言傳}{お|こと|づけ}でしたので
\ruby{出}{で}ましたのですが、
%
\ruby{御用}{ご|よう}を
\ruby{何樣}{ど|う}か
\ruby{伺}{うかゞ}ひたいもので。
』

\原本頁{}
と
\ruby{促}{うなが}すが
\ruby{如}{ごと}くに
\ruby{云}{い}い
\ruby{出}{い}づれば、

\原本頁{}
『イヤー、
%
\ruby{何樣}{ど|う}もハヤ
\ruby{詰}{つま}らん
\ruby{事}{こと}で、
』

\原本頁{}
と
\ruby{磊落}{らい|らく}らしく
\ruby{右}{みぎ}の
\ruby{手}{て}を
\ruby{上}{あ}げて
\ruby{頭髮}{あた|ま}を
\ruby{撫}{な}でしが、
%
やがて
\ruby{然}{さ}も〳〵
\ruby{決心}{けつ|しん}したりといふやうに
\ruby{眞面目}{ま|じ|め}になつて
\ruby{自己}{お|の}が
\ruby{膝}{ひざ}を
\ruby{見詰}{み|つ}め、

\原本頁{}
『
\ruby[g]{水野}{みづの}さん
\ruby{決}{けつ}して
\ruby{御怒}{お|おこ}りなすつてはいけませんよ。
%
\ruby{萬已}{ばん|や}むを
\ruby{得}{え}んから
\ruby{是非無}{ぜ|ひ|な}く
\ruby{御話}{お|はなし}を
\ruby{致}{いた}しますがネ。
%
これも
\ruby{小生}{わた|くし}の
\ruby{地位}{ち|ゐ}から
\ruby{致}{いた}しまして
\ruby{詮方}{せん|かた}が
\ruby{無}{な}いので、
%
\ruby{何樣}{ど|う}か
\ruby{惡}{あし}からず
\ruby{御解釋}{ご|かい|しやく}を
\ruby{願}{ねが}ひますのです。
%
\ruby{實}{じつ}は
\ruby{貴下}{あな|た}の
\ruby{御}{ご}
\ruby{{\換字{評}}{\換字{判}}}{ひやう|ばん}が
\ruby{甚}{はなは}だ
\ruby{思}{おも}はしくないので。
%
イヤ
\ruby{小生}{わた|くし}は
\ruby{何{\換字{所}}}{ど|こ}までも
\ruby{貴下}{あな|た}を
\ruby{信}{しん}じて
\ruby{居}{を}りまするから、
%
\ruby{他}{ひと}が
\ruby{何}{なん}と
\ruby{申}{まを}しても
\ruby{關}{かま}ひませんが、
%
\ruby{何樣}{ど|う}も
\ruby{種々}{いろ|〳〵}の
\ruby{事}{こと}を
\ruby{申}{まを}しまするので。
%
ハヽヽ、
%
\ruby{世間}{せ|けん}といふものは
\ruby{煩}{うるさ}いものでしてナア、
%
\ruby{信仰}{しん|かう}の
\ruby{自由}{じ|ゆう}といふ
\ruby{事}{こと}は
\ruby{嚴然}{ちや|ん}と
\ruby{許}{ゆる}されて
\ruby{居}{を}りまするのに、
%
\ruby{貴下}{あな|た}の
\ruby{事}{こと}を
\ruby{妄信}{まう|しん}に
\ruby{陷}{おちい}つたの
\ruby{何}{なん}のと
\ruby{申}{まを}しましてナ、
%
\ruby{其}{それ}は
\ruby{{\換字{又}}}{また}
\ruby{斯樣}{か|う}いふ
\ruby{理由}{わ|け}からだの
\ruby{彼樣}{あ|ゝ}いふ
\ruby{仔細}{し|さい}からだのと
\ruby{下}{くだ}らん
\ruby{事}{こと}を
\ruby{云}{い}ひましてナ、
%
それで
\ruby{何樣}{ど|う}も
\ruby{兎角}{と|かく}
\ruby{小生}{わた|くし}の
\ruby{耳}{みゝ}へ
\ruby{煩}{うるさ}い
\ruby{事}{こと}が
\ruby{入}{はい}ります。
%
\ruby{就}{つ}きましては
\ruby{小生}{わた|くし}の
\ruby{考}{かんが}へまするには、
%
\ruby{貴下}{あな|た}も
\ruby{其}{それ}では
\ruby{生徒}{せい|と}の
\ruby{{\換字{父}}兄}{ふ|けい}の
\ruby{手{\換字{前}}}{て|まへ}や
\ruby{何}{なん}ぞ、
%
どうも
\ruby{敎職}{けう|しよく}を
お
\ruby{執}{と}りなさり
\ruby{難}{にく}いやうな
\ruby{譯}{わけ}ですから、
%
\ruby{一應}{いち|おう}
\ruby{此村}{こ|ゝ}の
\ruby{校}{かう}の
\ruby{方}{はう}を
\ruby{御{\換字{退}}}{お|ひ}きなすつて
\ruby{頂}{いたゞ}いて、
%
\ruby{他}{た}の
\ruby{校}{かう}へ
\ruby{行}{い}つて
\ruby{頂}{いたゞ}いた
\ruby{方}{はう}が
\ruby{貴下}{あな|た}の
\ruby{御利益}{ご|り|えき}で、
%
\ruby{{\換字{又}}}{また}
\ruby{{\換字{延}}}{ひ}いては
\ruby{校}{かう}の
\ruby{爲}{ため}にも
\ruby{聊}{いさゝ}か
\ruby{利益}{り|えき}かと
\ruby{勘考致}{かん|かう|いた}しましたです。
%
\ruby{御轉校}{ご|てん|かう}の
\ruby{事}{こと}は
\ruby{貴下}{あな|た}の
\ruby{御不都合}{ご|ふ|つ|がふ}にならんように、
%
\ruby{必}{かなら}ず
\ruby{小生}{わた|くし}が
\ruby{取計}{とり|はか}らひまするから。
』

\原本頁{}
と、
%
\ruby{辛}{から}くして
\ruby{云}{い}ひ
\ruby{出}{いだ}したる
\ruby{其}{そ}の
\ruby{眞意}{しん|い}は、
%
\ruby{我}{われ}をして
\ruby{職}{しよく}を
\ruby{辭}{じ}さしめんといふことなりけり。

\原本頁{}
\ruby{高田}{たか|た}は
\ruby{重大}{ぢゆう|だい}の
\ruby{事}{こと}と
\ruby{思}{おも}へるなるべし、
%
\ruby[g]{水野}{みづの}は
\ruby{斯}{か}ばかりの
\ruby{事}{こと}かと
\ruby{毛}{け}より
\ruby{輕}{かろ}く
\ruby{思}{おも}ひて、

\原本頁{}
『
\ruby{解}{わか}りました。
%
\ruby{早{\換字{速}}}{さつ|そく}
\ruby{御諭}{お|さとし}しの
\ruby{{\換字{通}}}{とほ}りに
\ruby{致}{いた}しましやう。
』

\原本頁{}
と
\ruby{心易}{こゝろ|やす}く
\ruby{答}{こた}ふれば、
%
\ruby{高田}{たか|た}はホツト
\ruby{息}{いき}をつける
\ruby{樣}{さま}なり。

