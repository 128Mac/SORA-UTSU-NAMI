\Entry{其三}

\原本頁{}%
\ruby{我}{わ}が
\ruby{職務}{つと|め}を
\ruby{卑}{いやし}む
\ruby{意}{こゝろ}などは
\ruby{露}{つゆ}ばかりも
\ruby{有}{あ}らざりしが、
%
もとより
\ruby{一生}{いつ|しやう}を
\ruby{其任}{そ|れ}に
\ruby{委}{ゆだ}ねんとも
\ruby{思}{おも}はざりしなれば、
%
\ruby[g]{水野}{みづの}は
\ruby{{\換字{難}}}{はゞか}る
\ruby{色}{いろ}も
\ruby{無}{な}く
\ruby{職}{しよく}を
\ruby{辭}{じ}せんと
\ruby{云}{い}へるに、
%
\ruby{高田}{たか|た}は
\ruby{我}{わ}が
\ruby{意}{こゝろ}の
\ruby{{\換字{通}}}{とほ}りたるより
\ruby{胸}{むね}は
\ruby{安}{やす}くせしものゝ、
%
\ruby{却}{かへ}つて
\ruby{{\換字{又}}}{また}
\ruby{對手}{あひ|て}の
\ruby{餘}{あま}りに
\ruby{未練氣}{み|れん|げ}
\ruby{無}{な}きに
\ruby{薄氣味}{うす|き|み}
\ruby{惡}{あし}く、
%
\ruby{懸念}{け|ねん}らしく
\ruby{小}{ちひさ}き
\ruby{眼}{め}を
\ruby{瞬}{しばた}きて
\ruby[g]{水野}{みづの}を
\ruby{見居}{み|ゐ}たり。

\原本頁{}%
『\換字{志}かし
\ruby[g]{水野}{みづの}さん
\ruby{決}{けつ}して
\ruby{御不快}{ご|ふ|かい}に
\ruby{御思}{お|おも}ひなすつてはいけません、
%
\ruby{何樣}{ど|う}か
\ruby{感{\換字{情}}}{かん|じやう}を
\ruby{{\換字{害}}}{がい}して
\ruby{下}{くだ}さらんやうに
\ruby{願}{ねが}ひます。
%
\ruby{小生}{わた|くし}は
\ruby{何處迄}{ど|こ|まで}も
\ruby{貴下}{あな|た}を
\ruby{信}{しん}じて
\ruby{居}{を}るのですから、
%
\ruby{貴下}{あな|た}に
\ruby{校}{かう}から
\ruby{離}{はな}れて
\ruby{頂}{いたゞ}きたい
\ruby{心}{こゝろ}は
\ruby{{\換字{更}}}{さら}に
\ruby{無}{な}いのでして、
%
\ruby{長}{なが}く
\ruby{貴下}{あな|た}と
\ruby{圓滿}{ゑん|まん}な
\ruby{御{\換字{交}}際}{ご|かう|さい}を
\ruby{繼續}{つ|な}いで
\ruby{參}{まゐ}りたいのです。
%
\ruby{貴下}{あな|た}は
\ruby{失禮}{しつ|れい}ながら
\ruby{學力}{がく|りよく}は
\ruby{御有}{お|あ}りなさるし、
%
なか〳〵
\ruby{長}{なが}く
\ruby{小學}{せう|がく}の
\ruby{敎師}{けう|し}などを
\ruby{仕}{し}て
\ruby{居}{ゐ}らつしやる
\ruby{御仁}{ご|じん}では
\ruby{無}{な}いのです。
%
が、
%
\ruby{差當}{さし|あた}つて
\ruby{校}{かう}の
\ruby{方}{はう}を
\ruby{離}{はな}れて
\ruby{戴}{いたゞ}いては
\ruby{御困}{お|こま}りでもございましやうから、
%
\ruby{小生}{わた|くし}は
\ruby{小生}{わた|くし}の
\ruby{貴下}{あな|た}に
\ruby{對}{たい}する
\ruby{眞{\換字{情}}}{しん|じやう}を
\ruby{表}{へう}して、
%
\ruby{貴下}{あな|た}を
\ruby{他{\換字{所}}}{よ|そ}の
\ruby{校}{かう}へ
\ruby{御周旋}{ご|しう|せん}
\ruby{致}{いた}しましようと
\ruby{存}{ぞん}じて
\ruby{居}{を}ります。
%
\ruby{何樣}{ど|う}か
\ruby{小生}{わた|くし}が
\ruby{貴下}{あな|た}に
\ruby{對}{たい}する
\ruby{敬意}{けい|い}を
\ruby{御汲}{お|く}み
\ruby{取}{と}り
\ruby{下}{くだ}すつて
\ruby{頂}{いたゞ}きたいもので。
』

\原本頁{}%
と、
%
\ruby{是}{これ}もまた
\ruby{三十匁}{さん|じう|め}の
\ruby{茶}{ちや}を
\ruby{入}{い}るゝに
\ruby{湯}{ゆ}を
\ruby{冷}{さ}まして
\ruby{後}{のち}にするが
\ruby{如}{ごと}く
\ruby{叮嚀}{てい|ねい}に
\ruby{言}{い}へば、
%
\ruby[g]{水野}{みづの}は
\ruby{他}{ひと}に
\ruby{憎}{にく}まれじとする
\ruby{心{\換字{遣}}}{こゝろ|づか}ひの、
%
いと
\ruby{明}{あき}らかに
\ruby{見}{み}ゆる
\ruby{此}{こ}の
\ruby{{\換字{半}}白}{はん|ぱく}の
\ruby{敎育家}{けう|いく|か}を、
%
\ruby{憫然}{あは|れ}に
\ruby{思}{おも}ふやうの
\ruby{{\換字{情}}}{こゝろ}も
\ruby{起}{おこ}りて、

\原本頁{}%
『はい、
%
\ruby{有}{あ}り
\ruby{{\換字{難}}}{がた}うございます。
%
\ruby{御高{\換字{情}}}{ご|かう|せい}はまことに
\ruby{有}{あ}り
\ruby{{\換字{難}}}{がた}うございます。
%
\ruby{御言葉}{お|こと|ば}に
\ruby{甘}{あま}えまして
\ruby{何處}{いづ|れ}かへ
\ruby{御周旋}{ご|しう|せん}を
\ruby{願}{ねが}はなくつてはならんのですが、
\換字{志}かし
\ruby{小生}{わた|くし}は
\ruby{何樣}{ど|う}も
\ruby{敎鞭}{けう|べん}を
\ruby{執}{と}るには
\ruby{{\換字{適}}}{てき}せんやうに
\ruby{思}{おも}ひますから、
%
\ruby{差當}{さし|あた}つて
\ruby{他{\換字{所}}}{よ|そ}の
\ruby{校}{かう}へ
\ruby{參}{まゐ}りたいとも
\ruby{存}{ぞん}じませんです。
%
\ruby{御厚意}{ご|かう|い}は
\ruby{何處}{ど|こ}までも
\ruby{有}{あ}り
\ruby{{\換字{難}}}{がた}く
\ruby{存}{ぞん}じますけれども、
%
\ruby{當{\換字{分}}}{たう|ぶん}は
\ruby{{\換字{遊}}}{あそ}んで
\ruby{見}{み}たいと
\ruby{思}{おも}つて
\ruby{居}{を}りまする。
%
それでは
\ruby{辭表}{じ|へう}は
\ruby{明日}{みやう|にち}
\ruby{早{\換字{速}}}{さつ|そく}
\ruby{差}{さ}し
\ruby{出}{だ}しまするから、
%
\ruby{何{\換字{分}}}{なに|ぶん}
\ruby{宜}{よろ}しく
\ruby{御計}{お|はか}らひを
\ruby{願}{ねが}ひまする。
』

\原本頁{}%
と、
%
\ruby{{\換字{飽}}}{あく}まで
\ruby{{\換字{謙}}{\換字{退}}}{けん|たい}して
\ruby{柔和}{にう|わ}に
\ruby{應}{こた}へたり。

\原本頁{}%
\ruby[g]{水野}{みづの}が
\ruby{面}{おもて}に
\ruby{怨氣}{ゑん|き}をも
\ruby{盛}{も}らずして、
%
\ruby{{\換字{平}}常}{ふだ|ん}の
\ruby{如}{ごと}く
\ruby{何氣}{なに|げ}なき
\ruby{言}{ことば}の
\ruby{調子}{てう|し}に
\ruby{職}{しよく}を
\ruby{辭}{じ}せんといふを
\ruby{聞}{き}き、
%
\ruby{高田}{たか|た}はやうやく
\ruby{荷}{に}を
\ruby{下}{おろ}したる
\ruby{心地}{こゝ|ち}してか、

\原本頁{}%
『ヤ、
%
それでは
\ruby{當{\換字{分}}}{たう|ぶん}
\ruby{御{\換字{遊}}}{お|あそ}びも
\ruby{宜}{よろ}しうございましやう。
%
\ruby{疾}{とう}から
\ruby{小生}{わた|くし}は
\ruby{貴下}{あな|た}を
\ruby{目}{もく}して、
%
\ruby{蛟龍}{かう|りう}
\ruby{永}{なが}く
\ruby{池中}{ち|ちゆう}のものたらずと
\ruby{申}{まを}して
\ruby{居}{を}りましたのです。
%
ハヽヽ。
%
\ruby{何樣}{ど|う}か
\ruby{今後}{こん|ご}
\ruby{何{\換字{分}}}{なに|ぶん}
\ruby{御見棄無}{お|み|すて|な}く
\ruby{御{\換字{交}}際}{ご|かう|さい}を
\ruby{願}{ねが}ひまする。
』

\原本頁{}%
と
\ruby{可笑}{を|か}しくも
\ruby{無}{な}きところに
\ruby{磊落}{らい|らく}めかして
\ruby{妙}{めう}に
\ruby{笑}{わら}つて、
%
\ruby{最後}{さい|ご}には
\ruby{改}{あらた}めて
\ruby{肘}{ひぢ}を
\ruby{張}{は}つて
\ruby{堅}{かた}くろしく
\ruby{頭}{かうべ}を
\ruby{下}{さ}げて
\ruby{一禮}{いち|れい}すれば、
%
\ruby[g]{水野}{みづの}も
\ruby{是非}{ぜ|ひ}なく
\ruby{禮}{れい}を
\ruby{{\換字{返}}}{かへ}して、

\原本頁{}%
『いや
\ruby{今後}{こん|ご}の
\ruby{御{\換字{交}}際}{ご|かう|さい}は
\ruby{小生}{わた|くし}の
\ruby{方}{はう}からこそ
\ruby{願}{ねが}ふべきで、
%
では
\ruby{今日}{こん|にち}はこれで
\ruby{失禮致}{しつ|れい|いた}します。
』

\原本頁{}%
と
\ruby{慇懃}{いん|ぎん}に
\ruby{挨拶}{あい|さつ}して
\ruby{辭}{じ}し
\ruby{歸}{かへ}りたり。

\原本頁{}%
\ruby{區々}{く|ゝ}たる
\ruby{職}{しよく}と
\ruby{些々}{さ|ゝ}たる
\ruby{俸給}{ほう|きふ}とは、
%
\ruby{之}{これ}を
\ruby{得}{う}るも
\ruby{之}{これ}を
\ruby{失}{うしな}ふも
\ruby{一顰}{いつ|ぴん}
\ruby{一笑}{いつ|せう}にだに
\ruby{價}{あたひ}せずと、
%
\ruby[g]{水野}{みづの}は
\ruby{其}{その}
\ruby{事}{こと}を
\ruby{繰}{く}り
\ruby{{\換字{返}}}{かへ}しても
\ruby{思}{おも}はず、
%
たゞ
\ruby{{\換字{猶}}}{なほ}
\ruby{微}{かすか}に
\ruby{殘}{のこ}れる
\ruby{醉}{よひ}を% 「醉」は原本通り「よ」で調整
\ruby{吹}{ふ}く
\ruby{風}{かぜ}の
\ruby{薄{\換字{寒}}}{うす|さむ}きを
\ruby{覺}{おぼ}えつゝ
\ruby{歸}{かへ}り
\ruby{着}{つ}けば、
%
お
\ruby{濱}{はま}は
\ruby{待}{ま}ち
\ruby{{\換字{兼}}}{か}ねしが
\ruby{如}{ごと}く
\ruby{飛}{とん}で
\ruby{出}{い}でゝ、
%
\ruby{茶}{ちや}の
\ruby{間}{ま}に
\ruby[<j|]{{\換字{迎}}}{むかへ}
\ruby{入}{い}るゝや
\ruby{否}{いな}や、
%
\ruby{滿面}{まん|めん}に
\ruby{笑}{ゑみ}を
\ruby{輝}{かゞや}かしつ、
%
\ruby{他人}{ひ|と}には
\ruby{何言}{なに|い}ふ
\ruby{間}{ひま}をも
\ruby{與}{あた}へずして、

\原本頁{}%
『
\ruby{今}{いま}
\ruby{先生}{せん|せい}と
\ruby{入}{い}れ
\ruby{{\換字{違}}}{ちが}つてネ、
%
\ruby{彼}{あ}の
\ruby[g]{尾竹}{をだけ}が
\ruby{變}{へん}に
\ruby{威張}{ゐ|ば}つて
\ruby{{\換字{遣}}}{や}つて
\ruby{來}{き}ましてネ。
%
とう〳〵
\ruby{此方}{こつ|ち}のものに
\ruby{仕}{し}た、
%
もう
\ruby{大{\換字{丈}}夫}{だい|ぢやう|ぶ}だ、
%
もう
\ruby{屹度}{きつ|と}
\ruby{保證}{うけ|あ}ひます、
%
もう
\ruby{宜}{よ}うございます、
%
もう
\ruby{是}{これ}からは
\ruby{快癒}{な|ほ}るばかりです、
%
\ruby{必}{かなら}ず
\ruby[g]{五十子}{いそこ}さんは
\ruby{本復}{ほん|ぷく}するといふ
\ruby{見{\換字{込}}}{み|こ}みが
\ruby{立}{た}ちました。
%
\ruby[g]{水野}{みづの}さんに
\ruby{十{\換字{分}}}{じう|ぶん}
\ruby{悅}{よろこ}んで
\ruby{貰}{もら}はなくちやあ、
%
と
\ruby{云}{い}つて
\ruby{今}{いま}まで
\ruby{饒舌}{しや|べ}つて
\ruby{行}{ゆ}きましたよ。
%
\ruby{嬉}{うれ}しいのネエ
\ruby{先生}{せん|せい}。
%
\ruby[<j|]{妾}{わたし}
\ruby{嬉}{うれ}しくつて!。
%
ほんとに
\ruby[<j|]{妾}{わたし}
\ruby{嬉}{うれ}しくつて〳〵!。
』

\原本頁{}%
と
\ruby{急}{せ}きに
\ruby{急}{せ}きて
\ruby{喜悅}{よろ|こび}の
\ruby{音信}{おと|づれ}を
\ruby{傳}{つた}へたり。

\原本頁{}%
お
\ruby{濱}{はま}は
\ruby{我}{わ}が
\ruby{此}{こ}の
\ruby{言葉}{こと|ば}を
\ruby{聞}{き}くと
\ruby{齊}{ひと}しく
\ruby[g]{水野}{みづの}の
\ruby{如何}{い|か}に
\ruby{悅}{よろこ}びて
\ruby{笑}{ゑ}むならんと
\ruby{思}{おも}ひ
\ruby{設}{まう}けつ、
%
\ruby{心樂}{こゝろ|たのし}みにして
\ruby[g]{水野}{みづの}の
\ruby{面}{おもて}を
\ruby{差覗}{さし|のぞ}けるに、
%
\ruby{悅}{よろこ}び
\ruby{極}{きは}まつてか
\ruby{其}{その}
\ruby{人}{ひと}は
\ruby{笑}{ゑ}まず、
%
\ruby{目}{ま}のあたりに
\ruby{神佛}{かみ|ほとけ}を
\ruby{拜}{をが}めるが
\ruby{如}{ごと}き、
%
\ruby{敬}{つゝし}みに
\ruby{敬}{つゝし}めるが
\ruby{中}{なか}に
\ruby{和}{やさ}しさ
\ruby{見}{み}ゆる
\ruby{面}{おもて}になつて、
%
\ruby{抑々}{そも|〳〵}
\ruby{何}{なに}をか
\ruby{見詰}{み|つ}むるや
\ruby{頭}{かしら}を
\ruby{斜}{なゝめ}に、
%
\ruby{物}{もの}も
\ruby{無}{な}き
\ruby{{\換字{空}}中}{くう|ちゆう}を
\ruby{凝然}{じ|つ}と
\ruby{仰}{あふ}ぎたるが、
%
\ruby{見}{み}る〳〵
\ruby{動}{うご}かざる
\ruby{其}{そ}の
\ruby{眼}{め}の
\ruby{中}{うち}よりは、
%
\ruby{汪々}{わう|〳〵}
\ruby{漣々}{れん|〳〵}として
\ruby{涙}{なみだ}の
\ruby{溢}{あふ}れたり。
%
\ruby{悅}{よろこ}び
\ruby{涙}{なみだ}とはこれなるべきにや。
