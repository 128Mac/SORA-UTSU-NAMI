\Entry{其三}

% メモ 校正終了 2024-05-10 2024-06-06
\原本頁{14-3}%
\ruby{我}{わ}が
\ruby[|g|]{職務}{つとめ}を
\ruby{卑}{いやし}む
\ruby{意}{こゝろ}などは
\ruby{露}{つゆ}
ばかりも
\ruby{有}{あ}らざりしが、
%
もとより
\ruby[||j>]{一}{いつ}
\ruby[||j>]{生}{しやう}を
% \ruby{一生}{いつ|しやう}を
\ruby[g]{其任}{そ れ }に
\ruby{委}{ゆだ}ねんとも
\ruby{思}{おも}はざりしなれば、
%
\ruby[g]{水野}{みづの }は
\ruby{{\換字{難}}}{はゞか}る
\ruby{色}{いろ}も
\ruby{無}{な}く
\ruby{職}{しよく}を
\ruby{辭}{じ}せんと
\ruby{云}{い}へるに、
%
\ruby[g]{高田}{たかた }は
\ruby{我}{わ}が
\ruby{意}{こゝろ}の
\ruby{{\換字{通}}}{とほ}り
たるより
\ruby{胸}{むね}は
\ruby{安}{やす}く
せしものゝ、
%
\ruby{却}{かへ}つて
\ruby{{\換字{又}}}{また}
\ruby[g]{對手}{あひて }の% ルビ調整(原本通り)非グループルビ
\ruby{餘}{あま}りに
\ruby{未練氣}{み|れん|げ}
\ruby{無}{な}きに
\ruby{薄氣味}{うす|き|み}
\ruby{惡}{あし}く、
%
\ruby[g]{懸念}{け ねん}らしく
\ruby{小}{ちひさ}き
\ruby{眼}{め}を
\ruby[<j>]{瞬}{しばたゝ}きて
\ruby[g]{水野}{みづの }を
\ruby[g]{見居}{み ゐ }たり。

\原本頁{14-8}%
『
\換字{志}かし
\ruby[g]{水野}{みづの }さん
\ruby{決}{けつ}して
\ruby{御不快}{ご|ふ|かい}に
\ruby[g]{御思}{お おも}ひ
なすつては
いけません、
%
\ruby[g]{何樣}{ど う }か
\ruby[||j>]{{\換字{感}}}{かん}
\ruby[||j>]{{\換字{情}}}{じやう}を
% \ruby{{\換字{感}}{\換字{情}}}{かん|じやう}を
\ruby{{\換字{害}}}{がい}して
\ruby{下}{くだ}さらん
やうに
\ruby{願}{ねが}ひます。
%
\ruby[g]{小生}{わたくし}は
\ruby[g]{何處}{ど こ }
\ruby{迄}{まで}も
\ruby[|g|]{貴下}{あなた}を
\ruby{信}{しん}じて
\ruby{居}{を}る
のですから、
%
\ruby[|g|]{貴下}{あなた}に
\ruby{校}{かう}から
\ruby{離}{はな}れて
\ruby{頂}{いたゞ}きたい
\ruby[<j||]{心}{こゝろ}
\原本頁{15-1}\改行%
は
\ruby{{\換字{更}}}{さら}に
\ruby{無}{な}い
のでして、
%
\ruby{長}{なが}く
\ruby[|g|]{貴下}{あなた}と
\ruby[g]{圓滿}{ゑんまん}な
\ruby{御{\換字{交}}際}{ご|かう|さい}を
\ruby[g]{繼續}{つ な }いで
\ruby{參}{まゐ}りたいのです。
%
\ruby[|g|]{貴下}{あなた}は
\ruby[g]{失禮}{しつれい}
ながら
\ruby[||j>]{學}{がく}
\ruby[||j>]{力}{りよく}は
% \ruby{學力}{がく|りよく}は
\ruby[g]{御有}{お あ }りなさるし、
%
なかなか% ルビ調整(原本通り)非踊り字表記(行末行頭の境界付近)
\ruby{長}{なが}く
\ruby[g]{小學}{せうがく}の
\ruby[g]{敎師}{けうし }などを
\ruby{仕}{し}て
\ruby{居}{ゐ}らつしやる
\ruby[g]{御仁}{ご じん}では
\ruby{無}{な}いのです
\改行% 校正作業の簡略化のため
。
%
\原本頁{15-4}\改行%
が、
%
\ruby[g]{差當}{さしあた}つて
\ruby{校}{かう}の
\ruby{方}{はう}を
\ruby{離}{はな}れて
\ruby[g]{戴い}{いたゞ }
ては
\ruby[g]{御困}{お こま}り
でも
ございましやう
から、
%
\ruby[g]{小生}{わたくし}は
\ruby[g]{小生}{わたくし}の
\ruby[|g|]{貴下}{あなた}に
\ruby{對}{たい}する
\ruby[||j>]{眞}{しん}
\ruby[||j>]{{\換字{情}}}{じやう}を
% \ruby{眞{\換字{情}}}{しん|じやう}を
\ruby{表}{へう}して、
%
\ruby[|g|]{貴下}{あなた}を
\ruby[g]{他{\換字{所}}}{よ そ }の
\ruby{校}{かう}へ
\ruby{御周旋}{ご|しう|せん}
\ruby{致}{いた}しましようと
\ruby{存}{ぞん}じて
\ruby{居}{を}ります。
%
\ruby[g]{何樣}{ど う }か
\ruby[g]{小生}{わたくし}が
\ruby[g]{貴下}{あなた }に% ルビ調整(原本通り)非グループルビ
\ruby{對}{たい}する
\ruby[g]{敬意}{けいい }を
\ruby[g]{御汲}{お く }み
\ruby{取}{と}り
\ruby{下}{くだ}すつて
\ruby{頂}{いたゞ}きたい
もので。
』

\原本頁{15-8}%
と、
%
\ruby{是}{これ}も
また
\ruby{三十匁}{さん|じふ|め}の
\ruby{茶}{ちや}を
\ruby{入}{い}るゝに
\ruby{湯}{ゆ}を
\ruby{冷}{さ}まして
\ruby{後}{のち}にするが
\ruby{如}{ごと}く
\ruby[g]{叮嚀}{ていねい}に
\ruby{言}{い}へば、
%
\ruby[g]{水野}{みづの }は
\ruby{他}{ひと}に
\ruby{憎}{にく}まれじ
\ruby{恨}{うら}まれじとする
\ruby[||j>]{心}{こゝろ}
\ruby[||j>]{{\換字{遣}}}{ づか}ひの
% \ruby{心{\換字{遣}}}{こゝろ|づか}ひの
\改行% 校正作業の簡略化のため
、
%
\原本頁{15-10}\改行%
いと
\ruby{明}{あき}らかに
\ruby{見}{み}ゆる
\ruby{此}{こ}の
\ruby[g]{{\換字{半}}白}{はんぱく}の
\ruby{敎育家}{けう|いく|か}を、
%
\ruby[g]{憫然}{あはれ }に
\ruby{思}{おも}ふやうの
\ruby[<j||]{{\換字{情}}}{こゝろ}も% 行末行頭の境界付近なので特例処置を施す
\ruby{起}{おこ}りて、

\原本頁{16-1}%
『
はい、
%
\ruby{有}{あ}り
\ruby{{\換字{難}}}{がた}う
ございます。
%
\ruby{御高{\換字{情}}}{ご|かう|せい}は
まことに
\ruby{有}{あ}り
\ruby{{\換字{難}}}{がた}う
ございます。
%
\ruby{御言葉}{お|こと|ば}に
\ruby{甘}{あま}えまして
\ruby[|g|]{何處}{いづれ}かへ
\ruby{御周旋}{ご|しう|せん}を
\ruby{願}{ねが}はなくつては
\原本頁{16-3}\改行%
ならんのですが、
\換字{志}かし
\ruby[g]{小生}{わたくし}は
\ruby[g]{何樣}{ど う }も
\ruby[g]{敎鞭}{けうべん}を
\ruby{執}{と}るには
\ruby{{\換字{適}}}{てき}せん
やうに
\ruby{思}{おも}ひます
から、
%
\ruby[g]{差當}{さしあた}つて
\ruby[g]{他{\換字{所}}}{よ そ }の
\ruby{校}{かう}へ
\ruby{參}{まゐ}りたいとも
\ruby{存}{ぞん}じませんです。
%
\ruby{御厚意}{ご|かう|い}は
\ruby[g]{何處}{ど こ }までも
\ruby{有}{あ}り
\ruby{{\換字{難}}}{がた}く
\ruby{存}{ぞん}じます
けれども、
%
\ruby[g]{當{\換字{分}}}{たうぶん}は
\ruby{{\換字{遊}}}{あそ}んで
\ruby{見}{み}たいと
\ruby{思}{おも}つて
\ruby{居}{を}りまする。
%
それでは
\ruby[g]{辭表}{じ へう}は
\ruby[<j||]{明}{みやう}
\ruby[<j||]{日}{にち}
% \ruby{明日}{みやう|にち}
\ruby[g]{早{\換字{速}}}{さつそく}
\ruby{差}{さ}し
\ruby{出}{だ}しまするから、
%
\ruby[g]{何{\換字{分}}}{なにぶん}
\ruby{宜}{よろ}しく
\ruby[g]{御計}{お はか}らひを
\ruby{願}{ねが}ひまする。
』

\原本頁{16-8}%
と、
%
\ruby{{\換字{飽}}}{あく}まで
\ruby[g]{{\換字{謙}}{\換字{退}}}{けんたい}して
\ruby[g]{柔和}{にうわ }に
\ruby{應}{こた}へたり。

\原本頁{16-9}%
\ruby[g]{水野}{みづの }が
\ruby{面}{おもて}に
\ruby[g]{怨氣}{ゑんき }をも
\ruby{盛}{も}らずして、
%
\ruby[|g|]{{\換字{平}}常}{ふだん}の
\ruby{如}{ごと}く
\ruby[g]{何氣}{なにげ }なき
\ruby{言}{ことば}の
\ruby[g]{調子}{てうし }に
\ruby{職}{しよく}を
\ruby{辭}{じ}せん
といふを
\ruby{聞}{き}き、
%
\ruby[g]{高田}{たかた }は
やうやく
\ruby{荷}{に}を
\ruby{下}{おろ}したる
\ruby[g]{心地}{こゝち }してか、

\原本頁{17-1}%
『
ヤ、
%
それでは
\ruby[g]{當{\換字{分}}}{たうぶん}
\ruby[g]{御{\換字{遊}}}{お あそ}びも
\ruby{宜}{よろ}しう
ございましやう。
%
\ruby{疾}{とう}から% 「疾から」早くから。前々から。とっくに
\ruby[g]{小生}{わたくし}は
\ruby[|g|]{貴下}{あなた}を
\ruby{目}{もく}して、
%
\ruby[g]{蛟龍}{かうりう}
%「蛟龍」中国の伝説上の生き物。
% ながく水中に潜んでいるが、
% 時がくれば雨や雲に乗り、
% 天に昇って竜となる竜の幼生。
% みずち。
% (比喩)時運に恵まれずに実力を発揮できない英雄や豪傑。
\ruby{永}{なが}く
\ruby[g]{池中}{ちちゆう}の
ものたらずと
\ruby{申}{まを}して
\ruby{居}{を}りました
のです。
%
ハヽヽ。
%
\ruby[g]{何樣}{ど う }か
\ruby[g]{今後}{こんご }
\ruby[g]{何{\換字{分}}}{なにぶん}
\ruby{御見棄}{お|み|すて}
\ruby{無}{な}く
\ruby{御{\換字{交}}際}{ご|かう|さい}を
\ruby{願}{ねが}ひまする。
』

\原本頁{17-5}%
と
\ruby[|g|]{可笑}{をかし}くも
\ruby{無}{な}き
ところに
\ruby[g]{磊落}{らいらく}
%「磊落」気が大きく朗らかで、小さいことにこだわらないこと。
%        気性がさっぱりしていること。
めかして
\ruby{妙}{めう}に
\ruby{笑}{わら}つて、
%
\ruby[g]{最後}{さいご }には
\ruby[<j||]{改}{あらた}めて% 行末行頭の境界付近なので特例処置を施す
\ruby{肘}{ひぢ}を
\ruby{張}{は}つて
\ruby{堅}{かた}くろしく
\ruby{頭}{かうべ}を
\ruby{下}{さ}げて
\ruby[g]{一禮}{いちれい}すれば、
%
\ruby[g]{水野}{みづの }も
\ruby[g]{是非}{ぜ ひ }
なく
\ruby{禮}{れい}を
\ruby{{\換字{返}}}{かへ}して、

\原本頁{17-8}%
『
いや
\ruby[g]{今後}{こんご }の
\ruby{御{\換字{交}}際}{ご|かう|さい}は
\ruby[g]{小生}{わたくし}の
\ruby{方}{はう}から
こそ
\ruby{願}{ねが}ふべきで。
%
では
\ruby[g]{今日}{こんにち}は
これで
\ruby[g]{失禮}{しつれい}
\ruby{致}{いた}します。
』

\原本頁{17-10}%
と
\ruby[g]{慇懃}{いんぎん}に
\ruby[g]{挨拶}{あいさつ}して
\ruby{辭}{じ}し
\ruby{歸}{かへ}りたり。

\原本頁{17-11}%
\ruby[g]{區々}{く ゝ }たる
\ruby{職}{しよく}と
\ruby[g]{些々}{さ ゝ }たる
\ruby[g]{俸給}{ほうきふ}とは、
%
\ruby{之}{これ}を
\ruby{得}{う}るも
\ruby{之}{これ}を
\ruby{失}{うしな}ふも
\ruby[g]{一顰}{いつぴん}
\ruby[g]{一笑}{いつせう}にだに
\ruby{價}{あたひ}せずと、
%
\ruby[g]{水野}{みづの }は
\ruby{其}{その}
\ruby{事}{こと}を
\ruby{繰}{く}り
\ruby{{\換字{返}}}{かへ}しても
\ruby{思}{おも}はず、
%
たゞ
\ruby{{\換字{猶}}}{なほ}
\原本頁{18-2}\改行%
\ruby{微}{かすか}に
\ruby{殘}{のこ}れる
\ruby{醉}{よひ}を% 「醉」は原本通り「よ」で調整
\ruby{吹}{ふ}く
\ruby{風}{かぜ}の
\ruby{薄}{うす}
\ruby{{\換字{寒}}}{さむ}きを
\ruby{覺}{おぼ}えつゝ
\ruby{歸}{かへ}り
\ruby{着}{つ}けば、
%
お
\ruby{濱}{はま}は
\ruby{待}{ま}ち
\ruby{{\換字{兼}}}{か}ねしが
\ruby{如}{ごと}く
\ruby{飛}{とん}で
\ruby{出}{い}でゝ、
%
\ruby{茶}{ちや}の
\ruby{間}{ま}に
\ruby[||j>]{{\換字{迎}}}{むかへ}
\ruby[||j>]{入}{ い }るゝや
\ruby{否}{いな}や、
%
\ruby[g]{滿面}{まんめん}に
\ruby{笑}{ゑみ}を
\ruby{輝}{かゞや}かしつ、
%
\ruby[g]{他人}{ひ と }には
\ruby{何}{なに}
\ruby{言}{い}ふ
\ruby{間}{ひま}をも
\ruby{與}{あた}へずして、

\原本頁{18-5}%
『
\ruby{今}{いま}
\ruby[g]{先生}{せんせい}と
\ruby{入}{い}れ
\ruby{{\換字{違}}}{ちが}つてネ、
%
\ruby{彼}{あ}の
\ruby[g]{尾竹}{を だけ}が
\ruby{變}{へん}に
\ruby[g]{威張}{ゐ ば }つて
\ruby{{\換字{遣}}}{や}つて
\ruby{來}{き}ましてネ。
%
とう〳〵
\ruby[|g|]{此方}{こつち}の
ものに
\ruby{仕}{し}た、
%
もう
\ruby{大{\換字{丈}}夫}{だい|ぢやう|ぶ}だ、
%
もう
\ruby[g]{屹度}{きつと }% ルビ調整(原本通り)非グループルビ
\原本頁{18-7}\改行%
\ruby[|g|]{保證}{うけあ}ひます、
%
もう
\ruby{宜}{よ}う
ございます、
%
もう
\ruby{是}{これ}からは
\ruby[g]{快癒}{な ほ }る
ばかりです、
%
\ruby{必}{かなら}ず
\ruby{五十子}{い|そ|こ}さんは
\ruby[g]{本復}{ほんぷく}する
といふ
\ruby[g]{見{\換字{込}}}{み こ }みが
\ruby{立}{た}ちました。
%
\原本頁{18-9}\改行%
\ruby[g]{水野}{みづの }さんに
\ruby[g]{十{\換字{分}}}{じふぶん}
\ruby{悅}{よろこ}んで
\ruby{貰}{もら}は
なくちやあ、
%
と
\ruby{云}{い}つて
\ruby{今}{いま}まで
\ruby[|g|]{饒舌}{しやべ}つて
\ruby{行}{ゆ}きましたよ。
%
\ruby{嬉}{うれ}しいのネエ
\ruby[g]{先生}{せんせい}。
%
\ruby[||j>]{妾}{わたし}
\ruby[||j>]{嬉}{ うれ}しくつて!。
%
ほんとに
\ruby[||j>]{妾}{わたし}
\ruby[||j>]{嬉}{ うれ}しくつて〳〵!。
』

\原本頁{19-1}%
と
\ruby{急}{せ}きに
\ruby{急}{せ}きて
\ruby[g]{喜悅}{よろこび}の
\ruby[g]{音信}{おとづれ}を
\ruby{傳}{つた}へたり。

\原本頁{19-2}%
お
\ruby{濱}{はま}は
\ruby{我}{わ}が
\ruby{此}{こ}の
\ruby[g]{言葉}{ことば }を
\ruby{聞}{き}くと
\ruby{齊}{ひと}しく
\ruby[g]{水野}{みづの }の
\ruby[g]{如何}{い か }に
\ruby{悅}{よろこ}びて
\ruby{笑}{ゑ}むならんと
\ruby{思}{おも}ひ
\ruby{設}{まう}けつ、
%
\ruby[<j||]{心}{こゝろ}
\ruby[||j>]{樂}{だのし}み
にして
\ruby[g]{水野}{みづの }の
\ruby{面}{おもて}を
\ruby{差}{さし}
\ruby{覗}{のぞ}けるに、
%
\ruby[||j>]{悅}{よろこ}び%
%%%%
\ruby{極}{きは}まつてか
\ruby{其}{その}
\ruby{人}{ひと}は
\ruby{笑}{ゑ}まず、
%
\ruby{目}{ま}の
あたりに
\ruby[||j>]{神}{かみ}
\ruby[||j>]{佛}{ほとけ}
% \ruby{神佛}{かみ|ほとけ}
をも
\ruby{拜}{をが}めるが
\ruby{如}{ごと}き
\改行% 校正作業の簡略化のため
、
%
\原本頁{19-5}\改行%
\ruby{敬}{つゝし}みに
\ruby{敬}{つゝし}めるが
\ruby{中}{なか}に
\ruby{和}{やさ}しさ
\ruby{見}{み}ゆる
\ruby{面}{おもて}に
なつて、
%
\ruby[g]{抑々}{そも〳〵}
\ruby{何}{なに}をか
\ruby[g]{見詰}{み つ }むるや
\ruby{頭}{かしら}を
\ruby{斜}{なゝめ}に、
%
\ruby{物}{もの}も
\ruby{無}{な}き
\ruby[||j>]{{\換字{空}}}{くう}
\ruby[||j>]{中}{ちゆう}を
% \ruby{{\換字{空}}中}{くう|ちゆう}を
\ruby[g]{凝然}{じ つ }と
\ruby{仰}{あふ}ぎたるが、
%
\ruby{見}{み}る〳〵
\原本頁{19-7}\改行%
\ruby{動}{うご}かざる
\ruby{其}{そ}の
\ruby{眼}{め}の
\ruby{中}{うち}よりは、
%
\ruby[g]{汪々}{わう〳〵}
\ruby[g]{漣々}{れん〳〵}として
\ruby{涙}{なみだ}の
\ruby{溢}{あふ}れたり。
%
\ruby[<j||]{悅}{よろこ}び
\ruby{涙}{なみだ}
とは
これ
なる
べき
にや。
