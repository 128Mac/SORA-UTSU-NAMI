\Entry{其四}

% メモ 校正終了 2024-05-10 2024-06-06
\原本頁{19-10}%
\ruby[|g|]{相良}{さがら}にも
\ruby{尾竹}{を|だけ}にも
\ruby[||j>]{囘}{くわい}
\ruby[||j>]{復}{ ふく}の% 原本通り「囘」
% \ruby{囘復}{くわい|ふく}の% 原本通り「囘」
\ruby{望無}{のぞみ|な}し
とこそは
\ruby{言}{い}はれざりつれ、
%
\ruby{十}{じふ}に
\原本頁{20-1}\改行%
\ruby{六七}{ろく|しち}% 原本には漢数字「七」のルビ無し
までは
\ruby{危}{あやふ}く
\ruby{思}{おも}はれたる
らしき
\ruby{徴}{しるし}には、
%
\ruby{變狀}{へ|ん}さへ
\ruby{無}{な}くば、
%
\ruby{變狀}{へ|ん}さへ
\ruby{無}{な}くばと、
%
\ruby{遁路}{にげ|みち}のある
\ruby{保證}{うけ|あひ}の
\ruby{仕方}{し|かた}を
\ruby{爲}{さ}れたる、
%
\ruby{其}{そ}の
\ruby{重}{おも}き
\ruby{病}{やまひ}に% 原本では「病」の直後にアキあるが
\ruby{惱}{なや}みし
\ruby{人}{ひと}の、
%
\ruby{今}{いま}は
\ruby{必}{かなら}ず
\ruby{癒}{なほ}る
べしとは
\ruby[|g|]{眞實}{まこと}の
\ruby{事}{こと}なりや、
%
\ruby{覺}{さ}めての
\ruby{後}{のち}の
\ruby{口惜}{く|や}しかる
べき
\ruby{夢}{ゆめ}の
\ruby{中}{うち}の
\ruby{果敢無}{は|か|な}き
\ruby{悅}{よろこ}び
には
あらざるや。
%
あゝ、
%
\ruby{夢}{ゆめ}には
あらず、
%
\ruby{確}{たしか}に
\ruby{現}{うつゝ}なり、
%
\ruby{虛妄}{いつ|はり}には
あらず、
%
\ruby{確}{たしか}に
\原本頁{20-6}\改行%
\ruby[|g|]{眞實}{まこと}なり。
%
かつては
\ruby{人}{ひと}の
\ruby{{\換字{運}}命}{うん|めい}の
\ruby{頼}{たの}み
\ruby{無}{な}きを
\ruby{悲}{かなし}みて、
%
\ruby{訴}{うつた}ふる
\ruby{方}{かた}
\ruby{無}{な}き
\ruby{我}{わ}が
\ruby{思}{おもひ}の、
%
\ruby{{\換字{空}}}{むな}しく
\ruby{流水}{なが|れ}に
\ruby{描}{ゑが}く
\ruby{{\換字{文}}字}{もん|じ}
となつて
\ruby{{\換字{消}}}{き}ゆべきかを
\ruby{歎}{なげ}きしも、
%
\ruby{今}{いま}は
\ruby{天地}{てん|ち}の
\ruby{間}{あひだ}に
\ruby[|g|]{愛{\換字{情}}}{なさけ}
\ruby{有}{あ}り
\ruby{{\換字{道}}義}{み|ち}
\ruby{有}{あ}つて、
%
\ruby{神明}{か|み}
\ruby[<g>]{佛陀}{ほとけ}の
\ruby{慈愍}{じ|みん}の
\ruby{御}{おん}
\ruby[|-|]{眥}{まなじり}は
\ruby{人間}{ひ|と}の
\ruby{上}{うへ}を
\ruby{離}{はな}れず、
%
\ruby{愛護}{あい|ご}の
\ruby{御手}{おん|て}は
\ruby{一切}{いつ|さい}の
\ruby[||j>]{衆}{しゆ}
\ruby[||j>]{生}{じやう}を
% \ruby{衆生}{しゆ|じやう}を
\ruby{攝取}{せつ|しゆ}して
\ruby{捨}{す}て
ざらんと
\ruby{仕玉}{し|たま}へることを
\ruby{思}{おも}ひ
\ruby[<j>]{奉}{たてまつ}り、
%
\ruby{愚}{おろか}しき
\ruby{一念}{いち|ねん}の
\ruby{誠}{まこと}を
\ruby{籠}{こ}めて、
%
\ruby{他人}{ひ|と}には
\ruby{言}{い}へぬ
\ruby{心中}{しん|ちう}の
\ruby{秘事}{ひめ|ごと}に、
%
あはれ
\ruby{彼}{か}の
\ruby{人}{ひと}の
\ruby{壽}{いのち}の
\ruby{無}{な}きに
\ruby{定}{さだ}まれる
ならば、
%
\ruby{我}{わ}が
\ruby{生命}{いの|ち}を
\ruby{殺}{そ}ぎ
\ruby{縮}{ちゞ}めても
\ruby{助}{たす}けさせ
\ruby{玉}{たま}へ、
%
かかる
\ruby{{\換字{道}}理}{こと|わり}
\ruby{無}{な}き
\ruby{願}{ねが}ひを
\ruby{掛}{か}け
\ruby[<j>]{奉}{たてまつ}ることの、
%
\ruby{愚}{おろか}にも
\ruby{愚}{おろか}なるをば
\ruby{知}{し}らぬには
あらねど、
%
\ruby{知}{し}りて
\ruby{{\換字{猶}}}{なほ}
\ruby{已}{や}まんとして
\ruby{已}{や}み
\ruby{{\換字{難}}}{がた}き
\ruby{胸}{むね}の
\ruby{苦}{くる}しさは、
%
\原本頁{21-4}\改行%
\ruby{御覽}{み|そな}はさぬ
ところ
\ruby{無}{な}き
\ruby{神明}{か|み}
\ruby[|g|]{佛陀}{ほとけ}の
\ruby{見{\換字{透}}}{み|とほ}し
たまひて、
%
\ruby{憫然}{あは|れ}とも
\ruby{思}{おぼ}して
\ruby{我}{わ}が
\ruby{心}{こゝろ}をば
\ruby{納}{い}れ
させたまへ、
%
と
\ruby{祈}{いの}りたり
しが、
%
\ruby{彼}{か}の
\ruby{人}{ひと}の
\ruby[||j>]{壽}{じゆ}
\ruby[||j>]{命}{みやう}の
% \ruby{壽命}{じゆ|みやう}の
\ruby{本}{もと}より
\ruby{有}{あ}りしか、
%
\ruby{我}{わ}が
\ruby{命}{いのち}の
\ruby{彼}{か}の
\ruby{人}{ひと}の
\ruby{命}{いのち}を
\ruby{補}{おぎな}ひしかは
\ruby{知}{し}らず
\改行% 校正作業の簡略化のため
、
%
\原本頁{21-7}\改行%
\ruby{大旱}{ひ|でり}に
\ruby{萎}{しな}れし
\ruby{玉苗}{たま|なえ}の、
%
\ruby{一夜}{いち|や}の
\ruby{露}{つゆ}に
\ruby[<j>]{蘇}{よみがへ}つて、
%
\ruby{田面}{た|づら}を
\ruby{渡}{わた}る
\ruby{曉風}{あさ|かぜ}には
\ruby{{\換字{猶}}}{なほ}
\ruby{{\換字{弱}}々}{よわ|〳〵}と
\ruby{戰}{そよ}ぎ
ながらも、
%
はや
\ruby{行末}{ゆく|すゑ}の
\ruby{頼}{たの}もしき
\ruby{榮}{さかえ}を
\ruby{見}{み}する
\ruby{其}{そ}の
\ruby{色}{いろ}の
\ruby{靑々}{あを|〳〵}と
\ruby[<j>]{勢}{いきほひ}
\ruby[||j>]{好}{ よ }きが
\ruby{如}{ごと}く、
%
\ruby{危}{あやふ}くも
\ruby[||j>]{心}{こゝろ}
\ruby[||j>]{細}{ ぼそ}かりし
% \ruby{心細}{こゝろ|ぼそ}かりし
\ruby{病}{やまひ}の
\ruby{瀬}{せ}を
\ruby{{\換字{過}}}{す}ぎて、
%
\ruby[<j||]{全}{まつた}く% 行末行頭の境界付近なので特例処置を施す
\ruby{復}{また}
\ruby[|g|]{現世}{このよ}の
\ruby{光}{ひかり}に
\ruby{美}{うつく}しう
\ruby{照}{て}らさるゝ
やうになりし
\ruby{彼}{か}の
\ruby{人}{ひと}の
\ruby{{\換字{運}}}{うん}の
\ruby{目出度}{め|で|た}さ、
%
\ruby{我}{わ}が
\ruby{心}{こゝろ}の
\ruby{嬉}{うれ}しさ。
%
\ruby{思}{おも}へば
\ruby{神明}{か|み}も
\ruby[|g|]{佛陀}{ほとけ}も
\ruby{確}{たしか}に
\ruby{御坐}{お|は}す
\ruby{世}{よ}なり。
%
\ruby{人間}{ひ|と}を
\ruby{包}{つゝ}める
\ruby{{\換字{運}}命}{うん|めい}は
\ruby{雲}{くも}
\ruby{霧}{きり}と
\ruby{冥}{くら}くして
\ruby{得}{え}
\ruby{知}{し}れねども、
%
\ruby{其}{その}
\ruby{中}{うち}に
\原本頁{22-2}\改行%
\ruby{神明}{か|み}の
\ruby{御心}{み|こゝろ}
\ruby[g]{佛陀}{ほとけ}の
\ruby{御心}{み|こゝろ}は
\ruby{動}{うご}き
\ruby{働}{はたら}きて、
%
\ruby{人間}{ひ|と}の
\ruby{抱}{いだ}く% ルビ調整(原本通り)(いだ)
\ruby{心}{こゝろ}の
さまに
\ruby{酬}{むく}ひ
たまふやうの
\ruby{氣}{き}ぞする。
%
\ruby{冥々}{めい|〳〵}の
\ruby{中}{うち}に
\ruby{靈}{く}しき
\ruby{力}{ちから}
ありて
\ruby{神佛}{しん|ぶつ}の
\ruby{意}{い}を
\ruby{受}{う}け、
%
\ruby{吉}{よき}も
\ruby{凶}{あしき}も
\ruby{皆}{みな}
\ruby{其力}{そ|れ}の
\ruby{爲}{す}る
\ruby{事}{こと}の
やうにぞ
\ruby{思}{おも}はるゝ。
%
\ruby{神明}{か|み}も
\原本頁{22-5}\改行%
\ruby{{\換字{遠}}}{とほ}からず、
%
\ruby[g]{佛陀}{ほとけ}も
\ruby{{\換字{遠}}}{とほ}からず、
%
\ruby{一念}{いち|ねん}の
\ruby{微}{かすか}なる
\ruby{動}{うご}きも
\ruby{洩}{も}らさず
\ruby{知}{し}り
たまふと
\ruby{覺}{おぼ}ゆ。
%
\ruby{嗚呼}{あ|ゝ}、
%
\ruby{神明}{か|み}も
\ruby[g]{佛陀}{ほとけ}も
\ruby{{\換字{猶}}}{なほ}
\ruby{御覽}{み|そな}はせ、
%
\ruby{我}{わ}が
\ruby{心}{こゝろ}の
\ruby{誠}{まこと}を
\改行% 校正作業の簡略化のため
。
%
\原本頁{22-7}\改行%
\ruby{邪無}{よこしま|な}く、
%
\ruby{汚無}{けがれ|な}く、
%
\ruby[<j>]{僞}{いつはり}
\ruby[||j>]{無}{ な }く
\ruby{人}{ひと}を
\ruby{思}{おも}ひて、
%
\ruby{我}{わ}が
\ruby{如何}{い|か}に
してか
\ruby{有}{あ}り
\ruby{果}{は}つべき
\ruby{我}{わ}が
\ruby{世}{よ}の
\ruby{末}{すゑ}を
\ruby{見}{み}んとぞ
\ruby{思}{おも}ふ。
%
\ruby{實}{げ}に
\ruby{地}{つち}を
\ruby{掘}{ほ}れば
\ruby{水}{みづ}に
\ruby{逢}{あ}ひ、
%
\原本頁{22-9}\改行%
\ruby{壁}{かべ}を
\ruby{穿}{うが}てば
\ruby{光}{ひかり}に
\ruby{逢}{あ}ひ、
%
\ruby{人}{ひと}の
\ruby{心}{こゝろ}の
\ruby{奧}{おく}に
\ruby{入}{い}れば
\ruby{必}{かなら}ず
\ruby{神明}{か|み}
\ruby[g]{佛陀}{ほとけ}に
\ruby{逢}{あ}ひ
\ruby[|-|]{奉}{たてまつ}る
ものと
\ruby{云}{い}へるも
\ruby{言}{い}ひ
\ruby{得}{え}たる
ことかな。
%
\ruby{我}{われ}
\ruby{今}{いま}
\ruby[|-|]{幸}{さいはひ}にして
\ruby{眼}{ま}の
あたりに
\ruby{利生}{り|しやう}を
\ruby{仰}{あふ}ぎ
\ruby{得}{え}、
%
\ruby{冥々}{めい|〳〵}の
\ruby{中}{うち}に
\ruby{御坐}{お|は}して
\ruby{果敢無}{は|か|な}き
\ruby{此}{こ}の
\ruby{我}{われ}を
\ruby{愛}{いつく}しみ
たまふ
\ruby{大慈}{だい|じ}
\ruby{大悲}{だい|ひ}の
\ruby[||j>]{御}{おん}
\ruby[||j>]{心}{こゝろ}の
% \ruby{御心}{おん|こゝろ}の
\ruby[|->]{忝}{かたじけな}きを
\ruby{{\換字{感}}}{かん}じて、
%
\ruby{此}{こ}の
\ruby{嬉}{うれ}しさ
\原本頁{23-2}\改行%
\ruby{有}{あ}り
\ruby{{\換字{難}}}{がた}さ
\ruby{肺腑}{はい|ふ}に
\ruby{浸}{し}み
\ruby{徹}{とほ}りぬ。
%
\ruby{願}{ねが}はくは
\ruby{我}{われ}
\ruby{長}{なが}く
\ruby[||j>]{此}{この}
\ruby[||j>]{心}{こゝろ}を
% \ruby{此心}{この|こゝろ}を
\ruby{失}{うしな}はず
して
\ruby{頼}{たの}み
\ruby[<->]{奉}{たてまつ}らん
ほどに、
%
\ruby{{\換字{猶}}}{なほ}
\ruby{行末}{ゆく|すゑ}
\ruby{掛}{か}けて
\ruby{彼}{か}の
\ruby{人}{ひと}の
\ruby{上}{うへ}に
\ruby{幸福}{さい|はひ}
\ruby{多}{おほ}からしめ
\ruby{給}{たま}へ。
%
\ruby{我}{わ}が
\ruby{身}{み}の
\ruby{幸福}{さい|はひ}をば
\ruby{祈}{いの}り
\ruby{求}{もと}むれば
こそ、
%
たゞ
\ruby{彼}{か}の
\ruby{人}{ひと}
\ruby{好}{よ}かれ
とのみ
\ruby{思}{おも}ふ
こゝろの、
%
\ruby{此}{こ}の
\ruby{虛僞}{いつ|はり}
\ruby{無}{な}き
\ruby[|g|]{眞實}{まこと}を
\ruby{汲}{く}ませたまへ、
%
と
\ruby{水野}{みづ|の}は
\ruby{默念}{もく|ねん}したり。

\原本頁{23-7}%
\ruby{其}{その}
\ruby{夜}{よ}
\ruby{水野}{みづ|の}は
\ruby{何事}{なに|ごと}を
\ruby{思}{おも}ひ
つゞけしにや、
%
\ruby{{\換字{更}}}{ふ}くる
まで
\ruby{{\換字{終}}}{つひ}に
\ruby{睡}{ねむ}りに
\ruby{入}{い}らで、
%
\ruby{二番鷄}{に|ばん|とり}の
\ruby{唱}{うた}ふ
\ruby{頃}{ころ}
\ruby{辛}{から}くも
\ruby{夢}{ゆめ}を
\ruby{結}{むす}びぬ。
%
たゞ
\ruby{思}{おも}ふ
\ruby{人}{ひと}の
\ruby{病}{やまひ}の
\ruby{快}{よ}き
\ruby{方}{かた}に
\ruby{向}{むか}へるを
\ruby{悅}{よろこ}んで、
%
おのが
\ruby{職}{しよく}を
\ruby{失}{うしな}へる
こと
なんどは
\ruby{悔}{くや}みも
せざりし
なるべし。
