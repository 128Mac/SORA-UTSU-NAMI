\Entry{其五}

% メモ 校正終了 2024-05-10 2024-06-06
\原本頁{24-2}%
お
\ruby{濱}{はま}は
\ruby{可笑}{をか|し}さに% 原本通り非グループルビ
\ruby{堪}{た}へぬ
\ruby{如}{ごと}く
\ruby{笑}{わら}ひを
\ruby{耐}{こら}へ
ながら、

\原本頁{24-3}%
『
マア
\ruby[|g|]{如是}{こんな}に
\ruby{晏}{おそ}く
\ruby{起}{お}きて
\ruby{置}{お}いて、
%
\ruby{而}{さう}して
\ruby{變}{へん}に
\ruby{沈着}{おち|つ}いて
\ruby{居}{ゐ}らつしやるの\換字{子}。
%
\ruby{先生}{せん|せい}、
%
\ruby{今日}{け|ふ}は
\ruby{日曜}{にち|えう}ぢやあ
\ruby{有}{あ}りませんよ。
%
\ruby[|g|]{早{\換字{速}}}{さつさ}と
なさらないと
もう
\ruby{遲}{おく}れますよ。
%
\ruby{彼人}{な|に}が
\ruby{快}{い}いもんで
\ruby{安心}{あん|しん}して
\ruby{仕舞}{し|ま}つて
\改行% 校正作業の簡略化のため
、
%
\原本頁{24-6}\改行%
それで
\ruby{全然}{すつ|かり}
\ruby{氣}{き}が
\ruby{弛}{ゆる}んで
\ruby{御仕舞}{お|し|ま}ひ
なすつたの?。
%
\ruby{妙}{めう}に
\ruby{今{\換字{朝}}}{け|さ}は
ゆつたり
として
\ruby{澄}{す}まして
\ruby{居}{ゐ}らつしやるのネエ。
%
\ruby{何樣}{ど|う}なすつたの?、
%
\ruby{餘}{あんま}りだは!、
%
をかしくつてよ。
』

\原本頁{24-9}%
と
\ruby{戲}{たはむ}るゝが
\ruby{如}{ごと}く
\ruby{云}{い}ひしが
\ruby{{\換字{終}}}{つひ}に
\ruby{堪}{こら}へ
かねて、

\原本頁{24-10}%
『
ホヽホヽヽヽ。
』

\原本頁{25-1}%
と
\ruby{笑}{わら}ひ
\ruby{出}{いだ}したり。

\原本頁{25-2}%
\ruby{夢}{ゆめ}の
\ruby{名殘}{な|ごり}を
\ruby{洗}{あら}ふ
\ruby{{\換字{朝}}茶}{あさ|ぢや}の
\ruby{淡}{あは}き
\ruby[<j>]{味}{あぢはひ}を
\ruby{樂}{たのし}みて、
%
\ruby{悠然}{いう|ぜん}として
\ruby{湯呑}{ゆ|のみ}を
\ruby{手}{て}にして
\ruby{居}{ゐ}たりし
\ruby{水野}{みづ|の}は
\ruby{此}{こ}の
\ruby{笑}{わらひ}に
\ruby{驚}{おどろ}かされつ、
%
\ruby{實}{げ}に
\ruby{我}{わ}が
\ruby{心}{こゝろ}の
\ruby{中}{うち}の
\ruby{昨日}{きの|ふ}に% 行末で行替えが発生しているのでここはグループルビにはしない
\ruby{今日}{け|ふ}は
\ruby{甚}{いた}く
\ruby{異}{こと}なりて
\ruby{寛}{ゆたか}なれば、
%
\ruby{外}{そと}に
\ruby{現}{あらは}るゝ
\ruby{身}{み}の
\ruby{樣子}{やう|す}も、
%
\ruby{他}{ひと}には
\ruby{可笑}{を|か}しき
ほど
\ruby{變}{かは}れる
なるべし。
%
\ruby{特}{こと}に
\ruby{掌上}{ての|ひら}に
\ruby{乘}{の}るばかりの
\ruby{微少}{わづ|か}なる% 原本通り非グループルビ
\ruby{俸米}{ほう|まい}に
\ruby{繋}{つな}がれても、
%
\ruby[|g|]{職務}{つとめ}と
\ruby{思}{おも}へば
\ruby{其職}{そ|れ}を
\ruby{疎畧}{おろ|そか}にせん% 131-3-40-其四十.tex では 疎略(おろ|そか) とある
\ruby{心}{こゝろ}は
\原本頁{25-7}\改行%
\ruby{無}{な}くて、
%
\ruby[|g|]{身體}{からだ}の
\ruby{疲}{つか}れ
きつたる
\ruby{時}{とき}にも、
%
\ruby{氣合}{き|あひ}の
\ruby{如何}{い|か}にしても
\ruby{{\換字{進}}}{すゝ}まざる
\ruby{折}{をり}にも、
%
\ruby{{\換字{強}}}{し}ひて
\ruby{勉}{つと}めて
\ruby{果}{はた}すべき
だけの
\ruby{事}{こと}を
\ruby{果}{はた}したる、
%
\ruby{其}{そ}の
\原本頁{25-9}\改行%
\ruby{苦}{くる}しさを
\ruby{今}{いま}は
\ruby{免}{まぬか}れて、
%
\ruby{起}{お}きるも
\ruby{睡}{ね}るも
\ruby{心}{こゝろ}の
まゝの、
%
\ruby{肩}{かた}に
\ruby{荷}{に}は
\ruby{無}{な}き
\ruby[||j>]{境}{きよう}
\ruby[||j>]{界}{ がい}
% \ruby{境界}{きよう|がい}
となりたるを、
%
まだ
\ruby{知}{し}らねば
お
\ruby{濱}{はま}の
\ruby{怪}{あやし}むも
\ruby{無理}{む|り}ならずと
\ruby{微笑}{ほゝ|ゑ}まれ、

\原本頁{26-1}%
『
ハヽヽ、
%
\ruby{何}{なんに}も
\ruby[|g|]{可笑}{をかし}い
ことも
\ruby{何}{なに}も
\ruby{有}{あ}りは
\ruby{仕}{し}ないよ。
%
\ruby{今日}{け|ふ}は
もう
%\原本頁{26-2}\改行%
\ruby{學校}{がく|かう}へも
\ruby{何}{なに}へも
\ruby{出}{で}やあ
\ruby{仕}{し}ない
のだもの、
%
いくら
\ruby{沈着}{おち|つ}いて
\ruby{居}{ゐ}ても
\ruby[|g|]{可笑}{をかし}い
\ruby{事}{こと}あ
\ruby{有}{あ}りやあ
\ruby{仕}{し}ない。
』

\原本頁{26-4}%
と
\ruby{輕}{かろ}く
\ruby{答}{こた}へたり。

\原本頁{26-5}%
『
ぢやあ
\ruby{今日}{け|ふ}は
\ruby{怠}{なま}けて
\ruby{御休}{お|やす}み
なさるの?。
%
\ruby{{\換字{嫌}}}{いや}な
\ruby{先生}{せん|せい}ネエ!、
%
\ruby{何故}{な|ぜ}
\ruby{御休}{お|やす}み
なさるの?。
』% 原本では「』閉じカッコ」は植字されていない

\原本頁{26-7}%
『
なあに
\ruby{怠}{なま}けて
\ruby{休}{やす}む
\ruby{譯}{わけ}ぢやあ
\ruby{無}{な}いが、
%
\ruby{今日}{け|ふ}ツ
からは
\ruby{私}{わたし}にやあ
\ruby{毎日}{まい|にち}
\ruby{日曜}{にち|えう}なのだ。
%
だから
もう
\ruby{先生}{せん|せい}〳〵つて
\ruby{云}{い}ふのも
\ruby{止}{よ}して
\ruby{貰}{もら}は
なくつちやあ。
%
\ruby{仕方}{し|かた}が
\ruby{無}{な}いから
\ruby{今}{いま}までは
\ruby{呼}{よ}ばれて
\ruby{居}{ゐ}た
けれども、
%
\原本頁{26-10}\改行%
\ruby{先生}{せん|せい}〳〵つて
\ruby{云}{い}はれるなあ、
%
\ruby{先}{せん}から
\ruby{私}{わたし}あ
\ruby{好}{す}きぢやあ
\ruby{無}{な}かつた
のだからネ。
』

\原本頁{27-1}%
『
あら、
%
それ
ぢやあ
\ruby{學校}{がく|かう}を
もう
\ruby{御止}{お|よ}し
なすつたの?。
』

\原本頁{27-2}%
『
あゝ。
%
\ruby{高田}{たか|た}さんが
\ruby{止}{よ}したら
\ruby{宜}{よ}からう
といふ
から
\ruby{止}{よ}して
\ruby{仕舞}{し|ま}ふ
ことにした。
』

\原本頁{27-4}%
『
\ruby{何故}{な|ぜ}
\ruby{高田}{たか|た}さんが
\ruby{其樣}{そ|ん}なことを
\ruby{云}{い}ひ
\ruby{出}{だ}したの。
%
\ruby{憎}{にく}らしい
\ruby{高田}{たか|た}さん
だことネエ、
%
\ruby{何故}{な|ぜ}
\ruby{先生}{せん|せい}に
\ruby{御止}{お|よ}しなさい
つて
\ruby{云}{い}つたの?。
』

\原本頁{27-6}%
\ruby{問}{と}はれては
\ruby[|g|]{流石}{さすが}に
\ruby{勇}{いさ}んで
\ruby{答}{こた}ふ
べき
ならねば、
%
\ruby{水野}{みづ|の}は
\ruby{唯}{たゞ}
\ruby{默然}{もく|ぜん}として
\ruby{笑}{わら}つて
\ruby{語}{かた}らず。

\原本頁{27-8}%
『
\ruby{昨夜}{ゆふ|べ}
\ruby{高田}{たか|た}さん
ところへ
\ruby{入}{い}らしつたのは
\ruby{其}{そ}の
\ruby{事}{こと}でしたか。
』

\原本頁{27-9}%
『
あゝ、
』

\原本頁{27-10}%
『
ほんとに
\ruby{可厭}{い|や}な
\ruby{高田}{たか|た}さん
だこと!。
%
\ruby{可}{い}いは、
%
\ruby[||j>]{祖}{おぢい}
\ruby[||j>]{{\換字{父}}}{ さん}に
% \ruby{祖{\換字{父}}}{おぢい|さん}に
\ruby{左樣}{さ|う}いつて
\ruby{叱}{しか}らせて
\ruby{{\換字{遣}}}{や}るは。
%
\ruby{左樣}{さ|う}して
\ruby{復}{また}
\ruby{先生}{せん|せい}を
\ruby{舊}{もと}の
\ruby{{\換字{通}}}{とほ}り
にする
やうに
\ruby{爲}{さ}せるは。
』

\原本頁{28-2}%
『
ハヽヽ。
%
\ruby{折角}{せつ|かく}
\ruby{丁度}{ちやう|ど}
\ruby{止}{よ}して
\ruby{仕舞}{し|ま}つた
ものを、
%
\ruby{其樣}{そ|ん}な
\ruby{世話}{せ|わ}を
\ruby{燒}{や}かれちやあ
\ruby{却}{かへ}つて
\ruby{困}{こま}るよ。
%
\ruby{打棄}{うつ|ちや}つて
\ruby{置}{お}いて
\ruby{吳}{く}れ
\ruby{無}{な}くちやあ。
』

\原本頁{28-4}%
『
だつて、
%
それ
ぢやあ
\ruby{先生}{せん|せい}は、
%
\ruby{何處}{ど|こ}か
\ruby{他{\換字{所}}}{よ|そ}へ
\ruby{行}{い}つて
\ruby{御仕舞}{お|し|ま}ひ
なさるん
でしやう。
%
\ruby{此樣}{こ|ん}な
\ruby{詰}{つま}らない
\ruby{村}{むら}にやあ
\ruby{居}{ゐ}て
\ruby{下}{くだ}さらない
でしやう。
%
\ruby{屹度}{きつ|と}% 原本通り非グループルビ
\ruby{妾}{わたし}の
\ruby{家}{うち}を
\ruby{出}{で}て
\ruby{行}{い}つて
お
\ruby{仕舞}{し|ま}ひ
なさるん
でしやう。
』

\原本頁{28-7}%
と
\ruby{云}{い}ひさして
\ruby{水野}{みづ|の}の
\ruby{面}{おもて}を
\ruby{凝然}{じ|つ}と
\ruby{見居}{み|ゐ}たりしが、

\原本頁{28-8}%
『
\ruby{{\換字{嫌}}}{いや}だは、
%
\ruby{{\換字{嫌}}}{いや}だは、
%
\ruby[<j||]{妾}{わたし}
\ruby{{\換字{嫌}}}{いや}だは!。
%
\ruby[||j>]{祖}{おぢい}
\ruby[||j>]{{\換字{父}}}{ さん}に
% \ruby{祖{\換字{父}}}{おぢい|さん}に
\ruby{左樣}{さ|う}
\ruby{云}{い}つて
\ruby{高田}{たか|た}さんを
\ruby{叱}{しか}らせるから
\ruby{宜}{い}いは。
』

\原本頁{28-10}%
と
\ruby{眼}{め}に
\ruby{露}{つゆ}
\ruby{持}{も}つて
\ruby{腹立}{はら|だた}しげに% 原本では非踊り字表現
\ruby{悶}{もだ}えたり。

\原本頁{28-11}%
『
ハヽヽ。
%
\ruby[||j>]{祖}{おぢい}
\ruby[||j>]{{\換字{父}}}{ さん}が
% \ruby{祖{\換字{父}}}{おぢい|さん}が
\ruby{何程}{いく|ら}% 原本通り非グループルビ
\ruby{幅利}{はゞ|きゝ}でも、
%
\ruby{高田}{たか|た}さんは
\ruby{高田}{たか|た}さん
だから、
%
%\原本頁{29-1}\改行%
\ruby{左樣}{さ|う}
\ruby{自由}{じ|ゆう}の
\ruby{利}{き}く
わけの
もの
でも
\ruby{無}{な}い。
%
また
\ruby{私}{わたし}は
\ruby{今}{いま}
\ruby{何處}{ど|こ}へ
\ruby{行}{ゆ}く
といふ
ことも
\ruby{有}{あ}りや
\ruby{仕無}{し|な}いから、
%
\ruby{矢張}{やつ|ぱり}
いつまでも
\ruby{此村}{こ|ゝ}に
\ruby{居}{ゐ}る
つもり
だよ。
』

\原本頁{29-4}%
『
\ruby[|g|]{眞實}{ほんと}?、
%
\ruby{眞實}{ほん|たう}?、
%
\ruby{矢張}{やつ|ぱり}
いつまでも
\ruby{此家}{う|ち}に
\ruby{居}{ゐ}らつしやるの?。
』

\原本頁{29-5}%
『
あゝ。
%
\ruby{別}{べつ}に
\ruby{何處}{ど|こ}へ
\ruby{行}{ゆ}かうと
\ruby{云}{い}ふ
\ruby{料簡}{れう|けん}も
\ruby{無}{な}いから。
』

\原本頁{29-6}%
『
\ruby{嬉}{うれ}しい!。
%
それ
ぢやあ
\ruby{學校}{がく|かう}へも
\ruby{出}{で}ないで
\ruby{始{\換字{終}}}{し|じゆう}
\ruby{此家}{う|ち}に
\ruby{居}{ゐ}らつしやる。
%
あゝ、
%
そんなら
\ruby{學校}{がく|かう}
なんか
\ruby{先生}{せん|せい}が
\ruby{止}{よ}し
\ruby{仕舞}{ち|ま}つた
\ruby{方}{はう}が
\ruby{宜}{い}いは。
%
\ruby[|g|]{澤山}{たんと}
\ruby{先生}{せん|せい}が
\ruby{此家}{う|ち}に
\ruby{居}{ゐ}らつしやる
のだから。
%
\ruby{今後}{これ|から}
また
\ruby{先}{せん}
のやうに
\ruby{種々}{いろ|〳〵}の
\ruby{面白}{おも|しろ}い
\ruby{御話}{お|はなし}を
\ruby{仕}{し}て
\ruby{頂}{いたゞ}けるはネ。
』

\原本頁{29-10}%
\ruby{人}{ひと}の
\ruby{胸}{むね}の
\ruby{中}{うち}は
\ruby{{\換字{更}}}{さら}に
\ruby{知}{し}らず、
%
\ruby{{\換字{飽}}}{あく}まで
\ruby{我儘}{わが|まゝ}なる
\ruby[|g|]{處女}{をとめ}
\ruby{氣}{ぎ}の
\ruby[|g|]{長閑}{のどか}さに、
%
\ruby{水野}{みづ|の}は
\ruby{笑}{わら}つて
\ruby{點頭}{うな|づ}かざるを
\ruby{得}{え}ざりき。

\原本頁{30-1}%
『
これで
もう
\ruby{淺草}{あさ|くさ}へも
\ruby{行}{い}らつしやらないと、
%
\ruby[|g|]{眞實}{ほんと}に
\ruby{好}{い}い
のだけれども。
』

\原本頁{2}%
\ruby{{\換字{猶}}}{なほ}
\ruby{不足氣}{ふ|そく|げ}に
\ruby{如是}{か|く}
\ruby{云}{い}ひて
\ruby{嫣然}{につ|こり}と
\ruby{笑}{ゑ}める
\ruby{面}{おもて}つき、
%
また
\ruby{無}{な}く
\ruby{美}{うるは}し。
