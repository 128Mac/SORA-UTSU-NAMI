\Entry{其六}

% メモ 校正終了 2024-05-10 2024-06-06
\原本頁{30-5}%
\ruby{罪}{つみ}も
\ruby{無}{な}く
\ruby{念}{おもひ}も
\ruby{無}{な}き
お
\ruby{濱}{はま}の
\ruby{願}{ねがひ}の
\ruby{是}{かく}の
\ruby{如}{ごと}きには
\ruby{引}{ひき}かへて、
%
\ruby{水野}{みづ|の}が
\ruby{今{\換字{朝}}}{け|さ}
\ruby{差當}{さし|あた}つて
\ruby{先}{ま}づ
\ruby{思}{おも}へるは、
%
\ruby{淺草}{あさ|くさ}の
\ruby{御堂}{み|だう}に
\ruby{詣}{まゐ}りて
\ruby[||j>]{心}{こゝろ}
\ruby[||j>]{靜}{ しづ}かに
% \ruby{心靜}{こゝろ|しづ}かに
\ruby{報恩}{はう|おん}
\ruby{謝徳}{しや|とく}の
\ruby[|g|]{誠意}{まこと}を
\ruby{{\換字{運}}}{はこ}び、
%
かつは
\ruby{{\換字{猶}}}{なほ}
\ruby{行末}{ゆく|すゑ}を
\ruby{頼}{たの}み
\ruby[<j>]{奉}{たてまつ}らん
との
\ruby{事}{こと}
のみなり
しなり。

\原本頁{30-9}%
されど
\ruby{淺草}{あさ|くさ}に
\ruby{詣}{まゐ}らんと
\ruby{思}{おも}ふ
\ruby{意}{こゝろ}の
\ruby{側}{わき}には、
%
\ruby{{\換字{強}}}{し}ひて
\ruby{求}{もと}むる
といふ
ほどには
あらねど、
%
\ruby{{\換字{若}}}{も}し
\ruby{機會}{を|り}よく
\ruby{我}{わ}が
\ruby{御堂}{み|だう}に
\ruby{詣}{まゐ}りて
より
\ruby{歸}{かへ}る
までの
\ruby{間}{あひだ}に、
%
\ruby{彼}{か}の
\ruby[||j>]{同}{おも }
\ruby[||j>]{{\換字{情}}}{ひやり}
% \ruby{同{\換字{情}}}{おも|ひやり}
\ruby[||j>]{深}{ ふか}く
\ruby{信心}{しん|〴〵}
\ruby{深}{ふか}き
\ruby{優}{やさ}しく
\ruby{懷}{なつか}しき
\ruby{不幸福}{ふ|しあ|はせ}の%「幸福」ここは「は」
\ruby{人}{ひと}に
\ruby{相}{あひ}
\ruby{逢}{あ}ふ
ことを
\ruby{得}{え}ば、
%
\ruby{願}{ねが}はくば
\ruby{相}{あひ}
\ruby{逢}{あ}ひて
\ruby{一}{ひ}ト
\ruby{言}{こと}
\ruby{二}{ふ}タ
\ruby{言}{こと}の
\ruby{言葉}{こと|ば}
をも
\ruby{{\換字{交}}}{まじ}へたき
やうの
\ruby{念}{おもひ}も
\ruby{潜}{ひそ}める% 【潛 u6f5b 「先先」】【潜 u6f5c 「夫夫」】併用されている
なり。
%
\ruby[|g|]{昨日}{きのふ}の
\ruby[|g|]{談話}{はなし}にて、
%
\ruby{其}{その}
\ruby{人}{ひと}の
\ruby{詣}{まうづ}るは、
%
\原本頁{31-4}\改行%
\ruby{毎日}{まい|にち}
\ruby{大抵}{たい|てい}
\ruby{午{\換字{前}}}{ひる|まへ}の
\ruby{事}{こと}にして、
\ruby{午後}{ひる|すぎ}に
\ruby{詣}{まう}でしは
\ruby{昨日}{きの|ふ}のみ% ルビ調整(原本通り)
なりと
\ruby{知}{し}りたれば、
%
\ruby[|g|]{職務}{つとめ}に
\ruby{縛}{しば}らるゝ
\ruby{身}{み}の
\ruby{午{\換字{前}}}{ひる|まへ}は
\ruby{我}{わ}が
\ruby{自由}{ま|ま}ならで
\ruby{其}{その}
\ruby{人}{ひと}に
\ruby{再}{ふたゝ}び
\原本頁{31-6}\改行%
\ruby{行}{ゆき}
\ruby{逢}{あ}ふことも
\ruby{無}{な}かるべきを
\ruby[||j>]{{\換字{遺}}}{のこり}
\ruby[||j>]{憾}{ をし}く
% \ruby{{\換字{遺}}憾}{のこり|をし}く
\ruby{思}{おも}ひ
\ruby{居}{ゐ}たるが、
%
\ruby[|g|]{昨日}{きのふ}に
\ruby{今日}{け|ふ}は
\ruby{變}{かは}れる
\ruby{我}{わ}が
\ruby{上}{うへ}の、
%
\ruby{今}{いま}は
\ruby{何時}{い|つ}
\ruby{參}{まゐ}らんも
\ruby{心}{こゝろ}の
\ruby{自由}{ま|ま}なるまゝ、
%
\ruby{先}{ま}づ
\ruby{彼}{か}の
\ruby{人}{ひと}の
\ruby{詣}{まゐ}るといふ
\ruby{午{\換字{前}}}{ひる|まへ}に
\ruby{詣}{まゐ}りて、
%
\ruby[<j>]{幸}{さいはひ}にして
\ruby{{\換字{若}}}{も}し
\ruby{相}{あひ}
\ruby{見}{まみ}ゆる
ことを
\ruby{得}{え}たらんには、
%
\ruby{我}{わ}が
\ruby{五十子}{い|そ|こ}の
\ruby[|g|]{病氣}{やまひ}の
\ruby{本復}{ほん|ぷく}
\ruby{疑}{うたが}ひ
\ruby{無}{な}きに
\ruby{至}{いた}りたる
\ruby{事}{こと}をも
\ruby{告}{つ}げて、
%
\ruby{御佛}{み|ほとけ}の
\ruby{加護}{か|ご}を
\ruby{悅}{よろこ}び、
%
\ruby{彼}{か}の
\ruby{人}{ひと}の
\ruby{親切}{しん|せつ}を
\ruby{謝}{しや}し
も
せんとの
\ruby{念}{おもひ}の
\ruby{潜}{ひそ}める% 【潛 u6f5b 「先先」】【潜 u6f5c 「夫夫」】併用されている
なり。
%
\ruby{優}{やさ}しく
\ruby{懷}{なつか}しき
\ruby{彼}{か}の
\ruby{人}{ひと}に、
%
\ruby{我}{わ}が
\ruby{五十子}{い|そ|こ}の
\ruby[<j||]{甚}{いと}% 行末行頭の境界付近なので特例処置を施す
\ruby[<j||]{危}{あやふ}
% \ruby{甚危}{いと|あやふ}
きところを
\ruby{免}{のが}れて、
%
\ruby{復}{ふたゝ}び
\ruby[|g|]{現世}{このよ}の
\ruby{日}{ひ}に
\ruby{照}{て}らさるゝに
\ruby{至}{いた}りし
ことを
\改行% 校正作業の簡略化のため
、
%
\原本頁{32-2}\改行%
\ruby{人}{ひと}を
\ruby{吸}{す}ひ
\ruby{入}{い}るゝが
\ruby{如}{ごと}き
\ruby{其}{そ}の
\ruby{愛}{あい}
\ruby{深}{ふか}き
\ruby{笑顏}{ゑ|がほ}に
\ruby{悅}{よろこ}び
\ruby{欣}{よろこ}びて
\ruby{貰}{もら}ひたき
\ruby[<j||]{念}{おもひ}の% 行末行頭の境界付近なので特例処置を施す
\ruby{潜}{ひそ}めるなり。% 【潛 u6f5b 「先先」】【潜 u6f5c 「夫夫」】併用されている

\原本頁{32-4}%
\ruby{水野}{みづ|の}は
お
\ruby{濱}{はま}が
\ruby{假初}{かり|そめ}の
\ruby{語}{ことば}には
\ruby{耳}{みゝ}を
\ruby{假}{か}すことも
\ruby{無}{な}く、
%
やがて
\ruby{淺草}{あさ|くさ}
さして
\ruby{立出}{たち|い}でたり。

\原本頁{32-6}%
\ruby{幾度}{いく|たび}か
\ruby{往來}{ゆき|き}し% ルビ調整(原本通り)非グループルビ
\ruby{馴}{な}れたる
\ruby{路}{みち}の、
%
\ruby{眼}{め}に
\ruby{{\換字{古}}}{ふ}りたる
\ruby{景色}{け|しき}は
\ruby{心}{こゝろ}の
\ruby{{\換字{留}}}{と}まる
\ruby{方}{かた}も
\ruby{無}{な}くて、
%
\ruby{早}{はや}くも
\ruby{御堂}{み|だう}に
\ruby{到}{いた}り
\ruby{着}{つ}きたり。
%
\ruby{先}{ま}づ
\ruby{常例}{つ|ね}の
\ruby{如}{ごと}く
\ruby{祈念}{き|ねん}を
\原本頁{32-8}\改行%
\ruby{寵}{こ}めて、
%
\ruby[|g|]{少時}{しばし}は
\ruby{何事}{なに|ごと}をも
\ruby{思}{おも}はざりしが、
%
\ruby{念}{ねん}じ
\ruby{{\換字{終}}}{をは}りて
\ruby{閉}{と}ぢたる
\ruby{眼}{め}を
\ruby{開}{ひら}き、
%
\ruby{下}{さ}げたる
\ruby{頭}{かうべ}を
\ruby{擡}{あ}げ、
%
\ruby{身}{み}を
\ruby{起}{おこ}して
\ruby{我}{わ}が
\ruby{居}{ゐ}たる
\ruby[|g|]{四邊}{あたり}を
\ruby{見}{み}れば、
%
\ruby{夢}{ゆめ}の
\ruby{裏}{うち}に
\ruby{現}{あらは}れ
\ruby{來}{きた}る
\ruby{人}{ひと}の
\ruby{跫音}{あし|おと}も
\ruby{無}{な}く
\ruby{衣}{きぬ}の
\ruby{音}{おと}も
せずして
\ruby[|g|]{俄然}{にはか}に
\原本頁{32-11}\改行%
\ruby{我}{わ}が
\ruby{{\換字{前}}}{まへ}に
\ruby{湧}{わ}き
\ruby{出}{い}づるが
\ruby{如}{ごと}くに、
%
\ruby{何時}{い|つ}か
\ruby{知}{し}らず、
%
\ruby{我}{わ}が
\ruby[||j>]{傍}{かたへ}に
\ruby[<->]{跪}{ひざまづ}きて% 「|-|」「|->」 でも送り仮名が行替えしてしまう
\原本頁{33-1}\改行%
\ruby{御佛}{み|ほとけ}を
\ruby{念}{ねん}ぜる
\ruby{人}{ひと}
あり
たり。
%
\ruby{其}{そ}の
\ruby{柔}{やはら}かに
\ruby{合}{あは}せたる
\ruby{掌}{て}の
\ruby{白々}{しろ|〴〵}と
\ruby[<j||]{殊}{しゆ }% 行末行頭の境界付近なので特例処置を施す
\ruby[<j||]{{\換字{勝}}}{しよう}
% \ruby{殊{\換字{勝}}}{しゆ|しよう}
\ruby{氣}{げ}なる、
%
\ruby{其}{そ}の
\ruby{領}{えり}の
すらり
として
\ruby{見}{み}
\ruby{好}{よ}き、
%
\ruby{其}{そ}の
\ruby{髮}{かみ}の
めでたき、
%
\ruby{其}{そ}の
\ruby{肩}{かた}つきの
\ruby{如何}{い|か}にも
\ruby{女}{をんな}らしく
\ruby{優}{やさ}しき、
%
\ruby{其}{そ}の
\ruby{横顏}{よこ|がほ}の
\ruby{能}{よ}くは
\ruby{見}{み}えぬ
ながら
\ruby[||j>]{櫻}{さくら}
\ruby[||j>]{色}{ いろ}に
% \ruby{櫻色}{さくら|いろ}に
\ruby{美}{うる}はしきは、
%
\ruby{嗚呼}{あ|ゝ}
\ruby{我}{わ}が
\ruby{相}{あひ}
\ruby{見}{み}んと
\ruby{希}{ねが}ひたりし
\ruby{其}{そ}の
\ruby{人}{ひと}に
あらずや。
%
\ruby{正}{まさ}しく
\ruby[|g|]{昨日}{きのふ}は
\ruby{見}{み}、
%
\ruby{今{\換字{朝}}}{け|さ}は
\ruby{思}{おも}ひたりし
\ruby{其}{そ}の
お
\ruby{龍}{りう}
ならずや。
%
\ruby{御佛}{み|ほとけ}を
\ruby{念}{ねん}ぜし
\ruby{今}{いま}
\ruby[|g|]{少時}{しばし}の
\ruby{間}{あひだ}のみ
\ruby{忘}{わす}れ
\ruby{居}{ゐ}たりし
\ruby{其}{そ}の
\ruby{優}{やさ}しく
\ruby[<j||]{懷}{なつか}しき% 行末行頭の境界付近なので特例処置を施す
\ruby{親切}{しん|せつ}の
\ruby{人}{ひと}
ならずや。
%
\ruby{我}{わ}が
\ruby{涙}{なみだ}を
\ruby{濺}{そゝ}ぎて
\ruby{聞}{き}きし
\ruby{不幸福}{ふ|しあ|はせ}の%「幸福」ここは「は」
\ruby[||j>]{物}{もの}
\ruby[||j>]{語}{がたり}を
% \ruby{物語}{もの|がたり}を
\原本頁{33-8}\改行%
\ruby{有}{いう}せる
\ruby{悲}{かな}しき
\ruby{薄命}{はく|めい}の
\ruby{{\換字{婦}}人}{ふ|じん}% ルビ調整(原本通り)非グループルビ
ならずや。
%
\ruby{何}{なん}ぞ
\ruby{其}{そ}の
\ruby{掌}{て}を
\ruby{合}{あは}せて
\ruby{念}{ねん}ぜる
さまの
\ruby{哀}{あは}れ
\ruby{深}{ふか}くして、
%
\ruby{首}{かうべ}を
\ruby{垂}{た}れて
\ruby{思}{おもひ}を
\ruby{凝}{こ}らせる
さまの
\ruby{人}{ひと}の
\ruby{心}{こゝろ}を
\原本頁{33-10}\改行%
\ruby{動}{うご}かすや。
%
\ruby{不思議}{ふ|し|ぎ}にも
\ruby{何時}{い|つ}の
\ruby{間}{ま}にか
\ruby{此堂}{こ|ゝ}には
\ruby{參}{まゐ}り
\ruby{合}{あは}せたる!\inhibitglue{}%
と
\原本頁{33-11}\改行%
\ruby{思}{おも}ふ
\ruby{時}{とき}
\ruby{漸}{やうや}くに
\ruby{念}{ねん}じ
\ruby{{\換字{終}}}{をは}りてか、
%
\ruby{身}{み}じろぎして
\ruby{靜}{しづか}に
\ruby{女}{をんな}は
\ruby{立上}{たち|あが}りたり
\改行% 校正作業の簡略化のため
。

\原本頁{34-1}%
『‥‥‥‥』

\原本頁{34-2}%
『‥‥‥‥』

\原本頁{34-3}%
\ruby{聲}{こゑ}
\ruby{無}{な}くして
\ruby{其處}{そ|こ}に
\ruby{呼}{よ}ぶ
\ruby{聲}{こゑ}ありたり、
%
\ruby{應}{こた}ふる
\ruby{聲}{こゑ}ありたり、
%
\ruby{言無}{ことば|な}くして
\ruby{其處}{そ|こ}に
\ruby{語}{かた}れる
\ruby{言}{ことば}ありたり、
%
\ruby{酬}{むく}ひたる
\ruby{言}{ことば}ありたり。
