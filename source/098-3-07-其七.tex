\Entry{其七}

% メモ 校正終了 2024-05-10 2024-06-06
\原本頁{34-6}%
『
\ruby[|g|]{昨日}{きのふ}は
いろ〳〵
\ruby{御厄介}{ご|やく|かい}に、
』

\原本頁{34-7}%
『
いゝえ、
%
\ruby{却}{かへ}つて
\ruby{御{\換字{迷}}惑}{ご|めい|わく}
でございましたらう。
%
おとうさんが
\ruby[g]{彼樣}{あ ん }な
\ruby[g]{氣合}{き あひ}の
\ruby{人}{ひと}
だもん
ですから、
%
\ruby{御{\換字{遠}}慮}{ご|えん|りよ}の
\ruby{無}{な}い
こと
ばかり
\ruby{致}{いた}す
やうに
なりまして。
%
\ruby{定}{さだ}めし
\ruby{御蔑視}{お|さげ|すみ}
なすつた
\ruby{事}{こと}
だらうと、
%
\ruby{後}{あと}になつて
\ruby[|g|]{二人}{ふたり}で
\ruby[g]{左樣}{さ う }
\ruby{申}{まを}して
\ruby{居}{を}りました。
』

\原本頁{35-1}%
『
イヤ、
%
どうして
\ruby[g]{其樣}{そ ん }なことを
\ruby{思}{おも}ふ
もの
ですか。
%
たゞ
\ruby{私}{わたし}は
\ruby{何}{なん}の
% \原本頁{35-2}\改行%
\ruby[|g|]{因緣}{いはれ}も
\ruby{無}{な}い
\ruby{方}{かた}に
お
\ruby[g]{世話}{せ わ }を
かけたのが
\ruby{濟}{す}まぬ
\ruby{樣}{やう}な
\ruby{氣}{き}が
\ruby{仕}{し}ます。
%
お
\ruby{會}{あ}ひ
なすつたら
\ruby{彼}{あ}の
\ruby{方}{かた}に
\ruby{宜}{よろ}しく
\ruby{仰}{おつし}あつて
\ruby{下}{くだ}さいまし。
』

\原本頁{35-4}%
『
ホヽ
\ruby[g]{大層}{たいそう}
\ruby{折目高}{をり|め|だか}に
\ruby{物}{もの}を
\ruby{仰}{おつし}あること。
%
\ruby{彼}{あ}の
\ruby{人}{ひと}は
\ruby[g]{彼樣}{あ ゝ }した
\ruby{人}{ひと}なの
% \原本頁{35-5}\改行%
ですもの、
%
\ruby[g]{御氣}{お き }に
お
\ruby{掛}{か}け
なさる
\ruby{事}{こと}は
ありやあ
\ruby{仕}{し}ません。
%
それは
\原本頁{35-6}\改行%
まあ
\ruby[g]{何樣}{ど う }でも
\ruby{宜}{い}いと
しまして、
%
\ruby[g]{今日}{け ふ }は
\ruby{何}{なん}でも
\ruby{無}{な}い
\ruby{日}{ひ}で
ございます
のに、
%
どうして
\ruby{今}{いま}
\ruby{頃}{ごろ}
\ruby{御}{お}いでに
なりましたの?。
%
\ruby[|g|]{貴下}{あなた}の
\ruby{拜}{をが}んで
\原本頁{35-7}\改行%
\ruby{居}{ゐ}らしつた
\ruby{御}{お}
\ruby[<j||]{後}{うしろ}
\ruby[||j>]{姿}{すがた}を
\ruby{見}{み}まして、
%
\ruby{妾}{わたし}は
\ruby{初}{はじめ}は
\ruby{氣}{き}の
\ruby{{\換字{迷}}}{まよ}ひ
かと
\ruby{思}{おも}ひ
ましたよ。
%
だつて
\ruby[|g|]{貴下}{あなた}が
\ruby{今}{いま}
\ruby{頃}{ごろ}
\ruby{御}{お}いで
なさらう
\ruby{譯}{わけ}は
\ruby{無}{な}いと
\ruby{思}{おも}ひ
\ruby{定}{き}つて
\原本頁{35-10}\改行%
\ruby{居}{ゐ}た
のです
もの。
』

\原本頁{35-11}%
『
ハヽヽ、
%
\ruby{私}{わたし}は
また
\ruby[g]{何時}{い つ }の
\ruby{間}{ま}にか
\ruby{私}{わたし}の
\ruby{傍}{そば}に
\ruby[|g|]{貴孃}{あなた}の
\ruby{來}{き}て
\ruby{居}{ゐ}られたのに
\ruby[g]{吃驚}{びつくり}
しました。
』

\原本頁{36-2}%
『
ホヽヽ、
%
\ruby[|g|]{貴下}{あなた}が
\ruby[g]{一心}{いつしん}に
なつて
\ruby{拜}{をが}んで
\ruby{居}{ゐ}らしつたから、
%
\ruby[g]{吃驚}{びつくり}
なさらない
やうにと
\ruby{思}{おも}つて
\ruby{悄々地}{そー|つ|と}
\ruby{妾}{わたし}も
\ruby{拜}{をが}んで
\ruby{居}{を}りましたのよ。
』

\原本頁{36-4}%
『
それは
\ruby{兎}{と}も
\ruby{角}{かく}も、
%
\ruby[g]{今日}{け ふ }
\ruby{{\換字{若}}}{も}し
\ruby[|g|]{貴孃}{あなた}に
\ruby[g]{御目}{お め }に
かゝれたら、
%
\ruby{先}{ま}づ
\ruby[g]{第一}{だいいち}に
\ruby[g]{御話}{おはなし}を
して、
%
\ruby{悅}{よろこ}んで
\ruby{戴}{いたゞ}きたいと
\ruby{思}{おも}つて
\ruby{居}{を}りましたが、
%
\ruby{御蔭樣}{お|かげ|さま}で
\ruby[||j>]{病}{びやう}
\ruby[||j>]{人}{ にん}も
% \ruby{病人}{びやう|にん}も
\ruby[g]{何樣}{ど う }やら
\ruby[g]{持直}{もちなほ}して、
%
\ruby[g]{醫者}{い しや}が
\ruby[g]{屹度}{きつと }% ルビ調整(原本通り)非グループルビ
\ruby[g]{本復}{ほんぷく}すると
\ruby[g]{保證}{うけあ }つて
\原本頁{36-7}\改行%
\ruby{吳}{く}れた
やうな
ところ
\ruby{迄}{まで}には
\ruby{漕}{こ}ぎ
つけました。
%
もう
\ruby[g]{心配}{しんぱい}は
\ruby{無}{な}さゝう
になりました。
%
\ruby[g]{御案}{お あん}じ
\ruby{下}{くだ}すつた
\ruby[g]{甲{\換字{斐}}}{か ひ }も
あつて、
%
\ruby{御親切}{ご|しん|せつ}も
まあ
\ruby{屆}{とゞ}いたと% 「屆」「届」 原本通り「屆」
\ruby{申}{まを}す
もので
ございます。
%
ほんとに
\ruby[||j>]{病}{びやう}
\ruby[||j>]{人}{ にん}とは
% \ruby{病人}{びやう|にん}とは
\ruby[g]{御緣}{ご えん}も
\ruby{薄}{うす}い
\原本頁{36-10}\改行%
\ruby[|g|]{貴卿}{あなた}が、
%
かうして
\ruby[g]{毎日}{まいにち}
\ruby[g]{々々}{ 〳〵 }
\ruby{歩}{あゆみ}を
\ruby{{\換字{運}}}{はこ}んで
\ruby{下}{くだ}すつて、
%
\ruby[g]{御願}{ごぐわん}を
\ruby[g]{御掛}{お か }け
\ruby{下}{くだ}さつた
\ruby{御芳{\換字{情}}}{お|こゝろ|もち}は
おろそかには
\ruby{思}{おも}ひません、
%
\ruby[||j>]{病}{びやう}
\ruby[||j>]{人}{ にん}が
% \ruby{病人}{びやう|にん}が
\ruby{快}{よ}く
なりましたに
つけても
\ruby{有}{あ}り
\ruby{{\換字{難}}}{がた}く
\ruby{思}{おも}ひます。
%
\ruby{今}{いま}と
いつて
\ruby{今}{いま}は
\ruby[g]{何樣}{ど う }
\ruby[g]{御禮}{お れい}の
\原本頁{37-2}\改行%
\ruby{爲}{し}やうも
\ruby{存}{ぞん}じませんが、
%
\ruby{何}{なん}ぞの
\ruby{折}{をり}には
\ruby[g]{屹度}{きつと }% ルビ調整(原本通り)非グループルビ
\ruby[|g|]{貴卿}{あなた}の
ために、
%
\ruby[|g|]{貴卿}{あなた}の
\ruby{優}{やさ}しい
\ruby{御芳{\換字{情}}}{お|こゝろ|もち}に
\ruby{對}{たい}して
\ruby[g]{其{\換字{丈}}}{それだけ}の
\ruby[|g|]{御{\換字{返}}禮}{おかへし}を
\ruby{爲}{し}やうとは
\ruby{思}{おも}つて
\ruby{居}{を}ります。
%
\ruby[|g|]{貴卿}{あなた}の
\ruby{御芳{\換字{情}}}{お|こゝろ|もち}は
\ruby{長}{なが}く
\ruby{忘}{わす}れません。
』

\原本頁{37-5}%
\ruby{此}{こ}の
\ruby{事}{こと}を
\ruby{言}{い}はん
とおもふ
\ruby{意}{こゝろ}の
\ruby{充}{み}ち
\ruby{滿}{み}ちたるに、
%
\ruby[g]{言葉}{ことば }も
\ruby{自}{おのづ}か
\ruby[<g>]{ら勢}{いきほひ}% 行末行頭の境界付近なので特例処置を施す
\ruby{籠}{こも}りて、
%
\ruby{口}{くち}
ばかりの
\ruby[g]{挨拶}{あいさつ}ならぬは
\ruby[g]{確乎}{しつかり}としたる
\ruby{眼}{め}つきにも
\ruby{著}{しる}し
\改行% 校正作業の簡略化のため
。
%
\原本頁{37-7}\改行%
お
\ruby{龍}{りう}は
\ruby{生眞面目}{き|ま|じ|め}に
\ruby[g]{如是}{か く }
\ruby{云}{い}はれては、
%
\ruby[g]{眞舳}{ま とも}には
\ruby{當}{あた}り
\ruby{得}{え}ざる
やうの
\ruby{氣}{き}も
\ruby{仕}{し}て、
%
\ruby{安}{やす}からぬ
\ruby[g]{心地}{こゝち }の
\ruby{竊}{ひそか}に
\ruby{爲}{す}ればにや、
%
たゞしは
\ruby{{\換字{又}}}{また}
\ruby{他}{ひと}
\ruby{知}{し}らぬ
\ruby[<j>]{考}{かんがへ}の
\ruby{別}{べつ}に
\ruby{有}{あ}ればにや、
%
\ruby{我}{わ}が
\ruby[g]{祈願}{きぐわん}の
\ruby[g]{甲{\換字{斐}}}{か ひ }の
\ruby{見}{み}えしを
\ruby{悅}{よろこ}ぶ
とも
\ruby{無}{な}く、
%
\ruby[g]{水野}{みづの }に
\ruby{斯}{か}ばかり
\ruby{禮}{れい}を
\ruby{云}{い}はれしを
\ruby{嬉}{うれ}しと
\ruby{思}{おも}ふとも
\ruby{見}{み}えず、
%
\ruby{却}{かへ}つて
\ruby{物}{もの}
\ruby{羞}{はぢ}したるが
\ruby{如}{ごと}く
\ruby{沈}{おち}
\ruby{着}{つ}かぬ
\ruby[g]{樣子}{やうす }に
なりて、
%
\ruby[g]{時々}{とき〴〵}は
\ruby{見}{み}でも
\ruby{宜}{よ}き
\ruby[|g|]{{\換字{遠}}方}{とほく}の
\ruby{額}{がく}などに
ちら〳〵と
\ruby{其}{そ}の
\ruby{美}{うつく}しき
\ruby{眼}{め}を
\ruby{辷}{すべ}らせて
\ruby{聞}{き}き
\ruby{居}{ゐ}し
\改行% 校正作業の簡略化のため
が、

\原本頁{38-3}%
『
まあ
\ruby[|g|]{眞實}{ほんと}に
そりやあ
\ruby{何}{なに}よりの
\ruby{事}{こと}で、
%
こんな
\ruby{嬉}{うれ}しい
ことは
もう
% \原本頁{38-4}\改行%
ございません。
%
どんなにか
\ruby[|g|]{貴下}{あなた}の
\ruby[g]{御嬉}{お うれ}しい
ことで
ございましやう!。
%
\ruby[|g|]{貴下}{あなた}の
\ruby[g]{御胸}{お むね}の
\ruby{中}{うち}を
\ruby{思}{おも}つて
\ruby{見}{み}ますと、
%
\ruby{妾}{わたし}も
\ruby{何}{なん}だか
\ruby{嬉}{うれ}し
\ruby{涙}{なみだ}が
\ruby{出}{で}さう
になります。
%
\ruby{何}{なに}も
\ruby{妾}{わたし}
なんぞが
\ruby[g]{御願}{お ねが}ひ
\ruby{申}{まを}した
から
といふ
\ruby{譯}{わけ}
では
ございます
まいが、
%
あれ
\ruby{程}{ほど}に
\ruby[g]{一心}{いつしん}に
なつて
\ruby[g]{御願}{お ねが}ひ
なすつた
\原本頁{38-8}\改行%
\ruby[|g|]{貴下}{あなた}の
\ruby{御念力}{ご|ねん|りき}
だけでも、
%
\ruby[||j>]{佛}{ほとけ}
\ruby[||j>]{樣}{ さま}が
% \ruby{佛樣}{ほとけ|さま}が
\ruby[g]{打棄}{うつちや}つては
\ruby[g]{御置}{お お }き
なされ
なくつて、
%
それで
\ruby{五十子}{い|そ|こ}さんが
\ruby{快}{よ}く
\ruby{御}{お}なり
なので
ございましやう。
%
ほんとに
\ruby{五十子}{い|そ|こ}さんは
\ruby{御}{お}
\ruby{羨}{うらや}ましい、
%
\ruby{御不幸}{お|ふし|あはせ}のやうで%「幸福」ここは「は」
\ruby{御幸福}{お|しあ|はせ}の%「幸福」ここは「は」
\ruby{方}{かた}です。
%
\ruby[g]{神樣}{かみさま}
\ruby[||j>]{佛}{ほとけ}
\ruby[||j>]{樣}{ さま}の
% \ruby{佛樣}{ほとけ|さま}の
\ruby{御憐愍}{お|あは|れみ}さへ
かゝつて
\ruby{居}{ゐ}る
\ruby{方}{かた}
ですもの!。
』

\原本頁{39-1}%
と
\ruby{末}{すゑ}は
\ruby{誰}{たれ}に
\ruby{云}{い}ふとも
\ruby{無}{な}く
\ruby{言}{い}ひ
たりしが、
%
はしたなしと
\ruby{思}{おも}ひてや
\改行% 校正作業の簡略化のため
、
%
\原本頁{39-2}\改行%
\ruby[g]{調子}{てうし }を
\ruby{變}{か}へて、

\原本頁{39-3}%
『
\ruby{歸}{かへ}りましたら
\ruby[g]{早{\換字{速}}}{さつそく}
\ruby[g]{師匠}{しゝやう}にも
\ruby[g]{左樣}{さ う }
\ruby{申}{まを}しまして、
%
\ruby{御丹精甲{\換字{斐}}}{ご|たん|せい|が|ひ}の
\ruby{有}{あ}つた
\ruby{事}{こと}を
\ruby{聽}{き}かせ
まして
\ruby{悅}{よろこ}ばせ
ましやう。
%
\ruby{定}{さだ}めし
\ruby[g]{屹度}{きつと }
\ruby{有}{あ}り
\ruby{{\換字{難}}}{がた}がる
\ruby{事}{こと}で
ございましやう。
』

\原本頁{39-6}%
と
\ruby{言}{ことば}を
\ruby{添}{そ}へたり。
