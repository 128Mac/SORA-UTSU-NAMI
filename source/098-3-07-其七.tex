\Entry{其七}

\原本頁{}
『
\ruby{昨日}{きの|ふ}はいろ〳〵
\ruby{御厄介}{ご|やく|かい}に、
』

\原本頁{}
『いゝえ、
%
\ruby{却}{かへ}つて
\ruby{御{\換字{迷}}惑}{ご|めい|わく}でございましらう。
%
おとうさんが
\ruby{彼樣}{あ|ん}な
\ruby{氣合}{き|あひ}の
\ruby{人}{ひと}だもんですから、
%
\ruby{御{\換字{遠}}慮}{ご|えん|りよ}の
\ruby{無}{な}いことばかり
\ruby{致}{いた}すやうになりまして。
%
\ruby{定}{さだ}めし
\ruby{御蔑視}{お|さげ|すみ}なすつた
\ruby{事}{こと}だらうと、
%
\ruby{後}{あと}になつて
\ruby{二人}{ふた|り}で
\ruby{左樣}{さ|う}
\ruby{申}{まを}して
\ruby{居}{を}りました。
』

\原本頁{}
『イヤ、
%
どうして
\ruby{其樣}{そ|ん}なことを
\ruby{思}{おも}ふものですか。
%
たゞ
\ruby{私}{わたくし}は
\ruby{何}{なん}の
\ruby{因緣}{い|はれ}も
\ruby{無}{な}い
\ruby{方}{かた}に
お
\ruby{世話}{せ|わ}をかけたのが
\ruby{濟}{す}まぬ
\ruby{樣}{やう}な
\ruby{氣}{き}が
\ruby{仕}{し}ます。
%
お
\ruby{會}{あ}ひなすつたら
\ruby{彼}{あ}の
\ruby{方}{かた}に
\ruby{宜}{よろ}しく
\ruby{仰}{おつし}あつて
\ruby{下}{くだ}さいまし。
』

\原本頁{}
『ホヽ
\ruby{大層}{たい|そう}
\ruby{折目高}{をり|め|だか}に
\ruby{物}{もの}を
\ruby{仰}{おつし}あること。
%
\ruby{彼}{あ}の
\ruby{人}{ひと}は
\ruby{彼樣}{あ|ゝ}した
\ruby{人}{ひと}なのですもの、
%
\ruby{御氣}{お|き}に
お
\ruby{掛}{か}けなさる
\ruby{事}{こと}はありやあ
\ruby{仕}{し}ません。
%
それはまあ
\ruby{何樣}{ど|う}でも
\ruby{宜}{い}いとしまして、
%
\ruby{今日}{け|ふ}は
\ruby{何}{なん}でも
\ruby{無}{な}い
\ruby{日}{ひ}でございますのに、
%
どうして
\ruby{今}{いま}
\ruby{頃}{ごろ}
\ruby{御}{お}いでになりましたの?。
%
\ruby{貴下}{あな|た}の
\ruby{拜}{をが}んで
\ruby{居}{ゐ}らしつ
\ruby{御後姿}{お|うしろ|すがた}を
\ruby{見}{み}まして、
%
\ruby{妾}{わたし}は
\ruby{初}{はじめ}は
\ruby{氣}{き}の
\ruby{{\換字{迷}}}{まよ}ひかと
\ruby{思}{おも}ひましたよ。
%
だつて
\ruby{貴下}{あな|た}が
\ruby{今}{いま}
\ruby{頃}{ごろ}
\ruby{御}{お}いでなさらう
\ruby{譯}{わけ}は
\ruby{無}{な}いと
\ruby{思}{おも}ひ
\ruby{定}{き}つて
\ruby{居}{ゐ}たのですもの。
』

\原本頁{}
『ハヽヽ、
%
\ruby{私}{わたし}はまた
\ruby{何時}{い|つ}の
\ruby{間}{ま}にか
\ruby{私}{わたし}の
\ruby{傍}{そば}に
\ruby{貴孃}{あな|た}の
\ruby{來}{き}て
\ruby{居}{ゐ}られたのに
\ruby{吃驚}{びつ|くり}しました。
』

\原本頁{}
『ホヽヽ、
%
\ruby{貴下}{あな|た}が
\ruby{一心}{いつ|しん}になつて
\ruby{拜}{をが}んで
\ruby{居}{ゐ}らしつたから、
%
\ruby{吃驚}{びつ|くり}なさらないやうにと
\ruby{思}{おも}つて
\ruby[g]{悄々地}{そーつと}
\ruby{妾}{わたし}も
\ruby{拜}{をが}んで
\ruby{居}{を}りましたのよ。
』

\原本頁{}
『それは
\ruby{兎}{と}も
\ruby{角}{かく}も、
%
\ruby{今日}{け|ふ}
\ruby{{\換字{若}}}{も}し
\ruby{貴孃}{あな|た}に
\ruby{御目}{お|め}にかゝれたら、
%
\ruby{先}{ま}づ
\ruby{第一}{だい|いち}に
\ruby{御話}{お|はなし}をして、
%
\ruby{悅}{よろこ}んで
\ruby{戴}{いたゞ}きたいと
\ruby{思}{おも}つて
\ruby{居}{を}りましたが、
%
\ruby{御蔭樣}{お|かげ|さま}で
\ruby{病人}{びやう|にん}も
\ruby{何樣}{ど|う}やら
\ruby{持直}{もち|なほ}して、
%
\ruby{醫者}{い|しや}が
\ruby{屹度}{きつ|と}
\ruby{本復}{ほん|ぷく}すると
\ruby{保證}{うけ|あ}つて
\ruby{吳}{く}れたやうなところ
\ruby{迄}{まで}には
\ruby{漕}{こ}ぎつけました。
%
もう
\ruby{心配}{しん|ぱい}は
\ruby{無}{な}さゝうになりました。
%
\ruby{御案}{お|あん}じ
\ruby{下}{くだ}すつた
\ruby{甲{\換字{斐}}}{か|ひ}もあつて、
%
\ruby{御親切}{ご|しん|せつ}もまあ
\ruby{屆}{とゞ}いたと
\ruby{申}{まを}すものでございます。
%
ほんとに
\ruby{病人}{びやう|にん}とは
\ruby{御緣}{ご|えん}も
\ruby{薄}{うす}い
\ruby{貴卿}{あな|た}が、
%
かうして
\ruby{毎日}{まい|にち}
\g詰めruby{々々}{〳〵}
\ruby{歩}{あゆみ}を
\ruby{{\換字{運}}}{はこ}んで
\ruby{下}{くだ}すつて、
%
\ruby{御願}{ご|ぐわん}を
\ruby{御掛}{お|か}け
\ruby{下}{くだ}さつた
\ruby{御芳{\換字{情}}}{お|こゝ|ろもち}はおろそかには
\ruby{思}{おも}ひません、
%
\ruby{病人}{びやう|にん}が
\ruby{快}{よ}くなりましたにつけても
\ruby{有}{あ}り
\ruby{難}{がた}く
\ruby{思}{おも}ひます。
%
\ruby{今}{いま}といつて
\ruby{今}{いま}は
\ruby{何樣}{ど|う}
\ruby{御禮}{お|れい}の
\ruby{爲}{し}やうも
\ruby{存}{ぞん}じませんが、
%
\ruby{何}{なん}ぞの
\ruby{折}{をり}には
\ruby{屹度}{きつ|と}
\ruby{貴卿}{あな|た}のために、
%
\ruby{貴卿}{あな|た}の
\ruby{優}{やさし}しい
\ruby{御芳{\換字{情}}}{お|こゝ|ろもち}に
\ruby{對}{たい}して
\ruby{其{\換字{丈}}}{それ|だけ}の
\ruby{御{\換字{返}}禮}{お|かへ|し}を
\ruby{爲}{し}やうとは
\ruby{思}{おも}つて
\ruby{居}{を}ります。
%
\ruby{貴卿}{あな|た}の
\ruby{御芳{\換字{情}}}{お|こゝ|ろもち}は
\ruby{長}{なが}く
\ruby{忘}{わす}れません。
』

\原本頁{}
\ruby{此}{こ}の
\ruby{事}{こと}を
\ruby{言}{い}はんとおもふ
\ruby{意}{こゝろ}の
\ruby{充}{み}ち
\ruby{滿}{み}ちたるに、
%
\ruby{言葉}{こと|ば}も
\ruby{自}{おのづ}から
\ruby[<h||]{勢}{いきほひ}
\ruby{籠}{こも}りて、
%
\ruby{口}{くち}ばかりの
\ruby{挨拶}{あい|さつ}ならぬは
\ruby{確乎}{しつ|かり}としたる
\ruby{眼}{め}つきにも
\ruby{著}{しる}し、
%
お
\ruby{龍}{りう}は
\ruby{生眞面目}{き|ま|じ|め}に
\ruby{如是}{か|く}
\ruby{云}{い}はれては、
%
\ruby{眞舳}{ま|とも}には
\ruby{當}{あた}り
\ruby{得}{え}ざるやうの
\ruby{氣}{き}も
\ruby{仕}{し}て、
%
\ruby{安}{やす}からぬ
\ruby{心地}{こゝ|ち}の
\ruby{竊}{ひそか}に
\ruby{爲}{す}ればにや、
%
たゞしは
\ruby{{\換字{又}}}{また}
\ruby{他知}{ひと|し}らぬ
\ruby{考}{かんがへ}の
\ruby{別}{べつ}に
\ruby{有}{あ}ればにや、
%
\ruby{我}{わ}が
\ruby{祈願}{き|ぐわん}の
\ruby{甲{\換字{斐}}}{か|ひ}の
\ruby{見}{み}えしを
\ruby{悅}{よろこ}ぶとも
\ruby{無}{な}く、
%
\ruby{水野}{みづ|の}に
\ruby{斯}{か}ばかり
\ruby{禮}{れい}を
\ruby{云}{い}はれしを
\ruby{嬉}{うれ}しと
\ruby{思}{おも}ふとも
\ruby{見}{み}えず、
%
\ruby{却}{かへ}つて
\ruby{物羞}{もの|はぢ}したるが
\ruby{如}{ごと}く
\ruby{沈着}{おち|つ}かぬ
\ruby{樣子}{やう|す}になりて、
%
\ruby{時々}{とき|〴〵}は
\ruby{見}{み}でも
\ruby{宜}{よ}き
\ruby{{\換字{遠}}方}{とほ|く}の
\ruby{額}{がく}などにちら〳〵と
\ruby{其}{そ}の
\ruby{美}{うつく}しき
\ruby{眼}{め}を
\ruby{辷}{すべ}らせて
\ruby{聞}{き}き
\ruby{居}{ゐ}しが、

\原本頁{}
『まあ
\ruby[g]{眞實}{ほんと}にそりやあ
\ruby{何}{なに}よりの
\ruby{事}{こと}、
%
こんな
\ruby{嬉}{うれ}しいことはもうございません。
%
どんなにか
\ruby{貴下}{あな|た}の
\ruby{御嬉}{お|うれ}しいことでございましやう!。
%
\ruby{貴下}{あな|た}の
\ruby{御胸}{お|むね}の
\ruby{中}{うち}を
\ruby{思}{おも}つて
\ruby{見}{み}ますと、
%
\ruby{妾}{わたし}も
\ruby{何}{なん}だか
\ruby{嬉}{うれ}し
\ruby{涙}{なみだ}が
\ruby{出}{で}さうになります。

\原本頁{}
\ruby{何}{なに}も
\ruby{妾}{わたし}なんぞが
\ruby{御願}{お|ねが}ひ
\ruby{申}{まを}したからといふ
\ruby{譯}{わけ}ではございますまいが、
%
あれ
\ruby{程}{ほど}に
\ruby{一心}{いつ|しん}になつて
\ruby{御願}{お|ねが}ひなすつた
\ruby{貴下}{あな|た}の
\ruby{御念力}{ご|ねん|りき}だけでも、
%
\ruby{佛樣}{ほとけ|さま}が
\ruby{打棄}{うつ|ちや}つては
\ruby{御置}{お|お}きなされなくつて、
%
それで
\ruby[g]{五十子}{いそこ}さんが
\ruby{快}{よ}く
\ruby{御}{お}なりなのでございましやう。
%
ほんとに
\ruby[g]{五十子}{いそこ}さんは
\ruby{御羨}{お|うらや}ましい、
%
\ruby{御不幸}{お|ふし|あはせ}のやうで%「幸福」ここは「は」
\ruby{御幸福}{お|しあ|はせ}の%「幸福」ここは「は」
\ruby{方}{かた}です。
%
\ruby{神樣佛樣}{かみ|さま|ほとけ|さま}の
\ruby{御憐愍}{お|あは|れみ}さへかゝつて
\ruby{居}{ゐ}る
\ruby{方}{かた}ですもの!。
』

\原本頁{}
と
\ruby{末}{すゑ}は
\ruby{誰}{たれ}に
\ruby{云}{い}ふとも
\ruby{無}{な}く
\ruby{言}{い}ひたりしが、
%
はしたなしと
\ruby{思}{おも}ひてや、
%
\ruby{調子}{てう|し}を
\ruby{變}{か}へて、

\原本頁{}
『
\ruby{歸}{かへ}りましたら
\ruby{早{\換字{速}}}{さつ|そく}
\ruby{師匠}{し|ゝやう}にも
\ruby{左樣}{さ|う}
\ruby{申}{まを}しまして、
%
\ruby{御丹精甲{\換字{斐}}}{ご|たん|せい|が|ひ}の
\ruby{有}{あ}つた
\ruby{事}{こと}を
\ruby{聽}{き}かせまして
\ruby{悅}{よろこ}ばせましやう。
%
\ruby{定}{さだ}めし
\ruby{屹度}{きつ|と}
\ruby{有}{あ}り
\ruby{難}{がた}がる
\ruby{事}{こと}でございましやう。
』

\原本頁{}
と
\ruby{言}{ことば}を
\ruby{添}{そ}へたり。
