\Entry{其八}

% メモ 校正終了 2024-05-12 2024-06-07
\原本頁{39-8}%
\ruby[|g|]{際限}{はてし}
\ruby{無}{な}く
\ruby{御堂}{み|だう}の
\ruby{内}{うち}に
\ruby{{\換字{若}}}{わか}き
\ruby{女}{ひと}と
\ruby{立}{た}ち
\ruby{話}{ばなし}して
\ruby{參詣}{さん|けい}の
\ruby[||j>]{老}{らう}
\ruby[||j>]{{\換字{若}}}{にやく}に
% \ruby{老{\換字{若}}}{らう|にやく}に
\ruby{面見}{おもて|み}られん
ごとの
\ruby{好}{この}ましからぬ
\ruby{心地}{こゝ|ち}すれば、
%
\ruby{水野}{みづ|の}は
\ruby{談}{はなし}の
\ruby{切目}{きれ|め}に
\ruby{本{\換字{尊}}}{ほん|ぞん}の
\ruby{方}{かた}を
\ruby{一拜}{いつ|ぱい}して、
%
\ruby{漸}{やうや}く
\ruby{下向}{げ|かう}の
\ruby{路}{みち}に
\ruby{就}{つ}かん
とするに、
%
お
\ruby{龍}{りう}は
\ruby[||j>]{間}{あはひ}
\ruby[||j>]{隔}{ へだ}たらず
% \ruby{間隔}{あはひ|へだ}たらず
\ruby{{\換字{連}}}{つ}れ
\ruby{立}{だ}ちては
\ruby{來}{き}ながら、
%
\ruby{遲々}{ち|ゝ}として
\ruby{却}{かへ}つて
\ruby{水野}{みづ|の}の
\ruby{歩}{あゆみ}を
\ruby{澁}{しぶ}らせん
とするが
\ruby{如}{ごと}し。

\原本頁{40-3}%
\ruby{御堂}{み|だう}の
\ruby{階段}{きざ|はし}は
\ruby{降}{お}り
\ruby{盡}{つく}しぬ。
%
\ruby{貴賤}{き|せん}
\ruby{行}{ゆき}
\ruby{{\換字{交}}}{ちが}ふ
\ruby[g]{長々}{なが〳〵}しき
\ruby[||j>]{石}{いし}
\ruby[||j>]{疊}{だゝみ}の
% \ruby{石疊}{いし|だゝみ}の
\ruby{路}{みち}を
\ruby[|g|]{二人}{ふたり}は
\ruby{辿}{たど}れり。
%
こゝは
\ruby{賑}{にぎ}はしからぬ
\ruby{時}{とき}も
\ruby{無}{な}き
ところ
とて、
%
ぽつくりの
\ruby{響}{ひゞ}き、
%
\ruby{{\換字{雪}}駄}{せつ|た}の
\ruby{鳴}{なり}、
%
\ruby{人聲}{ひと|ごゑ}
\ruby{物音}{もの|おと}
\ruby{一}{ひと}つ
になりて、
%
たゞ
がや〳〵と
\ruby{譯}{わけ}
\原本頁{40-6}\改行%
\ruby{無}{な}く
\ruby{騷}{さわ}がしく、
%
\ruby{七子}{なゝ|こ}の% 原本には漢数字「七」のルビ有り
\ruby{袖}{そで}は
\ruby{擦}{す}れ
\ruby{{\換字{違}}}{ちが}ふ
\ruby{縮緬}{ちり|めん}の
\ruby{袂}{たもと}、
%
\ruby{矢}{や}の
\ruby{字}{じ}の
\ruby{帶}{おび}は
\ruby{觸}{さは}る
\ruby{海軍帽}{かい|ぐん|ばう}、
%
\ruby[|g|]{甲家}{かしこ}の
\ruby{旦那樣}{だん|な|さま}
\ruby{乙家}{そ|こ}の
\ruby{奧樣}{おく|さま}、
%
\ruby{女}{をんな}の
\ruby{兒}{こ}も
\ruby{男}{をとこ}の
\ruby{兒}{こ}も
\ruby{目}{め}まぐるしく
\ruby[|g|]{徃來}{ゆきき}すれば、
%
\ruby{{\換字{遂}}}{と}げては
\ruby{我}{われ}も
\ruby{他}{ひと}を
\ruby{見}{み}るに
\ruby{由}{よし}
\ruby{無}{な}く、
%
\ruby{他}{ひと}も
また
\原本頁{40-9}\改行%
\ruby{我}{われ}を
\ruby{見}{み}るに
\ruby{由}{よし}
\ruby{無}{な}く、
%
\ruby{能}{よ}くは
\ruby{他}{ひと}の
\ruby{談}{はなし}も
\ruby{耳}{みゝ}に
\ruby{入}{い}らねば、
%
\ruby{我}{わ}が
\ruby{談}{はなし}も
また
\ruby{他}{ひと}には
\ruby{聞}{きこ}えぬなり。
%
お
\ruby{龍}{りう}は
\ruby{此}{こ}の
\ruby{中}{なか}を
\ruby{{\換字{連}}}{つ}れ
\ruby{立}{だ}ちて
\ruby{歩}{ある}きつゝ、
%
ややも
すれば
\ruby{獨立}{ひとり|だ}ちて
\ruby{先}{さき}に
\ruby{行}{ゆ}かんとする
\ruby{水野}{みづ|の}を
\ruby{{\換字{追}}}{お}ひかくる
やうにして、

\原本頁{41-2}%
『
アノ、
%
\ruby{今日}{け|ふ}は
\ruby{御休}{お|やす}みの
\ruby{日}{ひ}ぢやあ
ございます
まいのにネエ。
%
わざわざ% ルビ調整(原本通り)非踊り字表記(行末行頭の境界付近)
\ruby{御休}{お|やす}み
なすつて
\ruby{御禮}{お|れい}
\ruby{參}{まゐ}りにいらしつた
\ruby{譯}{わけ}なの?。
』

\原本頁{41-4}%
と、
%
\ruby{{\換字{若}}}{も}し
\ruby{然}{さ}も
あらば、
%
\ruby{餘}{あま}りに
\ruby{彼}{か}の
\ruby{人}{ひと}の
\ruby{事}{こと}を
\ruby{思}{おも}ふ
\ruby{心}{こゝろ}のみ
\ruby{{\換字{強}}}{つよ}くして
\改行% 校正作業の簡略化のため
、
%
\原本頁{41-5}\改行%
\ruby{何}{なに}も
\ruby{彼}{か}も
\ruby{忘}{わす}れ
\ruby{果}{は}てたるが
\ruby[<j>]{甚}{はなはだ}し
\ruby{{\換字{過}}}{す}ぎたり
といふ
やうに、
%
\ruby{聊}{いさゝ}か
\ruby{笑}{わらひ}を
\ruby{含}{ふく}んで
\ruby{問}{と}ひ
かけたり。

\原本頁{41-7}%
\ruby{先刻}{さ|き}にも
\ruby{受}{う}けたる
\ruby{問}{とひ}
ながら、
%
\ruby{答}{こた}ふるも
\ruby{煩}{わづら}はしと
\ruby{思}{おも}ひて
\ruby[<j>]{顧}{かへりみ}ざりしが、
%
\ruby{今}{いま}
\ruby{{\換字{又}}}{また}
\ruby{如是}{かゝ|る}
\ruby{樣子}{やう|す}に
\ruby{問}{と}はれては
\ruby{默}{だま}りても
\ruby{居{\換字{難}}}{ゐ|がた}く、

\原本頁{41-8}%
『
ハヽヽ、
%
まさか
\ruby{左樣}{さ|う}いふ
\ruby{譯}{わけ}でも
\ruby{無}{な}いのですが、
%
\ruby{丁度}{ちやう|ど}
\ruby[|g|]{職務}{つとめ}は
\ruby{辭}{よ}して
\ruby{仕舞}{し|ま}つたので、
%
それで
\ruby{萬一}{ひよ|つと}
したら
\ruby[|g|]{貴卿}{あなた}に
\ruby{御目}{お|め}に
かゝれ
やうか
といふ
\ruby[<j>]{考}{かんがへ}も
\ruby{有}{あ}つて、
%
\ruby{{\換字{平}}日}{いつ|も}よりは
\ruby{早}{はや}く
\ruby{出}{で}て
\ruby{來}{き}たのです。
%
\ruby{仕合}{し|あはせ}に
\ruby{巧}{うま}く
\ruby{御目}{お|め}にかゝる
\ruby{事}{こと}が
\ruby{出來}{で|き}て、
%
\ruby{聞}{き}いて
\ruby{戴}{いたゞ}かうと
\ruby{思}{おも}つて
\ruby{居}{ゐ}たことも
\ruby{聞}{き}いて
\ruby{戴}{いたゞ}いたので、
%
\ruby{悉皆}{すつ|かり}
\ruby{思}{おも}つた
\ruby{{\換字{通}}}{とほ}り
に
なりましたが、
%
これも
\ruby{下}{くだ}らない
\ruby[|g|]{職務}{つとめ}なんか
\ruby{廢}{よ}して
\ruby{仕舞}{し|ま}つた
\ruby{故}{せゐ}でしやう、% せ(ゐ)
%
ハヽハヽ。
』

\原本頁{42-4}%
と
\ruby{輕}{かろ}く
\ruby{打}{うち}
\ruby{笑}{わら}ひたり。

\原本頁{42-5}%
\ruby{水野}{みづ|の}は
\ruby{輕}{かろ}く
\ruby{打}{うち}
\ruby{笑}{わら}ひたれども、
%
\ruby[|g|]{職務}{つとめ}を
\ruby{棄}{す}てたり
といふ
\ruby{事}{こと}の、
%
お
\ruby{龍}{りう}には
\ruby{輕}{かろ}からず
\ruby{聞}{きこ}えや
しけん、
%
\ruby{其}{そ}の
\ruby{眉}{まゆ}を
\ruby{顰}{ひそ}めて
\ruby{心配}{しん|ぱい}げに、

\原本頁{42-7}%
『
お
\ruby[|g|]{職務}{つとめ}を
\ruby{御止}{お|よ}し
なすつた
のですつて!。
%
\ruby{何故}{な|ぜ}
\ruby{其樣}{そ|ん}な
ことを
なすつたの?。
%
\ruby{何}{なに}も
\ruby{御困}{お|こま}りなさる
\ruby{樣}{やう}な
\ruby{事}{こと}は
\ruby{御有}{お|あ}んなさりやあ
\ruby{仕}{し}ます
まい
けれどもネエ、
%
\ruby{何}{なん}だつて
\ruby{其樣}{そ|ん}な
\ruby{事}{こと}を
なさい
ましたの。
%
そんな
\ruby{事}{こと}を
なさら
\ruby{無}{な}くても
ぢやあ
\ruby{有}{あ}りませんか。
』

\原本頁{42-11}%
と
\ruby{滿腔}{まん|こう}の
\ruby[||j>]{同}{どう}
\ruby[||j>]{{\換字{情}}}{じやう}より
% \ruby{同{\換字{情}}}{どう|じやう}より
\ruby{私}{ひそか}に
\ruby[||j>]{生}{せい}
\ruby[||j>]{活}{くわつ}の
% \ruby{生活}{せい|くわつ}の
\ruby{{\換字{道}}}{みち}の
\ruby[|g|]{便宜}{たより}
\ruby{惡}{あし}かる
べきを
\ruby{氣{\換字{遣}}}{き|づか}ふ
ものの
\ruby{如}{ごと}し。

\原本頁{43-2}%
『
ナニ、
%
\ruby{別}{べつ}に
\ruby{無理}{む|り}に
\ruby{辭}{や}めたいと
\ruby{思}{おも}つた
のでも
\ruby{無}{な}い
のです
けれども、
%
\ruby{辭}{や}め
させられて
\ruby{見}{み}れば
\ruby{仕方}{し|かた}が
ないわけ
ですもの。
』

\原本頁{43-4}%
『
だつて、
%
\ruby{何故}{な|ぜ}
ネエ?。
%
\ruby{餘}{あんま}り
\ruby{御}{ご}
\ruby{不{\換字{勤}}}{ふ|づとめ}
でも
なすつた
の?。
』

\原本頁{43-5}%
『
イヽヤ、
%
そんな
\ruby{事}{こと}は
\ruby{決}{けつ}して
\ruby{爲}{せ}ん
\ruby{私}{わたし}です。
』

\原本頁{43-6}%
『
ぢやあ
\ruby{其樣}{そ|ん}な
\ruby{事}{こと}になる
\ruby{譯}{わけ}が
\ruby{無}{な}いぢやあ
\ruby{有}{あ}りませんか。
%
もし
それぢやあ
\ruby{萬一}{ひよ|つと}
したら
\ruby{五十子}{い|そ|こ}さんの
\ruby{事}{こと}で
\ruby[||j>]{{\換字{評}}}{ひやう}
\ruby[||j>]{{\換字{判}}}{ ばん}でも
% \ruby{{\換字{評}}{\換字{判}}}{ひやう|ばん}でも
\ruby{立}{た}つて、
%
\ruby{其}{そ}の
\ruby{爲}{ため}
といふ
やうな
\ruby{譯}{わけ}ぢやあ
\ruby{無}{な}くつて?。
』

\原本頁{43-9}%
『
ハヽ、
%
\ruby{云}{い}はゞ
\ruby{其樣}{そ|ん}な
\ruby{事}{こと}の
\ruby{爲}{ため}
なんで
しやうが、
%
\ruby{何樣}{ど|う}でも
\ruby{其樣}{そ|ん}な
\ruby{事}{こと}は
\ruby{構}{かま}やあ
\ruby{仕}{し}ません。
%
まさか
\ruby{下}{くだ}らない
\ruby[|g|]{職務}{つとめ}を
\ruby{止}{よ}した
から
といつて
\ruby{困}{こま}りも
\ruby{仕}{し}ます
まいから、
%
いつそ
\ruby{卑小}{け|ち}な
\ruby[|g|]{職務}{つとめ}
なんかに
\ruby{縛}{しば}られない
\ruby{今日}{け|ふ}の
\ruby{方}{はう}が
\ruby{宜}{い}い
\ruby[||j>]{心}{こゝろ}
\ruby[||j>]{持}{ もち}が
% \ruby{心持}{こゝろ|もち}が
\ruby{仕}{し}ます。
』

\原本頁{44-2}%
『
そりやあ
\ruby{左樣}{さ|う}でも
\ruby{御有}{お|あ}ん
なさり
ましやうが、
%
でも
まあ
\ruby{差當}{さし|あた}つて
‥‥。
%
ほんたう
なら
\ruby{五十子}{い|そ|こ}さんの
\ruby{御母}{おつ|か}さんが
\ruby{何樣}{ど|う}にでも
\ruby{仕}{し}て
あげるのが
\ruby{{\換字{道}}}{みち}
なんです
けれども。
』

\原本頁{44-5}%
\ruby{何}{なに}をか
\ruby{思}{おも}ふ、
%
お
\ruby{龍}{りう}は
\ruby{言}{い}ひ
\ruby{澱}{よど}んで
\ruby[<j>]{考}{かんがへ}に
\ruby{沈}{しづ}みしが、
%
\ruby{水野}{みづ|の}は
\ruby{却}{かへ}つて
\ruby{冴}{さえ}% ルビ調整(原本通り)非踊り字表記(行末行頭の境界付近)
\ruby{冴}{ざえ}として、

\原本頁{44-7}%
『
ハヽ、
%
\ruby{決}{けつ}して
\ruby{何}{なに}も
\ruby{心配}{しん|ぱい}して
\ruby{下}{くだ}さらんでも
\ruby{可}{い}いのです、
%
\ruby{考案}{かん|がへ}が
ある
のですから。
%
\ruby{信心}{しん|〴〵}を
\ruby{仕}{し}て、
%
\ruby{愚}{ぐ}だと
\ruby{云}{い}はれて、
%
\ruby{擯{\換字{斥}}}{ひん|せき}されて
\ruby{仕舞}{し|ま}つた、
%
こんな
\ruby{馬鹿}{ば|か}でも、
%
\ruby{男}{をとこ}は
\ruby[|g|]{矢張}{やつぱ}り
\ruby{男}{をとこ}
ですからネ。
%
イヤ
\ruby{此處}{こ|ゝ}で
\ruby{失敬}{しつ|けい}
しましやう、
%
\ruby{左樣}{さ|やう}なら。
』

\原本頁{44-11}%
と
\ruby{書生風}{しよ|せい|ふう}に
\ruby{淡泊}{たん|ぱく}に
\ruby{挨拶}{あい|さつ}して
\ruby{別}{わか}れ
\ruby{去}{さ}らんとす。
%
\ruby{何時}{い|つ}か
\ruby{石路}{せき|ろ}は
\ruby{既}{すで}に
\ruby{歩}{あゆ}み
\ruby{盡}{つく}せる
なり。

\原本頁{45-2}%
\ruby{何}{なに}にか
\ruby{心}{こゝろ}を
\ruby{奪}{と}られ
\ruby{居}{ゐ}し
お
\ruby{龍}{りう}は、
%
\ruby{水野}{みづ|の}の
\ruby[|g|]{告別}{わかれ}の
\ruby{辭}{ことば}に
\ruby{打}{うち}
\ruby{慌}{あは}てゝ、

\原本頁{45-3}%
『
ぢやあ
\ruby[|g|]{明日}{あした}
また
\ruby{御眼}{お|め}に
かゝれますの?。
』

\原本頁{45-4}%
と
\ruby{辛}{から}くも
\ruby{一句}{いつ|く}
\ruby{問}{と}ひ
かくれば、
%
\ruby{既}{すで}に
\ruby{十餘歩}{じふ|よ|ほ}を
\ruby{隔}{へだ}たりし
\ruby{水野}{みづ|の}は
\ruby{無言}{む|ごん}に
\ruby[|g|]{點頭}{うなづ}きて、
%
\ruby{{\換字{情}}}{つれ}
\ruby{無}{な}きが
\ruby{如}{ごと}く
\ruby{其}{その}
\ruby{儘}{まゝ}
\ruby{{\換字{終}}}{つひ}に
\ruby{東}{ひがし}に
\ruby{去}{さ}りたり。

\原本頁{45-6}%
\ruby{去}{さ}り
\ruby{去}{さ}る
\ruby{百歩餘}{ひやつ|ぽ|あま}り
にして、
%
\ruby{水野}{みづ|の}は
\ruby{徐}{おもむ}ろに
\ruby{首}{かうべ}を
\ruby{囘}{かへ}して% 原本通り「囘」
\ruby{見}{み}れば、
%
\ruby{人}{ひと}の
\ruby{繁}{しげ}く
\ruby{車}{くるま}の
\ruby{煩}{うるさ}きが
\ruby{中}{なか}に
\ruby{{\換字{猶}}}{なほ}
\ruby{悠然}{いう|ぜん}と
\ruby{立}{た}つて、
%
\ruby{我}{わ}が
\ruby{方}{かた}をや
\ruby{見{\換字{送}}}{み|おく}り
\ruby{居}{ゐ}たる、
%
お
\ruby{龍}{りう}の
\ruby{面}{おもて}の
\ruby{花}{はな}と
\ruby{白}{しろ}きが
\ruby{仄}{ほの}かに
\ruby{見}{み}えたり。
