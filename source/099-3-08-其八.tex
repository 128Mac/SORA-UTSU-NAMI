\Entry{其八}

\ruby{際限無}{はて|し|な}く
\ruby{御堂}{み|だう}の
\ruby{内}{うち}に
\ruby{若}{わか}き
\ruby{女}{ひと}と
\ruby{立}{た}ち
\ruby{話}{ばなし}して
\ruby{參詣}{さん|けい}の
\ruby{老若}{らう|にやく}に
\ruby{面見}{おも|てみ}られんごとの
\ruby{好}{この}ましからぬ
\ruby{心地}{こゝ|ち}すれば、
\ruby{水野}{みづ|の}は
\ruby{談}{はなし}の
\ruby{切目}{きれ|め}に
\ruby{本{\換字{尊}}}{ほん|ぞん}の
\ruby{方}{かた}を
\ruby{一拜}{いつ|ぱい}して、
\ruby{漸}{やうや}く
\ruby{下向}{げ|かう}の
\ruby{路}{みち}に
\ruby{就}{つ}かんとするに、
お
\ruby{龍}{りゆう}は
\ruby{間隔}{あはひ|へだ}たらず
\ruby{連}{つ}れ
\ruby{立}{だ}ちては、
\ruby{遲々}{ち|ゝ}として
\ruby{却}{かへ}つて
\ruby{水野}{みづ|の}の
\ruby{歩}{あゆみ}を
\ruby{澁}{しぶ}らせんとするが
\ruby{如}{ごと}し。

\ruby{御堂}{み|だう}の
\ruby{階段}{きざ|はし}は
\ruby{降}{お}り
\ruby{盡}{つく}しぬ。
\ruby{貴賤行{\換字{交}}}{き|せん|ゆき|ちが}ふ
\ruby{長々}{なが|〳〵}しき
\ruby{石疊}{いし|だゝみ}の
\ruby{路}{みち}を
\ruby{二人}{ふた|り}は
\ruby{辿}{たど}れり。
こゝは
\ruby{賑}{にぎ}はしからぬ
\ruby{時}{とき}も
\ruby{無}{な}きところとて、ぽつくりの
\ruby{響}{ひゞ}き、
\ruby{雪駄}{せつ|た}の
\ruby{鳴}{なり}、
\ruby{人聲}{ひと|ごゑ}
\ruby{物音}{もの|おと}
\ruby{一}{ひと}つになりて、たゞがや〳〵と
\ruby{譯無}{わけ|な}く
\ruby{騷}{さわ}がしく、
\ruby{七子}{なゝ|こ}の
\ruby{袖}{そで}は
\ruby{擦}{す}れ
\ruby{{\換字{違}}}{ちが}ふ
\ruby{縮緬}{ちり|めん}の
\ruby{袂}{たもと}、
\ruby{矢}{や}の
\ruby{字}{じ}の
\ruby{帶}{おび}は
\ruby{觸}{さわ}る
\ruby{海軍帽}{かい|ぐん|ばう}、
\ruby{甲家}{かし|こ}の
\ruby{旦那樣}{だん|な|さま}
\ruby{乙家}{そ|こ}の
\ruby{奧樣}{おく|さま}、
\ruby{女}{をんな}の
\ruby{兒}{こ}も
\ruby{男}{をとこ}の
\ruby{兒}{こ}も
\ruby{目}{め}まぐるしく
\ruby{往來}{ゆき|き}すれば、
\ruby{{\換字{遂}}}{と}げては
\ruby{我}{われ}も
\ruby{他}{ひと}を
\ruby{見}{み}るに
\ruby{由無}{よし|な}く、
\ruby{他}{ひと}もまた
\ruby{我}{われ}を
\ruby{見}{み}るに
\ruby{由無}{よし|な}く、
\ruby{能}{よ}くは
\ruby{他}{ひと}の
\ruby{談}{はなし}も
\ruby{耳}{みゝ}に
\ruby{入}{い}らねば、
\ruby{我}{わ}が
\ruby{談}{はなし}もまた
\ruby{他}{ひと}には
\ruby{聞}{きこ}えぬなり。
お
\ruby{龍}{りゆう}は
\ruby{此}{こ}の
\ruby{中}{なか}を
\ruby{{\換字{連}}}{つ}れ
\ruby{立}{だ}ちて
\ruby{歩}{ある}きつゝ、ややもすねば
\ruby{獨立}{ひとり|だ}ちて
\ruby{先}{さき}に
\ruby{行}{ゆ}かんとする
\ruby{水野}{みづ|の}を
\ruby{{\換字{追}}}{お}ひかくるやうにして、

『アノ、
\ruby{今日}{け|ふ}は
\ruby{御休}{お|やす}みの
\ruby{日}{ひ}ぢやあございますまいのにネエ。
わざわざ
\ruby{御休}{お|やす}みなすつて
\ruby{御禮參}{お|れい|まゐ}りにいらしつた
\ruby{譯}{わけ}なの?。
』

と、
\ruby{若}{も}し
\ruby{然}{さ}もあらば、
\ruby{餘}{あま}りに
\ruby{彼}{か}の
\ruby{人}{ひと}の
\ruby{事}{こと}を
\ruby{思}{おも}ふ
\ruby{心}{こゝろ}の
\ruby{{\換字{強}}}{つよ}くして、
\ruby{何}{なに}も
\ruby{彼}{か}も
\ruby{忘}{わす}れ
\ruby{果}{は}てたるが
\ruby{甚}{はなはだ}し
\ruby{{\換字{過}}}{す}ぎたりといふやうに、
\ruby{聊}{いさゝ}か
\ruby{笑}{わらひ}を
\ruby{含}{ふく}んで
\ruby{問}{と}ひかけたり。

\ruby{先刻}{さ|き}にも
\ruby{受}{う}けたる
\ruby{問}{とひ}ながら、
\ruby{答}{こた}ふるも
\ruby{煩}{わづら}はしと
\ruby{思}{おも}ひて
\ruby{顧}{かへりみ}ざりしが、
\ruby{今}{いま}
\ruby{{\換字{又}}}{また}
\ruby{如是}{かゝ|る}
\ruby{樣子}{やう|す}に
\ruby{問}{と}はれては
\ruby{默}{だま}りても
\ruby{居難}{ゐ|がた}く、

『ハヽヽ、まさか
\ruby{左樣}{さ|う}いふ
\ruby{譯}{わけ}でも
\ruby{無}{な}いのですが、
\ruby{丁度}{ちやう|ど}
\ruby{職務}{つと|め}は
\ruby{辭}{よ}して
\ruby{仕舞}{し|ま}つたので、それで
\ruby{萬一}{ひよ|つと}したら
\ruby{貴卿}{あな|た}に
\ruby{御目}{お|め}にかゝれやうかといふ
\ruby{考}{かんがへ}も
\ruby{有}{あ}つて、
\ruby{{\換字{平}}日}{いつ|も}よりは
\ruby{早}{はや}く
\ruby{出}{で}て
\ruby{來}{き}たのです。
\ruby{仕合}{しあ|わせ}に
\ruby{巧}{うま}く
\ruby{御目}{お|め}にかゝる
\ruby{事}{こと}が
\ruby{出來}{で|き}て、
\ruby{聞}{き}いて
\ruby{戴}{いたゞ}かうと
\ruby{思}{おも}つて
\ruby{居}{ゐ}たことも
\ruby{聞}{き}いて
\ruby{戴}{いたゞ}いたので、
\ruby{悉皆}{すつ|かり}
\ruby{思}{おも}つた
\ruby{通}{とほ}りになりましたが、これも
\ruby{下}{くだ}らない
\ruby{職務}{つと|め}なんか
\ruby{廢}{よ}して
\ruby{仕舞}{し|ま}つた
\ruby{故}{せゐ}でしやう、ハヽハヽ。
』

と
\ruby{輕}{かろ}く
\ruby{打笑}{うち|わら}ひたり。

\ruby{水野}{みづ|の}は
\ruby{輕}{かろ}く
\ruby{打笑}{うち|わら}ひたれども、
\ruby{職務}{つと|め}を
\ruby{棄}{す}てたりといふ
\ruby{事}{こと}の、
お
\ruby{龍}{りう}には
\ruby{輕}{かろ}からず
\ruby{聞}{きこ}えやしけん、
\ruby{其}{そ}の
\ruby{眉}{まゆ}を
\ruby{顰}{ひそ}めて
\ruby{心配}{しん|ぱい}げに、

『お
\ruby{職務}{つと|め}を
\ruby{御止}{お|よ}しなすつたのですつて!。
\ruby{何故}{な|ぜ}
\ruby{其樣}{そ|ん}なことをなすつたの?。
\ruby{何}{なに}も
\ruby{御困}{お|こま}りなさる
\ruby{樣}{やう}な
\ruby{事}{こと}は
\ruby{御有}{お|あ}んなさりやあ
\ruby{仕}{し}ますまいけれどもネエ、
\ruby{何}{なん}だつて
\ruby{其樣}{そ|ん}な
\ruby{事}{こと}をなさいましたの。
そんな
\ruby{事}{こと}をなさら
\ruby{無}{な}くてもぢやあ
\ruby{有}{あ}りませんか。
』

と
\ruby{滿腔}{まん|こう}の
\ruby{同{\換字{情}}}{どう|じやう}より
\ruby{私}{ひそか}に
\ruby{生活}{せい|くわつ}の
\ruby{{\換字{道}}}{みち}の
\ruby{便宜惡}{たよ|り|あし}かるべきを
\ruby{氣{\換字{遣}}}{き|づか}ふものの
\ruby{如}{ごと}し。

『ナニ、
\ruby{別}{べつ}に
\ruby{無理}{む|り}に
\ruby{辭}{や}めたいと
\ruby{思}{おも}つたのでも
\ruby{無}{な}いのですけれども、
\ruby{辭}{や}めさせられて
\ruby{見}{み}れば
\ruby{仕方}{し|かた}がないわけですもの。
』

『だつて、
\ruby{何故}{な|ぜ}ネエ?。
\ruby{餘}{あんま}り
\ruby{御不{\換字{勤}}}{ご|ふ|づとめ}でもなすつたの?。
』

『イヽヤ、そんな
\ruby{事}{こと}は
\ruby{決}{けつ}して
\ruby{爲}{せ}ん
\ruby{私}{わたし}です。
』

『ぢやあ
\ruby{其樣}{そ|ん}な
\ruby{事}{こと}になる
\ruby{譯}{わけ}が
\ruby{無}{な}いぢやあ
\ruby{有}{あ}りませんか。
もしそれぢやあ
\ruby{萬一}{ひよ|つと}したら
\ruby{五十子}{い|そ|こ}さんの
\ruby{事}{こと}で
\ruby{{\換字{評}}{\換字{判}}}{ひやう|ばん}でも
\ruby{立}{た}つて、
\ruby{其}{そ}の
\ruby{爲}{ため}といふやうな
\ruby{譯}{わけ}ぢやあ
\ruby{無}{な}くつて?。
』

『ハヽ、
\ruby{云}{い}はゞ
\ruby{其樣}{そ|ん}な
\ruby{事}{こと}の
\ruby{爲}{ため}なんでしやうが、
\ruby{何樣}{ど|う}でも
\ruby{其樣}{そ|ん}な
\ruby{事}{こと}は
\ruby{構}{かま}やあ
\ruby{仕}{し}ません。
まさか
\ruby{下}{くだ}らない
\ruby{職務}{つと|め}を
\ruby{止}{よ}したからといつて
\ruby{困}{こま}りも
\ruby{仕}{し}ますまいから、いつそ
\ruby{卑小}{け|ち}な
\ruby{職務}{つと|め}なんかに
\ruby{縛}{しば}られない
\ruby{今日}{け|ふ}の
\ruby{方}{はう}が
\ruby{宜}{よ}い
\ruby{心持}{こゝろ|もち}が
\ruby{仕}{し}ます。
』

『そりやあ
\ruby{左樣}{さ|う}でも
\ruby{御有}{お|あ}んなさりましやうが、でもまあ
\ruby{差當}{さし|あた}つて…………。
ほんたうなら
\ruby{五十子}{い|そ|こ}さんの
\ruby{御母}{お|つか}さんが
\ruby{何樣}{ど|う}にでも
\ruby{仕}{し}てあげるのが
\ruby{{\換字{道}}}{みち}なんですけれども。
』

\ruby{何}{なに}をか
\ruby{思}{おも}ふ、
お
\ruby{龍}{りゆう}は
\ruby{言}{い}ひ
\ruby{澱}{よど}んで
\ruby{考}{かんがへ}に
\ruby{沈}{しづ}みしが、
\ruby{水野}{みづ|の}は
\ruby{却}{かへ}つて
\ruby{冴}{さえ}
\ruby{冴}{ざえ}として、

『ハヽ、
\ruby{決}{けつ}して
\ruby{何}{なに}も
\ruby{心配}{しん|ぱい}して
\ruby{下}{くだ}さらんでも
\ruby{可}{い}いのです、
\ruby{考案}{かん|がへ}があるのですから。
\ruby{信心}{しん|〴〵}を
\ruby{仕}{し}て、
\ruby{愚}{ぐ}だと
\ruby{云}{い}はれて、
\ruby{擯斥}{ひん|せき}されて
\ruby{仕舞}{し|ま}つた、こんな
\ruby{馬鹿}{ば|か}でも、
\ruby{男}{をとこ}は
\ruby{矢張}{やつ|ぱ}り
\ruby{男}{をとこ}ですからネ。
イヤ
\ruby{此處}{こ|ゝ}で
\ruby{失敬}{しつ|けい}しましやう、
\ruby{左樣}{さ|やう}なら。
』

と
\ruby{書生風}{しよ|せい|ふう}に
\ruby{淡泊}{たん|ぱく}に
\ruby{挨拶}{あい|さつ}して
\ruby{別}{わか}れ
\ruby{去}{さ}らんとす。
\ruby{何時}{い|つ}か
\ruby{石路}{せき|ろ}は
\ruby{既}{すで}に
\ruby{歩}{あゆ}み
\ruby{盡}{つく}せるなり。

\ruby{何}{なに}にか
\ruby{心}{こゝろ}を
\ruby{奪}{と}られ
\ruby{居}{ゐ}し
お
\ruby{龍}{りゆう}は、
\ruby{水野}{みづ|の}の
\ruby{告別}{わか|れ}の
\ruby{辭}{ことば}に
\ruby{打慌}{うち|あは}てゝ、

『ぢやあ
\ruby{明日}{あし|た}また
\ruby{御眼}{お|め}にかゝれますの?。
』

と
\ruby{辛}{から}くも
\ruby{一句問}{いつ|く|と}ひかくれば、
\ruby{既}{すで}に
\ruby{十餘歩}{じう|よ|ほ}を
\ruby{隔}{へだ}たりし
\ruby{水野}{みづ|の}は
\ruby{無言}{む|ごん}に
\ruby{點頭}{うな|づ}きて、
\ruby{{\換字{情}}無}{つれ|な}きが
\ruby{如}{ごと}く
\ruby{其儘{\換字{終}}}{その|まゝ|つひ}に
\ruby{東}{ひがし}に
\ruby{去}{さ}りたり。

\ruby{去}{さ}り
\ruby{去}{さ}る
\ruby{百歩餘}{ひやく|ほ|あま}りにして、
\ruby{水野}{みづ|の}は
\ruby{徐}{おもむ}ろに
\ruby{首}{かうべ}を
\ruby{囘}{かへ}して
\ruby{見}{み}れば、
\ruby{人}{ひと}の
\ruby{繁}{しげ}く
\ruby{車}{くるま}の
\ruby{煩}{うるさ}きが
\ruby{中}{なか}に
\ruby{{\換字{猶}}}{なほ}
\ruby{悠然}{いう|ぜん}と
\ruby{立}{た}つて、
\ruby{我}{わ}が
\ruby{方}{かた}をや
\ruby{見{\換字{送}}}{み|おく}り
\ruby{居}{ゐ}たる、
お
\ruby{龍}{りゆう}の
\ruby{面}{おもて}の
\ruby{花}{はな}と
\ruby{白}{しろ}きが
\ruby{仄}{ほの}かに
\ruby{見}{み}えたり。

