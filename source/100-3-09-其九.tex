\Entry{其九}

\原本頁{}
\ruby{疾病}{やま|ひ}のやうやく
\ruby{快}{よ}くなり
\ruby{行}{ゆ}くさまを、
%
\ruby{薄紙}{うす|がみ}を
\ruby{剝}{は}ぐが
\ruby{如}{ごと}しとは
\ruby{誰}{たれ}が
\ruby{云}{い}ひ
\ruby{初}{そ}めけん、
%
さしもに
\ruby{一時}{いち|じ}は
\ruby{危}{あやふ}かりし
\ruby[g]{五十子}{いそこ}の、
%
\ruby{天壽}{てん|じゆ}いまだ
\ruby{盡}{つ}きねば
\ruby{人力効}{じん|りよく|かひ}ありて、
%
\ruby{實}{げ}に
\ruby{此頃}{この|ごろ}は
\ruby{薄紙}{うす|がみ}を
\ruby{剝}{は}ぐが
\ruby{如}{ごと}く、
%
\ruby{日}{ひ}に
\ruby{日}{ひ}に
\ruby{少}{すこ}しづつ
\ruby{快}{よ}くなりゆけば、
%
\ruby{年齡}{と|し}の
\ruby{勢}{いきほひ}も
\ruby{藥餌}{やく|じ}の
\ruby{能}{のう}もこゝに
\ruby{現}{あらは}れ
\ruby{來}{きた}りて、
%
\ruby{一陽來復}{いち|やう|らい|ふく}の
\ruby{機}{をり}
\ruby{待}{ま}ち
\ruby{得}{え}たる
\ruby{{\換字{若}}樹}{わか|ぎ}の、
%
\ruby{{\換字{猶}}}{なほ}
\ruby{{\換字{雪}}}{ゆき}には
\ruby{籠}{こ}められ
\ruby{氷}{こほり}には
\ruby{{\換字{鎖}}}{とざ}されながらも、
%
\ruby{既}{すで}に
\ruby{漸}{やうや}く
\ruby{芽}{め}をも
\ruby{蕾}{いばら}をも
\ruby{含}{ふく}み
\ruby{居}{ゐ}て、
%
やがて
\ruby{春風}{はる|かぜ}の
\ruby{渡}{わた}らん
\ruby{曉}{あした}に
\ruby{誇}{ほこ}らんとするが
\ruby{如}{ごと}く、
%
\ruby{窶}{やつ}れ
\ruby{果}{は}てたるが
\ruby{中}{なか}にも、
%
はや
\ruby{行末}{ゆく|すゑ}の
\ruby{榮}{さか}ゆる
\ruby{色}{いろ}は
\ruby{微見}{ほの|み}ゆるに
\ruby{至}{いた}れり。

\原本頁{}
されば
\ruby{愁}{うれひ}の
\ruby{雲厚}{くも|あつ}く
\ruby{蔽}{おほ}ひて、
%
\ruby{火}{ひ}の
\ruby{{\換字{消}}}{き}えたるやうに
\ruby{陰氣}{いん|き}なりし
\ruby{此}{こ}の
\ruby{家}{いへ}の、
%
\ruby[g]{五十子}{いそこ}が
\ruby{面}{おもて}の
\ruby{色}{いろ}の
\ruby{美}{よ}くなり
\ruby{行}{ゆ}くに
\ruby{{\換字{連}}}{つ}れて、
%
\ruby{一室}{ひと|ま}の
\ruby{中}{うち}は
\ruby{日}{ひ}の
\ruby{出}{い}でし
\ruby{如}{ごと}く
\ruby{賑}{にぎ}やかになり、
%
\ruby{先}{ま}づ
\ruby{年少}{とし|わか}の
\ruby{松之助}{まつ|の|すけ}より
\ruby{何}{なん}ぞに
\ruby{付}{つ}けて
\ruby{笑聲}{わら|ひ}を
\ruby{洩}{もら}せば、
%
\ruby{元氣溢}{げん|き|あふ}るゝばかりの
\ruby{看護{\換字{婦}}}{かん|ご|ふ}も
\ruby{折{\換字{節}}}{をり|ふし}は
\ruby{高笑}{たか|わら}ひして、
%
こゝは
\ruby{人々}{ひと|〴〵}の
\ruby{機{\換字{嫌}}}{き|げん}も
\ruby{好}{よ}く、
%
\ruby{談話聲}{はな|し|ごゑ}も
\ruby{冴}{さ}ゆる、
%
\ruby{陽氣}{やう|き}の
\ruby{家}{いへ}と
\ruby{打}{うつ}て
\ruby{變}{かは}りたり。

\原本頁{}
\ruby{體溫}{たい|おん}は
\ruby{高下}{かう|げ}
\ruby{少}{すくな}くなりて
\ruby{漸}{やうや}く
\ruby{{\換字{平}}常}{つ|ね}に
\ruby{復}{ふく}さんとするの
\ruby{勢}{いきほひ}を
\ruby{示}{しめ}し、
%
\ruby{脉搏}{みやく|はく}は
\ruby{{\換字{猶}}}{なほ}
\ruby{{\換字{弱}}}{よわ}けれども
\ruby{走}{はし}らず
\ruby{澁}{しぶ}らずして
\ruby{危險}{き|けん}の
\ruby{{\換字{虞}}}{おそれ}の
\ruby{既}{すで}に
\ruby{去}{さ}りたるを
\ruby{現}{あらは}せり。
%
\ruby{恐}{おそ}ろしき
\ruby{熱}{ねつ}に
\ruby{惱}{なや}める
\ruby{日}{ひ}の
\ruby{少}{すくな}からざりしかば、
%
\ruby{肉}{にく}は
\ruby{落}{お}ち
\ruby{骨}{ほね}は
\ruby{立}{た}ちて、
%
\ruby{今}{いま}
\ruby{{\換字{猶}}}{なほ}
\ruby{一人}{ひと|り}しては
\ruby{何}{なん}とする
\ruby{事}{こと}も
\ruby{叶}{かな}はぬほどに
\ruby{衰}{おとろ}へ
\ruby{果}{は}てたれど、
%
\ruby{一昨日}{を|とゝ|ひ}より
\ruby{昨日}{きの|ふ}は
\ruby{好}{よ}く、
%
\ruby{昨日}{きの|ふ}より
\ruby{今日}{け|ふ}は
\ruby{確乎}{しつ|かり}として、
%
\ruby{病勢}{びやう|せい}の
\ruby{烈}{はげ}しかりしに
\ruby{纎{\換字{弱}}}{か|よわ}き
\ruby{{\換字{婦}}人}{をん|な}の
\ruby{身}{み}なれば
\ruby{衰{\換字{弱}}}{すゐ|じやく}こそ
\ruby{{\換字{尋}}常越}{な|み|こ}えて
\ruby{甚}{はなはだ}しけれ、
%
これより
\ruby{五六週間}{ご|ろく|しう|かん}も
\ruby{立}{た}たば、
%
\ruby{必}{かなら}ず
\ruby{病}{や}まぬ
\ruby{徃日}{むか|し}の
\ruby{健康}{すこ|やかさ}に
\ruby{囘}{かへ}つて、% 原本通り「囘」
%
\ruby{日々}{にち|〳〵}の
\ruby{{\換字{勤}}務}{つ|とめ}を
\ruby{執}{と}るに
\ruby{至}{いた}るを
\ruby{得}{う}べしとの
\ruby{相良}{さが|ら}
\ruby{尾竹}{を|だけ}の
\ruby{言葉}{こと|ば}も
\ruby{僞}{いつは}りなるまじく
\ruby{思}{おも}はれぬ。

\原本頁{}
\ruby[g]{五十子}{いそこ}が
\ruby{狀態}{やう|す}
\ruby{是}{かく}の
\ruby{如}{ごと}くなれば、
%
\ruby{松之助}{まつ|の|すけ}は
\ruby{自}{みづか}ら
\ruby{熱}{あつ}き
\ruby{{\換字{乳}}}{ちゝ}を
\ruby{薦}{すゝ}めたる
\ruby{或曉}{ある|あさ}、
%
\ruby{其}{そ}の
\ruby{姊}{あね}の
\ruby{面}{おもて}をつく〴〵と
\ruby{打護}{うち|まも}りて、

\原本頁{}
『もう
\ruby{大{\換字{丈}}夫}{だい|ぢやう|ぶ}だ、
%
もう
\ruby{大{\換字{丈}}夫}{だい|ぢやう|ぶ}だ!。
%
ほんとに
\ruby{怖}{こは}いと
\ruby{思}{おも}つた
\ruby{時}{とき}も
\ruby{有}{あ}つたけれ
\ruby{共}{ども}、
%
とう〳〵
\ruby{僕}{ぼく}の
\ruby{姊}{ねえ}さんは
\ruby{僕}{ぼく}の
\ruby{姊}{ねえ}さんになつた!。
』

\原本頁{}
と
\ruby{無邪氣}{む|じや|き}に
\ruby{叫}{さけ}び
\ruby{出}{だ}して
\ruby{笑}{わら}ひ
\ruby{悅}{よろこ}び、
%
\ruby{相良}{さが|ら}が
\ruby{手}{て}より
\ruby{來}{きた}れる
\ruby{看護{\換字{婦}}}{かん|ご|ふ}の
\ruby{芳野}{よし|の}は、
%
\ruby{或夜}{ある|よ}
\ruby{體溫表}{たい|おん|へう}を
\ruby{記}{しる}し
\ruby{{\換字{終}}}{をは}れる
\ruby{次}{ついで}に、
%
\ruby{其表}{その|へう}をつく〴〵
\ruby{眺}{なが}めながら、

\原本頁{}
『マア
\ruby{宜}{よ}かつた
\ruby{事}{こと}、
%
もう
\ruby{如是}{こ|う}いふ
\ruby{樣子}{やう|す}になつて
\ruby{來}{く}れば
\ruby{心配}{しん|ぱい}は
\ruby{無}{な}い。
%
\ruby{一時}{いち|じ}はほんとに
\ruby{何樣}{ど|う}なるかと
\ruby{思}{おも}つたけれど、
%
マア
\ruby{患者}{くわん|じや}さんも
\ruby{幸福}{しあ|はせ}、%「幸福」ここは「は」
%
\ruby{私}{わたし}も
\ruby{幸福}{しあ|はせ}で、%「幸福」ここは「は」
%
\ruby{患者}{くわん|じや}さんは
\ruby{辛棒}{しん|ぼう}% 「辛抱」の誤植かもしれないがそのまま
\ruby{甲{\換字{斐}}}{が|ひ}があり、
%
\ruby{私}{わたし}は
\ruby{看護甲{\換字{斐}}}{かん|ご|が|ひ}がある
\ruby{事}{こと}になつて、
%
\ruby{相良}{さが|ら}さんに
\ruby{對}{むか}つても
\ruby{面目}{めん|ぼく}がある!。
』

\原本頁{}
と
\ruby{獨語}{ひとり|ご}ち、
%
\ruby{{\換字{又}}}{また}、
%
\ruby[g]{吉右衛門}{きちゑもん}に
\ruby{命}{いひつ}けられて
お
\ruby{澤}{さは}が
\ruby{許}{もと}にありて
\ruby{人々}{ひと|〴〵}が
\ruby{爲}{ため}に
\ruby{雜事}{ざつ|じ}の
\ruby{勞}{らう}を
\ruby{執}{と}れる
\ruby{下婢}{か|ひ}の
お
\ruby{鹽}{しほ}も、

\原本頁{}
『
\ruby[g]{水野}{みづの}さんの
\ruby{念力}{おも|ひ}だけでも
\ruby{治癒}{な|ほ}ると
\ruby{人}{ひと}が
\ruby{言}{い}つたゞが、
%
ほんに
\ruby{可怖}{おつ|かな}いもんだ!とう〳〵
\ruby{治癒}{な|ほ}るだあ。
%
\ruby{病}{やまひ}の
\ruby{高}{こう}じた
\ruby{時}{とき}あハア、
%
\ruby{何樣}{ど|う}しても
\ruby{彼世}{あの|よ}へ
\ruby{辷}{すべ}り
\ruby{{\換字{込}}}{こ}みさうな
\ruby{樣}{やう}な
\ruby{顏}{かほ}を
\ruby{仕}{し}て
\ruby{御座}{ご|ざ}つた
\ruby{彼}{あ}の
\ruby{人}{ひと}{---}{---}
\ruby{彼}{あ}の
\ruby{危}{あぶな}かつた
\ruby{人}{ひと}を
\ruby{取}{と}り
\ruby{止}{と}めることが
\ruby{出來}{で|き}たかと
\ruby{思}{おも}ふと
\ruby{不思議}{ふ|し|ぎ}でならない。
%
おらあハア
\ruby{始}{はじ}めて
\ruby{人}{ひと}の
\ruby{念力}{ねん|りき}といふ
\ruby{可怖}{おつ|かな}いものを
\ruby{目}{め}の
\ruby{{\換字{前}}}{まへ}に
\ruby{見}{み}て
\ruby{魂{\換字{消}}}{たま|げ}た。
%
\ruby{醫者業}{い|しや|わざ}ぢやあ
\ruby{無}{な}いだ、
%
\ruby{全}{まつた}く
\ruby{醫者業}{い|しや|わざ}ぢやあ
\ruby{無}{な}いだ!。
』

\原本頁{}
と
\ruby{下司}{げ|す}の
\ruby{常}{つね}とて
\ruby{言葉}{こと|ば}こそ
\ruby{多}{おほ}けれ、
%
これもまた
\ruby[g]{五十子}{いそこ}が
\ruby{囘復}{くわい|ふく}を% 原本通り「囘」
\ruby{悅}{よろこ}べる
\ruby{數}{かず}には
\ruby{洩}{も}れぬに、
%
たゞ
\ruby{彼}{か}の
\ruby{{\換字{強}}慾}{がう|よく}の
お
\ruby{澤}{さは}
\ruby{婆}{ばゞ}のみは、

\原本頁{}
『
\ruby{生}{い}きたつて
\ruby{面白}{おも|しろ}いとも
\ruby{定}{きま}つて
\ruby{居}{ゐ}ない
\ruby{世}{よ}の
\ruby{中}{なか}に、
%
とう〳〵
\ruby{彼}{あ}の
\ruby{人}{ひと}も
\ruby{生殘}{いき|のこ}つたやうだ!。
%
まだ
\ruby{業}{ごふ}が
\ruby{滅}{めつ}しないので
\ruby{死}{し}ねないと
\ruby{見}{み}えるだ!。
%
\ruby[g]{水野}{みづの}の
\ruby{世話}{せ|わ}で
\ruby{死}{し}なゝかつた
\ruby{{\換字{丈}}}{だけ}に、
%
\ruby{却}{かへ}つて
\ruby{今後}{これ|から}が
\ruby{面倒}{めん|だう}らしい。
%
\ruby{無錢}{た|ゞ}で
\ruby{買}{か}へるものは
\ruby{一}{ひと}つも
\ruby{無}{な}いだ!、
%
\ruby{借}{かり}は
\ruby{{\換字{返}}}{かへ}さずには
\ruby{眩度}{きつ|と}
\ruby{濟}{す}まないだ!。
%
\ruby{物}{もの}を
\ruby{取}{と}れば
\ruby{代}{かは}りを
\ruby{與}{や}る、
%
\ruby{借}{か}りた
\ruby{茶}{ちや}は
\ruby{茶}{ちや}で
\ruby{{\換字{返}}}{かへ}す、
%
\ruby{酒}{さけ}は
\ruby{酒}{さけ}で
\ruby{{\換字{返}}}{かへ}す!。
%
\ruby{人}{ひと}の
\ruby{親切}{しん|せつ}は
\ruby{何}{なん}で
\ruby{{\換字{返}}}{かへ}す?。
%
\ruby{生命}{いの|ち}の
\ruby{恩}{おん}は
\ruby{何}{なん}で
\ruby{{\換字{返}}}{かへ}す?。
%
\ruby{生}{い}きたが
\ruby{彼}{あ}の
\ruby{人}{ひと}の
\ruby{幸福}{しあ|はせ}だか%「幸福」ここは「は」
\ruby{何樣}{ど|う}だか?。
%
\ruby{病氣}{びやう|き}は
\ruby{無}{な}くなつたゞらうが、
%
\ruby{可厭}{い|や}なものが
\ruby{殘}{のこ}らう!。
%
\ruby{死損}{しに|そこな}つて
\ruby{氣}{き}の
\ruby{毒}{どく}の
\ruby{樣}{やう}な!。

\原本頁{}
\ruby{治}{なほ}つてから
\ruby{彼}{あ}の
\ruby{人}{ひと}が
\ruby{何樣}{ど|ん}な
\ruby{氣持}{き|もち}がさつしやらうかサ!。
%
\ruby{業}{ごふ}が
\ruby{盡}{つ}きないだ、
%
\ruby{業}{ごふ}が
\ruby{殘}{のこ}つたゞ!、
%
\ruby{何癒}{なに|なほ}ることが
\ruby{芽出度}{め|で|た}いに
\ruby{決}{きま}るかい!。
』

\原本頁{}
と、
%
\ruby{頻}{しき}りに
\ruby{松之助}{まつ|の|すけ}やら
\ruby{看護{\換字{婦}}}{かん|ご|ふ}やらの
\ruby{尾}{を}に
\ruby{從}{つ}いて
\ruby{悅}{よろこ}べる
お
\ruby{鹽}{しほ}に
\ruby{對}{むか}つて、
%
\ruby{例}{れい}の
\ruby{如}{ごと}く
\ruby{憎}{にく}さげに
\ruby{冷笑}{あざ|わら}ひて
\ruby{言}{い}ひ
\ruby{聞}{き}かせたり。
