\Entry{其十}

\ruby{凝}{こ}れるものを
\ruby{觀}{み}れば
\ruby{石}{いし}あり
\ruby{璧}{たま}あり。
\ruby{生}{お}ふるものを
\ruby{觀}{み}れば
\ruby{雜草}{ざつ|さう}あり
\ruby{百合}{ゆ|り}あり。
\ruby{同}{おな}じ
\ruby{人間}{ひ|と}にも、
\ruby{一生}{いつ|しやう}おろかしく
\ruby[g]{衣食}{いゝし}のために
\ruby{{\換字{逐}}}{お}ひ
\ruby{使}{つか}はれて、
\ruby{{\換字{猶}}}{なほ}
\ruby{其}{そ}の
\ruby{足}{た}らざるを
\ruby{憂}{うれ}ふる
\ruby{額}{ひたひ}の
\ruby{皺}{しわ}を
\ruby{深々}{ふか|〳〵}と
\ruby{疊}{たゝ}み、おのが
\ruby{働}{はたら}きの
\ruby{無}{な}きは
\ruby{省}{かへり}みずに、
\ruby{他人}{ひ|と}を
\ruby{恨}{うら}み
\ruby{世}{よ}を
\ruby{謗}{そし}りて
\ruby{甲斐無}{か|ひ|な}く
\ruby{悶}{もだ}えながら
\ruby{老境}{お|い}に
\ruby{入}{い}るもあり、
\ruby{{\換字{又}}生}{また|うま}れつきの
\ruby{心}{こゝろ}の
\ruby{{\換字{丈}}高}{たけ|たか}く
\ruby{胸}{むね}の
\ruby{海濶}{うみ|ひろ}くして、
\ruby{此}{こ}のむづかしき
\ruby{世}{よ}に
\ruby{身}{み}の
\ruby{取}{と}り
\ruby{置}{お}き
\ruby{拙}{つたな}からず、
\ruby{憂}{う}さも
\ruby{苦}{くる}しさも、するりと
\ruby{切}{き}り
\ruby{拔}{ぬ}けて、
\ruby{屈託}{くつ|たく}せぬ
\ruby{顏色}{かほ|つき}の
\ruby{何時}{い|つ}も
\ruby{若々}{わか|〳〵}と、
\ruby{雲}{くも}より
\ruby{上}{うへ}に
\ruby{居}{ゐ}る
\ruby{月}{つき}の、
\ruby{澄}{すま}し
\ruby{{\換字{返}}}{かへ}つて
\ruby{暮}{くら}すやうなる
\ruby{優}{すぐ}れ
\ruby{者}{もの}もあるなり。

お
\ruby{龍}{りう}は
\ruby{自己}{お|の}が
\ruby{身}{み}の
\ruby{上}{うへ}の
\ruby{今}{いま}の
\ruby{果敢無}{は|か|な}さを
\ruby{羞}{はじ}らひて、
\ruby{我}{わ}が
\ruby{口}{くち}より
\ruby{我}{わ}が
\ruby{友}{とも}なりとは
\ruby{憚}{はゞか}りて
\ruby{云}{い}はねど、
\ruby[g]{彼方}{かなた}は
\ruby{何處}{ど|こ}までも
\ruby[g]{隔意無}{へだてごゝろな}く、お
\ruby{龍}{りう}を
\ruby{友}{とも}とも
\ruby{妹}{いもと}とも
\ruby[g]{待遇}{あしら}ひて、
\ruby[g]{親身}{しんみ}も
\ruby{及}{およ}ばず
\ruby{優}{やさ}しくするお
\ruby{彤}{とう}といへる
\ruby{一美人}{いち|び|じん}あり。

\ruby{叔母}{を|ば}が
\ruby{無理壓制}{む|り|おし|つけ}の
\ruby{婿取沙汰}{むこ|とり|ざ|た}を
\ruby{厭}{いと}ひて、
\ruby{駿府}{すん|ぷ}を
\ruby{{\換字{脱}}}{ぬ}け
\ruby{出}{い}でゝ
\ruby{東京}{とう|きやう}に
\ruby{來}{きた}りし
\ruby{時}{とき}、お
\ruby{龍}{りう}が
\ruby{先}{ま}づ
\ruby{頼}{たよ}りしは
\ruby{此女}{この|ひと}にして、お
\ruby{龍}{りう}と
\ruby{共}{とも}に
\ruby{淺草}{あさ|くさ}に
\ruby{{\換字{遊}}}{あそ}びし
\ruby[g]{日水野}{ひみづの}に
\ruby{{\換字{遇}}}{あ}ひて、
\ruby{水野}{みづ|の}をして
\ruby{其}{そ}の
\ruby{美}{び}に
\ruby{驚}{おどろ}かしめしも
\ruby{此女}{この|をんな}なりけるなり。

お
\ruby{彤}{とう}が
\ruby{身分}{み|ぶん}を
\ruby{問}{と}へば、
\ruby{世}{よ}に
\ruby{聞}{きこ}えたる
\ruby{一代分限}{いち|だい|ぶ|げん}の
\ruby[g]{筑波何某}{つくばなにがし}といへる
\ruby[g]{六十男}{むそをとこ}の
\ruby{外妾}{ぐわい|せう}に
\ruby{{\換字{過}}}{す}ぎぬなり。
\ruby{然}{さ}なり、
\ruby[g]{藥研堀附{\換字{近}}}{やげんぼりあたり}に
\ruby{數寄}{す|き}を
\ruby{凝}{こ}らせる
\ruby{家}{いへ}を
\ruby{構}{かま}へて、
\ruby{賑}{にぎ}やかなるが
\ruby{中}{なか}に
\ruby[g]{靜閑}{しづか}に
\ruby{暮}{くら}すほどの
\ruby{贅澤}{ぜい|たく}を
\ruby{縱}{ほしいまゝ}にし、
\ruby{美衣}{び|い}を
\ruby{纒}{まと}ひ
\ruby{美饌}{び|せん}を
\ruby{口}{くち}にし、
\ruby[g]{萬般幸福}{よろづしあわせ}に
\ruby{世}{よ}を
\ruby{經}{ふ}るとはいへ、
\ruby{實}{まこと}に
\ruby{其}{そ}の
\ruby{身分}{み|ぶん}を
\ruby{問}{と}へば
\ruby[g]{外妾}{めかけ}には
\ruby{{\換字{過}}}{す}ぎぬなり。

されどお
\ruby{彤}{とう}は
\ruby{人}{ひと}の
\ruby{正室}{つ|ま}たるを
\ruby{得}{え}ざるが
\ruby{故}{ゆゑ}に
\ruby{身}{み}を
\ruby[g]{日陰者}{ひかげもの}の
\ruby{其位}{そ|れ}に
\ruby{安}{やす}んぜるにはあらず。
\ruby{今}{いま}を
\ruby{去}{さ}ること
\ruby{七年}{しち|ねん}ほど
\ruby{前}{まへ}の
\ruby{事}{こと}なりき。
\ruby[g]{筑波}{つくば}が
\ruby{其}{そ}の
\ruby{正妻}{つ|ま}を
\ruby{失}{うしな}ひし
\ruby{時}{とき}、
\ruby{面}{おもて}の
\ruby{美}{うつく}しさばかりに
\ruby{{\換字{迷}}}{まよ}ひ
\ruby{溺}{おぼ}るゝがごとき
\ruby[g]{痴漢}{おろかもの}ならぬ
\ruby[g]{筑波}{つくば}は、よく〳〵
\ruby{見定}{み|さだ}めたるところやありけん、お
\ruby{彤}{とう}を
\ruby{引上}{ひき|あ}げて
\ruby{正室}{つ|ま}とせんとは
\ruby{云}{い}ひたりしなり。
されば
\ruby{其時}{その|とき}お
\ruby{彤}{とう}にして
\ruby{{\換字{強}}}{し}ひて
\ruby{辭}{いな}み
\ruby{立}{だて}だにせざりしならば、
\ruby{今}{いま}は
\ruby{此}{こ}の
\ruby{世}{よ}の
\ruby{表面}{おも|て}に
\ruby{立}{た}ちて、
\ruby[g]{立派}{りつぱ}に
\ruby[g]{筑波夫人}{つくばふじん}と
\ruby{崇}{あが}め
\ruby{仰}{あふ}がれ、
\ruby{夫}{おつと}の
\ruby{勢力}{せい|りよく}の
\ruby{及}{およ}べる
\ruby[g]{境域}{さかひ}には
\ruby[g]{反身}{そりみ}になりて
\ruby{誇}{ほこ}りて
\ruby[g]{生活}{くら}すことの
\ruby{叶}{かな}ふべき
\ruby{筈}{はず}なるを、
\ruby{我}{われ}から
\ruby{我}{わ}が
\ruby{出世}{しゆ|つせ}を
\ruby{遮}{さへぎ}り
\ruby{止}{とど}めて
\ruby{今}{いま}も
\ruby[g]{{\換字{猶}}外妾}{なほめかけ}たるなり。

\ruby[g]{筑波}{つくば}が
\ruby{引上}{ひき|あ}げて
\ruby{正室}{つ|ま}とせんと
\ruby{云}{い}ひし
\ruby{時}{とき}、お
\ruby{彤}{とう}は
\ruby{如何}{い|か}なる
\ruby{意}{こゝろ}にて
\ruby{之}{これ}を
\ruby{辭}{いな}みしか
\ruby{知}{し}らず。
されど
\ruby{其}{そ}の
\ruby{外}{そと}に
\ruby{現}{あら}はれたるところにては、お
\ruby{彤}{とう}は
\ruby[g]{一向謹}{ひたすらつゝし}み
\ruby{愼}{つゝし}みて、

『
\ruby{妾}{わたし}を
\ruby{引上}{ひき|あ}げて
\ruby{下}{くだ}さらうとい
\ruby[g]{御思召}{おぼしめし}は
\ruby{嬉}{うれ}しうございますが、
\ruby{妾}{わたし}は
\ruby{實家}{さ|と}も
\ruby{無}{な}く
\ruby{後楯}{うしろ|だて}も
\ruby{無}{な}い
\ruby{身}{み}ですから、
\ruby[g]{左樣仰}{さうおつし}あつて
\ruby{下}{くだ}さるから
\ruby{好}{い}いはで
\ruby{成}{な}り
\ruby{上}{あが}りましたら、
\ruby{人}{ひと}の
\ruby{謗}{そし}り
\ruby{嘲}{あざけ}りは
\ruby{何}{ど}の
\ruby{樣}{やう}でございましやう。
\ruby{其}{それ}も
\ruby{妾}{わたし}が
\ruby{惡}{わる}く
\ruby{云}{い}はれるだけで
\ruby{濟}{す}めば
\ruby{宜}{よ}うございますが、
\ruby{針}{はり}ほどの
\ruby{事}{こと}も
\ruby{棒}{ぼう}ほどに
\ruby{云}{い}ひたがる
\ruby{人}{ひと}の
\ruby{口}{くち}ですもの
\ruby{何}{なん}ぞの
\ruby{折}{をり}には
\ruby{妾}{わたし}のことを
\ruby{云}{い}ひ
\ruby{出}{だ}して、
\ruby{彼樣}{あ|ん}なものを
\ruby{引上}{ひき|あ}げたのは
\ruby{何事}{なに|ごと}だと、
\ruby[g]{屹度貴下}{きつとあなた}を
\ruby{惡}{わる}く
\ruby{云}{い}はずには
\ruby{居}{を}りません。
よし
\ruby{何}{なに}を
\ruby{人}{ひと}が
\ruby{云}{い}つたつて
\ruby{氣}{き}になさるほどの
\ruby{{\換字{弱}}}{よわ}い
\ruby[g]{貴下}{あなた}では
\ruby{無}{な}くつても、
\ruby{妾}{わたし}の
\ruby{所爲}{せ|ゐ}で
\ruby[g]{貴下}{あなた}の
\ruby{金箔}{は|く}を
\ruby{剝{\換字{脱}}}{お|と}すのは
\ruby{妾}{わたし}は
\ruby{{\換字{嫌}}}{いや}です。
どうせ
\ruby{今}{いま}まで
\ruby[g]{日陰者}{ひかげもの}で
\ruby{濟}{す}まして
\ruby{來}{き}た
\ruby{妾}{わたし}ですもの、いつそ
\ruby[g]{一生日陰者}{いつしやうひかげもの}で
\ruby{濟}{す}まして
\ruby{{\換字{終}}}{しま}つて、
\ruby{人}{ひと}に
\ruby{目角}{め|かど}を
\ruby{立}{た}てられずに
\ruby[g]{生活}{くら}した
\ruby{方}{はう}が
\ruby{性}{しやう}に
\ruby{合}{あ}ひさうです。
\ruby[g]{貴下}{あなた}さへ
\ruby{見棄}{み|す}てゝ
\ruby{下}{くだ}さらなければ、
\ruby{自分}{じ|ぶん}が
\ruby{出世}{しゆ|つせ}して
\ruby[g]{貴下}{あなた}を
\ruby{惡}{わる}く
\ruby{云}{い}はせやう
\ruby{氣}{き}はございません。
』

と、いと
\ruby{眞面目}{ま|じ|め}に
\ruby{道理正}{だう|り|たゞ}しく
\ruby{斷}{ことわ}れるのみか、
\ruby{扨打解}{さて|うち|と}けて
\ruby{碎}{くだ}けて
\ruby{笑}{わら}ふ
\ruby{醉}{よひ}の
\ruby{後}{あと}などには、
\ruby{面}{めん}と
\ruby{對}{むか}ひて
\ruby{{\換字{遠}}慮}{ゑん|りよ}も
\ruby{無}{な}く
\ruby{直接}{うち|つけ}に、

『
\ruby{正室}{おく|さま}になりやあ
\ruby{正室}{おく|さま}だけの
\ruby{荷}{に}を
\ruby{背負}{し|よ}はなけりやあなりませんからネ。
\ruby{力}{ちから}の
\ruby{無}{な}い
\ruby{妾}{わたし}が
\ruby{其樣}{そ|ん}な
\ruby{事}{こと}を
\ruby{仕}{し}て
\ruby{肩}{かた}を
\ruby{凝}{こ}らすよりやあ、
\ruby{氣樂}{き|らく}にして
\ruby{斯樣}{か|う}して
\ruby{居}{ゐ}る
\ruby{方}{はう}がマア
\ruby{宜}{よ}さゝうですから。
』

と
\ruby{云}{い}ひて
\ruby{肯}{うけが}はず。
\ruby{乘}{の}らば
\ruby{乘}{の}るべかりし
\ruby{玉}{たま}の
\ruby{輿}{こし}を
\ruby{自}{みづか}ら
\ruby{棄}{す}てゝ
\ruby{吝}{おし}まざりしかば、
\ruby[g]{某子爵}{なにがしゝしやく}の
\ruby{姫君}{ひめ|ぎみ}は
\ruby[g]{筑波}{つくば}の
\ruby{妻}{つま}として
\ruby{今}{いま}の
\ruby{榮華}{えい|ぐわ}を
\ruby{受}{う}け
\ruby{得}{え}たまふに
\ruby{至}{いた}りしなり。

されば
\ruby[g]{筑波}{つくば}はお
\ruby{彤}{とう}を
\ruby{日陰者}{ひ|かげ|もの}として
\ruby{世}{よ}にこそ
\ruby{隱}{かく}し
\ruby{居}{を}れ、
\ruby{之}{これ}を
\ruby{愛}{め}で
\ruby{重}{おも}んずることは
\ruby{今}{いま}の
\ruby{正室}{つ|ま}にも
\ruby{勝}{まさ}れり。

お
\ruby{彤}{とう}は
\ruby{是}{かく}の
\ruby{如}{ごと}くにして
\ruby{此}{こ}の
\ruby{世}{よ}にたゞ
\ruby[g]{一人}{ひとり}の
\ruby[g]{筑波}{つくば}の
\ruby{意}{こゝろ}を
\ruby{失}{うしな}はざらんとする
\ruby{外}{ほか}には、
\ruby{何}{なん}の
\ruby{心}{こゝろ}を
\ruby{用}{もち}ひ
\ruby{氣}{き}を
\ruby{勞}{つか}らすことも
\ruby{無}{な}く、
\ruby{年}{とし}の
\ruby{首}{はじめ}より
\ruby{年}{とし}の
\ruby{尾}{をはり}まで、
\ruby{身}{み}の
\ruby[g]{周圍}{まはり}の
\ruby{物}{もの}より
\ruby{庭}{には}の
\ruby{隅}{すみ}の
\ruby[g]{草木}{くさき}まで、
\ruby{一切}{いつ|さい}を
\ruby{榮華}{えい|ぐわ}の
\ruby{頂上}{てつ|ぺん}の
\ruby[g]{仕度三昧}{したいざんまい}に
\ruby{振舞}{ふる|ま}ひて、
\ruby{誰}{たれ}に
\ruby[g]{苦情}{くじやう}を
\ruby{云}{い}はるゝことも
\ruby{無}{な}く
\ruby{日}{ひ}を
\ruby{{\換字{過}}}{す}ごせるなり。

