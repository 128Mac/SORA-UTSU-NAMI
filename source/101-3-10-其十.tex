\Entry{其十}

% メモ 校正終了 2024-05-12
\原本頁{51-3}%
\ruby{凝}{こ}れる
ものを
\ruby{觀}{み}れば
\ruby{石}{いし}あり
\ruby{璧}{たま}あり。
%
\ruby{生}{お}ふる
ものを
\ruby{觀}{み}れば
\ruby{雜草}{ざつ|さう}あり
\ruby{百合}{ゆ|り}あり。
%
\ruby{同}{おな}じ
\ruby{人間}{ひ|と}にも、
%
\ruby[||j>]{一}{いつ}
\ruby[||j>]{生}{しやう}
% \ruby{一生}{いつ|しやう}
おろかしく
\ruby[|g|]{衣食}{い・し}% (いいし)では??ので「衣(い)」「食(し)」の並列化のため(・)に見える
のために
\ruby{{\換字{逐}}}{お}ひ
\原本頁{51-5}\改行%
\ruby{使}{つか}はれて、
%
\ruby{{\換字{猶}}}{なほ}
\ruby{其}{そ}の
\ruby{足}{た}らざるを
\ruby{憂}{うれ}ふる
\ruby{額}{ひたひ}の
\ruby{皺}{しわ}を
\ruby{深々}{ふか|〳〵}と
\ruby{疊}{たゝ}み、
%
おのが
\ruby{働}{はたら}きの
\ruby{無}{な}きは
\ruby{省}{かへり}みずに、
%
\ruby{他人}{ひ|と}を
\ruby{恨}{うら}み
\ruby{世}{よ}を
\ruby{謗}{そし}りて
\ruby{甲{\換字{斐}}}{か|ひ}
\ruby{無}{な}く
\ruby{悶}{もだ}え
ながら
\ruby{老境}{お|い}に
\ruby{入}{い}る
もあり、
%
\ruby{{\換字{又}}}{また}
\ruby{生}{うま}れ
つきの
\ruby{心}{こゝろ}の
\ruby{{\換字{丈}}}{たけ}
\ruby{高}{たか}く
\ruby{胸}{むね}の
\ruby{海}{うみ}
\ruby{濶}{ひろ}くして、
%
\ruby{此}{こ}の
むづかしき
\ruby{世}{よ}に
\ruby{身}{み}の
\ruby{取}{と}り
\ruby{置}{お}き
\ruby{拙}{つたな}からず、
%
\ruby{憂}{う}さも
\ruby{苦}{くる}しさも、
%
するりと
\ruby{切}{き}り
\ruby{拔}{ぬ}けて、
%
\ruby{屈託}{くつ|たく}せぬ
\ruby{顏色}{かほ|つき}の
\ruby{何時}{い|つ}も
\ruby{{\換字{若}}々}{わか|〳〵}と、
%
\ruby{雲}{くも}より
\ruby{上}{うへ}に
\ruby{居}{ゐ}る
\ruby{月}{つき}の、
%
\ruby{澄}{すま}し
\ruby{{\換字{返}}}{かへ}つて
\ruby{暮}{くら}す
やうなる
\ruby{優}{すぐ}れ
\ruby{者}{もの}も
あるなり。

\原本頁{52-1}%
お
\ruby{龍}{りう}は
\ruby{自己}{お|の}が
\ruby{身}{み}の
\ruby{上}{うへ}の
\ruby{今}{いま}の
\ruby{果敢無}{は|か|な}さを
\ruby{羞}{はぢ}らひて、
%
\ruby{我}{わ}が
\ruby{口}{くち}より
\ruby{我}{わ}が
\ruby{友}{とも}なり
とは
\ruby{憚}{はゞか}りて% 「憚 は(ゞ)か」
\ruby{云}{い}はねど、
%
\ruby{彼方}{かな|た}は
\ruby{何處}{ど|こ}までも
\ruby[<j||]{隔}{へだて}
\ruby[||j>]{意}{ごゝろ}
\ruby[||j>]{無}{ な }く、
%
お
\ruby{龍}{りう}を
\ruby{友}{とも}とも
\ruby{妹}{いもと}とも
\ruby[|g|]{待{\換字{遇}}}{あしら}ひて、
%
\ruby{親身}{しん|み}も
\ruby{及}{およ}ばず
\ruby{優}{やさ}しくする
お
\ruby{彤}{とう}
といへる
\ruby{一}{いち}
\ruby{美人}{び|じん}あり。

\原本頁{52-5}%
\ruby{叔母}{を|ば}が
\ruby{無理}{む|り}
\ruby{壓制}{おし|つけ}の
\ruby{婿}{むこ}% (婿 5a7f) 聟 805f
\ruby{取沙汰}{とり|ざ|た}を
\ruby{厭}{いと}ひて、
%
\ruby{駿府}{すん|ぷ}を
\ruby{脫}{ぬ}け
\ruby{出}{い}でゝ
\ruby[||j>]{東}{とう}
\ruby[||j>]{京}{きやう}に
% \ruby{東京}{とう|きやう}に
\ruby{來}{きた}りし
\ruby{時}{とき}、
%
お
\ruby{龍}{りう}が
\ruby{先}{ま}づ
\ruby{頼}{たよ}りしは
\ruby{此}{この}
\ruby{女}{ひと}
にして、
%
お
\ruby{龍}{りう}と
\ruby{共}{とも}に
\ruby{淺草}{あさ|くさ}に
\ruby{{\換字{遊}}}{あそ}びし
\ruby{日}{ひ}
\ruby{水野}{みづ|の}に
\ruby{{\換字{遇}}}{あ}ひて、
%
\ruby{水野}{みづ|の}をして
\ruby{其}{そ}の
\ruby{美}{び}に
\ruby{驚}{おどろ}かし
めしも
\ruby[||j>]{此}{この}
\ruby[||j>]{女}{をんな}
% \ruby{此女}{この|をんな}
なりけるなり。

\原本頁{52-9}%
お
\ruby{彤}{とう}が
\ruby{身{\換字{分}}}{み|ぶん}を
\ruby{問}{と}へば、
%
\ruby{世}{よ}に
\ruby{聞}{きこ}えたる
\ruby{一代{\換字{分}}限}{いち|だい|ぶ|げん}の
\ruby{筑波}{つく|ば}
\ruby{何某}{なに|がし}
といへる
\ruby{六十男}{む|そ |をとこ}の
\ruby[||j>]{外}{ぐわい}
\ruby[||j>]{妾}{ せう}に
% \ruby{外妾}{ぐわい|せう}に
\ruby{{\換字{過}}}{す}ぎぬ
なり。
%
\ruby{然}{さ}なり、
%
\ruby{藥研堀}{や|げん|ぼり}
\ruby{附{\換字{近}}}{あた|り}に
\ruby{數寄}{す|き}を
\ruby{凝}{こ}らせる
\ruby{家}{いへ}を
\ruby{構}{かま}へて、
%
\ruby{賑}{にぎ}やか
なるが
\ruby{中}{なか}に
\ruby{靜閑}{しづ|か}に
\ruby{暮}{くら}す
ほどの
\ruby{贅澤}{ぜい|たく}を
\原本頁{53-1}\改行%
\ruby[|-|]{縱}{ほしいまゝ}にし、
%
\ruby{美衣}{び|い}を
\ruby{纒}{まと}ひ
\ruby{美饌}{び|せん}を
\ruby{口}{くち}にし、
%
\ruby[|g|]{萬般}{よろづ}
\ruby{幸福}{しあ|はせ}に%「幸福」ここは「は」
\ruby{世}{よ}を
\ruby{經}{ふ}る
とはいへ、
%
\ruby{實}{まこと}に
\ruby{其}{そ}の
\ruby{身{\換字{分}}}{み|ぶん}を
\ruby{問}{と}へば
\ruby{外妾}{めか|け}には
\ruby{{\換字{過}}}{す}ぎぬ
なり。

\原本頁{53-3}%
されど
お
\ruby{彤}{とう}は
\ruby{人}{ひと}の
\ruby{正室}{つ|ま}たるを
\ruby{得}{え}ざるが
\ruby{故}{ゆゑ}に
\ruby{身}{み}を
\ruby{日陰者}{ひ|かげ|もの}の
\ruby{其位}{そ|れ}に
\ruby{安}{やす}んぜる
にはあらず。
%
\ruby{今}{いま}を
\ruby{去}{さ}ること
\ruby{七年}{なな|ねん}ほど
\ruby{{\換字{前}}}{まへ}の
\ruby{事}{こと}なりき。
%
\ruby{筑波}{つく|ば}が
\ruby{其}{そ}の
\ruby{正妻}{つ|ま}を
\ruby{失}{うしな}ひし
\ruby{時}{とき}、
%
\ruby{面}{おもて}の
\ruby{美}{うつく}しさ
ばかりに
\ruby{{\換字{迷}}}{まよ}ひ
\ruby{溺}{おぼ}るゝ
がごとき
\ruby[||j>]{痴}{おろか}
\ruby[||j>]{{\換字{漢}}}{ もの}
% \ruby{痴{\換字{漢}}}{おろか|もの}
ならぬ
\ruby{筑波}{つく|ば}は、
%
よく〳〵
\ruby{見}{み}
\ruby{定}{さだ}め
たる
ところや
ありけん、
%
\原本頁{53-7}\改行%
お
\ruby{彤}{とう}を
\ruby{引上}{ひき|あ}げて
\ruby{正室}{つ|ま}と
せんとは
\ruby{云}{い}ひたり
しなり。
%
されば
\ruby{其}{その}
\ruby{時}{とき}
お
\ruby{彤}{とう}にして
\ruby{{\換字{強}}}{し}ひて
\ruby{辭}{いな}み
\ruby{立}{だて}
だに
せざりし
ならば、
%
\ruby{今}{いま}は
\ruby{此}{こ}の
\ruby{世}{よ}の
\ruby[|g|]{表面}{おもて}に
\ruby{立}{た}ちて、
%
\ruby{立派}{りつ|ぱ}に
\ruby{筑波}{つく|ば}
\ruby{夫人}{ふ|じん}と
\ruby{崇}{あが}め
\ruby{仰}{あふ}がれ、
%
\ruby{夫}{おつと}の
\ruby[||j>]{勢}{せい}
\ruby[||j>]{力}{りよく}の
% \ruby{勢力}{せい|りよく}の
\ruby{及}{およ}べる
\ruby{境域}{さか|ひ}には
\ruby{反身}{そり|み}に
なりて
\ruby{誇}{ほこ}りて
\ruby{生活}{く|ら}す
ことの
\ruby{叶}{かな}ふべき
\ruby{筈}{はず}なるを、
%
\ruby{我}{われ}から
\ruby{我}{わ}が
\ruby{出世}{しゆつ|せ}を
\ruby{{\換字{遮}}}{さへぎ}り
\ruby{止}{とゞ}めて
\ruby{今}{いま}も
\ruby{{\換字{猶}}}{なほ}
\ruby{外妾}{めか|け}
たる
なり。

\原本頁{54-1}%
\ruby{筑波}{つく|ば}が
\ruby{引上}{ひき|あ}げて
\ruby{正室}{つ|ま}と
せんと
\ruby{云}{い}ひし
\ruby{時}{とき}、
%
お
\ruby{彤}{とう}は
\ruby{如何}{い|か}なる
\ruby{意}{こゝろ}にて
\ruby{之}{これ}を
\ruby{辭}{いな}みしか
\ruby{知}{し}らず。
%
されど
\ruby{其}{そ}の
\ruby{外}{そと}に
\ruby{現}{あら}はれたる
ところ
にては、
%
お
\ruby{彤}{とう}は
\ruby{一向}{ひた|すら}
\ruby{謹}{つゝし}み
\ruby{愼}{つゝし}みて、

\原本頁{54-4}%
『
\ruby{妾}{わたし}を
\ruby{引上}{ひき|あ}げて
\ruby{下}{くだ}さらう
といふ
\ruby{御思召}{お|ぼし|めし}は
\ruby{嬉}{うれ}しう
ございますが、
%
\原本頁{54-5}\改行%
\ruby{妾}{わたし}は
\ruby{實家}{さ|と}も
\ruby{無}{な}く
\ruby[||j>]{後}{うしろ}
\ruby[||j>]{楯}{ だて}も
% \ruby{後楯}{うしろ|だて}も
\ruby{無}{な}い
\ruby{身}{み}
ですから、
%
\ruby{左樣}{さ|う}
\ruby{仰}{おつし}あつて
\ruby{下}{くだ}さるから
\ruby{好}{い}いはで
\ruby{成}{な}り
\ruby{上}{あが}り
ましたら、
%
\ruby{人}{ひと}の
\ruby{謗}{そし}り
\ruby{嘲}{あざけ}りは
\ruby{何}{ど}の
\ruby{樣}{やう}で
ございましやう。
%
\ruby{其}{それ}も
\ruby{妾}{わたし}が
\ruby{惡}{わる}く
\ruby{云}{い}はれる
だけで
\ruby{濟}{す}めば
\ruby{宜}{よ}う
ございますが、
%
\ruby{針}{はり}ほどの
\ruby{事}{こと}も
\ruby{棒}{ぼう}ほどに
\ruby{云}{い}ひたがる
\ruby{人}{ひと}の
\ruby{口}{くち}ですもの、
%
\ruby{何}{なん}ぞの
\原本頁{54-9}\改行%
\ruby{折}{をり}には
\ruby{妾}{わたし}の
ことを
\ruby{云}{い}ひ
\ruby{出}{だ}して、
%
\ruby{彼樣}{あ|ん}な
ものを
\ruby{引上}{ひき|あ}げたのは
\ruby{何事}{なに|ごと}
だと、
%
\ruby{屹度}{きつ|と}
\ruby[|g|]{貴下}{あなた}を
\ruby{惡}{わる}く
\ruby{云}{い}はずには
\ruby{居}{を}りません。
%
よし
\ruby{何}{なに}を
\ruby{人}{ひと}が
\ruby{云}{い}つたつて
\ruby{氣}{き}に
なさる
ほどの
\ruby{{\換字{弱}}}{よわ}い
\ruby[|g|]{貴下}{あなた}では
\ruby{無}{な}く
つても、
%
\ruby{妾}{わたし}の
\ruby{{\換字{所}}爲}{せ|ゐ}で
\ruby[|g|]{貴下}{あなた}の
\ruby{金箔}{は|く}を
\ruby{剝脫}{お|と}すのは
\ruby{妾}{わたし}は
\ruby{{\換字{嫌}}}{いや}です。
%
どうせ
\ruby{今}{いま}まで
\ruby{日陰者}{ひ|かげ|もの}で
\原本頁{55-2}\改行%
\ruby{濟}{す}まして
\ruby{來}{き}た
\ruby{妾}{わたし}ですもの、
%
いつそ
\ruby[||j>]{一}{いつ}
\ruby[||j>]{生}{しやう}
% \ruby{一生}{いつ|しやう}
\ruby{日陰者}{ ひ|かげ|もの}で
\ruby{濟}{す}まして
\ruby{{\換字{終}}}{しま}つて、
%
\原本頁{55-3}\改行%
\ruby{人}{ひと}に
\ruby{目角}{め|かど}を
\ruby{立}{た}てられずに
\ruby{生活}{く|ら}した
\ruby{方}{はう}が
\ruby{性}{しやう}に
\ruby{合}{あ}ひさうです。
%
\ruby[|g|]{貴下}{あなた}
さへ
\ruby{見}{み}
\ruby{棄}{す}てゝ
\ruby{下}{くだ}さらなければ、
%
\ruby{自{\換字{分}}}{じ|ぶん}が
\ruby{出世}{しゆつ|せ}して
\ruby[|g|]{貴下}{あなた}を
\ruby{惡}{わる}く
\ruby{云}{い}はせやう
\ruby{氣}{き}は
ございません。
』

\原本頁{55-6}%
と、
%
いと
\ruby{眞面目}{ま|じ|め}に
\ruby{{\換字{道}}理}{だう|り}
\ruby{正}{たゞ}しく
\ruby{斷}{ことわ}れる
のみか、
%
\ruby{扨}{さて}
\ruby{打}{うち}
\ruby{解}{と}けて
\ruby{碎}{くだ}けて
\ruby{笑}{わら}ふ
\ruby{醉}{よひ}の% 「醉」は原本通り「よ」で調整
\ruby{後}{あと}
など
には、
%
\ruby{面}{めん}と
\ruby{對}{むか}ひて
\ruby{{\換字{遠}}慮}{ゑん|りよ}も
\ruby{無}{な}く
\ruby{直接}{うち|つけ}に、

\原本頁{55-8}%
『
\ruby{正室}{おく|さま}
になりやあ
\ruby{正室}{おく|さま}
だけの
\ruby{荷}{に}を
\ruby{背負}{し|よ}はなけりやあ
なりませんからネ。
%
\ruby{力}{ちから}の
\ruby{無}{な}い
\ruby{妾}{わたし}が
\ruby{其樣}{そ|ん}な
\ruby{事}{こと}を
\ruby{仕}{し}て
\ruby{肩}{かた}を
\ruby{凝}{こ}らす
よりやあ、
%
\ruby{氣樂}{き|らく}にして
\ruby{斯樣}{か|う}して
\ruby{居}{ゐ}る
\ruby{方}{はう}が
マア
\ruby{宜}{よ}さゝう
ですから。
』

\原本頁{55-11}%
と
\ruby{云}{い}ひて
\ruby{肯}{うけが}はず。
%
\ruby{乘}{の}らば
\ruby{乘}{の}るべかりし
\ruby{玉}{たま}の
\ruby{輿}{こし}を
\ruby{自}{みづか}ら
\ruby{棄}{す}てゝ
\ruby{吝}{おし}まざりしかば、
%
\ruby[<j>]{某}{なにがし}
\ruby[||j>]{子}{ ゝ }
\ruby[||j>]{爵}{しやく}の
% \ruby{子爵}{ゝ|しやく}の
\ruby{姫君}{ひめ|ぎみ}は
\ruby{筑波}{つく|ば}の
\ruby{妻}{つま}として
\ruby{今}{いま}の
\ruby{榮華}{えい|ぐわ}を
\ruby{受}{う}け
\ruby{得}{え}
たまふに
\ruby{至}{いた}りしなり。

\原本頁{56-3}%
されば
\ruby{筑波}{つく|ば}は
お
\ruby{彤}{とう}を
\ruby{日陰者}{ひ|かげ|もの}
として
\ruby{世}{よ}にこそ
\ruby{隱}{かく}し
\ruby{居}{を}れ、
%
\ruby{之}{これ}を
\ruby{愛}{め}で
\ruby{重}{おも}んずる
ことは
\ruby{今}{いま}の
\ruby{正室}{つ|ま}にも
\ruby{{\換字{勝}}}{まさ}れり。

\原本頁{56-5}%
お
\ruby{彤}{とう}は
\ruby{是}{かく}の
\ruby{如}{ごと}く
にして
\ruby{此}{こ}の
\ruby{世}{よ}に
たゞ
\ruby{一人}{ひと|り}の
\ruby{筑波}{つく|ば}の
\ruby{意}{こゝろ}を
\ruby{失}{うしな}はざらん
とする
\ruby{外}{ほか}には、
%
\ruby{何}{なん}の
\ruby{心}{こゝろ}を
\ruby{用}{もち}ひ
\ruby{氣}{き}を
\ruby{勞}{つか}らすことも
\ruby{無}{な}く、
%
\ruby{年}{とし}の
\ruby[<j||]{首}{はじめ}より
\ruby{年}{とし}の
\ruby{尾}{をはり}まで、
%
\ruby{身}{み}の
\ruby[|g|]{周圍}{まはり}の
\ruby{物}{もの}より
\ruby{庭}{には}の
\ruby{隅}{すみ}の
\ruby{草木}{くさ|き}まで、
%
\ruby{一切}{いつ|さい}を
\原本頁{56-8}\改行%
\ruby{榮華}{えい|ぐわ}の
\ruby{頂上}{てつ|ぺん}の
\ruby{仕度}{し|たい}
\ruby{三昧}{ざん|まい}に
\ruby{振舞}{ふる|ま}ひて、
%
\ruby{誰}{たれ}に
\ruby{苦{\換字{情}}}{く|じやう}を
\ruby{云}{い}はるゝ
ことも
\ruby{無}{な}く
\ruby{日}{ひ}を
\ruby{{\換字{過}}}{す}ごせる
なり。
