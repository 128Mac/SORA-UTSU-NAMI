\Entry{其十一}

% メモ 校正終了 2024-05-12
\原本頁{57-1}%
\ruby{六疊}{ろく|でふ}の
\ruby{茶}{ちや}の
\ruby{間}{ま}、
%
\ruby{茶}{ちや}の
\ruby{間}{ま}
とは
いへ
\ruby{大抵}{たい|てい}の
\ruby{家}{いへ}の
\ruby{客室}{きやく|ま}より
\ruby{美}{うつく}しく、
%
\ruby[<j||]{柱}{はしら}より% 行末行頭の境界付近なので特例処置を施す
\ruby{敷居}{しき|ゐ}
\ruby{鴨居}{かも|ゐ}の
\ruby{木口}{き|ぐち}の
\ruby{結構}{けつ|こう}さ。
%
\ruby{格}{こ}の
\ruby{配}{くば}りに
\ruby{物}{もの}
\ruby{好}{ずき}を
\ruby{見}{み}せたる
\ruby{細骨}{ほそ|ぼね}の
\ruby{纖巧}{きや|しや}なる
\ruby{二間四枚}{に|けん|よん|まい}の
\ruby{障子}{しやう|じ}に、
%
\ruby{繼目}{つぎ|め}
\ruby{無}{な}しの
\ruby{紙}{かみ}は
\ruby{{\換字{雪}}}{ゆき}より
\ruby{白}{しろ}く
\ruby{椽}{えん}の
\原本頁{57-4}\改行%
\ruby{方}{かた}より
\ruby[||j>]{光}{くわう}
\ruby[||j>]{線}{ せん}を
% \ruby{光線}{くわう|せん}を
\ruby{取}{と}りて、
%
\ruby{上}{うへ}は
\ruby{{\換字{嫌}}味}{いや|み}
\ruby{氣}{け}
\ruby{無}{な}き
\ruby{柾}{まさ}の
\ruby[||j>]{天}{てん}
\ruby[||j>]{井}{じやう}、
% \ruby{天井}{てん|じやう}、
%
\ruby{下}{した}は
\ruby{緣}{へり}
\ruby{無}{な}しの
\ruby{備後表}{び|ん|ご}
といふ
\ruby{室}{ま}の
\ruby{内}{うち}の、
%
\ruby{好}{よ}きほどに
\ruby{据}{す}ゑられたる
\ruby[|g|]{多{\換字{分}}}{いずれ}
\ruby{太田}{おほ|た}あたりで
\ruby{指}{さ}させたる
らしき
\ruby{島桑}{しま|ぐは}の
\ruby{長火鉢}{なが|ひ|ばち}と、
%
\ruby{其}{そ}の
\ruby{横手}{よこ|て}に
\ruby{置}{お}かれたる
\原本頁{57-6}\改行%
\ruby{思}{おも}ひ
\ruby{切}{き}つて
\ruby{立派}{りつ|ぱ}なる
\ruby{支那}{し|な}
\ruby{製}{せい}の
\ruby{紫檀}{し|たん}の
\ruby{茶棚}{ちや|だな}
とは、
%
\ruby{先}{ま}づ
\ruby{入}{い}るものゝ
\ruby{目}{め}を
\ruby{惹}{ひ}きて、
%
\ruby{此家}{こ|ゝ}の
\ruby{女主人}{あ|る|じ}の
\ruby{十二{\換字{分}}}{じう|に|ぶん}に
\ruby{財}{たから}に
\ruby{富}{と}み
\ruby{足}{た}りて、
%
\ruby{且}{か}つは
\原本頁{57-8}\改行%
\ruby{其}{そ}の
\ruby{勸工場品}{くわん|こう|ば|もの}に% 原文通り「場」
\ruby{望}{のぞ}み
\ruby{足}{た}れり
とする
やう
なる
\ruby{沒趣味者}{わか|ら|ず|や}
ならぬを
\ruby{示}{しめ}し、
%
\ruby{壁}{かべ}の
\ruby{塗}{ぬ}り
\ruby{色}{いろ}、
%
\ruby{押入}{おし|いれ}の
\ruby{襖}{ふすま}の
\ruby{模樣}{も|やう}まで、
%
すべて
\ruby{釣合}{つり|あ}ひて
しつとりと
\ruby{整}{とゝの}ひたるが
\ruby{中}{なか}に、
%
おのづから
\ruby{薄手}{うす|て}ならず
\ruby{{\換字{又}}}{また}
わびしげ
ならで
\原本頁{58-1}\改行%
\ruby{{\換字{飽}}}{あく}まで
     『
     \ruby{良}{い}いもの
     \ruby{好}{ず}き
     』
     『
     \ruby{粗惡}{い|や}なもの
     \ruby{{\換字{嫌}}}{ぎら}ひ
     』
の
\ruby{趣}{おもむ}きは
\ruby{見}{み}えたり。

\原本頁{58-2}%
『
お
\ruby{龍}{りう}ちやん、
%
お
\ruby{{\換字{前}}}{まへ}
\ruby{御客樣}{お|きやく|さま}らしく
\ruby{仕無}{し|な}いでも、
%
もつと
\ruby{此方}{こつ|ち}へ
\ruby{寄}{よ}つて
\ruby{御}{お}あたりナ。
』

\原本頁{58-4}%
\ruby{大島紬}{おほ|し|ま}は
\ruby{好}{よ}い
もの
なれども、
%
\ruby{何處}{ど|こ}
となく
ぼやついて、
%
すつぺり
とせぬが
\ruby{厭}{いや}なり、
%
\ruby{{\換字{平}}常着}{ふだ|ん|ぎ}は
\ruby{此}{これ}に
\ruby{限}{かぎ}ると、
%
\ruby{{\換字{平}}生}{ひご|ろ}
\ruby[|g|]{御召縮緬}{おめし}を
\ruby{着}{き}
\ruby{{\換字{通}}}{とほ}せる
お
\ruby{彤}{とう}の、
%
\ruby{今}{いま}も
\ruby{相}{あひ}
\ruby{變}{かは}らず
\ruby{其品}{そ|れ}づくめの
\ruby{衣服}{な|り}つき
\ruby{見}{み}
\ruby{好}{よ}く、
%
\ruby{絹物}{き|ぬ}の
\ruby{坐蒲團}{ざ|ぶ|とん}の
\ruby{上}{うへ}に
\ruby{居}{ゐ}て、
%
\ruby{火鉢}{ひ|ばち}より
\ruby{南部}{なん|ぶ}の
\ruby{鐵瓶}{てつ|びん}を
\ruby{重}{おも}さうに
\ruby{取}{と}り
\ruby{下}{おろ}しながら
\ruby{斯}{か}く
\ruby{云}{い}へば、

\原本頁{58-9}%
『
えゝ、
%
\ruby{姊}{ねえ}さん
のところへ
\ruby{來}{き}て
\ruby{御客樣}{お|きやく|さま}
らしく
なんぞ
\ruby{仕}{し}や
\ruby{仕}{し}ませんがネ、
%
まだ
\ruby{火}{ひ}の
\ruby{傍}{そば}へ
\ruby{行}{い}きたい
ほど
\ruby{{\換字{寒}}}{さむ}かあ
\ruby{有}{あ}りません
もの。
』

\原本頁{58-11}%
と
\ruby{笑}{わら}ひ
つゝ
お
\ruby{龍}{りう}は
\ruby{言}{ことば}に
\ruby{從}{したが}つて
\ruby{聊}{いさゝ}か
\ruby{坐}{ざ}を
\ruby{{\換字{進}}}{すゝ}めたるが、
%
\ruby{實}{げ}に
\ruby{其}{そ}の
\ruby{顏}{かほ}は
\ruby{見}{み}るからが
\ruby{冴々}{さえ|〴〵}しく
\ruby[||j>]{櫻}{さくら}
\ruby[||j>]{色}{ いろ}に
% \ruby{櫻色}{さくら|いろ}に
\ruby{艶}{えん}にして、% 原本通り「えん」
%
\ruby{如何}{い|か}にも
\ruby{此}{こ}の
\ruby{頃}{ごろ}の
\ruby{{\換字{寒}}}{さむ}さ
\原本頁{59-2}\改行%
\ruby{位}{ぐらゐ}は
\ruby{何}{なん}とも
\ruby{思}{おも}はぬ
らしき
\ruby{樣子}{やう|す}
を
あらはせり。

\原本頁{59-3}%
お
\ruby{彤}{とう}は
\ruby{坐}{ざ}を
\ruby{{\換字{進}}}{すゝ}むる
お
\ruby{龍}{りう}が
\ruby{頭髮}{かし|ら}を
\ruby{一寸}{ちよ|つと}
\ruby{見}{み}しが、
%
\ruby[||j>]{女}{をんな}
\ruby{同士}{ どう|し}の
\ruby{談}{はなし}の
\ruby[|-|]{緖}{いとぐち}は
\ruby{先}{ま}づ
\ruby{其}{それ}より
\ruby{解}{ほご}るゝ
\ruby{{\換字{習}}}{ならひ}
なり。

\原本頁{59-5}%
『
\ruby{今日}{け|ふ}も
また
\ruby{束髮}{そく|はつ}
にして
おいでだネ。
%
\ruby{此}{この}
\ruby{{\換字{節}}}{せつ}は
\ruby{何時}{い|つ}
\ruby{見}{み}ても
\ruby{結}{い}つては
\ruby{居}{ゐ}ないのネ。
』

\原本頁{59-7}%
『
ハア。
%
\ruby{姊}{ねえ}さん
でさへ
\ruby{矢張}{やつ|ぱり}
\ruby{束髮}{こ|れ}に
なさる
ぢやあ
\ruby{有}{あ}りませんか。
%
\原本頁{59-8}\改行%
まして
\ruby{妾}{わたし}
なんか。
%
\ruby{出}{で}る
\ruby{先}{さき}に
\ruby{立}{た}つて
\ruby{一々}{いち|〳〵}
\ruby{人手}{ひと|で}を
\ruby{假}{か}りるのが
\ruby{億劫}{おつ|くう}な
ものですから、
%
つい
\ruby{自{\換字{分}}}{ひと|り}で
もつて
ぐる〳〵と
\ruby{卷}{ま}いて
\ruby{仕舞}{し|ま}ふので。
%
\ruby{似合}{に|あ}は
\ruby{無}{な}いで
\ruby{可笑}{を|か}しくつて?。
』

\原本頁{59-11}%
『
ナアニ
\ruby{似合}{に|あ}はない
\ruby{事}{こと}は
\ruby{有}{あ}りやあ
\ruby{仕}{し}ないよ、
%
ぢやあ
\ruby{今日}{け|ふ}も
もう
\ruby{何處}{ど|こ}かへ
\ruby{御出}{お|いで}
だつたのだネ。
』

\原本頁{60-2}%
『
ハア
\ruby{一寸}{ちよ|つと}。
』

\原本頁{60-3}%
こゝに
\ruby{至}{いた}りて
\ruby{女主人}{あ|る|じ}は
\ruby{其}{そ}の
\ruby{美}{うつく}しき
\ruby{面}{おもて}に
\ruby{微笑}{ゑ|み}を
\ruby{泛}{うか}めて、

\原本頁{60-4}%
『
\ruby{當}{あ}てゝ
\ruby{見}{み}やうかへ。
』

\原本頁{60-5}%
と
\ruby{戲}{たはむ}るゝが
\ruby{如}{ごと}く
\ruby{云}{い}へば、
%
お
\ruby{龍}{りう}は
\ruby{言}{ことば}も
\ruby{無}{な}く
\ruby{莞爾}{にこ|り}と
\ruby{笑}{ゑ}みて
\ruby{親}{した}しげに
\ruby{輕}{かろ}く
\ruby{點頭}{うな|づ}けり。

\原本頁{60-7}%
『
\ruby{屹度}{きつ|と}
また
\ruby{淺草}{あさ|くさ}へ
\ruby{御出}{お|いで}
だつたのさ。
』

\原本頁{60-8}%
『
いゝえ。
』

\原本頁{60-9}%
『
なに、
%
いゝえの
\ruby{事}{こと}が
\ruby{有}{あ}るものかネ。
%
ソラ〳〵
\ruby{口}{くち}は
\ruby{詐}{うそ}を
お
\ruby{云}{い}ひ
でも
\ruby{顏}{かほ}は
\ruby[||j>]{正}{しやう}
\ruby[||j>]{直}{ ぢき}だよ、
% \ruby{正直}{しやう|ぢき}だよ、
%
ハイ
\ruby[<j||]{觀}{くわん}% 「觀音」の読みは原本通り「くわん(の)ん」
\ruby[<j||]{音}{のん }
% \ruby{觀音}{くわん|のん}% 「觀音」の読みは原本通り「くわん(の)ん」
\ruby{樣}{さま}へ
\ruby{參}{まゐ}りました
と、
%
その
\ruby{笑}{わら}つて
\ruby{居}{ゐ}る
\ruby{眼}{め}が、
%
チヤーンと
\ruby{左樣}{さ|う}いつて
\ruby{居}{ゐ}るよ。
』

\原本頁{61-1}%
『
ホヽホヽホヽ。
』

\原本頁{61-2}%
『
ホヽホヽ、
%
それ
\ruby{御覽}{ご|らん}、
%
\ruby{御手}{お|て}の
\ruby{筋}{すぢ}
だらう。
%
\ruby{御精}{ご|せい}が
\ruby{出}{で}て
\ruby{眞實}{ほん|と}に
\ruby{御奇特}{ご|き|どく}の
\ruby{事}{こと}だネエ。
』

\原本頁{61-4}%
『
あら
\ruby{姊}{ねえ}さん、
%
\ruby{調戲}{から|か}つちやあ
\ruby{厭}{いや}
ですよ、
%
あんまり
ですは。
』

\原本頁{61-5}%
『
\ruby{左樣}{さ|う}さネエ。
%
\ruby{何}{なに}も
\ruby{彼}{あ}の
\ruby{人}{ひと}に
\ruby{御會}{お|あ}ひ
でも
\ruby{無}{な}かつた
らうに、
%
\ruby[|g|]{調戲}{からか}はれちやあ
\ruby[||j>]{愍}{かは}
\ruby[||j>]{然}{いさう}% 「愍然 か(は)いさう」
% \ruby{愍然}{かは|いさう}% 「愍然 か(は)いさう」
だつたネ。
』

\原本頁{61-7}%
『
もう
よう
ござんす
は、
%
\ruby{澤山}{たん|と}
いろんな
\ruby{事}{こと}を
\ruby{仰}{おつし}あいよ。
%
\ruby{今日}{け|ふ}も
\ruby{不思議}{ふ|し|ぎ}に
\ruby{落}{おち}
\ruby{合}{あ}つて
\ruby{會}{あ}つて
\ruby{來}{き}ました
は。
』

\原本頁{61-9}%
『
オヤツ。
%
そんな
\ruby{譯}{わけ}は
\ruby{無}{な}いぢやあ
\ruby{無}{な}いか。
%
\ruby{今日}{け|ふ}は
\ruby{{\換字{平}}常}{た|ゞ}の
\ruby{日}{ひ}だし、
%
\原本頁{61-10}\改行%
\ruby{彼}{あ}の
\ruby{人}{ひと}は
\ruby{職務}{つと|め}が
\ruby{有}{あ}るつて
いふ
\ruby{談}{はなし}
だつた
もの。
%
ぢやあ
\ruby{矢張}{やつ|ぱり}
\ruby[||j>]{打}{うち}
\ruby[||j>]{合}{あはせ}でも
% \ruby{打合}{うち|あはせ}でも
\ruby{仕}{し}て
\ruby{御置}{お|お}き
だつたの?。
』

\原本頁{62-1}%
『
いゝえ、
%
そんな
\ruby{事}{こと}は
\ruby{有}{あ}りあ
\ruby{仕}{し}ませんがネ。
%
\ruby{彼}{あ}の
\ruby{人}{ひと}が
\ruby{職務}{つと|め}の
\ruby{方}{はう}を
\ruby{辭}{よ}して
\ruby{仕舞}{し|ま}つたので、
%
それで
\ruby{今日}{け|ふ}は
\ruby{御午{\換字{前}}}{お|ひる|まへ}に
\ruby{出}{で}て
\ruby{來}{き}たつて
\ruby{云}{い}ふんで。
%
ひよつくりと
\ruby{御堂}{み|だう}で
\ruby{會}{あ}つた
わけ
なの
ですよ。
』

\原本頁{62-4}%
『
ヘーエ、
%
\ruby{職務}{つと|め}の
\ruby{方}{はう}を
\ruby{辭}{よ}したつて‥‥。
%
あゝ
\ruby{解}{わか}つた
\ruby{免}{よ}されたんだネ。
』

\原本頁{62-6}%
『
\ruby{左樣}{さ|う}なのよ、
%
\ruby{事實}{まつ|たく}は
\ruby{免}{よ}された
のですつて。
%
\ruby{其}{それ}に
ついて
\ruby{姊}{ねえ}さんに
\ruby{些}{ちつと}
お
\ruby{願}{ねがひ}が
あつて
\ruby{來}{き}たのです
がネ。
』

\原本頁{62-8}%
と、
%
やゝ
\ruby{眞}{しん}に
なつて
\ruby{談話}{はな|し}を
せんとする
お
\ruby{龍}{りう}の
\ruby{眼色}{め|いろ}を
\ruby{見}{み}て、
%
お
\ruby{彤}{とう}は
\ruby{輕}{かろ}く
\ruby{一寸}{ちよ|つと}
\ruby{制止}{と|ゞ}めつ、

\原本頁{62-10}%
『
\ruby{御待}{お|ま}ちよ
お
\ruby{龍}{りう}ちやん。
%
\ruby[|g|]{彼室}{あつち}へ
\ruby{行}{い}つてから
\ruby{{\換字{緩}}々}{ゆつ|くり}と
\ruby{談}{はなし}を
\ruby{聞}{き}かうから。
』

\原本頁{63-1}%
と、
%
\ruby{奧}{おく}の
\ruby{方}{かた}を
\ruby{指}{ゆび}
さし、

\原本頁{63-2}%
『
あら
\ruby{姊}{ねえ}さん、
%
\ruby{此室}{こ|ゝ}で
\ruby{澤山}{たく|さん}
だは。
』

\原本頁{63-3}%
と
お
\ruby{龍}{りう}の
\ruby{云}{い}ふを
\ruby{打}{うち}
\ruby{{\換字{消}}}{け}して、

\原本頁{63-4}%
『
\ruby{妾}{わたし}が
\ruby{茶}{ちや}の
\ruby{間}{ま}に
\ruby{居}{ゐ}るのゝ
\ruby{{\換字{嫌}}}{きらひ}なのは
お
\ruby{{\換字{前}}}{まへ}も
\ruby{知}{し}つて
\ruby{居}{ゐ}るぢや
\ruby{無}{な}いか。
』

\原本頁{63-5}%
と
\ruby{{\換字{遮}}}{さへぎ}り、
%
さて
\ruby{下手}{しも|て}へ
\ruby{向}{むか}つて
\ruby{小間使}{こ|ま|づかひ}の
お
\ruby{春}{はる}と
いへる
\ruby{可愛}{か|はい}らしき
\ruby{兒}{こ}を
\ruby{喚}{よ}び
\ruby{出}{いだ}し、

\原本頁{63-7}%
『
\ruby{妾}{わたし}の
\ruby{部屋}{へ|や}の
\ruby{茶{\換字{道}}具}{ちや|だう|ぐ}を
\ruby{能}{よ}く
\ruby{淸潔}{きれ|い}に
\ruby{仕}{し}てネ、
%
そして
また
\ruby[|g|]{彼室}{あつち}へ
\ruby{持}{も}つて
\ruby{行}{い}つて
お
\ruby{吳}{く}れ。
%
お
\ruby{茶}{ちや}は
\ruby{妾}{わたし}が
\ruby{自{\換字{分}}}{じ|ぶん}で
\ruby{淹}{い}れるからネ、
%
お
\ruby{{\換字{前}}}{まへ}は
\ruby{御菓子}{お|くわ|し}を
\ruby{出}{だ}して、
‥‥
ア
\ruby{羊羮}{やう|かん}は
いけない、
%
\ruby{玉簾}{たま|だれ}の
\ruby{方}{はう}を
\ruby{切}{き}つて
おいで。
』

\原本頁{63-11}%
と
\ruby[|g|]{命令}{いひつ}け、

\原本頁{64-1}%
『
さあ
\ruby{此方}{こつ|ち}へ
\ruby{御}{お}いで。
』

\原本頁{64-2}%
と
\ruby{立}{たち}
\ruby{上}{あが}つて
お
\ruby{龍}{りう}を
\ruby{奧}{おく}へ
\ruby{{\換字{伴}}}{ともな}へる
\ruby{時}{とき}、
%
\ruby{恰}{あだか}も% 恰も「あ(だ)かも」
\ruby{時計}{と|けい}の
\ruby{音}{おと}は
\ruby{三時}{さん|じ}を
\ruby{報}{はう}じたり。

\原本頁{64-4}%
\ruby{男}{をとこ}にも
いろ〳〵
あれば、
%
\ruby{女}{をんな}にも
いろ〳〵
ありて、
%
まことに
お
\ruby{彤}{とう}は
\ruby{今}{いま}
みづから
\ruby{言}{い}へるが
\ruby{如}{ごと}くに、
%
\ruby[|g|]{{\換字{平}}生}{ひごろ}
\ruby{長火鉢}{なが|ひ|ばち}の
\ruby{{\換字{前}}}{まへ}に
\ruby{坐}{すわ}りて
\ruby{茶}{ちや}の
\ruby{間}{ま}に
\ruby{在}{あ}ることは
\ruby{悅}{よろこ}ばずして、
%
おのが
\ruby{室}{ま}と
\ruby{定}{さだ}めたる
\ruby{小座敷}{こ|ざ|しき}に
\ruby{端然}{しや|ん}として
\ruby{居}{ゐ}ることを
\ruby{好}{この}めるなり。
%
されば
\ruby{是}{これ}
\ruby{程}{ほど}の
\ruby{好}{よ}き
\ruby{茶}{ちや}の
\ruby{室}{ま}をも、
%
\ruby{一}{ひ}ト
\ruby{風}{ふう}ある
\ruby{氣性}{き|しやう}
からは、
%
\ruby{床}{とこ}の
\ruby{間}{ま}さへ
\ruby{無}{な}き
\ruby{室}{へや}と
\ruby{賤}{いや}しく
\ruby{思}{おも}ふ
なる
べし。
