\Entry{其十一}

\ruby{六疊}{ろく|でふ}の
\ruby{茶}{ちや}の
\ruby{間}{ま}、
\ruby{茶}{ちや}の
\ruby{間}{ま}とはいへ
\ruby{大抵}{たい|てい}の
\ruby{家}{いへ}の
\ruby[g]{客室}{きやくま}より
\ruby{美}{うつく}しく、
\ruby{柱}{はしら}より
\ruby{敷居鴨居}{しき|ゐ|かも|ゐ}の
\ruby[g]{木口}{きぐち}の
\ruby{結構}{けつ|こう}さ。
\ruby{格}{こ}の
\ruby{配}{くば}りに
\ruby{物好}{もの|ずき}を
\ruby{見}{み}せたる
\ruby{細骨}{ほそ|ぼね}の
\ruby{纖巧}{きや|しや}なる
\ruby{二間四枚}{に|けん|よん|まい}の
\ruby{障子}{しやう|じ}に、
\ruby[g]{繼目無}{つぎめな}しの
\ruby{紙}{かみ}は
\ruby{{\GWI{u96ea-k}}}{ゆき}より
\ruby{白}{しろ}く
\ruby{椽}{ゑん}の
\ruby{方}{かた}より
\ruby{光線}{くわう|せん}を
\ruby{取}{と}りて、
\ruby{上}{うへ}は
\ruby[g]{{\GWI{u5acc-k}}味氣無}{いやみけな}き
\ruby{柾}{まさ}の
\ruby[g]{天井}{てんじやう}、
\ruby{下}{した}は
\ruby[g]{緣無}{へりな}しの
\ruby{備後表}{び|ん|ご}といふ
\ruby{室}{ま}の
\ruby{内}{うち}の、
\ruby{好}{よ}きほどに
\ruby{据}{す}ゑられたる
\ruby[g]{多分太田}{いずれおほた}あたりで
\ruby{指}{さ}させたるらしき
\ruby{島桑}{しま|ぐは}の
\ruby[g]{長火鉢}{ながひばち}と、
\ruby{其}{そ}の
\ruby[g]{横手}{よこて}に
\ruby{置}{お}かれたる
\ruby{思}{おも}ひ
\ruby{切}{き}つて
\ruby[g]{立派}{りつぱ}なる
\ruby[g]{支那製}{しなせい}の
\ruby{紫檀}{し|たん}の
\ruby{茶棚}{ちゃ|だな}とは、
\ruby{先}{ま}づ
\ruby{入}{い}るものヽ
\ruby{目}{め}を
\ruby{惹}{ひ}きて、
\ruby{此家}{こ|ヽ}の
\ruby{女主人}{あ|る|じ}の
\ruby{十二分}{じう|に|ぶん}に
\ruby{財}{たから}に
\ruby{富}{と}み
\ruby{足}{た}りて、
\ruby{且}{か}つは
\ruby{其}{そ}の
\ruby[g]{勸工場品}{くわんこうばもの}に
\ruby{望}{のぞ}み
\ruby{足}{た}れりとするやうなる
\ruby[g]{沒趣味者}{わからずや}ならぬを
\ruby{示}{しめ}し、
\ruby{壁}{かべ}の
\ruby{塗}{ぬ}り
\ruby{色}{いろ}、
\ruby{押入}{おし|いれ}の
\ruby{襖}{ふすま}の
\ruby{模樣}{も|やう}まで、すべて
\ruby[g]{釣合}{つりあ}ひてしつとりと
\ruby{整}{とヽの}ひたるが
\ruby{中}{なか}に、おのづから
\ruby{薄手}{うす|で}ならず
\ruby{{\GWI{u53c9-k}}}{また}わびしげならで
\ruby{飽}{あく}まで『
\ruby{良}{よ}いもの
\ruby{好}{ず}き』『
\ruby{粗惡}{い|や}なもの
\ruby{{\GWI{u5acc-k}}}{ぎら}ひ』の
\ruby{趣}{おもむ}きは
\ruby{見}{み}えたり。

『お
\ruby{龍}{りう}ちやん、お
\ruby[g]{前御客樣}{まへおきやくさま}らしく
\ruby{仕無}{し|な}いでも、もつと
\ruby[g]{此方}{こつち}へ
\ruby{寄}{よ}つて
\ruby{御}{お}あたりナ。
』

\ruby[g]{大島紬}{おほしま}は
\ruby{好}{よ}いものなれども、
\ruby{何處}{ど|こ}となくぼやついて、すつぺりとせぬが
\ruby{厭}{いや}なり、
\ruby[g]{{\GWI{u5e73-k}}常着}{ふだんぎ}は
\ruby{此}{これ}に
\ruby{限}{かぎ}ると、
\ruby[g]{{\GWI{u5e73-k}}生御召縮緬}{ひごろおめし}を
\ruby{着{\GWI{u901a-k}}}{き|とほ}せるお
\ruby{彤}{とう}の、
\ruby{今}{いま}も
\ruby{相變}{あひ|かは}らず
\ruby{其品}{そ|れ}づくめの
\ruby{衣服}{な|り}つき
\ruby{見好}{み|よ}く、
\ruby{絹物}{き|ぬ}の
\ruby{坐蒲團}{ざ|ぶ|とん}の
\ruby{上}{うへ}に
\ruby{居}{ゐ}て、
\ruby{火鉢}{ひ|ばち}より
\ruby[g]{南部}{なんぶ}の
\ruby{鐵瓶}{てつ|びん}を
\ruby{重}{おも}さうに
\ruby{取}{と}り
\ruby{下}{おろ}しながら
\ruby{斯}{か}く
\ruby{云}{い}へば、

『えヽ、
\ruby{姊}{ね{\GWI{u1b001}}}さんのところへ
\ruby{來}{き}て
\ruby[g]{御客樣}{おきやくさま}らしくなんぞ
\ruby{仕}{し}や
\ruby{仕}{し}ませんがネ、まだ
\ruby{火}{ひ}の
\ruby{傍}{そば}へ
\ruby{行}{い}きたいほど
\ruby{{\GWI{u5bd2-k}}}{さむ}かあ
\ruby{有}{あ}りませんもの。
』

と
\ruby{笑}{わら}ひつゝお
\ruby{龍}{りう}は
\ruby{言}{ことば}に
\ruby{從}{したが}つて
\ruby{聊}{いさヽ}か
\ruby{坐}{ざ}を
\ruby{{\GWI{u9032-k}}}{すヽ}めたるが、
\ruby{實}{げ}に
\ruby{其}{そ}の
\ruby{顏}{かほ}は
\ruby{見}{み}るからが
\ruby{冴々}{さえ|〴〵}しく
\ruby{櫻色}{さくら|いろ}に
\ruby{艶}{えん}にして、
\ruby{如何}{い|か}にも
\ruby{此}{こ}の
\ruby{頃}{ごろ}の
\ruby{{\GWI{u5bd2-k}}}{さむ}さ
\ruby{位}{ぐらゐ}は
\ruby{何}{なん}とも
\ruby{思}{おも}はぬらしき
\ruby[g]{樣子}{やうす}をあらはせり。

お
\ruby{彤}{とう}は
\ruby{坐}{ざ}を
\ruby{{\GWI{u9032-k}}}{すヽ}むるお
\ruby{龍}{りう}が
\ruby[g]{頭髮}{かしら}を
\ruby{一寸見}{ちよ|つと|み}しが、
\ruby[g]{女同士}{をんなどうし}の
\ruby{談}{はなし}の
\ruby{緒}{いとぐち}は
\ruby{先}{ま}づ
\ruby{其}{それ}より
\ruby{解}{ほご}るヽ
\ruby{{\GWI{u7fd2-k}}}{ならひ}なり。

『
\ruby{今日}{け|ふ}もまた
\ruby{束髮}{そく|はつ}にしておいでだネ。
\ruby{此{\GWI{u7bc0-k}}}{この|せつ}は
\ruby{何時見}{い|つ|み}ても
\ruby{結}{い}つては
\ruby{居}{ゐ}ないのネ。
』

『ハア。
\ruby{姊}{ねえ}さんでさへ
\ruby[g]{矢張}{やつぱり}
\ruby{束髮}{ %全角空白
こ|れ}になさるぢやあ
\ruby{有}{あ}りませんか。
まして
\ruby{妾}{わたし}なんか。
\ruby{出}{で}る
\ruby{先}{さき}に
\ruby{立}{た}つて
\ruby{一々人手}{いち|〳〵|ひと|で}を
\ruby{假}{か}りるのが
\ruby{億劫}{おつ|くう}なものですから、つい
\ruby[g]{自分}{ひとり}でもつてぐる〳〵と
\ruby{卷}{ま}いて
\ruby{仕舞}{し|ま}ふので。
\ruby{似合}{に|あ}は
\ruby{無}{な}いで
\ruby[g]{可笑}{おかし}くつて?。
』

『ナアニ
\ruby{似合}{に|あ}はない
\ruby{事}{こと}は
\ruby{有}{あ}りやあ
\ruby{仕}{し}ないよ、ぢやあ
\ruby{今日}{け|ふ}ももう
\ruby{何處}{ど|こ}かへ
\ruby[g]{御出}{おいで}だつたのだネ。
』

『ハア
\ruby[g]{一寸}{ちよつと}。
』

こゝに
\ruby{至}{いた}りて
\ruby{女主人}{あ|る|じ}は
\ruby{其}{そ}の
\ruby{美}{うつく}しき
\ruby{面}{おもて}に
\ruby{微笑}{ゑ|み}を
\ruby{泛}{うか}めて、

『
\ruby{當}{あ}てヽ
\ruby{見}{み}やうかへ。
』

と
\ruby{戯}{たはむ}るヽが
\ruby{如}{ごと}く
\ruby{云}{い}へば、お
\ruby{龍}{りう}は
\ruby{言}{ことば}も
\ruby{無}{な}く
\ruby{莞爾}{にこ|り}と
\ruby{笑}{ゑ}みて
\ruby{親}{した}しげに
\ruby{輕}{かろ}く
\ruby[g]{點頭}{うなづ}けり。

『
\ruby[g]{屹度}{きつと}また
\ruby{淺草}{あさ|くさ}へ
\ruby{御出}{お|いで}だつたのさ。
』

『いゝえ。
』

『なに、いヽえの
\ruby{事}{こと}が
\ruby{有}{あ}るものかネ。
ソラ〳〵
\ruby{口}{くち}は
\ruby{詐}{うそ}をお
\ruby{云}{い}ひでも
\ruby{顏}{かほ}は
\ruby{正直}{しやう|ぢき}だよ、ハイ
\ruby{觀音樣}{くわん|のん|さま}へ
\ruby{參}{まゐ}りましたと、その
\ruby{笑}{わら}つて
\ruby{居}{ゐ}る
\ruby{眼}{め}が、チヤーンと
\ruby{左樣}{さ|う}いつて
\ruby{居}{ゐ}るよ。
』

『ホヽホヽホヽ。
』

『ホヽホヽ、それ
\ruby{御覽}{ご|らん}、
\ruby{御手}{お|て}の
\ruby{筋}{すぢ}だらう。
\ruby{御{\GWI{u7cbe-k}}}{ご|せい}が
\ruby{出}{で}て
\ruby{眞實}{ほん|と}に
\ruby[g]{御奇特}{ごきとく}の
\ruby{事}{こと}だネエ。
』

『あら
\ruby{姊}{ねえ}さん、
\ruby{調戯}{から|か}つちやあ
\ruby{厭}{いや}ですよ、あんまりですは。
』

『
\ruby{左樣}{さ|う}さネエ。
\ruby{何}{なに}も
\ruby{彼}{あ}の
\ruby{人}{ひと}に
\ruby{御會}{お|あ}ひでも
\ruby{無}{な}かつたらうに、
\ruby{調戯}{から|か}はれちやあ
\ruby[g]{愍然}{かはいさう}だつたネ。
』

『もうようござんすは、
\ruby[g]{澤山}{たんと}いろんな
\ruby{事}{こと}を
\ruby{仰}{おつし}あいよ。
\ruby{今日}{け|ふ}も
\ruby{不思議}{ふ|し|ぎ}に
\ruby[g]{落合}{おちあ}つて
\ruby{會}{あ}つて
\ruby{來}{き}ましたは。
』

『オヤツ。
そんな
\ruby{譯}{わけ}は
\ruby{無}{な}いぢやあ
\ruby{無}{な}いか。
\ruby{今日}{け|ふ}は
\ruby{{\GWI{u5e73-k}}常}{た|ヾ}の
\ruby{日}{ひ}だし、
\ruby{彼}{あ}の
\ruby{人}{ひと}は
\ruby[g]{職務}{つとめ}が
\ruby{有}{あ}るつていふ
\ruby{談}{はなし}だつたもの。
ぢやあ
\ruby[g]{矢張打合}{やつぱりうちあはせ}でも
\ruby{仕}{し}て
\ruby{御置}{お|お}きだつたの?。
』

『いゝえ、そんな
\ruby{事}{こと}は
\ruby{有}{あ}りあ
\ruby{仕}{し}ませんがネ。
\ruby{彼}{あ}の
\ruby{人}{ひと}が
\ruby[g]{職務}{つとめ}の
\ruby{方}{はう}を
\ruby{辭}{よ}して
\ruby{仕舞}{し|ま}つたので、それで
\ruby{今日}{け|ふ}は
\ruby[g]{御午前}{おひるまへ}に
\ruby{出}{で}て
\ruby{來}{き}たつて
\ruby{云}{い}ふんで。
ひよつくりと
\ruby{御堂}{み|だう}で
\ruby{會}{あ}つたわけなのですよ。
』

『ヘーエ、
\ruby[g]{職務}{つとめ}の
\ruby{方}{はう}を
\ruby{辭}{よ}したつて……。
あヽ
\ruby{解}{わか}つた
\ruby{免}{よ}されたんだネ。
』

『
\ruby{左樣}{さ|う}なのよ、
\ruby[g]{事實}{まつたく}は
\ruby{免}{よ}されたのですつて。
\ruby{其}{それ}について
\ruby{姊}{ねえ}さんに
\ruby{些}{ちつと}お
\ruby{願}{ねがひ}があつて
\ruby{來}{き}たのですがネ。
』

と、やヽ
\ruby{眞}{しん}になつて
\ruby[g]{談話}{はなし}をせんとするお
\ruby{龍}{りう}の
\ruby{眼色}{め|いろ}を
\ruby{見}{み}て、お
\ruby{彤}{とう}は
\ruby{輕}{かろ}く
\ruby[g]{一寸制止}{ちよつととヾ}めつ、

『
\ruby{御待}{お|ま}ちよお
\ruby{龍}{りう}ちやん。
\ruby[g]{彼室}{あつち}へ
\ruby{行}{い}つてから
\ruby{{\GWI{u7de9-k}}々}{ゆつ|くり}と
\ruby{談}{はなし}を
\ruby{聞}{き}かうから。
』

と、
\ruby{奧}{おく}の
\ruby{方}{かた}を
\ruby{指}{ゆび}さし、

『あら
\ruby{姊}{ねえ}さん、
\ruby{此室}{こ|ヽ}で
\ruby{澤山}{たく|さん}
だは。
』

とお
\ruby{龍}{りう}の
\ruby{云}{い}ふを
\ruby[g]{打{\GWI{u6d88-k}}}{うちけ}して、

『
\ruby{妾}{わたし}が
\ruby{茶}{ちや}の
\ruby{間}{ま}に
\ruby{居}{ゐ}るのヽ
\ruby{{\GWI{u5acc-k}}}{きらひ}なのはお
\ruby{前}{まへ}も
\ruby{知}{し}つて
\ruby{居}{ゐ}るぢや
\ruby{無}{な}いか。
』

と
\ruby{{\GWI{u906e-k}}}{さへぎ}り、さて
\ruby{下手}{しも|て}へ
\ruby{向}{むか}つて
\ruby[g]{小間使}{こまづかひ}のお
\ruby{春}{はる}といへる
\ruby{可愛}{か|はい}らしき
\ruby{兒}{こ}を
\ruby{喚}{よ}び
\ruby{出}{いだ}し、

『
\ruby{妾}{わたし}の
\ruby{部屋}{へや|}の
\ruby[g]{茶{\GWI{u9053-k}}具}{ちやだうぐ}を
\ruby{能}{よ}く
\ruby[g]{清潔}{きれい}に
\ruby{仕}{し}てネ、そしてまた
\ruby[g]{彼室}{あつち}へ
\ruby{持}{も}つて
\ruby{行}{い}つてお
\ruby{{\GWI{u5449-itaiji-002}}}{く}れ。
お
\ruby{茶}{ちや}は
\ruby{妾}{わたし}が
\ruby{自分}{じ|ぶん}で
\ruby{淹}{い}れるからネ、お
\ruby{前}{まへ}は
\ruby[g]{御菓子}{おくわし}を
\ruby{出}{だ}して、……ア
\ruby{羊羮}{やう|かん}はいけない、
\ruby{玉簾}{たま|だれ}の
\ruby{方}{はう}を
\ruby{切}{き}つておいで。
』

と
\ruby{命令}{いひ|つけ}け、

『さあ
\ruby[g]{此方}{こつち}へ
\ruby{御}{お}いで。
』

と
\ruby{立上}{たち|あが}つてお
\ruby{龍}{りう}を
\ruby{奧}{おく}へ
\ruby{{\換字{伴}}}{ともな}へる
\ruby{時}{とき}、
\ruby{恰}{あだか}も
\ruby{時計}{と|けい}の
\ruby{音}{おと}は
\ruby{三時}{さん|じ}を
\ruby{報}{はう}じたり。

\ruby{男}{をとこ}にもいろ〳〵あれば、
\ruby{女}{をんな}にもいろ〳〵ありて、まことにお
\ruby{彤}{とう}は
\ruby{今}{いま}みづから
\ruby{言}{い}へるが
\ruby{如}{ごと}くに、
\ruby[g]{{\GWI{u5e73-k}}生長火鉢}{ひごろながひばち}の
\ruby{前}{まへ}に
\ruby{坐}{すわ}りて
\ruby{茶}{ちや}の
\ruby{間}{ま}に
\ruby{在}{あ}ることは
\ruby{{\GWI{u6085-k}}}{よろこ}ばずして、おのが
\ruby{室}{ま}と
\ruby{定}{さだ}めたる
\ruby{小座敷}{こ|ざ|しき}に
\ruby[g]{端然}{しやん}として
\ruby{居}{ゐ}ることを
\ruby{好}{この}めるなり。
されば
\ruby{是程}{これ|ほど}の
\ruby{好}{よ}き
\ruby{茶}{ちや}の
\ruby{室}{ま}をも、
\ruby{一}{ひ}ㇳ
\ruby{風}{ふう}ある
\ruby{氣性}{きし|やう}からは、
\ruby{床}{とこ}の
\ruby{間}{ま}さへ
\ruby{無}{な}き
\ruby{室}{へや}と
\ruby{賤}{いや}しく
\ruby{思}{おも}ふなるべし。

