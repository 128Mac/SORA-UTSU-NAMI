\Entry{其十二}

% メモ 校正終了 2024-05-13
\原本頁{64-11}%
\ruby{市中}{まち|なか}の
\ruby{事}{こと}なれば
\ruby{廣}{ひろ}く
はあらねど、
%
\ruby{特}{わざ}と
\ruby{花物}{はな|もの}を
\ruby{{\換字{嫌}}}{きら}ひたる
\ruby{常磐木}{とき|は|ぎ}
のみの
\ruby{庭}{には}の、
%
\ruby{見}{み}えぬところに
\ruby{人}{ひと}の
\ruby{手}{て}の
\ruby{十{\換字{分}}}{じふ|ぶん}に
\ruby{用}{もち}ひられたる
\ruby[<j||]{證}{しるし}とて、
%
\原本頁{65-2}\改行%
\ruby{枝々}{えだ|〳〵}は
\ruby{好}{よ}き
ほどに
\ruby{折}{を}り
\ruby{合}{あ}ひて
\ruby{茂}{しげ}り
ながら、
%
\ruby{隈々}{くま|〴〵}は
\ruby{汚}{むさ}からで
\ruby{明}{あか}るく、
%
わづかに
\ruby{大}{おほき}からず
\ruby{小}{ちひ}さからぬ
\ruby{燈籠}{とう|ろう}
\ruby{一}{ひと}つの
\ruby{形狀}{かた|ち}も
\ruby{佳}{よ}く
\ruby{時代}{じ|だい}もありて
\ruby{一寸}{ちよ|つと}
\ruby{面白}{おも|しろ}きが
ほかには、
%
\ruby{別}{べつ}に
\ruby{此}{これ}といふ
\ruby{價}{ね}の
\ruby{高}{たか}き
\ruby{樹}{き}も
\ruby{珍}{めづ}らしき
\ruby{石}{いし}も
\ruby{無}{な}けれど、
%
\ruby{一體}{いつ|たい}の
\ruby{調子}{てう|し}の
\ruby{蟠屈無}{わだ|かまり|な}く
すらりと、
%
\ruby{幽閑}{しづ|か}にして、
%
\ruby[|g|]{特設}{こしら}へ
\ruby{氣}{ぎ}も
\ruby{無}{な}く、
%
\ruby{見}{み}る
\ruby{眼}{め}
\ruby{安}{やす}く
\ruby{穩和}{おだ|やか}なる
ところに
\ruby[||j>]{自}{おのづ}% ルビが重なるので調整
\ruby[||j>]{然}{ から}
% \ruby{自然}{おのづ|から}
\ruby[||j>]{{\換字{飽}}}{ あ }かぬ
\原本頁{65-7}\改行%
\ruby{床}{ゆか}しさ
ありて、
%
\ruby{夏}{なつ}は
\ruby{{\換字{梢}}}{こずゑ}に
\ruby{新月}{にひ|づき}の
\ruby{低}{ひく}う
\ruby{懸}{かゝ}る
\ruby{{\換字{宵}}}{よひ}、
%
\ruby[|g|]{不如歸}{ほとゝぎす}の
\ruby{一}{ひ}ト
\ruby{聲}{こゑ}をも
\ruby{待}{ま}ち
\ruby{得}{え}ば
\ruby{嘸}{さぞ}
と
おもはれ、
%
\ruby{{\換字{冬}}}{ふゆ}は
\ruby[||j>]{雀}{すゞめ}
\ruby[||j>]{膨}{ ふく}るゝ
% \ruby{雀膨}{すゞめ|ふく}るゝ
\ruby{{\換字{寒}}}{さむ}き
\ruby{日}{ひ}の
\ruby{雲破}{くも|やぶ}れて
\ruby{時雨}{しぐ|れ}
\原本頁{65-9}\改行%
はら〳〵と
\ruby{落}{お}つる
\ruby{夕}{ゆふべ}、
%
\ruby{或}{ある}は
\ruby{{\換字{又}}}{また}
\ruby{{\換字{雪}}}{ゆき}の
\ruby{薄綿}{うす|わた}
\ruby{萬物}{ばん|ぶつ}を
\ruby{包}{つゝ}む
\ruby{曉}{あした}など、
%
\ruby{如何}{い|か}にと
\ruby{{\換字{忍}}}{しの}ばるゝ
ばかり
なり。

\原本頁{65-11}%
されば
\ruby{折}{をり}ふしは
\ruby{此家}{こ|ゝ}にも
\ruby{出入}{で|い}りする
\ruby{筑波}{つく|ば}が
\ruby{氣}{き}に
\ruby{入}{い}りの
\ruby{骨董屋}{だう|ぐ|や}の
\原本頁{66-1}\改行%
\ruby{老{\換字{漢}}}{ぢ|ゞ}に、
%
\ruby{利齋}{り|さい}と
いひて、
%
\ruby{内々}{ない|〳〵}は
\ruby{茶{\換字{道}}}{ちや|だう}
\ruby{天狗}{てん|ぐ}の
\ruby{小賢}{こ|ざか}しき
\ruby{男}{をとこ}、
%
\ruby{此}{こ}の
\ruby{庭}{には}を
\ruby{見}{み}て、

\原本頁{66-3}%
『
\ruby{猫}{ねこ}の
\ruby{額}{ひたひ}
ぐらゐの
\ruby{庭}{には}だが
\ruby{彼}{あ}の
\ruby{人}{ひと}の
\ruby{住居}{すま|ゐ}に
\ruby{彼}{あ}の
\ruby{庭}{には}は
\ruby{何}{なん}とも
いへない。
%
\ruby{庭}{には}の
\ruby{出來}{で|き}が
\ruby{好}{い}い
ばかり
では
\ruby{無}{な}い、
%
\ruby{彼}{あ}の
こつくり
した
\ruby{素樸}{ぢ|み}の
\ruby{景色}{け|しき}の
\ruby{中}{なか}に、
%
\ruby{繪}{ゑ}の
\ruby{{\換字{浮}}}{う}いて
\ruby{出}{で}た
やうに
\ruby{美麗}{き|れい}な
\ruby{福相}{ふく|さう}の
\ruby{美人}{び|じん}の
\ruby{彼}{あ}の
\原本頁{66-6}\改行%
\ruby{人}{ひと}が
\ruby{澄}{す}まして
\ruby{居}{ゐ}る
\ruby[||j>]{對}{うつり}
\ruby[||j>]{照}{ あひ}
% \ruby{對照}{うつり|あひ}
といふものは、
%
\ruby{何}{なん}のことは
\ruby{無}{な}い、
%
\ruby{茶壁}{ちや|かべ}の、
%
\原本頁{66-7}\改行%
\ruby{何}{なに}も
\ruby{無}{な}い
\ruby{床}{とこ}に
\ruby{一輪}{いち|りん}の
\ruby{白牡丹}{はく|ぼ|たん}を
\ruby{活}{い}けた
やうな
もので、
%
\ruby{一}{ひ}ト
\ruby{層}{きは}
\ruby{人}{ひと}の
\原本頁{66-8}\改行%
\ruby{眼}{め}を
\ruby{驚}{おどろ}かす。
%
\ruby{彼}{あ}の
\ruby{人}{ひと}が
\ruby{花}{はな}だから
\ruby{花}{はな}は
\ruby{要}{い}らない。
%
これを
\ruby{思}{おも}へば
\ruby{花}{はな}と
\ruby{見}{み}られる
ほどの
\ruby{容姿}{きり|よう}も
\ruby{無}{な}い
\ruby{女}{をんな}
なぞが、
%
\ruby{自{\換字{分}}}{じ|ぶん}の% 原本通り非グループルビ
\ruby{庭{\換字{前}}}{には|さき}に
\ruby{花}{はな}を
\ruby{植}{う}ゑたり
なんぞ
して、
\ruby{妙}{めう}に
\ruby[|g|]{優美}{やさし}がつて
\ruby{好}{い}い
\ruby{氣}{き}になつて
\ruby{居}{ゐ}ても、
%
\ruby{下手}{へ|た}に
\ruby{花}{はな}の
\ruby{{\換字{近}}傍}{そ|ば}に
でも
\ruby[|g|]{彷徨}{まごつ}かう
ものなら、
%
\ruby[|g|]{宛然}{まるで}
\ruby{海棠}{かい|だう}の
\ruby{下}{した}で
\ruby{狸}{たぬき}が
チンチン
でも
\ruby{仕}{し}て
\ruby{居}{ゐ}るやうに
\ruby{見}{み}えるのが
\ruby{多}{おほ}い。
%
\ruby{茶{\換字{道}}}{ち|や}を
\ruby{知}{し}らない
\ruby{奴}{やつ}は
\原本頁{67-2}\改行%
まあ
\ruby{其樣}{そ|ん}な
ものだが、
%
\ruby{彼庭}{あ|れ}が
\ruby{彼}{あ}の
\ruby{人}{ひと}の
\ruby{好}{この}みで
\ruby{出來}{で|き}た
といへば、
%
\原本頁{67-3}\改行%
\ruby{彼}{あ}の
お
\ruby{彤}{とう}さん
といふ
\ruby{人}{ひと}は
\ruby{顏}{かほ}が
\ruby{美}{い}い
ばかり
ぢやあ
\ruby{無}{な}い、
%
\ruby{何}{なに}も
\ruby{彼}{か}も
\原本頁{67-4}\改行%
\ruby{解}{わか}る
\ruby{人}{ひと}だ、
%
\ruby{中々}{なか|〳〵}
\ruby{一}{ひ}ト
\ruby{{\換字{通}}}{とほ}りや
\ruby{二}{ふ}タ
\ruby{{\換字{通}}}{とほ}りの
\ruby{人}{ひと}で
\ruby{無}{な}い。
%
\ruby{{\換字{道}}理}{だう|り}で
\ruby{物品}{も|の}を
\原本頁{67-5}\改行%
\ruby{買}{か}つても
\ruby{買}{か}ひつ
\ruby{振}{ぷ}りが
\ruby{可}{い}い。
%
そして
\ruby{倦}{あ}きつぽい
\ruby{彼}{あ}の
\ruby{筑波}{つく|ば}さんが、
%
\原本頁{67-6}\改行%
\ruby{何年}{なん|ねん}
といふもの
こびり
\ruby{付}{つ}いて
\ruby{居}{ゐ}る。
%
どうも
\ruby{偉}{えら}い、
%
\ruby{茶{\換字{道}}}{ち|や}を
\ruby{知}{し}つて
\ruby{居}{ゐ}るから
\ruby{何樣}{ど|う}も
\ruby{偉}{えら}い。
』

\原本頁{67-8}%
と、
%
\ruby{自己}{お|の}が
\ruby{高慢}{かう|まん}を
\ruby{{\換字{交}}}{ま}ぜて
\ruby{{\換字{評}}}{ひやう}したる
\ruby{事}{こと}ありき。

\原本頁{67-9}%
\ruby{家}{いへ}の
\ruby{一角}{いつ|かく}の
\ruby{小座敷}{こ|ざ|しき}の、
%
\ruby[<j||]{僅}{わづか}
\ruby{四疊{\換字{半}}}{よ|でふ|はん}
には
\ruby{{\換字{過}}}{す}ぎねど、
%
\ruby{此}{こ}の
\ruby{庭}{には}を
\ruby{東南}{たつ|み}に
\原本頁{67-10}\改行%
\ruby{受}{う}けて、
%
\ruby{陽氣}{やう|き}
なれど
\ruby{廂}{ひさし}を
\ruby{長}{なが}く
\ruby{仕}{し}たれば
\ruby{明}{あか}る
\ruby{{\換字{過}}}{す}ぎず
\ruby{建}{た}てられたるが
\ruby{中}{なか}に
\ruby{今}{いま}しも
お
\ruby{彤}{とう}
お
\ruby{龍}{りう}は
\ruby{相}{あひ}
\ruby{對}{たい}して
\ruby{坐}{すわ}れり。
%
\ruby[|g|]{薩{\換字{摩}}杉}{さつま}の
\ruby{天}{てん}
\ruby[|g|]{井板}{じやう}の
\ruby{木理}{も|く}
\ruby{美}{うる}はしく、
%
\ruby{根岸茶}{ね|ぎ|し}の
\ruby{壁}{かべ}の
\ruby{色}{いろ}
\ruby{沈着}{おち|つ}きて、
%
\ruby{床}{とこ}には
お
\ruby{彤}{とう}が
\ruby{好}{この}みか
\ruby{筑波}{つく|ば}が
\ruby{好}{この}みかは
\ruby{知}{し}らず
\ruby{明人}{みん|ひと}らしき% 「明人」(盲人に対する)目の見える人/光明正大な人(https://kotobank.jp/ から)
\ruby{書}{しよ}の
\ruby{小幅}{せう|ふく}を
\ruby{掛}{か}けて、
%
\ruby{棚}{たな}には
これ
\原本頁{68-3}\改行%
は
\ruby{慥}{たしか}に
\ruby[|g|]{主人}{あるじ}が
\ruby[||j>]{玩}{もて}
\ruby[||j>]{弄}{あそび}に
% \ruby{玩弄}{もて|あそび}に
\ruby{疑}{うたが}ひ
\ruby{無}{な}き
\ruby{繪卷}{ゑ|まき}
など
\ruby{取}{と}り
\ruby{繕}{つくろ}はず
\ruby{載}{の}せたり。
%
\ruby{出入口}{で|いり|ぐち}、
%
\ruby{窓}{まど}の
\ruby{取}{と}り
\ruby{方}{かた}
なんど
\ruby{總}{す}べて
\ruby{茶室}{ちや|しつ}めきたれど、
%
\ruby{釜}{かま}を
\ruby{掛}{か}くる
ことは
\ruby{{\換字{嫌}}}{きら}へる
にや
\ruby{爐}{ろ}は
\ruby{切}{き}りて
あらず、
%
\ruby{一面}{いち|めん}に
\ruby{美}{うつく}しき
\ruby{敷物}{しき|もの}の
\ruby{敷}{し}き
つめられて、
%
\ruby{一方}{いつ|ぱう}の
\ruby{隅}{すみ}には
\ruby{今}{いま}
\ruby{物}{もの}ならぬ
\ruby[<j||]{女}{をんな}% 原本では「女」「用」の直後に空きがあるが、詰めた
\ruby[||j>]{用}{もちひ}の
\ruby{螺塡}{ら|でん}の
\ruby{黑}{くろ}き
\ruby{小机}{こ|づくゑ}の、
%
\原本頁{68-7}\改行%
\ruby{漆光}{て|り}は
\ruby{既}{すで}に
\ruby{脫}{ぬ}けて
\ruby{好}{よ}き
\ruby{頃}{ころ}に
\ruby{{\換字{古}}}{ふる}びたる
\ruby{善美}{けつ|こう}
いふ
ばかり
\ruby{無}{な}きが
\ruby{上}{うへ}に、
%
\原本頁{68-8}\改行%
\ruby{同}{おな}じ
やう
なる
\ruby{手}{て}の
\ruby{小}{ちひ}さき
\ruby[||j>]{硯}{すゞり}
\ruby[||j>]{箱}{ ばこ}
% \ruby{硯箱}{すゞり|ばこ}
\ruby{置}{お}かれ、
%
\ruby{机下}{し|た}にも
\ruby{同}{おな}じ
やう
なる
\ruby{手匣}{て|ばこ}の
\ruby{置}{お}かれたる、
%
\ruby{此}{こ}の
\ruby{{\換字{前}}}{まへ}は
\ruby{女主人}{あ|る|じ}が
\ruby{常}{つね}の
\ruby{座處}{ゐ|どころ}
なるべし。

\原本頁{68-10}%
お
\ruby{彤}{とう}は
\ruby{今}{いま}
\ruby{其座}{そ|れ}を
\ruby[|g|]{背後}{うしろ}に
して、
%
\ruby{是眞}{ぜ|しん}が% 幕末・明治の日本画家、漆芸家の柴田是真(しばたぜしん)のようである
\ruby{蒔繪}{まき|ゑ}の
\ruby{桐胴}{きり|どう}の
\ruby{手爐}{て|あぶり}の
\ruby{小}{ちひ}さきを
\ruby{横手}{よこ|て}に、
%
\ruby[|g|]{此方}{こなた}を
\ruby{向}{む}きて
\ruby{茶}{ちや}を
\ruby{淹}{い}れ
\ruby{居}{を}れば、
%
お
\ruby{龍}{りう}は
\ruby{淸楚}{さつ|ぱり}
とこそ
\ruby{仕}{し}て
\ruby{居}{を}れ、
%
おのが
\ruby[|g|]{銘仙織}{めいせん}
づくめの
\ruby{衣服}{な|り}の
\ruby{身}{み}の、
%
\ruby{居}{を}るには
\ruby{憚}{はゞか}らるゝ% 「憚 は(ゞ)か」
\原本頁{69-2}\改行%
ほどの
お
\ruby{納{\換字{戸}}}{なん|ど}
\ruby{緞子}{ゞん|す}の
\ruby{蒲團}{ふ|とん}に、
%
やゝ
\ruby{安}{おちつ}きかぬるが
\ruby{如}{ごと}く
\ruby{坐}{すわ}りて、
%
\ruby[<j||]{客}{きやく}と
いへば
\ruby{客}{きやく}
ながら、
%
おのづから
\ruby{{\換字{貧}}富}{ひん|ぷ}の
\ruby[|g|]{相{\換字{違}}}{たがひ}に
\ruby{壓}{お}さるゝ
\ruby{氣味}{き|み}
あるを
\ruby[<g>]{如何}{いかん}とも
\ruby{仕{\換字{難}}}{し|がた}く、
%
たゞ
おとなしく
\ruby{内端}{うち|ば}に
\ruby{控}{ひか}へたるが、
%
\ruby{{\換字{猶}}}{なほ}
\ruby{持}{も}つて
\ruby{生}{うま}れし
\ruby{氣象}{き|しやう}の
\ruby{徳}{とく}には
\ruby{少}{すこ}しも
\ruby{萎}{め}げぬ
\ruby{顏}{かほ}つきの
\ruby{我}{われ}は
\ruby{我}{われ}だけに
\ruby{冴}{さ}えて、
%
\ruby{毫末}{いさ|ゝか}の
\ruby{隔}{へだ}て
\ruby{氣}{ぎ}も
\ruby{無}{な}く
\ruby{人}{ひと}を
\ruby{親}{したし}む
\ruby{眼}{め}の
\ruby{中}{うち}
\ruby{凉}{すゞ}しく
\ruby{相}{あひ}
\ruby{對}{むか}へるさま、
%
\原本頁{69-8}\改行%
たとへば
\ruby[|g|]{一人}{ひとり}は
\ruby{晴}{はれ}の
\ruby{日}{ひ}の
\ruby{晝}{ひる}に
\ruby{笑}{ゑま}へる
\ruby{牡丹}{ぼ|たん}
ならば、
%
\ruby[|g|]{一人}{ひとり}は
\ruby{野}{の}の
\ruby{風}{かぜ}の
そよ
\ruby{吹}{ふ}く
\ruby{秋}{あき}に、
%
\ruby{{\換字{寒}}}{さむ}さ
\ruby{知}{し}らぬ
\ruby{色}{いろ}して
\ruby{{\換字{咲}}}{さ}ける
\ruby[|g|]{木芙蓉}{ふよう}
とも
いひつ
べし。
