\Entry{其十二}

\ruby{市中}{まち|なか}の
\ruby{事}{こと}なれば
\ruby{廣}{ひろ}くはあらねど、
\ruby{特}{わざ}と
\ruby{花物}{はな|もの}を
\ruby{{\GWI{u5acc-k}}}{きら}ひたる
\ruby[g]{常磐木}{ときはぎ}のみの
\ruby{庭}{には}の、
\ruby{見}{み}えぬところに
\ruby{人}{ひと}の
\ruby{手}{て}の
\ruby{十分}{じう|ぶん}に
\ruby{用}{もち}ひられたる
\ruby{證}{しるし}とて、
\ruby{枝々}{えだ|〳〵}は
\ruby{好}{よ}きほどに
\ruby{折}{お}り
\ruby{合}{あ}ひて
\ruby{茂}{しげ}りながら、
\ruby{隈々}{くま|〴〵}は
\ruby{汚}{むさ}からで
\ruby{明}{あか}るく、わづかに
\ruby{大}{おほき}からず
\ruby{小}{ちひ}さからぬ
\ruby{燈籠一}{とう|ろう|ひと}つの
\ruby[g]{形狀}{かたち}も
\ruby{佳}{よ}く
\ruby{時代}{じ|だい}もありて
\ruby{一寸面白}{ちよ|つと|おも|しろ}きがほかには、
\ruby{別}{べつ}に
\ruby{此}{これ}といふ
\ruby{價}{ね}の
\ruby{高}{たか}き
\ruby{樹}{き}も
\ruby{珍}{めづ}らしき
\ruby{石}{いし}も
\ruby{無}{な}けれど、
\ruby{一體}{いつ|たい}の
\ruby{調子}{てう|し}の
\ruby[g]{蟠屈無}{わだかまりな}くすらりと、
\ruby[g]{幽閑}{しづか}にして、
\ruby[g]{特設}{こしら}へ
\ruby{氣}{ぎ}も
\ruby{無}{な}く、
\ruby{見}{み}る
\ruby{眼安}{め|やす}く
\ruby{穩和}{おだ|やか}なるところに
\ruby[g]{自然{\GWI{u98fd-k}}}{おのづからあ}かぬ
\ruby{床}{ゆか}しさありて、
\ruby{夏}{なつ}は
\ruby{{\GWI{u68a2-k}}}{こずえ}に
\ruby{新月}{にひ|づき}の
\ruby{低}{ひく}う
\ruby{懸}{かヽ}る
\ruby{{\GWI{u5bb5-k}}}{よひ}、
\ruby[g]{不如歸}{ほとヽぎす}の
\ruby{一}{ひ}ㇳ
\ruby{聲}{こゑ}をも
\ruby{待}{ま}ち
\ruby{得}{え}ば
\ruby{嘸}{さぞ}とおもはれ、
\ruby{{\GWI{u51ac-k}}}{ふゆ}は
\ruby{雀膨}{すヾめ|ふく}るヽ
\ruby{{\GWI{u5bd2-k}}}{さむ}き
\ruby{日}{ひ}の
\ruby{雲破}{くも|やぶ}れて
\ruby[g]{時雨}{しぐれ}はら〳〵と
\ruby{落}{お}つる
\ruby{夕}{ゆふべ}、
\ruby{或}{ある}は
\ruby{{\GWI{u53c8-k}}{\GWI{u96ea-k}}}{また|ゆき}の
\ruby{薄綿萬物}{うす|わた|ばん|ぶつ}を
\ruby{包}{つヽ}む
\ruby{曉}{あした}など、
\ruby{如何}{い|か}にと
\ruby{忍}{しの}ばるヽばかりなり。

されば
\ruby{折}{をり}ふしは
\ruby{此家}{こ|ヽ}にも
\ruby{出入}{で|い}りする
\ruby[g]{筑波}{つくば}が
\ruby{氣}{き}に
\ruby{入}{い}りの
\ruby[g]{骨董屋}{だうぐや}の
\ruby{老漢}{ぢ|ヾ}に、
\ruby{利齋}{り|さい}といひて、
\ruby{内々}{ない|〳〵}は
\ruby{茶{\GWI{u9053-k}}天狗}{ちや|だう|てん|ぐ}の
\ruby[g]{小賢}{こざか}しき
\ruby{男}{をとこ}、
\ruby{此}{こ}の
\ruby{庭}{には}を
\ruby{見}{み}て、

『
\ruby{猫}{ねこ}の
\ruby{額}{ひたひ}ぐらゐの
\ruby{庭}{には}だが
\ruby{彼}{あ}の
\ruby{人}{ひと}の
\ruby[g]{住居}{すまゐ}に
\ruby{彼}{あ}の
\ruby{庭}{には}は
\ruby{何}{なん}ともいへない。
\ruby{庭}{には}の
\ruby{出來}{で|き}が
\ruby{好}{よ}いばかりでは
\ruby{無}{な}い、
\ruby{彼}{あ}のこつくりした
\ruby{素樸}{ぢ|み}の
\ruby{景色}{け|しき}の
\ruby{中}{なか}に、
\ruby{繪}{ゑ}の
\ruby{{\GWI{u6d6e-k}}}{う}いて
\ruby{出}{で}たやうに
\ruby{美麗}{き|れい}な
\ruby{福相}{ふく|さう}の
\ruby{美人}{び|じん}の
\ruby{彼}{あ}の
\ruby{人}{ひと}が
\ruby{澄}{す}まして
\ruby{居}{ゐ}る
\ruby[g]{對照}{うつりあひ}といふものは、
\ruby{何}{なん}のことは
\ruby{無}{な}い、
\ruby{茶壁}{ちや|かべ}の、
\ruby{何}{なに}も
\ruby{無}{な}い
\ruby{床}{とこ}に
\ruby{一輪}{いち|りん}の
\ruby[g]{白牡丹}{はくぼたん}を
\ruby{活}{い}けたやうなもので、
\ruby{一}{ひ}ㇳ
\ruby{層人}{きは|ひと}の
\ruby{眼}{め}を
\ruby{驚}{おどろ}かす。

\ruby{彼}{あ}の
\ruby{人}{ひと}が
\ruby{花}{はな}だから
\ruby{花}{はな}は
\ruby{要}{い}らない。
これを
\ruby{思}{おも}へば
\ruby{花}{はな}と
\ruby{見}{み}られるほどの
\ruby{容姿}{きり|よう}も
\ruby{無}{な}い
\ruby{女}{をんな}なぞが、
\ruby{自分}{じ|ぶん}の
\ruby{庭前}{には|さき}に
\ruby{花}{はな}を
\ruby{植}{う}ゑたりなんぞして
\ruby{妙}{めう}に
\ruby[g]{優美}{やさし}がつて
\ruby{好}{い}い
\ruby{氣}{き}になつて
\ruby{居}{ゐ}ても、
\ruby{下手}{へ|た}に
\ruby{花}{はな}の
\ruby{近傍}{そ|ば}にでも
\ruby[g]{彷徨}{まごつ}かうものなら、
\ruby[g]{宛然海棠}{まるでかいだう}の
\ruby{下}{した}で
\ruby{狸}{たぬき}がチンチンでも
\ruby{仕}{し}て
\ruby{居}{ゐ}るやうに
\ruby{見}{み}えるのが
\ruby{多}{おほ}い。
\ruby{茶{\GWI{u9053-k}}}{ち|や}を
\ruby{知}{し}らない
\ruby{奴}{やつ}はまあ
\ruby{其樣}{そ|ん}なものだが、
\ruby{彼庭}{あ|れ}が
\ruby{彼}{あ}の
\ruby{人}{ひと}の
\ruby{好}{この}みで
\ruby{出來}{で|き}たといへば
\ruby{彼}{あ}のお
\ruby{彤}{とう}さんといふ
\ruby{人}{ひと}は
\ruby{顏}{かほ}が
\ruby{美}{い}いばかりぢやあ
\ruby{無}{な}い、
\ruby{何}{なに}も
\ruby{彼}{か}も
\ruby{解}{わか}る
\ruby{人}{ひと}だ、
\ruby{中々}{なか|〳〵}
\ruby{一}{ひ}ㇳ
\ruby{{\GWI{u901a-k}}}{とほ}りや
\ruby{二}{ふ}タ
\ruby{{\GWI{u901a-k}}}{とほ}りの
\ruby{人}{ひと}で
\ruby{無}{な}い。
\ruby{道理}{だう|り}で
\ruby{物品}{も|の}
\ruby{買}{か}つても
\ruby{買}{か}ひつ
\ruby{振}{ぷ}りが
\ruby{可}{い}い。
そして
\ruby{倦}{あ}きつぽい
\ruby{彼}{あ}の
\ruby[g]{筑波}{つくば}さんが、
\ruby{何年}{なん|ねん}といふものこびり
\ruby{付}{つ}いて
\ruby{居}{ゐ}る。
どうも
\ruby{偉}{えら}い、
\ruby{茶{\GWI{u9053-k}}}{ち|や}を
\ruby{知}{し}つて
\ruby{居}{ゐ}るから
\ruby{何樣}{ど|う}も
\ruby{偉}{えら}い。
』

と、
\ruby{自己}{お|の}が
\ruby{高慢}{かう|まん}を
\ruby{交}{ま}ぜて
\ruby{{\GWI{u8a55-k}}}{ひよう}したる
\ruby{事}{こと}ありき。

\ruby{家}{いへ}の
\ruby{一角}{いつ|かく}の
\ruby{小座敷}{こ|ざ|しき}の、
\ruby[g]{僅四疊{\GWI{u534a-k}}}{わづかよでふはん}には
\ruby{{\GWI{u904e-k}}}{す}ぎねど、
\ruby{此}{こ}の
\ruby{庭}{には}を
\ruby[g]{東南}{たつみ}に
\ruby{受}{う}けて、
\ruby[g]{陽氣}{やうき}なれど
\ruby{廂}{ひさし}を
\ruby{長}{なが}く
\ruby{仕}{し}たれば
\ruby{明}{あか}る
\ruby{{\GWI{u904e-k}}}{す}ぎず
\ruby{建}{た}てられたるが
\ruby{中}{なか}に
\ruby{今}{いま}しもお
\ruby{彤}{とう}お
\ruby{龍}{りう}は
\ruby{相對}{あひ|たい}して
\ruby{坐}{すわ}れり。
\ruby{薩摩杉}{さ|つ|ま}の
\ruby[g]{天井板}{てんじやう}の
\ruby[g]{木理美}{もくうる}はしく、
\ruby{根岸茶}{ね|ぎ|し}の
\ruby{壁}{かべ}の
\ruby[g]{色沈着}{いろおちつ}きて、
\ruby{床}{とこ}にはお
\ruby{彤}{とう}が
\ruby{好}{この}みか
\ruby{筑波}{つく|ば}が
\ruby{好}{この}みかは
\ruby{知}{し}らず
\ruby{明人}{みん|ひと}らしき
\ruby{書}{しよ}の
\ruby{小幅}{せう|ふく}を
\ruby{掛}{か}けて、
\ruby{棚}{たな}にはこれは
\ruby{慥}{たしか}に
\ruby[g]{主人}{あるじ}が
\ruby[g]{玩弄}{もてあそび}に
\ruby{疑}{うたが}ひ
\ruby{無}{な}き
\ruby{繪卷}{ゑ|まき}など
\ruby{取}{と}り
\ruby{繕}{つくろ}はず
\ruby{載}{の}せたり。
\ruby[g]{出入口}{でいりぐち}、
\ruby{窓}{まど}の
\ruby{取}{と}り
\ruby{方}{かた}なんど
\ruby{總}{す}べて
\ruby{茶室}{ちや|しつ}めきたれど、
\ruby{釜}{かま}を
\ruby{掛}{か}くることは
\ruby{{\GWI{u5acc-k}}}{きら}へるにや
\ruby{爐}{ろ}は
\ruby{切}{き}りてあらず、
\ruby{一面}{いち|めん}に
\ruby{美}{うつく}しき
\ruby{敷物}{しき|もの}の
\ruby{敷}{し}きつめられて、
\ruby{一方}{いつ|ぱう}の
\ruby{隅}{すみ}には
\ruby{今物}{いま|もの}ならぬ
\ruby{女用}{をんな|もちひ}の
\ruby{螺塡}{ら|でん}の
\ruby{黑}{くろ}き
\ruby{小机}{こづ|くゑ}の、
\ruby{漆光}{て|り}は
\ruby{既}{すで}に
\ruby{{\GWI{u812b-k}}}{ぬ}けて
\ruby{好}{よ}き
\ruby{頃}{ころ}に
\ruby{古}{ふる}びたる
\ruby{善美}{けつ|こう}いふばかり
\ruby{無}{な}きが
\ruby{上}{うへ}に、
\ruby{同}{おな}じやうなる
\ruby{手}{て}の
\ruby{小}{ちひ}さき
\ruby[g]{硯箱置}{すヾりばこお}かれ、
\ruby{机下}{し|た}にも
\ruby{同}{おな}じやうなる
\ruby{手匣}{て|ばこ}の
\ruby{置}{お}かれたる、
\ruby{此}{こ}の
\ruby{前}{まへ}は
\ruby{女主人}{あ|る|じ}が
\ruby{常}{つね}の
\ruby{座處}{ゐど|ころ}なるべし。

お
\ruby{彤}{とう}は
\ruby[g]{今其座}{いまそれ}を
\ruby[g]{背後}{うしろ}にして、
\ruby[g]{是眞}{ぜしん}が
\ruby{蒔繪}{まき|ゑ}の
\ruby{桐胴}{きり|どう}の
\ruby{手爐}{てあ|ぶり}の
\ruby{小}{ちひ}さきを
\ruby[g]{横手}{よこて}に、
\ruby[g]{此方}{こなた}を
\ruby{向}{む}きて
\ruby{茶}{ちや}を
\ruby{淹}{い}れ
\ruby{居}{を}れば、お
\ruby{龍}{りう}は
\ruby{清楚}{さつ|ぱり}とこそ
\ruby{仕}{し}て
\ruby{居}{を}れ、おのが
\ruby[g]{銘仙織}{めいせん}づくめの
\ruby{衣服}{な|り}の
\ruby{身}{み}の、
\ruby{居}{を}るには
\ruby{憚}{はばか}らるヽほどのお
\ruby[g]{納戸緞子}{なんどどんす}の
\ruby{蒲團}{ふ|とん}に、やヽ
\ruby{安}{おちつ}きかぬるが
\ruby{如}{ごと}く
\ruby{坐}{すわ}りて、
\ruby{客}{きやく}といへば
\ruby{客}{きやく}ながら、おのづから
\ruby{貧富}{ひん|ぷ}の
\ruby[g]{相違}{たがひ}に
\ruby{壓}{お}さるヽ
\ruby{氣味}{き|み}あるを
\ruby[g]{如何}{いかん}とも
\ruby{仕難}{し|がた}く、たヾおとなしく
\ruby{内端}{うち|ば}に
\ruby{控}{ひか}へたるが、
\ruby{{\GWI{u7336-k}}持}{なほ|も}つて
\ruby{生}{うま}れし
\ruby{氣象}{きし|やう}の
\ruby{徳}{とく}には
\ruby{少}{すこ}しも
\ruby{萎}{め}げぬ
\ruby{顏}{かほ}つきの
\ruby{我}{われ}は
\ruby{我}{われ}だけに
\ruby{冴}{さ}えて、
\ruby{毫末}{いさ|さか}の
\ruby{隔}{へだ}て
\ruby{氣}{ぎ}も
\ruby{無}{な}く
\ruby{人}{ひと}を
\ruby{親}{したし}む
\ruby{眼}{め}の
\ruby{中凉}{うち|すヾ}しく
\ruby{相對}{あひ|むか}へるさま、たとへば
\ruby{一人}{ひと|り}は
\ruby{晴}{はれ}の
\ruby{日}{ひ}の
\ruby{晝}{ひる}に
\ruby{笑}{わら}へる
\ruby{牡丹}{ぼ|たん}ならば、
\ruby[g]{一人}{ひとり}は
\ruby{野}{の}の
\ruby{風}{かぜ}のそよ
\ruby{吹}{ふ}く
\ruby{秋}{あき}に、
\ruby{{\GWI{u5bd2-k}}}{さむ}さ
\ruby{知}{し}らぬ
\ruby{色}{いろ}して
\ruby{咲}{さ}ける
\ruby[g]{木芙蓉}{ふよう}ともいひつべし。

