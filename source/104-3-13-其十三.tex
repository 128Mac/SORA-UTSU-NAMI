\Entry{其十三}

% メモ 校正終了 2024-05-13
\原本頁{70-1}%
\ruby{{\換字{古}}薩{\換字{摩}}}{こ|さつ|ま}か
\ruby{{\換字{古}}九谷}{こ|くた|に}と
ありさうな
ところを
\ruby{然}{さ}は
\ruby{無}{な}くて、
%
\ruby{永樂}{えい|らく}
あたりの
\ruby{稀品}{き|ひん}
なるべし、
%
\ruby{形狀}{かた|ち}
\ruby{品格}{ひ|ん}
\ruby{佳}{よ}くして
\ruby{{\換字{彩}}釉}{いろ|ゑ}
\ruby[<j>]{快}{こゝろよ}く
\ruby{麗}{うる}はしき
\ruby[||j>]{京}{きやう}
\ruby[||j>]{燒}{ やき}の
% \ruby{京燒}{きやう|やき}の
\ruby[|-|]{茶}{ちや}% 行末行頭の境界付近なので特例処置を施す
\ruby{器}{き}を、
%
\ruby{五指}{ご|し}
\ruby[||j>]{白}{はく}
\ruby[||j>]{玉}{ぎよく}の
% \ruby{白玉}{はく|ぎよく}の
\ruby{如}{ごと}く
\ruby{美}{うつく}しき
\ruby{手}{て}に
\ruby{自}{みづか}ら
\ruby{扱}{あつか}ひて、
%
\ruby{既}{すで}に
\ruby{鎚目}{つち|め}の
\ruby{銀}{ぎん}
\ruby{瓶}{びん}の
\ruby{湯}{ゆ}を
\ruby[|g|]{徐々}{しづか}に
\ruby{注}{さ}し
\ruby{{\換字{終}}}{をは}り、
%
\ruby{今}{いま}や
\ruby{一盞}{いつ|さん}に
\ruby{玉露}{ぎよく|ろ}の
\ruby{花香}{はな|か}を
\ruby{湛}{たゝ}へて、
%
お
\ruby{彤}{とう}は
これをば
\ruby{與}{あた}へ
\ruby{{\換字{遣}}}{や}りつ、
%
\ruby{鍋島}{なべ|しま}の
\ruby{菓子皿}{くわ|し|ざら}
をば
\ruby{{\換字{又}}}{また}
\ruby{聊}{いさゝ}か
お
\ruby{龍}{りう}が
\ruby{方}{かた}へと
\ruby{推{\換字{進}}}{おし|すゝ}めたり。

\原本頁{70-7}%
お
\ruby{龍}{りう}は
\ruby{心底}{しん|そこ}より
\ruby{悅}{よろこ}びて
\ruby{茶}{ちや}を
\ruby{味}{あぢ}はひつ。

\原本頁{70-8}%
『
いつでも
\ruby{眞個}{ほん|たう}に
\ruby{勿體}{もつ|たい}ない
やうな
\ruby{佳良}{けつ|こう}な
\ruby{御茶}{お|ちや}ネ。
』

\原本頁{70-9}%
『
ホヽヽ、
%
お
\ruby{茶}{ちや}ばかり
\ruby{褒}{ほ}めずとも
\ruby{淹}{い}れ
\ruby{方}{かた}も
\ruby{褒}{ほ}めて、
%
お
\ruby{吳}{く}れな。
』

\原本頁{70-10}%
『
ホヽヽ、
%
そりやあ
もう、
%
\ruby{口}{くち}へ
\ruby{出}{だ}しては
\ruby{云}{い}はなくつても
‥‥。
』

\原本頁{70-11}%
『
オヤ
\ruby{左樣}{さ|う}、
%
\ruby{嬉}{うれ}しい
\ruby{人}{ひと}ネエ。
%
ぢやあ
まあ
\ruby{澤山}{たん|と}
\ruby{御菓子}{お|くわ|し}でも
\ruby{御食}{お|あが}り
なすつて。
』

\原本頁{71-2}%
『
\ruby{厭}{いや}ネエ、
%
ふざけて!。
%
\ruby{姊}{ねえ}さんは
\ruby{人}{ひと}が
\ruby{惡}{わる}いは。
』

\原本頁{71-3}%
と
お
\ruby{龍}{りう}は
\ruby{一寸}{ちよ|つと}
\ruby{瞋}{おこ}つたる
やうな
\ruby{顏}{かほ}して
\ruby{云}{い}ひ、

\原本頁{71-4}%
『
それに
\ruby{此}{こ}の
\ruby{御菓子}{お|くわ|し}は
\ruby{妾}{わたし}は
\ruby{澤山}{たく|さん}ですよ。
』

\原本頁{71-5}%
といふ。

\原本頁{71-6}%
『
\ruby{{\換字{嫌}}}{きら}ひ?。
』

\原本頁{71-7}%
と
\ruby{女主人}{あ|る|じ}は
\ruby{輕}{かろ}く
\ruby{眞面目}{ま|じ|め}に
\ruby{問}{と}ふ。
%
\ruby{問}{と}はれて
\ruby{莞爾}{にこ|やか}なる
\ruby{舊}{もと}に
\ruby{復}{かへ}りながら、

\原本頁{71-9}%
『
まあ
\ruby{左樣}{さ|う}なの。
』

\原本頁{71-10}%
と
\ruby{氣}{き}の
\ruby{毒}{どく}さうに
\ruby{答}{こた}へたるは、
%
\ruby{思}{おも}はず
\ruby{我}{わ}が
\ruby{好}{す}き
\ruby{{\換字{嫌}}}{きら}ひの
\ruby{我儘}{わが|まゝ}を
\ruby{口走}{くち|ばし}つたる
\ruby{無{\換字{遠}}慮}{ぶ|ゑん|りよ}を
\ruby{羞}{は}ぢて、
%
\ruby{今}{いま}さら
\ruby{詮方}{せん|かた}
\ruby{無}{な}くも
\ruby{{\換字{猶}}}{なほ}
\ruby{少}{すこ}し
\ruby{曖昧}{あい|まい}に
\ruby{言葉}{こと|ば}を
\ruby{濁}{にご}せる
なるべし。

\原本頁{72-2}%
『
いけなかつたネエ、
%
\ruby{甘味{\換字{嫌}}}{あま|い|ぎら}ひ
と
ばつかり
\ruby{思}{おも}つて
\ruby{居}{ゐ}て
\ruby{此品}{こ|れ}が
\ruby{{\換字{嫌}}}{きら}ひ
だつたとは
\ruby{知}{し}らなかつたよ。
%
もつともネ、
%
\ruby{一體}{いつ|たい}
\ruby{此}{これ}は
\ruby{御茶}{お|ちや}にも
\原本頁{72-4}\改行%
\ruby{餘}{あま}り
\ruby{賞}{ほ}めた
もの
ぢやあ
\ruby{無}{な}いの。
%
そればかり
ぢやあ
\ruby{無}{な}い、
%
\ruby{鳥貝}{とり|がひ}の
\ruby{御鮨}{お|す}もじ
だの
\ruby{玉簾}{たま|だれ}
だの
といふものは、
%
\ruby{惡}{わる}く
\ruby{氣取}{き|ど}つた
\ruby{女}{ひと}に
\ruby{食}{た}べさせて
\ruby{{\換字{遣}}}{や}れ
なんぞ
といふ
\ruby{位}{くらゐ}
のもの
だつた
のに、
%
つい
\ruby{妾}{わたし}が
\ruby{氣}{き}が
\ruby{注}{つ}かなかつたよ、
%
\ruby{堪{\換字{忍}}}{かん|にん}% 原文通り「堪忍」
おし。
%
\ruby{今}{いま}
\ruby{他}{ほか}の
ものを
\ruby{何}{なん}ぞ
あげるから。
』

\原本頁{72-8}%
『
\ruby{何故}{な|ぜ}?。
%
\ruby{氣取}{き|ど}つた
\ruby{女}{ひと}が
\ruby{何樣}{ど|う}か
\ruby{仕}{し}でも
するの?。
』

\原本頁{72-9}%
『
ソレ
\ruby{鳥貝}{とり|がひ}は% ルビが(とりがひ)なので「烏(からす)」でなく「鳥(とり)」
お
\ruby{{\換字{前}}}{まへ}
\ruby{早}{はや}くは
\ruby{咬}{か}み
\ruby{切}{き}れないし、
%
\ruby{玉簾}{たま|だれ}は
ホロ〳〵と
\ruby{零}{こぼ}れ
\ruby{{\換字{勝}}}{かち}
だし
\ruby{辛}{から}くは
あるし
するからネ。
%
いつまでも
\ruby{口}{くち}を
ムグ〳〵
させて
\ruby{居}{ゐ}たり、
%
だらし
\ruby{無}{な}く
\ruby{膝}{ひざ}を
\ruby{汚}{よご}して、
%
そして
\ruby{辛}{から}さを
\ruby{辛抱}{しん|ぼう}する
\ruby{泣顏}{なき|がほ}を
\ruby{仕}{し}て
\ruby{居}{ゐ}たり
するの
なんぞは
\ruby{見}{み}
\ruby{好}{い}い
もの
ぢやあ
\ruby{無}{な}い
からさ。
』

\原本頁{73-2}%
『
あらツ!、
%
\ruby{妾}{わたし}あ
\ruby{其樣}{そ|ん}な
\ruby{譯}{わけ}で
\ruby{{\換字{嫌}}}{きら}ひ
だつて
いふのぢやあ
\ruby{有}{あ}りませんは。
%
\ruby{姊}{ねえ}さんの
ところへ
\ruby{來}{き}て
\ruby{一寸}{ちよ|いと}だつて
\ruby{氣}{き}を
\ruby{置}{お}いて
なんぞ
\ruby{居}{ゐ}やあ
\ruby{仕}{し}ませんのに。
%
\ruby{好}{よ}う
ござんすよ、
%
\ruby[|g|]{一人}{ひとり}で
\ruby[|g|]{悉皆}{みんな}
\ruby{頂}{いたゞ}いて
\ruby{仕舞}{し|ま}つて、
%
\ruby{其邊}{そこ|いら}
\ruby[||j>]{中}{ぢゆう}
\ruby[||j>]{食}{ た }べ
\ruby{零}{こぼ}して、
%
そうして
\ruby{澤山}{たん|と}
\ruby{見}{み}つとも
\ruby{無}{な}い
\ruby{泣顏}{なき|がほ}を
して、
%
\ruby{笑}{わら}つて
いたゞきます
から。
』

\原本頁{73-7}%
『
ホヽホヽホヽ、
%
ホラ
\ruby{始}{はじ}まつたよ
お
\ruby{龍}{りう}ちやんの
\ruby{癇癖}{む|し}が。
%
だが
お
\ruby{{\換字{前}}}{まへ}が
\ruby{一寸}{ちよ|いと}
\ruby{口惜}{く|や}しい
といふ
\ruby[||j>]{思}{おもひ}
\ruby[||j>]{入}{ いれ}
% \ruby{思入}{おもひ|いれ}
をすると、
%
\ruby{色艶}{いろ|つや}は
\ruby{好}{よ}し、
%
\ruby{眼}{め}は
\ruby{淸}{すゞ}しいし、
%
\ruby[|g|]{眉毛}{まみえ}は
\ruby{奇麗}{き|れい}だし、
%
それが
\ruby[|g|]{悉皆}{みんな}
\ruby{役}{やく}に
\ruby{立}{た}つて
\ruby[||j>]{顏}{かほ}
\ruby[||j>]{中}{ぢゆう}が
% \ruby{顏中}{かほ|ぢゆう}が
\ruby{活}{い}きて
\ruby{見}{み}えて
\ruby{來}{き}て、
%
ほんとに
\ruby{婀娜}{あ|だ}で
\ruby[|g|]{可憐}{かはい}らしいよ。
』

\原本頁{73-11}%
『
\ruby{好}{よ}う
ござんすよ。
』

\原本頁{74-1}%
\ruby{此度}{こ|たび}は
いよ〳〵
\ruby{瞋}{いか}りて
いよ〳〵
\ruby{言葉}{こと|ば}
\ruby{少}{すくな}く、
%
\ruby{恨}{うら}めしげに
\GWI{u1b048-u3099}ろりと% 「志」+「濁点」
お
\ruby{彤}{とう}を
\ruby{睨}{にら}みて、
%
つん
として
\ruby{其}{そ}の
\ruby{儘}{まゝ}
\ruby{横}{よこ}を
\ruby{向}{む}かん
とせしが、
%
\ruby{閑事}{あだ|ごと}は
\ruby{兎}{と}に
\ruby{角}{かく}、
%
\ruby{云}{い}はで
\ruby{叶}{かな}はざる
\ruby{用事}{よう|じ}は
あるなり、
%
\ruby{霎時}{しば|し}
\ruby{間}{ま}を
\ruby{置}{お}きて
\ruby{面}{おもて}を
\ruby{擡}{あ}げ、

\原本頁{74-5}%
『
ネエ、
%
\ruby{姊}{ねえ}さん、
%
\ruby{今}{いま}
\ruby[|g|]{彼室}{あつち}で
\ruby{云}{い}ひ
かけた
のはネ、
%
\ruby{眞個}{ほん|と}に
\ruby{妾}{わたし}の
\ruby{御願}{お|ねが}ひの
\ruby{事}{こと}
なんですから
\ruby{聽}{き}いて
\ruby{下}{くだ}さい
ましな。
』

\原本頁{74-7}%
と、
%
\ruby{心配}{しん|ぱい}
\ruby{氣}{げ}に
お
\ruby{彤}{とう}が
\ruby{面色}{かほ|つき}を
\ruby{見}{み}ながら、
%
いつはり
ならず
\ruby{心}{こゝろ}を
\ruby{籠}{こ}めて
\ruby{云}{い}ひ
\ruby{出}{いだ}したり。

\原本頁{74-9}%
『
あゝ
\ruby{可}{い}いとも。
%
お
\ruby{{\換字{前}}}{まへ}の
\ruby{御頼}{お|たの}みの
\ruby{事}{こと}なら
\ruby{何}{なん}でも
\ruby{聽}{き}いて
あげるとも。
』

\原本頁{74-11}%
\ruby{此}{これ}は
\ruby{極}{きは}めて
\ruby{易}{やす}らかなる
\ruby{語氣}{ご|き}の
いと
\ruby{輕}{かろ}き
\ruby{答}{こたへ}
なり。

\原本頁{75-1}%
『
ほんとに?。
』

\原本頁{75-2}%
\ruby[|g|]{此方}{こなた}は
\ruby{力}{ちから}を
\ruby{入}{い}れて
\ruby{重}{かさ}ねて
\ruby{問}{と}へば、
%
\ruby[|g|]{彼方}{かなた}は
\ruby{沈靜}{おち|つき}
きつて
\ruby{{\換字{平}}氣}{へい|き}に、

\原本頁{75-3}%
『
あゝ、
%
ほんたうにさ!。
』

\原本頁{75-4}%
と
\ruby{事}{こと}も
\ruby{無}{な}げなり。

\原本頁{75-5}%
『
あゝ
\ruby{姊}{ねえ}さん
\ruby{有}{あ}り
\ruby{{\換字{難}}}{がた}う
ございます、
%
\ruby[<j||]{一}{いつ }% ルビが重なるので調整
\ruby[<j||]{生}{しやう}
% \ruby{一生}{いつ|しやう}
\ruby{記}{おぼ}えて% 送り仮名は原本通り「え」
\ruby{居}{ゐ}ますよ。
%
ぢやあ
\ruby{申}{まを}しますがネ。
%
かういふ
\ruby{譯}{わけ}
なんです。
』

\原本頁{75-7}%
と
\ruby{說}{と}き
\ruby{出}{いだ}さん
とするを
お
\ruby{彤}{とう}は
\ruby{抑}{おさ}へて、

\原本頁{75-8}%
『
\ruby{可}{い}いよ
お
\ruby{龍}{りう}ちやん、
%
かういふ
のだらう。
%
\ruby{彼}{あ}の
\ruby{水野}{みづ|の}さん
ていふ
\原本頁{75-9}\改行%
\ruby{人}{ひと}が
\ruby{職務}{や|く}を
\ruby{離}{はな}れたに
\ruby{就}{つ}いちやあ、
%
\ruby{何樣}{ど|う}か
\ruby{彼}{あ}の
\ruby{人}{ひと}を
\ruby{困窮}{こ|ま}らせたく
\ruby{無}{な}いので、
%
\ruby{妾}{わたし}に
\ruby{口}{くち}を
きいて
\ruby{貰}{もら}つたら
\ruby{家}{うち}の
\ruby{旦那}{だん|な}の
\ruby{方}{はう}に
でも
\ruby{好}{い}い
\ruby{口}{くち}が
\ruby{有}{あ}りやあ
\ruby{仕}{し}まいか、
%
\ruby{出來}{で|き}る
\ruby{事}{こと}なら
\ruby{好}{い}い
\ruby{口}{くち}を
\ruby{搜}{さが}し
\ruby{出}{だ}して
\ruby{持}{も}つて
\原本頁{76-1}\改行%
\ruby{行}{い}つて
\ruby{{\換字{遣}}}{や}りたい。
%
と、
%
かういふ
ところからの
お
\ruby{{\換字{前}}}{まへ}の
\ruby{御頼}{お|たの}み
なのぢや
\ruby{無}{な}くつて?。
』

\原本頁{76-3}%
と
\ruby{全}{まつた}く
お
\ruby{龍}{りう}の
\ruby{胸}{むね}の
\ruby{奧}{おく}の
\ruby{{\換字{文}}}{あや}を
\ruby{鏡}{かゞみ}に
\ruby{取}{と}りて
\ruby{見}{み}る
\ruby{如}{ごと}く
\ruby{云}{い}ひ
\ruby{出}{だ}したり。

\原本頁{76-4}%
\ruby{云}{い}はれて
お
\ruby{龍}{りう}は
\ruby{驚}{おどろ}いて
\ruby{眼}{め}を
\ruby{{\換字{睜}}}{みは}り、

\原本頁{76-5}%
『
まあ、
%
\ruby{何樣}{ど|う}して
\ruby{然樣}{さ|う}
\ruby[|g|]{不殘}{みんな}
\ruby{姊}{ねえ}さんは
\ruby{知}{し}つてゝ?。
%
\ruby{姊}{ねえ}さんの
\ruby{智慧}{ち|ゑ}
\原本頁{76-6}\改行%
の
\ruby{深}{ふか}いのは
\ruby{{\換字{前}}}{せん}から
\ruby{知}{し}つてますが、
%
ほんとに
まあ、
%
\ruby{何樣}{ど|う}すれば
\ruby{其樣}{そん|な}に% 行末付近でもあり、原本通り非グループルビ
\ruby{人}{ひと}の
\ruby{意}{き}が
\ruby{解}{わか}るの?。
%
\ruby{妾}{わたし}あ
\ruby{餘}{あんま}り
\ruby{其}{そ}の
\ruby{{\換字{通}}}{とほ}り
なので
\ruby{怖}{こは}い
やうな
\ruby{氣}{き}が
\ruby{仕}{し}ますよ。
%
\ruby{全}{まつた}く
\ruby{然樣}{さ|う}いふ
\ruby{譯}{わけ}の
\ruby{御願}{お|ねがひ}で
わざ〳〵
\ruby{來}{き}たのですが、
\ruby{何樣}{ど|う}
いふ
もので
しやう?、
%
\ruby{姊}{ねえ}さん、
%
\ruby{聽}{き}いて
\ruby{下}{くだ}すつて?。
』

\原本頁{76-10}%
と
\ruby[||j>]{正}{しやう}
\ruby[||j>]{直}{ ぢき}に
なつて
% \ruby{正直}{しやう|ぢき}になつて
\ruby{頼}{たの}み
\ruby{聞}{きこ}ゆるを、
%
お
\ruby{彤}{とう}は
\ruby{憐}{あはれ}むが
\ruby{如}{ごと}く
\ruby{憐}{あはれ}まざるが
\ruby{如}{ごと}く
\ruby{冷}{ひやゝ}かに
\ruby{見}{み}やりて、

\原本頁{77-1}%
『
\ruby{頼}{たの}みを
\ruby{聽}{き}くも
\ruby{聽}{き}かないも
\ruby{有}{あ}りやあ
\ruby{仕}{し}ないがネ、
%
お
\ruby{龍}{りう}ちやん、
%
お
\ruby{{\換字{前}}}{まへ}
そりやあ
\ruby{詰}{つま}らない
\ruby{事}{こと}だらうよ。
』

\原本頁{77-3}%
と、
%
いと
\ruby{物}{もの}
\ruby{靜}{しづ}かに
\ruby{先}{ま}づ
\ruby{一句}{いつ|く}
\ruby{云}{い}ひ
\ruby{斷}{き}りたり。
