\Entry{其十三}

\ruby[g]{古薩摩}{こさつま}か
\ruby[g]{古九谷}{こくたに}とありさうなところを
\ruby{然}{さ}は
\ruby{無}{な}くて、
\ruby[g]{永樂}{えいらく}あたりの
\ruby[g]{稀品}{きひん}なるべし、
\ruby[g]{形状品格佳}{かたちひんよ}くして
\ruby[g]{彩釉快}{いろゑこゝろよ}く
\ruby{麗}{うる}はしき
\ruby[g]{京燒}{きやうやき}の
\ruby{茶器}{ちや|き}を、
\ruby{五指}{ご|し}
\ruby{白玉}{はく|ぎよく}の
\ruby{如}{ごと}く
\ruby{美}{うつく}しき
\ruby{手}{て}に
\ruby{自}{みづか}ら
\ruby{扱}{あつか}ひて、
\ruby{既}{すで}に
\ruby{鎚目}{つち|め}の
\ruby{銀瓶}{ぎん|びん}の
\ruby{湯}{ゆ}を
\ruby{徐々}{しづ|か}に
\ruby{注}{さ}し
\ruby{{\換字{終}}}{をわ}り、
\ruby{今}{いま}や
\ruby{一盞}{いつ|さん}に
\ruby[g]{玉露}{ぎよくろ}の
\ruby{花香}{はな|か}を
\ruby{湛}{たゝ}へて、
お
\ruby{彤}{とう}はこれをば
\ruby{與}{あた}へ
\ruby{{\換字{遣}}}{や}りつ、
\ruby{鍋島}{なべ|しま}の
\ruby[g]{菓子皿}{くわしざら}をば
\ruby[g]{{\換字{又}}聊}{またいさゝか}か
お
\ruby{龍}{りう}が
\ruby{方}{かた}へと
\ruby{推進}{おし|すゝ}めたり。

お
\ruby{龍}{りう}は
\ruby{心底}{しん|そこ}より
\ruby{{\換字{悅}}}{よろこ}びて
\ruby{茶}{ちや}を
\ruby{味}{あじ}はひつ。

『いつでも
\ruby{眞個}{ほん|たう}に
\ruby{勿體}{もつ|たい}ないやうな
\ruby{佳良}{けつ|こう}な
\ruby{御茶}{お|ちや}ネ。
』

『ホヽヽ、お
\ruby{茶}{ちや}ばかり
\ruby{褒}{ほ}めずとも
\ruby{淹}{い}れ
\ruby{方}{かた}も
\ruby{褒}{ほ}めて、
お
\ruby{{\換字{呉}}}{く}れな。
』

『ホヽヽ、そりやあもう、
\ruby{口}{くち}へ
\ruby{出}{だ}しては
\ruby{云}{い}はなくつても……。
』

『オヤ
\ruby{左樣}{さ|う}、
\ruby{嬉}{うれ}しい
\ruby{人}{ひと}ネエ。
ぢやあまあ
\ruby[g]{澤山御菓子}{たんとおくわし}でも
\ruby[g]{御食}{おあが}りなすつて。
』

『
\ruby{厭}{いや}ネエ、ふざけて!。
\ruby{姊}{ねえ}さんは
\ruby{人}{ひと}が
\ruby{惡}{わる}いは。
』

とお
\ruby{龍}{りう}は
\ruby{一寸}{ちよ|つと}
\ruby{瞋}{おこ}つたるやうな
\ruby{顏}{かほ}して
\ruby{云}{い}ひ、

『それに
\ruby{此}{こ}の
\ruby{御菓子}{お|くわ|し}は
\ruby{妾}{わたし}は
\ruby{澤山}{たく|さん}ですよ。
』

といふ。

『
\ruby{{\換字{嫌}}}{きら}ひ?。
』

と
\ruby{女主人}{あ|る|じ}は
\ruby{輕}{かろ}く
\ruby{眞面目}{ま|じ|め}に
\ruby{問}{と}ふ。
\ruby{問}{と}はれて
\ruby{莞爾}{にこ|やか}なる
\ruby{舊}{もと}に
\ruby{復}{かへ}りながら、

『まあ
\ruby{左樣}{さ|う}なの。
』

と
\ruby{氣}{き}の
\ruby{毒}{どく}さうに
\ruby{答}{こた}へたるは、
\ruby{思}{おも}はず
\ruby{我}{わ}が
\ruby{好}{す}き
\ruby{{\換字{嫌}}}{きら}ひの
\ruby{我儘}{わが|まゝ}を
\ruby{口走}{くち|ばし}つたる
\ruby{無遠慮}{ぶ|ゑん|りよ}を
\ruby{羞}{は}ぢて、
\ruby{今}{いま}さら
\ruby[g]{詮方無}{せんかたな}くも
\ruby{{\換字{猶}}}{なほ}
\ruby{少}{すこ}し
\ruby{曖昧}{あい|まい}に
\ruby{言葉}{こと|ば}を
\ruby{濁}{にご}せるなるべし。

『いけなかつたネエ、
\ruby[g]{甘味{\換字{嫌}}}{あまいぎら}ひとばつかり
\ruby{思}{おも}つて
\ruby{居}{ゐ}て
\ruby{此品}{こ|れ}が
\ruby{{\換字{嫌}}}{きら}ひだつたとは
\ruby{知}{し}らなかつたよ。
もつともネ、
\ruby[g]{一體此}{いつたいこれ}は
\ruby{御茶}{お|ちや}にも
\ruby{餘}{あま}り
\ruby{賞}{ほ}めたものぢやあ
\ruby{無}{な}いの。
そればかりぢやあ
\ruby{無}{な}い、
\ruby{鳥貝}{とり|がひ}の
\ruby{御鮨}{お|す}もじだの
\ruby{玉簾}{たま|だれ}だのといふものは、
\ruby{惡}{わる}く
\ruby{氣取}{き|ど}つた
\ruby{女}{ひと}に
\ruby{食}{た}べさせて
\ruby{{\換字{遣}}}{や}れなんぞといふ
\ruby{位}{くらゐ}のものだつたのに、つい
\ruby{妾}{わたし}が
\ruby{氣}{き}が
\ruby{注}{つ}かなかつたよ、
\ruby{堪忍}{かん|にん}おし。
\ruby{今他}{いま|ほか}のものを
\ruby{何}{なん}ぞあげるから。
』

『
\ruby{何故}{な|ぜ}?。
\ruby{氣取}{き|ど}つた
\ruby{女}{ひと}が
\ruby{何樣}{ど|う}か
\ruby{仕}{し}でもするの?。
』

『ソレ
\ruby{烏貝}{とり|がひ}は
お
\ruby{前早}{まへ|はや}くは
\ruby{咬}{か}み
\ruby{切}{き}れないし、
\ruby{玉簾}{たま|だれ}はホロ〳〵と
\ruby{零}{こぼ}れ
\ruby{{\換字{勝}}}{かち}だし
\ruby{辛}{から}くはあるしするからネ。
いつまでも
\ruby{口}{くち}をムグ〳〵させて
\ruby{居}{ゐ}たり、だらし
\ruby{無}{な}く
\ruby{膝}{ひざ}を
\ruby{汚}{よご}して、そして
\ruby{辛}{から}さを
\ruby{辛抱}{しん|ぼう}する
\ruby{泣顏}{なき|がほ}を
\ruby{仕}{し}て
\ruby{居}{ゐ}たりするのなんぞは
\ruby{見好}{み|い}いものぢやあ
\ruby{無}{な}いからさ。
』

『あらツ!、
\ruby{妾}{わたし}あ
\ruby{其樣}{そ|ん}な
\ruby{譯}{わけ}で
\ruby{{\換字{嫌}}}{きら}ひだつていふのぢやあ
\ruby{有}{あ}りませんは。
\ruby{姊}{ねえ}さんのところへ
\ruby{來}{き}て
\ruby{一寸}{ちよ|いと}だつて
\ruby{氣}{き}を
\ruby{置}{お}いてなんぞ
\ruby{居}{ゐ}やあ
\ruby{仕}{し}ませんのに。
\ruby{好}{よ}うござんすよ、
\ruby{一人}{ひと|り}で
\ruby{悉皆}{みん|な}
\ruby{頂}{いたゞ}いて
\ruby{仕舞}{し|ま}つて、
\ruby[g]{其邊中食}{そこいらじゆうた}べ
\ruby{零}{こぼ}して、そうして
\ruby[g]{澤山見}{たんとみ}つとも
\ruby{無}{な}い
\ruby{泣顏}{なき|がほ}をして、
\ruby{笑}{わら}つていたゞきますから。
』

『ホヽホヽホヽ、ホラ
\ruby{始}{はじ}まつたよ
お
\ruby{龍}{りう}ちやんの
\ruby{癇癖}{む|し}が。
だがお
\ruby{前}{まへ}が
\ruby{一寸}{ちよ|いと}
\ruby{口惜}{く|や}しいといふ
\ruby{思入}{おもひ|いれ}をすると、
\ruby{色艶}{いろ|つや}は
\ruby{好}{よ}
し、
\ruby{眼}{め}は
\ruby{淸}{すゞ}しいし、
\ruby[g]{眉毛}{まみえ}は
\ruby{奇麗}{き|れい}だし、それが
\ruby{悉皆}{みん|な}
\ruby{役}{やく}に
\ruby{立}{た}つて
\ruby{顏中}{かほ|ぢゆう}が
\ruby{活}{い}きて
\ruby{見}{み}えて
\ruby{來}{き}て、ほんとに
\ruby{婀娜}{あ|だ}で
\ruby{可憐}{かは|い}らしいよ。
』

『
\ruby{好}{よ}うござんすよ。
』

\ruby{此度}{こ|たび}はいよ〳〵
\ruby{瞋}{いか}りていよ〳〵
\ruby{言葉}{こと|ば}
\ruby{少}{すくな}く、
\ruby{恨}{うら}めしげに\換字{志}ろりと
お
\ruby{彤}{とう}を
\ruby{睨}{にら}みて、つんとして
\ruby{其}{そ}の
\ruby{儘横}{まゝ|よこ}を
\ruby{向}{む}かんとせしが、
\ruby{閑事}{あだ|ごと}は
\ruby{兎}{と}に
\ruby{角}{かく}、
\ruby{云}{い}はで
\ruby{叶}{かな}はざる
\ruby{用事}{よう|じ}はあるなり、
\ruby{霎時間}{しば|し|ま}を
\ruby{置}{お}きて
\ruby{面}{おもて}を
\ruby{擡}{あ}げ、

『ネエ、
\ruby{姊}{ねえ}さん、
\ruby{今}{いま}
\ruby{彼室}{あつ|ち}で
\ruby{云}{い}ひかけたのはネ、
\ruby{眞個}{ほん|と}に
\ruby{妾}{わたし}の
\ruby{御願}{お|ねが}ひの
\ruby{事}{こと}なんですから
\ruby{聽}{き}いて
\ruby{下}{くだ}さいましな。
』

と、
\ruby{心配氣}{しん|ぱい|げ}に
お
\ruby{彤}{とう}が
\ruby{面色}{かほ|つき}を
\ruby{見}{み}ながら、いつはりならず
\ruby{心}{こゝろ}を
\ruby{籠}{こ}めて
\ruby{云}{い}ひ
\ruby{出}{いだ}したり。

『あゝ
\ruby{可}{い}いとも。
お
\ruby{前}{まへ}の
\ruby{御頼}{お|たの}みの
\ruby{事}{こと}なら
\ruby{何}{なん}でも
\ruby{聽}{き}いてあげるとも。
』

\ruby{此}{これ}は
\ruby{極}{きは}めて
\ruby{易}{やす}らかなる
\ruby{語氣}{ご|き}のいと
\ruby{輕}{かろ}き
\ruby{答}{こたへ}なり。

『ほんとに?。
』

\ruby{此方}{こな|た}は
\ruby{力}{ちから}を
\ruby{入}{い}れて
\ruby{重}{かさ}ねて
\ruby{問}{と}へば、
\ruby{彼方}{かな|た}は
\ruby{沈靜}{おち|つき}きつて
\ruby{{\換字{平}}氣}{へい|き}に、

『あゝ、ほんたうにさ!。
』

と
\ruby{事}{こと}も
\ruby{無}{な}げなり。

『あゝ
\ruby{姊}{ねえ}さん
\ruby{有}{あ}り
\ruby{難}{がた}うございます、
\ruby{一生記}{いつ|しやう|おぼ}えて
\ruby{居}{ゐ}ますよ。
ぢやあ
\ruby{申}{まを}しますがネ。
かういふ
\ruby{譯}{わけ}なんです。
』

と
\ruby{{\換字{説}}}{と}き
\ruby{出}{いだ}さんとするを
お
\ruby{彤}{とう}は
\ruby{抑}{おさ}へて、

『
\ruby{可}{い}いよ
お
\ruby{龍}{りう}ちやん、かういふのだらう。
\ruby{彼}{あ}の
\ruby{水野}{みづ|の}さんていふ
\ruby{人}{ひと}が
\ruby{職務}{や|く}を
\ruby{離}{はな}れたに
\ruby{就}{つ}いちやあ、
\ruby{何樣}{ど|う}か
\ruby{彼}{あ}の
\ruby{人}{ひと}を
\ruby{困窮}{こ|ま}らせたく
\ruby{無}{な}いので、
\ruby{妾}{わたし}に
\ruby{口}{くち}をきいて
\ruby{貰}{もら}つたら
\ruby{家}{うち}の
\ruby[g]{旦那}{だんな}の
\ruby{方}{はう}にでも
\ruby{好}{い}い
\ruby{口}{くち}が
\ruby{有}{あ}りやあ
\ruby{仕}{し}まいか、
\ruby{出來}{で|き}る
\ruby{事}{こと}なら
\ruby{好}{い}い
\ruby{口}{くち}を
\ruby{搜}{さが}し
\ruby{出}{だ}して
\ruby{持}{も}つて
\ruby{行}{い}つて
\ruby{{\換字{遣}}}{や}りたい。
と、かういふところからのお
\ruby{前}{まへ}の
\ruby{御頼}{お|たの}みなのぢや
\ruby{無}{な}くつて?。
』

と
\ruby{全}{まつた}く
お
\ruby{龍}{りう}の
\ruby{胸}{むね}の
\ruby{奧}{おく}の
\ruby{{\換字{文}}}{あや}を
\ruby{鏡}{かゞみ}に
\ruby{取}{と}りて
\ruby{見}{み}る
\ruby{如}{ごと}く
\ruby{云}{い}ひ
\ruby{出}{だ}したり。

\ruby{云}{い}はれて
お
\ruby{龍}{りう}は
\ruby{驚}{おどろ}いて
\ruby{眼}{め}を
\ruby{睜}{みは}り、

『まあ、
\ruby{何樣}{ど|う}して
\ruby[g]{然樣不殘}{さうして}
\ruby{姊}{ねえ}さんは
\ruby{知}{し}つてゝ?。
\ruby{姊}{ねえ}さんの
\ruby{智慧}{ち|ゑ}の
\ruby{深}{ふか}いのは
\ruby{前}{せん}から
\ruby{知}{し}つてますが、ほんとにまあ、
\ruby{何樣}{ど|う}すれば
\ruby{其樣}{そん|な}に
\ruby{人}{ひと}の
\ruby{意}{き}が
\ruby{解}{わか}るの?。
\ruby{妾}{わたし}あ
\ruby{餘}{あんま}り
\ruby{其}{そ}の
\ruby{通}{とほ}りなので
\ruby{怖}{こは}いやうな
\ruby{氣}{き}が
\ruby{仕}{し}ますよ。
\ruby{全}{まつた}く
\ruby{然樣}{さ|う}いふ
\ruby{譯}{わけ}の
\ruby{御願}{お|ねがひ}でわざ〳〵
\ruby{來}{き}たのですが
\ruby{何樣}{ど|う}いふものでしやう?、
\ruby{姊}{ねえ}さん、
\ruby{聽}{き}いて
\ruby{下}{くだ}すつて?。
』

と
\ruby{正直}{しやう|ぢき}になつて
\ruby{頼}{たの}み
\ruby{聞}{きこ}ゆるを、
お
\ruby{彤}{とう}は
\ruby{憐}{あはれ}むが
\ruby{如}{ごと}く
\ruby{憐}{あはれ}まざるが
\ruby{如}{ごと}く
\ruby{冷}{ひやゝ}かに
\ruby{見}{み}やりて、

『
\ruby{頼}{たの}みを
\ruby{聽}{き}くも
\ruby{聽}{き}かないも
\ruby{有}{あ}りやあ
\ruby{仕}{し}ないがネ、
お
\ruby{龍}{りう}ちやん、
お
\ruby{前}{まへ}そりやあ
\ruby{詰}{つま}らない
\ruby{事}{こと}だらうよ。
』

と、いと
\ruby{物靜}{もの|しづ}かに
\ruby{先}{ま}づ
\ruby{一句云}{いつ|く|い}ひ
\ruby{斷}{き}りたり。

