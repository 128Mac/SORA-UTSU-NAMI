\Entry{其十四}

% メモ 校正終了 2024-05-13 2024-06-09
\原本頁{77-5}%
\ruby{我}{わ}が
\ruby{胸}{むね}の
\ruby{中}{うち}の
\ruby{{\換字{所}}思}{おも|はく}の
\ruby{底}{そこ}を
\ruby{盡}{つく}して
\ruby{說}{と}き
\ruby{中}{あ}てられたるに、
%
\ruby{一度}{ひと|たび}は
\ruby[|-|]{先}{ま}づ
\ruby{驚}{おどろ}き
\ruby{服}{ふく}したるも、
%
\ruby{其}{そ}れを
\ruby{詰}{つま}らぬことゝ
\ruby{唯}{たゞ}
\ruby{一言}{ひと|こと}に
\ruby{{\換字{斥}}}{しりぞ}けられては
\改行% 校正作業の簡略化のため
、
%
\原本頁{77-7}\改行%
\ruby{物}{もの}に
\ruby{堪}{こら}へぬ
お
\ruby{龍}{りう}の
\ruby[||j>]{心}{こゝろ}
\ruby[||j>]{{\換字{平}}}{ たひ}らか
% \ruby{心{\換字{平}}}{こゝろ|たひ}らか
ならず、
%
\ruby{思}{おも}はず
\ruby{顏}{かほ}を
\ruby{突}{つ}と
\ruby{擡}{あ}げて、

\原本頁{77-8}%
『
\ruby{何故}{な|ぜ}ネエ。
』

\原本頁{77-9}%
と
\ruby{詰}{なじ}り
\ruby{氣味}{ぎ|み}に
\ruby{咄嗟}{とつ|さ}に
\ruby{言葉}{こと|ば}を
\ruby{{\換字{返}}}{かへ}しゝが、
%
\ruby{見}{み}れば
\ruby{{\換字{古}}風}{こ|ふう}の
\ruby{内裏雛}{だい|り|びな}の
\ruby{如}{ごと}くに
\ruby[|g|]{端然}{しやん}と
したる
\ruby{面}{かほ}つきの、
%
\ruby{細}{ほそ}けれど
\ruby{亘}{わたり}の
\ruby{長}{なが}くして
\ruby{特}{こと}に
はつきりと
\ruby{明}{あき}らか
なる
\ruby{眼}{め}を、
%
\ruby{我}{わ}が
\ruby{上}{うへ}に
\換字{志゛}つと% 「志」+「濁点」
お
\ruby{彤}{とう}の
\ruby{注}{そゝ}ぎ
\ruby{居}{ゐ}たるに、
%
\ruby{其}{そ}の
\ruby{沈靜}{おち|つ}きたる
\ruby[|g|]{態度}{やうす}の
\ruby{中}{うち}に
\ruby{具}{そな}はれる
\ruby[||j>]{自}{おのづ}
\ruby[||j>]{然}{ から}の
% \ruby{自然}{おのづ|から}の
\ruby{威}{ゐ}は、
%
\ruby{輕々}{かろ|〴〵}しく
\ruby{慌}{あわ}たゞしき
\ruby{我}{われ}を
\ruby{壓}{お}す
\ruby{如}{ごと}く
\ruby{覺}{おぼ}えて、
%
\ruby{何}{なん}
といふ
\ruby{事}{こと}は
\ruby{無}{な}けれど
\ruby{當}{あた}り
\ruby{{\換字{難}}}{がた}き
\ruby{心地}{こゝ|ち}の
\ruby{爲}{し}、
%
\ruby{氣勢}{いき|ほひ}
\ruby{忽}{たちま}ち
\ruby{挫}{くじ}けて
\ruby{語氣}{ご|き}も
\ruby{萎々}{なえ|〳〵}と、

\原本頁{78-5}%
『
\ruby{詰}{つま}らない
つて、
%
\ruby{其}{そ}りやあ
\ruby{然樣}{さ|う}かも
\ruby{知}{し}りません
けれども、
%
\ruby{妾}{わたし}にやあ
\ruby{些}{ちつと}も
\ruby{然樣}{さ|う}は
\ruby{思}{おも}へませんは。
%
\ruby{下}{くだ}らない
かも
\ruby{知}{し}りません
けれども
%
\ruby{妾}{わたし}の
\ruby{思}{おも}つてる
\ruby{事}{こと}を、
%
ネエ
\ruby{姊}{ねえ}さん
どうか
\ruby{一}{ひ}ト
\ruby{{\換字{通}}}{とほ}り
\ruby{聞}{き}いて
\ruby{見}{み}て
\ruby{下}{くだ}さいな。
』

\原本頁{78-9}%
と、
%
\ruby{憐愍}{あは|れみ}を
\ruby{乞}{こ}ふが
\ruby{如}{ごと}くに
\ruby{云}{い}ひ
\ruby{足}{た}したり。

\原本頁{78-10}%
\ruby{人}{ひと}に
\ruby{頼}{たの}み
ごと
する
ものゝ
\ruby{心}{こゝろ}の
\ruby{中}{うち}ほど
\ruby{苦}{くる}しきは
\ruby{無}{な}し。
%
\ruby{{\換字{強}}}{し}ひる
ほどに
\ruby{頼}{たの}まねば
\ruby[|g|]{願望}{ねがひ}は
\ruby{成}{な}り
\ruby{{\換字{難}}}{がた}く、
%
\ruby{{\換字{強}}}{し}ひ
\ruby{{\換字{過}}}{す}ぎて
\ruby{怒}{おこ}られて
\ruby{仕舞}{し|ま}へば
それまで
なれば、
%
\ruby{願}{ねが}ふ
\ruby{意}{こゝろ}の
\ruby{切}{せつ}なれば
\ruby{切}{せつ}なるだけ、
%
\ruby{我}{わ}が
\ruby{言葉}{こと|ば}の
\ruby[||j>]{斟}{しん}
\ruby[||j>]{{\換字{酌}}}{しやく}に
% \ruby{斟{\換字{酌}}}{しん|しやく}に
\原本頁{79-2}\改行%
\ruby{氣}{き}を
\ruby{使}{つか}ひて、
%
\ruby{斯樣}{か|う}
\ruby{云}{い}ひて
\ruby{宜}{よ}かる
べきか
\ruby{惡}{あし}かる
べきかの
\ruby{心配}{しん|ぱい}に、
%
\原本頁{79-3}\改行%
\ruby{人知}{ひと|し}れず
\ruby[|g|]{幾干}{いくそ}の
\ruby{胸}{むね}を
\ruby{痛}{いた}むるなり。
%
お
\ruby{彤}{とう}は
\ruby{我}{わ}が
\ruby{愛}{あい}する
お
\ruby{龍}{りう}が
いぢらしき
\ruby{心}{こゝろ}の
\ruby{中}{うち}を、
%
\ruby{早}{はや}くも
\ruby{其}{そ}の
\ruby{目色}{め|いろ}
\ruby[||j>]{語}{ことば}
\ruby[||j>]{氣}{ つき}
% \ruby{語氣}{ことば|つき}
に
\ruby{猜}{すゐ}し
\ruby{知}{し}りて、
%
たちまちに
\ruby{面}{おもて}を
\ruby{和}{やは}らげ
\ruby{笑}{ゑみ}を
\ruby{爲}{つく}りつ、

\原本頁{79-6}%
『
まあ
お
\ruby{龍}{りう}ちやんの
\ruby{思}{おも}つてる
\ruby{事}{こと}つて
\ruby{何樣}{ど|う}いふ
\ruby{事}{こと}なの?。
』

\原本頁{79-7}%
と、
%
\ruby{云}{い}ひ
\ruby{出}{い}で
\ruby{易}{やす}き
やうに
\ruby{路}{みち}を
\ruby{開}{ひら}きたり。

\原本頁{79-8}%
お
\ruby{龍}{りう}は
これに
\ruby[<j>]{勢}{いきほひ}を
\ruby{得}{え}て、

\原本頁{79-9}%
『
\ruby{經{\換字{過}}}{ゆく|たて}を
\ruby{御話}{お|はな}し
\ruby{仕}{し}ない
ぢやあ、
%
\ruby{何}{なん}だか
\ruby{單}{たゞ}、
%
\ruby{妾}{わたし}の
\ruby{餘計}{よ|けい}な
\ruby{物數寄}{もの|ず|き}の
やうに
\ruby{聞}{きこ}えますからネ、
%
\ruby{長}{なが}つたらしくても
\ruby{最初}{さい|しよ}つから
いひますよ。
%
まあ
\ruby{一番}{いち|ばん}
\ruby{初}{はじめ}つから
いひますとネ。
』

\原本頁{80-1}%
と、
%
\ruby{先}{ま}づ
\ruby{語}{かた}り
\ruby{出}{いだ}して
\ruby{縷々}{る|ゝ}と
\ruby{語}{かた}り
つゞけぬ。

\原本頁{80-2}%
『
もと
\ruby{彼}{あ}の
\ruby{水野}{みづ|の}つて
いふ
\ruby{人}{ひと}は
\ruby{妾}{わたし}の
\ruby{知}{し}つてた
\ruby{人}{ひと}でも
\ruby{何}{なん}でも
\ruby{有}{あ}りやあ
\ruby{仕}{し}ませんがネ。
%
\ruby{今}{いま}
\ruby{妾}{わたし}の
\ruby{世話}{せ|わ}に
なつてる
お
\ruby{師匠}{し|よ}さんに
\ruby{義女}{まゝ|つこ}が
あるのです。
%
\ruby{會}{あ}
つた
\ruby{事}{こと}が
\ruby{無}{な}いから
\ruby{面}{かほ}は
\ruby{知}{し}りませんが
\ruby{好}{い}い
\ruby{容貌}{きり|やう}だ
さうだし、
%
\ruby{學問}{がく|もん}も
\ruby{中々}{なか|〳〵}ある
さうで
\ruby{敎師}{けう|し}さんを
\ruby{仕}{し}て
\ruby{居}{ゐ}るんです。
%
お
\ruby{五十}{い|そ}さん
といつて、
%
\ruby{沈毅者}{しつ|かり|もん}でネ、
%
もとつ
から
\ruby[||j>]{繼}{おつ}
\ruby[||j>]{母}{かさん}とは
% \ruby{繼母}{おつ|かさん}とは
\ruby{氣}{き}が
\ruby{合}{あ}はないので
\ruby[|g|]{全然}{まるで}
\ruby{離}{はな}れて
\ruby{居}{ゐ}て、
%
\ruby[|g|]{一人}{ひとり}
\ruby{立}{だち}で
\ruby{何樣}{ど|う}か
\ruby{斯樣}{か|う}か
\ruby{{\換字{遣}}}{や}つて
\ruby{行}{い}つてたのです。
%
\ruby{世話}{せ|わ}に
なつて
\ruby{居}{ゐ}て
\ruby{惡}{わる}く
\ruby{云}{い}つちやあ
\ruby{濟}{す}みませんがネ、
%
\原本頁{80-9}\改行%
お
\ruby{師匠樣}{し|よ|さん}は
\ruby{隨{\換字{分}}}{ずゐ|ぶん}
\ruby{我儘}{わが|まゝ}ぢやあ
\ruby{有}{あ}り、
%
\ruby{品行}{おこ|なひ}
だつて
\ruby{堅}{かた}い
\ruby{方}{はう}
ぢやあ
\ruby{無}{な}い
\原本頁{80-10}\改行%
\ruby{{\換字{勝}}手}{かつ|て}な
\ruby{人}{ひと}
ですから、
%
\ruby{眞正}{ほん|たう}の
\ruby{理屈}{り|くつ}を
\ruby{云}{い}やあ
\ruby[|g|]{端正}{しやん}として
\ruby{居}{ゐ}る
お
\ruby{五十}{い|そ}さんの
\ruby{方}{はう}が
\ruby{正}{い}いのでしやうサ。
%
だけれども
お
\ruby{師匠}{し|よ}さんに
\ruby{云}{い}はせりやあ、
%
\ruby{變}{へん}に
\ruby{高慢}{かう|まん}で、
%
\ruby{執拗}{かた|いぢ}な
\ruby{可厭}{い|や}な
\ruby{女}{ひと}だつて
\ruby{云}{い}ふん
です。
%
まあ
\原本頁{81-2}\改行%
\ruby{其}{それ}あ
\ruby[|g|]{何方}{どつち}が
\ruby{眞正}{ほん|た}
\footnote{「眞正」のルビは(ほんたう)が妥当だが原本通り(ほんた)とする
(国会図書館 コマ番号4/146 p-081 l-02)}%
だか% ルビ調整(原本通り)
\ruby{會}{あ}つて
\ruby{見}{み}ない
\ruby{人}{ひと}の
\ruby{事}{こと}
ですから
\ruby{{\換字{分}}}{わか}りません
けどもネ、
%
\ruby{其}{そ}の
お
\ruby{五十}{い|そ}さん
ていふのが
\ruby[<->]{弟}{おとうと}の
\ruby{世話}{せ|わ}まで
\ruby{燒}{や}いて
ゐるのに、
%
お
\ruby{師匠}{し|よ}さんは
\ruby{何}{なんに}も
\ruby{少}{すこし}も
\ruby{管}{かま}はないで、
%
\ruby{自{\換字{分}}}{じ|ぶん}で% ルビ調整(原本通り)非グループルビ
\ruby{取}{と}るものは
\ruby{自{\換字{分}}}{じ|ぶん}で% ルビ調整(原本通り)非グループルビ
\ruby{使}{つか}つて
お
\ruby{酒}{さけ}
なんぞを
\ruby{飮}{の}んでる
のですもの、
%
まあ
\ruby{何樣}{ど|う}しても
お
\原本頁{81-6}\改行%
\ruby{師匠樣}{し|よ|さん}の
\ruby{方}{はう}に
\ruby[|g|]{阿{\換字{扇}}}{うちは}は
\ruby{上}{あ}げられません
やネ。
%
ところが
\ruby{其}{そ}の
お
\ruby{五十}{い|そ}さん
といふ
\ruby{人}{ひと}が
\ruby{窒扶斯}{ち|ぶ|す}を
\ruby{患}{わづ}らつて、
%
\ruby{生死}{いき|しに}の
\ruby{{\換字{分}}}{わか}らない
\ruby{怖}{こは}い
\ruby{瀬}{せ}に
かかつた
のです。
%
それを
\ruby{何樣}{ど|う}でしやう
\ruby{家}{うち}の
\ruby{御師匠樣}{お|し|よ|さん}は
\ruby{振}{ふ}り
\ruby{向}{む}いても
\ruby{見}{み}ないのです。
%
もとより
お
\ruby{五十}{い|そ}さんが
\ruby{財產}{も|の}を
\ruby{有}{も}つて
\ruby{居}{ゐ}やう
ぢやあ
\ruby{無}{な}し
\ruby[|->]{弟}{おとうと}
ツ
\ruby{兒}{こ}
は
まだ
\ruby{一向}{いつ|かう}の
\ruby{小兒}{こ|ども}% ルビ調整(原本通り)非グループルビ
なんですもの、
%
\ruby{困}{こま}つて
\ruby{仕舞}{し|ま}ふのは
\ruby{知}{し}れ
\ruby{切}{き}つて
\ruby{居}{ゐ}ます。
%
\ruby{其處}{そ|こ}で
\ruby{彼}{あ}の
\ruby{水野}{みづ|の}さん
て
いふ
\ruby{人}{ひと}が
\ruby{世話}{せ|わ}を
\ruby{仕}{し}た
のでしてネ、
%
\ruby{彼}{あ}の
\ruby{人}{ひと}は
お
\ruby{師匠樣}{し|よ|さん}にも
お
\ruby{五十}{い|そ}さんにも
\ruby{赤}{あか}の
\ruby{他人}{た|にん}
なのです!。
』
