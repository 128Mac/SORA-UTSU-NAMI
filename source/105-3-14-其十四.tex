\Entry{其十四}

\ruby{我}{わ}が
\ruby{胸}{むね}の
\ruby{中}{うち}の
\ruby{{\換字{所}}思}{おも|はく}の
\ruby{底}{そこ}を
\ruby{盡}{つく}して
\ruby{{\換字{説}}}{と}き
\ruby{中}{あ}てられたるに、
\ruby{一度}{ひと|たび}は
\ruby{先}{ま}づ
\ruby{驚}{おどろ}き
\ruby{服}{ふく}したるも、
\ruby{其}{そ}れを
\ruby{詰}{つま}らぬことゝ
\ruby{唯一言}{たゞ|ひと|こと}に
\ruby{斥}{しりぞ}けられては、
\ruby{物}{もの}に
\ruby{堪}{こら}へぬ
お
\ruby{龍}{りう}の
\ruby{心{\換字{平}}}{こゝろ|たひ}らかならず、
\ruby{思}{おも}はず
\ruby{顏}{かほ}を
\ruby{突}{つ}と
\ruby{擡}{あ}げて、

『
\ruby{何故}{な|ぜ}ネエ。
』

と
\ruby{詰}{なじ}り
\ruby{氣味}{ぎ|み}に
\ruby{咄嗟}{とつ|さ}に
\ruby{言葉}{こと|ば}を
\ruby{{\換字{返}}}{かへ}しゝが、
\ruby{見}{み}れば
\ruby{{\換字{古}}風}{こ|ふう}の
\ruby{内裏雛}{だい|り|びな}の
\ruby{如}{ごと}くに
\ruby{端然}{しや|ん}としたる
\ruby{面}{かほ}つきの、
\ruby{細}{ほそ}けれど
\ruby{亘}{わたり}の
\ruby{長}{なが}くして
\ruby{特}{こと}にはつきりと
\ruby{明}{あき}らかなる
\ruby{眼}{め}を、
\ruby{我}{わ}が
\ruby{上}{うへ}に\換字{志}つと
お
\ruby{彤}{とう}の
\ruby{注}{そゝ}ぎ
\ruby{居}{ゐ}たるに、
\ruby{其}{そ}の
\ruby{沈靜}{おち|つ}きたる
\ruby{態度}{やう|す}の
\ruby{中}{うち}に
\ruby{具}{そな}はれる
\ruby{自然}{おのづ|から}の
\ruby{威}{ゐ}は、
\ruby{輕々}{かろ|〴〵}しく
\ruby{慌}{あわ}たゞしき
\ruby{我}{われ}を
\ruby{壓}{お}す
\ruby{如}{ごと}く
\ruby{覺}{おぼ}えて、
\ruby{何}{なん}といふ
\ruby{事}{こと}は
\ruby{無}{な}けれど
\ruby{當}{あた}り
\ruby{難}{がた}き
\ruby{心地}{こゝ|ち}の
\ruby{爲}{し}、
\ruby{氣勢忽}{いき|ほい|たちま}ち
\ruby{挫}{くじ}けて
\ruby{語氣}{ご|き}も
\ruby{萎々}{なえ|〳〵}と、

『
\ruby{詰}{つま}らないつて、
\ruby{其}{そ}りやあ
\ruby{然樣}{さ|う}かも
\ruby{知}{し}りませんけれども、
\ruby{妾}{わたし}にやあ
\ruby{些}{ちつと}も
\ruby{然樣}{さ|う}は
\ruby{思}{おも}へませんは。
\ruby{下}{くだ}らないかも
\ruby{知}{し}りませんけれども、
\ruby{妾}{わたし}の
\ruby{思}{おも}つてる
\ruby{事}{こと}を、ネエ
\ruby{姊}{ねえ}さんどうか
\ruby{一}{ひ}ト
\ruby{通}{とほ}り
\ruby{聞}{き}いて
\ruby{見}{み}て
\ruby{下}{くだ}さいな。
』

と、
\ruby{憐愍}{あは|れみ}を
\ruby{乞}{こ}ふが
\ruby{如}{ごと}くに
\ruby{云}{い}ひ
\ruby{足}{た}したり。

\ruby{人}{ひと}に
\ruby{頼}{たの}みごとするものゝ
\ruby{心}{こゝろ}の
\ruby{中}{うち}ほど
\ruby{苦}{くる}しきは
\ruby{無}{な}し。
\ruby{{\換字{強}}}{し}ひるほどに
\ruby{頼}{たの}まねば
\ruby{願望}{ねが|ひ}は
\ruby{成}{な}り
\ruby{難}{がた}く、
\ruby{{\換字{強}}}{し}ひ
\ruby{{\換字{過}}}{す}ぎて
\ruby{怒}{おこ}られて
\ruby{仕舞}{し|ま}へばそれまでなれば、
\ruby{願}{ねが}ふ
\ruby{意}{こゝろ}の
\ruby{切}{せつ}なるだけ、
\ruby{我}{わ}が
\ruby{言葉}{こと|ば}の
\ruby{斟酌}{しん|しやく}に
\ruby{氣}{き}を
\ruby{使}{つか}ひて、
\ruby{斯樣云}{か|う|い}ひて
\ruby{宜}{よ}かるべきか
\ruby{惡}{あし}かるべきかの
\ruby{心配}{しん|ぱい}に、
\ruby{人知}{ひと|し}れず
\ruby{幾干}{いく|そ}の
\ruby{胸}{むね}を
\ruby{痛}{いた}むるなり。
お
\ruby{彤}{とう}は
\ruby{我}{わ}が
\ruby{愛}{あい}する
お
\ruby{龍}{りう}がいぢらしき
\ruby{心}{こゝろ}の
\ruby{中}{うち}を、
\ruby{早}{はや}くも
\ruby{其}{そ}の
\ruby{目色語氣}{め|いろ|ことば|つき}に
\ruby{猜}{すゐ}し
\ruby{知}{し}りて、たちまちに
\ruby{面}{おもて}を
\ruby{和}{やは}らげ
\ruby{笑}{ゑみ}を
\ruby{爲}{つく}りつ、

『まあお
\ruby{龍}{りう}ちやんの
\ruby{思}{おも}つてる
\ruby{事}{こと}つて
\ruby{何樣}{ど|う}いふ
\ruby{事}{こと}なの?。
』

と、
\ruby{云}{い}ひ
\ruby{出}{い}で
\ruby{易}{やす}きやうに
\ruby{路}{みち}を
\ruby{開}{ひら}きたり。

お
\ruby{龍}{りう}はこれに
\ruby{勢}{いきほひ}を
\ruby{得}{え}て、

『
\ruby{經{\換字{過}}}{ゆく|たて}を
\ruby{御話}{お|はなし}
\ruby{仕}{し}ないぢやあ、
\ruby{何}{なん}だか
\ruby{單}{たゞ}、
\ruby{妾}{わたし}の
\ruby{餘計}{よ|けい}な
\ruby{物數寄}{も|の|ずき}のやうに
\ruby{聞}{きこ}えますからネ、
\ruby{長}{なが}つたらしくても
\ruby{最初}{さい|しよ}つからいひますよ。
まあ
\ruby{一番初}{いち|ばん|はじめ}つからいひますとネ。
』

と、
\ruby{先}{ま}づ
\ruby{語}{かた}り
\ruby{出}{いだ}して
\ruby{縷々}{る|ゝ}と
\ruby{語}{かた}りつゞけぬ。

『もと
\ruby{彼}{あ}の
\ruby{水野}{みづ|の}つていふ
\ruby{人}{ひと}は
\ruby{妾}{わたし}の
\ruby{知}{し}つてた
\ruby{人}{ひと}でも
\ruby{何}{なん}でも
\ruby{有}{あ}りやあ
\ruby{仕}{し}ませんがネ。
\ruby{今}{いま}
\ruby{妾}{わたし}の
\ruby{世話}{せ|わ}になつてる
お
\ruby{師匠}{し|よ}さんに
\ruby{義女}{まゝ|つこ}があるのです。
\ruby{會}{あ}
つた
\ruby{事}{こと}が
\ruby{無}{な}いから
\ruby{面}{かほ}は
\ruby{知}{し}りませんが
\ruby{好}{い}い
\ruby{容貌}{きり|やう}ださうだし、
\ruby{學問}{がく|もん}も
\ruby{中々}{なか|〳〵}あるさうで
\ruby{敎師}{けう|し}さんを
\ruby{仕}{し}て
\ruby{居}{ゐ}るんです。
お
\ruby{五十}{い|そ}さんといつて、
\ruby{沈毅者}{しつ|かり|もん}でネ、もとつから
\ruby{繼母}{おつ|かさん}とは
\ruby{氣}{き}が
\ruby{合}はないので
\ruby{全然}{まる|で}
\ruby{離}{はな}れて
\ruby{居}{ゐ}て、
\ruby{一人}{ひと|り}
\ruby{立}{だち}で
\ruby{何樣}{ど|う}か
\ruby{斯樣}{か|う}か
\ruby{{\換字{遣}}}{や}つて
\ruby{行}{い}つてたのです。
\ruby{世話}{せ|わ}になつて
\ruby{居}{ゐ}て
\ruby{惡}{わる}く
\ruby{云}{い}つちやあ
\ruby{濟}{す}みませんがネ、
お
\ruby{師匠樣}{し|よ|さん}は
\ruby{隨{\換字{分}}}{ずゐ|ぶん}
\ruby{我儘}{わが|まゝ}ぢやあ
\ruby{有}{あ}り、
\ruby{品行}{おこ|なひ}だつて
\ruby{堅}{かた}い
\ruby{方}{はう}ぢやあ
\ruby{無}{な}い
\ruby{{\換字{勝}}手}{かつ|て}な
\ruby{人}{ひと}ですから、
\ruby{眞正}{ほん|たう}の
\ruby{理屈}{り|くつ}を
\ruby{云}{い}やあ
\ruby{端正}{しや|ん}として
\ruby{居}{ゐ}る
お
\ruby{五十}{い|そ}さんの
\ruby{方}{はう}が
\ruby{正}{い}いのでしやうサ。
だけれどもお
\ruby{師匠}{し|よ}さんに
\ruby{云}{い}はせりやあ、
\ruby{變}{へん}に
\ruby{高慢}{かう|まん}で、
\ruby{執拗}{かた|いぢ}な
\ruby{可厭}{い|や}な
\ruby{女}{ひと}だつて
\ruby{云}{い}ふんです。
まあ
\ruby{其}{それ}あ
\ruby{何方}{どつ|ち}が
\ruby{眞正}{ほん|たう}だか
\ruby{會}{あ}つて
\ruby{見}{み}ない
\ruby{人}{ひと}の
\ruby{事}{こと}ですから
\ruby{{\換字{分}}}{わか}りませんけともネ、
\ruby{其}{そ}の
お
\ruby{五十}{い|そ}さんていふのが
\ruby{弟}{おとうと}の
\ruby{世話}{せ|わ}まで
\ruby{燒}{や}いてゐるのに、
お
\ruby{師匠}{し|よ}さんは
\ruby{何}{なんに}も
\ruby{少}{すこし}も
\ruby{管}{かま}はないで、
\ruby{自{\換字{分}}}{じ|ぶん}で
\ruby{取}{と}るものは
\ruby{自{\換字{分}}}{じ|ぶん}で
\ruby{使}{つか}つて
お
\ruby{酒}{さけ}なんぞを
\ruby{飮}{の}んでるのですもの、まあ
\ruby{何樣}{ど|う}しても
お
\ruby{師匠樣}{し|よ|さん}の
\ruby{方}{はう}に
\ruby{阿{\換字{扇}}}{うち|は}は
\ruby{上}{あ}げられませんやネ。
ところが
\ruby{其}{そ}の
お
\ruby{五十}{い|そ}さんといふ
\ruby{人}{ひと}が
\ruby{窒扶斯}{ち|ぶ|す}を
\ruby{患}{わづ}らつて、
\ruby{生死}{いき|しに}の
\ruby{{\換字{分}}}{わか}らない
\ruby{怖}{こは}い
\ruby{瀬}{せ}にかかつたのです。
それを
\ruby{何樣}{ど|う}でしやう
\ruby{家}{うち}の
\ruby{御師匠樣}{お|し|よ|さん}は
\ruby{振}{ふ}り
\ruby{向}{む}いても
\ruby{見}{み}ないのです。
もとよりお
\ruby{五十}{い|そ}さんが
\ruby{財産}{も|の}を
\ruby{有}{も}つて
\ruby{居}{ゐ}やうぢやあ
\ruby{無}{な}し
\ruby{弟}{おとうと}ツ
\ruby{兒}{こ}はまだ
\ruby{一向}{いつ|かう}の
\ruby{小兒}{こ|ども}なんですもの、
\ruby{困}{こま}つて
\ruby{仕舞}{し|ま}ふのは
\ruby{知}{し}れ
\ruby{切}{き}つて
\ruby{居}{ゐ}ます。
\ruby{其處}{そ|こ}で
\ruby{彼}{あ}の
\ruby{水野}{みづ|の}さんていふ
\ruby{人}{ひと}が
\ruby{世話}{せ|わ}を
\ruby{仕}{し}たのでしてネ、
\ruby{彼}{あ}の
\ruby{人}{ひと}は
お
\ruby{師匠樣}{し|よ|さん}にも
お
\ruby{五十}{い|そ}さんにも
\ruby{赤}{あか}の
\ruby{他人}{た|にん}なのです!。
』

