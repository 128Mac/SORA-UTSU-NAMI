\Entry{其十五}

% メモ 校正終了 2024-05-13 2024-06-09
\原本頁{82-4}%
『
\ruby{{\換字{過}}日}{こな|ひだ}も
\ruby{一寸}{ちよ|つと}
\ruby{御話}{お|はな}しを
\ruby{仕}{し}た
のですから
\ruby{諄}{くど}くは
\ruby{云}{い}ひませんが、
%
\ruby{其}{そ}の
\ruby{赤}{あか}の
\ruby{他人}{た|にん}の
\ruby{彼}{あ}の
\ruby{人}{ひと}と
お
\ruby{五十}{い|そ}さん
との
\ruby{間}{あひだ}は、
%
たゞ
\ruby{互}{たがひ}に
\ruby{同}{おな}じ
\ruby{學校}{がく|かう}に
\ruby{奉職}{つ|と}めて
\ruby{居}{ゐ}る
といふ
だけの
\ruby{事}{こと}です。
%
そりやあ
\ruby{成程}{なる|ほど}
お
\ruby{五十}{い|そ}さんを
\ruby{思}{おも}つて
\ruby{居}{ゐ}る
から
とは
いふ
ものゝ、
%
\ruby{何}{なに}も
\ruby{有}{あ}り
\ruby{餘}{あま}つて
\ruby{居}{ゐ}る
\ruby{人}{ひと}ぢやあ
\ruby{無}{な}し、
%
\ruby{學校}{がく|かう}の
\ruby{先生}{せん|せい}
なんぞを
\ruby{仕}{し}て
\ruby{居}{ゐ}る
のですもの、
%
その
\ruby{懷中}{ふと|ころ}
\ruby{合}{あひ}も
\ruby{知}{し}れて
\ruby{居}{ゐ}ますはネ。
%
その
\ruby{樂}{らく}でも
\ruby{無}{な}い
\ruby{人}{ひと}が
\ruby{無}{な}け
\ruby{無}{な}しの
\ruby{中}{なか}で
\ruby{何樣}{ど|う}か
\ruby{工夫}{く|ふう}
を
して、
%
お
\ruby{醫者}{い|しや}さんも
\ruby{頼}{たの}んで
\ruby{來}{く}る、
%
\ruby{看護{\換字{婦}}}{かん|ご|ふ}も
\ruby{附}{つ}ける、
%
\ruby[||j>]{下}{した}
\ruby[||j>]{働}{ばたら}きの
% \ruby{下働}{した|ばたら}きの
\ruby{小婢}{こ|をんな}まで
\ruby{添}{そ}へて
\ruby{置}{お}いたと
\ruby{云}{い}ふなあ、
%
\ruby{普{\換字{通}}}{な|み}
\ruby{大抵}{たい|てい}の
\ruby{親切}{しん|せつ}ぢやあ
\ruby{出來}{で|き}ません。
%
でも
また
お
\ruby{五十}{い|そ}さんが
\ruby{彼}{あ}の
\ruby{人}{ひと}
と
\ruby{思}{おも}ひ
\ruby{合}{あ}つて
\ruby{居}{ゐ}て
\改行% 校正作業の簡略化のため
、
%
\原本頁{83-2}\改行%
あの
\ruby{人}{ひと}の
\ruby{親切}{しん|せつ}を
\ruby{身}{み}に
\ruby{沁}{し}みて
\ruby{悅}{よろこ}んで
\ruby{心底}{しん|そこ}から
\ruby{嬉}{うれ}しい
とでも
\ruby{思}{おも}ふ
と
いふのなら、
%
\ruby{隨{\換字{分}}}{ずゐ|ぶん}
\ruby{彼}{あ}の
\ruby{人}{ひと}も
\ruby{苦}{くるし}み
\ruby{甲{\換字{斐}}}{が|ひ}が
ありましやうが、
%
\ruby{性}{しやう}が
\ruby{合}{あ}はない
とでも
\ruby{云}{い}ふ
のでしやうか、
%
\ruby{御師匠}{お|し|よ}さんの
\ruby{談}{はなし}では
\ruby{{\換字{嫌}}}{きら}つて
\ruby{{\換字{嫌}}}{きら}ひ
\ruby{拔}{ぬ}いて、
%
\ruby{有{\換字{難}}}{あり|がた}いとも
\ruby{嬉}{うれ}しいとも
\ruby{思}{おも}ひ
さうも
\ruby{無}{な}い
といふん
ですもの、
%
\ruby{彼}{あ}の
\ruby{人}{ひと}の
\ruby{立}{た}つ
\ruby{瀬}{せ}は
\ruby{有}{あ}りやあ
\ruby{仕}{し}ませんはネ。
%
それに
\ruby{段々}{だん|〴〵}と
\原本頁{83-8}\改行%
\ruby{吾家}{う|ち}の
\ruby{御師匠}{お|し|よ}さんの
\ruby{口}{くち}
\ruby{占}{うら}を
\ruby{引}{ひ}いて
\ruby{見}{み}ますと、
%
\ruby{今度}{こん|ど}の
\ruby{事}{こと}の
\ruby{起}{おこ}る
ずつと
\ruby{{\換字{前}}}{まへ}から、
%
お
\ruby{師匠}{し|よ}さんは
\ruby{彼}{あ}の
\ruby{人}{ひと}が
お
\ruby{五十}{い|そ}さんを
\ruby{思}{おも}つてるのに
\原本頁{83-10}\改行%
\ruby{附}{つけ}
\ruby{{\換字{込}}}{こ}んでネ、
%
\ruby{將來}{ゆく|〳〵}は
お
\ruby{五十}{い|そ}を
あげましやう
といふ
やうな
\ruby{事}{こと}を
\ruby{巧}{うま}く
\ruby{匂}{にほ}はせて、
%
\ruby{何}{なん}とか
\ruby{彼}{か}とか
\ruby{口實}{いひ|ぐさ}を
\ruby{拵}{こしら}へては
\ruby[|g|]{{\換字{若}}干金}{いくら}かづつ% 原本通り非踊り字表記「づつ」
\ruby{絞}{しぼ}つた
\原本頁{84-1}\改行%
らしいので、
%
どうも
\ruby{後}{あと}
\ruby{{\換字{前}}}{さき}を
\ruby{能}{よう}く
\ruby{考}{かんが}へて
\ruby{見}{み}ると
\ruby{屹度}{きつ|と}% ルビ調整(原本通り)非グループルビ
さう
なの
です
\改行% 校正作業の簡略化のため
よ。
』

\原本頁{84-3}%
『
へーエ、
%
\ruby{罪}{つみ}な
\ruby{事}{こと}を
\ruby{仕}{し}た
ものだネエ!、
%
お
\ruby{關}{せき}さん
といふ
\ruby{人}{ひと}は。
』

\原本頁{84-4}%
『
\ruby{罪}{つみ}ですとも
ほんとに!。
%
あんな
\ruby{生眞面目}{き|ま|じ|め}な
\ruby{初心}{う|ぶ}な
\ruby{人}{ひと}を
\ruby{欺}{だま}す
のですもの。
』

\原本頁{84-6}%
『
ぢやあ、
%
お
\ruby{{\換字{前}}}{まへ}の
\ruby{御師匠}{お|し|よ}さん
ていふ
\ruby{人}{ひと}は
\ruby{惡}{わる}い
\ruby{人}{ひと}ちやあ
\ruby{無}{な}いか。
』

\原本頁{84-7}%
『
\ruby{唯}{えゝ}、
%
まあ
\ruby{善}{い}い
\ruby{人}{ひと}たあ
\ruby{御師匠樣}{お|し|よ|さん}
ですけれども
\ruby{云}{い}へませんネエ。
%
\原本頁{84-8}\改行%
で、
%
\ruby{吾家}{う|ち}の
お
\ruby{師匠樣}{し|よ|さん}が
\ruby{萬一}{も|し}
\ruby{普{\換字{通}}}{ひと|なみ}に
\ruby[||j>]{人}{にん}
\ruby[||j>]{{\換字{情}}}{じよう}
% \ruby{人{\換字{情}}合}{にん|じよう}
\ruby[||j>]{合}{ あひ}の
\ruby{{\換字{分}}}{わか}る
\ruby{人}{ひと}
ならば、
%
\ruby{從{\換字{前}}}{いま|まで}の% ルビ調整(原本通り)非踊り字表記
\ruby{事}{こと}は
\ruby{何樣}{ど|う}でも
\ruby{斯樣}{か|う}でも
\ruby{濟}{す}んだ
こと
だから
\ruby{仕方}{し|かた}が
\ruby{無}{な}い
としても
\改行% 校正作業の簡略化のため
、
%
\原本頁{84-10}\改行%
\ruby{今度}{こん|ど}は
\ruby{云}{い}はゞ
\ruby{水野}{みづ|の}さんの
\ruby{世話}{せ|わ}
\ruby{一}{ひと}ツで
お
\ruby{五十}{い|そ}さんを
\ruby{取}{と}り
\ruby{{\換字{留}}}{と}めた
のですから、
%
\ruby{床上}{とこ|あ}げでも
\ruby{濟}{す}んだ
\ruby{其}{そ}の
\ruby[<j>]{曉}{あかつき}にやあ、
%
たとひ
お
\ruby{五十}{い|そ}さんが
\ruby{何}{なん}と
\ruby{云}{い}はうとも
\ruby{割}{わつ}つ
\ruby{口說}{く|ど}いつして、
%
\ruby{水野}{みづ|の}さんに
\ruby{嫁}{や}る
やうにでも
\ruby{仕}{し}なくちやあ
ならない
\ruby{筈}{はず}だと
\ruby{思}{おも}ひますは。
%
ネエ
\ruby{姊}{ねえ}さん、
%
\ruby{然樣}{さ|う}ぢやあ
\ruby{有}{あ}りませんか、
%
\ruby{義理}{ぎ|り}つてえ
ものがネエ。
』

\原本頁{85-4}%
『
\ruby{成程}{なる|ほど}
お
\ruby{{\換字{前}}}{まへ}が
お
\ruby{五十}{い|そ}さんの
\ruby{御母}{お|つか}さん
だつたら
\ruby{然樣}{さ|う}も
\ruby{御爲}{お|し}だらうと
おもはれるよ。
』

\原本頁{85-6}%
お
\ruby{龍}{りう}は
\ruby{此}{こ}の
お
\ruby{彤}{とう}が
\ruby{答}{こたへ}に
\ruby{少}{すくな}からぬ
\ruby{不足}{ふ|そく}の
\ruby{色}{いろ}を
\ruby[||j>]{現}{あらは}
したり。

\原本頁{85-7}%
『
ぢやあ
\ruby{姊}{ねえ}さんが
\ruby{{\換字{若}}}{も}し
\ruby{御師匠}{お|し|よ}さん
だつたら?。
』

\原本頁{85-8}%
『
ホヽヽ、
%
\ruby{挨拶}{あい|さつ}が
\ruby[||j>]{些}{ちつと}
\ruby[||j>]{氣}{ き}に
\ruby{入}{い}らなかつたネ。
%
\ruby{妾}{わたし}が
お
\ruby{五十}{い|そ}さんの
\ruby[<j||]{母}{おつか}さん% 行末行頭の境界付近なので特例処置を施す
ならカエ。
%
さうさねエ、
%
\ruby{妾}{わたし}
なら
まあ、
%
\ruby{先}{さき}へ
\ruby{恩{\換字{返}}}{おん|がへ}しを
\ruby{仕}{し}て
\ruby{置}{お}いてネ、
%
‥‥
\ruby{世話}{せ|わ}に
なつた
\ruby{恩}{おん}は
\ruby{恩}{おん}で
\ruby{水野}{みづ|の}さんに
\ruby{恩{\換字{返}}}{おん|がへ}しを
\ruby{仕}{し}てネ
\改行% 校正作業の簡略化のため
、
%
\原本頁{85-11}\改行%
\ruby{緣}{えん}の
\ruby{事}{こと}は
\ruby{其}{それ}から
\ruby{後}{あと}で
\ruby{決}{き}めやうと
\ruby{思}{おも}ふネ。
』

\原本頁{86-1}%
『
\ruby{然樣}{さ|う}!。
%
それなら
それで
\ruby{其}{それ}も
また
\ruby{譯}{わけ}の
\ruby{{\換字{分}}}{わか}つた
\ruby{大變}{たい|へん}に
\ruby{良}{い}い
\ruby{仕方}{し|かた}
だと
\ruby{妾}{わたし}も
おもひますは。
%
ところが
\ruby{吾家}{う|ち}の
\ruby{御師匠}{お|し|よ}さんは
\ruby{妾}{わたし}の
\ruby{云}{い}つたやうに
\ruby{仕}{し}やうでも
\ruby{無}{な}けりやあ、
%
\ruby{姊}{ねえ}さんの
お
\ruby{云}{い}ひのやうに
\ruby{仕}{し}やうでも
\ruby{無}{な}いんで、
%
たゞ
\ruby{病患}{わ|る}い
\ruby{時}{とき}やあ
\ruby{人}{ひと}
まかせに
\ruby{仕}{し}て
\ruby{置}{お}いて、
%
\ruby{治}{なほ}
りやあ
\ruby{自{\換字{分}}}{じ|ぶん}の% ルビ調整(原本通り)非グループルビ
\ruby{子}{こ}つて
いふやうな
\ruby{{\換字{勝}}手}{かつ|て}な
\ruby{料簡}{れう|けん}で、
%
いつまでも
\ruby{水野}{みづ|の}さんは
\ruby{釣}{つ}りつぱなしに
\ruby{仕}{し}て
\ruby{打棄}{うつ|ちや}つて
\ruby{置}{お}かう
といふん
ですもの、
%
\ruby{酷}{ひど}いぢやあ
\ruby{有}{あ}りませんか。
』

\原本頁{86-8}%
『
そりやあ
\ruby{酷}{ひど}いとも!。
%
\ruby{酷}{ひど}い
\ruby{人}{ひと}だよ。
%
\ruby{聞}{き}いて
\ruby{見}{み}りやあ
\ruby[|g|]{眞個}{ほんと}に
お
\ruby{{\換字{前}}}{まへ}の
\ruby{御師匠}{お|し|よ}さんて
\ruby{云}{い}ふのは
\ruby{惡}{わる}い
\ruby{人}{ひと}だよ。
』

\原本頁{86-10}%
『
でも
まあ
\ruby{緣}{えん}の
\ruby{事}{こと}は
\ruby{當人}{たう|にん}
\ruby{同士}{どう|し}の
\ruby{事}{こと}で、
%
\ruby{親}{おや}の
\ruby{思}{おも}ふ
やうに
ばかりも
ならない
\ruby{理}{すぢ}も
\ruby{有}{あ}りましやう。
%
ですから
お
\ruby{五十}{い|そ}さんが
\ruby{{\換字{嫌}}}{いや}なら
\ruby{{\換字{嫌}}}{いや}で
\原本頁{87-1}\改行%
\ruby{{\換字{強}}}{し}ひる
わけには
\ruby{行}{ゆ}かない
として、
%
\ruby{其}{それ}あ
\ruby{其}{それ}で
\ruby{可}{い}い
とした
ところが
\ruby{恩}{おん}は
\ruby{恩}{おん}ですもの、
%
\ruby{恩}{おん}は
\ruby{何處}{ど|こ}までも
\ruby{着}{き}なけりやあ
なりません。
%
まして
\ruby{水野}{みづ|の}さんが
\ruby{困}{こま}るといふ
\ruby{時{\換字{節}}}{は|め}になりやあ、
%
\ruby{何樣}{ど|う}しても
\ruby{知}{し}らん
\ruby{顏}{かほ}ぢやあ
\ruby{居}{ゐ}られない
\ruby{譯}{わけ}で、
%
\ruby{出來}{で|き}ない
までも
\ruby{心配}{しん|ぱい}
だけ
なりと
\ruby{仕}{し}なくちやあ
なりませんはネ。
』
