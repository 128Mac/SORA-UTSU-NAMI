\Entry{其十五}

% メモ 校正終了 2024-05-13 2024-06-09
\原本頁{82-4}%
『
\ruby[g]{{\換字{過}}日}{こなひだ}も
\ruby[g]{一寸}{ちよつと}
\ruby[g]{御話}{お はな}しを
\ruby{仕}{し}た
のですから
\ruby{諄}{くど}くは
\ruby{云}{い}ひませんが、
%
\ruby{其}{そ}の
\ruby{赤}{あか}の
\ruby[g]{他人}{た にん}の
\ruby{彼}{あ}の
\ruby{人}{ひと}と
お
\ruby[g]{五十}{い そ }さん
との
\ruby{間}{あひだ}は、
%
たゞ
\ruby{互}{たがひ}に
\ruby{同}{おな}じ
\ruby[g]{學校}{がくかう}に
\ruby[g]{奉職}{つ と }めて
\ruby{居}{ゐ}る
といふ
だけの
\ruby{事}{こと}です。
%
そりやあ
\ruby[g]{成程}{なるほど}
お
\ruby[g]{五十}{い そ }さんを
\ruby{思}{おも}つて
\ruby{居}{ゐ}る
から
とは
いふ
ものゝ、
%
\ruby{何}{なに}も
\ruby{有}{あ}り
\ruby{餘}{あま}つて
\ruby{居}{ゐ}る
\ruby{人}{ひと}ぢやあ
\ruby{無}{な}し、
%
\ruby[g]{學校}{がくかう}の
\ruby[g]{先生}{せんせい}
なんぞを
\ruby{仕}{し}て
\ruby{居}{ゐ}る
のですもの、
%
その
\ruby[g]{懷中}{ふところ}
\ruby{合}{あひ}も
\ruby{知}{し}れて
\ruby{居}{ゐ}ますはネ。
%
その
\ruby{樂}{らく}でも
\ruby{無}{な}い
\ruby{人}{ひと}が
\ruby{無}{な}け
\ruby{無}{な}しの
\ruby{中}{なか}で
\ruby[g]{何樣}{ど う }か
\ruby[g]{工夫}{く ふう}
を
して、
%
お
\ruby[g]{醫者}{い しや}さんも
\ruby{頼}{たの}んで
\ruby{來}{く}る、
%
\ruby{看護{\換字{婦}}}{かん|ご|ふ}も
\ruby{附}{つ}ける、
%
\ruby[||j>]{下}{した}
\ruby[||j>]{働}{ばたら}きの
% \ruby{下働}{した|ばたら}きの
\ruby[g]{小婢}{こをんな}まで
\ruby{添}{そ}へて
\ruby{置}{お}いたと
\ruby{云}{い}ふなあ、
%
\ruby[g]{普{\換字{通}}}{な み }
\ruby[g]{大抵}{たいてい}の
\ruby[g]{親切}{しんせつ}ぢやあ
\ruby[g]{出來}{で き }ません。
%
でも
また
お
\ruby[g]{五十}{い そ }さんが
\ruby{彼}{あ}の
\ruby{人}{ひと}
と
\ruby{思}{おも}ひ
\ruby{合}{あ}つて
\ruby{居}{ゐ}て
\改行% 校正作業の簡略化のため
、
%
\原本頁{83-2}\改行%
あの
\ruby{人}{ひと}の
\ruby[g]{親切}{しんせつ}を
\ruby{身}{み}に
\ruby{沁}{し}みて
\ruby{悅}{よろこ}んで
\ruby[g]{心底}{しんそこ}から
\ruby{嬉}{うれ}しい
とでも
\ruby{思}{おも}ふ
と
いふのなら、
%
\ruby[g]{隨{\換字{分}}}{ずゐぶん}
\ruby{彼}{あ}の
\ruby{人}{ひと}も
\ruby{苦}{くるし}み
\ruby[g]{甲{\換字{斐}}}{が ひ }が
ありましやうが、
%
\ruby{性}{しやう}が
\ruby{合}{あ}はない
とでも
\ruby{云}{い}ふ
のでしやうか、
%
\ruby{御師匠}{お|し|よ}さんの
\ruby{談}{はなし}では
\ruby{{\換字{嫌}}}{きら}つて
\ruby{{\換字{嫌}}}{きら}ひ
\ruby{拔}{ぬ}いて、
%
\ruby[g]{有{\換字{難}}}{ありがた}いとも
\ruby{嬉}{うれ}しいとも
\ruby{思}{おも}ひ
さうも
\ruby{無}{な}い
といふん
ですもの、
%
\ruby{彼}{あ}の
\ruby{人}{ひと}の
\ruby{立}{た}つ
\ruby{瀬}{せ}は
\ruby{有}{あ}りやあ
\ruby{仕}{し}ませんはネ。
%
それに
\ruby[g]{段々}{だん〴〵}と
\原本頁{83-8}\改行%
\ruby[g]{吾家}{う ち }の
\ruby{御師匠}{お|し|よ}さんの
\ruby{口}{くち}
\ruby{占}{うら}を
\ruby{引}{ひ}いて
\ruby{見}{み}ますと、
%
\ruby[g]{今度}{こんど }の
\ruby{事}{こと}の
\ruby{起}{おこ}る
ずつと
\ruby{{\換字{前}}}{まへ}から、
%
お
\ruby[g]{師匠}{し よ }さんは
\ruby{彼}{あ}の
\ruby{人}{ひと}が
お
\ruby[g]{五十}{い そ }さんを
\ruby{思}{おも}つてるのに
\原本頁{83-10}\改行%
\ruby{附}{つけ}
\ruby{{\換字{込}}}{こ}んでネ、
%
\ruby[g]{將來}{ゆく〳〵}は
お
\ruby[g]{五十}{い そ }を
あげましやう
といふ
やうな
\ruby{事}{こと}を
\ruby{巧}{うま}く
\ruby{匂}{にほ}はせて、
%
\ruby{何}{なん}とか
\ruby{彼}{か}とか
\ruby[g]{口實}{いひぐさ}を
\ruby{拵}{こしら}へては
\ruby[|g|]{{\換字{若}}干金}{いくら}かづつ% 原本通り非踊り字表記「づつ」
\ruby{絞}{しぼ}つた
\原本頁{84-1}\改行%
らしいので、
%
どうも
\ruby{後}{あと}
\ruby{{\換字{前}}}{さき}を
\ruby{能}{よう}く
\ruby{考}{かんが}へて
\ruby{見}{み}ると
\ruby[g]{屹度}{きつと }% ルビ調整(原本通り)非グループルビ
さう
なの
です
\改行% 校正作業の簡略化のため
よ。
』

\原本頁{84-3}%
『
へーエ、
%
\ruby{罪}{つみ}な
\ruby{事}{こと}を
\ruby{仕}{し}た
ものだネエ!、
%
お
\ruby{關}{せき}さん
といふ
\ruby{人}{ひと}は。
』

\原本頁{84-4}%
『
\ruby{罪}{つみ}ですとも
ほんとに!。
%
あんな
\ruby{生眞面目}{き|ま|じ|め}な
\ruby[g]{初心}{う ぶ }な
\ruby{人}{ひと}を
\ruby{欺}{だま}す
のですもの。
』

\原本頁{84-6}%
『
ぢやあ、
%
お
\ruby{{\換字{前}}}{まへ}の
\ruby{御師匠}{お|し|よ}さん
ていふ
\ruby{人}{ひと}は
\ruby{惡}{わる}い
\ruby{人}{ひと}ちやあ
\ruby{無}{な}いか。
』

\原本頁{84-7}%
『
\ruby{唯}{えゝ}、
%
まあ
\ruby{善}{い}い
\ruby{人}{ひと}たあ
\ruby{御師匠樣}{お|し|よ|さん}
ですけれども
\ruby{云}{い}へませんネエ。
%
\原本頁{84-8}\改行%
で、
%
\ruby[g]{吾家}{う ち }の
お
\ruby{師匠樣}{し|よ|さん}が
\ruby[g]{萬一}{も し }
\ruby[g]{普{\換字{通}}}{ひとなみ}に
\ruby[||j>]{人}{にん}
\ruby[||j>]{{\換字{情}}}{じよう}
% \ruby{人{\換字{情}}合}{にん|じよう}
\ruby[||j>]{合}{ あひ}の
\ruby{{\換字{分}}}{わか}る
\ruby{人}{ひと}
ならば、
%
\ruby[g]{從{\換字{前}}}{いままで}の% ルビ調整(原本通り)非踊り字表記
\ruby{事}{こと}は
\ruby[g]{何樣}{ど う }でも
\ruby[g]{斯樣}{か う }でも
\ruby{濟}{す}んだ
こと
だから
\ruby[g]{仕方}{し かた}が
\ruby{無}{な}い
としても
\改行% 校正作業の簡略化のため
、
%
\原本頁{84-10}\改行%
\ruby[g]{今度}{こんど }は
\ruby{云}{い}はゞ
\ruby[g]{水野}{みづの }さんの
\ruby[g]{世話}{せ わ }
\ruby{一}{ひと}ツで
お
\ruby[g]{五十}{い そ }さんを
\ruby{取}{と}り
\ruby{{\換字{留}}}{と}めた
のですから、
%
\ruby[g]{床上}{とこあ }げでも
\ruby{濟}{す}んだ
\ruby{其}{そ}の
\ruby[<j>]{曉}{あかつき}にやあ、
%
たとひ
お
\ruby[g]{五十}{い そ }さんが
\ruby{何}{なん}と
\ruby{云}{い}はうとも
\ruby{割}{わつ}つ
\ruby[g]{口說}{く ど }いつして、
%
\ruby[g]{水野}{みづの }さんに
\ruby{嫁}{や}る
やうにでも
\ruby{仕}{し}なくちやあ
ならない
\ruby{筈}{はず}だと
\ruby{思}{おも}ひますは。
%
ネエ
\ruby{姊}{ねえ}さん、
%
\ruby[g]{然樣}{さ う }ぢやあ
\ruby{有}{あ}りませんか、
%
\ruby[g]{義理}{ぎ り }つてえ
ものがネエ。
』

\原本頁{85-4}%
『
\ruby[g]{成程}{なるほど}
お
\ruby{{\換字{前}}}{まへ}が
お
\ruby[g]{五十}{い そ }さんの
\ruby[g]{御母}{お つか}さん
だつたら
\ruby[g]{然樣}{さ う }も
\ruby[g]{御爲}{お し }だらうと
おもはれるよ。
』

\原本頁{85-6}%
お
\ruby{龍}{りう}は
\ruby{此}{こ}の
お
\ruby{彤}{とう}が
\ruby{答}{こたへ}に
\ruby{少}{すくな}からぬ
\ruby[g]{不足}{ふ そく}の
\ruby{色}{いろ}を
\ruby[||j>]{現}{あらは}
したり。

\原本頁{85-7}%
『
ぢやあ
\ruby{姊}{ねえ}さんが
\ruby{{\換字{若}}}{も}し
\ruby{御師匠}{お|し|よ}さん
だつたら?。
』

\原本頁{85-8}%
『
ホヽヽ、
%
\ruby[g]{挨拶}{あいさつ}が
\ruby[||j>]{些}{ちつと}
\ruby[||j>]{氣}{ き}に
\ruby{入}{い}らなかつたネ。
%
\ruby{妾}{わたし}が
お
\ruby[g]{五十}{い そ }さんの
\ruby[<j||]{母}{おつか}さん% 行末行頭の境界付近なので特例処置を施す
ならカエ
\footnote{「カエ」「かエ」の使用頻度はそれぞれ 2箇所、31箇所 であるが「ネエ」などの用法もあり原本通りとする
(国会図書館 コマ番号 46 / 146 p-085 l-09)}%
。
%
さうさねエ、
%
\ruby{妾}{わたし}
なら
まあ、
%
\ruby{先}{さき}へ
\ruby[g]{恩{\換字{返}}}{おんがへ}しを
\ruby{仕}{し}て
\ruby{置}{お}いてネ、
%
‥‥
\ruby[g]{世話}{せ わ }に
なつた
\ruby{恩}{おん}は
\ruby{恩}{おん}で
\ruby[g]{水野}{みづの }さんに
\ruby[g]{恩{\換字{返}}}{おんがへ}しを
\ruby{仕}{し}てネ
\改行% 校正作業の簡略化のため
、
%
\原本頁{85-11}\改行%
\ruby{緣}{えん}の
\ruby{事}{こと}は
\ruby{其}{それ}から
\ruby{後}{あと}で
\ruby{決}{き}めやうと
\ruby{思}{おも}ふネ。
』

\原本頁{86-1}%
『
\ruby[g]{然樣}{さ う }!。
%
それなら
それで
\ruby{其}{それ}も
また
\ruby{譯}{わけ}の
\ruby{{\換字{分}}}{わか}つた
\ruby[g]{大變}{たいへん}に
\ruby{良}{い}い
\ruby[g]{仕方}{し かた}
だと
\ruby{妾}{わたし}も
おもひますは。
%
ところが
\ruby[g]{吾家}{う ち }の
\ruby{御師匠}{お|し|よ}さんは
\ruby{妾}{わたし}の
\ruby{云}{い}つたやうに
\ruby{仕}{し}やうでも
\ruby{無}{な}けりやあ、
%
\ruby{姊}{ねえ}さんの
お
\ruby{云}{い}ひのやうに
\ruby{仕}{し}やうでも
\ruby{無}{な}いんで、
%
たゞ
\ruby[g]{病患}{わ る }い
\ruby{時}{とき}やあ
\ruby{人}{ひと}
まかせに
\ruby{仕}{し}て
\ruby{置}{お}いて、
%
\ruby{治}{なほ}
りやあ
\ruby[g]{自{\換字{分}}}{じ ぶん}の% ルビ調整(原本通り)非グループルビ
\ruby{子}{こ}つて
いふやうな
\ruby[g]{{\換字{勝}}手}{かつて }な
\ruby[g]{料簡}{れうけん}で、
%
いつまでも
\ruby[g]{水野}{みづの }さんは
\ruby{釣}{つ}りつぱなしに
\ruby{仕}{し}て
\ruby[g]{打棄}{うつちや}つて
\ruby{置}{お}かう
といふん
ですもの、
%
\ruby{酷}{ひど}いぢやあ
\ruby{有}{あ}りませんか。
』

\原本頁{86-8}%
『
そりやあ
\ruby{酷}{ひど}いとも!。
%
\ruby{酷}{ひど}い
\ruby{人}{ひと}だよ。
%
\ruby{聞}{き}いて
\ruby{見}{み}りやあ
\ruby[|g|]{眞個}{ほんと}に
お
\ruby{{\換字{前}}}{まへ}の
\ruby{御師匠}{お|し|よ}さんて
\ruby{云}{い}ふのは
\ruby{惡}{わる}い
\ruby{人}{ひと}だよ。
』

\原本頁{86-10}%
『
でも
まあ
\ruby{緣}{えん}の
\ruby{事}{こと}は
\ruby[g]{當人}{たうにん}
\ruby[g]{同士}{どうし }の
\ruby{事}{こと}で、
%
\ruby{親}{おや}の
\ruby{思}{おも}ふ
やうに
ばかりも
ならない
\ruby{理}{すぢ}も
\ruby{有}{あ}りましやう。
%
ですから
お
\ruby[g]{五十}{い そ }さんが
\ruby{{\換字{嫌}}}{いや}なら
\ruby{{\換字{嫌}}}{いや}で
\原本頁{87-1}\改行%
\ruby{{\換字{強}}}{し}ひる
わけには
\ruby{行}{ゆ}かない
として、
%
\ruby{其}{それ}あ
\ruby{其}{それ}で
\ruby{可}{い}い
とした
ところが
\ruby{恩}{おん}は
\ruby{恩}{おん}ですもの、
%
\ruby{恩}{おん}は
\ruby[g]{何處}{ど こ }までも
\ruby{着}{き}なけりやあ
なりません。
%
まして
\ruby[g]{水野}{みづの }さんが
\ruby{困}{こま}るといふ
\ruby[g]{時{\換字{節}}}{は め }になりやあ、
%
\ruby[g]{何樣}{ど う }しても
\ruby{知}{し}らん
\ruby{顏}{かほ}ぢやあ
\ruby{居}{ゐ}られない
\ruby{譯}{わけ}で、
%
\ruby[g]{出來}{で き }ない
までも
\ruby[g]{心配}{しんぱい}
だけ
なりと
\ruby{仕}{し}なくちやあ
なりませんはネ。
』
