\Entry{其十五}

\原本頁{}%
『
\ruby{{\換字{過}}日}{こな|ひだ}も
\ruby{一寸}{ちよ|つと}
\ruby{御話}{お|はなし}を
\ruby{仕}{し}たのですから
\ruby{諄}{くど}くは
\ruby{云}{い}ひませんが、
%
\ruby{其}{そ}の
\ruby{赤}{あか}の
\ruby{他人}{た|にん}の
\ruby{彼}{あ}の
\ruby{人}{ひと}と
お
\ruby[g]{五十}{いそ}さんとの
\ruby{間}{あひだ}は、
%
たゞ
\ruby{互}{たがひ}に
\ruby{同}{おな}じ
\ruby{學校}{がく|かう}に
\ruby{奉職}{つ|と}めて
\ruby{居}{ゐ}るといふだけの
\ruby{事}{こと}です。
%
そりやあ
\ruby{成程}{なる|ほど}
お
\ruby[g]{五十}{いそ}さんを
\ruby{思}{おも}つて
\ruby{居}{ゐ}るからとはいふものゝ、
%
\ruby{何}{なに}も
\ruby{有}{あ}り
\ruby{餘}{あま}つて
\ruby{居}{ゐ}る
\ruby{人}{ひと}ぢやあ
\ruby{無}{な}し、
%
\ruby{學校}{がく|かう}の
\ruby{先生}{せん|せい}なんぞを
\ruby{仕}{し}て
\ruby{居}{ゐ}るのですもの、
%
その
\ruby{懷中}{ふと|ころ}
\ruby{合}{あひ}も
\ruby{知}{し}れて
\ruby{居}{ゐ}ますはネ。
%
その
\ruby{樂}{らく}でも
\ruby{無}{な}い
\ruby{人}{ひと}が
\ruby{無}{な}け
\ruby{無}{な}しの
\ruby{中}{なか}で
\ruby{何樣}{ど|う}か
\ruby{工夫}{く|ふう}をして、
%
お
\ruby{醫者}{い|しや}さんも
\ruby{頼}{たの}んで
\ruby{來}{く}る、
%
\ruby{看護{\換字{婦}}}{かん|ご|ふ}も
\ruby{附}{つ}ける、
%
\ruby{下働}{した|ばたら}きの
\ruby{小婢}{こ|をんな}まで
\ruby{添}{そ}へて
\ruby{置}{お}いたと
\ruby{云}{い}ふなあ、
%
\ruby{普{\換字{通}}}{な|み}
\ruby{大抵}{たい|てい}の
\ruby{親切}{しん|せつ}ぢやあ
\ruby{出來}{で|き}ません。
%
でもまたお
\ruby[g]{五十}{いそ}さんが
\ruby{彼}{あ}の
\ruby{人}{ひと}
と
\ruby{思}{おも}ひ
\ruby{合}{あ}つて
\ruby{居}{ゐ}て、
%
あの
\ruby{人}{ひと}の
\ruby{親切}{しん|せつ}を
\ruby{身}{み}に
\ruby{沁}{し}みて
\ruby{悅}{よろこ}んで
\ruby{心底}{しん|そこ}から
\ruby{嬉}{うれ}しいとでも
\ruby{思}{おも}ふといふのなら、
%
\ruby{隨{\換字{分}}}{ずゐ|ぶん}
\ruby{彼}{あ}の
\ruby{人}{ひと}も
\ruby{苦}{くるし}み
\ruby{甲{\換字{斐}}}{が|ひ}がありましやうが、
%
\ruby{性}{しやう}が
\ruby{合}{あ}はないとでも
\ruby{云}{い}ふのでしやうか、
%
\ruby{御師匠}{お|し|よ}さんの
\ruby{談}{はなし}では
\ruby{{\換字{嫌}}}{きら}つて
\ruby{{\換字{嫌}}}{きら}ひ
\ruby{拔}{ぬ}いて、
%
\ruby{有{\換字{難}}}{あり|がた}いとも
\ruby{嬉}{うれ}しいとも
\ruby{思}{おも}ひさうも
\ruby{無}{な}いといふんですもの、
%
\ruby{彼}{あ}の
\ruby{人}{ひと}の
\ruby{立}{た}つ
\ruby{瀬}{せ}は
\ruby{有}{あ}りやあ
\ruby{仕}{し}ませんはネ。
%
それに
\ruby{段々}{だん|〴〵}と
\ruby{吾家}{う|ち}の
\ruby{御師匠}{お|し|よ}さんの
\ruby{口占}{くち|うら}を
\ruby{引}{ひ}いて
\ruby{見}{み}ますと、
%
\ruby{今度}{こん|ど}の
\ruby{事}{こと}の
\ruby{起}{おこ}るずつと
\ruby{{\換字{前}}}{まへ}から、
%
お
\ruby{師匠}{し|よ}さんは
\ruby{彼}{あ}の
\ruby{人}{ひと}が
お
\ruby[g]{五十}{いそ}さんを
\ruby{思}{おも}つてるのに
\ruby{附{\換字{込}}}{つけ|こ}んでネ、
%
\ruby{將來}{ゆく|〳〵}は
お
\ruby[g]{五十}{いそ}をあげましやうといふやうな
\ruby{事}{こと}を
\ruby{巧}{うま}く
\ruby{匂}{にほ}はせて、
%
\ruby{何}{なん}とか
\ruby{彼}{か}とか
\ruby{口實}{いひ|ぐさ}を
\ruby{拵}{こしら}へては
\ruby{{\換字{若}}干金}{い|く|ら}かづつ
\ruby{絞}{しぼ}つたらしいので、
%
どうも
\ruby{後{\換字{前}}}{あと|さき}を
\ruby{能}{よう}く
\ruby{考}{かんが}へて
\ruby{見}{み}ると
\ruby{屹度}{きつ|と}さうなのですよ。
』

\原本頁{}%
『へーエ、
%
\ruby{罪}{つみ}な
\ruby{事}{こと}を
\ruby{仕}{し}たものだネエ!、
%
お
\ruby{關}{せき}さんといふ
\ruby{人}{ひと}は。
』

\原本頁{}%
『
\ruby{罪}{つみ}ですともほんとに!。
%
あんな
\ruby{生眞面目}{き|ま|じ|め}な
\ruby{初心}{う|ぶ}な
\ruby{人}{ひと}を
\ruby{欺}{だま}すのですもの。
』

\原本頁{}%
『ぢやあ、
%
お
\ruby{{\換字{前}}}{まへ}の
\ruby{御師匠}{お|し|よ}さんていふ
\ruby{人}{ひと}は
\ruby{惡}{わる}い
\ruby{人}{ひと}ちやあ
\ruby{無}{な}いか。
』

\原本頁{}%
『
\ruby{唯}{えゝ}、
%
まあ
\ruby{善}{い}い
\ruby{人}{ひと}たあ
\ruby{御師匠樣}{お|し|よ|さん}ですけれども
\ruby{云}{い}へませんネエ。
%
で、
%
\ruby{吾家}{う|ち}の
\ruby{御師匠樣}{お|し|よ|さん}が
\ruby{萬一}{も|し}
\ruby{普{\換字{通}}}{ひと|なみ}に
\ruby{人{\換字{情}}合}{にん|じよう|あひ}の
\ruby{{\換字{分}}}{わか}る
\ruby{人}{ひと}ならば、
%
\ruby{從{\換字{前}}}{いま|ゝで}の
\ruby{事}{こと}は
\ruby{何樣}{ど|う}でも
\ruby{斯樣}{か|う}でも
\ruby{濟}{す}んだことだから
\ruby{仕方}{し|かた}が
\ruby{無}{な}いとしても、
%
\ruby{今度}{こん|ど}は
\ruby{云}{い}はゞ
\ruby[g]{水野}{みづの}さんの
\ruby{世話一}{せ|わ|ひと}ツで
お
\ruby[g]{五十}{いそ}さんを
\ruby{取}{と}り
\ruby{{\換字{留}}}{と}めたのですから、
%
\ruby{床上}{とこ|あ}げでも
\ruby{濟}{す}んだ
\ruby{其}{そ}の
\ruby{曉}{あかつき}にやあ、
%
たとひ
お
\ruby[g]{五十}{いそ}さんが
\ruby{何}{なん}と
\ruby{云}{い}はうとも
\ruby{割}{わつ}つ
\ruby{口說}{く|ど}いつして、
%
\ruby[g]{水野}{みづの}さんに
\ruby{嫁}{や}るやうにでも
\ruby{仕}{し}なくちやあならない
\ruby{筈}{はず}だと
\ruby{思}{おも}ひますは。
%
ネエ
\ruby{姊}{ねえ}さん、
%
\ruby{然樣}{さ|う}ぢやあ
\ruby{有}{あ}りませんか、
%
\ruby{義理}{ぎ|り}つてえものがネエ。
』

\原本頁{}%
『
\ruby{成程}{なる|ほど}
お
\ruby{{\換字{前}}}{まへ}が
お
\ruby[g]{五十}{いそ}さんの
\ruby{御母}{お|つか}さんだつたら
\ruby{然樣}{さ|う}も
\ruby{御爲}{お|し}だらうとおもはれるよ。
』

\原本頁{}%
お
\ruby{龍}{りう}は
\ruby{此}{こ}の
お
\ruby{彤}{とう}が
\ruby{答}{こたへ}に
\ruby{少}{すくな}からぬ
\ruby{不足}{ふ|そく}の
\ruby{色}{いろ}を
\ruby[<j|]{現}{あらは}
したり。

\原本頁{}%
『ぢやあ
\ruby{姊}{ねえ}さんが
\ruby{{\換字{若}}}{も}し
\ruby{御師匠}{お|し|よ}さんだつたら?。
』

\原本頁{}%
『ホヽヽ、
%
\ruby{挨拶}{あい|さつ}が
\ruby[<j|]{些}{ちつと}
\ruby{氣}{き}に
\ruby{入}{い}らなかつたネ。
%
\ruby{妾}{わたし}が
お
\ruby[g]{五十}{いそ}さんの
\ruby{母}{おつか}さんならカエ。
%
さうさねエ、
%
\ruby{妾}{わたし}ならまあ、
%
\ruby{先}{さき}へ
\ruby{恩{\換字{返}}}{おん|がへ}しを
\ruby{仕}{し}て
\ruby{置}{お}いてネ、‥‥
\ruby{世話}{せ|わ}になつた
\ruby{恩}{おん}は
\ruby{恩}{おん}で
\ruby[g]{水野}{みづの}さんに
\ruby{恩{\換字{返}}}{おん|がへ}しを
\ruby{仕}{し}てネ、
%
\ruby{緣}{えん}の
\ruby{事}{こと}は
\ruby{其}{それ}から
\ruby{後}{あと}で
\ruby{決}{き}めやうと
\ruby{思}{おも}ふネ。
』

\原本頁{}%
『
\ruby{然樣}{さ|う}!。
%
それならそれで
\ruby{其}{それ}もまた
\ruby{譯}{わけ}の
\ruby{{\換字{分}}}{わか}つた
\ruby{大變}{たい|へん}に
\ruby{良}{い}い
\ruby{仕方}{し|かた}だと
\ruby{妾}{わたし}もおもひますは。
%
ところが
\ruby{吾家}{う|ち}の
\ruby{御師匠}{お|し|よ}さんは
\ruby{妾}{わたし}の
\ruby{云}{い}つたやうに
\ruby{仕}{し}やうでも
\ruby{無}{な}けりやあ、
%
\ruby{姊}{ねえ}さんの
お
\ruby{云}{い}ひのやうに
\ruby{仕}{し}やうでも
\ruby{無}{な}いんで、
%
たゞ
\ruby{病患}{わ|る}い
\ruby{時}{とき}やあ
\ruby{人}{ひと}まかせに
\ruby{仕}{し}て
\ruby{置}{お}いて、
%
\ruby{治}{なほ}
りやあ
\ruby{自{\換字{分}}}{じ|ぶん}の
\ruby{子}{こ}つていふやうな
\ruby{{\換字{勝}}手}{かつ|て}な
\ruby{料簡}{れう|けん}で、
%
いつまでも
\ruby[g]{水野}{みづの}さんは
\ruby{釣}{つ}りつばなしに
\ruby{仕}{し}て
\ruby{打棄}{うつ|ちや}つて
\ruby{置}{お}かうといふんですもの、
%
\ruby{酷}{ひど}いぢやあ
\ruby{有}{あ}りませんか。
』

\原本頁{}%
『そりやあ
\ruby{酷}{ひど}いとも!。
%
\ruby{酷}{ひど}い
\ruby{人}{ひと}だよ。
%
\ruby{聞}{き}いて
\ruby{見}{み}りやあ
\ruby[g]{眞個}{ほんと}に
お
\ruby{{\換字{前}}}{まへ}の
\ruby{御師匠}{お|し|よ}さんて
\ruby{云}{い}ふのは
\ruby{惡}{わる}い
\ruby{人}{ひと}だよ。
』

\原本頁{}%
『でもまあ
\ruby{緣}{えん}の
\ruby{事}{こと}は
\ruby{當人}{たう|にん}
\ruby{同士}{どう|し}の
\ruby{事}{こと}で、
%
\ruby{親}{おや}の
\ruby{思}{おも}ふやうにばかりもならない
\ruby{理}{すぢ}も
\ruby{有}{あ}りましやう。
%
ですからお
\ruby[g]{五十}{いそ}さんが
\ruby{{\換字{嫌}}}{いや}なら
\ruby{{\換字{嫌}}}{いや}で
\ruby{{\換字{強}}}{し}ひるわけには
\ruby{行}{ゆ}かないとして、
%
\ruby{其}{それ}あ
\ruby{其}{それ}で
\ruby{可}{い}いとしたところが
\ruby{恩}{おん}は
\ruby{恩}{おん}ですもの、
%
\ruby{恩}{おん}は
\ruby{何處}{ど|こ}までも
\ruby{着}{き}なけりやあなりません。
%
まして
\ruby[g]{水野}{みづの}さんが
\ruby{困}{こま}るといふ
\ruby{時{\換字{節}}}{は|め}になりやあ、
%
\ruby{何樣}{ど|う}しても
\ruby{知}{し}らん
\ruby{顏}{かほ}ぢやあ
\ruby{居}{ゐ}られない
\ruby{譯}{わけ}で、
%
\ruby{出來}{で|き}ないまでも
\ruby{心配}{しん|ぱい}だけなりと
\ruby{仕}{し}なくちやあなりませんはネ。
』
