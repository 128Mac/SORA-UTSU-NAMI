\Entry{其十六}

『ところが
\ruby{吾家}{う|ち}の
\ruby{御師匠}{お|し|よ}さんと
\ruby{來}{き}た
\ruby{日}{ひ}にやあ
\ruby[g]{眞個}{ほんと}に
\ruby{酷}{ひど}い
\ruby{人}{ひと}で、
\ruby{妾}{わたし}がこれ〳〵だといふ
\ruby{話}{はなし}を
\ruby{仕}{し}て
\ruby{聞}{き}かせても、フーン
\ruby{然樣}{さ|う}かエと
\ruby{云}{い}つたばかりで
\ruby{氣}{き}の
\ruby{毒}{どく}とも
\ruby{云}{い}はずに、
\ruby{默}{だま}つて
\ruby[g]{懐手}{ふところで}で
\ruby[g]{高處}{たかみ}で
\ruby{見物}{けん|ぶつ}しやうといふんですもの、
\ruby{餘}{あんま}りぢやあ
\ruby{有}{あ}りませんか。
それも
\ruby{水野}{みづ|の}さんが
\ruby{職}{しよく}を
\ruby{辭}{じ}すやうになつた
\ruby{其}{そ}の
\ruby{原因}{も|と}が、
\ruby{何}{なに}も
\ruby{關係}{かけ|かまひ}の
\ruby{無}{な}いことなら
\ruby{其}{それ}で
\ruby{宜}{い}いかも
\ruby{知}{し}りませんが、
\ruby{彼}{あ}の
\ruby{人}{ひと}が
\ruby{學藝}{わ|ざ}が
\ruby{出來}{で|き}ないといふのぢやあ
\ruby{無}{な}し、
\ruby[g]{怠惰}{なまけ}たといふのぢやあ
\ruby{無}{な}し、たゞお
\ruby{五十}{い|そ}さんに
\ruby{親切}{しん|せつ}にして、
\ruby{信心}{しん|〴〵}まで
\ruby{仕}{し}た
\ruby{其事}{そ|れ}が
\ruby{人目}{ひと|め}に
\ruby{立}{た}つて、
\ruby{傍}{はた}の
\ruby[g]{風評}{うはさ}が
\ruby[g]{矢鱈}{やたら}に
\ruby{喧}{やか}ましくなつて、
\ruby{其}{それ}が
\ruby{爲}{ため}に
\ruby{職}{やく}を
\ruby{退}{ひ}いたといふのですから、
\ruby{云}{い}はゞ
\ruby{此方}{こつ|ち}の
\ruby{爲}{ため}に
\ruby{然樣}{さ|う}いふ
\ruby{譯}{わけ}になつたのですもの、
\ruby[g]{石佛}{いしぼとけ}だつて
\ruby{氣}{き}の
\ruby{毒}{どく}と
\ruby{思}{おも}はずには
\ruby{居}{ゐ}られさうも
\ruby{無}{な}いところです。
それを
\ruby{何樣}{ど|う}でしやう
\ruby{全然}{まる|で}
\ruby{知}{し}らん
\ruby{顏}{かほ}で、
\ruby{濟}{す}まして
\ruby{行}{ゆ}かうといふのです!。
\ruby{人間}{にん|げん}も
\ruby{其}{そ}の
\ruby{位}{くらひ\ }
\ruby{身{\換字{勝}}手}{み|がつ|て}になれりやあ
\ruby{澤山}{たく|さん}だと
\ruby{思}{おも}ひますは。
』

『だつて
\ruby{惡}{わる}い
\ruby{人}{ひと}なら
\ruby{其}{そ}の
\ruby{位}{くらゐ}の
\ruby{事}{こと}は
\ruby{{\換字{平}}氣}{へい|き}で
\ruby{仕}{し}やうぢあ
\ruby{無}{な}いか。
』

『そりやあ
\ruby{云}{い}つて
\ruby{見}{み}ればまあ
\ruby{其樣}{そ|ん}なもので
\ruby{不思議}{ふ|し|ぎ}はありますまいがネ、
\ruby[g]{丁度中}{ちやうどなか}に
\ruby{介}{はさ}まつてゐる
\ruby{妾}{わたし}が
\ruby{兩方}{りやう|はう}を
\ruby{見}{み}ますとネ、つくづく
\ruby{吾家}{う|ち}のお
\ruby{師匠}{し|しやう}さんを
\ruby{餘}{あんま}りだと
\ruby{思}{おも}ふ
\ruby{其}{それ}に
\ruby{{\換字{連}}}{つ}れて
\ruby{水野}{みづ|の}さんが
\ruby{愍然}{かはい|さう}で
\ruby{愍然}{かはい|さう}で、ほんとに
\ruby{何}{なん}といふ
\ruby{愍然}{かはい|さう}な
\ruby{人}{ひと}だらうと
\ruby{身}{み}に
\ruby{浸}{し}みて
\ruby{思}{おも}ひますは。
』

『さうさネエ、まあ
\ruby{愍然}{かはい|さう}で
\ruby{無}{な}い
\ruby{事}{こと}も
\ruby{無}{な}いネエ。
』

『あらツ!、まあ
\ruby{愍然}{かはい|さう}で
\ruby{無}{な}い
\ruby{事}{こと}も
\ruby{無}{な}いネエだなんて、
\ruby{餘}{あんま}りですは。
いくら
\ruby{自{\換字{分}}}{じ|ぶん}が
\ruby{{\換字{迷}}}{まよ}つたのだから
\ruby{仕方}{し|かた}が
\ruby{無}{な}いとは
\ruby{云}{い}ふものゝ、
\ruby{助}{たす}かるか
\ruby{死}{し}ぬかも
\ruby{知}{し}れない
\ruby{病人}{びやう|にん}に
\ruby{對}{むか}つて、
\ruby{心配}{しん|ぱい}も
\ruby{仕}{し}て
\ruby{{\換字{遣}}}{や}る、お
\ruby{金}{かね}も
\ruby{掛}{か}ける、
\ruby{書生}{しよ|せい}さん
\ruby{風}{ふう}の
\ruby{人}{ひと}だのに
\ruby{信心}{しん|〴〵}まで
\ruby{仕}{し}て、
\ruby{此}{こ}の
\ruby{節}{せつ}の
\ruby{人}{ひと}
の
\ruby{爲}{し}さうにも
\ruby{無}{な}い
\ruby{觀音樣}{くわん|のん|さま}に
\ruby{手}{て}を
\ruby{合}{あは}せるといふやうな
\ruby{事}{こと}まで
\ruby{爲}{し}たのは、まあよく〳〵の
\ruby{事}{こと}で
\ruby{無}{な}くつちやあ
\ruby{出來}{で|き}ませんは。
それだのに
\ruby[g]{其程思}{それほどおも}つてる
\ruby{人}{ひと}にやあ
\ruby{酷}{ひど}く
\ruby{{\換字{嫌}}}{きら}はれて、そして
\ruby{吾家}{う|ち}のお
\ruby{師匠}{し|よ}さんにやあ
\ruby{口頭}{くち|さき}だけで
\ruby{綾}{あや}なされて、
\ruby[g]{御腹}{おなか}の
\ruby{中}{なか}ぢやあ
\ruby{舌}{した}を
\ruby{出}{だ}して
\ruby{笑}{わら}つて
\ruby{居}{ゐ}られて、
\ruby{揚句}{あげ|く}の
\ruby{果}{はて}に
\ruby{取}{と}るものも
\ruby{取}{と}れ
\ruby{無}{な}い
\ruby{身}{み}になつて
\ruby{仕舞}{し|ま}ふなんて、そりやあ
\ruby[g]{男兒}{をとこ}のことですから
\ruby{胸}{むね}
\ruby{濶}{ひろ}いで\換字{志}やうし、
\ruby{氣性}{きし|やう}も
\ruby[g]{毅然}{しつかり}と
\ruby{仕}{し}て
\ruby{居}{ゐ}るらしい
\ruby{人}{ひと}ですから、まんざらくよ〳〵も
\ruby{仕}{し}ますまいが、
\ruby{妾}{わたし}が
\ruby{若}{も}し
\ruby{彼}{あ}の
\ruby{人}{ひと}の
\ruby{身}{み}だつたら、まあ
\ruby{何樣}{ど|ん}なでしやう!。
\ruby{此}{こ}の
\ruby{先}{さき}お
\ruby{五十}{い|そ}さんの
\ruby{氣}{き}が
\ruby{折}{を}れて
\ruby{優}{やさし}しくでもなつたら
\ruby{濟}{す}みも
\ruby{仕}{し}ましやうが、
\ruby{若}{も}しお
\ruby{五十}{い|そ}さんはお
\ruby{五十}{い|そ}さんで
\ruby{何處}{ど|こ}までも
\ruby{剛{\換字{情}}}{がう|じやう}を
\ruby{張}{は}り、お
\ruby{師匠}{し|よ}さんはお
\ruby{師匠}{し|よ}さんで
\ruby{鼻}{はな}の
\ruby{尖}{さき}ばかりで
\ruby[g]{待遇}{あしら}つて
\ruby{行}{い}つたら、
\ruby[g]{何程男兒}{いくらをとこ}だつて
\ruby{{\換字{迷}}}{まよ}つた
\ruby{心持}{こゝろ|もち}の
\ruby{苦}{くる}しさは
\ruby{女}{をんな}と
\ruby{異}{ちが}ひも
\ruby{仕}{し}ますまいもの、
\ruby{何樣}{ど|ん}なにか
\ruby{泣}{な}きも
\ruby{仕}{し}ましやう、
\ruby{恨}{うら}みも
\ruby{仕}{し}ましやう、
\ruby{口惜}{く|やし}がりも
\ruby{仕}{し}ましやう。
\ruby[g]{愍然}{かはいさう}に
\ruby{彼}{あ}の
\ruby{人}{ひと}は
\ruby{云}{い}はゞ
\ruby[g]{淸玄見}{せいげんみ}たやうなものになつて、
\ruby[g]{{\換字{終}}局}{しまひ}にやあ
\ruby{段々}{だん|〳〵}との
\ruby{行掛}{ゆき|がかり}づくから、
\ruby{何樣}{ど|ん}な
\ruby{怖}{おそ}ろしい
\ruby{恐}{こは}い
\ruby{場}{ば}に
\ruby{行}{ゆ}き
\ruby{着}{つ}かうかも
\ruby{知}{し}れません。
もし
\ruby{然樣}{さ|う}なつたところでお
\ruby{五十}{い|そ}さんやお
\ruby{師匠}{し|よ}さんは、
\ruby{身}{み}から
\ruby{出}{で}た
\ruby{錆}{さび}だから
\ruby{仕方}{し|かた}が
\ruby{無}{な}いとしても、
\ruby{別}{べつ}に
\ruby{何}{なに}も
\ruby{惡}{わる}い
\ruby{事}{こと}は
\ruby{仕}{し}ない
\ruby{彼}{あ}の
\ruby{{\換字{情}}}{じやう}の
\ruby{厚}{あつ}い、
\ruby{正直}{しやう|ぢき}な、
\ruby{生無垢}{き|む|く}な、
\ruby{彼}{あ}の
\ruby{前{\換字{途}}}{おひ|さき}が
\ruby{有}{あ}りさうな
\ruby{彼}{あ}の
\ruby{人}{ひと}が……
\ruby{見}{み}す〳〵
\ruby[g]{一人廢}{ひとりすた}つて
\ruby{仕舞}{し|ま}ふのは
\ruby[g]{愍然}{かはいさう}ぢやあ
\ruby{有}{あ}りませんか。
ネエ
\ruby{姊}{ねえ}さん、
\ruby{察}{さつ}しの
\ruby{宜}{い}い
\ruby{姊}{ねえ}さんに
\ruby{其處}{そ|こ}が
\ruby{解}{わか}らない
\ruby{事}{こと}はありますまい。
\ruby{惡}{わる}い
\ruby{事}{こと}も
\ruby{仕}{し}ない
\ruby{人}{ひと}が
\ruby{見}{み}す〳〵
\ruby[g]{人一人廢}{ひとひとるすた}りさうな、それが
\ruby{愍然}{かはい|さう}で
\ruby{無}{な}い
\ruby{事}{こと}はありますまい、ねエ
\ruby{姊}{ねえ}さん。
』

\ruby[g]{{\換字{情}}激}{じやうげき}してやお
\ruby{龍}{りう}が
\ruby{面}{おもて}はやゝ
\ruby{紅}{あか}くなり、
\ruby{其}{そ}の
\ruby{眼}{め}は
\ruby{濡}{ぬ}れ
\ruby{色}{いろ}を
\ruby{帶}{お}びて
\ruby{異}{あや}しく
\ruby{光}{ひかり}を
\ruby{{\換字{増}}}{ま}せり。

