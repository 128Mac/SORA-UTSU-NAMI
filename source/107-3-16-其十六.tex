\Entry{其十六}

% メモ 校正終了 2024-05-13
\原本頁{87-7}%
『
ところが
\ruby{吾家}{う|ち}の
\ruby{御師匠}{お|し|よ}さんと
\ruby{來}{き}た
\ruby{日}{ひ}にやあ
\ruby[|g|]{眞個}{ほんと}に
\ruby{酷}{ひど}い
\ruby{人}{ひと}で、
%
\原本頁{87-8}\改行%
\ruby{妾}{わたし}が
これ〳〵だ
といふ
\ruby{話}{はなし}を
\ruby{仕}{し}て
\ruby{聞}{き}かせても、
%
フーン
\ruby{然樣}{さ|う}かエと
\ruby{云}{い}つた
ばかりで
\ruby{氣}{き}の
\ruby{毒}{どく}とも
\ruby{云}{い}はずに、
%
\ruby{默}{だま}つて
\ruby[<j>]{懷}{ふところ}
\ruby[||j>]{手}{ で }で
% \ruby{懷手}{ふところ|で}で
\ruby{高處}{たか|み}で
\ruby{見物}{けん|ぶつ}
しやうと
いふん
ですもの、
%
\ruby{餘}{あんま}り
ぢやあ
\ruby{有}{あ}りませんか。
%
それも
\ruby{水野}{みづ|の}さんが
\ruby{職}{やく}を
\ruby{辭}{よ}す
やうに
なつた
\ruby{其}{そ}の
\ruby{原因}{も|と}が、
%
\ruby{何}{なに}も
\ruby[||j>]{關}{かけ}
\ruby[||j>]{係}{かまひ}の
% \ruby{關係}{かけ|かまひ}の
\ruby{無}{な}い
ことなら
\ruby{其}{それ}で
\ruby{宜}{い}いかも
\ruby{知}{し}りませんが、
%
\ruby{彼}{あ}の
\ruby{人}{ひと}が
\ruby{學藝}{わ|ざ}が
\ruby{出來}{で|き}ないと
\原本頁{88-3}\改行%
いふのぢやあ
\ruby{無}{な}し、
%
\ruby{怠惰}{なま|け}たと
いふのぢやあ
\ruby{無}{な}し、
%
たゞ
お
\ruby{五十}{い|そ}さんに
\ruby{親切}{しん|せつ}にして、
%
\ruby{信心}{しん|〴〵}まで
\ruby{仕}{し}た
\ruby{其事}{そ|れ}が
\ruby{人目}{ひと|め}に
\ruby{立}{た}つて、
%
\ruby{傍}{はた}の
\ruby[|g|]{風{\換字{評}}}{うはさ}が
\ruby{矢鱈}{や|たら}に
\ruby{喧}{やか}ましく
なつて、
%
\ruby{其}{それ}が
\ruby{爲}{ため}に
\ruby{職}{やく}を
\ruby{{\換字{退}}}{ひ}いた
といふ
のですから、
%
\ruby{云}{い}はゞ
\ruby{此方}{こつ|ち}の
\ruby{爲}{ため}に
\ruby{然樣}{さ|う}いふ
\ruby{譯}{わけ}に
なつた
のです
もの、
%
\ruby[||j>]{石}{いし}
\ruby[||j>]{佛}{ぼとけ}
% \ruby{石佛}{いし|ぼとけ}
だつて
\ruby{氣}{き}の
\ruby{毒}{どく}と
\ruby{思}{おも}はずには
\ruby{居}{ゐ}られ
さうも
\ruby{無}{な}い
ところ
です。
%
それを
\原本頁{88-8}\改行%
\ruby{何樣}{ど|う}
でしやう
\ruby{全然}{まる|で}
\ruby{知}{し}らん
\ruby{顏}{かほ}で、
%
\ruby{濟}{す}まして
\ruby{行}{ゆ}かう
といふ
のです!。
%
\原本頁{88-9}\改行%
\ruby{人間}{にん|げん}も
\ruby{其}{そ}の
\ruby{位}{くらゐ}
\ruby{身{\換字{勝}}手}{ み|がつ|て}% ルビが連なるので対処
になれりやあ
\ruby{澤山}{たく|さん}だと
\ruby{思}{おも}ひますは。
』

\原本頁{88-10}%
『
だつて
\ruby{惡}{わる}い
\ruby{人}{ひと}なら
\ruby{其}{そ}の
\ruby{位}{くらゐ}の
\ruby{事}{こと}は
\ruby{{\換字{平}}氣}{へい|き}で
\ruby{仕}{し}やう
ぢあ
\ruby{無}{な}いか。
』

\原本頁{88-11}%
『
そりやあ
\ruby{云}{い}つて
\ruby{見}{み}れば
まあ
\ruby{其樣}{そ|ん}な
もので
\ruby{不思議}{ふ|し|ぎ}は
あります
ま
\原本頁{89-1}\改行%
いがネ、
%
\ruby{丁度}{ちやう|ど}
\ruby{中}{なか}に
\ruby{介}{はさ}まつて
ゐる
\ruby{妾}{わたし}が
\ruby[||j>]{兩}{りやう}
\ruby[||j>]{方}{ はう}を
% \ruby{兩方}{りやう|はう}を
\ruby{見}{み}ますとネ、
%
つくづく% 行末行頭禁則につき非踊り字表記
\ruby{吾家}{う|ち}の
お
\ruby{師匠}{し|よ}さんを
\ruby{餘}{あんま}りだと
\ruby{思}{おも}ふ
\ruby{其}{それ}に
\ruby{{\換字{連}}}{つ}れて
\ruby{水野}{みづ|の}さんが
\ruby[<j||]{愍}{かは }% 「愍然 か(は)いさう」
\ruby[<j||]{然}{いさう}で
% \ruby{愍然}{かは|いさう}で% 「愍然 か(は)いさう」
\ruby[||j>]{愍}{かは}% 「愍然 か(は)いさう」
\ruby[||j>]{然}{いさう}で、
% \ruby{愍然}{かは|いさう}で、
%
ほんとに
\ruby{何}{なん}
といふ
\ruby[||j>]{愍}{かは}% 「愍然 か(は)いさう」
\ruby[||j>]{然}{いさう}な
% \ruby{愍然}{かは|いさう}な% 「愍然 か(は)いさう」
\ruby{人}{ひと}
だらうと
\ruby{身}{み}に
\ruby{浸}{し}みて
\ruby{思}{おも}ひますは。
』

\原本頁{89-5}%
『
さうさネエ、
%
まあ
\ruby[||j>]{愍}{かは}% 「愍然 か(は)いさう」
\ruby[||j>]{然}{いさう}で
% \ruby{愍然}{かは|いさう}で% 「愍然 か(は)いさう」
\ruby{無}{な}い
\ruby{事}{こと}も
\ruby{無}{な}いネエ。
』

\原本頁{89-6}%
『
あらツ!、
%
まあ
\ruby[||j>]{愍}{かは}% 「愍然 か(は)いさう」
\ruby[||j>]{然}{いさう}で
% \ruby{愍然}{かは|いさう}で% 「愍然 か(は)いさう」
\ruby{無}{な}い
\ruby{事}{こと}も
\ruby{無}{な}いネエ
だなんて、
%
\ruby{餘}{あんま}りですは。
%
いくら
\ruby{自{\換字{分}}}{じ|ぶん}が
\ruby{{\換字{迷}}}{まよ}つた
の
だから
\ruby{仕方}{し|かた}が
\ruby{無}{な}いとは
\ruby{云}{い}ふ
ものゝ、
%
\原本頁{89-8}\改行%
\ruby{助}{たす}かるか
\ruby{死}{し}ぬかも
\ruby{知}{し}れない
\ruby[||j>]{病}{びやう}
\ruby[||j>]{人}{ にん}に
% \ruby{病人}{びやう|にん}に
\ruby{對}{むか}つて、
%
\ruby{心配}{しん|ぱい}も
\ruby{仕}{し}て
\ruby{{\換字{遣}}}{や}る、
%
お
\ruby{金}{かね}も
\ruby{掛}{か}ける、
%
\ruby{書生}{しよ|せい}さん
\ruby{風}{ふう}の
\ruby{人}{ひと}だのに
\ruby{信心}{しん|〴〵}まで
\ruby{仕}{し}て、
%
\ruby{此}{こ}の
\ruby{{\換字{節}}}{せつ}の
\ruby{人}{ひと}
\原本頁{89-10}\改行%
の
\ruby{爲}{し}さうにも
\ruby{無}{な}い
\ruby[<j||]{觀}{くわん}% 「觀音」の読みは原本通り「くわん(の)ん」
\ruby[<j||]{音}{のん }
\ruby{樣}{さま}に
\ruby{手}{て}を
\ruby{合}{あは}せる
といふ
やうな
\ruby{事}{こと}まで
\ruby{爲}{し}たのは、
%
まあ
よく〳〵の
\ruby{事}{こと}で
\ruby{無}{な}くつちやあ
\ruby{出來}{で|き}ませんは。
%
それだのに
\ruby{其}{それ}
\ruby{程}{ほど}
\ruby{思}{おも}つてる
\ruby{人}{ひと}にやあ
\ruby{酷}{ひど}く
\ruby{{\換字{嫌}}}{きら}はれて、
%
そして
\ruby{吾家}{う|ち}の
お
\ruby{師匠}{し|よ}さん
にやあ
\ruby{口}{くち}
\ruby{頭}{さき}だけで
\ruby{綾}{あや}なされて、
%
\ruby{御腹}{お|なか}の
\ruby{中}{なか}ぢやあ
\ruby{舌}{した}を
\ruby{出}{だ}して
\原本頁{90-3}\改行%
\ruby{笑}{わら}つて
\ruby{居}{ゐ}られて、
%
\ruby{揚句}{あげ|く}の
\ruby{果}{はて}に
\ruby{取}{と}るものも
\ruby{取}{と}れ
\ruby{無}{な}い
\ruby{身}{み}になつて
\ruby{仕舞}{し|ま}ふ
なんて、
%
そりやあ
\ruby[|g|]{男兒}{をとこ}
のこと
ですから
\ruby{胸}{むね}も
\ruby{濶}{ひろ}いで
\換字{志}やうし、
%
\原本頁{90-5}\改行%
\ruby{氣性}{き|しやう}も
\ruby{毅然}{しつ|かり}と
\ruby{仕}{し}て
\ruby{居}{ゐ}るらしい
\ruby{人}{ひと}
ですから、
%
まんざら
くよ〳〵も
\ruby{仕}{し}ますまいが、
%
\ruby{妾}{わたし}が
\ruby{{\換字{若}}}{も}し
\ruby{彼}{あ}の
\ruby{人}{ひと}の
\ruby{身}{み}だつたら、
%
まあ
\ruby{何樣}{ど|ん}なでしやう!。
%
\ruby{此}{こ}の
\ruby{先}{さき}
お
\ruby{五十}{い|そ}さんの
\ruby{氣}{き}が
\ruby{折}{を}れて
\ruby{優}{やさし}しく
でも
なつたら
\ruby{濟}{す}みも
\ruby{仕}{し}ましやうが、
%
\ruby{{\換字{若}}}{も}し
お
\ruby{五十}{い|そ}さんは
お
\ruby{五十}{い|そ}さんで
\ruby{何處}{ど|こ}までも
\原本頁{90-9}\改行%
\ruby[||j>]{剛}{がう}
\ruby[||j>]{{\換字{情}}}{じやう}を
% \ruby{剛{\換字{情}}}{がう|じやう}を
\ruby{張}{は}り、
%
お
\ruby{師匠}{し|よ}さんは
お
\ruby{師匠}{し|よ}さんで
\ruby{鼻}{はな}の
\ruby{尖}{さき}
ばかりで
\ruby{待{\換字{遇}}}{あし|ら}つて
\ruby{行}{い}つたら、
%
\ruby{何程}{いく|ら}
\ruby[|g|]{男兒}{をとこ}
だつて
\ruby{{\換字{迷}}}{まよ}つた
\ruby[||j>]{心}{こゝろ}
\ruby[||j>]{持}{ もち}の
% \ruby{心持}{こゝろ|もち}の
\ruby{苦}{くる}しさは
\ruby{女}{をんな}と
\ruby{異}{ちが}ひも
\原本頁{90-11}\改行%
\ruby{仕}{し}ますまいもの、
%
\ruby{何樣}{ど|ん}なにか
\ruby{泣}{な}きも
\ruby{仕}{し}ましやう、
%
\ruby{恨}{うら}みも
\ruby{仕}{し}ましやう、
%
\ruby{口惜}{く|やし}がりも
\ruby{仕}{し}ましやう。
%
\ruby[||j>]{愍}{かは}% 「愍然 か(は)いさう」
\ruby[||j>]{然}{いさう}に
% \ruby{愍然}{かは|いさう}に% 「愍然 か(は)いさう」
\ruby{彼}{あ}の
\ruby{人}{ひと}は
\ruby{云}{い}はゞ
\ruby{淸玄}{せい|げん}% 「清水清玄(きよみずせいげん)」かしら?
\ruby{見}{み}たやうな
ものに
なつて、
%
\ruby[|g|]{{\換字{終}}局}{しまひ}
にやあ
\ruby{段々}{だん|〴〵}
との
\ruby[||j>]{行}{ゆき}
\ruby[||j>]{掛}{がかり}づくから、
% \ruby{行掛}{ゆき|がかり}づくから、
%
\ruby{何樣}{ど|ん}な
\ruby{怖}{おそ}ろしい
\ruby{恐}{こは}い
\ruby{場}{ば}に% 原文通り「場」
\ruby{行}{ゆ}き
\ruby{着}{つ}かう
かも
\ruby{知}{し}れません。
%
よし% 普通は「もし」だと思うが原文通り「よし」
\ruby{然樣}{さ|う}
なつた
\原本頁{91-4}\改行%
ところで
お
\ruby{五十}{い|そ}さんや
お
\ruby{師匠}{し|よ}さんは、
%
\ruby{身}{み}から
\ruby{出}{で}た
\ruby{錆}{さび}
だから
\ruby{仕方}{し|かた}が
\ruby{無}{な}い
としても、
%
\ruby{別}{べつ}に
\ruby{何}{なに}も
\ruby{惡}{わる}い
\ruby{事}{こと}は
\ruby{仕}{し}ない
\ruby{彼}{あ}の
\ruby{{\換字{情}}}{じやう}の
\ruby{厚}{あつ}い、
%
\ruby[<j||]{正}{しやう}% 行末行頭の境界付近なので特例処置を施す
\ruby[<j||]{直}{ ぢき}な、
% \ruby{正直}{しやう|ぢき}な、
%
\ruby{生無垢}{き|む|く}な、
%
\ruby{彼}{あ}の
\ruby{{\換字{前}}{\換字{途}}}{おひ|さき}が
\ruby{有}{あ}りさうな
\ruby{彼}{あ}の
\ruby{人}{ひと}が
‥‥
\ruby{見}{み}す〳〵
\ruby{人}{ひと}
\ruby[|g|]{一人}{ひとり}
\ruby{廢}{すた}つて
\ruby{仕舞}{し|ま}ふのは
\ruby[||j>]{愍}{かは}% 「愍然 か(は)いさう」
\ruby[||j>]{然}{いさう}ぢやあ
% \ruby{愍然}{かは|いさう}ぢやあ% 「愍然 か(は)いさう」
\ruby{有}{あ}りませんか。
%
ネエ
\ruby{姊}{ねえ}さん、
%
\原本頁{91-8}\改行%
\ruby{察}{さつ}しの
\ruby{宜}{い}い
\ruby{姊}{ねえ}さんに
\ruby{其處}{そ|こ}が
\ruby{解}{わか}らない
\ruby{事}{こと}は
ありますまい。
%
\ruby{惡}{わる}い
\ruby{事}{こと}も
\ruby{仕}{し}ない
\ruby{人}{ひと}が
\ruby{見}{み}す〳〵
\ruby{人}{ひと}
\ruby[|g|]{一人}{ひとり}
\ruby{廢}{すた}りさうな、
%
それが
\ruby[||j>]{愍}{かは}% 「愍然 か(は)いさう」
\ruby[||j>]{然}{いさう}で
% \ruby{愍然}{かは|いさう}で% 「愍然 か(は)いさう」
\ruby{無}{な}い
\ruby{事}{こと}は
ありますまい、
%
ねエ
\ruby{姊}{ねえ}さん。
』

\原本頁{91-11}%
\ruby[||j>]{{\換字{情}}}{じやう}
\ruby[||j>]{激}{ げき}
% \ruby{{\換字{情}}激}{じやう|げき}
してや
お
\ruby{龍}{りう}が
\ruby{面}{おもて}は
やゝ
\ruby{紅}{あか}くなり、
%
\ruby{其}{そ}の
\ruby{眼}{め}は
\ruby{濡}{ぬ}れ
\ruby{色}{いろ}を
\ruby{帶}{お}びて
\ruby{異}{あや}しく
\ruby{光}{ひかり}を
\ruby{增}{ま}せり。
