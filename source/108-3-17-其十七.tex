\Entry{其十七}

『そりやあもう
\ruby{屹度}{きつ|と}
お
\ruby{前}{まへ}の
\ruby{御云}{お|い}ひの
\ruby{通}{とほ}りだよ。
そのお
\ruby{五十}{い|そ}さんといふ
\ruby{人}{ひと}や
お
\ruby{前}{まへ}の
\ruby{御師匠}{お|し|よ}さんが、いつまでも〳〵
\ruby{然樣}{さ|う}いつた
\ruby{調子}{てう|し}で
\ruby{居}{ゐ}りやあ、それほど
\ruby{迄}{まで}に
\ruby{思}{おも}ひ
\ruby{込}{こ}んだ
\ruby{彼}{あ}の
\ruby{水野}{みづ|の}つていふ
\ruby{人}{ひと}の、
\ruby{落}{お}ちて
\ruby{行}{ゆ}く
\ruby{前{\換字{途}}}{さ|き}は
\ruby{知}{し}れて
\ruby{居}{ゐ}るよ。
\ruby{學問}{がく|もん}もあるといふ
\ruby{人}{ひと}の
\ruby{事}{こと}だから、まさかに
\ruby{無{\換字{分}}別沙汰}{む|ふん|べつ|ざ|た}も
\ruby{仕}{し}まいけれどもネエ、
\ruby{彼}{あ}の
\ruby{人}{ひと}が
\ruby{若}{もし}
\ruby{愚人}{ば|か}かなんかだと、それこそ
\ruby{怖}{おそろ}しい
\ruby{事}{こと}にもなり
\ruby{{\換字{兼}}}{かね}ない
\ruby{話}{はなし}たよ。
』

『
\ruby{然樣}{さ|う}ですとも、ほんとに!。
もし
\ruby{彼}{あ}の
\ruby{人}{ひと}が
\ruby{無茶}{む|ちや}な
\ruby{人}{ひと}だつた
\ruby{日}{ひ}にやあ、
\ruby{隨{\換字{分}}刄物}{ずい|ぶん|は|もの}でも
\ruby{持}{も}ち
\ruby{出}{だ}し
\ruby{{\換字{兼}}}{かね}ないとおもひますよ。
さうすりやあ
\ruby[g]{差詰}{さしず}め
\ruby{吾家}{う|ち}の
\ruby{御師匠}{お|し|よ}さんが
\ruby{目}{め}ざされる
\ruby{人}{ひと}ですネエ。
』

『あゝさうとも!。
お
\ruby{前}{まへ}の
\ruby{御師匠}{お|し|よ}さんといふ
\ruby{人}{ひと}は
\ruby{小}{けち}な
\ruby{惡}{わる}い
\ruby{人}{ひと}なんだけれど、
\ruby{仕方}{し|かた}が
\ruby{餘}{あんま}り
\ruby{罪}{つみ}な
\ruby{仕方}{し|かた}だからネ、
\ruby{隨{\換字{分}}鰺切}{ずゐ|ぶん|あぢ|きり}で
\ruby{突}{つゝつ}かれる
\ruby{位}{くらゐ}の
\ruby{事}{こと}は
\ruby{出來}{で|き}ても
\ruby{是非}{ぜ|ひ}が
\ruby{無}{な}いよ。
』

『ですが
\ruby{彼}{あ}の
\ruby{人}{ひと}が
\ruby{無茶}{む|ちや}な
\ruby{人}{ひと}で
\ruby{無}{な}いだけに、
\ruby{何樣間{\換字{違}}}{ど|う|ま|ちが}つたつて
\ruby{下}{くだ}らない
\ruby{事}{こと}なんかは
\ruby{仕}{し}や
\ruby{仕}{し}ますまい。
\ruby{百}{ひやく}のものならまあ
\ruby[g]{九十九}{くじうく}までは\換字{志}つと
\ruby{堪}{こら}へるだらうと
\ruby{思}{おも}ひますが、
\ruby{何處}{ど|こ}までも\換字{志}つと
\ruby{堪}{こら}へて
\ruby{獨}{ひと}りで
\ruby{苦}{くる}しんで、
\ruby{思}{おも}ひ
\ruby{死}{じに}に
\ruby{死}{し}んで
\ruby{仕舞}{し|ま}ふまでも
\ruby{穩}{おとな}しく
\ruby{仕}{し}て
\ruby{居}{ゐ}やうかと
\ruby{思}{おも}ふと、
\ruby{{\換字{分}}別}{ふん|べつ}や
\ruby{堪}{こら}へ
\ruby{{\換字{情}}}{じやう}が
\ruby{有}{あ}る
\ruby{人}{ひと}だけに
\ruby{{\換字{猶}}}{なほ}の
\ruby{事氣}{こと|き}の
\ruby{毒}{どく}で、ほんとに
\ruby{何}{なん}といふ
\ruby{愍然}{かは|いさう}な
\ruby{人}{ひと}だらうと
\ruby{思}{おも}はずには
\ruby{居}{ゐ}られません。

それでもまた
\ruby{彼}{あ}の
\ruby{人}{ひと}が
\ruby{困}{こま}らずにでも
\ruby{居}{ゐ}たら、
\ruby{同}{おな}じ
\ruby{胸}{むね}の
\ruby{苦}{くる}しい
\ruby{中}{なか}でも
\ruby{氣}{き}の
\ruby{樂}{らく}なところも
\ruby{有}{あ}りましやうが、
\ruby{職務}{や|く}は
\ruby{無}{な}し、
\ruby{身體}{から|だ}は
\ruby{閑}{ひま}なり、
\ruby{懐中合}{ふと|ころ|あひ}は
\ruby{惡}{わる}し、
\ruby{差當}{さし|あた}り
\ruby{段々困}{だん|〳〵|こま}つて
\ruby{來}{く}るといふところで、
\ruby{其}{そ}の
\ruby{困}{こま}るやうになつた
\ruby{原因}{も|と}の
お
\ruby{五十}{い|そ}さんは
\ruby{{\換字{情}}無}{つれ|な}いし、
お
\ruby{師匠}{し|よ}さんは
\ruby{薄{\換字{情}}}{はく|じやう}の
\ruby{地金}{ぢ|がね}を
\ruby{露}{だ}して、
\ruby{一昨日}{を|とゝ|ひ}
お
\ruby{出}{いで}といふやうは
\ruby{挨拶}{あい|さつ}を
\ruby{仕}{し}たなら、
\ruby{彼}{あ}の
\ruby{人}{ひと}の
\ruby{胸}{むね}の
\ruby{中}{うち}はまあ
\ruby{何樣}{ど|ん}なになるでしやう。
\ruby{火水}{ひ|みづ}が
\ruby{一諸}{いつ|しよ}になつたやうになつて、
\ruby{居}{ゐ}ても
\ruby{立}{た}つても
\ruby{居}{ゐ}られや
\ruby{仕}{し}ますまい。

ですから
\ruby{妾}{わたし}が
\ruby{吾家}{う|ち}の
\ruby{御師匠}{お|し|よ}さんの
\ruby{子}{こ}とか
\ruby{姪}{めひ}とか、
\ruby{何}{なに}か
\ruby{親眷}{しん|み}のものでゞも
\ruby{有}{あ}るのならば、よしんば
お
\ruby{師匠}{し|よ}さんと
\ruby{論爭}{いひ|あひ}を
\ruby{仕}{し}ても
お
\ruby{五十}{い|そ}さんを
\ruby{與}{や}るとか、
\ruby{恩{\換字{返}}}{おん|がへ}しをするとか、
\ruby{何}{ど}の
\ruby{{\換字{道}}}{みち}にせよ
\ruby{彼}{あ}の
\ruby{人}{ひと}の
\ruby{立}{た}つ
\ruby{瀬}{せ}のあるやうに、
\ruby{何樣}{ど|う}にか
\ruby{仕}{し}て
\ruby{{\換字{遣}}}{や}るのですが、
お
\ruby{師匠}{し|よ}さんと
\ruby{妾}{わたし}たあ
\ruby{他人同士}{た|にん|どう|し}、
\ruby{養女}{むす|め}になれ
\ruby{養女}{むす|め}にするつて
\ruby{此頃}{この|ごろ}ぢや
\ruby{大切}{だい|じ}にして
\ruby{優}{やさ}しくは
\ruby{仕}{し}て
\ruby{{\換字{呉}}}{く}れても、
\ruby{此方}{こつ|ち}あ
\ruby{食客}{かゝ|りうど}てす、
\ruby{論爭}{いひ|あ}ふまでにやあ
\ruby{何}{なに}も
\ruby{云}{い}へません、また
\ruby{論爭}{いひ|あ}つたつて
\ruby{無{\換字{益}}}{む|だ}なのは
\ruby{知}{し}れてます。
ですけれど
\ruby{御師匠}{お|し|よ}さんの
\ruby{代}{かはり}になつて
\ruby{行}{い}つて、
\ruby{彼}{あ}の
\ruby{人}{ひと}と
\ruby{知}{し}り
\ruby{合}{あひ}になつてからいろ〳〵のいきさつを
\ruby{聞}{き}いて
\ruby{一々知}{いち|〳〵|し}つて
\ruby{見}{み}ると、
\ruby{妾}{わたし}あ
\ruby{眞個}{ほん|と}に
\ruby{彼}{あ}の
\ruby{人}{ひと}が
\ruby{氣}{き}の
\ruby{毒}{どく}で〳〵、
お
\ruby{五十}{い|そ}さんていふ
\ruby{人}{ひと}が
\ruby{小憎}{こ|にく}らしい
\ruby{位}{くらゐ}に
\ruby{思}{おも}つて
\ruby{居}{ゐ}たところへ、これこれで
\ruby{職}{やく}も
\ruby{無}{な}くなつたといふ
\ruby{話}{はなし}を
\ruby{聞}{き}いて
\ruby{見}{み}るとハア
\ruby{然樣}{さ|う}ですかと
\ruby{云}{い}つた
\ruby{限}{き}りにやあ
\ruby{出來無}{で|き|な}いやうな
\ruby{氣}{き}もすれば、
\ruby{何}{なん}だか
\ruby{知}{し}らん
\ruby{顏}{かほ}で
\ruby{打棄}{うつ|ちや}つて
\ruby{置}{お}いちやあ
\ruby[g]{不人情}{ふにんじやう}のやうな
\ruby{氣}{き}もするんですよ。
で、
\ruby{姊}{ねえ}さんが
\ruby{口}{くち}さへきいて
\ruby{下}{くだ}さりやあ
\ruby[g]{必定譯}{きつとわけ}は
\ruby{無}{な}い
\ruby{事}{こと}、
\ruby{多勢}{おほ|ぜい}の
\ruby{人}{ひと}を
お
\ruby{使}{つか}ひなさる
\ruby{筑波}{つく|ば}さんところで
\ruby{人}{ひと}
\ruby{一人}{ひと|り}
\ruby{位}{ぐらひ}に
\ruby{授}{さづ}けて
\ruby{下}{くだ}さる
\ruby{職}{やく}の
\ruby{無}{な}い
\ruby{事}{こと}は
\ruby{有}{あ}るまいからと、
\ruby{然樣思}{さ|う|おも}つて、それで
\ruby{餘計}{よ|けい}なおせつかいか
\ruby{知}{し}りませんが
\ruby{御願}{お|ねが}ひに
\ruby{來}{き}たのです。

\ruby{一體}{いつ|たい}ならば
\ruby{吾家}{う|ち}の
\ruby{御師匠}{お|し|よ}さんが
\ruby{出來}{で|き}ないまでもかういふ
\ruby{苦勞}{く|らう}を
\ruby{仕}{し}て
\ruby{見}{み}なけりやあならない
\ruby{處}{ところ}なので、
\ruby{妾}{わたし}が
\ruby{爲}{す}るのは
\ruby{出{\換字{過}}}{で|す}ぎても
\ruby{居}{ゐ}ましやうが、
お
\ruby{師匠}{し|よ}さんは
お
\ruby{師匠}{し|よ}さんで
\ruby{澄}{す}まして
\ruby{{\換字{平}}氣}{へい|き}で
\ruby{居}{ゐ}ても、
\ruby{妾}{わたし}あ
\ruby{妾}{わたし}の
\ruby[g]{苦勞性}{くらうしやう}で
\ruby{安然}{じ|つ}としちやあ
\ruby{居}{ゐ}られなくつて、
\ruby{斯樣}{か|う}して
\ruby{出}{で}て
\ruby{來}{き}て
\ruby{姊}{ねえ}さんに
\ruby{縋}{すが}るのです。
まさか
\ruby{如是}{こ|れ}だけに
\ruby{細}{こまか}い
\ruby{理由}{わ|け}を
\ruby{御話}{お|はなし}
\ruby{仕}{し}たら、そりやあ
お
\ruby{前詰}{まへ|つま}らないよと
\ruby{云}{い}つても
\ruby{下}{くだ}さいますまいが、ネエ
\ruby{姊}{ねえ}さん、
\ruby{妾}{わたし}の
\ruby{慾得}{よく|とく}で
\ruby{御願}{お|ねが}ひをするのぢやあ
\ruby{無}{な}いし、
\ruby{姊}{ねえ}さんだつて
\ruby{彼}{あ}の
\ruby{人}{ひと}を
\ruby{愍然}{かは|いさう}ちや
\ruby{無}{な}いと
お
\ruby{思}{おも}ひなさるやうな
\ruby{事}{こと}は
\ruby{有}{あ}りやあ
\ruby{仕}{し}ますまいもの、
お
\ruby{願}{ねがひ}ですから
\ruby{妾}{わたし}の
\ruby[g]{所思}{おもひ}の
\ruby{無}{む}にならないやうに
\ruby{仕}{し}て
\ruby{下}{くだ}さいな、ねエ
\ruby{姊}{ねえ}さん。
』

\ruby{思}{おも}ひ
\ruby{入}{い}つて
\ruby{頼}{たの}み
\ruby{聞}{きこ}ゆる
お
\ruby{龍}{りう}を
\ruby{優}{やさ}しき
\ruby{眼}{め}して
\ruby{見居}{み|ゐ}たる
お
\ruby{彤}{とう}は、
\ruby{先刻}{さ|き}より
\ruby{今}{いま}に
\ruby{至}{いた}つて
\ruby{{\換字{猶}}未}{なほ|いま}だ
\ruby{鬢}{びん}の
\ruby{毛}{け}の
\ruby{一筋}{ひと|すじ}をだに
\ruby{動}{ゆる}がさず、
\ruby{端然}{たん|ねん}として
\ruby{坐}{すわ}りたるまゝなり。

