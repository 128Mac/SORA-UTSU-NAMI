\Entry{其十七}

% メモ 校正終了 2024-05-13 2024-06-09
\原本頁{92-3}%
『
そりやあ
もう
\ruby{屹度}{きつ|と}% ルビ調整(原本通り)非グループルビ
お
\ruby{{\換字{前}}}{まへ}の
\ruby{御云}{お|い}ひの
\ruby{{\換字{通}}}{とほ}りだよ。
%
その
お
\ruby{五十}{い|そ}さん
といふ
\ruby{人}{ひと}や
お
\ruby{{\換字{前}}}{まへ}の
\ruby{御師匠}{お|し|よ}さんが、
%
いつまでも〳〵
\ruby{然樣}{さ|う}いつた
\ruby{調子}{てう|し}で
\ruby{居}{ゐ}りやあ、
%
それほど
\ruby{迄}{まで}に
\ruby{思}{おも}ひ
\ruby{{\換字{込}}}{こ}んだ
\ruby{彼}{あ}の
\ruby{水野}{みづ|の}
つて
いふ
\ruby{人}{ひと}の
\改行% 校正作業の簡略化のため
、
%
\原本頁{92-6}\改行%
\ruby{落}{お}ちて
\ruby{行}{ゆ}く
\ruby{{\換字{前}}{\換字{途}}}{さ|き}は
\ruby{知}{し}れて
\ruby{居}{ゐ}るよ。
%
\ruby{學問}{がく|もん}
もある
といふ
\ruby{人}{ひと}の
\ruby{事}{こと}
だから、
%
まさかに
\ruby{無{\換字{分}}別}{む|ふん|べつ}
\ruby{沙汰}{ざ|た}も
\ruby{仕}{し}まい
けれどもネエ、
%
\ruby{彼}{あ}の
\ruby{人}{ひと}が
\ruby{{\換字{若}}}{もし}
\ruby{愚人}{ば|か}か
なんかだと、
%
それ
こそ
\ruby{怖}{おそろ}しい
\ruby{事}{こと}にもなり
\ruby{{\換字{兼}}}{かね}ない
\ruby{話}{はなし}たよ。
』

\原本頁{92-9}%
『
\ruby{然樣}{さ|う}
ですとも、
%
ほんとに!。
%
もし
\ruby{彼}{あ}の
\ruby{人}{ひと}が
\ruby{無茶}{む|ちや}な
\ruby{人}{ひと}だつた
\ruby{日}{ひ}
にやあ、
%
\ruby{隨{\換字{分}}}{ずゐ|ぶん}
\ruby{刄物}{は|もの}でも
\ruby{持}{も}ち
\ruby{出}{だ}し
\ruby{{\換字{兼}}}{かね}ない
と
おもひますよ。
%
さうすりやあ
\ruby{差詰}{さし|ず}め
\ruby{吾家}{う|ち}の
\ruby{御師匠}{お|し|よ}さんが
\ruby{目}{め}ざされる
\ruby{人}{ひと}ですネエ。
』

\原本頁{93-2}%
『
あゝ
さうとも!。
%
お
\ruby{{\換字{前}}}{まへ}の
\ruby{御師匠}{お|し|よ}さん
といふ
\ruby{人}{ひと}は
\ruby{小}{けち}な
\ruby{惡}{わる}い
\ruby{人}{ひと}
なんだ
けれど、
%
\ruby{仕方}{し|かた}が
\ruby{餘}{あんま}り
\ruby{罪}{つみ}な
\ruby{仕方}{し|かた}だからネ、
%
\ruby{隨{\換字{分}}}{ずゐ|ぶん}
\ruby{鰺切}{あぢ|きり}で% 「鰺」しか「切(れない)」包丁からか?
\ruby{突}{つゝつ}かれる
\ruby{位}{くらゐ}の
\ruby{事}{こと}は
\ruby{出來}{で|き}ても
\ruby{是非}{ぜ|ひ}が
\ruby{無}{な}いよ。
』

\原本頁{93-5}%
『
ですが
\ruby{彼}{あ}の
\ruby{人}{ひと}が
\ruby{無茶}{む|ちや}な
\ruby{人}{ひと}で
\ruby{無}{な}いだけに、
%
\ruby{何樣}{ど|う}
\ruby{間{\換字{違}}}{ま|ちが}つたつて
\ruby{下}{くだ}らない
\ruby{事}{こと}なんかは
\ruby{仕}{し}や
\ruby{仕}{し}ますまい。
%
\ruby{百}{ひやく}の
ものなら
まあ
\ruby{九十九}{く|じふ|く}までは
\換字{志゛}つと% 「志」+「濁点」
\ruby{堪}{こら}へる
だらうと
\ruby{思}{おも}ひますが、
%
\ruby{何處}{ど|こ}までも
\換字{志゛}つと% 「志」+「濁点」
\ruby{堪}{こら}へて
\ruby{獨}{ひと}りで
\ruby{苦}{くる}しんで、
%
\ruby{思}{おも}ひ
\ruby{死}{じに}に
\ruby{死}{し}んで
\ruby{仕舞}{し|ま}ふ
までも
\ruby{穩}{おとな}しく
\ruby{仕}{し}て
\ruby{居}{ゐ}やうかと
\ruby{思}{おも}ふと、
%
\ruby{{\換字{分}}別}{ふん|べつ}や
\ruby{堪}{こら}へ
\ruby{{\換字{情}}}{じやう}が
\ruby{有}{あ}る
\ruby{人}{ひと}
だけに
\ruby{{\換字{猶}}}{なほ}の
\ruby{事}{こと}
\ruby{氣}{き}の
\ruby{毒}{どく}で
\改行% 校正作業の簡略化のため
、
%
\原本頁{93-10}\改行%
ほんとに
\ruby{何}{なん}
といふ
\ruby[||j>]{愍}{かは}% 「愍然 か(は)いさう」
\ruby[||j>]{然}{いさう}な
% \ruby{愍然}{かは|いさう}な% 「愍然 か(は)いさう」
\ruby{人}{ひと}だらうと
\ruby{思}{おも}はずには
\ruby{居}{ゐ}られません。
\原本頁{93-11}\改行%
それでも
また
\ruby{彼}{あ}の
\ruby{人}{ひと}が
\ruby{困}{こま}らずに
でも
\ruby{居}{ゐ}たら、
%
\ruby{同}{おな}じ
\ruby{胸}{むね}の
\ruby{苦}{くる}しい
\ruby{中}{なか}
\原本頁{94-1}\改行%
でも
\ruby{氣}{き}の
\ruby{樂}{らく}な
ところも
\ruby{有}{あ}りましやうが、
%
\ruby{職務}{や|く}は
\ruby{無}{な}し、
%
\ruby[|g|]{身體}{からだ}は
\ruby{閑}{ひま}なり、
%
\ruby{懷中}{ふと|ころ}
\ruby{合}{あひ}は
\ruby{惡}{わる}し、
%
\ruby{差當}{さし|あた}り
\ruby[g]{段々}{だん〴〵}
\ruby{困}{こま}つて
\ruby{來}{く}る
といふ
ところで、
%
\原本頁{94-3}\改行%
\ruby{其}{そ}の
\ruby{困}{こま}る
やうに
なつた
\ruby{原因}{も|と}の
お
\ruby{五十}{い|そ}さんは
\ruby{{\換字{情}}}{つれ}
\ruby{無}{な}いし、
%
お
\ruby{師匠}{し|よ}さんは
\ruby[||j>]{薄}{はく}
\ruby[||j>]{{\換字{情}}}{じやう}の
% \ruby{薄{\換字{情}}}{はく|じやう}の
\ruby{地金}{ぢ|がね}を
\ruby{露}{だ}して、
%
\ruby[|g|]{一昨日}{をとゝひ}
お
\ruby{出}{いで}と
いふやうな
\ruby{挨拶}{あい|さつ}を
\ruby{仕}{し}たら、
%
\ruby{彼}{あ}の
\ruby{人}{ひと}の
\ruby{胸}{むね}の
\ruby{中}{うち}は
まあ
\ruby{何樣}{ど|ん}なに
なるで
しやう。
%
\ruby{火水}{ひ|みづ}が
\ruby{一諸}{いつ|しよ}に
なつた
やうに
なつて、
%
\ruby{居}{ゐ}ても
\ruby{立}{た}つても
\ruby{居}{ゐ}られや
\ruby{仕}{し}ますまい。
%
\原本頁{94-7}\改行%
ですから
\ruby{妾}{わたし}が
\ruby{吾家}{う|ち}の
\ruby{御師匠}{お|し|よ}さんの
\ruby{子}{こ}とか
\ruby{姪}{めひ}とか、
%
\ruby{何}{なに}か
\ruby{親眷}{しん|み}の
ものでゞも
\ruby{有}{あ}る
のならば、
%
よしんば
お
\ruby{師匠}{し|よ}さんと
\ruby{論爭}{いひ|あひ}を
\ruby{仕}{し}ても、
\原本頁{94-9}\改行%
お
\ruby{五十}{い|そ}さんを
\ruby{與}{や}るとか、
%
\ruby{恩{\換字{返}}}{おん|がへ}しを
するとか、
%
\ruby{何}{ど}の
\ruby{{\換字{道}}}{みち}にせよ
\ruby{彼}{あ}の
\原本頁{94-10}\改行%
\ruby{人}{ひと}の
\ruby{立}{た}つ
\ruby{瀬}{せ}の
ある
やうに、
%
\ruby{何樣}{ど|う}にか
\ruby{仕}{し}て
\ruby{{\換字{遣}}}{や}る
のですが、
%
お
\ruby{師匠}{し|よ}さんと
\ruby{妾}{わたし}たあ
\ruby{他人}{た|にん}
\ruby{同士}{どう|し}、
%
\ruby[|g|]{養女}{むすめ}に
なれ
\ruby[|g|]{養女}{むすめ}に
するつて
\ruby{此}{この}
\ruby{頃}{ごろ}ぢや
\ruby{大切}{だい|じ}にして
\ruby{優}{やさ}しくは
\ruby{仕}{し}て
\ruby{吳}{く}れても、
%
\ruby[|g|]{此方}{こつち}あ
\ruby[||j>]{食}{かゝ}
\ruby[||j>]{客}{りうど}てす
\footnote{原本の前後の「て」「で」と比較し原本通り「てす」とする
(国会図書館 コマ番号51/146 p-095 l-01)}%
、
% \ruby{食客}{かゝ|りうど}てす、
%
\ruby[|g|]{論爭}{いひあ}ふ
までにやあ
\ruby{何}{なに}も
\ruby{云}{い}へません、
%
また
\ruby[|g|]{論爭}{いひあ}つたつて
\ruby{無益}{む|だ}なのは
\ruby{知}{し}れてます。
%
です
けれど
\ruby{御師匠}{お|し|よ}さんの
\ruby{代}{かはり}に
なつて
\ruby{行}{い}つて、
%
\ruby{彼}{あ}の
\ruby{人}{ひと}と
\ruby{知}{し}り
\ruby{合}{あひ}に
なつてから
いろ〳〵の
いきさつを
\ruby{聞}{き}いて
\ruby[g]{一々}{いち〳〵}
\ruby{知}{し}つて
\ruby{見}{み}ると、
%
\ruby{妾}{わたし}あ
\ruby[|g|]{眞個}{ほんと}に
\ruby{彼}{あ}の
\ruby{人}{ひと}が
\ruby{氣}{き}の
\ruby{毒}{どく}で〳〵、
%
お
\ruby{五十}{い|そ}さん
ていふ
\ruby{人}{ひと}が
\原本頁{95-6}\改行%
\ruby{小憎}{こ|にく}らしい
\ruby{位}{くらゐ}に
\ruby{思}{おも}つて
\ruby{居}{ゐ}た
ところへ、
%
これこれで% ルビ調整(原本通り)非踊り字表記
\ruby{職}{やく}も
\ruby{無}{な}くなつた
といふ
\ruby{話}{はなし}を
\ruby{聞}{き}いて
\ruby{見}{み}ると、
ハア
\ruby{然樣}{さ|う}ですかと
\ruby{云}{い}つた
\ruby{限}{ぎ}りにやあ
\ruby{出來}{で|き}
\ruby{無}{な}いやうな
\ruby{氣}{き}も
すれば、
%
\ruby{何}{なん}だか
\ruby{知}{し}らん
\ruby{顏}{かほ}で
\ruby{打棄}{うつ|ちや}つて
\ruby{置}{お}いちやあ
\ruby{不人{\換字{情}}}{ふ|にん|じやう}
のやうな
\ruby{氣}{き}も
するんですよ。
%
で、
%
\ruby{姊}{ねえ}さんが
\ruby{口}{くち}さへ
きいて
\ruby{下}{くだ}さりやあ
\ruby[|g|]{必定}{きつと}
\ruby{譯}{わけ}は
\ruby{無}{な}い
\ruby{事}{こと}、
%
\ruby{多勢}{おほ|ぜい}の
\ruby{人}{ひと}を
お
\ruby{使}{つか}ひ
なさる
\ruby{筑波}{つく|ば}さん
ところで
\ruby{人}{ひと}
\ruby[|g|]{一人}{ひとり}
\ruby{位}{ぐらゐ}に
\ruby{授}{さづ}けて
\ruby{下}{くだ}さる
\ruby{職}{やく}の
\ruby{無}{な}い
\ruby{事}{こと}は
\ruby{有}{あ}るまい
\原本頁{96-1}\改行%
からと、
%
\ruby{然樣}{さ|う}
\ruby{思}{おも}つて、
%
それで
\ruby{餘計}{よ|けい}な
おせつかいか
\ruby{知}{し}りませんが
\原本頁{96-2}\改行%
\ruby{御願}{お|ねが}ひに
\ruby{來}{き}たのです。
%
\ruby{一體}{いつ|たい}
ならば
\ruby{吾家}{う|ち}の
\ruby{御師匠}{お|し|よ}さんが
\ruby{出來}{で|き}ない
までも
かういふ
\ruby{苦勞}{く|らう}を
\ruby{仕}{し}て
\ruby{見}{み}なけりやあ
ならない
\ruby{處}{ところ}なので、
%
\ruby[<j||]{妾}{わたし}
\原本頁{96-4}\改行%
が
\ruby{爲}{す}るのは
\ruby{出{\換字{過}}}{で|す}ぎても
\ruby{居}{ゐ}ましやうが、
%
お
\ruby{師匠}{し|よ}さんは
お
\ruby{師匠}{し|よ}さんで
\ruby{澄}{す}まして
\ruby{{\換字{平}}氣}{へい|き}で
\ruby{居}{ゐ}ても、
%
\ruby{妾}{わたし}あ
\ruby{妾}{わたし}の
\ruby{苦勞性}{く|らう|しやう}で
\ruby{安然}{じ|つ}と
しちやあ
\ruby{居}{ゐ}られ
なくつて、
%
\ruby{斯樣}{か|う}して
\ruby{出}{で}て
\ruby{來}{き}て
\ruby{姊}{ねえ}さんに
\ruby{縋}{すが}るのです。
%
まさか
\原本頁{96-7}\改行%
\ruby{如是}{こ|れ}だけに
\ruby{細}{こまか}い
\ruby{理由}{わ|け}を
\ruby{御話}{お|はなし}
\ruby{仕}{し}たら、
%
そりやあ
お
\ruby{{\換字{前}}}{まへ}
\ruby{詰}{つま}らないよと
\原本頁{96-8}\改行%
\ruby{云}{い}つても
\ruby{下}{くだ}さいますまいが、
%
ネエ
\ruby{姊}{ねえ}さん、
%
\ruby{妾}{わたし}の
\ruby{慾得}{よく|とく}で
\ruby{御願}{お|ねが}ひを
するのぢやあ
\ruby{無}{な}いし、
%
\ruby{姊}{ねえ}さん
だつて
\ruby{彼}{あ}の
\ruby{人}{ひと}を
\ruby[||j>]{愍}{かは}% 「愍然 か(は)いさう」
\ruby[||j>]{然}{いさう}ぢや
% \ruby{愍然}{かは|いさう}ぢや% 「愍然 か(は)いさう」
\ruby{無}{な}いと
お
\原本頁{96-10}\改行%
\ruby{思}{おも}ひ
なさる
やうな
\ruby{事}{こと}は
\ruby{有}{あ}りやあ
\ruby{仕}{し}ますまいもの、
%
お
\ruby{願}{ねがひ}ですから
\原本頁{96-11}\改行%
\ruby{妾}{わたし}の
\ruby[|g|]{{\換字{所}}思}{おもひ}の
\ruby{無}{む}に
ならない
やうに
\ruby{仕}{し}て
\ruby{下}{くだ}さいな、
%
ねエ
\ruby{姊}{ねえ}さん。
』

\原本頁{97-1}%
\ruby{思}{おも}ひ
\ruby{入}{い}つて
\ruby{頼}{たの}み
\ruby{聞}{きこ}ゆる
お
\ruby{龍}{りう}を
\ruby{優}{やさ}しき
\ruby{眼}{め}して
\ruby{見}{み}
\ruby{居}{ゐ}たる
お
\ruby{彤}{とう}は、
%
\ruby{先刻}{さ|き}より
\ruby{今}{いま}に
\ruby{至}{いた}つて
\ruby{{\換字{猶}}}{なほ}
\ruby{未}{いま}だ
\ruby{鬢}{びん}の
\ruby{毛}{け}の
\ruby{一筋}{ひと|すぢ}をだに
\ruby{動}{ゆる}がさず、
%
\ruby{端然}{たん|ねん}として
\ruby{坐}{すわ}りたる
まゝ
なり。
