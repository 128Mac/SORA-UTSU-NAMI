\Entry{其十八}

\原本頁{}
お
\ruby{彤}{とう}は
\ruby{其}{そ}の
\ruby{美}{うつく}しき
\ruby{手}{て}に
\ruby{手爐}{て|あぶり}の
\ruby{緣}{ふち}を
\ruby{撫}{な}づるとも
\ruby{無}{な}く
\ruby{撫}{な}でながら、
%
いと
\ruby{靜}{しづか}に
\ruby{口}{くち}を
\ruby{開}{ひら}きて、

\原本頁{}
『お
\ruby{{\換字{前}}}{まへ}の
\ruby{云}{い}ふ
\ruby{事}{こと}は、
%
ようく
\ruby{{\換字{分}}}{わか}つたよ、
%
だがネエ
お
\ruby{龍}{りう}ちやん!。
』

\原本頁{}
と
\ruby{親}{した}しげに
\ruby{呼}{よ}びかくれば
お
\ruby{龍}{りう}も、

\原本頁{}
『ハア。
』

\原本頁{}
と
\ruby{甘}{あま}ゆるが
\ruby{如}{ごと}く
\ruby{輕}{かろ}く
\ruby{答}{こた}へて
お
\ruby{彤}{とう}を
\ruby{見}{み}つ、
%
\ruby{我}{わ}が
\ruby{姊}{あね}の
\ruby{如}{ごと}くに
\ruby{頼}{たの}み
\ruby{思}{おも}へる
\ruby{人}{ひと}は
\ruby{何}{なに}と
\ruby{云}{い}ひ
\ruby{出}{い}づるならん、
%
\ruby{多{\換字{分}}}{た|ぶん}は
\ruby{我}{わ}が
\ruby{頼}{たの}みを
\ruby{聞}{き}いては
\ruby{吳}{く}るゝならんがと
\ruby{思}{おも}ひながらも、
%
だがネエと
\ruby{云}{い}へる
\ruby{發語}{いひ|だし}に、
%
\ruby{少}{すこ}し
\ruby{氣{\換字{遣}}氣味}{き|づかひ|ぎ|み}の、
%
\ruby{心配}{しん|ぱい}らしき
\ruby{眼}{め}して
\ruby{他}{ひと}の
\ruby{眼}{め}を
\ruby{見}{み}たり。

\原本頁{}
『
\ruby{成程}{なる|ほど}
お
\ruby{{\換字{前}}}{まへ}の
\ruby{御云}{お|いひ}の
\ruby{{\換字{通}}}{とほ}り
\ruby{水野}{みづ|の}つていふ
\ruby{人}{ひと}も
\ruby{愍然}{かは|いさう}だし、% 「愍然 か(は)いさう」
%
お
\ruby{{\換字{前}}}{まへ}の
\ruby{御師匠}{お|し|よ}さんていふ
\ruby{人}{ひと}の
\ruby{仕方}{し|かた}も
\ruby{惡}{わる}いがネエ、
%
お
\ruby{龍}{りう}ちやん、
%
お
\ruby{{\換字{前}}}{まへ}が
\ruby{何}{なに}も
\ruby{彼}{あ}の
お
\ruby{師匠}{し|よ}さんの
\ruby{眷屬}{み|うち}といふのぢやあ
\ruby{無}{な}いし、
%
\ruby{{\換字{又}}}{また}
\ruby{深}{ふか}しい
\ruby{關係}{ひつ|かゝり}のある
\ruby{免}{のが}れない
\ruby{仲}{なか}といふのぢやあ
\ruby{無}{な}いしさ、
%
お
\ruby{{\換字{前}}}{まへ}が
\ruby{彼}{あ}の
お
\ruby{師匠}{し|よ}さんのところから
\ruby{身}{み}さへ
\ruby{引}{ひ}いて
\ruby{{\換字{終}}}{しま}へば、
%
\ruby{其}{そ}の
\ruby{話}{はなし}あ
\ruby{全然}{まつ|たく}
お
\ruby{{\換字{前}}}{まへ}にやあ
\ruby{飛沫}{しぶ|き}も
\ruby{飛}{と}んで
\ruby{來}{こ}ない
\ruby{話}{はなし}になつて
\ruby{仕舞}{し|ま}つて、
%
たとへ
\ruby{何樣}{ど|ん}な
\ruby{喧嘩}{けん|くわ}が
\ruby{始}{はじ}まるにしても
\ruby{泥仕合}{どろ|じ|あひ}が
\ruby{始}{はじ}まるにしても、
%
\ruby{彼方}{むか|ふ}が
\ruby{彼方}{むか|ふ}だけで
\ruby{何樣}{ど|う}にでも
\ruby{{\換字{遣}}}{や}り
\ruby{合}{あ}つて
\ruby{居}{ゐ}やうつていふ
\ruby{譯}{わけ}ぢやあ
\ruby{無}{な}いか。
%
\ruby{彼方}{むか|ふ}
\ruby{同士}{どう|し}あ
\ruby{一團}{た|ま}になつてこんがらかつて
\ruby{居}{ゐ}る
\ruby{絲}{いと}だよ、
%
お
\ruby{{\換字{前}}}{まへ}は
\ruby{其}{その}
\ruby{一團}{た|ま}の
\ruby{中}{なか}に
\ruby{入}{はい}つては
\ruby{居}{ゐ}てもこんがらかつては
\ruby{居無}{ゐ|な}い{---}{---}
\ruby{引張}{ひつ|ぱ}ればするりと
\ruby{脫}{ぬ}けて
\ruby{仕舞}{し|ま}ふ
\ruby{事}{こと}の
\ruby{出來}{で|き}る
\ruby{絲}{いと}だよ。
%
だから
\ruby{早}{はや}い
\ruby{話}{はなし}を
\ruby{云}{い}やあ
\ruby{汝}{おまへ}が
\ruby{其}{そ}のこんがらかりの
\ruby{一團}{た|ま}の
\ruby{中}{なか}に
\ruby{入}{はい}つて、
%
\ruby{氣}{き}を
\ruby{使}{つか}つたり
\ruby{目}{め}を
\ruby{使}{つか}つたりしてまごついて
\ruby{居}{ゐ}るよりやあ、
%
するりと
\ruby{脫}{ぬ}けて
\ruby{仕舞}{し|ま}つた
\ruby{方}{はう}が
\ruby{何程}{いく|ら}
\ruby{好}{い}いか
\ruby{知}{し}れないよ。
%
\ruby{譯}{わけ}は
\ruby{無}{な}いやあネ、
%
\ruby{妾}{わたし}のところへ
\ruby{來}{き}て
お
\ruby{仕舞}{し|ま}ひな、
%
\ruby{以{\換字{前}}}{ま|へ}のやうに
\ruby{妾}{わたし}のところで
\ruby{氣}{き}を
\ruby{長閑}{のん|き}に
\ruby{仕}{し}て、
%
\ruby{小說}{せう|せつ}でも
\ruby{讀}{よ}んで
\ruby{{\換字{遊}}}{あそ}んでおいでが
\ruby{宜}{い}いぢやあ
\ruby{無}{な}いか。
%
\ruby{彼}{あ}の
お
\ruby{師匠}{し|よ}さんていふ
\ruby{人}{ひと}が
\ruby{何}{なに}かぶつ〳〵
\ruby{云}{い}つたにしても、
%
\ruby{金錢}{おか|ね}のぽつちりも
\ruby{與}{や}りやあ
\ruby{尾}{を}を
\ruby{振}{ふ}つちまふ
\ruby{人}{ひと}だらうから、
%
\ruby{何}{なに}もむづかしい
\ruby{事}{こと}は
\ruby{有}{あ}りやあ
\ruby{仕無}{し|な}いはネ。
%
お
\ruby{{\換字{前}}}{まへ}の
\ruby{爲}{ため}の
\ruby{好}{い}いやうになら
\ruby{何樣}{ど|ん}なにでも
\ruby{仕}{し}てあげるつもりなのだし、
%
お
\ruby{{\換字{前}}}{まへ}の
\ruby{身}{み}の
\ruby{上}{うへ}に
\ruby{就}{つ}いちやあ
\ruby{妾}{わたし}も
\ruby[<h||]{些}{ちよつと}
\ruby{考}{かんが}へてる
\ruby{事}{こと}もあるんだし、
%
\ruby{{\換字{又}}}{また}
\ruby{何處}{ど|こ}までも
\ruby{引受}{ひき|う}けて
\ruby{世話}{せ|わ}を
\ruby{仕度}{し|た}いといふ
\ruby{{\換字{道}}理}{わ|け}も
\ruby{有}{あ}るんだからネ。
%
\ruby{決}{けつ}して
\ruby{惡}{わる}い
\ruby{事}{こと}は
\ruby{云}{い}はないから
\ruby{脫}{ぬ}けて
\ruby{仕舞}{し|ま}つたら
\ruby{何樣}{ど|う}だエ。
%
\ruby{第一}{だい|いち}
お
\ruby{{\換字{前}}}{まへ}の
\ruby{話}{はなし}でも
\ruby{{\換字{分}}}{わか}つて
\ruby{居}{ゐ}る
お
\ruby{{\換字{前}}}{まへ}の
\ruby{御師匠}{お|し|よ}さんネ、
%
そんな
\ruby{可厭}{い|や}な
\ruby{人}{ひと}と
\ruby{一緖}{いつ|しよ}に
\ruby{居}{ゐ}て
\ruby{末々}{すゑ|〴〵}は
お
\ruby{{\換字{前}}}{まへ}
\ruby{何樣}{ど|う}
\ruby{仕}{し}やうつて
\ruby{氣}{き}なのだエ。
%
お
\ruby{{\換字{前}}}{まへ}
\ruby{程}{ほど}にも
\ruby{無}{な}い、
%
\ruby{{\換字{分}}}{わか}らないぢやあないか。
』

\原本頁{}
『そりやあもう
\ruby{段々}{だん|〴〵}と
\ruby{彼}{あ}の
\ruby{人}{ひと}の
\ruby{御腹}{お|なか}の
\ruby{中}{なか}が
\ruby{讀}{よ}めて
\ruby{來}{き}て
\ruby{見}{み}ると、
%
\ruby{到底}{とて|も}
\ruby{末長}{すゑ|なが}く
\ruby{一緖}{いつ|しよ}になんぞ
\ruby{居}{ゐ}られる
\ruby{人}{ひと}ぢやあ
\ruby{無}{な}いのですし、
%
\ruby{妾}{わたし}に
\ruby{仕}{し}た
\ruby{{\換字{前}}}{まへ}
\ruby{々}{〳〵}の
\ruby{{\換字{所}}行}{しう|ち}も
\ruby{此頃}{この|ごろ}になつて
\ruby{見}{み}りやあ、
%
\ruby{合點}{が|てん}の
\ruby{行}{ゆ}く
\ruby{恨}{うら}めしいことが
\ruby{澤山}{たん|と}あるのですもの。
%
ですから
\ruby{表面}{うは|べ}こそは
\ruby{奇麗}{き|れい}にして
\ruby{居}{ゐ}ますが、
%
\ruby{些}{ちつと}も
\ruby{一處}{いつ|しよ}に
\ruby{居}{ゐ}たい
\ruby{事}{こと}なんか
\ruby{有}{あ}りやあ
\ruby{仕}{し}ませんの!。
%
たゞ、
%
\ruby{今}{いま}
\ruby{直}{すぐ}に
\ruby{何樣}{ど|う}
\ruby{思}{おも}つたからつて
\ruby{思}{おも}つたやうにもならない
\ruby{身}{み}だもんですから……。
』

\原本頁{}
『それで
\ruby{彼家}{あす|こ}に
\ruby{居}{ゐ}ると
お
\ruby{云}{い}ひのかエ。
%
それ
\ruby{御覽}{ご|らん}、
%
\ruby{彼}{あ}の
\ruby{人}{ひと}は
\ruby{{\換字{前}}}{まへ}つから
\ruby{妾}{わたし}が
\ruby{推量}{すゐ|りやう}した
\ruby{{\換字{通}}}{とほ}りだつたらう、
%
\ruby{云}{い}はない
\ruby{事}{こつ}ちやあ
\ruby{無}{な}い。
%
だから
\ruby{今}{いま}
お
\ruby{{\換字{前}}}{まへ}をちやほや
\ruby{云}{い}つて
\ruby{家}{うち}に
\ruby{置}{お}いて
\ruby{居}{ゐ}る
\ruby{料簡}{れう|けん}だつて、
』

\原本頁{}
『つまり
\ruby{妾}{わたし}を
\ruby{猿{\換字{廻}}}{さる|まは}しの
\ruby{猿}{さる}にして、
%
\ruby{自{\換字{分}}}{じ|ぶん}が
\ruby{食}{た}べやうつていふ
\ruby{腹}{はら}なんですよ。
%
その
\ruby{位}{ぐらゐ}の
\ruby{事}{こと}は
\ruby{妾}{わたし}だつて、
%
\ruby{氣}{き}のつかない
\ruby{程人}{ほど|ひと}が
\ruby{好}{よ}くももうありませんからネ。
%
それを
\ruby{何時}{い|つ}までも
\ruby{小兒}{こ|ども}かと
\ruby{思}{おも}つて、
%
\ruby{馬鹿}{ば|か}にして
\ruby{居}{ゐ}る
\ruby{氣}{き}の
\ruby{御師匠}{お|し|よ}さんの
\ruby{仕方}{し|かた}にやあ
\ruby{腹}{はら}が
\ruby{立}{た}ちますは。
』

\原本頁{}
『ホヽホヽホヽ、
%
\ruby{澤山}{たん|と}
\ruby{苦勞}{く|らう}を
お
\ruby{仕}{し}だつたから、
%
\ruby{{\換字{前}}}{まへ}の
お
\ruby{龍}{りう}ちやんぢやあ
\ruby{無}{な}いものネエ。
%
だが
\ruby{然樣}{さ|う}
\ruby{知}{し}り
\ruby{切}{き}つて
\ruby{居}{ゐ}てそれであどけ
\ruby{無}{な}い
\ruby{風}{ふう}を
\ruby{仕}{し}ておいでのなんざあ、
%
お
\ruby{{\換字{前}}}{まへ}の
\ruby{方}{はう}が
お
\ruby{師匠}{し|よ}さんよりも
\ruby{人}{ひと}の
\ruby{惡}{わる}さが
\ruby{一枚}{いち|まい}
\ruby{上}{うへ}ぢやあ
\ruby{無}{な}いか
\ruby{知}{し}らん、
%
ホヽホヽホヽ。
』

\原本頁{}
『ホヽホヽホヽ、
%
だつて
\ruby{妾}{わたし}あ、
%
あんな
\ruby{眞}{しん}の
\ruby{惡}{わる}い
\ruby{憎}{にく}い
\ruby{人}{ひと}にだから
\ruby{然樣}{さ|う}して
\ruby{居}{ゐ}られるのですは。
%
\ruby{善}{い}いとおもふ
\ruby{人}{ひと}に
\ruby{對}{むか}つちやあ
\ruby{此}{これ}つばかりだつて
\ruby{作}{つく}り
\ruby{{\換字{飾}}}{かざ}りは
\ruby{仕}{し}やあ
\ruby{仕}{し}ませんよ。
』

\原本頁{}
『ホヽホヽホー。
%
いゝよ。
%
\ruby{誰}{たれ}も
お
\ruby{{\換字{前}}}{まへ}を
\ruby[g]{眞個}{ほんと}に
\ruby{惡}{わる}い
\ruby{人}{ひと}におなりだつて
\ruby{云}{い}ひや
\ruby{仕}{し}ないから。
%
で、
%
さういふわけなら
\ruby{{\換字{猶}}}{なほ}の
\ruby{事}{こと}ぢや
\ruby{無}{な}いか。
%
\ruby{一日}{いち|にち}も
\ruby{早}{はや}く
\ruby{其樣}{そ|ん}な
\ruby{人}{ひと}と
\ruby{一}{ひと}つ
\ruby{御釜}{お|かま}の
\ruby{御飯}{ご|はん}を
\ruby{食}{た}べあつて
\ruby{緣}{えん}を
\ruby{深}{ふか}くする
\ruby{樣}{やう}な
\ruby{事}{こと}を、
%
\ruby{仕無}{し|な}い
\ruby{樣}{やう}に
\ruby{仕}{し}た
\ruby{方}{はう}が
\ruby{宜}{よ}からうぢやあ
\ruby{無}{な}いか。
』

\原本頁{}
『そりやあ
\ruby{其}{そ}の
\ruby{譯}{わけ}はもう
\ruby{能}{よう}く
\ruby{{\換字{分}}}{わか}つてますが、
%
ぢやあ、
%
\ruby{姊}{ねえ}さんの
\ruby{心持}{こゝろ|もち}ぢやあ
\ruby{水野}{みづ|の}さんの
\ruby{事}{こと}は、
%
まあ
\ruby{一體}{いつ|たい}
\ruby{何樣}{ど|う}したら
\ruby{好}{い}いんだと
\ruby{御思}{お|おも}ひなんでしやう?。
%
\ruby{構}{かま}ふ
\ruby{事}{こと}は
\ruby{無}{な}い、
%
\ruby{何}{なに}も
\ruby{彼}{か}も
\ruby{抛}{はふ}つて
お
\ruby{仕舞}{し|ま}ひと
\ruby{御思}{お|おも}ひの?。
』

\原本頁{}
\ruby{此}{これ}は
\ruby{恨}{うら}むるに
\ruby{似}{に}て
\ruby{云}{い}へど
\ruby{彼}{かれ}は
\ruby{感}{かん}ぜざるがごとし。

\原本頁{}
『
\ruby{一體}{いつ|たい}
\ruby{水野}{みづ|の}つて
\ruby{人}{ひと}は
\ruby{彼}{あ}りやあ
お
\ruby{{\換字{前}}}{まへ}の
\ruby{何}{なん}に
\ruby{當}{あた}るのだエ?。
』

\原本頁{}
『…………』

\原本頁{}
『お
\ruby{{\換字{前}}}{まへ}あの
\ruby{人}{ひと}に
\ruby{其樣}{そ|ん}なに
\ruby{肩}{かた}を
\ruby{入}{い}れて
\ruby{何樣}{ど|う}
\ruby{仕}{し}やうつて
お
\ruby{思}{おも}ひのだエ?。
』

\原本頁{}
『…………』

\原本頁{}
『
\ruby{考}{かんが}へて
\ruby{御覽}{ご|らん}、
%
\ruby{餘}{あんま}り
\ruby{詰}{つま}らな
\ruby{{\換字{過}}}{す}ぎるぢやあ
\ruby{無}{な}いかエ。
』

\原本頁{}
『……だつて
\ruby{姊}{ねえ}さん。
』

\原本頁{}
『だつてぢやあ
\ruby{無}{な}いよ。
%
え、
%
お
\ruby{龍}{りう}ちやん、
%
\ruby{妾}{わたし}あ
\ruby{何}{なん}だか
\ruby{意地}{い|ぢ}の
\ruby{惡}{わる}い
\ruby{事}{こと}を
\ruby{云}{い}ふやうだがネ、
%
ようく
\ruby{考}{かんが}へてごらんな。
%
どうだエ、それ、
%
お
\ruby{龍}{りう}ちやん。
』

\原本頁{}
『……だつて
\ruby{姊}{ねえ}さん、
』

\原本頁{}
『いゝえ。
%
だつてぢやあ
\ruby{有}{あ}りまとんよ。
%
\ruby{能}{よう}く
\ruby{考}{かんが}へてごらん。
%
\ruby{詰}{つま}らない
\ruby{事}{こと}は
\ruby{{\換字{終}}局}{しま|ひ}まで
\ruby{行}{い}つても
\ruby{矢張}{やつ|ぱ}り
\ruby{詰}{つま}らないよ。
』

\原本頁{}
『だつて
\ruby{姊}{ねえ}さん……。
%
だつて
\ruby{姊}{ねえ}さん……。
%
でもそれぢやあ
\ruby{餘}{あんま}り
\ruby{怜悧}{り|こう}
\ruby{{\換字{過}}}{す}ぎて
\ruby{薄{\換字{情}}}{はく|じやう}ぢやあ
\ruby{無}{な}くつて?。
』
