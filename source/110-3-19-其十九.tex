\Entry{其十九}

% メモ 校正終了 2024-05-14 2024-06-10
\原本頁{104-9}%
お
\ruby{龍}{りう}は
\ruby{自己}{お|の}が
\ruby{身}{み}の
\ruby{凡}{すべ}て
お
\ruby{彤}{とう}に
\ruby{及}{およ}ばざるを
\ruby{知}{し}れるなり。
%
\ruby{第一}{だい|いち}
\ruby{今}{いま}の
\ruby{身}{み}の
\ruby{境{\換字{遇}}}{う|へ}は
\ruby{掛}{か}けても
\ruby{及}{およ}ばざるを
\ruby{知}{し}れるなり、
%
\ruby{有}{も}つて
\ruby{生}{うま}れたる
\ruby{容貌}{きり|やう}も
もとより
\ruby{及}{およ}ばざるを
\ruby{知}{し}れるなり、
%
\ruby{智慧}{ち|ゑ}は
\ruby{特}{こと}さらに
\ruby{及}{およ}ばざるを
\ruby{知}{し}れるなり、
%
\ruby{讀書筆札}{よ|み|か|き}も
\ruby{二年三年}{に|ねん|さん|ねん}
\ruby{苦}{くる}しみたり
とて
\ruby{及}{およ}ぶ
べきに
あらず、
%
\ruby{挿花}{は|な}
\ruby{茶湯}{ちや|のゆ}
は
いふまでも
\ruby{無}{な}く、
%
\ruby{我}{わ}が
\ruby{最}{もつと}も
\ruby{好}{す}ける
\ruby{絲竹}{いと|たけ}の
\ruby{{\換字{道}}}{みち}
% 《「絲/糸」は琴・三味線などの弦楽器、「竹」は笛・笙(しょう)などの管楽器》
% 1 和楽器の総称。しちく。「—の調べ」
% 2 音楽。音曲。「—の道」
%
\改行% 校正作業の簡略化のため
、
\原本頁{105-4}\改行%
\ruby{彼}{かれ}の% ルビ調整(判読困難)「彼」のルビ判読し難い
\ruby{最}{もつと}も
\ruby{悅}{よろこ}ばぬ
\ruby{縫針}{ぬひ|はり}の
\ruby{{\換字{道}}}{みち}に
\ruby{掛}{か}けてすら
\ruby{{\換字{猶}}}{なほ}
\ruby{且}{かつ}
\ruby{及}{およ}ばず、
%
\ruby{隨{\換字{分}}}{ずゐ|ぶん}
\ruby{人}{ひと}には
\ruby{負}{ま}くる
\ruby{{\換字{嫌}}}{ぎら}ひの、
%
\ruby{何事}{なに|ごと}を
\ruby{仕}{し}ても
\ruby{人後}{あ|と}には
\ruby{立}{た}つまじと
\ruby{思}{おも}ふ
\ruby{身}{み}ながら
\改行% 校正作業の簡略化のため
、
%
\原本頁{105-6}\改行%
\ruby{何事}{なに|ごと}を
\ruby{仕}{し}ても
お
\ruby{彤}{とう}には
\ruby{及}{およ}びかぬるを
\ruby{知}{し}りて、
%
\ruby{心}{こゝろ}の
\ruby{底}{そこ}の
\ruby{底}{そこ}より
\ruby{深}{ふか}く
\ruby{深}{ふか}く
\ruby{{\換字{尊}}}{たつと}び
\ruby{敬}{うやま}へるなり。
%
されど
\ruby{唯}{たゞ}
\ruby{一}{ひと}つ、
%
\ruby[||j>]{{\換字{情}}}{じやう}
\ruby[||j>]{合}{ あひ}の
% \ruby{{\換字{情}}合}{じやう|あひ}の
\ruby{深}{ふか}き
\ruby{淺}{あさ}き
といふ
\原本頁{105-8}\改行%
\ruby{事}{こと}のみに
\ruby{掛}{か}けては、
%
ひそかに
\ruby{姊}{あね}と
\ruby{頼}{たの}む
お
\ruby{彤}{とう}にも
\ruby{讓}{ゆづ}らざる
\ruby{心地}{こゝ|ち}して、
%
\ruby{我}{われ}は
\ruby{何}{なん}ぞの
\ruby{折}{をり}には
\ruby{慾}{よく}も
\ruby{得}{とく}も
\ruby{何}{なに}も
\ruby{彼}{か}も
\ruby{棄}{す}てゝ
\ruby{仕舞}{し|ま}ふ
\ruby{馬鹿}{ば|か}なれ
\原本頁{105-10}\改行%
\ruby{共}{ども}、
%
\ruby{彼}{か}の
\ruby{人}{ひと}は
\ruby[|g|]{怜悧}{りこう}
だけに
\ruby{同}{おな}じ
\ruby{其}{そ}の
\ruby{時}{とき}に
\ruby{然樣}{さ|う}は
\ruby{爲}{す}まじき
\ruby{人}{ひと}と、
%
\ruby{却}{かへ}つて
\ruby[|g|]{流石}{さすが}に
\ruby{崇}{あが}め
\ruby{慕}{した}へる
\ruby{其}{その}
\ruby{人}{ひと}をも、
%
\ruby{聊}{いさゝ}か
\ruby{物}{もの}
\ruby{足}{た}らず
\ruby{{\換字{飽}}}{あ}かず
\ruby{思}{おも}へる
\ruby{氣味}{き|み}
さへ
あるなり。

\原本頁{106-2}%
されば
\ruby{今}{いま}
お
\ruby{龍}{りう}が
\ruby{云}{い}ひ
\ruby{出}{い}でしは、
%
もとより
\ruby{{\換字{率}}然}{そつ|ぜん}の
\ruby{語}{ご}
なれども、
%
\ruby[<j||]{意}{こゝろ}を% 行末行頭の境界付近なので特例処置を施す
\ruby{用}{もち}ひざる
\ruby{其}{そ}の
\ruby[|g|]{僅少}{わづか}なる
\ruby{語}{ことば}の
\ruby{中}{うち}に、
%
お
\ruby{龍}{りう}は
おのづから
お
\ruby{龍}{りう}の
\ruby{氣}{き}% 行末行頭の境界付近なので特例処置を施す
\ruby{性}{しやう}の、
%
\ruby{然}{さ}ばかりに
\ruby{崇}{あが}め
\ruby{思}{おも}へる
お
\ruby{彤}{とう}
のためにも
\ruby{枉}{ま}げられず
\ruby{屈}{くつ}せられぬ
ものあるを
\ruby{露}{あらは}し
\ruby{出}{いだ}して、
%
\ruby{抑}{おさ}へんとして
\ruby{抑}{おさ}へかねたる
\ruby{不服}{ふ|ふく}の
\ruby{氣}{き}を
\ruby{我}{われ}
\ruby{知}{し}らず
\ruby{洩}{も}らせる
なり。

\原本頁{106-7}%
お
\ruby{龍}{りう}の
\ruby{持{\換字{前}}}{もち|まへ}を
\ruby{知}{し}りきつたる
お
\ruby{彤}{とう}は、
%
\ruby{走}{はし}り
\ruby{來}{きた}れる
\ruby{矢}{や}を
\ruby{幕}{まく}もて
\ruby{止}{とゞ}むる
\ruby{如}{ごと}く、
%
\ruby{柔軟}{やはら|か}なる
\ruby{語氣}{ご|き}に
\ruby{却}{かへ}つて
\ruby{問}{と}ひ
\ruby{反}{かへ}しぬ。

\原本頁{106-9}%
『
\ruby[||j>]{薄}{はく}
\ruby[||j>]{{\換字{情}}}{じやう}
% \ruby{薄{\換字{情}}}{はく|じやう}
ぢやあ
\ruby{無}{な}くつて
ツて。
%
\ruby{何故}{な|ぜ}
また
ネエ。
』

\原本頁{106-10}%
『
\ruby{何故}{な|ぜ}つて、
%
\ruby{姊}{ねえ}さん。
%
そりやあ
\ruby{妾}{わたし}さへ
\ruby{{\換字{退}}}{ひ}いて
\ruby{仕舞}{し|ま}へば
\ruby{妾}{わたし}の
\ruby{身}{み}の
\ruby{好}{い}いのは
\ruby{知}{し}れて
\ruby{居}{ゐ}ますが、
%
それぢやあ
\ruby{彼}{あ}の
\ruby{人}{ひと}は
\ruby[|g|]{否{\換字{運}}}{わるい}
まんまで
\ruby{{\換字{遺}}}{のこ}るので、
%
\ruby{矢張}{やつ|ぱ}り% ルビ調整(原本通り)非グループルビ
\ruby{彼}{あ}の
\ruby{人}{ひと}は
\ruby[||j>]{愍}{かは}% 「愍然 か(は)いさう」
\ruby[||j>]{然}{いさう}ちやあ
% \ruby{愍然}{かは|いさう}ちやあ% 「愍然 か(は)いさう」
\ruby{有}{あ}りませんか、
%
ですから
\ruby{其}{そ}れぢや
\ruby[||j>]{薄}{はく}
\ruby[||j>]{{\換字{情}}}{じやう}
% \ruby{薄{\換字{情}}}{はく|じやう}
になりますはネ。
%
\ruby{妾}{わたし}あ
\ruby{詰}{つま}る
\ruby{詰}{つま}らないは
\ruby{何樣}{ど|う}だつて
\ruby{好}{い}いんですよ。
%
\ruby{妾}{わたし}あ
たゞ
\ruby{彼}{あ}の
\ruby{人}{ひと}が
\ruby[||j>]{愍}{かは}% 「愍然 か(は)いさう」
\ruby[||j>]{然}{いさう}だから
% \ruby{愍然}{かは|いさう}だから% 「愍然 か(は)いさう」
\ruby{何樣}{ど|う}か
\ruby{仕}{し}て
\ruby{{\換字{遣}}}{や}りたいつて
\ruby{云}{い}ふんぢやあ
\ruby{有}{あ}りませんか。
』

\原本頁{107-5}%
『
いゝえ、
%
お
\ruby{{\換字{前}}}{まへ}の
\ruby[||j>]{心}{こゝろ}
\ruby[||j>]{持}{ もち}は
% \ruby{心持}{こゝろ|もち}は
もう
\ruby{悉皆}{すつ|かり}
\ruby{解}{わか}つて
\ruby{居}{ゐ}るのだがネ。
%
\ruby{妾}{わたし}あ
\ruby{{\換字{又}}}{また}
ただ% ルビ調整(原本通り)非踊り字表記
お
\ruby{{\換字{前}}}{まへ}の
\ruby{朋友}{とも|だち}で、
%
お
\ruby{{\換字{前}}}{まへ}の
\ruby{利益}{た|め}になる
\ruby{事}{こと}を
\ruby{仕}{し}てあげたいのだから。
%
{---}{---}%
いゝかエ。
%
だから
\ruby{妾}{わたし}あ
\ruby{{\換字{前}}{\換字{途}}}{さ|き}の
\ruby{{\換字{前}}{\換字{途}}}{さ|き}まで
\ruby{考}{かんが}へるので、
%
お
\ruby{{\換字{前}}}{まへ}の
\ruby{詰}{つま}る
\ruby{詰}{つま}らないを
\ruby{關}{かま}はない
なんて、
%
そんな
\ruby{事}{こと}は
\ruby{出來}{で|き}ないよ。
』

\原本頁{107-9}%
『
でも
\ruby{詰}{つま}る
\ruby{詰}{つま}らないで
\ruby{云}{い}やあ、
%
\ruby{何}{なん}だつて
\ruby{詰}{つま}らないは!。
%
\ruby{妾}{わたし}
みた
やうな
\ruby[|g|]{種々}{いろん}な
\ruby{目}{め}に
あつて
\ruby{來}{き}た
ものは
\ruby{活}{い}きて
\ruby{居}{ゐ}るのからして
\ruby{詰}{つま}らないは!。
%
\ruby{何樣}{ど|う}せ
\ruby{妾}{わたし}が
\ruby{彼}{あ}の
\ruby{人}{ひと}を
\ruby[||j>]{愍}{かは}% 「愍然 か(は)いさう」
\ruby[||j>]{然}{いさう}だから
% \ruby{愍然}{かは|いさう}だから% 「愍然 か(は)いさう」
\ruby{何樣}{ど|う}して
\ruby{{\換字{遣}}}{や}りたいと
\原本頁{108-1}\改行%
\ruby{思}{おも}つたつて、
%
\ruby[||j>]{結局}{つま|り}
\ruby[<j||]{妾}{わたし}にやあ% ルビ調整(原本通り)
\ruby{何}{なん}にもならない%
{---}{---}%
\ruby{詰}{つま}らないなあ
\ruby{知}{し}れてますは%
‥‥。
%
でも
\ruby{妾}{わたし}の
\ruby{氣}{き}が
\ruby{屆}{とゞ}けば% 「屆」「届」 原本通り「屆」
\ruby{妾}{わたし}の
\ruby[||j>]{心}{こゝろ}
\ruby[||j>]{持}{ もち}は
% \ruby{心持}{こゝろ|もち}は
\ruby{宜}{よ}うござんすは。
%
\ruby{知}{し}らん
\ruby{顏}{かほ}で
\ruby{濟}{す}ますなあ
\ruby[||j>]{薄}{はく}
\ruby[||j>]{{\換字{情}}}{じやう}
% \ruby{薄{\換字{情}}}{はく|じやう}
なやうな
\ruby{氣}{き}が
\ruby{爲}{し}ますは。
』

\原本頁{108-4}%
『
オヤ、
%
\ruby{妾}{わたし}あ
\ruby{爲}{し}なくちやあ
ならない
\ruby{事}{こと}を
\ruby{爲}{し}ないのが
\ruby[||j>]{薄}{はく}
\ruby[||j>]{{\換字{情}}}{じやう}
% \ruby{薄{\換字{情}}}{はく|じやう}
つて
いふ
ものかと
\ruby{思}{おも}つて
\ruby{居}{ゐ}たが、
%
お
\ruby{{\換字{前}}}{まへ}のは
\ruby{爲}{し}なくても
\ruby{濟}{す}むことを
\ruby{仕無}{し|な}いのも
\ruby[||j>]{薄}{はく}
\ruby[||j>]{{\換字{情}}}{じやう}
% \ruby{薄{\換字{情}}}{はく|じやう}
といふのだネ。
』

\原本頁{108-7}%
『
\ruby{爲}{し}なくちやあ
ならない
\ruby{事}{こと}を
\ruby{仕無}{し|な}いのは、
%
そりあ
\ruby{不義理}{ふ|ぎ|り}ですは。
%
\ruby{爲}{し}なくても
\ruby{濟}{す}むことでも、
%
\ruby{爲}{し}て
やりやあ
\ruby{他人}{ひ|と}の
\ruby{利益}{た|め}になる、
%
それを
\ruby{爲}{し}ないのが
\ruby{妾}{わたし}あ
\ruby[||j>]{薄}{はく}
\ruby[||j>]{{\換字{情}}}{じやう}
% \ruby{薄{\換字{情}}}{はく|じやう}
かと
\ruby{思}{おも}つて
\ruby{居}{ゐ}ますよ。
』

\原本頁{108-10}%
『
お
\ruby{龍}{りう}ちやんのやうに
\ruby{云}{い}つた
\ruby{日}{ひ}にやあ、
%
お
\ruby{龍}{りう}ちやんの
\ruby{他}{ほか}の
\ruby{出間}{せ|けん}
の
\ruby{人}{ひと}
は
\ruby{悉皆}{みん|な}% ルビ調整(原本通り)非グループルビ
\ruby[<j||]{薄}{はく }
\ruby[<j||]{{\換字{情}}}{じやう}
% \ruby{薄{\換字{情}}}{はく|じやう}
\ruby{者}{もの}
の
やうに
なつて
\ruby{仕舞}{し|ま}ふよ。
%
ホヽヽ、
%
まあ
\ruby{其}{そ}りやあ
\ruby{何樣}{ど|う}でも
\ruby{宜}{い}いが、
%
それぢやあ
\ruby{詰}{つま}つても
\ruby{詰}{つま}らなくつても
\ruby{水野}{みづ|の}
つて
いふ
\ruby{人}{ひと}は
\ruby{妾}{わたし}が
\ruby{引受}{ひき|う}けて
\ruby{何樣}{ど|う}か
\ruby{仕}{し}てあげる
とすると
\ruby{決}{き}めて
\ruby{置}{お}くがネ。
』
