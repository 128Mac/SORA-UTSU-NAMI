\Entry{其十九}

お
\ruby{龍}{りう}は
\ruby{自己}{お|の}が
\ruby{身}{み}の
\ruby{凡}{すべ}てお
\ruby{彤}{とう}に
\ruby{及}{およ}ばざるを
\ruby{知}{し}れるなり。
\ruby{第一今}{だい|いち|いま}の
\ruby{身}{み}の
\ruby{境{\換字{遇}}}{う|へ}は
\ruby{掛}{か}けても
\ruby{及}{およ}ばざるを
\ruby{知}{し}れるなり、
\ruby{有}{も}つて
\ruby{生}{うま}れたる
\ruby[g]{容貌}{きりやう}ももとより
\ruby{及}{およ}ばざるを
\ruby{知}{し}れるなり、
\ruby{智慧}{ち|ゑ}は
\ruby{特}{こと}さらに
\ruby{及}{およ}ばざるを
\ruby{知}{し}れるなり、
\ruby{讀書筆札}{よ|み|か|き}も
\ruby[g]{二年三年苦}{にねんさんねんくる}しみたりとて
\ruby{及}{およ}ぶべきにあらず、
\ruby[g]{挿花茶湯}{はなちやのゆ}はいふまでも
\ruby{無}{な}く、
\ruby{我}{わ}が
\ruby{最}{もつと}も
\ruby{好}{す}ける
\ruby{絲竹}{いと|たけ}の
\ruby{{\換字{道}}}{みち}、
\ruby{彼}{かれ}の
\ruby{最}{もつと}も
\ruby{{\換字{悅}}}{よろこ}ばぬ
\ruby{縫針}{ぬひ|はり}の
\ruby{{\換字{道}}}{みち}に
\ruby{掛}{か}けてすら
\ruby{{\換字{猶}}且及}{なほ|かつ|およ}ばず、
\ruby[g]{隨{\換字{分}}人}{ずひぶんひと}には
\ruby{負}{ま}くる
\ruby{{\換字{嫌}}}{ぎら}ひの、
\ruby{何事}{なに|ごと}を
\ruby{仕}{し}ても
\ruby{人後}{あ|と}には
\ruby{立}{た}つまじと
\ruby{思}{おも}ふ
\ruby{身}{み}ながら、
\ruby{何事}{なに|ごと}を
\ruby{仕}{し}てもお
\ruby{彤}{とう}には
\ruby{及}{およ}びかぬるを
\ruby{知}{し}りて、
\ruby{心}{こゝろ}の
\ruby{底}{そこ}の
\ruby{底}{そこ}より
\ruby{深}{ふか}く
\ruby{深}{ふか}く
\ruby{尊}{たつと}び
\ruby{敬}{うやま}へるなり。
されど
\ruby{唯一}{たゞ|ひと}つ、
\ruby{{\換字{情}}合}{じやう|あひ}の
\ruby{深}{ふか}き
\ruby{淺}{あさ}きといふ
\ruby{事}{こと}のみに
\ruby{掛}{か}けては、ひそかに
\ruby{姊}{あね}と
\ruby{頼}{たの}むお
\ruby{彤}{とう}にも
\ruby{讓}{ゆづ}らざる
\ruby[g]{心地}{こゝち}して、
\ruby{我}{われ}は
\ruby{何}{なん}ぞの
\ruby{折}{をり}には
\ruby{慾}{よく}も
\ruby{得}{とく}も
\ruby{何}{なに}も
\ruby{彼}{か}も
\ruby{棄}{す}てゝ
\ruby{仕舞}{し|ま}ふ
\ruby{馬鹿}{ば|か}なれ
\ruby{共}{ども}、
\ruby{彼}{か}の
\ruby{人}{ひと}は
\ruby[g]{恰悧}{りかう}だけに
\ruby{同}{おな}じ
\ruby{其}{そ}の
\ruby{時}{とき}に
\ruby{然樣}{さ|う}は
\ruby{爲}{す}まじき
\ruby{人}{ひと}と、
\ruby{却}{かへ}つて
\ruby{流石}{さす|が}に
\ruby{崇}{あが}め
\ruby{慕}{した}へる
\ruby{其人}{その|ひと}をも、
\ruby{聊}{いさゝ}か
\ruby{物足}{もの|た}らず
\ruby{{\換字{飽}}}{あ}かず
\ruby{思}{おも}へ
\ruby{氣味}{き|み}さへあるなり。

されば
\ruby{今}{いま}お
\ruby{龍}{りう}が
\ruby{云}{い}ひ
\ruby{出}{い}でしは、もとより
\ruby{率然}{そつ|ぜん}の
\ruby{語}{ご}なれども、
\ruby{意}{こゝろ}を
\ruby{用}{もち}ひざる
\ruby{其}{そ}の
\ruby[g]{僅少}{わづか}なる
\ruby{語}{ことば}の
\ruby{中}{うち}に、お
\ruby{龍}{りう}はおのづからお
\ruby{龍}{りう}の
\ruby{氣性}{きし|やう}の、
\ruby{然}{さ}ばかりに
\ruby{崇}{あが}め
\ruby{思}{おも}へるお
\ruby{彤}{とう}のためにも
\ruby{枉}{ま}げられず
\ruby{屈}{くつ}せられぬものあるを
\ruby{露}{あらは}し
\ruby{出}{いだ}して、
\ruby{抑}{おさ}へんとして
\ruby{抑}{おさ}へかねたる
\ruby{不服}{ふ|ふく}の
\ruby{氣}{き}を
\ruby{我知}{われ|し}らず
\ruby{洩}{も}らせるなり。

お
\ruby{龍}{りう}の
\ruby{持前}{もち|まへ}を
\ruby{知}{し}りきつたるお
\ruby{彤}{とう}は、
\ruby{走}{はし}り
\ruby{來}{きた}れる
\ruby{矢}{や}を
\ruby{幕}{まく}もて
\ruby{止}{とゞ}むる
\ruby{如}{ごと}く、
\ruby{柔軟}{やは|らか}なる
\ruby{語氣}{ご|き}に
\ruby{却}{かへ}つて
\ruby{問}{と}ひ
\ruby{反}{かへ}しぬ。

『
\ruby[g]{薄情}{はくじやう}ぢやあ
\ruby{無}{な}くつてツて。
\ruby{何故}{な|ぜ}またネエ。
』

『
\ruby{何故}{な|ぜ}つて、
\ruby{姊}{ねえ}さん。
そりやあ
\ruby{妾}{わたし}さへ
\ruby{退}{ひ}いて
\ruby{仕舞}{し|ま}へば
\ruby{妾}{わたし}の
\ruby{身}{み}の
\ruby{好}{い}いのは
\ruby{知}{し}れて
\ruby{居}{ゐ}ますが、それぢやあ
\ruby{彼}{あ}の
\ruby{人}{ひと}は
\ruby{否}{わるい}まんまで
\ruby{{\換字{遺}}}{のこ}るので、
\ruby[g]{矢張}{やつぱ}り
\ruby{彼}{あ}の
\ruby{人}{ひと}は
\ruby{愍然}{かはい|さう}ちやあ
\ruby{有}{あ}りませんか、ですから
\ruby{其}{そ}れぢや
\ruby[g]{薄{\換字{情}}}{はくじやう}になりますはネ。
\ruby{妾}{わたし}あ
\ruby{詰}{つま}る
\ruby{詰}{つま}らないは
\ruby{何樣}{ど|う}だつて
\ruby{好}{い}いんですよ。
\ruby{妾}{わたし}あたゞ
\ruby{彼}{あ}の
\ruby{人}{ひと}が
\ruby[g]{愍然}{かはいさう}だから
\ruby{何樣}{ど|う}か
\ruby{仕}{し}て
\ruby{{\換字{遣}}}{や}りたいつて
\ruby{云}{い}ふんぢやあ
\ruby{有}{あ}りませんか。
』

『いゝえ、お
\ruby{前}{まへ}の
\ruby[g]{心持}{こゝろもち}はもう
\ruby[g]{悉皆解}{すつかりわか}つて
\ruby{居}{ゐ}るのだがネ。
\ruby{妾}{わたし}あ
\ruby{{\換字{又}}}{また}たゞお
\ruby{前}{まへ}の
\ruby{朋友}{とも|だち}で、お
\ruby{前}{まへ}の
\ruby{利{\換字{益}}}{た|め}になる
\ruby{事}{こと}を
\ruby{仕}{し}てあげたいのだから。
 \------ いゝかエ。
だから
\ruby{妾}{わたし}あ
\ruby{前{\換字{途}}}{さ|き}の
\ruby{前{\換字{途}}}{さ|き}まで
\ruby{考}{かんが}へるので、お
\ruby{前}{まへ}の
\ruby{詰}{つま}る
\ruby{詰}{つま}らないを
\ruby{關}{かま}はないなんて、そんな
\ruby{事}{こと}は
\ruby{出來}{で|き}ないよ。
』

『でも
\ruby{詰}{つま}る
\ruby{詰}{つま}らないで
\ruby{云}{い}やあ、
\ruby{何}{なん}だつて
\ruby{詰}{つま}らないは!。
\ruby{妾}{わたし}みたやうな
\ruby[g]{種々}{いろん}な
\ruby{目}{め}にあつて
\ruby{來}{き}たものは
\ruby{活}{い}きて
\ruby{居}{ゐ}るのからして
\ruby{詰}{つま}らないは!。
\ruby{何樣}{ど|う}せ
\ruby{妾}{わたし}が
\ruby{彼}{あ}の
\ruby{人}{ひと}を
\ruby{愍然}{かはい|さう}だから
\ruby{何樣}{ど|う}して
\ruby{{\換字{遣}}}{や}りたいと
\ruby{思}{おも}つたつて、
\ruby[g]{結局妾}{つまりわたし}にやあ
\ruby{何}{なん}にもならない \------
\ruby{詰}{つま}らないなあ
\ruby{知}{し}れてますは……。
でも
\ruby{妾}{わたし}の
\ruby{氣}{き}が
\ruby{屆}{とゞ}けば
\ruby{妾}{わたし}の
\ruby{心持}{こゝろ|もち}は
\ruby{宜}{よ}うござんすは。
\ruby{知}{し}らん
\ruby{顏}{かほ}で
\ruby{濟}{す}ますなあ
\ruby[g]{薄{\換字{情}}}{はくじやう}なやうな
\ruby{氣}{き}が
\ruby{爲}{し}ますは。
』

『オヤ、
\ruby{妾}{わたし}あ
\ruby{爲}{し}なくちやあならない
\ruby{事}{こと}を
\ruby{爲}{し}ないのが
\ruby[g]{薄{\換字{情}}}{はくじやう}つていふものかと
\ruby{思}{おも}つて
\ruby{居}{ゐ}たが、お
\ruby{前}{まへ}のは
\ruby{爲}{し}なくても
\ruby{濟}{す}むことを
\ruby{仕無}{し|な}いのに
\ruby[g]{薄{\換字{情}}}{はくじやう}といふのだネ。
』

『
\ruby{爲}{し}なくちやあならない
\ruby{事}{こと}を
\ruby{仕無}{し|な}いのは、そりあ
\ruby{不義理}{ふ|ぎ|り}ですは、
\ruby{爲}{し}なくても
\ruby{濟}{す}むことでも、
\ruby{爲}{し}てやりやあ
\ruby{他人}{ひ|と}の
\ruby{利{\換字{益}}}{た|め}になる、それを
\ruby{爲}{し}ないのが
\ruby{妾}{わたし}あ
\ruby[g]{薄{\換字{情}}}{はくじやう}かと
\ruby{思}{おも}つて
\ruby{居}{ゐ}ますよ。
』

『お
\ruby{龍}{りう}ちやんのやうに
\ruby{云}{い}つた
\ruby{日}{ひ}にやあ、お
\ruby{龍}{りう}ちやんの
\ruby{他}{ほか}の
\ruby[g]{出間}{せけん}の
\ruby{人}{ひと}
は
\ruby[g]{悉皆}{みんな}
\ruby[g]{薄{\換字{情}}者}{はくじやうもの}のやうになつて
\ruby{仕舞}{し|ま}ふよ。
ホヽヽ、まあ
\ruby{其}{そ}りやあ
\ruby{何樣}{ど|う}でも
\ruby{宜}{い}いが、それぢやあ
\ruby{詰}{つま}つても
\ruby{詰}{つま}らなくつても
\ruby{水野}{みづ|の}つていふ
\ruby{人}{ひと}は
\ruby{妾}{わたし}が
\ruby{引受}{ひき|う}けて
\ruby{何樣}{ど|う}か
\ruby{仕}{し}てあげるとすると
\ruby{決}{き}めて
\ruby{置}{お}くがネ。
』

