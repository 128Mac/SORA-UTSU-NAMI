\Entry{其二十}

% メモ 校正終了 2024-05-14
\原本頁{109-5}%
『
ぢやあ
\ruby{姊}{ねえ}さん、
%
ほんとに
\ruby{受合}{うけ|あ}つて
\ruby{下}{くだ}さるの。
』

\原本頁{109-6}%
お
\ruby{龍}{りう}の
\ruby{眼}{め}は
\ruby{既}{すで}に
\ruby{罪}{つみ}も
\ruby{無}{な}く
\ruby{悅}{よろこ}びて
\ruby{笑}{ゑ}める
なり。
%
お
\ruby{彤}{とう}は
\ruby{其}{そ}の
\ruby{樣子}{やう|す}を
\ruby{見}{み}て
\ruby{却}{かへ}つて
\ruby{微}{かすか}に
\ruby{愁}{うれ}ふる
\ruby{色}{いろ}あり。

\原本頁{109-8}%
『
あゝ。
%
\ruby{彼}{あ}の
\ruby{人}{ひと}が
\ruby{困}{こま}らない
やうに
する
だけの
\ruby{事}{こと}
なんぞは、
\ruby{旦那}{だん|な}
\原本頁{109-9}\改行%
に
\ruby{云}{い}ふ
までも
\ruby{無}{な}い、
%
\ruby{妾}{わたし}が
\ruby{何樣}{ど|う}にでも
\ruby{必定}{きつ|と}
\ruby{爲}{し}てあげるがネ。
%
お
\ruby{龍}{りう}ちやんは
\ruby{{\換字{又}}}{また}
\ruby{何}{なん}だつて
\ruby{然樣}{さ|う}
\ruby{彼}{あ}の
\ruby{人}{ひと}の
\ruby{事}{こと}に
\ruby{肩}{かた}を
\ruby{御入}{お|い}れ
のだらう?。
』

\原本頁{110-1}%
『
だつて
\ruby{姊}{ねえ}さん
\ruby[||j>]{愍}{かは}% 「愍然 か(は)いさう」
\ruby[||j>]{然}{いさう}
% \ruby{愍然}{かは|いさう}% 「愍然 か(は)いさう」
なのですもの!。
』

\原本頁{110-2}%
『
たゞ
\ruby[||j>]{愍}{かは}% 「愍然 か(は)いさう」
\ruby[||j>]{然}{いさう}
% \ruby{愍然}{かは|いさう}% 「愍然 か(は)いさう」
だつて
いふ
ばかりで?。
』

\原本頁{110-3}%
『
ハア
\ruby{然樣}{さ|う}ですは。
』

\原本頁{110-4}%
『
\ruby{全}{まつた}く
たゞ?。
』

\原本頁{110-5}%
『
いやだ
\ruby{事}{こと}ネエ、
%
\ruby{何}{なん}だか
\ruby{異}{をか}アしく
\ruby{御聞}{お|き}き
なさるのネ。
』

\原本頁{110-6}%
\ruby{面}{おもて}は
\ruby{漸}{やうや}く
\ruby{不安}{ふ|あん}を
\ruby{現}{あらは}し、
%
\ruby{言}{ことば}は
\ruby{忙}{せは}しく
\ruby{其}{そ}の
\ruby{問}{とひ}を
\ruby{{\換字{遮}}}{さへぎ}り
\ruby{止}{とゞ}めんとしたり。
%
\原本頁{110-7}\改行%
お
\ruby{彤}{とう}は
\ruby{口}{くち}の
ほとりに
\ruby{見}{み}ゆるか
\ruby{見}{み}えざるかの
\ruby{笑}{ゑみ}を
\ruby{{\換字{浮}}}{うか}めて、
%
\ruby{{\換字{猶}}}{なほ}
\ruby{{\換字{追}}求}{つゐ|きう}して
\ruby{已}{や}まず。

\原本頁{110-9}%
『
もしや
お
\ruby{龍}{りう}ちやん、
%
お
\ruby{{\換字{前}}}{まへ}、
%
あの
\ruby{人}{ひと}が
\ruby{好}{すき}
になつたの
ぢやあ
\ruby{無}{な}くつて?。
』

\原本頁{110-11}%
『
エ。
』

\原本頁{111-1}%
『
ひよつと
したら
お
\ruby{{\換字{前}}}{まへ}、
%
\ruby{胸}{むね}の
\ruby{底}{そこ}ぢやあ
\ruby{彼}{あ}の
\ruby{人}{ひと}を
\ruby{思}{おも}つてる
のぢやあ
\ruby{無}{な}くつて?。
』

\原本頁{111-3}%
\ruby{眼}{め}の
\ruby{上}{うへ}に
\ruby{白刄}{はく|じん}を
\ruby{閃}{ひらめ}めかさるゝが
\ruby{如}{ごと}く、
%
\ruby{一語}{いち|ご}は
\ruby{一語}{いち|ご}より
\ruby{急}{きふ}に
\ruby{逼}{せま}り
\ruby{立}{た}てられて、
%
お
\ruby{龍}{りう}は
さつと
\ruby{面}{おもて}を
\ruby{紅}{あか}くし、

\原本頁{111-5}%
『
あら
\ruby{姊}{ねえ}さん、
%
\ruby{其樣}{そ|ん}な
\ruby{事}{こと}を
\ruby{云}{い}つちやあ
\ruby{妾}{わたし}あ
\ruby{{\換字{嫌}}}{いや}
ですよ。
%
\ruby{妾}{わたし}やあ
\ruby{基樣}{そ|ん}な
\ruby{氣}{き}なんぞを
\ruby{些}{ちつと}も
\ruby{有}{も}つて
\ruby{居}{ゐ}やあ
\ruby{仕}{し}ませんは。
』

\原本頁{111-7}%
と、
%
\ruby{明}{あき}らかには
\ruby{答}{こた}へたれど、
%
\ruby{驚}{おどろ}き
\ruby{慌}{あわ}て
\ruby{{\換字{狼}}狽}{うろ|た}へて
どぎまぎせる
\ruby{態}{さま}は
あり〳〵と
\ruby{見}{み}えたり。
%
お
\ruby{彤}{とう}は
\ruby{此度}{こ|たび}は
\ruby[|g|]{嫣然}{にこり}と
\ruby{笑}{ゑみ}を
つくつて、

\原本頁{111-9}%
『
\ruby{必然}{きつ|と}?。
』

\原本頁{111-10}%
と
\ruby{重}{かさ}ねて
\ruby{問}{と}へば、
%
お
\ruby{龍}{りう}は
\ruby{既}{はや}
\ruby{{\換字{浮}}}{う}き
\ruby{足}{あし}を
\ruby{踏}{ふみ}
\ruby{堪}{こた}へ
\ruby{身構}{み|がま}へを
\ruby{仕直}{し|なほ}して、

\原本頁{111-11}%
『
だつて、
%
\ruby{知}{し}れきつてる
\ruby{事}{こと}ぢやあ
\ruby{有}{あ}りませんか。
%
\ruby{彼}{あ}の
\ruby{人}{ひと}は
お
\ruby{五十}{い|そ}さん
ていふ
\ruby{人}{ひと}を
\ruby{思}{おも}ひに
\ruby{思}{おも}ひ
ぬいてるのですもの、
%
\ruby{横合}{よこ|あひ}から
\ruby{妾}{わたし}が
\ruby{思}{おも}つたつて
\ruby{何樣}{ど|う}なりましやう!。
%
いくら
\ruby{妾}{わたし}が
\ruby{馬鹿}{ば|か}だつて
\ruby[<j||]{醉}{すゐ }% 行末行頭の境界付近なので特例処置を施す
\ruby[<j||]{狂}{きやう}
% \ruby{醉狂}{すゐ|きやう}
\原本頁{112-3}\改行%
だつて、
%
\ruby{其}{そ}の
\ruby{位}{くらゐ}の
\ruby{事}{こと}は
\ruby{知}{し}つてますから、
\ruby{{\換字{空}}店}{あき|だな}へ
\ruby{郵便}{いう|びん}を
\ruby{抛}{はふ}り
\ruby{{\換字{込}}}{こ}む
やうな
\ruby{事}{こと}を
\ruby{何}{なん}で
\ruby{爲}{し}ます
ものかネ。
%
ホヽホヽホヽ。
』

\原本頁{112-5}%
と
\ruby[||j>]{戲}{じやう}
\ruby[||j>]{言}{ だん}まで
% \ruby{戲言}{じやう|だん}まで
\ruby{云}{い}つて
\ruby{自}{みづか}ら
\ruby{笑}{わら}つて
\ruby{何氣}{なに|げ}なき
\ruby{態}{さま}なり。

\原本頁{112-6}%
お
\ruby{彤}{とう}は
お
\ruby{龍}{りう}の
\ruby{言}{ことば}を
\ruby{信}{しん}じたりや
\ruby{信}{しん}ぜざりしや
\ruby{知}{し}らず、

\原本頁{112-7}%
『
\ruby{然樣}{さ|う}かエ。
%
そんなら
\ruby{何}{なに}も
\ruby{既}{もう}
\ruby{云}{い}ふことは
\ruby{無}{な}い
のだがネ。
%
\ruby{妾}{わたし}あ
\ruby{{\換字{又}}}{また}、
%
お
\ruby{{\換字{前}}}{まへ}が
\ruby{彼}{あ}の
\ruby{人}{ひと}を
\ruby{好}{す}いて
でも
\ruby{居}{ゐ}る
といふ
ことなら、
%
\ruby{次第}{し|だい}に
\ruby{依}{よ}つちやあ
お
\ruby{{\換字{前}}}{まへ}の
\ruby{爲}{ため}に
\ruby{一}{ひ}ト
\ruby{苦勞}{く|らう}して、
%
お
\ruby{{\換字{前}}}{まへ}の
\ruby{身}{み}の
\ruby{收}{をさ}まりの
\ruby{好}{い}いやうに
\ruby{仕}{し}て
あげやうか
とも、
\ruby{初手}{しよ|て}には
ふつと
\ruby{思}{おも}つたのだよ。
』

\原本頁{112-11}%
『
エ。
』

\原本頁{113-1}%
\ruby[|g|]{全然}{まるで}
おもひの
\ruby{外}{ほか}なりし
\ruby{言葉}{こと|ば}に
お
\ruby{龍}{りう}は
\ruby{復}{また}
\ruby{驚}{おどろ}かされつ、
%
\ruby{我}{われ}
\ruby{知}{し}らず
\ruby[<j||]{心}{こゝろ}を% 行末行頭の境界付近なので特例処置を施す
\ruby{動}{うご}かして
\ruby{答}{こたへ}さへ
\ruby{答}{こた}へ
\ruby{鈍}{にぶ}りしが、
%
お
\ruby{彤}{とう}は
\ruby{早}{はや}くも
その
\ruby{眼色}{め|いろ}を
\ruby{見}{み}て
\ruby{取}{と}りたり。

\原本頁{113-4}%
『
だが
\ruby{彼}{あ}の
\ruby{人}{ひと}は
\ruby{彼樣}{あ|あ}だし、% 原本通りのルビ
%
\ruby{何樣}{ど|ん}な
もの
だらうかと
\ruby{思}{おも}つて
\ruby{居}{ゐ}る
\ruby{中}{うち}、
%
また
\ruby{別}{べつ}に
\ruby[|g|]{一條}{ひとつ}の
\ruby{話}{はなし}が
\ruby{出}{で}て
\ruby{來}{き}たので、
%
お
\ruby{{\換字{前}}}{まへ}の
\ruby{爲}{ため}に
\ruby{彼}{あ}の
\ruby{人}{ひと}は
\ruby{棄}{す}てる
\ruby{者}{もの}に
\ruby{仕}{し}た
\ruby{方}{はう}が
\ruby{宜}{い}いと
\ruby{決}{き}めて
\ruby{居}{ゐ}た
ところ、
%
\ruby{丁度}{ちやう|ど}
お
\ruby{{\換字{前}}}{まへ}も
\ruby{左樣}{さ|う}いふ
\ruby{氣}{き}だと
\ruby{今}{いま}
\ruby{聞}{き}いて
\ruby{妾}{わたし}も
\ruby{安心}{あん|しん}
したよ。
%
さうで
\ruby{無}{な}けりやあ
\ruby{彼}{あ}の
\ruby{人}{ひと}を
\ruby{思}{おも}つたつて
\ruby{詰}{つま}らない
といふ
\ruby{事}{こと}を
\ruby{云}{い}はうかと
\ruby{思}{おも}つて
\ruby{居}{ゐ}た
ところだよ。
』

\原本頁{113-9}%
\ruby{其}{そ}の
\ruby{云}{い}ふ
ところは
\ruby{假設}{う|そ}にや
\ruby[|g|]{實際}{ほんと}にや、
%
お
\ruby{龍}{りう}は
たゞ
\ruby{我}{わ}が
\ruby{心}{こゝろ}の
\ruby{蜘蛛}{く|も}の
\ruby{圍}{す}に
% 「圍 570d」
% 音読み「イ」/ 訓読み「かこむ」「かこう」
% 意味 かこむ / 周りをとりまく / かこい / かこみ / まわり / 周辺
\ruby{搦}{から}められ
\ruby{行}{ゆ}きて、
%
\ruby[|g|]{抵抗}{あらが}はんに
\ruby[|g|]{抵抗}{あらが}ふべき
\ruby{力}{ちから}の
\ruby{入}{い}れ
どころも
\ruby{知}{し}らぬ
\ruby{中}{うち}、
%
\ruby{次第}{し|だい}
\ruby[g]{々々}{ 〳〵 }に
\ruby{自由}{じ|いう}を% ルビは原本通り「じ(い)う」
\ruby{奪}{うば}はれ
\ruby{奪}{うば}はるゝが
\ruby{如}{ごと}く
\ruby{覺}{おぼ}ゆる
のみ、

\原本頁{114-2}%
『
ネエ
お
\ruby{龍}{りう}ちやん、
%
\ruby{仕樣}{し|やう}が
\ruby{無}{な}いやネ、
%
あゝ
いふ
\ruby{人}{ひと}は。
%
お
\ruby{{\換字{前}}}{まへ}
\ruby{彼}{あ}の
\原本頁{114-3}\改行%
\ruby{人}{ひと}を
\ruby{何樣}{ど|う}いふ
\ruby{人}{ひと}だと
お
\ruby{思}{おも}ひだエ?。
%
なる
\ruby{程}{ほど}
\ruby{{\換字{情}}}{じやう}も
\ruby{有}{あ}らう、
%
\ruby[||j>]{正}{しやう}
\ruby[||j>]{直}{ ぢき}
% \ruby{正直}{しやう|ぢき}
でも
あらう、
%
\ruby{學藝}{わ|ざ}も
\ruby{出來}{で|き}やうがネ、
%
\ruby[||j>]{一}{いつ}
\ruby[||j>]{生}{しやう}の
% \ruby{一生}{いつ|しやう}の
\ruby{{\換字{所}}天}{てい|しゆ}に
するにやあ、
%
\ruby{氣}{き}むづかしやで、
%
\ruby{{\換字{貧}}乏}{びん|ばう}
\ruby{性}{しやう}らしくつて、
%
ヘチ
\ruby{頑固}{ぐわん|こ}な
ところが
\ruby{有}{あ}つて、
%
\ruby{彼}{あれ}あ
\ruby{餘}{あんま}り
\ruby{有}{あ}り
\ruby{{\換字{難}}}{がた}くは
\ruby{無}{な}さゝうだネ。
%
といつて
\ruby{{\換字{情}}夫}{い|ろ}に
するにやあ、
%
\原本頁{114-7}\改行%
\ruby[|g|]{容貌}{をとこ}が
\ruby{惡}{わる}かあ
\ruby{無}{な}いが
\ruby{愛嬌}{あい|けう}の
\ruby{足}{た}りない、
%
\ruby{面白味}{おも|しろ|み}の
\ruby{薄}{うす}い、
%
\ruby[|g|]{無粹}{ぶいき}の、
%
\原本頁{114-8}\改行%
\ruby{世間}{せ|けん}を
\ruby{知}{し}らな
\ruby{{\換字{過}}}{す}ぎる%
{---}{---}%
\ruby{何樣}{ど|う}も
お
\ruby{{\換字{前}}}{まへ}の
\ruby{相手}{あひ|て}にやあ
\ruby[<j||]{些}{ちつと}
\ruby{不足}{ふ|そく}な
\ruby{男}{をとこ}ぢやあ
\ruby{無}{な}いか!。
』
