\Entry{其二十}

『ぢやあ
\ruby{姊}{ねえ}さん、ほんとに
\ruby{受合}{うけ|あ}つて
\ruby{下}{くだ}さるの。
』

お
\ruby{龍}{りう}の
\ruby{眼}{め}は
\ruby{既}{すで}に
\ruby{罪}{つみ}も
\ruby{無}{な}く
\ruby{{\換字{悅}}}{よろこ}びて
\ruby{笑}{ゑ}めるなり。
お
\ruby{彤}{とう}は
\ruby{其}{そ}の
\ruby{樣子}{やう|す}を
\ruby{見}{み}て
\ruby{却}{かへ}つて
\ruby{微}{かすか}に
\ruby{愁}{うれ}ふる
\ruby{色}{いろ}あり。

『あゝ、
\ruby{彼}{あ}の
\ruby{人}{ひと}が
\ruby{困}{こま}らないやうにするだけの
\ruby{事}{こと}なんぞは
\ruby{旦郡}{だん|な}に
\ruby{云}{い}ふまでも
\ruby{無}{な}い、
\ruby{妾}{わたし}が
\ruby{何樣}{ど|う}にでも
\ruby[g]{必定爲}{きつとし}てあげるがネ。
お
\ruby{龍}{りう}ちやんは
\ruby{{\換字{又}}}{また}
\ruby{何}{なん}だつて
\ruby{然樣彼}{さ|う|あ}の
\ruby{人}{ひと}の
\ruby{事}{こと}に
\ruby{肩}{かた}を
\ruby{御入}{お|い}れのだらう?。
』

『だつて
\ruby{姊}{ねえ}さん
\ruby{愍然}{かは|いさう}なのですもの!。
』

『たゞ
\ruby{愍然}{かは|いさう}だつていふばかりで?。
』

『ハア
\ruby{然樣}{さ|う}ですは。
』

『
\ruby{全}{まつた}くたゞ?。
』

『いやだ
\ruby{事}{こと}ネエ、
\ruby{何}{なん}だか
\ruby{異}{をか}アしく
\ruby{御聞}{お|き}きなさるのネ。
』

\ruby{面}{おもて}は
\ruby{漸}{やうや}く
\ruby{不安}{ふ|あん}を
\ruby{現}{あらは}し、
\ruby{言}{ことば}は
\ruby{忙}{せは}しく
\ruby{其}{そ}の
\ruby{問}{とひ}を
\ruby{遮}{さへぎ}り
\ruby{止}{とど}めんとしたり。
お
\ruby{彤}{とう}は
\ruby{口}{くち}のほとりに
\ruby{見}{み}ゆるか
\ruby{見}{み}えざるかの
\ruby{笑}{ゑみ}を
\ruby{{\換字{浮}}}{うか}めて、
\ruby{{\換字{猶}}}{なほ}
\ruby{{\換字{追}}求}{つゐ|きう}して
\ruby{已}{や}まず。

『もしやお
\ruby{龍}{りう}ちやん、
お
\ruby{前}{まへ}、あの
\ruby{人}{ひと}が
\ruby{好}{すき}になつたのぢやあ
\ruby{無}{な}くつて?。
』

『エ。
』

『ひよつとしたらお
\ruby{前}{まへ}、
\ruby{胸}{むね}の
\ruby{底}{そこ}ぢやあ
\ruby{彼}{あ}の
\ruby{人}{ひと}を
\ruby{思}{おも}つてるのぢやあ
\ruby{無}{な}くつて?。
』

\ruby{眼}{め}の
\ruby{上}{うへ}に
\ruby{白刄}{はく|じん}を
\ruby{閃}{ひらめ}めかさるゝが
\ruby{如}{ごと}く、
\ruby{一語}{いち|ご}は
\ruby{一語}{いち|ご}より
\ruby{急}{きふ}に
\ruby{逼}{せま}り
\ruby{立}{た}てられて、
お
\ruby{龍}{りう}はさつと
\ruby{面}{おもて}を
\ruby{紅}{あか}くし、

『あら
\ruby{姊}{ねえ}さん、
\ruby{其樣}{そ|ん}な
\ruby{事}{こと}を
\ruby{云}{い}つちやあ
\ruby{妾}{わたし}あ
\ruby{{\換字{嫌}}}{いや}ですよ。
\ruby{妾}{わたし}やあ
\ruby{基樣}{そ|ん}な
\ruby{氣}{き}なんぞを
\ruby{些}{ちつと}も
\ruby{有}{も}つて
\ruby{居}{ゐ}やあ
\ruby{仕}{し}ませんは。
』

と、
\ruby{明}{あき}らかには
\ruby{答}{こた}へたれど、
\ruby{驚}{おどろ}き
\ruby{慌}{あわ}て
\ruby{狼狽}{うろ|た}へてどぎまぎせる
\ruby{態}{さま}はあり〳〵と
\ruby{見}{み}えたり。
お
\ruby{彤}{とう}は
\ruby{此度}{こ|たび}は
\ruby{婿然}{にこ|り}と
\ruby{笑}{ゑみ}をつくつて、

『
\ruby{必然}{きつ|と}?。
』

と
\ruby{重}{かさ}ねて
\ruby{問}{と}へば、
お
\ruby{龍}{りう}は
\ruby{既{\換字{浮}}}{はや|う}き
\ruby{足}{あし}を
\ruby{踏堪}{ふみ|こた}へ
\ruby{身構}{み|がま}へを
\ruby{仕直}{し|なほ}して、

『だつて、
\ruby{知}{し}れきつてる
\ruby{事}{こと}ぢやあ
\ruby{有}{あ}りませんか。
\ruby{彼}{あ}の
\ruby{人}{ひと}は
お
\ruby{五十}{い|そ}さんていふ
\ruby{人}{ひと}を
\ruby{思}{おも}ひに
\ruby{思}{おも}ひぬいてるのですもの、
\ruby{横合}{よこ|あひ}から
\ruby{妾}{わたし}が
\ruby{思}{おも}つたつて
\ruby{何樣}{ど|う}なりましやう!。
いくら
\ruby{妾}{わたし}が
\ruby{馬鹿}{ば|か}だつて
\ruby{醉狂}{すゐ|きやう}だつて、
\ruby{其}{そ}の
\ruby{位}{くらゐ}の
\ruby{事}{こと}は
\ruby{知}{し}つてますから
\ruby{空店}{あき|だな}へ
\ruby{郵便}{いう|びん}を
\ruby{抛}{はふ}り
\ruby{込}{こ}むやうな
\ruby{事}{こと}を
\ruby{何}{なん}で
\ruby{爲}{し}ますものかネ。
ホヽホヽホヽ。
』

と
\ruby{戯言}{じやう|だん}まで
\ruby{云}{い}つて
\ruby{自}{みづか}ら
\ruby{笑}{わら}つて
\ruby{何氣}{なに|げ}なき
\ruby{態}{さま}なり。

お
\ruby{彤}{とう}は
お
\ruby{龍}{りう}の
\ruby{言}{ことば}を
\ruby{信}{しん}じたりや
\ruby{信}{しん}ぜざりしや
\ruby{知}{し}らず、

『
\ruby{然樣}{さ|う}かエ。
そんなら
\ruby{何}{なに}も
\ruby{既云}{もう|い}ふことは
\ruby{無}{な}いのだがネ。
\ruby{妾}{わたし}あ
\ruby{{\換字{又}}}{また}、
お
\ruby{前}{まへ}が
\ruby{彼}{あ}の
\ruby{人}{ひと}を
\ruby{好}{す}いてでも
\ruby{居}{ゐ}るといふことなら、
\ruby{次第}{し|だい}に
\ruby{依}{よ}つちやあ
お
\ruby{前}{まへ}の
\ruby{爲}{ため}に
\ruby{一}{ひ}ㇳ
\ruby{苦勞}{く|らう}して、
お
\ruby{前}{まへ}の
\ruby{身}{み}の
\ruby{收}{をさ}まりの
\ruby{好}{い}いやうに
\ruby{仕}{し}てあげやうかとも
\ruby{初手}{しよ|て}にはふつと
\ruby{思}{おも}つたのだよ。
』

『エ。
』

\ruby{全然}{まる|で}おもひの
\ruby{外}{ほか}なりし
\ruby{言葉}{こと|ば}に
お
\ruby{龍}{りう}は
\ruby{復驚}{また|おどろ}かされつ、
\ruby{我知}{われ|し}らず
\ruby{心}{こゝろ}を
\ruby{動}{うご}かして
\ruby{答}{こたへ}さへ
\ruby{答}{こた}へ
\ruby{鈍}{にぶ}りしが、
お
\ruby{彤}{とう}は
\ruby{早}{はや}くもその
\ruby{眼色}{め|いろ}を
\ruby{見}{み}て
\ruby{取}{と}りたり。

『だが
\ruby{彼}{あ}の
\ruby{人}{ひと}は
\ruby{彼樣}{あ|れ}だし、
\ruby{何樣}{ど|ん}なものだらうかと
\ruby{思}{おも}つて
\ruby{居}{ゐ}る
\ruby{中}{うち}、また
\ruby{別}{べつ}に
\ruby{一條}{ひと|つ}の
\ruby{話}{はなし}が
\ruby{出}{で}て
\ruby{來}{き}たので、
お
\ruby{前}{まへ}の
\ruby{爲}{ため}に
\ruby{彼}{あ}の
\ruby{人}{ひと}は
\ruby{棄}{す}てる
\ruby{者}{もの}に
\ruby{仕}{し}た
\ruby{方}{はう}が
\ruby{宜}{い}いと
\ruby{決}{き}めて
\ruby{居}{ゐ}たところ、
\ruby{丁度}{ちやう|ど}
お
\ruby{前}{まへ}も
\ruby{左樣}{さ|う}いふ
\ruby{氣}{き}だと
\ruby{今}{いま}
\ruby{聞}{き}いて
\ruby{妾}{わたし}も
\ruby{安心}{あん|しん}したよ。
さうで
\ruby{無}{な}けりやあ
\ruby{彼}{あ}の
\ruby{人}{ひと}を
\ruby{思}{おも}つたつて
\ruby{詰}{つま}らないといふ
\ruby{事}{こと}を
\ruby{云}{い}はうかと
\ruby{思}{おも}つて
\ruby{居}{ゐ}たところたよ。
』

\ruby{其}{そ}の
\ruby{云}{い}ふところは
\ruby{假設}{う|そ}にや
\ruby{實際}{ほん|と}にや、
お
\ruby{龍}{りゆ}はたゞ
\ruby{我}{わ}が
\ruby{心}{こゝろ}の
\ruby{蜘蛛}{く|も}の
\ruby{圖}{づ}に
\ruby{搦}{から}められ
\ruby{行}{ゆ}きて、
\ruby{抵抗}{あら|が}はんに
\ruby{抵抗}{あら|が}ふべき
\ruby{力}{ちから}の
\ruby{入}{い}れどころも
\ruby{知}{し}らぬ
\ruby{中}{うち}、
\ruby{次第}{し|だい}
\ruby[g]{々々}{〳〵}に
\ruby{自由}{じ|いう}を
\ruby{奪}{うば}はれ
\ruby{奪}{うば}はるゝが
\ruby{如}{ごと}く
\ruby{覺}{おぼ}ゆるのみ、

『ネエお
\ruby{龍}{りう}ちやん、
\ruby{仕樣}{し|やう}が
\ruby{無}{な}いやネ、あゝいふ
\ruby{人}{ひと}は。
お
\ruby{前彼}{まへ|あ}の
\ruby{人}{ひと}を
\ruby{何樣}{ど|う}いふ
\ruby{人}{ひと}だと
お
\ruby{思}{おも}ひだエ?。
なる
\ruby{程{\換字{情}}}{ほど|じやう}も
\ruby{有}{あ}らう、
\ruby{正直}{しやう|ぢき}でもあらう、
\ruby{學藝}{わ|ざ}も
\ruby{出來}{で|き}やうがネ、
\ruby{一生}{いつ|しやう}の
\ruby{{\換字{所}}天}{てい|しゆ}にするにやあ、
\ruby{氣}{き}むづかしやで、
\ruby{貧乏性}{びん|ばう|しやう}らしくつて、ヘチ
\ruby{頑固}{ぐわん|こ}なところが
\ruby{有}{あ}つて、
\ruby{彼}{あれ}あ
\ruby{餘}{あんま}り
\ruby{有}{あ}り
\ruby{難}{がた}くは
\ruby{無}{な}さゝうだネ。
といつて
\ruby{{\換字{情}}夫}{い|ろ}にするにやあ、
\ruby{容貌}{をと|こ}が
\ruby{惡}{わる}かあ
\ruby{無}{な}いが
\ruby{愛嬌}{あい|けう}の
\ruby{足}{た}りない、
\ruby{面白味}{おも|しろ|み}の
\ruby{薄}{うす}い、
\ruby{無粹}{ぶ|いき}の、
\ruby{世間}{せ|けん}を
\ruby{知}{し}らな
\ruby{{\換字{過}}}{す}ぎる \------
\ruby{何樣}{ど|う}も
お
\ruby{前}{まへ}の
\ruby{相手}{あひ|て}にやあ
\ruby[g]{些不足}{ちつとふそく}な
\ruby{男}{をとこ}ぢやあ
\ruby{無}{な}いか!。
』

