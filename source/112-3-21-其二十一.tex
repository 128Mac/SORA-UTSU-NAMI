\Entry{其二十一}

% メモ 校正終了 2024-05-14 2024-06-10
\原本頁{115-1}%
お
\ruby{龍}{りう}は
お
\ruby{彤}{とう}の
\ruby{水野}{みづ|の}を
\ruby{{\換字{評}}}{ひやう}するに
\ruby{{\換字{平}}}{たひ}らか
ならねども、
%
\ruby{反駁}{いひ|かへ}さんも
\ruby{何}{なん}と
\ruby{無}{な}く
\ruby[||j>]{後}{うしろ}
\ruby[||j>]{見}{ み}らるゝ
\ruby{心地}{こゝ|ち}せしが、
%
\ruby{其}{そ}の
\ruby{言}{い}ふ
ところ
\ruby{多}{おほ}くは
\ruby{當}{あた}れるを
\ruby[<g>]{如何}{いかん}とも
\ruby{爲}{す}る
\ruby{能}{あた}はず、
%
たゞ
\ruby{僅}{わづか}に、

\原本頁{115-4}%
『
あら
\ruby{姊}{ねえ}さん。
%
てんで
\ruby{妾}{わたし}あ
\ruby{全然}{まる|きり}
\ruby{其樣}{そ|ん}な
\ruby{事}{こと}を
\ruby{思}{おも}つてや
\ruby{仕無}{し|な}い
のですから、
%
\ruby{彼}{あ}の
\ruby{人}{ひと}が
\ruby{{\換字{貧}}乏}{びん|ばう}
\ruby{性}{しやう}だつて
\ruby[|g|]{無粹}{ぶいき}だつて
\ruby{何樣}{ど|う}だつて
\ruby{宜}{い}いぢや
\原本頁{115-6}\改行%
\ruby{有}{あ}りませんか、
%
\ruby{不足}{ふ|そく}でも
\ruby{{\換字{過}}}{す}ぎて
\ruby{居}{ゐ}ても
\ruby[||j>]{關}{かゝり}
\ruby[||j>]{係}{ あひ}の
% \ruby{關係}{かゝり|あひ}の
\ruby{無}{な}い
\ruby{事}{こと}ですは。
%
\ruby{隨{\換字{分}}}{ずゐ|ぶん}
\ruby{酷}{ひど}い
\ruby{事}{こと}ネエ、
%
\ruby{姊}{ねえ}さんの
\ruby{言}{くち}も。
』

\原本頁{115-8}%
と、
%
\ruby{知}{し}らざるを
\ruby{粧}{よそほ}ひて
\ruby{我}{われ}には
\ruby{聞}{き}き
\ruby{辛}{づら}き
\ruby{談}{はなし}を
\ruby{少}{すこ}しも
\ruby{早}{はや}く
\ruby{外}{はづ}さんと
\ruby{仕}{し}たり。

\原本頁{115-10}%
『
\ruby{然樣}{さ|う}さネエ。
%
ホヽヽ
\ruby[||j>]{關}{かゝり}
\ruby[||j>]{係}{ あひ}の
% \ruby{關係}{かゝり|あひ}の
\ruby{無}{な}いものを
\ruby{兎}{と}や
\ruby{角}{かく}
いふのには
\ruby{當}{あた}らない
のだがネ、
%
\ruby{此}{これ}あ
まあ
\ruby{無意}{た|ゞ}の
\ruby{話}{はなし}だと
\ruby{思}{おも}つて
\ruby{聞}{き}いて
\ruby{居}{ゐ}て
\ruby{御覽}{ご|らん}よ
\改行% 校正作業の簡略化のため
。
%
\原本頁{116-1}\改行%
お
\ruby{{\換字{前}}}{まへ}は
どうせ
\ruby{彼}{あ}の
\ruby{人}{ひと}を
\ruby{何樣}{だ|う}の
\ruby{彼樣}{か|う}の
となんぞ
\ruby{思}{おも}つては
\ruby{御}{お}いでゞ
\ruby{無}{な}い
といふの
だから、
%
\ruby{別}{べつ}に
\ruby{何}{なん}にも
\ruby{心配}{しん|ぱい}は
\ruby{無}{な}いがネ。
%
こゝに
\ruby{氣}{き}が
\原本頁{116-3}\改行%
\ruby{優}{やさ}しくつて
\ruby{而}{そ}して
\ruby{俠氣}{をとこ|ぎ}
のある
やうな
\ruby{{\換字{若}}}{わか}い
\ruby{女}{ひと}があつて、
%
\ruby{何樣}{ど|う}かした
\ruby{心}{こゝろ}の
\ruby[|g|]{機勢}{はずみ}から
\ruby{彼}{あ}の
\ruby{人}{ひと}を
\ruby{思}{おも}ふ
やうなことが
\ruby{有}{あ}る
とするとネ、
%
\ruby{早}{はや}く
\ruby{氣}{き}がついて
\ruby{引{\換字{返}}}{ひつ|かへ}して
\ruby{仕舞}{し|ま}へば
\ruby[||j>]{其}{それつ}
\ruby[||j>]{限}{ きり}で
% \ruby{其限}{それつ|きり}で
\ruby{濟}{す}むけれども、
%
\ruby[|g|]{田舎}{ゐなか}
\ruby{{\換字{道}}}{みち}
なんか
\ruby{歩}{ある}いても
\ruby{能}{よ}くある
\ruby{事}{こと}で、
%
\ruby{二十丁}{に|じふ|ちやう}
\ruby{三十}{さん|じふ}
\ruby[||j>]{丁}{ちやう}も
\ruby{間{\換字{違}}}{ま|ちが}つた
\ruby{路}{みち}へ
\ruby{踏{\換字{込}}}{ふみ|こ}んで
\ruby{仕舞}{し|ま}ふと、
%
あゝ
\ruby{間{\換字{違}}}{ま|ちが}つたと
\ruby{氣}{き}が
\ruby{付}{つ}いても
\ruby{後}{あと}へ
\ruby{{\換字{返}}}{かへ}る
\ruby{氣}{き}
には
なれないで、
%
\ruby{何樣}{ど|う}かして
\ruby{出拔}{で|ぬ}けやう
\ruby{出拔}{で|ぬ}けやうつて
\ruby{云}{い}ふんで
\ruby{餘計}{よ|けい}
\原本頁{116-9}\改行%
\ruby{變}{へん}な
\ruby{路}{みち}へ
\ruby{入}{はい}つて、
%
\ruby{下}{くだ}らない
\ruby{苦}{くるし}み
を
することが
\ruby{得}{え}て
\ruby{有}{あ}る
ものだが
\改行% 校正作業の簡略化のため
、
\原本頁{116-10}\改行%
\ruby{丁度}{ちやう|ど}
\ruby{其樣}{そ|ん}な
\ruby{譯}{わけ}で
\ruby{下手}{へ|た}に
\ruby{人}{ひと}を
\ruby{思}{おも}つて、
%
\ruby{少}{すこ}し
\ruby{宛}{づつ}
\ruby{少}{すこ}し
\ruby{宛}{づつ}
\ruby{深}{ふか}みへ
\ruby{入}{はい}つて
\原本頁{116-11}\改行%
\ruby{行}{ゆ}くと、
%
\ruby{{\換字{終}}}{しまひ}にやあ
\ruby{飛}{と}んだ
\ruby{目}{め}を
\ruby{見}{み}
\ruby{無}{な}けりやあ
ならない
やうな、
\ruby{馬鹿}{ば|か}な
ところへ
\ruby{行}{い}つて
\ruby{突}{つき}
\ruby{當}{あた}り
もするよ。
%
\ruby{何}{なん}でも
\ruby{{\換字{前}}{\換字{途}}}{さ|き}の
\ruby{知}{し}れない
\ruby{怪}{あや}しい
\ruby{路}{みち}へ
\ruby{入}{はい}つたら、
%
\ruby{一二}{いち|に}
\ruby{丁}{ちやう}
しか
\ruby{歩}{ある}かない
\ruby{中}{うち}に
\ruby{立}{たち}
\ruby{止}{どま}つてネ、
\換字{志゛}つと% 「志」+「濁点」
\ruby{考}{かんが}へるか
\ruby{人}{ひと}に
\ruby{聞}{き}くかして、
%
\ruby{引{\換字{返}}}{ひつ|かへ}すのが
まあ
\ruby{肝心}{かん|じん}で、
%
\ruby{無暗}{む|やみ}に
\ruby{歩}{ある}いて
\ruby{行}{ゆ}くのは
\ruby{一番}{いち|ばん}
\ruby{危}{あぶな}い
\ruby{事}{こと}だよ。
%
\ruby{彼}{あ}の
\ruby{水野}{みづ|の}
つて
いふ
\ruby{人}{ひと}は
\ruby{一}{ひ}ト
\ruby{目見}{め|み}ても
\ruby{{\換字{分}}}{わか}る、
%
\ruby{性}{しやう}は
\ruby{良}{い}い、
%
\ruby{眞人間}{ま|にん|げん}だよ、
%
\ruby{不實}{ふ|じつ}な
\ruby{人}{ひと}ぢや
\ruby{無}{な}い。
%
だから
\原本頁{117-6}\改行%
\ruby{彼}{あ}の
\ruby{人}{ひと}が
\ruby{別}{べつ}に
\ruby{人}{ひと}を
\ruby{思}{おも}つてるので
\ruby{無}{な}けりやあ、
%
\ruby{彼}{あ}の
\ruby{人}{ひと}を
\ruby{好}{す}いた
といふ
\ruby{女}{ひと}が
\ruby{有}{あ}りやあ
\ruby{其}{そ}りやあ
\ruby{好}{す}いたで
\ruby{宜}{い}いのさ。
%
\ruby{而}{そ}して
\ruby{其}{そ}の
\ruby{女}{ひと}の
\ruby{思}{おもひ}も
\ruby{屹度}{きつ|と}% ルビ調整(原本通り)非グループルビ
\ruby{彼}{あ}の
\ruby{人}{ひと}に
\ruby{{\換字{分}}}{わか}つて、
%
\ruby{小說}{せう|せつ}
ならば
まあ
\ruby{芽出度}{め|で|たし}
\ruby{芽出度}{め|で|たし}
といふ
ところにも
なるだらうがネ。
%
\ruby{彼}{あ}の
\ruby{人}{ひと}が
\ruby{他}{ほか}の
\ruby{人}{ひと}を
\ruby{一心}{いつ|しん}に
\ruby{思}{おも}つてる
からにやあ、
%
\ruby{性}{しやう}の
\ruby{良}{い}い
\ruby{人}{ひと}だけに
\ruby{傍}{わき}からの
\ruby{思}{おも}ひは
\ruby{受}{う}け
\ruby{付}{つ}けまい、
%
\原本頁{117-11}\改行%
\ruby{眞人間}{ま|にん|げん}だけに
\ruby[||j>]{二}{ふた}
\ruby[||j>]{心}{ごゝろ}は
% \ruby{二心}{ふた|ごゝろ}は
\ruby{持}{も}つまいよ。
%
\ruby{然樣}{さ|う}すりやあ
\ruby{彼}{あ}の
\ruby{人}{ひと}を
\ruby{思}{おも}ふなあ
\ruby[||j>]{死}{つき}
\ruby[||j>]{路}{あたり}へ
% \ruby{死路}{つき|あたり}へ
\ruby{向}{むか}つて
\ruby{行}{い}く
やうな
もので、
%
\ruby{行}{い}けば
\ruby{行}{い}くだけの
\ruby{草臥儲}{くた|びれ|まう}け
だから、
%
そんな
\ruby{路}{みち}へ
\ruby{{\換字{若}}}{も}し
\ruby{一寸}{ちよ|つと}でも
\ruby{歩}{あし}が
\ruby{向}{む}いて
\ruby{居}{ゐ}たらば、
%
\ruby[|g|]{其方}{そつち}へ
\原本頁{118-3}\改行%
\ruby{踏{\換字{込}}}{ふみ|こ}んだか
\ruby{踏}{ふ}み
\ruby{{\換字{込}}}{こ}まない
\ruby{中}{うち}に
\ruby{後}{あと}へ
\ruby{引{\換字{返}}}{ひつ|かへ}して
\ruby{仕舞}{し|ま}ふと、
%
\ruby{然程}{さ|ほど}
\ruby{苦}{く}にも
ならない、
%
\ruby{損}{そん}も
\ruby{仕}{し}
\ruby{無}{な}いで
\ruby{濟}{す}む
といふ
\ruby{譯}{わけ}
なのだよ。
%
\ruby{誰}{たれ}しも
\ruby{損}{そん}
\ruby{路}{みち}を
\ruby{仕}{し}ないで
\ruby{世}{よ}の
\ruby{中}{なか}を
\ruby{歩}{ある}いて
\ruby{來}{く}るものは
\ruby[g]{中々}{なか〳〵}
\ruby{無}{な}い。
%
お
\ruby{{\換字{前}}}{まへ}は
お
\ruby{知}{し}り
でないが
\ruby{妾}{わたし}
だつて
\ruby{損}{そん}
\ruby{{\換字{道}}}{みち}を
\ruby[|g|]{澤山}{たんと}
\ruby{仕}{し}て
\ruby{來}{き}て
\ruby{居}{ゐ}る。
%
お
\ruby{{\換字{前}}}{まへ}は
\ruby{妾}{わたし}も
\ruby{知}{し}つてるが
\ruby{既}{もう}
\ruby{一度}{いち|ど}
\ruby{甚}{ひど}い
\ruby{冗}{むだ}
\ruby{{\換字{道}}}{みち}を
\ruby{歩}{ある}いて、
%
\ruby{踏拔}{ふみ|ぬき}も
\ruby{仕}{し}て
おいでだし
\ruby{生爪}{なま|づめ}も
\ruby{剝}{は}がして
おいでだし、
%
\ruby[g]{散々}{さん〴〵}な
\ruby{目}{め}に
お
\ruby{會}{あ}ひだつた
\ruby{人}{ひと}だから、
%
\ruby{今}{いま}さら
\原本頁{118-9}\改行%
また
\ruby{{\換字{前}}{\換字{途}}}{さ|き}の
\ruby{知}{し}れない
\ruby{怪}{あや}しい
\ruby{路}{みち}へ
なんぞ、
%
\ruby{無暗}{む|やみ}には
\ruby{入}{はい}つて
\ruby{御}{お}いででは
\ruby{有}{あ}るまいから
\ruby{宜}{い}いがネ。
』

\原本頁{118-11}%
お
\ruby{彤}{とう}は
\ruby{云}{い}ひ
\ruby{{\換字{終}}}{をは}つて
\ruby{默}{もく}し、
%
お
\ruby{龍}{りう}は
\ruby{聞}{き}き
\ruby{{\換字{終}}}{をは}つて
\ruby{默}{もく}し、
%
\ruby{互}{たがひ}に
\ruby{言葉}{こと|ば}の
\ruby{{\換字{絕}}}{た}えたる
ところへ、
%
\ruby{小間}{こ|ま}
\ruby[<j||]{使}{づかひ}の
お
\ruby{春}{はる}は
\ruby{次室}{つぎ|のま}より
\ruby{現}{あら}はれ、

\原本頁{119-2}%
『
あの
\ruby[|g|]{昨日}{きのふ}
お
\ruby{來臨}{い|で}
なすつた
お
\ruby{婆}{ばあ}さんの
\ruby{方}{かた}が
\ruby{御出}{お|いで}
になりました。
』

\原本頁{119-3}%
と
\ruby{云}{い}へば、

\原本頁{119-4}%
『
おゝ
\ruby{丁度}{ちやう|ど}
\ruby{好}{い}い
ところへ
だつた、
%
\ruby[|g|]{此方}{こちら}へと
\ruby{御云}{お|い}ひ。
%
お
\ruby{龍}{りう}ちやん、
%
お
\ruby{{\換字{前}}}{まへ}、
%
\ruby{吃驚}{びつ|くり}
おしで
\ruby{無}{な}いよ。
%
お
\ruby{{\換字{前}}}{まへ}の
\ruby[||j>]{大}{だい}
\ruby[||j>]{{\換字{嫌}}}{きらひ}の
% \ruby{大{\換字{嫌}}}{だい|きらひ}の
\ruby{靜岡}{しづ|をか}の
\ruby{叔母}{を|ば}さんだよ。
』

\原本頁{119-6}%
と、
%
お
\ruby{彤}{とう}は
\ruby{笑}{ゑみ}を
\ruby{含}{ふく}んで
\ruby{云}{い}ひたり。
