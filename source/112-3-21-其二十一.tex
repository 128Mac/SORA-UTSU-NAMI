\Entry{其二十一}

お
\ruby{龍}{りう}は
お
\ruby{彤}{とう}の
\ruby{水野}{みづ|の}を
\ruby{{\換字{評}}}{ひやう}するに
\ruby{{\換字{平}}}{たひ}らかならねども、
\ruby{反駁}{いひ|かへ}さんも
\ruby{何}{なん}と
\ruby{無}{な}く
\ruby{後見}{うしろ|み}らるゝ
\ruby{心地}{こゝ|ち}せしが、
\ruby{其}{そ}の
\ruby{言}{い}ふところ
\ruby{多}{おほ}くは
\ruby{當}{あた}れるを
\ruby{如何}{い|かん}とも
\ruby{爲}{す}る
\ruby{能}{あた}はず、たゞ
\ruby{僅}{わづか}に、

『あら
\ruby{姊}{ねえ}さん。
てんで
\ruby{妾}{わたし}あ
\ruby{全然}{まる|きり}
\ruby{其樣}{そ|ん}な
\ruby{事}{こと}を
\ruby{思}{おも}つてや
\ruby{仕無}{し|な}いのですから、
\ruby{彼}{あ}の
\ruby{人}{ひと}が
\ruby{貧乏性}{びん|ばう|しやう}だつて
\ruby{無粹}{ぶ|いき}だつて
\ruby{何樣}{ど|う}だつて
\ruby{宜}{い}いぢや
\ruby{有}{あ}りませんか、
\ruby{不足}{ふ|そく}でも
\ruby{{\換字{過}}}{す}ぎて
\ruby{居}{ゐ}ても
\ruby{關係}{かゝり|あひ}の
\ruby{無}{な}い
\ruby{事}{こと}ですは。

\ruby{隨{\換字{分}}酷}{ずゐ|ぶん|ひど}い
\ruby{事}{こと}ネエ、
\ruby{姊}{ねえ}さんの
\ruby{言}{くち}も。
』

と、
\ruby{知}{し}らざるを
\ruby{粧}{よそほ}ひて
\ruby{我}{われ}には
\ruby{聞}{き}き
\ruby{辛}{づら}き
\ruby{談}{はなし}を
\ruby{少}{すこ}しも
\ruby{早}{はや}く
\ruby{外}{はづ}さんと
\ruby{仕}{し}たり。

『
\ruby{然樣}{さ|う}さネエ。
ホヽヽ
\ruby{關係}{かゝり|あひ}の
\ruby{無}{な}いものを
\ruby{兎}{と}や
\ruby{角}{かく}いふのには
\ruby{當}{あた}らないのだがネ、
\ruby{此}{これ}あまあ
\ruby{無意}{た|ゞ}の
\ruby{話}{ばなし}だと
\ruby{思}{おも}つて
\ruby{聞}{き}いて
\ruby{居}{ゐ}て
\ruby{御覽}{ご|らん}よ。
お
\ruby{前}{まへ}はどうせ
\ruby{彼}{あ}の
\ruby{人}{ひと}を
\ruby{何樣}{だ|う}の
\ruby{彼樣}{か|う}のとなんぞ
\ruby{思}{おも}つては
\ruby{御}{お}いでゞ
\ruby{無}{な}いといふのだから、
\ruby{別}{べつ}に
\ruby{何}{なん}にも
\ruby{心配}{しん|ぱい}は
\ruby{無}{な}いがネ。
こゝに
\ruby{氣}{き}が
\ruby{優}{やさ}しくつて
\ruby{而}{そ}して
\ruby{侠氣}{をと|こぎ}のあるやうな
\ruby{若}{わか}い
\ruby{女}{ひと}があつて、
\ruby{何樣}{ど|う}かした
\ruby{心}{こゝろ}の
\ruby{機勢}{はず|み}から
\ruby{彼}{あ}の
\ruby{人}{ひと}を
\ruby{思}{おも}ふやうなことが
\ruby{有}{あ}るとするとするとネ、
\ruby{早}{はや}く
\ruby{氣}{き}がついて
\ruby{引{\換字{返}}}{ひつ|かへ}して
\ruby{仕舞}{し|ま}へば
\ruby{其限}{それつ|きり}で
\ruby{濟}{す}むけれども、
\ruby[g]{田舍{\換字{道}}}{ゐなかみち}なんか
\ruby{歩}{ある}いても
\ruby{能}{よ}くある
\ruby{事}{こと}で、
\ruby[g]{二十丁三十丁}{にじうちやうさんじうちやう}も
\ruby{間{\換字{違}}}{ま|ちが}つた
\ruby{路}{みち}
\ruby{踏込}{ふみ|こ}んで
\ruby{仕舞}{し|ま}ふと、あゝ
\ruby{間{\換字{違}}}{ま|ちが}つたと
\ruby{氣}{き}が
\ruby{付}{つ}いても
\ruby{後}{あと}へ
\ruby{{\換字{返}}}{かへ}る
\ruby{氣}{き}にはなれないで、
\ruby{何樣}{ど|う}かして
\ruby{出拔}{で|ぬ}けやう
\ruby{出拔}{で|ぬ}けやうつて
\ruby{云}{い}ふんで
\ruby{餘計變}{よ|けい|へん}な
\ruby{路}{みち}へ
\ruby{入}{はい}つて、
\ruby{下}{くだ}らない
\ruby{苦}{くるし}みをすることが
\ruby{得}{え}て
\ruby{有}{あ}るものだが
\ruby[g]{丁度其樣}{ちやうどそん}な
\ruby{譯}{わけ}で
\ruby{下手}{へ|た}に
\ruby{人}{ひと}を
\ruby{思}{おも}つて、
\ruby{少}{すこ}し
\ruby{宛少}{づつ|すこ}し
\ruby{宛深}{づつ|ふか}みへ
\ruby{入}{はい}つて
\ruby{行}{ゆ}くと、
\ruby{{\換字{終}}}{しまひ}にやあ
\ruby{飛}{と}んだ
\ruby{目}{め}を
\ruby{見無}{み|な}けりやあならないやうな
\ruby{馬鹿}{ば|か}なところへ
\ruby{行}{い}つて
\ruby{突當}{つき|あた}りもするよ。
\ruby{何}{なん}でも
\ruby{前{\換字{途}}}{さ|き}の
\ruby{知}{し}れない
\ruby{怪}{あや}しい
\ruby{路}{みち}へ
\ruby{入}{はい}つたら、
\ruby{一二丁}{いち|に|ちやう}しか
\ruby{歩}{ある}かない
\ruby{中}{うち}
\ruby{立止}{たち|どま}つてネ、\換字{志}つと
\ruby{考}{かんが}へるか
\ruby{人}{ひと}に
\ruby{聞}{き}くかして、
\ruby{引{\換字{返}}}{ひつ|かへ}すのがまあ
\ruby{肝心}{かん|じん}で、
\ruby{無暗}{む|やみ}に
\ruby{歩}{ある}いて
\ruby{行}{ゆ}くのは
\ruby{一番危}{いち|ばん|あぶな}い
\ruby{事}{こと}だよ。
\ruby{彼}{あ}の
\ruby{水野}{みづ|の}つていふ
\ruby{人}{ひと}は
\ruby{一}{ひ}ㇳ
\ruby{目見}{め|み}ても
\ruby{{\換字{分}}}{わか}る、
\ruby{性}{しやう}は
\ruby{良}{よ}い、
\ruby{眞人間}{ま|にん|げん}だよ、
\ruby{不實}{ふ|じつ}な
\ruby{人}{ひと}ぢや
\ruby{無}{な}い。
だから
\ruby{彼}{あ}の
\ruby{人}{ひと}が
\ruby{別}{べつ}に
\ruby{人}{ひと}を
\ruby{思}{おも}つてるので
\ruby{無}{な}けりやあ、
\ruby{彼}{あ}の
\ruby{人}{ひと}を
\ruby{好}{す}いたといふ
\ruby{女}{ひと}が
\ruby{有}{あ}りやあ
\ruby{其}{そ}りやあ
\ruby{好}{す}いたで
\ruby{宜}{い}いのさ。
\ruby{而}{そ}して
\ruby{其}{そ}の
\ruby{女}{ひと}の
\ruby{思}{おもひ}も
\ruby{屹度}{きつ|と}
\ruby{彼}{あ}の
\ruby{人}{ひと}に
\ruby{{\換字{分}}}{わか}つて、
\ruby{小{\換字{説}}}{せう|せつ}ならばまあ
\ruby{芽出度}{め|で|たし}
\ruby{芽出度}{め|で|たし}といふところにもなるだらうがネ。
\ruby{彼}{あ}の
\ruby{人}{ひと}が
\ruby{他}{ほか}の
\ruby{人}{ひと}を
\ruby{一心}{いつ|しん}に
\ruby{思}{おも}つてるからにやあ、
\ruby{性}{しやう}の
\ruby{良}{よ}い
\ruby{人}{ひと}だけに
\ruby{傍}{わき}からの
\ruby{思}{おも}ひは
\ruby{受}{う}け
\ruby{付}{つ}けまい、
\ruby{眞人間}{ま|にん|げん}だけに
\ruby{二心}{ふた|ごころ}は
\ruby{持}{も}つまいよ。
\ruby{然樣}{さ|う}すりやあ
\ruby{彼}{あ}の
\ruby{人}{ひと}を
\ruby{思}{おも}ふなあ
\ruby[g]{死路}{つきあたり}へ
\ruby{向}{むか}つて
\ruby{行}{い}くやうなもので、
\ruby{行}{い}けば
\ruby{行}{い}くだけの
\ruby[g]{草臥儲}{くたびれまう}けたから、そんな
\ruby{路}{みち}へ
\ruby{若}{も}し
\ruby{一寸}{ちよ|つと}でも
\ruby{歩}{あし}が
\ruby{向}{む}いて
\ruby{居}{ゐ}たらば、
\ruby{其方}{そつ|ち}へ
\ruby{踏込}{ふみ|こ}んだか
\ruby{踏}{ふ}み
\ruby{込}{こ}まない
\ruby{中}{うち}
\ruby{後}{あと}へ
\ruby{引{\換字{返}}}{ひつ|かへ}して
\ruby{仕舞}{し|ま}ふと、
\ruby[g]{然程苦}{さほどく}にもならない、
\ruby{損}{そん}も
\ruby{仕無}{し|な}いで
\ruby{濟}{す}むといふ
\ruby{譯}{わけ}なのだよ。
\ruby{誰}{たれ}しも
\ruby{損路}{そん|みち}を
\ruby{仕}{し}ないで
\ruby{世}{よ}の
\ruby{中}{なか}を
\ruby{歩}{ある}いて
\ruby{來}{く}るものは
\ruby[g]{中々無}{なか〳〵な}い。
お
\ruby{前}{まへ}は
お
\ruby{知}{し}りでないが
\ruby{妾}{わたし}だつて
\ruby{損{\換字{道}}}{そん|みち}を
\ruby[g]{澤山仕}{たくさんし}て
\ruby{來}{き}て
\ruby{居}{ゐ}る。
お
\ruby{前}{まへ}は
\ruby{妾}{わたし}も
\ruby{知}{し}つてるが
\ruby{既一度甚}{もう|いち|ど|ひど}い
\ruby{冗{\換字{道}}}{むだ|みち}を
\ruby{歩}{ある}いて、
\ruby{踏拔}{ふみ|ぬき}も
\ruby{仕}{し}ておいでだし
\ruby{生爪}{なま|づめ}も
\ruby{剝}{は}がしておいでだし、
\ruby{散々}{さん|〴〵}な
\ruby{目}{め}に
お
\ruby{會}{あ}ひだつた
\ruby{人}{ひと}だから、
\ruby{今}{いま}さらまた
\ruby{前{\換字{途}}}{さ|き}の
\ruby{知}{し}れない
\ruby{怪}{あや}しい
\ruby{路}{みち}へなんぞ、
\ruby[g]{無暗}{むやみ}には
\ruby{入}{はい}つて
\ruby{御}{お}いででは
\ruby{有}{あ}るまいから
\ruby{宜}{い}いがネ。
』

お
\ruby{彤}{とう}は
\ruby{云}{い}ひ
\ruby{{\換字{終}}}{をは}つて
\ruby{默}{もく}し、
お
\ruby{龍}{りう}は
\ruby{聞}{き}き
\ruby{{\換字{終}}}{をは}つて
\ruby{默}{もく}し、
\ruby{互}{たがひ}に
\ruby{言葉}{こと|ば}の
\ruby{{\換字{絕}}}{た}えたるところへ、
\ruby{小間使}{こ|ま|づかひ}の
お
\ruby{春}{はる}は
\ruby{次室}{つぎ|のま}より
\ruby{現}{あら}はれ、

『あの
\ruby{昨日}{きの|ふ}
お
\ruby{來臨}{い|で}なすつた
お
\ruby{婆}{ばあ}さんの
\ruby{方}{かた}が
\ruby{御出}{お|いで}になりました。
』

と
\ruby{云}{い}へば、

『おゝ
\ruby{丁度好}{ちやう|ど|よ}いところへだつた、
\ruby{此方}{こち|ら}へと
\ruby{御云}{お|い}ひ。
お
\ruby{龍}{りう}ちやん、
お
\ruby{前}{まへ}、
\ruby{吃驚}{びつ|くり}おしで
\ruby{無}{な}いよ。
お
\ruby{前}{まへ}の
\ruby{大{\換字{嫌}}}{だい|きらひ}の
\ruby{靜岡}{しづ|をか}の
\ruby{叔母}{を|ば}さんだよ。
』

と、お
\ruby{彤}{とう}は
\ruby{笑}{ゑみ}を
\ruby{含}{ふく}んで
\ruby{云}{い}ひたり。

