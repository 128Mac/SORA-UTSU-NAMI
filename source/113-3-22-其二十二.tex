\Entry{其二十二}

% メモ 校正終了 2024-05-14 2024-06-10
\原本頁{119-8}%
お
\ruby{彤}{とう}と
\ruby{我}{わ}が
\ruby{叔母}{を|ば}とは
\ruby{相識}{ちか|づき}なるべき
\ruby{筈}{はず}の
\ruby{無}{な}ければ、
%
\ruby{此家}{こ|ゝ}にて
\ruby{叔母}{を|ば}に
\ruby{會}{あ}はんとは
\ruby{夢}{ゆめ}にも
\ruby{思}{おも}ひがけざりし
お
\ruby{龍}{りう}の、
%
\ruby[|g|]{主人}{あるじ}の
\ruby{言葉}{こと|ば}を
\ruby{聞}{き}きても
\ruby{{\換字{猶}}}{なほ}
\ruby{信}{しん}じ
かねて、
%
よもやと
\ruby{疑}{うたが}ひ
\ruby{訝}{いぶ}かれる
\ruby{間}{ま}も
\ruby{無}{な}く、
%
\ruby{既}{はや}
お
\ruby{春}{はる}に
\原本頁{120-1}\改行%
\ruby{導}{みち}びかれて、
%
\ruby[|g|]{身體}{からだ}は
\ruby{一體}{いつ|たい}が
\ruby{小粒}{こ|つぶ}なる
\ruby{上}{うへ}に
\ruby{老}{お}いたれば
いと
\ruby{小}{ちひ}さく
\ruby{見}{み}ゆれど、
%
\ruby{石}{いし}の
\ruby{如}{ごと}く
こつつりと% ルビ調整(原本通り)非踊り字表記
\ruby{堅}{かた}さうに
\ruby{緊}{しま}り
\ruby{切}{き}つたる
\ruby{小}{ちひ}さき
\makeatletter
\@ifundefined{デバッグ@ビルド}{%
  \ruby{顏}{かほ}、
}{%
  \ruby{顏}{かほ}
  \改行% 校正作業の簡略化のため
  、
}%
\makeatother
%
\原本頁{120-3}\改行%
\ruby{薄}{うす}く
なりたる
\ruby{癖毛}{くせ|げ}の
びつたりと
\ruby{地}{ぢ}に
\ruby{緊着}{かじり|つ}ける
\ruby{小}{ちひ}さなる
\ruby{頭}{あたま}、
%
\ruby{負}{ま}けぬ
\ruby{氣}{き}が
\ruby{尖}{とが}つて
\ruby{露}{あらは}れたる
やうなる
\ruby{小}{ちひ}さき
\ruby{三角}{さん|かく}の
\ruby{眼}{め}、
%
\ruby{都}{す}べて
\ruby{小}{ちひ}さきが
\ruby{中}{なか}に
\ruby{毫}{もつと}も
\ruby{{\換字{緩}}}{たる}みの
\ruby{無}{な}き、
%
\ruby{我}{わ}が
\ruby{叔母}{を|ば}の
お
\ruby{{\換字{近}}}{ちか}は
\ruby{忽}{たちま}ちに
\ruby{現}{あら}はれたり。
%
\原本頁{120-6}\改行%
\ruby{監}{あゐ}の
\ruby{味噌}{み|そ}
\ruby{漉縞}{こし|じま}の
\ruby{衣}{きもの}を
\ruby{襟元}{えり|もと}
\ruby{窄}{せま}く
\ruby{着}{き}て、
%
\ruby{疊}{たゝ}み
\ruby{皺}{じわ}
\ruby{見}{み}ゆる
\ruby{黑}{くろ}の
\ruby{紬}{つむぎ}の
\ruby{羽織}{は|おり}に、
%
\ruby{{\換字{古}}}{ひ}ねて
\ruby{堅}{かた}くなつた
\ruby{茶}{ちや}の
\ruby{細紐}{ほそ|ひも}を
\ruby{少}{すこ}し
\ruby{胸高}{むな|だか}に
きつちりと
\ruby{結}{むす}び、
%
\原本頁{120-8}\改行%
\ruby{妙}{めう}に
\ruby{角張}{かど|ば}つて
\ruby{坐}{すわ}つて
しなやか
ならず
\ruby{挨拶}{あい|さつ}せる
さまは、
%
\ruby{何樣}{ど|う}
\ruby{見}{み}ても
\ruby{靜岡}{しづ|をか}の
\ruby{在}{ざい}より
\ruby{出}{い}で
\ruby{來}{きた}りたる
\ruby[|g|]{田舎}{ゐなか}
\ruby{婆}{ばゞ}と
\ruby{見}{み}えて
\ruby{律義}{りち|ぎ}
\ruby{臭}{くさ}し。
%
されど
\原本頁{120-10}\改行%
\ruby{明治}{めい|ぢ}の
\ruby[|g|]{初年}{はじめ}に
\ruby{兩親}{ふた|おや}に
\ruby{{\換字{連}}}{つ}れられて、
%
\ruby[||j>]{東}{とう}
\ruby[||j>]{京}{きやう}を
% \ruby{東京}{とう|きやう}を
\ruby{離}{はな}れし
まゝ
\ruby[|j||]{茶圃}{ちや|ばたけ}% 原本でのアキを再現
\ruby[|j||]{麥圃}{むぎ|ばたけ}の
\ruby{間}{なか}に
\ruby{齷齪}{あく|せく}として
\ruby{年}{とし}を
\ruby{取}{と}りは
\ruby{仕}{し}たれ、
%
\ruby{根}{か}からの% ルビ調整(原本通り)(か)←(ね)の誤植?
\ruby[|g|]{田舎}{ゐなか}
\ruby{者}{もの}ならぬに
\原本頁{121-1}\改行%
\ruby{言語}{もの|いひ}
だけは
\ruby{然}{さ}のみ
をかしからず。

\原本頁{121-2}%
『
\ruby{何樣}{ど|う}も
\ruby{昨日}{さく|じつ}は
まことに
お
\ruby{喧}{やかま}しう
ございましたらう。
%
\ruby{老年}{とし|より}では
ございますし、
%
\ruby{我張}{がつ|ぱ}り
\ruby{婆}{ばゞあ}では
ございますし、
%
それに
\ruby[|g|]{田舎}{ゐなか}に
\ruby{居}{を}りますので
\ruby{自然}{し|ぜん}と% ルビ調整(原本通り)非グループルビ
\ruby{馬士}{ま|ご}
かなんぞ
のやうな
\ruby{大聲}{おほ|ごゑ}に
なつて
\ruby{仕舞}{し|ま}ひまして、
%
\ruby{自{\換字{分}}}{じ|ぶん}の% ルビ調整(原本通り)非グループルビ
\ruby{{\換字{勝}}手}{かつ|て}
ばかり
\ruby[|g|]{饒舌}{しやべ}り
\ruby{散}{ち}らし
ました
から
\ruby{嘸}{さぞ}
\ruby{御{\換字{迷}}惑}{ご|めい|わく}で
ございました
らうと、
%
\ruby{是}{これ}でも
\ruby{{\換字{又}}}{また}
\ruby[||j>]{殊}{しゆ}
\ruby[||j>]{{\換字{勝}}}{しよう}
% \ruby{殊{\換字{勝}}}{しゆ|しよう}
らしいもので、
%
\ruby{後}{あと}では
\ruby{御氣}{お|き}の
\ruby{毒}{どく}に
\ruby{存}{ぞん}じました
ので
ございます。
%
どうも
\ruby[g]{種々}{いろ〳〵}
\ruby{何}{なに}や
\ruby{彼}{か}や
\ruby{御深切}{ご|しん|せつ}さまに
\ruby{有}{あ}り
\ruby{{\換字{難}}}{がた}う
\ruby{存}{ぞん}じました。
%
それに
\ruby{御馳走}{ご|ち|そう}に
まで
なりまして、
%
\ruby{夜{\換字{分}}}{や|ぶん}
まで
お
\ruby{邪{\換字{魔}}}{じや|ま}を
\ruby{致}{いた}しましたり
なんぞ
して、
%
まことに
\ruby{既}{はや}
\ruby{年甲{\換字{斐}}}{とし|が|ひ}も
\ruby{無}{な}い
\ruby{自{\換字{分}}{\換字{勝}}手}{じ|ぶん|かつ|て}% ルビ調整(原本通り)非グループルビ
ばかりの
\ruby{婆}{ばゞあ}だと、
%
\ruby{御蔑視}{お|さげ|すみ}の
ところも
\ruby{御羞}{お|はづか}しう
ございました。
%
\ruby{{\換字{若}}}{も}し
\ruby{萬一}{ひよ|つと}
さて〳〵
\ruby{{\換字{勝}}手者}{かつ|て|もの}だと
\ruby{御愛想盡}{お|あい|そ|づ}かしも
\ruby{有}{あ}らうかと、
%
\ruby{宿}{やど}へ
\ruby{歸}{かへ}りまして
から
\ruby{些}{ちと}
\ruby{心配}{しん|ぱい}
\ruby{致}{いた}しましたが、
%
ナアニ
\ruby{馬鹿}{ば|か}にやあ
\ruby{怜悧}{り|こう}な% ルビ調整(原本通り)(りこう)非グループルビ
\ruby{方}{かた}の
\ruby{事}{こと}は
\ruby{{\換字{分}}}{わか}らなくつても
\ruby{怜悧}{り|こう}にやあ% ルビ調整(原本通り)(りこう)非グループルビ
\ruby{馬鹿}{ば|か}な
ものゝ
\ruby{事}{こと}は
\原本頁{122-3}\改行%
\ruby{能}{よ}く
\ruby{{\換字{分}}}{わか}るだらうから、
%
\ruby[|g|]{此方}{こつち}の
\ruby{何程}{どれ|ほど}か
\ruby{有}{あ}り
\ruby{{\換字{難}}}{がた}く
\ruby{思}{おも}つて
\ruby{居}{ゐ}る
\ruby{位}{ぐらゐ}の
\ruby{事}{こと}は
\ruby{御{\換字{分}}}{お|わか}り
だらう
から
まあ
\ruby{安心}{あん|しん}だ、
%
\ruby[|g|]{屹度}{きつと}
\ruby{馬鹿}{ば|か}
\ruby{婆}{ばゞあ}
だけれど
\ruby{腹}{おなか}の
\ruby{中}{なか}は
\原本頁{122-5}\改行%
\ruby{人並}{ひと|なみ}だ
\ruby{位}{ぐらゐ}には
\ruby{思}{おも}つて
\ruby{居}{ゐ}て
\ruby{下}{くだ}さる
だらう
から、
%
と
\ruby{斯樣}{か|う}
まづ
\ruby{{\換字{勝}}手}{かつ|て}に
\ruby{決}{き}めて
\ruby{仕舞}{し|ま}つて、
%
\ruby{安堵}{おち|つ}いた
ので
ございます。
%
ハヽヽ、
%
\ruby{何樣}{ど|う}か
\ruby{御恩}{ご|おん}には
\ruby{必}{かな}らず
\ruby{着}{き}ますから
\ruby{宜}{よろ}しく
\ruby{御願}{お|ねが}ひ
\ruby{申}{まを}しまする。
%
では
\ruby{此女}{こ|れ}も
\原本頁{122-8}\改行%
もう
\ruby[|g|]{貴女}{あなた}
\ruby{樣}{さま}が
\ruby{今日}{け|ふ}
お
\ruby{招}{よ}び
\ruby{下}{くだ}さいました
ので?。
』

\原本頁{122-9}%
と、
%
\ruby{人}{ひと}の
\ruby{云}{い}ふ
\ruby{事}{こと}は
\ruby{餘}{あま}り
\ruby{聞}{き}かずに
\ruby{獨}{ひと}りで
\ruby[|g|]{饒舌}{しやべ}つて、
%
お
\ruby{彤}{とう}には
\ruby{語}{ことば}を
\ruby{挿}{はさ}む
\ruby{間}{ま}を
さへ
ほと〳〵
\ruby{與}{あた}へざる
ほど、
%
\ruby{身體}{な|り}には
\ruby{似合}{に|あ}はず
\ruby[<j||]{大}{おほき}な
\ruby[<j||]{頑}{じやう}% 行末行頭の境界付近なので特例処置を施す
\ruby{健}{ぶ}なる
\ruby{聲}{こゑ}もて
\ruby{先}{ま}づ
\ruby{語}{かた}りたり。
