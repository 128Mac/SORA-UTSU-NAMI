\Entry{其二十三}

% メモ 校正終了 2024-05-14
\原本頁{123-2}%
お
\ruby{彤}{とう}は
お
\ruby{{\換字{近}}}{ちか}が
\ruby{言}{ものい}へる
\ruby{間}{あひだ}にも、
%
\ruby{少}{すこ}しの
\ruby{受答}{うけ|こた}へを
\ruby{爲}{し}つ、
%
\ruby{語}{くち}を
\ruby{挿}{はさ}まん
とせざる
には
あらざりしも、
%
\ruby{立板}{たて|いた}に
\ruby{水}{みづ}とは
いふべき
ならねど
\ruby{下}{くだ}り
\ruby{坂}{ざか}に
\ruby{走}{はし}る
\ruby{小車}{を|ぐるま}
のやうに
\ruby{騷}{さわ}がしく
\ruby{忙}{せは}しく
\ruby{話}{はな}し
つゞけられて
\ruby{口}{くち}を
\原本頁{123-5}\改行%
\ruby{入}{い}れ
\ruby{{\換字{兼}}}{か}ね
\ruby{居}{ゐ}しが、
%
\ruby{今}{いま}
\ruby{斯}{か}く
\ruby{問}{と}ひ
かけられて
\ruby{僅}{わづか}に
\ruby{言葉}{こと|ば}を
\ruby{出}{いだ}し、

\原本頁{123-6}%
『
いゝえ
\ruby{然樣}{さ|う}ぢやあ
\ruby{有}{あ}りませんが
\ruby{他}{ほか}の
\ruby{事}{こと}で
もつて、
%
\ruby{丁度}{ちやう|ど}
\ruby{自然}{ひと|りで}に
\ruby[|g|]{先刻}{さつき}
\ruby{方}{がた}
\ruby{見}{み}えたので、
』

\原本頁{123-8}%
と
\ruby{云}{い}ひ
かけて
お
\ruby{龍}{りう}の
\ruby{方}{はう}を
\ruby{莞爾}{に|こ}やかに
\ruby{見}{み}やり、

\原本頁{123-9}%
『
お
\ruby{龍}{りう}ちやん
お
\ruby{{\換字{前}}}{まへ}、
%
\ruby{默}{だま}つて
おいで
ぢやあ
\ruby{不可}{いけ|ない}よ、
%
\ruby{叔母}{を|ば}さん
ぢやあ
\ruby{無}{な}いかネ。
』

\原本頁{124-1}%
と
\ruby{輕}{かろ}き
\ruby{一句}{いつ|く}を
\ruby{與}{あた}へつ、
%
また
お
\ruby{{\換字{近}}}{ちか}に
\ruby{向}{むか}ひて、

\原本頁{124-2}%
『
きまりが
\ruby{惡}{わる}い
もので
\ruby{羞澁}{はに|か}んで
\ruby{困}{こま}つて
\ruby{居}{ゐ}る
のですよ。
%
ホヽヽ
まだ
\ruby{{\換字{若}}}{わか}くつて、
%
いつそ
\ruby{可憐}{か|はい}% 原本通り非グループルビ
らしい
ぢやあ
\ruby{有}{あ}りませんか。
%
どうか
まあ
\ruby{今日}{け|ふ}の
ところは
\ruby{御叱}{お|しか}り
なさらないでネ、
%
\ruby[|g|]{貴卿}{あなた}が
\ruby{御目上}{お|め|うへ}ですから
\ruby{優}{やさ}しく
\ruby{仕}{し}て
\ruby{御與}{お|や}り
なすつてネ。
』

\原本頁{124-6}%
と、
%
\ruby[|g|]{二人}{ふたり}の
\ruby{間}{あひだ}をば
\ruby{取}{と}り
\ruby{繕}{つくろ}ふ
やうに
\ruby{云}{い}へり。

\原本頁{124-7}%
\ruby{此}{こ}の
\ruby{叔母}{を|ば}が
\ruby{擇}{えら}み
\ruby{定}{さだ}めし
\ruby{婿}{むこ}を% (婿 5a7f) 聟 805f
\ruby{{\換字{嫌}}}{きら}ひし
より、
%
\ruby{{\換字{朝}}}{あさ}と
なく
\ruby{夜}{よる}と
\ruby{無}{な}く
\ruby{論}{い}ひ
\ruby{合}{あ}ひ
\ruby{睨}{にら}み
\ruby{合}{あ}ひて、
%
さらぬだに
\ruby{性}{しやう}の
\ruby{合}{あ}はぬ
\ruby{中}{なか}の、
いよ〳〵
おもしろからず、
えゝ、
%
あた
\ruby{忌々}{いま|〳〵}しい、
%
\ruby{何}{なん}と
なるものぞと、
%
\ruby{後}{あと}の
\ruby{{\換字{迷}}惑}{めい|わく}も
\ruby{思}{おも}はずに
\ruby{無言}{だ|ま}つて
\ruby{駈}{か}け
\ruby{出}{だ}したる
まゝ、
%
\ruby{恩}{おん}の
ある
\ruby{事}{こと}は
\ruby{知}{し}つて
\ruby{居}{ゐ}れど
\ruby{憎}{にく}らしさ
もあるに、
%
\ruby{手紙}{て|がみ}
\ruby{一本}{いつ|ぽん}も
\ruby{出}{だ}さで
\ruby{知}{し}らぬ
\ruby{顏}{かほ}に
\ruby{濟}{す}まし
\ruby{來}{きた}りし
\ruby{今日}{け|ふ}、
%
\ruby{突然}{だし|ぬけ}に
\ruby{此處}{こ|ゝ}に
\ruby{相}{あひ}
\ruby{會}{あ}ひては
お
\ruby{龍}{りう}も
\ruby{聊}{いさゝ}か
\ruby{驚}{おどろ}きつ、
%
\ruby{顏}{かほ}を
\ruby{見}{み}ては
\ruby[|g|]{流石}{さすが}
\ruby{氣}{き}の
\ruby{毒}{どく}さに
\ruby{面伏}{おも|ぶせ}の
\ruby{思}{おも}ひ
も
すれど、
%
\ruby{{\換字{勝}}手}{かつ|て}のみ
\ruby{{\換字{強}}}{つよ}くして
\ruby{{\換字{遠}}慮}{ゑん|りよ}を
\ruby{知}{し}らぬ
\ruby{性急}{せつ|かち}の
\ruby[||j>]{話}{はなし}
\ruby[||j>]{聲}{ ごゑ}
% \ruby{話聲}{はなし|ごゑ}
の、
%
いつも
ながら
\ruby{喧}{やかま}しく
\ruby{耳}{みゝ}に
\ruby{響}{ひゞ}くを
\ruby{聞}{き}きては、
%
もう
\ruby{薄腹}{うす|はら}の
\ruby{立}{た}つほど
\ruby{蟲}{むし}が
\ruby{{\換字{嫌}}}{きら}つて
\ruby{厭}{いや}で〳〵
\ruby{堪}{たま}らず、
%
\ruby{出}{で}ずとも
\ruby{可}{い}い
\ruby{人}{ひと}が
\ruby{出}{で}て
\ruby{來}{き}てと
\ruby{{\換字{迷}}惑}{めい|わく}
がりて、
%
\ruby{出}{で}るも
\ruby{引}{ひ}くも
ならぬに
\ruby{心}{こゝろ}
そげて
\ruby{居}{ゐ}たりしが、
%
お
\ruby{彤}{とう}に
\ruby{斯}{か}く
\ruby{云}{い}はれては
\ruby{横}{よこ}を
\ruby{向}{む}いて
ばかりも
\ruby{居}{ゐ}られず、
%
\ruby{不承}{ふ|しよう}
\ruby[g]{々々}{ 〴〵 }に、

\原本頁{124-8}%
『
\ruby{叔母}{を|ば}さん
‥‥
』

\原本頁{124-9}%
と
\ruby{云}{い}ひし
\ruby{限}{ぎ}り、
%
あとは
ぐず〴〵と
\ruby{口}{くち}の
\ruby{内}{うち}にて
\ruby{何}{なに}を
\ruby{云}{い}ひし
やら
\ruby{知}{し}れず、
%
\ruby{{\換字{術}}無}{じゆつ|な}げに
\ruby{頭}{かしら}を
\ruby{下}{さ}げて
\ruby{漸}{やつ}と
\ruby{挨拶}{あい|さつ}
すれば、
%
\ruby{叔母}{を|ば}は
なか〳〵
もう
\ruby{默}{だま}つては
\ruby{居}{ゐ}ず、
%
\ruby{三角}{さん|かく}の
\ruby{眼}{め}を
きらりと
\ruby{光}{ひか}らせ、

\原本頁{126-1}%
『
でも
まあ
\ruby{能}{よ}く
\ruby{忘}{わす}れずに
\ruby{叔母}{を|ば}さんと
\ruby{御云}{お|い}ひ
だつたネ。
%
ハイ、
%
\ruby{其}{その}
\原本頁{126-2}\改行%
\ruby{後}{のち}は% 行末行頭の境界付近なのでルビも非踊り字表記
\換字{志}ばらく。
%
お
\ruby{{\換字{前}}}{まへ}も
\ruby{御{\換字{達}}者}{お|たつ|しや}で、
%
\ruby{別}{べつ}に
\ruby{御天{\換字{道}}樣}{お|てん|たう|さま}にも
\ruby{愛想}{あい|そ}を
\ruby{盡}{つ}かされずに
\ruby{御暮}{お|くら}しで、
%
まあ
\ruby{結構}{けつ|こう}だネ。
%
まことに
お
\ruby{{\換字{前}}}{まへ}の
\ruby{御蔭}{お|かげ}ぢやあ
\原本頁{126-4}\改行%
\ruby{恐}{おそ}ろしい
\ruby{沸湯}{にえ|ゆ}を
\ruby{飮}{の}ませ
られました。
%
\ruby{會}{あ}つたら
\ruby[||j>]{引}{ひつ}
\ruby[||j>]{捉}{つかま}へて
% \ruby{引捉}{ひつ|つかま}へて
\ruby{耳}{みゝ}でも
\ruby{扯}{ちぎ}り
\ruby{取}{と}つて
あげて、
%
\ruby{何}{ど}の
%%\ruby[|-|]{位}{くらゐ}
%%\ruby[||-|]{妾}{わたし}が% 原本の空きを再現
\ruby[<j||]{位}{くらゐ}% 原本の空きを再現を試みたがアキが目立ちすぎなので詰めることにした
\ruby{妾}{わたし}が
\ruby{痛}{いた}かつたか
\ruby{苦}{くる}しかつたか、
%
\ruby{此樣}{こ|ん}な
ものだつた
よと、
%
\ruby{察}{さつ}して
\ruby{貰}{もら}ひ
ましやうと
\ruby{思}{おも}つて
\ruby{居}{ゐ}ましたがネ、
%
\原本頁{126-7}\改行%
\ruby[|g|]{此方}{こちら}
\ruby{樣}{さま}の
\ruby{御言葉}{お|こと|ば}
だから
\ruby{堪{\換字{忍}}}{かん|にん}% 原文通り「堪忍」
して
あげる。
%
\換字{志}かし
\ruby{彼}{あ}の
\ruby{事}{こと}は
\ruby{何樣}{ど|う}か
\ruby{此樣}{か|う}か% 原本通りルビは「かう」とする
\ruby{既}{もう}% 原本のルビ位置がおかしいので正しておく
\ruby{濟}{す}んで
\ruby{仕舞}{し|ま}つたが、
%
\ruby{一}{ひと}つ
\ruby{濟}{す}めば
\ruby{{\換字{又}}}{また}
\ruby{一}{ひと}つで
お
\ruby{{\換字{前}}}{まへ}の
\ruby{御蔭樣}{お|かげ|さま}で、
%
\ruby{斯樣}{か|う}して
\ruby[|j|]{砂塵}{すなつ|ぼこり}
ばかり
\ruby{立}{た}つ
\ruby[||j>]{東}{とう}
\ruby[||j>]{京}{きやう}
% \ruby{東京}{とう|きやう}
くんだりへ、
%
\ruby[|g|]{田舎}{ゐなか}
\ruby{婆}{ばあ}さんが
\原本頁{126-10}\改行%
ゑつちら
おつちらと
\ruby{得々}{わざ|〳〵}
\ruby{出}{で}かけて
\ruby{來}{き}て、
%
\ruby[|g|]{此方}{こちら}
\ruby{樣}{さま}へも
\ruby{御厄介}{ご|やく|かい}を
\ruby{掛}{か}けたり
なんぞ
\ruby{仕}{し}ます。
%
\ruby{婆}{ばあ}さんを
\ruby{苦勞}{く|らう}
ばかり
させて
\ruby{御手柄}{お|て|がら}の
\ruby{事}{こと}ですネ。
%
ほんとに
お
\ruby{{\換字{前}}}{まへ}の
\ruby{仕}{し}た
\ruby{事}{こと}に
\ruby{碌}{ろく}な
\ruby{事}{こと}は
\ruby{有}{あ}りやあ
\ruby{仕}{し}ない。
%
お
\ruby{{\換字{前}}}{まへ}の
\ruby{仕}{し}た
\ruby{事}{こと}の
\ruby{中}{うち}で
\ruby{好}{い}い
\ruby{事}{こと}
といふのは、
%
\ruby[|g|]{此方}{こちら}
\ruby{樣}{さま}に
\ruby{可愛}{か|はい}がつて
\ruby{頂}{いたゞ}いて
\原本頁{127-3}\改行%
\ruby{居}{ゐ}る
といふ
\ruby{事}{こと}
ばつかりだ。
%
\ruby[|g|]{此方}{こちら}
\ruby{樣}{さま}に
でも
\ruby{見離}{み|はな}されりやあ
お
\ruby{{\換字{前}}}{まへ}の
やうな
ものは、
%
それこそ
\ruby[|g|]{最{\換字{終}}}{しまひ}は
\ruby{倒}{のた}れ
\ruby{死}{じに}
だよ。
%
\ruby{身}{み}に
\ruby{染}{し}みて
\ruby{覺}{おぼ}えて
\原本頁{127-5}\改行%
おいでなさい、
%
もう
お
\ruby{{\換字{前}}}{まへ}の
\ruby[|g|]{身體}{からだ}は
お
\ruby{{\換字{前}}}{まへ}の
\ruby{料簡}{れう|けん}ぢやあ
\ruby{{\換字{勝}}手}{かつ|て}には
なりません。
%
\ruby{妾}{わたし}が
すつかりと
\ruby{願}{ねが}つて
\ruby{置}{お}きました。
%
もう
\ruby{何}{なに}も
\ruby{彼}{か}も
\ruby{此方樣}{こち|ら|さま}の% 原本通り非グループルビ
\ruby{仰}{おつし}やる
\ruby{{\換字{通}}}{とほ}りに
するのです。
%
\ruby{三絃}{さみ|せん}の
\ruby{師匠}{し|ゝやう}
だ
なんて、
%
\ruby{彼樣}{あ|ん}な
\ruby{惡}{わる}い
\ruby{人}{ひと}の
ところへ、
%
\ruby{身}{み}を
\ruby{置}{お}いては
\ruby{決}{けつ}して
なりません、
%
\ruby{出入}{で|はい}りしても
なりません。
%
\ruby{早{\換字{速}}}{さつ|そく}
これから
\ruby{其家}{そ|こ}を
\ruby{出}{で}て
\ruby[|g|]{此方}{こちら}へ
\ruby{御厄介}{ご|やく|かい}に
なつて、
%
\ruby[|g|]{此方}{こちら}
\ruby{樣}{さま}を
\ruby{有}{あ}り
\ruby{{\換字{難}}}{がた}い
と
おもつて
\ruby{身}{み}を
\ruby{責}{せ}めて
\ruby{御働}{お|はたら}きなさい。
』

\原本頁{127-11}%
と
\ruby{獨}{ひと}り
\ruby{合點}{が|てん}して、
%
まくし
\ruby{立}{た}てゝ
\ruby{指揮}{さし|づ}したり。

\原本頁{128-1}%
お
\ruby{彤}{とう}は
\ruby{訝}{いぶか}り
\ruby{疑}{うたが}ふ
お
\ruby{龍}{りう}を
\ruby{見}{み}て、

\原本頁{128-2}%
『
\ruby{叔母}{を|ば}さん、
%
\ruby{其}{それ}ぢやあ
\ruby{此}{こ}の
\ruby{人}{ひと}
にやあ
\ruby{{\換字{分}}}{わか}りますまい。
%
かういふ
\ruby{事}{こと}なのだよ
お
\ruby{龍}{りう}ちやん。
』

\原本頁{128-4}%
と
\ruby{靜}{しづか}に
\ruby{說}{と}き
\ruby{出}{いだ}したり。
