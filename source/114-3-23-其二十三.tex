\Entry{其二十三}

お
\ruby{彤}{とう}は
お
\ruby{近}{ちか}が
\ruby{言}{ものい}へる
\ruby{間}{あひだ}にも、
\ruby{少}{すこ}しの
\ruby{受答}{うけ|こた}へを
\ruby{爲}{し}つ、
\ruby{語}{くち}を
\ruby{挿}{はさ}まんとせざるにはあらざりしも、
\ruby{立板}{たて|いた}に
\ruby{水}{みづ}とはいふべきならねど
\ruby{下}{くだ}り
\ruby{坂}{ざか}に
\ruby{走}{はし}る
\ruby{小車}{をぐ|るま}のやうに
\ruby{騷}{さわ}がしく
\ruby{忙}{せは}しく
\ruby{話}{はな}しつづけられて
\ruby{口}{くち}を
\ruby{入}{い}れ
\ruby{{\換字{兼}}}{か}ね
\ruby{居}{ゐ}しが、
\ruby{今斯}{いま|か}く
\ruby{問}{と}ひかけられて
\ruby{僅}{わずか}に
\ruby{言葉}{こと|ば}を
\ruby{出}{いだ}し、

『いゝえ
\ruby{然樣}{さ|う}ぢやあ
\ruby{有}{あ}りませんが
\ruby{他}{ほか}の
\ruby{事}{こと}でもつて、
\ruby{丁度}{ちや|うど}
\ruby{自然}{ひと|りで}に
\ruby[g]{先刻方見}{さつきがたみ}えたので、』

と
\ruby{云}{い}ひかけて
お
\ruby{龍}{りう}の
\ruby{方}{はう}を
\ruby{莞爾}{に|こ}やかに
\ruby{見}{み}やり、

『お
\ruby{龍}{りう}ちやん
お
\ruby{前}{まへ}、
\ruby{默}{だま}つておいでぢやあ
\ruby{不可}{いけ|ない}よ、
\ruby{叔母}{を|ば}さんぢやあ
\ruby{無}{な}いかネ。
』

と
\ruby{輕}{かろ}き
\ruby{一句}{いつ|く}を
\ruby{與}{あた}へつ、また
お
\ruby{近}{ちか}に
\ruby{向}{むか}ひて、

『きまりが
\ruby{惡}{わる}いもので
\ruby[g]{羞澁}{はにか}んで
\ruby{困}{こま}つて
\ruby{居}{ゐ}るのですよ。
ホヽヽまだ
\ruby{若}{わか}くつて、いつそ
\ruby{可憐}{か|はい}らしいぢやあ
\ruby{有}{あ}りませんか。
どうかまあ
\ruby{今日}{け|ふ}のところは
\ruby{御叱}{お|しか}りなさらないでネ、
\ruby{貴卿}{あな|た}が
\ruby[g]{御目上}{おめうへ}ですから
\ruby{優}{やさ}しく
\ruby{仕}{し}て
\ruby{御與}{お|や}りなすつてネ。
』

と、
\ruby{二人}{ふた|り}の
\ruby{間}{あひだ}をば
\ruby{取}{と}り
\ruby{繕}{つくろ}ふやうに
\ruby{云}{い}へり。

\ruby{此}{こ}の
\ruby{叔母}{を|ば}が
\ruby{擇}{えら}み
\ruby{定}{さだ}めし
\ruby{婿}{むこ}を
\ruby{{\換字{嫌}}}{きら}ひしより、
\ruby{朝}{あさ}となく
\ruby{夜}{よる}と
\ruby{無}{な}く
\ruby{論}{い}ひ
\ruby{合}{あ}ひ
\ruby{睨}{にら}み
\ruby{合}{あ}ひて、さらだに
\ruby{性}{しやう}の
\ruby{合}{あ}はぬ
\ruby{中}{なか}の、いよ〳〵おもしろからず、えゝ、あた
\ruby{忌々}{いま|〳〵}しい、
\ruby{何}{なん}となるものぞと、
\ruby{後}{あと}の
\ruby{迷惑}{めい|わく}も
\ruby{思}{おも}はずに
\ruby{無言}{だ|ま}つて
\ruby{駈}{か}け
\ruby{出}{だ}したるまゝ、
\ruby{恩}{おん}のある
\ruby{事}{こと}は
\ruby{知}{し}つて
\ruby{居}{ゐ}れど
\ruby{憎}{にく}らしさもあるに、
\ruby[g]{手紙一本}{てがみいつぽん}も
\ruby{出}{だ}さで
\ruby{知}{し}らぬ
\ruby{顏}{かほ}に
\ruby{濟}{す}まし
\ruby{來}{きた}りし
\ruby{今日}{け|ふ}、
\ruby{突然}{だし|ぬけ}に
\ruby{此處}{こ|こ}に
\ruby{相會}{あひ|あ}ひては
お
\ruby{龍}{りう}も
\ruby{聊}{いさゝ}か
\ruby{驚}{おどろ}きつ、
\ruby{顏}{かほ}を
\ruby{見}{み}ては
\ruby{流石}{さす|が}
\ruby{氣}{き}の
\ruby{毒}{どく}さに
\ruby{面伏}{おも|ぶせ}の
\ruby{思}{おも}ひもすれど、
\ruby{{\換字{勝}}手}{かつ|て}のみ
\ruby{{\換字{強}}}{つよ}くして
\ruby{遠慮}{ゑん|りよ}を
\ruby{知}{し}らぬ
\ruby{性急}{せつ|かち}の
\ruby[g]{話聲}{はなしごゑ}の、いつもながら
\ruby{喧}{やかま}しく
\ruby{耳}{みゝ}に
\ruby{響}{ひび}くを
\ruby{聞}{き}きては、もう
\ruby{薄腹}{うす|はら}の
\ruby{立}{た}つほど
\ruby{蟲}{むし}が
\ruby{{\換字{嫌}}}{きら}つて
\ruby{厭}{いや}で〳〵
\ruby{堪}{たま}らず、
\ruby{出}{で}ずとも
\ruby{可}{い}い
\ruby{人}{ひと}が
\ruby{出}{で}て
\ruby{來}{き}てと
\ruby{迷惑}{めい|わく}がりて、
\ruby{出}{で}るも
\ruby{引}{ひ}くもならぬに
\ruby{心}{こゝろ}そげて
\ruby{居}{ゐ}たりしが、
お
\ruby{彤}{とう}に
\ruby{斯}{か}く
\ruby{云}{い}はれては
\ruby{横}{よこ}を
\ruby{向}{む}いてばかりも
\ruby{居}{ゐ}られず、
\ruby[g]{不承々々}{ふしよう〴〵}に、

『
\ruby{叔母}{を|ば}さん……』

と
\ruby{云}{い}ひし
\ruby{限}{ぎ}り、あとはぐず〴〵と
\ruby{口}{くち}の
\ruby{内}{うち}にて
\ruby{何}{なに}を
\ruby{云}{い}ひしやら
\ruby{知}{し}れず、
\ruby[g]{術無}{じゆつな}げに
\ruby{頭}{かしら}を
\ruby{下}{さ}げて
\ruby{漸}{やつ}と
\ruby{挨拶}{あい|さつ}すれば、
\ruby{叔母}{を|ば}はなか〳〵もう
\ruby{默}{だま}つては
\ruby{居}{ゐ}ず、
\ruby{三角}{さん|かく}の
\ruby{眼}{め}をきらりと
\ruby{光}{ひか}らせ、

『でもまあ
\ruby{能}{よ}く
\ruby{忘}{わす}れずに
\ruby{叔母}{を|ば}さんと
\ruby{御云}{お|い}ひだつたネ。
ハイ、
\ruby{其後}{その|のち}は\換字{志}ばらく。
お
\ruby{前}{まへ}も
\ruby[g]{御達者}{おたつしや}で、
\ruby{別}{べつ}に
\ruby[g]{御天{\換字{道}}樣}{おてんたうさま}にも
\ruby{愛想}{あい|そ}を
\ruby{盡}{つ}かされずに
\ruby{御暮}{お|くら}しで、まあ
\ruby{結構}{けつ|かう}だネ。
まことにお
\ruby{前}{まへ}の
\ruby{御蔭}{お|かげ}ぢやあ
\ruby{恐}{おそ}ろしい
\ruby[g]{沸湯}{にえゆ}を
\ruby{飮}{の}ませられました。
\ruby{會}{あ}つたら
\ruby{引捉}{ひつ|つかま}へて
\ruby{耳}{みゝ}でも
\ruby{扯}{ちぎ}り
\ruby{取}{と}つてあげて、
\ruby{何}{ど}の
\ruby{位妾}{くらゐ|わたし}が
\ruby{痛}{いた}かつたか
\ruby{苦}{くる}しかつたか、
\ruby{此樣}{こ|ん}なものだつたよと、
\ruby{察}{さつ}して
\ruby{貰}{もら}ひましやうと
\ruby{思}{おも}つて
\ruby{居}{ゐ}ましたがネ、
\ruby{此方樣}{こち|ら|さま}の
\ruby{御言葉}{お|こと|ば}だから
\ruby{堪忍}{かん|にん}してあげる。
\換字{志}かし
\ruby{彼}{あ}の
\ruby{事}{こと}は
\ruby{何樣}{ど|う}か
\ruby{此樣}{か|う}か
\ruby{既濟}{もう|す}んで
\ruby{仕舞}{し|ま}つたが、
\ruby{一}{ひと}つ
\ruby{濟}{す}めば
\ruby[g]{{\換字{又}}一}{またひと}つで
お
\ruby{前}{まへ}の
\ruby[g]{御蔭樣}{おかげさま}で、
\ruby{斯樣}{か|う}して
\ruby[g]{砂塵}{すなつぼこり}ばかり
\ruby{立}{た}つ
\ruby{東京}{とう|きやう}くんだりへ、
\ruby{田舍婆}{ゐな|か|ばあ}さんがゑつちらおつちらと
\ruby[g]{得々出}{わざ〳〵で}かけて
\ruby{來}{き}て、
\ruby{此方樣}{こち|ら|さま}へも
\ruby{御厄介}{ご|やく|かい}を
\ruby{掛}{か}けたりなんぞ
\ruby{仕}{し}ます。
\ruby{婆}{ばあ}さんを
\ruby{苦勞}{く|らう}ばかりさせて
\ruby[g]{御手柄}{おてがら}の
\ruby{事}{こと}ですネ。
ほんとにお
\ruby{前}{まへ}の
\ruby{仕}{し}た
\ruby{事}{こと}に
\ruby{碌}{ろく}な
\ruby{事}{こと}は
\ruby{有}{あ}りやあ
\ruby{仕}{し}ない。
お
\ruby{前}{まへ}の
\ruby{仕}{し}た
\ruby{事}{こと}の
\ruby{中}{うち}で
\ruby{好}{い}い
\ruby{事}{こと}といふのは、
\ruby{此方樣}{こち|ら|さま}に
\ruby{可愛}{か|はい}がつて
\ruby{頂}{いたゞ}いて
\ruby{居}{ゐ}るといふ
\ruby{事}{こと}ばつかりだ。
\ruby{此方樣}{こち|ら|さま}にでも
\ruby[g]{見離}{みはな}されりやあ
お
\ruby{前}{まへ}のやうなものは、それこそ
\ruby[g]{最{\換字{終}}}{しまひ}は
\ruby{倒}{のた}れ
\ruby{死}{じに}だよ。

\ruby{身}{み}に
\ruby{染}{し}みて
\ruby{覺}{おぼ}えておいでなさい、もう
お
\ruby{前}{まへ}の
\ruby{身體}{から|だ}は
お
\ruby{前}{まへ}の
\ruby{料簡}{れう|けん}ぢやあ
\ruby{{\換字{勝}}手}{かつ|て}にはなりません。
\ruby{妾}{わたし}がすつかりと
\ruby{願}{ねが}つて
\ruby{置}{お}きました。
もう
\ruby{何}{なに}も
\ruby{彼}{か}も
\ruby{此方樣}{こち|ら|さま}の
\ruby{仰}{おつし}やる
\ruby{{\換字{通}}}{とほ}りにするのです。
\ruby{三絃}{さみ|せん}の
\ruby{師匠}{し|しやう}だなんて、
\ruby[g]{彼樣惡}{あんなわる}い
\ruby{人}{ひと}のところへ、
\ruby{身}{み}を
\ruby{置}{お}いては
\ruby{決}{けつ}してなりません、
\ruby{出入}{で|はい}りしてもなりません。
\ruby{早{\換字{速}}}{さつ|そく}これから
\ruby{其家}{そ|こ}を
\ruby{出}{で}て
\ruby{此方}{こち|ら}へ
\ruby{御厄介}{ご|やく|かい}になつて、
\ruby{此方樣}{こち|ら|さま}を
\ruby{有}{あ}り
\ruby{難}{がた}いとおもつて
\ruby{身}{み}を
\ruby{責}{せ}めて
\ruby[g]{御働}{おはたら}きなさい。
』

と
\ruby{獨}{ひと}り
\ruby{合點}{が|てん}して、まくし
\ruby{立}{た}てゝ
\ruby{指揮}{さし|ず}したり。

お
\ruby{彤}{とう}は
\ruby{訝}{いぶか}り
\ruby{疑}{うたが}ふ
お
\ruby{龍}{りう}を
\ruby{見}{み}て、

『
\ruby{叔母}{を|ば}さん、
\ruby{其}{それ}ぢやあ
\ruby{此}{こ}の
\ruby{人}{ひと}
にやあ
\ruby{{\換字{分}}}{わか}りますまい。
かういふ
\ruby{事}{こと}なのだよ
お
\ruby{龍}{りう}ちやん。
』

と
\ruby{靜}{しづか}に
\ruby{{\換字{説}}}{と}き
\ruby{出}{いだ}したり。

