\Entry{其二十三}

\原本頁{}%
お
\ruby{彤}{とう}は
お
\ruby{{\換字{近}}}{ちか}が
\ruby{言}{ものい}へる
\ruby{間}{あひだ}にも、
%
\ruby{少}{すこ}しの
\ruby{受答}{うけ|こた}へを
\ruby{爲}{し}つ、
%
\ruby{語}{くち}を
\ruby{挿}{はさ}まんとせざるにはあらざりしも、
%
\ruby{立板}{たて|いた}に
\ruby{水}{みづ}とはいふべきならねど
\ruby{下}{くだ}り
\ruby{坂}{ざか}に
\ruby{走}{はし}る
\ruby{小車}{を|ぐるま}のやうに
\ruby{騷}{さわ}がしく
\ruby{忙}{せは}しく
\ruby{話}{はな}しつゞけられて
\ruby{口}{くち}を
\ruby{入}{い}れ
\ruby{{\換字{兼}}}{か}ね
\ruby{居}{ゐ}しが、
%
\ruby{今}{いま}
\ruby{斯}{か}く
\ruby{問}{と}ひかけられて
\ruby{僅}{わづか}に
\ruby{言葉}{こと|ば}を
\ruby{出}{いだ}し、

\原本頁{}%
『いゝえ
\ruby{然樣}{さ|う}ぢやあ
\ruby{有}{あ}りませんが
\ruby{他}{ほか}の
\ruby{事}{こと}でもつて、
%
\ruby{丁度}{ちやう|ど}
\ruby{自然}{ひと|りで}に
\ruby{先刻方見}{さつ|き|がた|み}えたので、
』

\原本頁{}%
と
\ruby{云}{い}ひかけて
お
\ruby{龍}{りう}の
\ruby{方}{はう}を
\ruby{莞爾}{に|こ}やかに
\ruby{見}{み}やり、

\原本頁{}%
『お
\ruby{龍}{りう}ちやん
お
\ruby{{\換字{前}}}{まへ}、
%
\ruby{默}{だま}つておいでぢやあ
\ruby{不可}{いけ|ない}よ、
%
\ruby{叔母}{を|ば}さんぢやあ
\ruby{無}{な}いかネ。
』

\原本頁{}%
と
\ruby{輕}{かろ}き
\ruby{一句}{いつ|く}を
\ruby{與}{あた}へつ、
%
また
お
\ruby{{\換字{近}}}{ちか}に
\ruby{向}{むか}ひて、

\原本頁{}%
『きまりが
\ruby{惡}{わる}いもので
\ruby{羞澁}{はに|か}んで
\ruby{困}{こま}つて
\ruby{居}{ゐ}るのですよ。
%
ホヽヽまだ
\ruby{{\換字{若}}}{わか}くつて、
%
いつそ
\ruby{可憐}{か|はい}らしいぢやあ
\ruby{有}{あ}りませんか。
%
どうかまあ
\ruby{今日}{け|ふ}のところは
\ruby{御叱}{お|しか}りなさらないでネ、
%
\ruby{貴卿}{あな|た}が
\ruby{御目上}{お|め|うへ}ですから
\ruby{優}{やさ}しく
\ruby{仕}{し}て
\ruby{御與}{お|や}りなすつてネ。
』

\原本頁{}%
と、
%
\ruby{二人}{ふた|り}の
\ruby{間}{あひだ}をば
\ruby{取}{と}り
\ruby{繕}{つくろ}ふやうに
\ruby{云}{い}へり。

\原本頁{}%
\ruby{此}{こ}の
\ruby{叔母}{を|ば}が
\ruby{擇}{えら}み
\ruby{定}{さだ}めし
\ruby{婿}{むこ}を% (婿 5a7f) 聟 805f
\ruby{{\換字{嫌}}}{きら}ひしより、
%
\ruby{{\換字{朝}}}{あさ}となく
\ruby{夜}{よる}と
\ruby{無}{な}く
\ruby{論}{い}ひ
\ruby{合}{あ}ひ
\ruby{睨}{にら}み
\ruby{合}{あ}ひて、
%
さらだに
\ruby{性}{しやう}の
\ruby{合}{あ}はぬ
\ruby{中}{なか}の、いよ〳〵おもしろからず、えゝ、
%
あた
\ruby{忌々}{いま|〳〵}しい、
%
\ruby{何}{なん}となるものぞと、
%
\ruby{後}{あと}の
\ruby{{\換字{迷}}惑}{めい|わく}も
\ruby{思}{おも}はずに
\ruby{無言}{だ|ま}つて
\ruby{駈}{か}け
\ruby{出}{だ}したるまゝ、
%
\ruby{恩}{おん}のある
\ruby{事}{こと}は
\ruby{知}{し}つて
\ruby{居}{ゐ}れど
\ruby{憎}{にく}らしさもあるに、
%
\ruby{手紙}{て|がみ}
\ruby{一本}{いつ|ぽん}も
\ruby{出}{だ}さで
\ruby{知}{し}らぬ
\ruby{顏}{かほ}に
\ruby{濟}{す}まし
\ruby{來}{きた}りし
\ruby{今日}{け|ふ}、
%
\ruby{突然}{だし|ぬけ}に
\ruby{此處}{こ|ゝ}に
\ruby{相}{あひ}
\ruby{會}{あ}ひては
お
\ruby{龍}{りう}も
\ruby{聊}{いさゝ}か
\ruby{驚}{おどろ}きつ、
%
\ruby{顏}{かほ}を
\ruby{見}{み}ては
\ruby{流石}{さす|が}
\ruby{氣}{き}の
\ruby{毒}{どく}さに
\ruby{面伏}{おも|ぶせ}の
\ruby{思}{おも}ひもすれど、
%
\ruby{{\換字{勝}}手}{かつ|て}のみ
\ruby{{\換字{強}}}{つよ}くして
\ruby{{\換字{遠}}慮}{ゑん|りよ}を
\ruby{知}{し}らぬ
\ruby{性急}{せつ|かち}の
\ruby{話聲}{はなし|ごゑ}の、
%
いつもながら
\ruby{喧}{やかま}しく
\ruby{耳}{みゝ}に
\ruby{響}{ひゞ}くを
\ruby{聞}{き}きては、
%
もう
\ruby{薄腹}{うす|はら}の
\ruby{立}{た}つほど
\ruby{蟲}{むし}が
\ruby{{\換字{嫌}}}{きら}つて
\ruby{厭}{いや}で〳〵
\ruby{堪}{たま}らず、
%
\ruby{出}{で}ずとも
\ruby{可}{い}い
\ruby{人}{ひと}が
\ruby{出}{で}て
\ruby{來}{き}てと
\ruby{{\換字{迷}}惑}{めい|わく}がりて、
%
\ruby{出}{で}るも
\ruby{引}{ひ}くもならぬに
\ruby{心}{こゝろ}そげて
\ruby{居}{ゐ}たりしが、
%
お
\ruby{彤}{とう}に
\ruby{斯}{か}く
\ruby{云}{い}はれては
\ruby{横}{よこ}を
\ruby{向}{む}いてばかりも
\ruby{居}{ゐ}られず、
%
\ruby{不承}{ふ|しよう}
\g詰めruby{々々}{〴〵}に、

\原本頁{}%
『
\ruby{叔母}{を|ば}さん‥‥』

\原本頁{}%
と
\ruby{云}{い}ひし
\ruby{限}{ぎ}り、
%
あとはぐず〴〵と
\ruby{口}{くち}の
\ruby{内}{うち}にて
\ruby{何}{なに}を
\ruby{云}{い}ひしやら
\ruby{知}{し}れず、
%
\ruby{{\換字{術}}無}{じゆつ|な}げに
\ruby{頭}{かしら}を
\ruby{下}{さ}げて
\ruby{漸}{やつ}と
\ruby{挨拶}{あい|さつ}すれば、
%
\ruby{叔母}{を|ば}は
なか〳〵もう
\ruby{默}{だま}つては
\ruby{居}{ゐ}ず、
%
\ruby{三角}{さん|かく}の
\ruby{眼}{め}をきらりと
\ruby{光}{ひか}らせ、

\原本頁{}%
『でもまあ
\ruby{能}{よ}く
\ruby{忘}{わす}れずに
\ruby{叔母}{を|ば}さんと
\ruby{御云}{お|い}ひだつたネ。
%
ハイ、
%
\ruby{其}{その}
\ruby{後}{ゝち}は
\換字{志}ばらく。
%
お
\ruby{{\換字{前}}}{まへ}も
\ruby{御{\換字{達}}者}{お|たつ|しや}で、
%
\ruby{別}{べつ}に
\ruby{御天{\換字{道}}樣}{お|てん|たう|さま}にも
\ruby{愛想}{あい|そ}を
\ruby{盡}{つ}かされずに
\ruby{御暮}{お|くら}しで、
%
まあ
\ruby{結構}{けつ|こう}だネ。
%
まことにお
\ruby{{\換字{前}}}{まへ}の
\ruby{御蔭}{お|かげ}ぢやあ
\ruby{恐}{おそ}ろしい
\ruby{沸湯}{にえ|ゆ}を
\ruby{飮}{の}ませられました。
%
\ruby{會}{あ}つたら
\ruby{引捉}{ひつ|つかま}へて
\ruby{耳}{みゝ}でも
\ruby{扯}{ちぎ}り
\ruby{取}{と}つてあげて、
%
\ruby{何}{ど}の
\ruby[<j|]{位}{くらゐ}
\ruby{妾}{わたし}が
\ruby{痛}{いた}かつたか
\ruby{苦}{くる}しかつたか、
%
\ruby{此樣}{こ|ん}なものだつたよと、
%
\ruby{察}{さつ}して
\ruby{貰}{もら}ひましやうと
\ruby{思}{おも}つて
\ruby{居}{ゐ}ましたがネ、
%
\ruby{此方樣}{こち|ら|さま}の
\ruby{御言葉}{お|こと|ば}だから
\ruby{堪{\換字{忍}}}{かん|にん}してあげる。% 原文通り「堪忍」
%
\換字{志}かし
\ruby{彼}{あ}の
\ruby{事}{こと}は
\ruby{何樣}{ど|う}か
\ruby{此樣}{か|う}か% 原文通りルビは「かう」とする
\ruby{既}{もう}
\ruby{濟}{す}んで
\ruby{仕舞}{し|ま}つたが、
%
\ruby{一}{ひと}つ
\ruby{濟}{す}めば
\ruby{{\換字{又}}}{また}
\ruby{一}{ひと}つで
お
\ruby{{\換字{前}}}{まへ}の
\ruby{御蔭樣}{お|かげ|さま}で、
%
\ruby{斯樣}{か|う}して
\ruby{砂塵}{すなつ|ぼこり}ばかり
\ruby{立}{た}つ
\ruby{東京}{とう|きやう}くんだりへ、
%
\ruby{田舎}{ゐな|か}
\ruby{婆}{ばあ}さんがゑつちらおつちらと
\ruby{得々}{わざ|〳〵}
\ruby{出}{で}かけて
\ruby{來}{き}て、
%
\ruby{此方樣}{こち|ら|さま}へも
\ruby{御厄介}{ご|やく|かい}を
\ruby{掛}{か}けたりなんぞ
\ruby{仕}{し}ます。
%
\ruby{婆}{ばあ}さんを
\ruby{苦勞}{く|らう}ばかりさせて
\ruby{御手柄}{お|て|がら}の
\ruby{事}{こと}ですネ。
%
ほんとにお
\ruby{{\換字{前}}}{まへ}の
\ruby{仕}{し}た
\ruby{事}{こと}に
\ruby{碌}{ろく}な
\ruby{事}{こと}は
\ruby{有}{あ}りやあ
\ruby{仕}{し}ない。
%
お
\ruby{{\換字{前}}}{まへ}の
\ruby{仕}{し}た
\ruby{事}{こと}の
\ruby{中}{うち}で
\ruby{好}{い}い
\ruby{事}{こと}といふのは、
%
\ruby{此方樣}{こち|ら|さま}に
\ruby{可愛}{か|はい}がつて
\ruby{頂}{いたゞ}いて
\ruby{居}{ゐ}るといふ
\ruby{事}{こと}ばつかりだ。
%
\ruby{此方樣}{こち|ら|さま}にでも
\ruby{見離}{み|はな}されりやあ
お
\ruby{{\換字{前}}}{まへ}のやうなものは、
%
それこそ
\ruby{最{\換字{終}}}{しま|ひ}は
\ruby{倒}{のた}れ
\ruby{死}{じに}だよ。

\原本頁{}%
\ruby{身}{み}に
\ruby{染}{し}みて
\ruby{覺}{おぼ}えておいでなさい、
%
もう
お
\ruby{{\換字{前}}}{まへ}の
\ruby{身體}{から|だ}は
お
\ruby{{\換字{前}}}{まへ}の
\ruby{料簡}{れう|けん}ぢやあ
\ruby{{\換字{勝}}手}{かつ|て}にはなりません。
%
\ruby{妾}{わたし}がすつかりと
\ruby{願}{ねが}つて
\ruby{置}{お}きました。
%
もう
\ruby{何}{なに}も
\ruby{彼}{か}も
\ruby{此方樣}{こち|ら|さま}の
\ruby{仰}{おつし}やる
\ruby{{\換字{通}}}{とほ}りにするのです。
%
\ruby{三絃}{さみ|せん}の
\ruby{師匠}{し|ゝやう}だなんて、
%
\ruby{彼樣惡}{あん|な|わる}い
\ruby{人}{ひと}のところへ、
%
\ruby{身}{み}を
\ruby{置}{お}いては
\ruby{決}{けつ}してなりません、
%
\ruby{出入}{で|はい}りしてもなりません。
%
\ruby{早{\換字{速}}}{さつ|そく}これから
\ruby{其家}{そ|こ}を
\ruby{出}{で}て
\ruby{此方}{こち|ら}へ
\ruby{御厄介}{ご|やく|かい}になつて、
%
\ruby{此方樣}{こち|ら|さま}を
\ruby{有}{あ}り
\ruby{{\換字{難}}}{がた}いとおもつて
\ruby{身}{み}を
\ruby{責}{せ}めて
\ruby{御働}{お|はたら}きなさい。
』

\原本頁{}%
と
\ruby{獨}{ひと}り
\ruby{合點}{が|てん}して、
%
まくし
\ruby{立}{た}てゝ
\ruby{指揮}{さし|づ}したり。

\原本頁{}%
お
\ruby{彤}{とう}は
\ruby{訝}{いぶか}り
\ruby{疑}{うたが}ふ
お
\ruby{龍}{りう}を
\ruby{見}{み}て、

\原本頁{}%
『
\ruby{叔母}{を|ば}さん、
%
\ruby{其}{それ}ぢやあ
\ruby{此}{こ}の
\ruby{人}{ひと}
にやあ
\ruby{{\換字{分}}}{わか}りますまい。
%
かういふ
\ruby{事}{こと}なのだよ
お
\ruby{龍}{りう}ちやん。
』

\原本頁{}%
と
\ruby{靜}{しづか}に
\ruby{說}{と}き
\ruby{出}{いだ}したり。
