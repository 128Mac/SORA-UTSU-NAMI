\Entry{其二十四}

\原本頁{}%
『
\ruby{最初}{さい|しよ}つから
\ruby{云}{い}ふと
\ruby{如是}{か|う}なのだよ
お
\ruby{龍}{りう}ちやん。
%
それ
\ruby{一昨年}{を|とゝ|し}の
\ruby{夏}{なつ}の
\ruby{事}{こと}だつたね、
%
これこれで
\ruby{此度}{こん|ど}
\ruby{叔母}{を|ば}に
\ruby{{\換字{伴}}}{つ}れられて、
%
\ruby{厭}{いや}だけれども
\ruby{靜岡}{しづ|をか}へ
\ruby{行}{ゆ}きますからつて、
%
お
\ruby{{\換字{前}}}{まへ}が
\ruby{暇乞}{いとま|ごひ}に
\ruby{御}{お}いでだつたことがあつた、
%
\ruby{其時}{そ|れ}からといふものは
\ruby{隨{\換字{分}}}{ずゐ|ぶん}
\ruby{長}{なが}い
\ruby{間}{あひだ}、
%
\ruby{此方}{こつ|ち}から
\ruby{手紙}{て|がみ}をあげても
\ruby{{\換字{返}}辭}{へん|じ}は
\ruby{少}{すくな}いし、
%
たまに
\ruby{御{\換字{遣}}}{お|よこ}しでも
\ruby{極々短}{ごく|〳〵|みじ}つかい
\ruby{眞}{ほん}の
\ruby{義理}{ぎ|り}
\ruby{濟}{す}ましだけの
\ruby{事}{こと}だし、
%
\ruby{是}{これ}あ
\ruby{何}{なに}か
\ruby{知}{し}らないけれども
\ruby{甚}{ひど}く
\ruby{氣}{き}を
\ruby{取}{と}れておいでの
\ruby{事}{こと}があるのだらう、
%
と
\ruby{思}{おも}つて
\ruby{居}{ゐ}る
\ruby{中}{うち}に
\ruby{今年}{こ|とし}の
\ruby{三月}{さん|ぐわつ}、
%
ふらりつと
\ruby{妾}{わたし}の
\ruby{處}{ところ}へ
\ruby{御}{お}いでだつたが、
%
\ruby{顏付}{かほ|つき}は
\ruby{全然}{まる|で}
\ruby{變}{かは}つて
\ruby{仕舞}{し|ま}つて、
%
\ruby{{\換字{前}}}{まへ}に
\ruby{見}{み}た
\ruby{處女}{むす|め}らしいところは
\ruby{無}{な}くなつて
\ruby{御{\換字{終}}}{お|しま}ひだし、
%
\ruby{樣子}{やう|す}は
\ruby{何}{なん}だか
\ruby{知}{し}らないがそは〳〵としておいでゞ、
%
\ruby{妾}{わたし}に
\ruby{御話}{お|はなし}の
\ruby{談話}{はな|し}にも
\ruby{辻褄}{つぢ|つま}の
\ruby{合}{あ}はないところは
\ruby{有}{あ}り、
%
\ruby{何樣}{ど|う}も
\ruby{氣}{き}になる
\ruby{事}{こと}ばかしだから
\ruby{妾}{わたし}は
\ruby{心配}{しん|ぱい}して、
%
すこし
\ruby{置}{お}いて
\ruby{吳}{く}れと
\ruby{御言}{お|い}ひのことだから、
%
あゝ
\ruby{宜}{い}いともと、
%
\ruby{表面}{うは|べ}は
\ruby{何}{なん}の
\ruby{氣}{き}もつかない
\ruby{風}{ふう}で
\ruby{家}{うち}へは
\ruby{置}{お}いて
\ruby{{\換字{進}}}{あ}げたものゝ、
%
\ruby{何樣}{ど|ん}なにいろ〳〵と
\ruby{物}{もの}をおもつたか
\ruby{知}{し}れないよ。
%
\ruby{此處}{こ|ゝ}に
\ruby{居}{ゐ}ることを
\ruby{靜岡}{しづ|をか}へ
\ruby{知}{し}らせては
\ruby{吳}{く}れるなと、
%
\ruby{念}{ねん}に
\ruby{念}{ねん}を
\ruby{押}{お}しての
\ruby{御依頼}{お|たの|み}だつたけれども、
%
\ruby{今}{いま}
\ruby{白狀}{はく|じやう}して
お
\ruby{{\換字{前}}}{まへ}に
\ruby{謝罪}{あや|ま}るがネ、
%
\ruby{何樣}{ど|う}も
\ruby{物}{もの}の
\ruby{{\換字{道}}理}{だう|り}が
\ruby{然樣}{さ|う}は
\ruby{行}{い}かないと
\ruby{思}{おも}つたので、
%
お
\ruby{{\換字{前}}}{まへ}には
\ruby{内密}{ない|しよ}でもつて
\ruby{靜岡}{しづ|をか}の
\ruby{叔母}{を|ば}さんへ、
%
これ〳〵の
\ruby{樣子}{やう|す}で、
%
\ruby{如是}{か|う}
\g詰めruby{々々}{〳〵}して
お
\ruby{龍}{りう}ちやんは
\ruby{妾}{わたし}の
\ruby{方}{はう}に
\ruby{御}{お}いでだと、
%
\ruby{妾}{わたし}が
\ruby{全然}{すつ|かり}
\ruby{知}{し}らせて
\ruby{仕舞}{し|ま}つたのだよ。
』

\原本頁{}%
\ruby{此}{これ}まで
\ruby{語}{かた}り
\ruby{掛}{か}けし
\ruby{時}{とき}、
%
\ruby{叔母}{を|ば}は
お
\ruby{龍}{りう}を
\ruby{見}{み}て、

\原本頁{}%
『それ
\ruby{御覽}{ご|らん}。
%
\ruby{汝}{おまへ}のやうな
\ruby{{\換字{分}}}{わか}らないものゝ
\ruby{云}{い}ふ
\ruby{事}{こと}や
\ruby{思}{おも}ふことばかりが
\ruby{何}{なん}で
\ruby{{\換字{通}}}{とほ}るものかエ。
%
\ruby{此方樣}{こち|ら|さま}のやうな
\ruby{方}{かた}は
\ruby{何程}{いく|ら}
\ruby{御優}{お|やさ}しくつても、
%
\ruby{角々}{かど|〳〵}は
\ruby{嚴然}{きつ|ぱり}と
\ruby{{\換字{道}}理}{だう|り}のある
\ruby{方}{はう}へ
\ruby{御就}{お|つ}きになる!。
%
お
\ruby{{\換字{前}}}{まへ}は
\ruby{知}{し}らないで
\ruby{好}{い}い
\ruby{氣}{き}になつておいでだつたらうが、
%
ちやんと
\ruby{妾}{わたし}の
\ruby{方}{はう}へ
\ruby{御知}{お|し}らせくだすつて、
%
いろ〳〵と
\ruby{御注意}{お|こゝろ|づけ}まで
\ruby{仕}{し}て
\ruby{下}{くだ}すつたのだ。
%
\ruby{七{\換字{分}}{\換字{通}}}{しち|ぶ|どほ}り
\ruby{八{\換字{分}}{\換字{通}}}{はち|ぶ|どほ}り
\ruby{話}{はなし}の
\ruby{定}{きま}つた
\ruby{婿}{むこ}を
\ruby{{\換字{嫌}}}{きら}つて
お
\ruby{{\換字{前}}}{まへ}には
\ruby{出}{で}られる、
%
\ruby{何處}{ど|こ}へ
\ruby{行}{い}つたかもかいくれ
\ruby{知}{し}れず、
%
また
\ruby{短氣}{たん|き}を
\ruby{仕}{し}て
\ruby{{\換字{若}}}{も}しや
\ruby{淵川}{ふち|かは}へでもかと
\ruby{何程}{どれ|ほど}
\ruby{妾}{わたし}が
\ruby{苦勞}{く|らう}して
\ruby{困}{こま}り
\ruby{拔}{ぬ}いたか
\ruby{知}{し}れない、
%
\ruby{其處}{そ|こ}へ
\ruby{此方樣}{こち|ら|さま}からの
\ruby{行屆}{ゆき|とゞ}いた% 「屆」「届」 原本通り「屆」
\ruby{御手紙}{お|て|がみ}で、
%
やつと
\ruby{胸}{むね}の
\ruby{凝塊}{かた|まり}がすこし
\ruby{下}{さが}つた。
%
\ruby{居{\換字{所}}}{ゐ|どこ}は
\ruby{知}{し}れたし、
%
\ruby{引捉}{ひつ|つかま}へてとも
\ruby{思}{おも}はないでは
\ruby{無}{な}かつたが、
%
\ruby{何樣}{ど|う}せ
\ruby{其程}{それ|ほど}
\ruby{{\換字{嫌}}}{きら}つて
\ruby{居}{ゐ}る
\ruby{婿}{むこ}ならば、
%
\ruby{仕方}{し|かた}がないからいつそ
\ruby{破談}{は|だん}になすつたが
\ruby{宜}{よ}からうし、
%
\ruby{破談}{は|だん}になさるなら
\ruby{{\換字{又}}}{また}
\ruby{當人}{たう|にん}が
\ruby{其地}{そち|ら}に
\ruby{居}{ゐ}ないで、
%
\ruby{何處}{ど|こ}へ
\ruby{行}{い}つたか
\ruby{知}{し}れないといふ
\ruby{{\換字{分}}}{ぶん}になすつた
\ruby{方}{はう}が、
%
\ruby{事}{こと}が
\ruby{濟}{す}み
\ruby{易}{やす}からうし、
%
\ruby{{\換字{若}}}{も}し
\ruby{{\換字{強}}}{し}ひて
\ruby{無理}{む|り}な
\ruby{事}{こと}をなさるやうでは
\ruby{當人}{たう|にん}の
\ruby{爲}{ため}にも、
%
\ruby{却}{かへ}つてならないやうな
\ruby{事}{こと}になりは
\ruby{爲}{し}まいかと
\ruby{思}{おも}はれるから、
%
\ruby{次第}{し|だい}によつたら
\ruby{姑}{しばら}く
\ruby{此儘}{この|まゝ}
\ruby{御預}{お|あづ}かり
\ruby{申}{まを}しても
\ruby{宜}{よ}い、
%
と
\ruby{能}{よ}く
\ruby{{\換字{分}}}{わか}つた
\ruby{此方樣}{こち|ら|さま}の
\ruby{御親切}{ご|しん|せつ}な
\ruby{御仰}{お|つし}ありやうでもあり、
%
また
\ruby{此方樣}{こち|ら|さま}の
\ruby{御噂}{お|うはさ}も
\ruby{豫}{かね}て
\ruby{聞}{き}いて
\ruby{何樣}{ど|う}いふ
\ruby{方}{かた}かと
\ruby{合點}{が|てん}しても
\ruby{居}{ゐ}たので、
%
とても
\ruby{妾}{わたし}には
\ruby{制{\換字{道}}}{せい|だう}の
\ruby{就}{つ}きません
\ruby{我儘者}{わが|まゝ|もの}でございますから
\ruby{既}{もう}
\ruby{諦}{あき}らめました、
%
\ruby{御甘}{お|あま}え
\ruby{申}{まを}しては
\ruby{濟}{す}みませんが
\ruby{然樣}{さ|う}いふ
\ruby{譯}{わけ}でございますれば、
%
\ruby{此方}{こち|ら}の
\ruby{話}{はなし}も
\ruby{解}{と}けて
\ruby{濟}{す}んで
\ruby{仕舞}{し|ま}ふまで
\ruby{御預}{お|あづ}かりを
\ruby{願}{ねが}ひます、
%
\ruby{成程}{なる|ほど}
\ruby{今}{いま}
\ruby{妾}{わたし}が
\ruby{出}{で}て
\ruby{參}{まゐ}りまして
\ruby{當人}{たう|にん}に
\ruby{會}{あ}つても
\ruby{何}{なん}にもなりますまいから、
%
\ruby{御{\換字{迷}}惑}{ご|めい|わく}でもござりましやうが
\ruby{其}{それ}では
\ruby{何{\換字{分}}}{なに|ぶん}
\ruby{宜}{よろ}しく
\ruby{願}{ねが}ひまする、
%
\ruby{{\換字{若}}}{も}し
\ruby{{\換字{又}}}{また}
\ruby{當人}{たう|にん}が
\ruby{不心得}{ふ|こゝろ|え}なぞを
\ruby{致}{いた}して、
%
\ruby{御厄介}{ご|やく|かい}を
\ruby{掛}{か}けまするやうなことがございますれば
\ruby{屹度}{きつ|と}
\ruby{引受}{ひき|う}けまする、
%
と
\ruby{斯樣}{か|う}いふ
\ruby{御挨拶}{ご|あい|さつ}を
\ruby{仕}{し}て
\ruby{願}{ねが}つて
\ruby{置}{お}いたのだ。
%
\ruby{今}{いま}
\ruby{解}{わか}つたかエ、
%
\ruby{妾}{わたし}の
\ruby{心持}{こゝろ|もち}も
\ruby{此方樣}{こち|ら|さま}の
\ruby{御思慮}{お|かん|がへ}も。
%
それほど
\ruby{妾}{わたし}にも
\ruby{此方樣}{こち|ら|さま}にも
\ruby{人知}{ひと|し}れず
\ruby{氣}{き}を
\ruby{揉}{も}ませて
\ruby{置}{お}いて、
%
それだのに
\ruby{何}{なん}だエ、
%
\ruby{月日}{つき|ひ}も
\ruby{經}{たゝ}ない
\ruby{中}{うち}に
\ruby{{\換字{又}}}{また}
\ruby{此方樣}{こち|ら|さま}を
\ruby{駈}{か}け
\ruby{出}{だ}して、{---}{---}
\ruby{妹}{いもと}のやうに
\ruby{思}{おも}ふ
\ruby{子}{こ}のやうに
\ruby{思}{おも}ふとまで
\ruby{云}{い}つてくださる
\ruby{此方樣}{こち|ら|さま}の
\ruby{御親切}{ご|しん|せつ}も、
%
\ruby{妾}{わたし}は
お
\ruby{{\換字{前}}}{まへ}の
\ruby[g]{眞實}{ほんと}の
\ruby{叔母}{を|ば}だけれども
\ruby{然樣}{さ|う}は
\ruby{濃}{こま}かに
お
\ruby{{\換字{前}}}{まへ}のためを
\ruby{思}{おも}ふことは
\ruby{出來}{で|き}ないと
\ruby{我}{が}の
\ruby{折}{を}れるほどに
\ruby{仕}{し}て
\ruby{下}{くだ}さる
\ruby{有}{あ}り
\ruby{難}{がた}い
\ruby{此方樣}{こち|ら|さま}の
\ruby{御恩}{ご|おん}をも
\ruby{全}{まる}で
\ruby{餘{\換字{所}}}{よ|そ}にして、
%
\ruby{何}{なに}が
\ruby{不足}{ふ|そく}で
\ruby{無言}{だん|まり}で
\ruby{三絃}{さみ|せん}の
\ruby{師匠}{し|ゝやう}だなんて
\ruby{彼}{あ}んな
\ruby{惡}{わる}い
\ruby{奴}{やつ}のところへ
\ruby{行}{い}つた。
%
これ、
%
\ruby{何故}{な|ぜ}
\ruby{此方樣}{こち|ら|さま}を
\ruby{後}{あと}にして
\ruby{稽{\換字{古}}{\換字{所}}}{けい|こ|じよ}なんぞの
\ruby{手助}{て|だすけ}けを
\ruby{仕}{し}て
\ruby{自墮落}{じ|だ|らく}に
\ruby{暮}{くら}したのだエ。
%
\ruby{彼女}{あ|れ}あ
お
\ruby{{\換字{前}}}{まへ}、
%
お
\ruby{{\換字{前}}}{まへ}に
\ruby{碌}{ろく}でも
\ruby{無}{な}い
\ruby{男}{をとこ}なんぞを
\ruby{取}{と}り
\ruby{持}{も}つた
\ruby{狸婆}{たぬき|ばゞあ}ぢや
\ruby{無}{な}いか。
%
\ruby{性凝}{しやう|こ}りも
\ruby{無}{な}く、
%
まだ
\ruby{{\換字{浮}}氣}{うは|き}が
\ruby{仕}{し}たくつて、
%
\ruby{彼樣}{あ|ん}な
\ruby{奴}{やつ}に
\ruby[g]{末}{すゑ}
\ruby[g]{始{\換字{終}}}{すゑしじう}は% ルビは原本通り「ゆ」無し
\ruby{食}{く}はれるのも
\ruby{知}{し}らないで、
%
\ruby{此方樣}{こち|ら|さま}を
\ruby{出}{で}たのかエ。
%
\ruby{猫}{ねこ}!。
%
いやらしい
\ruby{猫}{ねこ}!。
%
ほんとにいやらしい
\ruby{猫}{ねこ}!。
%
\ruby{猫}{ねこ}だつて
\ruby{畜}{か}はれた
\ruby{恩}{おん}を
\ruby[g]{三日}{みつか}
\ruby{經}{た}つてから
\ruby{忘}{わす}れる、
%
\ruby{汝}{おまへ}あ
\ruby{畜}{か}はれて
\ruby{居}{ゐ}て
\ruby{可愛}{か|はい}がられて
\ruby{居}{ゐ}て
\ruby{既時}{す|ぐ}に
\ruby{忘}{わす}れたのだ。
%
\ruby{妾}{わたし}にも
\ruby{然樣}{さ|う}だつた、
%
\ruby{此方樣}{こち|ら|さま}にも
\ruby{然樣}{さ|う}だつた。
%
お
\ruby{{\換字{前}}}{まへ}のやうな
\ruby{好}{い}い
\ruby{姪}{めひ}をもつて
\ruby{人樣}{ひと|さま}の
\ruby{{\換字{前}}}{まへ}で、
%
\ruby{妾}{わたし}あほんとに
\ruby{肩身}{かた|み}が
\ruby{廣}{ひろ}くつて
\ruby{何樣}{ど|ん}なにか
\ruby{嬉}{うれ}しいよ。
』

\原本頁{}%
と、
%
\ruby{例}{れい}の
\ruby{眼}{め}を
\ruby{動}{うご}かし〳〵
\ruby{思}{おも}ふさまに
\ruby{罵}{のゝし}つたり。

