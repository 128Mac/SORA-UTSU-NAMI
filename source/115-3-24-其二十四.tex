\Entry{其二十四}

% メモ 校正終了 2024-05-14 2024-06-10
\原本頁{128-6}%
『
\ruby[g]{最初}{さいしよ}つから
\ruby{云}{い}ふと
\ruby[g]{如是}{か う }なのだよ
お
\ruby{龍}{りう}ちやん。
%
それ
\ruby[|g|]{一昨年}{をとゝし}の
\ruby{夏}{なつ}
の
\ruby{事}{こと}だつたね、
%
これこれで% ルビ調整(原本通り)非踊り字表記
\ruby[g]{此度}{こんど }
\ruby[g]{叔母}{を ば }に
\ruby{{\換字{伴}}}{つ}れられて、
%
\ruby{厭}{いや}だ
けれども
\ruby[g]{靜岡}{しづをか}へ
\ruby{行}{ゆ}きます
からつて、
%
お
\ruby{{\換字{前}}}{まへ}が
\ruby[||j>]{暇}{いとま}
\ruby[||j>]{乞}{ ごひ}に
% \ruby{暇乞}{いとま|ごひ}に
\ruby{御}{お}いで
だつた
ことが
あつた、
%
\ruby[g]{其時}{そ れ }から
といふものは
\ruby[g]{隨{\換字{分}}}{ずゐぶん}
\ruby{長}{なが}い
\ruby{間}{あひだ}、
%
\ruby[|g|]{此方}{こつち}から
\ruby[g]{手紙}{て がみ}を
あげても
\ruby[g]{{\換字{返}}辭}{へんじ }は
\ruby{少}{すくな}いし、
%
たまに
\ruby[g]{御{\換字{遣}}}{お よこ}し
でも
\ruby{極々短}{ごく|〳〵|みじ}つかい
\ruby{眞}{ほん}の
\ruby[g]{義理}{ぎ り }
\原本頁{129-1}\改行%
\ruby{濟}{す}ましだけの
\ruby{事}{こと}だし、
%
\ruby{是}{これ}あ
\ruby{何}{なに}か
\ruby{知}{し}らない
けれども
\ruby{甚}{ひど}く
\ruby{氣}{き}を
\ruby{取}{と}られて
おいでの
\ruby{事}{こと}
が
あるのだらう、
%
と
\ruby{思}{おも}つて
\ruby{居}{ゐ}る
\ruby{中}{うち}に
\ruby[g]{今年}{こ とし}の
\ruby[<j||]{三}{さん }
\ruby[<j||]{月}{ぐわつ}% 行末行頭の境界付近なので特例処置を施す
% \ruby{三月}{さん|ぐわつ}
\改行% 校正作業の簡略化のため
、
%
\原本頁{129-3}\改行%
ふらりつと
\ruby{妾}{わたし}の
\ruby{處}{ところ}へ
\ruby{御}{お}いで
だつたが、
%
\ruby[g]{顏付}{かほつき}は
\ruby[|g|]{全然}{まるで}
\ruby{變}{かは}つて
\ruby[g]{仕舞}{し ま }つて、
%
\ruby{{\換字{前}}}{まへ}に
\ruby{見}{み}た
\ruby[|g|]{處女}{むすめ}
らしい
ところは
\ruby{無}{な}く
なつて
\ruby[g]{御{\換字{終}}}{お しま}ひだし、
%
\ruby[g]{樣子}{やうす }は
\ruby{何}{なん}だか
\ruby{知}{し}らないが
そは〳〵
として
おいでゞ、
%
\ruby{妾}{わたし}に
\ruby[g]{御話}{お はな}しの
\ruby[g]{談話}{はなし }にも% ルビ調整(原本通り)非グループルビ
\ruby[g]{辻褄}{つぢつま}の
\ruby{合}{あ}はない
ところは
\ruby{有}{あ}り、
%
\ruby[g]{何樣}{ど う }も
\ruby{氣}{き}になる
\ruby{事}{こと}
ばかり
\原本頁{129-7}\改行%
だから
\ruby{妾}{わたし}は
\ruby[g]{心配}{しんぱい}して、
%
すこし
\ruby{置}{お}いて
\ruby{吳}{く}れと
\ruby[g]{御言}{お い }ひのことだから
\改行% 校正作業の簡略化のため
、
%
\原本頁{129-8}\改行%
あゝ
\ruby{宜}{い}いともと、
%
\ruby[|g|]{表面}{うはべ}は
\ruby{何}{なん}の
\ruby{氣}{き}もつかない
\ruby{風}{ふう}で
\ruby{家}{うち}へは
\ruby{置}{お}いて
\ruby{{\換字{進}}}{あ}げた
ものゝ、
%
\ruby[g]{何樣}{ど ん }なに
いろ〳〵と
\ruby{物}{もの}を
おもつたか
\ruby{知}{し}れないよ。
%
\原本頁{129-10}\改行%
\ruby[g]{此處}{こ ゝ }に
\ruby{居}{ゐ}ることを
\ruby[g]{靜岡}{しづをか}へ
\ruby{知}{し}らせては
\ruby{吳}{く}れるなと、
%
\ruby{念}{ねん}に
\ruby{念}{ねん}を
\ruby{押}{お}しての
\ruby[|g|]{御依頼}{おたのみ}
だつた
けれども、
%
\ruby{今}{いま}
\ruby[||j>]{白}{はく}
\ruby[||j>]{狀}{じやう}して
% \ruby{白狀}{はく|じやう}して
お
\ruby{{\換字{前}}}{まへ}に
\ruby[|g|]{謝罪}{あやま}るがネ、
%
\ruby[g]{何樣}{ど う }も
\ruby{物}{もの}の
\ruby[g]{{\換字{道}}理}{だうり }が
\ruby[g]{然樣}{さ う }は
\ruby{行}{い}かないと
\ruby{思}{おも}つたので、
%
お
\ruby{{\換字{前}}}{まへ}には
\ruby[g]{内密}{ないしよ}で
もつて
\ruby[g]{靜岡}{しづをか}の
\ruby[g]{叔母}{を ば }さんへ、
%
これ〳〵の
\ruby[g]{樣子}{やうす }で、
%
\ruby[g]{如是}{か う }
\ruby[g]{々々}{ 〳〵 }して
お
\ruby{龍}{りう}ちやんは
\ruby{妾}{わたし}の
\ruby{方}{はう}に
\ruby{御}{お}いでだと、
%
\ruby{妾}{わたし}が
\ruby[g]{全然}{すつかり}
\ruby{知}{し}らせて
\ruby[g]{仕舞}{し ま }つたのだよ。
』

\原本頁{130-5}%
\ruby{此}{これ}まで
\ruby{語}{かた}り
\ruby{掛}{か}けし
\ruby{時}{とき}、
%
\ruby[g]{叔母}{を ば }は
お
\ruby{龍}{りう}を
\ruby{見}{み}て、

\原本頁{130-6}%
『
それ
\ruby[g]{御覽}{ご らん}。
%
\ruby{汝}{おまへ}の
やうな
\ruby{{\換字{分}}}{わか}らない
ものゝ
\ruby{云}{い}ふ
\ruby{事}{こと}や
\ruby{思}{おも}ふ
こと
ばかりが
\ruby{何}{なん}で
\ruby{{\換字{通}}}{とほ}るものかエ。
%
\ruby[|g|]{此方}{こちら}
\ruby{樣}{さま}の
やうな
\ruby{方}{かた}は
\ruby[|g|]{何程}{いくら}
\ruby[g]{御優}{お やさ}しくつても、
%
\ruby[g]{角々}{かど〳〵}は
\ruby[g]{嚴然}{きつぱり}と
\ruby[g]{{\換字{道}}理}{だうり }
のある
\ruby{方}{はう}へ
\ruby[g]{御就}{お つ }きになる!。
%
お
\ruby{{\換字{前}}}{まへ}は
\ruby{知}{し}らないで
\ruby{好}{い}い
\ruby{氣}{き}に
なつて
おいで
だつた
らうが、
%
ちやんと
\ruby{妾}{わたし}の
\ruby{方}{はう}へ
\原本頁{130-10}\改行%
\ruby[g]{御知}{お し }らせ
くだすつて、
%
いろ〳〵と
\ruby{御注意}{お|こゝろ|づけ}まで
\ruby{仕}{し}て
\ruby{下}{くだ}すつたのだ
\改行% 校正作業の簡略化のため
。
%
\原本頁{130-11}\改行%
\ruby{七{\換字{分}}{\換字{通}}}{しち|ぶ|どほ}り% 原本には漢数字「七」のルビ無し
\ruby{八{\換字{分}}{\換字{通}}}{はち|ぶ|どほ}り
\ruby{話}{はなし}の
\ruby{定}{きま}つた
\ruby{婿}{むこ}を% (婿 5a7f) 聟 805f
\ruby{{\換字{嫌}}}{きら}つて
お
\ruby{{\換字{前}}}{まへ}には
\ruby{出}{で}られる、
%
\ruby[g]{何處}{ど こ }へ
\ruby{行}{い}つたかもかいくれ
\ruby{知}{し}れず、
%
また
\ruby[g]{短氣}{たんき }を
\ruby{仕}{し}て
\ruby{{\換字{若}}}{も}しや
\ruby[g]{淵川}{ふちかは}へ
でもかと、
\ruby[g]{何程}{どれほど}
\ruby{妾}{わたし}が
\ruby[g]{苦勞}{く らう}して
\ruby{困}{こま}り
\ruby{拔}{ぬ}いたか
\ruby{知}{し}れない、
%
\ruby[g]{其處}{そ こ }へ
\ruby[g]{此方}{こ ちら}% % 行末行頭の境界付近なので特例処置を施す
\ruby{樣}{さま}からの
\ruby[g]{行屆}{ゆきとゞ}いた% 「屆」「届」 原本通り「屆」
\ruby{御手紙}{お|て|がみ}で、
%
やつと
\ruby{胸}{むね}の
\ruby[g]{凝塊}{かたまり}が
すこし
\ruby{下}{さが}つた
\改行% 校正作業の簡略化のため
。
%
\原本頁{131-4}\改行%
\ruby[g]{居{\換字{所}}}{ゐ どこ}は
\ruby{知}{し}れたし、
%
\ruby[||j>]{引}{ひつ}
\ruby[||j>]{捉}{つかま}へてとも
% \ruby{引捉}{ひつ|つかま}へてとも
\ruby{思}{おも}はないでは
\ruby{無}{な}かつたが、
%
\ruby[g]{何樣}{ど う }せ
\ruby[g]{其程}{それほど}
\ruby{{\換字{嫌}}}{きら}つて
\ruby{居}{ゐ}る
\ruby{婿}{むこ}ならば、% (婿 5a7f) 聟 805f
%
\ruby[g]{仕方}{し かた}が
ないから
いつそ
\ruby[g]{破談}{は だん}に
なすつたが
\ruby{宜}{よ}からうし、
%
\ruby[g]{破談}{は だん}に
なさるなら
\ruby{{\換字{又}}}{また}
\ruby[g]{當人}{たうにん}が
\ruby[|g|]{其地}{そちら}に
\ruby{居}{ゐ}ないで
\改行% 校正作業の簡略化のため
、
%
\原本頁{131-7}\改行%
\ruby[g]{何處}{ど こ }へ
\ruby{行}{い}つたか
\ruby{知}{し}れない
といふ
\ruby{{\換字{分}}}{ぶん}に
なすつた
\ruby{方}{はう}が、
%
\ruby{事}{こと}が
\ruby{濟}{す}み
\ruby{易}{やす}からうし、
%
\ruby{{\換字{若}}}{も}し
\ruby{{\換字{強}}}{し}ひて
\ruby[g]{無理}{む り }な
\ruby{事}{とと}を% ルビ調整(原本遠り)(とと)
なさる
やうでは
\ruby[g]{當人}{たうにん}の
\ruby{爲}{ため}にも
\改行% 校正作業の簡略化のため
、
%
\原本頁{131-9}\改行%
\ruby{却}{かへ}つて
ならない
やうな
\ruby{事}{こと}に
なりは
\ruby{爲}{し}まいかと
\ruby{思}{おも}はれる
から、
%
\ruby[g]{次第}{し だい}に
よつたら
\ruby[g]{姑く}{しばら }
\ruby[g]{此儘}{このまゝ}
\ruby[g]{御預}{お あづ}かり
\ruby{申}{まを}しても
\ruby{宜}{よ}い、
%
と
\ruby{能}{よ}く
\ruby{{\換字{分}}}{わか}つた
\ruby[g]{此方}{こ ちら}% ルビ調整(原本通り)非グループルビ
\ruby{樣}{さま}の
\ruby{御親切}{ご|しん|せつ}な
\ruby[g]{御仰}{お つし}あり
やうでも
あり、
%
また
\ruby[|g|]{此方}{こちら}
\ruby{樣}{さま}の
\ruby[g]{御噂}{おうはさ}も
\ruby{豫}{かね}て
\ruby{聞}{き}いて
\ruby[g]{何樣}{ど う }いふ
\ruby{方}{かた}かと
\ruby[g]{合點}{が てん}しても
\ruby{居}{ゐ}たので、
%
とても
\ruby{妾}{わたし}には
\ruby[g]{制{\換字{道}}}{せいだう}の
\ruby{就}{つ}きません
\ruby{我儘者}{わが|まゝ|もの}
で
ございます
から
\ruby{既}{もう}
\ruby{諦}{あき}らめました、
%
\ruby[g]{御甘}{お あま}え
\ruby{申}{まを}しては
\ruby{濟}{す}みませんが
\ruby[g]{然樣}{さ う }いふ
\ruby{譯}{わけ}
で
ございますれば、
%
\ruby[|g|]{此方}{こちら}の
\原本頁{132-4}\改行%
\ruby{話}{はなし}も
\ruby{解}{と}けて
\ruby{濟}{す}んで
\ruby[g]{仕舞}{し ま }ふ
まで
\ruby[g]{御預}{お あづ}かりを
\ruby{願}{ねが}ひます、
%
\ruby[g]{成程}{なるほど}
\ruby{今}{いま}
\ruby{妾}{わたし}が
\ruby{出}{で}て
\ruby{參}{まゐ}りまして
\ruby[g]{當人}{たうにん}に
\ruby{會}{あ}つても
\ruby{何}{なん}にも
なりますまいから、
%
\ruby{御{\換字{迷}}惑}{ご|めい|わく}
でも
ござりましやうが
\ruby{其}{それ}では
\ruby[g]{何{\換字{分}}}{なにぶん}
\ruby{宜}{よろ}しく
\ruby{願}{ねが}ひまする、
%
\ruby{{\換字{若}}}{も}し
\ruby{{\換字{又}}}{また}
\原本頁{132-7}\改行%
\ruby[g]{當人}{たうにん}が
\ruby{不}{ふ}
\ruby[g]{心得}{こゝろえ}
なぞを
\ruby{致}{いた}して、
%
\ruby{御厄介}{ご|やく|かい}を
\ruby{掛}{か}けまする
やうなことが
ございますれば
\ruby[g]{屹度}{きつと }% ルビ調整(原本通り)非グループルビ
\ruby[g]{引受}{ひきう }けまする、
%
と
\ruby[g]{斯樣}{か う }いふ
\ruby{御挨拶}{ご|あい|さつ}を
\ruby{仕}{し}て
\ruby{願}{ねが}つて
\ruby{置}{お}いたのだ。
%
\ruby{今}{いま}
\ruby{解}{わか}つたかエ、
%
\ruby{妾}{わたし}の
\ruby[||j>]{心}{こゝろ}
\ruby[||j>]{持}{ もち}も
% \ruby{心持}{こゝろ|もち}も
\ruby[|g|]{此方}{こちら}
\ruby{樣}{さま}の
\ruby{御思慮}{お|かん|がへ}も
\改行% 校正作業の簡略化のため
。
%
\原本頁{132-10}\改行%
それほど
\ruby{妾}{わたし}にも
\ruby[|g|]{此方}{こちら}
\ruby{樣}{さま}にも
\ruby[g]{人知}{ひとし }れず
\ruby{氣}{き}を
\ruby{揉}{も}ませて
\ruby{置}{お}いて、
%
それだのに
\ruby{何}{なん}だエ、
%
\ruby[g]{月日}{つきひ }も
\ruby{經}{たゝ}ない
\ruby{中}{うち}に
\ruby{{\換字{又}}}{また}
\ruby[|g|]{此方}{こちら}
\ruby{樣}{さま}を
\ruby{駈}{か}け
\ruby{出}{だ}して、%
{---}{---}%
\原本頁{133-1}\改行%
\ruby{妹}{いもと}のやうに
\ruby{思}{おも}ふ
\ruby{子}{こ}のやうに
\ruby{思}{おも}ふ
とまで
\ruby{云}{い}つて
くださる
\ruby[|g|]{此方}{こちら}
\ruby{樣}{さま}の
\ruby{御親切}{ご|しん|せつ}も、
%
\ruby{妾}{わたし}は
お
\ruby{{\換字{前}}}{まへ}の
\ruby[|g|]{眞實}{ほんと}の
\ruby[g]{叔母}{を ば }
だけれども
\ruby[g]{然樣}{さ う }は
\ruby{濃}{こま}かに
お
\ruby{{\換字{前}}}{まへ}
のためを
\ruby{思}{おも}ふことは
\ruby[g]{出來}{で き }ないと
\ruby{我}{が}の
\ruby{折}{を}れる
ほどに
\ruby{仕}{し}て
\ruby{下}{くだ}さる
\ruby{有}{あ}り
\ruby{{\換字{難}}}{がた}い
\ruby[|g|]{此方}{こちら}
\ruby{樣}{さま}の
\ruby[g]{御恩}{ご おん}をも
\ruby{全}{まる}で
\ruby[g]{餘{\換字{所}}}{よ そ }にして、
%
\ruby{何}{なに}が
\ruby[g]{不足}{ふ そく}で
\ruby[g]{無言}{だんまり}で
\ruby{三}{さ}
\ruby{絃}{みせん}の
% \ruby{三絃}{さ|みせん}の
\ruby[g]{師匠}{しゝやう}だ
なんて
\ruby{彼}{あ}んな
\ruby{惡}{わる}い
\ruby{奴}{やつ}の
ところへ
\ruby{行}{い}つた。
%
これ、
%
\ruby[g]{何故}{な ぜ }
\原本頁{133-6}\改行%
\ruby[|g|]{此方}{こちら}
\ruby{樣}{さま}を
\ruby{後}{あと}にして
\ruby{稽{\換字{古}}{\換字{所}}}{けい|こ|じよ}
なんぞの
\ruby[g]{手助}{て だす}けを
\ruby{仕}{し}て
\ruby{自墮落}{じ|だ|らく}に
\ruby{暮}{くら}したのだエ。
%
\ruby[g]{彼女}{あ れ }あ
お
\ruby{{\換字{前}}}{まへ}、
%
お
\ruby{{\換字{前}}}{まへ}に
\ruby{碌}{ろく}でも
\ruby{無}{な}い
\ruby{男}{をとこ}
なんぞを
\ruby{取}{と}り
\ruby{持}{も}つた
\ruby[<j||]{狸}{たぬき}
\ruby[||j>]{婆}{ばゞあ}ぢや
\ruby{無}{な}いか。
%
\ruby[g]{性凝}{しやうこ}りも
\ruby{無}{な}く、
%
まだ
\ruby[g]{{\換字{浮}}氣}{うはき }が
\ruby{仕}{し}たくつて、
%
\ruby[g]{彼樣}{あ ん }な
\ruby{奴}{やつ}に
\ruby{末始{\換字{終}}}{すゑ|し|じう}は% ルビ調整(原本通り)「ゆ」無し
\ruby{食}{く}はれるのも
\ruby{知}{し}らないで、
%
\ruby[|g|]{此方}{こちら}
\ruby{樣}{さま}を
\ruby{出}{で}たのかエ。
%
\ruby{猫}{ねこ}!。
%
いやらしい
\ruby{猫}{ねこ}!。
%
ほんとに
いやらしい
\ruby{猫}{ねこ}!。
%
\ruby{猫}{ねこ}
だつて
\ruby{畜}{か}はれた
\ruby{恩}{おん}を
\ruby[g]{三日}{みつか }
\ruby{經}{た}つてから
\ruby{忘}{わす}れる、
%
\ruby{汝}{おまへ}あ
\ruby{畜}{か}はれて
\ruby{居}{ゐ}て
\ruby[g]{可愛}{か はい}がられて
\ruby{居}{ゐ}て
\ruby[g]{既時}{す ぐ }に
\ruby{忘}{わす}れたのだ。
%
\ruby{妾}{わたし}にも
\ruby[g]{然樣}{さ う }だつた、
%
\ruby[|g|]{此方}{こちら}
\ruby{樣}{さま}にも
\ruby[g]{然樣}{さ う }だつた。
%
お
\ruby{{\換字{前}}}{まへ}の
やうな
\ruby{好}{い}い
\ruby{姪}{めひ}を
もつて
\ruby[g]{人樣}{ひとさま}の
\ruby{{\換字{前}}}{まへ}で、
%
\ruby{妾}{わたし}あ
ほんとに
\ruby[g]{肩身}{かたみ }が
\ruby{廣}{ひろ}くつて
\ruby[g]{何樣}{ど ん }なにか
\ruby{嬉}{うれ}しいよ。
』

\原本頁{133-4}%
と、
%
\ruby{例}{れい}の
\ruby{眼}{め}を
\ruby{動}{うご}かし〳〵
\ruby{思}{おも}ふさまに
\ruby{罵}{のゝし}つたり。
