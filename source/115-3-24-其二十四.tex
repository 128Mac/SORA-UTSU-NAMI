\Entry{其二十四}

\ruby{最初}{さい|しよ}つから
\ruby{云}{い}ふと
\ruby{如是}{か|う}なのだよお
\ruby{龍}{りう}ちやん。
それ
\ruby[g]{一昨年}{をとヽし}の
\ruby{夏}{なつ}の
\ruby{事}{こと}だつたね、これこれで
\ruby[g]{此度叔母}{こんどをば}に
\ruby{伴}{つ}れられて、
\ruby{厭}{いや}だけれども
\ruby{靜岡}{しづ|をか}へ
\ruby{行}{ゆ}きますからつて、お
\ruby{前}{まへ}が
\ruby{暇乞}{いとま|ごひ}に
\ruby{御}{お}いでだつたことがあつた、
\ruby{其時}{そ|れ}からといふものは
\ruby{隨分長}{ずゐ|ぶん|なが}い
\ruby{間}{あひだ}、
\ruby[g]{此方}{こつち}から
\ruby{手紙}{て|がみ}をあげても
\ruby[g]{{\換字{返}}辭}{へんじ}は
\ruby{少}{すくな}いし、たまに
\ruby{御{\GWI{u9063-k}}}{お|よこ}しでも
\ruby{極々短}{ごく|〳〵|みじ}つかい
\ruby{眞}{ほん}の
\ruby{義理濟}{ぎ|り|す}ましだけの
\ruby{事}{こと}だし、
\ruby{是}{これ}あ
\ruby{何}{なに}か
\ruby{知}{し}らないけれども
\ruby{甚}{ひど}く
\ruby{氣}{き}を
\ruby{取}{と}れておいでの
\ruby{事}{こと}があるのだらう、と
\ruby{思}{おも}つて
\ruby{居}{ゐ}る
\ruby{中}{うち}に
\ruby{今年}{こ|とし}の
\ruby{三月}{さん|ぐわつ}、ふらりつと
\ruby{妾}{わたし}の
\ruby{處}{ところ}へ
\ruby{御}{お}いでだつたが、
\ruby{顏付}{かほ|つき}は
\ruby[g]{全然變}{まるでかは}つて
\ruby{仕舞}{し|ま}つて、
\ruby{前}{まへ}に
\ruby{見}{み}た
\ruby[g]{處女}{むすめ}らしいところは
\ruby{無}{な}くなつて
\ruby{御{\換字{終}}}{お|しま}ひだし、
\ruby[g]{様子}{やうす}は
\ruby{何}{なん}だか
\ruby{知}{し}らないがそは〳〵としておいでヾ、
\ruby{妾}{わたし}に
\ruby{御話}{お|はな}しの
\ruby[g]{談話}{はなし}にも
\ruby{辻褄}{つぢ|つま}の
\ruby{合}{あ}はないところは
\ruby{有}{あ}り、
\ruby{何樣}{ど|う}も
\ruby{氣}{き}になる
\ruby{事}{こと}ばかしだから
\ruby{妾}{わたし}は
\ruby{心配}{しん|ぱい}して、すこし
\ruby{置}{お}いて
\ruby{{\換字{呉}}}{く}れと
\ruby{御言}{お|い}ひのことだから,あヽ
\ruby{宜}{い}いともと、
\ruby{表面}{うは|べ}は
\ruby{何}{なん}の
\ruby{氣}{き}もつかない
\ruby{風}{ふう}で
\ruby{家}{うち}へは
\ruby{置}{お}いて
\ruby{{\GWI{u9032-k}}}{あ}げたものヽ、
\ruby{何樣}{ど|ん}なにいろ〳〵と
\ruby{物}{もの}をおもつたか
\ruby{知}{し}れないよ。
\ruby{此處}{こ|ヽ}に
\ruby{居}{ゐ}ることを
\ruby{靜岡}{しづ|をか}へ
\ruby{知}{し}らせては
\ruby{{\換字{呉}}}{く}れるなと、
\ruby{念}{ねん}に
\ruby{念}{ねん}を
\ruby{押}{お}しての
\ruby[g]{御依頼}{おたのみ}だつたけれども、
\ruby{今白状}{いま|はく|じやう}してお
\ruby{前}{まへ}に
\ruby[g]{謝罪}{あやま}るがネ、
\ruby{何樣}{ど|う}も
\ruby{物}{もの}の
\ruby{{\GWI{u9053-k}}理}{だう|り}が
\ruby{然樣}{さ|う}は
\ruby{行}{い}かないと
\ruby{思}{おも}つたので、お
\ruby{前}{まへ}には
\ruby{内密}{ない|しよ}でもつて
\ruby{靜岡}{しづ|をか}の
\ruby{叔母}{を|ば}さんへ、これ〳〵の
\ruby[g]{樣子}{やうす}で、
\ruby[g]{如是々々}{かう〳〵}してお
\ruby{龍}{りう}ちやんは
\ruby{妾}{わたし}の
\ruby{方}{はう}に
\ruby{御}{お}いでだと、
\ruby{妾}{わたし}が
\ruby[g]{全然知}{すつかりし}らせて
\ruby{仕舞}{し|ま}つたのだよ。
』

\ruby{此}{これ}まで
\ruby{語}{かた}り
\ruby{掛}{か}けし
\ruby{時}{とき}、
\ruby{叔母}{を|ば}はお
\ruby{龍}{りう}を
\ruby{見}{み}て、

『それ
\ruby[g]{御覽}{ごらん}。
\ruby{汝}{おまへ}のやうな
\ruby{分}{わか}らないものヽ
\ruby{云}{い}ふ
\ruby{事}{こと}や
\ruby{思}{おも}ふことばかりが
\ruby{何}{なん}で
\ruby{通}{とほ}るものかエ。
\ruby[g]{此方樣}{こちらさま}のやうな
\ruby{方}{かた}は
\ruby[g]{何程御優}{いくらおやさ}しくつても、
\ruby{角々}{かど|〳〵}は
\ruby{嚴然}{きつ|ぱり}と
\ruby{道理}{だう|り}のある
\ruby{方}{はう}へ
\ruby{御就}{お|つ}きになる!。
お
\ruby{前}{まへ}は
\ruby{知}{し}らないで
\ruby{好}{い}い
\ruby{氣}{き}になつておいでだつたらうが、ちやんと
\ruby{妾}{わたし}の
\ruby{方}{はう}へ
\ruby{御知}{お|し}らせくだすつて、いろ〳〵と
\ruby[g]{御注意}{おこヽろづけ}まで
\ruby{仕}{し}て
\ruby{下}{くだ}すつたのだ。
\ruby[g]{七分{\GWI{u901a-k}}}{しちぶどほ}り
\ruby[g]{八分{\GWI{u901a-k}}}{はちぶどほ}り
\ruby{話}{はなし}の
\ruby{定}{きま}つた
\ruby{婿}{むこ}を
\ruby{{\換字{嫌}}}{きら}つてお
\ruby{前}{まへ}には
\ruby{出}{で}られる、
\ruby{何處}{ど|こ}へ
\ruby{行}{い}つたかもかいくれ
\ruby{知}{し}れず、
また
\ruby{短氣}{たん|き}を
\ruby{仕}{し}て
\ruby{若}{も}しや
\ruby{淵川}{ふち|かは}へでもかと
\ruby{何程妾}{どれ|ほど|わたし}が
\ruby{苦勞}{く|らう}して
\ruby{困}{こま}り
\ruby{拔}{ぬ}いたか
\ruby{知}{し}れない、
\ruby{其處}{そ|こ}へ
\ruby[g]{此方樣}{こちらさま}からの
\ruby{行屆}{ゆき|とど}いた
\ruby[g]{御手紙}{おてがみ}で、やつと
\ruby{胸}{むね}の
\ruby{凝塊}{かた|まり}がすこし
\ruby{下}{さが}つた。
\ruby{居所}{ゐ|どこ}は
\ruby{知}{し}れたし、
\ruby[g]{引捉}{ひつつかま}へてとも
\ruby{思}{おも}はないでは
\ruby{無}{な}かつたが、
\ruby{何樣}{ど|う}せ
\ruby[g]{其程{\換字{嫌}}}{それほどきら}つて
\ruby{居}{ゐ}る
\ruby{婿}{むこ}ならば、
\ruby{仕方}{し|かた}がないからいつそ
\ruby{破談}{は|だん}になすつたが
\ruby{宜}{よ}からうし、
\ruby{破談}{は|だん}になさるなら
\ruby{{\換字{又}}當人}{また|たう|にん}が
\ruby[g]{其地}{そちら}に
\ruby{居}{ゐ}ないで、
\ruby{何處}{ど|こ}へ
\ruby{行}{い}つたか
\ruby{知}{し}れないといふ
\ruby{分}{ぶん}になすつた
\ruby{方}{はう}が、
\ruby{事}{こと}が
\ruby{濟}{す}み
\ruby{易}{やす}からうし、
\ruby{若}{も}し
\ruby{{\換字{強}}}{し}ひて
\ruby{無理}{む|り}な
\ruby{事}{こと}をなさるやうでは
\ruby{當人}{たう|にん}の
\ruby{爲}{ため}にも、
\ruby{却}{かへ}つてならないやうな
\ruby{事}{こと}になりは
\ruby{爲}{し}まいかと
\ruby{思}{おも}はれるから、
\ruby{次第}{し|だい}によつたら
\ruby{姑}{しばら}く
\ruby[g]{此儘御預}{このまヽおあづ}かり
\ruby{申}{まを}しても
\ruby{宜}{よ}い、と
\ruby{能}{よ}く
\ruby{{\換字{分}}}{わか}つた
\ruby[g]{此方樣}{こちらさま}の
\ruby{御親切}{ご|しん|せつ}な
\ruby[g]{御仰}{おつし}ありやうでもあり、また
\ruby[g]{此方樣}{こちらさま}の
\ruby[g]{御噂}{おうはさ}も
\ruby{豫}{かね}て
\ruby{聞}{き}いて
\ruby{何樣}{ど|う}いふ
\ruby{方}{かた}かと
\ruby{合點}{が|てん}しても
\ruby{居}{ゐ}たので、とても
\ruby{妾}{わたし}には
\ruby{制{\GWI{u9053-k}}}{せい|だう}の
\ruby{就}{つ}きません
\ruby{我儘者}{わが|まヽ|もの}でございますから
\ruby{既諦}{もう|あき}らめました、
\ruby{御甘}{お|あま}え
\ruby{申}{まを}しては
\ruby{濟}{す}みませんが
\ruby{然樣}{さ|う}いふ
\ruby{譯}{わけ}でございますれば、
\ruby[g]{此方}{こちら}の
\ruby{話}{はなし}も
\ruby{解}{と}けて
\ruby{濟}{す}んで
\ruby{仕舞}{し|ま}ふまで
\ruby{御預}{お|あづ}かりを
\ruby{願}{ねが}ひます、
\ruby{成程今妾}{なる|ほど|いま|わたし}が
\ruby{出}{で}て
\ruby{參}{まゐ}りまして
\ruby{當人}{たう|にん}に
\ruby{會}{あ}つても
\ruby{何}{なん}にもなりますまいから、
\ruby[g]{御{\換字{迷}}惑}{ごめいわく}でもござりましやうが
\ruby{其}{それ}では
\ruby{何分宜}{なに|ぶん|よろ}しく
\ruby{願}{ねが}ひまする、
\ruby{若}{も}し
\ruby{{\換字{叉}}}{また}
\ruby{當人}{たう|にん}が
\ruby{不心得}{ふ|こヽろ|\GWI{u1b001}}なぞを
\ruby{致}{いた}して、
\ruby{御厄介}{ご|やく|かい}を
\ruby{掛}{か}けまするやうなことがございますれば
\ruby[g]{屹度引受}{きつとひきう}けまする、と
\ruby{斯樣}{か|う}いふ
\ruby[g]{御挨拶}{ごあいさつ}を
\ruby{仕}{し}て
\ruby{願}{ねが}つて
\ruby{置}{お}いたのだ。
\ruby{今解}{いま|わか}つたかエ、
\ruby{妾}{わたし}の
\ruby{心持}{こヽろ|もち}も
\ruby[g]{此方樣}{こちらさま}の
\ruby[g]{御思慮}{おかんがへ}も。
それほど
\ruby{妾}{わたし}にも
\ruby[g]{此方樣}{こちらさま}にも
\ruby{人知}{ひと|し}れず
\ruby{氣}{き}を
\ruby{揉}{も}ませて
\ruby{置}{お}いて、それだのに
\ruby{何}{なん}だエ、
\ruby{月日}{つき|ひ}も
\ruby{經}{たヽ}ない
\ruby{中}{うち}に
\ruby[g]{{\換字{叉}}此方樣}{またこちらさま}を
\ruby{駈}{か}け
\ruby{出}{だ}して、\------
\ruby{妹}{いもと}のやうに
\ruby{思}{おも}ふ
\ruby{子}{こ}のやうに
\ruby{思}{おも}ふとまで
\ruby{云}{い}つてくださる
\ruby[g]{此方樣}{こちらさま}の
\ruby{御親切}{ご|しん|せつ}も、
\ruby{妾}{わたし}はお
\ruby{前}{まへ}の
\ruby{眞實}{ほん|と}の
\ruby{叔母}{を|ば}だけれども
\ruby{然樣}{さ|う}は
\ruby{濃}{こま}かにお
\ruby{前}{まへ}のためを
\ruby{思}{おも}ふことは
\ruby{出來}{で|き}ないと
\ruby{我}{が}の
\ruby{折}{お}れるほどに
\ruby{仕}{し}て
\ruby{下}{くだ}さる
\ruby{有}{あ}り
\ruby{難}{がた}い
\ruby[g]{此方樣}{こちらさま}の
\ruby[g]{御恩}{ごおん}をも
\ruby{全}{まる}で
\ruby{餘所}{よ|そ}にして、
\ruby{何}{なに}が
\ruby{不足}{ふ|そく}で
\ruby[g]{無言}{だんまり}で
\ruby{三絃}{さみ|せん}の
\ruby{師匠}{し|しやう}だなんて
\ruby{彼}{あ}んな
\ruby{惡}{わる}い
\ruby{奴}{やつ}のところへ
\ruby{行}{い}つた。
これ、
\ruby[g]{何故此方樣}{なぜこちらさま}を
\ruby{後}{あと}にして
\ruby[g]{稽古所}{けいこじよ}なんぞの
\ruby{手助}{てだ|すけ}けを
\ruby{仕}{し}て
\ruby[g]{自墮落}{じだらく}に
\ruby{暮}{くら}したのだエ。
\ruby{彼女}{あ|れ}あお
\ruby{前}{まへ}、お
\ruby{前}{まへ}に
\ruby{碌}{ろく}でも
\ruby{無}{な}い
\ruby{男}{をとこ}なんぞを
\ruby{取}{と}り
\ruby{持}{も}つた
\ruby{狸婆}{たぬき|ばヾあ}ぢや
\ruby{無}{な}いか。
\ruby{性凝}{しや|うこ}りも
\ruby{無}{な}く、まだ
\ruby{{\換字{浮}}氣}{うは|き}が
\ruby{仕}{し}たくつて、
\ruby{彼樣}{あ|ん}な
\ruby{奴}{やつ}に
\ruby[g]{末始{\換字{終}}}{すゑしじう}は
\ruby{食}{く}はれるのも
\ruby{知}{し}らないで、
\ruby[g]{此方樣}{こちらさま}を
\ruby{出}{で}たのかエ。
\ruby{猫}{ねこ}!。
いやらしい
\ruby{猫}{ねこ}!。
ほんとにいやらしい
\ruby{猫}{ねこ}!。
\ruby{猫}{ねこ}だつて
\ruby{畜}{か}はれた
\ruby{恩}{おん}を
\ruby[g]{三日經}{みつかた}つてから
\ruby{忘}{わす}れる、
\ruby{汝}{おまへ}あ
\ruby{畜}{か}はれて
\ruby{居}{ゐ}て
\ruby[g]{可愛}{かはい}がられて
\ruby{居}{ゐ}て
\ruby{既時}{す|ぐ}に
\ruby{忘}{わす}れたのだ。
\ruby{妾}{わたし}にも
\ruby{然樣}{さ|う}だつた、
\ruby[g]{此方樣}{こちらさま}にも
\ruby{然樣}{さ|う}だつた。
お
\ruby{前}{まへ}のやうな
\ruby{好}{よ}い
\ruby{姪}{めひ}をもつて
\ruby{人樣}{ひと|さま}の
\ruby{前}{まへ}で、
\ruby{妾}{わたし}あほんとに
\ruby{肩身}{かた|み}が
\ruby{廣}{ひろ}くつて
\ruby{何樣}{ど|ん}なにか
\ruby{嬉}{うれ}しいよ。
』

と、
\ruby{例}{れい}の
\ruby{眼}{め}を
\ruby{動}{うご}かし〳〵
\ruby{思}{おも}ふさまに
\ruby{罵}{のヽし}つたり。

