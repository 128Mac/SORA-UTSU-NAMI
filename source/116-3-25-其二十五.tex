\Entry{其二十五}

% メモ 校正終了 2024-05-15 2024-06-11
\原本頁{134-6}%
\ruby{然樣}{さ|う}いふ
\ruby{氣性}{き|しやう}の
\ruby{人}{ひと}と
\ruby{思}{おも}へば
\ruby{腹}{はら}は
\ruby{立}{た}たぬ
ながら、
%
\ruby{理由}{わ|け}も
\ruby{知}{し}らず
\ruby{唯}{たゞ}
\ruby{一槪}{いち|がい}に
\ruby{猫}{ねこ}よ
\ruby[||j>]{畜}{ちく}
\ruby[||j>]{生}{しやう}よ
% \ruby{畜生}{ちく|しやう}よ
\ruby{猫}{ねこ}にも
\ruby{劣}{おと}るとは
\ruby[|g|]{何程}{いくら}
\ruby{叔母樣}{を|ば|さま}
なれば
とて
\ruby{餘}{あま}りなる
\ruby{言葉}{こと|ば}。
%
\ruby{靜岡}{しづ|をか}から
\ruby{唯}{たゞ}
\ruby{一}{ひと}つの
\ruby{頼}{たのみ}にして
\ruby{出}{で}て
\ruby{來}{き}たほどの
\ruby{此家}{こ|ゝ}を
\ruby{無言}{だん|まり}で
\ruby{出}{で}たのは、
%
よく〳〵
\ruby{口惜}{く|や}しい
\ruby{悲}{かな}しい
\ruby{事}{こと}の
\ruby{有}{あ}つたれば
こそ、
%
\ruby{生}{い}きて
\ruby{復}{また}
\ruby{顏}{かほ}を
\ruby{見}{み}たり
\ruby{見}{み}られたりする
\ruby{氣}{き}が
\ruby{些}{すこし}でも
あつては、
%
お
\ruby{彤}{とう}さんの
\ruby{親切}{しん|せつ}を
\ruby{餘{\換字{所}}}{よ|そ}に
して、
%
\ruby{何樣}{ど|う}して
\ruby{彼事}{あ|ん}な
\ruby{事}{こと}の
\ruby{出來}{で|き}る
ものでは
\ruby{無}{な}し、
%
\ruby{全}{まつた}く
\ruby{憎}{にく}い
\ruby{憎}{にく}い
\ruby{源}{げん}を
\ruby{殺}{ころ}して
\ruby{自{\換字{分}}}{じ|ぶん}も% 原本通り非グループルビ
\ruby{死}{し}んで
\ruby{仕舞}{し|ま}ふ
\ruby{氣}{き}で、
%
\ruby{濟}{す}まない
ことは
\ruby{悉皆}{みん|な}% 原本通り非グループルビ
\ruby[<j||]{冷}{つめた}く
なつてから
\ruby{謝罪}{わ|び}る
\ruby{積}{つも}りの、
%
\ruby{{\換字{遺}}書}{かき|おき}さへ
\ruby{身}{み}に
\ruby{着}{つ}けて
\ruby{持}{も}つて
\ruby{居}{ゐ}て
\ruby{此家}{こ|ゝ}を
\ruby{脫}{ぬ}けて、
%
\ruby{出會}{で|あ}つたが
\ruby{最後}{さい|ご}
\ruby{一發}{ひと|うち}と
\ruby{思}{おも}つて
\ruby{居}{ゐ}た
\改行% 校正作業の簡略化のため
、
%
\原本頁{135-5}\改行%
\ruby{其}{それ}は
\ruby{其}{そ}の
\ruby{事}{こと}
\ruby{無}{な}くて
\ruby{其}{そ}の
\ruby{意}{おもひ}の
\ruby{見}{み}えずに
\ruby{濟}{す}んだ
ゆゑ、
%
たゞ
\ruby{{\換字{勝}}手}{かつ|て}
\ruby{淫奔}{いた|づら}の
\ruby{心}{こゝろ}から
\ruby{彼樣}{あ|ん}な
ところへ
\ruby{行}{い}つて、
%
\ruby{身}{み}を
\ruby{自墮落}{じ|だ|らく}に
\ruby{稽{\換字{古}}{\換字{所}}}{けい|こ|じよ}に
\ruby{置}{お}くと
\原本頁{135-7}\改行%
\ruby{思}{おも}はれても
\ruby{仕方}{し|かた}
\ruby{無}{な}けれど、
%
\ruby{自{\換字{分}}}{じ|ぶん}の% 原本通り非グループルビ
\ruby{姪}{めひ}を
\ruby{其樣}{そ|ん}なに
\ruby{惡}{わる}いものにして
\ruby{罵詈}{く|さ}せば
\ruby{何}{なに}が
\ruby{面白}{おも|しろ}いのか、
%
\ruby{辯解}{いひ|わけ}% 弁 瓣 辦 辧 辨 辩 (辯)
すれば
\ruby{{\換字{又}}}{また}
\ruby{男}{をとこ}を
\ruby{殺}{ころ}さう
とした
\ruby{叔母}{を|ば}の
\ruby{知}{し}らぬ
\ruby{一條}{ひと|すぢ}の
\ruby{談}{はなし}を、
%
こゝで
\ruby{新規}{しん|き}に
\ruby{仕出}{し|だ}さねば
ならぬ
\ruby{故}{ゆゑ}、
%
\ruby{知}{し}らぬを
\ruby{幸}{さいは}ひにして
\ruby{默}{だま}つて
\ruby{惡}{わる}く
\ruby{云}{い}はれて
\ruby{濟}{す}ませば、
%
それで
\ruby{濟}{す}むこと% 行末行頭の境界なので原本は非踊り字表記
\原本頁{135-11}\改行%
と
\ruby{濟}{す}ましも
\ruby{仕}{し}やう
なれど、
%
\ruby{餘}{あま}り
といへば
\ruby[||j>]{同}{おも}
\ruby[||j>]{{\換字{情}}}{ひやり}の
% \ruby{同{\換字{情}}}{おも|ひやり}の
\ruby{無}{な}い、
%
\ruby{我}{が}
ばかりの
\ruby{人}{ひと}と、
%
\ruby{私}{ひそか}に
\ruby[|g|]{口惜}{くやし}く
\ruby{思}{おも}ふか
\ruby{眼}{め}さへ
\ruby{沾}{うる}ませて、
%
お
\ruby{龍}{りう}は
\ruby{小}{ちひ}さくなりし
まゝ
\ruby{咳嗽}{しは|ぶき}
\ruby{一}{ひと}つ
せず、
%
たゞ
\ruby{頸垂}{うな|だ}れて
\ruby{凝然}{じ|つ}と
したる
さまは、
%
\ruby{首}{くび}の
\原本頁{136-3}\改行%
\ruby{座}{ざ}に
\ruby{直}{なほ}れる
\ruby{罪人}{ざい|にん}の
\ruby[||j>]{罪}{ざい}
\ruby[||j>]{狀}{じやう}
% \ruby{罪狀}{ざい|じやう}
\ruby[||j>]{讀}{ よ}まるゝを、
%
\ruby{何}{なん}と
\ruby{詮方}{せん|かた}も
\ruby{無}{な}く
\ruby{聞}{き}き
\ruby{居}{を}るにも
\ruby{似}{に}たり。

\原本頁{136-5}%
『
\ruby{其樣}{そ|ん}なに
まあ
\ruby{苛}{ひど}いことを
\ruby{仰}{おつし}あらないでもの
\ruby{事}{こと}で、
%
お
\ruby{龍}{りう}ちやんが
\ruby{妾}{わたし}の
ところを
\ruby{出}{で}て
\ruby[|g|]{彼家}{あすこ}へ
\ruby{行}{い}つて
\ruby{居}{ゐ}るやうな
\ruby{經歷}{ゆく|たて}に
なつたのには、
%
いろ〳〵の
\ruby{理由}{わ|け}も
ある
ことで
\ruby{我儘}{わが|まゝ}
ばかり
ぢやあ
\ruby{有}{あ}りません
\改行% 校正作業の簡略化のため
。
%
\原本頁{136-8}\改行%
それは
\ruby{濟}{す}んで
\ruby{居}{ゐ}ること
だから
\ruby{何樣}{ど|う}でも
\ruby{好}{い}いとして、
%
\ruby{此度}{こん|ど}
\ruby{叔母}{を|ば}さんが
\ruby[|g|]{此地}{こつち}へ
\ruby{出}{で}ておいでのは、
%
お
\ruby{龍}{りう}ちやん
お
\ruby{{\換字{前}}}{まへ}の
\ruby{今}{いま}
\ruby{居}{ゐ}る
\ruby{家}{うち}の
\ruby{彼}{あ}の
\原本頁{136-10}\改行%
\ruby{御師匠}{お|し|よ}さんネ、
%
\ruby{彼}{あ}の
\ruby{人}{ひと}が
お
\ruby{{\換字{前}}}{まへ}を
\ruby{吳}{く}れろと
\ruby{叔母}{を|ば}さんの
ところへ、
%
\ruby{何}{なん}だか
\ruby{變}{へん}に
\ruby{搦}{から}んで
\ruby{云}{い}ひ
\ruby{{\換字{込}}}{こ}んで
\ruby{行}{い}つた
といふ
\ruby{其}{それ}から
\ruby{事}{こと}が
\ruby{起}{おこ}つたのだよ。
』

\原本頁{137-2}%
『
ほんとに
お
\ruby{{\換字{前}}}{まへ}は
\ruby{何處}{ど|こ}
\ruby{迄}{まで}
\ruby{人}{ひと}に
\ruby{世話}{せ|わ}を
\ruby{燒}{や}かせるのだか
\ruby{數}{すう}が
\ruby{知}{し}れない
\ruby{人}{ひと}だよ。
%
お
\ruby{{\換字{前}}}{まへ}が
\ruby[|g|]{此方}{こちら}
\ruby{樣}{さま}に
\ruby{御厄介}{ご|やく|かい}
になつて
\ruby[|g|]{靜穩}{おとな}しく
さへ
\ruby{仕}{し}て
\ruby{居}{ゐ}れば
\ruby{{\換字{紛}}紜}{いさ|くさ}の
\ruby{無}{な}いものを、
%
\ruby{性}{しやう}の
\ruby{知}{し}れない
\ruby{人}{ひと}の
\ruby{世話}{せ|わ}に
なんぞ
なるから、
%
\ruby{下}{くだ}らない
\ruby{苦勞}{く|らう}を
\ruby{無益}{む|だ}に
させられる!。
%
\ruby[|g|]{此方}{こちら}
\ruby{樣}{さま}の
\ruby[|g|]{御音信}{おたより}で
\ruby[<j||]{汝}{おまへ}の% 行末行頭の境界付近なので特例処置を施す
\ruby{樣子}{やう|す}も
\ruby{大抵}{たい|てい}は
\ruby{知}{し}つて
\ruby{居}{ゐ}たが、
%
\ruby{此頃}{この|ごろ}
になつて
\ruby{汝}{おまへ}の
\ruby{師匠}{し|しやう}% 原本通り、ルビ部分非踊り字表記
といふ
\ruby{人}{ひと}から、
%
\ruby{何}{なん}でも
お
\ruby{{\換字{前}}}{まへ}を
\ruby{貰}{もら}ひ
\ruby{度}{た}い
からとの
\ruby{再々}{さい|〳〵}の
\ruby{云}{い}ひ
\ruby{{\換字{込}}}{こ}みだ。
%
これ
よく
\ruby{御聞}{お|き}きなさい。
%
\ruby{一體}{いつ|たい}
なら
お
\ruby{{\換字{前}}}{まへ}
のやうな
ものは
\ruby{{\換字{遣}}}{や}つて
\ruby{仕舞}{し|ま}ふ
%%\原本頁{137-9}\改行%
\ruby{方}{はう}が
\ruby{苦勞拂}{く|らう|ばら}ひ
だから、
%
\ruby{鰹{\換字{節}}}{かつ|ぶし}でも
\ruby{付}{つ}けて
\ruby{{\換字{遣}}}{や}つて
\ruby{宜}{い}いのだが、
%
\ruby{見}{み}す% 行末行頭の境界付近なので非踊り字表記
\ruby{見}{み}す
\ruby{食物}{くひ|もの}に
なつて
\ruby{仕舞}{し|ま}ふ
\ruby{{\換字{前}}{\換字{途}}}{さ|き}が
\ruby{見}{み}えて
\ruby{居}{ゐ}るから、
%
\ruby{然樣}{さ|う}は
なりません
といつて
\ruby{挨拶}{あい|さつ}
したら、
%
まあ
\ruby{何}{なん}
といふ
こと
だらう、
%
\ruby{直}{すぐ}に
\ruby[|j|]{{\換字{狼}}}{おほかみ}% 行末行頭の境界付近なので特例処置を施す
%\原本頁{138-1}\改行%
\ruby{物}{もの}の
\ruby[||j>]{本}{ほん}
\ruby[||j>]{性}{しやう}を
% \ruby{本性}{ほん|しやう}を
\ruby{出}{だ}して、
%
\ruby{長}{なが}い
\ruby[||j>]{間}{あひだ}
\ruby[||j>]{御世話}{ お|せ|わ}を
\ruby{仕}{し}て
\ruby{居}{ゐ}た
\ruby{費用}{もの|いり}が
これ〳〵だ
\改行% 校正作業の簡略化のため
、
%
\原本頁{138-2}\改行%
お
\ruby{龍}{りう}さんを
\ruby{下}{くだ}さらなけりやあ
\ruby{御立替}{お|たて|かへ}を
\ruby{如何}{ど|う}か
なすつてと、
%
\ruby{吃驚}{びつ|くり}
するやうな
\ruby[||j>]{法}{はふ}
\ruby[||j>]{外}{ぐわい}の
% \ruby{法外}{はふ|ぐわい}の
お
\ruby{金}{かね}を
\ruby{妾}{わたし}から
\ruby{取}{と}らう
といふのだ。
%
\ruby{人}{ひと}を
\ruby{田舎}{ゐな|か}% 原本通り非グループルビ
\ruby[<j||]{婆}{ばゞあ}に
\ruby{仕}{し}て
\ruby{小馬鹿}{こ|ば|か}に
\ruby{仕}{し}たつて、
%
\ruby{野}{の}へ
\ruby{出}{で}ても
\ruby{座敷}{ざ|しき}へ
\ruby{上}{あが}つても
\ruby{人}{ひと}にやあ
\原本頁{138-5}\改行%
\ruby{負}{ま}けない
\ruby{婆}{ばゞあ}だ、
%
\ruby{先方}{さ|き}が
\ruby{然樣}{さ|う}
\ruby{出}{で}るなら、
%
\ruby[|g|]{此方}{こつち}も
\ruby{出樣}{で|やう}がある、
%
お
\ruby{龍}{りう}は
\ruby{妾}{わたし}の
\ruby{姪}{めひ}だ、
%
\ruby{妾}{わたし}が
\ruby{{\換字{連}}}{つ}れて
\ruby{歸}{かへ}ります、
%
お
\ruby{龍}{りう}に
\ruby{御注}{お|つ}ぎ
\ruby{{\換字{込}}}{こ}みなすつたのは
\ruby{汝}{おまへ}さんの
\ruby{御親切樣}{ご|しん|せつ|さま}だ、
%
\ruby{妾}{わたし}あ
\ruby[|g|]{些少}{ちつと}でも
\ruby{御恩}{ご|おん}になつた
\ruby{覺}{おぼ}えは
ありません、
%
\ruby{何}{なに}も
\ruby[||j>]{誘}{かど}
\ruby[||j>]{拐}{わかし}を
% \ruby{誘拐}{かど|わかし}を
\ruby{御商賣}{ご|しやう|ばい}にやあ
なさります
まいから、
%
\ruby{人}{ひと}の
\原本頁{138-9}\改行%
\ruby{姪}{めひ}
\ruby{甥}{をひ}に
\ruby{指}{ゆび}を
\ruby{御}{お}さし
になる
\ruby{事}{こと}は
\ruby{有}{あ}ります
まいと、
%
お
\ruby{{\換字{前}}}{まへ}を
\ruby[||j>]{拉}{ひつ}
\ruby[||j>]{去}{ちよび}いて
% \ruby{拉去}{ひつ|ちよび}いて
\原本頁{138-10}\改行%
\ruby{大手}{おほ|で}を
\ruby{振}{ふつ}て
\ruby{靜岡}{しづ|をか}へ
\ruby{歸}{かへ}つて、
%
\ruby{何樣}{ど|ん}な
\ruby{顏}{かほ}を
\ruby{仕}{し}て
\ruby{膨}{ふく}れるか
\ruby{見}{み}て
\ruby{{\換字{遣}}}{や}らうと
\ruby{思}{おも}つて、
%
\ruby[||j>]{東}{とう}
\ruby[||j>]{京}{きやう}の
% \ruby{東京}{とう|きやう}の
\ruby{生狡}{いけ|ずる}い
\ruby[<j||]{狸}{たぬき}
\ruby[||j>]{婆}{ばゞあ}の
\ruby{皮}{かは}を% 原本通り「皮 か(は)」
\ruby{剝}{む}く
\ruby{氣}{き}で
\ruby{出}{で}て
\ruby{來}{き}たのがネ。
』

\原本頁{139-1}%
と、
%
\ruby[||j>]{面}{まの}
\ruby[||j>]{{\換字{前}}}{あたり}にでも
% \ruby{面{\換字{前}}}{まの|あたり}にでも
お
\ruby{關}{せき}が
\ruby{居}{ゐ}るやうに
\ruby{怒}{おこ}り
\ruby{立}{た}つて
\ruby{力}{りき}んで
\ruby{云}{い}へる
\ruby{語氣}{もの|いひ}
\ruby{面色}{かほ|つき}、
%
なか〳〵
\ruby{當}{あた}り
\ruby{{\換字{難}}}{がた}く
あしらひ
\ruby{{\換字{難}}}{にく}き
\ruby{婆}{ばゞ}なり。
