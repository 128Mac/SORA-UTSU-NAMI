\Entry{其二十六}

『だがお
\ruby{龍}{りう}、お
\ruby{聞}{き}きなさい、
\ruby{妾}{わたし}あ
\ruby[g]{敵手}{あひて}が
\ruby{角}{つの}で
\ruby{向}{むか}つて
\ruby{來}{く}りやあ
\ruby[g]{此方}{こつち}も
\ruby{角}{つの}で
\ruby{向}{むか}つて
\ruby{行}{い}くけれど、お
\ruby{前}{まへ}のやうに
\ruby{眞}{しん}になつて
\ruby{世話}{せ|わ}を
\ruby{仕}{し}て
\ruby{{\換字{呉}}}{く}れる
\ruby{叔母}{を|ば}にも
\ruby{自分}{じ|ぶん}の
\ruby[g]{{\換字{勝}}手}{かつて}ぢやあお
\ruby{尻}{しり}を
\ruby{向}{む}けたり、
\ruby{折角優}{せつ|かく|やさ}しく
\ruby{仕}{し}て
\ruby{下}{くだ}さる
\ruby[g]{此方樣}{こちらさま}をも
\ruby{時}{とき}の
\ruby{都合}{つ|がふ}ぢやあ
\ruby{袖}{そで}にするやうな、
\ruby{其樣}{そ|ん}な
\ruby[g]{自分{\換字{勝}}手}{じぶんかつて}ばかりは
\ruby{夢}{ゆめ}にも
\ruby{仕}{し}ません。
お
\ruby{前}{まへ}は
\ruby{何}{なん}ぞに
\ruby{付}{つ}けちやあ、
\ruby{叔母}{を|ば}さんは
\ruby[g]{無理壓制}{むりおしつけ}だ、
\ruby{頑固}{ぐわん|こ}だ、
\ruby[g]{自分流義}{じぶんりうぎ}で
\ruby{何}{なん}でも
\ruby{押}{お}して
\ruby{行}{ゆ}かうとすると
\ruby{御云}{お|い}ひだが、そりやあ
\ruby{頑固}{ぐわん|こ}でもあらう、
\ruby[g]{自分流義}{じぶんりうぎ}でもあらう、
\ruby{然}{しか}し
\ruby{恩}{おん}は
\ruby{恩}{おん}、
\ruby{仇}{あだ}は
\ruby{仇}{あだ}でちやんと
\ruby{記}{おぼ}えて
\ruby{居}{ゐ}ます、お
\ruby{前}{まへ}のやうに
\ruby{恩}{おん}も
\ruby{仇}{あだ}も
\ruby{見}{み}さかひの
\ruby{無}{な}い
\ruby{事}{こと}は
\ruby{妾}{わたし}あ
\ruby{仕}{し}ません。
だから
\ruby{今}{いま}そのお
\ruby{關}{せき}つていふ
\ruby{奴}{やつ}のところへ
\ruby{押}{お}し
\ruby{込}{こ}んで
\ruby{行}{い}つて、
\ruby[g]{田舍婆}{ゐなかばゞあ}は
\ruby[g]{田舍婆}{ゐなかばゞあ}だけの
\ruby{意地}{い|ぢ}も
\ruby{有}{あ}りやあ
\ruby{根性}{こん|ぢやう}つ
\ruby{骨}{ほね}も
\ruby[g]{突張}{つゝぱ}つてゐるところを
\ruby{見}{み}せつけて
\ruby{{\換字{遣}}}{や}つて、
\ruby{間違}{ま|ちが}つたことは
\ruby{云}{い}はない
\ruby{妾}{わたし}だもの
\ruby{何負}{なに|ま}けるものか、
\ruby{思}{おも}ふさま
\ruby{捩}{ね}ぢ
\ruby{合}{あ}つて
\ruby{捩}{ね}ぢ
\ruby{合}{あ}ひ
\ruby{拔}{ぬ}いて、
\ruby{{\換字{勝}}鬨}{と|き}を
\ruby{吐}{は}いて
\ruby{歸}{かへ}らうと
\ruby{思}{おも}つたが、まづ
\ruby{其}{そ}の
\ruby{前}{まへ}に
\ruby[g]{此方樣}{こちらさま}へ
\ruby{伺}{うかゞ}つて、
\ruby{段々御世話}{だん|〳〵|お|せ|わ}になつた
\ruby{御禮}{お|れい}も
\ruby{云}{い}つたり、またお
\ruby{前}{まへ}が
\ruby{我儘}{わが|まゝ}に
\ruby[g]{此方樣}{こちらさま}を
\ruby{出}{で}て
\ruby{御親切}{ご|しん|せつ}を
\ruby{無}{む}にした
\ruby{御謝罪}{お|わ|び}も
\ruby{仕}{し}たり、
\ruby{一應}{いち|おう}は
\ruby[g]{此方樣}{こちらさま}の
\ruby[g]{御思召}{おぼしめし}も
\ruby{伺}{うかが}つてから、それから
\ruby{爭}{や}り
\ruby{合}{あ}ふなら
\ruby{爭}{や}り
\ruby{合}{あ}はなくつては
\ruby{義理}{ぎ|り}が
\ruby{惡}{わる}いと、それで
\ruby[g]{突掛}{つゝか}けに
\ruby[g]{此方樣}{こちらさま}へ
\ruby{伺}{うかが}つて、
\ruby{御噂}{おう|はさ}にばかり
\ruby{伺}{うかが}つて
\ruby{居}{ゐ}た
\ruby{方}{かた}にはじめて
\ruby{御目}{お|め}にかゝつたのだよ。
ところが、これお
\ruby{龍}{りう}、お
\ruby{聞}{き}きなさいよ。
\ruby{道理}{だう|り}に
\ruby{違}{ちが}つたことを
\ruby{云}{い}は
\ruby{無}{な}いものは
\ruby{何處}{ど|こ}にでも
\ruby[g]{味方}{みかた}があります。
いろ〳〵とお
\ruby{前}{まへ}のことを
\ruby{御話}{お|はな}し
\ruby{申}{まを}したところ、
\ruby[g]{悉皆妾}{すつかりわたし}の
\ruby{云}{い}ふことを
\ruby{道理}{もつ|とも}だと
\ruby{仰}{おつし}あつて
\ruby{下}{くだ}すつて、お
\ruby{前}{まへ}は
\ruby{何}{なん}ぞの
\ruby{時}{とき}には
\ruby[g]{此方樣}{こちらさま}を
\ruby{楯}{たて}に
\ruby{取}{と}つて、
\ruby{妾}{わたし}の
\ruby{云}{い}ふ
\ruby{事}{こと}を
\ruby{肯}{き}くまいなんぞと
\ruby{思}{おも}つてるか
\ruby{知}{し}らないが、もう
\ruby{然樣}{さ|う}は
\ruby{行}{い}きません
\ruby[g]{御生憎樣}{おあいにくさま}!、
\ruby{何樣}{ど|う}して
\ruby{何樣}{ど|う}して
\ruby{判然}{はつ|きり}と
\ruby{物}{もの}の
\ruby{道理}{だう|り}を
\ruby{御見{\換字{分}}}{お|み|わ}けなさる
\ruby[g]{此方樣}{こちらさま}だもの、
\ruby{可憐}{か|はい}いからつて
\ruby{御前}{お|まへ}の
\ruby[g]{味方}{みかた}にはなつて
\ruby{下}{くだ}さらない、すつかりと
\ruby{既妾}{もう|わたし}の
\ruby[g]{味方}{みかた}になり
\ruby{切}{き}つて
\ruby{下}{くだ}すつたのだよ。
\ruby{彼樣}{あ|ん}なところに
\ruby{居}{ゐ}るのなんぞは
\ruby{全}{まつた}くお
\ruby{前}{まへ}が
\ruby{惡}{わる}い、と
\ruby{散々}{さん|〴〵}に
\ruby{仰}{おつし}あつて、
\ruby[g]{彼家}{あすこ}を
\ruby{出}{で}させるやうにとの
\ruby[g]{御思召}{おぼしめし}なのだ。
\ruby{然}{しか}し
\ruby{何}{なに}も
\ruby{態々}{わざ|〳〵}とムキになつて
\ruby{惡}{わる}い
\ruby{奴}{やつ}を
\ruby{相手}{あひ|て}に
\ruby{爭}{や}り
\ruby{合}{あ}つても
\ruby{仕方}{し|かた}が
\ruby{無}{な}からう、お
\ruby{前}{まへ}が
\ruby{彼}{あ}の
\ruby{御師匠}{お|し|よ}さんていふ
\ruby{人}{ひと}の
\ruby{腹}{おなか}さへ
\ruby{解}{よ}めたら
\ruby[g]{彼家}{あすこ}に
\ruby{居}{ゐ}やう
\ruby{氣}{き}も
\ruby{有}{あ}るまいから、
\ruby{力}{ちから}を入れてお
\ruby{前}{まへ}を
\ruby{椀}{も}ぎ
\ruby{取}{と}りに
\ruby{行}{い}かなくつても
\ruby{濟}{す}む
\ruby{譯}{わけ}だ、と
\ruby{仰}{おつし}あつて
\ruby{下}{くだ}すつたから、
\ruby{成程}{なる|ほど}と
\ruby{妾}{わたし}も
\ruby{思}{おも}ひついて、
\ruby{何}{なに}も
\ruby{老年}{とし|より}が
\ruby{皺}{しわ}つ
\ruby{顏}{かほ}へ
\ruby{筋}{すぢ}を
\ruby{立}{た}てゝ
\ruby{喧嘩}{けん|くわ}しずとも
\ruby{濟}{す}むことならば、と
\ruby{狸婆}{たぬき|ばゞ}の
\ruby{面}{つら}の
\ruby{皮}{かは}を
\ruby{拗}{むし}りに
\ruby{行}{い}くことだけは
\ruby{思}{おも}ひ
\ruby{止}{と}まつたが、』

\ruby{此處}{こ|ゝ}まで
\ruby{語}{かた}れる
\ruby{時}{とき}、お
\ruby{彤}{とう}は
\ruby{後}{あと}を
\ruby{取}{と}つて、

『で、ネエ、お
\ruby{龍}{りう}ちやん、
\ruby{叔母}{を|ば}さんも
\ruby{實}{じつ}のところは、お
\ruby{前}{まへ}を
\ruby{直}{すぐ}に
\ruby{前}{ぜん}のやうにまた
\ruby{{\換字{連}}}{つ}れて
\ruby{歸}{かへ}つても、
\ruby{何樣}{ど|う}も
\ruby[g]{田舍}{ゐなか}の
\ruby{人}{ひと}は
\ruby{{\換字{嫌}}}{きら}ひだなんて
\ruby{云}{い}つて
\ruby{取}{と}つて
\ruby{{\換字{遣}}}{や}る
\ruby{婿}{むこ}を
\ruby{{\換字{嫌}}}{きら}ふやうでは
\ruby{始末}{し|まつ}が
\ruby{着}{つ}かないからつて、あぐんで
\ruby{居}{ゐ}らつしやるのだから、そこで
\ruby{妾}{わたし}が
\ruby{叔母}{を|ば}さんに
\ruby{對}{むか}つて、
\ruby{何樣}{ど|う}にでも
\ruby{彼樣}{あ|ん}な
\ruby{可厭}{い|や}な
\ruby{人}{ひと}の
\ruby{傍}{そば}からお
\ruby{龍}{りう}さんを
\ruby{離}{はな}して
\ruby{御仕舞}{お|し|ま}ひなさるのは
\ruby{其}{そ}りやあ
\ruby{宜}{よ}うございましやうが、それもお
\ruby{龍}{りう}さんが
\ruby{彼}{あ}の
\ruby{御師匠}{お|し|よ}さんの
\ruby{腹}{おなか}の
\ruby{惡}{わる}いのを
\ruby{自分}{じ|ぶん}から
\ruby{氣}{き}が
\ruby{付}{つ}いてで
\ruby{無}{な}くちやあ
\ruby{可}{い}けません。
それから
\ruby[g]{田舍}{ゐなか}へ
\ruby{{\換字{連}}}{つ}れて
\ruby{御歸}{お|かへ}りなさるのも、
\ruby[g]{矢張}{やつぱ}りお
\ruby{龍}{りう}さんが
\ruby{其}{そ}の
\ruby{氣}{き}にならなけりやあ、
\ruby[g]{末始{\換字{終}}}{すへしじう}が
\ruby{詰}{つま}りますまい。
\ruby{妾}{わたし}のところへ
\ruby{來}{き}て
\ruby{氣樂}{き|らく}に
\ruby{{\換字{遊}}}{あそ}んで
\ruby{居}{ゐ}るのが
\ruby{一番}{いち|ばん}お
\ruby{龍}{りう}さんの
\ruby{利{\換字{盆}}}{た|め}だとも
\ruby{思}{おも}ふし、
\ruby{{\換字{又}}妾}{また|わたし}が
\ruby{此樣}{こ|ん}な
\ruby{境遇}{ざ|ま}で
\ruby{居}{ゐ}ながら
\ruby[g]{立派}{りつぱ}な
\ruby{口}{くち}をきくのでは
\ruby[g]{夢更無}{ゆめさらな}いけれども、
\ruby{其}{そ}の
\ruby{中}{うち}には
\ruby{末々}{すへ|〴〵}のお
\ruby{龍}{りう}さんの
\ruby{身}{み}の
\ruby{收}{をさ}まりも
\ruby{妾}{わたし}の
\ruby{分別}{ふん|べつ}や
\ruby{力}{ちから}て
\ruby{出來}{で|き}るだけは
\ruby{仕}{し}て
\ruby{上}{あ}げたいともおもひますが、これもお
\ruby{龍}{りう}さんが
\ruby{妾}{わたし}のところへ
\ruby{來}{き}て
\ruby{居}{ゐ}るのを
\ruby{{\換字{嫌}}}{きら}つちやあ
\ruby{仕方}{し|かた}は
\ruby{無}{な}いし、
\ruby{若}{も}し
\ruby[g]{{\換字{又}}餘所}{またよそ}の
\ruby{堅}{かた}いところへ
\ruby[g]{奉公住}{ほうこうず}みでも
\ruby{仕}{し}やうといふやうな
\ruby{氣}{き}でもあるなら、それもお
\ruby{龍}{りう}さんの
\ruby{料簡次第}{れう|けん|し|だい}だし、
\ruby{{\換字{又}}些}{また|すこし}は
\ruby{遲}{おそ}けれども
\ruby[g]{此節柄}{このせつがら}の
\ruby{事}{こと}では
\ruby{有}{あ}り、
\ruby[g]{學校{\換字{通}}}{がくかうがよ}ひでも
\ruby{仕}{し}て、
\ruby{何}{なん}でも
\ruby{女一人}{をん|なひ|とり}で
\ruby{人}{ひと}の
\ruby{世話}{せ|わ}にならずに
\ruby{{\換字{遣}}}{や}つて
\ruby{行}{ゆ}かうといふのなら、それも
\ruby{其}{それ}で
\ruby{妾}{わたし}の
\ruby{手}{て}で
\ruby{三年}{さん|ねん}や
\ruby{五年}{ご|ねん}は
\ruby[g]{蝦茶袴}{えびちやばかま}さんで
\ruby{{\換字{過}}}{すご}させても
\ruby{上}{あ}げたいと
\ruby{思}{おも}ひますから、
\ruby{何事}{なに|ごと}も
\ruby[g]{無理壓制}{むりおしつけ}は
\ruby{可}{い}けません、ようく
\ruby{當人}{たう|にん}の
\ruby[g]{所存}{おなか}もゆつくりと
\ruby{聞}{き}いて
\ruby{見}{み}て、
\ruby{其}{そ}の
\ruby{上}{うへ}で
\ruby{何樣}{ど|う}ともする
\ruby{方}{はう}が
\ruby{宜}{よ}うございます。
お
\ruby{師匠}{し|よ}さんといふ
\ruby{人}{ひと}にやあ、お
\ruby{金}{かね}を
\ruby{{\換字{遣}}}{よこ}せなら
\ruby{{\換字{遣}}}{や}つても
\ruby{宜}{よ}うございますが、
\ruby{餘}{あま}り
\ruby{仕方}{し|かた}が
\ruby{憎}{にく}いから、お
\ruby{金}{かね}は
\ruby{惜}{をし}くは
\ruby{無}{な}いけれ
\ruby{共奪}{ども|と}られるのは
\ruby{業腹}{ごふ|はら}です、お
\ruby{龍}{りう}さんの
\ruby[g]{心次第}{こゝろしだい}で、
\ruby{何樣}{ど|う}とも
\ruby{仕}{し}て
\ruby{{\換字{遣}}}{や}りましやうつて、
\ruby{斯樣}{か|う}いつて
\ruby{妾}{わたし}あ
\ruby{御挨拶}{ご|あい|さつ}を
\ruby{仕}{し}たのだよ。
』

と、
\ruby{張}{は}りも
\ruby{弛}{ゆる}みもせぬ
\ruby{例}{れい}の
\ruby{調子}{てう|し}
に
\ruby{{\換字{述}}}{の}べたり。

