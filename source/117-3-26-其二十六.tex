\Entry{其二十六}

% メモ 校正終了 2024-05-15 2024-06-11
\原本頁{139-4}%
『
だが
お
\ruby{龍}{りう}、
%
お
\ruby{聞}{き}きなさい、
%
\ruby{妾}{わたし}あ
\ruby[g]{敵手}{あひて }が
\ruby{角}{つの}で
\ruby{向}{むか}つて
\ruby{來}{く}りやあ
\ruby[g]{此方}{こつち }も% ルビ調整(原本通り)非グループルビ
\ruby{角}{つの}で
\ruby{向}{むか}つて
\ruby{行}{い}く
けれど、
%
お
\ruby{{\換字{前}}}{まへ}
のやうに
\ruby{眞}{しん}になつて
\ruby[g]{世話}{せ わ }を
\ruby{仕}{し}て
\ruby{吳}{く}れる
\ruby[g]{叔母}{を ば }にも
\ruby[g]{自{\換字{分}}}{じ ぶん}の% ルビ調整(原本通り)非グループルビ
\ruby[g]{{\換字{勝}}手}{かつて }
ぢやあ
お
\ruby{尻}{しり}を
\ruby{向}{む}けたり、
%
\ruby[g]{折角}{せつかく}
\ruby{優}{やさ}しく
\ruby{仕}{し}て
\ruby{下}{くだ}さる
\ruby[|g|]{此方}{こちら}
\ruby{樣}{さま}をも
\ruby{時}{とき}の
\ruby[g]{都合}{つ がふ}
ぢやあ
\ruby{袖}{そで}に
するやうな、
%
\ruby[g]{其樣}{そ ん }な
\ruby{自{\換字{分}}{\換字{勝}}手}{じ|ぶん|かつ|て}% ルビ調整(原本通り)非グループルビ
ばかりは
\ruby{夢}{ゆめ}にも
\ruby{仕}{し}ません。
%
お
\ruby{{\換字{前}}}{まへ}は
\ruby{何}{なん}ぞに
\ruby{付}{つ}けちやあ
\改行% 校正作業の簡略化のため
、
%
\原本頁{139-9}\改行%
\ruby[g]{叔母}{を ば }さんは
\ruby[g]{無理}{む り }
\ruby[g]{壓制}{おしつけ}だ、
%
\ruby[g]{頑固}{ぐわんこ}だ、
%
\ruby[g]{自{\換字{分}}}{じ ぶん}% ルビ調整(原本通り)非グループルビ
\ruby[g]{流義}{りうぎ }で
\ruby{何}{なん}でも
\ruby{押}{お}して
\ruby{行}{ゆ}かう
とすると
\ruby[g]{御云}{お い }ひ
だが、
%
そりやあ
\ruby[g]{頑固}{ぐわんこ}
でもあらう、
%
\ruby[g]{自{\換字{分}}}{じ ぶん}% ルビ調整(原本通り)非グループルビ
\ruby[g]{流義}{りうぎ }
でもあらう、
%
\ruby{然}{しか}し
\ruby{恩}{おん}は
\ruby{恩}{おん}、
%
\ruby{仇}{あだ}は
\ruby{仇}{あだ}で
ちやんと
\ruby{記}{おぼ}えて% 送り仮名は原本通り「え」
\ruby{居}{ゐ}ます、
%
お
\ruby{{\換字{前}}}{まへ}のやうに
\ruby{恩}{おん}も
\ruby{仇}{あだ}も
\ruby{見}{み}さかひの
\ruby{無}{な}い
\ruby{事}{こと}は
\ruby{妾}{わたし}あ
\ruby{仕}{し}ません。
%
だから
\ruby{今}{いま}
\原本頁{140-3}\改行%
その
お
\ruby{關}{せき}つていふ
\ruby{奴}{やつ}の
ところへ
\ruby{押}{お}し
\ruby{{\換字{込}}}{こ}んで
\ruby{行}{い}つて、
%
\ruby[|g|]{田舎}{ゐなか}
\ruby[g]{婆は}{ばゞあ }
\ruby[|g|]{田舎}{ゐなか}% ルビ調整(原本通り)非グループルビ
\ruby[g]{婆だ}{ばゞあ }
けの
\ruby[g]{意地}{い ぢ }も
\ruby{有}{あ}りやあ
\ruby[||j>]{根}{こん}
\ruby[||j>]{性}{じやう}つ
% \ruby{根性}{こん|じやう}つ
\ruby{骨}{ほね}も
\ruby[g]{突張}{つゝぱ }つて
ゐる
ところを
\ruby{見}{み}せつけて
\ruby{{\換字{遣}}}{や}つて、
%
\ruby[g]{間{\換字{違}}}{ま ちが}つたことは
\ruby{云}{い}はない
\ruby{妾}{わたし}だもの
\ruby[g]{何負}{なにま }けるものか、
%
\ruby{思}{おも}ふさま
\ruby{{\換字{捩}}}{ね}ぢ
\ruby{合}{あ}つて
\ruby{{\換字{捩}}}{ね}ぢ
\ruby{合}{あ}ひ
\ruby{拔}{ぬ}いて、
%
\ruby[g]{{\換字{勝}}鬨}{と き }を
\ruby{吐}{ふ}いて
\ruby{歸}{かへ}らうと
\ruby{思}{おも}つたが、
%
まづ
\ruby{其}{そ}の
\ruby{{\換字{前}}}{まへ}に
\ruby[|g|]{此方}{こちら}
\ruby{樣}{さま}
 %
\footnote{「\ruby{樣}{さま}」に続く文字は原本では「に」の断片のように見えるが判読不可のため空白とする
(国会図書館 コマ番号 74 / 146 p-140 l-07)}%
\ruby{伺}{うかゞ}つて、
%
\ruby[g]{段々}{だん〴〵}
\ruby{御世話}{お|せ|わ}
になつた
\ruby[g]{御禮}{お れい}も
\ruby{云}{い}つたり、
%
また
お
\ruby{{\換字{前}}}{まへ}が
\ruby[g]{我儘}{わがまゝ}に
\ruby[|g|]{此方}{こちら}
\ruby{樣}{さま}を
\ruby{出}{で}て
\ruby{御親切}{ご|しん|せつ}を
\原本頁{140-9}\改行%
\ruby{無}{む}にした
\ruby{御謝罪}{お|わ|び}も
\ruby{仕}{し}たり、
%
\ruby[g]{一應}{いちおう}は
\ruby[|g|]{此方}{こちら}
\ruby{樣}{さま}の
\ruby{御思召}{お|ぼし|めし}も
\ruby{伺}{うかゞ}つてから
\改行% 校正作業の簡略化のため
、
%
\原本頁{140-10}\改行%
それから
\ruby{爭}{や}り
\ruby{合}{あ}ふなら
\ruby{爭}{や}り
\ruby{合}{あ}はなくつては
\ruby[g]{義理}{ぎ り }が
\ruby{惡}{わる}いと、
%
それで
\ruby[g]{突掛}{つゝか }けに
\ruby[|g|]{此方}{こちら}
\ruby{樣}{さま}へ
\ruby{伺}{うかゞ}つて、
%
\ruby[g]{御噂}{おうはさ}に
ばかり
\ruby{伺}{うかゞ}つて
\ruby{居}{ゐ}た
\ruby{方}{かた}に
はじめて
\ruby[g]{御目}{お め }に
かゝつたのだよ。
%
ところが、
%
これ
お
\ruby{龍}{りう}、
%
お
\ruby{聞}{き}きなさいよ。
%
\ruby[g]{{\換字{道}}理}{だうり }に% ルビ調整(補正)国書データベースでは(う)の印字が見えない
\ruby{{\換字{違}}}{ちが}つたことを
\ruby{云}{い}は
\ruby{無}{な}いものは
\ruby[g]{何處}{ど こ }にでも
\ruby[g]{味方}{み かた}が
あります。
%
いろ〳〵と
お
\ruby{{\換字{前}}}{まへ}の
ことを
\ruby[g]{御話}{お はな}し
\ruby{申}{まを}した
ところ、
%
\makeatletter
\@ifundefined{デバッグ@ビルド}{%
  \ruby[<g>]{悉皆}{すつかり}
}{%
  \ruby[<j||]{悉}{すつか}% 行末行頭の境界付近なので特例処置を施す
  \ruby[<j||]{皆}{ り }% 行末行頭の境界付近なので特例処置を施す
}%
\makeatother
\ruby{妾}{わたし}の% 行末行頭の境界付近なので特例処置を施す
\ruby{云}{い}ふことを
\ruby[g]{{\換字{道}}理}{もつとも}だと
\ruby{仰}{おつし}あつて
\ruby{下}{くだ}すつて、
%
お
\ruby{{\換字{前}}}{まへ}は
\ruby{何}{なん}ぞの
\ruby{時}{とき}には
\原本頁{141-5}\改行%
\ruby[|g|]{此方}{こちら}
\ruby{樣}{さま}を
\ruby{楯}{たて}に
\ruby{取}{と}つて、
%
\ruby{妾}{わたし}の
\ruby{云}{い}ふ
\ruby{事}{こと}を
\ruby{肯}{き}くまい
なんぞと
\ruby{思}{おも}つてるか
\ruby{知}{し}らないが、
%
もう
\ruby[g]{然樣}{さ う }は
\ruby{行}{い}きません
\ruby{御生憎樣}{お|あい|にく|さま}!、% ルビ調整(原本通り)(おあ(い)にくさま)
%
\ruby[g]{何樣}{ど う }して% ルビ調整(原本通り)非踊り字表記(行末行頭の境界付近)
\ruby[g]{何樣}{ど う }して
\ruby[g]{{\換字{判}}然}{はつきり}と
\ruby{物}{もの}の
\ruby[g]{{\換字{道}}理}{だうり }を
\ruby{御見{\換字{分}}}{お|み|わ}け
なさる
\ruby[|g|]{此方}{こちら}
\ruby{樣}{さま}だもの、
%
\ruby[g]{可憐}{か はい}い% ルビ調整(原本通り)非グループルビ
\原本頁{141-8}\改行%
からつて
\ruby[g]{御{\換字{前}}}{お まへ}の
\ruby[g]{味方}{み かた}には
なつて
\ruby{下}{くだ}さらない、
%
すつかりと
\ruby{既}{もう}
\ruby{妾}{わたし}の
\原本頁{141-9}\改行%
\ruby[g]{味方}{み かた}になり
\ruby{切}{き}つて
\ruby{下}{くだ}すつたのだよ。
%
\ruby[g]{彼樣}{あ ん }な
ところに
\ruby{居}{ゐ}るのなんぞは
\ruby{全}{まつた}く
お
\ruby{{\換字{前}}}{まへ}が
\ruby{惡}{わる}い、
%
と
\ruby[g]{散々}{さん〴〵}に
\ruby[g]{仰あ}{おつし }
つて、
%
\ruby[|g|]{彼家}{あすこ}を
\ruby{出}{で}させる
やうにとの
\ruby{御思召}{お|ぼし|めし}
なのだ。
%
\ruby{然}{しか}し
\ruby{何}{なに}も
\ruby[g]{態々}{わざ〳〵}と
ムキになつて
\ruby{惡}{わる}い
\ruby{奴}{やつ}を
\ruby[g]{相手}{あひて }に
\ruby{爭}{や}り
\ruby{合}{あ}つても
\ruby[g]{仕方}{し かた}が
\ruby{無}{な}からう、
%
お
\ruby{{\換字{前}}}{まへ}が
\ruby{彼}{あ}の
\ruby{御師匠}{お|し|よ}さん
ていふ
\ruby{人}{ひと}の
\ruby{腹}{おなか}さへ
\ruby{解}{よ}めたら
\ruby[|g|]{彼家}{あすこ}に
\ruby{居}{ゐ}やう
\ruby{氣}{き}も
\ruby{有}{あ}るまいから、
%
\ruby{力}{ちから}を
\ruby{入}{い}れて
お
\ruby{{\換字{前}}}{まへ}を
\ruby{椀}{も}ぎ
\ruby{取}{と}りに
\ruby{行}{い}かなくつても
\ruby{濟}{す}む
\ruby{譯}{わけ}だ、
%
と
\ruby{仰}{おつし}あつて
\ruby{下}{くだ}すつたから、
%
\ruby[g]{成程}{なるほど}と
\ruby{妾}{わたし}も
\ruby{思}{おも}ひついて、
%
\ruby{何}{なに}も
\ruby[g]{老年}{としより}が
\ruby{皺}{しわ}つ
\ruby{顏}{かほ}へ
\ruby{筋}{すぢ}を
\原本頁{142-5}\改行%
\ruby{立}{た}てゝ
\ruby[g]{喧嘩}{けんくわ}しずとも
\ruby{濟}{す}む
ことならば、
%
と
\ruby[<j||]{狸}{たぬき}
\ruby[||j>]{婆}{ばゞあ}の
% \ruby{狸婆}{たぬき|ばゞあ}の
\ruby{面}{つら}の
\ruby{皮}{かは}を% 原本通り「皮 か(は)」
\ruby{拗}{むし}りに
\ruby{行}{い}くことだけは
\ruby{思}{おも}ひ
\ruby{止}{と}まつたが、
』

\原本頁{142-7}%
\ruby[g]{此處}{こ ゝ }まで
\ruby{語}{かた}れる
\ruby{時}{とき}、
%
お
\ruby{彤}{とう}は
\ruby{後}{あと}を
\ruby{取}{と}つて、

\原本頁{142-9}%
『
で、
%
ネエ、
%
お
\ruby{龍}{りう}ちやん、
%
\ruby[g]{叔母}{を ば }さんも
\ruby{實}{じつ}の
ところは、
%
お
\ruby{{\換字{前}}}{まへ}を
\ruby{直}{すぐ}に
\ruby{{\換字{前}}}{せん}のやうに
また
\ruby{{\換字{連}}}{つ}れて
\ruby{歸}{かへ}つても、
%
\ruby[g]{何樣}{ど う }も
\ruby[|g|]{田舎}{ゐなか}の
\ruby{人}{ひと}は
\ruby{{\換字{嫌}}}{きら}ひ
だなんて
\ruby{云}{い}つて
\ruby{取}{と}つて
\ruby{{\換字{遣}}}{や}る
\ruby{婿}{むこ}を% (婿 5a7f) 聟 805f
\ruby{{\換字{嫌}}}{きら}ふ
やうでは
\ruby[g]{始末}{し まつ}が
\ruby{着}{つ}かないからつて、
%
あぐんで
\ruby{居}{ゐ}らつしやる
のだから、
%
そこで
\ruby{妾}{わたし}が
\ruby[g]{叔母}{を ば }さんに
\ruby{對}{むか}つて、
%
\ruby[g]{何樣}{ど う }にでも
\ruby[g]{彼樣}{あ ん }な
\ruby[g]{可厭}{い や }な
\ruby{人}{ひと}の
\ruby{傍}{そば}から
お
\ruby{龍}{りう}さんを
\ruby{離}{はな}して
\ruby{御仕舞}{お|し|ま}ひ
なさるのは
\ruby{其}{そ}りやあ
\ruby{宜}{よ}う
ございましやうが、
%
それも
お
\ruby{龍}{りう}さんが
\ruby{彼}{あ}の
\ruby{御師匠}{お|し|よ}さんの
\ruby{腹}{おなか}の
\ruby{惡}{わる}いのを
\ruby[g]{自{\換字{分}}}{じ ぶん}から% ルビ調整(原本通り)非グループルビ
\ruby{氣}{き}が
\ruby{付}{つ}いてで
\ruby{無}{な}くちやあ
\ruby{可}{い}けません。
%
それから
\ruby[|g|]{田舎}{ゐなか}へ
\ruby{{\換字{連}}}{つ}れて
\ruby[g]{御歸}{お かへ}りなさるのも
\改行% 校正作業の簡略化のため
、
%
\原本頁{143-5}\改行%
\ruby[g]{矢張}{やつぱ }り% ルビ調整(原本通り)非グループルビ
お
\ruby{龍}{りう}さんが
\ruby{其}{そ}の
\ruby{氣}{き}に
ならなけりやあ、
%
\ruby{末始{\換字{終}}}{すゑ|し|じう}が% ルビ調整(原本通り)「ゆ」無し
\ruby{詰}{つま}りますまい。
%
\ruby{妾}{わたし}の
ところへ
\ruby{來}{き}て
\ruby[g]{氣樂}{き らく}に
\ruby{{\換字{遊}}}{あそ}んで
\ruby{居}{ゐ}るのが
\ruby[g]{一番}{いちばん}
お
\ruby{龍}{りう}さんの
\原本頁{143-7}\改行%
\ruby[g]{利益}{た め }だとも
\ruby{思}{おも}ふし、
%
\ruby{{\換字{又}}}{また}
\ruby{妾}{わたし}が
\ruby[g]{此樣}{こ ん }な
\ruby[g]{境{\換字{遇}}}{ざ ま }で
\ruby{居}{ゐ}ながら
\ruby[g]{立派}{りつぱ }な
\ruby{口}{くち}を
きくのでは
\ruby{夢}{ゆめ}
\ruby{{\換字{更}}}{さら}
\ruby{無}{な}いけれども、
%
\ruby{其}{そ}の
\ruby{中}{うち}には
\ruby[g]{末々}{すゑ〴〵}の
お
\ruby{龍}{りう}さんの
\ruby{身}{み}の
\原本頁{143-9}\改行%
\ruby{收}{をさ}まりも
\ruby[g]{妾の}{わたし }
\ruby[g]{{\換字{分}}別}{ふんべつ}や
\ruby{力}{ちから}て
\ruby[g]{出來}{で き }るだけは
\ruby{仕}{し}て
\ruby{上}{あ}げたいとも
おもひますが、
%
これも
お
\ruby{龍}{りう}さんが
\ruby{妾}{わたし}の
ところへ
\ruby{來}{き}て
\ruby{居}{ゐ}るのを
\ruby{{\換字{嫌}}}{きら}つちやあ
\ruby[g]{仕方}{し かた}は
\ruby{無}{な}いし、
%
\ruby{{\換字{若}}}{も}し
\ruby{{\換字{又}}}{また}
\ruby[g]{餘{\換字{所}}}{よ そ }の
\ruby{堅}{かた}い
ところへ
\ruby{奉公住}{ほう|こう|ず}みでも
\ruby{仕}{し}やう
といふ
やうな
\ruby{氣}{き}
でも
ある
なら、
%
それも
お
\ruby{龍}{りう}さんの
\ruby[g]{料簡}{れうけん}
\ruby[g]{次第}{し だい}だし、
%
\ruby{{\換字{又}}}{また}
\ruby{些}{すこし}は
\ruby{遲}{おそ}けれども
\ruby{此節柄}{この|せつ|がら}の
\ruby{事}{こと}では
\ruby{有}{あ}り、
%
\ruby{學校{\換字{通}}}{がく|かう|がよ}ひでも
\ruby{仕}{し}て
\改行% 校正作業の簡略化のため
、
%
\原本頁{144-3}\改行%
\ruby{何}{なん}でも
\ruby{女一人}{をんな|ひ|とり}で% ルビ調整(原本通り)非グループルビ
\ruby{人}{ひと}の
\ruby[g]{世話}{せ わ }に
ならずに
\ruby{{\換字{遣}}}{や}つて
\ruby{行}{ゆ}かう
といふのなら
\改行% 校正作業の簡略化のため
、
%
\原本頁{144-4}\改行%
それも
\ruby{其}{それ}で
\ruby{妾}{わたし}の
\ruby{手}{て}で
\ruby[g]{三年}{さんねん}や
\ruby[g]{五年}{ご ねん}は
\ruby[||j>]{蝦}{えび}
\ruby[||j>]{茶}{ちや}
\ruby[||j>]{袴}{ばかま}
さんで
\ruby{{\換字{過}}}{すご}させても
\ruby{上}{あ}げたいと
\ruby{思}{おも}ひますから、
%
\ruby[g]{何事}{なにごと}も
\ruby[g]{無理}{む り }
\ruby[g]{壓制}{おしつけ}は
\ruby{可}{い}けません、
%
ようく
\ruby[g]{當人}{たうにん}の
\ruby[|g|]{{\換字{所}}存}{おなか}
も
ゆつくりと
\ruby{聞}{き}いて
\ruby{見}{み}て、
%
\ruby{其}{そ}の
\ruby{上}{うへ}で
\ruby[g]{何樣}{ど う }ともする
\ruby{方}{はう}が
\原本頁{144-7}\改行%
\ruby{宜}{よ}うございます。
%
お
\ruby[g]{師匠}{し よ }さん
といふ
\ruby{人}{ひと}にやあ、
%
お
\ruby{金}{かね}を
\ruby{{\換字{遣}}}{よこ}せなら
\原本頁{144-8}\改行%
\ruby{{\換字{遣}}}{や}つても
\ruby{宜}{よ}うございますが、
%
\ruby{餘}{あま}り
\ruby[g]{仕方}{し かた}が
\ruby{憎}{にく}いから、
%
お
\ruby{金}{かね}は
\ruby{惜}{をし}くは
\ruby{無}{な}いけれ
\ruby{共}{ども}
\ruby{奪}{と}られるのは
\ruby[g]{業腹}{ごふはら}です、
%
お
\ruby{龍}{りう}さんの
\ruby{心次第}{こゝろ|し|だい}で、
%
\ruby[g]{何樣}{ど う }とも
\ruby{仕}{し}て
\ruby{{\換字{遣}}}{や}りましやうつて、
%
\ruby[g]{斯樣}{か う }いつて
\ruby{妾}{わたし}あ
\ruby{御挨拶}{ご|あい|さつ}を
\ruby{仕}{し}たのだよ。
』

\原本頁{145-1}%
と、
%
\ruby{張}{は}りも
\ruby{弛}{ゆる}みもせぬ
\ruby{例}{れい}の
\ruby[g]{調子}{てうし }
に
\ruby{{\換字{述}}}{の}べたり。
