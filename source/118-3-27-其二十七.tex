\Entry{其二十七}

『
\ruby{解}{わか}つたかエ
お
\ruby{龍}{りう}、まあ
\ruby{何}{なん}といふ
\ruby{有}{あ}り
\ruby{難}{がた}い
\ruby{御優}{お|やさ}しい
\ruby[g]{御思召}{おぼしめし}だらう。
\ruby[g]{小兒}{こども}の
\ruby{時}{とき}から
\ruby{可愛}{か|はい}がつて
\ruby{下}{くだ}すつた
\ruby{上}{うへ}、
お
\ruby{前}{まへ}は
\ruby{御恩}{ご|おん}に
\ruby{負}{そむ}いて
\ruby{狗猫}{いぬ|ねこ}のやうな
\ruby{事}{こと}を
\ruby{仕}{し}ても、
\ruby{別}{べつ}に
\ruby{愛想}{あい|そ}づかしも
\ruby{仕}{し}て
\ruby{下}{くだ}さらないで、
お
\ruby{前}{まへ}が
\ruby[g]{稽古事}{けいこごと}を
\ruby{仕}{し}たければ
\ruby{其}{それ}も
\ruby{爲}{さ}せて
\ruby{{\換字{遣}}}{や}らう、
\ruby{家}{うち}に
\ruby{居}{ゐ}たいなら
\ruby{家}{うち}に
\ruby{置}{お}いて
\ruby{{\換字{遣}}}{や}らう、
\ruby{末々}{すゑ|〴〵}の
\ruby{身}{み}の
\ruby{{\換字{終}}局}{をさ|まり}も
\ruby{頼}{たの}むなら
\ruby{心配}{しん|ぱい}して
\ruby{{\換字{遣}}}{や}らうと、
\ruby{斯樣}{こ|ん}なに
\ruby{親切}{しん|せつ}にして
\ruby{下}{くだ}さる
\ruby{方}{かた}が
\ruby{何處}{ど|こ}にあると
\ruby{御思}{おお|も}ひだ。
\ruby{早}{はや}く
\ruby{料簡}{れう|けん}を
\ruby{入}{い}れかへて
\ruby[g]{眞人間}{まにんげん}になつて、\換字{志}やんと
\ruby{女}{をんな}は
\ruby{女}{をんな}
\ruby{一人}{ひと|り}だけ
\ruby{羞}{はづ}かしくないやうな
\ruby{今日}{こん|にち}の
\ruby{{\換字{送}}}{おく}り
\ruby{方}{かた}をする
\ruby{身}{み}になつて、
\ruby[g]{御恩返}{ごおんがへ}しは
\ruby{出來無}{で|き|な}いまでも
\ruby{御親切}{ご|しん|せつ}を
\ruby{無}{む}に
\ruby{爲}{し}ないやう
\ruby{仕}{し}なければ、
\ruby{叔母}{を|ば}の
\ruby{此}{こ}の
\ruby{妾}{わたし}にやきもきと
\ruby[g]{幾干}{いくそ}の
\ruby{苦勞}{く|らう}させる、
\ruby{其}{そ}の
\ruby{罸}{ばち}はよしんば
お
\ruby{前}{まへ}に
\ruby{當}{あた}らない
\ruby{迄}{まで}も、
\ruby{此方樣}{こち|ら|さま}の
\ruby{罰}{ばち}が
\ruby[g]{末始{\換字{終}}}{すゑしじう}は
\ruby{屹度}{きつ|と}
\ruby{當}{あた}つて、
お
\ruby{前}{まへ}は
\ruby{碌}{ろく}な
\ruby{死狀}{しに|ざま}は
\ruby{出來}{で|き}ますまいよ。
\ruby{花}{はな}が
\ruby{奇麗}{き|れい}だ、
\ruby{蝶々}{てふ|〳〵}が
\ruby{可憐}{か|はい}い、
\ruby{人形}{にん|ぎやう}が
\ruby{氣}{き}に
\ruby{入}{い}つたなんぞと、
\ruby{其樣}{そ|ん}な
\ruby{下}{くだ}らない
\ruby{{\換字{浮}}々}{うか|〳〵}としたことを
\ruby{云}{い}つて
\ruby{居}{ゐ}て
\ruby{{\換字{過}}}{すご}せるものぢや
\ruby{無}{な}い
\ruby{世}{よ}の
\ruby{中}{なか}だから、
\ruby{宜}{い}い
\ruby{加減}{か|げん}に
\ruby{目}{め}を
\ruby{覺}{さ}まして
\ruby{確乎}{しつ|かり}とした
\ruby{氣}{き}になつて、
\ruby{片目}{めつ|かち}でも
\ruby[g]{跛足}{びつこ}でも
\ruby{構}{かま}はないから
\ruby{食}{く}ふに
\ruby{困}{こま}らない
\ruby{男}{をとこ}を
\ruby{持}{も}つて、そして
\ruby{子}{こ}でも
\ruby{生}{う}んで
\ruby{末}{すゑ}の
\ruby{安堵}{おち|つき}を
\ruby{見}{み}るやうに
\ruby{仕無}{し|な}くつては
\ruby{濟}{す}む
\ruby{譯}{わけ}ぢやあ
\ruby{無}{な}い。
\ruby{自惚}{うぬ|ぼ}れて
\ruby{居}{ゐ}たつて
\ruby{可}{い}けは
\ruby{仕}{し}ない、
\ruby[g]{{\換字{情}}夫}{をとこ}に
\ruby{棄}{す}てられる
\ruby{位}{ぐらゐ}の
\ruby{容貌}{きり|やう}で
\ruby{居}{ゐ}て、
\ruby{飛}{と}び
\ruby{拔}{ぬ}けて
\ruby{何}{なに}が
\ruby{一}{ひと}つ
\ruby{出來}{で|き}るでも
\ruby{無}{な}い
\ruby[g]{天禀}{うまれつき}の
お
\ruby{前}{まへ}なんぞは、
\ruby{自{\換字{分}}}{じ|ぶん}で
\ruby{理屈}{り|くつ}を
\ruby{付}{つ}けりやあ
\ruby{理屈}{り|くつ}も
\ruby{有}{あ}るだらうが、
\ruby{世界}{せ|かい}から
\ruby{云}{い}つて
\ruby{見}{み}りやあ
\ruby[g]{圃中}{はたけ}の
\ruby[g]{蠻南瓜}{たうなす}か
\ruby{茄子}{な|す}か
\ruby{白瓜}{しろ|うり}で、
\ruby{何樣}{ど|う}せ
\ruby[g]{其邊中}{そこらぢう}にある
\ruby{數物}{かず|もの}なのだもの、
\ruby{好}{い}い
\ruby[g]{加減}{かげん}に
\ruby{熟}{で}きた
\ruby{時{\換字{分}}}{じ|ぶん}に
\ruby{何樣}{ど|う}かなつて
\ruby{仕舞}{し|ま}ふのが
\ruby[g]{當然}{あたりまへ}の
\ruby{事}{こと}で、
\ruby{早{\換字{速}}}{さつ|さ}と
\ruby{緣}{えん}のあるところへ
\ruby{行}{い}つて
\ruby{一代働}{いち|だい|はた}らいて、
\ruby{種子}{た|ね}でも
\ruby{{\換字{遺}}}{のこ}すより
\ruby{他}{ほか}にいざもこざも
\ruby{有}{あ}りやあ
\ruby{仕}{し}ないのだよ。
だから
\ruby{妾}{わたし}が
\ruby{其}{そ}の
\ruby{積}{つも}りで
\ruby{世話}{せ|わ}を
\ruby{燒}{や}いて
\ruby{{\換字{遣}}}{や}つたのに、
\ruby{何}{なん}だの
\ruby{彼}{か}だのとだゞを
\ruby{捏}{こ}ねて
\ruby{妾}{ひと}を
\ruby[g]{御困}{おこま}らせだつたが、
\ruby{其}{それ}もまあ
\ruby{緣}{えん}が
\ruby{無}{な}かつたのだと
\ruby{其}{そ}の
\ruby{事}{こと}は
\ruby{濟}{す}まして
\ruby{仕舞}{し|ま}つたところで。
\ruby[g]{蠻南瓜}{たうなす}を
\ruby[g]{眞綿}{まわた}に
\ruby{包}{くる}んで
\ruby{藏}{しま}ひ
\ruby{通}{とほ}したつて
\ruby{何}{なん}になるものでもない、
\ruby{矢張}{やつ|ぱり}
\ruby{何樣}{ど|う}かして
\ruby{片}{かた}づくところへ
\ruby{片}{かた}づけてやつて、
\ruby{持}{も}つて
\ruby{生}{うま}れた
\ruby{役}{やく}を
\ruby{濟}{す}まさせなけりやあなら
\ruby{無}{な}いから、そこで
\ruby{妾}{わたし}が
お
\ruby{願}{ねがひ}を
\ruby{仕}{し}て、それでは
\ruby{靜岡}{しづ|をか}に
\ruby{{\換字{連}}}{つ}れて
\ruby{歸}{かへ}ることは
\ruby{廢案}{や|め}に
\ruby{仕}{し}まして、
\ruby{御甘}{お|あま}え
\ruby{申}{まを}して
\ruby{濟}{す}みませんが
\ruby{何樣}{ど|う}か
\ruby{此方樣}{こち|ら|さま}で
\ruby[g]{御使}{おつか}ひなすつて
\ruby{頂}{いたゞ}きたうございます、
\ruby{何}{なん}でも
\ruby{手}{て}や
\ruby{足}{あし}に
\ruby{皸垢切}{ひび|あか|ぎれ}のきれますやうにこき
\ruby{使}{つか}つて
\ruby{下}{くだ}さいまして、
\ruby{其}{そ}の
\ruby{中}{うち}に
\ruby{破鍋}{われ|なべ}に
\ruby{綴蓋}{とぢ|ぶた}で、
\ruby{彼樣}{あ|ん}な
\ruby{奴}{やつ}ても
\ruby{貰}{もら}つて
\ruby{{\換字{遣}}}{や}らうといふ
\ruby{方}{かた}でもございましたら、
\ruby{此方樣}{こち|ら|さま}の
\ruby[g]{御鑑識次第}{おめがねしだい}で
\ruby[g]{豆腐屋}{とうふや}へでも
\ruby[g]{炭團屋}{たどんや}へでも
\ruby{何}{なん}でも
\ruby{宜}{よろ}しうございますから
\ruby{身}{み}を
\ruby{固}{かた}めさせて
\ruby{頂}{いたゞ}きたうございます、と
\ruby{斯樣}{か|う}いつて
\ruby{妾}{わたし}が
\ruby{御願}{お|ねが}ひ
\ruby{申}{まを}して
\ruby{居}{ゐ}るのですよ。
もう
\ruby{可}{い}けません、
\ruby{我儘}{わが|まゝ}は
\ruby{云}{い}はせません、
\ruby{何}{なん}でも
\ruby{彼}{かん}でも
\ruby{妾}{わたし}の
\ruby{云}{い}ふ
\ruby{通}{とほ}りに
\ruby{此方樣}{こち|ら|さま}の
\ruby{御世話}{お|せ|わ}を
\ruby{御願}{お|ねが}ひなさい。
\ruby{朝}{あさ}は
\ruby{昧}{くら}いから
\ruby{起}{お}きて
\ruby{夜遲}{よる|おそ}くまで、
\ruby{火}{ひ}も
\ruby{焚}{た}き
\ruby{水}{みづ}も
\ruby{汲}{く}み、
\ruby[g]{炊事雜巾掛}{にたきざうきんが}け、
\ruby{何}{なに}から
\ruby{何}{なに}まで
\ruby[g]{御奉公人}{ごほうこうにん}と
\ruby{勵}{はげ}み
\ruby{合}{あ}つて
\ruby{働}{はたら}かなくてはいけません。
\ruby{{\換字{嫌}}}{いや}だなんぞと
\ruby{云}{い}つても
\ruby[g]{既承知仕}{もうしようちし}ません。
さあ
\ruby[g]{丁度宜}{ちやうどい}い、
\ruby{妾}{わたし}と
\ruby{一緖}{いつ|しよ}に、
\ruby{{\換字{判}}然}{はつ|きり}と
\ruby{改}{あらた}めて
\ruby[g]{今後}{こんご}の
\ruby{御世話}{お|せ|わ}を
\ruby{御願}{お|ねが}ひ
\ruby{御仕}{お|し}なさい。
\ruby{考}{かんが}へて
\ruby{居}{ゐ}る
\ruby{事}{こと}も
\ruby{何}{なに}も
\ruby{有}{あ}りは
\ruby{仕}{し}ません。
』

