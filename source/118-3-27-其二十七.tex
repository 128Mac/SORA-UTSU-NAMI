\Entry{其二十七}

% メモ 校正終了 2024-05-15 2024-06-11
\原本頁{145-3}%
『
\ruby{解}{わか}つたかエ
お
\ruby{龍}{りう}、
%
まあ
\ruby{何}{なん}といふ
\ruby{有}{あ}り
\ruby{{\換字{難}}}{がた}い
\ruby{御優}{お|やさ}しい
\ruby{御思召}{お|ぼし|めし}
だらう。
%
\ruby{小兒}{こ|ども}の% ルビ調整(原本通り)非グループルビ
\ruby{時}{とき}から
\ruby{可愛}{か|はい}がつて
\ruby{下}{くだ}すつた
\ruby{上}{うへ}、
%
お
\ruby{{\換字{前}}}{まへ}は
\ruby{御恩}{ご|おん}に
\ruby{負}{そむ}いて
\原本頁{145-5}\改行%
\ruby{狗猫}{いぬ|ねこ}のやうな
\ruby{事}{こと}を
\ruby{仕}{し}ても、
%
\ruby{別}{べつ}に
\ruby{愛想}{あい|そ}づかしも
\ruby{仕}{し}て
\ruby{下}{くだ}さらないで
\改行% 校正作業の簡略化のため
、
%
\原本頁{145-6}\改行%
お
\ruby{{\換字{前}}}{まへ}が
\ruby{稽{\換字{古}}事}{けい|こ|ごと}を
\ruby{仕}{し}たければ
\ruby{其}{それ}も
\ruby{爲}{さ}せて
\ruby{{\換字{遣}}}{や}らう、
%
\ruby{家}{うち}に
\ruby{居}{ゐ}たいなら
\原本頁{145-7}\改行%
\ruby{家}{うち}に
\ruby{置}{お}いて
\ruby{{\換字{遣}}}{や}らう、
%
\ruby[g]{末々}{すゑ〴〵}の
\ruby{身}{み}の
\ruby{{\換字{終}}局}{をさ|まり}も
\ruby{頼}{たの}むなら
\ruby{心配}{しん|ぱい}して
\ruby{{\換字{遣}}}{や}らうと、
%
\ruby{斯樣}{こ|ん}なに
\ruby{親切}{しん|せつ}にして
\ruby{下}{くだ}さる
\ruby{方}{かた}が
\ruby{何處}{ど|こ}にあると
\ruby{御思}{お|おも}ひだ。
%
\ruby{早}{はや}く
\ruby{料簡}{れう|けん}を
\ruby{入}{い}れかへて
\ruby{眞人間}{ま|にん|げん}に
なつて、
\換字{志}やんと
\ruby[||j>]{女}{をんな}は
\ruby[||j>]{女}{をんな}%
\ruby[||j>]{一}{ ひと}
\ruby[||j>]{人}{ り}だけ% ルビ調整(原本通り)非グループルビ
\原本頁{145-10}\改行%
\ruby{羞}{はづ}かしくない
やうな
\ruby{今日}{こん|にち}の
\ruby{{\換字{送}}}{おく}り
\ruby{方}{かた}を
する
\ruby{身}{み}になつて、
%
\ruby{御恩{\換字{返}}}{ご|おん|がへ}しは
\ruby{出來無}{で|き|な}いまでも
\ruby{御親切}{ご|しん|せつ}を
\ruby{無}{む}に
\ruby{爲}{し}ないやうに
\ruby{仕}{し}なければ、
%
\ruby{叔母}{を|ば}の
\ruby{此}{こ}の
\ruby{妾}{わたし}に
やきもきと
\ruby{幾干}{いく|そ}の% ルビ調整(原本通り)非グループルビ
\ruby{苦勞}{く|らう}させる、
%
\ruby{其}{そ}の
\ruby{罸}{ばち}は
よしんば
お
\ruby{{\換字{前}}}{まへ}に
\ruby{當}{あた}らない
\ruby{迄}{まで}も、
%
\ruby[|g|]{此方}{こちら}
\ruby{樣}{さま}の
\ruby{罰}{ばち}が
\ruby{末始{\換字{終}}}{すゑ|し|じう}は% ルビ調整(原本通り)「ゆ」無し
\ruby{屹度}{きつ|と}% ルビ調整(原本通り)非グループルビ
\ruby{當}{あた}つて、
%
お
\ruby{{\換字{前}}}{まへ}は
\原本頁{146-4}\改行%
\ruby{碌}{ろく}な
\ruby{死狀}{しに|ざま}は
\ruby{出來}{で|き}ますまいよ。
%
\ruby{花}{はな}が
\ruby{奇麗}{き|れい}だ、
%
\ruby[g]{蝶々}{てふ〳〵}が
\ruby{可憐}{か|はい}い、% ルビ調整(原本通り)非グループルビ
%
\ruby[<j||]{人}{にん }% 行末行頭の境界付近なので特例処置を施す
\ruby[<j||]{形}{ぎやう}が
% \ruby{人形}{にん|ぎやう}が
\ruby{氣}{き}に
\ruby{入}{い}つたなんぞと、
%
\ruby{其樣}{そ|ん}な
\ruby{下}{くだ}らない
\ruby{{\換字{浮}}々}{うか|〳〵}
としたことを
\ruby{云}{い}つて
\ruby{居}{ゐ}て
\ruby{{\換字{過}}}{すご}せるものぢや
\ruby{無}{な}い
\ruby{世}{よ}の
\ruby{中}{なか}
だから、
%
\ruby{宜}{い}い
\ruby{加減}{か|げん}に
\ruby{目}{め}を
\ruby{覺}{さ}まして
\ruby{確乎}{しつ|かり}
とした
\ruby{氣}{き}
になつて、
%
\ruby{片目}{めつ|かち}でも
\ruby{跛足}{びつ|こ}でも
\ruby{構}{かま}はないから
\ruby{食}{く}ふに
\ruby{困}{こま}らない
\ruby{男}{をとこ}を
\ruby{持}{も}つて、
%
そして
\ruby{子}{こ}でも
\ruby{生}{う}んで
\ruby{末}{すゑ}の
\ruby{安堵}{おち|つき}を
\ruby{見}{み}るやうに
\ruby{仕無}{し|な}くつては
\ruby{濟}{す}む
\ruby{譯}{わけ}ぢやあ
\ruby{無}{な}い。
%
\ruby{自惚}{うぬ|ぼ}れて
\ruby{居}{ゐ}たつて
\ruby{可}{い}けは
\ruby{仕}{し}ない、
%
\ruby[|g|]{{\換字{情}}夫}{をとこ}に
\ruby{棄}{す}てられる
\ruby{位}{ぐらゐ}の
\ruby{容貌}{きり|やう}で
\ruby{居}{ゐ}て、
%
\ruby{飛}{と}び
\ruby{拔}{ぬ}けて
\ruby{何}{なに}が
\原本頁{146-11}\改行%
\ruby{一}{ひと}つ
\ruby{出來}{で|き}るでも
\ruby{無}{な}い
\ruby[||j>]{天}{うま}
\ruby[||j>]{禀}{れつき}の
% \ruby{天禀}{うま|れつき}の
お
\ruby{{\換字{前}}}{まへ}
なんぞは、
%
\ruby{自{\換字{分}}}{じ|ぶん}で% ルビ調整(原本通り)非グループルビ
\ruby{理屈}{り|くつ}を
\ruby{付}{つ}けりやあ
\ruby{理屈}{り|くつ}も
\ruby{有}{あ}るだらうが、
%
\ruby{世界}{せ|かい}から
\ruby{云}{い}つて
\ruby{見}{み}りやあ
\ruby[|g|]{圃中}{はたけ}の
\ruby[|g|]{蠻南}{たうな}% 行末行頭の境界付近なので特例処置を施す
\ruby{蠻}{す}か
\ruby{茄子}{な|す}か
\ruby{白瓜}{しろ|うり}で、
%
\ruby{何樣}{ど|う}せ
\ruby[|g|]{其邊}{そこら}
\ruby{中}{ぢう}にある
\ruby{數物}{かず|もの}
なのだもの、
%
\ruby{好}{い}い
\原本頁{147-3}\改行%
\ruby{加減}{か|げん}に
\ruby{熟}{で}きた
\ruby{時{\換字{分}}}{じ|ぶん}に
\ruby{何樣}{ど|う}かなつて
\ruby{仕舞}{し|ま}ふのが
\ruby[||j>]{當}{あたり}
\ruby[||j>]{然}{ まへ}の
% \ruby{當然}{あたり|まへ}の
\ruby{事}{こと}で、
%
\ruby[|g|]{早{\換字{速}}}{さつさ}と
\ruby{緣}{えん}
のある
ところへ
\ruby{行}{い}つて
\ruby{一代}{いち|だい}
\ruby{働}{はた}らいて、
%
\ruby{種子}{た|ね}でも
\ruby{{\換字{遺}}}{のこ}すより
\ruby{他}{ほか}に
いざもこざも
\ruby{有}{あ}りやあ
\ruby{仕}{し}ないのだよ。
%
だから
\ruby{妾}{わたし}が
\ruby{其}{そ}の
\ruby{積}{つも}りで
\原本頁{147-6}\改行%
\ruby{世話}{せ|わ}を
\ruby{燒}{や}いて
\ruby{{\換字{遣}}}{や}つたのに、
%
\ruby{何}{なん}だの
\ruby{彼}{か}だのと
だゞを
\ruby{捏}{こ}ねて
\ruby{妾}{ひと}を
\ruby{御困}{お|こま}らせだつたが、
%
\ruby{其}{それ}も
まあ
\ruby{緣}{えん}が
\ruby{無}{な}かつたのだと
\ruby{其}{そ}の
\ruby{事}{こと}は
\ruby{濟}{す}まして
\ruby{仕舞}{し|ま}つた
ところで。
%
\ruby[|g|]{蠻南瓜}{たうなす}を
\ruby{眞綿}{ま|わた}に
\ruby{包}{くる}んで
\ruby{藏}{しま}ひ
\ruby{{\換字{通}}}{とほ}したつて
\ruby{何}{なん}になるものでもない、
%
\ruby{矢張}{やつ|ぱり}
\ruby{何樣}{ど|う}かして
\ruby{片}{かた}づく
ところへ
\ruby{片}{かた}づけて
やつて、
%
\ruby{持}{も}つて
\ruby{生}{うま}れた
\ruby{役}{やく}を
\ruby{濟}{す}まさせ
なけりやあ
なら
\ruby{無}{な}いから、
%
\原本頁{147-11}\改行%
そこで
\ruby{妾}{わたし}が
お
\ruby{願}{ねがひ}を
\ruby{仕}{し}て、
%
それでは
\ruby{靜岡}{しづ|をか}に
\ruby{{\換字{連}}}{つ}れて
\ruby{歸}{かへ}ることは
\ruby{廢案}{や|め}に
\ruby{仕}{し}まして、
%
\ruby{御甘}{お|あま}え
\ruby{申}{まを}して
\ruby{濟}{す}みませんが
\ruby{何樣}{ど|う}か
\ruby[|g|]{此方}{こちら}
\ruby{樣}{さま}で
\ruby{御使}{お|つか}ひ
なすつて
\ruby{頂}{いたゞ}きたう
ございます、
%
\ruby{何}{なん}でも
\ruby{手}{て}や
\ruby{足}{あし}に
\ruby{皸}{ひび}
\ruby{垢切}{あか|ぎれ}の
きれますやうに
こき
\ruby{使}{つか}つて
\ruby{下}{くだ}さいまして、
%
\ruby{其}{そ}の
\ruby{中}{うち}に
\ruby{破鍋}{われ|なべ}に
\ruby{綴蓋}{とぢ|ぶた}で、
%
\ruby{彼樣}{あ|ん}な
\ruby{奴}{やつ}ても
\ruby{貰}{もら}つて
\ruby{{\換字{遣}}}{や}らうといふ
\ruby{方}{かた}でも
ございましたら、
%
\ruby[|g|]{此方}{こちら}
\ruby{樣}{さま}の
\ruby[|g|]{御鑑識}{おめがね}
\ruby{次第}{し|だい}で
\ruby{豆腐屋}{とう|ふ|や}へでも
\ruby[|g|]{炭團}{たどん}
\ruby{屋}{や}へでも
\ruby{何}{なん}でも
\ruby{宜}{よろ}しう
ございますから
\ruby{身}{み}を
\ruby{固}{かた}めさせて
\ruby{頂}{いただ} たう% 「( )たう」の部分、判読できず
\footnote{「\ruby{頂}{いただ} たう」の空白部分は判読できず空白とした
(国会図書館 コマ番号78/146 p-148 l-06)}%
ございます、
%
と
\ruby{斯樣}{か|う}
いつて
\ruby[<j||]{妾}{わたし}が% 行末行頭の境界付近なので特例処置を施す
\ruby{御願}{お|ねが}ひ
\ruby{申}{まを}して
\ruby{居}{ゐ}るのですよ。
%
もう
\ruby{可}{い}けません、
%
\ruby{我儘}{わが|まゝ}は
\ruby{云}{い}はせません、
%
\ruby{何}{なん}でも
\ruby{彼}{かん}でも
\ruby{妾}{わたし}の
\ruby{云}{い}ふ
\ruby{{\換字{通}}}{とほ}りに
\ruby[|g|]{此方}{こちら}
\ruby{樣}{さま}の
\ruby{御世話}{お|せ|わ}を
\ruby{御願}{お|ねが}ひなさい。
%
\ruby{{\換字{朝}}}{あさ}は
\ruby{昧}{くら}いから% 「昧」1. 夜のあけ方のうすぐらい時。あけぼの。/2. うすぐらい。くらくてはっきりしない。
\ruby{起}{お}きて
\ruby{夜}{よる}は
\ruby{遲}{おそ}くまで、
%
\ruby{火}{ひ}も
\ruby{焚}{た}き
\ruby{水}{みづ}も
\ruby{汲}{く}み
\改行% 校正作業の簡略化のため
、
%
\原本頁{148-10}\改行%
\ruby[|g|]{炊事}{にたき}
\ruby{雜巾掛}{ざう|きん|が}け、
%
\ruby{何}{なに}から
\ruby{何}{なに}まで
\ruby{御奉公人}{ご|ほう|こう|にん}と
\ruby{勵}{はげ}み
\ruby{合}{あ}つて
\ruby{働}{はたら}かなくつては
いけません。
%
\ruby{{\換字{嫌}}}{いや}だ
なんぞと
\ruby{云}{い}つても
\ruby{既}{もう}
\ruby{承知}{しよう|ち}
\ruby{仕}{し}ません。
%
さあ
\原本頁{149-1}\改行%
\ruby{丁度}{ちやう|ど}
\ruby{宜}{い}い、
%
\ruby{妾}{わたし}と
\ruby{一緖}{いつ|しよ}に、
%
\ruby{{\換字{判}}然}{はつ|きり}と
\ruby{改}{あらた}めて
\ruby{今後}{こん|ご}の
\ruby{御世話}{お|せ|わ}を
\ruby{御願}{お|ねが}ひ
\ruby{御仕}{お|し}なさい。
%
\ruby{考}{かんが}へて
\ruby{居}{ゐ}る
\ruby{事}{こと}も
\ruby{何}{なに}も
\ruby{有}{あ}りは
\ruby{仕}{し}ません。
』
