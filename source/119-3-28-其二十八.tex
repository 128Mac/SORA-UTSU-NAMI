\Entry{其二十八}

『
\ruby{然樣}{さ|う}まあ
\ruby{叔母}{を|ば}さんの
\ruby{御言}{お|いひ}のやうにばかりも
お
\ruby{龍}{りう}ちやんにやあなるまいけどもネ、ネエ
お
\ruby{龍}{りう}ちやん、
\ruby{聞}{き}けば
お
\ruby{前}{まへ}も
\ruby{彼}{あ}の
\ruby{御師匠}{お|し|よ}さんていふ
\ruby{人}{ひと}の
\ruby{胸}{むね}の
\ruby{中}{なか}が
\ruby{解}{わか}つて
\ruby{居}{ゐ}ないぢやあ
\ruby{無}{な}いしするのだから、
\ruby{他}{ほか}のいろ〳〵の
\ruby{事}{こと}は
\ruby{後廻}{あと|まは}しに
\ruby{仕}{し}て
\ruby{置}{お}いて。
\ruby{何樣}{ど|う}だエ、
\ruby{彼家}{あす|こ}を
\ruby{出}{で}ることだけは
\ruby{先}{ま}あ
\ruby{兎}{と}も
\ruby{角}{かく}も
\ruby{出}{で}ると
\ruby{決}{き}めては。
』

もとよりお
\ruby{關}{せき}には
\ruby{密}{ひそか}に
\ruby[g]{愛想}{あいそ}を
\ruby{盡}{つ}かし
\ruby{居}{を}れるなれば
\ruby{彼家}{かし|こ}に
\ruby{居}{を}りたき
\ruby{事}{こと}は
\ruby[g]{微塵}{みぢん}ほども
\ruby{無}{な}きなり、
\ruby{且}{か}つ
お
\ruby{彤}{とう}に
\ruby[g]{如是優}{かくやさ}しく
\ruby{云}{い}はれては
\ruby{背}{そむ}かうやうは
\ruby{無}{な}けれど、
\ruby[g]{今彼處}{いまかしこ}を
\ruby{去}{さ}りて
\ruby{離}{はな}れんは、
\ruby{春}{はる}の
\ruby[g]{野行}{のある}きしたる
\ruby{折}{をり}、
\ruby{圖}{はか}らずも
\ruby{乘}{の}つたる
\ruby[g]{田舍渡}{いなかわた}しの
\ruby[g]{襤褸舟}{ぼろぶね}より
\ruby{振顧}{ふり|かへ}り
\ruby{視}{み}たる
\ruby{岸}{きし}に、
\ruby{落}{お}ち
\ruby{零}{こぼ}れの
\ruby{{\換字{菜}}}{な}の
\ruby{花}{はな}の\換字{志}をらしくも
\ruby{{\換字{咲}}}{さ}きて、
\ruby{歪}{ゆが}める
\ruby{茅屋}{かや|や}の
\ruby{背門}{せ|ど}に
\ruby{桃}{もゝ}の
\ruby{盛}{さか}りなる
\ruby[g]{風{\換字{情}}}{ふぜい}などを
\ruby{見出}{み|いだ}し、とても
\ruby{何時}{い|つ}までも
\ruby{眺}{なが}むべきにはあらずと
\ruby{思}{おも}ひながらも
\ruby{今少時}{いま|しば|し}
\ruby{目}{め}にしたきを、
\ruby[g]{野川}{のがは}の
\ruby{甲斐無}{か|ひ|な}く
\ruby{小}{ちひさ}くて
\ruby{早}{はや}くも
\ruby{着}{つ}きたりとて
\ruby{{\換字{逐}}}{お}ひ
\ruby{上}{あ}げらるゝ
\ruby{時}{とき}、
\ruby[g]{{\換字{猶}}未練}{なほみれん}に
\ruby{其}{そ}の
\ruby{船}{ふね}の
\ruby{中}{うち}の
\ruby{戀}{こひ}しき
\ruby{樣}{やう}なる
\ruby{心地}{こゝ|ち}のして、
\ruby{頓}{とみ}には
\ruby{何}{なん}とも
\ruby{答}{こた}へわづらひたり。
されども
\ruby{何處}{ど|こ}から
\ruby{何處}{ど|こ}まで
\ruby{氣}{き}の
\ruby{走}{はし}る
お
\ruby{彤}{とう}に、
\ruby{彼處}{かし|こ}を
\ruby{去}{さ}りてはおのづからに
\ruby{水野}{みづ|の}と
\ruby{緣}{えん}の
\ruby{{\換字{遠}}}{とほ}くなるべきまゝ
\ruby{其}{それ}を
\ruby{厭}{いと}ひて
\ruby{見}{み}す〳〵
\ruby{惡}{わる}い
\ruby{人}{ひと}と
\ruby{知}{し}れる
お
\ruby{關}{せき}が
\ruby{許}{もと}に
\ruby{居}{ゐ}たがるかと
\ruby{思}{おも}はれんほども
\ruby{物憂}{もの|う}くて、

『そりやあ
\ruby{妾}{わたし}だつて
\ruby{彼家}{あす|こ}に
\ruby{居}{ゐ}たいことは
\ruby{有}{あ}りませんが、でも
\ruby{彼家}{あす|こ}を
\ruby{出}{で}てからの
\ruby{妾}{わたし}の
\ruby{行先}{いき|さき}が
\ruby{定}{き}まらなくつちやあ。
』

と
\ruby{僅}{わずか}に
\ruby{語}{ことば}のみを
\ruby{出}{いだ}して
\ruby{煮}{に}え
\ruby{切}{き}れぬ
\ruby{答}{こたへ}をすれば、

『だから
\ruby{此方樣}{こち|ら|さま}に
\ruby{置}{お}いて
\ruby{頂}{いたゞ}くやうに
\ruby{妾}{わたし}が
\ruby{願}{ねが}つて
\ruby{居}{ゐ}るでは
\ruby{無}{な}いか、
\ruby{{\換字{分}}}{わか}らないネエ
お
\ruby{前}{まへ}つて
\ruby{人}{ひと}は。
』

と
\ruby{横合}{よこ|あひ}より
\ruby{叔母}{を|ば}は
\ruby{焦燥}{じ|れ}に
\ruby{焦燥}{じ|れ}ぬ。

『ホヽヽヽ
\ruby{叔母}{を|ば}さん
\ruby[g]{其樣}{そんな}に
\ruby{御急}{お|せ}きなさらなくつてもの
\ruby{事}{こと}ですよ。
ぢやあお
\ruby{龍}{りう}ちやん、
お
\ruby{前}{まへ}も
\ruby{彼家}{あす|こ}に
\ruby{居}{ゐ}たい
\ruby{事}{こと}は
\ruby{無}{な}
いのだから、
\ruby{彼家}{あす|こ}は
\ruby{出}{で}ることに
\ruby{定}{き}め
\ruby{御置}{お|お}きで、そして
\ruby{其}{そ}の
\ruby{次}{つぎ}に
お
\ruby{前}{まへ}の
\ruby{行}{い}く
\ruby{先}{さき}を
\ruby[g]{腹一杯}{はらいつぱい}に
\ruby[g]{御考}{おかんが}へが
\ruby{宜}{い}いぢやあ
\ruby{無}{な}いか。
\ruby{何日}{い|つ}だつたか
\ruby{何}{なに}かの
\ruby{話}{はなし}の
\ruby{序}{つひで}に、
\ruby{妾}{わたし}あ
\ruby{自家}{う|ち}が
\ruby[g]{富裕}{ゆたか}で
お
\ruby{孃樣}{ぢやう|さま}で
\ruby{居}{ゐ}られるやうな
\ruby{身}{み}なら、
\ruby{畫}{ゑ}をかいて
\ruby[g]{一生遊}{いつしやうあそ}んで
\ruby{居}{ゐ}たいと
\ruby{御云}{お|い}ひの
\ruby{事}{こと}があつたが、
\ruby{今}{いま}でも
\ruby{若}{も}し
\ruby{其樣}{そ|ん}な
\ruby{心持}{こゝろ|もち}を
\ruby{有}{も}つておいでゞ、そして
\ruby{畫}{ゑ}でもつて
\ruby{{\換字{遣}}}{や}つて
\ruby{行}{い}かうといふやうな
\ruby{氣}{き}でも
\ruby{御有}{お|あ}りなら、そりやあ
\ruby{其}{それ}でもつて
\ruby{妾}{わたし}が
\ruby{何樣}{ど|う}でも
\ruby{仕}{し}てあげるが……。
\ruby[g]{{\換字{遠}}慮無}{ゑんりよな}しに
\ruby{何}{なん}でも
\ruby{思}{おも}ふ
\ruby{通}{とほ}りを
\ruby{云}{い}つて
\ruby{御覽}{ご|らん}な。
\ruby{畫}{ゑ}を
\ruby{{\換字{習}}}{なら}はうといふやうな
\ruby{氣}{き}も
\ruby{今}{いま}ぢやあ
\ruby{無}{な}いの?。
\ruby{{\換字{習}}}{なら}やあ
お
\ruby{前}{まへ}は
\ruby{屹度}{きつ|と}
\ruby{出來}{で|き}る
\ruby{人}{ひと}
だよ。
』

『いゝえ、もう
\ruby{其樣}{そ|ん}な
\ruby{事}{こと}は
\ruby{些}{ちつと}も
\ruby{思}{おも}つてや
\ruby{仕}{し}ませんは。
これでも
\ruby{自{\換字{分}}}{じ|ぶん}の
\ruby[g]{天禀}{うまれつき}が
\ruby{大}{たい}した
\ruby{上手}{じや|うず}になれない
\ruby{位}{ぐらゐ}の
\ruby{事}{こと}も
\ruby{{\換字{分}}}{わか}らないほどの
\ruby[g]{盲目}{めくら}ぢや
\ruby{無}{な}いのですもの!。
』

『ぢやあ
\ruby{鳴物}{なり|もの}は
\ruby{一體}{いつ|たい}
お
\ruby{前}{まへ}の
\ruby{性}{しやう}に
\ruby{合}{あ}つては
\ruby{居}{ゐ}るし、
\ruby{身}{み}に
\ruby{染}{し}みてほんとに
\ruby{好}{すき}ちやあ
\ruby{有}{あ}るし、
\ruby{若}{も}し
\ruby{音樂}{おん|がく}でも
\ruby{學}{や}つて
\ruby{見}{み}やうといふやうな
\ruby{氣}{き}なんぞも
\ruby{無}{な}くつて!。
』

『まあ
\ruby{厭}{いや}ですネエ、
\ruby{人}{ひと}に
\ruby{教}{おし}へたり
\ruby{人}{ひと}に
\ruby{聞}{き}かれたりするのは
\ruby{妾}{わたし}あ
\ruby{餘}{あま}り
\ruby{好}{すき}ぢやあ
\ruby{無}{な}いんですもの。
』

『ホヽホヽホ。
\ruby{他}{ほか}に
お
\ruby{龍}{りう}ちやんの
\ruby{好}{すき}な
\ruby{事}{こと}は
\ruby{無}{な}いし。
ぢやあ
\ruby{藝事}{げい|ごと}で
\ruby{身}{み}を
\ruby{立}{た}てやうつて
\ruby{氣}{き}も
\ruby{先}{ま}あ
\ruby{無}{な}いのだから、
\ruby[g]{修業沙汰}{しゆげふざた}なんかは
\ruby[g]{一切御}{いつさいお}やめなのだネエ。
』

『だつて
\ruby{今更}{いま|さら}、
\ruby{何}{なに}か
\ruby{爲}{し}て
\ruby{一人}{ひと|り}で
\ruby{何樣}{ど|う}の
\ruby{彼樣}{か|う}の
\ruby{仕}{し}やうつていふやうなことも
\ruby{思}{おも}つては
\ruby{居}{ゐ}ないんですもの!。
』

