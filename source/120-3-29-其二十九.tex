\Entry{其二十九}

『でも、それかと
\ruby{云}{い}つて
\ruby{叔母}{を|ば}さんと
\ruby{一緖}{いつ|しよ}に
\ruby{田舍}{ゐな|か}へ
\ruby{引込}{ひつ|こ}んで
\ruby{仕舞}{し|ま}つて、
\ruby{叔母}{を|ば}さんの
\ruby{鑑識}{め|がね}で
\ruby{持}{も}たせて
\ruby{下}{くだ}さる
お
\ruby{婿}{むこ}を
\ruby{持}{も}つて
\ruby{暮}{くら}さうといふ
\ruby{氣}{き}は
\ruby{無}{な}いと
\ruby{再々御云}{さい|〳〵|お|い}ひぢやあ
\ruby{無}{な}いか。
』

『そりやあもう
\ruby{然樣}{さ|う}ですとも!。
\ruby{妾}{わたし}あ
\ruby{何樣}{ど|う}あつても、
\ruby{何}{なん}だか
\ruby{{\換字{分}}}{わか}らないで
\ruby{牛}{うし}か
\ruby{馬}{うま}みたやうに
\ruby{挊}{かせ}いでる
\ruby{田舍}{ゐな|か}の
\ruby{人}{ひと}の、
\ruby{御飯}{ご|はん}を
\ruby{喫}{た}べるために
\ruby{生}{い}きてるつて
\ruby{云}{い}つたやうな
\ruby{其樣}{そ|ん}な
\ruby{{\換字{分}}}{わか}らない
\ruby{人}{ひと}と、
\ruby{一生暮}{いつ|しやう|くら}すなんかつていふ
\ruby{事}{こと}は
\ruby{到底出來}{とて|も|で|き}ないんですから。
』

\ruby{叔母}{を|ば}は
\ruby{堪}{た}へかねて
\ruby{口}{くち}を
\ruby{挿}{はさ}みたり。

『それ、それ、
\ruby{其}{そ}の
\ruby{根性}{こん|ぢやう}が
\ruby{碌}{ろく}で
\ruby{無}{な}い、
\ruby{正當}{まつ|たう}で
\ruby{無}{な}いのだよ。
\ruby[g]{傍目}{わきめ}もふらずにせつせと
\ruby{挊}{かせ}ぎ
\ruby{通}{とほ}すのが
\ruby{上人}{じやう|にん}といふもので、
お
\ruby{前}{まへ}のやうに
\ruby{何}{なん}だの
\ruby{彼}{か}だのと
\ruby{下}{くだ}らない
\ruby{事}{こと}ばかり
\ruby{云}{い}つて
\ruby{居}{ゐ}るのが
\ruby{間{\換字{違}}}{ま|ちが}ひきつて
\ruby{居}{ゐ}るのだ。
\ruby[g]{皆誰}{みんなだれ}だつて
\ruby{御飯}{ご|はん}を
\ruby{喫}{た}べるために
\ruby{挊}{かせ}ぐのぢやあ
\ruby{無}{な}いか。
\ruby{喫}{た}べる
\ruby{爲}{ため}に
\ruby{挊}{かせ}が
\ruby{無}{な}くつて
\ruby{何樣}{ど|う}なるものかネ、
\ruby{下}{くだ}らない。
』

『だつて
\ruby{其樣}{そ|ん}なに
\ruby{大騷}{おほ|さわ}ぎを
\ruby{{\換字{遣}}}{や}つて
\ruby{御膳}{ご|ぜん}を
\ruby{食}{た}べりやあ
\ruby{其}{それ}でもつて
\ruby{何}{なに}が
\ruby{嬉}{うれ}しいの?。
』

『そんな
\ruby{馬鹿}{ば|か}な
\ruby{氣樂}{き|らく}なことを
\ruby{云}{い}つて
\ruby{居}{ゐ}るから
\ruby{皆}{みんな}
お
\ruby{前}{まへ}の
\ruby{考}{かんがへ}は
\ruby{間{\換字{違}}}{ま|ちが}つて
\ruby{居}{ゐ}るのだよ。
\ruby{人間}{ひ|と}つてものは
\ruby{三度三度}{さん|ど|さん|ど}
\ruby{御膳}{ご|ぜん}さへ
\ruby{滿足}{まん|ぞく}にいたゞいて
\ruby{行}{ゆ}かれりやあ
\ruby{其}{それ}で
\ruby{結構}{けつ|こう}なので、
\ruby{嬉}{うれ}しいも
\ruby{嬉}{うれ}しくないも
\ruby{要}{い}つた
\ruby{事}{こと}あ
\ruby{有}{あ}りや
\ruby{仕無}{し|な}い。
お
\ruby{前}{まへ}なんざあ
\ruby{甚}{ひど}い
\ruby{苦勞}{く|らう}といふものを
\ruby{仕}{し}た
\ruby{事}{こと}が
\ruby{無}{な}いものだから、
\ruby{其樣}{そ|ん}な
\ruby{下}{くだ}らない
\ruby{事}{こと}ばかり
\ruby{云}{い}つて
\ruby{居}{ゐ}るんだよ。
』

『
\ruby{御膳}{ご|ぜん}を
\ruby{食}{た}べるばかりに
\ruby{齷齪}{あく|せく}して
\ruby{死}{し}んで
\ruby{仕舞}{し|ま}ふのだつて、
\ruby[g]{何程下}{いくらくだ}らないか
\ruby{知}{し}れや
\ruby{仕無}{し|な}いは。
』

『ホヽヽ、お
\ruby{龍}{りう}ちやん
お
\ruby{前}{まへ}が
\ruby{惡}{わる}いよ、
\ruby{目上}{め|うへ}に
\ruby{{\換字{逆}}}{さか}らつて!。
\ruby{第一}{だい|いち}
\ruby{談話}{はな|し}に
\ruby{枝}{えだ}が
\ruby{{\換字{咲}}}{さ}いて
\ruby{仕舞}{し|ま}ふはネ。
ぢやあお
\ruby{前}{まへ}は
\ruby{稽古事}{けい|こ|ごと}は
\ruby{爲}{す}る
\ruby{氣}{き}は
\ruby{無}{な}し、
\ruby{靜岡}{しづ|をか}へは
\ruby{行}{ゆ}くまいと
\ruby{云}{い}ふし、
\ruby{何樣}{ど|う}
\ruby{仕}{し}やうと
\ruby{御云}{お|い}ひなの?。
\ruby{妾}{わたし}の
\ruby{處}{ところ}へ
\ruby{來}{き}て
\ruby{妾}{わたし}の
\ruby{{\換字{遊}}}{あそ}び
\ruby{相手}{あひ|て}になつて
お
\ruby{{\換字{呉}}}{く}れの
\ruby{積}{つも}りなの?。
』

『…………、』

『いエもう
\ruby{{\換字{遊}}}{あそ}び
\ruby{相手}{あひ|て}なんぞと
\ruby{仰}{おつし}あやると
\ruby{直}{すぐ}に
\ruby[g]{増長致}{ぞうちやういた}します、
\ruby{矢張}{やつ|ぱ}り
\ruby{引{\換字{遣}}}{ひつ|つか}つて
\ruby{{\換字{遣}}}{や}ると
\ruby{仰}{おつし}あつて
\ruby{下}{くだ}さいまし。
』

『お
\ruby{龍}{りう}ちやんが
\ruby{默}{だま}つて
\ruby{居}{ゐ}ちやあ
\ruby{仕樣}{し|やう}が
\ruby{無}{な}いぢやあ
\ruby{無}{な}いか。
\ruby{默}{だま}つてるところを
\ruby{見}{み}ると
\ruby{吾家}{う|ち}へ
\ruby{來}{く}るのも
\ruby{厭}{いや}なの?。
』

『
\ruby{厭}{いや}つて
\ruby{事}{こと}は
\ruby{毫末}{ちつ|と}も
\ruby{有}{あ}りやあ
\ruby{仕}{し}ませんけれども……』

『ぢやあ
\ruby{何}{なに}も
\ruby{其樣}{そ|ん}なに
\ruby{考}{かんが}へてゐる
\ruby{事}{こと}は
\ruby{有}{あ}りさうも
\ruby{無}{な}いものぢや
\ruby{無}{な}いか。
』

『でも
\ruby{姊}{ねえ}さんのところへ
\ruby{來}{き}て
\ruby{居}{ゐ}ると……』

『
\ruby{何}{なに}か
\ruby{厭}{いや}な
\ruby{事}{こと}があつて?。
』

『いえ、
\ruby{然樣}{さ|う}なのぢや
\ruby{有}{あ}りませんけども
\ruby{餘}{あんま}
り
\ruby{叮嚀}{てい|ねい}に
\ruby{仕}{し}て
\ruby{下}{くだ}さるんで、\------ まるで
\ruby{眞實}{ほん|と}の
\ruby{妹}{いもうと}かなんぞのやうに、
\ruby{御孃樣}{お|ぢやう|さま}あつかひに
\ruby{仕}{し}てくださるので、
\ruby{何}{なん}だか
\ruby[g]{居辛}{ゐづら}くつて
\ruby{仕方}{し|かた}が
\ruby{無}{な}いんですもの。
\ruby{此}{こ}の
\ruby{春}{はる}だつて
\ruby{然樣}{さ|う}なのですよ。
\ruby{彼}{あ}の
\ruby{時}{とき}は
\ruby{彼樣}{あ|あ}した
\ruby{譯}{わけ}で
\ruby{二度}{に|ど}と
\ruby{姊}{ねえ}さんにやあ
\ruby{御目}{お|め}に
\ruby{掛}{かゝ}らないつもりで
\ruby{出}{で}たんですけれとも、
\ruby{後}{あと}になつても
\ruby{一}{ひと}つは
\ruby{其}{そ}の
\ruby{爲}{ため}に
\ruby{此方}{こち|ら}へは
\ruby{歸}{かへ}つて
\ruby{來}{こ}なかつたので。
\ruby{彼}{あ}の
お
\ruby{師匠}{し|よ}さんのところに
\ruby{居}{ゐ}ることに
\ruby{仕}{し}ましたのも、いろ〳〵の
\ruby{事}{こと}を
\ruby{云}{い}つて
\ruby{引{\換字{留}}}{ひき|と}められるからばかりぢやあ
\ruby{有}{あ}りませんので。
\ruby{彼家}{あす|こ}に
\ruby{居}{ゐ}りやあ
\ruby{居}{ゐ}るだけの
\ruby{事}{こと}を
\ruby{爲}{し}て
\ruby{報復}{か|へ}しますけれども、
\ruby{姊}{ねえ}さんの
\ruby{處}{ところ}に
\ruby{居}{ゐ}ますと、
\ruby{何一}{なに|ひと}つ
\ruby{用事}{よう|じ}を
\ruby{爲}{す}るのぢやあ
\ruby{無}{な}し、
\ruby{着物}{き|もの}も
\ruby{美麗}{き|れい}に
\ruby{仕}{し}て
\ruby{下}{くだ}さりやあ
\ruby{髮}{かみ}から
\ruby{穿物}{はき|もの}まで
\ruby{氣}{き}をつけて
\ruby{下}{くだ}さる、それで
\ruby{三度}{さん|ど}が
\ruby{三度}{さん|ど}とも
\ruby{据膳}{すゑ|ぜん}に
\ruby{對}{むか}つて、
\ruby{姊}{ねえ}さん
\ruby{同樣}{どう|やう}に
\ruby{御給仕}{お|きふ|じ}をされて
\ruby{御膳}{ご|ぜん}を
\ruby{頂}{いたゞ}くのは、
\ruby{妾}{わたし}にやあ
\ruby{何}{なん}だか
\ruby[g]{結構{\換字{過}}}{けつこうす}ぎて
\ruby{濟}{す}まないやうな
\ruby{氣}{き}がするのですもの!。
\ruby{小間使}{こ|ま|づかひ}や
\ruby{何}{なん}かと
\ruby{一緖}{いつ|しよ}になつて
\ruby{何}{なに}か
\ruby{用}{よう}を
\ruby{仕}{し}やうとすりやあ、
お
\ruby{止}{よ}し、
お
\ruby{止}{よ}し、
\ruby{不見識}{ふ|けん|しき}だよ、つて
\ruby{姊}{ねえ}さんが
\ruby{御止}{お|と}めなさるのですら、あれだけ
\ruby{御厄介}{ご|やく|かい}になつて
\ruby{居}{ゐ}た
\ruby{中}{うち}に
\ruby{姊}{ねえ}さんの
\ruby{爲}{ため}に
\ruby{何}{なに}か
\ruby{仕}{し}たと
\ruby{云}{い}つたら、たつた
\ruby{一遍相思鳥}{いつ|ぺん|さう|し|てう}の
\ruby{餌}{ゑ}を
\ruby{摺}{す}つたことが
\ruby{有}{あ}るつ
\ruby{限}{き}りなのですもの。
\ruby{何程兒童}{い|くら|こ|ども}の
\ruby{時}{とき}から
\ruby{一緖}{いつ|しよ}に
\ruby{寢}{ね}たりなんか
\ruby{仕}{し}て、
\ruby{{\換字{姉}}妹}{きやう|だい}よりも
\ruby{仲好}{なか|よ}く
\ruby{暮}{くら}して
\ruby{來}{き}たからつて、
\ruby{妾}{わたし}あ
\ruby{姊}{ねえ}さんにやあ
\ruby{緣}{えん}も
\ruby[g]{由緣}{ゆかり}も
\ruby{何}{なんに}も
\ruby{無}{な}い
\ruby{身}{み}だし、そりやあ
\ruby{今}{いま}が
\ruby{今}{いま}でも
\ruby{姊}{ねえ}さんの
\ruby{爲}{ため}になら
\ruby{火水}{ひ|みづ}の
\ruby{中}{なか}へなり
\ruby{入}{はい}らうつていふ
\ruby{氣}{き}だけは
\ruby{有}{も}つて
\ruby{居}{ゐ}ますけれども、
\ruby{今日}{け|ふ}までのところぢやあ
\ruby{何一}{なに|ひと}つ
\ruby{姊}{ねえ}さんの
\ruby{爲}{ため}に
\ruby{仕}{し}た
\ruby{事}{こと}でも
\ruby{有}{あ}るぢやあ
\ruby{無}{な}し、たゞ
\ruby{甘}{あま}つたれて
\ruby{可愛}{か|はい}がつて
\ruby{貰}{もら}つて
\ruby{居}{ゐ}たと
\ruby{云}{い}ふだけの
\ruby{事}{こと}なんですから、そんなに
\ruby{好}{よ}くされるやう
\ruby{譯}{わけ}は
\ruby{有}{あ}る
\ruby{筈}{はず}が
\ruby{無}{な}いので、
\ruby{何樣}{ど|う}も
\ruby{妾}{わたし}やあ
\ruby{氣}{き}が
\ruby{狹小}{け|ち}なんでしやうけれども
\ruby{氣}{き}が
\ruby{咎}{とが}めてならないのです。
ですから、いつそ
\ruby{叔母}{を|ば}の
\ruby{言葉}{こと|ば}の
\ruby{通}{とほ}りに
\ruby{扱}{こ}き
\ruby{使}{つか}つて
\ruby{下}{くだ}さるならば、
\ruby{願}{ねが}つても
\ruby{姊}{ねえ}さんの
\ruby{傍}{そば}へ
\ruby{置}{お}いて
\ruby{頂}{いたゞ}きたいのですけれど、
\ruby{何樣}{ど|う}も
\ruby{姊}{ねえ}さんは
\ruby{姊}{ねえ}さんの
\ruby{氣象}{き|しやう}でもつて
\ruby{然樣}{さ|う}は
\ruby{仕}{し}て
\ruby{下}{くだ}さるまいと
\ruby{思}{おも}ふと、
\ruby{何}{なに}も
\ruby{仕}{し}も
\ruby{仕無}{し|な}いものを
\ruby{餘}{あま}り
\ruby{好}{よ}くして
\ruby{下}{くだ}さるのが、
\ruby{妾}{わたし}にやあ
\ruby{心苦}{こゝろ|ぐる}しくつて
\ruby{居}{ゐ}られないのですから。
』

『オヤ、オヤ、お
\ruby{龍}{りう}ちやんは
\ruby{大層}{たい|そう}
\ruby{他人兒}{た|にん|こ}におなりネエ。
わかつたよお
\ruby{前}{まへ}の
\ruby{優}{やさ}しい
\ruby{奇麗}{き|れい}な
\ruby{心持}{こゝろ|もち}は
\ruby{善}{よ}く
\ruby{解}{わか}つたよ。
\ruby{何}{なに}かと
\ruby{思}{おも}つたら、ホヽホヽホヽ
\ruby{其樣}{そ|ん}な
\ruby{事}{こと}だつたの!。
つい
\ruby{{\換字{過}}般}{こな|ひだ}までの
お
\ruby{龍}{りう}ちやんは
\ruby{此樣}{こ|ん}な
\ruby{人}{ひと}ぢやあ
\ruby{無}{な}くつて、
\ruby{花簪}{はな|かんざし}の
\ruby{大}{おほき}いのを
お
\ruby{{\換字{悅}}}{よろこ}びだつた
\ruby{頃}{ころ}といふものは
\ruby{何}{なに}を
\ruby{買}{か}つて
\ruby{{\換字{呉}}}{く}れ、
\ruby{彼}{か}を
\ruby{買}{か}つて
\ruby{{\換字{呉}}}{く}れつて
\ruby{妾}{わたし}をせびつちやあ、
\ruby{稀}{たま}に
\ruby{買}{か}つて
\ruby{上}{あ}げ
\ruby{無}{な}からうものならプーツと
お
\ruby{膨}{ふく}れでネ、
\ruby{夜}{よる}になつて
\ruby{一緖}{いつ|しよ}に
\ruby{寢}{ね}ても
\ruby{彼方}{むか|う}を
\ruby{向}{む}いて
\ruby{口一}{くち|ひと}つきかないで、そして
\ruby{足}{あし}でもつてぼん〳〵と
\ruby{妾}{わたし}を
お
\ruby{蹴}{け}だつたぢやあ
\ruby{無}{な}いか。
』

『あら
\ruby{厭}{いや}な
\ruby{姊}{ねえ}さんだこと!。
\ruby{兒童}{こ|ども}の
\ruby{時}{とき}の
\ruby{事}{こと}なんか
\ruby{御云}{お|い}ひ
\ruby{出}{だ}しなすつちやあ。
』

『ホヽヽ、そのお
\ruby{龍}{りう}ちやんがまあ
\ruby{大層}{たい|そう}にませて、ほんとに
\ruby{{\換字{遠}}慮深}{ゑん|りよ|ぶか}く
お
\ruby{成}{な}りのネ!。
いゝよ、
\ruby{其}{それ}なら
\ruby{其}{それ}で
\ruby{其}{そ}の
\ruby{樣}{やう}に
\ruby{爲}{す}るから。
ぢやあ
\ruby{吾家}{う|ち}に
\ruby{居}{ゐ}ることに
お
\ruby{定}{き}めが
\ruby{好}{い}いぢやあ
\ruby{無}{な}いか。
』

お
\ruby{龍}{りう}は
\ruby{辭}{じ}せんとして
\ruby{今}{いま}は
\ruby{辭}{じ}する
\ruby{能}{あた}はざる
\ruby{境}{さかひ}に
\ruby{臨}{のぞ}みぬ。
お
\ruby{關}{せき}の
\ruby{許}{もと}を
\ruby{離}{はな}れて
お
\ruby{彤}{とう}の
\ruby{世話}{せ|わ}になる
\ruby{事}{こと}の
\ruby{{\換字{嫌}}}{いや}なるにはあらねど、
\ruby{何故}{なに|ゆゑ}にや
\ruby{前}{さき}の
\ruby{日}{ひ}と
\ruby{今日}{け|ふ}とは
お
\ruby{彤}{とう}の
\ruby{語氣}{くち|ぶり}の
\ruby{異}{ちが}ひて、
\ruby{彼}{か}の
\ruby{水野}{みづ|の}をば
\ruby{{\換字{悅}}}{よろこ}ばぬ
\ruby{氣}{き}なるが
\ruby{何}{なん}と
\ruby{無}{な}く
\ruby{心}{こゝろ}にかゝりて、
\ruby{此}{こ}の
\ruby{人}{ひと}の
\ruby{許}{もと}に
\ruby{明日}{あ|す}よりの
\ruby{我}{わ}が
\ruby{身}{み}を
\ruby{寄}{よ}せんことの
\ruby{何}{なに}かは
\ruby{知}{し}らねど
\ruby{窮屈}{きう|くつ}らしき
\ruby{心地}{こゝ|ち}して、
\ruby{嬉}{うれ}しかるべき
\ruby{筈}{はず}の
\ruby{事}{こと}ながら
\ruby{然}{さ}のみは
\ruby{嬉}{うれ}しからぬなり。

