\Entry{其三十}

% メモ 校正終了 2024-05-15
\原本頁{162-1}%
\ruby{色}{いろ}ある
\ruby{蓋}{かさ}の
いと
\ruby{艶}{えん}に% 原本通り「えん」
\ruby{美}{うつく}しき
\ruby{電燈}{でん|とう}の
\ruby{下}{もと}、
%
\ruby{上座}{じやう|ざ}に
お
\ruby{彤}{とう}、
%
やゝ
\ruby{隔}{へだ}たり
\ruby{下}{さが}つて
お
\ruby{龍}{りう}の
\ruby{叔母}{を|ば}、
%
それより
また
\ruby{下}{さが}つて
\ruby{坐}{すわ}れる
お
\ruby{龍}{りう}の
\ruby{三人}{さん|にん}は、
\ruby{今}{いま}しも
\ruby{夜食}{や|しよく}の
\ruby{膳}{ぜん}の
\ruby{既}{すで}に
\ruby{引}{ひ}き
\ruby{去}{さ}られたる
\ruby{後}{あと}を、
%
\ruby[||j>]{心}{こゝろ}
\ruby[||j>]{靜}{ しづ}かに
% \ruby{心靜}{こゝろ|しづ}かに
\ruby{茶}{ちや}に
\ruby{物語}{もの|がた}るなり。

\原本頁{162-5}%
\ruby{三人}{さん|にん}
\ruby{三樣}{さん|やう}の
\ruby{心}{こゝろ}の
\ruby{思}{おもひ}あれば
\ruby{面}{おもて}の
\ruby{色}{いろ}あり。
%
お
\ruby{龍}{りう}は
おのが
\ruby{頼}{たの}まんと
\ruby{思}{おも}ひて
\ruby{來}{き}しことは
\ruby[|g|]{自然}{おのづ}と
\ruby[|g|]{{\換字{半}}{\換字{分}}}{なかば}は
\ruby{餘{\換字{所}}}{よ|そ}に
されて、
%
\ruby{思}{おも}ひも
かけざりし
\ruby{我}{わ}が
\ruby{身}{み}の
\ruby{上}{うへ}の
\ruby[|g|]{彼家}{かしこ}を
\ruby{出}{い}でて
\ruby{此家}{こ|ゝ}に
\ruby{居}{を}るべきやう
\ruby{定}{さだ}められたるに、
%
\原本頁{162-8}\改行%
\ruby{可厭}{い|や}といふでは
\ruby{無}{な}けれど
\ruby{何}{なん}となく
\ruby{勇}{いさ}まぬ
\ruby{心地}{こゝ|ち}のするか、
%
\ruby{常}{つね}とは
\ruby{{\換字{違}}}{ちが}ひて
\ruby{沈}{しづ}めるやうなり。
%
お
\ruby{龍}{りう}が
\ruby{叔母}{を|ば}は、
%
\ruby{全}{まつた}く
\ruby{我}{わ}が
\ruby{思}{おも}ふ
\ruby{如}{ごと}くに
なりたりと
\ruby{云}{い}ふには
あらねど、
%
\ruby{兎}{と}に
\ruby{角}{かく}
お
\ruby{龍}{りう}を
\ruby{我}{わ}が
\ruby{{\換字{嫌}}}{きら}ふ
お
\ruby{關}{せき}が
\ruby{許}{もと}より
\ruby{移}{うつ}し
\ruby{奪}{うば}ひて、
%
\ruby{豫}{かね}て
お
\ruby{龍}{りう}より
\ruby{聞}{き}きしに
\ruby{{\換字{違}}}{たが}はず
\ruby{富}{と}みて
\ruby{美}{うつく}しく
\ruby{智慧}{ち|ゑ}
\原本頁{163-1}\改行%
\ruby{深}{ふか}き
\ruby{此家}{この|や}の
\ruby[|g|]{主人}{あるじ}が
\ruby{許}{もと}に
\ruby{預}{あづ}かり
\ruby{貰}{もら}ふ
\ruby{事}{こと}
となりたるに、
%
\ruby{心安堵}{こゝろ|おち|つ}きて
\ruby{莞爾}{に|こ}つき
\ruby{{\換字{勝}}}{がち}なれば、
%
\ruby{根}{ね}は
\ruby{善}{よ}き
\ruby{人}{ひと}の
\ruby{徴}{しるし}とて
\ruby{顏}{かほ}に
\ruby{曇}{くも}りなく、
%
\ruby{例}{れい}の
\ruby[<j||]{小}{ちひさ}なる% 行末行頭の境界付近なので特例処置を施す
\ruby{三角}{さん|かく}の
\ruby{眼}{め}さへ、
%
\ruby{其}{そ}の
\ruby{眼尻}{まな|じり}に
\ruby{寄}{よ}る
\ruby{小皺}{こ|じわ}に
\ruby{却}{かへ}つて
\ruby{可愛}{か|はい}らしく
\ruby{見}{み}ゆ。
%
たゞ
お
\ruby{彤}{とう}のみは
\ruby{心}{こゝろ}の
\ruby{動}{うご}くこと
\ruby{無}{な}くてや、
%
\ruby{能}{よ}く
\ruby{笑}{わら}ひ
\ruby{能}{よ}く
\ruby{語}{かた}れども
\ruby{悅}{よろこ}べるともなく
\ruby{樂}{たのし}まぬとも
\ruby{無}{な}く
\ruby{{\換字{平}}然}{へい|ぜん}として、
%
\ruby{今}{いま}
\ruby{{\換字{猶}}}{なほ}
\ruby{{\換字{前}}刻}{さ|き}の
\ruby{如}{ごと}く
\ruby{澄}{す}まし
\ruby{{\換字{返}}}{かへ}つたり。

\原本頁{163-7}%
お
\ruby{龍}{りう}は
\ruby{何}{なに}をか
\ruby{思}{おも}へる、
%
\ruby{沈默}{おし|だま}りて
\ruby{頭}{かうべ}を
\ruby{垂}{た}れつ、
%
\ruby{頻}{しきり}に
\ruby{譯}{わけ}も
\ruby{無}{な}く
\ruby{自己}{お|の}が
\ruby{衣服}{き|もの}の
\ruby{袖}{そで}
\ruby{膝}{ひざ}
なんどに
\ruby{吸}{す}ひ
\ruby{出}{だ}されたる
\ruby{綿}{わた}を
\ruby{摘}{つ}みては
\ruby{除}{と}り
\ruby{摘}{つ}みては
\ruby{除}{と}り
ながら、
%
\ruby{人}{ひと}
の
\ruby{話}{はなし}を
のみ
\ruby{聞}{き}きて
\ruby{居}{を}れば、
%
\ruby{叔母}{を|ば}は
お
\ruby{龍}{りう}が
\ruby{樣子}{やう|す}
などには
\ruby{眼}{め}も
\ruby{{\換字{遣}}}{や}らずして、

\原本頁{163-11}%
『
どうも
\ruby{誠}{まこと}に
\ruby{種々}{いろ|〳〵}
\ruby{有}{あ}り
\ruby{{\換字{難}}}{がた}う
ございます、
%
お
\ruby{蔭樣}{かげ|さま}で
\ruby[|j|]{私}{わたくし}も
\ruby{安心}{あん|しん}
いたしました。
%
では
\ruby[<j>]{私}{わたくし}は
\ruby{直接}{ぢ|か}には
お
\ruby{關}{せき}に
\ruby{會}{あ}ひませず、
%
\ruby{此儘}{この|まゝ}で
\ruby{國}{くに}へ
\原本頁{164-2}\改行%
\ruby{歸}{かへ}りまして、
%
\ruby{憚}{はばか}り% 「憚 は(ば)か」
さまで
ございますが
お
\ruby{關}{せき}の
\ruby{方}{はう}の
\ruby{事}{こと}は、
%
\ruby{一切}{いつ|さい}
\ruby{此方樣}{こち|ら|さま}
\ruby{次第}{し|だい}に
\ruby{願}{ねが}ひます。
%
\ruby{{\換字{若}}}{もし}
\ruby{{\換字{又}}}{また}
\ruby{全然}{ま|まる}% 「(まるまる)」のようにも思えるが、原本通りのルビ表記
\ruby{握}{にぎ}り
\ruby{{\換字{拳}}}{こぶし}でも
\ruby{濟}{す}みませぬやうの
\ruby{事}{こと}
で
ございました
ならば、
%
\ruby{惡}{わる}い
\ruby{奴}{やつ}に
\ruby{關}{かゝ}りあつたのが
\ruby{不祥}{ふ|しやう}
で
ございますから、
%
\ruby{三十}{さん|じふ}
\ruby{四十}{し|じふ}の
\ruby{金}{かね}を
\ruby{出}{だ}し
\ruby{惜}{をし}みは
\ruby{致}{いた}しません、
%
\ruby{御話}{お|はなし}さへ
ございますれば
\ruby{直}{すぐ}にも
\ruby{差出}{さし|だ}します。
%
\ruby{何}{なに}も
\ruby{彼}{か}も
\ruby{此女}{こ|れ}の
\ruby{爲}{ため}
\ruby{宜}{よ}かれと
\ruby{思}{おも}ふからの
\ruby{事}{こと}
で
ございますから
\ruby{{\換字{忍}}耐}{が|まん}も
\ruby{致}{いた}します。
%
\ruby{全}{まつた}く
\ruby{彼樣}{あ|ん}な
\ruby{奴}{やつ}に
\原本頁{164-8}\改行%
\ruby{鐚錢}{び|た}
\ruby{一}{ひと}つ
\ruby{吳}{く}れて
\ruby{{\換字{遣}}}{や}ります
\ruby{因緣}{いん|ねん}は
\ruby{無}{な}いと
\ruby{思}{おも}ひます
けれど
\ruby[<j>]{些}{いささか}、% 原本にはルビがないが、一文字なので(いささか)を補う
\ruby[|g|]{些少}{すこし}
ばかりの
\ruby{事}{こと}で
\ruby{煩}{うるさ}い
\ruby[||j>]{關}{ひつ}
\ruby[||j>]{係}{かゝり}を
% \ruby{關係}{ひつ|かゝり}を
\ruby{殘}{のこ}すのも
\ruby{可厭}{い|や}ですし、
%
\ruby{此女}{こ|れ}と
\ruby{彼}{あ}の
\ruby{婆}{ばゞあ}と
\原本頁{164-10}\改行%
\ruby{往來}{わう|らい}で
\ruby{逢}{あ}ひました
\ruby{時}{とき}、
%
\ruby{此女}{こ|れ}に
\ruby{氣}{き}の
\ruby{怯}{ひ}けるやうな
\ruby{思}{おも}ひを
させるのも
\ruby{可厭}{い|や}
で
ございますから、
%
\ruby{其}{そ}の
\ruby{位}{くらゐ}の
\ruby{事}{こと}なら
\ruby{出}{だ}しも
\ruby{致}{いた}しましやうと
\ruby{思}{おも}つて
\ruby{居}{を}りますのです。
%
\ruby{其邊}{そこ|いら}は
\ruby{御含}{お|ふく}み
\ruby{下}{くだ}さいまして、
%
\ruby{何樣}{ど|う}でも
\ruby{宜}{よろ}しいやうに
\ruby{御計}{お|はか}らひを
\ruby{願}{ねが}ひまする。
%
\ruby{此女}{こ|れ}の
\ruby{上}{うへ}は
\ruby{改}{あらた}めて
\ruby{今日}{け|ふ}
\原本頁{165-3}\改行%
\ruby{私か}{わた|くし}ら% [||j>]ではアキが生じてしまうので(か)を親文字に入れた
\ruby{御縋}{お|すが}り
\ruby{申}{まを}して
\ruby{御願}{お|ねが}ひ
\ruby{申}{まを}しまする。
%
\ruby{至}{いた}つて
\ruby{我儘}{わが|まゝ}な
\ruby{無{\換字{分}}別}{む|ふん|べつ}
\ruby{者}{もの}
では
ございまするが、
%
\ruby{心}{しん}から
\ruby{底}{そこ}から
\ruby{惡}{わる}い
\ruby{奴}{やつ}
といふのでも
\ruby{無}{な}い
やうで
ございますから、
%
\ruby{何樣}{ど|う}か
\ruby{十{\換字{分}}}{じふ|ぶん}に
\ruby{御斟{\換字{酌}}}{ご|しん|しやく}なく
\ruby{御使}{お|つか}ひ
なすつて、
%
\原本頁{165-6}\改行%
そして
\ruby{其}{その}
\ruby{中}{うち}
\ruby{相應}{さう|おう}な
ものでも
ございました
\ruby{時}{とき}に、
%
\ruby{御鑑識}{お|め|がね}で
\ruby{夫}{をとこ}でも
\ruby{持}{も}たせて
\ruby{{\換字{遣}}}{や}つて
\ruby{下}{くだ}されば
\ruby{其}{その}
\ruby{上}{うへ}は
ございません。
%
\ruby[<j>]{私}{わたくし}は
\ruby{斯樣}{こ|ん}な
がさつ
\ruby{者}{もの}
で
ございましても、
%
\ruby{姪}{めひ}
\ruby[|g|]{一人}{ひとり}
\ruby{叔母}{を|ば}
\ruby[|g|]{一人}{ひとり}
で
ございますから
\ruby{此女}{こ|れ}を
\ruby{棄}{す}てる
\ruby{氣}{き}
は
ございません。
%
\ruby{何處}{ど|こ}までも
\ruby{好}{よ}くして
\ruby{{\換字{遣}}}{や}りたいのは
\ruby{山々}{やま|〳〵}
で
ございますが、
%
とても
\ruby[|j|]{私}{わたくし}には
\ruby{制{\換字{道}}}{せい|だう}の
\ruby{付}{つ}きかねる
\ruby{氣}{き}まぐれ
\ruby{者}{もの}め
で
ございますので、
%
\ruby[|g|]{此方}{こちら}
\ruby{樣}{さま}へ
\ruby{願}{ねが}ふより
ほかには
\ruby{願}{ねが}はう
ところも
\ruby{無}{な}いやうな
\ruby{譯}{わけ}
で
ございます
ゆゑ、
%
\ruby{御{\換字{迷}}惑}{ご|めい|わく}
でも
ございましやうが
\ruby{何樣}{ど|う}か
\ruby[|g|]{御世話}{おせわ}
を
なすつて
\ruby{下}{くだ}さいますやうに、
%
\ruby{汚}{きたな}い
\ruby{婆}{ばゞあ}
で
ございますが
\ruby{是}{これ}でも
\ruby{人樣}{ひと|さま}の
\ruby{御恩}{ご|おん}を
\ruby{忘}{わす}れるやうな
\ruby{獸畜}{けだ|もの}
でも
ございません
\ruby[|g|]{田舎}{ゐなか}
\ruby{者}{もの}が、
%
\ruby{折入}{をり|い}つて
\ruby{此}{こ}の
\ruby{{\換字{通}}}{とほ}りに
お
\ruby{願}{ねが}ひ
\ruby{申}{まを}します。
』

\原本頁{166-5}%
と、
%
\ruby{云}{い}ひさま
\ruby{頭}{かしら}を
\ruby{下}{さ}げて
\ruby{染々}{しみ|〴〵}と
\ruby{眞心}{ま|ごゝろ}せめて
\ruby{頼}{たの}み
\ruby{聞}{きこ}えつ、

\原本頁{166-6}%
『
\ruby{歸}{かへ}りましたら
\ruby{早{\換字{速}}}{さつ|そく}
\ruby{衣類}{い|るい}も
\ruby{{\換字{送}}}{おく}りましやうし、
%
\ruby{{\換字{又}}}{また}、
%
\ruby{當人}{たう|にん}の
\ruby{小{\換字{遣}}}{こづ|かひ}
なんぞは
\ruby{御厄介}{ご|やく|かい}に
ならないやうに
\ruby{致}{いた}しましやう。
%
\ruby{萬々一}{まん|〳〵|いち}
\ruby{當人}{たう|にん}が
\ruby{不都合}{ふ|つ|がふ}な
\ruby{事}{こと}でも
\ruby{仕出}{し|だ}しましたらば、
%
\ruby{決}{けつ}して
\ruby{御{\換字{迷}}惑}{ご|めい|わく}は
\ruby{掛}{か}けませぬやうに、
%
\ruby{屹度}{きつ|と}
\ruby[|j|]{私}{わたくし}が
\ruby{引{\換字{請}}}{ひき|うけ}まするから、
%
\ruby[|g|]{何卒}{どうぞ}
\ruby{御奉公人}{ご|ほう|こう|にん}
\ruby{同樣}{どう|やう}に
\ruby{御扱}{お|あつか}ひなすつて、
%
\ruby{末々}{すゑ|〴〵}を
\ruby{宜}{よろ}しく
\ruby{御願}{お|ねが}ひ
\ruby{申}{まを}しまする。
%
ほんとに
\ruby{少}{ちひさ}い
\ruby{時}{とき}から
\ruby[|g|]{御馴染}{おなじみ}
\ruby{申}{まを}したのが
\ruby{當人}{たう|にん}の
\ruby{幸福}{しあ|はせ}とは%「幸福」ここは「は」
\ruby{申}{まを}しながら、
%
\ruby{是}{これ}といふ
\ruby{譯}{わけ}も
\原本頁{167-1}\改行%
\ruby{無}{な}いのに
\ruby{斯樣}{こ|ん}な
\ruby{我儘者}{わが|まゝ|もの}を
\ruby{御願}{お|ねが}ひ
\ruby{申}{まを}しまして、
%
そして
\ruby{快}{こゝろ}よく
\ruby{御引受}{お|ひき|う}け
くだすつて
\ruby{頂}{いたゞ}くといふのも、
%
\ruby{思}{おも}へば
\ruby{餘}{あま}り
\ruby{有}{あ}り
\ruby{{\換字{難}}{\換字{過}}}{がた|す}ぎまして、
%
\原本頁{167-3}\改行%
\ruby{何}{なん}だか
\ruby{不思議}{ふ|し|ぎ}なやうな
\ruby{氣}{き}が
\ruby{致}{いた}します
\ruby{位}{くらゐ}でございます。
』

\原本頁{167-4}%
と
\ruby{眞顏}{ま|がほ}になつて
\ruby{恩}{おん}を
\ruby{謝}{しや}するを、
%
お
\ruby{彤}{とう}は
\ruby[|g|]{嫣然}{にこ|り}と
\ruby{打}{うち}
\ruby{笑}{わら}つて、

\原本頁{167-5}%
『
なあに、
%
\ruby{其樣}{そ|ん}なに
\ruby{恩}{おん}に
\ruby{被}{き}て
\ruby{下}{くだ}さる
\ruby{事}{こと}は
\ruby{有}{あ}りやあ
\ruby{仕}{し}ません、
%
\ruby{人}{ひと}は
\ruby{各自}{めい|〳〵}の
\ruby{氣性}{き|しやう}で
\ruby[|g|]{種々}{いろん}な
\ruby{事}{こと}を
\ruby{爲}{す}るのですもの!。
%
\ruby{好}{す}いた
\ruby[|g|]{{\換字{盆}}栽}{うゑき}の
\ruby{世話}{せ|わ}を
\ruby{仕}{し}たからつて、
%
\ruby[|g|]{{\換字{盆}}栽}{うゑき}に
\ruby{御禮}{お|れい}を
\ruby{云}{い}はれやうつて
\ruby{思}{おも}ふ
\ruby{人}{ひと}は
\ruby[|g|]{一人}{ひとり}
\ruby{有}{あ}りやあ
\ruby{仕}{し}ません、
%
たゞ
\ruby{其}{そ}の
\ruby{樹}{き}が
\ruby{好}{よ}くさへなりやあ
\ruby{其}{それ}が
\ruby{嬉}{うれ}しいので。
%
\ruby{不思議}{ふ|し|ぎ}な
\ruby{事}{こと}も
\ruby{何}{なに}も
\ruby{有}{あ}りやあ
\ruby{仕}{し}ませんは、
%
\ruby{妾}{わたし}あ
\ruby{一體}{いつ|たい}
お
\ruby{龍}{りう}ちやんが
\ruby{好}{す}きなんですもの!。
%
たゞ
お
\ruby{龍}{りう}ちやんが
\ruby{好}{よ}くさへ
なつて
お
\ruby{吳}{く}れ
なら
それで
\ruby{本望}{ほん|まう}なので、
%
\ruby{何樣}{ど|ん}なにか
\ruby{嬉}{うれ}しく
\ruby{思}{おも}ふか
\ruby{知}{し}れや
\ruby{仕}{し}ません。
』

\原本頁{168-2}%
と
\ruby{輕}{かろ}く
\ruby{答}{こた}ふれば、
%
\ruby{何}{なに}
\ruby{不足}{ふ|そく}
\ruby{無}{な}き
\ruby{人}{ひと}の
\ruby{氣}{き}の
\ruby{持}{も}ち
\ruby{方}{かた}は
また
\ruby{{\換字{違}}}{ちが}ふもの、
%
\ruby{世}{よ}には
\ruby{此}{こ}の
\ruby{樣}{やう}な
\ruby{人}{ひと}も
\ruby{有}{あ}ることか、
%
と
\ruby[|g|]{田舎}{ゐなか}
\ruby{者}{もの}の
\ruby{我}{わ}が
\ruby{心}{こゝろ}の
\ruby{狹}{せま}く
\ruby{堅}{かた}くろしきに
\ruby{比}{くら}べて
つく〴〵
\ruby{{\換字{感}}}{かん}じ
\ruby{入}{い}る
\ruby{時}{とき}、

\原本頁{168-5}%
『
あの、
%
お
\ruby{富}{とみ}の
\ruby[|g|]{親{\換字{父}}}{おやぢ}
で
ございますつて、
%
\ruby{妙}{めう}な
\ruby[|g|]{老夫}{おぢい}さんが
\ruby{御臺{\換字{所}}口}{お|だい|どころ|ぐち}
へ
まゐり
ましたが、
%
お
\ruby{杉}{すぎ}さんも
\ruby{知}{し}つて
\ruby{居}{ゐ}る
\ruby{人}{ひと}のやうに
\ruby{見}{み}えます、
%
\ruby{何樣}{ど|う}
\ruby{致}{いた}しましやう。
』

\原本頁{168-8}%
と、
%
\ruby{其}{そ}の
\ruby{來}{きた}れる
\ruby{客}{きやく}の
\ruby{如何}{い|か}なる
\ruby{人}{ひと}なるかを
\ruby{小}{ちひさ}き
\ruby{胸}{むね}に
\ruby{危}{あやぶ}むが
\ruby{如}{ごと}き
\ruby{眼色}{め|いろ}して、
%
\ruby{年}{とし}
\ruby{{\換字{若}}}{わか}く
\ruby{可憐}{か|はい}らしき% 原本通り非グループルビ
お
\ruby{春}{はる}は
\ruby{取次}{とり|つ}ぎたり。

\原本頁{168-10}%
『
いゝよ。
%
\ruby[|g|]{彼方}{あちら}へ
\ruby{行}{い}つて
\ruby{會}{あ}ふのも
\ruby{面倒}{めん|だう}だから、
%
\ruby{此室}{こ|ゝ}へ
\ruby{{\換字{連}}}{つ}れて
おいで!。
』

\原本頁{169-1}%
『
お
\ruby{富}{とみ}の
\ruby{親}{おや}つて、
%
\ruby{彼}{あ}の
\ruby{妾}{わたし}の
\ruby{好}{す}きな
お
\ruby{富}{とみ}さんの?。
』

\原本頁{169-2}%
『
アヽ、
%
\ruby{彼女}{あ|れ}の。
』

\原本頁{169-3}%
『
\ruby{彼女}{あの|ひと}は
\ruby{{\換字{退}}}{さが}つたの?。
』

\原本頁{169-4}%
『
いゝえ、
%
\ruby{然樣}{さ|う}
\ruby{定}{き}まつた
\ruby{譯}{わけ}ちやあ
\ruby{無}{な}いが、
%
\ruby{大方}{おほ|かた}
それで
\ruby{來}{き}た
のだらう。
』

\原本頁{169-6}%
お
\ruby{龍}{りう}と
お
\ruby{彤}{とう}との
\ruby{間}{あひだ}に
\ruby{問}{とひ}と
\ruby{答}{こた}へとの
\ruby{{\換字{交}}}{か}はさるゝ
\ruby{間}{ま}も
\ruby{無}{な}く、
%
お
\ruby{春}{はる}に
\ruby{導}{みちび}かれて
\ruby{屈}{かゞ}みながら
\ruby[|g|]{此方}{こなた}へ
\ruby{來}{きた}れる
\ruby{男}{をとこ}は、
%
お
\ruby{彤}{とう}の
\ruby{面}{おもて}をば
\ruby{見}{み}るや
\ruby{見}{み}ざるや、
%
\ruby{室}{へや}の
\ruby{内}{うち}へは
\ruby{入}{はい}りも
\ruby{得}{え}せず
\ruby{恐}{おそ}れ〳〵て
\ruby{鴫居}{しき|ゐ}の
\ruby{外}{そと}に
\ruby{坐}{すわ}りつ、
%
\原本頁{169-9}\改行%
\ruby{先}{ま}づ
\ruby{其}{そ}の
\ruby{癯}{や}せ
\ruby{枯}{から}びて
いと
\ruby{薄}{うす}く
\ruby{長}{なが}う
\ruby{見}{み}ゆる
\ruby{掌}{て}を
\ruby{疊}{たゝみ}に
\ruby{並}{なら}べ
\ruby{貼}{つ}けて、
%
\原本頁{169-10}\改行%
\ruby{頭}{かしら}を
\ruby{其}{そ}の
\ruby{上}{うへ}に
\ruby{摺}{す}りつけ
\ruby{叮嚀}{てい|ねい}に
\ruby{挨拶}{あい|さつ}したるが、
%
\ruby{電燈}{でん|とう}の
\ruby{鮮}{あざ}やかなる
\原本頁{169-11}\改行%
\ruby{光}{ひか}りは、
%
\ruby{光澤}{つ|や}
\ruby{無}{な}き
\ruby{細}{ほそ}き
\ruby{毛}{け}の
\ruby{烟}{けむり}のやうに
ほや〳〵と
\ruby{薄}{うす}く
\ruby{殘}{のこ}れる
\ruby{頭顱}{かう|べ}を
\ruby{照}{て}らして、
%
\ruby{悲}{かな}しき
\ruby{老}{おい}
のさまを
\ruby{見}{あら}はし、
%
\ruby{左}{さ}のみ
\ruby{見苦}{み|ぐる}しき
\ruby[|g|]{襤褸}{つゞれ}を
\ruby{纒}{まと}へり
とには
あらねども、
%
\ruby{肩}{かた}
\ruby{窄}{すぼ}りて
\ruby{何處}{ど|こ}と
\ruby{無}{な}く
\ruby{{\換字{寒}}}{さむ}げなる
\ruby{樣子}{やう|す}は、
%
\ruby{見}{み}るものをして
\ruby{此}{こ}の
\ruby{人}{ひと}
\ruby{{\換字{貧}}}{ひん}に
\ruby{窶}{やつ}れて
\ruby{苦}{くるし}めるには
あらずやと
\ruby{思}{おも}はしめたり。
