\Entry{其三十一}

% メモ 校正終了 2024-03-18 2024-06-11
\原本頁{170-6}%
『
\ruby{好}{よ}く
お
\ruby[g]{入來}{い で }だつた、
%
さあ
\ruby[g]{{\換字{遠}}慮}{ゑんりよ}
\ruby[g]{仕無}{し な }いで
\ruby[|g|]{此方}{こつち}へ
\ruby[g]{御入}{お はい}り。
』

\原本頁{170-7}%
と、
%
お
\ruby{彤}{とう}に
\ruby{優}{やさ}しく
\ruby[g]{言葉}{ことば }を
\ruby{掛}{か}けられて、
%
\ruby[g]{老人}{らうじん}は
\ruby{漸}{やうや}くに
\ruby{頭}{かしら}をこそ
\ruby{擡}{あ}げたれ、

\原本頁{170-9}%
『
ハイ、
%
ハイ。
』

\原本頁{170-10}%
と
ばかり
にて
\ruby{{\換字{猶}}}{なほ}
\ruby[g]{中々}{なか〳〵}に
\ruby{席}{せき}を
\ruby{{\換字{進}}}{すゝ}まず。

\原本頁{171-1}%
『
お
\ruby{富}{とみ}は
\ruby[g]{何樣}{ど う }
\ruby{仕}{し}ましたえ?。
』

\原本頁{171-2}%
と、
%
\ruby{親}{した}しげに
\ruby{復}{また}
\ruby{問}{と}はれて、

\原本頁{171-3}%
『
ハイ、
%
ハイ。
%
イエ、
%
どうも
\ruby{不都合}{ふ|つ|がふ}な
\ruby{奴}{やつ}でございまして、
%
\ruby[g]{何共}{なんとも}
\改行% 校正作業の簡略化のため
、
\原本頁{171-4}\改行%
ハヤ、
%
どうも
\ruby[g]{申上}{まをしあ}げやうも
ございませんで。
』

\原本頁{171-5}%
と、
%
\ruby{脫}{ぬ}け
\ruby{上}{あが}りたる
\ruby{額}{ひたひ}、
%
\ruby{細}{ほそ}き
\ruby{鼻}{はな}、
%
たださへ% ルビ調整(原本通り)非踊り字表記
\ruby[g]{{\換字{貧}}相}{ひんさう}の
\ruby{面}{おもて}に
\ruby[g]{虛僞}{いつはり}ならぬ
\ruby[g]{當惑}{たうわく}の
\ruby{色}{いろ}を
\ruby{見}{あらは}し、
%
\ruby{甚}{いた}く
\ruby[<j||]{恐}{きよう}
\ruby[||j>]{縮}{しゆく}
して
\ruby{同}{おな}じ
\ruby{樣}{やう}の
\ruby{事}{こと}
のみを
\ruby{云}{い}へるは、
%
\ruby[g]{傍眼}{わきめ }の
お
\ruby{龍}{りう}にさへ
もどかしく
\ruby{聞}{きこ}えたり。

\原本頁{171-8}%
\ruby{身}{み}に
\ruby[g]{光澤}{て り }も
\ruby{無}{な}く
\ruby{氣}{き}に
\ruby{張}{は}りも
\ruby{無}{な}くて、
%
たゞ
\ruby[g]{老猫}{ふるねこ}の
\ruby{寢}{ね}ぼれたる
やうの、
%
\ruby{此}{こ}の
\ruby[g]{老人}{らうじん}の
\ruby[g]{樣子}{やうす }を、
%
お
\ruby{彤}{とう}は
\ruby[g]{心底}{しんそこ}より
\ruby[|g|]{可笑}{をかし}がりてか、
%
\ruby{唇}{くち}の
\ruby[<j||]{邊}{あたり}に
ちらりと
\ruby[g]{笑を}{わらひ }
ば
\ruby{上}{のぼ}せしが、
%
\ruby[g]{忽地}{たちまち}にして
\ruby{自}{みづか}ら
\ruby{抑}{おさ}へて、

\原本頁{171-11}%
『
そんなに
\ruby[|g|]{謝罪}{あやま}つて
ばかり
おいで
ぢやあ
\ruby{話}{はなし}が
\ruby[g]{出來}{で き }ませんよ。
%
\ruby[g]{何樣}{ど う }したのだえ
お
\ruby{富}{とみ}は?。
』

\原本頁{172-2}%
と、
%
\ruby{極}{きは}めて
\ruby[g]{{\換字{平}}穩}{おだやか}に
\ruby{問}{と}へば、
%
\ruby[g]{老人}{らうじん}は
\ruby{辛}{から}くも
\ruby{力}{ちから}を
\ruby{得}{え}たりと
\ruby{覺}{おぼ}しく、

\原本頁{172-3}%
『
ハイ。
%
イエ、
%
どうも
\ruby{飛}{と}んでも
\ruby{無}{な}い
\ruby[g]{大變}{たいへん}な
\ruby[g]{{\換字{過}}失}{あやまち}を
\ruby[g]{彼女}{あ れ }が
\ruby{致}{いた}しまして、
』

\原本頁{172-5}%
と
\ruby{云}{い}ひかけて
\ruby{復}{また}
\ruby[g]{叮嚀}{ていねい}に
\ruby{頭}{かしら}を
\ruby{下}{さ}げたり。

\原本頁{172-6}%
\ruby{笑}{わら}ふべき
\ruby{事}{こと}には
あらねど
\ruby{何}{なん}と
\ruby{無}{な}く
\ruby{其}{そ}の
\ruby{眞面目}{ま|じ|め}
\ruby{{\換字{過}}}{す}ぎ
\ruby[|g|]{萎縮}{いぢけ}
\ruby{{\換字{過}}}{す}ぎたる
\ruby{樣}{さま}の、
%
\ruby{氣}{き}の
\ruby{毒}{どく}らしきを
\ruby{越}{こ}して
\ruby{稍}{や}% 「稍(Shāo)」→「少し、ちょっと、わずか、やや」
\footnote{「稍(Shāo)」の意味等は(少し・ちょっと・わずか・やや)なので(やや)が妥当と思われるが原本通りとする
(国会図書館 コマ番号90/146 p-172 l-07)}%
\ruby[|g|]{可笑}{をかし}きに、
%
お
\ruby{龍}{りう}は
\ruby{思}{おも}はず
\ruby{眼}{め}のみに
\ruby{笑}{わら}ひたり。

\原本頁{172-9}%
『
そんなに
\ruby[|g|]{謝罪}{あやま}つて
ばかり
\ruby{居}{ゐ}ないでも
\ruby{宜}{よ}う
ござんす
といふのに
\改行% 校正作業の簡略化のため
。
』

\原本頁{172-10}%
『
ハイ、
%
イエ、
%
\ruby[g]{然樣}{さ う }
\ruby{仰}{おつし}あつて
\ruby{下}{くだ}さいますと、
%
\ruby[<j>]{愈}{いよ〳〵}
\ruby[||j>]{恐}{ おそ}れ% ルビ調整(特殊処理)ルビが重なるので調整
\ruby{入}{い}りますので。
%
\ruby{{\換字{廻}}}{まは}り
くどう
ございましやうが
\ruby[g]{御詫}{お わび}を
\ruby{申}{まを}し
\ruby{上}{あ}げます、
%
\ruby[|g|]{何卒}{どうぞ}
\ruby[g]{御聞}{お き }き
\ruby{下}{くだ}さいます
やうに。
%
\原本頁{173-1}%
もう
これ
お
\ruby{詫}{わび}にも
\ruby{出}{で}そびれて
\ruby[g]{十日}{とほか }
ばかりに
なりましたが。
%
\ruby[g]{然樣}{さ よう}、
エヽト、
%
コート、
%
\ruby[g]{丁度}{ちやうど}
\ruby[g]{今日}{こんにち}で
\ruby{十一日}{じふ|いち|にち}に
なります。
%
\ruby[g]{彼女}{あ れ }が
\ruby[|g|]{貴女}{あなた}、
%
\ruby[g]{眞靑}{まつさを}な
\ruby{顏}{かほ}をして
\ruby{駈}{か}け
\ruby{{\換字{込}}}{こ}んで
まゐりまして、
%
\ruby{御主人樣}{ご|しゆ|じん|さま}の
\ruby[|g|]{御大切}{おだいじ}な
\ruby{御菓子鉢}{お|くわ|し|ばち}を
\ruby[g]{仕舞}{し ま }はう
とする
\ruby{時}{とき}、
%
つい
\ruby{取}{と}り
\ruby{落}{おと}して
\ruby{割}{わ}つて
\ruby[g]{仕舞}{し ま }つたと
\ruby{申}{まを}す
ので
ございます。
』

\原本頁{173-6}%
『
ハア、
%
\ruby[g]{大方}{おほかた}
\ruby[g]{其故}{そ れ }で
\ruby{駈}{か}け
\ruby{出}{だ}して
\ruby{行}{い}つて
\ruby[g]{仕舞}{し ま }つた
のだ
らうと
\ruby{妾}{わたし}も
\ruby{思}{おも}つて
\ruby{居}{ゐ}たが、
%
\ruby{今}{いま}に
\ruby{何}{なん}とか
\ruby{云}{い}つて
おいで
だらうと
\ruby{思}{おも}つて
\ruby{人}{ひと}も
あげなかつたの。
%
\ruby[g]{然樣}{さ う }です、
%
\ruby[g]{{\換字{古}}渡}{こ わた}りの
\ruby{繪南京}{ゑ|なん|きん}の、
%
\ruby[g]{一寸}{ちよつと}
\ruby{無}{な}い
\ruby{鉢}{はち}を
\ruby{破}{わ}つて
\ruby[g]{仕舞}{し ま }つたので。
』

\原本頁{173-10}%
『
ハ、
%
ハイ、
%
ハイ。
%
どうも
\ruby{飛}{と}んでも
\ruby{無}{な}い
\ruby[g]{麁怱}{そ さう}を
\ruby{致}{いた}しました
\ruby{事}{こと}で。
%
\ruby[g]{其品}{そ れ }は
\ruby[g]{利齋}{り さい}とか
\ruby{仰}{おつし}ある
\ruby{方}{かた}が
\ruby{納}{をさ}めました
\ruby{品}{もの}で
ございまして、
%
\ruby{其}{その}
\ruby{折}{をり}
\原本頁{174-1}\改行%
\ruby[g]{色々}{いろ〳〵}と
\ruby{其}{そ}の
\ruby{仁}{じん}が
\ruby{其}{そ}の
\ruby{御}{お}
\ruby{器}{うつは}の
\ruby[g]{結構}{けつこう}な
\ruby{事}{こと}を
\ruby[g]{御話}{お はな}し
なさいました
\ruby[g]{其談}{そ れ }を
ちら〳〵
\ruby[g]{彼女}{あ れ }が
\ruby[|j|]{承}{うけたま}はつて
\ruby{居}{を}つたさうで、
%
\ruby{何}{なに}も
\ruby{{\換字{分}}}{わか}りません
\ruby[g]{彼女}{あ れ }でも
\ruby[g]{大層}{たいそう}
\ruby[g]{結構}{けつこう}な
\ruby{貴}{たつと}い
\ruby[g]{御品}{お しな}だ
といふ
\ruby{事}{こと}だけは
\ruby{存}{ぞん}じて
\ruby{居}{を}りました
\原本頁{174-4}\改行%
\ruby{故}{ゆゑ}、
%
これは
\ruby[g]{御詫}{お わび}の
\ruby{仕}{し}やうも
\ruby{無}{な}い
\ruby{事}{こと}を
\ruby{仕}{し}たと、
%
ト
\ruby{胸}{むね}を
\ruby{衝}{つ}いたと
\ruby{申}{まを}す
ので
ございまして。
%
\ruby[g]{何樣}{ど う }も
\ruby{何}{なん}とも
ハヤ
\ruby{相}{あひ}
\ruby{濟}{す}みません
\ruby{事}{こと}で。
%
ハイ、
%
ハイ。
%
それから
\ruby[|j|]{私}{わたくし}が
\ruby[|g|]{貴女}{あなた}、
%
\ruby{代}{かは}りの
\ruby{品}{しな}を
\ruby{差}{さし}
\ruby{出}{いだ}しまして
\ruby{御勘辨}{ご|かん|べん}を% 弁 瓣 辦 辧 (辨) 辩 辯
\ruby{願}{ねが}はうと
\ruby{存}{ぞん}じまして、
%
\ruby[g]{彼女}{あ れ }と
\ruby[|g|]{二人}{ふたり}で
\ruby[<j||]{東}{とう }% ルビ調整(特殊処理)ルビが重なるので特例処置
\ruby[<j||]{京}{きやう}
% \ruby{東京}{とう|きやう}
\ruby{中}{ぢゆう}を
\ruby{搜}{さが}しましたが
\改行% 校正作業の簡略化のため
、
%
\原本頁{174-8}\改行%
\ruby[g]{中々}{なか〳〵}
どう
\ruby{致}{いた}しまして
\ruby{似}{に}たやうな
\ruby{品}{もの}も
ございません。
』

\原本頁{174-9}%
『
まあ
\ruby{詰}{つま}らない
そんな
\ruby[g]{餘計}{よ けい}な
\ruby[g]{苦勞}{く らう}を
\ruby{仕}{し}て
\ruby{貰}{もら}はうとも
\ruby{何}{なん}とも
\ruby[|g|]{此方}{こちら}
ぢやあ
\ruby{思}{おも}つて
\ruby{居}{ゐ}も
\ruby{仕}{し}ないものを!。
』

\原本頁{174-9}%
『
ハイ、
%
ハイ。
%
まことに
\ruby[g]{何樣}{ど う }も
\ruby{恐}{おそ}れ
\ruby{入}{い}りましたことで。
%
\ruby[g]{然樣}{さ う }
\ruby[<j||]{仰}{おつし}あつて% 行末行頭の境界付近なので特例処置を施す
\ruby{下}{くだ}さいましても、
%
\ruby{夫}{それ}では
\ruby{濟}{す}みません
\ruby{譯}{わけ}で。
%
\ruby[|g|]{貴女}{あなた}、
%
\ruby[g]{彼女}{あ れ }が
\ruby[|g|]{此方}{こなた}
\ruby{樣}{さま}へ
まゐります
\ruby{{\換字{前}}}{まへ}に
\ruby{御奉公}{ご|ほう|こう}
\ruby{致}{いた}して
\ruby{居}{を}りました
\ruby[g]{御邸}{おやしき}は
\ruby[<j||]{伯}{はく }% ルビ調整(特殊処理)連続するルビ対策
\ruby[<j||]{爵}{しやく}
% \ruby{伯爵}{はく|しやく}
\ruby{樣}{さま}とかで
いらつしやいましたが、
%
\ruby[|g|]{彼方}{かなた}
\ruby{樣}{さま}では
\ruby{都}{す}べて
\ruby[||j>]{女}{ぢよ}
\ruby[||j>]{中}{ちゆう}の
% \ruby{女中}{ぢよ|ちゆう}の
\ruby{毀}{こは}しましたものは
\ruby[<j||]{皆}{みんな}
\ruby{其}{そ}の
\ruby{毀}{こは}したものが
\ruby{償}{つぐな}ひまする
\ruby{御}{お}
\ruby[|g|]{定規}{さだめ}
でございまして、
%
\ruby[g]{彼女}{あ れ }なぞは
\ruby[||j>]{頂}{ちやう}
\ruby[||j>]{戴}{ だい}
% \ruby{頂戴}{ちやう|だい}
するものが
\ruby[g]{少う}{すくな }
ございますから、
%
\ruby[g]{始{\換字{終}}}{し じう}% ルビ調整(原本通り)「ゆ」無し
\ruby[g]{持出}{もちだ }し
になります
やうな
\ruby{事}{こと}
で
ございました
\ruby{位}{くらゐ}で。
』

\原本頁{175-7}%
『
ヘーエ!。
』

\原本頁{175-8}%
『
でございますから
\ruby[|g|]{貴女}{あなた}、
%
\ruby[<j>]{私}{わたくし}は
\ruby[<j||]{一}{いつ }
\ruby[<j||]{生}{しやう}
% \ruby{一生}{いつ|しやう}
\ruby[g]{懸命}{けんめい}に
\ruby{搜}{さが}しまして、
%
\ruby{{\換字{終}}}{しまひ}には
\ruby[g]{利齋}{り さい}といふ
\ruby{人}{ひと}まで
\ruby{{\換字{尋}}}{たづ}ねまして
\ruby[g]{仔細}{し さい}を
\ruby{話}{はな}しまして、
%
これ〳〵の
\ruby{鉢}{はち}が
\ruby{欲}{ほし}しいと
\ruby{申}{まを}しました
ところ、
%
\ruby{今}{いま}
\ruby{欲}{ほ}しいと
\ruby{云}{い}つても
\ruby{今}{いま}
\ruby{有}{あ}るものでも
\ruby{無}{な}いし、
%
\ruby{有}{あ}つたに
\ruby{致}{いた}しても
\ruby[g]{如是}{これ〳〵}の
\ruby{價}{ね}のものだと
\ruby[<j>]{承}{うけたま}はり
まして、
%
\ruby[<j>]{私}{わたくし}
\ruby[||j>]{{\換字{連}}}{ づれ}の
\ruby{力}{ちから}には
\ruby{及}{およ}びかねます
\ruby[g]{大變}{たいへん}なもので
ございましたので
\改行% 校正作業の簡略化のため
、
%
\原本頁{176-2}\改行%
いよ〳〵
\ruby[g]{吃驚}{びつくり}
\ruby{致}{いた}しまして、
%
とても
のめ〳〵と
\ruby[g]{御詫}{お わび}に
\ruby{出}{で}られた
\ruby{段}{だん}では
ございませんが、
%
\ruby{死}{し}ぬやうな
\ruby{氣}{き}になつて
\ruby{漸}{や}つと
\ruby[|g|]{今日}{こんち}
\ruby[g]{御詫}{お わび}に
\ruby{出}{で}ましたで。
』

\原本頁{176-5}%
こゝまで
\ruby{云}{い}ひさして
\ruby{埋}{うづ}むるが
\ruby{如}{ごと}く
\ruby{疊}{たゝみ}に
\ruby{頭}{かうべ}を
\ruby{擦}{す}りつけたる
\ruby{時}{とき}、
%
\ruby{薄}{うす}き
\ruby{髮}{かみ}の
\ruby{下}{した}に
\ruby{{\換字{透}}}{す}きて
\ruby{見}{み}えたる
\ruby[|g|]{頭顱}{あたま}の
\ruby{地}{ぢ}には、
%
\ruby[g]{如何}{い か }ばかり
\ruby{{\換字{弱}}}{よわ}き
\ruby[<j||]{心}{こゝろ}の% 行末行頭の境界付近なので特例処置を施す
\原本頁{176-7}\改行%
\ruby{苦}{くる}しくや
\ruby{{\換字{感}}}{かん}じけん、
%
\ruby{慚}{はづ}かしさと
\ruby[g]{切無}{せつな }さに% ルビ調整(補正)原本のルビは(せ な)とはっきりしないが、意味的に補正した
\ruby{絞}{しぼ}り
\ruby{出}{いだ}されたる
\ruby{熱}{あつ}き
\ruby{汗}{あせ}
\原本頁{176-8}\改行%
の
\ruby[g]{點々}{てん〳〵}と
\ruby{玉}{たま}を
なして、
%
\ruby[g]{蒸氣}{ゆ げ }さへ
いさゝか
\ruby{立}{た}つ
ごとく
\ruby{見}{み}えたり。
