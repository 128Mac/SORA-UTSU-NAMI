\Entry{其三十一}
% メモ 校正終了 2024-03-18

\原本頁{170-6}
『
\ruby{好}{よ}く
お
\ruby{入來}{い|で}だつた、
%
さあ
\ruby{{\換字{遠}}慮}{ゑん|りよ}
\ruby{仕無}{し|な}いで
\ruby{此方}{こつ|ち}へ
\ruby{御入}{お|はい}り。
』

\原本頁{170-6}
と、
%
お
\ruby{彤}{とう}に
\ruby{優}{やさ}しく
\ruby{言葉}{こと|ば}を
\ruby{掛}{か}けられて、
%
\ruby{老人}{らう|じん}は
\ruby{漸}{やうや}くに
\ruby{頭}{かしら}をこそ
\ruby{擡}{あ}げたれ、

\原本頁{170-9}
『ハイ、
%
ハイ。
』

\原本頁{170-10}
とばかりにて
\ruby{{\換字{猶}}}{なほ}
\ruby{中々}{なか|〳〵}に
\ruby{席}{せき}を
\ruby{{\換字{進}}}{すゝ}まず。

\原本頁{171-1}
『お
\ruby{富}{とみ}は
\ruby{何樣}{ど|う}
\ruby{仕}{し}ましたえ?。
』

\原本頁{171-2}
と、
%
\ruby{親}{した}しげに
\ruby{復}{また}
\ruby{問}{と}はれて、

\原本頁{171-3}
『ハイ、
%
ハイ。
%
イエ、
%
どうも
\ruby{不都合}{ふ|つ|がふ}な
\ruby{奴}{やつ}でございまして、
%
\ruby{何共}{なん|とも}ハヤ、
%
どうも
\ruby{申上}{まをし|あ}げやうもございませんで。
』

\原本頁{171-5}
と、
%
\ruby{脫}{ぬ}け
\ruby{上}{あが}りたる
\ruby{額}{ひたひ}、
%
\ruby{細}{ほそ}き
\ruby{鼻}{はな}、
%
たゞさへ
\ruby{{\換字{貧}}相}{ひん|さう}の
\ruby{面}{おもて}に
\ruby{虛僞}{いつ|はり}ならぬ
\ruby{當惑}{たう|わく}の
\ruby{色}{いろ}を
\ruby{見}{あらは}し、
%
\ruby{甚}{いた}く
\ruby{恐縮}{きよう|しゆく}して
\ruby{同}{おな}じ
\ruby{樣}{やう}の
\ruby{事}{こと}のみを
\ruby{云}{い}へるは、
%
\ruby{傍眼}{わき|め}の
お
\ruby{龍}{りう}にさへ
もどかしく
\ruby{聞}{きこ}えたり。

\原本頁{171-8}
\ruby{身}{み}に
\ruby{光澤}{て|り}も
\ruby{無}{な}く
\ruby{氣}{き}に
\ruby{張}{は}りも
\ruby{無}{な}くて、
%
たゞ
\ruby{老猫}{ふる|ねこ}の
\ruby{寢}{ね}ぼれたるやうの、
%
\ruby{此}{こ}の
\ruby{老人}{らう|じん}の
\ruby{樣子}{やう|す}を、
%
お
\ruby{彤}{とう}は
\ruby{心底}{しん|そこ}より
\ruby{可笑}{を|か}しがりてか、
%
\ruby{唇}{くち}の
\ruby{邊}{あたり}に
ちらりと
\ruby{笑}{わらひ}をば
\ruby{上}{のぼ}せしが、
%
\ruby{忽地}{たちま|ち}にして
\ruby{自}{みづか}ら
\ruby{抑}{おさ}へて、

\原本頁{171-11}
『そんなに
\ruby{謝罪}{あや|ま}つて
ばかりおいでぢやあ
\ruby{話}{はなし}が
\ruby{出來}{で|き}ませんよ。
%
\ruby{何樣}{ど|う}したのだえ
お
\ruby{富}{とみ}は?。
』

\原本頁{172-2}
と、
%
\ruby{極}{きは}めて
\ruby{{\換字{平}}穩}{おだ|やか}に
\ruby{問}{と}へば、
%
\ruby{老人}{らう|じん}は
\ruby{辛}{から}くも
\ruby{力}{ちから}を
\ruby{得}{え}たりと
\ruby{覺}{おぼ}しく、

\原本頁{172-3}
『ハイ。
%
イエ、
%
どうも
\ruby{飛}{と}んでも
\ruby{無}{な}い
\ruby{大變}{たい|へん}な
\ruby{{\換字{過}}失}{あや|まち}を
\ruby{彼女}{あ|れ}が
\ruby{致}{いた}しまして、
』

\原本頁{172-5}
と
\ruby{云}{い}ひかけて
\ruby{復}{また}
\ruby{叮嚀}{てい|ねい}に
\ruby{頭}{かしら}を
\ruby{下}{さ}げたり。

\原本頁{172-6}
\ruby{笑}{わら}ふべき
\ruby{事}{こと}には
あらねど
\ruby{何}{なん}と
\ruby{無}{な}く
\ruby{其}{そ}の
\ruby{眞面目}{ま|じ|め}
\ruby{{\換字{過}}}{す}ぎ
\ruby{萎縮}{いぢ|け}
\ruby{{\換字{過}}}{す}ぎたる
\ruby{樣}{さま}の、
%
\ruby{氣}{き}の
\ruby{毒}{どく}らしきを
\ruby{越}{こ}して
\ruby{稍}{や}
\ruby{可笑}{を|かし}きに、
%
お
\ruby{龍}{りう}は
\ruby{思}{おも}はず
\ruby{眼}{め}のみに
\ruby{笑}{わら}ひたり。

\原本頁{172-9}
『そんなに
\ruby{謝罪}{あや|ま}つてばかり
\ruby{居}{ゐ}ないでも
\ruby{宜}{よ}うござんすと
いふのに。
』

\原本頁{172-10}
『ハイ、
%
イエ、
%
\ruby{然樣}{さ|う}
\ruby{仰}{おつし}あつて
\ruby{下}{くだ}さいますと、
%
\ruby[<-||]{愈}{いよ〳〵}
\ruby{恐}{おそ}れ
\ruby{入}{い}りますので。
%
\ruby{{\換字{廻}}}{まは}り
くどう
ございましやうが
\ruby{御詫}{お|わび}を
\ruby{申}{まを}し
\ruby{上}{あ}げます、
%
\ruby{何卒}{どう|ぞ}
\ruby{御聞}{お|き}き
\ruby{下}{くだ}さいますやうに。
%
\原本頁{173-1}
もうこれ
お
\ruby{詫}{わび}にも
\ruby{出}{で}そびれて
\ruby{十日}{とほ|か}ばかりに
なりましたが。
%
\ruby{然樣}{さ|よう}、
エヽト、
%
コート、
%
\ruby{丁度}{ちやう|ど}
\ruby{今日}{こん|にち}で
\ruby{十一日}{じう|いち|にち}に
なります。
%
\ruby{彼女}{あ|れ}が
\ruby{貴女}{あな|た}、
%
\ruby{眞靑}{まつ|さを}な
\ruby{顏}{かほ}をして
\ruby{駈}{か}け
\ruby{{\換字{込}}}{こ}んで
まゐりまして、
%
\ruby{御主人樣}{ご|しゆ|じん|さま}の
\ruby[g]{御大切}{おだいじ}な
\ruby{御菓子鉢}{お|くわ|し|ばち}を
\ruby{仕舞}{し|ま}はうとする
\ruby{時}{とき}、
%
つい
\ruby{取}{と}り
\ruby{落}{おと}して
\ruby{割}{わ}つて
\ruby{仕舞}{し|ま}つたと
\ruby{申}{まを}すので
ございます。
』

\原本頁{173-6}
『ハア、
%
\ruby{大方}{おほ|かた}
\ruby{其故}{そ|れ}で
\ruby{駈}{か}け
\ruby{出}{だ}して
\ruby{行}{い}つて
\ruby{仕舞}{し|ま}つたのだらうと
\ruby{妾}{わたし}も
\ruby{思}{おも}つて
\ruby{居}{ゐ}たが、
%
\ruby{今}{いま}に
\ruby{何}{なん}とか
\ruby{云}{い}つておいでだらうと
\ruby{思}{おも}つて
\ruby{人}{ひと}も
あげなかつたの。
%
\ruby{然樣}{さ|う}です、
%
\ruby{{\換字{古}}渡}{こ|わた}りの
\ruby{繪南京}{ゑ|なん|きん}の、
%
\ruby{一寸}{ちよ|つと}
\ruby{無}{な}い
\ruby{鉢}{はち}を
\ruby{破}{わ}つて
\ruby{仕舞}{し|ま}つたので。
』

\原本頁{173-10}
『ハ、ハイ、
%
ハイ。
%
どうも
\ruby{飛}{と}んでも
\ruby{無}{な}い
\ruby{麁怱}{そ|さう}を
\ruby{致}{いた}しました
\ruby{事}{こと}で。
%
\ruby{其品}{そ|れ}は
\ruby{利齋}{り|さい}とか
\ruby{仰}{おつし}ある
\ruby{方}{かた}が
\ruby{納}{をさ}めました
\ruby{品}{もの}でございまして、
%
\ruby{其折}{その|をり}
\原本頁{174-1}
\ruby{色々}{いろ|〳〵}と
\ruby{其}{そ}の
\ruby{仁}{じん}が
\ruby{其}{そ}の
\ruby{御器}{お|うつは}の
\ruby{結構}{けつ|こう}な
\ruby{事}{こと}を
\ruby{御話}{お|はなし}しなさいました
\ruby{其談}{そ|れ}を
ちら〳〵と
\ruby{彼女}{あ|れ}が
\ruby{承}{うけたま}はつて
\ruby{居}{を}つたさうで、
%
\ruby{何}{なに}も
\ruby{{\換字{分}}}{わか}りません
\ruby{彼女}{あ|れ}でも
\ruby{大層}{たい|そう}
\ruby{結構}{けつ|こう}な
\ruby{貴}{たつと}い
\ruby{御品}{お|しな}だといふ
\ruby{事}{こと}だけは
\ruby{存}{ぞん}じて
\ruby{居}{を}りました
\ruby{故}{ゆゑ}、
%
これは
\ruby{御詫}{お|わび}の
\ruby{仕}{し}やうも
\ruby{無}{な}い
\ruby{事}{こと}を
\ruby{仕}{し}たと、
%
ト
\ruby{胸}{むね}を
\ruby{衝}{つ}いたと
\ruby{申}{まを}すので
ございまして。
%
\ruby{何樣}{ど|う}も
\ruby{何}{なん}ともハヤ
\ruby{相}{あひ}
\ruby{濟}{す}みません
\ruby{事}{こと}で。
%
ハイ、
%
ハイ。
%
それから
\ruby{私}{わたくし}が
\ruby{貴女}{あな|た}、
%
\ruby{代}{かは}りの
\ruby{品}{しな}を
\ruby{差出}{さし|いだ}しまして
\ruby{御勘辨}{ご|かん|べん}を% 弁 瓣 辦 辧 (辨) 辩 辯
\ruby{願}{ねが}はうと
\ruby{存}{ぞん}じまして、
%
\ruby{彼女}{あ|れ}と
\ruby{二人}{ふた|り}で
\ruby{東京中}{とう|きやう|ぢゆう}を
\ruby{搜}{さが}しましたが、
%
\ruby{中々}{なか|〳〵}
どう
\ruby{致}{いた}しまして
\ruby{似}{に}たやうな
\ruby{品}{もの}もございません。
』

\原本頁{174-9}
『まあ
\ruby{詰}{つま}らない
そんな
\ruby{餘計}{よ|けい}な
\ruby{苦勞}{く|らう}を
\ruby{仕}{し}て
\ruby{貰}{もら}はうとも
\ruby{何}{なん}とも
\ruby{此方}{こち|ら}ぢやあ
\ruby{思}{おも}つて
\ruby{居}{ゐ}も
\ruby{仕}{し}ないものを!。
』

\原本頁{174-9}
『ハイ、
%
ハイ。
%
まことに
\ruby{何樣}{ど|う}も
\ruby{恐}{おそ}れ
\ruby{入}{い}りましたことで。
%
\ruby{然樣}{さ|う}
\ruby{仰}{おつし}あつて
\ruby{下}{くだ}さいましても、
%
\ruby{夫}{それ}では
\ruby{濟}{す}みません
\ruby{譯}{わけ}で。
%
\ruby{貴女}{あな|た}、
%
\ruby{彼女}{あ|れ}が
\ruby{此方樣}{こち|ら|さま}へ
まゐります
\ruby{{\換字{前}}}{まへ}に
\ruby{御奉公}{ご|ほう|こう}
\ruby{致}{いた}して
\ruby{居}{を}りました
\ruby{御邸}{お|やしき}は
\ruby{伯爵樣}{はく|しやく|さま}とかで
いらつしやいましたが、
%
\ruby{彼方}{かな|た}
\ruby{樣}{さま}では
\ruby{都}{す}べて
\ruby{女中}{ぢよ|ちゆう}の
\ruby{毀}{こは}しましたものは
\ruby{皆}{みな}
\ruby{其}{そ}の
\ruby{毀}{こは}したものが
\ruby{償}{つぐな}ひまする
\ruby{御定規}{お|さだ|め}でございまして、
%
\ruby{彼女}{あ|れ}なぞは
\ruby{頂戴}{ちやう|だい}するものが
\ruby{少}{すくな}うございますから、
%
\ruby[g]{始{\換字{終}}}{しじう}% ルビは原本通り「ゆ」無し
\ruby{持出}{もち|だ}しになりますやうな
\ruby{事}{こと}でございました
\ruby{位}{ぐらゐ}で。
』

\原本頁{175-7}
『ヘーエー!。
』

\原本頁{175-8}
『でございますから
\ruby{貴女}{あな|た}、
%
\ruby{私}{わたくし}は
\ruby{一生懸命}{いつ|しやう|けん|めい}に
\ruby{搜}{さが}しまして、
%
\ruby{{\換字{終}}}{しまひ}には
\ruby{利齋}{り|さい}といふ
\ruby{人}{ひと}まで
\ruby{{\換字{尋}}}{たづ}ねまして
\ruby{仔細}{し|さい}を
\ruby{話}{はな}しまして、
%
これ〳〵の
\ruby{鉢}{はち}が
\ruby{欲}{ほし}しいと
\ruby{申}{まを}しましたところ、
%
\ruby{今}{いま}
\ruby{欲}{ほ}しいと
\ruby{云}{い}つても
\ruby{今}{いま}
\ruby{有}{あ}るものでも
\ruby{無}{な}いし、
%
\ruby{有}{あ}つたに
\ruby{致}{いた}しても
\ruby{如是}{これ|〳〵}の
\ruby{價}{ね}のものだと
\ruby{承}{うけたま}はりまして、
%
\原本頁{176-1}
\ruby{私{\換字{連}}}{わたくし|づれ}の
\ruby{力}{ちから}には
\ruby{及}{およ}びかねます
\ruby{大變}{たい|へん}なもので
ございましたので
いよ〳〵
\ruby{吃驚}{びつ|くり}
\ruby{致}{いた}しまして、
%
とても
のめ〳〵と
\ruby{御詫}{お|わび}に
\ruby{出}{で}られた
\ruby{段}{だん}では
ございませんが、
%
\ruby{死}{し}ぬやうな
\ruby{氣}{き}になつて
\ruby{漸}{や}つと
\ruby[g]{今日}{こんち}
\ruby{御詫}{お|わび}に
\ruby{出}{で}ましたで。
』

\原本頁{176-5}
こゝまで
\ruby{云}{い}ひさして
\ruby{埋}{うづ}むるが
\ruby{如}{ごと}く
\ruby{疊}{たゝみ}に
\ruby{頭}{かうべ}を
\ruby{擦}{す}りつけたる
\ruby{時}{とき}、
%
\ruby{薄}{うす}き
\ruby{髮}{かみ}の
\ruby{下}{した}に
\ruby{{\換字{透}}}{す}きて
\ruby{見}{み}えたる
\ruby{頭顱}{あた|ま}の
\ruby{地}{ぢ}には、
%
\ruby{如何}{い|か}ばかり
\ruby{{\換字{弱}}}{よわ}き
\ruby{心}{こゝろ}の
\ruby{苦}{くる}しくや
\ruby{感}{かん}じけん、
%
\ruby{慚}{はづ}かしさと
\ruby{切無}{せつ|な}さに
\ruby{絞}{しぼ}り
\ruby{出}{いだ}されたる
\ruby{熱}{あつ}き
\ruby{汗}{あせ}の
\ruby{點々}{てん|〳〵}と
\ruby{玉}{たま}をなして、
%
\ruby{蒸氣}{ゆ|げ}さへ
いさゝか
\ruby{立}{た}つごとく
\ruby{見}{み}えたり。
