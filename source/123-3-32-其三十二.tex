\Entry{其三十二}

\原本頁{}%
『
\ruby{何樣}{ど|う}も
\ruby{何}{なん}と
\ruby{申上}{まをし|あげ}ましても
\ruby{相}{あひ}
\ruby{濟}{す}みません
\ruby{無調法}{ぶ|てふ|はふ}で。
%
ハイ。
%
\ruby{口}{くち}ばかりで
\ruby{何}{なに}を
\ruby{申}{まを}し
\ruby{上}{あ}げましても、
%
\ruby{實以}{じつ|もつ}て
\ruby{相}{あひ}
\ruby{濟}{す}みません
\ruby{譯}{わけ}で、
%
ハイ。
%
お
\ruby{羞}{はづか}しいことを
\ruby{申}{まを}し
\ruby{上}{あ}げませんければ
\ruby{理}{り}が
\ruby{聞}{きこ}えませぬが、
%
\ruby{實}{じつ}は
\ruby{段々}{だん|〴〵}と
\ruby{不幸}{ふ|しあはせ}は%「幸」ここは「は」
\ruby{續}{つゞ}きますし、
%
\ruby{私}{わたくし}は
\ruby{病身}{びやう|しん}で
\ruby{商法}{しやう|はふ}は
\ruby{止}{や}めて
\ruby{居}{を}りますし、
%
\ruby{少}{すこ}しばかりの
\ruby{地{\換字{所}}}{ぢ|しよ}
\ruby{家作}{か|さく}で
\ruby{細々}{ほそ|〴〵}と
\ruby{{\換字{遣}}}{や}つて
\ruby{居}{を}ります
\ruby{中}{なか}を、
%
\ruby{不孝者}{ふ|かう|もの}めの
\ruby{伜}{せがれ}に
\ruby{大無}{だい|な}しにされまして、
%
まことにはや
\ruby{何樣}{ど|う}も
\ruby{斯樣}{か|う}もならないやうになつて
\ruby{居}{を}りまするので、
%
たゞもう
\ruby{明暮}{あけ|くれ}、
%
\ruby{伜}{せがれ}めの
\ruby{碌}{ろく}で
\ruby{無}{な}しの
\ruby{料簡}{れう|けん}の
\ruby{直}{なほ}りますやうにと、
%
\ruby{信心}{しん|〴〵}を
\ruby{致}{いた}すのを
\ruby{今日}{こん|にち}の
\ruby{{\換字{勤}}}{つとめ}に
\ruby{致}{いた}して
\ruby{居}{を}るやうな
\ruby{意氣地}{い|く|ぢ}
の
\ruby{無}{な}い
\ruby{次第}{し|だい}でございますから、
%
\ruby{何共}{なん|とも}
\ruby{恐}{おそ}れ
\ruby{入}{い}りまする
\ruby{身{\換字{勝}}手}{み|がつ|て}な
\ruby{申{\換字{分}}}{まをし|ぶん}ではございますが、
%
\ruby{今}{いま}が
\ruby{今}{いま}
\ruby{何樣}{ど|う}にか
\ruby{致}{いた}さうと
\ruby{致}{いた}しますれば、
%
\ruby[<j|]{私}{わたくし}
\ruby{一人}{ひと|り}のところへ
\ruby{夫{\換字{婦}}掛向}{ふう|ふ|かけ|むか}ひの
\ruby{人}{ひと}を
\ruby{置}{お}きまして、
%
その
\ruby{貸間}{かし|ま}の
\ruby{料}{れう}で
\ruby{食}{た}べて
\ruby{居}{を}りまする
\ruby{住家}{すま|ゐ}をでも、
%
\ruby{何樣}{ど|う}か
\ruby{致}{いた}して
\ruby{算段}{さん|だん}
\ruby{致}{いた}すより
\ruby{他}{ほか}はございませんので、
%
それでは
\ruby{何樣}{ど|う}も
\ruby{後々}{あと|〳〵}のところが‥‥』

\原本頁{}%
\ruby{{\換字{貧}}相}{ひん|さう}な
\ruby{顏}{かほ}を
いよ〳〵
\ruby{{\換字{貧}}相}{ひん|さう}に
\ruby{仕}{し}て
\ruby{困{\換字{難}}}{こん|なん}の
\ruby{趣}{おもむ}きを
\ruby{{\換字{述}}}{の}べ
\ruby{哀愍}{あは|れみ}を
\ruby{乞}{こ}はんとする、
%
\ruby{其}{そ}の
\ruby{言語}{もの|いひ}は
\ruby{人}{ひと}の
\ruby{同{\換字{情}}}{どう|じやう}を
\ruby{惹}{ひ}くに
\ruby{足}{た}るほどの
\ruby{氣合}{き|あひ}さへ
\ruby{乏}{とぼ}しけれど、
%
\ruby{其}{そ}の
くど〳〵しく
\ruby{惡叮嚀}{わる|てい|ねい}なるに
\ruby{愚直}{ぐ|ちよく}さは
\ruby{盡}{こと〴〵}く
\ruby{知}{し}られたり。

\原本頁{}%
お
\ruby{彤}{とう}は
\ruby{最早}{も|はや}
\ruby{聞}{き}き
\ruby{居}{ゐ}るに
\ruby{堪}{た}へかねてや、
%
\ruby{言葉}{こと|ば}の
\ruby{澱}{よど}みに
\ruby{付}{つ}け
\ruby{入}{い}りて
\ruby{{\換字{又}}}{また}
\ruby{靜}{しづか}に
\ruby{{\換字{又}}}{また}
\ruby{爽快}{さわや|か}に、

\原本頁{}%
『まあ
\ruby{其}{それ}は
\ruby{大層}{たい|そう}に
\ruby{心配}{しん|ぱい}を
お
\ruby{爲}{し}だつたねえ。
%
お
\ruby{{\換字{前}}}{まへ}さんは
\ruby{當世}{たう|せい}にあ
\ruby{珍}{めづ}らしい
\ruby{律義}{りち|ぎ}な
\ruby{氣性}{き|しやう}なこと!。
%
なあに
\ruby{彼樣}{あ|ん}な
\ruby{鉢}{はち}の
\ruby{一}{ひと}つや
\ruby{{\換字{半}}{\換字{分}}}{はん|ぶん}、
%
\ruby{麁怱}{そ|さう}で
\ruby{毀}{こは}したものを
\ruby{何}{なん}で
\ruby{妾}{わたし}が
\ruby{償}{つくの}へなんぞといふものですかネ。
』

\原本頁{}%
と
\ruby{云}{い}ひ
\ruby{出}{いだ}せば、
%
\ruby{老人}{らう|じん}は
\ruby{何}{なん}と
\ruby{聞}{き}き
\ruby{取}{と}つてか
\ruby{慌}{あわ}てゝ
\ruby{{\換字{遮}}}{さへぎ}りて、

\原本頁{}%
『ど、
%
\ruby{何樣}{ど|う}
\ruby{致}{いた}しまして
\ruby{貴女}{あな|た}、
%
\ruby{伯爵樣}{はく|しやく|さま}の
\ruby{御邸}{お|やしき}でさへ、
』

\原本頁{}%
と、
%
\ruby{身}{み}に
\ruby{入}{し}みて
\ruby{記}{おぼ}えたる% 送り仮名は原本通り「え」
\ruby{事}{こと}にても
\ruby{有}{あ}るなるべし、
%
\ruby{伯爵邸}{はく|しやく|てい}の
\ruby{定規}{さだ|め}を
\ruby{例}{れい}に
\ruby{引}{ひ}きかくるを、
%
\ruby{二}{に}の
\ruby{句}{く}を
\ruby{續}{つ}がせず、
%
お
\ruby{彤}{とう}は
\ruby{冷}{ひや}やかに
\ruby{笑}{わら}つたり。

\原本頁{}%
『まあ
\ruby{御聞}{お|き}きなさいよ。
%
\ruby{伯爵樣}{はく|しやく|さま}の
\ruby{御邸}{お|やしき}は
\ruby{伯爵樣}{はく|しやく|さま}の
\ruby{御邸}{お|やしき}で、
%
\ruby{妾}{わたし}の
\ruby{家}{うち}は
\ruby{妾}{わたし}の
\ruby{家}{うち}ですよ。
%
いゝ
\ruby{身{\換字{分}}}{み|ぶん}の
\ruby{方}{かた}の
\ruby{眞似}{ま|ね}を
\ruby{妾等}{わたし|ら}が
\ruby{仕}{し}ちやあ
\ruby{成}{な}りませんからネ。
%
\ruby{金屬}{か|ね}でゞも
\ruby{有}{あ}りやあ
\ruby{仕}{し}まいし、
%
\ruby{根}{ね}が
\ruby{磁器}{やき|もの}ですもの、
%
\ruby{破}{わ}れることも
\ruby{有}{あ}りましやう、
%
\ruby{其}{そ}の
\ruby{磁器}{やき|もの}が
\ruby{麁怱}{そ|さう}で
\ruby{破}{わ}れたのを
\ruby{何樣}{ど|う}まあ
\ruby{酷}{むご}く
\ruby{咎}{とが}め
\ruby{立}{だて}を
\ruby{仕}{し}ましやう!。
』

\原本頁{}%
『ハ、ハイ、ハイ、
%
ハイ。
』

\原本頁{}%
\ruby{激}{はげ}しく
\ruby{感}{かん}じたるならん、
%
\ruby{氣息}{い|き}の
\ruby{詰}{つ}まるやうに
\ruby{老人}{らう|じん}は
\ruby{急}{せ}き
\ruby{{\換字{込}}}{こ}みて
\ruby{挨拶}{あい|さつ}したり。

\原本頁{}%
『それも
\ruby{{\換字{平}}常}{ふだ|ん}の
\ruby{{\換字{勤}}}{つと}め
\ruby{方}{かた}でも
\ruby{惡}{わる}いといふのなら
\ruby{叱言}{こ|ごと}を
\ruby{云}{い}ふまいものでも
\ruby{有}{あ}りませんが、
%
\ruby{何}{なに}も
\ruby{彼}{か}も
\ruby{悉皆}{みん|な}
\ruby{好}{よ}く
\ruby{爲}{し}て
\ruby{吳}{く}れて
\ruby{居}{ゐ}る
\ruby{彼}{あ}の
お
\ruby{富}{とみ}の
\ruby{爲}{し}た
\ruby{{\換字{過}}失}{あや|まち}ですもの!。
』

\原本頁{}%
『ハ、ハ、ハイ、
%
ハイ。
』

\原本頁{}%
『
\ruby{少}{すこ}し
\ruby{位}{くらゐ}の
\ruby{品}{もの}を
\ruby{毀}{こは}したからつて
\ruby{何}{なに}を
\ruby{云}{い}ひましやう!。
%
\ruby{使}{つか}つてる
\ruby{中}{うち}に
\ruby{器物}{も|の}が
\ruby{毀}{こは}れるのは
\ruby{當然}{あたり|まへ}の
\ruby{事}{こと}で、
%
\ruby{其}{それ}を
\ruby{厭}{いと}やあ
\ruby{箱}{はこ}の
\ruby{中}{なか}へでも
\ruby{藏}{しま}つて
\ruby{置}{お}くより
\ruby{他有}{ほか|あ}りやあ
\ruby{仕無}{し|な}いと
\ruby{思}{おも}ひますよ。
%
\ruby{器物}{も|の}をいたはつて
\ruby{人}{ひと}をいたはらないやうな
\ruby{事}{こと}は
\ruby{妾}{わたし}あ
\ruby{大{\換字{嫌}}}{だい|きら}ひで、
%
あんな
\ruby{磁物}{やき|もの}を
\ruby{十個集}{と|を|よ}せたつて
\ruby{百集}{ひやく|よ}せたつて
お
\ruby{富}{とみ}が
\ruby{出來}{で|き}るのぢやあ
\ruby{無}{な}いんですもの、
%
\ruby{幾干}{いく|ら}
お
\ruby{富}{とみ}の
\ruby{方}{はう}を
\ruby{大切}{だい|じ}に
\ruby{思}{おも}つてるか
\ruby{知}{し}れや
\ruby{仕}{し}ません。
』

\原本頁{}%
『ハ、ハ、ハイ、
%
ハイ。
』

\原本頁{}%
『だから
\ruby{{\換字{過}}失}{あや|まち}は
\ruby{{\換字{過}}失}{あや|まち}で、
%
\ruby{一言}{ひと|こと}
\ruby{詫}{わび}を
\ruby{云}{い}はれりやあそれまでゞ
\ruby{濟}{す}まして
\ruby{仕舞}{し|ま}ふがネ、
%
それよりやあ
お
\ruby{富}{とみ}が
\ruby{大變}{たい|へん}に
\ruby{濟}{す}まない
\ruby{事}{こと}がありますよ。
』

\原本頁{}%
『ハハツ、ハイ、ハイ、
%
ヘイ。
』

\原本頁{}%
『
\ruby{其}{それ}あ
\ruby{默}{だま}つて
\ruby{駈}{か}け
\ruby{出}{だ}して
\ruby{仕舞}{し|ま}つて
\ruby{妾}{わたし}に
\ruby{不自由}{ふ|じ|ゆう}をさせたことです。
%
\ruby{何}{なに}も
\ruby{彼}{か}も
\ruby{彼女}{あ|れ}にさせて
\ruby{居}{ゐ}るのに、
%
\ruby{急}{きふ}に
\ruby{出}{で}て
\ruby{行}{い}かれちやあ
\ruby{何樣}{ど|ん}なに
\ruby{不自由}{ふ|じ|ゆう}に
\ruby{思}{おも}ふか
\ruby{知}{し}れません。
%
\ruby{丁度}{ちやう|ど}
\ruby{好}{い}い
\ruby{代}{かは}りが
\ruby{有}{あ}りは
\ruby{有}{あ}つたやうなものゝ、
%
\ruby{眞底}{しん|そこ}
\ruby{詫}{わ}びる
\ruby{氣}{き}があるなら、
%
\ruby{歸}{かへ}つて
\ruby{來}{き}てちやんと
\ruby{{\換字{勤}}}{つと}めつゞく
\ruby{方}{はう}が
\ruby{何程}{いく|ら}
\ruby{好}{い}いか
\ruby{知}{し}れやしません。
』

\原本頁{}%
『ハヽツ、ハイ、
%
ハイ。
%
で、
%
では
\ruby{麁怱}{そ|さう}を
\ruby{致}{いた}しましたのは
\ruby{御免}{お|ゆる}し
\ruby{下}{くだ}さいまして、そ、
%
そして
\ruby{今迄{\換字{通}}}{いま|ゝで|どほ}り
\ruby{御使}{お|つか}ひ
\ruby{下}{くだ}さいまするので。
』

\原本頁{}%
『
\ruby{使}{つか}つて
\ruby{{\換字{遣}}}{や}りますとも、
%
\ruby{使}{つか}つて
\ruby{{\換字{遣}}}{や}りますとも!。
%
あんな
\ruby{忠義}{ちう|ぎ}ものゝ
\ruby{氣立}{き|だて}の
\ruby{好}{い}い
\ruby{兒}{こ}が、
%
\ruby{磁器}{やき|もの}の
\ruby{三}{み}つや
\ruby{四}{よ}つ
\ruby{破}{こは}したつて
\ruby{何}{なん}の
\ruby{何}{なん}とも
\ruby{思}{おも}ふもんで。
』

\原本頁{}%
『ハアーツ、
%
\ruby{有}{あ}り
\ruby{{\換字{難}}}{がた}うございます、
%
\ruby{有}{あ}り
\ruby{{\換字{難}}}{がた}うございます。
%
\ruby{早{\換字{速}}}{さつ|そく}
\ruby{彼女}{あ|れ}に
\ruby{唯}{たゞ}
\ruby{今}{いま}の
\ruby{有}{あ}り
\ruby{{\換字{難}}}{がた}い
\ruby{御思召}{お|ぼし|めし}を
\ruby{申聞}{まをし|き}かせませんでは。
』

\原本頁{}%
\ruby{老人}{らう|じん}は
\ruby{嬉}{うれ}しさに
\ruby{泣}{な}かぬばかりの
\ruby{顏}{かほ}して、
%
\ruby{許}{ゆる}しをさへ
\ruby{得}{え}ば
\ruby{立}{た}たんとし
\ruby{{\換字{追}}立尻}{おつ|たて|じり}になつたり。

\原本頁{}%
『お
\ruby{富}{とみ}に
\ruby{話}{はな}すつて、
%
\ruby{{\換字{近}}處}{きん|じよ}へでも
\ruby{{\換字{連}}}{つ}れて
\ruby{來}{き}て
\ruby{居}{ゐ}るの?。
』

\原本頁{}%
『ハイ、
%
イエ。
%
\ruby{一緖}{いつ|しよ}に
\ruby{{\換字{連}}}{つ}れてはまゐりましたが、
%
\ruby{御裏口}{お|うら|ぐち}の
\ruby{{\換字{戸}}外}{そ|と}に
\ruby{立}{た}たせて
\ruby{置}{お}きましたので。
』

\原本頁{}%
『ホヽホヽ、
%
\ruby{愍然}{かは|いさう}に!。% 「愍然 か(は)いさう」
%
\ruby{何}{なん}だつて
\ruby{{\換字{戸}}外}{そ|と}になんか
\ruby{立}{た}たせて
\ruby{置}{お}くのだらう、
%
\ruby{早}{はや}く
\ruby{此方}{こつ|ち}へ
\ruby{{\換字{連}}}{つ}れておいでなさい。
』
