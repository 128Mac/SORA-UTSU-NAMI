\Entry{其三十二}

% メモ 校正終了 2024-03-18 2024-06-12
\原本頁{176-10}%
『
\ruby[g]{何樣}{ど う }も
\ruby{何}{なん}と
\ruby[||j>]{申}{まをし}
\ruby[||j>]{上}{ あげ}
% \ruby{申上}{まをし|あげ}
ましても
\ruby{相}{あひ}
\ruby{濟}{す}みません
\ruby{無調法}{ぶ|てふ|はふ}で。
%
ハイ。
%
\ruby{口}{くち}
ばかりで
\ruby{何}{なに}を
\ruby{申}{まを}し
\ruby{上}{あ}げましても、
%
\ruby{實}{じつ}
\ruby{以}{もつ}て
\ruby{相}{あひ}
\ruby{濟}{す}みません
\ruby{譯}{わけ}で、
%
ハイ
\改行% 校正作業の簡略化のため
。
%
\原本頁{177-2}\改行%
お
\ruby{羞}{はづか}しいことを
\ruby{申}{まを}し
\ruby{上}{あ}げませんければ
\ruby{理}{り}が
\ruby{聞}{きこ}えませぬが、
%
\ruby{實}{じつ}は
\原本頁{177-3}\改行%
\ruby[g]{段々}{だん〴〵}と
\ruby{不}{ふ}
\ruby[<j>]{幸}{しあはせ}%「幸」ここは「は」
は
\ruby{續}{つゞ}きますし、
%
\ruby[<j>]{私}{わたくし}は
\ruby[||j>]{病}{びやう}
\ruby[||j>]{身}{ しん}で
% \ruby{病身}{びやう|しん}で
\ruby[||j>]{商}{しやう}
\ruby[||j>]{法}{ はふ}は
% \ruby{商法}{しやう|はふ}は
\ruby{止}{や}めて
\ruby{居}{を}りますし
\改行% 校正作業の簡略化のため
、
%
\原本頁{177-4}\改行%
\ruby{少}{すこ}し
ばかりの
\ruby[g]{地{\換字{所}}}{ぢ しよ}
\ruby[g]{家作}{か さく}で
\ruby[g]{細々}{ほそ〴〵}と
\ruby{{\換字{遣}}}{や}つて
\ruby{居}{を}ります
\ruby{中}{なか}を、
%
\ruby{不孝者}{ふ|かう|もの}めの
\ruby{伜}{せがれ}に
\ruby[g]{大無}{だいな }し
に
されまして、
%
まことに
はや
\ruby[g]{何樣}{ど う }も
\ruby[g]{斯樣}{か う }も
ならぬ
やうに
なつて
\ruby{居}{を}りまするので、
%
ただ% ルビ調整(原本通り)非踊り字表記
もう
\ruby{明}{あけ}
\ruby{暮}{くれ}、
%
\ruby{伜}{せがれ}めの
\ruby{碌}{ろく}で
\ruby{無}{な}しの
\ruby[g]{料簡}{れうけん}の
\ruby{直}{なほ}ります
やうにと、
%
\ruby[g]{信心}{しん〴〵}を
\ruby{致}{いた}すのを
\ruby[g]{今日}{こんにち}の
\ruby{{\換字{勤}}}{つとめ}に
\ruby{致}{いた}して
\原本頁{177-8}\改行%
\ruby{居}{を}るやうな
\ruby{意氣地}{い|く|ぢ}
の
\ruby{無}{な}い
\ruby[g]{次第}{し だい}
で
ございますから、
%
\ruby[g]{何共}{なんとも}
\ruby{恐}{おそ}れ
\ruby{入}{い}りまする
\ruby{身{\換字{勝}}手}{み|がつ|て}な
\ruby[||j>]{申}{まをし}
\ruby[||j>]{{\換字{分}}}{ ぶん}
% \ruby{申{\換字{分}}}{まをし|ぶん}
では
ございますが、
%
\ruby{今}{いま}が
\ruby{今}{いま}
\ruby[g]{何樣}{ど う }にか
\ruby{致}{いた}さうと
\ruby{致}{いた}しますれば、
%
\makeatletter
\@ifundefined{デバッグ@ビルド}{%
  \ruby[<j|]{私}{わたくし}
}{%
  \ruby[<j>]{私}{わたくし}
}%
\makeatother
\ruby[g]{一人}{ひとり}% ルビ調整(長いルビ対策)直前の長いルビ対策% ルビ調整(原本通り)非グループルビ
のところへ
\ruby[g]{夫{\換字{婦}}}{ふうふ }
\ruby[g]{掛向}{かけむか}ひの
\ruby{人}{ひと}を
\ruby{置}{お}きまして、
%
その
\ruby[g]{貸間}{かしま }の
\ruby{料}{れう}で
\ruby{食}{た}べて
\ruby{居}{を}りまする
\ruby[g]{住家}{すまゐ }をでも、
%
\ruby[g]{何樣}{ど う }か
\ruby{致}{いた}して
\ruby[g]{算段}{さんだん}
\ruby{致}{いた}すより
\ruby{他}{ほか}
は
ございませんので、
%
それでは
\ruby[g]{何樣}{ど う }も
\ruby[g]{後々}{あと〳〵}
の
ところが
‥‥
』

\原本頁{178-3}%
\ruby[g]{{\換字{貧}}相}{ひんさう}な
\ruby{顏}{かほ}を
いよ〳〵
\ruby[g]{{\換字{貧}}相}{ひんさう}に
\ruby{仕}{し}て
\ruby[g]{困{\換字{難}}}{こんなん}の
\ruby{趣}{おもむ}きを
\ruby{{\換字{述}}}{の}べ
\ruby[g]{哀愍}{あはれみ}を
\ruby{乞}{こ}はんとする、
%
\ruby{其}{そ}の
\ruby[g]{言語}{ものいひ}は
\ruby{人}{ひと}の
\ruby[||j>]{同}{どう}
\ruby[||j>]{{\換字{情}}}{じやう}
% \ruby{同{\換字{情}}}{どう|じやう}
を
\ruby{惹}{ひ}くに
\ruby{足}{た}るほどの
\ruby[g]{氣合}{き あひ}さへ
\ruby{乏}{とぼ}しけれど、
%
\ruby{其}{そ}の
くど〳〵しく
\ruby{惡}{わる}
\ruby[g]{叮嚀}{ていねい}なるに
\ruby[g]{愚直}{ぐちよく}さは
\ruby[<j>]{盡}{こと〴〵}く
\ruby{知}{し}られた
\改行% 校正作業の簡略化のため
り。

\原本頁{178-7}%
お
\ruby{彤}{とう}は
\ruby[g]{最早}{も はや}
\ruby{聞}{き}き
\ruby{居}{ゐ}るに
\ruby{堪}{た}へかねてや、
%
\ruby[g]{言葉}{ことば }の
\ruby{澱}{よど}みに
\ruby{付}{つ}け
\ruby{入}{い}りて
\ruby{{\換字{又}}}{また}
\ruby{靜}{しづか}に
\ruby{{\換字{又}}}{また}
\ruby[g]{爽快}{さわやか}に、

\原本頁{178-9}%
『
まあ
\ruby{其}{それ}は
\ruby[g]{大層}{たいそう}に
\ruby[g]{心配}{しんぱい}を
お
\ruby{爲}{し}だつたねえ。
%
お
\ruby{{\換字{前}}}{まへ}さんは
\ruby[g]{當世}{たうせい}にあ
\原本頁{178-10}\改行%
\ruby{珍}{めづ}らしい
\ruby[g]{律義}{りちぎ }な
\ruby[g]{氣性}{きしやう}なこと!。
%
なあに
\ruby[g]{彼樣}{あ ん }な
\ruby{鉢}{はち}の
\ruby{一}{ひと}つや
\ruby[g]{{\換字{半}}{\換字{分}}}{はんぶん}、
%
\ruby[g]{麁怱}{そ さう}で
\ruby{毀}{こは}したものを
\ruby{何}{なん}で
\ruby{妾}{わたし}が
\ruby{償}{つくの}へ
なんぞ
と
いふ
ものですかネ。
』

\原本頁{179-1}%
と
\ruby{云}{い}ひ
\ruby{出}{いだ}せば、
%
\ruby[g]{老人}{らうじん}は
\ruby{何}{なん}と
\ruby{聞}{き}き
\ruby{取}{と}つてか
\ruby{慌}{あわ}てゝ
\ruby{{\換字{遮}}}{さへぎ}りて、

\原本頁{179-2}%
『
ど、
%
\ruby[g]{何樣}{ど う }
\ruby{致}{いた}しまして
\ruby[|g|]{貴女}{あなた}、
%
\ruby[||j>]{伯}{はく}
\ruby[||j>]{爵}{しやく}
% \ruby{伯爵}{はく|しやく}
\ruby[||j>]{樣}{ さま}の
\ruby{御}{お}
\ruby{邸}{やしき}でさへ、
』

\原本頁{179-3}%
と、
%
\ruby{身}{み}に
\ruby{入}{し}みて
\ruby{記}{おぼ}えたる% 送り仮名は原本通り「え」
\ruby{事}{こと}にても
\ruby{有}{あ}るなるべし、
%
\ruby[||j>]{伯}{はく}
\ruby[||j>]{爵}{しやく}
% \ruby{伯爵}{はく|しやく}
\ruby[||j>]{邸}{ てい}の
\ruby[|g|]{定規}{さだめ}を
\ruby{例}{れい}に
\ruby{引}{ひ}きかくるを、
%
\ruby{二}{に}の
\ruby{句}{く}を
\ruby{續}{つ}がせず、
%
お
\ruby{彤}{とう}は
\ruby{冷}{ひや}やかに
\ruby{笑}{わら}つたり。

\原本頁{179-6}%
『
まあ
\ruby[g]{御聞}{お き }きなさいよ。
%
\ruby[||j>]{伯}{はく}
\ruby[||j>]{爵}{しやく}
% \ruby{伯爵}{はく|しやく}
\ruby[||j>]{樣}{ さま}の
\ruby[g]{御邸}{おやしき}は
\ruby[||j>]{伯}{はく}
\ruby[||j>]{爵}{しやく}
% \ruby{伯爵}{はく|しやく}
\ruby[||j>]{樣}{ さま}の
\ruby[g]{御邸}{おやしき}で、
%
\ruby{妾}{わたし}の
\ruby{家}{うち}は
\ruby{妾}{わたし}の
\ruby{家}{うち}ですよ。
%
いゝ
\ruby[g]{身{\換字{分}}}{み ぶん}の
\ruby{方}{かた}の
\ruby[g]{眞似}{ま ね }を
\ruby[g]{妾等}{わたしら}が
\ruby{仕}{し}ちやあ
\ruby{成}{な}りませんからネ。
%
\ruby[g]{金屬}{か ね }
でゞも
\ruby{有}{あ}りやあ
\ruby{仕}{し}まいし、
%
\ruby{根}{ね}が
\ruby[g]{磁器}{やきもの}
ですもの、
%
\ruby{破}{わ}れることも
\ruby{有}{あ}りましやう、
%
\ruby{其}{そ}の
\ruby[g]{磁器}{やきもの}が
\ruby[g]{麁怱}{そ さう}で
\ruby{破}{わ}れたのを
\改行% 校正作業の簡略化のため
、
\原本頁{179-10}\改行%
\ruby[g]{何樣}{ど う }
まあ
\ruby{酷}{むご}く
\ruby{咎}{とが}め
\ruby{立}{だて}を
\ruby{仕}{し}ましやう!。
』

\原本頁{179-11}%
『
ハ、
%
ハイ、
%
ハイ、
%
ハイ。
』

\原本頁{180-1}%
\ruby{激}{はげ}しく
\ruby{{\換字{感}}}{かん}じたる
ならん、
%
\ruby[g]{氣息}{い き }の
\ruby{詰}{つ}まるやうに
\ruby[g]{老人}{らうじん}は
\ruby{急}{せ}き
\ruby{{\換字{込}}}{こ}みて
\ruby[g]{挨拶}{あいさつ}したり。

\原本頁{180-3}%
『
それも
\ruby[|g|]{{\換字{平}}常}{ふだん}の
\ruby{{\換字{勤}}}{つと}め
\ruby{方}{かた}でも
\ruby{惡}{わる}い
といふのなら
\ruby[g]{叱言}{こ ごと}を
\ruby{云}{い}ふまいものでも
\ruby{有}{あ}りませんが、
%
\ruby{何}{なに}も
\ruby{彼}{か}も
\ruby[|g|]{悉皆}{みんな}
\ruby{好}{よ}く
\ruby{爲}{し}て
\ruby{吳}{く}れて
\ruby{居}{ゐ}る
\ruby{彼}{あ}の
お
\ruby{富}{とみ}の
\ruby{爲}{し}た
\ruby[g]{{\換字{過}}失}{あやまち}
ですもの!。
』

\原本頁{180-6}%
『
ハ、
%
ハ、
%
ハイ、
%
ハイ。
』

\原本頁{180-7}%
『
\ruby{少}{すこ}し
\ruby{位}{くらゐ}の
\ruby{品}{もの}を
\ruby{毀}{こは}した
からつて
\ruby{何}{なに}を
\ruby{云}{い}ひましやう!。
%
\ruby{使}{つか}つてる
\ruby{中}{うち}に
\ruby[g]{器物}{も の }が
\ruby{毀}{こは}れるのは
\ruby[||j>]{當}{あたり}
\ruby[||j>]{然}{ まへ}の
% \ruby{當然}{あたり|まへ}の
\ruby{事}{こと}で、
%
\ruby{其}{それ}を
\ruby{厭}{いと}やあ
\ruby{箱}{はこ}の
\ruby{中}{なか}へでも
\ruby{藏}{しま}つて
\ruby{置}{お}くより
\ruby{他}{ほか}
\ruby{有}{あ}りやあ
\ruby[g]{仕無}{し な }いと
\ruby{思}{おも}ひますよ。
%
\ruby[g]{器物}{も の }を
いたはつて
\ruby{人}{ひと}を
いたはらない
やうな
\ruby{事}{こと}は
\ruby{妾}{わたし}あ
\ruby[g]{大{\換字{嫌}}}{だいきら}ひで、
%
あんな
\ruby[g]{磁物}{やきもの}を
\ruby[g]{十個}{と を }
\ruby{集}{よ}せたつて
\ruby{百}{ひやく}
\ruby{集}{ よ}せたつて
お
\ruby{富}{とみ}が
\ruby[g]{出來}{で き }るのぢやあ
\ruby{無}{な}いんですもの、
%
\ruby[|g|]{幾干}{いくら}
お
\ruby{富}{とみ}の
\ruby{方}{はう}を
\ruby[g]{大切}{だいじ }に
\ruby{思}{おも}つてるか
\ruby{知}{し}れや
\ruby{仕}{し}ません。
』

\原本頁{181-2}%
『
ハ、
%
ハ、
%
ハイ、
%
ハイ。
』

\原本頁{181-3}%
『
だから
\ruby[g]{{\換字{過}}失}{あやまち}は
\ruby[g]{{\換字{過}}失}{あやまち}で、
%
\ruby[g]{一言}{ひとこと}
\ruby{詫}{わび}を
\ruby{云}{い}はれりやあ
それまでゞ
\ruby{濟}{す}まして
\ruby[g]{仕舞}{し ま }ふがネ、
%
それよりやあ
お
\ruby{富}{とみ}が
\ruby[g]{大變}{たいへん}に
\ruby{濟}{す}まない
\ruby{事}{こと}がありますよ。
』

\原本頁{181-6}%
『
ハハツ、
%
ハイ、
%
ハイ、
%
ヘイ。
』

\原本頁{181-7}%
『
\ruby{其}{それ}あ
\ruby{默}{だま}つて
\ruby{駈}{か}け
\ruby{出}{だ}して
\ruby[g]{仕舞}{し ま }つて
\ruby{妾}{わたし}に
\ruby{不自由}{ふ|じ|いう}を% ルビ調整(原本通り)「ふじ(い)う」
させたことです
\改行% 校正作業の簡略化のため
。
%
\原本頁{181-7}\改行%
\ruby{何}{なに}も
\ruby{彼}{か}も
\ruby[g]{彼女}{あ れ }に
させて
\ruby{居}{ゐ}るのに、
%
\ruby{急}{きふ}に
\ruby{出}{で}て
\ruby{行}{い}かれちやあ
\ruby[g]{何樣}{ど ん }なに
\ruby{不自由}{ふ|じ|いう}に% ルビ調整(原本通り)「ふじ(い)う」
\ruby{思}{おも}ふか
\ruby{知}{し}れません。
%
\ruby[g]{丁度}{ちやうど}
\ruby{好}{い}い
\ruby{代}{かは}りが
\ruby{有}{あ}りは
\ruby{有}{あ}つた
やうな
ものゝ、
%
\ruby[g]{眞底}{しんそこ}
\ruby{詫}{わ}びる
\ruby{氣}{き}
が
あるなら、
%
\ruby{歸}{かへ}つて
\ruby{來}{き}てちやんと
\ruby{{\換字{勤}}}{つと}め
つづく% ルビ調整(原本通り)非踊り字表記
\ruby{方}{はう}が
\ruby[|g|]{何程}{いくら}
\ruby{好}{い}いか
\ruby{知}{し}れや
しません。
』

\原本頁{182-1}%
『
ハヽツ、
%
ハイ、
%
ハイ。
%
で、
%
では
\ruby[g]{麁怱}{そ さう}を
\ruby{致}{いた}しましたのは
\ruby[g]{御免}{お ゆる}し
\ruby{下}{くだ}さいまして、
%
そ、
%
そして
\ruby[g]{今迄}{いまゝで}
\ruby{{\換字{通}}}{どほ}り
\ruby[g]{御使}{お つか}ひ
\ruby{下}{くだ}さいまするので。
』

\原本頁{182-3}%
『
\ruby{使}{つか}つて
\ruby{{\換字{遣}}}{や}りますとも、
%
\ruby{使}{つか}つて
\ruby{{\換字{遣}}}{や}りますとも!。
%
あんな
\ruby[g]{忠義}{ちうぎ }ものゝ% 原本通り(ちう)(国会図書館 コマ番号 95/146 p142 l3)
\ruby[g]{氣立}{き だて}の
\ruby{好}{い}い
\ruby{兒}{こ}が、
%
\ruby[g]{磁器}{やきもの}の
\ruby{三}{み}つや
\ruby{四}{よ}つ
\ruby{破}{こは}したつて
\ruby{何}{なん}の
\ruby{何}{なん}とも
\ruby{思}{おも}ふもんで。
』

\原本頁{182-6}%
『
ハアーツ、
%
\ruby{有}{あ}り
\ruby{{\換字{難}}}{がた}う
ございます、
%
\ruby{有}{あ}り
\ruby{{\換字{難}}}{がた}う
ございます。
%
\ruby[g]{早{\換字{速}}}{さつそく}
\ruby[g]{彼女}{あ れ }に
\ruby{唯}{たゞ}
\ruby{今}{いま}の
\ruby{有}{あ}り
\ruby{{\換字{難}}}{がた}い
\ruby{御思召}{お|ぼし|めし}を
\ruby[g]{申聞}{まをしき}かせませんでは。
』

\原本頁{182-8}%
\ruby[g]{老人}{らうじん}は
\ruby{嬉}{うれ}しさに
\ruby{泣}{な}かぬ
ばかりの
\ruby{顏}{かほ}して、
%
\ruby{許}{ゆる}しをさへ
\ruby{得}{え}ば
\ruby{立}{た}たんとして
\ruby{{\換字{追}}立尻}{おつ|たて|じり}
に
なつたり。

\原本頁{182-10}%
『
お
\ruby{富}{とみ}に
\ruby{話}{はな}すつて、
%
\ruby[g]{{\換字{近}}處}{きんじよ}へでも
\ruby{{\換字{連}}}{つ}れて
\ruby{來}{き}て
\ruby{居}{ゐ}るの?。
』

\原本頁{182-11}%
『
ハイ、
%
イエ。
%
\ruby[g]{一緖}{いつしよ}に
\ruby{{\換字{連}}}{つ}れては
まゐりましたが、
%
\ruby{御裏口}{お|うら|ぐち}の
\ruby[g]{{\換字{戸}}外}{そ と }に
\ruby{立}{た}たせて
\ruby{置}{お}きましたので。
』

\原本頁{183-2}%
『
ホヽホヽ、
%
\ruby[||j>]{愍}{かは}% 「愍然 か(は)いさう」
\ruby[||j>]{然}{いさう}に!。
% \ruby{愍然}{かは|いさう}に!。% 「愍然 か(は)いさう」
%
\ruby{何}{なん}だつて
\ruby[g]{{\換字{戸}}外}{そ と }
に
なんか
\ruby{立}{た}たせて
\ruby{置}{お}くのだらう、
%
\ruby{早}{はや}く
\ruby[|g|]{此方}{こつち}へ
\ruby{{\換字{連}}}{つ}れて
おいでなさい。
』
