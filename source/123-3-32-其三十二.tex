\Entry{其三十二}

『
\ruby{何樣}{ど|う}も
\ruby{何}{なん}と
\ruby[g]{申上}{まをしあげ}ましても
\ruby[g]{相濟}{あひす}みません
\ruby[g]{無調法}{ぶてふはふ}で。
ハイ。
\ruby{口}{くち}ばかりで
\ruby{何}{なに}を
\ruby{申}{まを}し
\ruby{上}{あ}げましても、
\ruby{實以}{じつ|もつ}て
\ruby{相濟}{あひ|す}みません
\ruby{譯}{わけ}で、ハイ。
お
\ruby{羞}{はづか}しいことを
\ruby{申}{まを}し
\ruby{上}{あ}げませんければ
\ruby{理}{り}が
\ruby{聞}{きこ}えませぬが、
\ruby{實}{じつ}は
\ruby{段々}{だん|〳〵}と
\ruby{不幸}{ふし|あわせ}は
\ruby{續}{つゞ}きますし、
\ruby{私}{わたくし}は
\ruby[g]{病身}{びやうしん}で
\ruby{商法}{しやう|はふ}は
\ruby{止}{や}めて
\ruby{居}{を}りますし、
\ruby{少}{すこ}しばかりの
\ruby[g]{地所家作}{ぢしよかさく}で
\ruby{細々}{ほそ|〴〵}と
\ruby{{\換字{遣}}}{や}つて
\ruby{居}{を}ります
\ruby{中}{なか}を、
\ruby[g]{不孝者}{ふかうもの}めの
\ruby{伜}{せがれ}に
\ruby[g]{大無}{だいな}しにされまして、まことにはや
\ruby{何樣}{ど|う}も
\ruby{斯樣}{か|う}もならないやうになつて
\ruby{居}{を}りまするので、ただもう
\ruby{明暮}{あけ|くれ}、
\ruby{伜}{せがれ}めの
\ruby{碌}{ろく}で
\ruby{無}{な}しの
\ruby{料簡}{れう|けん}の
\ruby{直}{なほ}りますやうにと、
\ruby{信心}{しん|〴〵}を
\ruby{致}{いた}すのを
\ruby{今日}{こん|にち}の
\ruby{勤}{つとめ}に
\ruby{致}{いた}して
\ruby{居}{を}るやうな
\ruby{意氣地}{い|く|ぢ}
の
\ruby{無}{な}い
\ruby{次第}{し|だい}でございますから、
\ruby[g]{何共恐}{なんともおそ}れ
\ruby{入}{い}りまする
\ruby[g]{身{\換字{勝}}手}{みがつて}な
\ruby[g]{申分}{まをしぶん}ではございますが、
\ruby{今}{いま}が
\ruby[g]{今何樣}{いまどう}にか
\ruby{致}{いた}さうと
\ruby{致}{いた}しますれば、
\ruby[g]{私一人}{わたくしひとり}のところへ
\ruby[g]{夫婦掛向}{ふうふかけむか}ひの
\ruby{人}{ひと}を
\ruby{置}{お}きまして、その
\ruby[g]{貸間}{かしま}の
\ruby{料}{れう}で
\ruby{食}{た}べて
\ruby{居}{を}りまする
\ruby[g]{住家}{すまゐ}をでも、
\ruby{何樣}{ど|う}か
\ruby{致}{いた}して
\ruby{算段致}{さん|だん|いた}すより
\ruby{他}{ほか}はございませんので、それでは
\ruby{何樣}{ど|う}も
\ruby{後々}{あと|〳〵}のところが……』

\ruby{貧相}{ひん|さう}な
\ruby{顏}{かほ}をいよ〳〵
\ruby{貧相}{ひん|さう}に
\ruby{仕}{し}て
\ruby{困難}{こん|なん}の
\ruby{趣}{おもむ}きを
\ruby{{\換字{述}}}{の}べ
\ruby{哀愍}{あは|れみ}を
\ruby{乞}{こ}はんとする、
\ruby{其}{そ}の
\ruby{言語}{もの|いひ}は
\ruby{人}{ひと}の
\ruby[g]{同{\換字{情}}}{どうじやう}を
\ruby{惹}{ひ}くに
\ruby{足}{た}るほどの
\ruby{氣合}{き|あひ}さへ
\ruby{乏}{とぼ}しけれど、
\ruby{其}{そ}のくど〳〵しく
\ruby{惡叮嚀}{わる|てい|ねい}なるに
\ruby{愚直}{ぐち|よく}さは
\ruby{盡}{こと〴〵}く
\ruby{知}{し}られたり。

お
\ruby{彤}{とう}は
\ruby[g]{最早聞}{もはやき}き
\ruby{居}{ゐ}るに
\ruby{堪}{た}へかねてや、
\ruby[g]{言葉}{ことば}の
\ruby{澱}{よど}みに
\ruby{付}{つ}け
\ruby{入}{い}りて
\ruby[g]{{\換字{又}}靜}{またしづか}に
\ruby[g]{{\換字{又}}爽快}{またさわやか}に、

『まあ
\ruby{其}{それ}は
\ruby{大層}{たい|そう}に
\ruby{心配}{しん|ぱい}をお
\ruby{爲}{し}だつたねえ。
お
\ruby{前}{まへ}さんは
\ruby{當世}{たう|せい}にあ
\ruby{珍}{めづ}らしい
\ruby[g]{律義}{りちぎ}な
\ruby{氣性}{きし|やう}なこと!。
なあに
\ruby{彼樣}{あ|ん}な
\ruby{鉢}{はち}の
\ruby{一}{ひと}つや
\ruby{半分}{はん|ぶん}、
\ruby{麁忽}{そ|さう}で
\ruby{毀}{こは}したものを
\ruby{何}{なん}で
\ruby{妾}{わたし}が
\ruby{償}{つくの}へなんぞといふものですかネ。
』

と
\ruby{云}{い}ひ
\ruby{出}{いだ}せば、
\ruby{老人}{らう|じん}は
\ruby{何}{なん}と
\ruby{聞}{き}き
\ruby{取}{と}つてか
\ruby{慌}{あわ}てゝ
\ruby{遮}{さへぎ}りて、

『ど、
\ruby{何樣致}{ど|う|いた}しまして
\ruby[g]{貴女}{あなた}、
\ruby[g]{伯爵樣}{はくしやくさま}の
\ruby{御邸}{お|やしき}でさへ、』

と、
\ruby{身}{み}に
\ruby{入}{し}みて
\ruby{記}{おぼ}えたる
\ruby{事}{こと}にても
\ruby{有}{あ}るなるべし、
\ruby[g]{伯爵邸}{はくしやくてい}の
\ruby[g]{定規}{さだめ}を
\ruby{例}{れい}に
\ruby{引}{ひ}きかくるを、
\ruby{二}{に}の
\ruby{句}{く}を
\ruby{續}{つ}がせず、お
\ruby{彤}{とう}は
\ruby{冷}{ひや}やかに
\ruby{笑}{わら}つたり。

『まあ
\ruby{御聞}{お|き}きなさいよ。
\ruby[g]{伯爵樣}{はくしやくさま}の
\ruby{御邸}{おや|しき}は
\ruby[g]{伯爵樣}{はくしやくさま}の
\ruby{御邸}{おや|しき}で、
\ruby{妾}{わたし}の
\ruby{家}{うち}は
\ruby{妾}{わたし}の
\ruby{家}{うち}ですよ。
いゝ
\ruby{身分}{み|ぶん}の
\ruby{方}{かた}の
\ruby{眞似}{ま|ね}を
\ruby{妾等}{わた|しら}が
\ruby{仕}{し}ちやあ
\ruby{成}{な}りませんからネ。
\ruby{金屬}{か|ね}でゞも
\ruby{有}{あ}りやあ
\ruby{仕}{し}まいし、
\ruby{根}{ね}が
\ruby{磁器}{やき|もの}ですもの、
\ruby{破}{わ}れることも
\ruby{有}{あ}りましやう、
\ruby{其}{そ}の
\ruby{磁器}{やき|もの}が
\ruby{麁忽}{そ|さう}で
\ruby{破}{わ}れたのを
\ruby{何樣}{ど|う}まあ
\ruby{酷}{むご}く
\ruby{咎}{とが}め
\ruby{立}{だて}を
\ruby{仕}{し}ましやう!。
』

『ハ、ハイ、ハイ、ハイ。
』

\ruby{激}{はげ}しく
\ruby{感}{かん}じたるならん、
\ruby{氣息}{い|き}の
\ruby{詰}{つ}まるやうに
\ruby{老人}{らう|じん}は
\ruby{急}{せ}き
\ruby{込}{こ}みて
\ruby{挨拶}{あい|さつ}したり。

『それも
\ruby[g]{{\換字{平}}常}{ふだん}の
\ruby{勤}{つと}め
\ruby{方}{かた}でも
\ruby{惡}{わる}いといふのなら
\ruby[g]{叱言}{こごと}を
\ruby{云}{い}ふまいものでも
\ruby{有}{あ}りませんが、
\ruby{何}{なに}も
\ruby{彼}{か}も
\ruby[g]{悉皆好}{みんなよ}く
\ruby{爲}{し}て
\ruby{{\換字{呉}}}{く}れて
\ruby{居}{ゐ}る
\ruby{彼}{あ}のお
\ruby{富}{とみ}の
\ruby{爲}{し}た
\ruby{{\換字{過}}失}{あや|まち}ですもの!。
』

『ハ、ハ、ハイ、ハイ。
』

『
\ruby{少}{すこ}し
\ruby{位}{くらゐ}の
\ruby{品}{もの}を
\ruby{毀}{こは}したからつて
\ruby{何}{なに}を
\ruby{云}{い}ひましやう!。
\ruby{使}{つか}つてる
\ruby{中}{うち}に
\ruby{器物}{も|の}が
\ruby{毀}{こは}れるのは
\ruby[g]{當然}{あたりまへ}の
\ruby{事}{こと}で、
\ruby{其}{それ}を
\ruby{厭}{いと}やあ
\ruby{箱}{はこ}の
\ruby{中}{なか}へでも
\ruby{藏}{しま}つて
\ruby{置}{お}くより
\ruby[g]{他有}{ほかあ}りやあ
\ruby{仕無}{し|な}いと
\ruby{思}{おも}ひますよ。
\ruby{器物}{も|の}をいたはつて
\ruby{人}{ひと}をいたはらないやうな
\ruby{事}{こと}は
\ruby{妾}{わたし}あ
\ruby{大{\換字{嫌}}}{だい|きら}ひで、あんな
\ruby{磁物}{やき|もの}を
\ruby{十個集}{と|を|よ}せたつて
\ruby[g]{百集}{ひやくよ}せたつてお
\ruby{富}{とみ}が
\ruby{出來}{で|き}るのぢやあ
\ruby{無}{な}いんですもの、
\ruby[g]{幾干}{いくら}お
\ruby{富}{とみ}の
\ruby{方}{はう}を
\ruby{大切}{だい|じ}に
\ruby{思}{おも}つてるか
\ruby{知}{し}れや
\ruby{仕}{し}ません。
』

『ハ、ハ、ハイ、ハイ。
』

『だから
\ruby{{\換字{過}}失}{あや|まち}は
\ruby{{\換字{過}}失}{あや|まち}で、
\ruby{一言詫}{ひと|こと|わび}を
\ruby{云}{い}はれりやあそれまでゞ
\ruby{濟}{す}まして
\ruby{仕舞}{し|ま}ふがネ、それよりやあお
\ruby{富}{とみ}が
\ruby{大變}{たい|へん}に
\ruby{濟}{す}まない
\ruby{事}{こと}がありますよ。
』

『ハハツ、ハイ、ハイ、ヘイ。
』

『
\ruby{其}{それ}あ
\ruby{默}{だま}つて
\ruby{駈}{か}け
\ruby{出}{だ}して
\ruby{仕舞}{し|ま}つて
\ruby{妾}{わたし}に
\ruby{不自由}{ふ|じ|ゆう}をさせたことです。
\ruby{何}{なに}も
\ruby{彼}{か}も
\ruby{彼女}{あ|れ}にさせて
\ruby{居}{ゐ}るのに、
\ruby{急}{きふ}に
\ruby{出}{で}て
\ruby{行}{い}かれちやあ
\ruby{何樣}{ど|ん}なに
\ruby{不自由}{ふ|じ|ゆう}に
\ruby{思}{おも}ふか
\ruby{知}{し}れません。
\ruby[g]{丁度好}{ちやうどい}い
\ruby{代}{かは}りが
\ruby{有}{あ}りは
\ruby{有}{あ}つたやうなものゝ、
\ruby[g]{眞底詫}{しんそこわ}びる
\ruby{氣}{き}があるなら、
\ruby{歸}{かへ}つて
\ruby{來}{き}てちやんと
\ruby{勤}{つと}めつづく
\ruby{方}{はう}が
\ruby[g]{何程好}{いくらい}いか
\ruby{知}{し}れやしません。
』

『ハヽツ、ハイ、ハイ。
で、では
\ruby{麁忽}{そ|さう}を
\ruby{致}{いた}しましたのは
\ruby{御免}{お|ゆる}し
\ruby{下}{くだ}さいまして、そ、そして
\ruby[g]{今迄{\換字{通}}}{いままでどほ}り
\ruby{御使}{お|つか}ひ
\ruby{下}{くだ}さいまするので。
』

『
\ruby{使}{つか}つて
\ruby{{\換字{遣}}}{や}りますとも、
\ruby{使}{つか}つて
\ruby{{\換字{遣}}}{や}りますとも!。
あんな
\ruby[g]{忠義}{ちうぎ}ものゝ
\ruby{氣立}{き|だて}の
\ruby{好}{よ}い
\ruby{兒}{こ}が、
\ruby{磁器}{やき|もの}の
\ruby{三}{み}つや
\ruby{四}{よ}つ
\ruby{破}{こは}したつて
\ruby{何}{なん}の
\ruby{何}{なん}とも
\ruby{思}{おも}ふもんで。
』

『ハアーツ、
\ruby{有}{あ}り
\ruby{難}{がた}うございます、
\ruby{有}{あ}り
\ruby{難}{がた}うございます。
\ruby[g]{早{\換字{速}}彼女}{さつそくあれ}に
\ruby{唯今}{ただ|いま}の
\ruby{有}{あ}り
\ruby{難}{がた}い
\ruby[g]{御思召}{おぼしめし}を
\ruby{申聞}{まを|しき}かせませんでは。
』

\ruby{老人}{らう|じん}は
\ruby{嬉}{うれ}しさに
\ruby{泣}{な}かぬばかりの
\ruby{顏}{かほ}して、
\ruby{許}{ゆる}しをさへ
\ruby{得}{え}ば
\ruby{立}{た}たんとし
\ruby[g]{追立尻}{おつたてじり}になつたり。

『お
\ruby{富}{とみ}に
\ruby{話}{はな}すつて、
\ruby{近處}{きん|じよ}へでも
\ruby{{\換字{連}}}{つ}れて
\ruby{來}{き}て
\ruby{居}{ゐ}るの?。
』

『ハイ、イエ。
\ruby{一緒}{いつ|しよ}に
\ruby{{\換字{連}}}{つ}れてはまゐりましたが、
\ruby[g]{御裏口}{おうらぐち}の
\ruby{戸外}{そ|と}に
\ruby{立}{た}たせて
\ruby{置}{お}きましたので。
』

『ホヽホヽ、
\ruby{愍然}{かはい|さう}に!。
\ruby{何}{なん}だつて
\ruby{戸外}{そ|と}になんか
\ruby{立}{た}たせて
\ruby{置}{お}くのだらう、
\ruby{早}{はや}く
\ruby[g]{此方}{こつち}へ
\ruby{{\換字{連}}}{つ}れておいでなさい。
』

