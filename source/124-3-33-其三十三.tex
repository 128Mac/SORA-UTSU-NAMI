\Entry{其三十三}

% メモ 校正終了 2024-03-18
\原本頁{183-5}%
『
\ruby{妾}{わたし}は
\ruby[||j>]{東}{とう}
\ruby[||j>]{京}{きやう}にやあ
% \ruby{東京}{とう|きやう}にやあ
\ruby{今時}{いま|どき}
\ruby{彼樣}{あ|ゝ}いふ
\ruby{人}{ひと}は
\ruby{無}{な}からう
と
ばつかり
\ruby{思}{おも}つて
\ruby{居}{ゐ}ましたが、
%
たまには
\ruby{矢張}{やつ|ぱ}り% 原本通り非グループルビ
\ruby{彼樣}{あ|ん}な
\ruby[||j>]{正}{しやう}
\ruby[||j>]{直}{ ぢき}な
% \ruby{正直}{しやう|ぢき}な
\ruby{篤實}{こく|めい}の
\ruby{人}{ひと}
も
ございますのネエ。
』

\原本頁{183-8}%
お
\ruby{龍}{りう}の
\ruby{叔母}{を|ば}の
\ruby{如是}{か|く}
\ruby{云}{い}ひ
\ruby{出}{い}づるを
\ruby[|g|]{主人}{あるじ}に
\ruby{答}{こた}へさする
\ruby{迄}{まで}も
\ruby{無}{な}く、
%
お
\ruby{龍}{りう}は
\ruby{代}{かは}つて、

\原本頁{183-10}%
『
そりやあ
\ruby{叔母}{を|ば}さん
\ruby[||j>]{東}{とう}
\ruby[||j>]{京}{きやう}
% \ruby{東京}{とう|きやう}
だつて
\ruby{狡猾}{ず|る}い
\ruby{人}{ひと}
ばかり
ぢやあ
\ruby{有}{あ}りません、
%
\ruby{廣}{ひろ}いん
ですもの。
%
\ruby{今}{いま}の
\ruby{話}{はなし}の
\ruby[||j>]{伯}{はく}
\ruby[||j>]{爵}{しやく}
% \ruby{伯爵}{はく|しやく}
のやうな
\ruby{卑格}{け|ち}な
\ruby{人}{ひと}も
\ruby{有}{あ}る
\ruby{代}{かは}りにやあ、
%
\ruby{姊}{ねえ}さん
の
やうな
\ruby{氣}{き}の
\ruby{大}{おほ}きい
\ruby{人}{ひと}
も
あるぢやあ
\ruby{有}{あ}りませんか。
』

\原本頁{184-4}%
と
\ruby{云}{い}へば、

\原本頁{184-5}%
『
ほんにね!。
%
だが、
%
\ruby{其樣}{そ|ん}なに
\ruby{高}{たか}い
\ruby{磁器}{やき|もの}
なんかゞ
\ruby{有}{あ}るものか
\ruby{知}{し}ら?。
』

\原本頁{184-7}%
といふ。

\原本頁{184-8}%
『
なあに、
%
\ruby{高}{たか}いと
\ruby{云}{い}つた
ところで
\ruby{多寡}{た|くわ}の
\ruby{知}{し}れたものですが、
%
つまり
\ruby{氣}{き}の
\ruby{小}{ちひさ}い
\ruby{人}{ひと}にやあ
\ruby{何樣}{ど|ん}なものでも
\ruby{大}{たい}したものに
\ruby{思}{おも}へるのでねえ、
%
それで
\ruby{大變}{たい|へん}に
\ruby{心配}{しん|ぱい}
したのでしやう。
』

\原本頁{184-11}%
と
お
\ruby{彤}{とう}の
\ruby{打}{うち}
\ruby{笑}{わら}ふ
\ruby{此}{こ}の
\ruby{問答}{もん|だふ}の
\ruby{中}{うち}に
\ruby{老人}{らう|じん}は
\ruby{復}{また}
\ruby{入}{い}り
\ruby{來}{きた}りしが、
%
\ruby[|g|]{背後}{うしろ}には
\ruby{恐}{おそ}れ
\ruby{惶}{かしこ}みて
\ruby{小}{ちひさ}く
なりたる
\ruby{{\換字{若}}}{わか}き
\ruby{女}{こ}を
\ruby{{\換字{連}}}{つ}れたり。
%
お
\ruby{龍}{りう}の
\ruby{叔母}{を|ば}は
\ruby{何氣無}{なに|げ|な}く
\ruby{打}{うち}
\ruby{見}{み}やるに、
%
\ruby{面貌}{おも|ざし}は
\ruby{老人}{らう|じん}を
\ruby{其}{その}
\ruby{儘}{まゝ}に
\ruby{眼}{め}も
\ruby{細}{ほそ}く
\ruby{鼻}{はな}も
\ruby{細}{ほそ}けれど
\原本頁{185-3}\改行%
\ruby{醜}{みにく}き
かたには
あらず、
%
\ruby[|g|]{卵子}{たまご}
\ruby{形}{なり}の
\ruby{顏}{かほ}の
\ruby[||j>]{上}{じやう}
\ruby[||j>]{品}{ ひん}に
% \ruby{上品}{じやう|ひん}に
\ruby{優}{やさ}しくて、
%
\ruby{慾}{よく}には
\ruby{色}{いろ}の
やゝ
\ruby{靑白}{あを|じろ}く
\ruby{束髮}{そく|はつ}の
\ruby{毛}{け}の
\ruby{纖{\換字{過}}}{ほそ|す}ぎて
\ruby[||j>]{嵩}{かさ}
\ruby[||j>]{少}{すくな}きを
% \ruby{嵩少}{かさ|すくな}きを
\ruby{治}{なほ}して
\ruby{{\換字{遣}}}{や}りたけれど、
%
\原本頁{185-5}\改行%
\ruby{年齡}{と|し}には
\ruby{似氣}{に|げ}
\ruby{無}{な}く
\ruby{靜}{しづか}に
\ruby{沈着}{おち|つ}いたる
\ruby{樣}{さま}
\ruby{如何}{い|か}にも
\ruby[|g|]{怜悧}{りこう}
らしく、
%
お
\ruby{龍}{りう}には
\ruby{慥}{たしか}に
\ruby{三歳}{み|つ}
\ruby{四歳}{よ|つ}
\ruby{劣}{おと}りなるべけれど、
%
\ruby{見比}{み|くら}ぶれば
お
\ruby{龍}{りう}の
\ruby{方}{かた}
\ruby{{\換字{若}}}{わか}く
\原本頁{185-7}\改行%
\ruby{{\換字{浮}}々}{うき|〳〵}
として、
%
\ruby{既}{すで}に
\ruby{生死}{いき|しに}の
\ruby{苦勞}{く|らう}を
\ruby{知}{し}れるにも
\ruby{似}{に}ず
\ruby{{\換字{猶}}}{なほ}
あど
\ruby{無}{な}く
\ruby{見}{み}ゆ。
%
\原本頁{185-8}\改行%
\ruby{今}{いま}の
\ruby{談}{はなし}の
お
\ruby{富}{とみ}とは
\ruby{是}{これ}なるべし、
%
\ruby{成程}{なる|ほど}
\ruby[|g|]{{\換字{平}}常}{ふだん}は
\ruby{{\換字{過}}失}{あや|まち}など
\ruby{中々}{なか|〳〵}
\ruby{仕出}{し|いだ}すまじき
\ruby{愼}{つゝし}み
\ruby{深}{ふか}げの、
%
\ruby{氣}{き}の
\ruby{能}{よ}く
\ruby{{\換字{廻}}}{まは}りさうな、
%
くすみたる
\ruby{女}{をんな}かな、
%
\原本頁{185-10}\改行%
これで
\ruby{{\換字{若}}}{も}し
\ruby{此程}{これ|ほど}に
\ruby{縞}{しま}の
\ruby{粗}{あら}き
\ruby{銘撰}{めい|せん}を
\ruby{着居}{き|を}らずば、
%
\ruby{能}{よ}く
\ruby{見}{み}ぬものは
\ruby{二十歳}{は|た|ち}とも
\ruby{見做}{み|な}すべしと
\ruby{一度}{ひと|たび}は
\ruby{思}{おも}ひしが、
%
\ruby[|g|]{流石}{さすが}に
\ruby{年齡}{と|し}は
\ruby{年齡}{と|し}なり、
%
\ruby{主人}{しゆ|じん}と
\ruby{眼}{め}を
\ruby{見合}{み|あは}すや
\ruby{否}{いな}や、
%
いと
\ruby{幼}{をさな}き
\ruby{素振}{そ|ぶ}りの
\ruby{繕}{つくろ}ひ
\ruby{氣}{げ}も
\ruby{無}{な}く
\ruby{頭}{かうべ}を
\ruby{疊}{たゝみ}に
\ruby{着}{つ}けて、

\原本頁{186-3}%
『
\ruby{飛}{と}んでも
\ruby{無}{な}い
\ruby{麁怱}{そ|さう}を
\ruby{致}{いた}しましたのを、
%
\ruby{御免下}{ご|めん|くだ}さい
まして
\ruby{眞}{まこと}に
\原本頁{186-4}\改行%
\ruby{有}{あ}り
\ruby{{\換字{難}}}{がた}う
ございます。
%
それから
\ruby{御斷}{お|ことわ}りも
\ruby{致}{いた}しませんで
\ruby{宅}{たく}へ
まゐりましたのは
\ruby{{\換字{猶}}}{なほ}
\ruby{相}{あひ}
\ruby{濟}{す}みません
で
ございました。
』

\原本頁{186-6}%
と
\ruby{素直}{す|なほ}に
\ruby[|g|]{謝罪}{あやま}れば、
%
お
\ruby{彤}{とう}は
\ruby{莞爾}{に|こ}やかに、

\原本頁{186-7}%
『
\ruby[|g|]{{\換字{平}}常}{ふだん}の
お
\ruby{{\換字{前}}}{まへ}の
\ruby{仕方}{し|かた}が
\ruby{好}{い}いから
\ruby{叱}{しか}らうとも
\ruby{何}{なん}とも
\ruby{思}{おも}つてや
\ruby{仕}{し}ません。
%
\ruby{{\換字{過}}失}{あや|まち}は
\ruby{{\換字{過}}失}{あや|まち}
だから
\ruby{仕方}{し|かた}が
\ruby{無}{な}い。
%
これから
さへ
\ruby{氣}{き}を
\ruby{付}{つ}けて
\原本頁{186-9}\改行%
お
\ruby{吳}{く}れなら
\ruby{其}{それ}で
\ruby{可}{いゝ}よ。
%
さあ
もう
をかしな
\ruby{顏}{かほ}を
\ruby{仕}{し}ないで
お
\ruby{{\換字{前}}}{まへ}の
\ruby{馴染}{な|じみ}の
お
\ruby{龍}{りう}ちやん
にも
\ruby{挨拶}{あい|さつ}を
お
\ruby{爲}{し}。
』

\原本頁{186-11}%
といふ。
%
\ruby{叱}{しか}りだにされず
\ruby{免}{ゆる}されたる
\ruby{嬉}{うれ}しさに、
%
さしぐむ
\ruby{涙}{なみだ}の
\ruby{目}{め}を
あげて、
%
さて
そつと
お
\ruby{龍}{りう}を
\ruby{見}{み}て
\ruby{懷}{なつか}しげに
\ruby{叩頭}{じ|ぎ}
すれば、
%
お
\ruby{龍}{りう}も
\原本頁{187-2}\改行%
また
\ruby{懷}{なつ}かしげに
\ruby[|g|]{其方}{そなた}を
\ruby{見}{み}やりて、

\原本頁{187-3}%
『
お
\ruby{{\換字{前}}}{まへ}さんが
\ruby[|g|]{此方}{こちら}に
\ruby{見}{み}えなかつた
ので、
%
\ruby{妾}{わたし}あ
\ruby{何樣}{ど|ん}なにか
\ruby[|g|]{眞實}{ほんと}に
\ruby{淋}{さび}しく
\ruby{思}{おも}つたらう!。
%
\ruby{丁度}{ちやう|ど}
\ruby{好}{い}い
\ruby{事}{こと}ねえ、
%
かうして
\ruby{歸}{かへ}つて
おいでだつた
のだから、
%
また
これから
お
\ruby{{\換字{前}}}{まへ}さんと
\ruby{仲}{なか}を
\ruby{好}{よ}くして、
%
\ruby{先}{せん}のやうに
\ruby{{\換字{又}}}{また}
\ruby{毎{\換字{朝}}}{まい|あさ}
\ruby{起}{おこ}して
\ruby{貰}{もら}ひましやうかネエ。
%
ホヽヽ。
』

\原本頁{187-7}%
と
\ruby{埒}{らち}
\ruby{無}{な}き
ことを
\ruby{早語}{はや|かた}り
\ruby{掛}{か}く。

\原本頁{187-8}%
『
また
\ruby{其樣}{そ|ん}な
\ruby{下}{くだ}らない
\ruby{好}{い}い
\ruby{氣}{き}ぜんの
\ruby{事}{こと}を
お
\ruby{{\換字{前}}}{まへ}は
お
\ruby{云}{い}ひだよ。
』

\原本頁{187-9}%
\ruby{苦々}{にが|〴〵}しげに
\ruby{叔母}{を|ば}は
たしなむるを
お
\ruby{彤}{とう}は
\ruby{餘{\換字{所}}}{よ|そ}に
\ruby{聽}{き}きて、
\ruby{茶}{ちや}をや
\ruby{得}{え}んとする、
%
お
\ruby{春}{はる}〳〵と
\ruby{呼}{よ}ぶに、
%
お
\ruby{春}{はる}は
\ruby{如何}{い|か}にしけん
\ruby{{\換字{更}}}{さら}に
\ruby{出}{い}で
\ruby{來}{きた}らず。
%
かゝる
\ruby{事}{こと}を
\ruby{甚}{いた}く
\ruby{悅}{よろこ}ばぬ
お
\ruby{彤}{とう}の、
%
\ruby{聲}{こゑ}
こそは
\ruby{仂無}{はし|たな}く
\ruby{高}{たか}めね、

\原本頁{188-1}%
『
お
\ruby{春}{はる}、
%
お
\ruby{春}{はる}、% 読点があるから非踊り字表記なのかな?
』

\原本頁{188-2}%
と
\ruby{復}{また}
\ruby{呼}{よ}べども
\ruby{{\換字{更}}}{さら}に
\ruby{答}{こた}へなし。

\原本頁{188-3}%
『
お
\ruby{春}{はる}!。
%
\ruby{何樣}{ど|う}したえ?\inhibitglue{}%
お
\ruby{春}{はる}!。
』

\原本頁{188-4}%
\ruby{一}{ひ}ト
\ruby{聲}{こゑ}は
\ruby{一}{ひ}ト
\ruby{聲}{こゑ}に
\ruby{癇}{かん}の
\ruby{募}{つの}るさま
\ruby{歷々}{あり|〳〵}と
\ruby{見}{み}ゆるに、

\原本頁{188-5}%
『
\ruby{何}{なん}でございますか、
%
\ruby{妾}{わたし}が
』

\原本頁{188-6}%
と
お
\ruby{富}{とみ}の
\ruby{立}{た}ちに
かゝる
\ruby{時}{とき}、
%
\ruby[||j>]{臺}{だい}
\ruby[||j>]{{\換字{所}}}{どころ}
% \ruby{臺{\換字{所}}}{だい|どころ}
と
おぼしき
ところ
にて、

\原本頁{188-7}%
『
お
\ruby{春}{はる}さん、
%
お
\ruby{春}{はる}さん、% ここも非踊り字表記
%
\ruby{御召}{お|め}し
なさるやうぢや
\ruby{無}{な}いかえ。
%
おや、
%
お
\ruby{{\換字{前}}}{まへ}さん、
%
\ruby{何}{なに}を
\ruby{泣}{な}いて
\ruby{居}{ゐ}るの?。
』

\原本頁{188-9}%
と
お
\ruby{杉}{すぎ}が
\ruby[|g|]{{\換字{平}}素}{いつも}
\ruby{馬士聲}{ま|ご|ゞゑ}とて
\ruby{叱}{しか}らるゝ
いと
\ruby{大}{おほ}きなる
\ruby{{\換字{丈}}夫}{ぢやう|ぶ}さうな
\ruby{其}{そ}の
\ruby{馬士聲}{ま|ご|ゞゑ}の
\ruby{聞}{きこ}えぬ。
