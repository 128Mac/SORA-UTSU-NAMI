\Entry{其三十三}

『
\ruby{妾}{わたし}は
\ruby{東京}{とう|きやう}にやあ
\ruby{今時彼樣}{いま|どき|あ|ゝ}いふ
\ruby{人}{ひと}は
\ruby{無}{な}からうとばつかり
\ruby{思}{おも}つて
\ruby{居}{ゐ}ましたが、たまには
\ruby{矢張}{やつ|ぱ}り
\ruby{彼樣}{あ|ん}な
\ruby{正直}{しやう|ぢき}な
\ruby{篤實}{こく|めい}の
\ruby{人}{ひと}もございますのネエ。
』

お
\ruby{龍}{りう}の
\ruby{叔母}{を|ば}の
\ruby{如是}{か|く}
\ruby{云}{い}ひ
\ruby{出}{い}づるを
\ruby{主人}{ある|じ}に
\ruby{答}{こた}へさする
\ruby{迄}{まで}も
\ruby{無}{な}く、
お
\ruby{龍}{りう}は
\ruby{代}{かは}つて、

『そりやあ
\ruby{叔母}{を|ば}さん
\ruby{東京}{とう|きやう}だつて
\ruby{狡猾}{ず|る}い
\ruby{人}{ひと}ばかりぢやあ
\ruby{有}{あ}りません、
\ruby{廣}{ひろ}いんですもの。
\ruby{今}{いま}の
\ruby{話}{はなし}の
\ruby{伯爵}{はく|しやく}のやうな
\ruby{卑格}{け|ち}な
\ruby{人}{ひと}も
\ruby{有}{あ}る
\ruby{代}{かは}りにやあ、
\ruby{姊}{ねえ}さんのやうな
\ruby{氣}{き}の
\ruby{大}{おほ}きい
\ruby{人}{ひと}もあるぢやあ
\ruby{有}{あ}りませんか。
』

と
\ruby{云}{い}へば、

『ほんにね!。
だが、
\ruby{其樣}{そ|ん}なに
\ruby{高}{たか}い
\ruby{磁器}{やき|もの}なんかゞ
\ruby{有}{あ}るものか
\ruby{知}{し}ら?。
』

といふ。

『なあに、
\ruby{高}{たか}いと
\ruby{云}{い}つたところで
\ruby{多寡}{た|か}の
\ruby{知}{し}れたものですが、つまり
\ruby{氣}{き}の
\ruby{小}{ちひさ}い
\ruby{人}{ひと}にやあ
\ruby{何樣}{ど|ん}なものでも
\ruby{大}{たい}したものに
\ruby{思}{おも}へるのでねえ、それで
\ruby{大變}{たい|へん}に
\ruby{心配}{しん|ぱい}したのでしやう。
』

とお
\ruby{彤}{とう}の
\ruby{打笑}{うち|わら}ふ
\ruby{此}{こ}の
\ruby{問答}{もん|だふ}の
\ruby{中}{うち}に
\ruby{老人}{らう|じん}は
\ruby{復}{また}
\ruby{入}{い}り
\ruby{來}{きた}りしが、
\ruby{背後}{うし|ろ}には
\ruby{恐}{おそ}れ
\ruby{惶}{かしこ}みて
\ruby{小}{ちひさ}くなりたる
\ruby{若}{わか}き
\ruby{女}{こ}を
\ruby{{\換字{連}}}{つ}れたり。
お
\ruby{龍}{りう}の
\ruby{叔母}{を|ば}は
\ruby{何氣無}{なに|げ|な}く
\ruby{打見}{うち|み}やるに、
\ruby{面貌}{おも|ざし}は
\ruby{老人}{らう|じん}を
\ruby{其儘}{その|まゝ}に
\ruby{眼}{め}も
\ruby{細}{ほそ}く
\ruby{鼻}{はな}も
\ruby{細}{ほそ}けれど
\ruby{醜}{みにく}きかたにはあらず、
\ruby{卵子形}{たま|ご|かた}の
\ruby{顏}{かほ}の
\ruby{上品}{じやう|ひん}に
\ruby{優}{やさ}しくて、
\ruby{慾}{よく}には
\ruby{色}{いろ}のやゝ
\ruby{靑白}{あを|じろ}く
\ruby{束髮}{そく|はつ}の
\ruby{毛}{け}の
\ruby{纖{\換字{過}}}{ほそ|すぎ}ぎて
\ruby{嵩少}{かさ|すくな}きを
\ruby{治}{なほ}して
\ruby{{\換字{遣}}}{や}りたけれど、
\ruby{年齡}{と|し}には
\ruby{似氣無}{に|げ|な}く
\ruby{靜}{しづか}に
\ruby{沈着}{おち|つ}いたる
\ruby{樣}{さま}
\ruby{如何}{い|か}にも
\ruby{怜悧}{り|こう}らしく、
お
\ruby{龍}{りう}には
\ruby{慥}{たしか}に
\ruby{三歳四歳劣}{み|つ|よ|つ|おと}りなるべけれど、
\ruby{見比}{み|くら}ぶれば
お
\ruby{龍}{りう}の
\ruby{方若}{かた|わか}く
\ruby{{\換字{浮}}々}{うき|うき}として、
\ruby{既}{すで}に
\ruby{生死}{いき|しに}の
\ruby{苦勞}{く|らう}を
\ruby{知}{し}れるにも
\ruby{似}{に}ず
\ruby{{\換字{猶}}}{なほ}あど
\ruby{無}{な}く
\ruby{見}{み}ゆ。
\ruby{今}{いま}の
\ruby{談}{はなし}の
お
\ruby{富}{とみ}とは
\ruby{是}{これ}なるべし、
\ruby{成程{\換字{平}}常}{なる|ほど|ふだ|ん}は
\ruby{{\換字{過}}失}{あや|まち}など
\ruby{中々仕出}{なか|〳〵|しい|だ}すまじき
\ruby{愼}{つゝし}み
\ruby{深}{ふか}げの、
\ruby{氣}{き}の
\ruby{能}{よ}く
\ruby{{\換字{廻}}}{まは}りさうなくすみたる
\ruby{女}{をんな}かな、これで
\ruby{若}{も}し
\ruby{此程}{これ|ほど}に
\ruby{縞}{しま}の
\ruby{粗}{あら}き
\ruby{銘撰}{めい|せん}を
\ruby{着居}{き|を}らずば、
\ruby{能}{よ}く
\ruby{見}{み}ぬものは
\ruby{二十歳}{は|た|ち}とも
\ruby{見做}{み|な}すべしと
\ruby{一度}{ひと|たび}は
\ruby{思}{おも}ひしが、
\ruby{流石}{さす|が}に
\ruby{年齡}{と|し}は
\ruby{年齡}{と|し}なり、
\ruby{主人}{しゆ|じん}と
\ruby{眼}{め}を
\ruby{見合}{み|あは}すや
\ruby{否}{いな}や、いと
\ruby{幼}{おさな}き
\ruby{素振}{そ|ぶ}りの
\ruby{繕}{つくろ}ひ
\ruby{氣}{げ}も
\ruby{無}{な}く
\ruby{頭}{かうべ}を
\ruby{疊}{たゝみ}に
\ruby{着}{つ}けて、

『
\ruby{飛}{と}んでも
\ruby{無}{な}い
\ruby{麁忽}{そ|さう}を
\ruby{致}{いた}しましたのを、
\ruby{御免下}{ご|めん|くだ}さいまして
\ruby{眞}{まこと}に
\ruby{有}{あ}り
\ruby{難}{がた}うございます。
それから
\ruby{御斷}{お|ことわ}りも
\ruby{致}{いた}しませんで
\ruby{宅}{たく}へまゐりましたのは
\ruby{{\換字{猶}}}{なほ}
\ruby{相濟}{あひ|す}みませんでございました』

と
\ruby{素直}{す|なほ}に
\ruby{謝罪}{あや|ま}れば、
お
\ruby{彤}{とう}は
\ruby{莞爾}{に|こ}やかに、

『
\ruby{{\換字{平}}常}{ふだ|ん}の
お
\ruby{{\換字{前}}}{まへ}の
\ruby{仕方}{し|かた}が
\ruby{好}{い}いから
\ruby{叱}{しか}らうとも
\ruby{何}{なん}とも
\ruby{思}{おも}つてや
\ruby{仕}{し}ません。
\ruby{{\換字{過}}失}{あや|まち}は
\ruby{{\換字{過}}失}{あや|まち}だから
\ruby{仕方}{し|かた}が
\ruby{無}{な}い。
これからさへ
\ruby{氣}{き}を
\ruby{付}{つ}けて
お
\ruby{吳}{く}れなら
\ruby{其}{それ}で
\ruby{可}{いゝ}よ。
さあもうをかしな
\ruby{顏}{かほ}を
\ruby{仕}{し}ないで
お
\ruby{{\換字{前}}}{まへ}の
\ruby{馴染}{な|じみ}の
お
\ruby{龍}{りう}ちやんにも
\ruby{挨拶}{あい|さつ}を
お
\ruby{爲}{し}。
』

といふ。
\ruby{叱}{しか}りだにされず
\ruby{免}{ゆる}されたる
\ruby{嬉}{うれ}しさに、さしぐむ
\ruby{淚}{なみだ}の
\ruby{目}{め}をあげて、さてそつと
お
\ruby{龍}{りう}を
\ruby{見}{み}て
\ruby{懷}{なつか}しげに
\ruby{叩頭}{じ|ぎ}すれば、
お
\ruby{龍}{りう}もまた
\ruby{懷}{なつ}かしげに
\ruby{其方}{そな|た}を
\ruby{見}{み}やりて、

『お
\ruby{{\換字{前}}}{まへ}さんが
\ruby{此方}{こち|ら}に
\ruby{見}{み}えなかつたので、
\ruby{妾}{わたし}あ
\ruby{何樣}{ど|ん}なにか
\ruby{眞實}{ほん|と}に
\ruby{淋}{さび}しく
\ruby{思}{おも}つたらう!。
\ruby{丁度}{ちやう|ど}
\ruby{好}{い}い
\ruby{事}{こと}ねえ、かうして
\ruby{歸}{かへ}つておいでだつたのだから、またこれから
お
\ruby{{\換字{前}}}{まへ}さんと
\ruby{仲}{なか}を
\ruby{好}{よ}くして、
\ruby{先}{せん}のやうに
\ruby{{\換字{又}}}{また}
\ruby{毎{\換字{朝}}起}{まい|あさ|おこ}して
\ruby{貰}{もら}ひましやうかネエ。
ホヽヽ。
』

と
\ruby{埒無}{らち|な}きことを
\ruby{早語}{はや|かた}り
\ruby{掛}{か}く。

『また
\ruby{其樣}{そ|ん}な
\ruby{下}{くだ}らない
\ruby{好}{い}い
\ruby{氣}{き}ぜんの
\ruby{事}{こと}を
お
\ruby{{\換字{前}}}{まへ}は
お
\ruby{云}{い}ひだよ。
』

\ruby{苦々}{にが|〴〵}しげに
\ruby{叔母}{を|ば}はたしなむるを
お
\ruby{彤}{とう}は
\ruby{餘{\換字{所}}}{よ|そ}に
\ruby{聽}{き}きて
\ruby{茶}{ちや}をや
\ruby{得}{え}んとする、
お
\ruby{春}{はる}〳〵と
\ruby{呼}{よ}ぶに、
お
\ruby{春}{はる}は
\ruby{如何}{い|か}にしけん
\ruby{更}{さら}に
\ruby{出}{い}で
\ruby{來}{きた}らず。
かゝる
\ruby{事}{こと}を
\ruby{甚}{いた}く
\ruby{悅}{よろこ}ばぬ
お
\ruby{彤}{とう}の、
\ruby{聲}{こゑ}こそは
\ruby{仂無}{はし|たな}く
\ruby{高}{たか}めね、

『お
\ruby{春}{はる}、
お
\ruby{春}{はる}、』

と
\ruby{復}{また}
\ruby{呼}{よ}べども
\ruby{更}{さら}に
\ruby{答}{こた}へなし。

『お
\ruby{春}{はる}!。
\ruby{何樣}{ど|う}したえ?
お
\ruby{春}{はる}!。
』

\ruby{一}{ひ}ト
\ruby{聲}{こゑ}は
\ruby{一}{ひ}ト
\ruby{聲}{こゑ}に
\ruby{癇}{かん}の
\ruby{募}{つの}るさま
\ruby{歷々}{あり|〳〵}と
\ruby{見}{み}ゆるに、

『
\ruby{何}{なん}でございますか、
\ruby{妾}{わたし}が』

とお
\ruby{富}{とみ}の
\ruby{立}{た}ちにかゝる
\ruby{時}{とき}、
\ruby{臺{\換字{所}}}{だい|どころ}とおぼしきところにて、

『お
\ruby{春}{はる}さん、
お
\ruby{春}{はる}さん、
\ruby{御召}{お|め}しなさるやうぢや
\ruby{無}{な}いかえ。
おや、お
\ruby{{\換字{前}}}{まへ}さん、
\ruby{何}{なに}を
\ruby{泣}{な}いて
\ruby{居}{ゐ}るの?。
』

とお
\ruby{杉}{すぎ}が
\ruby{{\換字{平}}素}{いつ|も}
\ruby{馬士聲}{ま|ご|ゞゑ}とて
\ruby{叱}{しか}らるゝいと
\ruby{大}{おほ}きなる
\ruby{{\換字{丈}}夫}{ぢやう|ぶ}さうな
\ruby{其}{そ}の
\ruby{馬士聲}{ま|ご|ゞゑ}の
\ruby{聞}{きこ}えぬ。

