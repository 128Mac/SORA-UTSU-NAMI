\Entry{其三十四}

% メモ 校正終了 2024-03-18 2024-06-13
\原本頁{189-2}%
『
\ruby{鶉}{うづら}といふ
\ruby{鳥}{とり}は
\ruby[g]{自{\換字{分}}}{じ ぶん}の% ルビ調整(原本通り)非グループルビ
\ruby{身}{み}から
\ruby{出}{で}る
\ruby[|g|]{香氣}{にほひ}を
\ruby{止}{と}めて
\ruby[g]{仕舞}{し ま }つて、
%
\ruby[g]{獵犬}{かりいぬ}に
\ruby{嗅}{か}ぎ
\ruby{出}{だ}されない
やうにする
\ruby[g]{機能}{はたらき}を
\ruby{有}{も}つて
\ruby{居}{ゐ}ると
\ruby[|g|]{銃獵者}{とりうち}に
\ruby{聞}{き}いたが、
%
お
\ruby{彤}{とう}、
%
\ruby{汝}{きさま}は
\ruby[g]{一體}{いつたい}が
\ruby{{\換字{嫌}}}{いや}に
\ruby{治}{をさ}めきつて
\ruby{居}{ゐ}やがつて、
%
そして
\ruby[g]{時時}{ときどき}% ルビ調整(原本通り)非踊り字表記(行末行頭の境界付近)
\ruby{鶉}{うづら}のやうな
\ruby{藝}{げい}をする
\ruby{奴}{やつ}だなあ。
』

\原本頁{189-6}%
とは
\ruby{甞}{かつ}て
\ruby[g]{筑波}{つくば }が
\ruby[g]{爛醉}{らんすゐ}の
\ruby{後}{のち}に
\ruby{罵}{のゝし}りし
\ruby{語}{ことば}なるが、
%
\ruby{吉}{よき}に
\ruby{{\換字{遇}}}{あ}ひても
\ruby[|g|]{齒齦}{はぐき}を
\ruby{露}{あら}はして
\ruby{笑}{ゑ}み
くつがへる
\ruby{程}{ほど}は
\ruby{悅}{よろこ}ばず、
%
\ruby{凶}{あしき}に
\ruby{{\換字{遇}}}{あ}ひても
\ruby{眉}{まゆ}を
\ruby{皺}{しわ}めて
\ruby{沈}{しづ}み
\ruby{入}{い}る
\ruby{程}{ほど}は
\ruby{悲}{かなし}まで、
%
\ruby[g]{何時}{い つ }も
\ruby[g]{自{\換字{分}}}{じ ぶん}の% ルビ調整(原本通り)非グループルビ
\ruby{顏}{かほ}つきの
\ruby[g]{不齊}{む ら }の
\ruby{無}{な}いやうにと
\ruby{心}{こゝろ}がけて
\ruby{居}{ゐ}るでも
\ruby{有}{あ}るまじけれど、
%
\ruby[|g|]{自然}{おのづ}と
\ruby{胸}{むね}の
\ruby{中}{うち}のさまを
\原本頁{189-10}\改行%
\ruby{鮮}{あざ}やかに
\ruby[g]{他人}{ひ と }に
\ruby{讀}{よ}めるやうには
\ruby{面}{かほ}に
\ruby{出}{だ}さぬ
お
\ruby{彤}{とう}も、
%
\ruby[|g|]{烟草}{たばこ}には
\ruby[g]{烟草}{たばこ }の% ルビ調整(原本通り)非グループルビ
\ruby{蟲}{むし}の
\ruby{有}{あ}る
\ruby[g]{{\換字{道}}理}{だうり }にてや、
%
\ruby[g]{矢張}{や は }り
\ruby[g]{或機}{あるをり}には
\ruby{心}{こゝろ}の
\ruby{悶}{もだえ}をば
\ruby[<j>]{盡}{こと〴〵}く
\ruby{面}{おもて}に
\ruby{現}{あら}はすなり。

\原本頁{190-3}%
\ruby[g]{何時}{い つ }の
\ruby{事}{こと}なりけん、
%
\ruby{一}{ある}
\ruby[|g|]{劇場}{しばゐ}に% 原文通り「場」
\ruby{西洋{\換字{婦}}人}{せい|やう|ふ|じん}の
\ruby{奇}{き}
\ruby{{\換字{術}}}{じゆつ}の
\ruby[||j>]{興}{こう}
\ruby[||j>]{行}{ぎやう}の
% \ruby{興行}{こう|ぎやう}の
\ruby{有}{あ}りし
\makeatletter
\@ifundefined{デバッグ@ビルド}{%
  \ruby{時}{とき}、
}{%
  \ruby{時}{とき}
  \改行% 校正作業の簡略化のため
  、
}%
\makeatother

\原本頁{190-4}%
『
\ruby{姊}{ねえ}さん、
%
\ruby[g]{大變}{たいへん}に
\ruby[g]{面白}{おもしろ}いといふ
\ruby{噂}{うはさ}ですから
\ruby{{\換字{連}}}{つ}れて
\ruby{行}{い}つて
\ruby{見}{み}せて
\改行% 校正作業の簡略化のため
。
』

\原本頁{190-5}%
と
お
\ruby{彤}{とう}に
\ruby[g]{{\換字{請}}求}{ね だ }りけるに、

\原本頁{190-6}%
『
\ruby{觀}{み}たけりやあ
\ruby[||j>]{汝}{おまへ}
\ruby[||j>]{一人}{ ひ|とり}で
\ruby{行}{い}つて
\ruby[g]{御覽}{ご らん}な。
%
\ruby[g]{{\換字{魔}}{\換字{術}}}{て づま}は
\ruby{妾}{わたし}あ
\ruby[g]{大{\換字{嫌}}}{だいきら}ひだよ
\改行% 校正作業の簡略化のため
。
』

\原本頁{190-7}%
と
\ruby{膠}{にべ}も
\ruby{無}{な}く
\ruby{云}{い}はれしより
\ruby[g]{不圖}{ふ と }
お
\ruby{龍}{りう}は
\ruby[g]{心付}{こゝろづ}いて、
%
\ruby[g]{差當}{さしあた}り
\ruby{我}{わ}が
\ruby[g]{智慧}{ち ゑ }にて
\ruby[g]{何共}{なんとも}
\ruby{解}{わか}らぬ
\ruby{事}{こと}にあへば、
%
お
\ruby{彤}{とう}は
\ruby{甚}{いた}く
\ruby[g]{面白}{おもしろ}からず
\ruby{思}{おも}ふと
\ruby{見}{み}えて、
%
\ruby{必}{かな}らず
\ruby[g]{可厭}{い や }な
\ruby[g]{可厭}{い や }な
\ruby{顏}{かほ}して
\ruby[g]{不快}{ふくわい}さを
\ruby{示}{しめ}すを
\ruby{知}{し}りぬ。

\原本頁{190-10}%
\ruby[g]{何事}{なにごと}の
\ruby{悲}{かな}しくて
お
\ruby{春}{はる}は
\ruby{泣}{な}けるぞや、
%
\ruby{誰}{たれ}も
\ruby{其}{そ}の
\ruby{故}{ゆゑ}を
\ruby{思}{おも}ひ
\ruby{得}{え}しものは
\ruby{無}{な}けれど、
%
\ruby{誰}{たれ}も
また
\ruby{其}{そ}の
\ruby{故}{ゆゑ}の
\ruby{{\換字{分}}}{わか}らねば
とて
\ruby{何}{なに}と
\ruby{思}{おも}ふも
\ruby{無}{な}きに
\改行% 校正作業の簡略化のため
、
%
\原本頁{191-1}\改行%
お
\ruby{彤}{とう}は
\ruby{例}{れい}の
\ruby{我}{わ}が
\ruby[g]{合點}{が てん}の
\ruby{行}{ゆ}かぬ
といふことをば
\ruby{{\換字{強}}}{つよ}く
\ruby[g]{忌々}{いま〳〵}しがつて
\改行% 校正作業の簡略化のため
、
%
\原本頁{191-2}\改行%
\ruby{其}{そ}の
\ruby{故}{ゆゑ}を
\ruby{解}{と}かんと、
%
\ruby{苦}{くるし}み
\ruby{悶}{もだ}ゆる
なるべし、
%
たゞ
\ruby[g]{轉瞬}{またたき}
する
ほどの
\ruby[g]{刹那}{せつな }の
\ruby{間}{ま}なれど、
%
\ruby{星}{ほし}のやうなる
\ruby[||j>]{兩}{りやう}
\ruby[||j>]{眼}{ がん}
% \ruby{兩眼}{りやう|がん}
を
やゝ
\ruby{寄}{よ}せて
\ruby[g]{上眼}{うはめ }づかひ
したる
\ruby{其}{そ}の
\ruby[g]{樣子}{やうす }、
%
\ruby{何}{なん}とも
\ruby{云}{い}へぬ
\ruby[g]{可厭}{い や }な
ところ
ありて、
%
\ruby[g]{牙彫}{げ ぼり}の
\ruby[g]{小町}{こ まち}の
やうな
\ruby{申{\換字{分}}無}{まをし|ぶん|な}き
\ruby{眼鼻立}{め|はな|だち}の
\ruby{美}{うつく}しさをも
\ruby{人}{ひと}をして
\ruby{忘}{わす}れ
\ruby{果}{はて}しめたり
\改行% 校正作業の簡略化のため
。
%
\原本頁{191-6}\改行%
かねて
\ruby{心}{こゝろ}づき
\ruby{居}{ゐ}たればこそ、
%
お
\ruby{龍}{りう}
ただ% ルビ調整(原本通り)非踊り字表記
\ruby[|g|]{一人}{ひとり}は
お
\ruby{彤}{とう}が
\ruby{其}{そ}の
\ruby[g]{不快}{ふくわい}げなる
\ruby{面}{おもて}を
\ruby{爲}{な}したるを
\ruby{早}{はや}くも
\ruby{見}{み}たれ、
%
\ruby{他}{た}の
\ruby[g]{人々}{ひと〴〵}は
\ruby{{\換字{更}}}{さら}に
\ruby{氣}{き}の
\ruby{付}{つ}かぬ
\ruby{間}{ま}に、
%
\ruby{其}{その}
\ruby{人}{ひと}は
\ruby{復}{また}
\ruby{忽}{たちま}ち
\ruby{舊}{もと}の
\ruby[g]{樣子}{やうす }
に
なりたり。

\原本頁{191-9}%
お
\ruby{彤}{とう}は
お
\ruby{春}{はる}に
\ruby{復}{ふたゝ}び
\ruby{管}{かま}はず、
%
お
\ruby{富}{とみ}に
\ruby[|g|]{命令}{いひつ}くれば
お
\ruby{富}{とみ}は
\ruby[g]{心得}{こゝろえ}て、
%
\ruby[g]{人人}{ひとびと}に% ルビ調整(原本通り)非踊り字表記
\ruby{茶}{ちや}を
\ruby{侑}{すゝ}め
\ruby[g]{菓子}{くわし }を
\ruby{薦}{すゝ}め
などしけるが、
%
\ruby{其}{そ}の
\ruby{中}{うち}
\ruby{良}{やゝ}
\ruby{久}{ひさ}しく
お
\ruby{杉}{すぎ}
お
\ruby{春}{はる}は
\ruby{何}{なに}をか
\ruby{語}{かた}りける、
%
やがて
お
\ruby{杉}{すぎ}は
\ruby{次}{つぎ}の
\ruby{間}{ま}に
\ruby{來}{きた}りて
\ruby{打}{うち}
\ruby{笑}{わら}ひなが
\改行% 校正作業の簡略化のため
ら、

\原本頁{192-2}%
『
お
\ruby{春}{はる}さんの
\ruby{泣}{な}いて
\ruby{居}{を}りましたのは
\ruby[g]{斯樣}{か う }なので
ございますよ。
%
ほんとに
\ruby[|g|]{可憐}{かはい}らしい
ぢやあ
ございませんか、
%
あの
\ruby[g]{斯樣}{か う }なので
ございます。
%
お
\ruby{富}{とみ}さん
ていふ
\ruby{方}{かた}が
\ruby{歸}{かへ}つて
おいでに
なれば
\ruby{妾}{わたし}は
お
\ruby[<j||]{暇}{いとま}に
なるでしやう。
%
\ruby[g]{折角}{せつかく}
こんな
\ruby{好}{い}い
\ruby[g]{御家}{お うち}へ
\ruby[g]{來合}{き あは}せたのに、
%
また
\ruby[g]{吾家}{う ち }へ
\ruby{行}{ゆ}くのかと
\ruby{思}{おも}ふと
\ruby{餘}{あんま}り
\ruby[g]{{\換字{情}}無}{なさけな}いので、
%
\ruby{今}{いま}
\ruby{伺}{うかゞ}つて
\ruby{居}{ゐ}れば
\ruby[g]{結構}{けつこう}な
\原本頁{192-7}\改行%
お
\ruby[g]{{\換字{道}}具}{だうぐ }を
お
\ruby{富}{とみ}さん
ていふ
\ruby{方}{かた}が
\ruby[g]{麁怱}{そ さう}
なすつても、
%
\ruby[g]{器物}{しなもの}よりやあ
\ruby{人}{ひと}が
\ruby[g]{可愛}{か はい}いと
\ruby{仰}{おつし}あつて
\ruby{御叱言}{お|こ|ごと}も
\ruby{無}{な}くつて
\ruby{濟}{す}みましたが、
%
\ruby{其}{そ}の
お
\ruby{優}{やさ}しい
\ruby[g]{御話}{おはなし}を
\ruby{伺}{うかゞ}つて
\ruby{居}{ゐ}る
\ruby{中}{うち}に
\ruby{妾}{わたし}あ
\ruby{胸}{むね}が
\ruby{痛}{いた}くなつて
\ruby{參}{まゐ}りました。
%
つい
\ruby[g]{先月}{せんげつ}の
\ruby{末}{すゑ}、
%
\ruby{詰}{つま}らない
\ruby{茶飮茶碗}{ちや|のみ|ぢや|わん}
\ruby{一}{ひと}つ
\ruby{妾}{わたし}が
\ruby[g]{麁怱}{そ さう}して
\ruby{破}{わ}りました
\ruby{時}{とき}
\原本頁{192-11}\改行% 校正作業の簡略化のため
は、
%
そりやあ
\ruby[g]{繼母}{まゝはゝ}の
\ruby{事}{こと}ですから
\ruby[g]{仕方}{し かた}も
\ruby{無}{な}い
のです
けれども、
%
\ruby[<j||]{妾}{わたし}あ% 行末行頭の境界付近なので特例処置を施す
\ruby{一時間}{いち|じ|かん}も
\ruby{二時間}{に|じ|かん}も
\ruby{口}{くち}ぎたなく
\ruby{叱}{しか}られました
\ruby{上}{うへ}、
%
\ruby{{\換字{終}}}{しまひ}にやあ
\ruby{性}{しやう}の
\原本頁{193-2}\改行%
\ruby{付}{つ}くやうにつて
\ruby{火}{ひ}の
\ruby{點}{つ}いて
\ruby{居}{ゐ}る
\ruby[|g|]{{\換字{煙}}管}{きせる}の
\ruby[g]{雁首}{がんくび}を
\換字{志゛}つと% 「志」+「濁点」
\ruby{手}{て}の
\ruby{甲}{かふ}に
\ruby{捺}{お}し
\ruby{付}{つ}けられました。
%
\ruby{今}{いま}の
\ruby[g]{御話}{おはなし}を
\ruby{伺}{うかゞ}つて
\ruby{居}{ゐ}る
\ruby{中}{うち}に
\ruby{其}{そ}の
\ruby{事}{こと}を
\ruby{思}{おも}ひ
\ruby{出}{だ}しましたら、
%
\ruby{妾}{わたし}あ
\ruby{猫}{ねこ}に
なつても
\ruby{宜}{よ}う
ございますし、
%
\ruby[g]{御膳}{ご ぜん}を
\ruby{頂}{いたゞ}かなくつても
\ruby{宜}{よ}う
ございますから、
%
\ruby[g]{何樣}{ど う }か
\ruby[|g|]{此方}{こちら}の
\ruby[g]{御家}{お うち}の
\ruby[g]{何處}{ど こ }かの
\原本頁{193-6}\改行%
\ruby{隅}{すみ}へ
\ruby{置}{お}いて
\ruby{頂}{いたゞ}きたい
\ruby{氣}{き}が
\ruby{仕}{し}て
‥‥
\ruby[g]{何樣}{ど う }せ
\ruby{何}{なに}も
\ruby{知}{しり}ませんので
\ruby[g]{御役}{お やく}には
\ruby{立}{た}ちませんし、
%
\ruby[g]{無益}{む だ }ですから、
%
\ruby{置}{お}いては
\ruby{下}{くだ}さいますまいつて、
%
それで
つい、
%
\ruby{泣}{な}いて
\ruby[g]{仕舞}{し ま }つた
といふので
ございます。
%
ほんとに
\ruby{聞}{き}いて
\ruby{見}{み}ますりやあ
\ruby[g]{繼母}{まゝはゝ}
だもん
ですので
\ruby[||j>]{愍}{かは}% 「愍然 か(は)いさう」
\ruby[||j>]{然}{いさう}
% \ruby{愍然}{かは|いさう}% 「愍然 か(は)いさう」
でございますが、
%
\ruby{猫}{ねこ}に
でも
なりたい
なんかつて、
%
ホヽヽヽ
\ruby{何}{なん}ぼ
\ruby{何}{なん}でも
\ruby[|g|]{可笑}{をかし}う
ございます。
%
\ruby{併}{しか}し
それに
\ruby{付}{つ}けても
よく〳〵だと
\ruby{思}{おも}はれます。
』

\原本頁{194-1}%
と
\ruby{告}{つ}げたり。

\原本頁{194-2}%
\ruby{聞}{き}けば
\ruby{何}{なん}でも
\ruby{無}{な}き
\ruby{事}{こと}なるに
お
\ruby{彤}{とう}は
\ruby{晴}{はれ}やかなる
\ruby{面}{おもて}して、

\原本頁{194-3}%
『
ホヽヽ、
%
\ruby{何}{なに}かと
\ruby{思}{おも}つたら
\ruby[g]{其樣}{そ ん }な
\ruby{事}{こと}なのかえ。
%
\ruby[|g|]{愍然}{かはい}さうに、
%
\ruby[g]{其樣}{そ ん }なに
\ruby{居}{ゐ}たがるものなら
\ruby{置}{お}いて
\ruby{{\換字{遣}}}{や}りましやう。
%
\ruby[|g|]{怜悧}{りこう}で、% ルビ調整(原本通り)(りこう)
%
そして
\ruby[g]{毅然}{しつかり}
とした
ところが
ある
\ruby[g]{中々}{なか〳〵}の
\ruby[g]{好兒}{いゝこ }だから。
』

\原本頁{194-6}%
と
\ruby{云}{い}へば、
%
\ruby{其}{そ}の
\ruby{語}{ことば}を
\ruby{聞}{き}きて
\ruby[g]{物蔭}{ものかげ}に
\ruby{居}{ゐ}たりし
お
\ruby{春}{はる}は
\ruby[g]{如何}{い か }ばかり
\ruby{嬉}{うれ}しくや
\ruby{思}{おも}ひけん、
%
\ruby{誰}{た}が
\ruby[g]{面{\換字{前}}}{ま へ }に
\ruby{居}{を}るとも
\ruby{無}{な}き
ところ
にて
\ruby{唯}{ただ}% ルビ調整(原本通り)非踊り字表記
\ruby[g]{主人}{しゆじん}の
\ruby{方}{かた}に
\ruby{對}{むか}ひ、
%
\ruby{疊}{たゝみ}に
\ruby{手}{て}を
\ruby{突}{つ}き
\ruby{頭}{かしら}を
\ruby{下}{さ}げて
\ruby{恩}{おん}を
\ruby{謝}{しや}したり。

\原本頁{194-9}%
\ruby[g]{先刻}{さ き }より
\ruby[g]{始{\換字{終}}}{し じう}を% ルビ調整(原本通り)「ゆ」無し
\ruby[g]{見聞}{み き }きせるもの、
%
お
\ruby{富}{とみ}は
\ruby{云}{い}ふに
\ruby{及}{およ}ばず、
%
お
\ruby{富}{とみ}の
\ruby{{\換字{父}}}{ちゝ}、
%
お
\ruby{龍}{りう}、
%
お
\ruby{龍}{りう}の
\ruby[g]{叔母}{を ば }、
%
お
\ruby{春}{はる}、
%
お
\ruby{杉}{すぎ}の
\ruby{末}{すゑ}に
\ruby{至}{いた}るまで、
%
\ruby{誰}{たれ}か
\ruby{今}{いま}
\ruby[g]{寛大}{おほやう}にして
\ruby{{\換字{情}}}{なさけ}ある
\ruby{此}{こ}の
\ruby{家}{や}の
\ruby{美}{うつく}しき
\ruby{女主人}{ぢよ|しゆ|じん}に
\ruby{心}{こゝろ}を
\ruby{寄}{よ}せざるもの
\ruby{有}{あ}らん。
%
あはれ
お
\ruby{彤}{とう}は
\ruby{一}{ひと}つの
\ruby{器}{うつは}を
\ruby{失}{うしな}つて
\ruby[g]{六人}{ろくにん}の
\ruby{心}{こゝろ}を
\ruby{得}{え}たるなり。

\原本頁{195-2}%
お
\ruby{彤}{とう}も
\ruby[|g|]{流石}{さすが}に
\ruby[||j>]{心}{こゝろ}
\ruby[||j>]{樂}{ たの}しき
% \ruby{心樂}{こゝろ|たの}しき
なるべし、
%
\ruby{鶉}{うづら}のやうなる
\ruby{藝}{げい}をすると
\ruby{云}{い}はれし
\ruby{人}{ひと}ながら、
%
\ruby{例}{れい}の
\ruby{治}{をさ}め
\ruby{切}{き}つたる
\ruby{顏}{かほ}つきの
\ruby{口}{くち}の
\ruby{邊}{あたり}に、
%
\ruby{見}{み}ゆるか
\ruby{見}{み}えぬ
ほどの
\ruby{誇}{ほこ}りの
\ruby{笑}{わらひ}を
\ruby{含}{ふく}みたり。
