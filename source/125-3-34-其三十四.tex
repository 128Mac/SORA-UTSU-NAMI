\Entry{其三十四}

「
\ruby{鶉}{うづら}といふ
\ruby{鳥}{とり}は
\ruby{自{\換字{分}}}{じ|ぶん}の
\ruby{身}{み}から
\ruby{出}{で}る
\ruby{香氣}{に|ほひ}を
\ruby{止}{と}めて
\ruby{仕舞}{し|ま}つて、
\ruby{獵犬}{かり|いぬ}に
\ruby{嗅}{か}ぎ
\ruby{出}{だ}されないやうにする
\ruby{機能}{はた|らき}を
\ruby{有}{も}つて
\ruby{居}{ゐ}ると
\ruby{銃獵者}{と|り|うち}に
\ruby{聞}{き}いたが、
お
\ruby{彤}{とう}、
\ruby{汝}{きさま}は
\ruby{一體}{いつ|たい}が
\ruby{{\換字{嫌}}}{いや}に
\ruby{治}{をさ}めきつて
\ruby{居}{ゐ}やがつて、そして
\ruby[g]{時時鶉}{ときどきうづら}のやうな
\ruby{藝}{げい}をする
\ruby{奴}{やつ}だなあ。
』

とは
\ruby{甞}{かつ}て
\ruby{筑波}{つく|ば}が
\ruby{爛醉}{らん|すゐ}の
\ruby{後}{のち}に
\ruby{罵}{のゝし}りし
\ruby{語}{ことば}なるが、
\ruby{吉}{よき}に
\ruby{{\換字{遇}}}{あ}ひても
\ruby{齒齦}{は|ぐき}を
\ruby{露}{あら}はして
\ruby{笑}{ゑ}みくつがへる
\ruby{程}{ほど}は
\ruby{{\換字{悅}}}{よろこ}ばず、
\ruby{凶}{あしき}に
\ruby{{\換字{遇}}}{あ}ひても
\ruby{眉}{まゆ}を
\ruby{皺}{しわ}めて
\ruby{沈}{しづ}み
\ruby{入}{い}る
\ruby{程}{ほど}は
\ruby{悲}{かなし}まで、
\ruby{何時}{い|つ}も
\ruby{自{\換字{分}}}{じ|ぶん}の
\ruby{顏}{かほ}つきの
\ruby{不齊}{む|ら}の
\ruby{無}{な}いやうにと
\ruby{心}{こゝろ}がけて
\ruby{居}{ゐ}るでも
\ruby{有}{あ}るまじけれど、
\ruby{自然}{おの|づ}と
\ruby{胸}{むね}の
\ruby{中}{うち}のさまを
\ruby{鮮}{あざ}やかに
\ruby{他人}{ひ|と}に
\ruby{讀}{よ}めるやうには
\ruby{面}{かほ}に
\ruby{出}{だ}さぬ
お
\ruby{彤}{とう}も、
\ruby{烟草}{たば|こ}には
\ruby{烟草}{たば|こ}の
\ruby{蟲}{むし}の
\ruby{有}{あ}る
\ruby{道理}{だう|り}にてや、
\ruby{矢張}{や|は}り
\ruby{或機}{ある|をり}には
\ruby{心}{こゝろ}の
\ruby{悶}{もだ\換字{江}}をば
\ruby{盡}{こと〴〵}く
\ruby{面}{おもて}に
\ruby{現}{あら}はすなり。

\ruby{何時}{い|つ}の
\ruby{事}{こと}なりけん、
\ruby{一劇場}{ある|しば|ゐ}に
\ruby[g]{西洋婦人}{せいやうふじん}の
\ruby[g]{奇術}{きじゆつ}の
\ruby[g]{興行}{こうぎやう}の
\ruby{有}{あ}りし
\ruby{時}{とき}、

『
\ruby{姊}{ねえ}さん、
\ruby{大變}{たい|へん}に
\ruby{面白}{おも|しろ}いといふ
\ruby{噂}{うはさ}ですから
\ruby{{\換字{連}}}{つ}れて
\ruby{行}{い}つて
\ruby{見}{み}せて。
』

とお
\ruby{彤}{とう}に
\ruby{請求}{ね|だ}りけるに、

『
\ruby{觀}{み}たけりやあ
\ruby{汝}{おまへ}
\ruby{一人}{ひと|り}で
\ruby{行}{い}つて
\ruby{御覽}{ご|らん}な。
\ruby[g]{魔{\換字{術}}}{てづま}は
\ruby{妾}{わたし}あ
\ruby{大{\換字{嫌}}}{だい|きら}ひだよ。
』

と
\ruby{膠}{にべ}も
\ruby{無}{な}く
\ruby{云}{い}はれしより
\ruby{不圖}{ふ|と}
お
\ruby{龍}{りう}は
\ruby{心付}{こゝろ|づ}いて、
\ruby{差當}{さし|あた}り
\ruby{我}{わ}が
\ruby{智慧}{ち|ゑ}にて
\ruby{何共解}{なん|とも|わか}らぬ
\ruby{事}{こと}にあへば、
お
\ruby{彤}{とう}は
\ruby{甚}{いた}く
\ruby{面白}{おも|しろ}からず
\ruby{思}{おも}ふと
\ruby{見}{み}えて、
\ruby{必}{かな}らず
\ruby{可厭}{い|や}な
\ruby{可厭}{い|や}な
\ruby{顏}{かほ}して
\ruby{不快}{ふ|くわい}さを
\ruby{示}{しめ}すを
\ruby{知}{し}りぬ。

\ruby{何事}{なに|ごと}の
\ruby{悲}{かな}しくて
お
\ruby{春}{はる}は
\ruby{泣}{な}けるぞや、
\ruby{誰}{たれ}も
\ruby{其}{そ}の
\ruby{故}{ゆゑ}を
\ruby{思}{おも}ひ
\ruby{得}{え}しものは
\ruby{無}{な}けれど、
\ruby{誰}{たれ}もまた
\ruby{其}{そ}の
\ruby{故}{ゆゑ}の
\ruby{{\換字{分}}}{わか}らねばとて
\ruby{何}{なに}と
\ruby{思}{おも}ふも
\ruby{無}{な}きに、
お
\ruby{彤}{とう}は
\ruby{例}{れい}の
\ruby{我}{わ}が
\ruby{合點}{が|てん}の
\ruby{行}{ゆ}かぬといふことをば
\ruby{{\換字{強}}}{つよ}く
\ruby{忌々}{いま|〳〵}しがつて、
\ruby{其}{そ}
の
\ruby{故}{ゆゑ}を
\ruby{解}{と}かんと、
\ruby{苦}{くるし}み
\ruby{悶}{もだ}ゆるなるべし、たゞ
\ruby{轉瞬}{また|たき}するほどの
\ruby{刹那}{せつ|な}の
\ruby{間}{ま}なれど、
\ruby{星}{ほし}のやうなる
\ruby[g]{兩眼}{りやうがん}をやゝ
\ruby{寄}{よ}せて
\ruby{上眼}{うは|め}づかひしたる
\ruby{其}{そ}の
\ruby{樣子}{やう|す}、
\ruby{何}{なん}とも
\ruby{云}{い}へぬ
\ruby{可厭}{い|や}なところありて、
\ruby[g]{牙彫}{げぼり}の
\ruby[g]{小町}{こまち}のやうな
\ruby[g]{申{\換字{分}}無}{まをしぶんな}き
\ruby{眼鼻立}{め|はな|だち}の
\ruby{美}{うつく}しさをも
\ruby{人}{ひと}をして
\ruby{忘}{わす}れ
\ruby{果}{はて}しめたり。

かねて
\ruby{心}{こゝろ}づき
\ruby{居}{ゐ}ればこそ、
お
\ruby{龍}{りう}たゞ
\ruby{一人}{ひと|り}は
お
\ruby{彤}{とう}が
\ruby{其}{そ}の
\ruby{不快}{ふ|くわい}げなる
\ruby{面}{おもて}を
\ruby{爲}{な}したるを
\ruby{早}{はや}くも
\ruby{見}{み}たれ、
\ruby{他}{た}の
\ruby{人々}{ひと|〴〵}は
\ruby{更}{さら}に
\ruby{氣}{き}の
\ruby{付}{つ}かぬ
\ruby{間}{ま}に、
\ruby{其人}{その|ひと}は
\ruby[g]{復忽}{またたちま}ち
\ruby{舊}{もと}の
\ruby{樣子}{やう|す}になりたり。

お
\ruby{彤}{とう}は
お
\ruby{春}{はる}に
\ruby{復}{ふたゝ}び
\ruby{管}{かま}はず、
お
\ruby{富}{とみ}に
\ruby{命令}{いひ|つけ}くれば
お
\ruby{富}{とみ}は
\ruby{心得}{こゝろ|え}て、
\ruby{人人}{ひと|びと}に
\ruby{茶}{ちや}を
\ruby{侑}{すゝ}め
\ruby{菓子}{くわ|し}を
\ruby{薦}{すゝ}めなどしけるが、
\ruby{其}{そ}の
\ruby{中良久}{うち|やゝ|ひさ}しく
お
\ruby{杉}{すぎ}
お
\ruby{春}{はる}は
\ruby{何}{なに}をか
\ruby{語}{かた}りける、やがて
お
\ruby{杉}{すぎ}は
\ruby{次}{つぎ}の
\ruby{間}{ま}に
\ruby{來}{きた}りて
\ruby{打笑}{うち|わら}ひながら、

『お
\ruby{春}{はる}さんの
\ruby{泣}{な}いて
\ruby{居}{を}りましたのは
\ruby{斯樣}{か|う}なのでございますよ。
ほんとに
\ruby{可憐}{かは|い}らしいぢやあございませんか、あの
\ruby{斯樣}{か|う}なのでございます。
お
\ruby{富}{とみ}さんていふ
\ruby{方}{かた}が
\ruby{歸}{かへ}つておいでになれば
\ruby{妾}{わたし}は
お
\ruby{暇}{いとま}になるでしやう。
\ruby{折角}{せつ|かく}こんな
\ruby{好}{い}い
\ruby{御家}{お|うち}へ
\ruby[g]{來合}{きあは}せたのに、また
\ruby{吾家}{う|ち}へ
\ruby{行}{ゆ}くのかと
\ruby{思}{おも}ふと
\ruby{餘}{あんま}り
\ruby[g]{{\換字{情}}無}{なさけな}いので、
\ruby[g]{今伺}{いまうかゞ}つて
\ruby{居}{ゐ}れば
\ruby{結構}{けつ|こう}な
お
\ruby{道具}{だう|ぐ}を
お
\ruby{富}{とみ}さんていふ
\ruby{方}{かた}が
\ruby{麁忽}{そ|さう}なすつても、
\ruby{器物}{しな|もの}よりやあ
\ruby{人}{ひと}が
\ruby{可愛}{か|はい}いと
\ruby{仰}{おつし}あつて
\ruby[g]{御叱言}{おこごと}も
\ruby{無}{な}くつて
\ruby{濟}{す}みましたが、
\ruby{其}{そ}の
お
\ruby{優}{やさし}しい
\ruby{御話}{お|はなし}を
\ruby{伺}{うかゞ}つて
\ruby{居}{ゐ}る
\ruby{中}{うち}に
\ruby{妾}{わたし}あ
\ruby{胸}{むね}が
\ruby{痛}{いた}くなつて
\ruby{參}{まゐ}りました。
つい
\ruby{先月}{せん|げつ}の
\ruby{末}{すゑ}、
\ruby{詰}{つま}らない
\ruby[g]{茶飮茶碗一}{ちやのみぢやわんひと}つ
\ruby{妾}{わたし}が
\ruby{麁忽}{そ|さう}して
\ruby{破}{わ}りました
\ruby{時}{とき}は、そりやあ
\ruby{繼母}{まゝ|はゝ}の
\ruby{事}{こと}ですから
\ruby{仕方}{し|かた}も
\ruby{無}{な}いのですけれども、
\ruby{妾}{わたし}あ
\ruby{一時間}{いち|じ|かん}も
\ruby[g]{二時間}{にじかん}も
\ruby{口}{くち}ぎたなく
\ruby{叱}{しか}られました
\ruby{上}{うへ}、
\ruby{{\換字{終}}}{しまひ}にやあ
\ruby{性}{しやう}の
\ruby{付}{つ}くやうにつて
\ruby{火}{ひ}の
\ruby{點}{つ}いて
\ruby{居}{ゐ}る
\ruby[g]{煙管}{きせる}の
\ruby{雁首}{がん|くび}を\換字{志}つと
\ruby{手}{て}の
\ruby{甲}{かふ}に
\ruby{捺}{お}し
\ruby{付}{つ}けられました。
\ruby{今}{いま}の
\ruby{御話}{お|はなし}を
\ruby{伺}{うかゞ}つて
\ruby{居}{ゐ}る
\ruby{中}{うち}に
\ruby{其}{そ}の
\ruby{事}{こと}を
\ruby{思}{おも}ひ
\ruby{出}{だ}しましたら、
\ruby{妾}{わたし}あ
\ruby{猫}{ねこ}になつても
\ruby{宜}{よ}うございますし、
\ruby{御膳}{ご|ぜん}を
\ruby{頂}{いたゞ}かなくつても
\ruby{宜}{よ}うございますから、
\ruby{何樣}{ど|う}か
\ruby{此方}{こち|ら}の
\ruby{御家}{お|うち}の
\ruby{何處}{ど|こ}かの
\ruby{隅}{すみ}へ
\ruby{置}{お}いて
\ruby{頂}{いたゞ}きたい
\ruby{氣}{き}が
\ruby{仕}{し}て……
\ruby{何樣}{ど|う}せ
\ruby{何}{なに}も
\ruby{知}{しり}ませんので
\ruby{御役}{お|やく}には
\ruby{立}{た}ちませんし、
\ruby{無{\換字{益}}}{む|だ}ですから、
\ruby{置}{お}いては
\ruby{下}{くだ}さいますまいつて、それでつい、
\ruby{泣}{な}いて
\ruby{仕舞}{し|ま}つたといふのでございます。
ほんとに
\ruby{聞}{き}いて
\ruby{見}{み}ますりやあ
\ruby{繼母}{まゝ|はゝ}だもんですので
\ruby{愍然}{かはい|さう}でございますが、
\ruby{猫}{ねこ}にでもなりたいなんかつて、ホヽヽヽ
\ruby{何}{なん}ぼ
\ruby{何}{なん}でも
\ruby{可笑}{を|か}しうございます。
\ruby{併}{しか}しそれに
\ruby{付}{つ}けてもよく〳〵だと
\ruby{思}{おも}はれます。
』

と
\ruby{告}{つ}げたり。

\ruby{聞}{き}けば
\ruby{何}{なん}でも
\ruby{無}{な}き
\ruby{事}{こと}なるに
お
\ruby{彤}{とう}は
\ruby{晴}{はれ}やかなる
\ruby{面}{おもて}して、

『ホヽヽ、
\ruby{何}{なに}かと
\ruby{思}{おも}つたら
\ruby{其樣}{そ|ん}な
\ruby{事}{こと}なのかえ。
\ruby[g]{愍然}{かはい}さうに、
\ruby{其樣}{そ|ん}なに
\ruby{居}{ゐ}たがるものなら
\ruby{置}{お}いて
\ruby{{\換字{遣}}}{や}りましやう。
\ruby{恰悧}{り|かう}で、そして
\ruby{毅然}{しつ|かり}としたところがある
\ruby{中々}{なか|〳〵}の
\ruby{好兒}{いゝ|こ}だから。
』

と
\ruby{云}{い}へば、
\ruby{其}{そ}の
\ruby{語}{ことば}を
\ruby{聞}{き}きて
\ruby{物蔭}{もの|かげ}に
\ruby{居}{ゐ}たりし
お
\ruby{春}{はる}は
\ruby{如何}{い|か}ばかり
\ruby{嬉}{うれ}しくや
\ruby{思}{おも}ひけん、
\ruby{誰}{た}が
\ruby{面前}{ま|へ}に
\ruby{居}{を}るとも
\ruby{無}{な}きところにて
\ruby{唯}{たゞ}
\ruby{主人}{しゆ|じん}の
\ruby{方}{かた}に
\ruby{對}{むか}ひ、
\ruby{疊}{たゝみ}に
\ruby{手}{て}を
\ruby{突}{つ}き
\ruby{頭}{かしら}を
\ruby{下}{さ}げて
\ruby{恩}{おん}を
\ruby{謝}{しや}したり。

\ruby{先刻}{さ|き}より
\ruby[g]{始{\換字{終}}}{しじう}を
\ruby{見聞}{み|き}きせるもの、
お
\ruby{富}{とみ}は
\ruby{云}{い}ふに
\ruby{及}{およ}ばず、
お
\ruby{富}{とみ}の
\ruby{父}{ちゝ}、
お
\ruby{龍}{りう}の
\ruby{叔母}{を|ば}、
お
\ruby{春}{はる}、
お
\ruby{杉}{すぎ}の
\ruby{末}{すゑ}に
\ruby{至}{いた}るまで、
\ruby{誰}{た}が
\ruby[g]{今寛大}{いまおほやう}にして
\ruby{{\換字{情}}}{なさけ}ある
\ruby{此}{こ}の
\ruby{家}{や}の
\ruby{美}{うつく}しき
\ruby{女主人}{ぢよ|しゆ|じん}に
\ruby{心}{こゝろ}を
\ruby{寄}{よ}せざるもの
\ruby{有}{あ}らん。
あはれお
\ruby{彤}{とう}は
\ruby{一}{ひと}つの
\ruby{器}{うつは}を
\ruby{失}{うしな}つて
\ruby{六人}{ろく|にん}の
\ruby{心}{こゝろ}を
\ruby{得}{え}たるなり。

お
\ruby{彤}{とう}も
\ruby{流石}{さす|が}に
\ruby{心樂}{こゝろ|たの}しきなるべし、
\ruby{鶉}{うづら}のやうなる
\ruby{藝}{げい}をすると
\ruby{云}{い}はれし
\ruby{人}{ひと}ながら、
\ruby{例}{れい}の
\ruby{治}{をさ}め
\ruby{切}{き}つたる
\ruby{顏}{かほ}つきの
\ruby{口}{くち}の
\ruby{邊}{あたり}に、
\ruby{見}{み}ゆるか
\ruby{見}{み}えぬほどの
\ruby{誇}{ほこ}りの
\ruby{笑}{わらひ}を
\ruby{含}{ふく}みたり。

