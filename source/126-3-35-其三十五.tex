\Entry{其三十五}

\原本頁{}%
お
\ruby{彤}{とう}が
\ruby{{\換字{分}}別}{ふん|べつ}に
\ruby{長}{た}けたる
\ruby{事}{こと}は
\ruby{對談}{はな|し}の
\ruby{中}{うち}にも
\ruby{知}{し}りしが、
%
\ruby{今}{いま}
\ruby{{\換字{又}}}{また}
\ruby{眼}{ま}のあたりに
\ruby{其}{そ}の
\ruby{胸}{むね}の
\ruby{廣}{ひろ}く
\ruby{慈悲}{なさ|け}の
\ruby{厚}{あつ}きをば
\ruby{見}{み}て、
%
\ruby{隨{\換字{分}}}{ずゐ|ぶん}
\ruby{負}{ま}けぬ
\ruby{氣}{き}の
お
\ruby{龍}{りう}の
\ruby{叔母}{を|ば}も
\ruby{全}{まつた}く
\ruby{我}{が}を
\ruby{折}{を}り
\ruby{盡}{つく}くして、
%
\ruby{好}{い}いと
\ruby{思}{おも}ひ
\ruby{{\換字{込}}}{こ}めば
\ruby{何處}{ど|こ}までも
\ruby{好}{い}いに
\ruby{仕}{し}て
\ruby{{\換字{終}}}{しま}ふ
\ruby{田舎}{ゐな|か}
\ruby{氣}{ぎ}の
\ruby{正直三昧}{しやう|ぢき|ざん|まい}に、
%
\ruby{此}{こ}の
\ruby{人}{ひと}にさへ
\ruby{頼}{たの}み
\ruby{置}{お}けば
\ruby{何樣}{ど|う}
\ruby{轉}{ころ}んでも
\ruby{間{\換字{違}}}{ま|ちがひ}
\ruby{無}{な}しと
\ruby{盡}{こと〴〵}く
\ruby{信}{しん}じて、
%
\ruby{何{\換字{分}}}{なに|ぶん}
\ruby{宜}{よろ}しく
\ruby{願}{ねが}ひまするを
\ruby[||j>]{百}{ひやく}
\ruby[||j>]{{\換字{遍}}}{ ぺん}ほども
% \ruby{百{\換字{遍}}}{ひやく|ぺん}ほども
\ruby{云}{い}ひたる
\ruby{末}{すゑ}、
%
\ruby{何事}{なに|ごと}も
お
\ruby{彤}{とう}
\ruby{任}{まか}せにして
\ruby{其}{その}
\ruby{次}{つぎ}の
\ruby{日}{ひ}に
\ruby{靜岡}{しづ|をか}へ
\ruby{歸}{かへ}りぬ。

\原本頁{}%
『お
\ruby{龍}{りう}ちやん、
%
お
\ruby{{\換字{前}}}{まへ}
\ruby{一寸}{ちよ|つと}
\ruby{今}{いま}までの
\ruby{居處}{う|ち}へ
\ruby{歸}{かへ}つてネ、
%
\ruby{叔母}{を|ば}のいひつけで
\ruby{今後}{これ|から}これ〳〵のところに
\ruby{居}{ゐ}るやうになつたといふ
\ruby{事}{こと}だけを
\ruby{斷}{ことわ}つておいでな。
』

\原本頁{}%
\ruby{叔母}{を|ば}の
\ruby{歸郷}{かへ|り}を
\ruby{停車場}{てい|しや|じやう}まで% 原文通り「場」
\ruby{{\換字{送}}}{おく}つての
\ruby{後}{のち}、
%
\ruby{何}{なに}を
\ruby{思}{おも}ふにや
\ruby{茫然}{ばう|ぜん}として
\ruby{爲}{な}す
\ruby{事}{こと}も
\ruby{無}{な}く
\ruby{居}{ゐ}たる
お
\ruby{龍}{りう}に
\ruby{向}{むか}つて
お
\ruby{彤}{とう}はかくの
\ruby{如}{ごと}く
\ruby{云}{い}ひ
\ruby{出}{いだ}したり。
%
お
\ruby{龍}{りう}は
\ruby{{\換字{迷}}惑}{めい|わく}さうに
\ruby{眉根}{まゆ|ね}を
\ruby{寄}{よ}せながら、
%
\ruby{何}{なん}の
\ruby{思案}{し|あん}も
\ruby{無}{な}く、

\原本頁{}%
『
\ruby{行}{い}かなくつちやあいけませんかネ、
%
ネエ
\ruby{行}{い}かなくつちやあ。
』

\原本頁{}%
と、
%
\ruby{然}{さ}も〳〵
\ruby{其}{そ}の
\ruby{事}{こと}の
\ruby{宥免}{ゆる|し}を
\ruby{乞}{こ}ふが
\ruby{如}{ごと}くに
\ruby{云}{い}へり。

\原本頁{}%
『ホヽヽ、
%
\ruby{{\換字{嫌}}}{いや}なの?\inhibitglue{}%
\ruby{其樣}{そん|な}に。
%
\ruby{怖}{こは}いやうにでも
\ruby{思}{おも}つて?。
』

\原本頁{}%
『
\ruby{怖}{こは}いつて
\ruby{事}{こと}は
\ruby{有}{あ}りませんけれどもネ、
%
\ruby{今日}{け|ふ}つから
\ruby{御暇}{お|ひま}を
\ruby{致}{いた}します、
%
\ruby{左樣}{さ|やう}ならつて
\ruby{云}{い}ふのが
\ruby{何}{なん}だか
\ruby{云}{い}ひづらいやうな
\ruby[||j>]{心}{こゝろ}
\ruby[||j>]{持}{ もち}がするんですもの。
% \ruby{心持}{こゝろ|もち}がするんですもの。
』

\原本頁{}%
『だつて
\ruby{何}{なに}も
お
\ruby{{\換字{前}}}{まへ}が
\ruby{不義理}{ふ|ぎ|り}なことを
\ruby{爲}{す}るつて
\ruby{云}{い}ふのぢやあ
\ruby{無}{な}し、
%
お
\ruby{{\換字{前}}}{まへ}にも
\ruby{{\換字{分}}}{わか}つて
\ruby{居}{ゐ}るとほり
\ruby{先方}{む|かふ}の
お
\ruby{腹}{なか}の
\ruby{中}{なか}が
\ruby{良}{よ}くないんだから、
%
ことわりを
\ruby{云}{い}ふだけの
\ruby{事}{こと}に
\ruby{譯}{わけ}は
\ruby{無}{な}いぢやあ
\ruby{無}{な}いか。
』

\原本頁{}%
『そりやあ、
%
\ruby{理屈}{り|くつ}は、
%
もうほんとに
\ruby{其{\換字{通}}}{その|とほ}りなんですけれども。
』

\原本頁{}%
『ぢやあ、
%
また、
%
\ruby{何故}{な|ぜ}ネエ?。
』

\原本頁{}%
『
\ruby{何}{なん}だか
\ruby{妾}{わたし}にも
\ruby{理由}{わ|け}は
\ruby{{\換字{分}}}{わか}りませんけども、
%
\ruby{妾}{わたし}にやあ
\ruby{{\換字{判}}然}{はつ|きり}と
\ruby{斷}{ことわ}りが
\ruby{云}{い}へさうも
\ruby{無}{な}いんですもの!。
%
\ruby{心}{しん}はほんとに
\ruby{可厭}{い|や}な
\ruby{人}{ひと}ですけれども、
%
\ruby{表面}{うは|べ}だけにしろ
お
\ruby{龍}{りう}〳〵つて
\ruby{可愛}{か|はい}がつて
\ruby{吳}{く}れまして、
%
\ruby{斯樣}{か|う}やつて
\ruby{衣類}{き|もの}も
\ruby{着}{き}せて
\ruby{吳}{く}れますし、
%
\ruby{一個}{ひと|つ}あるものも
\ruby{{\換字{半}}{\換字{分}}}{はん|ぶん}は
\ruby{取}{と}り
\ruby{{\換字{分}}}{わ}けて
\ruby{吳}{く}れるやうに
\ruby{始{\換字{終}}}{しじ|う}% ルビは原本通り「ゆ」無し
\ruby{爲}{さ}れて
\ruby{居}{ゐ}るんですから、
%
いつそ
\ruby{惡口}{あく|たい}でも
\ruby{云}{い}はれて
\ruby{喧嘩}{けん|くわ}でも
\ruby{仕}{し}たら
\ruby{妾}{わたし}の
\ruby{胸}{むね}の
\ruby{中}{なか}を
\ruby{有}{あ}り
\ruby{體}{てい}に
\ruby{云}{い}ひ
\ruby{出}{だ}す
\ruby{事}{こと}も
\ruby{出來}{で|き}るか
\ruby{知}{し}れませんけど、
%
\ruby{嘘}{うそ}でも
\ruby{優}{やさ}しい
\ruby{顏}{かほ}を
\ruby{仕}{し}て
\ruby{吳}{く}れて
\ruby{居}{ゐ}るのに
\ruby{對}{むか}つちやあ、
%
\ruby{其樣}{そ|ん}な
\ruby{譯}{わけ}の
\ruby{有}{あ}る
\ruby{筈}{はず}は
\ruby{毫末}{ちつ|と}も
\ruby{無}{な}いんですが、
%
\ruby{何}{なん}だか
\ruby{彼家}{あす|こ}を
\ruby{出}{で}やうつて
\ruby{云}{い}ふのが
\ruby{我儘}{わが|まゝ}
\ruby{{\換字{過}}}{す}ぎる
\ruby{不人{\換字{情}}}{ふ|にん|じやう}のことのやうに
\ruby{思}{おも}はれてならないんですもの。
』

\原本頁{}%
『ホヽヽ、
%
\ruby{餘}{あんま}り
お
\ruby{{\換字{前}}}{まへ}は
\ruby[||j>]{性}{しやう}
\ruby[||j>]{{\換字{分}}}{ ぶん}が
% \ruby{性{\換字{分}}}{しやう|ぶん}が
\ruby{美麗}{き|れい}なものだから
\ruby{氣}{き}が
\ruby{{\換字{弱}}}{よわ}いねエ。
%
ぢやあ
\ruby{思}{おも}ひきつて
\ruby{特}{わざ}と
\ruby{冒頭}{のつ|け}から
\ruby{喧嘩}{けん|くわ}を
\ruby{仕}{し}たら
\ruby{何樣}{ど|う}だえ。
』

\原本頁{}%
『あら!、
%
\ruby{姊}{ねえ}さんはまあ
\ruby{甚}{ひど}い
\ruby{事}{こと}ねえ、
%
\ruby{喧嘩}{けん|くわ}つていふものは
\ruby{自然}{ひと|りで}に
\ruby{出來}{で|き}るものだのに、
%
わざと
\ruby{噴嘩}{けん|くわ}をするなんて、
%
そんな
\ruby{事}{こと}があるの?。
』

\原本頁{}%
『ホヽホヽヽ、
%
あゝ、
%
\ruby{有}{あ}るともサ。
%
\ruby{妾}{わたし}なんぞは
\ruby{仕馴}{し|な}れて
\ruby{居}{ゐ}る
\ruby{位}{くらゐ}だよ。
%
どうだえ、
%
\ruby{吃驚}{びつ|くり}
お
\ruby{仕}{し}かえ、
%
\ruby{人}{ひと}が
\ruby{惡}{わる}いだらうネエ。
』

\原本頁{}%
『ホヽヽ、
%
\ruby{眞實}{ほん|と}かと
\ruby{思}{おも}つて
\ruby{居}{ゐ}たら
\ruby[||j>]{戲}{じやう}
\ruby[||j>]{談}{ だん}ばつかり。
% \ruby{戲談}{じやう|だん}ばつかり。
』

\原本頁{}%
『イヽエ、
%
\ruby[||j>]{戲}{じやう}
\ruby[||j>]{談}{ だん}ぢやあ
% \ruby{戲談}{じやう|だん}ぢやあ
\ruby{無}{な}いよ、
%
\ruby{一寸}{ちよ|つと}
\ruby{行}{い}つておいでな。
%
\ruby{一人}{ひと|り}で
\ruby[||j>]{心}{こゝろ}
\ruby[||j>]{細}{ ぼそ}いなら
% \ruby{心細}{こゝろ|ぼそ}いなら
お
\ruby{富}{とみ}を
\ruby{付}{つ}けてあげやうはネ。
%
\ruby{年}{とし}は
\ruby{行}{ゆ}かないけれども
\ruby{大}{だい}のしつかり
\ruby{者}{もの}だから、
%
\ruby{彼女}{あ|れ}にすつかり
\ruby[||j>]{口}{こう}
\ruby[||j>]{上}{じやう}を
% \ruby{口上}{こう|じやう}を
\ruby{敎}{をし}へて
\ruby{{\換字{遣}}}{や}りましやう。
%
お
\ruby{{\換字{前}}}{まへ}が
\ruby{何}{なん}にも
\ruby{云}{い}はなくつても
\ruby{可}{い}いやうに。
』

\原本頁{}%
『まさか
\ruby{妾}{わたし}だつて
お
\ruby{富}{とみ}さんに
\ruby[||j>]{口}{こう}
\ruby[||j>]{上}{じやう}を
% \ruby{口上}{こう|じやう}を
\ruby{云}{い}つて
\ruby{貰}{もら}はなくつてもですが、
%
\ruby{眞實}{ほん|と}に
\ruby{何樣}{ど|う}しても
\ruby{行}{い}かなくつちやあ
\ruby{不可}{いけ|ない}のでしやうか?。
』

\原本頁{}%
\ruby{如何}{い|か}にも
\ruby{苦}{くる}しげに
お
\ruby{龍}{りう}は
\ruby{再}{ふたゝ}び
\ruby{{\換字{尋}}}{たづ}ぬれば、
%
お
\ruby{彤}{とう}も
\ruby{憐}{あはれ}みて
\ruby{一寸}{ちよ|つと}
\ruby{考}{かんが}しが、

\原本頁{}%
『お
\ruby{待}{ま}ちよ。
%
それほどお
\ruby{{\換字{前}}}{まへ}が
\ruby{困}{こま}るつて
\ruby{云}{い}ふのなら、
%
アヽ
\ruby{可}{い}いよ、
%
\ruby{仕方}{し|かた}が
\ruby{無}{な}い、
%
\ruby{手紙}{て|がみ}で
\ruby{云}{い}ふことに
お
\ruby{爲}{し}。
%
さうしたら
\ruby{向}{むかふ}から
\ruby{足}{あし}を
\ruby{{\換字{運}}}{はこ}んで
\ruby{來}{く}るだらう、
%
どうせ
\ruby{一度}{いち|ど}は
\ruby{膨}{ふく}れつ
\ruby{面}{つら}を
\ruby{持}{も}つて
\ruby{來}{く}るに
\ruby{定}{きま}つて
\ruby{居}{ゐ}るのだから。
』

\原本頁{}%
と
\ruby{負}{ま}けて
\ruby{答}{こた}へぬ。
%
\ruby{談話}{はな|し}は
\ruby{是}{これ}に
\ruby{{\換字{終}}}{をは}つて
お
\ruby{龍}{りう}は
\ruby{手紙}{て|がみ}を
\ruby{認}{したゝ}めはじめしが、
%
\ruby{三行書}{さん|ぎやう|か}きては
\ruby{破}{やぶ}り、
%
\ruby{五行書}{ご|ぎやう|か}きては
\ruby{丸}{まる}め、
%
\ruby{幾度}{いく|たび}と
\ruby{無}{な}く
\ruby{書}{か}き
\ruby{損}{そん}じたる
\ruby{後}{のち}やうやくと
\ruby{恐惶}{かし|く}まで
\ruby{纒}{まと}めて、
%
\ruby{先}{ま}づ
\ruby{初}{はじめ}に
\ruby{世話}{せ|わ}になりたる
\ruby{恩}{おん}を
\ruby{謝}{しや}し、
%
\ruby{次}{つぎ}
には
\ruby{田舎}{ゐな|か}
\ruby{氣質}{かた|ぎ}の
\ruby{叔母}{を|ば}の
\ruby{片意地}{かた|い|ぢ}なる
\ruby{指揮}{さし|づ}の
\ruby{負}{そむ}き
\ruby{{\換字{難}}}{がた}き
\ruby{由}{よし}を
\ruby{云}{い}ひ、
%
\ruby{扨}{さて}
\ruby{其}{その}
\ruby{後}{ゝち}に、
%
\ruby{我}{わ}が
\ruby{意}{こゝろ}よりの
\ruby{事}{こと}ならねども
\ruby{其方}{そち|ら}を
\ruby{離}{はな}れて
\ruby{此家}{こ|ゝ}に
\ruby{{\換字{留}}}{とゞ}まりあるやうになりたる
\ruby{趣}{おもむ}きを
\ruby{記}{しる}したりけり。

\原本頁{}%
\ruby{如何}{い|か}ばかり
\ruby{{\換字{文}}}{ふみ}の
\ruby{言葉}{こと|ば}は
\ruby{優}{やさ}しく
\ruby{書}{か}かれたりとも、
%
\ruby{吾}{わ}が
\ruby{物}{もの}と
\ruby{思}{おも}ひ
\ruby{{\換字{込}}}{こ}みたる
\ruby{禽}{とり}に
\ruby{他家}{よ|そ}の
\ruby{檐端}{のき|ば}で
\ruby{鳴}{な}かれては
\ruby{堪{\換字{忍}}}{が|まん}なり% 原文通り「堪忍」
\ruby{{\換字{難}}}{がた}く、
%
お
\ruby{關}{せき}
は
\ruby{慾}{よく}の
\ruby{算盤}{そろ|ばん}の
\ruby{置{\換字{違}}}{おき|ちが}ひとなりたるに
\ruby{手紙}{て|がみ}
\ruby{讀}{よ}む
\ruby{眼}{め}の
\ruby{玉}{たま}を
\ruby{頻々}{しき|り}とパチ〳〵させ
\ruby{居}{を}りしが、
%
やがて
\ruby{手紙}{て|がみ}を
\ruby{揉}{も}み
\ruby{丸}{まる}めて
\ruby{投礫}{つぶ|て}の
\ruby{如}{ごと}く
\ruby{投}{な}げ
\ruby{捨}{す}て、

\原本頁{}%
『
\ruby{彼女}{あい|つ}も
\ruby{彼女}{あい|つ}だが、
%
お
\ruby{彤}{とう}つて
\ruby{奴}{やつ}が
\ruby{忌々}{いま|〳〵}しい。
%
\ruby{誰}{たれ}が
\ruby{指}{ゆび}を
\ruby{啣}{くは}へて
\ruby{引{\換字{込}}}{ひつ|こ}む?。
%
\ruby{人}{ひと}を
\ruby{馬鹿}{ば|か}に
\ruby{仕}{し}あがる!。
』

\原本頁{}%
と
\ruby{男}{をとこ}のやうな
\ruby{言葉}{こと|ば}
\ruby{{\換字{遣}}}{づか}ひして
\ruby{獨}{ひと}り
\ruby{罵}{のゝし}りつ、
%
\ruby{紫色}{むらさき|いろ}になつて
\ruby{怒}{いか}り
\ruby{瞋}{いか}つたり。
