\Entry{其三十五}

% メモ 校正終了 2024-05-17 2024-06-13
\原本頁{195-6}%
お
\ruby{彤}{とう}が
\ruby[g]{{\換字{分}}別}{ふんべつ}に
\ruby{長}{た}けたる
\ruby{事}{こと}は
\ruby[|g|]{對談}{はなし}の
\ruby{中}{うち}にも
\ruby{知}{し}りしが、
%
\ruby{今}{いま}
\ruby{{\換字{又}}}{また}
\ruby{眼}{ま}のあたりに
\ruby{其}{そ}の
\ruby{胸}{むね}の
\ruby{廣}{ひろ}く
\ruby[|g|]{慈悲}{なさけ}の
\ruby{厚}{あつ}きをば
\ruby{見}{み}て、
%
\ruby[g]{隨{\換字{分}}}{ずゐぶん}
\ruby{負}{ま}けぬ
\ruby{氣}{き}の
お
\ruby{龍}{りう}の
\ruby[g]{叔母}{を ば }も
\ruby{全}{まつた}く
\ruby{我}{が}を
\ruby{折}{を}り
\ruby{盡}{つ}くして、
%
\ruby{好}{よ}いと
\ruby{思}{おも}ひ
\ruby{{\換字{込}}}{こ}めば
\ruby[g]{何處}{ど こ }までも
\ruby{好}{よ}いに
\ruby{仕}{し}て
\ruby{{\換字{終}}}{しま}ふ
\ruby[|g|]{田舎}{ゐなか}
\ruby{氣}{ぎ}の
\ruby[<j||]{正}{しやう}
\ruby[<j||]{直}{ぢき}
% \ruby{正直}{しやう|ぢき}
\ruby[g]{三昧}{ざんまい}に、
%
\ruby{此}{こ}の
\ruby{人}{ひと}にさへ
\ruby{頼}{たの}み
\ruby{置}{お}けば
\ruby[g]{何樣}{ど う }
\ruby{轉}{ころ}んでも
\ruby[g]{間{\換字{違}}}{まちがひ}
\ruby{無}{な}しと
\ruby[<j>]{盡}{こと〴〵}く
\ruby{信}{しん}じて、
%
\ruby[g]{何{\換字{分}}}{なにぶん}
\ruby{宜}{よろ}しく
\ruby{願}{ねが}ひまするを
\ruby[||j>]{百}{ひやつ}% 「遍(ぺん)」となっているので(ひやつ)とした
\ruby[||j>]{{\換字{遍}}}{ ぺん}ほども
% \ruby{百{\換字{遍}}}{ひやく|ぺん}ほども
\ruby{云}{い}ひたる
\ruby{末}{すゑ}、
%
\ruby[g]{何事}{なにごと}も
お
\ruby{彤}{とう}
\ruby{任}{まか}せにして
\ruby{其}{その}
\ruby{次}{つぎ}の
\ruby{日}{ひ}に
\ruby[g]{靜岡}{しづをか}へ
\ruby{歸}{かへ}りぬ。

\原本頁{196-3}%
『
お
\ruby{龍}{りう}ちやん、
%
お
\ruby{{\換字{前}}}{まへ}
\ruby[g]{一寸}{ちよつと}
\ruby{今}{いま}までの
\ruby[g]{居處}{う ち }へ
\ruby{歸}{かへ}つてネ、
%
\ruby[g]{叔母}{を ば }の
いひつけで
\ruby[g]{今後}{これから}
これ〳〵
の
ところに
\ruby{居}{ゐ}る
やうになつた
といふ
\ruby{事}{こと}だけを
\ruby{斷}{ことわ}つて
おいでな。
』

\原本頁{196-6}%
\ruby[g]{叔母}{を ば }の
\ruby[|g|]{歸郷}{かへり}を
\ruby[<j||]{停}{てい }
\ruby[<j||]{車}{しやう}% 「車」は(しや)でなく原本通り(しやう)
\ruby[||j>]{場}{じやう}% 原文通り「場」
まで
\ruby{{\換字{送}}}{おく}つての
\ruby{後}{のち}、
%
\ruby{何}{なに}を
\ruby{思}{おも}ふにや
\ruby[g]{茫然}{ばうぜん}として
\原本頁{196-7}\改行%
\ruby{爲}{な}す
\ruby{事}{こと}も
\ruby{無}{な}く
\ruby{居}{ゐ}たる
お
\ruby{龍}{りう}に
\ruby{向}{むか}つて
お
\ruby{彤}{とう}は
かくの
\ruby{如}{ごと}く
\ruby{云}{い}ひ
\ruby{出}{いだ}したり。
%
お
\ruby{龍}{りう}は
\ruby[g]{{\換字{迷}}惑}{めいわく}さうに
\ruby[g]{眉根}{まゆね }を
\ruby{寄}{よ}せながら、
%
\ruby{何}{なん}の
\ruby[g]{思案}{し あん}も
\ruby{無}{な}く、

\原本頁{196-9}%
『
\ruby{行}{い}かなくつちやあ
いけませんかネ、
%
ネエ
\ruby{行}{い}かなくつちやあ。
』

\原本頁{196-10}%
と、
%
\ruby{然}{さ}も〳〵
\ruby{其}{そ}の
\ruby{事}{こと}の
\ruby[|g|]{宥免}{ゆるし}を
\ruby{乞}{こ}ふが
\ruby{如}{ごと}くに
\ruby{云}{い}へり。

\原本頁{196-11}%
『
ホヽヽ、
%
\ruby{{\換字{嫌}}}{いや}なの?\inhibitglue{}%
\ruby[|g|]{其樣}{そんな}に。
%
\ruby{怖}{こは}いやうに
でも
\ruby{思}{おも}つて?。
』

\原本頁{197-1}%
『
\ruby{怖}{こは}いつて
\ruby{事}{こと}は
\ruby{有}{あ}りません
けれどもネ、
%
\ruby[g]{今日}{け ふ }つから
\ruby[g]{御暇}{おいとま}を
\ruby{致}{いた}します、
%
\ruby[g]{左樣}{さ やう}なら
つて
\ruby{云}{い}ふのが
\ruby{何}{なん}だか
\ruby{云}{い}ひ
づらい
やうな
\ruby[||j>]{心}{こゝろ}
\ruby[||j>]{持}{ もち}
% \ruby{心持}{こゝろ|もち}
が
するんです
もの。
』

\原本頁{197-4}%
『
だつて
\ruby{何}{なに}も
お
\ruby{{\換字{前}}}{まへ}が
\ruby{不義理}{ふ|ぎ|り}なことを
\ruby{爲}{す}るつて
\ruby{云}{い}ふのぢやあ
\ruby{無}{な}し、
%
お
\ruby{{\換字{前}}}{まへ}にも
\ruby{{\換字{分}}}{わか}つて
\ruby{居}{ゐ}る
とほり
\ruby[|g|]{先方}{むかふ}の
お
\ruby{腹}{なか}の
\ruby{中}{なか}が
\ruby{良}{よ}くないん
だから
\改行% 校正作業の簡略化のため
、
%
\原本頁{197-6}\改行%
ことわりを
\ruby{云}{い}ふだけの
\ruby{事}{こと}に
\ruby{譯}{わけ}は
\ruby{無}{な}いぢやあ
\ruby{無}{な}いか。
』

\原本頁{197-7}%
『
そりやあ、
%
\ruby[g]{理屈}{り くつ}は、
%
もう
ほんとに
\ruby[g]{其{\換字{通}}}{そのとほ}り
なんです
けれども。
』

\原本頁{197-8}%
『
ぢやあ、
%
また、
%
\ruby[g]{何故}{な ぜ }ネエ?。
』

\原本頁{197-9}%
『
\ruby{何}{なん}だか
\ruby{妾}{わたし}にも
\ruby[g]{理由}{わ け }は
\ruby{{\換字{分}}}{わか}りません
けども、
%
\ruby{妾}{わたし}にやあ
\ruby[g]{{\換字{判}}然}{はつきり}と
\ruby{斷}{ことわ}り
が
\ruby{云}{い}へさうも
\ruby{無}{な}いん
ですもの!。
%
\ruby{心}{しん}は
ほんとに
\ruby[g]{可厭}{い や }な
\ruby{人}{ひと}です
けれども、
%
\ruby[|g|]{表面}{うはべ}
だけに
しろ
お
\ruby{龍}{りう}〳〵つて
\ruby[g]{可愛}{か はい}がつて
\ruby{吳}{く}れまして、
%
\原本頁{198-1}\改行%
\ruby[g]{斯樣}{か う }やつて
\ruby[|g|]{衣類}{きもの}も
\ruby{着}{き}せて
\ruby{吳}{く}れますし、
%
\ruby[|g|]{一個}{ひとつ}
あるものも
\ruby[g]{{\換字{半}}{\換字{分}}}{はんぶん}は
\ruby{取}{と}り
\ruby{{\換字{分}}}{わ}けて
\ruby{吳}{く}れるやうに
\ruby[g]{始{\換字{終}}}{し じう}% ルビ調整(原本通り)「ゆ」無し
\ruby{爲}{さ}れて
\ruby{居}{ゐ}るん
ですから、
%
いつそ
\ruby[g]{惡口}{あくたい}でも
\ruby{云}{い}はれて
\ruby[g]{喧嘩}{けんくわ}でも
\ruby{仕}{し}たら
\ruby{妾}{わたし}の
\ruby{胸}{むね}の
\ruby{中}{なか}を
\ruby{有}{あ}り
\ruby{體}{てい}に
\ruby{云}{い}ひ
\ruby{出}{だ}す
\ruby{事}{こと}も
\ruby[g]{出來}{で き }るか
\ruby{知}{し}れませんけど、
%
\ruby{嘘}{うそ}でも
\ruby{優}{やさ}しい
\ruby{顏}{かほ}を
\ruby{仕}{し}て
\ruby{吳}{く}れて
\ruby{居}{ゐ}るのに
\ruby{對}{むか}つちやあ、
%
\ruby[g]{其樣}{そ ん }な
\ruby{譯}{わけ}の
\ruby{有}{あ}る
\ruby{筈}{はず}は
\ruby[|g|]{毫末}{ちつと}も
\ruby{無}{な}いんですが、
%
\ruby{何}{なん}だか
\ruby[|g|]{彼家}{あすこ}を
\ruby{出}{で}やうつて
\ruby{云}{い}ふのが
\ruby[g]{我儘}{わがまゝ}
\ruby{{\換字{過}}}{す}ぎる
\ruby{不}{ふ}
\ruby[||j>]{人}{にん}
\ruby[||j>]{{\換字{情}}}{じやう}
のことのやうに
\ruby{思}{おも}はれて
ならないん
ですもの。
』

\原本頁{198-8}%
『
ホヽヽ、
%
\ruby{餘}{あんま}り
お
\ruby{{\換字{前}}}{まへ}は
\ruby[||j>]{性}{しやう}
\ruby[||j>]{{\換字{分}}}{ ぶん}が
% \ruby{性{\換字{分}}}{しやう|ぶん}が
\ruby[g]{美麗}{き れい}な
もの
だから
\ruby{氣}{き}が
\ruby{{\換字{弱}}}{よわ}いねエ。
%
ぢやあ
\ruby{思}{おも}ひきつて
\ruby{特}{わざ}と
\ruby[|g|]{冒頭}{のつけ}から
\ruby[g]{喧嘩}{けんくわ}を
\ruby{仕}{し}たら
\ruby[g]{何樣}{ど う }だえ。
』

\原本頁{198-10}%
『
あら!、
%
\ruby{姊}{ねえ}さんは
まあ
\ruby{甚}{ひど}い
\ruby{事}{こと}ねえ、
%
\ruby[g]{喧嘩}{けんくわ}つて
いふものは
\ruby[g]{自然}{ひとりで}
に
\ruby[g]{出來}{で き }る
ものだのに、
%
わざと
\ruby[g]{噴嘩}{けんくわ}
を
するなんて、
%
そんな
\ruby{事}{こと}があるの?。
』

\原本頁{199-2}%
『
ホヽホヽヽ、
%
あゝ、
%
\ruby{有}{あ}るともサ。
%
\ruby{妾}{わたし}
なんぞは
\ruby[g]{仕馴}{し な }れて
\ruby{居}{ゐ}る
\ruby[<j||]{位}{くらゐ}だよ。% 行末行頭の境界付近なので特例処置を施す
%
どうだえ、
%
\ruby[g]{吃驚}{びつくり}
お
\ruby{仕}{し}かえ、
%
\ruby{人}{ひと}が
\ruby{惡}{わる}いだらうネエ。
』

\原本頁{199-4}%
『
ホヽヽ、
%
\ruby[|g|]{眞實}{ほんと}かと
\ruby{思}{おも}つて
\ruby{居}{ゐ}たら
\ruby[||j>]{戲}{じやう}
\ruby[||j>]{談}{ だん}
% \ruby{戲談}{じやう|だん}
ばつかり。
』

\原本頁{199-5}%
『
イヽエ、
%
\ruby[||j>]{戲}{じやう}
\ruby[||j>]{談}{ だん}
% \ruby{戲談}{じやう|だん}
ぢやあ
\ruby{無}{な}いよ、
%
\ruby[g]{一寸}{ちよつと}
\ruby{行}{い}つておいでな。
%
\ruby[|g|]{一人}{ひとり}で
\ruby[<j||]{心}{こゝろ}% 行末行頭の境界付近なので特例処置を施す
\ruby[<j||]{細}{ぼそ}いなら
% \ruby{心細}{こゝろ|ぼそ}いなら
お
\ruby{富}{とみ}を
\ruby{付}{つ}けてあげやうはネ。
%
\ruby{年}{とし}は
\ruby{行}{ゆ}かない
けれども
\ruby{大}{だい}の
\換字{志}つかり
\ruby{者}{もの}だから、
%
\ruby[g]{彼女}{あ れ }に
すつかり
\ruby[||j>]{口}{こう}
\ruby[||j>]{上}{じやう}を
% \ruby{口上}{こう|じやう}を
\ruby{敎}{をし}へて
\ruby{{\換字{遣}}}{や}りましやう。
%
お
\ruby{{\換字{前}}}{まへ}が
\ruby{何}{なん}にも
\ruby{云}{い}はなくつても
\ruby{可}{い}いやうに。
』

\原本頁{199-9}%
『
まさか
\ruby{妾}{わたし}だつて
お
\ruby{富}{とみ}さんに
\ruby[||j>]{口}{こう}
\ruby[||j>]{上}{じやう}を
% \ruby{口上}{こう|じやう}を
\ruby{云}{い}つて
\ruby{貰}{もら}はなくつてもですが、
%
\ruby[|g|]{眞實}{ほんと}に
\ruby[g]{何樣}{ど う }しても
\ruby{行}{い}かなくつちやあ
\ruby[g]{不可}{いけない}のでしやうか?。
』

\原本頁{199-11}%
\ruby[g]{如何}{い か }にも
\ruby{苦}{くる}しげに
お
\ruby{龍}{りう}は
\ruby{再}{ふたゝ}び
\ruby{{\換字{尋}}}{たづ}ぬれば、
%
お
\ruby{彤}{とう}も
\ruby{憐}{あはれ}みて
\ruby[g]{一寸}{ちよつと}
\ruby{考}{かんが}へ
\改行% 校正作業の簡略化のため
しが、

\原本頁{200-2}%
『
お
\ruby{待}{ま}ちよ。
%
それほど
お
\ruby{{\換字{前}}}{まへ}が
\ruby{困}{こま}るつて
\ruby{云}{い}ふのなら、
%
アヽ
\ruby{可}{い}いよ、
%
\ruby[g]{仕方}{し かた}が
\ruby{無}{な}い、
%
\ruby[g]{手紙}{て がみ}で
\ruby{云}{い}ふことに
お
\ruby{爲}{し}。
%
さうしたら
\ruby{向}{むかふ}から
\ruby{足}{あし}を
\ruby{{\換字{運}}}{はこ}んで
\ruby{來}{く}る
だらう、
%
どうせ
\ruby[g]{一度}{いちど }は
\ruby{膨}{ふく}れつ
\ruby{面}{つら}を
\ruby{持}{も}つて
\ruby{來}{く}るに
\ruby{定}{きま}つて
\原本頁{200-5}\改行%
\ruby{居}{ゐ}るのだから。
』

\原本頁{200-6}%
と
\ruby{負}{ま}けて
\ruby{答}{こた}へぬ。
%
\ruby[|g|]{談話}{はなし}は
\ruby{是}{これ}に
\ruby{{\換字{終}}}{をは}つて
お
\ruby{龍}{りう}は
\ruby[g]{手紙}{て がみ}を
\ruby{認}{したゝ}め
はじめしが、
%
\ruby{三行書}{さん|ぎやう|か}きては
\ruby{破}{やぶ}り、
%
\ruby{五}{ご}
\ruby[g]{行書}{ぎやうか}きては
\ruby{丸}{まる}め、
%
\ruby[g]{幾度}{いくたび}と
\ruby{無}{な}く
\ruby{書}{か}き
\ruby{損}{そん}じたる
\ruby{後}{のち}
やうやくと
\ruby[|g|]{恐惶}{かしく}まで
\ruby{纒}{まと}めて、
%
\ruby{先}{ま}づ
\ruby{初}{はじめ}に
\ruby[g]{世話}{せ わ }になりたる
\ruby{恩}{おん}を
\ruby{謝}{しや}し、
%
\ruby{次}{つぎ}
には
\ruby[|g|]{田舎}{ゐなか}
\ruby[|g|]{氣質}{かたぎ}の
\ruby[g]{叔母}{を ば }の
\ruby{片意地}{かた|い|ぢ}なる
\ruby[g]{指揮}{さしづ }の
\ruby{負}{そむ}き
\ruby{{\換字{難}}}{がた}き
\原本頁{200-10}\改行%
\ruby{由}{よし}を
\ruby{云}{い}ひ、
%
\ruby{扨}{さて}
\ruby{其}{その}
\ruby{後}{のち}に、% ルビ調整(原本通り)非踊り字表記
%
\ruby{我}{わ}が
\ruby{意}{こゝろ}よりの
\ruby{事}{こと}ならねども
\ruby[|g|]{其方}{そちら}を
\ruby{離}{はな}れて
\ruby[g]{此家}{こ こ }に% ルビ調整(原本通り)非踊り字表記
\ruby{{\換字{留}}}{とど}まり% ルビ調整(原本通り)非踊り字表記
あるやうに
なりたる
\ruby{趣}{おもむ}きを
\ruby{記}{しる}したりけり。

\原本頁{201-1}%
\ruby[g]{如何}{い か }
ばかり
\ruby{{\換字{文}}}{ふみ}の
\ruby[g]{言葉}{ことば }は
\ruby{優}{やさ}しく
\ruby{書}{か}かれたりとも、
%
\ruby{吾}{わ}が
\ruby{物}{もの}と
\ruby{思}{おも}ひ
\ruby{{\換字{込}}}{こ}みたる
\ruby{禽}{とり}に
\ruby[g]{他家}{よ そ }の
\ruby[g]{檐端}{のきば }で
\ruby{鳴}{な}かれては
\ruby[|g|]{堪{\換字{忍}}}{がまん}% 原文通り「堪忍」
なり
\ruby{{\換字{難}}}{がた}く、
%
お
\ruby{關}{せき}
は
\ruby{慾}{よく}の
\ruby[g]{算盤}{そろばん}の
\ruby[g]{置{\換字{違}}}{おきちが}ひ
と
なりたるに
\ruby[g]{手紙}{て がみ}
\ruby{讀}{よ}む
\ruby{眼}{め}の
\ruby{玉}{たま}を
\ruby[|g|]{頻々}{しきり}
と
パチ〳〵させ
\ruby{居}{を}りしが、
%
やがて
\ruby[g]{手紙}{て がみ}を
\ruby{揉}{も}み
\ruby{丸}{まる}めて
\ruby[|g|]{投礫}{つぶて}の
\ruby{如}{ごと}く
\ruby{投}{な}げ
\ruby{捨}{す}て、

\原本頁{201-5}%
『
\ruby[|g|]{彼女}{あいつ}も
\ruby[|g|]{彼女}{あいつ}だが、
%
お
\ruby{彤}{とう}つて
\ruby{奴}{やつ}が
\ruby[g]{忌々}{いま〳〵}しい。
%
\ruby{誰}{だれ}が
\ruby{指}{ゆび}を
\ruby{{\換字{啣}}}{くは}へて
\ruby[g]{引{\換字{込}}}{ひつこ }む?。
%
\ruby{人}{ひと}を
\ruby[g]{馬鹿}{ば か }に
\ruby{仕}{し}あがる!。
』

\原本頁{201-7}%
と
\ruby{男}{をとこ}のやうな
\ruby[g]{言葉}{ことば }
\ruby{{\換字{遣}}}{づか}ひして
\ruby{獨}{ひと}り
\ruby{罵}{のゝし}りつ、
%
\ruby[<j>]{紫}{むらさき}
\ruby[||j>]{色}{ いろ}に
なつて
\ruby{怒}{いか}り
\ruby{瞋}{いか}つたり。
