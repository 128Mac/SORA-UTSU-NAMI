\Entry{其三十六}

% メモ 校正終了 2024-05-17
\原本頁{201-10}%
『
えゝ、
%
ぢれつたいネ、
%
\ruby[|g|]{{\換字{煙}}草}{たばこ}
\ruby{一}{ひと}つ
\ruby{入}{い}れるのに
\ruby{何}{なに}を
\ruby[|g|]{其樣}{そんな}に
\ruby{愚圖愚圖}{ぐ|づ|ぐ|づ}して% 行末行頭禁則につき非踊り字表記
\ruby{居}{ゐ}るのだえ。
%
\ruby[|g|]{百足}{むかで}に
\ruby{足袋}{た|び}でも
\ruby{穿}{は}かせや
\ruby{仕}{し}まいし、
%
\ruby{宜}{い}い
\ruby{加減}{か|げん}に
\ruby[|g|]{早{\換字{速}}}{さつさ}と
\ruby{仕}{し}て
お
\ruby{吳}{く}れな。
』

\原本頁{202-3}%
\ruby{樫貪}{けん|どん}
\ruby{聲}{ごゑ}に
\ruby{罵}{のゝし}り
ながら、
%
\ruby{腹立}{はら|だ}ち
\ruby{{\換字{紛}}}{まぎ}れの
\ruby{力}{ちから}を
\ruby{籠}{こ}めて
ぎうと
\ruby{吾}{わ}が
\ruby{帶}{おび}を
\ruby{緊}{きつ}く
\ruby{締}{し}め、
%
\ruby{{\換字{猶}}}{なほ}
\ruby{帶揚}{おび|あげ}を
\ruby{締}{し}め、
%
\ruby{帶{\換字{留}}}{おび|どめ}を
\ruby{締}{し}むる
\ruby{時}{とき}、
%
\ruby{小婢}{こ|をんな}の
お
\ruby{熊}{くま}が
\ruby{馴}{な}れぬ
\ruby{手}{て}つきの
たど〳〵しく
\ruby{漸}{やうや}くにして
\ruby[|g|]{{\換字{煙}}草}{たばこ}を
\ruby{詰}{つ}めて
\ruby{差}{さ}し
\ruby{出}{いだ}す
\ruby[|g|]{{\換字{煙}}草}{たばこ}
\ruby{袋}{いれ}を
\ruby{引奪}{ひつ|たく}るやうに
\ruby{取}{と}つて
ばた〳〵と
\ruby{拂}{はた}き、

\原本頁{202-7}%
『
\ruby{仕}{し}やうが
\ruby{無}{な}いねえ、
%
\ruby[|g|]{此樣}{こんな}に
\ruby{外部}{そ|と}に
\ruby[|g|]{{\換字{煙}}草}{たばこ}を
くつつけちやあ。
%
まるで
\ruby{毛}{け}が
\ruby{生}{は}えた
やう
ぢやあ
\ruby{無}{な}いか。
%
フツフツフツ。
』

\原本頁{202-9}%
と
\ruby{吹}{ふ}けば、
%
\ruby[|g|]{{\換字{煙}}草}{たばこ}の
\ruby{{\換字{粉}}}{こな}は
\ruby{{\換字{空}}}{くう}に
\ruby{飛}{と}び
\ruby{飛}{と}んで、
%
うつかりと
\ruby{仰向}{あふ|む}いて、
%
\ruby{頻}{しき}りに
\ruby{怒}{いか}り
\ruby{立}{た}つ
\ruby{主人}{しゆ|じん}の
\ruby{面}{おもて}を
\ruby{訝}{いぶか}り
\ruby{呆}{あき}れ
ながら
\ruby{視}{み}
\ruby{居}{ゐ}たりし
お
\ruby{熊}{くま}が
\ruby{小}{ちひ}さき
\ruby{金壺}{かな|つぼ}
\ruby[||j>]{眼}{まなこ}に
むざんや
\ruby{舞}{ま}ひ
\ruby{入}{い}りたり。

\原本頁{203-1}%
『
アツ、
%
アヽ
\ruby{痛}{いた}い!。
%
あんまりだこと!。
』

\原本頁{203-2}%
\ruby{思}{おも}はず
\ruby{叫}{さけ}びて
\ruby{眼}{め}を
\ruby{抑}{おさ}へ、
%
\ruby{泣}{な}きながら
お
\ruby{熊}{くま}の
\ruby{俯伏}{うつ|ぶ}すを、
%
\ruby{愍}{あはれ}み
\ruby{氣}{げ}も
\ruby{無}{な}く
\ruby{見}{み}
\ruby{下}{おろ}して
\ruby{却}{かへ}つて
\ruby{冷笑}{あざ|わら}ひ、

\原本頁{203-4}%
『
\ruby{下}{くだ}らなく
\ruby{汝}{おまへ}が
ぽかんと
\ruby{仕}{し}て
\ruby{居}{ゐ}る
からだアネ。
%
\ruby{妾}{わたし}の
\ruby{知}{し}つた
\ruby{事}{こと}ぢやあ
\ruby{無}{な}いよ。
%
\ruby{痛}{いた}い
つても
\ruby{火}{ひ}が
\ruby{入}{はい}つた
\ruby{程}{ほど}ぢやあ
\ruby{有}{あ}るまいから、
%
\ruby{其樣}{そ|ん}なに
\ruby{泣}{な}く
\ruby{事}{こと}は
\ruby{無}{な}いやネ。
%
さあ
\ruby{下駄}{げ|た}を
\ruby{出}{だ}して
おくれ。
%
えゝ
うぢうぢして% 行末行頭禁則につき原本通り非踊り字表記
\ruby{居}{ゐ}るネ、
%
\ruby{{\換字{分}}}{わか}らない!、
%
\ruby[|g|]{跣足}{はだし}ぢやあ
\ruby{出}{で}られ
\ruby{無}{な}いぢや
\ruby{無}{な}いか。
%
\ruby{一々}{いち|〳〵}
\ruby{此樣}{こ|ん}な
\ruby{事}{こと}までも、
%
ソレ〳〵と
\ruby{云}{い}はれなくつちやあ
\ruby{{\換字{分}}}{わか}らないかえ、
%
\ruby{困}{こま}つた
\ruby{人}{ひと}だネエ。
%
チヨツ、
%
いつまで
\ruby{{\換字{半}}間}{はん|ま}な
\ruby{顏}{かほ}を
\ruby{仕}{し}て
\ruby{泣}{な}いて
\ruby{居}{ゐ}るんだネ、
%
\ruby[||j>]{鼠}{ねずみ}
\ruby[||j>]{色}{ いろ}の
% \ruby{鼠色}{ねずみ|いろ}の
\ruby{涙}{なみだ}なんか
\ruby{零}{こぼ}して。
%
\ruby[|g|]{火傷}{やけど}へ
\ruby{唐辛子}{たう|がら|し}
\ruby{味噌}{み|そ}を
つけられた
\ruby{狸}{たぬき}に
\ruby{其樣}{そ|ん}な
\ruby{顏}{かほ}を
\ruby{仕}{し}て
\ruby{居}{ゐ}るのが
\ruby{有}{あ}つたつけ。
』

\原本頁{204-1}%
と、
%
\ruby{自己}{う|ぬ}が
\ruby[<j||]{煩}{もしや}
\ruby[||j>]{悶}{くしや}の
\ruby{八}{や}ツあたりに
\ruby{口}{くち}ぎたなく
\ruby{叱}{しか}り
\ruby{嘲}{あざけ}れば、
%
\ruby{惡口}{あく|こう}を
\ruby{{\換字{浴}}}{あび}せらるゝには
\ruby{既}{はや}
\ruby{慣}{な}れたる
お
\ruby{熊}{くま}も
\ruby{膨}{ふく}れ
\ruby{{\換字{返}}}{かへ}つて、
%
\ruby{色}{いろ}
\ruby{黑}{くろ}き
\ruby{小}{ちひさ}き
\ruby{身體}{から|だ}を
プリ〳〵
と
させつ、
%
いと
\ruby{狹}{せま}き
\ruby{額越}{ひたひ|ご}しに
\ruby{恨}{うら}みの
\ruby{眼}{め}を
\ruby{{\換字{遣}}}{や}りて、
%
\ruby{言葉}{こと|ば}
\ruby{無}{な}くプイと
\ruby{立上}{たち|あが}り、
%
\ruby{疊}{たゝみ}に
\ruby{躓}{つまづ}ける
やうに
\ruby{歩}{ある}いて
\ruby{出口}{で|ぐち}の
\ruby{方}{はう}に
\ruby{至}{いた}り、
%
がたり
びしりと
\ruby{物音}{もの|おと}
\ruby{荒}{あら}く
\ruby{下駄箱}{げ|た|ばこ}に
\ruby{當}{あた}り
\ruby{散}{ち}らしたり。

\原本頁{204-6}%
『
ぢやあ
\ruby{一寸}{ちよ|つと}
\ruby{往}{い}つて
\ruby{來}{く}るから
\ruby{氣}{き}をつけて
\ruby{居}{ゐ}なくちやあ
\ruby{不可}{いけ|ない}よ。
%
\原本頁{204-7}\改行%
オヤ、
%
\ruby{狸}{たぬき}さん、
%
\ruby{怒}{おこ}つて
\ruby{膨}{ふく}れて
おいでだネ。
%
\ruby{怒}{おこ}つてりやあ
\ruby{睡}{ねむ}く
ならないから
\ruby{其}{それ}も
\ruby{宜}{い}いだらう。
%
\ruby{{\換字{留}}守番}{る|す|ばん}が
\ruby{性}{しやう}も
\ruby{無}{な}く
\ruby{坐睡}{ゐね|むり}を
\ruby{仕}{し}て、
%
\ruby{魂魄}{たま|しひ}が
\ruby{鼻}{はな}の
\ruby{穴}{あな}から
\ruby{獅子}{し|し}の
\ruby{洞}{ほら}
\ruby{入}{い}り
\ruby{洞}{ほら}
\ruby{{\換字{還}}}{がへ}り
なんかを
\ruby{仕}{し}て
\ruby{居}{ゐ}られるよりやあ、
%
\ruby{其}{そ}の
\ruby{方}{はう}が
\ruby{優}{まし}らしいから。
%
ハヽヽ、
%
ぢやあ
\ruby{頼}{たの}むよ
\ruby{御{\換字{留}}守番}{お|る|す|ばん}、
%
\ruby{好}{い}い
\ruby[|g|]{御土產}{おみやげ}を
\ruby{買}{か}つて
\ruby{來}{こ}やうネヱ。
』

\原本頁{205-1}%
\ruby{纔}{わづか}に
\ruby{胸}{むね}の
\ruby{中}{なか}の
\ruby{鬱々}{もや|くや}を
\ruby{洩}{もら}すか、
%
\ruby{益}{えき}も
\ruby{無}{な}い
\ruby{惡口}{あく|たい}に
\ruby{目下}{め|した}を
\ruby{嬲}{なぶ}つて
お
\ruby{關}{せき}は
\ruby{出}{い}で
\ruby{去}{さ}れば、
%
\ruby{主}{しゆ}を
\ruby{{\換字{送}}}{おく}り
\ruby{出}{だ}して
\ruby{後}{あと}に
\ruby{殘}{のこ}りし
お
\ruby{熊}{くま}は、
\ruby{室}{へや}の
\ruby{眞中}{まん|なか}に
\原本頁{205-3}\改行%
\ruby{取}{と}り
\ruby{散}{ち}らされたる
\ruby{主人}{しゆ|じん}の
\ruby{脫}{ぬぎ}つからしをば
\ruby{片付}{かた|づ}くるとて、
%
\ruby{其}{そ}の
\ruby{片手}{かた|て}に
\ruby{衣紋竹}{え|もん|だけ}を
\ruby{持}{も}ちたれども
\ruby{片手}{かた|て}は
\ruby{{\換字{更}}}{さら}に
\ruby{使}{つか}はで、
%
\ruby{足}{あし}の
\ruby{先}{さき}に
\ruby{幾度}{いく|たび}か
\ruby[|g|]{衣類}{きもの}を
\ruby{蹴{\換字{返}}}{け|かへ}し
\ruby{蹴{\換字{返}}}{け|かへ}しつ、
%
\ruby{{\換字{終}}}{つひ}に
\ruby{片手業}{かた|て|わざ}に
\ruby{衣紋竹}{え|もん|だけ}に
\ruby{引掛}{ひつ|か}けて
\ruby{壁}{かべ}に
\ruby{掛}{か}けたりしが、
%
たま〳〵
\ruby{催}{もよほ}したる
\ruby{噴嚏}{くし|やみ}を
\ruby{{\換字{遠}}慮}{ゑん|りよ}も
\ruby{無}{な}く
\ruby{大}{おほ}きくして、

\原本頁{205-7}%
『
ハツクシヨーン。
』

\原本頁{205-8}%
と
\ruby{特}{こと}さらに
\ruby{我}{わ}が
\ruby{顏}{かほ}を
\ruby{今}{いま}
\ruby{掛}{か}けたる
\ruby[|g|]{衣類}{きもの}の
\ruby{胴}{どう}の
あたりに
\ruby{持}{も}ち
\ruby{行}{ゆ}きつ、
\換字{志}たゝかに
\ruby{汚}{きたな}き
\ruby[|g|]{唾液}{つばき}の
\ruby{霧}{きり}を
\ruby{注}{そゝ}ぐが
\ruby{如}{ごと}く
\ruby{噴}{ふ}き
\ruby{掛}{か}けぬ。

\原本頁{205-10}%
\ruby{土瓶}{ど|びん}の
\ruby{底}{そこ}を
\ruby{拔}{ぬ}き、
%
\ruby{桶}{をけ}の
\ruby{箍}{たが}を
はじけさするなど、
%
\ruby{下司}{げ|す}の
\ruby{復讎}{しか|へし}は
\ruby{都}{すべ}て
\ruby{陰}{かげ}でする
\ruby{{\換字{習}}}{なら}ひなれば、
%
それより
お
\ruby{熊}{くま}の
\ruby{{\換字{戸}}棚}{と|だな}
\ruby{捜}{さが}し
\ruby{仕}{し}て、
%
\ruby{白砂糠}{しろ|ざ|たう}を
\ruby{舐}{な}め、
%
\ruby{奈良漬}{な|ら|づけ}を
\ruby{荒}{あら}し、
%
\ruby{自己}{お|の}が
\ruby{嗜}{す}きなものは
\ruby{暴}{あば}れ
\ruby{食}{ぐひ}して、
%
\ruby{蓋物}{ふた|もの}の
\ruby{蓋}{ふた}を
\ruby{除}{と}つて
\ruby{自己}{お|の}が
\ruby{好}{す}かぬ
\ruby{鹽辛}{しほ|から}
なんぞに
\ruby{{\換字{遇}}}{あ}へば
\ruby[|g|]{唾液}{つばき}を
\ruby{仕{\換字{込}}}{し|こん}で
\ruby{掻}{か}き
\ruby{{\換字{廻}}}{まは}し
\ruby{置}{お}くやうの
\ruby{事}{こと}を
\ruby{仕居}{し|ゐ}るとも
\ruby{知}{し}らず、
%
お
\ruby{關}{せき}は
\ruby[<j>]{勢}{いきほひ}
\ruby[||j>]{{\換字{込}}}{ こ}んで
お
\ruby{彤}{とう}が% 原本通りルビは(う)抜き出会ったが補正
\ruby{家}{いへ}を
\ruby{{\換字{尋}}}{たづ}ねたり。

\原本頁{206-5}%
\ruby{便利}{べん|り}なる
\ruby{場處}{ば|しよ}の% 原文通り「場」
\ruby{聊}{いさゝ}か
\ruby{引{\換字{退}}}{ひつ|こ}んで
\ruby{靜}{しづか}なる
ところに、
%
すべて
\ruby{金子}{か|ね}の
かかりたる
\ruby{{\換字{造}}}{つく}りの、
%
\ruby{見}{み}るから
\ruby{知}{し}らるゝ
\ruby{其}{そ}の
\ruby{贅澤}{ぜい|たく}さの
\ruby{小憎}{こ|にく}らしき
\ruby{家}{いへ}を、
%
\ruby{此家}{こ|ゝ}と
\ruby{{\換字{尋}}}{たづ}ね
\ruby{得}{え}て
お
\ruby{關}{せき}の
\ruby[|g|]{訪問}{おとな}へば、
%
\ruby{折}{をり}から
\ruby{此}{こ}の
むづかしい
\ruby{世}{よ}を
\ruby{餘{\換字{所}}}{よ|そ}にして、
%
\ruby{此{\換字{所}}}{こ|ゝ}は
\ruby{日}{ひ}の
\ruby{短}{みじか}い
\ruby{盛}{さか}りをも
\ruby{長}{なが}く
\ruby{暮}{くら}すやうなる
\ruby[|g|]{長閑}{のどか}さを
\ruby{現}{あらは}す
\ruby{賑}{にぎ}やかなる
\ruby{手物}{て|もの}の
\ruby{撥音}{ばち|おと}
\ruby{鮮}{あざ}やかに、
%
\ruby[|g|]{二人}{ふたり}して
\ruby{彈}{ひ}く
\ruby{絃}{いと}の
\ruby{音}{おと}の
\ruby{冴}{さ}えて、
%
\ruby{然}{さ}も
\ruby{面白}{おも|しろ}げに
\ruby[|g|]{樓上}{にかい}
あるべく
\ruby{思}{おも}はるゝ
\ruby{奧}{おく}の
\ruby{方}{かた}より
\ruby{洩}{も}れ
\原本頁{206-11}\改行%
\ruby{聞}{きこ}え
\ruby{來}{き}つ、
%
\ruby[||j>]{婢}{をんな}
\ruby[||j>]{等}{ ども}も
% \ruby{婢等}{をんな|ども}も
\ruby{其方}{そ|れ}に
\ruby{耳}{みゝ}や
\ruby{奪}{と}られ
\ruby{居}{ゐ}る、
%
\ruby{御免}{ご|めん}なさい、
%
\ruby{御免}{ご|めん}なさい、
%
と
\ruby{云}{い}へど
\ruby{應}{いら}ふるものも
\ruby{無}{な}く、
%
\ruby{拭}{ふ}いて
\ruby{除}{と}つたやうに
\ruby{奇麗}{き|れい}なる
\ruby{三和土}{た|た|き}の
\ruby{履脫}{くつ|ぬぎ}に
\ruby{良}{やゝ}
\ruby{久}{ひさ}しく
\ruby{立}{た}
たされたり。
