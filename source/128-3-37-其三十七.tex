\Entry{其三十七}

% メモ 校正終了 2024-05-17 2024-06-13
\原本頁{207-4}%
\ruby{何知}{なに|し}らぬ
\ruby{耳}{みゝ}にも
\ruby{面白}{おも|しろ}きは
\ruby{面白}{おも|しろ}く、
%
\ruby{{\換字{連}}彈}{つれ|びき}の
\ruby{三昧線}{し|や|み}の
\ruby{音}{ね}、
%
\ruby{急}{きふ}なる
\ruby{時}{とき}には
\ruby[|g|]{玉霰}{あられ}
\ruby{銀盤}{ぎん|ばん}を
\ruby{拍}{う}ち、
%
\ruby{{\換字{緩}}}{ゆる}き
\ruby{時}{とき}には
\ruby{{\換字{寒}}水}{み|ず}
せゝらぎに
\ruby{咽}{むせ}んで、
%
\ruby{一高一低}{いつ|かう|いつ|てい}、
%
\ruby{一挑一撥}{いつ|たう|いつ|ばつ}、
%
\ruby{{\換字{前}}聲}{ぜん|せい}は
\ruby{後聲}{こう|せい}を
\ruby{呼}{よ}び、
%
\ruby{後聲}{こう|せい}は
\ruby{{\換字{前}}聲}{ぜん|せい}に
\ruby{應}{こた}へて、
%
\ruby{斷}{た}えつ
\ruby{續}{つゞ}きつする
\ruby{間}{あひだ}に、
%
おのづと
\ruby{人}{ひと}の
\ruby{心}{こゝろ}を
\ruby{攝}{と}り
\ruby{去}{さ}れば、
%
\ruby{彼}{あれ}は
\ruby{何}{なに}といふ
\ruby{曲}{きよく}ぞとも
\ruby{知}{し}らぬ
お
\ruby{春}{はる}さへ
\ruby{聞}{きゝ}
\ruby{惚}{ほ}れて、
%
\ruby{身}{み}は
こゝに
\ruby{在}{あ}りながら
\ruby{思}{おもひ}を
\ruby[|g|]{彼方}{かなた}の
\ruby[||j>]{樓}{ろう}
\ruby[||j>]{上}{じよう}に
% \ruby{樓上}{ろう|じよう}に
\ruby{馳}{は}せて、
%
たゞ
\ruby{恍然}{うつ|とり}と
\ruby{我}{われ}を
\ruby{忘}{わす}れたる
\ruby{折}{をり}しも、
%
\ruby{怒}{いか}るが
\ruby{如}{ごと}く
\ruby{罵}{のゝし}るが
\ruby{如}{ごと}き
\ruby{案内}{あん|ない}
\ruby{乞}{こ}ふ
\ruby{聲}{こゑ}を
\ruby{聞}{き}き
つけて、
%
\ruby{吃驚}{びつ|くり}して
\ruby{我}{われ}に
\ruby{復}{かへ}り、
%
\ruby{周章}{あ|は}てゝ
\ruby{立}{たち}
\ruby{出}{い}で
\ruby{見}{み}れば、
%
\ruby[|g|]{衣服}{みなり}こそ
\ruby{見苦}{み|ぐる}しくは
あらね、
%
\ruby{五十}{ご|じふ}% もしかしたら(いそ)、(いそじ)かもしれない
\ruby{{\換字{近}}}{ぢか}き
\ruby{女}{をんな}の、
%
ただ% ルビ調整(原本通り)非踊り字表記
さへ
\ruby{下品}{げ|ひん}に
\ruby{肥}{ふと}りたる
\ruby{{\換字{平}}顏}{ひら|がほ}を、
%
\ruby{目}{め}に
\ruby{見}{み}ゆるほど
\ruby{膨}{ふく}らませきつたる
\ruby{不機{\換字{嫌}}}{ふ|き|げん}の
\ruby{氣色}{け|しき}
\ruby{怖}{おそ}ろしく、
%
\ruby{{\換字{嫌}}味}{いや|み}らしく
\ruby{細}{ほそ}く
\ruby{剃}{す}りつけたる
をかしき
\ruby{眉}{まゆ}を
\ruby{擧}{あ}げ、
%
\ruby{白睛}{しろ|め}の
\ruby{赤濁}{あか|にご}りせる
\ruby{汚}{きたな}き
\ruby{眼}{め}の
\ruby{小}{ちひさ}きに
\ruby{稜}{かど}
\原本頁{207-5}\改行%
\ruby{立}{た}てゝ、
%
     『
     \ruby{此}{こ}の
     \ruby{小}{こ}びつちよめが
     』
と
\ruby{云}{い}はぬばかりに
\ruby{頭}{あたま}から
\ruby{見}{み}
\ruby{下}{おろ}
\改行% 校正作業の簡略化のため
し
、
%
\原本頁{207-6}\改行%
その
\ruby{言葉}{こと|ば}
つきも
\ruby{憎}{にく}らしく
\ruby{刺々}{とげ|〴〵}しく、

\原本頁{208-7}%
『
お
\ruby{龍}{りう}に
\ruby{然樣}{さ|う}
\ruby{云}{い}つて
\ruby{下}{くだ}さい、
%
\ruby{本銀}{しろ|かね}
\ruby[||j>]{町}{ちやう}から
\ruby{來}{き}ましたと。
%
ハイ、
%
\ruby{然樣}{さ|う}
\ruby{云}{い}つて
\ruby{下}{くだ}されば
それで
\ruby{{\換字{分}}}{わか}るのですから。
%
\ruby{居不在}{ゐ|る|す}
なんぞは
\ruby{使}{つか}はせませんよ。
%
それ
あの
\ruby{上調子}{うは|でう|し}を
\ruby{付}{つ}けて
\ruby{居}{ゐ}る%
{---}{---}%
\ruby{彼}{あれ}は
\ruby{屹度}{きつ|と}% ルビ調整(原本通り)非グループルビ
お
\ruby{龍}{りう}に
\ruby{定}{きま}つてる
んですからネ。
』

\原本頁{208-11}%
と
\ruby{無{\換字{遠}}慮}{ぶ|ゑん|りよ}にも
\ruby{程度}{ほ|ど}のあるに、
%
\ruby{不在}{る|す}を
\ruby{使}{つか}はれやうかとの
\ruby{先潛}{さき|くぐ}りまでして、% 【潛 u6f5b 「先先」】【潜 u6f5c 「夫夫」】併用されている
% ルビ調整(原本通り)非踊り字表記の(くぐ)
%
\ruby{撥音}{ばち|おと}を
\ruby{聞}{き}いて
\ruby{其}{そ}の
\ruby{人}{ひと}を
\ruby{猜}{すゐ}することの
\ruby{出來}{で|き}るものやら
\ruby{出來}{で|き}ぬものやら
\ruby{知}{し}らねど、
%
\ruby{拔}{ぬ}けさせぬ
つもりからの
\ruby{當推}{あて|ずゐ}に、
%
\ruby{{\換字{硝}}子箱}{びい|どろ|ばこ}の
\ruby{中}{なか}のものを
\ruby{見}{み}でも
\ruby{仕}{し}たやうに
\ruby{確}{たしか}に
\ruby{其}{それ}と
\ruby{指}{さ}して
\ruby{云}{い}ひたき
まゝを
\ruby{云}{い}ひたり。

\原本頁{209-5}%
\ruby{其}{そ}の
\ruby{慳貪}{けん|どん}さ、
%
\ruby{其}{そ}の
\ruby{無作法}{ぶ|さ| ふ}さ、% ルビ調整(原本通り)(ぶさ ふ)
%
\ruby{其}{そ}の
\ruby{{\換字{尊}}大}{おほ|ふう}さ、
%
その
\ruby{下作}{げ|さく}さに、% ルビ調整(補正)国書データベースでは空白であるが国会図書館のには「作」のルビ(さく)が振られていた
% 【下作】げさく
% 1.《名》下等な出来あがり。できのわるい物。
% 2.《名ノナ》 いやしく下品なこと。
%
\ruby{優}{やさ}しき
お
\ruby{春}{はる}は
\ruby{驚}{おどろ}き
\ruby{呆}{あき}れつ、
%
\ruby{一寸}{いつ|すん}の
\ruby{蟲}{むし}にも
\ruby{五{\換字{分}}}{ご|ぶ}の
\ruby{魂魄}{たま|しひ}あれば、
%
\ruby{胸}{むね}の
\ruby{中}{うち}には
\ruby{可厭}{い|や}な〳〵
\ruby{人}{ひと}と
\ruby[|g|]{侮蔑}{さげす}みながら、

\原本頁{209-8}%
『
お
\ruby{待}{ま}ち
\ruby{下}{くだ}さいまし、
%
\ruby{然樣}{さ|う}
\ruby{申}{まを}しますから。
』

\原本頁{209-9}%
と
\ruby{冷}{ひや}やかに
\ruby{答}{こた}へて
\ruby[|g|]{徐々}{しづか}に
\ruby{身}{み}を
\ruby{起}{おこ}し、
%
\ruby{奧深}{おく|ふか}なる
\ruby[|g|]{樓上}{にかい}に
\ruby{至}{いた}りたり。
%
\原本頁{209-10}\改行%
\ruby{見}{み}れば
\ruby[|g|]{主人}{あるじ}の
お
\ruby{彤}{とう}は
\ruby{常}{つね}の
\ruby{如}{ごと}く
\ruby{沈着}{おち|つ}きたる
\ruby{面}{かほ}の
\ruby{色}{いろ}、
%
\ruby{逼}{せま}らず
\ruby{急}{せ}かず
\改行% 校正作業の簡略化のため
、
%
\原本頁{209-11}\改行%
たゞ
\ruby{白}{しろ}く、
%
\ruby{下品}{げ|ひん}の
\ruby{人}{ひと}を
\ruby{今}{いま}
\ruby{見}{み}たる
\ruby{目}{め}には
\ruby{宛}{あだか}も
\ruby{女雛}{め|びな}
なんどを
\ruby{見}{み}る
\ruby{如}{ごと}く
\ruby[||j>]{上}{じやう}
\ruby[||j>]{品}{ ひん}に
% \ruby{上品}{じやう|ひん}に
\ruby{見}{み}え、
%
お
\ruby{龍}{りう}は
また
\ruby{思}{おも}はず
\ruby{知}{し}らず
\ruby{興}{きよう}に
\ruby{乘}{の}り
\ruby{心}{こゝろ}を
はずませて
\ruby{我}{われ}
おもしろく
\ruby{彈}{ひ}くと
\ruby{思}{おぼ}しく、
%
\ruby{汗}{あせ}ばむ
といふ
ほどには
あらねど
\原本頁{210-3}\改行%
\ruby{氣勢}{いき|ほひ}
\ruby{{\換字{込}}}{こ}みたる
\ruby{面色}{かほ|つき}
やゝ
\ruby{紅色}{くれ|なゐ}
さして
\ruby{美}{うつく}しく
\ruby{見}{み}えしが、
%
\ruby{主人}{しゆ|じん}は
\ruby{我}{わ}が
\ruby{方}{かた}を
\ruby{見}{み}も
\ruby{{\換字{返}}}{かへ}らねど
お
\ruby{龍}{りう}は
\ruby{活々}{いき|〳〵}
としたる
\ruby{眼}{め}に
ちらりと
\ruby[|g|]{此方}{こなた}を
\ruby{見}{み}し
まゝ、
%
ただ% ルビ調整(原本通り)非踊り字表記
\ruby{一心}{いつ|しん}に
\ruby{彈}{ひ}き
つづけたり。% ルビ調整(原本通り)非踊り字表記
%
\ruby{{\換字{遠}}}{とほ}く
\ruby{聞}{き}きしにだに
\ruby{賑}{にぎ}やかなりしを、
%
\ruby{{\換字{近}}}{ちか}く
\ruby{聞}{き}けば
\ruby{{\換字{又}}}{また}
\ruby{一}{ひ}ト
\ruby{層}{しほ}
おもしろき
\ruby{絃}{いと}の
\ruby{色音}{いろ|ね}の、
%
\ruby{或}{あるひ}は
\ruby{{\換字{強}}}{つよ}く
\原本頁{210-7}\改行%
\ruby{撥}{ひ}き
%
\ruby{或}{あるひ}は
\ruby{輕}{かろ}く
\ruby{挑}{すく}ひ
%
\ruby{或}{あるひ}は
\ruby{彈}{はじ}く
%
\ruby{彼絃}{か|れ}の
\ruby{餘韵}{よ|ゐん}
\ruby{未}{いま}だ
\ruby{{\換字{消}}}{き}えずして
\ruby{此絃}{こ|れ}の
\ruby[<j||]{響}{ひゞき}% 行末行頭の境界付近なので特例処置を施す
\ruby{新}{あらた}に
\ruby{起}{おこ}る
\ruby{音}{おと}と
\ruby{音}{おと}とは、
%
\ruby{一條}{いち|でう}の
\ruby{玉}{たま}の
\ruby{{\換字{鎖}}}{くさり}の
\ruby{環}{わ}と
\ruby{環}{わ}と
\ruby[||j>]{相}{あひ}
\ruby[||j>]{{\換字{連}}}{つらな}り、
% \ruby{相{\換字{連}}}{あひ|つらな}り、
%
\ruby{一聯}{いち|れん}の
\原本頁{210-9}\改行%
\ruby{花輪}{はな|わ}の
\ruby{花}{はな}と
\ruby{花}{はな}と
\ruby{相}{あひ}
\ruby{襲}{かさ}なりて、
%
いづくに
\ruby{斷目}{きれ|め}も
\ruby{見}{み}えざるが
\ruby{如}{ごと}くなれば、
%
\ruby{言}{ことば}を
\ruby{出}{いだ}さん
\ruby{機會}{し|ほ}を
\ruby{知}{し}らずして、
%
\ruby{困}{こま}り〳〵て
\ruby{躊躇}{ちう|ちよ}しけるが
\改行% 校正作業の簡略化のため
、
%
\原本頁{210-11}\改行%
いつまで
かくては
\ruby{濟}{す}まじと
お
\ruby{龍}{りう}の
\ruby{傍}{かたへ}に
やゝ
\ruby{{\換字{近}}}{ちか}づきて、

\原本頁{211-1}%
『
お
\ruby{龍}{りう}さん、
%
あの、
%
\ruby{本銀}{しろ|かね}
\ruby[||j>]{町}{ちやう}から
まゐりましたつて
\ruby{何}{なん}だか
\ruby{可厭}{い|や}な
\ruby{人}{ひと}でございますが、
%
\ruby{五十}{ご|じふ}ばかりの
お
\ruby{方}{かた}が
‥‥。
』

\原本頁{211-3}%
と
\ruby{云}{い}へば
お
\ruby{龍}{りう}は
それと
\ruby{聞}{き}いて、
%
\ruby{彈}{ひ}く
\ruby{手}{て}は
\ruby{止}{と}めざれども
\ruby{眼}{め}は
お
\ruby{彤}{とう}の
\ruby{方}{かた}を
\ruby{見}{み}て、
%
\ruby[|g|]{許可}{ゆるし}をさへ
\ruby{得}{え}ば
\ruby{直}{すぐ}にも
\ruby{立}{た}つて
\ruby{下}{した}に
\ruby{行}{ゆ}かん
\ruby{素振}{そ|ぶり}を
あらはしたり。
%
お
\ruby{彤}{とう}は
これを
\ruby{見}{み}て
お
\ruby{龍}{りう}には
\ruby{答}{こた}へず、
%
\ruby{居}{ゐ}るか
\ruby{居}{ゐ}ぬか
\原本頁{211-6}\改行%
\ruby{知}{し}れざるやうに
\ruby{先刻}{さ|き}より
\ruby{我}{わ}が
\ruby{後}{うしろ}の
\ruby{隅}{すみ}に
かしこまりて
\ruby{控}{ひか}へ
\ruby{居}{ゐ}し
お
\ruby{富}{とみ}を
\ruby{一寸}{ちよ|つと}
\ruby{見}{み}れば、
%
お
\ruby{富}{とみ}は
\ruby{早}{はや}くも
\ruby{其}{そ}の
\ruby{意}{い}を
\ruby{悟}{さと}りて、
%
お
\ruby{春}{はる}の
\ruby{袂}{たもと}を
\ruby{引}{ひ}きに
\ruby{引}{ひ}きて
\ruby{樓下}{し|た}に
\ruby{去}{さ}りぬ。

\原本頁{211-9}%
『
\ruby{何}{なに}?、
%
お
\ruby{富}{とみ}さん、
%
\ruby{無理}{む|り}に
\ruby{妾}{わたし}の
\ruby{袂}{たもと}を
\ruby{引}{ひ}ぱつて。
』

\原本頁{211-10}%
\ruby{解}{かい}し
\ruby{得}{え}ぬ
お
\ruby{春}{はる}の
\ruby{訝}{いぶか}り
\ruby{問}{と}ふを
お
\ruby{富}{とみ}は
\ruby{冷笑}{あざ|わら}つて、

\原本頁{211-11}%
『
\ruby{何}{なに}ぢやあ
\ruby{有}{あ}りませんは、
%
\ruby{下}{くだ}らないよ、
%
お
\ruby{{\換字{前}}}{まへ}さんは。
%
あゝ
やつて
\ruby{{\換字{遊}}}{あそ}んで
\ruby{居}{ゐ}らつしやる
\ruby[||j>]{最}{さい}
\ruby[||j>]{中}{ちゆう}に
% \ruby{最中}{さい|ちゆう}に
\ruby{下}{くだ}らない
\ruby{事}{こと}なんぞ
\ruby{云}{い}つて
\ruby{行}{ゆ}くのだもの。
%
\ruby{御邪{\換字{魔}}}{お|じや|ま}
に
なる
ぢやあ
\ruby{無}{な}いかネ、
%
\ruby{何}{なん}でも
\ruby{自{\換字{分}}}{じ|ぶん}の% ルビ調整(原本通り)非グループルビ
\ruby{仕}{し}て
\ruby{居}{ゐ}らつしやる
\ruby{事}{こと}の
\ruby{腰}{こし}を
\ruby{折}{を}られたり
なんぞ
するのは
\ruby{大{\換字{嫌}}}{だい|きら}ひの
\ruby{御方}{お|かた}
なんだからネ。
%
もう
\ruby{今}{いま}
お
\ruby{龍}{りう}さんが
\ruby{立}{た}たうと
なすつただけで% ルビ調整(原本通り)非踊り字表記
\ruby{餘程}{よつ|ぽど}
\ruby{可厭}{い|や}に
\原本頁{212-5}\改行%
\ruby{思}{おもつ}て
\ruby{居}{ゐ}らつしやるのだよ。
%
\ruby{何樣}{ど|う}して、
%
そりやあ〳〵
\ruby{御行屆}{お|ゆき|とゞ}きなさる% 「屆」「届」 原本通り「屆」
\ruby{方}{かた}だけに
\ruby{恐}{おそ}ろしい
\ruby{高慢}{かう|まん}の
\ruby{{\換字{強}}}{つよ}い
\ruby{御}{ご}
\ruby{氣象}{き|しやう}
なんだからネ。
%
\ruby{人}{ひと}が
\ruby{來}{き}たら
\ruby{待}{ま}たして
\ruby{置}{お}いて
お
\ruby{濟}{す}みに
なつた
\ruby{時}{とき}
\ruby{申}{まを}し
\ruby{上}{あ}げ
さへ
すりやあ
それで、
%
\ruby{宜}{よ}いぢやあ
\ruby{無}{な}いかえ。
%
こんど
から
\ruby{氣}{き}を
\ruby{付}{つ}けないと、
%
\ruby{馬鹿}{ば|か}だ
といつて
\ruby{御笑}{お|わら}ひ
になるよ。
』

\原本頁{212-10}%
と
\ruby[|g|]{自己}{おのれ}も
\ruby{一度}{いち|ど}は
\ruby{笑}{わら}はれたる
\ruby{事}{こと}のあるなるべし、
%
\ruby{姊}{あね}ぶつて
\ruby{敎}{をし}へた
\改行% 校正作業の簡略化のため
り。

\原本頁{213-1}%
『
\ruby{然樣}{さ|う}、
%
だつて
\ruby{何}{なん}だか
ぶり〳〵
\ruby{怒}{おこ}つて
\ruby{居}{ゐ}る、
%
やかましい
\ruby{事}{こと}でも
\ruby{云}{い}ひさうな
\ruby{權幕}{けん|まく}の
\ruby{人}{ひと}が
\ruby{來}{き}たんですもの。
』

\原本頁{213-3}%
と、
%
\ruby{負惜}{まけ|をし}み
\ruby{氣味}{ぎ|み}に
\ruby{辯解}{いひ|わけ}を、% 弁 瓣 辦 辧 辨 辩 (辯)
%
\ruby{試}{こゝろ}みるを、

\原本頁{213-4}%
『
\ruby{何}{なん}だえ、
%
やかましいことでも
\ruby{云}{い}ひさうな
\ruby{人}{ひと}だつて。
%
ヘエー。
%
\原本頁{213-5}\改行%
ナニ、
%
\ruby{何樣}{ど|ん}な
\ruby{人}{ひと}だつて
\ruby{關}{かま}ふ
こと
が
あるもんかネ、
%
\ruby{下}{くだ}らない!。
%
\ruby[||j>]{妾}{わたし}
\ruby[||j>]{等}{ たち}あ
% \ruby{妾等}{わたし|たち}あ
\ruby{御主人樣}{ご|しゆ|じん|さま}の
\ruby{御氣}{お|き}に
\ruby{入}{い}る
やうにさへ
\ruby{爲}{す}りやあ
\ruby{宜}{い}いぢやあ
\ruby{無}{な}いか。
%
ぢやあ
\ruby{妾}{わたし}が
\ruby{待}{ま}つて
\ruby{居}{ゐ}ろつて、
%
\ruby{待}{ま}たせて
\ruby{置}{お}いて
\ruby{{\換字{遣}}}{や}りましや
\改行% 校正作業の簡略化のため
う。
』

\原本頁{213-9}%
と
\ruby{此}{これ}は
\ruby{{\換字{飽}}}{あく}まで
\ruby{姊}{あね}ぶつて
\ruby{入口}{いり|くち}の
\ruby{方}{かた}に
\ruby{行}{ゆ}きたり、
%
\ruby[||j>]{樓}{ろう}
\ruby[||j>]{上}{じよう}の
% \ruby{樓上}{ろう|じよう}の
\ruby{絃聲}{げん|せい}は
\ruby{盛}{さか}んに
\ruby{續}{つゞ}けり。
