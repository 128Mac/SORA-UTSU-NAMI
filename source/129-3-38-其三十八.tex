\Entry{其三十八}

\原本頁{}
\ruby{同}{おな}じ
\ruby{身{\換字{分}}}{み|ぶん}ながらも
\ruby{新參}{しん|ざん}だけに
\ruby{我}{わ}が
\ruby{下}{した}につける
お
\ruby{春}{はる}に
\ruby{對}{むか}ひては、
%
\ruby{神經質}{む|し|もち}の
\ruby{本性}{ほん|しやう}を
\ruby{露}{あらは}して
\ruby{偶然}{ふ|と}したる
\ruby{氣}{き}の
\ruby{向}{む}き
\ruby{方}{かた}のはずみにかゝり、
%
\ruby{意地}{い|ぢ}でも
\ruby{惡}{わる}き
\ruby{人}{ひと}のやうに、
%
つけ〳〵と
\ruby{思}{おも}ふまゝを
\ruby{自己}{お|の}が
\ruby{心}{こゝろ}の
\ruby{{\換字{廻}}}{まは}るに
\ruby{任}{まか}せて、
%
\ruby{年齡}{と|し}の
\ruby{十歳}{と|を}も
\ruby{{\換字{違}}}{ちが}ふほど
\ruby{大人}{おと|な}ぶりて
\ruby{{\換字{銳}}}{するど}くも
\ruby{言}{ものい}へ、
%
\ruby{根}{ね}が
\ruby{粗豪}{あ|ら}からぬ
\ruby{氣象}{き|しやう}の
\ruby{心細}{こゝろ|ぼそ}かければ、
%
\ruby{客}{きやく}に
\ruby{對}{むか}ひては
\ruby{打}{う}つて
\ruby{變}{かは}つて、
%
\ruby{顏色}{かほ|つき}も
\ruby{恭}{うや〳〵}しく
\ruby{言葉}{こと|ば}も
\ruby{慇懃}{いん|ぎん}に、

\原本頁{}
『さあ
\ruby{何樣}{ど|う}かまあ
\ruby{此方}{こち|ら}へ
\ruby{御上}{お|あが}りなさいまして、
』

\原本頁{}
と
\ruby{入口{\換字{近}}}{いり|くち|ちか}き
\ruby{一}{ひ}ト
\ruby{室}{ま}に
\ruby{{\換字{通}}}{とほ}して、
%
\ruby{會}{あ}ふとも
\ruby{會}{あ}はぬとも
\ruby{其}{そ}の
\ruby{挨拶}{あい|さつ}は
\ruby{云}{い}はず、
%
\ruby{待}{ま}てと
\ruby{特{\換字{更}}}{こと|さら}には
\ruby{告}{つ}げず
\ruby{默}{だま}つて
\ruby{待}{ま}たせ
\ruby{置}{お}き、
%
\ruby{物}{もの}の
\ruby{値}{ね}でも
\ruby{定}{ふ}むやうに
\ruby{室}{へや}の
\ruby{中}{うち}をきよろ〳〵
\ruby{眼}{め}に
\ruby{見囘}{み|まは}す% 原本通り「囘」
\ruby{客}{きやく}を
\ruby{其儘}{その|まゝ}
\ruby{殘}{のこ}して
\ruby{身}{み}は
\ruby{蔭}{かげ}に
\ruby{{\換字{退}}}{しりぞ}き、

\原本頁{}
『ほんたうにお
\ruby{春}{はる}さん、
%
\ruby{何}{なん}だか
\ruby{可厭}{い|や}な
\ruby{人}{ひと}ネエ。
%
でも
\ruby{宜}{い}いは、
%
お
\ruby{茶}{ちや}と
\ruby{火}{ひ}とだけ
\ruby{與}{や}つて
\ruby{置}{お}いて、
%
\ruby{默}{だま}つて
\ruby{引{\換字{込}}}{ひつ|こ}んでさへ
\ruby{居}{ゐ}りやあ、
%
それで
\ruby{濟}{す}むのだもの。
%
\ruby{關}{かま}ふことは
\ruby{有}{あ}りやあ
\ruby{仕}{し}ませんは、
%
\ruby{柔軟}{やはら|か}にあしらつて、
%
そして
\ruby{無言}{だん|まり}でさへ
\ruby{居}{ゐ}りやあ。
%
\ruby{妾}{わたし}あ
\ruby{彼方}{あち|ら}で
\ruby{御用}{ご|よう}があるか
\ruby{知}{し}れないから……』

\原本頁{}
と
\ruby{云}{い}ひさして
\ruby{既}{はや}
\ruby{樓}{にかい}の
\ruby{方}{かた}へ
\ruby{去}{さ}れば、
%
お
\ruby{春}{はる}は
\ruby{言葉}{こと|ば}の
\ruby{如}{ごと}く
\ruby{唯}{たゞ}
\ruby{謹}{つゝし}みて
\ruby{火}{ひ}を
\ruby{{\換字{運}}}{はこ}び
\ruby{茶}{ちや}を
\ruby{{\換字{運}}}{はこ}べり。

\原本頁{}
お
\ruby{富}{とみ}が
\ruby{樓}{にかい}へ
\ruby{上}{あが}りたる
\ruby{時}{とき}は
\ruby{曲}{きよく}は
\ruby{既}{はや}
\ruby{{\換字{終}}}{をは}りに
\ruby{{\換字{近}}}{ちか}く、
%
やがて
\ruby{二人}{ふた|り}は
\ruby{彈}{ひ}き
\ruby{仕舞}{し|ま}ひけるが、
%
お
\ruby{彤}{とう}は
\ruby{此}{こ}の
\ruby{時}{とき}はじめて
\ruby{莞爾}{にこ|り}として
お
\ruby{龍}{りう}を
\ruby{見{\換字{遣}}}{み|や}りつ、

\原本頁{}
『
\ruby{面白}{おも|しろ}かつたこと!
\ruby{久}{ひさ}しぶりで
\ruby{二人}{ふた|り}で
\ruby{彈}{ひ}いたので、
%
だが
\ruby[<h||]{妾}{わたし}
\ruby{樂}{らく}ぢやあ
\ruby{無}{な}かつたの、
%
たまに
\ruby{彈}{ひ}いたんだから。
』

\原本頁{}
と、
%
\ruby{何處}{ど|こ}に
\ruby{人}{ひと}が
\ruby{來}{き}て
\ruby{待}{ま}つて
\ruby{居}{ゐ}るかも
\ruby{知}{し}らぬやうに、
%
\ruby{悠然}{ゆつ|たり}と
\ruby{云}{い}へ、

\原本頁{}
『あら
\ruby{嘘}{うそ}ばつかり、
%
\ruby{妾}{わたし}こそ
\ruby{姊}{ねえ}さんと
\ruby{彈}{ひ}くと
\ruby{氣}{き}が
\ruby{詰}{つ}まるやうな
\ruby{氣}{き}が
\ruby{仕}{し}て
\ruby{樂}{らく}ぢやあ
\ruby{無}{な}いの!。
%
\ruby{姊}{ねえ}さんは
\ruby{餘}{あんま}り
\ruby{奇麗}{き|れい}に、
%
そして
\ruby{餘}{あんま}りきつかり〳〵に
\ruby{几帳面}{きち|やう|めん}に
お
\ruby{彈}{ひ}きなさるんですもの!。
』

\原本頁{}
と、
%
お
\ruby{龍}{りう}も
\ruby{是非無}{ぜ|ひ|な}く
\ruby{受答}{うけ|こた}へは
\ruby{仕}{し}て
\ruby{居}{ゐ}れど、
%
\ruby{此}{これ}は
\ruby{來客}{らい|きやく}の
\ruby{聊}{いさゝ}か
\ruby{早}{はや}く、
%
お
\ruby{彤}{とう}は
\ruby{今}{いま}しも
お
\ruby{富}{とみ}が
\ruby{薦}{すゝ}むる
\ruby{一碗}{いち|わん}の
\ruby{茶}{ちや}を
\ruby{然}{さ}も
\ruby{心好}{こゝろ|よ}げに
\ruby{飮}{の}み
\ruby{味}{あぢ}はふにも
\ruby{似}{に}ず、
%
\ruby{此}{これ}は
\ruby{茶碗}{ちや|わん}を
\ruby{手}{て}に
\ruby{取}{と}り
\ruby{上}{あ}ぐる
\ruby{事}{こと}だに
\ruby{爲}{な}さざるなり。

\原本頁{}
『
\ruby{然樣}{さ|う}ネエ、
%
どうも
\ruby{妾}{わたし}の
\ruby{彈}{ひ}き
\ruby{方}{かた}は
\ruby{器械}{き|かい}かなんかゞ
\ruby{動}{うご}く
\ruby{樣}{やう}で、
%
\ruby{味}{あぢ}が
\ruby{無}{な}くつていけないよ。
%
\ruby{詰}{つま}り
\ruby{{\換字{習}}}{なら}つて
\ruby{記}{おぼ}えたつて% 送り仮名は原本通り「え」
\ruby{云}{い}ふつ
\ruby{限}{き}りの
\ruby{{\換字{技}}}{わざ}で、
%
ほんたうは
\ruby{藝事}{げい|ごと}の
\ruby{出來}{で|き}るつて
\ruby{云}{い}ふ
\ruby{人}{ひと}の
\ruby{性質}{た|ち}ぢやあ
\ruby{無}{な}いのだネ。
%
お
\ruby{{\換字{前}}}{まへ}はまた
\ruby{大變}{たい|へん}に
\ruby{出來不出來}{で|き|ふ|で|き}が
お
\ruby{有}{あ}りのやうだけれど、
%
\ruby{今日}{け|ふ}のやうに
\ruby{機勢}{はず|み}に
\ruby{乘}{の}つて
お
\ruby{彈}{ひ}きのときは、
%
ほんとに
\ruby{憎}{にく}らしい
\ruby[<h||]{位}{くらゐ}
\ruby{見事}{み|ごと}に
\ruby{御出來}{お|で|き}だよ。
%
\ruby{詰}{つま}り
お
\ruby{{\換字{前}}}{まへ}のは、
%
\ruby{何樣}{ど|う}かした
\ruby{時}{とき}にやあ、
%
おぼえたつて
\ruby{云}{い}ふつ
\ruby{限}{き}りの
\ruby{{\換字{技}}}{わざ}ぢやあ
\ruby{無}{な}いものが
\ruby{何處}{ど|こ}からか
\ruby{知}{し}らないが
\ruby{出}{で}て
\ruby{來}{く}るんだネエ。
%
\ruby{生}{うま}れついて
\ruby{藝}{げい}の
\ruby{味}{あぢ}といふものを
\ruby{有}{も}つておいでなんだよ。
』

\原本頁{}
『なあに、
%
\ruby{然樣}{さ|う}ぢやあ
\ruby{無}{な}いんですけれどもネ、
%
\ruby{一人}{ひと|り}でなんか
\ruby{彈}{ひ}くと、
%
\ruby{妾}{わたし}あつまーらないと
\ruby{思}{おも}つて
\ruby{彈}{ひ}く
\ruby{時}{とき}が
\ruby{多}{おほ}いんですがネ、
%
\ruby{姊}{ねえ}さんと
\ruby{彈}{ひ}いたりなんぞすると、
%
\ruby{何樣}{ど|う}かすると
\ruby{不思議}{ふ|し|ぎ}に
\ruby{自{\換字{分}}}{じ|ぶん}でも
\ruby{面白}{おも|しろ}くなつて
\ruby{來}{く}ることがあるんですの、
%
そして
\ruby{然樣}{さ|う}いふ
\ruby{時}{とき}は
\ruby{屹度}{きつ|と}
\ruby{自{\換字{分}}}{じ|ぶん}の
\ruby{思}{おも}ふやうに
\ruby{自然}{ひと|りで}に
\ruby{彈}{ひ}けるんですよ。
%
やつばり
\ruby{一生懸命}{いつ|しやう|けん|めい}になるからなんでしやうかネエ?。
』

\原本頁{}
『ホヽヽヽ、
%
\ruby{一生懸命}{いつ|しやう|けん|めい}になりやあ
\ruby{巧}{うま}く
\ruby{彈}{ひ}けるけれども、
%
\ruby{然樣}{さ|う}でない
\ruby{時}{とき}あ
\ruby{彈}{ひ}けないつて
\ruby{云}{い}ふんぢやあ、
%
ぢやあ
お
\ruby{{\換字{前}}}{まへ}は
\ruby{横着者見}{わう|ちやく|もの|み}たやうだ
\ruby{事}{こと}ネエ。
』

\原本頁{}
『ホヽヽヽ、
%
\ruby{屹度}{きつ|と}
\ruby{然樣}{さ|う}なんかも
\ruby{知}{し}れませんよ。
%
でも
\ruby{妾}{わたし}あ
\ruby{故}{わざ}と
\ruby{然樣}{さ|う}やるのぢやあ
\ruby{無}{な}くつて、
%
\ruby{自然}{ひと|りで}に
\ruby{生}{うま}れついて
\ruby{居}{ゐ}る
\ruby{横着者}{わう|ちやく|もの}なんでしやうから……』

\原本頁{}
『
\ruby{惡}{わる}い
\ruby{横着者}{わう|ちやく|もの}ぢやあ
\ruby{有}{あ}るまいと
お
\ruby{云}{い}ひの?。
』

\原本頁{}
『ホヽホヽホヽ、
』

\原本頁{}
『ホヽホヽホヽ、
%
マア
\ruby{蟲}{むし}が
\ruby{宜}{い}いネエ。
』

\原本頁{}
『ホヽホヽ、
%
\ruby{美}{い}い
\ruby{横着者}{わう|ちやく|もの}でも
\ruby{惡}{わる}い
\ruby{横着者}{わう|ちやく|もの}でも
\ruby{其}{そ}りやあ
\ruby{關}{かま}ひませんが、
%
\ruby{樓下}{し|た}の
\ruby{彼}{あ}の
\ruby{人}{ひと}が
\ruby{待}{ま}つて
\ruby{居}{ゐ}ましやうから……』

\原本頁{}
\ruby{斯}{か}く
\ruby{云}{い}ひて
\ruby{立}{た}たんとする
お
\ruby{龍}{りう}を
\ruby{抑止}{と|ど}めて

\原本頁{}
『
\ruby{宜}{い}いよ、
%
お
\ruby{{\換字{前}}}{まへ}はまあ
\ruby{此室}{こ|ゝ}においで。
%
\ruby{妾}{わたし}が
\ruby{會}{あ}つて
\ruby{談}{はなし}を
\ruby{仕}{し}て
\ruby{仕舞}{し|ま}ふから。
』

\原本頁{}
とお
\ruby{彤}{とう}はやをら
\ruby{身}{み}を
\ruby{起}{おこ}したり。
