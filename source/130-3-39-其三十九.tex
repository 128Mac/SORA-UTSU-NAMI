\Entry{其三十九}

こゝに
\ruby{居}{ゐ}よと
\ruby{云}{い}はれては
\ruby{{\換字{逆}}}{さか}らふべくもあらねば、
お
\ruby{龍}{りう}は
\ruby{殘}{のこ}り
\ruby{止}{とど}まりて
\ruby{三昧線}{さ|み|せん}の
\ruby{絃}{いと}を
\ruby{戾}{もど}し
\ruby{{\換字{緩}}}{ゆる}めなど
\ruby{仕}{し}ながらも、
\ruby{我}{わ}が
\ruby{上}{うへ}に
\ruby{就}{つ}きて
\ruby{來}{きた}れる
\ruby{彼}{か}の
お
\ruby{關}{せき}が
\ruby{事}{こと}の
\ruby{氣}{き}になりてならねば、そこら
\ruby{取片付}{とり|かた|づ}くる
お
\ruby{富}{とみ}をば
\ruby{一寸}{ちよ|つと}
\ruby{視}{み}て、

『お
\ruby{春}{はる}さんの
\ruby{云}{い}つたやうに、ほんとに
\ruby{怒}{おこ}つて
\ruby{居}{ゐ}て?。
』

と
\ruby{問}{と}へば、
お
\ruby{富}{とみ}はさも〳〵
\ruby{其}{そ}の
\ruby{人}{ひと}を
\ruby{厭}{いと}ひ
\ruby{{\換字{嫌}}}{きら}ふといふやうに、さらでも
\ruby{淋}{さび}しき
\ruby{顏}{かほ}を
\ruby{妙}{めう}に
\ruby{皺}{しわ}めて、

『ほんとに
\ruby{恐}{おそ}ろしくぶり〳〵して
\ruby{居}{ゐ}ますの!。
まるで
\ruby{御酒}{ご|しゆ}にでも
\ruby{醉}{よ}つた
\ruby{人}{ひと}のやうな
\ruby{顏}{かほ}を
\ruby{仕}{し}まして、』

と
\ruby{先}{ま}づ
\ruby{答}{こた}へつ、

『
\ruby{何}{なん}だか
\ruby{自{\換字{分}}{\換字{勝}}手}{じ|ぶん|かつ|て}の
\ruby[g]{不理屈}{ふりくつ}でも
\ruby{云}{い}ひさうな
\ruby{可厭}{い|や}な
\ruby{人}{ひと}ですことネエ。
』

と
\ruby{添}{そ}へたり。

『マア
\ruby{可厭}{い|や}だことネエ!。
そんなやうに
\ruby{見}{み}えるほど
\ruby{恐}{おそ}ろしい
\ruby{怒}{おこ}つた
\ruby{顏}{かほ}を
\ruby{仕}{し}て
\ruby{居}{ゐ}て?。
』

『
\ruby{然樣}{さ|う}なんですよ、
\ruby{怒}{おこ}り
\ruby{切}{き}つて
\ruby{居}{ゐ}るといふ
\ruby{顏}{かほ}つきなんです。
それに
\ruby{一體}{いつ|たい}が
\ruby{地腫}{ぢ|ばれ}の
\ruby{仕}{し}たやうな
\ruby{顏}{かほ}なんでしやうかネエ、
\ruby{隨{\換字{分}}}{ずゐ|ぶん}おそろしく
\ruby{膨}{ふく}れかへつて、
\ruby{宛然}{とん|と}……』

『
\ruby{宛然何}{とん|と|なん}なの?。
\ruby{自{\換字{分}}}{じ|ぶん}でばかり
\ruby{承知}{しよ|うち}して
\ruby{笑}{わら}つて。
』

『マア
\ruby{止}{よ}して
\ruby{置}{お}きましやう
\ruby[g]{他人樣}{ひとさま}の
\ruby{惡口}{わる|くち}なんか。
』

『ホヽヽをかしな
\ruby{人}{ひと}ネエ、
\ruby{一人}{ひと|り}で
\ruby{合點}{が|てん}して
\ruby{一人}{ひと|り}で
\ruby{可笑}{を|か}しがつたりなんかして。
』

『ホヽヽ、でも
\ruby{惡}{わる}うございますもの。
』

\ruby{宛然}{とん|と}
\ruby{河豚}{ふ|ぐ}が
\ruby{五合}{ご|がふ}も
\ruby{引掛}{ひつ|か}けたやうと
\ruby{云}{い}はんと
\ruby{仕}{し}たりし
\ruby{歟}{か}、
\ruby{風{\換字{船}}玉}{ふう|せん|だま}に
\ruby{眼鼻}{め|はな}を
\ruby{付}{つ}けたやうと
\ruby{云}{い}はんと
\ruby{仕}{し}たりし
\ruby{歟}{か}、
\ruby{{\換字{終}}}{つひ}に
\ruby{口}{くち}を
\ruby{啓}{ひら}かねば
\ruby{知}{し}るものは
\ruby{當人}{たう|にん}の
\ruby{胸}{むね}のみ。

『マア
\ruby{勘忍}{か|に}して
\ruby{置}{お}いて
\ruby{頂戴}{ちやう|だい}よ。
』

と
\ruby{輕}{かろ}く
\ruby{謝}{わ}びて
\ruby{根問}{ね|どひ}さるゝを
\ruby{{\換字{遮}}}{さへぎ}り
\ruby{止}{とど}めつ
\ruby{樓下}{し|た}に
\ruby{去}{さ}りたり。

\ruby{人去}{ひと|さ}つて
\ruby{小樓靜}{せう|ろう|しづか}に、
\ruby{刳拔}{くり|ぬき}の
\ruby{桐}{きり}の
\ruby{手爐}{てあ|ぶり}の
\ruby{小}{ちひさ}なるを
\ruby{擁}{よう}して、
\ruby{雪}{ゆき}と
\ruby{白}{しろ}き
\ruby{蠣灰}{かき|ばひ}に
\ruby{纖}{ほそ}き
\ruby{火箸}{ひ|ばし}もて
\ruby{譯}{わけ}も
\ruby{無}{な}く
\ruby{假名文字}{か|な|も|じ}を
\ruby{書}{か}きては
\ruby{{\換字{消}}}{け}し
\ruby{書}{か}きては
\ruby{{\換字{消}}}{け}しつ、
お
\ruby{龍}{りう}はじつと
\ruby{心一筋}{こゝろ|ひと|すじ}に
\ruby{彼方}{かな|た}の
\ruby{談話}{はな|し}の
\ruby{何}{なん}となり
\ruby{行}{ゆ}くかを
\ruby{想}{おも}ひやりつゝ、

『
\ruby{彼}{あ}の
\ruby{{\換字{勝}}手}{かつ|て}の
\ruby{{\換字{強}}}{つよ}い
\ruby{慾}{よく}の
\ruby{深}{ふか}い
お
\ruby{師匠}{し|よ}さんがまあ
\ruby{何樣}{ど|ん}な
\ruby{事}{こと}を
お
\ruby{云}{い}ひのだらう。
そりやあもう
\ruby{智慧}{ち|ゑ}も
\ruby{{\換字{分}}別}{ふん|べつ}も
\ruby{確固}{しつ|かり}としておいでゞ、
\ruby{而}{さう}して
\ruby{言語}{もの|いひ}だつて
\ruby{拙}{まづ}い
\ruby{事}{こと}なんぞは
お
\ruby{云}{い}ひで
\ruby{無}{な}い
\ruby{姊}{ねえ}さんの
\ruby{事}{こと}だから、
\ruby{何}{なに}を
\ruby[g]{對手}{むかふ}で
\ruby{云}{い}つたつて
\ruby{譯}{わけ}も
\ruby{無}{な}く
\ruby{捌}{さば}いて
お
\ruby{仕舞}{し|ま}ひなさるには
\ruby{{\換字{違}}}{ちが}ひ
\ruby{無}{な}からうが、
\ruby{對手}{あひ|て}が
\ruby{無茶}{む|ちや}な
\ruby{人}{ひと}なだけに
\ruby{御困}{お|こま}りなさりは
\ruby{仕}{し}まいか
\ruby{知}{し}らん。
\ruby{自{\換字{分}}}{じ|ぶん}の
\ruby{{\換字{勝}}手}{かつ|て}づくに
\ruby{掛}{か}けちやあ
\ruby{理合}{り|あひ}や
\ruby{{\換字{情}}合}{じやう|あひ}に
\ruby{構}{かま}つて
\ruby{居}{ゐ}る
\ruby{樣}{やう}な
\ruby{其樣}{そ|ん}な
\ruby{上品}{じやう|ひん}な
\ruby{人}{ひと}ぢやあ
\ruby{無}{な}さゝうな
\ruby{彼}{あ}の
\ruby{人}{ひと}を
\ruby{對手}{あひ|て}にして、くだらない
\ruby{惡口}{あく|たい}や
\ruby{無理}{む|り}な
\ruby{難題}{なん|だい}でも
\ruby{云}{い}はれて
\ruby{困}{こま}つておいでゞは
\ruby{有}{あ}るまいか
\ruby{知}{し}ら。
\ruby{對手}{む|かう}が
\ruby{無茶}{む|ちや}な
\ruby{人}{ひと}でさへ
\ruby{無}{な}ければ
\ruby{宜}{よ}いのだけれども、
\ruby{男}{をとこ}にでも
\ruby{何}{なん}でも
\ruby{負}{ま}けては
\ruby{居}{ゐ}ない
\ruby{樣}{やう}な
\ruby{氣}{き}の
\ruby{{\換字{強}}}{つよ}い
\ruby{人}{ひと}ではあるし、また
\ruby{大變}{たい|へん}に
\ruby{怒}{おこ}り
\ruby{立}{た}つて
\ruby{來}{き}たのだとはいふし、
\ruby{一體}{いつ|たい}が
\ruby{{\換字{勝}}手}{かつ|て}のひどい
\ruby{甚}{ひど}い
\ruby{人}{ひと}だから、いくら
\ruby{姊樣}{ねえ|さん}が
\ruby[g]{冷悧}{りこう}でも
\ruby{扱}{あつか}ひ
\ruby{難}{にく}いかと
\ruby{思}{おも}はれるが、まあどんな
\ruby{事}{こと}を
\ruby{云}{い}つて
\ruby{來}{き}たもので
\ruby{有}{あ}らう。
\ruby{若}{も}し
\ruby{下}{くだ}らない
\ruby{事}{こと}を
\ruby{云}{い}つて
\ruby{哦鳴}{が|な}り
\ruby{立}{た}てでもされた
\ruby{日}{ひ}には、ほんとに
\ruby{姊}{ねえ}さんに
お
\ruby{氣}{き}の
\ruby{毒}{どく}で、
\ruby{妾}{わたし}はまあ
\ruby{何樣}{ど|う}したら
\ruby{宜}{よ}からう。
\ruby{何樣}{ど|う}か
\ruby{彼}{あ}の
\ruby{人}{ひと}が
\ruby{姊}{ねえ}さんの
\ruby{理解}{り|かい}に
\ruby{折}{を}れ
\ruby{{\換字{呉}}}{く}れゝば
\ruby{宜}{い}いが、いくら
\ruby{姊}{ねえ}さんでも
\ruby{對手}{あひ|て}が
\ruby{惡}{わる}いから、
\ruby{何}{なん}だか
\ruby{覺束無}{おぼ|つか|な}いやうな
\ruby{氣}{き}が
\ruby{仕}{し}てならない。
あゝ
\ruby{氣}{き}の
\ruby{揉}{も}める。
\ruby{一體}{いつ|たい}まあ
\ruby{今日}{け|ふ}の
\ruby{談}{はなし}は
\ruby{何樣}{ど|う}
\ruby{結局}{をさ|まり}がついて、そして
\ruby{妾}{わたし}はまあこれから
\ruby[g]{前{\換字{途}}何樣}{さきどう}なつて
\ruby{行}{ゆ}く
\ruby{身}{み}なのだらう。
』

と
\ruby{取}{と}り
\ruby{止}{と}まらず
\ruby{物}{もの}を
\ruby{案}{あん}じて
\ruby{耳}{みゝ}は
\ruby{彼方}{かな|た}にのみ
\ruby{走}{はし}れど、
\ruby{距離隔}{あは|ひ|へだ}てたれば
\ruby{音}{おと}も
\ruby{聞}{きこ}えず、
\ruby{人}{ひと}もあらぬが
\ruby{如}{ごと}く
\ruby[g]{此家靜}{このいへしづか}なり。

やゝ
\ruby{久}{ひさし}くして
\ruby{階段}{はし|ご}を
\ruby{上}{のぼ}り
\ruby{來}{く}る
\ruby{人}{ひと}の
\ruby{跫音}{あし|おと}し、やがて
お
\ruby{春}{はる}は
\ruby{襖}{ふすま}を
\ruby{開}{ひら}きて
\ruby{面}{おもて}を
\ruby{出}{いだ}せば、

『
\ruby{妾}{わたし}に
\ruby{來}{こ}いつて、』

とお
\ruby{龍}{りう}は
\ruby{此方}{こな|た}より
\ruby{問}{と}ひかけたり。

『ハイ、
\ruby{左樣仰}{さ|う|おつし}あいましたので。
』

\ruby{今}{いま}さら
\ruby{胸}{むね}のだくつくやうおぼえて、
\ruby{話}{ばなし}の
\ruby[g]{模樣}{もやう}を
\ruby{測}{はか}りかねつ、
お
\ruby{龍}{りう}は
\ruby{却}{かへ}つて
\ruby{頓}{とみ}には
\ruby{起}{た}たざりけり。

