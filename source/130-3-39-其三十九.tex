\Entry{其三十九}

% メモ 校正終了 2024-05-18
\原本頁{220-1}%
こゝに
\ruby{居}{ゐ}よと
\ruby{云}{い}はれては
\ruby{{\換字{逆}}}{さか}らふ
べくも
あらねば、
%
お
\ruby{龍}{りう}は
\ruby{殘}{のこ}り
\ruby{止}{とど}まりて% ルビ調整(原本通り)非踊り字表記
\ruby{三昧線}{さ|み|せん}の
\ruby{絃}{いと}を
\ruby{戾}{もど}し
\ruby{{\換字{緩}}}{ゆる}めなど
\ruby{仕}{し}ながらも、
%
\ruby{我}{わ}が
\ruby{上}{うへ}に
\ruby{就}{つ}きて
\ruby{來}{きた}れる
\ruby{彼}{か}の
お
\ruby{關}{せき}が
\ruby{事}{こと}の
\ruby{氣}{き}になりて
ならねば、% 原本では(ら)の部分が印刷不明瞭だったが補完
%
そこら
\ruby{取}{とり}
\ruby[g]{片付}{かたづ }くる
お
\ruby{富}{とみ}をば
\ruby[g]{一寸}{ちよつと}
\ruby{視}{み}て、

\原本頁{220-5}%
『
お
\ruby{春}{はる}さんの
\ruby{云}{い}つたやうに、
%
ほんとに
\ruby{怒}{おこ}つて% ルビ調整(補正)国書データベースではルビの(お)は植字されていないが国会図書館のを参考に補完
\ruby{居}{ゐ}て?。
』

\原本頁{220-6}%
と
\ruby{問}{と}へば、
%
お
\ruby{富}{とみ}は
さも〳〵
\ruby{其}{そ}の
\ruby{人}{ひと}を
\ruby{厭}{いと}ひ
\ruby{{\換字{嫌}}}{きら}ふ
といふ
やうに、
%
さら
でも
\ruby{淋}{さび}しき
\ruby{顏}{かほ}を
\ruby{妙}{めう}に
\ruby{皺}{しわ}めて、

\原本頁{220-8}%
『
ほんとに
\ruby{恐}{おそ}ろしく
ぶり〳〵して
\ruby{居}{ゐ}ますの!。
%
まるで
\ruby[g]{御酒}{ご しゆ}にでも
\ruby{醉}{よ}つた% 「醉」は原本通り「よ」で調整
\ruby{人}{ひと}のやうな
\ruby{顏}{かほ}を
\ruby{仕}{し}まして、
』

\原本頁{220-10}%
と
\ruby{先}{ま}づ
\ruby{答}{こた}へつ、

\原本頁{220-11}%
『
\ruby{何}{なん}だか
\ruby[g]{自{\換字{分}}}{じ ぶん}% ルビ調整(原本通り)非グループルビ
\ruby[g]{{\換字{勝}}手}{かつて }の
\ruby{不理屈}{ふ|り|くつ}でも
\ruby{云}{い}ひさうな
\ruby[g]{可厭}{い や }な
\ruby{人}{ひと}ですことネ
\改行% 校正作業の簡略化のため
エ。
』

\原本頁{221-2}%
と
\ruby{添}{そ}へたり。

\原本頁{221-3}%
『
マア
\ruby[g]{可厭}{い や }だことネエ!。
%
そんな
やうに
\ruby{見}{み}えるほど
\ruby{恐}{おそ}ろしい
\ruby{怒}{おこ}つた
\ruby{顏}{かほ}を
\ruby{仕}{し}て
\ruby{居}{ゐ}て?。
』

\原本頁{221-5}%
『
\ruby[g]{然樣}{さ う }なんですよ、
%
\ruby{怒}{おこ}り
\ruby{切}{き}つて
\ruby{居}{ゐ}る
といふ
\ruby{顏}{かほ}
つき
なんです。
%
それに
\ruby[g]{一體}{いつたい}が
\ruby[g]{地腫}{ぢ ばれ}の
\ruby{仕}{し}たやうな
\ruby{顏}{かほ}
なんで
しやうかネエ、
%
\ruby[g]{隨{\換字{分}}}{ずゐぶん}
おそろしく
\ruby{膨}{ふく}れ
かへつて、
%
\ruby[|g|]{宛然}{とんと}
‥‥
』

\原本頁{221-8}%
『
\ruby[|g|]{宛然}{とんと}
\ruby{何}{なん}なの?。
%
\ruby[g]{自{\換字{分}}}{じ ぶん}% ルビ調整(原本通り)非グループルビ
で
ばかり
\ruby[g]{承知}{しようち}して
\ruby{笑}{わら}つて。
』

\原本頁{221-9}%
『
マア
\ruby{止}{よ}して
\ruby{置}{お}きましやう
\ruby{他人樣}{ひ|と|さま}の
\ruby[g]{惡口}{わるくち}
なんか。
』

\原本頁{221-10}%
『
ホヽヽ
をかしな
\ruby{人}{ひと}ネエ、
%
\ruby[|g|]{一人}{ひとり}で
\ruby[g]{合點}{が てん}して
\ruby[|g|]{一人}{ひとり}で
\ruby[|g|]{可笑}{をかし}がつたり
なんか
して。
』

\原本頁{222-1}%
『
ホヽヽ、
%
でも
\ruby{惡}{わる}う
ございますもの。
』

\原本頁{2}%
\ruby[|g|]{宛然}{とんと}
\ruby[g]{河豚}{ふ ぐ }が
\ruby[g]{五合}{ごんがふ}も% 「五合」に関して:「五合(ごんごう)炊き」のようなご飯を炊くことに関してだろうか
\ruby[g]{引掛}{ひつか }けたやうと
\ruby{云}{い}はんと
\ruby{仕}{し}たりし
\ruby{歟}{か}、
%
\ruby{風{\換字{船}}玉}{ふう|せん|だま}に
\ruby{眼}{め}
\ruby{鼻}{はな}を
\ruby{付}{つ}けたやうと
\ruby{云}{い}はんと
\ruby{仕}{し}たりし
\ruby{歟}{か}、
%
\ruby{{\換字{終}}}{つひ}に
\ruby{口}{くち}を
\ruby{啓}{ひら}かねば
\ruby{知}{し}るものは
\ruby[g]{當人}{たうにん}の
\ruby{胸}{むね}のみ。

\原本頁{222-5}%
『
マア
\ruby[g]{勘{\換字{忍}}}{か に }して% 原文通り「勘忍」
\ruby{置}{お}いて
\ruby[||j>]{頂}{ちやう}
\ruby[||j>]{戴}{ だい}よ。
% \ruby{頂戴}{ちやう|だい}よ。
』

\原本頁{222-6}%
と
\ruby{輕}{かろ}く
\ruby{謝}{わ}びて
\ruby[g]{根問}{ね どひ}
さるゝを
\ruby{{\換字{遮}}}{さへぎ}り
\ruby{止}{とゞ}めつ
\ruby[g]{樓下}{し た }に
\ruby{去}{さ}りたり。

\原本頁{222-7}%
\ruby{人}{ひと}
\ruby{去}{さ}つて
\ruby[g]{小樓}{せうろう}
\ruby{靜}{しづか}に、
%
\ruby[g]{刳拔}{くりぬき}の
\ruby{桐}{きり}の
\ruby[g]{手爐}{てあぶり}の
\ruby{小}{ちひさ}なるを
\ruby{擁}{よう}して、
%
\ruby{{\換字{雪}}}{ゆき}と
\ruby{白}{しろ}き
\ruby[g]{蠣{\換字{灰}}}{かきばひ}に
\ruby{纖}{ほそ}き
\ruby[g]{火箸}{ひ ばし}
もて
\ruby{譯}{わけ}も
\ruby{無}{な}く
\ruby[g]{假名}{か な }
\ruby[g]{{\換字{文}}字}{も じ }を
\ruby{書}{か}きては
\ruby{{\換字{消}}}{け}し
\ruby{書}{か}きては
\ruby{{\換字{消}}}{け}しつ、
%
お
\ruby{龍}{りう}は
じつと
\ruby[<j||]{心}{こゝろ}
\ruby[g]{一筋}{ひとすぢ}に
\ruby[|g|]{彼方}{かなた}の
\ruby[|g|]{談話}{はなし}の
\ruby{何}{なに}となり
\ruby{行}{ゆ}くかを
\ruby{想}{おも}ひ
やりつゝ、

\原本頁{222-11}%
『
\ruby{彼}{あ}の
\ruby[g]{{\換字{勝}}手}{かつて }の
\ruby{{\換字{強}}}{つよ}い
\ruby{慾}{よく}の
\ruby{深}{ふか}い
お
\ruby[g]{師匠}{し よ }さんが
まあ
\ruby[g]{何樣}{ど ん }な
\ruby{事}{こと}を
お
\ruby{云}{い}ひ
のだらう。
%
そりやあ
もう
\ruby[g]{智慧}{ち ゑ }も
\ruby[g]{{\換字{分}}別}{ふんべつ}も
\ruby[g]{確固}{しつかり}として
おいでゞ、
%
\ruby{而}{さう}して
\ruby{言}{もの}
\ruby{語}{いひ}だつて
\ruby{拙}{まづ}い
\ruby{事}{こと}なんぞは
お
\ruby{云}{い}ひで
\ruby{無}{な}い
\ruby{姊}{ねえ}さんの
\ruby{事}{こと}だから
\改行% 校正作業の簡略化のため
、
%
\原本頁{223-3}\改行%
\ruby{何}{なに}を
\ruby[|g|]{對手}{むかふ}で
\ruby{云}{い}つたつて
\ruby{譯}{わけ}も
\ruby{無}{な}く
\ruby{捌}{さば}いて
お
\ruby[g]{仕舞}{し ま }ひ
なさるには
\ruby{{\換字{違}}}{ちが}ひ
\原本頁{223-4}\改行%
\ruby{無}{な}からうが、
%
\ruby[|g|]{對手}{あひて}が
\ruby[g]{無茶}{む ちや}な
\ruby{人}{ひと}なだけに
\ruby[g]{御困}{お こま}り
なさりは
\ruby{仕}{し}まいか
\原本頁{223-5}\改行%
\ruby{知}{し}らん。
%
\ruby[g]{自{\換字{分}}}{じ ぶん}の% ルビ調整(原本通り)非グループルビ
\ruby[g]{{\換字{勝}}手}{かつて }づくに
\ruby{掛}{か}けちやあ
\ruby[g]{理合}{り あひ}や
\ruby[||j>]{{\換字{情}}}{じやう}
\ruby[||j>]{合}{ あひ}に
% \ruby{{\換字{情}}合}{じやう|あひ}に
\ruby{構}{かま}つて
\ruby{居}{ゐ}る
\原本頁{223-6}\改行%
\ruby{樣}{やう}な
\ruby[g]{其樣}{そ ん }な
\ruby[||j>]{上}{じやう}
\ruby[||j>]{品}{ ひん}な
% \ruby{上品}{じやう|ひん}な
\ruby{人}{ひと}ぢやあ
\ruby{無}{な}さゝうな
\ruby{彼}{あ}の
\ruby{人}{ひと}を
\ruby[|g|]{對手}{あひて}にして、
%
くだらない
\ruby[g]{惡口}{あくたい}や
\ruby[g]{無理}{む り }な
\ruby[g]{{\換字{難}}題}{なんだい}でも
\ruby{云}{い}はれて
\ruby{困}{こま}つておいでゞは
\ruby{有}{あ}るまいか
\ruby{知}{し}ら。
%
\ruby[|g|]{對手}{むかふ}が
\ruby[g]{無茶}{む ちや}な
\ruby{人}{ひと}でさへ
\ruby{無}{な}ければ
\ruby{宜}{よ}い
のだ
けれども
\改行% 校正作業の簡略化のため
、
%
\原本頁{223-9}\改行%
\ruby{男}{をとこ}にでも
\ruby{何}{なん}でも
\ruby{負}{ま}けては
\ruby{居}{ゐ}ない
\ruby{樣}{やう}な
\ruby{氣}{き}の
\ruby{{\換字{強}}}{つよ}い
\ruby{人}{ひと}
では
あるし、
%
また
\ruby[g]{大變}{たいへん}に
\ruby{怒}{おこ}り
\ruby{立}{た}つて
\ruby{來}{き}た
のだ
とは
いふし、
%
\ruby[g]{一體}{いつたい}が
\ruby[g]{{\換字{勝}}手}{かつて }
の
ひどい
\ruby{甚}{ひど}い
\ruby{人}{ひと}だから、
%
いくら
\ruby[g]{姊樣}{ねえさん}が
\ruby[|g|]{怜悧}{りこう}でも% ルビ調整(原本通り)(りこう)
\ruby{扱}{あつか}ひ
\ruby{{\換字{難}}}{にく}いかと
\ruby{思}{おも}はれるが
\改行% 校正作業の簡略化のため
、
%
\原本頁{224-1}\改行%
まあ
どんな
\ruby{事}{こと}を
\ruby{云}{い}つて
\ruby{來}{き}たもので
\ruby{有}{あ}らう。
%
\ruby{{\換字{若}}}{も}し
\ruby{下}{くだ}らない
\ruby{事}{こと}を
\ruby{云}{い}つて
\ruby[g]{哦鳴}{が な }り
\ruby{立}{た}て
でも
された
\ruby{日}{ひ}には、
%
ほんとに
\ruby{姊}{ねえ}さんに
お
\ruby{氣}{き}の
\ruby{毒}{どく}
\原本頁{224-3}\改行%
で、
%
\ruby{妾}{わたし}は
まあ
\ruby[g]{何樣}{ど う }したら
\ruby{宜}{よ}からう。
%
\ruby[g]{何樣}{ど う }か
\ruby{彼}{あ}の
\ruby{人}{ひと}が
\ruby{姊}{ねえ}さんの
\ruby[g]{理解}{り かい}に
\ruby{折}{を}れて
\ruby{吳}{く}れゝば
\ruby{宜}{い}いが、
%
いくら
\ruby{姊}{ねえ}さんでも
\ruby[|g|]{對手}{あひて}が
\ruby{惡}{わる}いから
\改行% 校正作業の簡略化のため
、
%
\原本頁{224-5}\改行%
\ruby{何}{なん}だか
\ruby{覺束無}{おぼ|つか|な}い
やうな
\ruby{氣}{き}が
\ruby{仕}{し}て
ならない。
%
あゝ
\ruby{氣}{き}の
\ruby{揉}{も}める。
%
\ruby[g]{一體}{いつたい}
まあ
\ruby[g]{今日}{け ふ }の
\ruby{談}{はなし}は
\ruby[g]{何樣}{ど う }
\ruby[g]{結局}{をさまり}
が
ついて、
%
そして
\ruby{妾}{わたし}は
まあ
これから
\ruby[g]{{\換字{前}}{\換字{途}}}{さ き }
\ruby[g]{何樣}{ど う }なつて
\ruby{行}{ゆ}く
\ruby{身}{み}
なの
だらう。
』

\原本頁{224-8}%
と
\ruby{取}{と}り
\ruby{止}{と}まらず
\ruby{物}{もの}を
\ruby{案}{あん}じて
\ruby{耳}{みゝ}は
\ruby[|g|]{彼方}{かなた}
に
のみ
\ruby{走}{はし}れど、
%
\ruby[|g|]{距離}{あはひ}
\ruby{隔}{へだ}てたれば
\ruby{音}{おと}も
\ruby{聞}{きこ}えず、
%
\ruby{人}{ひと}も
あらぬが
\ruby{如}{ごと}く
\ruby{此}{この}
\ruby{家}{いへ}
\ruby{靜}{しづか}
なり。

\原本頁{224-10}%
やゝ
\ruby{久}{ひさし}くして
\ruby[|g|]{階段}{はしご}を
\ruby{上}{のぼ}り
\ruby{來}{く}る
\ruby{人}{ひと}の
\ruby{跫}{あし}
\ruby{音}{おと}し、
%
やがて
お
\ruby{春}{はる}は
\ruby{襖}{ふすま}を
\ruby{開}{ひら}きて
\ruby[g]{面を}{おもて }
\ruby{出}{いだ}せば、

\原本頁{225-1}%
『
\ruby{妾}{わたし}に
\ruby{來}{こ}いつて、
』

\原本頁{225-2}%
と
お
\ruby{龍}{りう}は
\ruby[|g|]{此方}{こなた}より
\ruby{問}{と}ひかけたり。

\原本頁{225-3}%
『
ハイ、
%
\ruby[g]{左樣}{さ う }
\ruby{仰}{おつし}あい
ましたので。
』

\原本頁{225-4}%
\ruby{今}{いま}さら
\ruby{胸}{むね}の
だくつく% 「だくつく」不安や驚きなどのために心臓がどきどきする。動悸どうきがする。
やう
おぼえて、
%
\ruby{話}{はなし}の
\ruby[g]{模樣}{も やう}を
\ruby{測}{はか}り
かねつ、
%
お
\ruby{龍}{りう}は
\ruby{却}{かへ}つて
\ruby{頓}{とみ}には
\ruby{起}{た}たざりけり。
