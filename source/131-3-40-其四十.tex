\Entry{其四十}

% メモ 校正終了 2024-05-18 2024-06-14
\原本頁{225-7}%
\ruby{我}{わ}が
\ruby{眼}{め}の
\ruby{力}{ちから}の
\ruby{及}{およ}ばぬ
\ruby{闇}{やみ}の
\ruby{夜}{よ}に
\ruby{歩}{あし}の
\ruby{{\換字{進}}}{すゝ}まぬ
やうに、
%
お
\ruby{龍}{りう}は
\ruby[|g|]{鬼胎}{おそれ}を
\ruby{懷}{いだ}き
ながら
\ruby{室}{へや}に
\ruby{入}{い}りて
\ruby{見}{み}れば、
%
\ruby{{\換字{朝}}日}{あさ|ひ}の
\ruby{光}{ひか}り
の
ある
ところ
\ruby[|g|]{自然}{おのづ}と
\ruby[||j>]{心}{こゝろ}
\ruby[||j>]{{\換字{強}}}{ づよ}き
% \ruby{心{\換字{強}}}{こゝろ|づよ}き
やうの
\ruby{{\換字{感}}}{おもひ}の
\ruby{仕}{し}て、
%
\ruby{先}{ま}づ
お
\ruby{彤}{とう}が
\ruby{{\換字{平}}常}{つ|ね}にも
\ruby{增}{ま}して
\ruby{位}{くらゐ}を
\ruby{取}{と}つて
\ruby{沈着}{おち|つ}き
\ruby{切}{き}つたる
\ruby{面}{おもて}の
\ruby{上}{うへ}に、
%
\ruby{掛}{かゝ}れる
\ruby{雲}{くも}の
\ruby{影}{かげ}だに
\ruby{無}{な}き
\ruby{樣}{さま}なるに
\ruby{氣}{き}も
\原本頁{228-1}\改行%
\ruby{勇}{いさ}み
\ruby{立}{た}ち、
%
\ruby{其}{そ}の
\ruby{横手}{よこ|て}の
\ruby{方}{かた}に、
%
やゝ
\ruby{下}{さが}りて
\ruby{坐}{すわ}りつ、
%
いろ〳〵の
\ruby[|g|]{思}{おもひ}に% 行末行頭の境界付近なので特例処置を施す
\ruby{小波}{さざ|なみ}の
\ruby{{\換字{文}}立}{あや|た}つ
\ruby{胸}{むね}を
\ruby{鎭}{しづ}めて、
%
\ruby{言葉}{こと|ば}は
\ruby{無}{な}けれど
\ruby{叮嚀}{てい|ねい}に
\ruby{挨拶}{あい|さつ}したり
\改行% 校正作業の簡略化のため
。
%
\原本頁{228-3}\改行%
ちらりと
\ruby{見}{み}し
お
\ruby{關}{せき}が
\ruby{顏色}{かほ|いろ}の、
%
お
\ruby{春}{はる}
お
\ruby{富}{とみ}が
\ruby{言葉}{こと|ば}とは
\ruby{{\換字{違}}}{ちが}ひて、
%
\ruby{思}{おも}ひ
の
ほか
\ruby{{\換字{平}}穩}{おだ|やか}
なる
やう
なる
に、
%
\ruby{心}{こゝろ}
ひそかに
\ruby{疑}{うたが}ひ
ながら
\ruby{徐}{しづか}に
\ruby{頭}{かしら}を
\ruby{擡}{あ}ぐれば、
%
これは
また
\ruby{如何}{い|か}なる
ことぞや
お
\ruby{關}{せき}は
\ruby{滿面}{まん|めん}に
\ruby{春}{はる}を
\ruby{湛}{たゝ}へて
\改行% 校正作業の簡略化のため
、
%
\原本頁{228-6}\改行%
さも〳〵
\ruby{親}{した}しげに
\ruby{{\換字{又}}}{また}
\ruby{懷}{なつ}かしげに、

\原本頁{226-7}%
『
マア
\ruby{立派}{りつ|ぱ}に
おなりな
こと!、
%
\ruby{吃驚}{びつ|くり}して
\ruby{仕舞}{し|ま}つたよ。
%
\ruby{少}{すこ}し
\ruby{粹}{いき}
だけれども
\ruby[|g|]{全然}{まるで}
\ruby{如是}{こ|れ}
ぢやあ
\ruby{立派}{りつ|ぱ}な
\ruby{御邸}{お|やしき}の
お
\ruby[||j>]{孃}{ぢやう}
\ruby[||j>]{樣}{ さま}だよ。
%
\ruby{好}{い}いことネエ、
%
お
\ruby{龍}{りう}ちやんは
\ruby{大變}{たい|へん}な
\ruby{幸福}{しあ|はせ}を%「幸福」ここは「は」
\ruby{御仕}{お|し}ねエ。
%
ほんとに
マア〳〵
\ruby{見{\換字{違}}}{み|ちが}へて
\ruby{仕舞}{し|ま}ふよ。
%
\ruby[|g|]{{\換字{平}}常}{ふだん}で
さへ
\ruby{斯樣}{か|う}
ぢやあ
\ruby{外}{そと}へ
でも
お
\ruby{出}{で}の
\ruby{時}{とき}は
マア
\ruby{何樣}{ど|ん}なに、
%
\ruby{見事}{み|ごと}に
お
\ruby{仕}{し}だらう!。
%
ほんとに
お
\ruby{{\換字{前}}}{まへ}さんは
マア
\ruby{大變}{たい|へん}な
\ruby{幸福}{しあ|はせ}な%「幸福」ここは「は」
\ruby{身}{み}に
おなりネエ。
%
\ruby{妾}{わたし}の
\ruby{處}{ところ}
なんぞに
\ruby{御在}{お|いで}で
ごらん、
%
\ruby{何程}{いく|ら}% ルビ調整(原本通り)非グループルビ
\ruby{妾}{わたし}が
やきもき
\ruby{思}{おも}つて
\ruby{好{\換字{遇}}}{よ|く}
して
あげた
からつて、
%
\ruby{精々}{せい|〴〵}
\ruby[|g|]{外出衣}{よそいき}が
\原本頁{227-3}\改行%
\ruby{銘仙}{めい|せん}か
\ruby{{\換字{節}}糸}{ふし|いと}% 玉繭からとった節の多い絹糸。玉糸。
\ruby{位}{ぐらゐ}の
\ruby{物}{もの}で、
%
それより
\ruby{上}{うへ}あ
\ruby{妾}{わたし}が
\ruby{千圓}{せん|ゑん}の
\ruby{籤}{くじ}にでも
\ruby{中}{あた}つたら
\ruby{知}{し}らないこと、
%
まあ〳〵
お
\ruby{{\換字{前}}}{まへ}さんに
\ruby[|g|]{御召縮緬}{おめし}
なんか
\ruby{引張}{ひつ|ぱ}らせて
あげる
ことあ
\ruby{出來}{で|き}つこは
\ruby{有}{あ}りやあ
\ruby{仕}{し}ないのに、
%
お
\ruby[||j>]{正}{しやう}
\ruby[||j>]{月}{ がつ}でも
% \ruby{正月}{しやう|がつ}でも
\ruby{無}{な}けりやあ
%
お
\ruby{{\換字{節}}句}{せつ|く}でも
\ruby{無}{な}い
\ruby{日}{ひ}に、
%
\ruby{然樣}{さ|う}いふ
\ruby{衣服}{な|り}を
\ruby{仕}{し}て
お
\ruby{在}{いで}の
やうに
おなりたあ、
%
\ruby[|g|]{眞實}{ほんと}に
マア
お
\ruby{{\換字{前}}}{まへ}さんは
\ruby{大變}{たい|へん}な
\ruby{幸福}{しあ|はせ}ネエ。%「幸福」ここは「は」
%
それも
\原本頁{227-8}\改行%
これも
\ruby[|g|]{悉皆}{みんな}
\ruby[|g|]{此方}{こちら}
\ruby{樣}{さま}の
お
\ruby{庇蔭}{か|げ}で、
%
\ruby[||j>]{私}{わたし}
\ruby[||j>]{等}{ たち}
% \ruby{私等}{わたし|たち}
の
\ruby{働}{はたら}きや
お
\ruby{{\換字{前}}}{まへ}さんの
\ruby{力}{ちから}
なんぞから
ぢやあ、
%
\ruby[<j||]{皺}{しやつ}% ルビ調整(長いルビ対策)
\ruby[<j||]{鉾}{ ちよ}
\ruby[||j>]{立}{こだち}を
% \ruby[<j>]{皺鉾立}{しやつち|よこ|だち}を
\ruby{仕}{し}たつて
\ruby{出來}{で|き}る
こつちやあ
\ruby{有}{あ}りませんよ。
%
だから
\ruby[|g|]{眞實}{ほんと}に
\ruby{仇}{あだ}や
\ruby{疎略}{おろ|そか}に% 096-3-05-其五.tex では 疎畧(おろ|そか) とある
\ruby{思}{おも}つちやあ
\ruby{濟}{す}みませんよ、
%
\ruby{何}{なん}でも
\原本頁{227-11}\改行%
\ruby[|g|]{此方}{こちら}
\ruby{樣}{さま}の
\ruby{仰}{おつし}あり
\ruby{次第}{し|だい}に
\ruby{身}{み}を
\ruby{{\換字{粉}}}{こ}にしても
\ruby{働}{はたら}か
\ruby{無}{な}くつちやあ
\ruby{濟}{す}みませんよ。
%
\ruby{{\換字{若}}}{も}し
お
\ruby{{\換字{前}}}{まへ}さんの
\ruby{仕方}{し|かた}に
そで
\ruby{無}{な}い
ことでも
\ruby{有}{あ}らう
もんなら、
%
\ruby[|g|]{此方}{こちら}
\ruby{樣}{さま}
ぢやあ
\ruby{容赦}{うつ|ちあ}つて
お
\ruby{置}{お}きなすつても
\ruby{私}{わたし}が
\ruby{承知}{しよう|ち}しや
\ruby{仕無}{し|な}い
\ruby[|g|]{心算}{つもり}で
\ruby{居}{ゐ}るからネ。
%
\ruby[|g|]{屹度}{きつと}
\ruby{妾}{わたし}が
\ruby{出}{で}て
\ruby{來}{き}て
お
\ruby{{\換字{前}}}{まへ}さんを
\ruby{折檻}{せつ|かん}すると
\原本頁{228-4}\改行%
\ruby{御思}{お|おも}ひよ。
%
ハヽホヽハヽヽ、
%
オヤマア
\ruby{此}{これ}あ
\ruby{下}{くだ}らない
ことを
\ruby{云}{い}つた
ものだネエ、
%
お
\ruby{龍}{りう}ちやんが
\ruby{如在}{じよ|さい}でも
\ruby{有}{あ}る
\ruby{人}{ひと}のやうに!。
%
ハヽハ、
だが、
%
ただ% 原本では非通り字表記
\ruby{此}{これ}あ
\ruby{其程}{それ|ほど}までに
\ruby{私}{わたし}あ
\ruby[|g|]{此方}{こちら}
\ruby{樣}{さま}を
お
\ruby{{\換字{前}}}{まへ}さんに
\ruby{取}{と}つちやあ
\ruby{有}{あ}りがたいと
\ruby{思}{おも}つてる
といふ
\ruby[||j>]{心}{こゝろ}
\ruby[||j>]{持}{ もち}を
% \ruby{心持}{こゝろ|もち}を
\ruby{打撒}{ぶち|ま}けた
ばかり
なんさ
\改行% 校正作業の簡略化のため
。
%
\原本頁{228-8}\改行%
ほんとに
\ruby[||j>]{戲}{じやう}
\ruby[||j>]{談}{ だん}ぢやあ
% \ruby{戲談}{じやう|だん}ぢやあ
\ruby{有}{あ}りませんよ、
%
\ruby{身}{み}に
\ruby{染}{し}みて
\ruby{有}{あ}り
\ruby{{\換字{難}}}{がた}いと
\ruby{思}{おも}はなくつちやあ
\ruby{罰}{ばち}が
\ruby{當}{あた}りますよ。
%
\ruby{妾}{わたし}もネエ、
%
お
\ruby{{\換字{前}}}{まへ}さん
から
\ruby{緣}{えん}を
\ruby{牽}{ひ}いた
お
\ruby{蔭}{かげ}で
もつてネエ、
%
\ruby[|g|]{此方}{こちら}
\ruby{樣}{さま}の
やうな
\ruby{結構}{けつ|こう}な
\ruby{方}{かた}にも
お
\ruby{目}{め}に
かかつたり、% ルビ調整(原本通り)非踊り字表記(行末行頭の境界付近)
%
それから
\ruby{{\換字{又}}}{また}
\ruby{種々}{いろ|〳〵}
\ruby{優}{やさ}しく
\ruby{仰}{おつし}あつて
\ruby{戴}{いたゞ}いたりなんかして
\改行% 校正作業の簡略化のため
、
%
\原本頁{229-1}\改行%
\ruby{此樣}{こ|ん}な
\ruby{嬉}{うれ}しい
ことは
\ruby{有}{あ}りません
のですよ。
%
\ruby{何樣}{ど|う}かネエ
お
\ruby{{\換字{前}}}{まへ}さん
からも
\ruby{能}{よう}く
\ruby{御禮}{お|れい}を
\ruby{申}{まを}してネ、
%
そしてネ、
%
\ruby{今後}{これ|から}も
\ruby{時々}{とき|〴〵}は
\ruby{御邪{\換字{魔}}}{お|じや|ま}でも
\ruby{御出入}{お|で|いり}を
させて
\ruby{戴}{いたゞ}く
やうにネ、
%
\ruby{何樣}{ど|う}か
お
\ruby{{\換字{前}}}{まへ}さん
からも
\ruby{能}{よう}く
\ruby{願}{ねが}つて
\ruby{下}{くだ}さいよ。
%
そして
\ruby{妾}{わたし}あ
\ruby{{\換字{又}}}{また}
お
\ruby{{\換字{前}}}{まへ}さんに
\ruby{一}{ひと}つ
\ruby{御願}{お|ねがひ}が
あるのだがネ。
%
ナアニ
\ruby{面倒}{めん|だう}な
\ruby{事}{こと}でも
\ruby{何}{なん}でも
\ruby{無}{な}いんで、
%
ただ% 原本では非通り字表記
\ruby{今度}{こん|ど}
\ruby{他}{よそ}へ
\ruby{出}{で}る
\ruby{時}{とき}
\原本頁{229-6}\改行%
\ruby{一寸}{ちよ|いと}
\ruby{囘}{まは}り% 原本通り「囘」
\ruby{{\換字{道}}}{みち}を
\ruby{仕}{し}てネ、
%
\ruby{汚}{きたな}くつても
\ruby{妾}{わたし}の
\ruby{宅}{うち}へ
\ruby{寄}{よ}つて
\ruby{御茶}{お|ちや}の
\ruby{一}{ひと}つも
\ruby{飮}{の}んで
\ruby{行}{い}つて
\ruby{貰}{もら}ひたい
のさ。
%
ただ% 原本では非通り字表記
もう、
%
お
\ruby{{\換字{前}}}{まへ}さんが
\ruby[|g|]{如是}{こんな}に
\ruby{立派}{りつ|ぱ}に
おなり
だといふ
ことを
\ruby{誰}{だれ}か
\ruby{知}{し}らに
\ruby{見}{み}せて、
%
\ruby{私}{わたし}が
\ruby{腹一杯}{はら|いつ|ぱい}に
\ruby{天狗}{てん|ぐ}を
\ruby{云}{い}つて
\ruby{威張}{ゐ|ばり}たい
んだから。
%
ア、
%
それから
\ruby{{\換字{又}}}{また}、
%
\ruby{此樣}{こ|ん}なに
\ruby{何不足}{なに|ふ|そく}ない
\ruby{結構}{けつ|こう}なところへ
\ruby{御}{お}いでのだから、
%
\ruby{何}{なに}も
\ruby{彼}{か}も
\ruby{要}{い}ることは
\ruby{御有}{お|あ}りぢや
\ruby{無}{な}からうがネエ、
%
\ruby{私}{わたし}の
ところに
お
\ruby{{\換字{前}}}{まへ}さんの
こざ〳〵した
\原本頁{230-1}\改行%
\ruby{物}{もの}や
\ruby{何}{なん}かが
そつくり
\ruby{仕}{し}て
\ruby{居}{ゐ}る、
%
\ruby{彼品}{あ|れ}は
\ruby[|g|]{悉皆}{みんな}
\ruby[|g|]{明日}{あした}に
でも
\ruby{持}{も}たして
\ruby{{\換字{遺}}}{よこ}しますからネ。
』

\原本頁{230-3}%
と、
%
\ruby[||j>]{{\換字{追}}}{つゐ}
\ruby[||j>]{從}{しよう}やら
% \ruby{{\換字{追}}從}{つゐ|しよう}やら
\ruby{諛辭}{せ|じ}やらを
\ruby{混淆}{ごた|まぜ}に、%「淆」コウ. まじる・みだす・にごる. 意味.
%
\ruby{叮嚀}{てい|ねい}と
\ruby{粗略}{ぞん|ざい}との
\ruby{虎斑}{とら|ぶち}の
\ruby{言葉}{こと|ば}
\ruby{{\換字{遣}}}{づか}ひに、
%
\ruby{何}{なに}かは
\ruby{知}{し}らず
\ruby{無上}{む|しやう}に
\ruby[|g|]{機{\換字{嫌}}}{きげん}
\ruby{好}{よ}く
\ruby[|g|]{饒舌}{しやべ}り
\ruby{立}{た}てられ、
%
お
\ruby{龍}{りう}は
ただ% 原本では非通り字表記
ただ% 原本では非通り字表記
\ruby{{\換字{煙}}}{けむ}に
\ruby{卷}{ま}かれて、
%
すべてが
\ruby{我}{わ}が
\ruby{思}{おもひ}の
ほか
なりしに
\ruby{{\換字{返}}辭}{へん|じ}に
さへ
\ruby{{\換字{迷}}}{まど}ひ
つゝ、
%
\ruby{如何}{い|か}に
\ruby[|g|]{應對}{あしら}ひて
\ruby{如是}{か|く}は
\ruby{虎}{とら}の
やうなるべき
お
\ruby{關}{せき}をば
\改行% 校正作業の簡略化のため
、
%
\原本頁{230-7}\改行%
\ruby{甘}{あま}へて
\ruby{戲}{ざ}るゝ
\ruby{猫}{ねこ}の
やうには
\ruby{仕}{し}たりしかと、
%
\ruby{不審}{いぶ|かし}さに
\ruby{堪}{た}へぬ
\ruby{眼}{め}を
\ruby{張}{は}つて
お
\ruby{彤}{とう}を
\ruby{見}{み}たり。
