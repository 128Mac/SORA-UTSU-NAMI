\Entry{其四十}

\ruby{我}{わ}が
\ruby{眼}{め}の
\ruby{力}{ちから}の
\ruby{及}{およ}ばぬ
\ruby{闇}{やみ}の
\ruby{夜}{よ}に
\ruby{歩}{あし}の
\ruby{{\換字{進}}}{すゝ}まぬやうに、
お
\ruby{龍}{りう}は
\ruby{鬼胎}{おそ|れ}を
\ruby{懷}{いだ}きながら
\ruby{室}{へや}に
\ruby{入}{い}りて
\ruby{見}{み}れば、
\ruby{{\換字{朝}}日}{あさ|ひ}の
\ruby{光}{ひか}りのあるところ
\ruby{自然}{おの|づ}と
\ruby{心{\換字{強}}}{こゝろ|づよ}きやうの
\ruby{感}{おもひ}の
\ruby{仕}{し}て、
\ruby{先}{ま}づ
お
\ruby{彤}{とう}が
\ruby{{\換字{平}}常}{つ|ね}にも
\ruby{增}{ま}して
\ruby{位}{くらゐ}を
\ruby{取}{と}つて
\ruby{沈着}{おち|つ}き
\ruby{切}{き}つたる
\ruby{面}{おもて}の
\ruby{上}{うへ}に、
\ruby{掛}{かゝ}れる
\ruby{雲}{くも}の
\ruby{影}{かげ}だに
\ruby{無}{な}き
\ruby{樣}{さま}なるに
\ruby{氣}{き}も
\ruby{勇}{いさ}み
\ruby{立}{た}ち、
\ruby{其}{そ}の
\ruby{横手}{よこ|て}の
\ruby{方}{かた}に、やゝ
\ruby{下}{さが}りて
\ruby{坐}{すわ}りつ、いろ〳〵の
\ruby{思}{おもひ}に
\ruby{小波}{さざ|なみ}の
\ruby{{\換字{文}}立}{あや|た}つ
\ruby{胸}{むね}を
\ruby{鎭}{しづ}めて、
\ruby{言葉}{こと|ば}は
\ruby{無}{な}けれど
\ruby{叮嚀}{てい|ねい}に
\ruby{挨拶}{あい|さつ}したり。

ちらりと
\ruby{見}{み}し
お
\ruby{關}{せき}が
\ruby{顏色}{かほ|いろ}の、
お
\ruby{春}{はる}
お
\ruby{富}{とみ}が
\ruby{言葉}{こと|ば}とは
\ruby{{\換字{違}}}{ちが}ひて、
\ruby{思}{おも}ひのほか
\ruby{{\換字{平}}穩}{おだ|やか}なるやうなるに、
\ruby{心}{こゝろ}ひそかに
\ruby{疑}{うたが}ひながら
\ruby{徐}{しづか}に
\ruby{頭}{かしら}を
\ruby{擡}{あ}ぐれば、これはまた
\ruby{如何}{い|か}なることぞや
お
\ruby{關}{せき}は
\ruby{滿面}{まん|めん}に
\ruby{春}{はる}を
\ruby{湛}{たゝ}へて、さも〳〵
\ruby{親}{した}しげに
\ruby{{\換字{又}}}{また}
\ruby{懷}{なつ}かしげに、

『マア
\ruby{立派}{りつ|ぱ}におなりなこと!、
\ruby{吃驚}{びつ|くり}して
\ruby{仕舞}{し|ま}つたよ。
\ruby{少}{すこ}し
\ruby{粹}{いき}だけれども
\ruby{全然}{まる|で}
\ruby{如是}{こ|れ}ぢやあ
\ruby{立派}{りつ|ぱ}な
\ruby{御邸}{お|やしき}の
お
\ruby{孃樣}{ぢやう|さま}だよ。
\ruby{好}{い}いことネエ、
お
\ruby{龍}{りう}ちやんは
\ruby{大變}{たい|へん}な
\ruby{幸福}{しあ|わせ}を
\ruby{御仕}{お|し}ねエ。
ほんとにマア〳〵
\ruby{見{\換字{違}}}{み|ちが}へて
\ruby{仕舞}{し|ま}ふよ。
\ruby{{\換字{平}}常}{ふだ|ん}でさへ
\ruby{斯樣}{か|う}ぢやあ
\ruby{外}{そと}へでも
お
\ruby{出}{で}の
\ruby{時}{とき}はマア
\ruby{何樣}{ど|ん}なに、
\ruby{見事}{み|ごと}に
お
\ruby{仕}{し}だらう!。
ほんとにお
\ruby{{\換字{前}}}{まへ}さんはマア
\ruby{大變}{たい|へん}な
\ruby{幸福}{しあ|わせ}な
\ruby{身}{み}におなりネエ。

\ruby{妾}{わたし}の
\ruby{處}{ところ}なんぞに
\ruby{御在}{お|いで}でごらん、
\ruby{何程}{いく|ら}
\ruby{妾}{わたし}がやきもき
\ruby{思}{おも}つて
\ruby{好{\換字{遇}}}{よ|く}してあげたからつて、
\ruby{精々}{せい|〴〵}
\ruby{外出衣}{よ|そ|いき}が
\ruby{銘仙}{めい|せん}か
\ruby{節糸}{ふし|いと}% 玉繭からとった節の多い絹糸。玉糸。
\ruby{位}{ぐらゐ}の
\ruby{物}{もの}で、それより
\ruby{上}{うへ}あ
\ruby{妾}{わたし}が
\ruby{千圓}{せん|ゑん}の
\ruby{籤}{くじ}にでも
\ruby{中}{あた}つたら
\ruby{知}{し}らないこと、まあ〳〵
お
\ruby{{\換字{前}}}{まへ}さんに
\ruby{御召縮緬}{お|め|し|}なんか
\ruby{引張}{ひつ|ぱ}らせてあげることあ
\ruby{出來}{で|き}つこは
\ruby{有}{あ}りやあ
\ruby{仕}{し}ないのに、
お
\ruby{正月}{しやう|がつ}でも
\ruby{無}{な}けりやあ
お
\ruby{節句}{せつ|く}でも
\ruby{無}{な}い
\ruby{日}{ひ}に、
\ruby{然樣}{さ|う}いふ
\ruby{衣服}{な|り}を
\ruby{仕}{し}て
お
\ruby{在}{いで}のやうにおなりたあ、
\ruby{眞實}{ほん|と}にマア
お
\ruby{{\換字{前}}}{まへ}さんは
\ruby{大變}{たい|へん}な
\ruby{幸福}{しあ|わせ}ネエ。
それもこれも
\ruby{悉皆}{みん|な}
\ruby{此方樣}{こち|ら|さま}の
お
\ruby{庇蔭}{か|げ}で、
\ruby{私等}{わたし|ら}の
\ruby{働}{はたら}きや
お
\ruby{{\換字{前}}}{まへ}さんの
\ruby{力}{ちから}なんぞからぢやあ、
\ruby{皺鉾立}{しやつ|ちよこ|だち}を
\ruby{仕}{し}たつて
\ruby{出來}{で|き}るこつちやあ
\ruby{有}{あ}りませんよ。
だから
\ruby{眞實}{ほん|と}に
\ruby{仇}{あだ}や
\ruby{疎略}{おろ|そか}に
\ruby{思}{おも}つちやあ
\ruby{濟}{す}みませんよ、
\ruby{何}{なん}でも
\ruby{此方樣}{こち|ら|さま}の
\ruby{仰}{おつし}あり
\ruby{次第}{し|だい}に
\ruby{身}{み}を
\ruby{{\換字{粉}}}{こ}にしても
\ruby{働}{はたら}か
\ruby{無}{な}くつちやあ
\ruby{濟}{す}みませんよ。
\ruby{若}{も}し
お
\ruby{{\換字{前}}}{まへ}さんの
\ruby{仕方}{し|かた}にそで
\ruby{無}{な}いことでも
\ruby{有}{あ}らうもんなら、
\ruby{此方樣}{こち|ら|さま}ぢやあ
\ruby{容赦}{うつ|ちあ}つて
お
\ruby{置}{お}きなすつても
\ruby{私}{わたくし}が
\ruby{承知}{しよう|ち}しや
\ruby{仕無}{し|な}い
\ruby{心算}{つも|り}で
\ruby{居}{ゐ}るからネ。

\ruby{屹度}{きつ|と}
\ruby{妾}{わたし}が
\ruby{出}{で}て
\ruby{來}{き}て
お
\ruby{{\換字{前}}}{まへ}さんを
\ruby{折檻}{せつ|かん}すると
\ruby{御思}{お|おも}ひよ。
ハヽホヽハヽヽ、オヤマア
\ruby{此}{これ}あ
\ruby{下}{くだ}らないことを
\ruby{云}{い}つたものだネエ、
お
\ruby{龍}{りう}ちやんが
\ruby{如在}{じよ|ざい}でも
\ruby{有}{あ}る
\ruby{人}{ひと}のやうに!。
ハヽハ、だが、たゞ
\ruby{此}{これ}あ
\ruby{其程}{それ|ほど}までに
\ruby{私}{わたし}あ
\ruby{此方樣}{こち|ら|さま}を
お
\ruby{{\換字{前}}}{まへ}さんに
\ruby{取}{と}つちやあ
\ruby{有}{あ}りがたいと
\ruby{思}{おも}つてるといふ
\ruby{心持}{こゝろ|もち}を
\ruby{打撒}{ぶち|ま}けたばかりなんさ。
ほんとに
\ruby{戯談}{じやう|だん}ぢやあ
\ruby{有}{あ}りませんよ、
\ruby{身}{み}に
\ruby{染}{し}みて
\ruby{有}{あ}り
\ruby{難}{がた}いと
\ruby{思}{おも}はなくつちやあ
\ruby{罰}{ばち}が
\ruby{當}{あた}りますよ。
\ruby{妾}{わたし}もネエ、
お
\ruby{{\換字{前}}}{まへ}さんから
\ruby{緣}{えん}を
\ruby{牽}{ひ}いた
お
\ruby{蔭}{かげ}でもつてネエ、
\ruby{此方樣}{こち|ら|さま}のやうな
\ruby{結構}{けつ|こう}な
\ruby{方}{かた}にも
お
\ruby{目}{め}にかかつたり、それから
\ruby{{\換字{又}}}{また}
\ruby{種々}{いろ|〳〵}
\ruby{優}{やさ}しく
\ruby{仰}{おつし}あつて
\ruby{戴}{いたゞ}いたりなんかして、
\ruby{此樣}{こ|んな}な
\ruby{嬉}{うれ}しいことは
\ruby{有}{あ}りませんのですよ。
\ruby{何樣}{ど|う}かネエ
お
\ruby{{\換字{前}}}{まへ}さんからも
\ruby{能}{よう}く
\ruby{御禮}{お|れい}を
\ruby{申}{まを}してネ、そしてネ、
\ruby{今後}{これ|から}も
\ruby{時々}{とき|〴〵}は
\ruby{御邪魔}{お|じや|ま}でも
\ruby{御出入}{お|で|いり}をさせて
\ruby{戴}{いたゞ}くやうにネ、
\ruby{何樣}{ど|う}か
お
\ruby{{\換字{前}}}{まへ}さんからも
\ruby{能}{よう}く
\ruby{願}{ねが}つて
\ruby{下}{くだ}さいよ。
そして
\ruby{妾}{わたし}あ
\ruby{{\換字{又}}}{また}
お
\ruby{{\換字{前}}}{まへ}さんに
\ruby{一}{ひと}つ
\ruby{御願}{お|ねがひ}があるのだがネ。
ナアニ
\ruby{面倒}{めん|だう}な
\ruby{事}{こと}でも
\ruby{何}{なん}でも
\ruby{無}{な}いんで、たゞ
\ruby{今度他}{こん|ど|よそ}へ
\ruby{出}{で}る
\ruby{時}{とき}
\ruby{一寸}{ちよ|いと}
\ruby{囘}{まは}り
\ruby{{\換字{道}}}{みち}を
\ruby{仕}{し}てネ、
\ruby{汚}{きたな}くつても
\ruby{妾}{わたし}の
\ruby{宅}{うち}へ
\ruby{寄}{よ}つて
\ruby{御茶}{お|ちや}の
\ruby{一}{ひと}つも
\ruby{飮}{の}んで
\ruby{行}{い}つて
\ruby{貰}{もら}ひたいのさ。
たゞもう、お
\ruby{{\換字{前}}}{まへ}さんが
\ruby{如是}{こ|んな}に
\ruby{立派}{りつ|ぱ}におなりだといふことを
\ruby{誰}{たれ}か
\ruby{知}{し}らに
\ruby{見}{み}せて、
\ruby{私}{わたくし}が
\ruby{腹一杯}{はら|いつ|ぱい}に
\ruby{天狗}{てん|ぐ}を
\ruby{云}{い}つて
\ruby{威張}{ゐ|ばり}たいんだから。
ア、それから
\ruby{{\換字{又}}}{また}、
\ruby{此樣}{こ|ん}なに
\ruby{何不足}{なに|ふ|そく}ない
\ruby{結構}{けつ|こう}なところへ
\ruby{御}{お}いでのだから、
\ruby{何}{なに}も
\ruby{彼}{か}も
\ruby{要}{い}ることは
\ruby{御有}{お|あ}りぢや
\ruby{無}{な}からうがネエ、
\ruby{私}{わたくし}のところに
お
\ruby{{\換字{前}}}{まへ}さんのこざ〳〵した
\ruby{物}{もの}や
\ruby{何}{なん}かゞそつくり
\ruby{仕}{し}て
\ruby{居}{ゐ}る、
\ruby{彼品}{あ|れ}は
\ruby{悉皆}{みん|な}
\ruby{明日}{あし|た}にでも
\ruby{持}{も}たして
\ruby{{\換字{遺}}}{よこ}しますからネ。
』

と、
\ruby{{\換字{追}}從}{つゐ|しよう}やら
\ruby{諛辭}{せ|じ}やらを
\ruby{混滯}{ごた|まぜ}に、
\ruby{叮嚀}{てい|ねい}と
\ruby{粗略}{ぞん|ざい}との
\ruby{虎斑}{とら|ぶち}の
\ruby{言葉}{こと|ば}
\ruby{{\換字{遣}}}{づか}ひに、
\ruby{何}{なに}か
\ruby{知}{し}らず
\ruby{無上}{むし|よう}に
\ruby{機{\換字{嫌}}好}{き|げん|よ}く
\ruby{饒舌}{しや|べ}り
\ruby{立}{た}てられ、
お
\ruby{龍}{りう}はたゞたゞ
\ruby{煙}{けむ}に
\ruby{卷}{ま}かれて、すべてが
\ruby{我}{わ}が
\ruby{思}{おもひ}のほかなりしに
\ruby{{\換字{返}}辭}{へん|じ}にさへ
\ruby{{\換字{迷}}}{まど}ひつゝ、
\ruby{如何}{い|か}に
\ruby{應對}{あし|ら}ひて
\ruby{如是}{か|く}は
\ruby{虎}{とら}のやうなるべき
お
\ruby{關}{せき}をば、
\ruby{甘}{あま}へて
\ruby{戲}{ざ}るゝ
\ruby{猫}{ねこ}のやうには
\ruby{仕}{し}たりしかと、
\ruby{不審}{いぶ|か}さに
\ruby{堪}{た}へぬ
\ruby{眼}{め}を
\ruby{張}{は}つて
お
\ruby{彤}{とう}を
\ruby{見}{み}たり。

