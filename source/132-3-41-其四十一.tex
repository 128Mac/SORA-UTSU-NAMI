\Entry{其四十一}

\ruby{尾}{を}もあらば
\ruby{振}{ふ}つて
\ruby{見}{み}すべき
\ruby{程{\換字{悅}}}{ほど|よろこ}びかへつて、お
\ruby{關}{せき}はおのが
\ruby{賤}{いや}しき
\ruby{詞}{ことば}の
\ruby{端々}{はし|〴〵}に
\ruby{下卑}{げ|び}たる
\ruby{心}{こゝろ}の
\ruby{隈々}{くま|〴〵}を
\ruby{殘}{のこ}りなく
\ruby{露}{あらは}すをも
\ruby{顧}{かへり}みず、
\ruby{知}{し}ら
\ruby{知}{じ}らしきまでお
\ruby{彤}{とう}お
\ruby{龍}{りう}に
\ruby{諛辭}{おは|むき}の
\ruby{數々}{かず|〴〵}を
\ruby{云}{い}ひ
\ruby{盡}{つく}したる
\ruby{後}{のち}、あまり
\ruby{長居}{なが|ゐ}して
\ruby[g]{愛想}{あいそ}をつかされてはと
\ruby{思}{おも}ひてか、
\ruby{但}{たゞ}しはお
\ruby{彤}{とう}が
\ruby{餘}{あま}り
\ruby{多}{おほ}くも
\ruby{言}{ものい}はず
\ruby{餘}{あま}り
\ruby{多}{おほ}くも
\ruby{笑}{わら}はで、いつまでも
\ruby{面正}{おも|たゞ}しくなし
\ruby{居}{ゐ}るに、
\ruby{流石}{さす|が}の
\ruby[g]{{\換字{勝}}手者}{かつてもの}も
\ruby{氣}{き}の
\ruby{置}{お}けてか、
\ruby{吳々}{くれ|〴〵}も
\ruby{此後}{この|のち}とも
\ruby{疎}{うと}み
\ruby{棄}{す}てられぬやうにと
\ruby{頼}{たの}み
\ruby{聞}{きこ}えて、お
\ruby{富}{とみ}お
\ruby{春}{はる}にまで
\ruby{無理捏}{む|り|づく}ねに
\ruby{捏}{つく}ねつけたるやうの
\ruby[g]{愛想}{あいそ}の
\ruby{有}{あ}る
\ruby{限}{かぎ}りを
\ruby{振}{ふ}り
\ruby{撒}{ま}き、
\ruby{來}{きた}りし
\ruby{時}{とき}の
\ruby{荒々}{あら|〳〵}しかりしには
\ruby{引}{ひき}かへ、
\ruby{歸}{かへ}る
\ruby{時}{とき}には
\ruby{疊}{たゝみ}もそつと
\ruby{踏}{ふ}むやうにして
\ruby{漸}{やうや}くに
\ruby{出去}{いで|さ}れば、
\ruby[g]{其背影}{そのうしろかげ}の
\ruby{見}{み}えずなるや
\ruby{否}{いな}や、
\ruby{{\換字{送}}}{おく}つて
\ruby{出}{い}でたるお
\ruby{春}{はる}は
\ruby{堪}{こら}へかねて、フヽワヽと
\ruby{笑}{わら}ひ
\ruby{出}{だ}し、

『マア、
\ruby{何}{なん}ていふ
\ruby{現金}{げん|きん}な
\ruby[g]{得手{\換字{勝}}手}{ゑてかつて}な
\ruby{人}{ひと}でしやう!。
\ruby{來}{き}た
\ruby{時}{とき}にやあ
\ruby[g]{宛然狂犬見}{まるでやまひいぬみ}た
\ruby{樣}{やう}に、
\ruby{手}{て}でも
\ruby{出}{だ}したら
\ruby{噬}{く}ひつきさうな
\ruby{怖}{おそろ}しい
\ruby{顏}{かほ}を
\ruby{仕}{し}て
\ruby{來}{き}て、
\ruby{歸}{かへ}る
\ruby{時}{とき}にやあ
\ruby{小狗}{ちん|ころ}かなんかの
\ruby{樣}{やう}にころ〳〵して
\ruby{{\換字{悅}}}{よろこん}で
\ruby{行}{ゆ}くんですもの!。
おゝ
\ruby{可厭}{い|や}なをかしなお
\ruby{婆}{ばあ}さんだこと!。
』

と、
\ruby{引{\換字{返}}}{ひつ|かへ}しながらお
\ruby{富}{とみ}と
\ruby{顏}{かほ}を
\ruby{見合}{み|あは}せて
\ruby{云}{い}ふを、これも
\ruby{何處}{ど|こ}やらに
\ruby{笑}{わらひ}を
\ruby{含}{ふく}みながらも
\ruby{叱}{しか}るが
\ruby{如}{ごと}く
\ruby[g]{上眼}{うはめ}つかひして
\ruby{制}{せい}し
\ruby{止}{とど}めつ、お
\ruby{富}{とみ}は
\ruby{小聲}{こ|ごゑ}に、

『でも
\ruby{彼樣}{あ|あ}いふのが
\ruby{正直}{しやう|ぢき}つて
\ruby{云}{い}ふんで、
\ruby{可愛}{か|はい}い
\ruby[g]{性{\換字{分}}}{しやうぶん}なんですかも
\ruby{知}{し}れませんよ。
\ruby{罪}{つみ}も
\ruby{何}{なん}も
\ruby{無}{な}くつてネエ。
』

と
\ruby{冷}{ひや}やかに
\ruby{罵}{のゝし}る。
お
\ruby{春}{はる}は
\ruby{此語}{こ|れ}を
\ruby{聞}{き}いて
\ruby{{\換字{猶}}笑}{なほ|わら}ひ
\ruby{止}{や}まず、

『
\ruby{左樣}{さ|う}ネエ、
\ruby{毫}{ちつと}も
\ruby{奧底}{おく|そこ}が
\ruby{無}{な}いんですからネエ。
だが、
\ruby{左樣}{さ|う}いへばお
\ruby{富}{とみ}さんなんぞは
\ruby{大變}{たい|へん}に
\ruby{可愛}{か|はい}らしくない
\ruby{人}{ひと}なの?。
\ruby{何}{なん}でも
\ruby{{\換字{遠}}慮深}{ゑん|りよ|ぶか}くつて、
\ruby{愼}{つゝし}みが
\ruby{深}{ふか}いのですもの!。
』

と
\ruby{小聲}{こ|ごゑ}に
\ruby{語}{かた}り
\ruby{合}{あ}ふ
\ruby{此方}{こな|た}は
\ruby{此方}{こな|た}、
\ruby{彼方}{かな|た}は
\ruby{彼方}{かな|た}にて、お
\ruby{龍}{りう}は
\ruby{先}{ま}づ
\ruby{訝}{いぶか}り
\ruby{糺}{たゞ}し、

『
\ruby{姊}{ねえ}さん、
\ruby{彼}{あ}の
\ruby{人}{ひと}を
\ruby{何樣}{ど|う}なすつたの?。
』

と
\ruby{問}{と}へば、お
\ruby{彤}{とう}は
\ruby{微}{すこ}しく
\ruby{笑}{ゑみ}
\ruby{含}{ふく}み、

『
\ruby{何故}{な|ぜ}?。
\ruby{別}{べつ}に
\ruby{何樣}{ど|う}も
\ruby{仕}{し}やうは
\ruby{有}{あ}りやあ
\ruby{仕無}{し|な}いぢや
\ruby{無}{な}いか。
』

と
\ruby{澄}{す}まし
\ruby{切}{き}つて
\ruby{云}{い}ふ。

『でも
\ruby{大變}{たい|へん}に
\ruby{怒}{おこ}つて
\ruby{來}{き}たといふのに、
\ruby{妾}{わたし}が
\ruby{下}{お}りて
\ruby{來}{き}て
\ruby{見}{み}りやあ、
\ruby{毫}{ちつと}もそんな
\ruby{樣子}{やう|す}は
\ruby{無}{な}くつて、
\ruby{怒}{おこ}るどころぢやあ
\ruby{無}{な}く、
\ruby{莞爾}{にこ|〳〵}してばかり
\ruby{居}{ゐ}るぢやあ
\ruby{有}{あ}りませんか。
』

『そりやあ
\ruby{何}{なに}お
\ruby{前}{まへ}、
\ruby{何}{なんに}も
\ruby{不思議}{ふ|し|ぎ}は
\ruby{有}{あ}りやあ
\ruby{仕}{し}ないはネ。
\ruby[g]{些少}{ぽつちり}ばかり
\ruby{金錢}{も|の}を
\ruby{與}{や}つたので
\ruby[g]{如是{\換字{悅}}}{ああよろこ}んで
\ruby{仕舞}{し|ま}つたのさ。
』

『
\ruby[g]{金錢}{おかね}を?。
』

『あゝ。
』

『あら!。
\ruby{何}{なに}も
\ruby{姊}{ねえ}さんがそんなものお
\ruby{與}{や}んなさる
\ruby{理由}{わ|け}は
\ruby{無}{な}いぢやあ
\ruby{有}{あ}りませんか。
さうして
\ruby{姊}{ねえ}さんも
\ruby{彼}{あ}の
\ruby{靜岡}{しづ|をか}のに、
お
\ruby{金}{かね}は
\ruby{惜}{をし}かないけれども
\ruby{取}{と}られるのは
\ruby{業腹}{ごふ|はら}だから、と
\ruby[g]{御自{\換字{分}}}{ごじぶん}でちやんと
\ruby[g]{然樣仰}{さうおつし}あつたぢやあ
\ruby{有}{あ}りませんか?。
』

『そりやあお
\ruby{前}{まへ}の
\ruby{叔母}{を|ば}さんには
\ruby{然樣云}{さ|う|い}つたけれどもネ、
\ruby{彼}{あ}りやあ
\ruby{云}{い}はば
\ruby{叔母}{を|ば}さんの
\ruby{氣}{き}の
\ruby{濟}{す}むやうに
\ruby{云}{い}つたゞけの
\ruby{事}{こと}でネ、
\ruby{何}{なに}も
\ruby{妾}{わたし}あ
\ruby{彼樣}{あ|ん}な
\ruby{慾張}{よく|ば}りの
\ruby{人}{ひと}と
\ruby{爭}{や}り
\ruby{合}{あ}はうといふ
\ruby{氣}{き}は
\ruby{最初}{さい|しよ}から
\ruby{無}{な}かつたのだよ。
』

『でも
\ruby{理由}{わ|け}も
\ruby{無}{な}い
\ruby{金錢}{も|の}を。
』

『
\ruby{取}{と}られたつて
\ruby{口惜}{く|や}しかあ
\ruby{無}{な}いぢやあ
\ruby{無}{な}いか、
\ruby{物事}{もの|ごと}さへすらりツとそれで
\ruby{濟}{す}んで
\ruby{仕舞}{し|ま}へば!。
\ruby{妾}{わたし}あ
\ruby{彼樣}{あ|ん}な
\ruby{人}{ひと}を
\ruby{對手}{あひ|て}に
\ruby{仕}{し}て
\ruby{爭}{や}り
\ruby{合}{あ}ふなあ
\ruby[g]{何程得}{いくらとく}がいつても
\ruby{可厭}{い|や}だよ。
』

『そりやあ
\ruby{然樣}{さ|う}でしやうけれども、
\ruby{餘}{あんま}りそれぢやあ……』

『だつて
\ruby{仕方}{し|かた}が
\ruby{有}{あ}りやあ
\ruby{仕}{し}ないやネ、
\ruby{蚊}{か}を
\ruby{拍}{はた}けばお
\ruby{前掌}{まへ|て}が
\ruby{汚}{よご}れやうぢやあ
\ruby{無}{な}いか、
\ruby{蚤}{のみ}を
\ruby{潰}{つぶ}しやあ
\ruby{矢張}{やつ|ぱり}
\ruby{爪}{つめ}が
\ruby{汚}{よご}れるはネ。
\ruby{下}{くだ}らない
\ruby{人}{ひと}を
\ruby{相手}{あひ|て}に
\ruby{仕}{し}て
\ruby{居}{ゐ}りやあ、
\ruby[g]{始{\換字{終}}下}{しよつちうくだ}らないことを
\ruby{仕}{し}て
\ruby{居}{ゐ}なけりやあならないやうな
\ruby{譯}{わけ}になるもの!。
』

