\Entry{其四十一}

% メモ 校正終了 2024-05-18 2024-06-14
\原本頁{230-10}%
\ruby{尾}{を}も
あらば
\ruby{振}{ふ}つて
\ruby{見}{み}すべき
\ruby{程}{ほど}
\ruby{悅}{よろこ}び
かへつて、
%
お
\ruby{關}{せき}は
おのが
\ruby{賤}{いや}しき
\ruby[g]{詞の}{ことば }
\ruby[g]{端々}{はし〴〵}に
\ruby[g]{下卑}{げ び }たる
\ruby{心}{こゝろ}の
\ruby[g]{隈々}{くま〴〵}を
\ruby{殘}{のこ}り
なく
\ruby{露}{あらは}すをも
\ruby{顧}{かへり}みず、
%
\ruby{知}{し}ら
\ruby{知}{じ}らしき
まで
お
\ruby{彤}{とう}
お
\ruby{龍}{りう}に
\ruby[g]{諛辭}{おはむき}の
\ruby[g]{數々}{かず〴〵}を
\ruby{云}{い}ひ
\ruby{盡}{つく}したる
\ruby{後}{のち}、
%
あまり
\ruby[g]{長居}{ながゐ }して
\ruby[g]{愛想}{あいそ }を
つかされてはと
\ruby{思}{おも}ひてか、
%
\ruby{但}{たゞ}しは
お
\ruby{彤}{とう}が
\ruby{餘}{あま}り
\原本頁{231-4}\改行%
\ruby{多}{おほ}くも
\ruby{言}{ものい}はず
%
\ruby{餘}{あま}り
\ruby{多}{おほ}くも
\ruby{笑}{わら}はで、
%
いつまでも
\ruby{面}{おも}
\ruby{正}{ただ}しくなし% 原本では非通り字表記
\ruby{居}{ゐ}るに、
%
\ruby[|g|]{流石}{さすが}の
\ruby{{\換字{勝}}手者}{かつ|て|もの}も
\ruby{氣}{き}の
\ruby{置}{お}けてか、
%
\ruby[g]{吳々}{くれ〴〵}も
\ruby[g]{此後}{こののち}とも
\ruby{疎}{うと}み
\ruby{棄}{す}てられぬ
やうにと
\ruby{頼}{たの}み
\ruby{聞}{きこ}えて、
%
お
\ruby{富}{とみ}
お
\ruby{春}{はる}にまで
\ruby[g]{無理}{む り }
\ruby{捏}{づく}ねに
\ruby{捏}{つく}ね
つけたる
やうの
\ruby[g]{愛想}{あいそ }の
\ruby{有}{あ}る
\ruby{限}{かぎ}りを
\ruby{振}{ふ}り
\ruby{撒}{ま}き、
%
\ruby{來}{きた}りし
\ruby{時}{とき}の
\ruby[g]{荒々}{あら〳〵}し
かりしには
\ruby{引}{ひき}かへ、
%
\ruby{歸}{かへ}る
\ruby{時}{とき}には
\ruby{疊}{たゝみ}も
そつと
\ruby{踏}{ふ}む
やうにして
\ruby[g]{漸く}{やうや }
に
\ruby{出}{いで}
\原本頁{231-9}\改行%
\ruby{去}{さ}れば、
%
\ruby{其}{その}
\ruby[||j>]{背}{うしろ}
\ruby[||j>]{影}{ かげ}の
% \ruby{背影}{うしろ|かげ}の
\ruby{見}{み}えずなるや
\ruby{否}{いな}や、
%
\ruby{{\換字{送}}}{おく}つて
\ruby{出}{い}でたる
お
\ruby{春}{はる}は
\ruby{堪}{こら}へ
かねて、
%
フヽワヽと
\ruby{笑}{わら}ひ
\ruby{出}{だ}し、

\原本頁{231-11}%
『
マア、
%
\ruby{何}{なん}て
いふ
\ruby[g]{現金}{げんきん}な
\ruby[g]{得手}{ゑ て }
\ruby[g]{{\換字{勝}}手}{かつて }な
\ruby{人}{ひと}でしやう!。
%
\ruby{來}{き}た
\ruby{時}{とき}にやあ
\ruby[|g|]{宛然}{まるで}
\ruby{狂犬見}{やまひ|いぬ|み}た% ルビ調整(長いルビ対策)前のルビが連続するので「見」を連結
\ruby{樣}{やう}に、
%
\ruby{手}{て}でも
\ruby{出}{だ}したら
\ruby{噬}{く}ひつき
さうな
\ruby{怖}{おそろ}しい
\ruby{顏}{かほ}を
\ruby{仕}{し}て
\ruby{來}{き}て、
%
\ruby{歸}{かへ}る
\ruby{時}{とき}にやあ
\ruby[g]{小狗}{ちんころ}か
なんかの
\ruby{樣}{やう}に
ころ〳〵し
\ruby[|g|]{て悅}{よろこん}% 行末行頭の境界付近なので特例処置を施す
で
\ruby{行}{ゆ}くんですもの!。
%
おゝ
\ruby[g]{可厭}{い や }な
をかしな
お
\ruby{婆}{ばあ}さん
だ
こと!
。
\改行% 校正作業の簡略化のため
』

\原本頁{232-4}%
と、
%
\ruby[g]{引{\換字{返}}}{ひつかへ}し
ながら
お
\ruby{富}{とみ}と
\ruby{顏}{かほ}を
\ruby{見}{み}
\ruby{合}{あは}せて
\ruby{云}{い}ふを、
%
これも
\ruby[g]{何處}{ど こ }やらに
\ruby{笑}{わらひ}を
\ruby{含}{ふく}み
ながらも
\ruby{叱}{しか}るが
\ruby{如}{ごと}く
\ruby[g]{上眼}{うはめ }
つかひ
して
\ruby{制}{せい}し
\ruby{止}{とど}めつ、% 原本では非通り字表記
%
お
\ruby{富}{とみ}は
\ruby[g]{小聲}{こ ごゑ}に、% 原本では非通り字表記

\原本頁{232-7}%
『
でも
\ruby[g]{彼樣}{あ あ }いふのが
\ruby[||j>]{正}{しやう}
\ruby[||j>]{直}{ ぢき}つて
% \ruby{正直}{しやう|ぢき}つて
\ruby{云}{い}ふんで、
%
\ruby[g]{可愛}{か はい}い
\ruby[||j>]{性}{しやう}
\ruby[||j>]{{\換字{分}}}{ ぶん}
% \ruby{性{\換字{分}}}{しやう|ぶん}
なんですかも
\ruby{知}{し}れませんよ。
%
\ruby{罪}{つみ}も
\ruby{何}{なんに}も
\ruby{無}{な}くつてネエ。
』

\原本頁{232-9}%
と
\ruby{冷}{ひや}やかに
\ruby{罵}{のゝし}る。
%
お
\ruby{春}{はる}は
\ruby[g]{此語}{こ れ }を
\ruby{聞}{き}いて
\ruby{{\換字{猶}}}{なほ}
\ruby{笑}{わら}ひ
\ruby{止}{や}まず、

\原本頁{232-10}%
『
\ruby[g]{左樣}{さ う }ネエ、
%
\ruby{毫}{ちつと}も
\ruby{奧}{おく}
\ruby{底}{そこ}が
\ruby{無}{な}いん
ですからネエ。
%
だが、
%
\ruby[g]{左樣}{さ う }
いへば
お
\ruby{富}{とみ}さん
なんぞは
\ruby[g]{大變}{たいへん}に
\ruby[g]{可愛}{か はい}らしくない
\ruby{人}{ひと}なの?。
%
\ruby{何}{なん}でも
\ruby[g]{{\換字{遠}}慮}{ゑんりよ}
\ruby{深}{ぶか}くつて、
%
\ruby[g]{愼み}{つゝし }
が
\ruby{深}{ふか}い
のですもの!。
』

\原本頁{233-2}%
と
\ruby[g]{小聲}{こ ごゑ}に% 原本では非通り字表記
\ruby{語}{かた}り
\ruby{合}{あ}ふ
\ruby[|g|]{此方}{こなた}は
\ruby[|g|]{此方}{こなた}、
%
\ruby[|g|]{彼方}{かなた}は
\ruby[|g|]{彼方}{かなた}にて、
%
お
\ruby{龍}{りう}は
\ruby{先}{ま}づ
\ruby[<j||]{訝}{いぶか}り% こゝまで29文字あるのでちょっとずれる。
\ruby{糺}{たゞ}し、

\原本頁{233-4}%
『
\ruby{姊}{ねえ}さん、
%
\ruby{彼}{あ}の
\ruby{人}{ひと}を
\ruby[g]{何樣}{ど う }
なすつたの?。
』

\原本頁{233-5}%
と
\ruby{問}{と}へば、
%
お
\ruby{彤}{とう}は
\ruby{微}{すこ}しく
\ruby{笑}{ゑみ}
を
\ruby{含}{ふく}み、

\原本頁{233-6}%
『
\ruby[g]{何故}{な ぜ }?。
%
\ruby{別}{べつ}に
\ruby[g]{何樣}{ど う }も
\ruby{仕}{し}やうは
\ruby{有}{あ}りやあ
\ruby[g]{仕無}{し な }い
ぢや
\ruby{無}{な}いか。
』

\原本頁{233-7}%
と
\ruby{澄}{す}まし
\ruby{切}{き}つて
\ruby{云}{い}ふ。

\原本頁{233-8}%
『
でも
\ruby[g]{大變}{たいへん}に
\ruby{怒}{おこ}つて
\ruby{來}{き}た
といふのに、
%
\ruby{妾}{わたし}が
\ruby{下}{お}りて
\ruby{來}{き}て
\ruby{見}{み}りやあ、
%
\ruby{毫}{ちつと}も
そんな
\ruby[g]{樣子}{やうす }は
\ruby{無}{な}くつて、
%
\ruby{怒}{おこ}る
どころ
ぢやあ
\ruby{無}{な}く、
%
\ruby[g]{莞爾}{にこ〳〵}
して
ばかり
\ruby{居}{ゐ}る
ぢやあ
\ruby{有}{あ}りませんか。
』

\原本頁{233-11}%
『
そりやあ
\ruby{何}{なに}
お
\ruby{{\換字{前}}}{まへ}、
%
\ruby{何}{なんに}も
\ruby{不思議}{ふ|し|ぎ}は
\ruby{有}{あ}りやあ
\ruby{仕}{し}ないはネ。
%
\ruby[g]{些少}{ぽつちり}
ばかり
\ruby[g]{金錢}{も の }を
\ruby{與}{や}つたので
\ruby[g]{如是}{あ ゝ }
\ruby{悅}{よろこ}んで
\ruby[g]{仕舞}{し ま }つたのさ。
』

\原本頁{234-2}%
『
\ruby[|g|]{金錢}{おかね}を?。
』


\原本頁{234-3}%
『
あゝ。
』

\原本頁{234-4}%
『
あら!。
%
\ruby{何}{なに}も
\ruby{姊}{ねえ}さんが
そんなもの
お
\ruby{與}{や}んなさる
\ruby[g]{理由}{わ け }は
\ruby{無}{な}い
ぢやあ
\ruby{有}{あ}りませんか。
%
さうして
\ruby{姊}{ねえ}さんも
\ruby{彼}{あ}の
\ruby[g]{靜岡}{しづをか}のに、
%
お
\ruby{金}{かね}は
\ruby{惜}{をし}かない
けれども
\ruby{取}{と}られるのは
\ruby[g]{業腹}{ごふはら}
だから、
%
と
\ruby{御自{\換字{分}}}{ご|じ|ぶん}で
ちやんと
\ruby[g]{然樣}{さ う }
\ruby[<j||]{仰}{おつし}あつた% 原本に合わせて調整
ぢやあ
\ruby{有}{あ}りませんか?。
』

\原本頁{234-8}%
『
そりやあ
お
\ruby{{\換字{前}}}{まへ}の
\ruby[g]{叔母}{を ば }さんには
\ruby[g]{然樣}{さ う }
\ruby{云}{い}つた
けれどもネ、
%
\ruby{彼}{あ}りやあ
\ruby{云}{い}はば
\ruby[g]{叔母}{を ば }さんの
\ruby{氣}{き}の
\ruby{濟}{す}むやうに
\ruby{云}{い}つただけの% 原本では非通り字表記
\ruby{事}{こと}でネ、
%
\ruby{何}{なに}も
\ruby{妾}{わたし}あ
\ruby[g]{彼樣}{あ ん }な
\ruby[g]{慾張}{よくば }りの
\ruby{人}{ひと}と
\ruby{爭}{や}り
\ruby{合}{あ}はう
といふ
\ruby{氣}{き}は
\ruby[g]{最初}{さいしよ}から
\ruby{無}{な}かつたのだよ。
』

\原本頁{235-1}%
『
でも
\ruby[g]{理由}{わ け }も
\ruby{無}{な}い
\ruby[g]{金錢}{も の }を。
』

\原本頁{235-2}%
『
\ruby{取}{と}られたつて
\ruby[g]{口惜}{く や }しかあ
\ruby{無}{な}い
ぢやあ
\ruby{無}{な}いか、
%
\ruby[g]{物事}{ものごと}さへ
すらりツと
それで
\ruby{濟}{す}んで
\ruby[g]{仕舞}{し ま }へば!。
%
\ruby{妾}{わたし}あ
\ruby[g]{彼樣}{あ ん }な
\ruby{人}{ひと}を
\ruby[|g|]{對手}{あひて}に
\ruby{仕}{し}て
\ruby{爭}{や}り
\ruby{合}{あ}ふなあ
\ruby[|g|]{何程}{いくら}
\ruby{得}{とく}が
いつても
\ruby[g]{可厭}{い や }だよ。
』

\原本頁{235-5}%
『
そりやあ
\ruby[g]{然樣}{さ う }
でしやう
けれども、
%
\ruby{餘}{あんま}り
それぢやあ
‥‥
』

\原本頁{235-6}%
『
だつて
\ruby[g]{仕方}{し かた}が
\ruby{有}{あ}りやあ
\ruby{仕}{し}ないやネ、
%
\ruby{蚊}{か}を
\ruby{拍}{はた}けば
お
\ruby{{\換字{前}}}{まへ}
\ruby{掌}{て}が
\ruby{汚}{よご}れやう
ぢやあ
\ruby{無}{な}いか、
%
\ruby{蚤}{のみ}を
\ruby{潰}{つぶ}しやあ
\ruby[g]{矢張}{やつぱり}
\ruby{爪}{つめ}が
\ruby{汚}{よご}れるはネ。
%
\ruby{下}{くだ}らない
\ruby{人}{ひと}を
\ruby[g]{相手}{あひて }に
\ruby{仕}{し}て
\ruby{居}{ゐ}りやあ、
%
\ruby[||j>]{始}{しよつ}
\ruby[||j>]{{\換字{終}}}{ ちう}
% \ruby{始{\換字{終}}}{しよつ|ちう}
\ruby[||j>]{下}{ くだ}らない
ことを
\ruby{仕}{し}て
\ruby{居}{ゐ}なけりやあ
ならない
やうな
\ruby{譯}{わけ}に
なるもの!。
』
