\Entry{其四十二}

% メモ 校正終了 2024-05-19 2024-06-14
\原本頁{236-1}%
\ruby{氣位}{き|ぐらゐ}
\ruby{高}{たか}しと
\ruby{云}{い}はば
\ruby{氣位}{き|ぐらゐ}
\ruby{高}{たか}しと
\ruby{云}{い}ふべし、
%
\ruby{憎}{にく}しと
\ruby{云}{い}はば
\ruby{憎}{にく}しと
\ruby{云}{い}ふべし、
%
お
\ruby{彤}{とう}は
\ruby{眉}{まゆ}をだに
\ruby{動}{うご}かさで
\ruby{澄}{す}まし
かへつて
\ruby{斯}{か}く
\ruby{云}{い}ひて、
%
\ruby{然}{さ}も
\ruby{然}{さ}も% 原本では非通り字表記
\ruby{我}{わ}が
\ruby{言}{ことば}に
\ruby{無理}{む|り}は
あらじ、
%
\ruby{然}{さ}は
\ruby{思}{おも}はずやと
\ruby{云}{い}はぬ
ばかりに
お
\ruby{龍}{りう}を
\ruby{徐}{しづか}に
\ruby{見}{み}けるが、
%
お
\ruby{龍}{りう}は
やゝ
\ruby{頭}{かしら}を
\ruby{垂}{た}れて
\ruby{獨}{ひと}り
\ruby{物}{もの}を
\ruby{思}{おも}ひ
\ruby{居}{ゐ}つ、
%
\ruby[|g|]{自己}{おのれ}は
おのれ
だけに
\ruby{何事}{なに|ごと}をか
\ruby{考}{かんが}へ
\ruby{居}{を}れり。

\原本頁{236-6}%
『
お
\ruby{龍}{りう}ちやん、
%
\ruby{何}{なに}を
\ruby[|g|]{其樣}{そんな}に
お
\ruby{{\換字{前}}}{まへ}は
\ruby{考}{かんが}へ
\ruby{{\換字{込}}}{こ}んで
\ruby{居}{ゐ}るの?。
』

\原本頁{236-7}%
\ruby{不快氣}{ふ|くわい|げ}
といふ
までには
あらねど、
%
\ruby{言葉}{こと|ば}の
\ruby{優}{やさ}しきには
\ruby{似}{に}ず
\ruby{聊}{いさゝ}か
\ruby[<j||]{悅}{よろこ}ばぬ% 行末行頭の境界付近なので特例処置を施す
\ruby{色}{いろ}して
お
\ruby{彤}{とう}は
\ruby{{\換字{尋}}}{たづ}ねたり。

\原本頁{236-9}%
『
\ruby{何}{なに}つて、
%
\ruby{何}{なんに}も
\ruby{考}{かんが}へてや
\ruby{仕}{し}ません
けど、
%
ただ% 原本では非通り字表記
\ruby{餘}{あんま}り
\ruby{何樣}{ど|う}も
‥‥
』

\原本頁{236-10}%
『
\ruby{餘}{あんま}り
\ruby{何樣}{ど|う}も
‥‥
\ruby{世話}{せ|わ}に
なり
\ruby{{\換字{過}}}{す}ぎる
と
でも
\ruby{思}{おも}つて
おいでの?。
』

\原本頁{236-11}%
『
\ruby{唯}{えゝ}。
%
だつて
\ruby{何樣}{ど|う}も
\ruby{何}{なん}だ
\ruby{彼}{か}だつて
\ruby{餘}{あんま}り
\ruby{御厄介}{ご|やく|かい}
ばかし
\ruby{掛}{か}けるんですもの!。
』

\原本頁{237-1}%
『
ぢやあ
\ruby{其}{それ}が
\ruby{可厭}{い|や}だ
とでも
\ruby{御思}{お|おも}ひ
なの?。
』

\原本頁{237-2}%
『
あら
\ruby{飛}{と}んでも
ない、
%
\ruby{然樣}{さ|う}
ぢや
\ruby{有}{あ}りません
けども、
%
\ruby{餘}{あんま}り
\ruby{重}{かさ}ね
\ruby{重}{がさ}ね% ルビ調整(原本通り)非踊り字表記(行末行頭の境界付近)
ですから、
%
\ruby{何}{なん}だか
\ruby{姊}{ねえ}さんに
\ruby{濟}{す}まない
やうな
\ruby{氣}{き}が
\ruby{仕}{し}て
\ruby{仕方}{し|かた}が
\ruby{無}{な}い
もんですから、
%
それで
\ruby{茫然}{ぼん|やり}と
\ruby{考}{かんが}へて
\ruby{居}{ゐ}た
んですよ。
』

\原本頁{237-6}%
『
\ruby{宜}{い}い
ぢやあ
\ruby{無}{な}い
かえ、
%
そんな
\ruby{事}{こと}を
\ruby{考}{かんが}へ
\ruby{無}{な}くつたつて。
%
\ruby{妾}{わたし}が
\ruby{好}{す}きで
\ruby{爲}{す}る
\ruby{事}{こと}
だから
\ruby{放擲}{うつ|ちや}つて
\ruby{任}{まか}して
お
\ruby{置}{お}きでも!。
%
\ruby{何}{なに}も
お
\ruby{{\換字{前}}}{まへ}に
\ruby{頼}{たの}まれた
から
\ruby{爲}{す}るつて
\ruby{云}{い}ふ
んぢやあ
\ruby{無}{な}い
の
だから、
%
\ruby{妾}{わたし}の
\ruby{{\換字{道}}樂}{だう|らく}で
\ruby{{\換字{勝}}手}{かつ|て}な
\ruby{事}{こと}を
\ruby{仕}{し}て
\ruby{居}{ゐ}るんだと
\ruby{思}{おも}つて
おいでな。
』

\原本頁{237-10}%
『
でも
\ruby{何}{なん}だか
\ruby{餘}{あんま}り
なんです
もの。
%
\ruby{彼樣}{あ|ん}な
\ruby{人}{ひと}にまで
\ruby{妾}{わたし}の
\ruby{故}{せゐ}で% せ(ゐ)
もつ
\改行% 校正作業の簡略化のため
て
‥‥
』

\原本頁{238-1}%
『
\ruby{宜}{い}いよ、
%
そんな
\ruby{詰}{つま}らない
ことを。
%
\ruby{氣}{き}に
お
\ruby{仕}{し}で
\ruby{無}{な}いといふのに。
%
ホヽヽ
お
\ruby{{\換字{前}}}{まへ}は
\ruby{{\換字{近}}頃}{この|ごろ}は
\ruby{氣}{き}が
\ruby{小}{ちひ}さく
おなりだネエ。
%
\ruby{構}{かま}はない
ぢやあ
\ruby{無}{な}いか。
%
そんな
\ruby{事}{こと}ばかり
\ruby{云}{い}つて
\ruby{御}{お}いで
のやう
ぢやあ、
%
お
\ruby{{\換字{前}}}{まへ}にやあ
まだ
\ruby{妾}{わたし}の
\ruby{氣性}{き|しやう}も
\ruby[||j>]{心}{こゝろ}
\ruby[||j>]{持}{ もち}も
% \ruby{心持}{こゝろ|もち}も
\ruby{能}{よ}くは
\ruby{解}{わか}らない
のだネエ、
%
いやな
\ruby{人}{ひと}だ
ことネ!。
』

\原本頁{238-6}%
『
いゝえ、
%
\ruby{姊}{ねえ}さんの
\ruby[||j>]{心}{こゝろ}
\ruby[||j>]{持}{ もち}
% \ruby{心持}{こゝろ|もち}
だつて
\ruby{氣性}{き|しやう}
だつて
\ruby{其}{それ}あ
\ruby{知}{し}つてますは。
%
\原本頁{238-7}\改行%
いくら
\ruby{妾}{わたし}が
\ruby[|g|]{怜悧}{りこう}% ルビ調整(原本通り)(りこう)
ぢや
\ruby{無}{な}くつても
\ruby{其}{それ}あ
ちやんと
\ruby{知}{し}つて
\ruby{居}{ゐ}ますよ。
』

\原本頁{238-8}%
『
\ruby{然樣}{さ|う}、
%
それ
ぢやあ
\ruby{宜}{い}い
ぢやあ
\ruby{無}{な}いか、
%
そんな
\ruby{事}{こと}を
\ruby{氣}{き}に
\ruby{仕}{し}なくつても。
%
\ruby{妾}{わたし}あ
お
\ruby{龍}{りう}ちやんの
\ruby{先}{せーん}から
\ruby{知}{し}つてる
\ruby{{\換字{通}}}{とほ}りにネ、
%
\ruby{何}{なん}にも
これ
といふ
\ruby{慾}{よく}も
\ruby{願}{ねがひ}も
\ruby{有}{あ}りやあ
\ruby{仕無}{し|な}い
けれども、
%
たゞ
\ruby{毎日}{まい|にち}
\ruby[g]{々々}{ 〳〵 }を
\ruby[<j||]{心}{こゝろ}% 行末行頭の境界付近なので特例処置を施す
\ruby[||j>]{持}{もち}
% \ruby{心持}{こゝろ|もち}
\ruby{宜}{よ}く、
%
\ruby{不快}{い|や}な
ことや
\ruby{馬鹿}{ば|か}な
\ruby{事}{こと}や
\ruby[|g|]{汚穢}{きたな}い
\ruby{事}{こと}に
たづさはらないで
\改行% 校正作業の簡略化のため
、
%
\原本頁{239-1}\改行%
それで
\ruby{{\換字{消}}光}{お|く}つて% 「消光 しようこう」 日々を送ること。 くらすこと。
\ruby{行}{い}きさへ
すりやあ、
%
\ruby{好}{い}いと
\ruby{思}{おも}つてる
の
だから。
』

\原本頁{239-2}%
『
そりやあ
もう
\ruby{姊}{ねえ}さん
ばかり
ぢやあ
\ruby{有}{あ}りませんは、
%
\ruby{妾}{わたし}だつて、
%
\ruby{誰}{だれ}だつて。
』

\原本頁{239-4}%
『
それ
\ruby{御覽}{ご|らん}な。
%
そんなら
\ruby{彼樣}{あ|ん}な
\ruby{人}{ひと}に
かゝり
あつて
\ruby{爭}{や}り
あつて
なんぞ
\ruby{居}{ゐ}る
より、
%
\ruby{些細}{ぽつ|ちり}
ばかしの
\ruby[|g|]{阿堵物}{もの}で% 「阿堵物(あとぶつ)」お金のこと
\ruby{奇麗事}{き|れい|ごと}に
\ruby{埓}{らち}を
\ruby{明}{あ}けた
\ruby{方}{はう}が
\改行% 校正作業の簡略化のため
、
%
\原本頁{239-6}\改行%
\ruby[|g|]{何程}{いくら}
\ruby{理屈}{り|くつ}が
\ruby{好}{い}いか
\ruby{知}{し}れや
\ruby{仕無}{し|な}いやネ。
%
\ruby{下}{くだ}らない
\ruby{人}{ひと}を
\ruby{相手}{あひ|て}に
する
\ruby[||j>]{位}{くらゐ}% 行末行頭の境界付近なので特例処置を施す
\ruby{下}{ くだ}らない
ことは
\ruby{有}{あ}りやあ
\ruby{仕無}{し|な}い
もの!。
』

\原本頁{239-8}%
『
そりやあ
もう
\ruby{然樣}{さ|う}には
\ruby{定}{きま}つてます
けれども、
%
\ruby{其}{そ}の
\ruby{些少}{ぼつ|ちり}
ばかしの
\ruby{物}{もの}だつて
ただ% 原本では非通り字表記
\ruby{湧}{わ}いて
\ruby{來}{き}やあ
\ruby{仕}{し}ません
から。
』

\原本頁{239-10}%
『
ホヽヽ、
%
そんな
\ruby{下}{くだ}らない
\ruby{見}{み}つとも
\ruby{無}{な}いことを
\ruby{二度}{に|ど}と
\ruby{云}{い}つて
お
\ruby{吳}{く}れ
ぢやあ
\ruby{可厭}{い|や}だよ。
%
\ruby{可惜}{あつ|たら}
お
\ruby{龍}{りう}ちやんの
\ruby{器量}{きり|やう}が
\ruby{下}{さが}つて
\ruby{仕舞}{し|ま}ふよ
\改行% 校正作業の簡略化のため
。
%
\原本頁{240-1}\改行%
\ruby{今}{いま}が
\ruby{今}{いま}の
\ruby[||j>]{心}{こゝろ}
\ruby[||j>]{持}{ もち}
% \ruby{心持}{こゝろ|もち}
さへ
\ruby{好}{よ}けりやあ
\ruby{其}{それ}で
\ruby{可}{い}いんだもの、
%
\ruby{何}{なんに}も
\ruby{悋}{をし}いものは
\ruby{無}{な}からう
ぢやあ
\ruby{無}{な}いか。
%
\ruby{妾}{わたし}あ
\ruby{妾}{わたし}の
\ruby[|g|]{身體}{からだ}
だつて
\ruby{悋}{をし}んで
\ruby{居}{ゐ}や
\ruby{仕無}{し|な}い
\ruby{身}{み}
ぢやあ
\ruby{無}{な}いか。
%
\ruby{何}{なん}でも
\ruby{可}{い}いから、
%
\ruby{妾}{わたし}あ
\ruby{妾}{わたし}の
\ruby[|g|]{周圍}{まはり}に
お
\ruby{{\換字{前}}}{まへ}の
やうな
%
\ruby{妾}{わたし}の
\ruby{好}{す}きな
\ruby{人{\換字{達}}}{ひと|たち}を
\ruby{置}{お}いて
%
\ruby{妾}{わたし}の
\ruby{好}{すき}な
ところに
\ruby{居}{ゐ}て
%
\ruby{妾}{わたし}の
\ruby{好}{す}きな
ことを
\ruby{仕}{し}て
\ruby{{\換字{遊}}}{あそ}んで
\ruby{居}{ゐ}りやあ
\ruby{其}{それ}で
\ruby{可}{い}い
のだよ。
』
