\Entry{其四十二}

\原本頁{}
\ruby{氣位}{き|ぐらゐ}
\ruby{高}{たか}しと
\ruby{云}{い}はば
\ruby{氣位}{き|ぐらゐ}
\ruby{高}{たか}しと
\ruby{云}{い}ふべし、
%
\ruby{憎}{にく}しと
\ruby{云}{い}はば
\ruby{憎}{にく}しと
\ruby{云}{い}ふべし、
%
お
\ruby{彤}{とう}は
\ruby{眉}{まゆ}をだに
\ruby{動}{うご}かさで
\ruby{澄}{す}ましかへつて
\ruby{斯}{か}く
\ruby{云}{い}ひて、
%
\ruby{然}{さ}も
\ruby{然}{さ}も
\ruby{我}{わ}が
\ruby{言}{ことば}に
\ruby{無理}{む|り}はあらじ、
%
\ruby{然}{さ}は
\ruby{思}{おも}はずやと
\ruby{云}{い}はぬばかりに
お
\ruby{龍}{りう}を
\ruby{徐}{しづか}に
\ruby{見}{み}けるが、
%
お
\ruby{龍}{りう}はやゝ
\ruby{頭}{かしら}を
\ruby{垂}{た}れて
\ruby{獨}{ひと}り
\ruby{物}{もの}を
\ruby{思}{おも}ひ
\ruby{居}{ゐ}つ、
%
\ruby{自己}{おの|れ}はおのれだけに
\ruby{何事}{なに|ごと}をか
\ruby{考}{かんが}へ
\ruby{居}{を}れり。

\原本頁{}
『お
\ruby{龍}{りう}ちやん、
%
\ruby{何}{なに}を
\ruby{其樣}{そん|な}に
お
\ruby{{\換字{前}}}{まへ}は
\ruby{考}{かんが}へ
\ruby{{\換字{込}}}{こ}んで
\ruby{居}{ゐ}るの?。
』

\原本頁{}
\ruby{不快氣}{ふ|くわい|げ}といふまでにはあらねど、
%
\ruby{言葉}{こと|ば}の
\ruby{優}{やさ}しきには
\ruby{似}{に}ず
\ruby{聊}{いさゝ}か
\ruby{悅}{よろこ}ばぬ
\ruby{色}{いろ}して
お
\ruby{彤}{とう}は
\ruby{{\換字{尋}}}{たづ}ねたり。

\原本頁{}
『
\ruby{何}{なに}つて、
%
\ruby{何}{なんに}も
\ruby{考}{かんが}へてや
\ruby{仕}{し}ませんけど、
%
たゞ
\ruby{餘}{あんま}り
\ruby{何樣}{ど|う}も……』

\原本頁{}
『
\ruby{餘}{あんま}り
\ruby{何樣}{ど|う}も……
\ruby{世話}{せ|わ}になり
\ruby{{\換字{過}}}{す}ぎるとでも
\ruby{思}{おも}つておいでの?。
』

\原本頁{}
『
\ruby{唯}{えゝ}。
%
だつて
\ruby{何樣}{ど|う}も
\ruby{何}{なん}だ
\ruby{彼}{か}だつて
\ruby{餘}{あんま}り
\ruby{御厄介}{ご|やく|かい}ばかし
\ruby{掛}{か}けるんですもの!。
』

\原本頁{}
『ぢやあ
\ruby{其}{それ}が
\ruby{可厭}{い|や}だとでも
\ruby{御思}{お|おも}ひなの?。
』

\原本頁{}
『あら
\ruby{飛}{と}んでもない、
%
\ruby{然樣}{さ|う}ぢや
\ruby{有}{あ}りませんけども、
%
\ruby{餘}{あんま}り
\ruby{重}{かさ}ね
\ruby{重}{がさ}ねですから、
%
\ruby{何}{なん}だか
\ruby{姊}{ねえ}さんに
\ruby{濟}{す}まないやうな
\ruby{氣}{き}が
\ruby{仕}{し}て
\ruby{仕方}{し|かた}が
\ruby{無}{な}いもんですから、
%
それで
\ruby{茫然}{ぼん|やり}と
\ruby{考}{かんが}へて
\ruby{居}{ゐ}たんですよ。
』

\原本頁{}
『
\ruby{宜}{い}いぢやあ
\ruby{無}{な}いかえ、
%
そんな
\ruby{事}{こと}を
\ruby{考}{かんが}へ
\ruby{無}{な}くつたつて。
%
\ruby{妾}{わたし}が
\ruby{好}{す}きで
\ruby{爲}{す}る
\ruby{事}{こと}だから
\ruby{放擲}{うつ|ちや}つて
\ruby{任}{まか}して
お
\ruby{置}{お}きでも!。

\原本頁{}
\ruby{何}{なに}も
お
\ruby{{\換字{前}}}{まへ}に
\ruby{頼}{たの}まれたから
\ruby{爲}{す}るつて
\ruby{云}{い}ふんぢやあ
\ruby{無}{な}いのだから、
%
\ruby{妾}{わたし}の
\ruby{{\換字{道}}樂}{だう|らく}で
\ruby{{\換字{勝}}手}{かつ|て}な
\ruby{事}{こと}を
\ruby{仕}{し}て
\ruby{居}{ゐ}るんだと
\ruby{思}{おも}つておいでな。
』

\原本頁{}
『でも
\ruby{何}{なん}だか
\ruby{餘}{あんま}りなんですもの。
%
\ruby{彼樣}{あ|ん}な
\ruby{人}{ひと}にまで
\ruby{妾}{わたし}の
\ruby{故}{せゐ}でもつて……』% せ(ゐ)

\原本頁{}
『
\ruby{宜}{い}いよ、
%
そんな
\ruby{詰}{つま}らないことを。
%
\ruby{氣}{き}に
お
\ruby{仕}{し}で
\ruby{無}{な}いといふのに。
%
ホヽヽお
\ruby{{\換字{前}}}{まへ}は
\ruby{{\換字{近}}頃}{この|ごろ}は
\ruby{氣}{き}が
\ruby{小}{ちひ}さくおなりだネエ。
%
\ruby{構}{かま}はないぢやに
\ruby{無}{な}いか。
%
そんな
\ruby{事}{こと}ばかり
\ruby{云}{い}つて
\ruby{御}{お}いでのやうぢやあ、
%
お
\ruby{{\換字{前}}}{まへ}にやあまだ
\ruby{妾}{わたし}の
\ruby{氣性}{き|しやう}も
\ruby{心持}{こゝろ|もち}も
\ruby{能}{よ}くは
\ruby{解}{わか}らないのだネエ、
%
いやな
\ruby{人}{ひと}だことネ!。
』

\原本頁{}
『いゝえ、
%
\ruby{姊}{ねえ}さんの
\ruby{心持}{こゝろ|もち}だつて
\ruby{氣性}{き|しやう}だつて
\ruby{其}{それ}あ
\ruby{知}{し}つてますは。
%
いくら
\ruby{妾}{わたし}が
\ruby{怜悧}{り|こう}ぢや
\ruby{無}{な}くつても
\ruby{其}{それ}あちやんと
\ruby{知}{し}つて
\ruby{居}{ゐ}ますよ。
』

\原本頁{}
『
\ruby{然樣}{さ|う}、
%
それぢやあ
\ruby{宜}{い}いぢやあ
\ruby{無}{な}いか、
%
そんな
\ruby{事}{こと}を
\ruby{氣}{き}に
\ruby{仕}{し}なくつても。
%
\ruby{妾}{わたし}あ
お
\ruby{龍}{りう}ちやんの
\ruby{先}{せん}から
\ruby{知}{し}つてる
\ruby{{\換字{通}}}{とほ}りにネ、
%
\ruby{何}{なん}にもこれといふ
\ruby{慾}{よく}も
\ruby{願}{ねがひ}も
\ruby{有}{あ}りやあ
\ruby{仕無}{し|な}いけれども、
%
たゞ
\ruby{毎日}{まい|にち}
\g詰めruby{々々}{〳〵}を
\ruby{心持}{こゝろ|もち}
\ruby{宜}{よ}く、
%
\ruby{不快}{い|や}なことや
\ruby{馬鹿}{ば|か}な
\ruby{事}{こと}や
\ruby{汚穢}{きた|な}い
\ruby{事}{こと}にたづさはらないで、
%
それで
\ruby{{\換字{消}}光}{お|く}つて% 「消光 しようこう」 日々を送ること。 くらすこと。
\ruby{行}{い}きさへすりやあ、
%
\ruby{好}{い}いと
\ruby{思}{おも}つてるのだから。
』

\原本頁{}
『そりやあもう
\ruby{姊}{ねえ}さんばかりぢやあ
\ruby{有}{あ}りませんは、
%
\ruby{妾}{わたし}だつて、
%
\ruby{誰}{たれ}だつて。
』

\原本頁{}
『それ
\ruby{御覽}{ご|らん}な。
%
そんなら
\ruby{彼樣}{あ|ん}な
\ruby{人}{ひと}にかゝりあつて
\ruby{爭}{や}りあつてなんぞ
\ruby{居}{ゐ}るより、
%
\ruby{些細}{ぽつ|ちり}ばかしの
\ruby[g]{阿堵物}{もの}で% 「阿堵物(あとぶつ)」お金のこと
\ruby{奇麗事}{き|れい|ごと}に
\ruby{埓}{らち}を
\ruby{明}{あ}けた
\ruby{方}{はう}が、
%
\ruby{何程}{いく|ら}
\ruby{理屈}{り|くつ}が
\ruby{好}{い}いか
\ruby{知}{し}れや
\ruby{仕無}{し|な}いやネ。
%
\ruby{下}{くだ}らない
\ruby{人}{ひと}を
\ruby{相手}{あひ|て}にする
\ruby[<h||]{位}{くらゐ}
\ruby{下}{くだ}らないことは
\ruby{有}{あ}りやあ
\ruby{仕無}{し|な}いもの!。
』

\原本頁{}
『そりやあもう
\ruby{然樣}{さ|う}には
\ruby{定}{きま}つてますけれども、
%
\ruby{其}{そ}の
\ruby{些少}{ぼつ|ちり}ばかしの
\ruby{物}{もの}だつてたゞ
\ruby{湧}{わ}いて
\ruby{來}{き}やあ
\ruby{仕}{し}ませんから。
』

\原本頁{}
『ホヽヽ、
%
そんな
\ruby{下}{くだ}らない
\ruby{見}{み}つとも
\ruby{無}{な}いことを
\ruby{二度}{に|ど}と
\ruby{云}{い}つて
お
\ruby{吳}{く}れぢやあ
\ruby{可厭}{い|や}だよ。
%
\ruby{可惜}{あつ|たら}
お
\ruby{龍}{りう}ちやんの
\ruby{器量}{きり|やう}が
\ruby{下}{さが}つて
\ruby{仕舞}{し|ま}ふよ。
%
\ruby{今}{いま}が
\ruby{今}{いま}の
\ruby{心持}{こゝろ|もち}さへ
\ruby{好}{よ}けりやあ
\ruby{其}{それ}で
\ruby{可}{い}いんだもの、
%
\ruby{何}{なんに}も
\ruby{悋}{をし}いものは
\ruby{無}{な}からうぢやあ
\ruby{無}{な}いか。
%
\ruby{妾}{わたし}あ
\ruby{妾}{わたし}の
\ruby{身體}{から|だ}だつて
\ruby{悋}{をし}んで
\ruby{居}{ゐ}や
\ruby{仕無}{し|な}い
\ruby{身}{み}ぢやあ
\ruby{無}{な}いか。
%
\ruby{何}{なん}でも
\ruby{可}{い}いから、
%
\ruby{妾}{わたし}あ
\ruby{妾}{わたし}の
\ruby{周圍}{まは|り}に
お
\ruby{{\換字{前}}}{まへ}のやうな
\ruby{妾}{わたし}の
\ruby{好}{す}きな
\ruby{人{\換字{達}}}{ひと|たち}を
\ruby{置}{お}いて
\ruby{妾}{わたし}の
\ruby{好}{すき}なところに
\ruby{居}{ゐ}て
\ruby{妾}{わたし}の
\ruby{好}{すき}なことを
\ruby{仕}{し}て
\ruby{{\換字{遊}}}{あそ}んで
\ruby{居}{ゐ}りやあ
\ruby{其}{それ}で
\ruby{可}{い}いのだよ。
』
