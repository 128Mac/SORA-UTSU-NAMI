\Entry{其四十三}

『そりやあもう
\ruby{姊}{ねえ}さんは
\ruby{何}{なに}をなさらうと
\ruby{隨意}{ま|ゝ}におなんなさる
\ruby{事}{こと}ですから、
\ruby{姊}{ねえ}さんの
\ruby[g]{氣性一}{きしやういつ}ぱいに
\ruby{生活}{く|ら}して
\ruby{行}{い}かうと
\ruby{御思}{お|おもひ}なさる、そりやあ
\ruby{其}{それ}で
\ruby{宜}{い}いんですが、
\ruby{妾}{わたし}あまた
\ruby{妾}{わたし}で、
\ruby{働}{はたら}きも
\ruby{意氣地}{い|く|ぢ}もないもんですから……』

『それで?』

『…………』

『あゝ
\ruby{解}{わか}つたよ!。
\ruby{恩}{おん}を
\ruby{受}{う}けるなあ
\ruby{可}{い}いやうなもんだけれど、
\ruby{{\換字{返}}}{かへ}しやうの
\ruby{目的}{あ|て}
が
\ruby{無}{な}いから
\ruby{困}{こま}ると
\ruby{御}{お}おもひなんだらう。
』

『
\ruby{困}{こま}るといふんでもありませんけど、まあ
\ruby{然樣}{さ|う}なの。

\ruby{何}{なに}も
\ruby{姊}{ねえ}さんが
\ruby{人}{ひと}に
\ruby[g]{恩{\換字{返}}}{おんがへ}しを
\ruby{仕}{し}てもらはうなんて
\ruby{云}{い}つたやうな
\ruby{其樣}{そ|ん}な
\ruby{氣}{き}を
\ruby{有}{も}つておいでぢやあ
\ruby{無}{な}いのは
\ruby{知}{し}りきつてますが、
\ruby{何樣}{ど|う}したら
\ruby{妾}{わたし}が
\ruby{嬉}{うれ}しいと
\ruby{身}{み}に
\ruby{染}{し}みて
\ruby{思}{おも}つて
\ruby{居}{ゐ}る
\ruby{此}{こ}の
\ruby{心持}{こゝろ|もち}を、
\ruby{何}{なに}かに
\ruby{爲}{し}て
\ruby{姊}{ねえ}さんに
\ruby{見}{み}ていたゞくことが
\ruby{出來}{で|き}るだらうと
\ruby{思}{おも}つて、それが
\ruby{氣}{き}になつてならないのです。
\ruby{妾}{わたし}あ
\ruby{如是}{こ|ん}なぶらんさんの
\ruby{身}{み}ぢやあ
\ruby{有}{あ}りますし、
\ruby{何一}{なに|ひと}つ
\ruby{{\換字{遂}}}{と}げて
\ruby{出來}{で|き}る
\ruby{技}{わざ}が
\ruby{有}{あ}るんぢや
\ruby{有}{あ}りませんし、これから
\ruby[g]{前{\換字{途}}何年}{さきどれ}だけ
\ruby{經}{た}ちやあ
\ruby{何樣}{ど|う}なる
\ruby{身}{み}だつて
\ruby{云}{い}ふんでも
\ruby{無}{な}いのですから、
\ruby{心}{こゝろ}にやあ
\ruby{斷}{た}えずに
\ruby{思}{おも}つて
\ruby{居}{ゐ}ても、
\ruby{何時}{い|つ}になつたらまあ
\ruby{些少}{ぼつ|ちり}ばかりでも
\ruby{御禮}{お|れい}らしいことが
\ruby{出來}{で|き}ることだらう!、と
\ruby{思}{おも}ふと
\ruby{何}{なん}だか
\ruby{妙}{めう}に
\ruby{味氣}{あじ|き}なくなつて、
\ruby{妾}{わたし}の
\ruby{行末}{ゆく|すゑ}が
\ruby[g]{{\換字{情}}無}{なさけな}い
\ruby{果敢無}{は|か|な}い……
\ruby{薄暗}{うす|くら}い
\ruby{路}{みち}を
\ruby{薄{\換字{寒}}}{うす|さむ}い
\ruby{日}{ひ}に
\ruby{辿}{たど}るやうな、
\ruby{何}{なん}とも
\ruby{云}{い}へない
\ruby{心細}{こゝろ|ぼそ}いやうな
\ruby{氣}{き}が
\ruby{仕}{し}て、とても
\ruby{自{\換字{分}}}{じ|ぶん}の
\ruby{氣}{き}の
\ruby{濟}{す}むだけの
\ruby{事}{こと}を
\ruby{仕}{し}て
\ruby{姊}{ねえ}さんに
\ruby{見}{み}ていたゞく
\ruby{事}{こと}なんかは、
\ruby{一生}{いつ|しやう}たつても
\ruby{出來無}{で|き|な}いやうな
\ruby[g]{可厭}{いやあ}な
\ruby{感}{おもひ}がするんです。
\ruby{斯樣}{か|う}いつたら
\ruby[g]{御笑}{おわら}ひなさるでしやうが
\ruby{嘘}{うそ}ぢやあ
\ruby{無}{な}いのです、
\ruby{今}{いま}になつて
\ruby{叔母}{を|ば}が
\ruby{云}{い}ひました
\ruby[g]{言葉}{ことば}が
\ruby{妙}{めう}に
\ruby{胸}{むね}に
\ruby{{\換字{浮}}}{うか}んで
\ruby{來}{き}て、いつそ
\ruby{前{\換字{途}}}{さ|き}も
\ruby{見}{み}えも
\ruby{仕}{し}ないのにうか〳〵と
\ruby{日}{ひ}を
\ruby{{\換字{過}}}{すご}すより
\ruby{鋤}{すき}や
\ruby{鍬}{くは}を
\ruby{擔}{かつ}ぐ
\ruby{男}{をとこ}でも
\ruby{實直}{こく|めい}な
\ruby{堅}{かた}い
\ruby{人}{ひと}を、
\ruby{自{\換字{分}}}{じ|ぶん}の
\ruby{一生}{いつ|しやう}の
\ruby{柱}{はしら}に
\ruby{頼}{たの}んで
\ruby{眞黒}{まつ|くろ}になつて
\ruby{働}{はたら}いて、さうして
\ruby{{\換字{適}}}{たま}には
\ruby{姊}{ねえ}さんのところへ
\ruby[g]{大根}{だいこ}や
\ruby{竹}{たけ}の
\ruby{子}{こ}を
\ruby{持}{も}つて
\ruby{來}{き}て、これは
\ruby{妾}{わたし}が
\ruby{作}{つく}りました、これはわたしの
\ruby{背戸}{せ|ど}の
\ruby{藪}{やぶ}で
\ruby{掘}{ほ}りましたつて
\ruby{云}{い}ふやうなことを
\ruby{云}{い}つて、ほんとにお
\ruby{龍}{りう}がまあ
\ruby[g]{田舍者}{ゐなかもの}になりきつて
\ruby[g]{御仕舞}{おしまひ}で、
\ruby{何}{なん}と
\ruby{好}{い}いお
\ruby[g]{土産}{みやげ}をお
\ruby{吳}{く}れぢやあ
\ruby{無}{な}いか、とお
\ruby{富}{とみ}さんやなんぞと
\ruby[g]{御笑}{おわら}ひ
\ruby{合}{あ}ひなすつて
\ruby{頂}{いたゞ}く
\ruby{樣}{やう}な
\ruby{其樣}{そ|ん}な
\ruby{身}{み}になつて
\ruby{仕舞}{し|ま}つたら、
\ruby{其}{そ}の
\ruby{方}{はう}が
\ruby{宜}{い}いか
\ruby{知}{し}らと
\ruby{思}{おも}ふ
\ruby{氣}{き}さへ
\ruby{仕}{し}ますが、まさかに
\ruby{然樣}{さ|う}も
\ruby{思}{おも}ひ
\ruby{切}{き}れないで……』

\ruby{眞面目}{ま|じ|め}に
\ruby{云}{い}ふ
\ruby[g]{言葉}{ことば}は、
\ruby[g]{笑聲}{わらひ}に
\ruby[g]{打{\換字{消}}}{うちけ}されたり。

『ホヽホヽホヽ、
\ruby{可笑}{を|か}しなお
\ruby{龍}{りう}ちやんだよ、ホヽホヽホヽ、
\ruby{何}{なん}だネエ
\ruby{急}{きふ}に
\ruby{年}{とし}をお
\ruby{取}{と}りだネ。
\ruby{詰}{つま}らない!。

\ruby{濕}{しめ}つぼい、そんなことを
\ruby{言}{い}ふものぢやあ
\ruby{無}{な}いよ。
\ruby[g]{大根}{だいこ}や
\ruby{竹}{たけ}の
\ruby{子}{こ}なんかあ
\ruby{妾}{わたし}あ
\ruby{可厭}{い|や}だよ、
\ruby{女}{をんな}は
\ruby[g]{{\換字{所}}天次第}{をとこしだい}ぢやあ
\ruby{無}{な}いか、
\ruby[g]{立派}{りつぱ}な
\ruby[g]{{\換字{所}}天}{をとこ}を
\ruby{御持}{お|も}ちで、そして
\ruby{妾}{わたし}にやあ
\ruby[g]{金剛石}{だいやもんど}の
\ruby{首飾}{くび|かざ}りでもなんでも
\ruby[g]{澤山}{たんと}お
\ruby{吳}{く}れ!。
\ruby{買物}{かい|もの}は
\ruby[g]{{\換字{勝}}手}{かつて}だあネ、
\ruby[g]{男子}{をとこ}は
\ruby{撰}{えら}み
\ruby{取}{ど}りにするが
\ruby{宜}{い}いぢやあ
\ruby{無}{な}いか、
\ruby{腕}{うで}のある
\ruby{確固}{しつ|かり}した
\ruby{男}{をとこ}さへ
\ruby{持}{も}ちやあ、
\ruby{何}{なに}も
\ruby{彼}{か}も
\ruby{湧}{わ}いて
\ruby{來}{こ}やうぢやあ
\ruby{無}{な}いかえ。
そりやあお
\ruby{前}{まへ}の
\ruby{胸}{むね}
\ruby{中}{なか}に
\ruby{働}{はたら}きのある
\ruby[g]{好漢}{いゝをとこ}が
\ruby{無}{な}いもんだから、
そんな
\ruby[g]{陰氣臭}{いんきくさ}いことを
\ruby{云}{い}ふやうになるんだよ。
いくら
\ruby{好}{い}い
\ruby{人}{ひと}でも
\ruby{手腕}{はた|らき}の
\ruby{無}{な}いなあ、
\ruby[g]{{\換字{所}}天}{をとこ}に
\ruby{仕}{し}やうとすりやあ
\ruby{淋}{さび}しくつていけないよ。
\ruby{彼}{あ}の
\ruby{人}{ひと}なんぞはまあ
\ruby{抛擲}{うつ|ちや}つて
\ruby{置}{お}いて、
\ruby{搜}{さが}してごらん、
\ruby[g]{何程}{いくら}も
\ruby{好}{い}い
\ruby{男}{をとこ}はあるよ。
お
\ruby{前}{まへ}に
\ruby[g]{一人見}{ひとりみ}せてあげやうかネエ。
\ruby{其男}{そ|れ}なら
\ruby[g]{屹度}{きつと}お
\ruby{前}{まへ}の
\ruby{行末}{ゆく|すゑ}を
\ruby{春}{はる}の
\ruby{日}{ひ}に
\ruby{好}{い}い
\ruby[g]{海邊}{うみのはた}でも
\ruby{歩}{ある}かせるやうに
\ruby{爲}{す}るに
\ruby{定}{きま}つて
\ruby{居}{ゐ}るよ。

\ruby{其}{それ}に
\ruby[g]{引代}{ひきか}へて
\ruby{水野}{みづ|の}つていふ
\ruby{人}{ひと}ネ、
\ruby{彼}{あ}の
\ruby{人}{ひと}ネ、
\ruby{彼}{あ}の
\ruby{人}{ひと}と
\ruby{{\換字{連}}}{つ}れ
\ruby{立}{だ}ちやあ、お
\ruby{前}{まへ}は
\ruby[g]{成程薄暗}{なるほどうすつくら}い
\ruby{路}{みち}を
\ruby{薄{\換字{寒}}}{うす|さむ}い
\ruby{日}{ひ}に
\ruby{辿}{たど}るよ。
』

