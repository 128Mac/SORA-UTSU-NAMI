\Entry{其四十三}

\原本頁{}%
『そりやあもう
\ruby{姊}{ねえ}さんは
\ruby{何}{なに}をなさらうと
\ruby{隨意}{ま|ゝ}におなんなさる
\ruby{事}{こと}ですから、
%
\ruby{姊}{ねえ}さんの
\ruby{氣性}{き|しやう}
\ruby{一}{いつ}ぱいに
\ruby{生活}{く|ら}して
\ruby{行}{い}かうと
\ruby{御思}{お|おもひ}なさる、
%
そりやあ
\ruby{其}{それ}で
\ruby{宜}{い}いんですが、
%
\ruby{妾}{わたし}あまた
\ruby{妾}{わたし}で、
%
\ruby{働}{はたら}きも
\ruby{意氣地}{い|く|ぢ}もないもんですから‥‥』

\原本頁{}%
『それで?』

\原本頁{}%
『‥‥‥‥』

\原本頁{}%
『あゝ
\ruby{解}{わか}つたよ!。
%
\ruby{恩}{おん}を
\ruby{受}{う}けるなあ
\ruby{可}{い}いやうなもんだけれど、
%
\ruby{{\換字{返}}}{かへ}しやうの
\ruby{目的}{あ|て}
が
\ruby{無}{な}いから
\ruby{困}{こま}ると
\ruby{御}{お}おもひなんだらう。
』

\原本頁{}%
『
\ruby{困}{こま}るといふんでもありませんけど、
%
まあ
\ruby{然樣}{さ|う}なの。

\原本頁{}%
\ruby{何}{なに}も
\ruby{姊}{ねえ}さんが
\ruby{人}{ひと}に
\ruby{恩{\換字{返}}}{おん|がへ}しを
\ruby{仕}{し}てもらはうなんて
\ruby{云}{い}つたやうな
\ruby{其樣}{そ|ん}な
\ruby{氣}{き}を
\ruby{有}{も}つておいでぢやあ
\ruby{無}{な}いのは
\ruby{知}{し}りきつてますが、
%
\ruby{何樣}{ど|う}したら
\ruby{妾}{わたし}が
\ruby{嬉}{うれ}しいと
\ruby{身}{み}に
\ruby{染}{し}みて
\ruby{思}{おも}つて
\ruby{居}{ゐ}る
\ruby{此}{こ}の
\ruby{心持}{こゝろ|もち}を、
%
\ruby{何}{なに}かに
\ruby{爲}{し}て
\ruby{姊}{ねえ}さんに
\ruby{見}{み}ていたゞくことが
\ruby{出來}{で|き}るだらうと
\ruby{思}{おも}つて、
%
それが
\ruby{氣}{き}になつてならないのです。
%
\ruby{妾}{わたし}あ
\ruby{如是}{こ|ん}なぶらんさんの
\ruby{身}{み}ぢやあ
\ruby{有}{あ}りますし、
%
\ruby{何一}{なに|ひと}つ
\ruby{{\換字{遂}}}{と}げて
\ruby{出來}{で|き}る
\ruby{{\換字{技}}}{わざ}が
\ruby{有}{あ}るんぢや
\ruby{有}{あ}りませんし、
%
これから
\ruby{{\換字{前}}{\換字{途}}}{さ|き}
\ruby{何年}{ど|れ}だけ
\ruby{經}{た}ちやあ
\ruby{何樣}{ど|う}なる
\ruby{身}{み}だつて
\ruby{云}{い}ふんでも
\ruby{無}{な}いのですから、
%
\ruby{心}{こゝろ}にやあ
\ruby{斷}{た}えずに
\ruby{思}{おも}つて
\ruby{居}{ゐ}ても、
%
\ruby{何時}{い|つ}になつたらまあ
\ruby{些少}{ぼつ|ちり}ばかりでも
\ruby{御禮}{お|れい}らしいことが
\ruby{出來}{で|き}ることだらう!、
%
と
\ruby{思}{おも}ふと
\ruby{何}{なん}だか
\ruby{妙}{めう}に
\ruby{味氣}{あぢ|き}なくなつて、
%
\ruby{妾}{わたし}の
\ruby{行末}{ゆく|すゑ}が
\ruby{{\換字{情}}無}{なさけ|な}い
\ruby{果敢無}{は|か|な}い‥‥
\ruby{薄暗}{うす|くら}い
\ruby{路}{みち}を
\ruby{薄{\換字{寒}}}{うす|さむ}い
\ruby{日}{ひ}に
\ruby{辿}{たど}るやうな、
%
\ruby{何}{なん}とも
\ruby{云}{い}へない
\ruby{心細}{こゝろ|ぼそ}いやうな
\ruby{氣}{き}が
\ruby{仕}{し}て、
%
とても
\ruby{自{\換字{分}}}{じ|ぶん}の
\ruby{氣}{き}の
\ruby{濟}{す}むだけの
\ruby{事}{こと}を
\ruby{仕}{し}て
\ruby{姊}{ねえ}さんに
\ruby{見}{み}ていたゞく
\ruby{事}{こと}なんかは、
%
\ruby{一生}{いつ|しやう}たつても
\ruby{出來無}{で|き|な}いやうな
\ruby{可厭}{い|やあ}な
\ruby{感}{おもひ}がするんです。
%
\ruby{斯樣}{か|う}いつたら
\ruby{御笑}{お|わら}ひなさるでしやうが
\ruby{嘘}{うそ}ぢやあ
\ruby{無}{な}いのです、
%
\ruby{今}{いま}になつて
\ruby{叔母}{を|ば}が
\ruby{云}{い}ひました
\ruby{言葉}{こと|ば}が
\ruby{妙}{めう}に
\ruby{胸}{むね}に
\ruby{{\換字{浮}}}{うか}んで
\ruby{來}{き}て、
%
いつそ
\ruby{{\換字{前}}{\換字{途}}}{さ|き}も
\ruby{見}{み}えも
\ruby{仕}{し}ないのに
うか〳〵と
\ruby{日}{ひ}を
\ruby{{\換字{過}}}{すご}すより
\ruby{鋤}{すき}や
\ruby{鍬}{くは}を
\ruby{擔}{かつ}ぐ
\ruby{男}{をとこ}でも
\ruby{實直}{こく|めい}な
\ruby{堅}{かた}い
\ruby{人}{ひと}を、
%
\ruby{自{\換字{分}}}{じ|ぶん}の
\ruby{一生}{いつ|しやう}の
\ruby{柱}{はしら}に
\ruby{頼}{たの}んで
\ruby{眞黒}{まつ|くろ}になつて
\ruby{働}{はたら}いて、
%
さうして
\ruby{{\換字{適}}}{たま}には
\ruby{姊}{ねえ}さんのところへ
\ruby{大根}{だい|こん}や
\ruby{竹}{たけ}の
\ruby{子}{こ}を
\ruby{持}{も}つて
\ruby{來}{き}て、
%
これは
\ruby{妾}{わたし}が
\ruby{作}{つく}りました、
%
これはわたしの
\ruby{背{\換字{戸}}}{せ|ど}の
\ruby{藪}{やぶ}で
\ruby{掘}{ほ}りましたつて
\ruby{云}{い}ふやうなことを
\ruby{云}{い}つて、
%
ほんとに
お
\ruby{龍}{りう}がまあ
\ruby{田舎}{ゐな|か}
\ruby{者}{もの}になりきつて
\ruby{御仕舞}{お|し|まひ}で、
%
\ruby{何}{なん}と
\ruby{好}{い}い
お
\ruby{土產}{みや|げ}を
お
\ruby{吳}{く}れぢやあ
\ruby{無}{な}いか、
%
と
お
\ruby{富}{とみ}さんやなんぞと
\ruby{御笑}{お|わら}ひ
\ruby{合}{あ}ひなすつて
\ruby{頂}{いたゞ}く
\ruby{樣}{やう}な
\ruby{其樣}{そ|ん}な
\ruby{身}{み}になつて
\ruby{仕舞}{し|ま}つたら、
%
\ruby{其}{そ}の
\ruby{方}{はう}が
\ruby{宜}{い}いか
\ruby{知}{し}らと
\ruby{思}{おも}ふ
\ruby{氣}{き}さへ
\ruby{仕}{し}ますが、
%
まさかに
\ruby{然樣}{さ|う}も
\ruby{思}{おも}ひ
\ruby{切}{き}れないで‥‥』

\原本頁{}%
\ruby{眞面目}{ま|じ|め}に
\ruby{云}{い}ふ
\ruby{言葉}{こと|ば}は、
%
\ruby{笑聲}{わら|ひ}に
\ruby{打{\換字{消}}}{うち|け}されたり。

\原本頁{}%
『ホヽホヽホヽ、
%
\ruby{可笑}{を|か}しな
お
\ruby{龍}{りう}ちやんだよ、
%
ホヽホヽホヽ、
%
\ruby{何}{なん}だネエ
\ruby{急}{きふ}に
\ruby{年}{とし}を
お
\ruby{取}{と}りだネ。
%
\ruby{詰}{つま}らない!。

\原本頁{}%
\ruby{濕}{しめ}つぼい、
%
そんなことを
\ruby{言}{い}ふものぢやあ
\ruby{無}{な}いよ。
%
\ruby{大根}{だい|こん}や
\ruby{竹}{たけ}の
\ruby{子}{こ}なんかあ
\ruby{妾}{わたし}あ
\ruby{可厭}{い|や}だよ、
%
\ruby{女}{をんな}は
\ruby{{\換字{所}}天}{をと|こ}
\ruby{次第}{し|だい}ぢやあ
\ruby{無}{な}いか、
%
\ruby{立派}{りつ|ぱ}な
\ruby{{\換字{所}}天}{をと|こ}を
\ruby{御持}{お|も}ちで、
%
そして
\ruby{妾}{わたし}にやあ
\ruby{金剛石}{だい|やも|んど}の
\ruby{首{\換字{飾}}}{くび|かざ}りでもなんでも
\ruby[g]{澤山}{たんと}
お
\ruby{吳}{く}れ!。
%
\ruby{買物}{かい|もの}は
\ruby{{\換字{勝}}手}{かつ|て}だあネ、
%
\ruby{男子}{をと|こ}は
\ruby{撰}{えら}み
\ruby{取}{ど}りにするが
\ruby{宜}{い}いぢやあ
\ruby{無}{な}いか、
%
\ruby{腕}{うで}のある
\ruby{確固}{しつ|かり}した
\ruby{男}{をとこ}さへ
\ruby{持}{も}ちやあ、
%
\ruby{何}{なに}も
\ruby{彼}{か}も
\ruby{湧}{わ}いて
\ruby{來}{こ}やうぢやあ
\ruby{無}{な}いかえ。
%
そりやあお
\ruby{{\換字{前}}}{まへ}の
\ruby{胸}{むね}
\ruby{中}{なか}に
\ruby{働}{はたら}きのある
\ruby{好漢}{いゝ|をとこ}が
\ruby{無}{な}いもんだから、
%
そんな
\ruby{陰氣臭}{いん|き|くさ}いことを
\ruby{云}{い}ふやうになるんだよ。
%
いくら
\ruby{好}{い}い
\ruby{人}{ひと}でも
\ruby{手腕}{はた|らき}の
\ruby{無}{な}いなあ、
%
\ruby{{\換字{所}}天}{をと|こ}に
\ruby{仕}{し}やうとすりやあ
\ruby{淋}{さび}しくつていけないよ。
%
\ruby{彼}{あ}の
\ruby{人}{ひと}なんぞはまあ
\ruby{抛擲}{うつ|ちや}つて
\ruby{置}{お}いて、
%
\ruby{搜}{さが}してごらん、
%
\ruby{何程}{いく|ら}も
\ruby{好}{い}い
\ruby{男}{をとこ}はあるよ。
%
お
\ruby{{\換字{前}}}{まへ}に
\ruby{一人}{ひと|り}
\ruby{見}{み}せてあげやうかネエ。
%
\ruby{其男}{そ|れ}なら
\ruby{屹度}{きつ|と}
お
\ruby{{\換字{前}}}{まへ}の
\ruby{行末}{ゆく|すゑ}を
\ruby{春}{はる}の
\ruby{日}{ひ}に
\ruby{好}{い}い
\ruby{海邊}{うみの|はた}でも
\ruby{歩}{ある}かせるやうに
\ruby{爲}{す}るに
\ruby{定}{きま}つて
\ruby{居}{ゐ}るよ。

\原本頁{}%
\ruby{其}{それ}に
\ruby{引代}{ひき|か}へて
\ruby[g]{水野}{みづの}つていふ
\ruby{人}{ひと}ネ、
%
\ruby{彼}{あ}の
\ruby{人}{ひと}ネ、
%
\ruby{彼}{あ}の
\ruby{人}{ひと}と
\ruby{{\換字{連}}}{つ}れ
\ruby{立}{だ}ちやあ、
%
お
\ruby{{\換字{前}}}{まへ}は
\ruby{成程}{なる|ほど}
\ruby{薄暗}{うすつ|くら}い
\ruby{路}{みち}を
\ruby{薄{\換字{寒}}}{うす|さむ}い
\ruby{日}{ひ}に
\ruby{辿}{たど}るよ。
』
