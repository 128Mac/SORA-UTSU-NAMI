\Entry{其四十三}

% メモ 校正終了 2024-05-19 2024-06-14
\原本頁{240-7}%
『
そりやあ
もう
\ruby{姊}{ねえ}さんは
\ruby{何}{なに}を
なさらうと
\ruby[g]{隨意}{ま ま }% 原本では非通り字表記
に
おなんなさる
\ruby{事}{こと}
ですから、
%
\ruby{姊}{ねえ}さんの
\ruby[g]{氣性}{きしやう}
\ruby{一}{いつ}ぱいに
\ruby[g]{生活}{く ら }して
\ruby{行}{い}かうと
\ruby[g]{御思}{おおもひ}なさる
\改行% 校正作業の簡略化のため
、
%
\原本頁{240-9}\改行%
そりやあ
\ruby{其}{それ}で
\ruby{宜}{い}いんですが、
%
\ruby{妾}{わたし}あ
また
\ruby{妾}{わたし}で、
%
\ruby{働}{はたら}きも
\ruby{意氣地}{い|く|ぢ}
も
ない
もん
ですから
‥‥
』

\原本頁{241-1}%
『
それで?
』

\原本頁{241-2}%
『
‥‥‥‥
』

\原本頁{241-3}%
『
あゝ
\ruby{解}{わか}つたよ!。
%
\ruby{恩}{おん}を
\ruby{受}{う}けるなあ
\ruby{可}{い}い
やうなもん
だけれど、
%
\ruby{{\換字{返}}}{かへ}し
やうの
\ruby[g]{目的}{あ て }
が
\ruby{無}{な}いから
\ruby{困}{こま}ると
\ruby{御}{お}おもひ
なんだらう。
』

\原本頁{241-5}%
『
\ruby{困}{こま}る
といふん
でも
ありません
けど、
%
まあ
\ruby[g]{然樣}{さ う }なの。
%
\ruby{何}{なに}も
\ruby{姊}{ねえ}さんが
\ruby{人}{ひと}に
\ruby[g]{恩{\換字{返}}}{おんがへ}しを
\ruby{仕}{し}て
もらはう
なんて
\ruby{云}{い}つた
やうな
\ruby[g]{其樣}{そ ん }な
\ruby{氣}{き}を
\原本頁{241-7}\改行%
\ruby{有}{も}つて
おいで
ぢやあ
\ruby{無}{な}いのは
\ruby{知}{し}り
きつてますが、
%
\ruby[g]{何樣}{ど う }したら
\ruby[<j||]{妾}{わたし}が% 行末行頭の境界付近なので特例処置を施す
\ruby{嬉}{うれ}しいと
\ruby{身}{み}に
\ruby{染}{し}みて
\ruby{思}{おも}つて
\ruby{居}{ゐ}る
\ruby{此}{こ}の
\ruby[||j>]{心}{こゝろ}
\ruby[||j>]{持}{ もち}を、
% \ruby{心持}{こゝろ|もち}を、
%
\ruby{何}{なに}かに
\ruby{爲}{し}て
\ruby{姊}{ねえ}さんに
\ruby{見}{み}て
いただく% 原本では非通り字表記
ことが
\ruby[g]{出來}{で き }る
だらうと
\ruby{思}{おもつ}て、
%
それが
\ruby{氣}{き}に
なつて
ならない
のです。
%
\ruby{妾}{わたし}あ
\ruby[g]{如是}{こ ん }な
ぶらんさんの
\ruby{身}{み}
ぢやあ
\ruby{有}{あ}りますし、
%
\ruby{何}{なに}
\ruby{一}{ひと}つ
\ruby{{\換字{遂}}}{と}げて
\ruby[g]{出來}{で き }る
\ruby{{\換字{技}}}{わざ}が
\ruby{有}{あ}るんぢや
\ruby{有}{あ}りませんし、
%
これから
\ruby[g]{{\換字{前}}{\換字{途}}}{さ き }
\ruby[g]{何年}{ど れ }だけ
\ruby{經}{た}ちやあ
\ruby[g]{何樣}{ど う }なる
\ruby{身}{み}だつて
\ruby{云}{い}ふんでも
\ruby{無}{な}いのですから、
%
\ruby{心}{こゝろ}にやあ
\ruby{斷}{た}えずに
\ruby{思}{おも}つて
\ruby{居}{ゐ}ても、
%
\ruby[g]{何時}{い つ }に
なつたら
まあ
\原本頁{242-3}\改行%
\ruby[g]{些少}{ぽつちり}
ばかり
でも
\ruby[g]{御禮}{お れい}らしい
ことが
\ruby[g]{出來}{で き }る
こと
だらう!、
%
と
\ruby{思}{おも}ふと
\ruby{何}{なん}だか
\ruby{妙}{めう}に
\ruby[g]{味氣}{あぢき }なく
なつて、
%
\ruby{妾}{わたし}の
\ruby[g]{行末}{ゆくすゑ}が
\ruby[g]{{\換字{情}}無}{なさけな}い
\ruby{果敢無}{は|か|な}い%
{---}{---}%
\原本頁{242-5}\改行%
\ruby{薄}{うす}
\ruby{暗}{くら}い
\ruby{路}{みち}を
\ruby{薄}{うす}
\ruby{{\換字{寒}}}{さむ}い
\ruby{日}{ひ}に
\ruby{辿}{たど}るやうな、
%
\ruby{何}{なん}とも
\ruby{云}{い}へない
\ruby[||j>]{心}{こゝろ}
\ruby[||j>]{細}{ ぼそ}い
% \ruby{心細}{こゝろ|ぼそ}い
やうな
\ruby{氣}{き}が
\ruby{仕}{し}て、
%
とても
\ruby[g]{自{\換字{分}}}{じ ぶん}の% ルビ調整(原本通り)非グループルビ
\ruby{氣}{き}の
\ruby{濟}{す}むだけの
\ruby{事}{こと}を
\ruby{仕}{し}て
\ruby{姊}{ねえ}さんに
\ruby{見}{み}て
いたゞく
\ruby{事}{こと}
なんかは、
%
\ruby[||j>]{一}{いつ}
\ruby[||j>]{生}{しやう}
% \ruby{一生}{いつ|しやう}
たつても
\ruby{出來無}{で|き|な}いやうな
\ruby[|g|]{可厭}{いやあ}な
\ruby[<j||]{{\換字{感}}}{おもひ}が
するんです。
%
\ruby[g]{斯樣}{か う }
いつたら
\ruby[g]{御笑}{お わら}ひ
なさる
でしやうが
\ruby{嘘}{うそ}ぢやあ
\原本頁{242-9}\改行%
\ruby{無}{な}いのです、
%
\ruby{今}{いま}に
なつて
\ruby[g]{叔母}{を ば }が
\ruby{云}{い}ひました
\ruby[g]{言葉}{ことば }が
\ruby{妙}{めう}に
\ruby{胸}{むね}に
\ruby{{\換字{浮}}}{うか}んで
\ruby{來}{き}て、
%
いつそ
\ruby[g]{{\換字{前}}{\換字{途}}}{さ き }も
\ruby{見}{み}えも
\ruby{仕}{し}ないのに
うか〳〵と
\ruby{日}{ひ}を
\ruby{{\換字{過}}}{すご}す
より
\ruby{鋤}{すき}や
\ruby{鍬}{くは}を
\ruby{擔}{かつ}ぐ
\ruby{男}{をとこ}でも
\ruby[g]{實直}{こくめい}な
\ruby{堅}{かた}い
\ruby{人}{ひと}を、
%
\ruby[g]{自{\換字{分}}}{じ ぶん}の% ルビ調整(原本通り)非グループルビ
\ruby[||j>]{一}{いつ}
\ruby[||j>]{生}{しやう}の
% \ruby{一生}{いつ|しやう}の
\ruby{柱}{はしら}に
\ruby{頼}{たの}んで
\ruby[g]{眞黒}{まつくろ}に
なつて
\ruby{働}{はたら}いて、
%
さうして
\ruby{{\換字{適}}}{たま}には
\ruby{姊}{ねえ}さんの
ところへ
\ruby[g]{大根}{だいこ }や% ルビ調整(原本通り)(ん)無し
\ruby{竹}{たけ}の
\ruby{子}{こ}を
\ruby{持}{も}つて
\ruby{來}{き}て、
%
これは
\ruby{妾}{わたし}が
\ruby{作}{つく}りました、
%
これは
わたしの
\ruby[g]{背{\換字{戸}}}{せ ど }の
\ruby{藪}{やぶ}で
\ruby{掘}{ほ}りました
つて
\ruby{云}{い}ふ
やうな
ことを
\ruby{云}{い}つて、
%
ほんとに
お
\ruby{龍}{りう}がまあ
\ruby[|g|]{田舎}{ゐなか}
\ruby{者}{もの}に
なりきつて
\ruby{御仕舞}{お|し|まひ}で、
%
\ruby{何}{なん}と
\ruby{好}{い}い
お
\ruby[g]{土產}{み や }を
お
\ruby{吳}{く}れ
ぢやあ
\ruby{無}{な}いか、
%
と
お
\ruby{富}{とみ}さんや
なんぞと
\ruby[g]{御笑}{お わら}ひ
\ruby{合}{あ}ひなすつて
\ruby{頂}{いただ}く% 原本では非通り字表記
\ruby{樣}{やう}な
\ruby[g]{其樣}{そ ん }な
\ruby{身}{み}になつて
\ruby[g]{仕舞}{し ま }つたら、
%
\ruby{其}{そ}の
\ruby{方}{はう}が
\ruby{宜}{い}いか
\ruby{知}{し}らと
\ruby{思}{おも}ふ
\ruby{氣}{き}さへ
\ruby{仕}{し}ますが、
%
まさかに
\ruby[g]{然樣}{さ う }も
\ruby{思}{おも}ひ
\ruby{切}{き}れない
で
‥‥
』

\原本頁{243-8}%
\ruby{眞面目}{ま|じ|め}に
\ruby{云}{い}ふ
\ruby[g]{言葉}{ことば }は、
%
\ruby[|g|]{笑聲}{わらひ}に
\ruby[g]{打{\換字{消}}}{うちけ }されたり。

\原本頁{243-9}%
『
ホヽホヽホヽ、
%
\ruby[g]{可笑}{を か }しな
お
\ruby{龍}{りう}ちやん
だよ、
%
ホヽホヽホヽ、
%
\ruby{何}{なん}だネエ
\ruby{急}{きふ}に
\ruby{年}{とし}を
お
\ruby{取}{と}りだネ。
%
\ruby{詰}{つま}らない!。
%
\ruby{濕}{しめ}つぽい、
%
そんな
ことを
\ruby{言}{い}ふ
ものぢやあ
\ruby{無}{な}いよ。
%
\ruby[g]{大根}{だいこ }や% ルビ調整(原本通り)(ん)無し
\ruby{竹}{たけ}の
\ruby{子}{こ}
なんかあ
\ruby{妾}{わたし}あ
\ruby[g]{可厭}{い や }だよ
\改行% 校正作業の簡略化のため
、
%
\原本頁{244-1}\改行%
\ruby{女}{をんな}は
\ruby[|g|]{{\換字{所}}天}{をとこ}
\ruby[g]{次第}{し だい}
ぢやあ
\ruby{無}{な}いか、
%
\ruby[g]{立派}{りつぱ }な
\ruby[|g|]{{\換字{所}}天}{をとこ}を
\ruby[g]{御持}{お も }ちで、
%
そして
\ruby[<j||]{妾}{わたし}にやあ% 行末行頭の境界付近なので特例処置を施す
\ruby{金剛石}{だい|やも|んど}の
\ruby[g]{首{\換字{飾}}}{くびかざ}り
でも
なんでも
\ruby[|g|]{澤山}{たんと}
お
\ruby{吳}{く}れ!。
%
\ruby[g]{買物}{かいもの}は
\ruby[g]{{\換字{勝}}手}{かつて }だあネ、
%
\ruby[|g|]{男子}{をとこ}は
\ruby{撰}{えら}み
\ruby{取}{ど}りに
するが
\ruby{宜}{い}い
ぢやあ
\ruby{無}{な}いか、
%
\ruby{腕}{うで}の
ある
\原本頁{244-4}\改行%
\ruby[g]{確固}{しつかり}した
\ruby{男}{をとこ}さへ
\ruby{持}{も}ちやあ、
%
\ruby{何}{なに}も
\ruby{彼}{か}も
\ruby{湧}{わ}いて
\ruby{來}{こ}やう
ぢやあ
\ruby{無}{な}いかえ。
%
そりやあ
お
\ruby{{\換字{前}}}{まへ}の
\ruby{胸}{むね}ん
\ruby{中}{なか}に
\ruby{働}{はたら}き
の
ある
\ruby[||j>]{好}{いゝ}
\ruby[||j>]{{\換字{漢}}}{をとこ}が
% \ruby{好{\換字{漢}}}{いゝ|をとこ}が
\ruby{無}{な}いもんだから
\改行% 校正作業の簡略化のため
、
%
\原本頁{244-6}\改行%
そんな
\ruby[g]{陰氣}{いんき }
\ruby{臭}{くさ}いことを
\ruby{云}{い}ふ
やうに
なるんだよ。
%
いくら
\ruby{好}{い}い
\ruby{人}{ひと}でも
\ruby[g]{手腕}{はたらき}の
\ruby{無}{な}いなあ、
%
\ruby[|g|]{{\換字{所}}天}{をとこ}に
\ruby{仕}{し}やうと
すりやあ
\ruby{淋}{さび}しくつて
いけないよ。
%
\ruby{彼}{あ}の
\ruby{人}{ひと}
なんぞは
まあ
\ruby[g]{抛擲}{うつちや}つて
\ruby{置}{お}いて、
%
\ruby{搜}{さが}して
ごらん、
%
\ruby[g]{何程}{い くら}も% 行末行頭の境界付近なので特例処置を施す
\ruby{好}{い}い
\ruby{男}{をとこ}は
あるよ。
%
お
\ruby{{\換字{前}}}{まへ}に
\ruby[|g|]{一人}{ひとり}
\ruby{見}{み}せて
あげやうかネエ。
%
\ruby[g]{其男}{そ れ }なら
\ruby[g]{屹度}{きつと }% ルビ調整(原本通り)非グループルビ
お
\ruby{{\換字{前}}}{まへ}の
\ruby[g]{行末}{ゆくすゑ}を
\ruby{春}{はる}の
\ruby{日}{ひ}に
\ruby{好}{い}い
\ruby[||j>]{海}{うみの}
\ruby[||j>]{邊}{ はた}でも
% \ruby{海邊}{うみの|はた}でも
\ruby{歩}{ある}かせるやうに
\ruby{爲}{す}るに
\ruby{定}{きま}つて
\ruby{居}{ゐ}るよ。
%
\ruby{其}{それ}に
\ruby[g]{引代}{ひきか }へて
\ruby[g]{水野}{みづの }つて
いふ
\ruby{人}{ひと}ネ
、
%
\ruby{彼}{あ}の
\ruby{人}{ひと}ネ
\改行% 校正作業の簡略化のため
、
%
\原本頁{245-1}\改行%
\ruby{彼}{あ}の
\ruby{人}{ひと}と
\ruby{{\換字{連}}}{つ}れ
\ruby{立}{だ}ちやあ、
%
お
\ruby{{\換字{前}}}{まへ}は
\ruby[g]{成程}{なるほど}
\ruby[||j>]{薄}{うすつ}
\ruby[||j>]{暗}{ くら}い
% \ruby{薄暗}{うすつ|くら}い
\ruby{路}{みち}を
\ruby{薄}{うす}
\ruby{{\換字{寒}}}{さむ}い
\ruby{日}{ひ}に
\ruby{辿}{たど}る
\改行% 校正作業の簡略化のため
よ。
』
