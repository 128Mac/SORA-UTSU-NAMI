\Entry{其四十四}

『いやですは
\ruby{姊}{ねえ}さん、また
\ruby{其樣}{そ|ん}な
\ruby{事}{こと}を
\ruby{云}{い}つて!。
\ruby{妾}{わたし}あ
\ruby{何}{なに}も
\ruby{彼}{あ}の
\ruby{人}{ひと}を
\ruby{何樣}{ど|う}の
\ruby{彼樣}{こ|う}のと
\ruby{其樣}{そ|ん}な
\ruby{事}{こと}なんか
\ruby{胸}{むね}の
\ruby{中}{なか}で
\ruby{思}{おも}つてや
\ruby{仕}{し}ませんて
\ruby{云}{い}つたぢやあ
\ruby{有}{あ}りませんか。
』

『あヽ
\ruby{然樣}{さ|う}だつけネエ。
』

と
\ruby{云}{い}ひたる
\ruby{限}{き}り
\ruby{後}{あと}は
\ruby{何}{なに}とも
\ruby{云}{い}はで
\ruby{止}{や}みたれども、お
\ruby{彤}{とう}はお
\ruby{龍}{りう}の
\ruby{言葉}{こと|ば}をば
\ruby{信}{しん}ずるが
\ruby{如}{ごと}く
\ruby{疑}{うたが}ふが
\ruby{如}{ごと}く
\ruby{其}{そ}の
\ruby{面}{おもて}を
\ruby{見}{み}やりて、
\ruby[g]{心解}{こヽろと}けてにもあらず、さればと
\ruby{云}{い}ひて
\ruby{嘲}{あざ}
みてにもあらず、ただにやりと
\ruby{笑}{わら}つたり。

\ruby{氣}{き}の
\ruby{直}{すぐ}なるお
\ruby{龍}{りう}はお
\ruby{彤}{とう}の
\ruby{言葉}{こと|ば}を
\ruby[g]{言葉通}{ことばどほ}りに
\ruby{聞}{き}けるなるべし。

『そして
\ruby{其樣}{そ|ん}な
\ruby{戯談}{じやう|だん}なんか
\ruby{御云}{お|い}ひなすつたつて、
\ruby{其}{そ}りやあ
\ruby{姊}{ねえ}さんみたやうに
\ruby{何}{なに}も
\ruby{彼}{か}も
\ruby{能}{よ}く
\ruby{出來}{で|き}て、おまけに
\ruby{世}{よ}の
\ruby{中}{なか}のほんとの
\ruby{事}{こと}が
\ruby[g]{悉皆解}{すつかりわか}つて
\ruby{居}{ゐ}て、
\ruby[g]{容貌}{きりやう}も
\ruby[g]{百人千人}{ひやくにんせんにん}に
\ruby{勝}{すぐ}れて
\ruby{美}{うつく}しいといふんなら、
\ruby{妾}{わたし}でも
\ruby{出來}{で|き}るか
\ruby{知}{し}れませんけれど、
\ruby[g]{男子}{をとこ}
\ruby{擇}{えら}み
\ruby{取}{ど}りだなんて、マア
\ruby{其樣}{そ|ん}なことは、
\ruby{生}{うま}れ
\ruby{代}{かは}つてでも
\ruby{來}{こ}なけりやあ
\ruby[g]{到底出來}{とてもでき}やしません。
\ruby{妾}{わたし}なんか
\ruby{圃}{はたけ}の
\ruby{中}{なか}の
\ruby[g]{蠻南瓜}{たうなす}や
\ruby{茄子}{な|す}だつて、ほんとに
\ruby{叔母}{を|ば}の
\ruby{云}{い}つた
\ruby{通}{とほ}りの
\ruby{下}{くだ}らない
\ruby[g]{禀賦}{うまれ}なんですもの。
\ruby{出世}{しゆ|つせ}しやうと
\ruby{思}{おも}つたつて、
\ruby{{\換字{運}}}{うん}に
\ruby{乘}{の}らうと
\ruby{思}{おも}つたつて、
\ruby{何}{なに}が
\ruby{何樣}{ど|う}なりましやう。
\ruby[g]{加之}{そして}もう〳〵
\ruby[g]{{\換字{所}}天}{をとこ}を
\ruby{持}{も}たうなんて、そんなことはふつ〳〵
\ruby{厭}{いや}に
\ruby{思}{おも}つて
\ruby{居}{ゐ}るんですから。
\ruby{持}{も}つ
\ruby{位}{くらゐ}なら
\ruby{虚言}{う|そ}ぢやあ
\ruby{有}{あ}りません、
\ruby[g]{蠻南瓜}{たうなす}や
\ruby{茄子}{な|す}に
\ruby{相應}{さう|あう}な
\ruby[g]{何首烏球}{かしゆうだま}に
\ruby{手足}{て|あし}の
\ruby{生}{は}えた
\ruby{樣}{やう}なお
\ruby{百姓}{ひやく|しやう}さんでも
\ruby{持}{も}ちましやうが、それも
\ruby[g]{矢張可厭}{やつぱりいや}ですから、
\ruby[g]{一生一人}{いつしやうひとり}で
\ruby{居}{ゐ}ます。
\ruby{氣}{き}の
\ruby{利}{き}いた
\ruby{男}{をとこ}を
\ruby{持}{も}ちたいの、
\ruby{出世}{しゆ|つせ}を
\ruby{仕}{し}て
\ruby{見度}{み|た}いのと、
\ruby{其樣}{そ|ん}な
\ruby{蟲}{むし}の
\ruby{好}{い}いことを
\ruby{考}{かんが}へて
\ruby{居}{ゐ}るほどに
\ruby{身}{み}の
\ruby{程}{ほど}を
\ruby{知}{し}らなかあ
\ruby{有}{あ}りません。
ですから
\ruby{前{\換字{途}}}{さ|き}の
\ruby{事}{こと}を
\ruby{思}{おも}ふと、
\ruby{心細}{こヽろ|ぼそ}くなつて
\ruby{仕舞}{し|ま}ふんです。
』

と
\ruby{云}{い}へば、

『オホヽヽ、
\ruby{何樣}{ど|う}か
\ruby{仕}{し}ておいでだよお
\ruby{龍}{りう}ちやんは。
そんな
\ruby{老}{ふ}けた
\ruby{事}{こと}ばかし
\ruby{云}{い}つて
\ruby{何樣}{ど|う}するつもりなんだらう。
\ruby{蟲}{むし}の
\ruby{好}{い}いことを
\ruby{考}{かんが}へてるからこそ
\ruby{人間}{ひ|と}は
\ruby{生}{い}きて
\ruby{居}{ゐ}られるんぢやあ
\ruby{無}{な}いかえ。
お
\ruby{前見}{まへ|み}たやうに
\ruby{其樣}{そ|ん}なことを
\ruby{云}{い}つてた
\ruby{日}{ひ}にやあ
\ruby[g]{{\換字{終}}局}{しまひ}にやあ
\ruby{坊}{ばう}さんにでもならなきやあ
\ruby[g]{{\換字{追}}付}{おつつ}かないことになるはネ。
いけないよいけないよ、そんな
\ruby{{\換字{弱}}}{よわ}い
\ruby{氣}{き}ぢやあ。
\ruby{何}{なに}も
\ruby{一生}{いつ|しやう}だはネ、
\ruby{面白}{おも|しろ}く
\ruby[g]{生活}{くら}すが
\ruby{可}{い}いぢやあ
\ruby{無}{な}いか。
\ruby{擇}{えら}み
\ruby{取}{ど}りに
\ruby{仕}{し}て
\ruby{取}{と}れ
\ruby{無}{な}くつたつて
\ruby{本}{もと}なんだもの!。
また
\ruby{擇}{えら}み、また
\ruby{擇}{えら}み
\ruby{仕}{し}て
\ruby{居}{ゐ}りやあ
\ruby{其}{そ}の
\ruby{中}{うち}にやあ
\ruby{氣}{き}に
\ruby{入}{い}つたので
\ruby{緣}{えん}の
\ruby{有}{あ}るのも
\ruby{出}{で}て
\ruby{來}{き}やうぢやあ
\ruby{無}{な}いか。
』

『あら!。
』

『ホヽヽ、
\ruby{何樣}{ど|う}だえ?、
\ruby{妾}{わたし}にやあ
\ruby[g]{愛想}{あいそ}が
\ruby{盡}{つ}きるかえ?。
』

