\Entry{其四十四}

% メモ 校正終了 2024-05-19 2024-06-14
\原本頁{245-4}%
『
いやですは
\ruby{姊}{ねえ}さん、
%
また
\ruby{其樣}{そ|ん}な
\ruby{事}{こと}を
\ruby{云}{い}つて!。
%
\ruby{妾}{わたし}あ
\ruby{何}{なに}も
\ruby{彼}{あ}の
\ruby{人}{ひと}を
\ruby{何樣}{ど|う}の
\ruby{彼樣}{こ|う}のと
\ruby{其樣}{そ|ん}な
\ruby{事}{こと}
なんか
\ruby{胸}{むね}の
\ruby{中}{なか}で
\ruby{思}{おも}つてや
\ruby{仕}{し}ませんて
\ruby{云}{い}つた
ぢやあ
\ruby{有}{あ}りませんか。
』

\原本頁{245-7}%
『
あゝ
\ruby{然樣}{さ|う}だつけネエ。
』

\原本頁{245-8}%
と
\ruby{云}{い}ひたる
\ruby{限}{ぎ}り
\ruby{後}{あと}は
\ruby{何}{なに}とも
\ruby{云}{い}はで
\ruby{止}{や}みたれども、
%
お
\ruby{彤}{とう}は
お
\ruby{龍}{りう}の
\ruby{言葉}{こと|ば}をば
\ruby{信}{しん}ずるが
\ruby{如}{ごと}く
\ruby{疑}{うたが}ふが
\ruby{如}{ごと}く
\ruby{其}{そ}の
\ruby{面}{おもて}を
\ruby{見}{み}やりて、
%
\ruby{心解}{こゝろ|と}けて
にも
あらず、
%
さればと
\ruby{云}{い}ひて
\ruby{嘲}{あざ}みて
にも
あらず、
%
ただ% 原本では非通り字表記
にやりと
\ruby{笑}{わら}つたり。

\原本頁{246-2}%
\ruby{氣}{き}の
\ruby{直}{すぐ}なる
お
\ruby{龍}{りう}は
お
\ruby{彤}{とう}の
\ruby{言葉}{こと|ば}を
\ruby{言葉}{こと|ば}
\ruby{{\換字{通}}}{どほ}りに
\ruby{聞}{き}けるなるべし。

\原本頁{246-3}%
『
そして
\ruby{其樣}{そ|ん}な
\ruby[||j>]{戲}{じやう}
\ruby[||j>]{談}{ だん}
% \ruby{戲談}{じやう|だん}
なんか
\ruby{御云}{お|い}ひなすつたつて、
%
\ruby{其}{そ}りやあ
\ruby{姊}{ねえ}さんみたやうに
\ruby{何}{なに}も
\ruby{彼}{か}も
\ruby{能}{よ}く
\ruby{出來}{で|き}て、
%
おまけに
\ruby{世}{よ}の
\ruby{中}{なか}の
ほんとの
\原本頁{246-5}\改行%
\ruby{事}{こと}が
\ruby{悉皆}{すつ|かり}
\ruby{解}{わか}つて
\ruby{居}{ゐ}て、
%
\ruby{容貌}{きり|やう}も
\ruby[<j||]{百}{ひやく}% ルビ調整(配置位置補正)「百人千人」
\ruby{人}{にん}
% \ruby{百人}{ひやく|にん}
\ruby{千人}{せん|にん}に
\ruby{{\換字{勝}}}{すぐ}れて
\ruby{美}{うつく}しい
といふんなら、
%
\ruby{妾}{わたし}
でも
\ruby{出來}{で|き}るか
\ruby{知}{し}れません
けれど、
%
\ruby[|g|]{男子}{をとこ}は
\ruby{擇}{えら}み
\ruby{取}{ど}り
だなんて、
%
マア
\ruby{其樣}{そ|ん}なことは、
%
\ruby{生}{うま}れ
\ruby{代}{かは}つてでも
\ruby{來}{こ}なけりやあ
\ruby[|g|]{到底}{とても}
\ruby{出來}{で|き}やしません。
%
\ruby{妾}{わたし}
なんか
\ruby{圃}{はたけ}の
\ruby{中}{なか}の
\ruby[|g|]{蠻南瓜}{たうなす}や
\ruby{茄子}{な|す}だつて、
%
ほんとに
\ruby{叔母}{を|ば}の
\ruby{云}{い}つた
\ruby{{\換字{通}}}{とほ}りの
\ruby{下}{くだ}らない
\ruby[|g|]{禀賦}{うまれ}
なんですもの。
%
\ruby{出世}{しゆつ|せ}
しやうと
\ruby{思}{おも}つたつて、
%
\ruby{{\換字{運}}}{うん}に
\ruby{乘}{の}らうと
\ruby{思}{おも}つたつて、
%
\ruby{何}{なに}が
\ruby{何樣}{ど|う}
なりましやう。
%
\ruby[|g|]{加之}{そして}
もう〳〵
\ruby[|g|]{{\換字{所}}天}{をとこ}を
\ruby{持}{も}たう
なんて、
%
そんなことは
ふつ〳〵
\原本頁{247-1}\改行%
\ruby{厭}{いや}に
\ruby{思}{おも}つて
\ruby{居}{ゐ}る
んですから。
%
\ruby{持}{も}つ
\ruby{位}{くらゐ}なら
\ruby{虛言}{う|そ}ぢやあ
\ruby{有}{あ}りません
\改行% 校正作業の簡略化のため
、
%
\原本頁{247-2}\改行%
\ruby[|g|]{蠻南瓜}{たうなす}や
\ruby{茄子}{な|す}に
\ruby{相應}{さう|おう}な
\ruby[|g|]{何首烏球}{かしゆうだま}に
\ruby{手足}{て|あし}の
\ruby{生}{は}えた
\ruby{樣}{やう}な
お
\ruby[<j||]{百}{ひやく}
\ruby[||j>]{姓}{しやう}さん
% \ruby{百姓}{ひやく|しやう}さん
\原本頁{247-3}\改行%
でも
\ruby{持}{も}ちましやうが、
%
それも
\ruby{矢張}{やつ|ぱり}
\ruby{可厭}{い|や}ですから、
%
\ruby[||j>]{一}{いつ }
\ruby[||j>]{生}{しやう}
% \ruby{一生}{いつ|しやう}
\ruby[|g|]{一人}{ ひとり}で
\ruby{居}{ゐ}ます。
%
\ruby{氣}{き}の
\ruby{利}{き}いた
\ruby{男}{をとこ}を
\ruby{持}{も}ちたいの、
%
\ruby{出世}{しゆつ|せ}を
\ruby{仕}{し}て
\ruby{見度}{み|た}いのと、
%
\ruby{其樣}{そ|ん}な
\ruby{蟲}{むし}の
\ruby{好}{い}いことを
\ruby{考}{かんが}へて
\ruby{居}{ゐ}る
ほどに
\ruby{身}{み}の
\ruby{程}{ほど}を
\ruby{知}{し}らなかあ
\ruby{有}{あ}りません。
%
ですから
\ruby{{\換字{前}}{\換字{途}}}{さ|き}の
\ruby{事}{こと}を
\ruby{思}{おも}ふと、
%
\ruby[||j>]{心}{こゝろ}
\ruby[||j>]{細}{ ぼそ}く
% \ruby{心細}{こゝろ|ぼそ}く
なつて
\ruby{仕舞}{し|ま}ふんで
\改行% 校正作業の簡略化のため
す。
』

\原本頁{247-8}%
と
\ruby{云}{い}へば、

\原本頁{247-0}%
『
オホヽヽ、
%
\ruby{何樣}{ど|う}か
\ruby{仕}{し}て
おいでだよ
お
\ruby{龍}{りう}ちやんは。
%
そんな
\ruby{老}{ふ}けた
\ruby{事}{こと}ばかし
\ruby{云}{い}つて
\ruby{何樣}{ど|う}
する
つもり
なんだらう。
%
\ruby{蟲}{むし}の
\ruby{好}{い}いことを
\ruby{考}{かんが}へてる
から
こそ
\ruby{人間}{ひ|と}は
\ruby{生}{い}きて
\ruby{居}{ゐ}られるんぢやあ
\ruby{無}{な}いかえ。
%
お
\ruby{{\換字{前}}}{まへ}
\ruby{見}{み}たやうに
\ruby{其樣}{そ|ん}な
ことを
\ruby{云}{い}つてた
\ruby{日}{ひ}にやあ
\ruby[|g|]{{\換字{終}}局}{しまひ}にやあ
\ruby{坊}{ばう}さん
にでも
ならなきやあ
\ruby{{\換字{追}}付}{おつ|つ}かない
ことに
なるはネ。
%
いけないよ
いけないよ、
%
そんな
\ruby{{\換字{弱}}}{よわ}い
\ruby{氣}{き}ぢやあ。
%
\ruby{何}{なに}も
\ruby[||j>]{一}{いつ}
\ruby[||j>]{生}{しやう}だはネ、
% \ruby{一生}{いつ|しやう}だはネ、
%
\ruby{面白}{おも|しろ}く
\ruby{生活}{く|ら}すが
\ruby{可}{い}いぢやあ
\ruby{無}{な}いか。
%
\ruby{擇}{えら}み
\ruby{取}{ど}りに
\ruby{仕}{し}て
\ruby{取}{と}れ
\ruby{無}{な}くつたつて
\ruby{本}{もと}なんだもの!。
%
また
\ruby{擇}{えら}み、
%
また
\ruby{擇}{えら}み
\ruby{仕}{し}て
\ruby{居}{ゐ}りやあ、
\ruby{其}{そ}の
\ruby{中}{うち}にやあ
\原本頁{248-6}\改行%
\ruby{氣}{き}に
\ruby{入}{い}つたので
\ruby{緣}{えん}の
\ruby{有}{あ}るのも
\ruby{出}{で}て
\ruby{來}{き}やうぢやあ
\ruby{無}{な}いか。
』

\原本頁{248-7}%
『
あら!。
』

\原本頁{248-8}%
『
ホヽヽ、
%
\ruby{何樣}{ど|う}だえ?、
%
\ruby{妾}{わたし}にやあ
\ruby{愛想}{あい|そ}が
\ruby{盡}{つ}きるかえ?。
』
