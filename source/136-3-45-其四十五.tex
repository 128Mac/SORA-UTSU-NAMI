\Entry{其四十五}

『ようござんすよ、お
\ruby{富}{とみ}さん、
\ruby{自{\換字{分}}}{じ|ぶん}で
\ruby{展}{と}りますから。
』

\ruby{讀}{よ}みさしたる
\ruby{何}{なに}やらの
\ruby{書物}{しよ|もつ}を
\ruby{燈}{ともしび}の
\ruby{下}{した}に
\ruby{置}{お}きて、
\ruby{身}{み}を
\ruby{反}{ひね}りて
お
\ruby{龍}{りう}は
お
\ruby{富}{とみ}を
\ruby{見}{み}かへりつ、
\ruby{愛想}{あい|そ}も
\ruby{深}{ふか}く
\ruby{制止}{と|ど}むれど、

『でも
\ruby[g]{御命令}{おいひつけ}なんですもの、
\ruby{妾}{わたし}が
\ruby{仕}{し}ませんぢやあ……。
マア
\ruby{其}{そ}のまんまに
\ruby{御本}{ご|ほん}を
\ruby{見}{み}て
\ruby{居}{ゐ}らつしやいまし。
』

と
\ruby{此室}{こ|ゝ}の
\ruby{次室}{つ|ぎ}の
\ruby{長四疊}{なが|よ|でふ}に
\ruby{附}{つ}ける
\ruby{押入}{おし|いれ}より、
お
\ruby{納戸絹}{な|ん|ど}の
\ruby{中型}{ちう|がた}の
\ruby{夜眼}{よ|め}には
\ruby{美}{うつく}しき
\ruby[g]{小掻{\換字{巻}}}{こかいまき}など
\ruby{輕}{かろ}げに
\ruby{取}{と}り
\ruby{出}{いだ}して、
お
\ruby{富}{とみ}は
\ruby{今}{いま}
\ruby{早{\換字{速}}}{さつ|き}と
\ruby{手}{て}ばしこく
お
\ruby{龍}{りう}の
\ruby{爲}{ため}に
\ruby{臥床}{ふし|ど}を
\ruby{設}{まう}くるなり。

『あら、ほんとに
\ruby{不要}{い|い}つて
\ruby{云}{い}ふのに
お
\ruby{富}{とみ}さん!。
お
\ruby{客}{きやく}さまぢやあ
\ruby{有}{あ}りやあ
\ruby{仕}{し}まいし、
\ruby{此樣}{こ|ん}な
\ruby{妾}{わたし}なんかが
\ruby{床}{とこ}の
\ruby[g]{上下}{あげおろし}まで
お
\ruby{前}{まへ}さんたちに
\ruby{仕}{し}て
\ruby{貰}{もら}つちやあ、それこそ
\ruby{罸}{ばち}が
\ruby{當}{あた}つて
\ruby{冥利}{みや|うり}が
\ruby{竭}{つ}きつちまふは。
』

\ruby{立上}{たち|あが}つて
\ruby{自}{みづか}ら
\ruby{上掛}{うは|がけ}の
\ruby{衣被}{よ|ぎ}を
\ruby{搬}{はこ}び
\ruby{來}{きた}れる
お
\ruby{龍}{りう}と
\ruby{共}{とも}に、
\ruby{{\換字{終}}}{つひ}に
\ruby{二人}{ふた|り}して
\ruby{展}{の}べ
\ruby{{\換字{終}}}{をは}りたり。

『
\ruby{風}{かぜ}も
\ruby{吹}{ふ}いてや
\ruby{仕}{し}ないやうですが
お
\ruby{{\換字{寒}}}{さむ}い
\ruby{晩}{ばん}ですことネ。
これで
\ruby{宜}{よ}うございますか、
\ruby{御薄}{お|うす}くは
\ruby{有}{あ}りませんか
\ruby{知}{し}ら?。
』

『いゝえ
\ruby{澤山}{たく|さん}ですよ。
\ruby{主人}{な|に}は?。
もうお
\ruby[g]{就眠}{やすみ}?。
』

『ハア、あなたにもお
\ruby[g]{就眠}{やすみ}つて
お
\ruby{云}{い}ひつて。
\ruby{今}{いま}しがた
\ruby{既}{もう}。
』

『
\ruby{然樣}{さ|う}。
お
\ruby{春}{はる}さんは?。
』

『まだ
\ruby[g]{裁縫}{しごと}を
\ruby{仕}{し}てゐます。
』

『なか〳〵の
\ruby{人}{ひと}ネエー。
』

『
\ruby{左樣}{さ|う}でございますとも、
\ruby{負}{ま}けない
\ruby{氣}{き}の
\ruby{人}{ひと}ですよ。
\ruby{何}{なん}でも
\ruby{妾}{わたし}にやあ
\ruby{負}{ま}けたくないと
\ruby{思}{おも}ひましてネ。
』

『ホヽヽ、だが、あけすけで
\ruby{可愛}{か|はい}らしい
\ruby{兒}{こ}ネエ。
』

『さうですよ、
\ruby{些}{ちつと}も
\ruby{毒}{どく}は
\ruby{無}{な}い
\ruby{人}{ひと}
で。
ですから
\ruby{今日}{け|ふ}の
お
\ruby{客}{きやく}さまの
\ruby{最初}{さい|しよ}の
\ruby{樣子}{やう|す}にやあ
\ruby{何樣}{ど|ん}なにか
\ruby{怒}{おこ}りましたらう!。
オホヽ、そりやあ
\ruby{可笑}{を|か}しいほどでしたよ。
』

『
\ruby{然樣}{さ|う}!。
そんなに
\ruby{最初}{さい|しよ}は
\ruby{彼方}{あつ|ち}で
\ruby{怒}{おこ}り
\ruby{立}{た}つてつん〳〵
\ruby{仕}{し}て
\ruby{{\換字{遣}}}{や}つて
\ruby{來}{き}たの?。
』

『さうですとも。
そりやあ
\ruby{甚}{ひど}い
\ruby{權幕}{けん|まく}でしたの!。
』

『それを
\ruby{何樣}{ど|う}して
\ruby{姊}{ねえ}さんが
\ruby{直}{ぢき}に
\ruby[g]{彼樣}{あんな}にヘイ〳〵するやうに
\ruby{仕}{し}て
お
\ruby{仕舞}{し|まひ}だつたの?。
』

『そりやあ
\ruby{何}{なん}ですもの!。
』

『
\ruby{何樣}{ど|う}したの?。
お
\ruby{前}{まへ}さん
\ruby[g]{悉皆知}{すつかりし}つてゝ?。
』

『すつかり
\ruby{知}{し}つてます、
\ruby{斯樣}{か|う}なんですよ。
』

お
\ruby{富}{とみ}は
\ruby{諄々}{じゆん|〳〵}として
\ruby{始末}{し|まつ}を
\ruby{{\換字{説}}}{と}き、
お
\ruby{龍}{りう}は
\ruby{默々}{もく|〳〵}として
\ruby{一切}{いつ|さい}を
\ruby{聞}{き}き
\ruby{{\換字{終}}}{をは}りたり。

