\Entry{其四十五}

% メモ 校正終了 2024-03-26 2024-05-19 2024-06-14
\原本頁{248-10}%
『
ようござんすよ、
%
お
\ruby{富}{とみ}さん、
%
\ruby{自{\換字{分}}}{じ|ぶん}で% ルビ調整(原本通り)非グループルビ
\ruby{展}{と}ります
から。
』

\原本頁{249-1}%
\ruby{讀}{よ}み
さしたる
\ruby{何}{なに}やらの
\ruby{書物}{しよ|もつ}を
\ruby[<j>]{燈}{ともしび}の
\ruby{下}{した}に
\ruby{置}{お}きて、
%
\ruby{身}{み}を
\ruby{反}{ひね}りて
お
\ruby{龍}{りう}は
お
\ruby{富}{とみ}を
\ruby{見}{み}かへりつ、
%
\ruby{愛想}{あい|そ}も
\ruby{深}{ふか}く
\ruby{制止}{と|ど}むれど、% 原本では非通り字表記

\原本頁{249-3}%
『
でも
\ruby{御命令}{お|いひ|つけ}
なんですもの、
%
\ruby{妾}{わたし}が
\ruby{仕}{し}ません
ぢやあ
‥‥
。
%
マア
\ruby{其}{そ}の
まんまに
\ruby{御本}{ご|ほん}を
\ruby{見}{み}て
\ruby{居}{ゐ}らつしやいまし。
』

\原本頁{249-5}%
と
\ruby{此室}{こ|こ}の
\ruby{次室}{つ|ぎ}の
\ruby{長四疊}{なが|よ|でふ}に
\ruby{附}{つ}ける
\ruby{押入}{おし|いれ}より、
%
お
\ruby{納{\換字{戸}}絹}{な|ん|ど}の
\ruby{中型}{ちう|がた}の
\ruby{夜眼}{よ|め}には
\ruby{美}{うつく}しき
\ruby{小掻卷}{こ|がい|まき}など
\ruby{輕}{かろ}げに
\ruby{取}{と}り
\ruby{出}{いだ}して、
%
お
\ruby{富}{とみ}は
\ruby{今}{いま}
\ruby[|g|]{早{\換字{速}}}{さつさ}と
\ruby{手}{て}ばしこく
お
\ruby{龍}{りう}の
\ruby{爲}{ため}に
\ruby{臥床}{ふし|ど}を
\ruby{設}{まう}くるなり。

\原本頁{249-8}%
『
あら、
%
ほんとに
\ruby{不要}{い|い}つて
\ruby{云}{い}ふのに
お
\ruby{富}{とみ}さん!。
%
お
\ruby{客}{きやく}さま
ぢやあ
\ruby{有}{あ}りやあ
\ruby{仕}{し}まいし、
%
\ruby{此樣}{こ|ん}な
\ruby{妾}{わたし}なんかが% 原本では非通り字表記
\ruby{床}{とこ}の
\ruby[||j>]{上}{あげ}
\ruby[||j>]{下}{おろし}まで
% \ruby{上下}{あげ|おろし}まで
お
\ruby{{\換字{前}}}{まへ}さんたちに
\ruby{仕}{し}て
\ruby{貰}{もら}つちやあ、
%
それこそ
\ruby{罸}{ばち}が
\ruby{當}{あた}つて
\ruby{冥利}{みや|うり}が
\ruby{竭}{つ}きつちまふは。
』


\原本頁{250-1}%
\ruby{立上}{たち|あが}つて
\ruby{自}{みづ}から
% 欠落部分
\ruby{爲}{な}さんと
すれば、
%
お
\ruby{富}{とみ}は
\ruby{笑}{ゑみ}を
\ruby{含}{ふく}み、
%

\原本頁{250-2}%
『
お
\ruby{客}{きやく}さま
ぢやあ
\ruby{無}{な}くつても、
%
でも、
%
\ruby[<j||]{妾}{わたし}の% ルビ調整(配置位置補正)「妾」と「妹」のルビが繋がってしまうのでアキを調整
\ruby[<j>]{妹}{いもうと}だと
\ruby{思}{おも}つて
\ruby{何}{なん}でも
\ruby{御仕}{お|し}と、
\ruby{嚴}{きつ}く
\ruby{御命令}{お|いひ|つけ}になって
\ruby{居}{ゐ}るんですもの。
』

\原本頁{250-4}%
と
\ruby{云}{い}ひて、

\原本頁{250-5}%
『
そりやあ
\ruby{其樣}{そ|う}でも
\ruby{妾}{わたし}あ
また
お
\ruby{{\換字{前}}}{まへ}さん
\ruby{{\換字{達}}}{たち}と
\ruby{異}{ちが}ふ
\ruby{身{\換字{分}}}{み|ぶん}だとは
\ruby{思}{おも}つて
\ruby{居}{ゐ}や
\ruby{仕}{し}ないん
だから、
』

\原本頁{250-7}%
と
\ruby{云}{い}ひ〳〵
\ruby{自}{みづか}ら
\ruby{上掛}{うは|がけ}の
\ruby{衣被}{よ|ぎ}を
\ruby{搬}{はこ}び
\ruby{來}{きた}れる
お
\ruby{龍}{りう}と
\ruby{共}{とも}に、
%
\ruby{{\換字{終}}}{つひ}に
\ruby[|g|]{二人}{ふたり}して
\ruby{展}{の}べ
\ruby{{\換字{終}}}{をは}りたり。

\原本頁{250-9}%
『
\ruby{風}{かぜ}も
\ruby{吹}{ふ}いてや
\ruby{仕}{し}ない
やうですが
お
\ruby{{\換字{寒}}}{さむ}い
\ruby{晩}{ばん}
です
ことネ。
%
これで
\ruby{宜}{よ}う
ございますか、
%
\ruby{御薄}{お|うす}くは
\ruby{有}{あ}りませんか
\ruby{知}{し}ら?。
』

\原本頁{250-11}%
『
いゝえ
\ruby{澤山}{たく|さん}ですよ。
%
\ruby{主人}{な|に}は?。
%
もう
お
\ruby[|g|]{就眠}{やすみ}?。
』

\原本頁{251-1}%
『
ハア、
%
あなた
にも
お
\ruby[|g|]{就眠}{やすみ}つて
お
\ruby{云}{い}ひつて。
%
\ruby{今}{いま}しがた
\ruby{既}{もう}。
』

\原本頁{251-2}%
『
\ruby{然樣}{さ|う}。
%
お
\ruby{春}{はる}さんは?。
』

\原本頁{251-3}%
『
まだ
\ruby[|g|]{裁縫}{しごと}を
\ruby{仕}{し}て
ゐます。
』

\原本頁{251-4}%
『
なか〳〵の
\ruby{人}{ひと}ネエ!。
』

\原本頁{251-5}%
『
\ruby{左樣}{さ|う}で
ございますとも、
%
\ruby{負}{ま}けない
\ruby{氣}{き}の
\ruby{人}{ひと}ですよ。
%
\ruby{何}{なん}でも
\ruby{妾}{わたし}にやあ
\ruby{負}{ま}けたくないと
\ruby{思}{おも}ひましてネ。
』

\原本頁{251-7}%
『
ホヽヽ、
%
だが、
%
あけすけで
\ruby{可愛}{か|はい}らしい
\ruby{兒}{こ}ネエ。
』

\原本頁{251-8}%
『
さうですよ、
%
\ruby{些}{ちつと}も
\ruby{毒}{どく}は
\ruby{無}{な}い
\ruby{人}{ひと}
で。
%
ですから
\ruby{今日}{け|ふ}の
お
\ruby{客}{きやく}さまの
\ruby{最初}{さい|しよ}の
\ruby{樣子}{やう|す}にやあ
\ruby{何樣}{ど|ん}なにか
\ruby{怒}{おこ}りましたらう!。
%
オホヽ、
%
そりやあ
\ruby[|g|]{可笑}{をかし}い
ほど
でしたよ。
』

\原本頁{251-11}%
『
\ruby{然樣}{さ|う}!。
%
そんなに
\ruby{最初}{さい|しよ}は
\ruby[|g|]{彼方}{あつち}で
\ruby{怒}{おこ}り
\ruby{立}{た}つて
つん〳〵
\ruby{仕}{し}て
\ruby{{\換字{遣}}}{や}つて
\ruby{來}{き}たの?。
』

\原本頁{252-2}%
『
さうですとも。
%
そりやあ
\ruby{甚}{ひど}い
\ruby{權幕}{けん|まく}
でしたの!。
』

\原本頁{252-3}%
『
それを
\ruby{何樣}{ど|う}して
\ruby{姊}{ねえ}さんが
\ruby{直}{ぢき}に
\ruby[|g|]{彼樣}{あんな}に
ヘイ〳〵
する
やうに
\ruby{仕}{し}て
お
\ruby{仕舞}{し|まひ}
だつたの?。
』

\原本頁{252-5}%
『
そりやあ
\ruby{何}{なん}ですもの!。
』

\原本頁{252-6}%
『
\ruby{何樣}{ど|う}したの?。
%
お
\ruby{{\換字{前}}}{まへ}さん
\ruby{悉皆}{すつ|かり}
\ruby{知}{し}つてゝ?。
』

\原本頁{252-7}%
『
すつかり
\ruby{知}{し}つてます、
%
\ruby{斯樣}{か|う}
なんですよ。
』

\原本頁{252-8}%
お
\ruby{富}{とみ}は
\ruby[||j>]{諄}{じゆん}
\ruby[||j>]{々}{ 〳〵}
% \ruby{諄々}{じゆん|〳〵}
として
\ruby{始末}{し|まつ}を
\ruby{說}{と}き、
%
お
\ruby{龍}{りう}は
\ruby{默々}{もく|〳〵}として
\ruby{一切}{いつ|さい}を
\ruby{聞}{き}き
\ruby{{\換字{終}}}{をは}りたり。
