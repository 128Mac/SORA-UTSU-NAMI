\Entry{其四十六}

% メモ 校正終了 2024-05-19 2024-06-14
\原本頁{253-1}%
\ruby{有}{あ}りつる
\ruby{事}{こと}の
いろ〳〵を
\ruby{語}{かた}りて
\ruby{後}{のち}、
%
\ruby{要}{えう}も% ルビ調整(原本通り)「えう」
\ruby{無}{な}き
\ruby{業}{わざ}
したりと
\ruby{聊}{いさゝ}か
\ruby{悔}{くや}みてか、
%
\ruby[|g|]{御就眠}{おやすみ}なさい
ましを
\ruby{最{\換字{終}}}{す|ゑ}の
\ruby{言葉}{こと|ば}
にして、
%
\ruby{年齡}{と|し}に
\ruby{似合}{に|あ}はず
くすみて
\ruby{老}{ふ}けたる
お
\ruby{富}{とみ}は
\ruby{靜}{しづか}に
\ruby{此室}{こ|こ}を% 原本では非通り字表記
\ruby{去}{さ}りぬ。

\原本頁{253-4}%
\ruby[|g|]{階子}{はしご}を
\ruby{下}{お}りし
\ruby{音}{おと}の
\ruby[|g|]{彼方}{かなた}に
\ruby{{\換字{消}}}{き}えて
よりは、
%
\ruby{室毎}{ま|ごと}
\ruby[g]{々々}{ 〴〵 }の
\ruby{襖}{ふすま}の
\ruby{隔}{へだ}てたればにや、
%
\ruby{但}{ただ}しは% 原本では非通り字表記
お
\ruby{春}{はる}も
\ruby{共}{とも}に
\ruby{皆}{みな}
\ruby{眠}{ねむ}りに
\ruby{就}{つ}きたればにや、
%
\ruby[|g|]{微少}{わづか}なる
\ruby{音響}{お|と}だに
\ruby{聞}{きこ}え
\ruby{來}{こ}ず、
%
\ruby{風}{かぜ}
\ruby{無}{な}き
\ruby{{\換字{冬}}}{ふゆ}の
\ruby{夜}{よ}の、
%
\ruby{{\換字{戸}}外}{そ|と}は
\ruby{定}{さだ}めし
\ruby{星斗}{ほ|し}
\ruby{燦然}{きら|〳〵}と
\ruby{霜}{しも}の
\ruby{降}{ふ}る
\ruby{最中}{も|なか}
なるべし、
%
\ruby{天地}{てん|ち}
\ruby{死}{し}せるが
\ruby{如}{ごと}く
\ruby{靜}{しづか}にて、
%
ただ% 原本では非通り字表記
\ruby[|g|]{流石}{さすが}
\ruby[||j>]{大}{おほ}
\ruby[||j>]{都}{みやこ}の
% \ruby{大都}{おほ|みやこ}の
\ruby{市}{まち}
\ruby{中}{なか}
なれば、
%
\ruby{此家}{こ|こ}% 原本では非通り字表記
よりは
やゝ
\ruby{離}{はな}れたれど、
%
\ruby{凍}{い}てたる
\ruby{路}{みち}に

\ruby{車}{くるま}の
\ruby{走}{はし}る
\ruby{轟}{とどろ}きの、% 原本では非通り字表記
%
\ruby{{\換字{遠}}}{とほ}く
より
\ruby{來}{きた}りては
\ruby{復}{また}
\ruby[|g|]{{\換字{遠}}方}{とほく}に
\ruby{去}{さ}るが
\ruby{斷}{た}えざる
のみ、
%
\ruby{犬}{いぬ}さへ
\ruby{鳴}{な}かず、
%
\ruby{穩}{おだ}やかに
\ruby{今{\換字{宵}}}{こ|よひ}は
\ruby{{\換字{更}}}{ふ}けたるなり。

\原本頁{253-11}%
\ruby{其}{その}
\ruby{故}{ゆゑ}は
\ruby[|g|]{主人}{あるじ}
ならでは
\ruby{知}{し}る
もの
なけれど、
%
\ruby[|g|]{樓上}{にかい}の
\ruby{此處}{こ|こ}には% 原本では非通り字表記
\ruby{特}{わざ}と
\ruby{電燈}{でん|とう}を
\ruby{忌}{い}みてか
\ruby{其}{そ}の
\ruby[|g|]{設備}{そなへ}
あらずして、
%
やゝ
\ruby{高}{たか}き
\ruby{置}{おき}
\ruby[|g|]{洋燈}{らんぷ}
の
いと
\ruby{美}{うつく}しきを
\ruby{用}{もち}ひたり。
%
\ruby{電燈}{でん|とう}は
これを
\ruby{細}{ほそ}むることも
%
\ruby{之}{これ}を
\ruby{太}{ふと}むることも
%
\ruby[<j||]{油}{あぶら}% 行末行頭の境界付近なので特例処置を施す
\ruby{燈}{ひ}の
\ruby{如}{ごと}く
\ruby{自在}{じ|ざい}には
あらで、
%
\ruby{點}{とも}せば
\ruby{明}{あか}る
\ruby{{\換字{過}}}{す}ぎ、
%
\ruby{點}{とも}さざれば
\ruby{全}{まつた}く
\ruby{暗}{くら}く、
%
\ruby{如}{し}くものも
\ruby{無}{な}き
\ruby{春}{はる}の
\ruby{朧夜}{おぼろ|よ}の
\ruby{朧氣}{おぼろ|げ}なる
\ruby{光}{ひかり}を、
%
\ruby{時々}{とき|〴〵}の
\ruby[||j>]{心}{こゝろ}
\ruby[||j>]{任}{ まか}せに
% \ruby{心任}{こゝろ|まか}せに
\原本頁{254-5}\改行%
\ruby{加減}{か|げん}して
\ruby{趣致}{おも|むき}を
\ruby{取}{と}る
やうなる
ことの
\ruby{叶}{かな}はねば、
%
\ruby{如何}{い|か}なる
\ruby{折}{をり}にか
\原本頁{254-6}\改行%
\ruby{面白}{おも|しろ}からぬ
ことの
\ruby{有}{あ}る
が
ためなるべし。
%
お
\ruby{龍}{りう}は
やがて
\ruby{衣}{い}を
\ruby{{\換字{更}}}{か}へ
\改行% 校正作業の簡略化のため
、
%
\原本頁{254-7}\改行%
\ruby[||j>]{枕}{まくら}
\ruby[||j>]{頭}{ もと}の
% \ruby{枕頭}{まくら|もと}の
\ruby{其}{その}
\ruby{燈}{ひ}を
\ruby{熄}{き}えん
と
するまで
\ruby{細}{ほそ}めて
\ruby{眠}{ねむ}りに
\ruby{就}{つ}きたり。

\原本頁{254-8}%
\ruby{燈火}{とも|しび}の
\ruby{光}{ひかり}は
\ruby{朦朧}{ぼん|やり}と
\ruby{一室}{いつ|しつ}を
\ruby{籠}{こ}めて、
%
\ruby{床間}{と|こ}には
\ruby{軸}{ぢく}を
\ruby{掛}{か}けずに
\ruby{此}{これ}のみを
\ruby{眺}{なが}めと
\ruby{挿}{さ}したる
\ruby{妙蓮寺}{めう|れん|じ}% TODO 暫定で「蓮 uf999」とする(参考「蓮 u84ee」)
\ruby[|g|]{山茶}{つばき}の、
%
\ruby{{\換字{半}}{\換字{咲}}}{なかば|さ}きたるが
\ruby{一輪}{いち|りん}、
%
\ruby{{\換字{咲}}}{さ}かざるが
\ruby{一點}{いつ|てん}、
%
\ruby{{\換字{浮}}}{う}き
\ruby{出}{い}づるが
\ruby{如}{ごと}く
\ruby{白}{しろ}く
\ruby{見}{み}えたる
\ruby{他}{ほか}には
\ruby{何}{なん}の
\ruby{心}{こゝろ}を
\ruby{惹}{ひ}くものも
\ruby{無}{な}し。
%
お
\ruby{龍}{りう}は
\ruby{此}{こ}の
\ruby{瀟洒}{せう|しや}にして
\ruby{淸}{きよ}らなる
\ruby{室}{しつ}の
\ruby{中}{うち}に、
%
\ruby{柔}{やは}らかなる
\ruby{美}{うつく}しき
\ruby{燈}{ひ}の
\ruby{光}{ひかり}を
\ruby{{\換字{浴}}}{あ}び、
%
\ruby{穩}{おだ}やかに
\ruby{沈々}{ちん|〳〵}と
\ruby{{\換字{更}}}{ふ}くる
\ruby{夜}{よ}を
\ruby{寢}{ね}て、
%
\ruby{優}{やさ}しく
\ruby{幸福}{さい|はひ}
\ruby{多}{おほ}かるべき
\ruby{夢}{ゆめ}に
\ruby{入}{い}らん
と
したり。
%
されど
\ruby{如何}{い|か}にしけん
\ruby{頓}{とみ}には
\ruby{夢}{ゆめ}に
\ruby{入}{い}りかねて、
%
\ruby{一度}{ひと|たび}
\ruby{二度}{ふた|たび}% 原本は非踊り字表記
\ruby{寢{\換字{返}}}{ね|がへ}りして、
%
\ruby{不圖}{ふ|と}
\ruby{眼}{め}を
\ruby{開}{ひら}き
\ruby{見}{み}れば、
%
\ruby{我}{わ}が
\ruby{頭}{かしら}の
\ruby{上}{うへ}に
\ruby{唯}{ただ}% 原本は非踊り字表記
\ruby{一羽}{いち|は}の
\ruby{白}{しろ}き
\ruby{鷺}{さぎ}の、
%
\ruby{羽}{はね}を
\ruby{斂}{をさ}め
\ruby{頸}{くび}を
\ruby{縮}{すく}めて
\ruby{物}{もの}
\ruby{思}{おも}ふが
\ruby{如}{ごと}く、
%
けろりと
\ruby{立}{た}ち
\ruby{居}{ゐ}たり。
%
\ruby{夢}{ゆめ}
にもあらず
\ruby{幻影}{まぼ|ろし}
にもあらず
\ruby{物}{もの}の
\ruby{精}{せい}
にもあらず、
%
\ruby{此}{これ}は
\ruby{是}{これ}
\ruby{豫}{かね}てより
\ruby{此樓}{こ|こ}に% 原本は非踊り字表記
\ruby{掛}{か}けられたる
\ruby{一面}{いち|めん}の
\ruby{額}{がく}の
\ruby{畫}{ゑ}なりしなり。

\原本頁{255-8}%
\ruby{鷺}{さぎ}は
\ruby{夕暮}{ゆふ|ぐれ}の
\ruby{小闇}{を|ぐら}きに
\ruby{立}{た}てるなり。
%
\ruby{燈火}{とも|しび}の
\ruby{光}{ひかり}は
\ruby{{\換字{弱}}々}{よわ|〳〵}
として
\ruby{其}{そ}の
\ruby{暗}{くら}さに
\ruby{同}{おな}じきなり。
%
\ruby{畫}{ゑ}には
\ruby{魂魄}{たま|しひ}ありや
\ruby{鷺}{さぎ}は
\ruby{今}{いま}
\ruby{動}{うご}き
\ruby{出}{いだ}さんとす。
