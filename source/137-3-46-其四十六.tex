\Entry{其四十六}

\ruby{有}{あ}りつる
\ruby{事}{こと}のいろ〳〵を
\ruby{語}{かた}りて
\ruby{後}{あと}、
\ruby{要}{えう}も
\ruby{無}{な}き
\ruby{業}{わざ}したりと
\ruby{聊}{いさゝ}か
\ruby{悔}{くや}みてか、
\ruby{御就眠}{お|や|すみ}なさいましを
\ruby{最{\換字{終}}}{す|ゑ}の
\ruby{言葉}{こと|ば}にして、
\ruby{年齡}{と|し}に
\ruby{似合}{に|あ}はずくすみて
\ruby{老}{ふ}けたる
お
\ruby{富}{とみ}は
\ruby{靜}{しづか}に
\ruby{此室}{こ|ゝ}を
\ruby{去}{さ}りぬ。

\ruby{階子}{はし|ご}を
\ruby{下}{くだ}りし
\ruby{音}{おと}の
\ruby{彼方}{かな|た}に
\ruby{{\換字{消}}}{き}えてよりは、
\ruby{室毎}{ま|ごと}の
\g詰めruby{々々}{〴〵}の
\ruby{襖}{ふすま}の
\ruby{隔}{へだ}てたればにや、
\ruby{但}{たゞ}しは
お
\ruby{春}{はる}も
\ruby{共}{とも}に
\ruby{皆眠}{みな|ねむ}りに
\ruby{就}{つ}きたればにや、
\ruby{微少}{わづ|か}なる
\ruby{音響}{お|と}だに
\ruby{聞}{きこ}え
\ruby{來}{こ}ず、
\ruby{風無}{かぜ|な}き
\ruby{{\換字{冬}}}{ふゆ}の
\ruby{夜}{よ}の、
\ruby{{\換字{戸}}外}{そ|と}は
\ruby{定}{さだ}めし
\ruby{星斗}{ほ|し}
\ruby{燦然}{きら|〳〵}と
\ruby{霜}{しも}の
\ruby{降}{ふ}る
\ruby{最中}{も|なか}なるべし、
\ruby{天地}{てん|ち}
\ruby{死}{し}せるが
\ruby{如}{ごと}く
\ruby{靜}{しづか}にて、たゞ
\ruby{流石}{さす|が}
\ruby{大都}{おほ|みやこ}の
\ruby{市中}{まち|なか}なれば、
\ruby{此家}{こ|ゝ}よりはやゝ
\ruby{離}{はな}れたれど、
\ruby{凍}{い}てたる
\ruby{路}{みち}に
\ruby{車}{くるま}の
\ruby{走}{はし}る
\ruby{轟}{とゞろ}きの、
\ruby{{\換字{遠}}}{とほ}くより
\ruby{來}{きた}りては
\ruby{復}{また}
\ruby{{\換字{遠}}方}{とほ|く}に
\ruby{去}{さ}るが
\ruby{斷}{た}えざるのみ、
\ruby{犬}{いぬ}さへ
\ruby{鳴}{な}かず、
\ruby{穩}{おだ}やかに
\ruby{今{\換字{宵}}}{こ|よひ}は
\ruby{更}{ふ}けたるなり。

\ruby{其故}{その|ゆゑ}は
\ruby{主人}{ある|じ}ならでは
\ruby{知}{し}るものなけれど、
\ruby{樓上}{にか|い}の
\ruby{此處}{こ|ゝ}には
\ruby{特}{わざ}と
\ruby{電燈}{でん|とう}を
\ruby{忌}{い}みてか
\ruby{其}{そ}の
\ruby{設備}{そな|へ}あらずして、やゝ
\ruby{高}{たか}き
\ruby{置洋燈}{おき|らん|ぷ}のいと
\ruby{美}{うつく}しきを
\ruby{用}{もち}ひたり。
\ruby{電燈}{でん|とう}はこれを
\ruby{細}{ほそ}むることも
\ruby{油燈}{あぶ|らひ}の
\ruby{如}{ごと}く
\ruby{自在}{じ|ざい}にはあらで、
\ruby{點}{とも}せば
\ruby{明}{あか}る
\ruby{{\換字{過}}}{す}ぎ、
\ruby{點}{とも}さざれば
\ruby{全}{まつた}く
\ruby{暗}{くら}く、
\ruby{如}{し}くものも
\ruby{無}{な}き
\ruby{春}{はる}の
\ruby{朧夜}{おぼ|ろよ}の
\ruby{朧氣}{おぼ|ろげ}なる
\ruby{光}{ひかり}を、
\ruby{時々}{とき|〴〵}の
\ruby{心任}{こゝろ|まか}せに
\ruby{加減}{か|げん}して
\ruby{趣致}{おも|むき}を
\ruby{取}{と}るやうなることの
\ruby{叶}{かな}はねば、
\ruby{如何}{い|か}なる
\ruby{折}{をり}にか
\ruby{面白}{おも|しろ}からぬことの
\ruby{有}{あ}るがためなるべし。
お
\ruby{龍}{りう}はやがて
\ruby{衣}{い}を
\ruby{更}{か}へ、
\ruby{枕頭}{まくら|もと}の
\ruby{其燈}{その|ひ}を
\ruby{熄}{き}えんとするまで
\ruby{細}{ほそ}めて
\ruby{眠}{ねむ}りに
\ruby{就}{つ}きたり。

\ruby{燈火}{とも|しび}の
\ruby{光}{ひかり}は
\ruby{朦朧}{ぼん|やり}と
\ruby{一室}{いつ|しつ}を
\ruby{籠}{こ}めて、
\ruby{床間}{と|こ}には
\ruby{軸}{ぢく}を
\ruby{掛}{か}けずに
\ruby{此}{これ}のみを
\ruby{眺}{なが}めと
\ruby{挿}{さ}したる
\ruby{妙蓮寺山茶}{めう|れん|じ|つば|き}の、
\ruby{{\換字{半}}{\換字{咲}}}{なかば|さ}きたるが
\ruby{一輪}{いち|りん}、
\ruby{{\換字{咲}}}{さ}かざるが
\ruby{一點}{いつ|てん}、
\ruby{{\換字{浮}}}{う}き
\ruby{出}{い}づるが
\ruby{如}{ごと}く
\ruby{白}{しろ}く
\ruby{見}{み}えたる
\ruby{他}{ほか}には
\ruby{何}{なん}の
\ruby{心}{こゝろ}を
\ruby{惹}{ひ}くものも
\ruby{無}{な}し。
お
\ruby{龍}{りう}は
\ruby{此}{こ}の
\ruby{瀟洒}{せう|しや}にして
\ruby{淸}{きよ}らなる
\ruby{室}{しつ}の
\ruby{中}{うち}に、
\ruby{柔}{やは}らかなる
\ruby{美}{うつく}しき
\ruby{燈}{ひ}の
\ruby{光}{ひかり}を
\ruby{浴}{あ}び、
\ruby{穩}{おだ}やかに
\ruby{沈々}{ちん|〳〵}と
\ruby{更}{ふ}くる
\ruby{夜}{よ}を
\ruby{寢}{ね}て、
\ruby{優}{やさ}しく
\ruby{幸福}{さい|はひ}
\ruby{多}{おほ}かるべき
\ruby{夢}{ゆめ}に
\ruby{入}{い}らんとしたり。
されど
\ruby{如何}{い|か}にしけん
\ruby{頓}{とみ}には
\ruby{夢}{ゆめ}に
\ruby{入}{い}りかねて、
\ruby{一度}{ひと|たび}
\ruby{二度}{ふた|たび}
\ruby{寢{\換字{返}}}{ね|がへ}りして、
\ruby{不圖眼}{ふ|と|め}を
\ruby{開}{ひら}き
\ruby{見}{み}れば、
\ruby{我}{わ}が
\ruby{頭}{かしら}の
\ruby{上}{うへ}に
\ruby{唯}{たゞ}
\ruby{一羽}{いち|は}の
\ruby{白}{しろ}き
\ruby{鷺}{さぎ}の、
\ruby{羽}{はね}を
\ruby{斂}{をさ}め
\ruby{頸}{くび}を
\ruby{縮}{すく}めて
\ruby{物思}{もの|おも}ふが
\ruby{如}{ごと}く、けろりと
\ruby{立}{た}ち
\ruby{居}{ゐ}たり。
\ruby{夢}{ゆめ}にもあらず
\ruby{幻影}{まぼ|ろし}にもあらず
\ruby{物}{もの}の
\ruby{精}{せい}にもあらず、
\ruby{此}{これ}は
\ruby{是}{これ}
\ruby{豫}{かね}てより
\ruby{此樓}{こ|ゝ}に
\ruby{掛}{か}けられたる
\ruby{一面}{いち|めん}の
\ruby{額}{がく}の
\ruby{畫}{ゑ}なりしなり。

\ruby{鷺}{さぎ}は
\ruby{夕暮}{ゆふ|ぐれ}の
\ruby{小闇}{を|ぐら}きに
\ruby{立}{た}てるなり。
\ruby{燈火}{とも|しび}の
\ruby{光}{ひかり}は
\ruby{{\換字{弱}}々}{よわ|〳〵}として
\ruby{其}{そ}の
\ruby{暗}{くら}さに
\ruby{同}{おな}じきなり。
\ruby{畫}{ゑ}には
\ruby{魂魄}{たま|しひ}ありや
\ruby{鷺}{さぎ}は
\ruby{今}{いま}
\ruby{動}{うご}き
\ruby{出}{いだ}さんとす。

