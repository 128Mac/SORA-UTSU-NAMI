\Entry{其四十七}

\原本頁{}%
\ruby{我}{わ}が
\ruby{眼}{め}の
\ruby{彼}{かれ}を
\ruby{見}{み}つむれば、
%
\ruby{彼}{かれ}の
\ruby{眼}{め}もまたあり〳〵と
\ruby{我}{われ}を
\ruby{見詰}{み|つ}めて、
%
\ruby{漸}{やうや}く
\ruby[g]{此方}{こなた}に
\ruby{{\換字{近}}}{ちか}づき
\ruby{來}{きた}らんとする
\ruby{氣勢}{いき|ほひ}するに、
%
お
\ruby{龍}{りう}は
\ruby{思}{おも}はず
\ruby{知}{し}らず
\ruby{慄然}{ぞ|つ}と
\ruby{仕}{し}たりしが、
%
\ruby{忽地}{たちま|ち}にまた
\ruby{自}{みづか}ら
\ruby{笑}{わら}つて、
%
\ruby{何}{なん}の、
%
\ruby{燈火}{とも|しび}の
\ruby{工合}{ぐ|あひ}にて
\ruby{{\換字{浮}}出}{うき|いだ}したるやうにこそ
\ruby{見}{み}ゆれ、
%
\ruby{不思議}{ふ|し|ぎ}も
\ruby{{\換字{更}}}{さら}に
\ruby{無}{な}き
\ruby{普{\換字{通}}}{た|だ}の
\ruby{繪}{ゑ}なるをやと
\ruby{思}{おも}へば、
%
\ruby{鷺}{さぎ}はまた
\ruby{凝然}{つく|ねん}として
\ruby{畫}{ゑ}の
\ruby{中}{うち}に
\ruby{靜}{しづか}に
\ruby{立}{た}てるのみ。

\原本頁{}%
\ruby{思}{おも}へば
\ruby{此}{こ}の
\ruby{畫}{ゑ}は
\ruby{{\換字{古}}}{ふる}くより
\ruby{姊}{ねえ}さんの
\ruby{有}{も}てる
\ruby{畫}{ゑ}にて、
%
\ruby{幾年}{いく|とせ}の
\ruby{{\換字{前}}}{まへ}なりしか
\ruby{明}{あき}らかならねど、
%
\ruby{我}{わ}が
\ruby{{\換字{猶}}}{なほ}
\ruby{年}{とし}ゆかで
\ruby{{\換字{遠}}慮氣}{ゑん|りよ|げ}も
\ruby{無}{な}く
\ruby{明暮}{あけ|くれ}に
\ruby{{\換字{遊}}}{あそ}びに
\ruby{來}{き}ては
\ruby{姊}{ねえ}さんに
\ruby{甘}{あま}へし
\ruby{十幾歳}{じふ|いく|つ}の
\ruby{頃}{ころ}、
%
\ruby{如何}{い|か}なる
\ruby{折}{をり}にか
\ruby{此}{こ}の
\ruby{額}{がく}を
\ruby{見}{み}て、
%
\ruby{姊}{ねえ}さん
\ruby{此}{こ}の
\ruby{繪}{ゑ}は
\ruby{淋}{さみ}しくて
\ruby{不厭}{い|や}な
\ruby{繪}{ゑ}なことネエ、
%
と
\ruby{云}{い}ひしに、
%
\ruby{其樣}{そ|ん}な
\ruby{事}{こと}を
\ruby{御云}{お|い}ひで
\ruby{無}{な}い、
%
\ruby{此}{こ}りやあ
お
\ruby{{\換字{前}}}{まへ}の
\ruby{書}{か}いた
\ruby{繪}{ゑ}ぢやあ
\ruby{無}{な}いか、
%
と
\ruby{云}{い}はれて、
%
\ruby{調戲}{から|か}はれたりとは
\ruby{知}{し}らず、
%
\ruby{氣味}{き|み}の
\ruby{惡}{わる}さに
\ruby{吃驚}{びつ|くり}して
\ruby{顏}{かほ}の
\ruby{色}{いろ}を
\ruby{變}{か}へ、
%
あ、
%
\ruby{惡}{わる}い
\ruby{戲談}{じやう|だん}を
\ruby{云}{い}つた、
%
\ruby{勘{\換字{忍}}}{か|に}して% 原文通り「勘忍」
お
\ruby{吳}{く}れ、
%
たゞ
\ruby{少}{すこ}し
\ruby{譯}{わけ}があつて
\ruby{妾}{わたし}が
\ruby{有}{も}つて
\ruby{居}{ゐ}る
\ruby{此}{この}
\ruby{繪}{ゑ}を
\ruby{可厭}{い|や}だつて
お
\ruby{云}{い}ひだつたのが
\ruby{甚}{ひど}く
\ruby{可厭}{い|や}に
\ruby{聞}{きこ}えたものだから、
%
\ruby{詰}{つま}らないことを
\ruby{云}{い}つて
お
\ruby{{\換字{前}}}{まへ}を
\ruby{吃驚}{びつ|くり}させた、
%
\ruby{妾}{わたし}が
\ruby{惡}{わる}かつた、
%
と
\ruby{謝罪}{あや|ま}られ、
%
\ruby{慰}{なぐさ}められし
\ruby[g]{記臆}{おぼえ}あり。% 原本通り「おぼえ」
%
\ruby{其}{そ}の
\ruby{時}{とき}
\ruby{我}{わ}が
\ruby{心直}{こゝろ|たゞち}におちつきて、
%
\ruby{何}{なに}、
%
\ruby{姊}{ねえ}さんが
\ruby{好}{すき}なのなら
\ruby{妾}{わたし}も
\ruby{好}{すき}になるは、
%
そして
\ruby[h|]{妾}{わたし}
\ruby{眞似}{ま|ね}をして
\ruby{畫}{か}いてあげるは、
%
と
\ruby{云}{い}ひて、
%
\ruby{其}{そ}の
\ruby{日}{ひ}
\ruby{筆}{ふで}を
\ruby{執}{と}つて
\ruby{見描}{み|うつ}しの
\ruby{覺束}{おぼ|つか}なくも、
%
\ruby{何樣}{ど|う}やら
\ruby{斯樣}{か|う}やら
\ruby{似}{に}つこらしきものを
\ruby{書}{か}きて
\ruby{與}{あた}へて、
%
\ruby{大}{おほき}に
\ruby{褒}{ほ}められ
\ruby{悅}{よろこ}ばれしことありたり。
%
されど
\ruby{其}{そ}の
\ruby{理由}{わ|け}といふことは
\ruby{聞}{き}きもせず、
%
\ruby{聞}{き}かんともせで、
%
\ruby{其儘}{その|まゝ}に
\ruby{打{\換字{過}}}{うち|す}ぎ、
%
それより
\ruby{後}{のち}
\ruby{幾度}{いく|たび}と
\ruby{無}{な}く
\ruby{此}{こ}の
\ruby{繪}{ゑ}を
\ruby{見}{み}、
%
\ruby{此}{こ}の
\ruby{繪}{ゑ}の
\ruby{下}{した}に
\ruby{寢}{ね}たる
\ruby{事}{こと}もありしが、
%
\ruby{氣}{き}にもかけず、
%
\ruby{心}{こゝろ}にも
\ruby{止}{と}めず
\ruby{今日}{け|ふ}に
\ruby{至}{いた}りしに、
%
\ruby{今{\換字{宵}}}{こ|よひ}はたま〳〵
\ruby{夜}{よ}の
\ruby{{\換字{更}}}{ふ}けて
\ruby{稀}{めづ}らしく
\ruby{靜寂}{しづ|か}に、
%
\ruby{燈火}{とも|しび}の
\ruby{光}{ひか}りの
\ruby{朦朧}{ぼん|やり}したる
\ruby{工合}{ぐ|あひ}の
\ruby{繪}{ゑ}に
\ruby{映}{うつ}り
\ruby{合}{あ}へるが
\ruby{上}{うへ}、
%
\ruby{我}{わ}が
\ruby{心}{こゝろ}のさま〴〵の
\ruby{事}{こと}を
\ruby{思}{おも}ひて
\ruby{異}{あや}しく
\ruby{冴}{さ}えたるあまり、
%
ふと
\ruby{我}{わ}が
\ruby{眼}{め}につきて、
%
\ruby{我}{わ}が
\ruby{思}{おもひ}の
\ruby{此}{これ}に
\ruby{牽}{ひ}かれしなるべし、
%
\ruby{此}{この}
\ruby{繪}{ゑ}の
\ruby{昨日}{きの|ふ}に
\ruby{今日}{け|ふ}は
\ruby{何一}{なに|ひと}つ
\ruby{異}{かは}りたることもあらぬを、
%
\ruby{何時}{い|つ}に
\ruby{無}{な}く
\ruby{鷺}{さぎ}の
\ruby{動}{うご}き
\ruby{出}{いで}もするやうに
\ruby{思}{おも}ひ
\ruby{做}{な}すも
\ruby{愚}{おろか}なることなりと
\ruby{思}{おも}ひ
\ruby{{\換字{消}}}{け}しつ、
%
お
\ruby{龍}{りう}は
\ruby{眠}{ねむ}らんとして
\ruby{{\換字{強}}}{し}ひて
\ruby{眼眶}{ま|ぶた}を
\ruby{合}{あは}せたり。

\原本頁{}%
\ruby{寢苦}{ね|ぐる}しきといふにはあらねど
\ruby{{\換字{猶}}}{なほ}
\ruby{夢}{ゆめ}に
\ruby{入}{い}りかねて、
%
ふとまた
\ruby{眼}{め}を
\ruby{開}{ひら}けば、
%
\ruby{鷺}{さぎ}は
\ruby{薄}{うす}き
\ruby{闇}{やみ}に
\ruby{動}{うご}きて
\ruby{今}{いま}や
\ruby[g]{此方}{こなた}に
\ruby{歩}{あゆ}まんとするなり。

\原本頁{}%
\ruby{少}{すこ}し
\ruby{理由}{わ|け}があつてわたしが
\ruby{有}{も}つて
\ruby{居}{ゐ}る
\ruby{繪}{ゑ}と
\ruby{慥}{たしか}に
\ruby{彼}{あ}の
\ruby{時}{とき}に
\ruby{姊}{ねえ}さんの
\ruby{云}{い}ひたる
\ruby{理由}{わ|け}とは
\ruby{何}{なに}の
\ruby{理由}{わ|け}なるべきか、
%
\ruby{彼}{か}の
\ruby{時}{とき}はうつかり
\ruby{聞}{きゝ}
\ruby{流}{なが}して
\ruby{其}{そ}の
\ruby{仔細}{し|さい}を
\ruby{{\換字{尋}}}{たづ}ねもせず、
%
\ruby{{\換字{又}}}{また}その
\ruby{後}{のち}は
\ruby{此}{こ}の
\ruby{繪}{ゑ}につきて
\ruby{一}{ひ}ト
\ruby{言}{こと}の
\ruby{談話}{はな|し}を
\ruby{仕}{し}たることもなければ、
%
\ruby{其}{それ}の
\ruby{解}{わか}らうやうは
\ruby{無}{な}けれど、
%
\ruby{今}{いま}
\ruby{思}{おも}へば
\ruby{此}{こ}の
\ruby{繪}{ゑ}につきては
\ruby{何}{なに}か
\ruby{深}{ふか}いわけの
\ruby{有}{あ}りさうな
\ruby{心持}{こゝろ|もち}のする!。
%
\ruby{姊}{ねえ}さんは
\ruby{自{\換字{分}}}{じ|ぶん}の
\ruby{{\換字{過}}去話}{むか|し|ばなし}なぞをなさつた
\ruby{事}{こと}は
\ruby{些少}{すこ|し}も
\ruby{無}{な}ければ、
%
\ruby{眼}{め}に
\ruby{看}{み}たるほかには
\ruby{我}{われ}は
\ruby{何一}{なに|ひと}つ
\ruby{知}{し}らねど、
%
\ruby{徃時}{むか|し}は
\ruby{一體}{いつ|たい}どういふ
\ruby{徑路}{すぢ|みち}を
\ruby{經}{へ}た
\ruby{人}{ひと}?、
%
\ruby{此}{こ}の
\ruby{繪}{ゑ}にはまた
\ruby{何}{ど}のやうな
\ruby{理由}{わ|け}があるやら?、
%
\ruby{妾}{わたし}の
\ruby{身}{み}にしても
\ruby{種々}{いろ|〳〵}の
\ruby{{\換字{過}}去}{むか|し}がある、
%
\ruby{姊}{ねえ}さんの
\ruby{徃時}{むか|し}にも
\ruby{何}{なに}も
\ruby{無}{な}い
\ruby{事}{こと}は
\ruby{有}{あ}るまい、
%
\ruby{他}{ほか}の
\ruby{事}{こと}は
\ruby{兎}{と}も
\ruby{角}{かく}も
\ruby{此}{こ}の
\ruby{繪}{ゑ}に
\ruby{就}{つ}いてだけでも!。
%
あゝ
\ruby{然}{しか}し
\ruby{此}{こ}の
\ruby{樣}{やう}な
\ruby{事}{こと}をおもうても
\ruby{何}{なん}の
\ruby{甲{\換字{斐}}}{か|ひ}も
\ruby{無}{な}きことを、
%
と
お
\ruby{龍}{りう}はいろ〳〵に
\ruby{思}{おも}へし
\ruby{末}{すゑ}には
\ruby{心}{こゝろ}をなだらかにして、
%
\ruby{彼}{か}の
\ruby{鷺}{さぎ}の
\ruby{繪}{ゑ}を
\ruby{何氣}{なに|げ}もなく
\ruby{見}{み}たり。
