\Entry{其四十八}

% メモ 校正終了 2024-05-19 2024-06-14
\原本頁{260-3}%
\ruby{幾度}{いく|たび}と
\ruby{無}{な}く
\ruby{此}{この}
\ruby{繪}{ゑ}も
\ruby{見}{み}たりしが、
%
\ruby{心}{こゝろ}の
\ruby{中}{うち}に
\ruby{物}{もの}の
ありし
\ruby{時}{とき}は、
%
ただ% 原本では非通り字表記
\原本頁{260-4}\改行%
\ruby{其}{それ}に
\ruby{屈托}{くつ|たく}して
\ruby{眼}{め}にも
\ruby[|g|]{自然}{おのづ}と
\ruby{着}{つ}かず、
%
また
\ruby{何事}{なに|ごと}も
\ruby{無}{な}き
\ruby{時}{とき}は
\ruby{氣}{き}にも
\原本頁{260-5}\改行%
\ruby{止}{と}めず
\ruby{其}{その}
\ruby{儘}{まゝ}に
\ruby{見}{み}て
\ruby{{\換字{過}}}{すご}したりし
\ruby{故}{ゆゑ}にや、
%
\ruby{今}{いま}まで
\ruby{幾年}{いく|とせ}の
\ruby{間}{あひだ}
ただの% 原本では非通り字表記
\ruby{一度}{いち|ど}も、
%
\ruby{{\換字{古}}}{ふる}き
\ruby[||j>]{疇}{その}
\ruby[||j>]{昔}{むかし}の
% \ruby{疇昔}{その|むかし}の
\ruby{事}{こと}などを
\ruby{思}{おも}ひ
\ruby{出}{だ}したる
\ruby{折}{をり}も
\ruby{無}{な}かりしに、
%
\ruby{今{\換字{宵}}}{こ|よひ}は
\ruby{差}{さし}
\ruby{當}{あた}りて
\ruby{口惜}{く|や}しい
といふ
\ruby{事}{こと}も
\ruby{悲}{かな}しい
といふ
\ruby{事}{こと}も
\ruby{{\換字{又}}}{また}
\ruby{氣{\換字{遣}}}{き|づか}はしい
といふ
ことも
あるには
あらず、
%
まして
\ruby{人}{ひと}には
\ruby{明}{あ}かせぬ
\ruby{羞}{はづ}かしき
\原本頁{260-9}\改行%
\ruby{思}{おも}ひに
\ruby{胸}{むね}の
\ruby{底}{そこ}を
\ruby{掻}{か}き
\ruby{挘}{むし}りたき
やうの
\ruby{心地}{こゝ|ち}する
といふ
\ruby{事}{こと}
なんどの
\ruby{有}{あ}るにも
あらねど、
%
さればとて
\ruby{{\換字{又}}}{また}
\ruby{全}{まつた}く
\ruby{雲}{くも}
\ruby{無}{な}き
\ruby{{\換字{空}}}{そら}の
ただ% 原本では非通り字表記
\ruby{美}{うつく}しく
\ruby{靑}{あを}きやうに
\ruby{胸}{むね}の
\ruby{中}{うち}の
さつぱりと
\ruby[|g|]{乾淨}{きれい}
なるにも
あらず、
%
\ruby{取}{と}り
\ruby{詰}{つ}めて
\原本頁{261-2}\改行%
\ruby{此}{これ}を
\ruby{思}{おも}ふといふ
\ruby{事}{こと}も
\ruby{無}{な}けれど、
%
\ruby{何}{なに}も
\ruby{彼}{か}も
\ruby{忘}{わす}れ
\ruby{果}{は}てゝ
\ruby{物}{もの}
\ruby{覺}{おぼ}えぬ
\ruby{夢路}{ゆめ|ぢ}に
\ruby{入}{い}る
といふ
ほどにも
なりかぬる
より、
%
\ruby{偶然}{ふ|と}、
%
\ruby{眼}{め}の
\ruby{{\換字{前}}}{まへ}の
\ruby{此}{こ}の
\原本頁{261-4}\改行%
\ruby{鷺}{さぎ}の
\ruby{繪}{ゑ}などの
\ruby{心}{こゝろ}に
\ruby{{\換字{留}}}{と}まりて、
%
\ruby[|g|]{昨日}{きのふ}
\ruby{今日}{け|ふ}の
\ruby{事}{こと}にも
あらぬ
\ruby{{\換字{古}}}{ふる}き
\ruby[|g|]{記臆}{おぼえ}の% 原本通り「おぼえ」
\ruby{新}{あらた}に
\ruby{{\換字{浮}}}{うか}び
\ruby{現}{あら}はれ
\ruby{來}{きた}れるにや。
%
お
\ruby{龍}{りう}は
\ruby{{\換字{猶}}}{なほ}
\ruby{忘}{わす}れん
として
\ruby{其}{そ}の
\ruby{鷺}{さぎ}を
\原本頁{261-6}\改行%
\ruby{忘}{わす}れ
\ruby{得}{え}かねたり。

\原本頁{261-7}%
『
それにしても
\ruby{晝間}{ひる|ま}の
\ruby{姊}{ねえ}さんの
\ruby{言葉}{こと|ば}は、
%
\ruby{妾}{わたし}が
\ruby{心}{こゝろ}を
\ruby{引}{ひき}
\ruby{立}{た}てゝ
\ruby{下}{くだ}さらうとからの
\ruby[||j>]{戲}{じやう}
\ruby[||j>]{談}{ だん}
% \ruby{戲談}{じやう|だん}
\ruby[||j>]{{\換字{交}}}{ まじ}りの
\ruby{其言}{そ|れ}には
\ruby{相{\換字{違}}}{さう|ゐ}% ルビ調整(原本通り)非グループルビ
\ruby{無}{な}けれど、
%
\ruby{餘}{あんま}り
\ruby{{\換字{強}}{\換字{過}}}{きつ|す}ぎて
\原本頁{261-9}\改行%
\ruby{{\換字{強}}{\換字{過}}}{きつ|す}ぎて
\ruby{一々}{いち|〳〵}
\ruby{妾}{わたし}の
\ruby{耳}{みゝ}には
\ruby{可厭}{い|や}に
\ruby{聞}{きこ}えて
ならざりしが、
%
\ruby{{\換字{若}}}{も}し
\ruby{彼言}{あ|れ}が
まあ
\ruby{姊}{ねえ}さんの
\ruby[|g|]{眞實}{ほんと}の
\ruby{意}{こゝろ}から
の
ことなら、
%
\ruby{姊}{ねえ}さんは
\ruby{矢張}{やつ|ぱり}
\ruby{靜岡}{しづ|をか}の
\原本頁{261-11}\改行%
\ruby{叔母}{を|ば}さんも
\ruby{同}{おな}じことの
\ruby{人}{ひと}!。
%
そりやあ
\ruby{智惠}{ち|ゑ}も
\ruby{有}{あ}り
\ruby{餘}{あま}るほど
\ruby{有}{あ}り
\改行% 校正作業の簡略化のため
、
%
\原本頁{262-1}\改行%
\ruby[||j>]{同}{おも}
\ruby[||j>]{{\換字{情}}}{ひやり}も
% \ruby{同{\換字{情}}}{おも|ひやり}も
\ruby{痒}{かゆ}い
ところへ
\ruby{手}{て}の
\ruby{屆}{とゞ}く% 「屆」「届」 原本通り「屆」
\ruby{程}{ほど}
\ruby{有}{あ}り、
%
\ruby{氣位}{き|ぐらゐ}も
\ruby{大層}{たい|そう}に
\ruby{{\換字{違}}}{ちが}つて、
%
\ruby{何}{なに}も
\ruby{彼}{か}も
\ruby{{\換字{勝}}}{すぐ}れては
お
\ruby{在}{いで}なさるには
\ruby{相{\換字{違}}}{さう|ゐ}% ルビ調整(原本通り)非グループルビ
\ruby{無}{な}いけれども、
%
\ruby{種々}{いろ|〳〵}の
ことが
\ruby{{\換字{勝}}}{すぐ}れて
\ruby{御在}{お|いで}なさる
だけに
\ruby{仰}{おつし}ある
\ruby{事}{こと}も
\ruby{輪}{わ}を
\ruby{掛}{か}けて、
%
\ruby{叔母}{を|ば}は
ただ% 原本では非通り字表記
\原本頁{262-4}\改行%
\ruby{堅人}{かた|じん}を
\ruby[|g|]{{\換字{丈}}夫}{をとこ}に
\ruby{有}{も}て
といつた
ところを、
%
\ruby{姊}{ねえ}さんは
\ruby{世}{よ}を
\ruby{渡}{わた}る
\ruby{伎倆}{う|で}の
\原本頁{262-5}\改行%
ある
\ruby{毅然}{しつ|かり}とした
\ruby{立派}{りつ|ぱ}な
\ruby[|g|]{{\換字{漢}}子}{をとこ}を
\ruby{擇}{よ}つて
\ruby{配偶}{つれ|あひ}に
しろと
\ruby{御云}{お|い}ひに
なつただけで、% 原本では非通り字表記
%
\ruby{心}{しん}は
\ruby{矢張}{やつ|ぱり}
\ruby[|g|]{差{\換字{違}}}{ちがひ}は
\ruby{有}{あ}りは
\ruby{仕}{し}
\ruby{無}{な}い。
%
まさかに
\ruby{姊}{ねえ}さんの
\ruby{本心}{し|ん}からとは
\ruby{思}{おも}へぬ
けれども、
%
\ruby{全然}{まる|〳〵}
\ruby{意}{こゝろ}にも
\ruby{無}{な}いことを
\ruby{御云}{お|い}ひでは
\原本頁{262-8}\改行%
\ruby{無}{な}かつた
\ruby{樣子}{やう|す}。
%
\ruby{一旦}{いつ|たん}
\ruby{斯樣}{か|う}いふ
\ruby{不}{ふ}
\ruby[<j>]{幸}{しあはせ}な%「幸」ここは「は」
\ruby{目}{め}を
\ruby{見}{み}て
\ruby{來}{き}た
\ruby{妾}{わたし}に、% 行末行頭の境界付近なので特例処置を施す
%
また
\ruby[<j||]{男}{をとこ}を
\ruby{有}{も}てと
\ruby{仰}{おつし}あつて、
%
\ruby[|g|]{眞實}{ほんと}に
\ruby{然樣}{さ|う}いふことを
\ruby{妾}{わたし}が
\ruby{唯々}{は|い}と
\ruby{云}{い}ひさうなやうに
\ruby{思}{おも}つて
おいでゞも
\ruby{有}{あ}らうか
\ruby{知}{し}らん。
%
あれほど
\ruby{能}{よ}く
\ruby{何}{なに}も
\原本頁{262-11}\改行%
\ruby{彼}{か}も
\ruby{御解}{お|わか}りの
\ruby{姊}{ねえ}さんで、
%
あれほど
\ruby{妾}{わたし}を
\ruby{可愛}{か|はい}がつて
\ruby{下}{くだ}さる
\ruby{彼}{あ}の
\ruby{姊}{ねえ}さんで、
%
そして
\ruby{現今}{い|ま}ぢやあ
\ruby{此}{こ}の
\ruby{廣}{ひろ}い
\ruby{世界}{せ|かい}の
\ruby{中}{なか}で
\ruby{妾}{わたし}に
\ruby{取}{と}つちやあ
\原本頁{263-2}\改行%
\ruby{叔母}{を|ば}よりも
\ruby{誰}{たれ}よりも
\ruby{一番}{いち|ばん}
\ruby{馴染}{なじ|み}の% 「{馴染}{な|じみ}」だと思うが原本通り
\ruby{深}{ふか}い
\ruby{彼}{あ}の
\ruby{姊}{ねえ}さんが、
%
よもや
\ruby{妾}{わたし}を
\原本頁{263-3}\改行%
\ruby{其樣}{そ|ん}なことを
\ruby{爲}{し}さうなものとは
\ruby{思}{おも}つて
\ruby{御在}{お|いで}ぢやあ
\ruby{有}{あ}るまいと
\ruby{思}{おも}つては
\ruby{居}{ゐ}る
けれど
‥‥
。
%
\ruby{成程}{なる|ほど}
\ruby{二度}{に|ど}
\ruby{三度}{さん|ど}
\ruby[|g|]{{\換字{丈}}夫}{をとこ}を
\ruby{有}{も}つ
\ruby{人}{ひと}も
\ruby{稀}{めづ}らしくは
\ruby{無}{な}いから、
%
\ruby{叔母}{を|ば}の
\ruby{云}{い}ふのも
\ruby{世間}{せ|けん}
\ruby{普{\換字{通}}}{あり|ふれ}では
\ruby{有}{あ}らうし、
%
\ruby[|g|]{不思議}{ふしぎ}は
\ruby{無}{な}からう
けれども、
%
そりやあ
\ruby{他}{よそ}の
\ruby{人}{ひと}の
\ruby{話}{はなし}で、
%
\ruby{妾}{わたし}は
\ruby{妾}{わたし}の
\ruby[||j>]{性}{しやう}
\ruby[||j>]{{\換字{分}}}{ ぶん}。
% \ruby{性{\換字{分}}}{しやう|ぶん}。
%
\ruby[<j||]{妾}{わたし}の
\ruby[||j>]{性}{しやう}
\ruby[||j>]{{\換字{分}}}{ ぶん}を
% \ruby{性{\換字{分}}}{しやう|ぶん}を
\ruby{知}{し}りきつて
\ruby{御在}{お|いで}の
あの
\ruby{姊}{ねえ}さんが、
%
\ruby{妾}{わたし}も
\ruby{矢張}{やつ|ぱり}
\ruby{他}{よそ}の
\ruby{人}{ひと}と
\ruby{同}{おな}じやうに、
%
\ruby{時}{とき}が
\ruby{經}{た}ちさへ
すりやあ
\ruby{{\換字{又}}}{また}
\ruby{新規}{しん|き}に
\ruby{男}{をとこ}を
\ruby{有}{も}つものと
\ruby{思}{おも}つて
\ruby{御在}{お|いで}ぢやあ
\ruby{有}{あ}るまい。
%
そんな
\ruby{氣}{き}に
なれる
やうな
\ruby[||j>]{薄}{はく}
\ruby[||j>]{{\換字{情}}}{じやう}な
% \ruby{薄{\換字{情}}}{はく|じやう}な
\ruby{妾}{わたし}ならば、
%
\ruby{人}{ひと}に
\ruby{棄}{す}てられた
からと
\ruby{云}{い}つて、
%
\ruby{彼樣}{あ|あ}は% ルビ調整(原本通り)非踊り字表記
\ruby{口惜}{く|やし}がらない。
%
\ruby{姊}{ねえ}さんは
\ruby{妾}{わたし}が
\ruby{何樣}{ど|ん}な
\ruby{女}{をんな}だ
といふ
\ruby{事}{こと}は
\ruby{知}{し}りきつて
お
\ruby{在}{いで}に
\ruby{{\換字{違}}}{ちが}ひ
\ruby{無}{な}い。
%
けれども
\ruby{{\換字{過}}日}{こな|ひだ}からの
\ruby{御談}{お|はなし}
といひ、
%
\ruby{今日}{け|ふ}の
\ruby{御言葉}{お|こと|ば}
といひ、
%
\ruby{何}{なん}だか
\ruby[<j||]{妾}{わたし}には
\ruby{可厭}{い|や}に
\ruby{聞}{きこ}えて
ならぬ。
%
\ruby{{\換字{若}}}{も}しや
\ruby{妾}{わたし}を
\ruby{矢張}{やつ|ぱり}
\ruby[|g|]{眞實}{ほんと}に
\ruby{今後}{これ|から}また
\ruby{男}{をとこ}でも
\ruby{持}{も}ちさうなものに
\ruby{思}{おも}つて
\ruby{御在}{お|いで}のか
\ruby{知}{し}らん。
%
まさか
\ruby{其樣}{そ|ん}な
\ruby{事}{こと}は
\原本頁{264-4}\改行%
\ruby{有}{あ}るまいが。
%
いや〳〵
\ruby{水野}{みづ|の}
といふ
\ruby{人}{ひと}の
\ruby{事}{こと}を
\ruby{幾度}{いく|ど}も
\ruby{御云}{お|い}ひで、
%
\ruby{然}{さ}も
\ruby{妾}{わたし}が
\ruby{其}{そ}の
\ruby{人}{ひと}を
\ruby{何樣}{ど|う}かでも
\ruby{思}{おも}つて
\ruby{居}{ゐ}るやうに
\ruby{御取}{お|と}りのやうに
\ruby{聞}{きこ}えた。
%
あ、
%
\ruby{{\換字{若}}}{も}し
\ruby{左樣}{さ|う}
\ruby{御取}{お|と}りの
やうなら、
%
\ruby{其}{そ}れあ
\ruby{働}{はたら}き
のある
\ruby{男}{をとこ}を
\ruby{有}{も}てと
\ruby{御勸}{お|すゝ}め
なさるのも
\ruby{{\換字{道}}理}{もつ|とも}
だけれども、
%
\ruby{何}{なに}
\ruby{妾}{わたし}が
\ruby{彼}{あ}の
\ruby{人}{ひと}を
\ruby{何樣}{ど|う}の
\ruby{斯樣}{か|う}のと
\ruby{思}{おも}つて
\ruby{居}{ゐ}やう。
%
\ruby{妾}{わたし}は
ただ% 原本では非通り字表記
\ruby{彼}{あ}の
\ruby{人}{ひと}を
\ruby{氣}{き}の
\ruby{毒}{どく}なと
\ruby{思}{おも}つて
\原本頁{264-9}\改行%
\ruby{居}{ゐ}る
ばかりで、
%
\ruby{妾}{わたし}は
ただ% 原本では非通り字表記
\ruby{彼}{あ}の
\ruby{人}{ひと}を
\ruby{{\換字{嫌}}}{きら}ひでは
\ruby{無}{な}い
けれども、
%
\ruby{何}{なん}で
\原本頁{264-10}\改行%
\ruby{妾}{わたし}に
\ruby[|g|]{乾淨}{きれい}で
\ruby{無}{な}い
\ruby[||j>]{底}{そこ}
\ruby[||j>]{心}{ごゝろ}が
% \ruby{底心}{そこ|ごゝろ}が
\ruby{有}{あ}らう!。
%
そりやあ
\ruby{妾}{わたし}は
\ruby{彼}{あ}の
\ruby{人}{ひと}を
\ruby{好}{す}いては
\ruby{居}{ゐ}る
けれども、
%
\ruby{好}{す}いて
\ruby{居}{ゐ}る
ばかりで
\ruby{何樣}{ど|う}の
\ruby{斯樣}{か|う}のとは
\ruby[|g|]{眞實}{ほんと}に
\ruby{思}{おも}つては
\ruby{居}{ゐ}ない。
%
\ruby[|g|]{眞個}{ほんと}に
\ruby{妾}{わたし}は
\ruby[|g|]{乾淨}{きれい}でない
\ruby{氣}{き}
なんぞは
\ruby{微塵}{み|ぢん}も
\ruby{有}{も}つては
\ruby{居}{ゐ}ない。
』
