\Entry{其四十八}

\ruby{幾度}{いく|たび}と
\ruby{無}{な}く
\ruby{此繪}{この|ゑ}も
\ruby{見}{み}たりしが、
\ruby{心}{こヽろ}の
\ruby{中}{うち}に
\ruby{物}{もの}のありし
\ruby{時}{とき}は、ただ
\ruby{其}{それ}に
\ruby{屈托}{くつ|たく}して
\ruby{眼}{め}にも
\ruby[g]{自然}{おのづ}と
\ruby{着}{つ}かず、また
\ruby{何事}{なに|ごと}も
\ruby{無}{な}き
\ruby{時}{とき}は
\ruby{氣}{き}にも
\ruby{止}{と}めず
\ruby{其儘}{その|まヽ}に
\ruby{見}{み}て
\ruby{{\GWI{u904e-k}}}{すご}したりし
\ruby{故}{ゆゑ}にや、
\ruby{今}{いま}まで
\ruby{幾年}{いく|とせ}の
\ruby{間}{あひだ}ただの
\ruby[g]{一度}{いちど}も、
\ruby{古}{ふる}き
\ruby[g]{疇昔}{そのむかし}の
\ruby{事}{こと}などを
\ruby{思}{おも}ひ
\ruby{出}{だ}したる
\ruby{折}{をり}も
\ruby{無}{な}かりしに、
\ruby[g]{今{\GWI{u5bb5-k}}}{こよひ}は
\ruby{差當}{さし|あた}りて
\ruby{口惜}{く|や}しいといふ
\ruby{事}{こと}も
\ruby{悲}{かな}しいといふ
\ruby{事}{こと}も
\ruby[g]{{\GWI{u53c8-k}}氣{\GWI{u9063-k}}}{またきづか}はしいといふこともあるにはあらず、まして
\ruby{人}{ひと}には
\ruby{明}{あ}かせぬ
\ruby{羞}{はづ}かしき
\ruby{思}{おも}ひに
\ruby{胸}{むね}の
\ruby{底}{そこ}を
\ruby{掻}{か}き
\ruby{挘}{むし}りたきやうの
\ruby[g]{心地}{こヽち}するといふ
\ruby{事}{こと}なんどの
\ruby{有}{あ}るにもあらねど、さればとて
\ruby[g]{{\GWI{u53c8-k}}全}{またまつた}く
\ruby{雲無}{くも|なき}き
\ruby{空}{そら}のただ
\ruby{美}{うつく}しく
\ruby{靑}{あお}きやうに
\ruby{胸}{むね}の
\ruby{中}{うち}のさつぱりと
\ruby[g]{乾淨}{きれい}なるにもあらず、
\ruby{取}{と}り
\ruby{詰}{つ}めて
\ruby{此}{これ}を
\ruby{思}{おも}ふといふ
\ruby{事}{こと}も
\ruby{無}{な}けれど、
\ruby{何}{なに}も
\ruby{彼}{か}も
\ruby{忘}{わす}れ
\ruby{果}{は}てヽ
\ruby{物覺}{もの|おぼ}えぬ
\ruby{夢路}{ゆめ|ぢ}に
\ruby{入}{い}るといふほどにもなりかぬるより、
\ruby{偶然}{ふ|と}、
\ruby{眼}{め}の
\ruby{前}{まへ}の
\ruby{此}{こ}の
\ruby{鷺}{さぎ}の
\ruby{繪}{ゑ}などの
\ruby{心}{こヽろ}に
\ruby{留}{と}まりて、
\ruby[g]{昨日今日}{きのふけふ}の
\ruby{事}{こと}にもあらぬ
\ruby{古}{ふる}き
\ruby[g]{記憶}{おぼえ}の
\ruby{新}{あらた}に
\ruby{{\GWI{u6d6e-k}}}{うか}び
\ruby{現}{あら}はれ
\ruby{來}{きた}れるにや。
お
\ruby{龍}{りう}は
\ruby{{\GWI{u7336-k}}忘}{なほ|わす}れんとして
\ruby{其}{そ}の
\ruby{鷺}{さぎ}を
\ruby{忘}{わす}れ
\ruby{得}{え}かねたり。

『それにしても
\ruby[g]{書間}{ひるま}の
\ruby{姊}{ねえ}さんの
\ruby{言葉}{こと|ば}は、
\ruby{妾}{わたし}が
\ruby{心}{こヽろ}を
\ruby{引立}{ひき|た}てヽ
\ruby{下}{くだ}さらうとからの
\ruby[g]{戲談{\GWI{u4ea4-k}}}{じやうだんまじ}りの
\ruby{其言}{そ|れ}には
\ruby[g]{相{\GWI{u9055-k}}無}{さうゐな}けれど、
\ruby{餘}{あんま}り
\ruby[g]{{\GWI{u5f3a-g}}{\GWI{u904e-k}}}{きつす}ぎて
\ruby[g]{{\GWI{u5f3a-g}}{\GWI{u904e-k}}}{きつす}ぎて
\ruby[g]{一々妾}{いち〳〵わたし}の
\ruby{耳}{みヽ}には
\ruby{可厭}{い|や}に
\ruby{聞}{きこ}えてならざりしが、
\ruby{若}{も}し
\ruby{彼言}{あ|れ}がまあ
\ruby{姊}{ねえ}さんの
\ruby[g]{眞實}{ほんと}の
\ruby{意}{こヽろ}からのことなら、
\ruby{姊}{ねえ}さんは
\ruby[g]{矢張靜岡}{やつぱりしづをか}の
\ruby{叔母}{を|ば}さんも
\ruby{同}{おな}じことの
\ruby{人}{ひと}!。
そりやあ
\ruby{智惠}{ち|ゑ}も
\ruby{有}{あ}り
\ruby{餘}{あま}るほど
\ruby{有}{あ}り、
\ruby[g]{同{\GWI{u60c5-k}}}{おもひやり}も
\ruby{痒}{かゆ}いところへ
\ruby{手}{て}の
\ruby{屆}{とど}く
\ruby{程有}{ほど|あ}り、
\ruby{氣位}{きぐ|らゐ}も
\ruby{大層}{たい|そう}に
\ruby{{\GWI{u9055-k}}}{ちが}つて、
\ruby{何}{なに}も
\ruby{彼}{か}も
\ruby{{\GWI{u52dd-k}}}{すぐ}れてはお
\ruby{在}{いで}なさるには
\ruby[g]{相{\GWI{u9055-k}}無}{さうゐな}いけれども、
\ruby{種々}{いろ|〳〵}のことが
\ruby{{\GWI{u52dd-k}}}{すぐ}れて
\ruby{御在}{お|いで}なさるるだけに
\ruby{仰}{おつし}ある
\ruby{事}{こと}も
\ruby{輪}{わ}を
\ruby{掛}{か}けて、
\ruby{叔母}{を|ば}はただ
\ruby{堅人}{かた|じん}を
\ruby[g]{丈夫}{をとこ}に
\ruby{有}{も}てといつたところを、
\ruby{姊}{ねえ}さんは
\ruby{世}{よ}を
\ruby{渡}{わた}る
\ruby{伎倆}{う|で}のある
\ruby[g]{毅然}{しつかり}とした
\ruby[g]{立派}{りつぱ}な
\ruby[g]{漢子}{をとこ}を
\ruby{擇}{よ}つて
\ruby{配偶}{つれ|あひ}にしろと
\ruby{御云}{お|い}ひになつただけで、
\ruby{心}{しん}は
\ruby[g]{矢張差{\GWI{u9055-k}}}{やつぱりちがひ}は
\ruby{有}{あ}りは
\ruby{仕無}{し|な}い。
まさかに
\ruby{姊}{ねえ}さんの
\ruby{本心}{し|ん}からとは
\ruby{思}{おも}へぬけれども、
\ruby[g]{全然意}{まる〳〵こヽろ}にも
\ruby{無}{な}いことを
\ruby{御云}{お|い}ひでは
\ruby{無}{な}かつた
\ruby[g]{樣子}{やうす}。
\ruby[g]{一旦斯樣}{いつたんかう}いふ
\ruby{不幸}{ふし|あわせ}な
\ruby{目}{め}を
\ruby{見}{み}て
\ruby{來}{き}た
\ruby{妾}{わたし}に、また
\ruby{男}{をとこ}を
\ruby{有}{も}てと
\ruby{仰}{おつし}あつて、
\ruby[g]{眞實}{ほんと}に
\ruby{然樣}{さ|う}いふことを
\ruby{妾}{わたし}が
\ruby{唯々}{は|い}と
\ruby{云}{い}ひさうなやうに
\ruby{思}{おも}つておいでヾも
\ruby{有}{あ}らうか
\ruby{知}{し}らん。
あれほど
\ruby{能}{よ}く
\ruby{何}{なに}も
\ruby{彼}{か}も
\ruby[g]{御解}{おわか}りの
\ruby{姊}{ねえ}さんで、あれほど
\ruby{妾}{わたし}を
\ruby{可愛}{か|はい}がつて
\ruby{下}{くだ}さる
\ruby{彼}{あ}の
\ruby{姊}{ねえ}さんで、そして
\ruby{現今}{い|ま}ぢやあ
\ruby{此}{こ}の
\ruby{廣}{ひろ}い
\ruby{世界}{せ|かい}の
\ruby{中}{なか}で
\ruby{妾}{わたし}に
\ruby{取}{と}つちやあ
\ruby{叔母}{を|ば}よりも
\ruby{誰}{たれ}よりも
\ruby[g]{一番馴染}{いちばんなじみ}の
\ruby{深}{ふか}い
\ruby{彼}{あ}の
\ruby{姊}{ねえ}さんが、よもや
\ruby{妾}{わたし}を
\ruby{其樣}{そ|ん}なことを
\ruby{爲}{し}さうなものとは
\ruby{思}{おも}つて
\ruby{御在}{お|いで}ぢやあ
\ruby{有}{あ}るまいと
\ruby{思}{おも}つては
\ruby{居}{ゐ}るけれど……。
\ruby[g]{成程二度三度丈夫}{なるほどにどさんどをとこ}を
\ruby{有}{も}つ
\ruby{人}{ひと}も
\ruby{稀}{めづ}らしくは
\ruby{無}{な}いから、
\ruby{叔母}{を|ば}の
\ruby{云}{い}ふのも
\ruby[g]{世間普通}{せけんありふれ}では
\ruby{有}{あ}らうし、
\ruby{不思議}{ふ|し|ぎ}は
\ruby{無}{な}からうけれども、そりやあ
\ruby{他}{よそ}の
\ruby{人}{ひと}の
\ruby{話}{はなし}で、
\ruby{妾}{わたし}は
\ruby{妾}{わたし}の
\ruby[g]{性分}{しやうぶん}。
\ruby{妾}{わたし}の
\ruby[g]{性分}{しやうぶん}を
\ruby{知}{し}りきつて
\ruby{御在}{お|いで}のあの
\ruby{姊}{ねえ}さんが、
\ruby{妾}{わたし}も
\ruby[g]{矢張他}{やつぱりよそ}の
\ruby{人}{ひと}と
\ruby{同}{おな}じやうに、
\ruby{時}{とき}が
\ruby{經}{た}ちさへすりやあ
\ruby[g]{{\GWI{u53c8-k}}新規}{またしんき}に
\ruby{男}{をとこ}を
\ruby{有}{も}つものと
\ruby{思}{おも}つて
\ruby{御在}{お|いで}ぢやあ
\ruby{有}{あ}るまい。
そんな
\ruby{氣}{き}になれるやうな
\ruby[g]{薄{\GWI{u60c5-k}}}{はくじやう}な
\ruby{妾}{わたし}ならば、
\ruby{人}{ひと}に
\ruby{棄}{す}てられたからと
\ruby{云}{い}つて、
\ruby{彼樣}{あ|あ}は
\ruby{口惜}{く|やし}がらない。
\ruby{姊}{ねえ}さんは
\ruby{妾}{わたし}が
\ruby{何樣}{ど|ん}な
\ruby{女}{をんな}だといふ
\ruby{事}{こと}は
\ruby{知}{し}りきつてお
\ruby{在}{いで}に
\ruby{{\GWI{u9055-k}}}{ちが}ひ
\ruby{無}{な}い。
けれども
\ruby{{\GWI{u904e-k}}日}{こな|ひだ}からの
\ruby[g]{御談}{おはなし}といひ、
\ruby{今日}{け|ふ}の
\ruby{御言葉}{お|こと|ば}といひ、
\ruby{何}{なん}だか
\ruby{妾}{わたし}には
\ruby{可厭}{い|や}に
\ruby{聞}{きこ}えてならぬ。
\ruby{若}{も}しや
\ruby{妾}{わたし}を
\ruby[g]{矢張眞實}{やつぱりほんと}に
\ruby{今後}{これ|から}また
\ruby{男}{をとこ}でも
\ruby{持}{も}ちさうなものに
\ruby{思}{おも}つて
\ruby{御在}{お|いで}のか
\ruby{知}{し}らん。
まさか
\ruby{其樣}{そ|ん}な
\ruby{事}{こと}は
\ruby{有}{あ}るまいが。
いや〳〵
\ruby[g]{水野}{みづの}といふ
\ruby{人}{ひと}の
\ruby{事}{こと}を
\ruby{幾度}{いく|ど}も
\ruby{御云}{お|い}ひで、
\ruby{然}{さ}も
\ruby{妾}{わたし}が
\ruby{其}{そ}の
\ruby{人}{ひと}を
\ruby{何樣}{ど|う}かでも
\ruby{思}{おも}つて
\ruby{居}{ゐ}るやうに
\ruby{御取}{お|と}りのやうに
\ruby{問}{きこ}えた。
あ、
\ruby{若}{も}し
\ruby[g]{左樣御取}{さうおと}りのやうなら、
\ruby{其}{そ}れあ
\ruby{働}{はたら}きのある
\ruby{男}{をとこ}を
\ruby{有}{も}てと
\ruby[g]{御勸}{おすヽ}めなさるのも
\ruby{道理}{もつ|とも}だけれども、
\ruby{何妾}{なに|わたし}が
\ruby{彼}{あ}の
\ruby{人}{ひと}を
\ruby{何樣}{ど|う}の
\ruby{斯樣}{か|う}のと
\ruby{思}{おも}つて
\ruby{居}{ゐ}やう。
\ruby{妾}{わたし}はただ
\ruby{彼}{あ}の
\ruby{人}{ひと}を
\ruby{氣}{き}の
\ruby{毒}{どく}なと
\ruby{思}{おも}つて
\ruby{居}{ゐ}るばかりで、
\ruby{妾}{わたし}はただ
\ruby{彼}{あ}の
\ruby{人}{ひと}を
\ruby{{\GWI{u5acc-k}}}{きら}ひでは
\ruby{無}{な}いけれども、
\ruby{何}{なん}で
\ruby{妾}{わたし}に
\ruby[g]{乾淨}{きれい}で
\ruby{無}{な}い
\ruby[g]{底心}{そこごヽろ}が
\ruby{有}{あ}らう!。
そりやあ
\ruby{妾}{わたし}は
\ruby{彼}{あ}の
\ruby{人}{ひと}を
\ruby{好}{す}いては
\ruby{居}{ゐ}るけれども、
\ruby{好}{す}いて
\ruby{居}{ゐ}るばかりで
\ruby{何樣}{ど|う}の
\ruby{斯樣}{か|う}のとは
\ruby[g]{眞實}{ほんと}に
\ruby{思}{おも}つては
\ruby{居}{ゐ}ない。
\ruby[g]{眞個}{ほんと}に
\ruby{妾}{わたし}は
\ruby[g]{乾淨}{きれい}でない
\ruby{氣}{き}なんぞは
\ruby[g]{微塵}{みぢん}も
\ruby{有}{も}つては
\ruby{居}{ゐ}ない。

