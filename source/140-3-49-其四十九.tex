\Entry{其四十九}

『
\ruby{妾}{わたし}は
\ruby{自分}{じ|ぶん}からは
\ruby{其樣}{そ|ん}な
\ruby{女}{をんな}では
\ruby{無}{な}いと
\ruby{思}{おも}つても
\ruby{居}{を}れ、
\ruby{人}{ひと}
には
\ruby[g]{矢張}{やつぱ}り
\ruby{其樣}{その|やう}な
\ruby{女}{もの}にも
\ruby{見}{み}えやう。
\ruby{成程其}{なる|ほど|それ}も
\ruby[g]{仕方}{しかた}の
\ruby{無}{な}い
\ruby{事}{こと}ゆゑ、
\ruby{世間}{せ|けん}の
\ruby{人}{ひと}の
\ruby{誰彼}{だれ|かれ}が
\ruby{妾}{わたし}の
\ruby{心}{こヽろ}を
\ruby{知}{し}つて
\ruby{吳}{く}れない
\ruby{其}{それ}を
\ruby{口惜}{く|や}しいとも
\ruby[g]{{\換字{情}}無}{なさけな}いとも
\ruby{思}{おも}ふでは
\ruby{無}{な}く、また
\ruby{叔母}{を|ば}は
\ruby{彼}{あ}の
\ruby{通}{とほ}りの
\ruby{木}{き}で
\ruby{{\GWI{u9020-k}}}{つく}つたやうの
\ruby{人}{ひと}の
\ruby{事}{こと}なれば、はじめから
\ruby{妾}{わたし}の
\ruby{心}{こヽろ}の
\ruby{分}{わか}らぬも
\ruby{少}{すこ}しも
\ruby{無理}{む|り}とは
\ruby{思}{おも}はず、
\ruby{解}{わか}つて
\ruby{吳}{く}れなければとて
\ruby[g]{{\換字{情}}無}{なさけな}いとも
\ruby{思}{おも}はぬけれど、
\ruby{姊}{ねえ}さんだけは
\ruby{妾}{わたし}が
\ruby{何樣}{ど|ん}な
\ruby{女}{ひと}だといふことを
\ruby{知}{し}り
\ruby{拔}{ぬ}いて
\ruby{居}{ゐ}て
\ruby{下}{くだ}さるとばかり
\ruby{思}{おも}つて
\ruby{居}{ゐ}たに、
\ruby{矢張}{やつ|ぱり}
\ruby{姊}{ねえ}さんも
\ruby{妾}{わたし}を
\ruby{知}{し}つて
\ruby{下}{くだ}さらないかと
\ruby{思}{おも}ふと、もう
\ruby{此}{こ}の
\ruby{廣}{ひろ}い
\ruby{世}{よ}の
\ruby{中}{なか}に
\ruby{眞實}{ほん|たう}の
\ruby{妾}{わたし}の
\ruby{心持}{こヽろ|もち}を
\ruby{知}{し}つて
\ruby{吳}{く}れる
\ruby{人}{ひと}は
\ruby[g]{一人}{ひとり}も
\ruby{無}{な}いことかとつく〴〵
\ruby[g]{{\換字{情}}無}{なさけな}くなる。
もつとも
\ruby{憎}{にく}い
\ruby{彼}{あ}の
\ruby{男}{をとこ}に
\ruby{欺}{だま}されたそも〳〵の
\ruby{始}{はじめ}から
\ruby[g]{{\GWI{u7d42-ue0101}}局}{しまひ}までの
\ruby{間}{あひだ}は、
\ruby[g]{始{\GWI{u7d42-ue0101}}}{しじう}
\ruby{姊}{ねえ}さんに
\ruby{{\GWI{u9060-k}}}{とほ}ざかつて
\ruby{居}{ゐ}て、
\ruby{何事}{なに|ごと}も
\ruby{姊}{ねえ}さんに
\ruby{隱}{かく}して
\ruby{居}{ゐ}た
\ruby{其}{それ}は
\ruby{惡}{わる}かつたなれど、
\ruby{後}{あと}では
\ruby{羞}{はづ}かしい
\ruby{蹊蹟}{いき|さつ}の
\ruby{何}{なに}も
\ruby{彼}{か}も
\ruby{話}{はな}して
\ruby{仕舞}{し|ま}つてある
\ruby{故}{ゆゑ}、
\ruby{{\換字{猶}}}{なほ}のこと
\ruby{妾}{わたし}の
\ruby{氣心}{きご|ヽろ}も
\ruby{御}{お}わかりの
\ruby{筈}{はず}なるに、
\ruby[g]{水野}{みづの}さんの
\ruby{事}{こと}について
\ruby{何樣}{ど|う}の
\ruby{斯樣}{か|う}のつて
\ruby{二度}{に|ど}も
\ruby{三度}{さん|ど}も
\ruby{御云}{お|い}ひなすつたばかりか
\ruby{働}{はたら}きのある
\ruby{男}{をとこ}を
\ruby{見}{み}せやうかの
\ruby{何}{なん}のと、
\ruby{戯談}{じやう|だん}には
\ruby{{\GWI{u9055-k}}}{ちが}ひないけれども
\ruby{可厭}{い|や}な
\ruby{事}{こと}を
\ruby{仰}{おつし}あつたのは、
\ruby{矢張}{やつ|ぱり}
\ruby{妾}{わたし}の
\ruby{眞實}{ほん|たう}の〳〵の
\ruby{心持}{こヽろ|もち}が
\ruby{御解}{お|わか}りが
\ruby{無}{な}いからかと
\ruby{思}{おも}はれる。
\ruby{年端}{と|し}のゆかない
\ruby{故}{せい}でつい
\ruby{欺}{だま}されたにしろ
\ruby{何}{なに}にしろ、
\ruby{女}{をんな}の
\ruby{廢}{すた}つて
\ruby{仕舞}{し|ま}つた
\ruby{斯樣}{こ|ん}な
\ruby{身}{み}の
\ruby{上}{うへ}でもつて、たとひ
\ruby{妾}{わたし}が
\ruby{彼}{あ}の
\ruby{人}{ひと}に
\ruby{{\GWI{u8ff7-k}}}{まよ}つたからにしてが、
\ruby{何樣}{ど|う}まあ
\ruby{正直}{しやう|ぢき}で
\ruby[g]{{\換字{清}}潔}{きれい}で
\ruby[g]{純粹}{いつぽんぎ}な
\ruby{實意}{じつ|い}の
\ruby{深}{ふか}い
\ruby[g]{水野}{みづの}さんのやうな
\ruby{彼樣}{あ|ん}な
\ruby{人}{ひと}を、
\ruby[g]{加之}{おまけ}に
\ruby{横合}{よこ|あひ}から
\ruby{何樣}{ど|う}することが
\ruby{出來}{で|き}やう。
そんな
\ruby{汚}{きたな}い
\ruby{心持}{こヽろ|もち}をもつて、のめ〳〵とした
\ruby{事}{こと}を
\ruby{仕}{し}やうと
\ruby{爲}{し}もする
\ruby{女}{をんな}の
\ruby{樣}{やう}に
\ruby{妾}{わたし}が
\ruby{見}{み}えやうかと
\ruby{思}{おも}ふと、
\ruby{餘}{あんま}り
\ruby[g]{{\換字{情}}無}{なさけな}くて
\ruby{味氣無}{あぢ|き|な}くなつて
\ruby{仕舞}{し|ま}ふ。
\GWI{koseki-900370}かし
\ruby{姊}{ねえ}さんにさへ
\ruby{妾}{わたし}の
\ruby{心持}{こヽろ|もち}がほんとには
\ruby{分}{わか}らぬのなら、
\ruby{然樣}{さ|う}いふ
\ruby[g]{不正直}{ふしやうぢき}のが
\ruby{一體}{いつ|たい}の
\ruby{世間}{せ|けん}の
\ruby{女}{ひと}の
\ruby{常}{つね}なので、
\ruby{妾}{わたし}のやうなのは、よく〳〵の
\ruby{馬鹿}{ば|か}なのだらう。
つい
\ruby{氣}{き}の
\ruby{毒}{どく}と
\ruby{思}{おも}ふ
\ruby{心}{こヽろ}が
\ruby{募}{つの}つていろ〳〵と
\ruby[g]{水野}{みづの}さんの
\ruby{爲}{ため}に
\ruby{頼}{たの}みごとなんぞを
\ruby{仕}{し}たので、
\ruby{姊}{ねえ}さんにまで
\ruby{可厭}{い|や}な
\ruby{事}{こと}を
\ruby{云}{い}はれる。
あヽ、これも
\ruby{妾}{わたし}が
\ruby[g]{愚鈍{\GWI{u904e-k}}}{たらなす}ぎるからの
\ruby{事}{こと}で、もう〳〵いつそ
\ruby{可厭}{い|や}になつて
\ruby{仕舞}{し|ま}ふ。
\ruby{姊}{ねえ}さんに
\ruby{頼}{たの}んだ
\ruby{事}{こと}さへ
\ruby[g]{首尾能}{しゆびよ}く
\ruby{出來}{で|き}たなら、
\ruby[g]{水野}{みづの}さんの
\ruby{水}{みづ}の
\ruby{字}{じ}ももう
\ruby{云}{い}ひ
\ruby{出}{だ}さないで、
\ruby{當分}{たう|ぶん}は
\ruby{尋}{たづ}ねもすまい、
\ruby{會}{あ}ひも
\ruby{仕}{し}ますまい。
\ruby{何}{なん}でも
\ruby{些少}{わづ|か}の
\ruby[g]{日數}{ひかず}の
\ruby{中}{うち}に、
\ruby{姊}{ねえ}さんが
\ruby[g]{水野}{みづの}さんの
\ruby{事}{こと}を
\ruby{御云}{お|い}ひなさるやうの
\ruby[g]{調子}{てうし}が、
\ruby{急}{きふ}に
\ruby{異}{ちが}つて
\ruby{來}{き}たやうに
\ruby{思}{おも}はれる。
\GWI{koseki-900370}かし、これも
\ruby{妾}{わたし}の
\ruby[g]{僻見}{ひがみ}か
\ruby{知}{し}れぬけれど、
\ruby{何樣}{ど|う}も
\ruby{何}{なに}かの
\ruby{譯}{わけ}かあつて、
\ruby{妾}{わたし}が
\ruby[g]{水野}{みづの}さんに
\ruby{{\GWI{u8fd1-k}}}{ちか}よるのを
\ruby[g]{御{\換字{嫌}}}{おきら}ひなさり
\ruby{出}{だ}したやうにも
\ruby{思}{おも}はれる!。
\ruby{此上}{この|うへ}も
\ruby{無}{な}い
\ruby{有}{あ}り
\ruby{難}{がた}い
\ruby{姊}{ねえ}さんの
\ruby{{\換字{所}}思}{おも|はく}が
\ruby{然樣}{さ|う}なら、
\ruby{其}{それ}ても
\ruby{無理}{む|り}に
\ruby{彼}{あ}の
\ruby{人}{ひと}を
\ruby{何樣}{ど|う}の
\ruby{斯樣}{か|う}のと
\ruby{思}{おも}つて
\ruby{居}{ゐ}る
\ruby[g]{仔細}{しさい}のあるのでは
\ruby{無}{な}いし、
\ruby{妾}{わたし}が
\ruby{彼}{あ}の
\ruby{人}{ひと}に
\ruby{{\GWI{u9060-k}}}{とほ}ざかるのに
\ruby{別}{べつ}に
\ruby{苦}{く}も
\ruby{無}{な}い
\ruby{譯}{わけ}、
\ruby{妾}{わたし}は
\ruby{何處}{ど|こ}までも
\ruby{姊}{ねえ}さんの
\ruby[g]{指揮}{さしず}を
\ruby{受}{う}けて、
\ruby{何}{なに}を
\ruby{修業}{しゆ|げふ}するにしろ、
\ruby{何}{なん}でも
\ruby{宜}{よ}い
\ruby[g]{一人立}{ひとりだち}の
\ruby{出來}{で|き}る
\ruby{身}{み}になつて、ちやんと
\ruby[g]{一人}{ひとり}で
\ruby{{\GWI{u904e-k}}}{すご}せるやうになつてから、それから
\ruby{自分}{じ|ぶん}の
\ruby[g]{勝手}{かつて}に
\ruby[g]{水野}{みづの}さんの
\ruby{世話}{せ|わ}でも
\ruby{誰}{だれ}の
\ruby{世話}{せ|わ}でも、
\ruby{自分}{じ|ぶん}が
\ruby{親切}{しん|せつ}にして
\ruby{{\GWI{u9063-k}}}{や}りたいと
\ruby{思}{おも}ふ
\ruby{人}{ひと}には
\ruby{親切}{しん|せつ}にして
\ruby{{\GWI{u9063-k}}}{や}りませう。
\ruby{彼}{あ}の
\ruby{優}{やさ}しい
\ruby{智惠}{ち|ゑ}の
\ruby{深}{ふか}い
\ruby{氣}{き}の
\ruby{大}{おほ}きい
\ruby{姊}{ねえ}さんでさへ
\ruby{妾}{わたし}の
\ruby{外}{ほか}には
\ruby[g]{眞實}{ほんと}に
\ruby{味方}{み|かた}は
\ruby{無}{な}い!。
\ruby[g]{然樣思}{さうおも}つては
\ruby{濟}{す}まない
\ruby{事}{こと}ながら、
\ruby{此}{こ}の
\ruby{繪}{ゑ}の
\ruby{中}{なか}の
\ruby{鷺}{さぎ}が
\ruby{物}{もの}を
\ruby{云}{い}つたなら、
\ruby[g]{屹度}{きつと}
\ruby{姊}{ねえ}さんの
\ruby[g]{徃時}{むかし}も
\ruby{分}{わか}らうけれど、
\ruby{姊}{ねえ}さんもやつぱり
\ruby{辛}{つら}いか
\ruby{悲}{かな}しいかの
\ruby{瀬}{せ}を
\ruby{越}{こ}して、そして
\ruby{今}{いま}のやうに
\ruby[g]{一人立同樣}{ひとりだちどうやう}な
\ruby{身}{み}におなりに
\ruby[g]{相{\GWI{u9055-k}}無}{さうゐな}い。
そして
\ruby{此}{こ}の
\ruby{鷺}{さぎ}は
\ruby{其}{そ}の
\ruby[g]{因緣}{いはれ}の
\ruby[g]{紀念}{かたみ}でもあらう。
\ruby{鷺}{さぎ}も
\ruby{物}{もの}を
\ruby{云}{い}はず、
\ruby{姊}{ねえ}さんも
\ruby[g]{御話}{おはな}しぢやあ
\ruby{無}{な}いけれど、
\ruby{自分}{じ|ぶん}に
\ruby{比}{くら}べて
\ruby{姊}{ねえ}さんの
\ruby[g]{徃時}{むかし}をおもふとあヽ
\ruby{何}{なん}と
\ruby{無}{な}く
\ruby{朦朧}{ぼん|やり}と
\ruby{解}{わか}るやうな
\ruby{氣}{き}がする!。
』

お
\ruby{龍}{りう}は
\ruby{眼}{め}を
\ruby{開}{ひら}いてまた
\ruby{彼}{か}の
\ruby{繪}{ゑ}を
\ruby{見}{み}れば、
\ruby{鷺}{さぎ}はただ
\ruby{心}{こヽろ}も
\ruby{無}{な}く
\ruby{水}{みづ}に
\ruby{立}{た}ち
\ruby{盡}{つく}して、
\ruby[g]{爾我}{なんぢわ}が
\ruby{心}{こヽろ}を
\ruby{知}{し}れりや、
\ruby{我}{われ}は
\ruby{謎}{なぞ}なり、と
\ruby{云}{い}はぬばかりに
\ruby{默々}{もく|〳〵}たり
\ruby{寂々}{じやく|〳〵}たり。

\vspace{10zw}
\Large{天うつ浪 {\normalsize 第三{\GWI{u7d42-ue0101}}}}

