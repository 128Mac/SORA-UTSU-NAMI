\Entry{其四十九}

% メモ 校正終了 2024-05-19 2024-06-14
\原本頁{265-4}%
『
\ruby{妾}{わたし}は
\ruby[g]{自{\換字{分}}}{じ ぶん}からは
\ruby[g]{其樣}{そ ん }な
\ruby{女}{をんな}では
\ruby{無}{な}いと
\ruby{思}{おも}つても
\ruby{居}{を}れ、
%
\ruby{人}{ひと}には
\ruby[g]{矢張}{やつぱ }り% ルビ調整(原本通り)非グループルビ
\ruby{其}{その}
\ruby{樣}{やう}な
\ruby{女}{もの}にも
\ruby{見}{み}えやう。
%
\ruby[g]{成程}{なるほど}
\ruby{其}{それ}も
\ruby[g]{仕方}{し かた}の
\ruby{無}{な}い
\ruby{事}{こと}ゆゑ、
%
\ruby[g]{世間}{せ けん}の
\ruby{人}{ひと}の
\ruby{誰}{たれ}
\ruby{彼}{かれ}が
\ruby{妾}{わたし}の
\ruby{心}{こゝろ}を
\ruby{知}{し}つて
\ruby{吳}{く}れない
\ruby{其}{それ}を
\ruby[g]{口惜}{く や }しい
とも
\ruby[g]{{\換字{情}}無}{なさけな}い
とも
\ruby{思}{おも}ふでは
\ruby{無}{な}く、
%
また
\ruby[g]{叔母}{を ば }は
\ruby{彼}{あ}の
\ruby{{\換字{通}}}{とほ}りの
\ruby{木}{き}で
\ruby{{\換字{造}}}{つく}つた
やうの
\ruby{人}{ひと}の
\ruby{事}{こと}
なれば、
%
はじめから
\ruby{妾}{わたし}の
\ruby{心}{こゝろ}の
\ruby{{\換字{分}}}{わか}らぬも
\ruby{少}{すこ}しも
\ruby[g]{無理}{む り }とは
\ruby{思}{おも}はず、
%
\ruby{解}{わか}つて
\ruby{吳}{く}れなければ
とて
\ruby[g]{{\換字{情}}無}{なさけな}い
とも
\ruby{思}{おも}はぬ
けれど、
%
\ruby{姊}{ねえ}さん
\原本頁{265-10}\改行%
だけは
\ruby{妾}{わたし}が
\ruby[g]{何樣}{ど ん }な
\ruby{女}{ひと}だ
といふ
ことを
\ruby{知}{し}り
\ruby{拔}{ぬ}いて
\ruby{居}{ゐ}て
\ruby{下}{くだ}さると
ばかり
\ruby{思}{おも}つて
\ruby{居}{ゐ}たに、
%
\ruby[g]{矢張}{やつぱり}
\ruby{姊}{ねえ}さんも
\ruby{妾}{わたし}を
\ruby{知}{し}つて
\ruby{下}{くだ}さらないかと
\ruby{思}{おも}ふと、
%
もう
\ruby{此}{こ}の
\ruby{廣}{ひろ}い
\ruby{世}{よ}の
\ruby{中}{なか}に
\ruby[g]{眞實}{ほんたう}の
\ruby{妾}{わたし}の
\ruby[||j>]{心}{こゝろ}
\ruby[||j>]{持}{ もち}を
% \ruby{心持}{こゝろ|もち}を
\ruby{知}{し}つて
\ruby{吳}{く}れる
\ruby{人}{ひと}は
\ruby[|g|]{一人}{ひとり}も
\ruby{無}{な}い
こと
かと
つく〴〵
\ruby[g]{{\換字{情}}無}{なさけな}く
なる。
%
もつとも
\ruby{憎}{にく}い
\ruby{彼}{あ}の
\ruby{男}{をとこ}に
\ruby{欺}{だま}された
そも〳〵の
\ruby{始}{はじめ}から
\ruby[|g|]{{\換字{終}}局}{しまひ}までの
\ruby{間}{あひだ}は、
%
\ruby[g]{始{\換字{終}}}{し じう}% ルビ調整(原本通り)「ゆ」無し
\ruby{姊}{ねえ}さんに
\原本頁{266-5}\改行%
\ruby{{\換字{遠}}}{とほ}ざかつて
\ruby{居}{ゐ}て、
%
\ruby[g]{何事}{なにごと}も
\ruby{姊}{ねえ}さんに
\ruby{隱}{かく}して
\ruby{居}{ゐ}た
\ruby{其}{それ}は
\ruby{惡}{わる}かつた
なれど、
%
\ruby{後}{あと}では
\ruby{羞}{はづ}かしい
\ruby[g]{蹊蹟}{いきさつ}の
\ruby{何}{なに}も
\ruby{彼}{か}も
\ruby{話}{はな}して
\ruby[g]{仕舞}{し ま }つてある
\ruby{故}{ゆゑ}、
%
\ruby{{\換字{猶}}}{なほ}のこと
\ruby{妾}{わたし}の
\ruby[g]{氣心}{きごゝろ}も
\ruby{御}{お}わかりの
\ruby{筈}{はず}なるに、
%
\ruby[g]{水野}{みづの }さんの
\ruby{事}{こと}について
\原本頁{266-8}\改行%
\ruby[g]{何樣}{ど う }の
\ruby[g]{斯樣}{か う }のつて
\ruby[g]{二度}{に ど }も
\ruby[g]{三度}{さんど }も
\ruby[g]{御云}{お い }ひ
なすつた
ばかりか
\ruby{働}{はたら}き
のある
\ruby{男}{をとこ}を
\ruby{見}{み}せやうかの
\ruby{何}{なん}のと、
%
\ruby[||j>]{戲}{じやう}
\ruby[||j>]{談}{ だん}には
% \ruby{戲談}{じやう|だん}には
\ruby{{\換字{違}}}{ちが}ひない
けれども
\ruby[g]{可厭}{い や }な
\ruby{事}{こと}を
\ruby{仰}{おつし}あつたのは、
%
\ruby[g]{矢張}{やつぱり}
\ruby{妾}{わたし}の
\ruby[g]{眞實}{ほんたう}の〳〵
\ruby[||j>]{心}{こゝろ}
\ruby[||j>]{持}{ もち}が
% \ruby{心持}{こゝろ|もち}が
\ruby[g]{御解}{お わか}りが
\ruby{無}{な}いからかと
\ruby{思}{おも}はれる。
%
\ruby[g]{年端}{と し }の
ゆかない
\ruby{故}{せゐ}で% せ(ゐ)
つい
\ruby{欺}{だま}されたにしろ
\ruby{何}{なに}にしろ、
%
\ruby{女}{をんな}の
\ruby{廢}{すた}つて
\ruby[g]{仕舞}{し ま }つた
\ruby[g]{斯樣}{こ ん }な
\ruby{身}{み}の
\ruby{上}{うへ}で
もつて、
%
たとひ
\ruby[<j||]{妾}{わたし}% 行末行頭の境界付近なので特例処置を施す
\原本頁{267-2}\改行%
が
\ruby{彼}{あ}の
\ruby{人}{ひと}に
\ruby{{\換字{迷}}}{まよ}つた
からに
してが、
%
\ruby[g]{何樣}{ど う }
まあ
\ruby[||j>]{正}{しやう}
\ruby[||j>]{直}{ ぢき}で
% \ruby{正直}{しやう|ぢき}で
\ruby[|g|]{淸潔}{きれい}で
\ruby[||j>]{純}{いつ}
\ruby[||j>]{粹}{ぽんぎ}な
% \ruby{純粹}{いつ|ぽんぎ}な
\改行% 校正作業の簡略化のため
、
%
\原本頁{267-3}\改行%
\ruby[g]{實意}{じつい }の
\ruby{深}{ふか}い
\ruby[g]{水野}{みづの }さん
のやうな
\ruby[g]{彼樣}{あ ん }な
\ruby{人}{ひと}を、
%
\ruby[|g|]{加之}{おまけ}に
\ruby[g]{横合}{よこあひ}から
\ruby[g]{何樣}{ど う }することが
\ruby[g]{出來}{で き }やう。
%
そんな
\ruby{汚}{きたな}い
\ruby[||j>]{心}{こゝろ}
\ruby[||j>]{持}{ もち}
% \ruby{心持}{こゝろ|もち}
を
もつて、
%
のめ〳〵とし
\原本頁{267-5}\改行%
た
\ruby{事}{こと}を
\ruby{仕}{し}やうと
\ruby{爲}{し}もする
\ruby{女}{をんな}の
\ruby{樣}{やう}に
\ruby{妾}{わたし}が
\ruby{見}{み}え
やうかと
\ruby{思}{おも}ふと、
%
\ruby[<j||]{餘}{あんま}り% 行末行頭の境界付近なので特例処置を施す
\ruby[g]{{\換字{情}}無}{なさけな}くて
\ruby{味氣無}{あぢ|き|な}くなつて
\ruby[g]{仕舞}{し ま }ふ。
%
\換字{志}かし
\ruby{姊}{ねえ}さんに
さへ
\ruby{妾}{わたし}の
\ruby[<j||]{心}{こゝろ}% 行末行頭の境界付近なので特例処置を施す
\ruby[||j>]{持}{もち}
% \ruby{心持}{こゝろ|もち}
が
ほんとには
\ruby{{\換字{分}}}{わか}らぬのなら、
%
\ruby[g]{然樣}{さ う }いふ
\ruby{不正直}{ふ|しやう|ぢき}のが
\ruby[g]{一體}{いつたい}の
\ruby[g]{世間}{せ けん}
\原本頁{267-8}\改行%
の
\ruby{女}{ひと}の
\ruby{常}{つね}なので、
%
\ruby{妾}{わたし}の
やうなのは、
%
よく〳〵の
\ruby[g]{馬鹿}{ば か }なのだらう
\原本頁{}\改行%
。
%
\原本頁{267-9}\改行%
つい
\ruby{氣}{き}の
\ruby{毒}{どく}と
\ruby{思}{おも}ふ
\ruby{心}{こゝろ}が
\ruby{募}{つの}つて
いろ〳〵と
\ruby[g]{水野}{みづの }さんの
\ruby{爲}{ため}に
\ruby{頼}{たの}み
ご
\原本頁{267-10}\改行%
となんぞを
\ruby{仕}{し}たので、
%
\ruby{姊}{ねえ}さんに
まで
\ruby[g]{可厭}{い や }な
\ruby{事}{こと}を
\ruby{云}{い}はれる。
%
あゝ
%
\原本頁{267-11}\改行%
これも
\ruby{妾}{わたし}が
\ruby[|g|]{愚鈍}{たらな}
\ruby{{\換字{過}}}{す}ぎる
からの
\ruby{事}{こと}で、
%
もう〳〵
いつそ
\ruby[g]{可厭}{い や }になつて
\ruby[g]{仕舞}{し ま }ふ。
%
\ruby{姊}{ねえ}さんに
\ruby{頼}{たの}んだ
\ruby{事}{こと}さへ
\ruby[g]{首尾}{しゆび }
\ruby{能}{よ}く
\ruby[g]{出來}{で き }たなら、
%
\ruby[g]{水野}{みづの }さんの
\ruby{水}{みづ}の
\ruby{字}{じ}も% 原本では非通り字表記
もう
\ruby{云}{い}ひ
\ruby{出}{だ}さないで、
%
\ruby[g]{當{\換字{分}}}{たうぶん}は
\ruby{{\換字{尋}}}{たづ}ねもすまい、
%
\ruby{會}{あ}ひも
\ruby{仕}{し}ますまい。
%
\ruby{何}{なん}でも
\ruby[|g|]{些少}{わづか}の
\ruby[g]{日數}{ひ かず}の
\ruby{中}{うち}に、
%
\ruby{姊}{ねえ}さんが
\ruby[g]{水野}{みづの }さんの
\原本頁{268-4}\改行%
\ruby{事}{こと}を
\ruby[g]{御云}{お い }ひ
なさる
やうの
\ruby[g]{調子}{てうし }が、
%
\ruby{急}{きふ}に
\ruby{異}{ちが}つて
\ruby{來}{き}たやうに
\ruby{思}{おも}はれる。
%
\換字{志}かし、
%
これも
\ruby{妾}{わたし}の
\ruby[g]{僻見}{ひがみ }か
\ruby{知}{し}れぬ
けれど、
%
\ruby[g]{何樣}{ど う }も
\ruby{何}{なに}かの
\ruby{譯}{わけ}
\原本頁{268-6}\改行%
が
あつて、
%
\ruby{妾}{わたし}が
\ruby[g]{水野}{みづの }さんに
\ruby{{\換字{近}}}{ちか}よるのを
\ruby[g]{御{\換字{嫌}}}{お きら}ひ
なさり
\ruby{出}{だ}した
やう
\原本頁{268-7}\改行%
にも
\ruby{思}{おも}はれる!。
%
\ruby{此}{この}
\ruby{上}{うへ}も
\ruby{無}{な}い
\ruby{有}{あ}り
\ruby{{\換字{難}}}{がた}い
\ruby{姊}{ねえ}さんの
\ruby[g]{{\換字{所}}思}{おもはく}が
\ruby[g]{然樣}{さ う }なら
\改行% 校正作業の簡略化のため
、
%
\原本頁{268-8}\改行%
\ruby{其}{それ}でも
\ruby[g]{無理}{む り }に
\ruby{彼}{あ}の
\ruby{人}{ひと}を
\ruby[g]{何樣}{ど う }の
\ruby[g]{斯樣}{か う }のと
\ruby{思}{おも}つて
\ruby{居}{ゐ}る
\ruby[g]{仔細}{し さい}の
あるのでは
\ruby{無}{な}いし、
%
\ruby{妾}{わたし}が
\ruby{彼}{あ}の
\ruby{人}{ひと}に
\ruby{{\換字{遠}}}{とほ}ざかるのに
\ruby{別}{べつ}に
\ruby{苦}{く}も
\ruby{無}{な}い
\ruby{譯}{わけ}、
%
\ruby{妾}{わたし}は
\原本頁{268-11}\改行%
\ruby[g]{何處}{ど こ }までも
\ruby{姊}{ねえ}さんの
\ruby[g]{指揮}{さしづ }を
\ruby{受}{う}けて、
%
\ruby{何}{なに}を
\ruby[g]{修業}{しゆげふ}
するにしろ、
%
\ruby{何}{なん}でも
\ruby{宜}{よ}い
\ruby[|g|]{一人}{ひとり}
\ruby{立}{だち}の
\ruby[g]{出來}{で き }る
\ruby{身}{み}に
なつて、
%
ちやんと
\ruby[|g|]{一人}{ひとり}で
\ruby{{\換字{過}}}{すご}せる
やうに
なつてから、
%
それから
\ruby[g]{自{\換字{分}}}{じ ぶん}の
\ruby[g]{{\換字{勝}}手}{かつて }に
\ruby[g]{水野}{みづの }さんの
\ruby[g]{世話}{せ わ }でも
\ruby{誰}{だれ}の
\原本頁{269-2}\改行%
\ruby[g]{世話}{せ わ }でも、
%
\ruby[g]{自{\換字{分}}}{じ ぶん}が
\ruby[g]{親切}{しんせつ}に
して
\ruby{{\換字{遣}}}{や}りたいと
\ruby{思}{おも}ふ
\ruby{人}{ひと}には
\ruby[g]{親切}{しんせつ}に
して
\原本頁{269-3}\改行%
\ruby{{\換字{遣}}}{や}りませう。
%
\ruby{彼}{あ}の
\ruby{優}{やさ}しい
\ruby[g]{智惠}{ち ゑ }の
\ruby{深}{ふか}い
\ruby{氣}{き}の
\ruby{大}{おほ}きい
\ruby{姊}{ねえ}さんでさへ
\ruby[<j||]{妾}{わたし}の
\ruby[g]{眞實}{ほんとう}の
\ruby[||j>]{心}{こゝろ}
\ruby[||j>]{持}{ もち}
% \ruby{心持}{こゝろ|もち}
が
\ruby{解}{わか}つて
\ruby{下}{くだ}さらないかも
\ruby{知}{し}れないのだもの、
\ruby[g]{一身}{いつしん}
\原本頁{269-6}\改行%
の
\ruby{外}{ほか}には
\ruby[|g|]{眞實}{ほんと}に
\ruby[g]{味方}{み かた}は
\ruby{無}{な}い!。
%
\ruby[g]{然樣}{さ う }
\ruby{思}{おも}つては
\ruby{濟}{す}まない
\ruby{事}{こと}ながら
\改行% 校正作業の簡略化のため
、
%
\原本頁{269-6}\改行%
\ruby{此}{こ}の
\ruby{繪}{ゑ}の
\ruby{中}{なか}の
\ruby{鷺}{さぎ}が
\ruby{物}{もの}を
\ruby{云}{い}つたなら、
%
\ruby[g]{屹度}{きつと }% ルビ調整(原本通り)非グループルビ
\ruby{姊}{ねえ}さんの
\ruby[|g|]{徃時}{むかし}も
\ruby{{\換字{分}}}{わか}らう
けれど、
%
\ruby{姊}{ねえ}さんも
やつぱり
\ruby{辛}{つら}いか
\ruby{悲}{かな}しいかの
\ruby{瀬}{せ}を
\ruby{越}{こ}して、
%
そして
\ruby{今}{いま}のやうに
\ruby[|g|]{一人}{ひとり}
\ruby{立}{だち}
\ruby[g]{同樣}{どうやう}な
\ruby{身}{み}に
おなりに
\ruby[g]{相{\換字{違}}}{さうゐ }% ルビ調整(原本通り)非グループルビ
\ruby{無}{な}い。
%
そして
\ruby{此}{こ}の
\原本頁{269-9}\改行%
\ruby{鷺}{さぎ}は
\ruby{其}{そ}の
\ruby[|g|]{因緣}{いはれ}の
\ruby[|g|]{紀念}{かたみ}でも
あらう。
%
\ruby{鷺}{さぎ}も
\ruby{物}{もの}を
\ruby{云}{い}はず、
%
\ruby{姊}{ねえ}さんも
\ruby{御}{お}
\原本頁{269-10}\改行%
\ruby{話}{はな}し
ぢやあ
\ruby{無}{な}い
けれど、
%
\ruby[g]{自{\換字{分}}}{じ ぶん}に
\ruby{比}{くら}べて
\ruby{姊}{ねえ}さんの
\ruby[|g|]{徃時}{むかし}を
おもふと
\改行% 校正作業の簡略化のため
、
%
\原本頁{269-11}\改行%
あゝ
\ruby{何}{なん}と
\ruby{無}{な}く
\ruby[g]{朦朧}{ぼんやり}と
\ruby{解}{わか}るやうな
\ruby{氣}{き}がする!。
』

\原本頁{270-1}%
お
\ruby{龍}{りう}は
\ruby{眼}{め}を
\ruby{開}{ひら}いて
また
\ruby{彼}{か}の
\ruby{繪}{ゑ}を
\ruby{見}{み}れば、
%
\ruby{鷺}{さぎ}は
ただ% 原本では非通り字表記
\ruby{心}{こゝろ}も
\ruby{無}{な}く
\ruby{水}{みづ}に
\ruby{立}{た}ち
\ruby{盡}{つく}して、
%
\ruby[<j||]{爾}{なんぢ}
\ruby{我}{わ}が
\ruby{心}{こゝろ}を
\ruby{知}{し}れりや、
%
\ruby{我}{われ}は
\ruby{謎}{なぞ}なり、
%
と
\ruby{云}{い}はぬ
ばかりに
\ruby[g]{默々}{もく〳〵}たり
\ruby{寂々}{じやく|〳〵}たり。

\makeatletter
\@ifundefined{全三巻@一括ビルド}{%
\vspace{10zw}
{\Large{天うつ浪 {\normalsize 第三{\換字{終}}}}}
}
\makeatother
