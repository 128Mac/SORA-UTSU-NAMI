% +++
% sequence = ["cluttex"]
% [programs.cluttex]
% command  = "cluttex"
% opts     = [ "--engine = uplatex", "--shell-escape", "--output-directory = myout" ]
% +++

% https://okumuralab.org/tex/mod/forum/discuss.php?d=3551 を参考に下記
% を上記のように変更
%#! cluttex --engine=uplatex --shell-escape --output-directory=myout
%   cluttex でビルド時に必要なディレクトリ myout 、 mygwi と myout/mygwi
%   また --includeonly=NAMEs を指定すると '\includeonly{NAMEs}' を仮挿入してくれる

\RequirePackage{plautopatch}
%\RequirePackage{exppl2e}% 警告メッセージ削減のためコメントアウト
\documentclass[
uplatex                                     ,% upLaTeX文書
dvipdfmx                                    ,%
book                                        ,%
tate                                        ,%
twoside                                     ,% even_running_head 有効化
paper                       = a5paper       ,%
open_bracket_pos            = nibu_tentsuki ,% 組み方 段落開始二分折り返し行頭天付き
hanging_punctuation                         ,% 組み方 ぶら下げ組
openany                                     ,%
jafontsize                  = 12pt          ,% 13pt 指定すると LaTeX Font Warning が表示される
%%%%%%%%%%%%%%%%%%%%%%%%%%%%%%%%%%%%%%%%%%%%%% 分冊版では以下は default 設定
% head_space                = 36mm          ,% 天の空き量
% gutter                    = 18mm          ,% のどの余白の大きさ
% headfoot_verticalposition = 24mm          ,%
% line_length               = 27zw          ,% 原本と比較するとき(自動で 30zw 位)
% number_of_lines           = 11            ,% 原本と比較するとき(自動で 15行)
]{jlreq}

\usepackage{bxpapersize}
\usepackage{pxrubrica}
\usepackage{sfkanbun}
\usepackage[deluxe,multi,jis2004]{otf}
\usepackage[directunicode*, noalphabet]{pxchfon}[2017/04/08]
\usepackage{plext}
\usepackage{graphicx}
\usepackage[dir=mygwi,cache=108000]{bxglyphwiki}
\usepackage{indent}
\usepackage{CJK-char-convert}
\rubysetup{||h>}% 無指定時のルビ:(||)前進入禁止、(h)肩付き、(>)後進入大
% \rubysetup{|h>}% 無指定時のルビ:(|)前進入禁止、(h)肩付き、(>)後進入大
\title{\Huge 天うつ浪 {\Large 第一}}
\author{幸田露伴}
\date{         {\small 明治三十九年一月} 春陽{\換字{堂}}}

\makeatletter
% \def\全三巻@一括ビルド{}
% \def\デバッグ@ビルド{}
\@ifundefined{デバッグ@ビルド}{%
  \newcommand{\原本頁}[1]{}% デバッグ用/本番は「空」
  \newcommand{\改行}{}%%%%%% デバッグ用/本番は「空」
  \newcommand{\会話開始}{}%% デバッグ用/本番は「空」
}{%
  \newcommand{\原本頁}[1]{\marginpar{\hfill p-#1}}%%%%% デバッグ用/本番は「空」
  \setlength{\marginparwidth}{20mm}%% 傍注欄の大きさ%%% デバッグ用/本番はコメントアウト
  \newcommand{\改行}{\par}%%%%%%%%%%%%%%%%%%%%%%%%%%%%% デバッグ用/本番は「空」
  \newcommand{\会話開始}{ }%
  %
  % 背景にグリッドを表示させる
  \plautopatchdisable{eso-pic}% https://okumuralab.org/tex/mod/forum/discuss.php?d=2956
  % \usepackage{xcolor} は eso-pic.sty 内部で呼び出している
  \usepackage[
  colorgrid    = true    ,
  gridBG       = true    ,
  gridunit     = mm      , % mm, in, bp, pt
  gridcolor    = red!25  ,
  subgridcolor = lime!75 ,
  texcoord     = true    ,
  ]{eso-pic}
}

\makeatother

\def\footnote#1{\endnote{#1}}
\jlreqsetup{
  endnote_position      = {_chapter} , %後注の表示位置
  endnote_second_indent = {3zw}      ,
  mainmatter_pagebreak  = clearpage  ,
}

\begin{document}
\maketitle
\pagestyle{myheadings}
\newcommand{\Entry}[1]{
	\section*{#1}
	\markboth{#1}{#1}
	\setcounter{equation}{0}}
\begin{indentation}{4zw}{3zw}
\parindent=0pt

\chapter*{}
%\newpage
%\ % 全角空白
%\newpage

\makeatletter
\@ifundefined{全三巻@一括ビルド}{%
{\huge
\ruby{天}{そら} う つ % 空白有り
\ruby{浪}{なみ}}  {\normalsize 第一}
\vspace*{3zw}

\Entry{其一}
}
\makeatother

% メモ 校正終了 2024-03-28 2024-05-22 2024-06-15
\原本頁{1-3}%
\ruby{秋}{あき}は
\ruby[g]{海樓}{かいろう}の
\ruby[g]{垂簾}{すだれ }に
\ruby{動}{うご}きて、
%
ばつと
\ruby{吹}{ふ}き
\ruby{來}{く}る
\ruby{沖}{おき}の
\ruby{風}{かぜ}は、
%
\ruby[g]{夕日}{ゆふひ }の
\ruby{餘}{よ}
\ruby[||j>]{光}{くわう}% 行末行頭の境界付近なので特例処置を施す
\ruby[||j>]{美}{ うる}はしきが
\ruby{中}{なか}に、
%
\ruby[g]{無限}{む げん}の
\ruby[||j>]{爽}{さう}
\ruby[||j>]{凉}{りやう}の
% \ruby{爽凉}{さう|りやう}の
\ruby{氣}{き}を
\ruby{齎}{もた}らせば、
%
\ruby[g]{白帆}{しらほ }
\ruby{明}{あか}るき
\ruby[g]{{\換字{遠}}方}{とほく }の
\ruby{{\換字{船}}}{ふね}の
\ruby[g]{數々}{かず〳〵}も、
%
\ruby[||j>]{{\換字{鉛}}}{なまり}
\ruby[||j>]{色}{ いろ}なして
% \ruby{{\換字{鉛}}色}{なまり|いろ}なして
\ruby[g]{漫々}{まん〳〵}たる
\ruby{潮}{うしほ}の
\ruby{果}{はて}に
\ruby{却}{かへ}つて
\ruby[g]{物淋}{ものさび}しう
\ruby{見}{み}え
\ruby{渡}{わた}りつ、
%
\ruby[g]{竹芝}{たけしば}の
\ruby{浦}{うら}の
\ruby{浪}{なみ}
\ruby{靜}{しづ}かに、
%
\ruby{增上寺}{ぞう|じやう|じ}の
\ruby[g]{鐘聲}{か ね }に
\ruby{暮}{く}れ
\ruby{行}{ゆ}かんとす。

\原本頁{1-8}%
\ruby{此}{こ}の
\ruby[||j>]{夕}{ゆふべ}
\ruby[||j>]{此}{ こ}の% 原本に合わせて調整
\ruby{時}{とき}、
%
『
\ruby{見}{み}はらし
』の
\ruby[||j>]{樓}{ろう}
\ruby[||j>]{上}{じよう}の
% \ruby{樓上}{ろう|じよう}の
\ruby[g]{一室}{いつしつ}に、
%
\ruby{貸}{か}し
\ruby[g]{{\換字{浴}}衣}{ゆ かた}の
\ruby[g]{胸元}{むなもと}% 「、『見はらし』」の部分の影響か改行制御できず
ゆたかに
くつろげて、
%
\ruby{醉}{ゑひ}に% 「醉」は原本通り「ゑ」で調整
\ruby{嘯}{うそぶ}く
\ruby{大胡坐}{おほ|あぐ|ら}、
%
たゞ% 原本は「ヾ」片仮名繰{\換字{返}}し記号(濁点)を使用してる
\ruby{秋}{あき}の
\ruby[g]{飮酒}{さ け }に
\ruby{宜}{よろ}しきを
\原本頁{2-1}\改行%
\ruby{知}{し}つて
\ruby{其}{そ}の
\ruby{他}{た}を
\ruby{知}{し}らぬ
\ruby[g]{面構}{つらがま}へ
きび〳〵と、
%
あはれも
\ruby[g]{絲瓜}{へちま }も
あるものか、
%
\ruby{鴫}{しぎ}が
\ruby{飛}{と}んだら
\ruby{撃}{う}つて
\ruby[g]{下物}{さかな }、
%
と
\ruby{云}{い}はぬ
ばかりの
\ruby{顏}{かほ}つきして、
%
いづれも
\ruby{勇}{いさ}みを
\ruby{含}{ふく}む
\ruby[g]{酒盃}{さかづき}の
\ruby{{\換字{遣}}}{や}り
\ruby{取}{と}り、
%
\ruby{火}{ひ}の
\ruby{珠}{たま}も
\ruby{挾}{はさ}んで
\ruby{食}{く}ふべき
\ruby[g]{年齡}{としばへ}の
\ruby{勢}{いきほ}ひに、
%
\ruby[g]{此方}{こなた }の% ルビ調整(原本通り)
\ruby[g]{壯語}{さうご }、
%
\ruby[g]{彼方}{かなた }の
\ruby[g]{傲語}{がうご }、
%
\ruby{或}{あるひ}は
\ruby{彼}{かれ}
\ruby{此}{これ}
\ruby[g]{哄然}{ど つ }と
\ruby[g]{一齊}{いちど }の
\ruby[g]{天狗}{てんぐ }
\ruby{笑}{わら}ひの
\ruby{響}{どよみ}の
\ruby{中}{うち}に、
%
\ruby[g]{間{\換字{近}}}{ま ぢか}く
\ruby{{\換字{通}}}{とほ}る
\ruby[g]{滊車}{き しや}の
\ruby{音}{おと}をも
\ruby{埋}{うづ}めて
\ruby[g]{仕舞}{し ま }ふまで、
%
\ruby{無邪氣}{む|じや|き}に
\ruby{睦}{むつ}み
\ruby{語}{かた}らへる
\ruby[g]{四人}{よ にん}
\ruby{{\換字{連}}}{づれ}
あり。

\原本頁{2-7}%
\ruby[g]{陽氣}{やうき }の
\ruby[||j>]{歡}{くわん}
\ruby[||j>]{笑}{ せう}は
% \ruby{歡笑}{くわん|せう}は
\ruby{一}{ひ}トしきり
\ruby{濟}{す}みて、
%
\ruby{今}{いま}しも
\ruby[g]{談話}{はなし }は
\ruby{少}{すこ}し
\ruby{沈}{しづ}みぬ。

\原本頁{2-8}%
\ruby{手}{て}さき
\ruby[g]{頸筋}{くびすぢ}に
\ruby[g]{洋服}{やうふく}の
\ruby{痕}{あと}
\ruby[g]{{\換字{判}}然}{はつきり}と
\ruby{知}{し}れて、
%
\ruby{誰}{た}が
\ruby{眼}{め}にも
\ruby[g]{{\換字{船}}人}{ふなのり}と
\ruby{映}{うつ}る
\ruby[<j||]{赭}{あから}% 行末行頭の境界付近なので特例処置を施す
\ruby[||j>]{顏}{がほ}の
% \ruby{赭顏}{あから|がほ}の
\ruby{日}{ひ}に
\ruby{焦}{や}けきつたる
\ruby[g]{羽{\換字{勝}}}{は がち}
\ruby[g]{千{\換字{造}}}{せんざう}は、
%
\ruby[g]{酒盃}{さかづき}を
\ruby{擧}{あ}げて
\ruby{一}{ひ}ト
\ruby{口}{くち}
\ruby{飮}{の}みしが、
%
\ruby{不興氣}{ふ|きよう|げ}に
\ruby[g]{復下}{またした}に
\ruby{置}{お}きて、

\原本頁{2-11}%
『
フーム
』

\原本頁{3-1}%
とばかり
\ruby[g]{力無}{ちからな}く
\ruby{答}{こた}へつ、
%
\ruby{{\換字{猶}}}{なほ}
\ruby{其}{そ}の
\ruby[g]{對手}{あひて }の
\ruby[g]{何事}{なにごと}をか
\ruby{語}{かた}り
\ruby{添}{そ}ふるを
\ruby{待}{ま}つが
\ruby{如}{ごと}き
\ruby{意}{こゝろ}を
\ruby{其}{そ}の
\ruby[g]{語氣}{ご き }に
\ruby{現}{あらは}したり。

\原本頁{3-3}%
\ruby[g]{羽{\換字{勝}}}{は がち}に
\ruby{對}{むか}ひて
\ruby{坐}{ざ}せる
\ruby[g]{小男}{こをとこ}の、
%
\ruby[||j>]{面}{おもて}% 原本に合わせて調整
\ruby[||j>]{淸}{ きよ}らにして
\ruby[g]{桃花}{とうくわ}の
\ruby{如}{ごと}き
\ruby[g]{山瀬}{やませ }
\ruby[g]{荒吉}{あらきち}は
\ruby{其}{その}
\ruby{意}{い}を
\ruby{悟}{さと}つて、
%
\ruby{果}{はた}して
\ruby{直}{たゞち}に
\ruby[g]{言葉}{ことば }を
\ruby{足}{た}しぬ。

\原本頁{3-5}%
『
ト
\ruby{云}{い}ふ
\ruby[g]{次第}{し だい}なので
\ruby[g]{水野}{みづの }
\ruby{君}{くん}は
\ruby{來}{こ}んのさ。
%
\ruby{今}{いま}
\ruby{話}{はな}した
\ruby[||j>]{内}{ない}
\ruby[||j>]{{\換字{情}}}{じやう}も
% \ruby{内{\換字{情}}}{ない|じやう}も
\ruby{解}{わか}つて
\ruby{居}{ゐ}たので、
%
\ruby[g]{今日}{け ふ }の
\ruby[||j>]{會}{くわい}
\ruby[||j>]{合}{ がふ}の
% \ruby{會合}{くわい|がふ}の
\ruby{發起人}{ほつ|き|にん}の
\ruby{僕}{ぼく}は、
%
\ruby[g]{十{\換字{分}}}{じふぶん}に
\ruby[g]{{\換字{情}}理}{じやうり}を
\ruby{盡}{つく}した
\ruby[g]{手紙}{て がみ}を
\ruby{與}{や}つて、
%
\ruby[g]{是非}{ぜ ひ }
\ruby{出}{で}て
\ruby{來}{く}るやうにと
\ruby{勸}{すゝ}めたんだが、
%
たゞ% 原本は「ヾ」片仮名繰{\換字{返}}し記号(濁点)を使用してる
\ruby[||j>]{差}{さし}
\ruby[||j>]{支}{つかへ}があつて
% \ruby{差支}{さし|つかへ}があつて
\ruby{行}{ゆ}かれないといふ
\ruby[g]{冷淡}{れいたん}
\ruby{極}{きは}まる
\ruby[g]{{\換字{返}}事}{へんじ }なんで、
%
\ruby[g]{仕方}{し かた}が
\ruby{無}{な}いと
\ruby[g]{斷念}{あきら }めて
\ruby[g]{仕舞}{し ま }つた。
%
\ruby{實}{じつ}に
\ruby[g]{水野}{みづの }
\ruby{君}{くん}にも
\ruby[g]{似合}{に あ }はない、
%
\ruby[g]{全然}{まるで }
\ruby[g]{無茶}{む ちや}
\ruby[g]{苦茶}{く ちや}になつて
\ruby{居}{ゐ}られるのだからね。
』

\原本頁{3-11}%
\ruby{見}{み}る〳〵
\ruby[g]{羽{\換字{勝}}}{は がち}が
\ruby{面}{おもて}には
\ruby[||j>]{憂}{いう}
\ruby[||j>]{色}{しよく}
% \ruby[||j>]{憂色}{いう|しよく}
\ruby[||j>]{現}{ あら}はれ、% 原本に合わせて調整
%
その
\ruby{眼}{め}は
\ruby[g]{沈思}{ちんし }に
\ruby[g]{凝然}{じ つ }と
\ruby{動}{うご}かずなりたり。

\原本頁{4-2}%
\ruby[g]{羽{\換字{勝}}}{は がち}が
\ruby[g]{左方}{ひだり }に
\ruby{坐}{ざ}して
\ruby[g]{默々}{もく〳〵}と
\ruby{飮}{の}み
\ruby{居}{ゐ}し
\ruby[g]{骨太}{ほねぶと}
\ruby[g]{岩疊}{がんでふ}づくりの
\ruby[g]{日方}{ひ かた}
\ruby[g]{八郞}{はちらう}は、
%
\ruby[g]{突然}{とつぜん}として
\ruby{牛}{うし}の
\ruby{吼}{ほ}ゆるが
\ruby{如}{ごと}くに
\ruby{呌}{さけ}び
\ruby{出}{だ}し、

\原本頁{4-4}%
『
\ruby[g]{山瀬}{やませ }、
%
\ruby[g]{貴樣}{き さま}も
\ruby{今}{いま}は
\ruby[g]{堂々}{だう〴〵}たる
\ruby{新聞記者}{しん|ぶん|き|しや}だ。
%
\ruby[g]{往時}{むかし }のやうに
\ruby{想像談}{さう|〴〵|だん}や
\ruby[g]{法螺}{ほ ら }
\ruby[<j||]{話}{ばなし}は
\ruby{語}{かた}るまいな。
』

\原本頁{4-6}%
と、
%
\ruby{詰}{なじ}り
\ruby[g]{氣味}{ぎ み }に
\ruby{問}{と}ひ
\ruby{糺}{たゞ}せば、
%
\ruby[g]{山瀬}{やませ }は
\ruby{聊}{いさゝ}か
\ruby[g]{怫然}{む つ }として、

\原本頁{4-7}%
『
\ruby[g]{日方}{ひ かた}
\ruby{陸軍少尉}{りく|ぐん|せう|ゐ}
\ruby{殿}{どの}に
\ruby{伺}{うかゞ}ひます。
%
\ruby[g]{報告}{はうこく}は
\ruby{無責任}{む|せき|にん}を
\ruby{以}{もつ}て
\ruby[g]{作爲}{さくゐ }すべきもので
ござりまする
\ruby{歟}{か}。
%
はゝはゝはゝ。% 踊り字調整「〻(二の字点、揺すり点)に見えるが(ゝ)」% 原本の最後の「〻」には見えない
』

\原本頁{4-9}%
と
\ruby{{\換字{遣}}}{や}り
\ruby{{\換字{返}}}{かへ}して
\ruby{笑}{わら}ふ。

\原本頁{4-10}%
\ruby[g]{日方}{ひ かた}は
\ruby[g]{山瀬}{やませ }の
\ruby[g]{戱言}{たはむれ}には
\ruby{頓着無}{とん|ぢやく|な}く、
%
\ruby{怒}{いか}れるが
\ruby{如}{ごと}く
\ruby{眞面目}{ま|じ|め}になりて
\改行% 校正作業の簡略化のため
、

\原本頁{4-11}%
『
ムゝ、% ルビ調整(原本通り)平仮名繰返し記号踊り字。本来は「ヽ]
%
して
\ruby{見}{み}れば
\ruby{全}{まつた}く
\ruby[g]{事實}{じ じつ}と
\ruby{見}{み}える。
%
イヤ
\ruby{怪}{け}しからん、
%
\ruby{實}{じつ}に
\ruby{怪}{け}しからん。
%
\ruby{何}{なん}だ!。
%
\ruby[g]{愚劣}{ぐ れつ}
\ruby{極}{きは}まる!。
%
\ruby[g]{馬鹿}{ば か }
\ruby[g]{々々}{ 〳〵 }しい。% 本来なら \ruby[g]{々々}{ 〴〵 }しい。
%
ナニ?。
%
\ruby[g]{戀愛}{れんあい}に
\ruby{陷}{おちい}つて
\ruby[g]{苦悶}{く もん}しちよる、
%
それで
\ruby[g]{朋友}{ほういう}の
\ruby[||j>]{集}{しふ}
\ruby[||j>]{會}{くわい}にも
% \ruby{集會}{しふ|くわい}にも
\ruby[||j>]{出}{しゆつ}
\ruby[||j>]{席}{ せき}しないと?。
% \ruby{出席}{しゆつ|せき}しないと?。
%
たツ
\ruby{白痴野郎}{た|はけ|や|らう}め、
%
\ruby{何}{なん}といふ
\ruby{事}{こつ}た。
%
そんな
\ruby{愚}{ぐ}な
\ruby{奴}{やつ}では
\ruby{無}{な}かつたが、
%
\ruby{{\換字{魔}}}{ま}にでも
\ruby{憑}{つ}かれ
\ruby{居}{を}つたか、
%
\ruby{下}{くだ}らない。
%
\ruby[g]{山瀬}{やませ }、
%
\ruby[g]{貴樣}{き さま}も
\ruby[g]{幹事}{かんじ }
\ruby[g]{甲{\換字{斐}}}{が ひ }がない。
%
\ruby[g]{其樣}{そ ん }な
\ruby[g]{生溫}{なまぬる}つこい
\ruby{事}{こと}を
\ruby{云}{い}はす
\ruby{法}{はふ}が
\ruby{有}{あ}るかい!。
%
\ruby[g]{領上}{えりがみ}に
\ruby{手}{て}を
\ruby{掛}{か}けて
\ruby[g]{引摺}{ひきず }つて
\ruby{來}{く}りやあ、
%
\ruby[g]{一同}{みんな }で
\ruby[g]{引擲}{ひつぱた}いて
\ruby[g]{正氣}{しやうき}に
\ruby{仕}{し}て
\ruby{{\換字{遣}}}{や}るのに。
%
ゑゝ、
%
\ruby[g]{理由}{わ け }を
\ruby{聞}{き}かぬ
\ruby{間}{うち}は
\ruby{知}{し}らぬが
\ruby{佛}{ほとけ}で
\ruby{腹}{はら}も
\ruby{立}{た}たなかつたが、
%
\ruby{聞}{き}いて
\ruby{見}{み}りやあ
\ruby[g]{馬鹿}{ば か }
\ruby[g]{々々}{ 〳〵 }しくつて% 本来なら \ruby[g]{々々}{ 〴〵 }しくつて
\ruby{腹}{はら}が
\ruby{立}{た}つ。
%
\ruby[g]{山瀬}{やませ }!
\改行% 校正作業の簡略化のため
。
%
\原本頁{5-9}\改行%
\ruby[g]{一體}{いつたい}
\ruby[g]{貴樣}{き さま}が
\ruby{薄}{うす}つぺらで
\ruby[g]{眞底}{しんそこ}からの
\ruby{信實氣}{しつ|じつ|ぎ}が
\ruby{足}{た}らん。
%
\ruby[g]{本來}{ほんらい}
\ruby[g]{我々}{われ〳〵}
\ruby[g]{七人}{しちにん}は% 原本には漢数字「七」のルビ無し
\ruby[g]{何樣}{ど う }いふ
\ruby[g]{{\換字{交}}{\換字{情}}}{な か }だ。
%
みんな
\ruby[g]{野州}{や しう}の
\ruby[g]{田舎}{ゐなか }
\ruby{{\換字{漢}}}{もの}、
%
\ruby{碌}{ろく}な
\ruby{親}{おや}を
\ruby{持}{も}つたものは
\ruby[g]{一人}{ひとり }も
\ruby{無}{な}くつて、
%
\ruby[g]{役塲}{やくば }の% 原文通り「塲」
\ruby[g]{書記}{しよき }や
\ruby[g]{小學}{せうがく}
\ruby[g]{敎師}{けうし }、
%
\ruby[g]{乃公}{お ら }あ
\ruby{人力車}{く|る|ま}も
\原本頁{6-1}%
\ruby{曳}{ひつ}ぱつた
\ruby{{\換字{貧}}}{ひん}
\ruby[g]{書生}{しよせい}だが、
%
\ruby[g]{自己}{う ぬ }が
\ruby[g]{腕臑}{うですね}で
\ruby{食}{く}ふ
\ruby[g]{{\換字{貧}}乏}{びんばふ}
\ruby[g]{同士}{どうし }、
%
\ruby[g]{何時}{い つ }と
\ruby{無}{な}く
\原本頁{6-2}\改行%
\ruby{知}{し}り
\ruby{合}{あ}ひになつた
\ruby[g]{七人}{しちにん}が、% 原本には漢数字「七」のルビ無し
%
\ruby[g]{男兒}{をとこ }と
\ruby{生}{うま}れて
\ruby[g]{此狀}{こ れ }ぢやあ
\ruby{死}{し}ねぬ
%
\makeatletter
\@ifundefined{デバッグ@ビルド}{%
  、%
  \ruby[<j>]{志}{こゝろざ}す% ルビ調整
  }{%
  \ruby[<g>]{、志}{こゝろざ}す% ルビ調整(行末の特殊処理)「、」部分にルビを押し込む
}%
\makeatother
ところは
\ruby{異}{ちが}つても
\ruby{互}{たがひ}に
\ruby{助}{たす}け
\ruby{幇}{たす}け
\ruby{合}{あ}つて、
%
\ruby[g]{或時}{あるとき}は
\ruby{兄}{あに}となつて
\ruby[g]{學資}{がくし }も
\ruby{貢}{みつ}ぎ、
%
\ruby[g]{或時}{あるとき}は
\ruby{弟}{おとゝ}となつて
\ruby{恩}{おん}を
\ruby{報}{はう}じ、
%
\ruby{勵}{はげ}み
\ruby{合}{あ}ひ
\ruby[g]{擁護}{か ば }ひ
\ruby{合}{あ}つて
\原本頁{6-5}\改行%
\ruby{{\換字{進}}}{すゝ}んで
\ruby{行}{い}つたら、
%
\ruby{世}{よ}に
\ruby{立}{た}つて
\ruby{生}{い}き
\ruby[g]{甲{\換字{斐}}}{が ひ }のある
\ruby{身}{み}
ともなれやうと
\改行% 校正作業の簡略化のため
、
%
\原本頁{6-6}\改行%
\ruby[g]{七人}{しちにん}% 原本には漢数字「七」のルビ無し
\ruby{集}{あつ}まつた
\ruby{宇都宮}{う|つの|みや}の
\ruby{二荒山神社}{ふた|あら|やま|じん|じや}の
\ruby[g]{廣{\換字{前}}}{ひろまへ}で、
%
\ruby{此}{こ}の
\ruby[||j>]{願}{ねがひ}
\ruby[||j>]{此}{ こ}の% 原本に合わせて調整
\ruby[||j>]{心}{こゝろ}
\ruby[||j>]{渝}{ かは}るまじ、% 原本に合わせて調整
%
\ruby{必}{かなら}ず
\ruby[g]{信義}{しんぎ }を
\ruby{盡}{つく}し
\ruby{合}{あ}はんと、
%
\ruby{神}{かみ}に
\ruby{誓}{ちか}つた
\ruby[g]{{\換字{交}}{\換字{情}}}{な か }では
\ruby{無}{な}いか。
%
\原本頁{6-8}\改行%
\ruby[g]{指折}{ゆびを }り
\ruby{數}{かぞ}ふれば
\ruby{{\換字{速}}}{はや}いもので
\ruby{既}{はや}
\ruby[g]{七年}{しちねん}の% 原本には漢数字「七」のルビ無し
\ruby[g]{往時}{むかし }になるが、
%
\ruby[g]{其時}{そ れ }からといふものは
\ruby[g]{段々}{だん〳〵}と、% ルビ調整(原本通り)踊り字表記
%
\ruby{苦}{くる}しい
\ruby[g]{同士}{どうし }で
\ruby[g]{無理}{む り }
\ruby[g]{才覺}{さいかく}、
%
\ruby[g]{三人}{さんにん}の
\ruby[g]{財布}{さいふ }を
\ruby{揮}{ふる}つては
\ruby[g]{一人}{いちにん}
\footnote{%
この文脈では、「一人」のルビは(ひとり)と思われるが、
原本では「\ruby[g]{一人}{  にん}」となっていたため、
漢数字のルビ化に伴い(いちにん)とした。
(国会図書館 コマ番号7/134 p-6 l-10)
}%
の
\ruby[g]{{\換字{遊}}學}{いうがく}の
\ruby[g]{支度}{し たく}を
\ruby{拵}{こしら}へ、
%
\ruby[g]{五人}{ご にん}の
\ruby[g]{着物}{き もの}を
\ruby{賣}{う}つては
\ruby[g]{一人}{いちにん}
\footnote{%
この文脈では、「一人」のルビは(ひとり)と思われるが、
原本では「\ruby[g]{一人}{  にん}」となっていたため、
漢数字のルビ化に伴い(いちにん)とした。
(国会図書館 コマ番号7/134 p-6 l-10)
}%
の
\ruby{身}{み}の
\ruby{立}{た}つ
\ruby[g]{本錢}{もとで }とするといふ
\ruby[g]{始末}{し まつ}で、
%
ボツリ〳〵と
\ruby{皆}{みな}
\ruby[||j>]{東}{とう}
\ruby[||j>]{京}{きやう}へ、
% \ruby{東京}{とう|きやう}へ、
%
\ruby{漸}{やうや}く
\原本頁{7-1}\改行%
\ruby{這}{は}ひ
\ruby{出}{だ}して
それ〴〵に、
%
\ruby[<j>]{志}{こゝろざ}す
\ruby{{\換字{道}}}{みち}へと
\ruby{身}{み}を
\ruby{入}{い}れた、
%
\ruby[g]{如是}{かういふ}
\ruby[g]{{\換字{交}}{\換字{情}}}{な か }だのに
\ruby{何}{なん}の
\ruby{事}{こつ}た!。
%
\ruby[g]{胸糞}{むなくそ}の
\ruby{惡}{わる}い
\ruby[g]{戀愛}{れんあい}なんぞに
\ruby[g]{水野}{みづの }が
\ruby{{\換字{迷}}}{まよ}つてるなら
\ruby[g]{何故}{な ぜ }
\ruby[g]{打棄}{うつちや}つて
\ruby{置}{お}く?。
%
{\換字{志}}かも
\ruby[g]{羽{\換字{勝}}}{は がち}が
\ruby{始}{はじ}めて
\ruby[g]{首尾}{しゆび }よく
\ruby{{\換字{遠}}洋漁業}{ゑん|やう|ぎよ|げふ}の
\ruby{長}{なが}
\原本頁{7-4}\改行%
い
\ruby[g]{航海}{かうかい}を、
%
\ruby{{\換字{終}}}{をは}つて
\ruby{來}{き}た
\ruby[g]{今日}{け ふ }の
\ruby[g]{欣喜}{よろこび}の
\ruby[g]{集會}{あつまり}に、
%
\ruby[g]{自己}{お の }が
\ruby[g]{{\換字{勝}}手}{かつて }の
\ruby[||j>]{女}{をんな}
\ruby[||j>]{沙}{ ざ}
\原本頁{7-5}\改行%
\ruby[||j>]{汰}{た}のために
\ruby[g]{不參}{ふ さん}とは、
%
\ruby[g]{我々}{われ〳〵}を
\ruby{踏}{ふ}み
\ruby{付}{つ}けた
\ruby{憎}{にく}い
\ruby[g]{我儘}{わがまゝ}。
%
\ruby[||j>]{山瀬}{やま|せ}
\ruby[<j||]{汝}{きさま}は% ルビ調整(原本通り)
\ruby{何}{な}
\原本頁{7-6}\改行%
\ruby{故}{ぜ}
\ruby[g]{打棄}{うつちや}つて
\ruby{置}{お}く?。
%
\ruby{汝}{きさま}が
\ruby{新聞記者}{しん|ぶん|き|しや}になつた
\ruby{時}{とき}は、
%
\ruby[g]{我々}{われ〳〵}
\ruby[g]{七人}{しちにん}% 原本には漢数字「七」のルビ無し
\ruby{皆}{みな}
\ruby{揃}{そろ}
\原本頁{7-7}\改行%
つた。
%
\ruby[g]{乃公}{お れ }が
\ruby{士官候補生}{し|くわん|こう|ほ|せい}になつた
\ruby{時}{とき}にも
\ruby{皆}{みな}
\ruby{集}{あつ}まつて
\ruby{悅}{よろこ}んで
\ruby{吳}{く}れた。
%
\ruby[g]{羽{\換字{勝}}}{は がち}
\ruby{君}{くん}の
\ruby[g]{今日}{け ふ }の
\ruby[g]{祝賀}{よろこび}の
\ruby{會}{くわい}には、
%
\ruby[g]{楢井}{ならい }は
\ruby{北海{\換字{道}}}{ほく|かい|だう}に
\ruby{行}{い}つて
\ruby{居}{を}り
\改行% 校正作業の簡略化のため
、
%
\原本頁{7-9}\改行%
\ruby[g]{名倉}{な ぐら}は
\ruby[g]{病氣}{びやうき}、
%
\ruby[g]{二人}{ふたり }
\ruby{缺}{か}けて
\ruby{居}{ゐ}るさへ
\ruby[g]{殘念}{ざんねん}なに、
%
\ruby[g]{水野}{みづの }まで
\ruby{來}{こ}ぬので
\原本頁{7-10}\改行%
\ruby[||j>]{只}{たつた}% 後突き出させないようにした
\ruby[||j>]{四人}{ よ|にん}、
%
\ruby[g]{第一}{だいいち}
\ruby[g]{羽{\換字{勝}}}{は がち}
\ruby{君}{くん}にも
\ruby{氣}{き}の
\ruby{毒}{どく}
\ruby[g]{千萬}{せんばん}だ。
%
\ruby[g]{戀愛}{れんあい}も
\ruby{糞}{くそ}もあるものか
\改行% 校正作業の簡略化のため
、
%
\原本頁{7-11}\改行%
\ruby[g]{世間}{せ けん}
\ruby[g]{一統}{いつとう}の
\ruby[g]{愚物}{ぐ ぶつ}は
\ruby{知}{し}らず、
%
\ruby[g]{何時}{い つ }でも
\ruby[g]{現在}{げんざい}に
\ruby[g]{滿足}{まんぞく}せいで、
%
\ruby[g]{永久}{えいきう}に
\原本頁{8-1}\改行%
\ruby{{\換字{進}}}{すゝ}んで
\ruby{{\換字{飽}}}{あ}くこと
\ruby{知}{し}らぬを
\ruby[g]{理想}{り さう}と
\ruby{定}{さだ}めた
\ruby[g]{我我}{われ〳〵}% 原本通り非踊り字表記「我我」
\ruby[g]{七人}{しちにん}、% 原本には漢数字「七」のルビ無し
%
\ruby[g]{戀愛}{れんあい}
なんぞといふ
アタ
\ruby{{\換字{嫌}}}{いや}らしい
\ruby[g]{濕氣}{しつけ }の
\ruby{蠹}{むし}に、% ここのみ「蠹」
%
\ruby[g]{魂魄}{たましひ}を
\ruby{蝕}{くは}せて
\ruby{居}{ゐ}る
\ruby{間}{ま}は
\ruby{無}{な}い
\ruby{筈}{はず}。
%
\原本頁{8-1}\改行%
\ruby[g]{一體}{いつたい}
\ruby[g]{全體}{ぜんたい}
\ruby{癪}{しやく}に
\ruby{觸}{さは}る!。
%
\ruby{何}{なに}を
\ruby{讀}{よ}んでも
\ruby[g]{何處}{ど こ }へ
\ruby{行}{い}つても、
%
\ruby[g]{此頃}{このごろ}は
\ruby[g]{戀愛}{れんあい}といふ
\ruby{奴}{やつ}ばかり
\ruby{轉}{ころ}げて
\ruby{居}{ゐ}をるが、% 原本通りで(を)がある(国会図書館 コマ番号 8/134 p-8 l-4)
%
\ruby[g]{戀愛}{れんあい}たあ
\ruby{何}{なん}だ?、
%
\ruby{何}{なん}だ
\ruby[<j||]{正}{しやう}% 行末行頭の境界付近なので特例処置を施す
\ruby[<j||]{體}{たい}は?。
% \ruby{正體}{しやう|たい}は?。
%
\ruby[g]{自己}{う ぬ }から
\ruby{見}{み}りやあ
\ruby{貴}{い}いか
\ruby{知}{し}らぬが、
%
\ruby{他}{ひと}から
\ruby{見}{み}りやあ
\ruby{石决明}{あは|びつ|かひ}を
\ruby{當}{あ}てがつて
\ruby{{\換字{遣}}}{や}る
\ruby[g]{價値}{ね うち}も
\ruby{無}{な}い
\ruby[g]{馬糞}{ば ふん}に
\ruby{劣}{おと}つた
\ruby[g]{貨物}{しろもの}で、
%
\ruby{高}{たか}が
\ruby{女}{をんな}に
びりつく
\ruby{事}{こと}だ!。
%
\ruby[g]{水野}{みづの }は
\ruby{釅}{きぶ}い
\ruby{醋}{す}のやうな
\ruby{恐}{おそ}ろしい
ところのある
\原本頁{8-8}\改行%
\ruby{奴}{やつ}ぢやつたが、
%
\ruby[g]{{\換字{浮}}世}{うきよ }に
\ruby[g]{{\換字{感}}染}{か ぶ }れたのは
\ruby{氣}{き}が
\ruby{{\換字{緩}}}{ゆる}んだ
\ruby{歟}{か}。
%
\ruby[g]{打棄}{うつちや}つて
\ruby{置}{おい}ては
\ruby[g]{利益}{た め }にならん。
%
\ruby{直}{すぐ}
これから
\ruby{行}{い}つて
\ruby[g]{引摺}{ひきず }つて
\ruby{來}{こ}やう。
%
さあ
\ruby[g]{山瀬}{やませ }!\inhibitglue{}%
\ruby[g]{一緖}{いつしよ}に
\ruby{行}{ゆ}け、
%
\ruby{立}{た}たぬかやい。
%
\ruby[g]{水野}{みづの }めを
\ruby[g]{引張}{ひつぱ }つて
\ruby{來}{き}て
\ruby[g]{此處}{こ ゝ }で
\原本頁{8-11}\改行%
\ruby{諫}{いさ}めて、
%
\ruby{諫}{いさ}めて
\ruby{聽}{き}かずば
\ruby{擲}{たゝ}き
\ruby{撲}{なぐ}つて、
%
\ruby[g]{正氣}{しやうき}に
\ruby{{\換字{返}}}{かへ}らせて
\ruby{吳}{く}れにやならぬ、
%
さあ
\ruby{立}{た}て
\ruby[g]{山瀬}{やませ }!。
』

\原本頁{9-2}%
と
\ruby{云}{い}ひざまに、
%
\ruby[g]{五{\換字{分}}}{ご ぶ }の
\ruby[g]{慷慨}{かうがい}、
%
\ruby[g]{五{\換字{分}}}{ご ぶ }の
\ruby{醉}{ゑひ}、% 「醉」は原本通り「ゑ」で調整
%
\ruby[g]{山瀬}{やませ }が
\ruby[g]{肩頭}{かたさき}を
\ruby[g]{引攫}{ひつつか}んで
\ruby[g]{氣勢}{いきほひ}
\ruby{猛}{もう}に
\ruby[g]{立上}{たちあが}つたり。

\Entry{其二}

% メモ 校正終了 2024-03-28 2024-05-22 2024-06-15
\原本頁{9-5}%
\ruby{薄墨}{うす|ずみ}の
\ruby{夕}{ゆふべ}の
\ruby{色}{いろ}は
\ruby{物蔭}{もの|かげ}より
\ruby{擴}{ひろ}まりて、
%
\ruby{廓然}{くわ|らり}と
\ruby{晴}{は}れやかなりし
\ruby{樓}{ろう}の
\ruby{上}{うへ}も、
%
\ruby{手許}{て|もと}
やうやく
\ruby{暗}{くら}くなり、
%
いづくに
\ruby{歸}{かへ}る
\ruby{鵜}{う}の
\ruby{鳥}{とり}の、
%
\ruby{浪}{なみ}を
\ruby{{\換字{摩}}}{す}つて
\ruby{飛}{と}ぶ
\ruby{羽音}{は|おと}も
\ruby{寂}{さ}びたり。
%
\ruby{右}{みぎ}の
\ruby{方}{かた}は
\ruby{高輪}{たか|なわ}
\ruby{八ツ山}{や|つ|やま}% 地名なので一つにした
\ruby{品川}{しな|がは}の
\ruby{一}{ひ}トつゞき、
%
\ruby{森}{もり}も
\ruby{人家}{じん|か}も
たゞ
\ruby{一}{ひ}ト
\ruby{筆}{ふで}の
なすり
\ruby{書}{がき}と
\ruby{黑}{くろ}み、
%
\ruby{左}{ひだり}に
\ruby{低}{ひく}き
\ruby{築地}{つき|ぢ}
\ruby{月島}{つき|しま}、
%
\ruby{洲崎}{す|さき}は
\ruby{微}{かすか}にして
\ruby{{\換字{消}}}{き}えん
とする
\ruby{時}{とき}、
%
\ruby{其處}{そ|こ}に
\ruby{電燈}{でん|とう}の
\ruby{白々}{しろ|〴〵}と
\ruby{輝}{かゞや}き
\ruby{出}{い}づれば、
%
\ruby{燈火}{とも|しび}
\ruby{華}{はな}やかに
\ruby{此家}{こ|こ}にも
\ruby{點}{つ}きて、
%
\ruby{室}{へや}の
\ruby{内}{うち}
ぱつと
\ruby{明}{あか}るくなり、
%
\原本頁{10-1}%
\ruby{外}{そと}は
\ruby{全}{まつた}く
\ruby{海}{うみ}
\ruby{玄}{くろ}く
\ruby{風}{かぜ}
\ruby{睡}{ねむ}れる
\ruby{穩}{おだ}やかなる
\ruby{夜}{よ}となり
\ruby{畢}{をは}んぬ。

\原本頁{10-2}%
\ruby{日方}{ひ|かた}が
\ruby{急}{せ}き
\ruby{{\換字{込}}}{こ}み
\ruby{調子}{てう|し}に
\ruby{物言}{もの|い}ひても、
%
\ruby{特{\換字{更}}}{こと|さら}に
\ruby{沈着}{おち|つき}を
\ruby{爲}{つく}れる
\ruby{山瀬}{やま|せ}
\ruby{荒吉}{あら|きち}は、
%
\ruby{言}{い}ひ
\ruby{爭}{あらそ}はん
ともせで
\ruby{良}{やゝ}
\ruby{少時}{しば|し}、
%
\ruby{何事}{なに|ごと}をか
\ruby{思}{おも}ひ
\ruby{{\換字{廻}}}{めぐ}らし
\ruby{居}{ゐ}けるが、
%
\ruby{今}{いま}しも
\ruby{燈火}{とも|しび}の
\ruby{光}{ひかり}を
\ruby{得}{え}て、
%
\ruby{心}{こゝろ}の
\ruby{中}{うち}に
\ruby{索}{たづ}ね
\ruby{得}{え}し
\ruby{言葉}{こと|ば}の
\ruby[<j>]{緖}{いとぐち}をや
\ruby{求}{もと}め
\ruby{得}{え}けん、
%
\ruby{逸}{はや}り
きつたる
\ruby{日方}{ひ|かた}の
\ruby{面}{おもて}の、
%
いさゝか
\ruby{怒}{いかり}をさへ
\ruby{帶}{お}びたるを、
%
\ruby{愛}{あい}するが
\ruby{如}{ごと}く
\ruby{打見}{うち|み}やりて、

\原本頁{10-7}%
『
マア
\ruby{坐}{すわ}つて
\ruby{吳}{く}れ、
%
\ruby{日方}{ひ|かた}!。
%
\ruby{成程}{なる|ほど}
\ruby{打棄}{うつ|ちや}つて
\ruby{置}{お}いては
\ruby{水野}{みづ|の}の
\ruby{不利益}{ふ|た|め}になるから、
%
\ruby{君}{きみ}と
\ruby{一緖}{いつ|しよ}に
\ruby{{\換字{尋}}}{たづ}ねて
\ruby{行}{い}つて、
%
\ruby{隨{\換字{分}}}{ずゐ|ぶん}
\ruby[||j>]{忠}{ちゆう}% 原本通り(ちゆう)(国会図書館 コマ番号 9/134 p10 l8)
\ruby[||j>]{告}{ こく}も
% \ruby{忠告}{ちゆう|こく}も
\ruby{試}{こゝろ}みやう
\改行% 校正作業の簡略化のため
。
%
\原本頁{10-9}\改行%
\ruby{併}{しか}し
\ruby{水野}{みづ|の}のところは
\ruby{大{\換字{分}}}{だい|ぶん}
\ruby{{\換字{遠}}}{とほ}い。
%
\ruby{{\換字{連}}}{つ}れて
\ruby{來}{く}るにしても
\ruby{時間}{と|き}が
かゝる。
%
もう
\ruby{此}{こ}の
\ruby{{\換字{通}}}{とほ}り
\ruby{夜}{よ}にも
\ruby{入}{い}つて
\ruby{居}{ゐ}る。
%
\ruby{{\換字{連}}}{つ}れて
\ruby{來}{き}たにしたところで
\ruby{話}{はな}す
\ruby{間}{ま}も
\ruby{無}{な}い。
%
\ruby{第一}{だい|いち}
\ruby{左樣}{さ|う}で
\ruby{無}{な}くつてさへ、
%
\ruby{七人}{しち|にん}の
\ruby{中}{うち}が
\ruby{三人}{さん|にん}
\ruby{缺}{か}けて、
%
\原本頁{11-1}%
\ruby{四人}{よ|にん}しか
\ruby{居}{を}らぬ
\ruby{此}{こ}の
\ruby{席}{せき}を、
%
\ruby{君}{きみ}と
\ruby{僕}{ぼく}と
\ruby{二人}{ふた|り}
\ruby{脫}{ぬ}けて
\ruby{仕舞}{し|ま}へば
\原本頁{11-2}\改行%
\ruby{後}{あと}は
\ruby{何樣}{ど|う}だ。
%
\ruby{羽{\換字{勝}}}{は|がち}
\ruby{君}{くん}と
\ruby{島木}{しま|き}
\ruby{君}{くん}とたつた
\ruby{二人}{ふた|り}だ。
%
\ruby{今日}{け|ふ}の
\ruby{客}{きやく}たる
\ruby{羽{\換字{勝}}}{は|がち}
\ruby{君}{くん}を、
%
\ruby{島木}{しま|き}
\ruby{君}{くん}と
\ruby[||j>]{只}{たつた}
\ruby[||j>]{二人}{ ふた|り}に% ルビ調整(原本通り)
\ruby{仕}{し}て
\ruby{仕舞}{し|ま}つて、
%
\ruby{僕等}{ぼく|ら}が
\ruby{出}{で}て
\ruby{行}{い}くといふのは
\ruby{{\換字{勝}}手}{かつ|て}
\ruby{{\換字{過}}}{す}ぎる。
%
それでは
\ruby{餘}{あんま}り
\ruby{無禮}{ぶ|れい}になる。
%
こゝを
\ruby{無理}{む|り}に
\ruby{君}{きみ}と
\ruby{二人}{ふた|り}で
\ruby{出}{で}て
\ruby{行}{い}つたら、
%
\ruby{水野}{みづ|の}には
\ruby{成程}{なる|ほど}
\ruby{親切}{しん|せつ}にも
ならう、
%
\ruby{併}{しか}し
\ruby{羽{\換字{勝}}}{は|がち}
\ruby{君}{くん}には
\ruby{失敬}{しつ|けい}に
\ruby{當}{あた}らう。
%
もと〳〵
\ruby{君}{きみ}が
\ruby{怒}{おこ}り
\ruby{立}{た}つたのも、
%
つまりは
\ruby{水野}{みづ|の}が
\ruby{羽{\換字{勝}}}{は|がち}
\ruby{君}{くん}に
\ruby{對}{たい}する
\ruby{仕方}{し|かた}が
\ruby{冷淡}{れい|たん}
だといふのにあらう。
%
\ruby{羽{\換字{勝}}}{は|がち}
\ruby{君}{くん}に
\ruby{滿足}{まん|ぞく}を
\ruby{{\換字{感}}}{かん}ぜしめぬ
\ruby{其事}{そ|れ}が
\ruby{惡}{にく}むべき
\ruby{我儘}{わが|まゝ}
だといふのだ。
%
それだのに
\ruby{今}{いま}
\ruby{僕等}{ぼく|ら}が
\ruby{此席}{こ|ゝ}を
\ruby{去}{さ}つては、
%
たゞ
\ruby{淋}{さび}しさを
\ruby{增}{ま}すばかりで、
%
\原本頁{11-10}\改行%
\ruby{羽{\換字{勝}}}{は|がち}
\ruby{君}{くん}は
いよ〳〵
おもしろく
\ruby{無}{な}く
\ruby{{\換字{感}}}{かん}じやう。
%
\ruby{今日}{け|ふ}は
\ruby{既}{もう}
\ruby{十{\換字{分}}}{じふ|ぶん}に
\ruby{談笑}{だん|せう}も% ルビ調整(原本通り)「だんせ(う)」
\ruby{仕}{し}て、
%
\ruby{大{\換字{分}}}{だい|ぶ}
\ruby{醉}{ゑひ}さえも% 「醉」は原本通り「ゑ」で調整
\ruby{{\換字{廻}}}{まは}つて
\ruby{居}{ゐ}る。
%
\ruby{談話}{はな|し}の
\ruby{序}{つひで}から
\ruby{不圖}{ふ|と}
\ruby{水野}{みづ|の}の
\原本頁{12-1}\改行%
\ruby{事}{こと}が
\ruby{出}{で}て、
%
\ruby{始}{はじ}めて
\ruby{君}{きみ}は
\ruby{其}{それ}を
\ruby{聞}{き}いた
ところから、
%
\ruby{大}{おほき}に
\ruby{忌}{いま}はしくも
\原本頁{12-2}\改行%
\ruby{{\換字{感}}}{かん}じたらうが、
%
\ruby{何}{なに}も
\ruby{今}{いま}が
\ruby{今}{いま}で
\ruby{無}{な}くちやならぬといふ
\ruby{事}{こと}では
\ruby{無}{な}いから、
%
\ruby{彼}{かれ}を
\ruby{訪}{と}ふのは
\ruby{明日}{あ|す}でも
\ruby{明後日}{あさ|つ|て}でもの
\ruby{事}{こと}として、
%
\ruby{其}{その}
\ruby{時}{とき}
\ruby{戀愛}{れん|あい}
\ruby{{\換字{嫌}}}{ぎら}ひの
\ruby{君}{きみ}の
\ruby{存{\換字{分}}}{ぞん|ぶん}に、
%
\ruby{諫}{いさ}めるとも
\ruby{擲}{なぐ}る
ともするが
\ruby{宜}{よ}からう。
%
\ruby{今日}{け|ふ}は
\ruby{先}{ま}づ
\ruby{堪{\換字{忍}}}{かん|にん}して% 原文通り「堪忍」
\ruby{一同}{みん|な}と
\ruby{共}{とも}に、
%
\ruby{飮}{の}んで
\ruby{居}{ゐ}て
\ruby{吳}{く}れたつて
\ruby{可}{よ}いでは
\ruby{無}{な}いか。
』

\原本頁{12-7}%
と、
%
\ruby{他}{ひと}の
\ruby{言}{い}ふ
ところは
\ruby{斜}{なゝめ}に
\ruby{外}{そ}らせて、
%
\ruby{我}{わ}が
\ruby{言}{い}ふ
ところは
\ruby{斜}{なゝめ}に
\ruby{徹}{とほ}す
\ruby{才子}{さい|し}の
\ruby{面}{おもて}は
\ruby{笑}{ゑみ}を
\ruby{湛}{たゝ}へて、
%
\ruby{巧}{たくみ}に
\ruby{粗獷}{ぶ|こつ}なる
\ruby{相手}{あひ|て}を
\ruby{制}{せい}すれば、
%
\ruby[<j||]{正}{しやう}
\ruby[<j||]{直}{ぢき}
% \ruby{正直}{しやう|ぢき}
\ruby{一三昧}{いつ|さん|まい}の
% いっ‐さんまい【一三昧】 の解説
% 1 仏語。雑念を去り一心に修行に専念すること。
% 2 ほかのことに構わず、一つのことだけに心を用いること
\ruby{日方}{ひ|かた}は、
%
\ruby{脆}{もろ}くも、
%
\ruby{羽{\換字{勝}}}{は|がち}を
\ruby{重}{おも}んずる
\ruby{{\換字{情}}}{こゝろ}より、

\原本頁{12-10}%
『
ムー、
%
\ruby{此}{こ}の
\ruby{席}{せき}が
\ruby{淋}{さび}しくなる?。
%
ア、
%
\ruby{其處}{そ|こ}へは
\ruby{些}{ちつと}も
\ruby{氣}{き}が
つかなかつた。
%
\ruby{成程}{なる|ほど}
\ruby{今}{いま}
\ruby{直}{すぐ}
\ruby{引張}{ひつ|ぱ}つて
\ruby{來}{こ}やうと
\ruby{云}{い}つたのは、
%
\ruby{乃公}{お|れ}が
\ruby{惡}{わる}かつた。
%
\原本頁{13-1}%
こいつは
\ruby{一番}{いち|ばん}
\ruby{山瀬}{やま|せ}に
やられた。
%
ハヽヽ。
%
どうも
\ruby{山瀬}{やま|せ}は
\ruby{乃公}{お|れ}より
\ruby{怜悧}{り|こう}だ。% ルビ調整(原本通り)(りこう)
%
ハヽハヽ。
』

\原本頁{13-3}%
と、
%
\ruby{露}{つゆ}ばかりの
\ruby{我執}{が|しふ}も
\ruby{無}{な}く
\ruby{笑}{わら}つて
\ruby{仕舞}{し|ま}つて、
%
\ruby{霽々}{はれ|〴〵}したる
\ruby{顏色}{かほ|つき}にも
\ruby{著}{しる}き
\ruby{胸}{むね}に
\ruby{何}{なに}も
\ruby{{\換字{遺}}}{のこ}さぬ
\ruby{有樣}{あり|さま}は、
%
\ruby{譬}{たと}へば
\ruby{風}{かぜ}
\ruby{{\換字{過}}}{す}ぎて
\ruby{林}{はやし}
おのづから
\ruby[<j||]{靜}{しづか}に、
%
\ruby{雲}{くも}
\ruby{去}{さ}つて
\ruby{山}{やま}
\ruby{{\換字{更}}}{さら}に
\ruby{靑}{あを}きが
\ruby{如}{ごと}くなりしが、
%
\ruby{例}{れい}の
\ruby{癖}{くせ}とて
\ruby{突然}{とつ|ぜん}と、

\原本頁{13-6}%
『
ヤ、
%
\ruby{時}{とき}に
\ruby{羽{\換字{勝}}}{は|がち}
\ruby{君}{くん}
\ruby{一盃}{いつ|ぱい}
\ruby{吳}{く}れたまへ。
』

\原本頁{13-7}%
と
\ruby{云}{い}ひ
\ruby{出}{いだ}したり。
%
\ruby{羽{\換字{勝}}}{は|がち}は
\ruby{機{\換字{嫌}}}{き|げん}
\ruby{好}{よ}く
\ruby[<j>]{盃}{さかづき}を
さして、

\原本頁{13-8}%
『
\ruby{相變}{あひ|かは}らず
\ruby{君}{きみ}は
\ruby{君}{きみ}の
\ruby{氣風}{き|ふう}で
\ruby{押{\換字{通}}}{おし|とほ}すナ。
%
どうだ
\ruby{軍{\換字{隊}}}{ぐん|たい}の
\ruby[||j>]{生}{せい}
\ruby[||j>]{活}{くわつ}は
% \ruby{生活}{せい|くわつ}は
\ruby{{\換字{愉}}快}{ゆ|くわい}かネ。
』

\原本頁{13-10}%
と
\ruby{懷}{なつ}かし
\ruby{氣}{げ}に
\ruby{問}{と}へば、

\原本頁{13-11}%
『
ムヽ。
%
\ruby{左樣}{さ|う}さ、
%
\ruby[<j||]{快}{くわい}
\ruby[||j>]{活}{くわつ}な
\ruby{事}{こと}ばかりといふ
\ruby{譯}{わけ}にも
\ruby{行}{ゆ}かん。
%
\ruby{僕等}{ぼく|ら}の
\原本頁{14-1}%
\ruby{身{\換字{分}}}{み|ぶん}では
\ruby{隨{\換字{分}}}{ずゐ|ぶん}
\ruby{箱詰}{はこ|づめ}に
なるのを
\ruby{甘}{あま}んじ
なけりやあならん
\ruby{事}{こと}もあるが、
%
\ruby{其}{それ}が
\ruby{{\換字{即}}}{すなは}ち
\ruby{紀律}{き|りつ}で、
%
\ruby{紀律}{き|りつ}が
\ruby{{\換字{即}}}{すなは}ち
\ruby{精神}{せい|しん}である、
%
といふやうに
\ruby[<j||]{考}{かんが}へて% 行末行頭の境界付近なので特例処置を施す
\ruby{居}{ゐ}りやあ、
%
\ruby{別}{べつ}に
\ruby{窮屈}{きう|くつ}にも
\ruby{{\換字{感}}}{かん}じない。
%
ホワイトシヤツを
\ruby{着慣}{き|な}
\改行% 校正作業の簡略化のため
れ
\原本頁{14-4}\改行% 原本では一行が29文字になっているため
て
\ruby{見}{み}ると、
%
\ruby{彼}{あ}の
\ruby{硬}{こは}いものを
\ruby{身}{み}につけるのが、
%
\ruby{却}{かへ}つて
\ruby{好}{い}い
\ruby[||j>]{心}{こゝろ}
\ruby[||j>]{持}{ もち}に
% \ruby{心持}{こゝろ|もち}に
% \原本頁{14-5}\改行%
\ruby{思}{おも}へて
\ruby{來}{く}る。
%
\ruby{丁度}{ちやう|ど}
それと
\ruby{同}{おな}じ
\ruby{事}{こと}で、
%
\ruby{慣}{な}れて
みると
\ruby[||j>]{嚴}{げん}
\ruby[||j>]{肅}{しゆく}な
% \ruby{嚴肅}{げん|しゆく}な
\ruby{中}{うち}には
\原本頁{14-6}\改行%
\ruby{{\換字{愉}}快}{ゆ|くわい}が
あるから、
%
\ruby{僕}{ぼく}は
まあ
\ruby{不}{ふ}
\ruby{{\換字{愉}}快}{ゆ|くわい}には
\ruby{日}{ひ}を
\ruby{{\換字{送}}}{おく}らん。
』

\原本頁{14-7}%
と
\ruby{答}{こた}へて
\ruby{其}{そ}の
\ruby[<j>]{盃}{さかづき}を
\ruby{乾}{ほ}して
\ruby{洗}{あら}ふ。

\原本頁{14-8}%
『
\ruby{左樣}{さ|う}だ。
%
\ruby{紀律}{き|りつ}を
\ruby[||j>]{{\換字{尊}}}{そん}
\ruby[||j>]{重}{ちやう}する
% \ruby{{\換字{尊}}重}{そん|ちやう}する
\ruby{中}{うち}には
\ruby{{\換字{愉}}快}{ゆ|くわい}が
ある。
%
そして
\ruby{何}{なん}の
\ruby{方面}{はう|めん}の
\ruby{事}{こと}でも
\ruby{紀律}{き|りつ}は
\ruby{大切}{たい|せつ}だ。
%
\ruby{{\換字{船}}}{ふね}の
\ruby{中}{うち}などは
\ruby{特}{こと}に
\ruby{然樣}{さ|う}だ。
%
そればかりぢやあ
\ruby{無}{な}い、
%
\ruby{僕}{ぼく}が
\ruby{私}{ひそか}に
\ruby{思}{おも}ふには、
%
\ruby{身體}{から|だ}を
\ruby{扱}{あつか}ふのに
\ruby{紀律}{き|りつ}が
\ruby{無}{な}いと
\ruby{身體}{から|だ}が
\ruby{衰}{おとろ}へる、
%
\ruby{心}{こゝろ}を
\ruby{扱}{あつか}ふにも
\ruby{紀律}{き|りつ}が
\ruby{無}{な}いと
\ruby{心}{こゝろ}が
\ruby{歪}{ゆが}んで、
%
そこで
\原本頁{15-1}%
\ruby{戀愛}{れん|あい}
などゝいふものに
\ruby{取}{と}り
\ruby{憑}{つ}かれるのだ。
』

\原本頁{15-2}%
と
\ruby{云}{い}ひながら
\ruby{徐}{しづか}に
\ruby{酒盃}{さか|づき}を
\ruby{受}{う}くれば、
%
\ruby{日方}{ひ|かた}は

\原本頁{15-3}%
『
\ruby{確論}{かく|ろん}、
%
\ruby{確論}{かく|ろん}。
』

\原本頁{15-4}%
と
\ruby{悅}{よろこ}び
\ruby{叫}{さけ}んで、
%
\ruby{自}{みづか}ら
\ruby{{\換字{酌}}}{しやく}を
\ruby{仕}{し}て
\ruby{{\換字{遣}}}{や}らんと
\ruby{徳利}{とく|り}を
\ruby{擧}{あ}ぐれば、
%
\ruby{既}{はや}
\ruby{飮}{の}み
\ruby{盡}{つく}して
\ruby{二三滴}{に|さん|てき}のみ。
%
\ruby{山瀬}{やま|せ}は
\ruby{急}{いそ}ぎ
\ruby{手}{て}を
\ruby{拍}{たゝ}き
\ruby{立}{た}つ。

\原本頁{15-6}%
\ruby{此時}{この|とき}まで
にや〳〵と
\ruby{笑}{わら}ひながら、
%
\ruby{人々}{ひと|〴〵}の
\ruby{談}{はなし}を
のみ
\ruby{聞}{き}き
\ruby{居}{ゐ}たりし
\ruby{布袋肥胖}{ほ|てい|ぶ|と}りに
\ruby{肥}{ふと}つたる、
%
\ruby{丸顏}{まる|がほ}の
\ruby{眼下}{め|さが}りなる
\ruby{島木}{しま|き}は
\ruby{笑}{わら}つて、

\原本頁{15-8}%
『
ハヽヽ、
%
\ruby{談話}{はな|し}が
\ruby{惡}{わる}つ
\ruby{固}{かた}いから
\ruby{堪}{たま}りやあ
\ruby{仕無}{し|な}い。
%
\ruby{婢}{をんな}だつて
\ruby{何}{なん}だつて
\ruby{{\換字{逃}}}{に}げたつきりだ。
%
\ruby{徳利}{とつ|くり}の
\ruby{番兵}{ばん|ぺい}は
\ruby{野暮}{や|ぼ}ぢやあ
\ruby{使}{つか}へ
\ruby{無}{ね}えからな
\改行% 校正作業の簡略化のため
。
%
\原本頁{15-10}\改行%
ハヽヽ。
%
\ruby{何}{なん}だい?。
%
\ruby{紀律}{き|りつ}が
\ruby{無}{な}いと
いけ
\ruby{無}{な}いつて?。
%
\ruby[||j>]{戱}{じやう}
\ruby[||j>]{談}{ だん}
% \ruby{戱談}{じやう|だん}
\ruby{言}{ い}つちやあ% 原本に合わせて調整
いけない、
%
\ruby{舞臺}{ぶ|たい}に
\ruby{障}{さは}るぜ。
%
\ruby{不紀律}{ふ|き|りつ}の
\ruby[||j>]{大}{たい}
\ruby[||j>]{將}{しやう}、
% \ruby{大將}{たい|しやう}、
%
\ruby{實業家}{じつ|げふ|か}
\ruby{{\換字{兼}}}{けん}
\ruby{虛業家}{きよ|げふ|か}
\改行% 校正作業の簡略化のため
、
%
\原本頁{16-1}\改行%
\ruby{相場師}{さう|ば|し}に% 原文通り「場」
なつたつて
%
\ruby{一同}{みん|な}に
\ruby{怒}{おこ}られた、
%
\ruby{御利益}{ご|り|やく}は
\ruby{未}{ま}だ
\ruby{蒙}{かうむ}ら
\ruby{無}{な}いが
% \原本頁{16-2}\改行%
\ruby{拜金宗}{はい|きん|しう}の
\ruby{信徒}{しん|と}の、
%
\ruby{島木}{しま|き}
\ruby{萬五郎}{まん|ご|らう}
\ruby{樣}{さま}が
\ruby{此處}{こ|ゝ}に
\ruby{御坐}{お|いで}なさるぜ。
%
\ruby{憚}{はゞか}りながら% 「憚 は(ゞ)か」
\ruby{乃公}{お|れ}が
\ruby{何時}{い|つ}
\ruby{戀愛}{れん|あい}に
\ruby{取}{と}り
\ruby{憑}{つ}かれた。
%
ハヽヽ。
%
\ruby{其}{そ}りやあ
\ruby{左樣}{さ|う}と
\原本頁{16-4}\改行%
\ruby{水野}{みづ|の}の
\ruby{談}{はなし}は
\ruby{譯}{わけ}
\ruby{有}{あ}つて
\ruby{一番}{いち|ばん}
\ruby{乃公}{お|れ}が
\ruby{知}{し}つてる。
%
どうも
\ruby{一同}{みん|な}が
\ruby{氣}{き}に
\ruby{仕}{し}て
\ruby{居}{ゐ}る。
%
\ruby{羽{\換字{勝}}}{は|がち}の
\ruby{腹}{はら}の
\ruby{中}{なか}では
\ruby{取}{と}り
\ruby{{\換字{分}}}{わ}け
\ruby{深}{ふか}く
\ruby{心配}{しん|ぱい}して
\ruby{居}{ゐ}る
やうすだから
\ruby{話}{はな}して
\ruby{聞}{き}かさうか。
』

\原本頁{16-7}%
と、
%
\ruby{始}{はじめ}は
\ruby{戱}{たはむ}れ、
%
\ruby{{\換字{終}}}{をはり}は
\ruby{眞面目}{ま|じ|め}に
\ruby{云}{い}ひ
\ruby{出}{い}づれば、
%
\ruby[||j>]{謹}{きん}
\ruby[||j>]{聽}{ちやう}の
% \ruby{謹聽}{きん|ちやう}の
\ruby{聲}{こゑ}は
\ruby{異口}{い|く}
\ruby{一齊}{いつ|せい}に
\ruby{出}{い}でぬ。

\Entry{其三}

% メモ 校正終了 2024-03-28 2024-06-15
\原本頁{16-10}%
\ruby[g]{島木}{しまき }は
\ruby{驕}{おご}れるにもあらず
\ruby{慢}{あなど}れるにもあらず、
%
たゞ
\換字{志}たゝかなる
\原本頁{17-1}\改行%
\ruby{放肆兒}{だゞ|つ|こ}の、
%
\ruby[g]{一家}{いつか }の
\ruby[||j>]{長}{ちやう}
\ruby[||j>]{者}{ じや}をも
% \ruby{長者}{ちやう|じや}をも
はゞからずして、
%
\ruby[g]{自己}{お の }が
\ruby[g]{{\換字{勝}}手}{かつて }に
\ruby{泣}{な}きも
\ruby{笑}{わら}ひもするやうに、
%
\換字{志}かも
\ruby{其}{そ}の
\ruby[g]{小兒}{こ ども}らしき
\ruby{顏}{かほ}に
\ruby[g]{微笑}{ゑ み }を
うかめ
\改行% 校正作業の簡略化のため
て、

\原本頁{17-4}%
『
ハヽヽ、
%
\ruby[g]{日方}{ひ かた}までが
\ruby[||j>]{謹}{きん}
\ruby[||j>]{聽}{ちやう}と
% \ruby{謹聽}{きん|ちやう}と
\ruby{吐}{ぬ}かし
\ruby{居}{を}つたな!。
%
\ruby[g]{一體}{いつたい}
\ruby{汝}{きさま}は
\ruby{人}{ひと}は
\ruby{好}{い}いが、
%
\ruby{我}{が}ばかり
\ruby{{\換字{強}}}{つよ}くつて
\ruby{思}{おも}ひ
\ruby{{\換字{遣}}}{や}りが
\ruby{足}{た}らない。
%
\ruby{此}{こ}の
\ruby{思}{おも}ひ
\ruby{{\換字{遣}}}{や}りの
\ruby{足}{た}らない
\ruby[g]{手合}{て あひ}が、
%
\ruby[g]{他人}{た にん}の
\ruby[g]{戀愛}{れんあい}の
\ruby{談}{はなし}などには、
%
\ruby[g]{兎角}{と かく}に
\ruby[g]{點頭}{がつてん}
しかねるものだ。
%
\ruby{線}{せん}の
\ruby{無}{な}い
\ruby{家}{うち}にやあ
\ruby[g]{電話}{でんわ }は
\ruby{{\換字{通}}}{つう}じない、
%
\ruby{思}{おも}ひ
\ruby{{\換字{遣}}}{や}りの
\ruby{足}{た}らない
\ruby[g]{奴等}{やつら }にやあ
\ruby[g]{戀愛}{れんあい}は
\ruby{解}{げ}せない。
%
そこへ
\ruby{行}{い}つちやあ
\ruby[g]{乃公}{お れ }
なんぞは、
%
\ruby{身}{み}に
\ruby[g]{經驗}{おぼえ }が
あつて
\ruby[||j>]{同}{おも}
\ruby[||j>]{{\換字{情}}}{ひやり}が
% \ruby{同{\換字{情}}}{おも|ひやり}が
\ruby{{\換字{強}}}{つよ}いから、
%
ツーと
\ruby{云}{い}やあ
カーと
\ruby[g]{合點}{が てん}が
いくので、
%
\ruby[g]{初心}{う ぶ }な
\ruby[g]{水野}{みづの }の
\ruby{譚}{はなし}なんざあ、
%
\ruby[g]{何程}{いくら }
\ruby{彼}{あれ}が
\ruby{心}{こゝろ}の
\ruby{奧}{おく}に
\ruby{秘}{かく}して
\ruby{居}{を}つても、
%
\ruby{深}{ふか}い
\ruby{井}{ゐど}の
\ruby{床}{そこ}を
\ruby{鏡}{かゞみ}で
\ruby{照}{て}らして、
%
\ruby{見}{み}て
\ruby{取}{と}るやうに
\原本頁{18-1}%
\ruby{譯}{わけ}も
\ruby{無}{な}く
\ruby[g]{見拔}{み ぬ }く。
%
\ruby[g]{本來}{ほんらい}
\ruby{戀}{こひ}といふ
\ruby{事}{こと}が
\ruby[g]{罪惡}{つ み }ぢやあ
\ruby{有}{あ}るまいし、
%
\ruby[g]{日方}{ひ かた}のやうな
\ruby[g]{暴論}{ばうろん}の
\ruby[g]{愚論}{ぐ ろん}‥‥』

\原本頁{18-3}%
と
\ruby{云}{い}ひかくる
\ruby{時}{とき}
\ruby[g]{日方}{ひ かた}は
\ruby{堪}{こら}へず、

\原本頁{18-4}%
『
\ruby{何}{なん}だ、
%
\ruby[g]{暴論}{ばうろん}だと!。
%
こりやあ
\ruby{怪}{け}しからん。
%
\ruby{汝}{きさま}も
\ruby[g]{戀愛}{れんあい}の
\ruby[g]{奴隷}{ど れい}
\ruby{臭}{くさ}いぞ。
%
\ruby{身}{み}に
\ruby[g]{經驗}{おぼえ }が
あつてとは
\ruby{何}{なん}たる
\ruby[g]{囈語}{ね ごと}だ。
%
\ruby{聞}{き}き
ぐるしいことを
\ruby{吐}{ぬか}さずとも、
%
さつさと
\ruby[g]{水野}{みづの }の
ことを
\ruby{話}{はな}すが
\ruby{可}{よ}い。
』

\原本頁{18-7}%
と
\ruby[g]{怒鳴}{ど な }り
つくれば、
%
\ruby[g]{此方}{こなた }は% ルビ調整(原本通り)
いよ〳〵
\ruby{笑}{わら}い
\ruby{傾}{かたむ}き、

\原本頁{18-8}%
『
\ruby[g]{安心}{あんしん}しろ
\ruby[g]{日方}{ひ かた}!。
%
\ruby[g]{乃公}{お ら }あ
\ruby[g]{戀愛}{れんあい}の
\ruby[g]{奴隷}{ど れい}にやあ
ならねえ。
%
\ruby[g]{乃公}{お ら }あ
\ruby{女}{をんな}に
\ruby{惚}{ほ}れて
\ruby{戀}{こひ}は
おぼえねえ。
%
ヘン
\ruby{惚}{ほ}れられて
\ruby{惚}{ほ}れられて
\ruby{戀}{こひ}といふものは
\ruby[g]{此樣}{こ ん }なものかと
\ruby{知}{し}つたんだからナ。
%
アハヽヽヽ、
%
\ruby[g]{何樣}{ど う }だい
\ruby{奴}{やつこ}さん、
%
\ruby[g]{如何}{い かゞ}でござる!。
%
そこで
\ruby{惚}{ほ}れられて
\ruby{惚}{ほ}れられて
\ruby{悟}{さと}つて
\ruby{見}{み}ると、
%
\原本頁{19-1}%
\ruby[g]{水野}{みづの }を
\ruby[g]{辯護}{べんご }するといふ% 弁 瓣 辦 辧 辨 辩 (辯)
\ruby{譯}{わけ}ぢやあ
\ruby{無}{な}いが、
%
\ruby{戀}{こひ}は
\ruby[g]{人間}{ひ と }
の
\ruby{{\換字{情}}}{じやう}の
\ruby[g]{自然}{し ぜん}の
\ruby[g]{發動}{うごき }で、
%
\ruby{何}{なに}も
\ruby{咎}{とが}め
\ruby{立}{だて}を
することは
\ruby{有}{あ}りやしない。
%
\原本頁{19-3}\改行%
\ruby[g]{日方}{ひ かた}にやあ
\ruby[g]{日方}{ひ かた}だけの
\ruby[g]{愚論}{ぐ ろん}もあらうが、
%
\ruby[g]{乃公}{お ら }あ
\ruby{戀}{こひ}に
\ruby{{\換字{迷}}}{まよ}つた
\ruby{彼}{あ}の
\ruby[g]{水野}{みづの }を、
%
\ruby[||j>]{憫}{かは}
\ruby[||j>]{然}{いさう}だたあ% 「憫然 か(は)いさう」
% \ruby{憫然}{かは|いさう}だたあ% 「憫然 か(は)いさう」
\ruby{思}{おも}ふが
\ruby{惡}{にく}かあ
\ruby{思}{おも}はねえ。
』

\原本頁{19-5}%
と
\ruby{云}{い}はせも
\ruby{果}{は}てず
\ruby[g]{日方}{ひ かた}は
\ruby{目}{め}を
\ruby{剝}{む}き、

\原本頁{19-6}%
『
\ruby{馬鹿野郎}{ば|か|や|らう}ツ。
』

\原本頁{19-7}%
と
\ruby{烈}{はげ}しく
\ruby{罵}{のゝ}しつたる
\ruby[g]{裂帛}{れつぱく}の
\ruby[g]{一聲}{いつせい}に
\ruby[g]{氣合}{き あひ}
\ruby{籠}{こも}つて、
%
\ruby{人}{ひと}の
\ruby[g]{肺腑}{はいふ }に
\ruby{響}{ひゞ}き
\ruby{徹}{てつ}したり。

\原本頁{19-9}%
『
マア
\ruby{待}{ま}ち
\ruby{玉}{たま}へ。
』

\原本頁{19-10}%
『
\ruby{爭}{あらそ}つちや
いかん。
』

\原本頁{19-11}%
と、
%
\ruby{口}{くち}を
\ruby{衝}{つ}いて
\ruby{出}{い}でたる
\ruby[g]{山瀬}{やませ }
\ruby[g]{羽{\換字{勝}}}{は がち}の
\ruby[g]{二人}{に にん}の
\ruby[g]{言葉}{ことば }は
\ruby[g]{一句}{いつく }と
\ruby[g]{一句}{いつく }と
\原本頁{20-1}%
\ruby{斷}{き}るゝ
\ruby{間}{ひま}
\ruby{無}{な}く
\ruby{巧}{たくみ}に
\ruby{續}{つゞ}きて、
%
\ruby[g]{突差}{とつさ }に
\ruby{緊}{きび}しく
\ruby{制}{せい}し
\ruby{止}{と}むれば、
%
\ruby[g]{流石}{さすが }に
\ruby[g]{日方}{ひ かた}も
\ruby[g]{羽{\換字{勝}}}{は がち}を
\ruby{憚}{はゞか}りて、% 「憚 は(ゞ)か」
%
\ruby{言}{ものい}はんとして
\ruby{言}{い}はず
\ruby{已}{や}みけるが、
%
\ruby{眼}{め}には
\ruby{{\換字{猶}}}{なほ}
\ruby[g]{稜角}{か ど }を
\ruby{立}{た}てゝ
\ruby[g]{島木}{しまき }を
\ruby{睨}{にら}み、
%
\ruby{此}{こ}の
\ruby{時}{とき}
\ruby{遲}{おそ}く
\ruby{彼}{か}の
\ruby{時}{とき}
\ruby{{\換字{速}}}{はや}く、

\原本頁{20-4}%
『
そら
\ruby{{\換字{又}}}{また}
\ruby{馬鹿野郎}{ば|か|や|らう}が
\ruby{御來臨}{お|い|で}なすつた。
%
ハヽヽ、
%
\ruby[g]{何程}{いくら }
\ruby[<j||]{罵}{のゝし}られても% ルビ調整(原本通り)
\ruby[g]{相手}{あひて }には
ならねえ。
%
\ruby{汝}{きさま}は
\ruby[g]{乃公}{お れ }に
\ruby{楯}{たて}をついても、
%
\ruby[g]{乃公}{お ら }あ
\ruby{汝}{きさま}を
\ruby[g]{生呑}{まるのみ}に
\ruby{吞}{の}んでゝ、
%
そして
\ruby{腹}{はら}にも
\ruby{障}{さは}らねえから。
』

\原本頁{20-7}%
と、
%
\ruby[g]{島木}{しまき }の
\ruby{冷}{ひや}やかに
\ruby[g]{一矢}{いつし }
\ruby{酬}{むく}ゆるに、

\原本頁{20-8}%
『
\ruby{何}{なん}だ、
%
\ruby{吞}{の}んで
\ruby{居}{ゐ}る。
%
\ruby[g]{可矣}{よ し }ツ、
%
\ruby{吞}{の}まれたつて
\ruby[g]{鐵釘}{かなくぎ}が
\ruby{何}{なん}となる!。
%
\ruby{曲}{まが}りも
\ruby[g]{仕無}{し な }いは!、
%
\ruby{丸}{まる}くも
ならんは!。
』

\原本頁{20-10}%
と、
%
\ruby[g]{日方}{ひ かた}は
\ruby{{\換字{又}}}{また}
\ruby[||j>]{直}{たゞち}に
\ruby{熱}{ねつ}して
\ruby{答}{こた}ふ。

\原本頁{20-11}%
\ruby[g]{悠然}{いうぜん}と
\ruby{笑}{ゑみ}を
\ruby{含}{ふく}める
\ruby[g]{羽{\換字{勝}}}{は がち}は
\ruby{靜}{しづ}かに、

\原本頁{21-1}%
『
\ruby{可}{い}いさ、
%
\ruby[g]{二人}{ふたり }とも、
%
もう
\ruby{可}{い}いさ。
%
ハヽヽ、
%
\ruby{互}{たがひ}に
\ruby{其}{そ}の
\ruby[||j>]{位}{くらゐ}
\ruby[g]{威張}{ゐ ば }つたら% ルビ調整(原本通り)
\ruby{可}{い}いぢあ
\ruby{無}{な}いか。
%
\ruby[g]{島木}{しまき }は
\ruby[g]{日方}{ひ かた}に
\ruby{關}{かま}はないで
\ruby{僕}{ぼく}に
\ruby{話}{はな}すつもりで
\ruby{話}{はな}して
\ruby{吳}{く}れ
\ruby{玉}{たま}へ。
%
\ruby[g]{日方}{ひ かた}は
また
\ruby[g]{島木}{しまき }に
\ruby{關}{かま}はないで
\ruby{僕}{ぼく}に
\ruby[g]{{\換字{交}}際}{つきあ }つて
\ruby{聞}{きい}て
\ruby{居}{ゐ}て
\ruby{吳}{く}れ
\ruby{玉}{たま}へな。
%
つまり
お
\ruby{互}{たがひ}に
\ruby[g]{水野}{みづの }の
\ruby{上}{うへ}が
\ruby{知}{し}りたいのだからネ。
』

\原本頁{21-6}%
と、
%
\ruby{優}{やさ}しく
\ruby{制}{せい}すれば、

\原本頁{21-7}%
『
ヤ、% 小書きにすると右寄りになるので
%
\ruby{濟}{す}まなかつた、
%
\ruby{僕}{ぼく}が
\ruby{惡}{わる}かつた。
』

\原本頁{21-8}%
『
ア、% 小書きにすると右寄りになるので
%
\ruby[g]{左樣}{さ う }
\ruby{云}{い}はれりやあ
\ruby[g]{乃公}{お れ }も
\ruby{下}{くだ}らなかつた。
』

\原本頁{21-9}%
と、
\ruby[g]{日方}{ひ かた}も
\ruby[g]{島木}{しまき }も
\ruby{爭}{あらそ}ひ
\ruby{止}{や}みて、
%
\ruby{誰}{たれ}
\ruby{勸}{すゝ}めねど
\ruby{同}{おな}じ
\ruby{思}{おも}ひの、
%
\ruby[g]{双方}{さうはう}
\ruby[g]{一時}{いちじ }に
\ruby[g]{酒盃}{さかづき}を
\ruby{{\換字{交}}}{かは}して、
%
\ruby{笑}{わら}つて
\ruby[g]{仕舞}{し ま }つて
\ruby[g]{痕跡}{あとかた}もなし。

\原本頁{21-11}%
\ruby[g]{島木}{しまき }は
\ruby[g]{此度}{こ たび}は
やゝ
\ruby{眞面目}{ま|じ|め}に、
%
\ruby[g]{羽{\換字{勝}}}{は がち}の
\ruby{方}{かた}に
\ruby{向}{むか}つて
\ruby{語}{かた}り
\ruby{出}{だ}したり。

\原本頁{22-1}%
『% この島木の語りは其五の途中まで続く
\ruby[g]{一同}{みんな }も
\ruby{知}{し}つてる
\ruby{{\換字{通}}}{とほ}り
\ruby{彼}{あ}の
\ruby[g]{水野}{みづの }は、
%
\ruby[g]{我等}{おれたち}の
\ruby{中}{なか}では
\ruby[g]{一番}{いちばん}
\ruby[g]{年下}{としした}、
%
\ruby[g]{乃公}{お れ }が
\ruby[g]{今年}{こ とし}は
\ruby{二十七}{に|じふ|しち}だから、% 原本には漢数字「七」のルビ無し
%
\ruby{七}{しち}、% 原本には漢数字「七」のルビ無し
%
\ruby{六}{ろく}、
%
\ruby{五}{ご}、
%
\ruby{四}{よん}と
\ruby{四}{よ}つ
\ruby{目}{め}で
\ruby[g]{丁度}{ちやうど}
\ruby{二十四}{に|じふ|し}だ。% 国会図書館 コマ番号 15/134 p22 l3
%
\ruby{宇都宮}{み| |や}から
\ruby[||j>]{東}{とう}
\ruby[||j>]{京}{きやう}へ
% \ruby{東京}{とう|きやう}へ
\ruby{上}{のぼ}る
\ruby{時}{とき}にも、
%
\ruby[g]{一番}{いちばん}
\ruby{先}{さき}へ
\ruby{出}{で}たのは
\ruby[g]{羽{\換字{勝}}}{は がち}だつたが、
%
\ruby[g]{一番}{いちばん}
\ruby{後}{あと}へ
\ruby{殘}{のこ}つたのは
\ruby[g]{水野}{みづの }だつた。
%
\ruby{{\換字{若}}}{わか}いに
\ruby[g]{似合}{に あ }はず
\ruby{能}{よ}く
\ruby[g]{出來}{で き }たから、
%
\ruby{君}{きみ}は
\ruby{{\換字{若}}}{わか}いけれども
\ruby[g]{學業}{わ ざ }が
\ruby[g]{出來}{で き }る、
%
\ruby{早}{はや}く
\ruby[||j>]{東}{とう}
\ruby[||j>]{京}{きやう}へ
% \ruby{東京}{とう|きやう}へ
\ruby{出}{で}て
\ruby{身}{み}を
\ruby{立}{た}てるが
\ruby{可}{い}いと、
%
\ruby{勸}{すゝ}めたのは
\ruby[g]{乃公}{お れ }
\ruby[g]{一人}{ひとり }で
\ruby{無}{な}かつたが、
%
いや
\ruby[g]{小生}{わたくし}の
\ruby[<j>]{志}{こゝろざ}す
ところは
\ruby{些}{ちと}
\ruby{{\換字{違}}}{ちが}ふから、
%
\ruby[g]{左樣}{さ う }
\ruby{急}{いそ}がないでも
\ruby{可}{い}い
\ruby{事}{こと}だ、
%
\原本頁{22-8}\改行%
\ruby{他}{ほか}の
\ruby{人}{ひと}は
\ruby[g]{一日}{いちにち}
\ruby{遲}{おそ}ければ
\ruby[g]{一日}{いちにち}
\ruby{損}{そん}、
%
\ruby{少}{すこ}しも
\ruby{疾}{はや}く
\ruby[g]{上京}{じやうきやう}
するが
\ruby{可}{い}い、
%
と
\原本頁{22-9}\改行%
\ruby{妙}{めう}に
\ruby{片意地}{かた|い|ぢ}に
\ruby[g]{{\換字{謙}}遜}{けんそん}して
\ruby{出}{で}ず。
%
\ruby[g]{二番}{に ばん}に
\ruby{出}{で}たが
\ruby[g]{日方}{ひ かた}
\ruby[g]{山瀬}{やませ }、
%
それから
\原本頁{22-10}\改行%
\ruby[g]{名倉}{な ぐら}、
%
それから
\ruby[g]{楢井}{ならい }、
%
それから
\ruby[g]{乃公}{お れ }で、
%
\ruby{其}{その}
\ruby{後}{あと}から
\ruby{漸}{やつ}と
\ruby[g]{上京}{じやうきやう}
した
\改行% 校正作業の簡略化のため
。
%
\原本頁{22-11}\改行%
\ruby{其}{そ}の
\ruby[||j>]{位}{くらゐ}
\ruby{異}{ おつ}に
\ruby{固}{かた}いところのある
\ruby{男}{をとこ}で、
%
\ruby[||j>]{東}{とう}
\ruby[||j>]{京}{きやう}へ
% \ruby{東京}{とう|きやう}へ
\ruby{出}{で}てからも
\ruby[g]{一同}{みんな }は
\ruby{誰}{たれ}
\原本頁{23-1}\改行%
しも、
%
\ruby{身}{み}を
\ruby{立}{た}てる
\ruby{{\換字{道}}}{みち}に
\ruby[g]{汲々}{きふ〳〵}として、
%
\ruby[g]{隨{\換字{分}}}{ずゐぶん}
\ruby{骨}{ほね}を
\ruby{折}{を}つて
それ〴〵に
\改行% 校正作業の簡略化のため
、
%
\原本頁{23-2}\改行%
\ruby{辛}{から}く
\ruby[g]{出世}{しゆつせ}も
\ruby{仕}{し}て
\ruby{來}{き}たに、
%
\ruby{彼}{あ}の
\ruby{男}{をとこ}ばかりは
\ruby{澄}{す}ましかへつて、
%
\ruby{今}{いま}でも
\ruby[g]{小學}{せうがく}
\ruby[g]{敎師}{けうし }で
\ruby{甘}{あま}んじて
\ruby{居}{を}る。
%
それで
\ruby{惰}{なま}けて
\ruby{居}{を}るのかと
\ruby{思}{おも}へば、
%
\原本頁{23-4}\改行%
\ruby[g]{一寸}{いつすん}の
\ruby{暇}{ひま}も
\ruby{惜}{をし}んで
\ruby[||j>]{勉}{べん}
\ruby[||j>]{{\換字{強}}}{きやう}
% \ruby{勉{\換字{強}}}{べん|きやう}
して、
%
あらゆる
\ruby[g]{方面}{はうめん}に
\ruby{行}{ゆ}き
\ruby{渡}{わた}つて
\ruby{居}{ゐ}る。
%
\原本頁{23-5}\改行%
\ruby{僕}{ぼく}は
\ruby[||j>]{一}{いつ}
\ruby[||j>]{生}{しやう}を
% \ruby{一生}{いつ|しやう}を
かけて
\ruby{此}{こ}の
\ruby{世}{よ}の
\ruby{中}{なか}に、
%
たゞ
\ruby[g]{一篇}{いつぺん}の
\ruby{詩}{し}を
\ruby{{\換字{留}}}{とゞ}めれば
\ruby{可}{い}いのだ。
%
\ruby{今}{いま}は
\ruby{其}{そ}の
\ruby[g]{準備}{ようい }に
\ruby{{\換字{勤}}}{つと}めて
\ruby{居}{ゐ}るので、
%
\ruby{他}{ほか}に
\ruby{慾}{よく}も
\ruby{無}{な}ければ
\ruby{望}{のぞみ}も
\原本頁{23-8}\改行%
\ruby{無}{な}い、
%
\ruby[g]{{\換字{半}}熟}{なまにえ}なものを
\ruby{世}{よ}に
\ruby{出}{だ}して、
%
\ruby{今}{いま}つから
\ruby[g]{{\換字{文}}人}{ぶんじん}
\ruby{顏}{がほ}するのも
\ruby{羞}{はづ}か
しいから、
%
もう
\ruby[g]{十年}{じふねん}ばかりは
\ruby{小學讀本}{と|く|ほ|ん}
いぢりで、
%
たゞ〳〵
\makeatletter
\@ifundefined{デバッグ@ビルド}{%
  \ruby[||j>]{勉}{べん }
  \ruby[||j>]{{\換字{強}}}{きやう}
}{%
  \ruby[<j||]{勉}{べん }
  \ruby[<j||]{{\換字{強}}}{きやう}
}%
\makeatother
% \ruby{勉{\換字{強}}}{べん|きやう}
\原本頁{23-9}\改行%
を
するつもりだ、
%
と
\ruby{隱君子}{いん|くん|し}
\ruby[g]{氣質}{かたぎ }で
\ruby{日}{ひ}を
\ruby{經}{へ}て
\ruby{居}{ゐ}たのは、
%
\ruby[g]{羽{\換字{勝}}}{は がち}はじめ
\ruby[g]{一同}{みんな }も
\ruby{知}{し}つて
\ruby{居}{ゐ}やう。
%
ところで
\ruby{此}{こ}の
\ruby[g]{乃公}{お れ }は
\ruby{金}{かね}まうけ
\ruby[g]{主義}{しゆぎ }、
%
\ruby{卑}{いや}しいと
\ruby{云}{い}つて
\ruby[g]{一同}{みんな }に
\ruby{罵}{のゝし}られた
\ruby{位}{くらゐ}だから、
%
\ruby{守}{まも}るところのある
\ruby[g]{浪人}{らうにん}
\原本頁{24-1}\改行%
\ruby{肌}{はだ}の、
%
\ruby[g]{水野}{みづの }と
\ruby{氣}{き}の
\ruby{合}{あ}ふ
\ruby{譯}{わけ}は
\ruby{毫}{ちつと}も
\ruby{無}{な}いが、
%
\ruby{他}{ほか}の
\ruby[g]{五人}{ご にん}は
\ruby[g]{上京}{じやうきやう}
して、
%
\原本頁{24-2}\改行%
\ruby[g]{二人}{ふたり }だけ
\ruby{宮}{みや}に
\ruby{殘}{のこ}つた
\ruby{時}{とき}、
%
\ruby{彼}{あれ}が
\ruby{熱}{ねつ}を
\ruby{病}{や}んだのを
\ruby[g]{介抱}{かいはう}して、
%
\ruby{長}{なが}い
\ruby[g]{看護}{み とり}を
\ruby{爲}{し}て
\ruby{{\換字{遣}}}{や}つた、
%
\ruby[g]{其事}{そ れ }が
\ruby{{\換字{鎖}}}{くさり}になつて
\ruby[g]{此地}{こつち }へ
\ruby{來}{き}ても、
%
\ruby{取}{と}り
\ruby{{\換字{分}}}{わ}け
\ruby[g]{二人}{ふたり }は
\ruby{親}{した}しく
\ruby{仕}{し}て
\ruby{居}{ゐ}た。
%
\換字{志}かし
\ruby[g]{乃公}{お ら }あ
\ruby[g]{俗物}{ぞくぶつ}、
%
\ruby[g]{水野}{みづの }は
\ruby[g]{仙骨}{せんこつ}、
%
\ruby[g]{此方}{こつち }は% ルビ調整(原本通り)
\原本頁{24-5}\改行%
\ruby{飛}{と}んだり
\ruby{跳}{はね}たりして
\ruby[g]{悶躁}{も が }いて
\ruby{居}{ゐ}るので、
%
\ruby[g]{中々}{なか〳〵}
\ruby[g]{往來}{ゆきき }することも
\ruby{多}{おほ}くは
\ruby{無}{な}かつた。
%
さあ
\ruby[g]{此處}{こ ゝ }で
\ruby[||j>]{白}{はく}
\ruby[||j>]{狀}{じやう}
% \ruby{白狀}{はく|じやう}
\ruby[||j>]{仕}{ し}なけりや
ならないが、
%
\ruby[g]{丁度}{ちやうど}
\makeatletter
\@ifundefined{デバッグ@ビルド}{%
  \ruby[|g|]{一昨年}{をとゝし}の
}{%
  \ruby{一昨年}{を|とゝ|し}の
}%
\makeatother
\ruby{暮}{くれ}だつた。
%
\ruby{實}{じつ}は
\ruby{此}{こ}の
\ruby[g]{乃公}{お れ }が
\ruby[g]{山氣}{やまぎ }に
\ruby{逸}{はや}つて、
%
\ruby{危}{あぶな}い
\ruby{橋}{はし}を
\ruby{渡}{わた}る
\原本頁{24-8}\改行%
\ruby[g]{輕業}{かるわざ}をやつたところ、
%
\ruby{{\換字{運}}}{うん}が
\ruby{惡}{わる}くつて
\ruby[g]{可厭}{い や }な
\ruby{目}{め}が
\ruby{出}{で}て、
%
\ruby{甘}{うま}く
\ruby{行}{い}きあ
\ruby{論}{ろん}は
\ruby{無}{な}いことが
\ruby[g]{打壞}{ぶつこわ}れたんで、
%
たつた
\ruby[g]{五十}{ご じふ}
\ruby{兩}{りやう}ばかりの
\ruby[g]{有無}{あるなし}で
\原本頁{24-10}\改行%
\ruby[g]{何樣}{ど う }にも
\ruby[g]{仕切}{し き }れない
\ruby[g]{機會}{は め }へ
\ruby{臨}{のぞ}んだ。
%
そも〳〵
\ruby[g]{投機}{や ま }を
\ruby{始}{はじ}めた
\ruby{其}{そ}の
\原本頁{24-11}\改行%
\ruby{時}{とき}から、
%
\ruby[g]{乃公}{お ら }あ
\ruby{危}{あぶな}い
\ruby{事}{こと}をする
\ruby{代}{かは}りにやあ、
%
\ruby[g]{乃公}{お れ }が
\ruby[g]{一六}{いちろく}% ばくち・双六 (すごろく) で二つの賽 (さい) を振って、その目に一と六とが同時に出ること
\ruby[g]{沙汰}{ざ た }を
\ruby{廢}{や}めぬ
\ruby{内}{うち}は、
%
\原本頁{25-1}%
\ruby[g]{金錢}{きんせん}に
\ruby{關}{かゝ}つた
\ruby{事}{こと}では
\ruby{決}{けつ}して
\ruby[g]{一同}{みんな }に、
%
\ruby[g]{苦勞}{く らう}は
\ruby{掛}{か}けぬと
\原本頁{25-2}\改行%
\ruby[g]{誓言}{ちかひ }を
\ruby{立}{た}つた
\ruby{表}{おもて}が
あるから
\ruby{誰}{たれ}にも
\ruby{云}{い}へず、
%
\ruby[g]{思案}{し あん}に
\ruby{餘}{あま}つて
\ruby[||j>]{獨}{ひとり}
\ruby[||j>]{語}{ ごと}のやうに、
%
\ruby{其}{その}
\ruby{譯}{わけ}を
\ruby[g]{水野}{みづの }に
\ruby{話}{はな}して
\ruby{見}{み}ると、
%
\ruby[g]{手箱}{て ばこ}の
\ruby{底}{そこ}から
\ruby{書}{か}いたものを
\ruby{出}{だ}して、
%
\ruby{此}{これ}を
\ruby[g]{山瀬}{やませ }
\ruby{君}{くん}に
\ruby{頼}{たの}んで
\ruby{賣}{う}つて
\ruby{貰}{もら}つたら、
%
\ruby[||j>]{其}{その}
\ruby[||j>]{位}{くらゐ}の
% \ruby{其位}{その|くらゐ}の
\ruby{金}{かね}は
\ruby[g]{出來}{で き }るか
\ruby{知}{し}れぬ、
%
\ruby[g]{出來}{で き }たら
\ruby{使}{つか}ひ
\ruby{玉}{たま}へ
といふ
\ruby{話}{はなし}。
%
\ruby{當}{あて}には
ならないと
\原本頁{25-6}\改行%
\ruby{思}{おも}つたが、
%
\ruby[g]{山瀬}{やませ }に
\ruby{頼}{たの}むと
\ruby[g]{其事}{そ れ }が
\ruby[g]{出來}{で き }て、
%
そこで
\ruby{大}{おほき}に
\ruby{助}{たす}かつた。
%
\原本頁{25-7}\改行%
\ruby{其}{そ}の
\ruby{味}{あぢ}を
\ruby{占}{し}めた
といふのでは
\ruby{無}{な}いが、
%
\ruby{其}{そ}の
\ruby{後}{のち}も
\ruby[g]{種子}{た ね }を
\ruby{耗}{す}つた
\ruby{其}{その}
\原本頁{25-8}\改行%
\ruby{時}{とき}は、
%
\ruby[g]{三度}{さんど }といふもの
\ruby{助}{たす}けて
\ruby{貰}{もら}つて、
%
\ruby[g]{矢種}{や だね}を
つぎ〳〵
\ruby{戰}{たゝか}つた
\ruby{末}{すゑ}
\改行% 校正作業の簡略化のため
、
%
\原本頁{25-9}\改行%
どうやら
\ruby{{\換字{遣}}}{や}つて
\ruby{行}{い}かれる
\ruby[g]{身體}{からだ }になつた。
%
そこで
\ruby[g]{水野}{みづの }に
\ruby{對}{むか}つて
\ruby[g]{乃公}{お れ }が
いふには、
%
\ruby{貰}{もら}つたものを
\ruby{{\換字{返}}}{かへ}さうとは
\ruby{云}{い}はないが、
%
\ruby{金}{かね}が
\ruby{要}{い}る
\原本頁{25-11}\改行%
\ruby{時}{とき}は
\ruby[g]{何時}{い つ }でも
\ruby{云}{い}ひたまへ、
%
\ruby[g]{乃公}{お れ }が
\ruby[g]{懷中}{ふところ}だけなら
\ruby{洗}{さら}け
\ruby{出}{だ}すから、
%
\原本頁{26-1}\改行%
と
\ruby{此}{こ}の
\ruby{春}{はる}
\ruby{{\換字{遇}}}{あ}つた
\ruby{時}{とき}
\ruby{云}{い}つて
\ruby{置}{お}いた。
%
ところが
\ruby{金}{かね}を
\ruby{使}{つか}ふ
\ruby[g]{水野}{みづの }では
\ruby{無}{な}し、
%
たゞ
\ruby[g]{其限}{それぎり}で
\ruby{濟}{す}んで
\ruby{居}{ゐ}たが、
%
\ruby{此}{こ}の
\ruby{夏}{なつ}になつて
\ruby{{\換字{遣}}}{や}つて
\ruby{來}{き}て、
%
\ruby[g]{眞赤}{まつか }な
\ruby{顏}{かほ}をして
きまり
\ruby{惡}{わる}さうに、
%
\ruby[g]{三十}{さんじふ}
\ruby{兩}{りやう}
ばかり
\ruby{貸}{か}して
\ruby{吳}{く}れろ、
%
と
\原本頁{26-4}\改行%
\ruby{云}{い}つたのが
\ruby[g]{最初}{はじまり}で
\ruby{其}{その}
\ruby{後}{のち}も、% ルビ調整(原本通り)非踊り字表記
%
ぼつり〳〵と
\ruby{持}{も}つて
\ruby{行}{ゆ}く。
%
\ruby[g]{其事}{そ れ }が
\ruby[g]{乃公}{お れ }が
\ruby{勘}{かん}を
\ruby{付}{つ}けた
はじまりだつた。

\Entry{其四}

% メモ 校正終了 2024-03-30 2024-05-22 2024-06-15
\原本頁{26-7}%
\ruby[||j>]{考}{かんが}へて
\ruby{見}{み}りやあ
\ruby[g]{合點}{が てん}が
いかない。
%
\ruby[g]{多{\換字{分}}}{たんと }では
\ruby{無}{な}いが
\ruby[g]{給料}{きふれう}も
\ruby{取}{と}るし
\改行% 校正作業の簡略化のため
、
%
\原本頁{26-8}\改行%
\ruby{別}{べつ}に
\ruby[g]{蕩樂}{だうらく}の
\ruby{無}{な}い
\ruby{男}{をとこ}だから
\ruby[g]{其金}{そ れ }で
\ruby{一人身}{ひと|り|み}の
\ruby[g]{今日}{こんにち}を
\ruby{濟}{す}ませて、
%
\ruby[g]{剩餘}{あまり }で
\ruby[g]{書物}{しよもつ}を
\ruby{買}{か}つて
\ruby{讀}{よ}む
\ruby{位}{くらゐ}の
\ruby{事}{こと}。
%
その
\ruby[g]{書物}{しよもつ}を
\ruby{買}{か}ふにも
たゞは
\ruby{買}{か}はないで、
%
\ruby[g]{何時}{い つ }でも
\ruby{讀}{よ}んで
\ruby{了}{しま}つたのを
\ruby{下}{した}に
\ruby{{\換字{遣}}}{や}つて、
%
まだ
\ruby{讀}{よ}まぬもの
\原本頁{27-1}\改行%
と
\ruby[g]{取換}{とりか }へる。
%
それを
\ruby[g]{自{\換字{分}}}{じ ぶん}でも
\ruby[g]{可笑}{をかし }がつて、
%
\ruby{何}{なん}の
\ruby{事}{こと}は
\ruby{無}{な}い
\ruby{僕}{ぼく}の
\ruby{爲}{す}ることは
\ruby[g]{書肆}{ほんや }のために、
%
\ruby[g]{一枚}{いちまい}
\ruby[g]{一枚}{いちまい}
\ruby[g]{蟲拂}{むしはら}ひを
\ruby[g]{叮嚀}{ていねい}に
\ruby{仕}{し}て
\ruby{{\換字{遣}}}{や}る
やうなものだと
\ruby{云}{い}つて
\ruby{居}{ゐ}た
\ruby{程}{ほど}。
%
\ruby{併}{しか}し
\ruby[g]{左樣}{さ う }いふ
\ruby{{\換字{遣}}}{や}り
\ruby{方}{かた}を
して
\ruby{少}{すくな}い
\ruby{錢}{ぜに}で
\原本頁{27-4}\改行%
\ruby{多}{おほ}く
\ruby{讀}{よ}む、
%
それだけ
\ruby[g]{始末}{し まつ}の
\ruby{好}{い}い
\ruby{賢}{かしこ}い
\ruby[g]{水野}{みづの }が、
%
\ruby{何}{なん}の
\ruby{彼}{か}のと
\ruby{云}{い}つては
\ruby{金}{かね}を
\ruby{持}{も}つて
\ruby{行}{ゆ}く。
%
ハテ
\ruby{是}{これ}にやあ
\ruby{何}{なん}ぞ
\ruby[g]{仔細}{し さい}があらう、
%
\ruby{譯}{わけ}が
\ruby{無}{な}くちやあ
\ruby{要}{い}らない
\ruby{金}{かね}だ。
%
いくら
\ruby[g]{表面}{うはべ }は
\ruby[g]{物柔}{ものやは}らかな
\ruby{君子風}{くん|し|ふう}で、
%
\ruby{腹}{はら}の
\原本頁{27-7}\改行%
\ruby{底}{そこ}の
\ruby{底}{そこ}にやあ
\ruby{恐}{おそろ}ろしい% ルビ調整(原本通り)「おそろ」
\ruby[g]{高慢}{かうまん}、
%
\ruby{世界中}{せ|かい|ぢゆう}の
\ruby{奴}{やつ}を
\ruby[g]{相手}{あひて }にしても、
%
\ruby{鼻}{はな}の
\原本頁{27-8}\改行%
\ruby{頭}{さき}で
\ruby{笑}{わら}つて
\ruby{居}{ゐ}やうといふ
\ruby{沈毅{\換字{漢}}}{しつ|かり|もの}の、
%
\ruby{彼}{あ}の
\ruby[g]{水野}{みづの }でも、
%
\ruby[g]{年齡}{と し }は
\ruby[g]{年齡}{と し }だ。
%
\ruby{桃}{もゝ}の
\ruby{{\換字{速}}}{はや}いのも
\ruby{柹}{かき}の
\ruby{遲}{おそ}いのも、
%
いづれ
\ruby{時}{とき}が
\ruby{來}{く}りやあ
\ruby{花}{はな}は
\ruby{{\換字{咲}}}{さ}き
\原本頁{27-10}\改行%
\ruby{出}{だ}す。
%
\ruby{才}{さい}
はじけたも
\ruby{謹}{つゝ}しまやか
\footnote{%
検討対象の「\ruby{謹}{つゝ}しまやか」とルビを振っている件について、
「謹」の読みの一つに「つつしむ」があるので原本通りとする
(国会図書館 コマ番号17/134 p-27 l-10)}%
なも、
%
\ruby[g]{時{\換字{節}}}{じ せつ}
\ruby[g]{因緣}{いんねん}で
\ruby{{\換字{情}}}{こゝろ}が
\ruby{萌}{も}える。
%
\原本頁{27-11}\改行%
\ruby[g]{乃公}{お れ }のやうな
\ruby[g]{早熟}{はやなり}やあ
\ruby{十七八}{じふ|しち|はち}から、% 原本には漢数字「七」のルビ無し
%
\ruby[g]{白{\換字{粉}}}{おしろい}や
\ruby{油}{あぶら}の
\ruby{香}{にほひ}に
\ruby{鼻}{はな}も
ぴこつかせたが、
%
%ひこつく  ... ひくひく動く。主に鼻についていう。
%びこつかす ... 小刻みに動かす。ちょっちょっと動かす。
%びこつく  ... りきむ。虚勢を張る。ぴこつく。
\原本頁{28-1}%
\ruby{其}{その}
\ruby{代}{かは}り
\ruby[g]{{\換字{浮}}氣}{うはき }の
\ruby{掛}{か}け
\ruby{流}{なが}しで、
%
\ruby{笑}{わら}ふのも
\ruby{泣}{な}くのも
\ruby[g]{二日}{ふつか }か
\原本頁{28-2}\改行%
\ruby[g]{三日}{みつか }
\ruby{限}{き}り、
%
\ruby{思}{おも}ふも
\ruby{思}{おも}はれるも
\ruby{實}{じつ}は
\ruby{無}{な}くつて、
%
のほゝんで
\ruby[g]{今日}{け ふ }まで
\ruby[g]{無事}{ぶ じ }に
\ruby{來}{き}たが、
%
\ruby[g]{水野}{みづの }のやうな
\ruby[g]{彼樣}{あ ん }な
\ruby{男}{をとこ}が、
%
\ruby{惡}{わる}くすると
\ruby{唯}{たゞ}
\ruby[g]{一{\換字{途}}}{いちづ }に
\ruby[||j>]{純}{いつ}
\ruby[||j>]{粹}{ぽんぎ}の、
% \ruby{純粹}{いつ|ぽんぎ}の、
%
\ruby{眞正直}{まつ|しやう|ぢき}な
\ruby{戀}{こひ}に
\ruby{落}{お}ちて、
%
\ruby{人}{ひと}にも
\ruby{知}{し}らさず
\ruby{獨}{ひと}り
\ruby{苦}{くる}しみ、
%
\原本頁{28-5}\改行%
\ruby{思}{おも}ひ
\ruby{詰}{つ}め
\ruby{思}{おも}ひ
\ruby{詰}{つ}めて
\ruby{忘}{わす}れる
\ruby{間}{ま}も
\ruby{無}{な}く、
%
\ruby{胸}{むね}に
\ruby{解}{と}けかねる
\ruby[g]{凝塊}{し こり}を
\ruby{出}{で}かして、
%
\ruby{長}{なが}く
〳〵
\ruby{悶}{もだ}へて
\ruby{惱}{なや}むとも
あるもの。
%
\ruby{{\換字{若}}}{もし}や
\ruby[g]{其樣}{そ ん }な
\ruby{事}{こと}でゞ
\原本頁{28-7}\改行%
もあるならば、
%
\ruby[g]{朋友}{ともだち}の
よしみ、
%
\ruby[g]{年上}{としうへ}の
\ruby[g]{甲{\換字{斐}}}{か ひ }、
%
\ruby{特}{こと}には
\ruby{誰}{たれ}にも
\ruby{知}{し}らさず
\ruby[g]{内々}{ない〳〵}で、
%
\ruby{恩}{おん}を
\ruby{受}{う}けて
\ruby{居}{ゐ}る
\ruby[g]{譯合}{わけあひ}もあり、
%
\ruby{一}{ひ}ト
\ruby[g]{心配}{しんぱい}
\ruby[g]{仕無}{し な }けりやあ
ならぬと
\ruby{意}{こゝろ}を
\ruby{定}{さだ}めて、
%
さて
\ruby[g]{其時}{そ れ }から
\ruby[g]{水野}{みづの }の
\ruby[g]{樣子}{やうす }を
\ruby{見}{み}ると
\makeatletter
\@ifundefined{デバッグ@ビルド}{%
  \ruby[||j>]{推}{すゐ }
  \ruby[||j>]{量}{りやう}
}{%
  \ruby[<j||]{推}{すゐ }
  \ruby[<j||]{量}{りやう}% 行末行頭の境界付近なので特例処置を施す
}%
\makeatother
% \ruby{推量}{すゐ|りやう}
の
\ruby{{\換字{通}}}{とほ}り。
%
\ruby{何}{なん}と
\ruby{無}{な}く
\ruby{人}{ひと}に
\ruby[<j||]{隔}{へだて}
\ruby[||j>]{心}{ごゝろ}
がある。
%
\ruby{何}{なん}と
\ruby{無}{な}く
そは〳〵としたところがある。
%
\ruby[g]{此方}{こつち }から% ルビ調整(原本通り)
\ruby{話}{はな}す
\ruby{談}{はなし}には
\ruby{身}{み}を
\ruby{入}{い}れて
\ruby{聞}{き}かぬ。
%
\ruby{彼}{あれ}が
\ruby{話}{はな}す
\原本頁{29-1}\改行%
\ruby{談}{はなし}には
\ruby[g]{氣焰}{いきほひ}が
\ruby{足}{た}らぬ。
%
\ruby{人}{ひと}と
\ruby{對}{むか}ひあつて
\ruby{坐}{すわ}つて
\ruby{居}{ゐ}ながら、
%
\ruby[g]{談話}{はなし }が
\原本頁{29-2}\改行%
\ruby[g]{一寸}{ちよつと}
\ruby{斷}{た}えれば
\ruby{胸}{むね}の
\ruby{中}{なか}では、
%
\ruby{既}{もう}
\ruby[g]{他方}{よ そ }の
\ruby{事}{こと}を
\ruby{思}{おも}つて
\ruby{居}{ゐ}る
\ruby[g]{樣子}{やうす }。
%
\ruby[g]{將來}{ゆくすゑ}の
\ruby[g]{希望}{の ぞみ}は
\ruby{餘}{あま}り
\ruby{言}{い}はずに、
%
やゝもすると
\ruby{{\換字{過}}}{す}ぎた
\ruby{事}{こと}を
\ruby{云}{い}ひ
\ruby{出}{だ}しては
\改行% 校正作業の簡略化のため
、
%
\原本頁{29-4}\改行%
\ruby{無邪氣}{む|じや|き}だつた
\ruby[g]{往時}{むかし }を
なつかしがる。
%
\ruby{試}{こゝろ}みに
\ruby{{\換字{浮}}世話}{うき|よ|ばなし}を
\ruby[g]{三種}{み いろ}
\ruby[g]{四種}{よ いろ}
\ruby{爲}{し}て、
%
\ruby{何}{ど}の
\ruby{話}{はなし}が
\ruby{彼}{あれ}の
\ruby{胸}{むね}の
\ruby{中}{うち}と
\ruby{響}{ひゞ}き
\ruby{合}{あ}ふかと、
%
\ruby{探}{さぐ}つて
\ruby{見}{み}れば
\ruby[g]{全然}{すつかり}
\ruby{{\換字{分}}}{わか}つて、
%
\ruby{此}{こ}の
\ruby{絃}{いと}に
\ruby{和}{あ}つて
\ruby{鳴}{な}るのは
\ruby{其}{そ}の
\ruby{絃}{いと}と、
%
\ruby[g]{{\換字{判}}然}{ちやん }と
\ruby[||j>]{正}{しやう}
\ruby[||j>]{體}{ たい}の
% \ruby{正體}{しやう|たい}の
\ruby[g]{合點}{が てん}が
\原本頁{29-7}\改行%
いつた。
%
さあ
\ruby[g]{打棄}{うつちや}つて
\ruby{置}{お}く
\ruby{譯}{わけ}にやあ
\ruby{行}{い}かない。
%
\ruby[g]{相手}{あひて }さへ
\ruby{好}{よ}けりやあ
\ruby[g]{仔細}{し さい}は
\ruby{無}{な}いこと。
%
\ruby[g]{南方}{みなみ }へ
\ruby{枝}{えだ}が
さして
\ruby{花}{はな}が
\ruby{{\換字{咲}}}{さ}くに
\ruby{何}{なん}の
\ruby{罪}{つみ}!。
%
\原本頁{29-9}\改行%
\ruby[g]{人{\換字{情}}}{じやう }の%ここは「にんじやう」でなく原本通り「じやう」
\ruby[||j>]{溫}{あつた}
\ruby[||j>]{{\換字{暖}}}{ かみ}を
% \ruby{溫{\換字{暖}}}{あつた|かみ}を
\ruby{得}{え}やうと
おもつて、
%
\ruby{{\換字{若}}}{わか}い
\ruby{心}{こゝろ}の
\ruby{動}{うご}き
\ruby{出}{だ}すのが
\ruby{何}{なに}
\ruby[g]{無理}{む り }だらう!。
%
\ruby[g]{年齡}{と し }が
\ruby[g]{年齡}{と し }だもの、
%
\ruby{有}{あ}り
\ruby{内}{うち}の
\ruby{事}{こと}だ。
%
\ruby{然}{しか}し
\ruby{緣}{えん}は
\ruby{異}{い}なもの
\ruby{危}{あぶな}いもの、
%
よもやとは
\ruby{思}{おも}ふけれど、
%
\ruby{萬}{まん}が
\ruby{一}{いち}にも、
%
\ruby[g]{素性}{すじやう}や
\ruby{筋}{すぢ}の
\原本頁{30-1}\改行%
\ruby{惡}{わる}い
\ruby{女}{をんな}が
\ruby[g]{相手}{あひて }だつた
\ruby{日}{ひ}には
\ruby[g]{水野}{みづの }の
\ruby[g]{不幸}{ふ かう}、
%
\ruby{止}{と}め
\ruby{立}{だて}も
\ruby{爭}{あらそ}ひ
\ruby{立}{だて}も
\ruby[g]{仕無}{し な }けりや
ならぬ。
%
\ruby{金}{かね}の
\ruby{要}{い}るだけに
\ruby{氣}{き}がゝりな
ところがある。
%
と
\ruby{思}{おも}つたので
\ruby[g]{乃公}{お れ }の
\ruby[g]{身體}{からだ }にやあ
\ruby{暇}{ひま}も
\ruby{無}{な}かつたが、
%
\ruby[g]{或日}{あるひ }
\ruby[g]{水野}{みづの }の
\ruby[g]{不在}{る す }を
\原本頁{30-4}\改行%
\ruby{覗}{ねら}つて、
%
\ruby[g]{水野}{みづの }を
\ruby{置}{お}いて
\ruby[g]{世話}{せ わ }をして
\ruby{居}{ゐ}る
\ruby[g]{山路}{やまぢ }の
\ruby[g]{老夫}{おやぢ }を
\ruby{捕}{つかま}へて
\ruby{糺}{たゞ}しかけると、
%
\ruby{彼}{あ}の
\ruby[g]{老夫}{おやぢ }も
\ruby[g]{中々}{なか〳〵}の
\ruby{親切者}{しん|せつ|もの}で、
%
\ruby{特}{こと}さら
\ruby[g]{水野}{みづの }の
\ruby[g]{{\換字{平}}生}{ひ ごろ}の% ルビ調整(原本通り)
\ruby[g]{品行}{み もち}に
\ruby{惚}{ほ}れて
\ruby{居}{ゐ}るので、
%
\ruby{實}{じつ}は
\ruby[g]{水野}{みづの }
\ruby{樣}{さん}の
\ruby{御利益}{お|た|め}を
\ruby{思}{おも}つて、
%
\ruby[g]{貴下}{あなた }でも
\原本頁{30-7}\改行%
\ruby{御來臨}{お|い|で}になつたら
\ruby{申}{まを}し
\ruby{上}{あ}げたいと、
%
\ruby[g]{内々}{ない〳〵}
\ruby{願}{ねが}つて
\ruby{居}{ゐ}た
ところでござりました、
%
といふので
\ruby[g]{一切}{いつさい}の
\ruby[g]{事{\換字{情}}}{じじやう}は
\ruby[g]{老夫}{おやぢ }の
\ruby{口}{くち}から
\ruby{知}{し}れた。

\Entry{其五}
% メモ 校正終わり 2024-3-16

\原本頁{30-10}
\ruby{老夫}{おや|ぢ}の
\ruby{談話}{はな|し}を
\ruby{聞}{き}いて
\ruby{見}{み}りやあ
\ruby[g]{水野}{みづの}は
\ruby{實}{じつ}に
\ruby{憫然}{かは|いさう}だ。% 「憫然 か(は)いさう」
%
\ruby{勿論}{もち|ろん}
\ruby{其}{そ}の
\ruby{老夫}{おや|ぢ}の
\原本頁{31-1}
\ruby{云}{い}つたことが
\ruby{一}{いち}から
\ruby{十}{じう}まで
\ruby[g]{眞實}{ほんと}とも
\ruby{限}{かぎ}るまいが、
%
\ruby{岡目}{をか|め}の
\ruby{{\換字{評}}{\換字{判}}}{ひやう|ばん}なり
\ruby{老夫}{とし|より}の
\ruby{言葉}{こと|ば}なり、
%
\ruby{大體}{だい|たい}は
\ruby{{\換字{違}}}{ちが}ふ
\ruby{氣{\換字{遣}}}{きづ|かひ}は
あるまい。
%
そも〳〵は
\ruby{今年}{こ|とし}の
\ruby{春}{はる}の
\ruby{始}{はじめ}、
%
\ruby[g]{水野}{みづの}の
\ruby{出}{で}て
\ruby{居}{ゐ}る
\ruby{學校}{がく|かう}の
\ruby{女敎師}{ぢよ|けう|し}が
\ruby{一人}{ひと|り}
\ruby{故鄕}{く|に}へ
\ruby{歸}{かへ}つたので
\ruby{闕員}{けつ|いん}が
\ruby{出來}{で|き}た、
%
\ruby{其}{そ}の
\ruby{補闕}{ほ|けつ}として
\ruby{新}{あらた}に
\ruby{來}{き}たのが、
%
まだ
\ruby{敎員}{けう|ゐん}になりたての、
%
\ruby{年}{とし}の
\ruby{{\換字{若}}}{わか}い
\ruby{岩崎}{いは|さき}% 原本のこの部分は「いわさき」
\ruby[g]{五十子}{いそこ}といふ
\ruby{女}{をんな}だつた。
%
\ruby{老夫}{おや|ぢ}も
\ruby{度々}{たび|〳〵}
\ruby{見}{み}て
\ruby{知}{し}つているさうだが、
%
\ruby{極}{ごく}
\ruby{可愛}{か|あい}らしい
\ruby{惚}{ほ}れ〴〵すると
いふやうな
\ruby{顏立}{かほ|だち}では
\ruby{無}{な}いけれど、
%
\ruby{眼}{め}の
\ruby{淸}{すゞ}しい
\ruby{鼻}{はな}の
\ruby{高}{たか}い
\ruby{端然}{しや|ん}とした
\ruby{女}{をんな}で、
%
まあ
\ruby{當世}{たう|せい}の
\ruby{下司}{げ|す}
\ruby{根性}{こん|じやう}から
\ruby{云}{い}へば、
%
あれだけの
\ruby{容貌}{きり|やう}をもつて
\ruby{居}{ゐ}ながら、
%
\ruby{何}{なん}だつて
\ruby{敎師}{けう|し}
なんぞになつて
\ruby{居}{ゐ}るだらう、
%
と
\ruby{蔭口}{かげ|ぐち}も
\ruby{云}{い}はれ
\ruby{{\換字{兼}}}{かね}ない
\ruby{女}{をんな}ぶり
ださうさ。
%
\換字{志}かも、
%
\ruby{容貌}{きり|やう}の
\ruby{佳}{い}い
\ruby{奴}{やつ}は
\ruby{十人}{じう|にん}が
\ruby{八人}{はち|にん}まで、
%
\ruby{兎角}{と|かく}
\ruby{他人}{た|にん}に
\ruby{甘}{あま}つたれるやうな
\ruby{調子}{てう|し}が
あつて、
%
\ruby{學問}{がく|もん}なぞは
\ruby{得}{え}て
\原本頁{32-1}
\ruby{出來}{で|き}ないが、
%
\ruby{中々}{なか|〳〵}
\ruby{其女}{その|をんな}は
\ruby{能}{よ}く
\ruby{出來}{で|き}る
\ruby{上}{うへ}、
%
それこそ
\ruby[g]{日方}{ひかた}の
\ruby{云}{い}ひ
\ruby{草}{ぐさ}ぢやあ
\ruby{無}{な}いが、
%
いつでも
\ruby{現在}{げん|ざい}に
\ruby{滿足}{まん|ぞく}しないで、
%
\ruby{永久}{えい|きう}に
\ruby{{\換字{進}}}{すゝ}んで
\ruby{{\換字{飽}}}{あ}くことを
\ruby{知}{し}らぬ
\ruby{歟}{か}、
%
\ruby{感心}{かん|しん}に
\ruby{自{\換字{分}}}{じ|ぶん}は
\ruby{自{\換字{分}}}{じ|ぶん}の
\ruby{勉{\換字{強}}}{べん|きやう}を
\ruby{仕}{し}て
\ruby{居}{ゐ}るさうだ。
%
\換字{志}て
\ruby{見}{み}りやあ
\ruby{容貌}{きり|やう}も
\ruby{佳}{よ}いし、
%
\ruby{心掛}{こゝろ|がけ}も
\ruby{可}{よ}いし、
%
\ruby{別}{べつ}に
\ruby{難}{なん}は
\ruby{無}{な}い
\ruby{女}{をんな}なんだ。
%
\ruby{左樣}{さ|う}いふ
\ruby{女}{をんな}が
\ruby{現}{あらは}れたので、
%
\ruby{學校}{がく|かう}の
\ruby{内}{うち}でも
\ruby{外}{そと}でも
\ruby{珍}{めづ}らしがつて、
%
\ruby{何}{なん}とか
\ruby{彼}{か}とか
\ruby{{\換字{評}}{\換字{判}}}{ひやう|ばん}が
\ruby{立}{た}つて
\ruby{居}{ゐ}たが、
%
\ruby{其内}{その|うち}に
\ruby[g]{水野}{みづの}が
\ruby{{\換字{迷}}}{まよ}ひ
\ruby{出}{だ}した。
%
\ruby{何樣}{ど|う}いふ
\ruby{機會}{は|め}から
\ruby[g]{水野}{みづの}の
\ruby{心}{こゝろ}が
\ruby{其女}{その|をんな}に
\ruby{傾}{かたむ}いたかは
\ruby{解}{わか}らないが、
%
\ruby{乃公}{お|れ}が
\ruby{思}{おも}ふにやあ
\ruby{別}{べつ}な
\ruby{事}{こと}はない。
%
\ruby{淨瑠璃}{じやう|る|り}の
\ruby{{\換字{文}}句}{もん|く}にある
\ruby{{\換字{通}}}{とほ}り、
%
\ruby{琥珀}{こ|はく}の
\ruby{塵}{ちり}や
\ruby{磁石}{じ|しやく}の
\ruby{針}{はり}で、
%
\ruby{眼}{め}に
\ruby{見}{み}えて
\ruby{何處}{ど|こ}が
\ruby{何樣}{ど|う}と
いふ
\ruby{事}{こと}は
\ruby{無}{な}いが、
%
たゞ
\ruby{譯}{わけ}も
\ruby{無}{な}く
\ruby{引}{ひ}き
\ruby{寄}{よ}せられて、
%
\ruby{心}{こゝろ}が
\ruby{其處}{そ|こ}へ
\ruby{行}{ゆ}くのが
\ruby{戀}{こひ}の
\ruby{{\換字{習}}}{なら}ひだ。
%
こりあ
\ruby{俗物}{ぞく|ぶつ}でも
\ruby{仙骨}{せん|こつ}でも
\ruby{同}{おな}じ
\ruby{事}{こと}、
%
いくら
\ruby[g]{水野}{みづの}が
\ruby{俊才}{すぐれ|もの}だつて、
%
\原本頁{33-1}
\ruby{生血}{なま|ち}を
\ruby{包}{つゝ}んだ
\ruby{五尺}{ご|しやく}の
\ruby{身體}{から|だ}を、
%
\ruby{抱}{かゝ}へて
\ruby{居}{ゐ}るのだもの
\ruby{無理}{む|り}も
\ruby{無}{な}い、
%
\ruby{矢張}{やつ|ぱ}り
\ruby{年齡}{と|し}が
\ruby{年齡}{と|し}だから
\ruby{{\換字{迷}}}{まよ}つたんだらう。
%
\換字{志}かし
\ruby{相手}{あひ|て}も
\ruby{商賣人}{しやう|ばい|にん}ぢあ
\ruby{無}{な}し、
%
\ruby[g]{水野}{みづの}も
\ruby{獨身}{ひとり|み}で
\ruby{居}{ゐ}なけりあ
ならぬといふので
\ruby{無}{な}いから、
%
\ruby{全}{まつた}く
\ruby{深}{ふか}く
\ruby{思}{おも}ひ
\ruby{{\換字{込}}}{こ}んだものならば、
%
\ruby{緣}{えん}を
\ruby{纏}{まと}めりやあ
\ruby{其}{それ}で
\ruby{可}{い}いのだが、
%
さあ、
%
\ruby[g]{水野}{みづの}の
\ruby{不仕合}{ふ|し|あはせ}といふのは
\ruby{其處}{そ|こ}の
\ruby{事}{こと}で、
%
\ruby{俗}{ぞく}にいふ
\ruby{蟲}{むし}が
\ruby{{\換字{嫌}}}{きら}ふと
いふものでゞもあらうか、
%
\ruby{其女}{その|をんな}が
\ruby[g]{水野}{みづの}の
\ruby{眞心}{ま|ごゝろ}を
\ruby{受}{う}け
\ruby{納}{い}れぬので、
%
それで
\ruby[g]{水野}{みづの}は
\ruby{懊惱}{あう|なう}して
\ruby{居}{ゐ}るといふのだ。
%
もつとも
\ruby[g]{水野}{みづの}が
\ruby{明}{あか}らさまに、
%
\ruby{其女}{その|をんな}に
\ruby{何事}{なに|ごと}を
\ruby{云}{い}つたでも
あるまいが、
%
これは
\ruby{世間}{せ|けん}に
\ruby{老}{お}いた
\ruby[g]{山路}{やまぢ}の
\ruby{老夫}{ぢゞ|い}が、
%
\ruby[g]{水野}{みづの}の
\ruby{樣子}{やう|す}を
\ruby{見}{み}て
\ruby{察}{さつ}しての
\ruby{話}{はなし}だ。
%
さて
\ruby{其}{それ}にしたところで
\ruby{其限}{それ|ぎ}りの
\ruby{事}{こと}なら、
%
\ruby{芥火}{あくた|び}の
\ruby{燃}{も}えるやうに
ぶすりぶすりと、
%
\原本頁{34-1}
\ruby[g]{水野}{みづの}が
\ruby{物}{もの}を
\ruby{思}{おも}つて
\ruby{居}{ゐ}るだけで
\ruby{濟}{す}むのだが、
%
こゝに
\ruby{其}{そ}の
\ruby[g]{五十子}{いそこ}の
\ruby{親}{おや}に
お
\ruby{關}{せき}といふ、
%
\ruby{可憎}{い|や}な
\ruby{{\換字{強}}欲}{がう|よく}な
\ruby{惡婆}{あく|ば}がある。
%
\ruby{勿論}{もち|ろん}
\ruby{生}{うみ}の
\ruby{母}{はゝ}では
\ruby{無}{な}くつて、
%
\ruby[g]{五十子}{いそこ}とは
\ruby{別々}{べつ|〳〵}に
\ruby{住}{す}んで
\ruby{居}{ゐ}るほど、
%
\ruby{氣性}{き|しやう}も
\ruby{合}{あ}はねば
\ruby{仲}{なか}も
\ruby{惡}{わる}いのだが、
%
\ruby{時々}{とき|〴〵}
\ruby[g]{五十子}{いそこ}のところへ
\ruby{來}{き}ては
\ruby{無理}{む|り}を
\ruby{云}{い}つて、
%
\ruby{無}{な}け
\ruby{無}{な}しの
\ruby{金}{かね}を
\ruby{絞}{しぼ}つて
\ruby{行}{ゆ}く。
%
\ruby{其奴}{そ|いつ}が
\ruby[g]{水野}{みづの}の
\ruby{腹}{はら}を
\ruby{見}{み}て
\ruby{取}{と}つて、
%
\ruby{其}{そ}の
\ruby{初心}{う|ぶ}な
ところに
\ruby{付}{つ}け
\ruby{{\換字{込}}}{こ}んで、
%
いろいろ
さまざまな
\ruby{事}{こと}を
\ruby{云}{い}ひ
\ruby{散}{ち}らしちやあ、
%
つまり
\ruby{幾干}{いく|ら}かづゝ
\ruby{捲}{ま}き
\ruby{上}{あ}げるさうだ。
%
\ruby{金}{かね}は
\ruby{些少}{わづ|か}の
\ruby{事}{こと}だから
\ruby{仔細}{し|さい}は
\ruby{無}{な}いが、
%
\ruby{金}{かね}を
\ruby{取}{と}らう
\ruby{爲}{ため}ばつかりに
\ruby{其}{その}
\ruby{婆}{ばゞあ}めが、
%
\ruby{好}{い}い
\ruby{加減}{か|げん}な
\ruby{事}{こと}を
\ruby{云}{い}つて
\ruby{煽}{あふ}り
\ruby{立}{た}つて
\ruby{燃}{も}え
\ruby{立}{た}たする。
%
ところが
\ruby{一方}{いつ|ぱう}ぢやあ
\ruby{{\換字{又}}}{また}、
%
\ruby{肝心}{かん|じん}の
\ruby{人}{ひと}に
よそ〳〵しく
\ruby{冷}{ひや}つこく
\ruby{待{\換字{遇}}}{あし|ら}はれる。
%
\ruby{火}{ひ}にあひ
\ruby{水}{みづ}にあふのだから
\ruby{敵}{かな}はない、
%
\ruby[g]{水野}{みづの}の
\ruby{心}{こゝろ}の
\ruby{靜穩}{しづ|か}なことは、
%
\ruby{今}{いま}は
\ruby{一時}{いつ|とき}でも
\ruby{有}{あ}りさうも
\ruby{無}{な}い
\ruby{譯}{わけ}。
%
そこで
\ruby{今}{いま}までの
\ruby{行狀}{みも|ち}とは
\ruby{打}{う}つて
\ruby{變}{かは}つて、
%
\ruby{家}{うち}に
\ruby{居}{ゐ}る
\ruby{時}{とき}は
\ruby{鬱々}{うつ|〳〵}として、
%
たゞ
\ruby{沈}{しづ}みきつて
\ruby{物}{もの}も
\ruby{言}{い}はず、
%
\ruby{机}{つくゑ}に
\ruby{對}{むか}つても
\ruby{書}{ほん}は
\ruby{讀}{よ}まずに、
%
\ruby{長太息}{た|め|いき}を
\ruby{吐}{つ}く
\ruby{時}{とき}のみ
\ruby{多}{おほ}く、
%
\ruby{{\換字{朝}}}{あさ}は
\ruby{心}{こゝろ}よく
\ruby{起}{お}きる
\ruby{日}{ひ}も
\ruby{無}{な}く、
%
\ruby{夜}{よ}も
\ruby{寐苦}{ね|ぐる}しく
\ruby{{\換字{過}}}{すご}すさうだ。
%
これは
\ruby{乃公}{お|れ}が
\ruby{老夫}{おや|ぢ}から
\ruby{聞}{き}いたゞけで、
%
\ruby{無論}{む|ろん}
\ruby[g]{山路}{やまぢ}の
\ruby{老夫}{おや|ぢ}の
つもりでは、
%
\ruby{乃公}{お|ら}に
\ruby{意見}{い|けん}して
\ruby{{\換字{遣}}}{や}れと
いふのだつた。
%
\換字{志}かし
\ruby{乃公}{お|れ}は
\ruby{乃公}{お|れ}の
\ruby{考}{かんがへ}で、
%
\ruby[g]{水野}{みづの}のためには
\ruby{幾干}{いく|ら}でも、
%
\ruby{盡力}{つ|く}したいと
\ruby{思}{おも}つて
\ruby{居}{ゐ}ることは
\ruby{思}{おも}つて
\ruby{居}{ゐ}るが、
%
\ruby{意見}{い|けん}を
\ruby{仕}{し}て
\ruby{利益}{た|め}になりさうな
\ruby{筋}{すぢ}では
\ruby{無}{な}いと、
%
\ruby{見切}{み|き}つてつい
\ruby{其儘}{その|まゝ}に
\ruby{{\換字{過}}}{す}ごして
\ruby{來}{き}たのだ。
』% 其三での羽勝の最後の語りが終えたところ

\原本頁{35-10}
\ruby{辛}{から}くも
\ruby{此時}{こ|ゝ}まで
\ruby{堪}{こら}へたりし
\ruby[g]{日方}{ひかた}は
\ruby{再}{ふたゝ}び
\ruby{叫}{さけ}び
\ruby{出}{いだ}しぬ。

\原本頁{35-11}
『
\ruby{何故}{な|ぜ}
\ruby{意見}{い|けん}を
\ruby{仕}{し}ても
\ruby{利益}{た|め}にならん?。
%
\ruby{意見}{い|けん}を
\ruby{仕無}{し|な}いで
\ruby{何}{なん}と
\ruby{爲}{す}るんだ?。
%
\原本頁{36-1}
\ruby{何樣}{ど|う}して
\ruby[g]{水野}{みづの}の
\ruby{爲}{ため}に
\ruby{盡力}{つ|く}す?。
』

\原本頁{36-2}
『
\ruby{乃公}{お|ら}あ
\ruby{出來}{で|き}る
\ruby{事}{こと}なら
\ruby[g]{水野}{みづの}の
\ruby{思}{おも}ひの、
%
\ruby{徹}{とほ}るやうに
\ruby{爲}{し}て
\ruby{{\換字{遣}}}{や}らうと
\ruby{思}{おも}つて
\ruby{居}{ゐ}るのだ。
』

\原本頁{36-4}
『
\ruby{何}{なん}だと、
%
\ruby{馬鹿野郎}{ば|か|や|らう}ツ!、
%
\ruby{愚}{ぐ}にもつかん!。
%
そんな
\ruby{下}{くだ}らんことがあるものか、
%
\ruby{貴樣}{き|さま}は
\ruby{一體}{いつ|たい}
\ruby{腐敗}{ふ|はい}して
\ruby{居}{ゐ}る!。
』

\原本頁{36-6}
『また
\ruby{馬鹿}{ば|か}
\ruby{呼}{よば}はりを
するナ!。
%
\ruby{汝}{きさま}こそ
\ruby{馬鹿}{ば|か}だ。
%
\ruby{意見}{い|けん}して
\ruby{役}{やく}に
\ruby{立}{た}つ
\ruby{位}{くらゐ}なら
\ruby{乃公}{お|れ}が
\ruby{爲}{す}るは。
%
\ruby{人}{ひと}は
\ruby{銘々}{めい|〳〵}に
\ruby{{\換字{所}}考}{かん|がへ}が
ある。
%
\ruby{乃公}{お|れ}は
\ruby{乃公}{お|れ}、
%
\ruby{汝}{きさま}は
\ruby{汝}{きさま}で
\ruby{可矣}{い|ゝ}ぢやあ
\ruby{無}{ね}えか。
%
\ruby{意見}{い|けん}が
\ruby{仕}{し}たけりやあ
\ruby[<h||]{汝}{きさま}
\ruby{爲}{し}ろ。
』

\原本頁{36-9}
『
\ruby{勿論}{もち|ろん}だ。
%
\ruby{諫}{いさ}めて
\ruby{{\換字{遣}}}{や}らないで
\ruby{何樣}{ど|う}するものか。
%
\ruby{女}{をんな}が
\ruby{美}{よ}くつても
\ruby{惡}{わる}くつても、
%
\ruby{何}{なん}だ!、
%
\ruby{女}{をんな}が!。
%
\ruby{苟}{いやし}くも
\ruby{大{\換字{丈}}夫}{だい|ぢやう|ぶ}たるものが
\ruby{高}{たか}が
\ruby{一{\換字{婦}}人}{いち|ぶ|じん}に、
%
\ruby{志}{こゝろざし}を
\ruby{喪}{うしな}ふとは
\ruby{何}{なん}たる
\ruby{事}{こつ}た。
%
\ruby{實}{じつ}に
\ruby{怪}{け}しからん、
%
はがゆい
\ruby{奴}{やつ}だ。
%
\原本頁{37-1}
\ruby{是非}{ぜ|ひ}
\ruby{{\換字{尋}}}{たづ}ねて
\ruby{行}{い}つて
\ruby{大}{おほい}に
\ruby{諫}{いさ}める。
』

\原本頁{37-2}
\ruby{二人}{ふた|り}の
\ruby{問答}{もん|だふ}は
こゝに
\ruby{已}{や}んで、
%
\ruby[g]{山瀬}{やませ}は
\ruby{爽}{さわ}やかに
\ruby{口}{くち}を
\ruby{開}{ひら}きぬ。

\原本頁{37-3}
『
\ruby{僕}{ぼく}は
\ruby{他人}{ひ|と}の
\ruby{意志}{い|し}
\ruby{感{\換字{情}}}{かん|じやう}の
\ruby{自由}{じ|いう}を
\ruby{{\換字{尊}}重}{そん|ちよう}するから、
%
\ruby{立入}{たち|い}つては
\ruby{敢}{あへ}て
\ruby{兎角}{と|かく}を
\ruby{言}{い}はぬ。
%
\換字{志}かし
これは
\ruby[g]{水野}{みづの}
\ruby{君}{くん}のために
\ruby{不利益}{ふ|り|えき}と
\ruby{思}{おも}ふから、
%
\ruby{一應}{いち|おう}は
\ruby{忠告}{ちゆう|こく}を
\ruby{試}{こゝろ}みるつもりだ。
』

\原本頁{37-6}
\ruby{人}{ひと}
\ruby{皆}{みな}
\ruby{語}{かた}れども
\ruby[g]{羽{\換字{勝}}}{はがち}は
\ruby{語}{かた}らず、
%
たゞ
\ruby{僅}{わづか}に
\ruby{吁然}{ほ|つ}と
\ruby{息}{いき}つけば、
%
\ruby{手}{て}にせし
\ruby{卷{\換字{煙}}草}{た|ば|こ}の
\ruby{{\換字{灰}}}{はい}の% ルビは「はひ」でなく原本通り
\ruby{長}{なが}く
\ruby{續}{つゞ}けるが、
%
ぼたりと
\ruby{膝}{ひざ}の
\ruby{上}{うへ}に
\ruby{落}{お}ちて
\ruby{脆}{もろ}く
\ruby{散}{ち}つたり。

\原本頁{37-9}
\ruby{夜色}{や|しよく}は
\ruby{樓外}{ろう|ぐわい}に
\ruby{沈々}{ちん|〳〵}として、
%
\ruby{澄}{す}みわたりたる
\ruby{天}{そら}に
かゝれる
\ruby{星斗}{ほ|し}は
\ruby{爛然}{らん|ぜん}と
\ruby{明}{あき}らかに、
%
\ruby{明日}{あ|す}は
\ruby{風}{かぜ}にや
\ruby{其}{そ}の
\ruby{大}{おほき}なるは、
%
いづれも
\ruby{煌々}{ひか|〳〵}と
\ruby{瞬目}{めは|じき}して、
%
\ruby{光}{ひかり}の
\ruby{芒}{のぎ}は
\ruby{搖}{ゆら}ぎに
\ruby{搖}{ゆら}げり。


\Entry{其六}

% メモ 校正終了 2024-03-29 2024-05-23 2024-06-15
\原本頁{38-2}%
\ruby[g]{山瀬}{やませ }が
\ruby{催}{もよほ}せし
\ruby[g]{小集}{せうしふ}の、
%
\ruby[g]{竹芝}{たけしば}の
\ruby{浦}{うら}に
\ruby{開}{ひら}かれし
\ruby{日}{ひ}なり、
%
これは
\ruby[<j||]{東}{とう }% 行末行頭の境界付近なので特例処置を施す
\ruby[<j||]{京}{きやう}
% \ruby{東京}{とう|きやう}
\原本頁{38-3}\改行%
を
\ruby[g]{丑寅}{うしとら}に
\ruby{離}{はな}れし
\ruby{東武線}{とう|ぶ|せん}の
\ruby[||j>]{鐘}{かねが}
\ruby[||j>]{淵}{ ふち}の
% \ruby{鐘淵}{かねが|ふち}の
\ruby[g]{停車}{ていしや}
\ruby{場}{じやう}より、% 原文通り「場」
%
\ruby{上}{のぼ}り
\ruby[g]{滊車}{き しや}の
\ruby{今}{いま}や
\ruby{出}{い}でんとするに
\ruby{駈}{か}け
\ruby{付}{つ}けて、
%
\ruby{辛}{から}くも
\ruby{乘}{の}り
\ruby{{\換字{込}}}{こ}みし
\ruby[g]{水野}{みづの }
\ruby{靜十郎}{せい|じふ|らう}は、
%
\ruby[g]{車室}{しやしつ}の
\ruby[g]{一隅}{いちぐう}に
\ruby{身}{み}を
おちつけて、
%
\ruby{煎}{い}りつくが
\ruby{如}{ごと}き
\ruby{急}{せ}き
\ruby{心}{ごゝろ}に
\ruby{少}{すくな}からぬ
\原本頁{38-6}\改行%
\ruby[g]{路程}{みちのり}を
\ruby{走}{はし}り
\ruby{來}{きた}りし
\ruby{胸}{むね}の
\ruby{轟}{とゞろ}きを
\ruby{纔}{わづか}に
\ruby{息}{やす}めぬ。

\原本頁{38-7}%
\ruby[g]{車窓}{しやそう}の
\ruby{外}{そと}は、
%
\ruby{目}{め}に
\ruby{障}{さは}るものも
\ruby{無}{な}く
\ruby[g]{廣々}{ひろ〴〵}としたる
\ruby[g]{葛{\換字{飾}}}{かつしか}の
\ruby{秋}{あき}の
\ruby[g]{稻田}{いなだ }に、
%
\ruby{黄金色}{こ|がね|いろ}の
\ruby[g]{夕陽}{ゆふひ }の
\ruby[g]{光線}{ひかり }
\ruby{明}{あか}るく
\ruby{斜}{なゝめ}に
\ruby{落}{お}ちて、
%
\ruby[g]{折々}{をり〳〵}
ばつと
\ruby{立}{た}つ
\ruby[||j>]{群}{ぐん}
\ruby[||j>]{雀}{じやく}の
% \ruby{群雀}{ぐん|じやく}の
\ruby{{\換字{空}}}{そら}に
\ruby{散}{ち}る
\ruby[g]{景色}{け しき}も、
%
\ruby[g]{土用}{ど よう}の
\ruby{旱}{てり}の
\ruby{足}{た}りて
\ruby{豊}{ゆたか}なる
\ruby{年}{とし}の
\換字{志}るしと
\ruby{好}{この}もしく、
%
\ruby{暑}{あつ}かりし
\ruby{夏}{なつ}の
\ruby{日}{ひ}の
\ruby{汗}{あせ}の
\ruby{滴}{しづく}は、
%
\ruby{今}{いま}
\ruby{皆}{みな}
やがて
\ruby[g]{粒々}{りふ〳〵}の
\ruby{實}{み}となつて
\ruby{現}{あらは}るべき
\ruby[<j>]{快}{こゝろよ}き
\ruby[g]{眺望}{ながめ }なり。

\原本頁{39-2}%
されば
\ruby{乗}{の}り
\ruby{合}{あ}はせし
\ruby[g]{人々}{ひと〴〵}も
\ruby{欣}{よろこ}び
\ruby{顏}{がほ}して、

\原本頁{39-3}%
『
\ruby{先}{ま}づ
\ruby{此}{こ}の
\ruby{{\換字{分}}}{ぶん}に
\ruby{行}{ゆ}きやあ
\ruby[g]{豐年}{ほうねん}でがす。
』

\原本頁{39-4}%
と
\ruby[g]{股引}{もゝひき}に
\ruby{草鞋穿}{わら|ぢ|ば}きの
\ruby[||j|]{農}{ひやく}% ルビ調整(原本通り)
\ruby[||j|]{夫}{しやう}らしきが
% \ruby{農夫}{ひやく|しやう}らしきが
\ruby[g]{眞先}{まつさき}に
\ruby{云}{い}ひ
\ruby{出}{だ}せば、

\原本頁{39-5}%
『
さうです、
%
\ruby{風}{かぜ}さへ
\ruby{無}{な}きやあ
\ruby{既}{もう}
\ruby{大{\換字{丈}}夫}{だい|ぢやう|ぶ}です。
%
おほかた
\ruby{不景氣}{ふ|けい|き}も
\ruby{直}{なほ}るでがせう。
』

\原本頁{39-7}%
と
\ruby{同}{おな}じ
\ruby{風}{ふう}の
\ruby{男}{をとこ}が
\ruby{云}{い}ふ。
%
その
\ruby{後}{あと}より
\ruby{髮}{かみ}の
\ruby{毛}{け}を
\ruby[g]{綺麗}{き れい}に
\ruby{{\換字{分}}}{わ}けたる
\ruby{生意氣}{なま|い|き}の
\ruby{{\換字{若}}}{わか}き
\ruby{男}{をとこ}の、
%
これは
\ruby[||j>]{商}{しやう}
\ruby[||j>]{人}{ にん}と
% \ruby{商人}{しやう|にん}と
\ruby{見}{み}えたるが、

\原本頁{39-9}%
『
\ruby{何}{なん}にしろ
\ruby[g]{此夏}{このなつ}の
\ruby[g]{暑氣}{あつさ }の
おかげですもの、
%
\ruby[||j>]{此}{この}
\ruby[||j>]{位}{ぐらゐ}の
% \ruby{此位}{この|ぐらゐ}の
\ruby{事}{こと}あ
\ruby{無}{な}くちやあ
なりませんや。
%
\ruby{暑}{あつ}かつた
\ruby{事}{こと}あ
\ruby[g]{無法}{む はふ}に
\ruby{暑}{あつ}うございましたが、
%
\ruby[g]{何樣}{ど う }でしやう
\ruby[g]{全國}{ぜんこく}ぢやあ
\ruby{其}{それ}がために、
%
\ruby[g]{去年}{きよねん}に
\ruby{比}{くら}べりやあ
\ruby{一千萬石}{いつ|せん|まん|ごく}も
\ruby[g]{餘計}{よ けい}に
\ruby{穫}{と}れる
\ruby[g]{算盤}{そろばん}だつて
\ruby{云}{い}ふんですからなア!。
%
\原本頁{40-1}%
\ruby[g]{一石}{いつこく}
\ruby[g]{十圓}{じふゑん}としても
\ruby{一億圓}{いち|おく|ゑん}、
%
\ruby{四千萬人}{よん|せん|まん|にん}に
\ruby{割}{わ}つて
みると、
%
\ruby{一人{\換字{前}}}{いち|にん|まへ}が
\ruby{二圓五十錢}{に|ゑん|ご|じふ|せん}
\ruby{宛}{づゝ}、
%
\ruby[g]{畢竟}{つ まり}
それだけ
\ruby{宛}{づゝ}
\ruby[g]{暑氣}{あつさ }の
\ruby[g]{堪{\換字{忍}}}{が まん}
\ruby{賃}{ちん}に
\ruby{貰}{もら}つたやうな
\ruby{譯}{わけ}に
\ruby{當}{あた}りますから、
%
\ruby[g]{隨{\換字{分}}}{ずゐぶん}
\ruby{暑}{あつ}かつたのも
\ruby[g]{無理}{む り }は
\ruby{有}{あ}りません。
%
\ruby{併}{しか}し
\ruby[g]{如是}{か う }なつて
\ruby{見}{み}りやあ
\ruby{有}{あ}り
\ruby{{\換字{難}}}{がた}いもんで、
%
\ruby[g]{屹度}{きつと }
\ruby[g]{景氣}{けいき }も
\ruby{好}{よ}くなりまさあネ。
』

\原本頁{40-6}%
などゝ
\ruby[g]{口々}{くち〴〵}に
\ruby{語}{かた}り
あへど、
%
\ruby[||j>]{思}{おもひ}
\ruby[||j>]{有}{ あ}る
\ruby{身}{み}の
\ruby[g]{水野}{みづの }
\ruby[g]{一人}{ひとり }は、
%
\ruby[g]{景色}{け しき}も
\ruby{眼}{め}に
\原本頁{40-7}\改行%
\ruby{{\換字{更}}}{さら}に
\ruby{見}{み}ざるが
ごとく、
%
\ruby[g]{談話}{はなし }も
\ruby{耳}{みゝ}に
\ruby{{\換字{更}}}{さら}に
\ruby{聞}{き}かぬが
\ruby{如}{ごと}く、
%
\ruby{身}{み}じろぎ
\原本頁{40-8}\改行%
も
\ruby{多}{おほ}くはせで
\ruby[||j>]{寂}{じやく}
\ruby[||j>]{然}{ ねん}と
% \ruby{寂然}{じやく|ねん}と
\ruby{坐}{すわ}りつ、
%
たゞ
\ruby{帶}{おび}の
\ruby{間}{あひだ}より
\ruby[g]{時計}{と けい}を
\ruby{出}{いだ}して、
%
\ruby[<j||]{恰}{あだか}% 恰も「あ(だ)かも」% 行末行頭の境界付近なので特例処置を施す
\原本頁{40-9}\改行%
も
\ruby[g]{滊車}{き しや}の
\ruby[g]{{\換字{速}}力}{はやさ }を
\ruby{疑}{うたが}ふやうに、
%
\ruby[g]{幾度}{いくたび}か
\ruby{其}{そ}の
\ruby{鍼}{はり}を
\ruby[g]{甲{\換字{斐}}}{か ひ }
\ruby{無}{な}く
\ruby[g]{視詰}{み つ }めぬ
\改行% 校正作業の簡略化のため
。
%
\原本頁{40-10}\改行%
\ruby[g]{淺黑}{あさぐろ}き
\ruby{其}{そ}の
\ruby{面}{おもて}は
\ruby{底}{そこ}に
\ruby[g]{蒼色}{あをみ }を
\ruby{帶}{お}びて、
%
\ruby[g]{鳳眼}{ほうがん}とやらん
\ruby{人}{ひと}のいふ
\ruby{魚尾上}{し|り|あが}りの
\ruby{眼}{め}は、
%
どんよりと
\ruby{曇}{くも}りて
\ruby{光}{ひか}り
\ruby{澱}{よど}み、
%
やゝ
\ruby{狭}{せま}き
\ruby{鼻}{はな}は
つんと
\原本頁{41-1}\改行%
\ruby{高}{たか}くして、
%
\ruby{血}{ち}の
\ruby[g]{色薄}{いろうす}き
\ruby{一}{いち}の
\ruby{字}{じ}
\ruby{口}{ぐち}の
\ruby[<j>]{唇}{くちびる}は、
%
\ruby{復}{ふたゝ}び
\ruby{開}{ひら}かるゝ
\ruby{時}{とき}の
\ruby{無}{な}からん
\ruby{如}{ごと}くに
\ruby{{\換字{飽}}}{あく}まで
\ruby{緊}{きび}しく
\ruby{閉}{とぢ}られたり。
%
\ruby{眼鼻立}{め|はな|だち}は
\ruby{醜}{あし}きに
あらぬ
\ruby[<j||]{男}{をとこ}ながら、
%
\ruby[g]{水野}{みづの }が
\ruby{今}{いま}の
\ruby{顏}{かほ}の
\ruby[g]{氣色}{やうす }は、
%
\ruby[g]{稚兒}{をさなご}は
\ruby{之}{これ}を
\ruby{望}{のぞ}まば
\ruby{怖}{おそ}れて
\ruby{泣}{な}くべし。

\原本頁{41-5}%
\ruby[g]{滊車}{き しや}の
やがて
\ruby{吾妻橋}{あづ|ま|ばし}% ルビ調整(原本通り)
\ruby[g]{停車}{ていしや}
\ruby{場}{じやう}に% 原文通り「場」
% 吾妻橋停車場 とうきょうスカイツリー駅
% 1902年(明治35年)に北千住駅から吾妻橋駅(現・とうきょうスカイツリー駅)へ延伸開業
% なので、ここで下車して吾妻橋を渡ることになる
\ruby{着}{つき}し
\ruby{時}{とき}には、
%
\ruby{暮}{く}れやすき
\ruby{秋}{あき}の
\ruby{日}{ひ}は
\原本頁{41-6}\改行%
\ruby{既}{はや}
\ruby{沒}{い}りて、
%
\ruby{千點萬點}{せん|てん|ばん|てん}の
\ruby[g]{燈火}{ともしび}に
\ruby{{\換字{飾}}}{かざ}られたる
\ruby{夜}{よる}の
\ruby[||j>]{東}{とう}
\ruby[||j>]{京}{きやう}は
% \ruby{東京}{とう|きやう}は
\ruby{眼}{め}の
\ruby{{\換字{前}}}{まへ}に
\ruby{現}{あら}はれぬ。

\原本頁{41-8}%
\ruby[g]{水野}{みづの }は
\ruby{人}{ひと}を
\ruby{突}{つ}き
\ruby{{\換字{退}}}{の}くるまでに
\ruby{忙}{いそ}がはしく
\ruby{歩}{あゆ}みて、
%
\ruby{忽}{たちま}ち
\ruby[g]{停車}{ていしや}
\ruby{場}{じやう}を% 原文通り「場」
\原本頁{41-9}\改行%
\ruby{出}{い}で、
%
\ruby{忽}{たちま}ち
\ruby{吾妻橋}{あづ|ま|ばし}を% ルビ調整(原本通り)
\ruby{越}{こ}え、
%
\ruby{忽}{たちま}ち
\ruby{茶屋町}{ちや|ゝ|まち}を% 現町名:台東区雷門二丁目
\ruby{{\換字{過}}}{す}ぎ、
%
\ruby{忽}{たちま}ち
\ruby[g]{並木}{なみき }を% 現町名:台東区雷門二丁目
\ruby{經}{へ}て
\改行% 校正作業の簡略化のため
、
%
\原本頁{41-10}\改行%
\ruby{忽}{たちま}ち
\ruby[g]{藏{\換字{前}}}{くらまへ}に
\ruby{至}{いた}り、
%
\ruby[g]{其處}{そ こ }に
\ruby{住}{すま}へる
\ruby[g]{月日}{つきひ }は
\ruby{未}{いま}だ
\ruby{長}{なが}からねど、
%
\ruby[g]{淺草}{あさくさ}
\ruby{一}{いち}
\原本頁{41-11}\改行%
との
\ruby{噂}{うはさ}を
\ruby{得}{え}たる
\ruby[g]{醫學士}{い がくし }
\ruby[g]{相良}{さがら }
\ruby[g]{公{\換字{平}}}{こうへい}の
\ruby[||j>]{玄}{げん}
\ruby[||j>]{關}{くわん}に
% \ruby{玄關}{げん|くわん}に
\ruby{至}{いた}り、

\原本頁{42-1}%
『
\ruby{頼}{たの}む。
』

\原本頁{42-2}%
と
\ruby[g]{一聲}{いつせい}
\ruby{音}{おと}づれたり。

\Entry{其七}

% メモ 校正終了 2024-03-30 2024-05-23 2024-06-15
\原本頁{42-4}%
\ruby{應}{おう}と
\ruby{答}{こた}へて
\ruby{出}{い}で
\ruby{來}{きた}れるは、
%
\ruby{盤臺面}{ばん|だい|づら}の% 【盤台面】平たくて大きい顔をあざけっていう語。
% 盤台 《「はんだい」とも》
% 魚屋が魚を運ぶのに用いる、浅く作った楕円形または円形の大きなたらい。
% 料理ですし飯をまぜたりするのにも用いる。
\ruby{鼻}{はな}の
\ruby{下}{した}に
\ruby[g]{薄髭}{うすひげ}
しよぼ〳〵と
\ruby{{\換字{煙}}}{けむり}の
\ruby{如}{ごと}く
\ruby{生}{は}えたる、
%
\ruby{二十七八}{に|じふ|しち|はち}の% 原本には漢数字「七」のルビ無し
\ruby[g]{物體}{もつたい}ぶつた
\ruby{男}{をとこ}なり。
%
\ruby[g]{水野}{みづの }が
\ruby{紺飛白}{こん|が|すり}の
\ruby[||j>]{單}{ひとへ}
\ruby[||j>]{衣}{ もの}に、
% \ruby{單衣}{ひとへ|もの}に、
%
\ruby[g]{着皺}{き じわ}も
\ruby{見}{み}ゆる
\ruby{薄羽織}{うす|ば|おり}といふ
\ruby{身}{み}の
\ruby[g]{周圍}{まはり }を
\ruby{見}{み}て、
%
\ruby[g]{突立}{つゝた }ちたる
まゝ
\ruby[g]{{\換字{尊}}大}{おほふう}に、

\原本頁{42-8}%
『
もう
\ruby[g]{診察}{しんさつ}の
\ruby[g]{時間}{じ かん}は
\ruby{濟}{す}んだが。
』

\原本頁{42-9}%
と
\ruby{云}{い}ひかけしが、
%
また
\ruby{其}{そ}の
\ruby[g]{顏色}{かほいろ}の
\ruby{好}{よ}からぬを
\ruby{見}{み}て、

\原本頁{42-10}%
『
お
\ruby{{\換字{前}}}{まへ}さんかネ。
』

\原本頁{43-1}%
と
\ruby{僅}{わづか}に
\ruby[g]{愛想}{あいそ }あり。

\原本頁{43-2}%
\ruby[g]{水野}{みづの }は
\ruby[g]{叮嚀}{ていねい}に
\ruby[g]{會釋}{ゑしやく}して、

\原本頁{43-3}%
『
イヤ
\ruby[|j|]{私}{わたくし}では% ルビ調整(原本通り)
ございません。
%
\ruby{御書{\換字{留}}}{お|かき|とめ}
\ruby{置}{お}き
\ruby{下}{くだ}すつたといふ
\ruby{事}{こと}ですが、
%
\ruby[g]{昨日}{さくじつ}
\ruby[g]{使丁}{つかひ }を
\ruby{以}{も}つて
\ruby{願}{ねが}ひました
\ruby{四木村}{よ|つ|ぎ}の
\ruby[g]{{\換字{平}}井}{ひらゐ }と
\ruby{申}{まを}す
\ruby{者}{もの}の
\ruby{方}{かた}
の
\ruby[||j>]{病}{びやう}
\ruby[||j>]{人}{ にん}、
% \ruby{病人}{びやう|にん}、
%
\ruby[g]{岩崎}{いはさき}% 原本のこの部分は「いわさき」
\ruby[g]{五十}{い そ }といふものを
\ruby{御來診}{ご|らい|しん}
\ruby{願}{ねが}ひたいので
\ruby{出}{で}ましたのです。
』

\原本頁{43-7}%
と
\ruby{云}{い}へば、

\原本頁{43-8}%
『
アヽ、
%
\ruby{其}{そ}の
\ruby{四ツ木}{よ| |ぎ}% 原本では縦書き用の小書き「ッ」だが ...
とかいふところは、
%
\ruby[g]{非常}{ひじやう}に
\ruby{{\換字{遠}}}{とほ}い
ところぢやさうだナ。
%
\ruby{知}{し}らんものだから
\ruby[g]{仕方}{し かた}が
\ruby{無}{な}い、
%
\ruby[g]{小梅}{こ うめ}か
\ruby[g]{{\換字{請}}地}{うけぢ }の
\ruby[g]{{\換字{近}}傍}{ちかく }かと
\ruby{思}{おも}うて、
%
ムヽ
\ruby[g]{可矣}{よ し }
\ruby{願}{ねが}つて
\ruby{置}{お}いて
\ruby{{\換字{遣}}}{や}ると
\ruby{僕}{ぼく}が
\ruby[g]{受合}{うけあ }つたが、
%
\ruby{後}{あと}で
\ruby[g]{先生}{せんせい}に
\ruby{酷}{ひど}く
\ruby{叱}{しか}られた!。
%
\ruby[||g>]{重病人}{ぢゆうびやうにん}% ルビ調整(原本通り)「重(ぢゆう)」
や
\ruby[||g>]{長病人}{ちやうびやうにん}
を
\ruby[g]{澤山}{たくさん}に
\ruby{扣}{ひか}へて
\ruby{居}{ゐ}られるから、
%
\ruby[g]{中々}{なか〳〵}
\ruby[g]{其樣}{そ ん }な
\ruby{{\換字{遠}}}{とほ}い
ところへ
\ruby{御往診}{お|い|で}には
なりかねる
といふ
\原本頁{44-2}\改行%
ことだ。
%
どうか
\ruby[g]{他家}{よ そ }へ
\ruby{行}{い}つて
\ruby{頼}{たの}んで
\ruby{見}{み}てくれ。
』

\原本頁{44-3}%
と、
%
\ruby{實}{まこと}に
\ruby{酷}{ひど}く
\ruby{叱}{しか}られや
\ruby{仕}{し}けむ、
%
\ruby{其}{そ}の
\ruby{時}{とき}の
\ruby[g]{不{\換字{平}}}{ふ へい}は
\ruby{今}{いま}の
\ruby{顏}{かほ}に
\ruby{膨}{ふく}れ
\ruby{出}{だ}して、
%
\ruby[g]{{\換字{逐}}拂}{おつぱら}つて
\ruby[g]{仕舞}{し ま }ふ
つもりの
\ruby[g]{物言}{ものい }ひ
\ruby[g]{仁慈}{なさけ }
\ruby{無}{な}し。

\原本頁{44-5}%
\ruby{二三度}{に|さん|ど}
\ruby{四五度}{し|ご|ど}
\ruby{呼}{よ}びに
\ruby{{\換字{遣}}}{や}りける、
%
といふ
\ruby{{\換字{前}}}{まへ}
\ruby{句}{く}に、
%
\ruby{引}{ひ}く
\ruby{息}{いき}の
\ruby{{\換字{絕}}}{た}ゆるに
\ruby[g]{醫者}{い しや}の
おどろかず、
%
と
\ruby{付}{つ}けたるを、
%
\ruby[||j>]{西}{さい}
\ruby[||j>]{鶴}{くわく}が
% \ruby{西鶴}{さい|くわく}が
\ruby{撰}{えら}みし
\ruby{其}{そ}の
\ruby[g]{疇昔}{むかし }より、
%
\ruby{世}{よ}に
\ruby[g]{勢威}{いきほひ}ある
\ruby[g]{醫者}{い しや}を、
%
\ruby{富}{とみ}も
\ruby{無}{な}く
\ruby{名}{な}も
\ruby{無}{な}き
\ruby[g]{賤人}{し づ }が
\ruby[g]{伏屋}{ふせや }に
\ruby{{\換字{請}}}{しやう}じ
\ruby{入}{い}れんとするほど、
%
\ruby{心}{こゝろ}に
\ruby{任}{まか}せで
\ruby[g]{口惜}{くちをし}きは
\ruby{無}{な}し。
%
\ruby[g]{相良}{さがら }が
\ruby[g]{書生}{しよせい}の
\ruby{冷}{ひや}やかなる
\ruby[g]{言葉}{ことば }も、
%
\ruby{今}{いま}さら
\ruby{珍}{めづ}らしからぬ
\ruby[g]{{\換字{浮}}世}{うきよ }の
\ruby{態}{さま}なれば、
%
\ruby{腹}{はら}は
\ruby{立}{た}てねども
\ruby[g]{差當}{さしあた}つて
\ruby{恨}{うら}めしく
\ruby{悲}{かな}しく、
%
\ruby[g]{水野}{みづの }は

\原本頁{44-11}%
『
\ruby[g]{左樣}{さ う }
\ruby{仰}{おつし}あつては
\ruby[g]{當惑}{たうわく}いたします。
%
\ruby{實}{じつ}は
\ruby[g]{昨日}{さくじつ}から
\ruby{今}{いま}
\ruby{御來臨}{お|い|で}か
\ruby{今}{いま}
\ruby{御來臨}{お|い|で}かと
\ruby[g]{御待}{お ま }ち
\ruby{申}{まを}して
\ruby{居}{をり}ました
\ruby{樣}{やう}な
\ruby{譯}{わけ}で
ございますから。
』

\原本頁{45-2}%
と
\ruby{云}{い}ひかくるを、
%
\ruby[g]{書生}{しよせい}は
\ruby[g]{面倒}{めんだう}と
\ruby{云}{い}はぬばかりに、

\原本頁{45-3}%
『
だから、
%
うつかり
\ruby[g]{受合}{うけあ }つた
\ruby{段}{だん}は
\ruby{僕}{ぼく}が
\ruby[g]{謝罪}{あやま }る。
%
たゞし
\ruby[g]{先生}{せんせい}は
\ruby[g]{御忙}{お いそ}がしくつて
\ruby{御來診}{お|い|で}に
なられんといふのぢやから
\ruby[g]{仕方}{し かた}が
\ruby{無}{な}いぢや
\ruby{無}{な}いか。
』

\原本頁{45-6}%
と
\ruby{後}{あと}を
\ruby{言}{い}はせぬやうに
\ruby{壓}{お}し
\ruby{被}{かぶ}せて
\ruby{云}{い}ふ。
%
それを
\ruby[g]{此方}{こなた }は% ルビ調整(原本通り)
\ruby[g]{押{\換字{返}}}{おしかへ}して
\改行% 校正作業の簡略化のため
、

\原本頁{45-7}%
『
では
\ruby[g]{御座}{ご ざ }いませうが
\ruby[g]{其處}{そ こ }を
\ruby[g]{何卒}{どうぞ }、
%
もう
\ruby[g]{一度}{いちど }
\ruby[g]{御願}{お ねが}ひ
\ruby{下}{くだ}すつて
\ruby{見}{み}て
\ruby{頂}{いたゞ}きたいのです。
%
\ruby[g]{先生}{せんせい}より
\ruby{他}{ほか}の
\ruby{方}{かた}を
\ruby{願}{ねが}はう
\ruby{氣}{き}は
\ruby{無}{な}くつて、
%
かうして
\ruby[g]{態々}{わざ〳〵}
\ruby{四ツ木}{よ| |ぎ}% 原本では縦書き用の小書き「ッ」だが ...
から、
%
\ruby[g]{御願}{お ねが}ひに
\ruby{出}{で}たので
ございますから。
』

\原本頁{45-10}%
と、
%
\ruby{低}{ひく}き
\ruby[g]{聲音}{こわね }に
\ruby[g]{顫動}{ふるひ }をさへ
\ruby{帶}{お}びて、
%
\ruby{思}{おも}ひ
\ruby{入}{い}つて
\ruby{頭}{かうべ}を
\ruby{下}{さ}げて
{\換字{志}}みじみ% 原本は行頭禁足で非踊り字表記
と
\ruby{頼}{たの}み
\ruby{聞}{きこ}えぬ。
%
\ruby{見}{み}れば
\ruby[||j>]{其}{その}
\ruby[||j>]{面}{おもて}は
% \ruby{其面}{その|おもて}は
\ruby{深}{ふか}き〳〵
\ruby[g]{憂愁}{うれひ }の
\ruby[g]{陰雲}{く も }に
\ruby[g]{生氣}{せいき }を
\原本頁{46-1}\改行%
\ruby{{\換字{鎖}}}{とざ}されて、
%
\ruby[g]{疑懼}{ぎ く }に
\ruby{潤}{うる}める
\ruby{眼}{め}の
\ruby{中}{うち}には、
%
\原本頁{46-1}%
\ruby[g]{限無}{かぎりな}き
\ruby[g]{悲痛}{ひ つう}の
\ruby{色}{いろ}を
\ruby{{\換字{浮}}}{うか}めたり。
%
\ruby[g]{至誠}{まこと }に
\ruby{動}{うご}かされて
\ruby{爭}{あらそ}ひかねたる
\ruby[g]{書生}{しよせい}は
\ruby[g]{是非}{ぜ ひ }
\ruby{無}{な}く
\ruby{立}{た}ち
\ruby{上}{あが}つて
\改行% 校正作業の簡略化のため
、

\原本頁{46-3}%
『
それぢやあ
\ruby{先}{まあ}
\ruby[||j>]{伺}{うかゞ}つて
\ruby{見}{み}て
\ruby{上}{あ}げやうから、
%
\ruby[g]{其處}{そ こ }へ
\ruby{上}{あが}つて
\ruby{待}{ま}つて
\ruby{居}{ゐ}なさい。
』

\原本頁{46-5}%
と、
%
\ruby{{\換字{猶}}}{なほ}
\ruby[g]{水野}{みづの }を
\ruby[g]{田舎}{ゐなか }
\ruby{{\換字{漢}}}{もの}
あしらひにして
\ruby{奧}{おく}へ
\ruby{行}{ゆ}きぬ。

\原本頁{46-6}%
\ruby[g]{丁度}{ちやうど}
\ruby{人}{ひと}の
\ruby[g]{{\換字{途}}{\換字{絕}}}{と だ }えし
\ruby[g]{夜食}{やしよく}の
\ruby{頃}{ころ}とて、
%
\ruby{人}{ひと}も
\ruby{無}{な}き
\ruby[||j>]{玄}{けん}% ルビ調整(原本通り)
\ruby[||j>]{關}{くわん}に
% \ruby{玄關}{げん|くわん}に
たゞ
\ruby{我}{われ}
ひとり
\改行% 校正作業の簡略化のため
、
%
\原本頁{46-6}\改行%
\ruby[g]{兀然}{つゝくり}として
\ruby{坐}{すわ}り
\ruby{居}{を}れば、
%
\ruby{我}{わ}が
\ruby[g]{影子}{か げ }
\ruby{淋}{さび}しく
\ruby[||j>]{{\換字{古}}}{ふる}
\ruby[||j>]{疊}{だゝみ}に
% \ruby{{\換字{古}}疊}{ふる|だゝみ}に
\ruby{浸}{し}みて、
%
\ruby[g]{偶然}{ふ と }
\ruby{見}{み}れば
\ruby{低}{ひく}く
\ruby{吊}{つ}りたる
\ruby[g]{電燈}{でんとう}の
\ruby[g]{蓋裏}{かさうら}に、
%
\ruby[g]{{\換字{弱}}々}{よわ〳〵}としたる
\ruby{白}{しろ}き
\ruby{蛾}{が}の、
%
\ruby{蝶}{てふ}
と
\原本頁{46-9}\改行%
いふほども
\ruby{無}{な}く
\ruby{小}{ちひさ}なるが、
%
やがて
\ruby[||j>]{力}{ちから}
\ruby[||j>]{盡}{ つ}きての
\ruby{身}{み}の
\ruby{果}{はて}をも
\ruby{思}{おも}はず
\改行% 校正作業の簡略化のため
、
%
\原本頁{46-10}\改行%
\ruby{飛}{と}んでは
\ruby{止}{と}まり、
%
\ruby{止}{と}まつては
\ruby{飛}{と}びて
\ruby{狂}{くる}ひ
\ruby{居}{を}れり。

\原本頁{46-11}%
\ruby{待}{ま}つこと
\ruby[g]{少時}{しばし }して
\ruby{間}{あひ}の
\ruby{劃}{しきり}の
\ruby[g]{唐紙}{からかみ}を
がらりと
\ruby{明}{あ}けて、
%
\ruby[g]{書生}{しよせい}は
\ruby{復}{ふたゝ}び
\ruby{入}{い}り
\ruby{來}{きた}りぬ。

\原本頁{47-2}%
『
\ruby[g]{何樣}{ど う }も
\ruby{他}{ほか}の
\ruby[g]{病家}{びやうか}の
\ruby[g]{都合}{つ がふ}も
あつて
\ruby{出}{で}られぬと
\ruby{仰}{おつし}ある。
%
\ruby{氣}{き}の
\ruby{毒}{どく}だけれども
\ruby{他}{ほか}へ
\ruby{行}{い}つて
\ruby{下}{くだ}さい。
』

\原本頁{47-4}%
\ruby[g]{言葉}{ことば }の
\ruby{柔}{やさ}しくなりたるだけに
\ruby[g]{拒{\換字{絕}}}{きよぜつ}の
\ruby{意}{こゝろ}は
いよ〳〵
\ruby{堅}{かた}し。
%
さりとて
\ruby{病}{や}める
\ruby{五十子}{い|そ|こ}が
\ruby{曾}{かつ}てより
\ruby{信}{しん}じて、
%
\ruby[g]{苦悶}{く もん}の
\ruby{床}{とこ}の
\ruby{上}{うへ}の
\ruby[||j>]{獨}{ひとり}
\ruby[||j>]{語}{ ごと}に
\ruby{頼}{たの}みたしと
いひしは、
%
たゞ
\ruby{此}{こ}の
\ruby{家}{いへ}の
\ruby[g]{主人}{あるじ }なるを、
%
いづくにか
\ruby{行}{ゆ}き
\ruby[g]{他人}{ひ と }を
\ruby{頼}{たの}まん。
%
\ruby[g]{水野}{みづの }は
ほとほと% 原本は非踊り字表記
\ruby{行}{ゆ}き
\ruby{詰}{つ}まりて、
%
\ruby[g]{言葉}{ことば }も
\ruby{無}{な}く
\ruby{力}{ちから}も
\ruby{無}{な}く
\ruby{首}{かうべ}を
\ruby{垂}{た}れしが、
%
\ruby{搏}{はたゝ}き
\ruby{已}{や}めぬ
\ruby{彼}{か}の
\ruby{白}{しろ}き
\ruby{蛾}{が}の、
%
\ruby[g]{電燈}{あかり }の
\ruby[g]{周圍}{まはり }を
\ruby{飛}{と}び
\原本頁{47-9}\改行%
\ruby{{\換字{廻}}}{めぐ}る
\ruby{其}{そ}の
\ruby{陰翳眼}{か|げ|め}の
\ruby{{\換字{前}}}{まへ}に
ちら〳〵と
\ruby{落}{お}つれば、
%
\ruby{噫}{あゝ}、
%
\ruby{我}{われ}も
\ruby{取}{と}りかぬる
\ruby{燈}{ひ}の
\ruby[g]{{\換字{近}}傍}{かたはら}を、
%
\ruby{{\換字{猶}}}{なほ}
\ruby{去}{さ}らぬ
\ruby{蟲}{むし}と
\ruby{愚}{おろか}にも
\ruby{愚}{おろか}なれど、
%
\ruby[g]{甲{\換字{斐}}}{か ひ }
\ruby{無}{な}くも
\ruby{飛}{と}び
\原本頁{47-11}\改行%
\ruby{直}{なほ}し〳〵するごとく、
%
\ruby[g]{言葉}{ことば }を
\ruby{換}{か}へて
\ruby{頼}{たの}みて
\ruby{見}{み}んと、
%
\ruby{其}{その}
\ruby{場}{ば}は% 原文通り「場」
\ruby{立}{た}たんともせざる
\ruby{折}{をり}から、
%
\原本頁{48-1}%
\ruby{奧}{おく}の
\ruby{方}{はう}より
\ruby{丁}{ちやう}といふ
\ruby[g]{石子}{い し }の
\ruby{響}{ひゞ}き、
%
\ruby{確}{たしか}に
\ruby{人}{ひと}の
\ruby{碁}{ご}を
\ruby{打}{う}てる
\ruby{音}{おと}の、
%
\ruby{幽}{かすか}に
\ruby[g]{此方}{こ なた}に% ルビ調整(原本通り)
\ruby{聞}{きこ}えたり。

\Entry{其八}

\ruby{人}{ひと}おの〳〵
\ruby{我}{わ}が
\ruby{娯樂}{たの|しみ}に
\ruby{使}{つか}はれるは
\ruby{無}{な}し。
\ruby{中}{なか}にも
\ruby{碁好}{ご|ずき}は
\ruby{聖}{せい}に
\ruby{{\換字{近}}}{ちか}く
\ruby{愚}{ぐ}に
\ruby{{\換字{近}}}{ちか}く、
\ruby{假}{かり}の
\ruby{與奪}{やり|とり}の
\ruby{白黑}{しろ|くろ}の
\ruby{石}{いし}に、
\ruby{氣}{き}を
\ruby{{\換字{遣}}}{つか}ひ
\ruby{心}{こゝろ}を
\ruby{苦}{くるし}めて
\ruby{一切}{いつ|さい}を
\ruby{忘}{わす}れ
\ruby{果}{は}て、
\ruby{一寸}{いつ|すん}の
\ruby{暇}{ひま}を
\ruby{偸}{ぬす}んで
\ruby{始}{はじ}めし
\ruby{爭戰}{あら|そひ}にも、
\ruby{思}{おも}はず
\ruby{{\換字{半}}日}{はん|にち}の
\ruby{尻}{しり}を
\ruby{腐}{くさ}らせて
\ruby{悔}{くや}まぬが
\ruby{常}{つね}なり。
されば
\ruby{殆}{ほとん}ど
\ruby{一日}{いち|にち}の
\ruby{忙}{せは}しき
\ruby{業務}{つと|め}を
\ruby{{\換字{終}}}{を}へし
\ruby{擧句}{あげ|く}、
\ruby[<h||]{心}{こゝろ}
\ruby{蘇生}{よみ|が}へる
\ruby{晩餐}{ばん|さん}の
\ruby{小酌}{せう|しやく}の
\ruby{後}{のち}に、
\ruby{憎}{にく}くも
\ruby{可愛}{かは|ゆ}くもある
\ruby{其敵}{その|てき}を
\ruby{得}{え}て、
\ruby{罪無}{つみ|な}き
\ruby{樂}{たのし}みを
\ruby{取}{と}る
\ruby{一手}{いつ|て}
\g詰めruby{々々}{〳〵}の、
\ruby{興}{きよう}の
\ruby{極}{きは}めて
\ruby{旺}{さかん}なるところへ、
\ruby{熟知}{なじ|み}にもあらぬ
\ruby{病家}{びよ|うか}の、\換字{志}かも
\ruby{普通}{な|み}
\ruby{外}{はづ}れて
\ruby{{\換字{遠}}}{とほ}きより、
\ruby{夜陰}{や|いん}に
\ruby{及}{およ}びて
\ruby{呼}{よ}び
\ruby{{\換字{迎}}}{むか}へんとするとも、
\ruby{門{\換字{前}}}{もん|ぜん}の
\ruby{雀羅}{じや|くら}、
\ruby{藥局}{やく|きよく}の
\ruby{蛛網}{しゆ|まう}、
\ruby{客}{きやく}に
\ruby{饑}{う}ゑきつたる
\ruby{庸醫}{よう|い}はいざ
\ruby{知}{し}らず、
\ruby{苟}{いやし}くも
\ruby{名}{な}の
\ruby{{\換字{通}}}{とほ}つたるほどの
\ruby{人}{ひと}の
\ruby{應}{おう}ぜざるべきは、
\ruby{思}{おも}へば
\ruby{無理}{む|り}も
\ruby{無}{な}き
\ruby{事{\換字{情}}}{わ|け}なりと、
\ruby{鈍}{にぶ}からぬ
\ruby{水野}{みづ|の}は
\ruby{早}{はや}くも
\ruby{悟}{さと}りしが、
\ruby{物}{もの}に
\ruby{脆}{もろ}からぬ
\ruby{性質}{せい|しつ}の
\ruby{{\換字{猶}}}{なほ}
\ruby{思}{おも}ひ
\ruby{棄}{す}てず、
\ruby{何}{なに}をか
\ruby{考}{かんが}へ
\ruby{得}{え}しや
\ruby{此度}{こ|たび}は
\ruby{氣輕}{き|がる}く、

『ヤ、たび〳〵
\ruby{御面倒}{ご|めん|だう}を
\ruby{願}{ねが}ひまして、
\ruby{有}{あ}り
\ruby{難}{がた}うございました。
』

と、
\ruby{云}{い}ひながら
\ruby{多少錢}{いく|ら|か}を
\ruby{手早}{て|ばや}く
\ruby{白色包}{か|み|づゝみ}にして、

『
\ruby{煙草}{た|ばこ}でも
\ruby{購}{と}つて
\ruby{參}{まゐ}つて
\ruby{獻}{あ}げるべきですが。
』

と、
\ruby{言葉}{こと|ば}を
\ruby{{\換字{飾}}}{かざ}つて
\ruby{取}{と}りつくろひ、
\ruby{流石}{さす|が}
\ruby{手}{て}を
\ruby{出}{いだ}しては
\ruby{取}{と}りかぬるを
\ruby{無理}{む|り}やりに
\ruby{握}{にぎ}らすれば、まさかに
\ruby{投}{な}げ
\ruby{{\換字{返}}}{かへ}すこともせず、

『どうも
\ruby{御氣}{お|き}の
\ruby{毒}{どく}で、』

と、
\ruby{我}{わ}が
\ruby{師}{し}の
\ruby{{\換字{迎}}}{むかへ}に
\ruby{應}{おう}ぜぬが
\ruby{氣}{き}の
\ruby{毒}{どく}なやら、
\ruby{我}{わ}が
\ruby{錢}{ぜに}
\ruby{使}{つか}はせしが
\ruby{氣}{き}の
\ruby{毒}{どく}なやら、どちら
\ruby{付}{つ}かぬ
\ruby{挨拶}{あい|さつ}して、うぢ〳〵と
\ruby{取}{と}りぬ。

\ruby{印}{いん}を
\ruby{結}{むす}び、
\ruby{呪}{じゆ}を
\ruby{誦}{じゆ}すること、
\ruby{今}{いま}は
\ruby{流行}{は|や}らず、
\ruby{世}{よ}にたゞ
\ruby{錢{\換字{術}}}{せん|じゆつ}ありて
\ruby{神}{かみ}に
\ruby{{\換字{通}}}{つう}ずるを、
\ruby{知}{し}らぬほど
\ruby{迂闊}{うく|わつ}にはあらざりし
\ruby{水野}{みづ|の}は、
\ruby{書生}{しよ|せい}が
\ruby{我}{わ}が
\ruby{人{\換字{情}}錢}{こゝ|ろ|づけ}を
\ruby{収}{おさ}めしを
\ruby{見}{み}て、

『
\ruby{何樣}{ど|う}いふものでございましやう?
\ruby{病人}{びやう|にん}が
\ruby{思}{おも}ひ
\ruby{込}{こ}んで
\ruby{居}{を}るのでございますから、
\ruby{一度}{いち|ど}だけなりと
\ruby{診}{み}て
\ruby{戴}{いたゞ}く
\ruby{譯}{わけ}には
\ruby{參}{まゐ}りますまいか。
こちらの
\ruby{先生}{せん|せい}の
\ruby{事}{こと}でございますから、
\ruby{澤山}{たく|さん}の
\ruby{御病家}{ご|びやう|か}の
\ruby{御都合}{ご|つ|がふ}もあつて、
\ruby{御暇}{お|ひま}の
\ruby{少}{すく}ないのは
\ruby{承知}{しよ|うち}して
\ruby{居}{を}りますから、
\ruby{始{\換字{終}}}{し|ゞう}
\ruby{來}{き}て
\ruby{戴}{いたゞ}きたいとは
\ruby{申}{まを}しますまいが、
\ruby{只一度}{たつた|いち|ど}おいでなすつて
\ruby{下}{くだ}さるほどの
\ruby{事}{こと}なら、
\ruby{然程}{さ|ほど}
\ruby{御暇}{お|ひま}の
\ruby{取}{と}れるでは
\ruby{無}{な}し、
\ruby{御都合}{ご|つ|がふ}の
\ruby{出來}{で|き}ぬでも
\ruby{無}{な}からうと
\ruby{存}{ぞん}じます。
\ruby{一度}{いち|ど}でも
\ruby{御診察}{ご|しん|さつ}
\ruby{下}{くだ}すつて、そして
\ruby{御指揮}{おさ|し|づ}を
\ruby{仕}{し}て
\ruby{戴}{いたゞ}いたら、あとは
\ruby{村醫}{そん|い}でも
\ruby{間}{ま}に
\ruby{合}{あ}はうかと
\ruby{存}{ぞん}じますが、
\ruby{病人}{びやう|にん}も
\ruby{信}{しん}じて
\ruby{居}{を}りませぬ
\ruby{村醫}{そん|い}ばかりでは、
\ruby{實以}{じつ|もつ}て
\ruby{傍観}{わき|め}にも
\ruby{案}{あん}じられまして、
\ruby{癒}{なほ}るものも
\ruby{癒}{なほ}るまいかと
\ruby{心配致}{しん|ぱい|いた}します。
\ruby{貴君}{あな|た}には
\ruby{御無理}{ご|む|り}を
\ruby{申}{まを}して
\ruby{濟}{す}みませんが、
\ruby{折}{を}り
\ruby{入}{い}つて
\ruby{一}{ひと}つ
\ruby{此}{こ}の
\ruby{譯}{わけ}を
\ruby{仰}{おつし}あつて、も
\ruby{一度}{いち|ど}
\ruby{何卒}{どう|ぞ}
\ruby{御願}{お|ねが}ひなすつて
\ruby{見}{み}てはいたゞけますまいか。
』

と
\ruby{泣}{な}かぬばかりに
\ruby{掻口{\換字{説}}}{かき|く|ど}けば、
\ruby{書生}{しよ|せい}の
\ruby{面}{おもて}には
\ruby{難色}{なん|しよく}
\ruby{見}{み}えしが、
\ruby{既}{すで}に
\ruby{毒}{どく}を
\ruby{盛}{も}られたれば
\ruby{爭}{あらそ}ひ
\ruby{難}{がた}く、
\ruby{無下}{む|げ}に
\ruby{酷}{むご}くは
\ruby{斥}{しりぞ}けかねて、

『では
\ruby{始{\換字{終}}}{し|ゞう}
\ruby{病人}{びやう|にん}を
\ruby{受合}{うけ|あ}つて
\ruby{吳}{く}れといふのでは
\ruby{無}{な}くつて、
\ruby{診斷}{しん|だん}だけで
\ruby{好}{い}いからといふのぢやネ。
』

『ハイ、それで
\ruby{滿足}{まん|ぞく}
\ruby{致}{いた}しませうと
\ruby{申}{まを}しますのですから、
\ruby{何卒}{どう|か}
\ruby{枉}{ま}げて
\ruby{御聞入}{お|きゝ|い}れ
\ruby{下}{くだ}さるやうに
\ruby{御願}{お|ねが}ひなすつて。
』

と
\ruby{一問一答}{いち|もん|いつ|たふ}の
\ruby{果}{は}てし
\ruby{後}{のち}、
\ruby{澁}{しぶ}る〳〵
\ruby{{\換字{弱}}}{よわ}つた
\ruby{氣色}{け|しき}して
\ruby{奥}{おく}へ
\ruby{行}{ゆ}きぬ。
\ruby{水野}{みづ|の}は
\ruby{病}{や}める
\ruby{我}{わ}が
\ruby{五十子}{い|そ|こ}が
\ruby{物憂}{もの|う}げに、
\ruby{此}{こ}の
\ruby{廣}{ひろ}き
\ruby{世}{よ}に
\ruby{只一人}{たゞ|ひ|とり}の
\ruby{誠意}{まこ|と}ある
\ruby{介抱者}{かい|はう|しや}をも
\ruby{有}{も}たずして、
\ruby{頼}{たの}み
\ruby{少}{すくな}き
\ruby{村醫}{そん|い}の
\ruby{怪}{あや}しき
\ruby{藥}{くすり}をのみ
\ruby{力}{ちから}としつゝ、
\ruby{心淋}{こゝろ|さび}しくも
\ruby{秋}{あき}の
\ruby{夜}{よ}
\ruby{悲}{かな}しき
\ruby{田舎家}{ゐ|な|か}の
\ruby{一室}{ひと|ま}の
\ruby{内}{うち}に
\ruby{横}{よこた}はれる
\ruby{光景}{あり|さま}を
\ruby{胸}{むね}のうちに
\ruby{描}{ゑが}きながら、こたびの
\ruby{{\換字{返}}事}{へん|じ}は
\ruby{如何}{い|か}にぞと、
\ruby{聞}{き}く
\ruby{耳}{みゝ}
\ruby{立}{た}てゝ
\ruby{意}{こゝろ}を
\ruby{注}{つ}くれば、

『うるさい!。
\換字{志}つゝこい!。
』

と
\ruby{叱}{しか}る
\ruby{聲}{こゑ}に
\ruby{次}{つ}いで、
\ruby{負}{ま}けかゝりたるに
\ruby{怒}{いかり}をや
\ruby{含}{ふく}みけん、パチリと
\ruby{{\換字{強}}}{つよ}く
\ruby{石}{いし}を
\ruby{下}{くだ}す
\ruby{音}{おと}して、やがて
\ruby{書生}{しよ|せい}は
\ruby{膨}{ふく}れかへつて
\ruby{出}{い}で
\ruby{來}{きた}りぬ。
\ruby{挨拶}{あい|さつ}は
\ruby{聞}{き}かずとも
\ruby{既}{はや}
\ruby{解}{わか}りたり。
されど
\ruby{如是}{か|く}ても
\ruby{水野}{みづ|の}は
\ruby{屈}{くつ}せず、
\ruby{書生}{しよ|せい}が
\ruby{何}{なに}を
\ruby{云}{い}ひしやらも
\ruby{知}{し}らずに、
\ruby{如何}{い|か}にしてか
\ruby{我}{わ}が
\ruby{念}{おもひ}を
\ruby{{\換字{遂}}}{と}げんと
\ruby{考}{かんが}へ
\ruby{沈}{しづ}みし
\ruby{後}{のち}、
\ruby{思}{おも}ひ
\ruby{得}{え}しところやありけん
\ruby{頭}{かうべ}を
\ruby{擡}{あ}げしが、
\ruby{其}{そ}の
\ruby{面}{おもて}は
\ruby{何時}{い|つ}か
\ruby{聊}{いさゝ}か
\ruby{色}{いろ}ざし
\ruby{來}{きた}り、
\ruby{其}{そ}の
\ruby{眼}{め}よりは
\ruby{今}{いま}まで
\ruby{潛}{ひそ}み
\ruby{居}{ゐ}たりし
\ruby{烱々}{けい|〳〵}たる
\ruby{光}{ひかり}の
\ruby{閃}{ひらめ}き
\ruby{出}{い}でゝ、
\ruby{見}{み}る〳〵
\ruby{如何}{い|か}なる
\ruby{任務}{つと|め}にも
\ruby{堪}{た}ふべく、
\ruby{如何}{い|か}なる
\ruby{人}{ひと}にも
\ruby{爭}{あらそ}つて
\ruby{{\換字{勝}}}{か}つべき
\ruby{峻烈}{しゆん|れつ}の
\ruby{氣象}{き|しやう}を
\ruby{現}{あらは}し
\ruby{出}{いだ}しぬ。
\ruby{折}{をり}から
\ruby{一}{ひと}つの
\ruby{彼}{か}の
\ruby{小}{ちひ}さき
\ruby{蛾}{が}は、
\ruby{力盡}{ちから|つ}き
\ruby{翼傷}{つばさ|きず}つきて
\ruby{翩々}{ひら|〳〵}として、
\ruby{落花}{らく|くわ}の
\ruby{枝}{えだ}を
\ruby{辭}{じ}せしが
\ruby{如}{ごと}くに、あはれにも
\ruby{水野}{みづ|の}が
\ruby{膝}{ひざ}の
\ruby{{\換字{前}}}{まへ}に
\ruby{墜}{お}ちぬ。

\Entry{其九}

% メモ 校正終了 2024-03-30 2024-05-23 2024-06-17
\原本頁{53-9}%
\ruby[||j>]{頭}{かしら}を
\ruby{下}{さ}げ
\ruby[g]{言葉}{ことば }を
\ruby{低}{ひく}くして、
%
\ruby{頼}{たの}む
ほどは
\ruby{頼}{たの}み
\ruby{盡}{つく}せしを、
%
\ruby[g]{膠無}{にべな }く
\ruby{色}{いろ}なく
\ruby[g]{斷り}{ことわ }に
\ruby{斷}{ことわ}られたり。
%
\ruby{今}{いま}は
\ruby{復}{また}
\ruby[||j>]{言}{ものい}ふべき
\ruby[g]{餘地}{よ ち }も
\ruby{無}{な}からんを、
%
\ruby[g]{水野}{みづの }は
そも〳〵
\ruby{何}{なん}とせんとかする。

\原本頁{54-2}%
\ruby{水}{みづ}をもて
\ruby{解}{と}くべからざるものは
\ruby{火}{ひ}をもて
\ruby{熔}{と}かすべし、
%
\ruby{刀}{たう}をもて
\ruby{截}{き}り
\ruby{{\換字{難}}}{がた}きものは
\ruby{槌}{つち}をもて
\ruby{碎}{くだ}き
\ruby{得}{え}ん。
%
\ruby{求}{もと}めて
\ruby{已}{や}まぬ
\ruby[g]{願望}{ねがひ }の
\ruby{心}{こゝろ}あれば、
%
おのづと
\ruby{働}{はたら}く
\ruby[g]{智慧}{ち ゑ }の
\ruby{眼}{まなこ}は、
%
\ruby{我}{わ}が
\ruby{思}{おも}へる
\ruby{地}{ち}に
\ruby{到}{いた}らんとするに
\ruby[g]{{\換字{平}}和}{おだやか}なる
\ruby{路}{みち}を
\ruby{取}{と}ることの
\ruby[g]{甲{\換字{斐}}}{か ひ }
\ruby{無}{な}きを
\ruby{悟}{さと}りたらん
\ruby[<j>]{曉}{あかつき}、
%
いかで
\ruby{{\換字{猶}}}{なほ}
\ruby{別}{べつ}に
\ruby{峻}{さか}しき
\ruby{一}{ひ}ト
\ruby{條}{すぢ}の
\ruby{徑}{こみち}ありて
\ruby[g]{其處}{そ こ }に
\ruby{{\換字{通}}}{つう}ずるを
\ruby[g]{見出}{み いだ}さゞらんや。
%
\ruby[g]{水野}{みづの }は
\ruby{今}{いま}その
\ruby{峻}{さか}しきを
\ruby[g]{見出}{み いだ}して
\ruby{攀}{よ}ぢ
\ruby{上}{のぼ}らんとするなり。
%
\ruby{火}{ひ}の
\ruby{力}{ちから}、
%
\ruby{槌}{つち}の
\ruby{力}{ちから}を
\ruby{試}{こゝろ}みんとするなり。

\原本頁{54-9}%
\ruby{其}{そ}の
\ruby{顏}{かほ}つきの
\ruby{變}{かは}れるが
\ruby{如}{ごと}くに、
%
\ruby[g]{言葉}{ことば }の
\ruby[g]{調子}{てうし }も
\ruby{俄}{にはか}に
\ruby{變}{かは}り、
%
\ruby{聲}{こゑ}も
\換字{志}たゝかに
\ruby{大}{おおき}くなりぬ。

% \原本頁{54-11}%
『
いよ〳〵
\ruby[g]{先生}{せんせい}は
\ruby{御來臨}{お|い|で}
\ruby{下}{くだ}さらんと
\ruby{仰}{おつし}あるのですか。
%
イヤ、
%
それは
\ruby[g]{失禮}{しつれい}ながら
\ruby[g]{左樣}{さ う }では
ございますまい、
%
\ruby{御取次}{お|とり|つぎ}の
\ruby{御言葉}{お|こと|ば}が
\ruby{足}{た}らんので、
%
\ruby[g]{先生}{せんせい}に
\ruby{御理解}{お|わか|り}が
\ruby{無}{な}いのでしやう。
%
\ruby[g]{{\換字{遠}}方}{ゑんぱう}だから
\ruby{行}{い}つて
\ruby{{\換字{遣}}}{や}らぬと、
%
そんな
\ruby{事}{こと}を
\ruby{仰}{おつし}ある
\ruby[g]{先生}{せんせい}では
\ruby{無}{な}い、
%
そんな
\ruby{無慈悲}{む|じ|ひ}な
\ruby[g]{先生}{せんせい}では
\ruby{無}{な}い。
%
‥‥』

\原本頁{55-5}%
と、
%
\ruby{今}{いま}までは
\ruby{頭}{あたま}の
\ruby{低}{ひく}かりし
\ruby{男}{をとこ}の、
%
\ruby{居{\換字{丈}}高}{ゐ|たけ|だか}になつて、
%
\ruby{思}{おも}ひの
\ruby{外}{ほか}なる
\ruby[g]{{\換字{強}}言}{しひごと}を
\ruby{云}{い}ひ
\ruby{出}{いだ}せば、
%
\ruby[g]{書生}{しよせい}は
\ruby{其}{そ}の
\ruby[g]{意外}{いぐわい}なるに
\ruby{度}{ど}を
\ruby{失}{うしな}つて、
%
\ruby[g]{{\換字{狼}}狽}{うろた }へながらも
\ruby[g]{怫然}{ふつぜん}として、
%
\ruby{急}{きふ}に
\ruby{{\換字{遮}}}{さへぎ}り
\ruby{止}{とゞ}めんと、

\原本頁{55-8}%
『
バ、
%
バ、
%
\ruby[g]{馬鹿}{ば か }な
\ruby{事}{こと}を、
』

\原本頁{55-9}%
と、
%
\ruby[g]{眞赤}{まつか }になりて
\ruby[g]{抗辯}{あらが }はんとしけるが、% 弁 瓣 辦 辧 辨 辩 (辯)
%
\ruby[g]{紫電}{し でん}
\ruby{閃}{ひら}めきて
\ruby{出}{い}づるが
\ruby{如}{ごと}き
\ruby[g]{水野}{みづの }が
\ruby{恐}{おそ}ろしき
\ruby{眼}{め}に
\ruby{眼}{め}を
\ruby[g]{見合}{み あは}せて、
%
\ruby{睨}{にら}み
\ruby{殺}{ころ}さん
ばかりに
\ruby{我}{われ}を
\ruby[g]{見据}{み す }ゑたる
\ruby{其}{そ}の
\ruby{異}{あや}しき
\ruby{力}{ちから}に
\ruby[g]{{\換字{所}}以}{いはれ }
\ruby{無}{な}くも
\ruby[g]{氣壓}{け お }され、
%
\ruby{云}{い}ひ
\ruby[g]{甲{\換字{斐}}}{が ひ }
\ruby{無}{な}くも
\ruby{當}{あた}り
\ruby{{\換字{難}}}{がた}く
おぼえて、
%
\ruby{我}{われ}
\ruby{知}{し}らず
\ruby{面}{おもて}を
\ruby[g]{背向}{そ む }け
\ruby[g]{言葉}{ことば }を
\ruby{吞}{の}みたり。
%
\原本頁{56-2}%
\ruby[g]{水野}{みづの }は
\ruby[g]{相手}{あひて }の
たぢろぎしに
\ruby{{\換字{緩}}}{ゆる}みを
\ruby{吳}{く}れず、
%
\ruby[g]{往來}{わうらい}にも
\ruby{鳴}{な}り
\ruby{渡}{わた}れ、
%
\ruby{奧}{おく}にも
\ruby{響}{ひゞ}けと、
%
いよ〳〵
\ruby{聲}{こゑ}を
\ruby{高}{たか}め、
%
\ruby[g]{言葉}{ことば }を
\ruby{荒}{あら}くして、

\原本頁{56-4}%
『
\ruby{御當家}{こ|ち|ら}の
\ruby[g]{先生}{せんせい}は
\ruby[g]{仁慈}{なさけ }
\ruby{深}{ぶか}い
\ruby[g]{先生}{せんせい}だ、
%
\ruby[g]{取次}{とりつぎ}の
\ruby{君}{きみ}がまだ
\ruby[g]{新參}{しんざん}で、
%
\ruby{御}{こ}
\ruby{當}{ち}
\ruby{家}{ら}の
\ruby{御風儀}{ご|ふう|ぎ}を
\ruby{知}{し}らんので、
%
\ruby[g]{中{\換字{途}}}{ちゆうと}で
\ruby[g]{間{\換字{違}}}{ま ちが}つた
\ruby[g]{忠義}{ちゆうぎ}% 原本通り(ちゆう)(国会図書館 コマ番号 32/134 p56 l5)
\ruby{立}{だて}で
\ruby{計}{はか}らつて
\改行% 校正作業の簡略化のため
、
%
\原本頁{56-6}\改行%
\ruby[g]{其樣}{そ ん }な
\ruby{好}{い}い
\ruby[g]{加減}{か げん}な
\ruby{事}{こと}を
\ruby[g]{御言}{お い }ひのだ。
%
\ruby{御慈悲}{お|じ|ひ}
\ruby{深}{ぶか}い
\ruby[g]{此方}{こちら }の% ルビ調整(原本通り)
\ruby[g]{先生}{せんせい}だもの、
%
\ruby[g]{{\換字{遠}}方}{ゑんぱう}だつて
\ruby{來}{き}て
\ruby{下}{くだ}さるのだ。
%
\ruby[g]{世間}{せ けん}に
\ruby{有}{あ}り
\ruby{觸}{ふ}れた
\ruby[g]{藥賣}{くすりう}り
\ruby[g]{坊主}{ばうず }
\原本頁{56-8}\改行%
と、
%
\ruby[g]{此方}{こちら }の% ルビ調整(原本通り)
\ruby[g]{先生}{せんせい}とは
\ruby{譯}{わけ}が
\ruby{{\換字{違}}}{ちが}ふ。
%
\ruby[||j>]{商}{しやう}
\ruby[||j>]{賣}{ ばい}づく
% \ruby{商賣}{しやう|ばい}づく
ばかりで
\ruby[||j>]{病}{びやう}
\ruby[||j>]{人}{ にん}を
% \ruby{病人}{びやう|にん}を
いぢる
\改行% 校正作業の簡略化のため
、
%
\原本頁{56-9}\改行%
\ruby[g]{其樣}{そ ん }な
\ruby[g]{卑劣}{ひ れつ}くさい
\ruby[g]{先生}{せんせい}では
\ruby{無}{な}いのだ、
%
\ruby[g]{先生}{せんせい}の
\ruby{御性{\換字{分}}}{ご|しやう|ぶん}の
\ruby{美}{うつく}しい
\ruby{御慈悲}{お|じ|ひ}
\ruby{深}{ぶか}いのは
\ruby{誰}{たれ}でも% 国会図書館 コマ番号 32/134 p56 l10
\ruby{知}{し}つて
\ruby{居}{ゐ}る。
%
\ruby[g]{他人}{ひ と }も
\ruby{知}{し}つて
\ruby{居}{ゐ}る、
%
\ruby[g]{自{\換字{分}}}{じ ぶん}も
\ruby{知}{し}つて
\ruby{居}{ゐ}る。
%
\ruby[g]{先生}{せんせい}で
\ruby{無}{な}くちやあ
ならんと
\ruby{云}{い}つて、
%
\ruby[g]{御願}{お ねが}ひ
\ruby{申}{まを}すのに
\原本頁{57-1}\改行%
\ruby{來}{き}て
\ruby{下}{くだ}さらん、
%
そんな
\ruby[g]{仁慈}{なさけ }の
\ruby{無}{な}い
\ruby[g]{先生}{せんせい}では
\ruby{無}{な}い。
%
\ruby[g]{先生}{せんせい}の
\ruby{御氣性}{ご|き|しやう}も
\ruby{知}{し}らないで、
%
\ruby{何}{なに}を
\ruby[g]{寢惚}{ね とぼ}けた
\ruby[g]{挨拶}{あいさつ}を
するのだ。
』

\原本頁{57-3}%
と、
%
\ruby{口}{くち}も
\ruby{開}{あ}かせず
\ruby{疊}{たゝ}みかけて、
%
\ruby{{\換字{猶}}}{なほ}も
\ruby{止}{と}め
\ruby{度}{ど}
\ruby{無}{な}く
\ruby{罵}{のゝし}らんとす。
%
\ruby{此}{こ}の
\ruby{時}{とき}
\ruby[||j>]{藥}{やく}
\ruby[||j>]{局}{きよく}の
% \ruby{藥局}{やく|きよく}の
\ruby{内}{うち}
こと〳〵と
\ruby{音}{おと}して、
%
\ruby[g]{物騷}{ものさわ}がしき
\ruby[g]{此場}{このば }の% 原文通り「場」
\ruby[g]{樣子}{やうす }を、
%
\ruby[g]{何事}{なにごと}かと
\ruby{他}{た}の
\ruby[g]{書生}{しよせい}の
\ruby{覗}{うかゞ}ひに
\ruby{來}{き}しと
おぼしく、
%
\ruby{{\換字{又}}}{また}
\ruby{今}{いま}の
\ruby{間}{ま}に
\ruby{來}{き}し
\ruby{二三人}{に|さん|にん}の
\ruby[g]{藥取}{くすりと}りは、
%
こそ〳〵と
\ruby{隅}{すみ}の
\ruby{方}{かた}に
\ruby{潛}{ひそ}み% 【潛 u6f5b 「先先」】【潜 u6f5c 「夫夫」】併用されている
\ruby{居}{ゐ}て
\ruby[g]{成行}{なりゆき}を
\ruby{見}{み}、
%
はや
\ruby{門}{もん}の
\ruby{外}{そと}には
ちらり
ほらりと、
%
\ruby{人}{ひと}さへ
\ruby{立}{た}ちて
\ruby[g]{見居}{み ゐ }るさまなり。

\原本頁{57-8}%
\ruby[g]{書生}{しよせい}は
\ruby{心}{こゝろ}も
\ruby{心}{こゝろ}ならず、

\原本頁{57-9}%
『
マア
\ruby[g]{左樣}{そ う }
\ruby{大}{おほき}な
\ruby{聲}{こゑ}を
\ruby{立}{た}てゝは
\ruby{困}{こま}るぢや
\ruby{無}{な}いか。
』

\原本頁{57-10}%
と、
\ruby{制}{せい}すれども
\ruby{耳}{みゝ}にも
\ruby{入}{い}るれば
こそ、

\原本頁{57-11}%
『
つまり
\ruby{君}{きみ}のやうな
\ruby[g]{取次}{とりつぎ}は
\ruby[g]{先生}{せんせい}の
\ruby{不利益}{ふ|た|め}だ、
%
\ruby[g]{先生}{せんせい}の
\ruby[||j>]{{\換字{評}}}{ひやう}
\ruby[||j>]{{\換字{判}}}{ ばん}を
% \ruby{{\換字{評}}{\換字{判}}}{ひやう|ばん}を
\ruby{惡}{わる}くする。
%
\ruby[g]{{\換字{技}}{\換字{術}}}{わ ざ }ばかり
\ruby{良}{よ}い
\ruby[g]{先生}{せんせい}では
\ruby{無}{な}い、
%
\ruby[g]{御優}{お やさ}しいので
\ruby[g]{人徳}{にんとく}のある
\原本頁{58-2}\改行%
\ruby[g]{先生}{せんせい}を
それぢやあ
\ruby[g]{臺無}{だいな }しに
\ruby{仕}{し}て
\ruby[g]{仕舞}{し ま }ふでは
\ruby{無}{な}いか、
%
さつさと
\ruby{{\換字{猶}}}{も}
\ruby[g]{一度}{いちど }
\ruby{奧}{おく}へ
\ruby{行}{い}つて
\ruby{願}{ねが}つて
\ruby{來}{き}てくれ。
%
\ruby{願}{ねが}ひ
\ruby{直}{なほ}して
\ruby{吳}{く}れなければ
\ruby[g]{此處}{こ ゝ }は
\ruby{動}{うご}かん。
%
\ruby[||j>]{病}{びやう}
\ruby[||j>]{人}{ にん}が
% \ruby{病人}{びやう|にん}が
\ruby[g]{先生}{せんせい}で
\ruby{無}{な}ければと
\ruby{云}{い}つて
\ruby{首}{くび}を
\ruby{{\換字{延}}}{の}ばして
\ruby{待}{ま}つて
\原本頁{58-5}\改行%
\ruby{居}{ゐ}るのだ、
%
\ruby[g]{先生}{せんせい}の
\ruby[g]{御供}{お とも}を
\ruby{仕}{し}て
\ruby{歸}{かへ}らなけりやあ
\ruby[g]{此處}{こ ゝ }は
\ruby{動}{うご}かん。
%
\ruby[g]{書生}{しよせい}の
\ruby{癖}{くせ}に
\ruby{有}{あ}る
\ruby[g]{間敷}{ま じき}
\ruby{事}{こと}だ。
%
\ruby{碁}{ご}なぞに
\ruby{凝}{こ}つて
\ruby{居}{ゐ}るやうだから
\ruby[g]{取次}{とりつぎ}が
\原本頁{58-7}\改行%
\ruby[g]{間{\換字{違}}}{ま ちが}ふのだ。
%
さあ
\ruby[g]{確乎}{しつかり}として
\ruby[g]{先生}{せんせい}に
\ruby{願}{ねが}つて
\ruby{見}{み}て
\ruby{吳}{く}れ。
%
うるさい
\改行% 校正作業の簡略化のため
、
%
\原本頁{58-8}\改行%
\換字{志}つゝこい、
%
とは
\ruby{何}{なん}の
\ruby{事}{こと}だ。
%
\換字{志}つゝこい
\ruby[g]{人間}{にんげん}に
\ruby{恨}{うら}まれたら、
%
\ruby[g]{先生}{せんせい}に
\ruby{飛}{と}んだ
\ruby{御{\換字{迷}}惑}{ご|めい|わく}が
\ruby{掛}{かゝ}らう、
%
\ruby{祟}{たゝ}りかね
\ruby{無}{な}いものだと
\ruby{思}{おも}ふか。
』

\原本頁{58-10}%
と、
%
\ruby[g]{次第}{し だい}〳〵に
\ruby[g]{聲高}{こわだか}に
\ruby{云}{い}へば、
%
\ruby[||j>]{門}{もん}
\ruby[||j>]{外}{ぐわい}に
% \ruby{門外}{もん|ぐわい}に
\ruby{人}{ひと}は
\ruby[g]{愈々}{いよ〳〵}
\ruby{嵩}{かさ}みて、
%
\ruby{奧}{おく}の
\ruby{方}{かた}は
\ruby{人}{ひと}の
\ruby{氣}{け}もせず
\ruby[g]{靜謐}{しづか }になりぬ。

\原本頁{59-1}%
\ruby{時}{とき}に
\ruby[g]{此室}{こ ゝ }と
\ruby{奧}{おく}との
\ruby[g]{劃域}{しきり }は
するりと
\ruby{開}{あ}いて、
%
\ruby[g]{立出}{たちい }でたる
\ruby{{\換字{猶}}}{なほ}
\ruby{{\換字{若}}}{わか}き
\ruby{此}{この}
\原本頁{59-2}\改行%
\ruby{家}{や}の
\ruby[g]{主人}{しゆじん}は、
%
\ruby[g]{福々}{ふく〴〵}しく
\ruby{肥}{ふと}りたる
\ruby{其}{その}
\ruby{顏}{かほ}に、
%
\ruby[g]{莞爾}{にこやか}なる
\ruby{笑}{ゑみ}を
つくりて
\改行% 校正作業の簡略化のため
、

\原本頁{59-3}%
『
ヤ、
%
\ruby[g]{取次}{とりつぎ}のものを
\ruby[g]{御叱}{お しか}りでは
\ruby{恐}{おそ}れ
\ruby{入}{い}る。
%
\ruby{直}{すぐ}と
\ruby{今}{いま}から
\ruby{出}{で}ますから、
%
さあ
\ruby{一}{ひ}
ト
\ruby{足}{あし}
\ruby[g]{御先}{お さき}へ。
%
\ruby[g]{相田}{あひだ }!、
%
\ruby{{\換字{所}}}{ところ}は
\ruby{{\換字{分}}}{わか}つて
\ruby{居}{ゐ}るだらうな、
%
ム
\原本頁{59-5}\改行%
ヽ
\ruby[g]{左樣}{さ う }か、
%
\ruby{直}{すぐ}と
\ruby{車}{くるま}の
\ruby[g]{支度}{し たく}を
させろ。
』

\原本頁{59-6}%
と、
%
\ruby[||j>]{卒}{そつ}
\ruby[||j>]{直}{ちよく}に
% \ruby{卒直}{そつ|ちよく}に
\ruby[g]{水野}{みづの }に
\ruby[g]{滿足}{まんぞく}を
\ruby{與}{あた}へぬ。

\原本頁{59-7}%
\ruby[g]{水野}{みづの }は、
%
\ruby{此}{こ}の
\ruby{己}{おのれ}に
\ruby{克}{か}つことを
\ruby{知}{し}つて
\ruby{非}{ひ}を
\ruby{{\換字{遂}}}{と}げん
ともせざる
\ruby[g]{良醫}{りやうい}の
\ruby{{\換字{前}}}{まへ}に、
%
\ruby{心}{こゝろ}よりの
\ruby[g]{{\換字{感}}謝}{かんしや}の
\ruby{禮}{れい}を
\ruby[g]{深々}{ふか〴〵}と
\ruby{施}{ほどこ}して、
%
\ruby{欣}{よろこ}び
\ruby{勇}{いさ}んで
\ruby[g]{室外}{おもて }に
\ruby{出}{い}でぬ。

\原本頁{59-10}%
\ruby{惡}{あし}き
\ruby{兆}{しるし}かと
\ruby{忌}{いま}はしかりし
\ruby{彼}{か}の
\ruby{蛾}{が}の
\ruby{弄}{なぶ}りし
\ruby[g]{電燈}{でんとう}の
\ruby{下}{した}は
\ruby{去}{さ}つて、
%
\ruby[||j>]{藍}{らん}
\ruby[||j>]{色}{しよく}
% \ruby{藍色}{らん|しよく}
\ruby[||g>]{滴}{したゝ}るが% ルビ調整(長いルビ対策)(る)を親文字に加える
\ruby{如}{ごと}き
\ruby{澄}{す}みたる
\ruby{天}{そら}に、
%
\ruby{星}{ほし}は
\ruby{梨子地}{な|し|じ}を
\ruby{描}{か}きたらんやうに
\ruby{光}{ひか}り
\ruby{輝}{かゞや}けるを、
%
\ruby{振}{ふ}り
\ruby{仰}{あふ}ぎて
\ruby{眺}{なが}めたる
\ruby[g]{可憐}{か れん}の
\ruby[g]{水野}{みづの }は、
%
\ruby{我}{わ}が
\ruby{意}{こゝろ}の
\ruby{中}{うち}の
\ruby{其}{その}
\ruby{人}{ひと}のために、
%
\ruby{思}{おも}ふ
\ruby{事}{こと}
\ruby{{\換字{遂}}}{と}げたる
\ruby{嬉}{うれ}しさに
\ruby[||j>]{頭}{かしら}
\ruby[||j>]{高}{ たか}き
\ruby[g]{心地}{こゝち }して、
%
\ruby[g]{水色}{みづいろ}の
\ruby{光}{ひか}り
\ruby{特}{こと}に
\ruby{優}{すぐ}れたる
\ruby{一}{ひと}つの
\ruby{星}{ほし}に
\ruby{眼}{まなこ}を
\ruby{止}{とゞ}めて、
%
\ruby[g]{少時}{しばし }は
\ruby[g]{人知}{ひとし }らぬ
\ruby{胸}{むね}の
\ruby{凉}{すゞ}しさを
\ruby{味}{あぢは}ひたり。

\Entry{其十}

\原本頁{}
\ruby{先挽後推}{さき|びき|あと|おし}の
\ruby{勢}{いきほひ}よく、
%
\ruby{矢}{や}を
\ruby{射}{い}る
\ruby{如}{ごと}くに
\ruby{走}{はし}れる
\ruby{相良}{さが|ら}の
\ruby{車}{くるま}は、
%
\ruby{長橋}{ちやう|けう}を
\ruby{東}{ひがし}に
\ruby{渡}{わた}つて
\ruby{小梅}{こ|うめ}にかゝり、
%
\ruby{引舟{\換字{通}}}{ひき|ふね|どほ}りを
\ruby{眞直}{まつ|すぐ}に
\ruby{北}{きた}へと、
%
\ruby{夜風}{よ|かぜ}のやや
\ruby{{\換字{寒}}}{さむ}きを
\ruby{衝}{つ}いて
\ruby{{\換字{進}}}{すゝ}みに
\ruby{{\換字{進}}}{すゝ}みぬ。
%
\ruby{{\換字{道}}}{みち}は
\ruby{砥}{と}の
\ruby{如}{ごと}し、
%
\ruby{人}{ひと}の
\ruby{往來}{ゆき|き}は
\ruby{無}{な}し、
%
\ruby{車夫}{しや|ふ}は
\ruby{脚一杯}{あし|いつ|ぱい}に
\ruby{駈}{か}くるほどに、
%
おほよその
\ruby{二里}{に|り}を
\ruby{瞬}{またゝ}く
\ruby{間}{ま}に
\ruby{{\換字{過}}}{す}ぎて、
%
\ruby{忽地}{たちま|ち}にして
\ruby{目}{め}ざす
\ruby{四}{よ}ツ
\ruby{木}{ぎ}へと
\ruby{着}{つ}きぬ。

\原本頁{}
\ruby{病人}{びやう|にん}の
\ruby{大切}{たい|せつ}さは
\ruby{{\換字{貧}}富}{ひん|ぷ}に
\ruby{關}{かゝ}はらぬ
\ruby{事}{こと}ながら、
%
\ruby{市街}{ま|ち}
\ruby{離}{はな}れたる
\ruby{{\換字{遠}}}{とほ}きところより、
%
\ruby{夜}{よ}にさへ
\ruby{入}{い}りたるに
\ruby{無理}{む|り}
\ruby{{\換字{強}}}{じひ}に
\ruby{{\換字{強}}}{し}ひて、
%
\ruby{我}{わ}が
\ruby{先生}{せん|せい}を
\ruby{{\換字{迎}}}{むか}へたるは、
%
\ruby{田舎}{ゐな|か}とは
\ruby{云}{い}へ、
%
\ruby{定}{さだ}めし
\ruby{門構}{もん|がま}への
\ruby{立派}{りつ|ぱ}に、
%
\ruby{庭{\換字{前}}}{には|さき}
\ruby{廣}{ひろ}く、
%
がつしりとしたる
\ruby{槻柱}{けやき|ばしら}の
\ruby{太}{ふと}きが、
%
\ruby{二尺}{に|しやく}も
\ruby{厚}{あつ}さのある
\ruby{茅葺屋根}{かや|ぶき|や|ね}のいと
\ruby{高}{たか}く
\ruby{大}{おほき}なるを
\ruby{支}{さゝ}へたるやうの
\ruby{家}{いへ}ならんと、
%
\ruby{車夫}{しや|ふ}は
\ruby{心}{こゝろ}の
\ruby{中}{うち}に
\ruby{算}{つも}り
\ruby{居}{ゐ}けるが、
%
\ruby{{\換字{分}}}{わか}り
\ruby{{\換字{兼}}}{か}ぬる
\ruby{闇}{やみ}の
\ruby{村逕}{むら|みち}を
\ruby{{\換字{迷}}}{まよ}ひ〳〵て、
%
やうやくに
\ruby{{\換字{尋}}}{たづ}ね
\ruby{當}{あ}てたるは
\ruby{是}{これ}は
\ruby{如何}{い|か}な
\ruby{事}{こと}、
%
\ruby{{\換字{寒}}竹}{かん|ちく}の
\ruby{藪疊}{やぶ|だゝみ}の
\ruby{不體裁}{ぶ|ざ|ま}に
\ruby{歪}{ゆが}みたる
\ruby{其}{そ}の
\ruby{構}{かまへ}の
\ruby{中}{うち}こそは
\ruby{意外}{い|ぐわい}に
\ruby{濶}{ひろ}けれ、
%
\ruby{{\換字{空}}}{むな}しく
\ruby{明}{あ}け
\ruby{置}{お}く
\ruby{地}{ち}を
\ruby{惜}{をし}んでか、
%
\ruby{{\換字{通}}}{かよ}ひ
\ruby{路}{ぢ}をも
\ruby{埋}{うづ}むるまでに
\ruby{作}{つく}りたる
\ruby{芋}{いも}の
\ruby{圃}{はたけ}の
\ruby{奧}{おく}に、
%
\ruby{微}{かす}けき
\ruby{星}{ほし}のひかりを
\ruby{{\換字{浴}}}{あ}びて
\ruby{黑}{くろ}みて
\ruby{立}{た}てる、
%
\ruby{見}{み}るからが
\ruby{悲}{かな}しき
\ruby{草}{くさ}の
\ruby{屋}{や}なり。

\原本頁{}
\ruby{餘}{あま}りの
\ruby{思}{おも}はくの
\ruby{{\換字{違}}}{ちが}ひの
\ruby{忌々}{いま|〳〵}しくてや、
%
\ruby{車夫}{しや|ふ}は
\ruby{憚}{はゞか}り% 「憚 は(ゞ)か」
\ruby{氣}{げ}
\ruby{無}{な}く
\ruby{人力車}{く|る|ま}を
\ruby{挽}{ひ}き
\ruby{入}{い}るれば、
%
\ruby{車輪}{しや|りん}に
\ruby{觸}{ふ}るゝ
\ruby{芋}{いも}の
\ruby{葉}{は}は
\ruby{左右}{さ|いう}に
\ruby{開}{ひら}けて、
%
\ruby{湛}{たゝ}へられし
\ruby{露}{つゆ}の
\ruby{珠}{たま}は
\ruby{墜}{お}ちて
\ruby{聲}{こゑ}あり。

\原本頁{}
\ruby{人}{ひと}ありや
\ruby{無}{な}しや
\ruby{岑閑}{しん|かん}として、
%
たゞ
\ruby{燈}{ひ}のみ
\ruby{洩}{も}るゝ
\ruby{板{\換字{戸}}}{いた|ど}を
\ruby{敲}{たゝ}き
\ruby{驚}{おどろ}かしつゝ
\ruby{車夫}{しや|ふ}は
\ruby{聲}{こゑ}
\ruby{明}{あき}らかにそれと
\ruby{云}{い}ひ
\ruby{入}{い}るれば、
%
\ruby{何}{なに}を
\ruby{擱}{さしお}きても
\ruby{飛}{と}んで
\ruby{出}{い}でゝ、
%
\ruby{喜}{よろこ}び〳〵て
\ruby{{\換字{迎}}}{むか}へ
\ruby{入}{い}るべきを、
%
\ruby{是}{これ}はまた
\ruby{何}{なん}たる
\ruby{事}{こと}ぞ
\ruby{沈着}{おち|つ}き
\ruby{拂}{はら}つて、

\原本頁{}
『ハア、
%
\ruby{左樣}{さ|う}ですかい!。
』

\原本頁{}
と、
%
\ruby{田舎}{ゐな|か}
\ruby{詞}{ことば}の
\ruby{素氣無}{す|げ|な}く
\ruby{答}{こた}へたるのみにて
\ruby{嬉}{うれ}しき
\ruby{顏}{かほ}もせねば、
%
\ruby{{\換字{請}}}{しやう}じ
\ruby{入}{い}れんともせず、
%
\ruby{折}{をり}から
\ruby{自裂}{は|じ}け
\ruby{{\換字{兼}}}{か}ねたる
\ruby{大豆}{ま|め}の
\ruby{莢}{さや}を
\ruby{取}{と}るにやあらん、
%
\ruby{箕}{み}を
\ruby{{\換字{前}}}{まへ}にして
\ruby{乾}{かは}きたる
\ruby{豆}{まめ}を
\ruby{弄}{いぢ}り
\ruby{居}{ゐ}し
\ruby{婆}{ばゞ}の、
%
\ruby{面}{おもて}は
\ruby{赭黄色}{あか|き|いろ}く
\ruby{焦}{や}け
\ruby{皺}{しわ}びて、
%
\ruby{髮}{かみ}は
\ruby{天蠶糸屑}{て|ぐ|す|くず}の
\ruby{如}{ごと}く
\ruby{白}{しろ}く
\ruby{光}{ひか}るが
\ruby{{\換字{交}}}{まじ}れる、
%
\ruby{年}{とし}の
\ruby{頃}{ころ}は
\ruby{六十}{ろく|じう}ばかりなるが、
%
\ruby{不承不承}{ふ|しよう|ぶ|しよう}に
\ruby{身}{み}を
\ruby{起}{おこ}して
\ruby{{\換字{戸}}口}{と|ぐち}に
\ruby{立塞}{たち|ふさ}がり、

\原本頁{}
『
\ruby{病人}{びやう|にん}は
\ruby{此處}{こ|ゝ}には
\ruby{居}{を}りましねえ。
%
\ruby{別室}{はな|れ}の
\ruby{方}{はう}に
\ruby{寢}{ね}て
\ruby{居}{を}りますから、
%
\ruby{直}{すぐ}とそつちらへ
\ruby{御座}{ご|ざ}らしつて
\ruby{下}{くだ}さい。
%
\ruby{暗}{くら}くつて
\ruby{{\換字{分}}}{わか}りますまいが
\ruby{足元}{あし|もと}は
\ruby{好}{い}いでがす。
%
\ruby{家}{うち}へさへ
\ruby{付}{つ}いて
\ruby{{\換字{廻}}}{まは}れば
\ruby{直}{ぢき}でがすよ。
%
あ、\換字{志}かし
\ruby{{\換字{菜}}}{な}
\ruby{圃}{ばたけ}へでも
\ruby{轉}{ころ}げられると
\ruby{詰}{つま}らない。
%
\ruby{水野}{みづ|の}さんが
\ruby{後}{あと}になつたゞから
\ruby{仕方}{し|かた}が
\ruby{無}{な}い、
%
\ruby{妾}{わし}が
\ruby{案内}{あん|ない}を
\ruby{仕}{し}てあげやう。
%
ヤ、
%
\ruby{車夫}{くる|まや}さん、
%
\ruby{提灯}{ちやう|ちん}があるの、
%
\ruby{其}{そ}の
\ruby{提灯}{ちやう|ちん}を
\ruby{妾}{わし}に
\ruby{貸}{か}さつせえ。
%
さあ
\ruby{先生}{せん|せい}さん、
%
\ruby{妾}{わし}に
\ruby{隨}{つ}いて
\ruby{御坐}{ご|ざ}らつせえ。
』

\原本頁{}
と、
%
\ruby{藁草履}{わら|ざう|り}つゝかけて
\ruby{先}{さき}に
\ruby{立}{た}つたり。
%
\ruby{相良}{さが|ら}は
\ruby{是非無}{ぜ|ひ|な}く
\ruby{後}{あと}に
\ruby{隨}{つ}きて、
%
\ruby{家}{いへ}の
\ruby{横手}{よこ|て}を
\ruby{斜}{なゝめ}に
\ruby{奧}{おく}へ、
%
\ruby{此方}{こ|なた}には
\ruby{燃料}{たき|れう}の
\ruby{柴木}{しば|き}の
\ruby{積}{つ}まれ、
%
\ruby{彼方}{かな|た}には
\ruby{玉蜀黍幹}{たう|もろ|こし|がら}の
\ruby{埒無}{らち|な}く
\ruby{置}{お}かれなどしたる
\ruby{間}{あひだ}を
\ruby{縫}{ぬ}ひて、
%
さて、
%
\ruby{下}{した}は
\ruby{夏蒔}{なつ|まき}の
\ruby{{\換字{菜}}}{な}の
\ruby{圃}{はたけ}の
\ruby{細徑}{ほそ|みち}の
\ruby{滑}{すべ}り
\ruby{易}{やす}く、
%
\ruby{上}{うへ}は
\ruby{柹}{かき}の
\ruby{樹}{き}の
\ruby{幾本}{いく|ほん}の
\ruby{枝低}{えだ|ひく}くして
\ruby{帽子}{ばう|し}
\ruby{危}{あやふ}きところを
\ruby{{\換字{過}}}{す}ぐれば、
%
\ruby{{\換字{前}}}{まへ}の
\ruby{家}{いへ}よりは
\ruby{彼}{かれ}
\ruby{是}{これ}
\ruby{二十間餘}{に|じう|けん|あま}りも
\ruby{距}{はな}れたりとおぼしきところに、
%
\ruby{椎}{しひ}の
\ruby{樹}{き}ならん
\ruby{眞黑}{まつ|くろ}に
\ruby{見}{み}ゆる
\ruby{{\換字{丈}}矮}{たけ|ひく}き
\ruby{樹}{き}のいと
\ruby{大}{おほい}なるを
\ruby{後楯}{うしろ|だて}に
\ruby{取}{と}りて、
%
\ruby{僅}{わづか}に
\ruby{二}{ふ}タ
\ruby{室}{ま}ほどなるべき
\ruby{離屋}{はな|れや}
\ruby{立}{た}てり。

\原本頁{}
『さあ
\ruby{此處}{こ|ゝ}でがあす、
%
\ruby{上}{あが}つて
\ruby{下}{くだ}さい。
』

\原本頁{}
と、
%
\ruby{婆}{ばゞ}は
\ruby{{\換字{戸}}}{と}を
\ruby{引}{ひ}き
\ruby{明}{あ}けてつか〳〵と
\ruby{上}{あが}りぬ。

\原本頁{}
『お
\ruby{{\換字{前}}}{まへ}さまが
\ruby{頼}{たの}み
\ruby{度}{た}いと
\ruby{云}{い}つた
\ruby{先生}{せん|せい}がござらしつた。
』

\原本頁{}
と、
%
\ruby{云}{い}ひながら
\ruby{次}{つぎ}の
\ruby{室}{ま}の
\ruby{長四疊}{なが|よ|でふ}を
\ruby{{\換字{過}}}{す}ぎて、
%
\ruby{六疊}{ろく|でふ}の
\ruby{其}{そ}の
\ruby{室}{ま}に
\ruby{至}{いた}りたれど、
%
\ruby{熱}{ねつ}の
\ruby{一}{ひ}ト
\ruby{{\換字{退}}}{ひき}
\ruby{{\換字{退}}}{ひ}きし
\ruby{汐合}{しほ|あひ}の
\ruby{時}{とき}にや、
%
\ruby{病人}{びやう|にん}は
\ruby{答}{こた}へも
\ruby{無}{な}く
\ruby{音}{おと}も
\ruby{無}{な}く
\ruby{眠}{ねむ}り
\ruby{居}{を}れり。

\原本頁{}
\ruby{醫師}{い|し}は
\ruby{婆}{ばゞ}につゞきて
\ruby{上}{あが}りけるが、
%
\ruby{先}{ま}ず
\ruby{此}{こ}の
\ruby{室}{ま}に
\ruby{籠}{こも}りたる
\ruby{不快}{ふ|くわい}の
\ruby{臭氣}{にほ|い}に、
%
\ruby{不審}{ふ|しん}の
\ruby{眉}{まゆ}を
\ruby{顰}{ひそ}めて\換字{志}ろりと
\ruby{見渡}{み|わた}せば、
%
\ruby{廣}{ひろ}からぬ
\ruby{一室}{ひと|ま}の
\ruby{内}{うち}
\ruby{法外}{はふ|ぐわい}に
\ruby{明}{あか}るく、
%
\ruby{病人}{びやう|にん}が
\ruby{枕上}{まくら|もと}の
\ruby{洋燈}{らん|ぷ}は
\ruby{何時}{いつ|か}か
\ruby{燃}{も}え
\ruby{高}{かう}じて、
%
\ruby{其}{そ}の
\ruby{火屋}{ほ|や}の
\ruby{上}{うへ}の
\ruby{方}{かた}は
\ruby{眞黑}{まつ|くろ}に
\ruby{煤}{すゝ}け、
%
\ruby{毒々}{どく|〴〵}しき
\ruby{黑}{くろ}き
\ruby{油{\換字{煙}}}{ゆ|えん}は
\ruby{今}{いま}やしたゝかに
\ruby{舞}{ま}ひ
\ruby{上}{あが}り
\ruby{居}{を}れり。

\原本頁{}
『オーヤ、
%
\ruby{洋燈}{らん|ぷ}が
\ruby{出{\換字{過}}}{で|す}ぎて
\ruby{居}{ゐ}る!。
%
\ruby{何}{なん}とマア
\ruby{危}{あぶな}い
\ruby{事}{こと}だつた!。
%
いくら
\ruby{病人}{びやう|にん}だつて、
%
\ruby{意氣地}{い|く|ぢ}が
\ruby{無}{な}いつて、
%
ハア、
%
\ruby{此樣}{こ|ん}な
\ruby{事}{こと}つて
\ruby{有}{あ}る
\ruby{譯}{わけ}で
\ruby{無}{な}い。
』

\原本頁{}
と
\ruby{婆}{ばゞ}は
\ruby{獨語}{ひとり|ごと}して
\ruby{其}{そ}の
\ruby{心}{しん}を
\ruby{引{\換字{込}}}{ひつ|こ}ませぬ。

\原本頁{}
\ruby{臭氣}{にほ|い}の
\ruby{源}{もと}は
\ruby{仔細無}{し|さい|な}き
\ruby{事}{こと}なりけるが、
%
\ruby{惱}{なや}み
\ruby{疲}{つか}れし
\ruby{後}{のち}の
\ruby{睡}{ねむ}りたる
\ruby{間}{ま}に、
%
\ruby{洋燈}{らん|ぷ}はおのづと
\ruby{燃}{も}え
\ruby{高}{かう}じて、\換字{志}たゝかに
\ruby{憫然}{あは|れ}なる
\ruby{人}{ひと}に
\ruby{惡氣}{あく|き}をや
\ruby{吸}{す}はせけん。
%
\ruby{相良}{さが|ら}は
\ruby{眼}{ま}のあたりに
\ruby{見}{み}たる
\ruby{此}{こ}の
\ruby{一事}{ひと|こと}と、
%
\ruby{婆}{ばゞ}が
\ruby{今}{いま}
\ruby{洩}{も}らしたる
\ruby{其}{そ}の
\ruby{一語}{ひと|こと}とに、
%
\ruby{誰}{たれ}
\ruby{看護}{み|まも}るものも
\ruby{無}{な}き
\ruby{此}{こ}の
\ruby{病人}{びやう|にん}の、
%
\ruby{何病}{なに|びやう}に
\ruby{惱}{なや}めるかはいざ
\ruby{知}{し}らず、
%
\ruby{萬般}{よろ|づ}のあはれさ
\ruby{推}{お}し
\ruby{測}{はか}り
\ruby{知}{し}られて、
%
\ruby{他}{ひと}の
\ruby{憂}{うき}を
\ruby{見}{み}るには
\ruby{馴}{な}れたる
\ruby{身}{み}も、
%
\ruby{先}{ま}づ
\ruby{惻然}{そく|ぜん}として
\ruby{心}{こゝろ}を
\ruby{動}{うご}かしぬ。


\Entry{其十一}

\ruby{思}{おも}ふまゝに
\ruby{世}{よ}を
\ruby{振舞}{ふる|ま}ふは
\ruby{下人}{げ|にん}の
\ruby{常}{つね}なり。
\ruby{相良}{さが|ら}の
\ruby{車夫等}{しや|ふ|ら}は
\ruby{此}{こ}の
\ruby{狀態}{やう|す}に
\ruby{呆}{あき}れ
\ruby{果}{は}てゝ、せめては
\ruby{番茶}{ばん|ちや}なりと
\ruby{飮}{の}んで
\ruby{寢}{ね}ころんで
\ruby{寛}{くつろ}がんと、
\ruby{母家}{おも|や}をさして
\ruby{戾}{もど}りけるが、

『
\ruby{何}{なん}と
\ruby{飛}{と}んだところへ
\ruby{來}{き}たぢや
\ruby{無}{ね}えか。
とても
\ruby{眼}{め}も
\ruby{鼻}{はな}も
\ruby{明}{あ}きさうぢや
\ruby{無}{ね}えぜ。
』

『ハヽヽ、あの
\ruby{婆}{ばあ}さんは
\ruby{大方}{おほ|かた}、
\ruby{御醫者}{お|い|しや}さんの
\ruby{御抱}{お|かゝへ}は
\ruby{澤山}{たく|さん}
\ruby{給金}{きふ|きん}を
\ruby{取}{と}るだらう
\ruby{位}{ぐらゐ}に
\ruby{思}{おも}つて
\ruby{居}{ゐ}るだらうよ。
』

『ウツ、
\ruby{{\換字{違}}}{ちげ}へ
\ruby{無}{ね}え。
\ruby{一體}{いつ|たい}
\ruby{吾家}{う|ち}の
\ruby{先生}{せん|せい}は
\ruby{人}{ひと}が
\ruby{好{\換字{過}}}{よ|す}ぎるからナア。
\ruby{此方等}{こ|ち|とら}あ
\ruby{何樣}{ど|う}したつて
\ruby{取}{と}るものあ
\ruby{取}{と}るが、
\ruby{先生}{せん|せい}が
\ruby{第一}{だい|いち}
\ruby{馬鹿}{ば|か}を
\ruby{見}{み}らあ。
』

と
\ruby{闇}{やみ}にはびこる
\ruby{胴魔聲}{どう|ま|ごゑ}
\ruby{太}{ふと}く、
\ruby{{\換字{遠}}慮}{ゑん|りよ}も
\ruby{無}{な}く
\ruby{二人}{ふた|り}して
\ruby{喚}{わめ}き
\ruby{散}{ち}らしたり。

\ruby{婆}{ばゞ}は
\ruby{此等}{これ|ら}の
\ruby{聲}{こゑ}を
\ruby{聞}{き}かざりしと
\ruby{見}{み}ゆ。
いたはり
\ruby{氣}{げ}も
\ruby{無}{な}く
\ruby{病人}{びやう|にん}を
\ruby{搖}{ゆ}り
\ruby{起}{おこ}こして、

『お
\ruby{{\換字{前}}}{めへ}さまが
\ruby{頼}{たの}みたいと
\ruby{云}{い}つた
\ruby{先生}{せん|せい}が
\ruby{御坐}{ご|ざ}らしつたよ。
』

と、
\ruby{同}{おな}じ
\ruby{言葉}{こと|ば}を
\ruby{冷}{ひや}やかに
\ruby{繰}{く}り
\ruby{{\換字{返}}}{かへ}しつ、
\ruby{重}{おも}き
\ruby{眶}{まぶた}を
\ruby{力}{ちから}
\ruby{無}{な}く
\ruby{擧}{あ}げて、

\ruby{微}{かすか}に
\ruby{點頭}{うな|づく}を
\ruby{見}{み}るより、

『さあ
\ruby{先生樣}{せん|せい|さん}、
\ruby{見}{み}てやつて
\ruby{下}{くだ}さい。
\ruby{濟}{す}んだらば
\ruby{別}{べつ}に
\ruby{水}{みづ}はあげ
\ruby{無}{な}いから、
\ruby{其處}{そ|こ}の
\ruby{椽先}{えん|さき}の
\ruby{手水鉢}{て|うづ|ばち}で、
\ruby{{\換字{勝}}手}{かつ|て}に
\ruby{手}{て}を
\ruby{洗}{あら}ふが
\ruby{可}{い}いでがあす。
ナアニ
\ruby{一昨日}{を|とゝ|ひ}
\ruby{汲}{く}んだばかりで、
\ruby{誰}{たれ}も
\ruby{使}{つか}はないから
\ruby{奇麗}{き|れい}でがあすよ。
そして
\ruby{彼方}{あつ|ち}へ
\ruby{寄}{よ}つて
\ruby{溫茶}{ぬる|ちや}でも
\ruby{上}{あが}らつしやい。
どれ
\ruby{妾}{わたし}は
\ruby{先}{さき}へ
\ruby{行}{い}つて
\ruby{火}{ひ}でも
\ruby{燃}{た}きましやう。
』

と、
\ruby{他人同士}{た|にん|どう|し}とは
\ruby{本}{もと}より
\ruby{一目}{ひと|め}にも
\ruby{知}{し}れわたりたれど、さりとては
\ruby{乾}{かわ}き
\ruby{切}{き}つたる
\ruby{心}{こゝろ}の
\ruby{鬼々}{おに|〳〵}しくも
\ruby{人{\換字{情}}}{なさ|け}
\ruby{無}{な}き
\ruby{婆}{ばゞ}かな、と
\ruby{竊}{ひそか}に
\ruby{驚}{おどろ}ける
\ruby{相良}{さが|ら}を
\ruby{後}{あと}にして、
\ruby{恰}{あたか}も
\ruby{機關仕掛}{ぜん|まい|じ|かけ}の
\ruby{人形}{にん|ぎやう}かなんぞの
\ruby{動}{うご}くやうに、
\ruby{四圍}{あた|り}への
\ruby{斟{\換字{酌}}}{しん|しやく}も
\ruby{氣{\換字{兼}}}{き|がね}も
\ruby{無}{な}く、
\ruby{我}{わ}が
\ruby{行}{ゆ}かんとする
\ruby{方}{かた}へ
\ruby{早{\換字{速}}}{さつ|さ}と
\ruby{行}{ゆ}きぬ。

『
\ruby{水野}{みづ|の}さんが
\ruby{居}{ゐ}ないで、ハア
\ruby{餘計}{よ|けい}な
\ruby{暇潰}{ひまつ|ゝぶ}しな。
アヽ
\ruby{江{\換字{戸}}}{え|ど}の
\ruby{人}{ひと}と
\ruby{挨拶}{あい|さつ}するのは
\ruby{面倒}{めん|だう}な。
』

と、つぶやきながら
\ruby{婆}{ばゞ}は
\ruby{火}{ひ}を
\ruby{焚}{た}きはじめたり。

\ruby{急}{いそ}ぎに
\ruby{急}{いそ}ぎて
\ruby{今}{いま}
\ruby{歸}{かへ}り
\ruby{來}{きた}れる
\ruby{水野}{みづ|の}は、
\ruby{額}{ひたひ}に
\ruby{汗}{あせ}の
\ruby{玉}{たま}を
\ruby{散}{ち}らして、
\ruby{蒸}{む}されたるが
\ruby{如}{ごと}くになりたる
\ruby{面}{おもて}は、
\ruby{薄紅}{うす|くれなゐ}に
\ruby{血}{ち}の
\ruby{色}{いろ}
\ruby{潮}{さ}したれば、
\ruby{引}{ひ}き
\ruby{立}{た}つて
\ruby{見}{み}ゆる
\ruby{眉目}{び|もく}のあたりに
\ruby{淸秀}{せい|しう}の
\ruby{氣}{き}
\ruby{滿}{み}ち
\ruby{溢}{あふ}れて、これこそ
\ruby{水野}{みづ|の}が
\ruby{往時}{むか|し}の
\ruby{面貌}{おも|わ}かと、
\ruby{天晴}{あつ|ぱ}れ
\ruby{美}{うつく}しく
\ruby{生々}{いき|〳〵}としたり。
\ruby{早}{はや}くも
\ruby{既}{すで}に
\ruby{相良}{さが|ら}の
\ruby{見}{み}えたるに
\ruby{欣}{よろこ}び
\ruby{悅}{よろこ}び、
\ruby{取}{と}り
\ruby{敢}{あ}へず
\ruby{先}{ま}ず
\ruby{車夫}{しや|ふ}を
\ruby{犒}{ねぎら}ひて
\ruby{手當}{て|あて}を
\ruby{與}{あた}へ、
\ruby{{\換字{更}}}{さら}に
\ruby{病室}{びやう|しつ}には
\ruby{行}{ゆ}かんともせずして、こゝに
\ruby{數{\換字{分}}間}{すう|ふん|かん}の
\ruby{後}{のち}
\ruby{我}{わ}が
\ruby{受}{う}くべき
\ruby{吉凶}{きつ|きよう}いづれかの
\ruby{報告}{しら|せ}の、
\ruby{醫}{い}によつて
\ruby{齎}{もた}らさるべきを
\ruby{恐}{おそ}る
\ruby{懼}{おそ}る
\ruby{待}{ま}ちたり。

\ruby{程經}{ほど|へ}て
\ruby{相良}{さが|ら}は
\ruby{歸}{かへ}り
\ruby{來}{きた}りぬ。
むさくろしき
\ruby{此}{こ}の
\ruby{婆}{ばゞ}が
\ruby{茶}{ちや}の
\ruby{間}{ま}の
\ruby{中}{うち}にて、
\ruby{水野}{みづ|の}と
\ruby{互}{たがひ}に
\ruby{挨拶}{あい|さつ}して、さて
\ruby{婆}{ばゞ}と
\ruby{水野}{みづ|の}とに
\ruby{向}{むか}つて
\ruby{徐}{おもむ}ろに、
\ruby{病人}{びやう|にん}の
\ruby{中々}{なか|〳〵}に
\ruby{重體}{ぢゆう|たい}なる
\ruby{事}{こと}、
\ruby{徴候}{ちよう|こう}の
\ruby{不完全}{ふ|くわん|ぜん}なるをもて
\ruby{今}{いま}までの
\ruby{醫}{い}は、
\ruby{何}{なん}と
\ruby{診斷}{しん|だん}したるか
\ruby{知}{し}らざれども、
\ruby{病氣}{びやう|き}は
\ruby{全}{まつた}く
\ruby{腸窒扶斯}{ちやう|ち|ぷ|す}なる
\ruby{事}{こと}、
\ruby{傳染}{でん|せん}の
\ruby{{\換字{虞}}}{おそれ}ある
\ruby{病氣}{びやう|き}なれば
\ruby{其}{そ}の
\ruby{心}{こゝろ}すべき
\ruby{事}{こと}、
\ruby{患者}{くわん|じや}のためには
\ruby{設備}{せつ|び}
\ruby{宜}{よろ}しき
\ruby{病院}{びやう|ゐん}に
\ruby{入}{い}らしむるを
\ruby{良}{よ}しとする
\ruby{事}{こと}、されども
\ruby{{\換字{遠}}路}{ゑん|ろ}を
\ruby{{\換字{伴}}}{ともな}ひ
\ruby{行}{ゆ}かんも
\ruby{難儀}{なん|ぎ}にして、
\ruby{聊}{いさゝ}か
\ruby{懸念}{け|ねん}も
\ruby{無}{な}きにあらねば、
\ruby{一軒建}{いつ|けん|だち}の
\ruby{離}{はな}れ
\ruby{家}{や}なるを
\ruby{幸}{さいは}ひ、
\ruby{彼處}{かし|こ}にて
\ruby{療養}{れう|やう}さするも
\ruby{惡}{あし}からぬ
\ruby{事}{こと}、たゞし
\ruby{此}{こ}の
\ruby{病}{やまひ}は
\ruby{藥劑}{くす|り}よりも
\ruby{寧}{むし}ろ
\ruby{看護}{かん|ご}の
\ruby{良否}{よし|あし}によりて、
\ruby{囘復}{くわい|ふく}すると
\ruby{爲}{せ}ざるとも
\ruby{生}{しやう}ずるもの
\ruby{故}{ゆゑ}、
\ruby{今}{いま}の
\ruby{如}{ごと}き
\ruby{狀}{さま}にては
\ruby{宜}{よろ}しからぬ
\ruby{事}{こと}、
\ruby{彼處}{かし|こ}にて
\ruby{其儘}{その|まゝ}
\ruby{療養}{れう|やう}せんには
\ruby{是非}{ぜ|ひ}とも
\ruby{智識}{ち|しき}
\ruby{經驗}{けい|けん}の
\ruby{十{\換字{分}}}{じう|ぶん}なる
\ruby{良看護{\換字{婦}}}{りやう|かん|ご|ふ}を
\ruby{添}{そ}ふべき
\ruby{事}{こと}、くれ〴〵も
\ruby{患者}{くわん|じや}をして
\ruby{{\換字{強}}}{つよ}き
\ruby{身動}{み|うご}きなど
\ruby{爲}{せ}しめざるやう、
\ruby{取}{と}り
\ruby{扱}{あつか}ひも
\ruby{極}{きは}めて
\ruby{手柔}{て|やはら}かにすべき
\ruby{事}{こと}、
\ruby{看護}{かん|ご}の
\ruby{力}{ちから}
\ruby{足}{た}らねば
\ruby{危}{あやふ}き
\ruby{事}{こと}、
\ruby{今}{いま}まで
\ruby{投劑}{とう|ざい}し
\ruby{居}{を}れる
\ruby{醫}{い}に
\ruby{此由}{この|よし}を
\ruby{語}{かた}りて、
\ruby{其}{そ}のつもりの
\ruby{處方}{しよ|はう}を
\ruby{乞}{こ}ひ、
\ruby{且}{か}つ
\ruby{種々}{いろ|〳〵}の
\ruby{注意}{ちゆう|い}を
\ruby{受}{う}くべき
\ruby{事}{こと}、
\ruby{其他}{その|た}さし
\ruby{當}{あた}つての
\ruby{樣々}{さま|〴〵}の
\ruby{處置}{しよ|ち}など、
\ruby{我}{わ}が
\ruby{職{\換字{分}}}{つと|め}の
\ruby{上}{うへ}より
\ruby{云}{い}ふべきほどの
\ruby{事}{こと}は、
\ruby{一々}{いち|〳〵}
\ruby{物柔}{もの|やは}らかに
\ruby{言}{い}ひ
\ruby{盡}{つく}して、
\ruby{御大切}{ご|たい|せつ}にと
\ruby{靜々}{しづ|〳〵}と
\ruby{歸}{かへ}りぬ。

\ruby{醫師}{い|し}が
\ruby{親切}{しん|せつ}の
\ruby{長々}{なが|〳〵}しき
\ruby{物語}{もの|がた}りの
\ruby{間}{あひだ}に、
\ruby{雲間}{くも|ま}の
\ruby{月}{つき}の
\ruby{如唯僅}{ごと|たゞ|わづか}の
\ruby{間}{あひだ}だけ
\ruby{美}{うつく}\換字{志}かりし
\ruby{水野}{みづ|の}は、
\ruby{其}{そ}の
\ruby{往時}{むか|し}の
\ruby{俤}{おもかげ}もいづくへやら、
\ruby{唇}{くちびる}は
\ruby{微}{かすか}に
\ruby{顚}{ふる}へて
\ruby{自然}{ひと|りで}に
\ruby{戰}{おのゝ}き、
\ruby{眼}{め}は
\ruby{洞然}{どう|ぜん}として
\ruby{何處}{いづ|く}を
\ruby{見}{み}るとも
\ruby{無}{な}く
\ruby{据}{すわ}りたるに、
\ruby{引}{ひ}きかへて
\ruby{冷酷}{れい|こく}なる
\ruby{主人}{ある|じ}の
\ruby{婆}{ばゞ}は、
\ruby{哂}{しや}れ
\ruby{{\換字{古}}}{ふる}したる
\ruby{木彫}{き|ぼり}の
\ruby{假面}{め|ん}の、いづくにも
\ruby{潤}{うるほ}ひの
\ruby{無}{な}きが
\ruby{如}{ごと}き
\ruby{顏}{かほ}して、

『
\ruby{傳染病}{うつ|りや|まひ}ぢやあハア
\ruby{大變}{たい|へん}な
\ruby{事}{こと}だ。
\ruby{死}{し}なれでも
\ruby{仕}{し}たらまあ、オヽ
\ruby{厭}{いや}な
\ruby{事}{こと}だ。
\ruby{早{\換字{速}}}{さつ|そく}に
\ruby{{\換字{逐}}}{ぼ}ひ
\ruby{出}{だ}して
\ruby{仕舞}{し|ま}は
\ruby{無}{な}けりやあ。
』

と、
\ruby{慈悲}{じ|ひ}も
\ruby{人{\換字{情}}}{なさ|け}も
\ruby{無}{な}く
\ruby{云}{い}ひ
\ruby{出}{いで}しさまは、たゞ
\ruby{地獄物語}{ぢ|ごく|もの|がたり}の
\ruby{奪衣婆}{だつ|え|ば}を、
\ruby{今}{いま}
\ruby{眼}{め}の
\ruby{{\換字{前}}}{まへ}に
\ruby{見}{み}るが
\ruby{如}{ごと}し。

\Entry{其十二}

% メモ 校正終了 2024-04-01 2024-05-24 2024-06-17
\原本頁{72-6}%
\ruby{重}{おも}き
\ruby[g]{風邪}{か ぜ }なりと
\ruby{村}{むら}の
\ruby{醫}{い}の
\ruby[g]{尾竹}{を だけ}の
\ruby{云}{い}ひし
\ruby{時}{とき}だに、
%
\ruby{其}{そ}の
\ruby[g]{容態}{ようだい}の
\ruby[g]{傍觀}{わきめ }にも
たゞならぬに、
%
\ruby{淺}{あさ}からず
\ruby{心}{こゝろ}をも
\ruby{使}{つか}ひ
\ruby{氣}{き}をも
\ruby{揉}{も}みしものを、
%
\ruby[g]{淺草}{あさくさ}
\ruby[g]{以北}{い ほく}にては
\ruby[g]{上無}{うへな }き
\ruby{人}{ひと}に
\ruby{頼}{たの}み
おもへる
\ruby[g]{相良}{さがら }に
\ruby{今}{いま}、
%
\ruby{病}{やまひ}は
これこれなり、
%
\ruby[g]{看護}{かんご }
\ruby{行}{ゆ}き
\ruby{屆}{とゞ}かずば% 「屆」「届」 原本通り「屆」
\ruby{危}{あやふ}からんと
\ruby{云}{い}はれては、
%
\ruby[g]{愕然}{がくぜん}として
\ruby[||j>]{打}{うち}
\ruby[||j>]{驚}{おどろ}きつ、
% \ruby{打驚}{うち|おどろ}きつ、
%
\ruby{胸}{むね}の
たゞ
\ruby{中}{なか}に
\ruby[g]{鐵槌}{てつゝゐ}の
\ruby[g]{一撃}{いちげき}を
\ruby{受}{う}けたるやう
おぼえて、
%
\原本頁{73-1}%
\ruby{我}{われ}
\ruby{先}{ま}づ
\ruby{死}{し}にも
すべく
\ruby{惱}{なや}ましきに、
%
\ruby[g]{垂死}{すゐし }の
\ruby{人}{ひと}を
\ruby{{\換字{逐}}}{お}ひ
\ruby{出}{いだ}さんと
いふ
\ruby[g]{苛酷}{いらひど}き
\ruby{婆}{ばゞ}の
\ruby[g]{言葉}{ことば }を
\ruby{聞}{き}きては、
%
\ruby[g]{怒火}{いかり }
\ruby[g]{心頭}{しんとう}に
\ruby{起}{おこ}つて
\ruby{堪}{た}ふるにも
\ruby{堪}{た}へられず、
%
\ruby{思}{おも}はず
\ruby{目}{め}に
\ruby{稜角立}{か|ど|た}てゝ
\ruby{峻}{けは}しく
\ruby{睨}{にら}みしが、
%
ハツト
\ruby{心}{こゝろ}づきて
\ruby{自}{みづか}ら
\ruby{警}{いまし}め、
%
\ruby{燃}{も}え
\ruby{立}{た}つ
\ruby[g]{瞋恚}{いかり }を
\ruby[g]{押鎭}{おしゝづ}め
\ruby[g]{押鎭}{おしゝづ}めて、
%
わざと
\ruby[g]{何氣}{なにげ }
\ruby{無}{な}く
\ruby{粧}{よそほ}ふ
\ruby[g]{言葉}{ことば }つき
\ruby[g]{{\換字{平}}穩}{なだらか}に、

\原本頁{73-6}%
『
そんな
\ruby{酷}{むご}らしいことを
\ruby{云}{い}つたつて
\ruby[g]{仕樣}{し やう}が
\ruby{無}{な}いぢや
\ruby{無}{な}いか、
%
\ruby{歩}{ある}けも
\ruby[g]{仕無}{し な }い
\ruby[||j>]{病}{びやう}
\ruby[||j>]{人}{ にん}を
% \ruby{病人}{びやう|にん}を
\ruby{{\換字{逐}}}{おひ}
\ruby{出}{た}すなんて。
』

\原本頁{73-8}%
と、
%
\ruby[g]{打碎}{うちくだ}けて
\ruby{云}{い}へど
\ruby{婆}{ばゞ}は
\ruby{應}{おう}ぜず、

\原本頁{73-9}%
『
\ruby{歩}{ある}けても
\ruby{歩}{ある}けないでも
\ruby{構}{かま}ひは
\ruby{有}{あ}りましねえ。
%
そんな
\ruby{病}{やまひ}で
\ruby{死}{し}なれた
\ruby{日}{ひ}には、
%
\ruby{彼}{あ}の
\ruby{家}{うち}へ
\ruby{入}{はい}る
\ruby{人}{ひと}は
\ruby{無}{な}くなつて、
%
\ruby{後}{あと}が
\ruby[g]{廃物}{だ め }になつて
\ruby[g]{仕舞}{し ま }ひます。
%
\ruby[g]{早{\換字{速}}}{さつさ }と
\ruby{出}{で}て
\ruby{貰}{もら}つて
\ruby[g]{掃除}{さうぢ }を
\ruby{仕}{し}て、
%
\ruby[g]{行者さ}{ぎやうじや }
んにでも
\ruby{淸}{きよ}めて
\ruby{貰}{もら}ひます。
%
\原本頁{74-1}%
\makeatletter
\@ifundefined{デバッグ@ビルド}{%
  \ruby[g]{行者さ}{ぎやうじや }
  んを
}{%
  \ruby[||j>]{行}{ぎやう}
  \ruby[||j>]{者}{ じや}さんを
}%
\makeatother
% \ruby{行者}{ぎやう|じや}さんを
\ruby{喚}{よ}ぶだけは
\ruby{痛}{いた}みになるが、
%
それだけは
\ruby{時}{とき}の
\ruby[g]{不祥}{ふしやう}と
\ruby[g]{勘辨}{かんべん}するでがあす。% 弁 瓣 辦 辧 (辨) 辩 辯
』

\原本頁{74-3}%
と、
%
\ruby{{\換字{飽}}}{あく}まで
\ruby[g]{我欲}{が よく}の
\ruby{云}{い}ひ
\ruby{草}{ぐさ}なり。

\原本頁{74-4}%
『
だつて
\ruby[||j>]{病}{びやう}
\ruby[||j>]{人}{ にん}が
% \ruby{病人}{びやう|にん}が
\ruby[g]{自{\換字{分}}}{じ ぶん}で
\ruby{出}{で}て
\ruby{行}{ゆ}きやうは
\ruby{無}{な}し、
%
\ruby{{\換字{又}}}{また}
\ruby{五十子}{い|そ|こ}さんの
お
\ruby{母}{つか}さんは、
%
\ruby{汝}{おまへ}の
\ruby{知}{し}つて
\ruby{居}{ゐ}る
\ruby{{\換字{通}}}{とほ}りの
\ruby[g]{自{\換字{分}}}{じ ぶん}
\ruby[g]{{\換字{勝}}手}{かつて }ばかりの
\ruby[g]{繼母}{まゝはゝ}さんで
\改行% 校正作業の簡略化のため
、
%
\原本頁{74-6}\改行%
\ruby[g]{{\換字{平}}常}{ふだん }から
\ruby{五十子}{い|そ|こ}さんには
\ruby[g]{無理}{む り }を
\ruby{云}{い}ふけれど、
%
\ruby{五十子}{い|そ|こ}さんの
\ruby[g]{世話}{せ わ }
\原本頁{74-7}\改行%
は
\ruby[g]{毫末}{ちつと }も
\ruby[g]{仕無}{し な }い、
%
\ruby{酷}{ひど}い〳〵
\ruby[||j>]{人}{にん}
\ruby[||j>]{{\換字{情}}}{じやう}の
% \ruby{人{\換字{情}}}{にん|じやう}の
\ruby{無}{な}い
\ruby{人}{ひと}ぢあ
\ruby{無}{な}いか。
%
\ruby[g]{今度}{こんど }の
\ruby[<j||]{病}{びやう}
\原本頁{74-8}\改行%
\ruby{氣}{き}を
\ruby{知}{し}らせて
\ruby{{\換字{遣}}}{や}つても、
%
\ruby{顏}{かほ}も
\ruby{出}{だ}さ
\ruby{無}{な}けりあ、
%
\ruby[g]{手紙}{て がみ}
\ruby{一}{ひと}つ
\ruby{{\換字{遣}}}{よこ}さない
\原本頁{74-9}\改行%
\ruby{位}{くらゐ}の
\ruby{人}{ひと}だもの、
%
\ruby[||j>]{病}{びやう}
\ruby[||j>]{人}{ にん}を
% \ruby{病人}{びやう|にん}を
\ruby[g]{引取}{ひきと }らうとは
\ruby{云}{い}ふまいぢあ
\ruby{無}{な}いか。
』

\原本頁{74-10}%
『
けれども
\ruby{親}{おや}は
\ruby{親}{おや}でがあす、
%
\ruby{引}{ひ}き
\ruby{取}{と}らないとは
\ruby{云}{い}はせましねえ。
%
\ruby{親}{おや}が
\ruby[g]{引取}{ひきと }らないほどの
\ruby{厄介者}{やく|かい|もの}を、
%
\ruby[g]{他人}{た にん}の
\ruby{婆}{ばゞあ}が
ハア
\ruby{擔}{かつ}がう
\ruby[g]{理由}{わ け }は
\原本頁{75-1}\改行%
\ruby{有}{あ}りましねえ。
%
たつて
\ruby[g]{引取}{ひきと }ら
\ruby{無}{な}けりやあ、
%
ナアニ
\ruby{譯}{わけ}は
\ruby{無}{な}い、
%
\ruby[<j||]{{\換字{巡}}}{じゆん}% 行末行頭の境界付近なので特例処置を施す
\ruby{査}{さ}さん
\ruby{頼}{たの}んで
\ruby[g]{引取}{ひきと }らせるだ。
%
ハア、
%
\ruby[g]{{\換字{道}}理}{す ぢ }の
\ruby{{\換字{違}}}{ちが}つた
\ruby{事}{こと}
\ruby{云}{い}はない
\ruby{婆}{ばゞ}
\原本頁{75-3}\改行%
だよ。
%
\ruby{婆}{ばゞ}は
\ruby[g]{他人}{た にん}だよ、
%
\ruby[g]{身寄}{み より}で
\ruby{無}{な}いだよ、
%
\ruby[g]{錢金}{ぜにかね}づくで
\ruby{彼}{あ}の
\ruby{家}{うち}に
\ruby{置}{お}いたばかりだよ。
%
\ruby[g]{貯金}{たくはへ}も
\ruby{有}{あ}るか
\ruby{無}{な}いか
\ruby{知}{し}れない
\ruby[||j>]{病}{びやう}
\ruby[||j>]{人}{ にん}を
% \ruby{病人}{びやう|にん}を
\ruby{預}{あづ}かる、
%
\原本頁{75-5}\改行%
{---}{---}%
\換字{志}かも
\ruby{傳染病}{うつ|り|やまひ}の
\ruby[||j>]{大}{たい}% ルビ調整(原本通り)
\ruby[||j>]{病}{びやう}
\ruby[||j>]{人}{ にん}
を
\ruby{預}{あづ}かる、
%
\ruby[g]{其樣}{そ ん }な
\ruby{鈍}{どん}くさい
\ruby{事}{こと}
\ruby[g]{出來}{で き }ないだよ。
%
お
\ruby[g]{{\換字{前}}樣}{めへさま}も
\ruby[||j>]{病}{びやう}
\ruby[||j>]{人}{ にん}には
% \ruby{病人}{びやう|にん}には
\ruby[g]{他人}{た にん}で
\ruby{無}{な}いか、
%
\ruby{恨}{うら}みつぽい
\ruby[g]{其樣}{そ ん }な
\ruby[<j||]{眼}{まなこ}
\原本頁{75-7}\改行%
つきをして
\ruby{何}{なに}も
\ruby{此}{この}
\ruby{婆}{ばゞあ}を
\ruby{視}{み}さつしやることは
\ruby{無}{な}い。
』

\原本頁{75-8}%
『
なるほど
\ruby{其}{それ}は
\ruby[g]{左樣}{さ う }でもあらうが、
%
いくら
\ruby[g]{他人}{た にん}でも
\ruby[||j>]{病}{びやう}
\ruby[||j>]{人}{ にん}を
% \ruby{病人}{びやう|にん}を
\ruby[g]{突出}{つきだ }さうといふのは、
%
それは
\ruby{餘}{あま}り
\ruby{酷}{むご}いぢやあ
\ruby{無}{な}いか。
』

\原本頁{75-10}%
『
\ruby[||j>]{病}{びやう}
\ruby[||j>]{人}{ にん}だから
% \ruby{病人}{びやう|にん}だから
\ruby{{\換字{逐}}}{お}ひ
\ruby{出}{だ}さうといふので、
%
\ruby{酷}{むご}かあ
\ruby{酷}{むご}いに
\ruby{仕}{し}て
\ruby{置}{お}かつしやい。
』

\原本頁{76-1}%
『
お
\ruby{婆}{ばあ}さん、
%
お
\ruby{{\換字{前}}}{まへ}、
%
そんな
\ruby{事}{こと}を
\ruby{云}{い}つたつて、
%
\ruby[g]{人間}{ひ と }には
\ruby[g]{人{\換字{道}}}{み ち }といふものがある。
%
\ruby{動}{うご}かしてさへ
\ruby{惡}{わる}いと
\ruby[g]{醫者}{い しや}の
\ruby{云}{い}つた
\ruby[||j>]{病}{びやう}
\ruby[||j>]{人}{ にん}を
% \ruby{病人}{びやう|にん}を
\ruby{{\換字{逐}}}{お}ひ
\ruby{出}{だ}さうとは
\ruby[g]{非{\換字{道}}}{ひ だう}では
\ruby{無}{な}いか。
』

\原本頁{76-4}%
『
\ruby[g]{非{\換字{道}}}{ひ だう}なら
\ruby[g]{非{\換字{道}}}{ひ だう}に
\ruby{仕}{し}て
\ruby{置}{お}かつしやい。
%
\ruby{金}{かね}の
\ruby[g]{出處}{で どこ}の
\ruby{覺束無}{おぼ|つか|な}い
\ruby[g]{其樣}{そ ん }な
\ruby{大病人}{たい|びやう|にん}を、
%
\ruby[g]{世話}{せ わ }を
して
\ruby{損}{そん}を
するのは
\ruby{婆}{ばゞあ}は
\ruby{{\換字{嫌}}}{きら}ひだ。
』

\原本頁{76-6}%
『
でも
\ruby{有}{あ}らうが
\ruby{汝}{おまへ}が
\ruby{今}{いま}
\ruby{{\換字{逐}}}{お}ひ
\ruby{出}{だ}して
\ruby[g]{仕舞}{し ま }へば、
%
よしんば
\ruby[g]{繼母}{おつか }さんが
\ruby{引}{ひ}き
\ruby{取}{と}るにしても、
%
あちこち
\ruby{持}{も}ち
\ruby{{\換字{廻}}}{まは}られるのは
\ruby[||j>]{病}{びやう}
\ruby[||j>]{人}{ にん}の
% \ruby{病人}{びやう|にん}の
\ruby{不利益}{ふ|た|め}
\改行% 校正作業の簡略化のため
、
%
\原本頁{76-8}\改行%
\換字{志}かも
\ruby[g]{何樣}{ど う }して
\ruby{彼}{あ}の
\ruby[g]{繼母}{おつか }さんが、
%
\ruby{碌}{ろく}な
\ruby[g]{世話}{せ わ }をする
\ruby{事}{こと}では
\ruby{無}{な}い。
%
\原本頁{76-9}\改行%
\ruby{仕}{し}て
\ruby{見}{み}れば
\ruby[g]{看護}{かんご }が
\ruby{惡}{わる}けりやあ
\ruby{危}{あぶな}いといふ
\ruby[g]{病氣}{びやうき}だもの、
%
\ruby{十}{とほ}に
\ruby{一}{ひと}つも
\ruby{助}{たすか}る
\ruby{瀬}{せ}は
\ruby{無}{な}い、
%
\ruby{見}{み}す〳〵
\ruby[||j>]{病}{びやう}
\ruby[||j>]{人}{ にん}は
% \ruby{病人}{びやう|にん}は
\ruby{殺}{ころ}されるやうなもの!。
%
\ruby{譯}{わけ}の
\原本頁{76-11}\改行%
\ruby{{\換字{分}}}{わか}らない
\ruby{汝}{おまへ}でも
\ruby{無}{な}し、
%
こゝのところを
\ruby{考}{かんが}へて、
%
\ruby{私}{わたし}が
\ruby{此}{こ}の
\ruby{{\換字{通}}}{とほ}り
\ruby{手}{て}を
ついて
\ruby{頼}{たの}むから、
%
\原本頁{77-1}%
どうか
\ruby[g]{左樣}{さ う }
あこぎな
\ruby{事}{こと}を
\ruby{云}{い}はないで、
%
\ruby[g]{當{\換字{分}}}{たうぶん}
\改行% 校正作業の簡略化のため
‥‥。
』

\原本頁{77-3}%
『
イヽエ、
%
あこぎな
\ruby{事}{こと}を
\ruby{云}{い}ふでがあすよ。
%
\ruby{手}{て}をついて
\ruby{頼}{たの}んだつて、
%
\ruby[g]{芋塊}{い も }が
\ruby{一}{ひと}つ
\ruby[g]{自然}{ひとりで}に
\ruby[g]{出來}{で き }て
\ruby{來}{く}るものぢやあ
ござら
\ruby{無}{な}い。
%
\ruby{頼}{たの}むなら
\ruby{頼}{たの}むやうにして
\ruby{頼}{たの}まつしやい。
』

\原本頁{77-6}%
『
\ruby{頼}{たの}むやうに
\ruby{仕}{し}ろつて、
%
\ruby[g]{何樣}{ど う }すれば
\ruby{好}{い}いと
\ruby{云}{い}ふのかえ。
』

\原本頁{77-7}%
『
\ruby{婆}{ばゞあ}は
\ruby{年}{とし}を
とつて
\ruby{氣}{き}が
\ruby{短}{みじか}い、
%
\ruby[g]{打撒}{ぶちま }けて
\ruby[||j>]{汝}{おまへ}
\ruby[||j>]{樣}{ さま}に
% \ruby{汝樣}{おまへ|さま}に
\ruby{云}{い}つて
\ruby{上}{あ}げやう。
%
\ruby[||j>]{病}{びやう}
\ruby[||j>]{人}{ にん}の
% \ruby{病人}{びやう|にん}の
\ruby[g]{月々}{つき〴〵}のものは
\ruby{今}{いま}まで
\ruby{{\換字{通}}}{どほ}りに
\ruby{屹}{きつ}と
お
\ruby[g]{{\換字{前}}樣}{めへさま}が
\ruby[g]{受合}{うけあ }つて、
%
それから
\ruby[||j>]{病}{びやう}
\ruby[||j>]{人}{ にん}が
% \ruby{病人}{びやう|にん}が
いけなかつたら、
%
\ruby{後}{あと}の
\ruby[g]{始末}{し まつ}は
\ruby{皆}{みな}
\ruby{此}{この}
\ruby{婆}{ばゞあ}に
\ruby[g]{{\換字{迷}}惑}{めいわく}を
\ruby{掛}{か}けないで、
%
そして
\ruby[g]{座敷}{ざ しき}に
\ruby[g]{死穢}{け がれ}を
\ruby{付}{つ}けた
\ruby[g]{謝罪}{あやまり}に
\ruby{二十兩}{に|じふ|りやう}、
%
\ruby{癒}{なほ}つたら
\ruby{祝}{いはひ}に
\ruby[||j>]{十}{じふ}
\ruby[||j>]{兩}{りやう}
% \ruby{十兩}{じふ|りやう}
\ruby[||j>]{{\換字{遣}}}{ よこ}すと、
%
\ruby[g]{確乎}{しつかり}
\ruby{御{\換字{前}}樣}{お|めへ|さま}が
\ruby[g]{呑{\換字{込}}}{のみこ }んで、
%
\ruby{先}{ま}づ
\ruby[||j>]{十}{じふ}
\ruby[||j>]{兩}{りやう}だけ
% \ruby{十兩}{じふ|りやう}だけ
\ruby{渡}{わた}して
\ruby{置}{お}かつしやい。
%
\原本頁{78-1}%
その
\ruby{代}{かは}り
\ruby[||j>]{病}{びやう}
\ruby[||j>]{人}{ にん}には
% \ruby{病人}{びやう|にん}には
\ruby{構}{かま}はないから、
%
どうなりと
\ruby[g]{{\換字{勝}}手}{かつて }に
\ruby[g]{介抱}{かいはう}さつしやい。
%
さ、
%
お
\ruby[g]{{\換字{前}}樣}{めへさま}も
あかの
\ruby[g]{他人}{た にん}、
%
これだけ
\ruby[g]{踏{\換字{込}}}{ふみこ }んで
\ruby[g]{世話}{せ わ }も
なるまい。
%
それとも
\ruby[||j>]{病}{びやう}
\ruby[||j>]{人}{ にん}が
% \ruby{病人}{びやう|にん}が
\ruby[||j>]{愍}{かは}
\ruby[||j>]{然}{いさう}で、% 「愍然 か(は)いさう」
% \ruby{愍然}{かは|いさう}で、% 「愍然 か(は)いさう」
%
\ruby{金}{かね}を
\ruby{出}{だ}してもと
\ruby{云}{い}はつしやるか、
%
どつちでも
お
\ruby[g]{{\換字{前}}樣}{めへさま}の
\ruby{好}{すき}に
さつしやい。
』

\原本頁{78-5}%
『
ムヽ
』

\原本頁{78-6}%
\ruby{手}{て}に
\ruby{在}{あ}らば
\ruby{千金萬金}{せん|きん|ばん|きん}も
\ruby{何}{なに}
\ruby{惜}{をし}かるべきを、
%
\ruby{及}{およ}ぶことの
\ruby{及}{およ}ばぬに
\ruby{口}{くち}
\ruby{惜}{をし}きは
\ruby{金沙汰}{かね|さ|た}なり。
%
\ruby[g]{水野}{みづの }は
\ruby{生}{うま}れて
はじめて
\ruby[g]{日頃}{ひ ごろ}
\ruby{此}{この}
\ruby[|g|]{阿堵物}{もの}を% 「阿堵物(あとぶつ)」お金のこと
\ruby{卑}{いやし}みしを
\ruby{悔}{く}いぬ。

\Entry{其十三}

% メモ 校正終了 2024-04-04 2024-05-24 2024-06-17
\原本頁{78-10}%
おのづから
\ruby{横}{よこ}さまに
\ruby{降}{ふ}る
\ruby{雨}{あめ}はあらじ、
%
\ruby{風}{かぜ}の
\ruby{添}{そ}はるにこそ、
%
\ruby{音}{おと}
あらけなく
\ruby{夜}{よる}の
\ruby{窓}{まど}をも
\ruby{打}{う}つなれ、
%
と
\ruby{胸}{むね}
ゆたかなる
\ruby{{\換字{古}}}{むかし}の
\ruby{人}{ひと}の
\ruby{云}{い}ひける。
%
かゝる
\ruby{鬼}{おに}くさき
\ruby{婆}{ばゞ}も、
%
\ruby{齡}{とし}の
\ruby{十七八}{じふ|しち|はち}には、% 原本には漢数字「七」のルビ無し
%
\ruby{女}{をんな}の
\ruby[||j>]{本}{うまれ}
\ruby[||j>]{性}{ つき}とて、
% \ruby{本性}{うまれ|つき}とて、
%
\ruby[g]{臙脂}{べ に }
\ruby[g]{白{\換字{粉}}}{おしろい}に
\ruby{色}{いろ}つくりて、
%
\ruby{人}{ひと}に
\ruby{悅}{よろこ}ばれんと
\ruby{願}{ねが}ひたる
\ruby{日}{ひ}もあるべきに
\改行% 校正作業の簡略化のため
、
%
\原本頁{7904}\改行%
\ruby{其}{そ}の
\ruby{後}{のち}
\ruby[g]{如何}{い か }なる
\ruby{世}{よ}の
\ruby{風}{かぜ}に
\ruby{吹}{ふ}き
\ruby{曲}{ゆが}められてか、
%
\ruby{今}{いま}は
\ruby[g]{如是}{か く }
\ruby{直}{すぐ}ならず
\ruby{人}{ひと}には
\ruby{當}{あた}るならん。
%
\ruby[g]{水野}{みづの }は
\ruby[g]{一度}{ひとたび}は
\ruby{此}{こ}の
\ruby{婆}{ばゞ}を
\ruby{憎}{にく}しと
\ruby{見}{み}しかど、
%
\ruby{憎}{にく}む
\ruby{心}{こゝろ}は
\ruby{忽}{たちま}ちに
\ruby{失}{う}せて、
%
\ruby{且}{かつ}は
\ruby{其}{そ}の
\ruby[g]{欲深}{よくふか}きに
\ruby{呆}{あき}れ、
%
\ruby{且}{かつ}は
\ruby{其}{そ}の
\ruby[||j>]{意}{こゝろ}
\ruby{剛}{ たけ}きを% 全角空白は「意(こゝろ)」の突出対策
\ruby{怪}{あやし}み、
%
\ruby{且}{かつ}は
\ruby{其}{そ}の
\ruby{人}{ひと}
らしからぬまでに
\ruby{{\換字{尊}}}{たつと}き
\ruby[g]{愛{\換字{情}}}{こゝろ }の
\ruby{既}{すで}に
\ruby{壞}{やぶ}れ
\ruby{盡}{つく}して、
%
\ruby{卑}{いや}しき
\ruby{我}{が}のみの
\ruby{殘}{のこ}りて
\ruby{高}{たか}ぶれるを、
%
\ruby{哀}{かなし}み
\ruby{愍}{あはれ}みて
\ruby[g]{打見}{うちみ }やりた
\改行% 校正作業の簡略化のため
り。

\原本頁{79-10}%
されど
\ruby{今}{いま}は
\ruby[g]{他人}{ひ と }を
\ruby{愍}{あはれ}みて
あるべき
\ruby{時}{とき}ならねば、
%
\ruby[g]{水野}{みづの }は
\ruby{直}{たゞち}に
\ruby{差}{さ}し
\ruby{當}{あた}つての
\ruby{我}{わ}が
\ruby{上}{うへ}に
\ruby{掛}{かゝ}れる
\ruby{事}{こと}に
\ruby{心}{こゝろ}を
\ruby{惱}{なや}ましめぬ。

\原本頁{80-1}%
\ruby{五十子}{い|そ|こ}を
\ruby[g]{如是}{か く }
\ruby{忌}{いま}はしく
\ruby[g]{親切}{しんせつ}
\ruby{無}{な}き
\ruby{婆}{ばゞ}の
\ruby{家}{いへ}に
\ruby{在}{あ}らせんよりは、
%
\ruby{良}{よ}き
\ruby[g]{病院に}{びやうゐん }
% \ruby{病院}{びやう|ゐん}に
\ruby{移}{うつ}さんかた
\ruby[g]{萬般}{すべて }に
\ruby{就}{つ}けて
\ruby[g]{心地}{こゝち }よし
とは
\ruby{思}{おも}ひながらも、
%
\ruby[g]{今{\換字{宵}}}{こ よひ}の
\ruby{如}{ごと}く
\ruby{穩}{おだ}やかに
\ruby{晴}{は}れてのみ
あるべくは
あらぬ
\ruby{秋}{あき}の
\ruby[g]{天候}{そ ら }の
\ruby[<j||]{{\換字{習}}}{ならひ}なれば、% 行末行頭の境界付近なので特例処置を施す
%
\ruby{時}{とき}に
\ruby{臨}{のぞ}みて
\ruby[g]{如何}{い か }なる
\ruby{雨}{あめ}
\ruby{風}{かぜ}の
\ruby[g]{妨{\換字{害}}}{さまたげ}に
\ruby{{\換字{遇}}}{あ}はんも
\ruby{知}{し}るべからず
\改行% 校正作業の簡略化のため
、
%
\原本頁{80-5}\改行%
\ruby{{\換字{又}}}{また}
\ruby[g]{然無}{さ な }きだに
\ruby[g]{{\換字{遠}}路}{ゑんろ }を
\ruby[g]{{\換字{伴}}ひ}{ともな }
\ruby{行}{ゆ}く
\ruby[g]{{\換字{途}}上}{とじやう}は
\ruby[||j>]{病}{びやう}
\ruby[||j>]{人}{ にん}も
% \ruby{病人}{びやう|にん}も
\ruby{特}{こと}に
\ruby[||j>]{心}{こゝろ}
\ruby[||j>]{惱}{ なや}ましかるべく、
% \ruby{心惱}{こゝろ|なや}ましかるべく、
%
それがために
\ruby[g]{萬一}{まんいち}
\ruby{惡}{あし}き
\ruby{事}{こと}もやとの
\ruby[g]{懸念}{け ねん}も
\ruby{少}{すくな}からぬに、
%
\ruby[g]{由無}{よしな }き
\ruby[g]{金錢}{きんせん}を
\ruby{婆}{ばゞ}に
\ruby{貪}{むさぼ}らるゝは
\ruby{愚}{おろか}なるに
\ruby{似}{に}たれど、
%
これも
\ruby{病}{や}める
\ruby{人}{ひと}のためと
\ruby{{\換字{忍}}}{しの}ばんには
\ruby{露}{つゆ}
\ruby{厭}{いと}はしからずと、
%
\ruby[g]{水野}{みづの }は
\ruby{{\換字{終}}}{つひ}に
\ruby{意}{こゝろ}を
\ruby{決}{けつ}して、
%
\原本頁{80-9}\改行%
\ruby{彼}{か}の
\ruby{離}{はな}れ
\ruby{室}{や}に
\ruby{置}{お}きたるまゝ
\ruby[g]{介抱}{かいはう}する
\ruby{事}{こと}と
\ruby{定}{さだ}めたり。
%
もとより
\ruby{一}{ひと}つには
\ruby{其}{そ}の
\ruby[g]{奧深}{おくふか}き
\ruby{底}{そこ}の
\ruby{底}{そこ}の
\ruby{心}{こゝろ}に、
%
\ruby{五十子}{い|そ|こ}と
\ruby{我}{われ}との
\ruby[g]{相距}{あひさ }らざらんを
\ruby{望}{のぞ}む
\ruby{思}{おもひ}の
\ruby{潜}{ひそ}めばなるべし。% 【潛 u6f5b 「先先」】【潜 u6f5c 「夫夫」】併用されている
%
たとひ
\ruby[g]{自己}{お の }が
\ruby{身}{み}は
\ruby[g]{如何}{い か }なる
\ruby{故}{ゆゑ}にか
\原本頁{81-1}\改行%
\ruby{五十子}{い|そ|こ}に
\ruby{{\換字{嫌}}}{きら}はれて、
%
\ruby{特}{こと}に
\ruby{病}{やまひ}のため
\ruby{癇}{かん}の
\ruby{高}{たか}ぶりて
\ruby{我}{が}の
\ruby{{\換字{強}}}{つよ}くなれる
\ruby{此}{こ}の
\ruby{頃}{ごろ}の
\ruby[g]{彼女}{か れ }には、
%
\ruby{面}{おもて}を
\ruby{會}{あ}はすを
さへ
\ruby{厭}{いと}はるゝより、
%
\makeatletter
\@ifundefined{デバッグ@ビルド}{%
  \ruby[||j>]{自}{みづか}ら% ルビ調整(特殊処理)(ルビ3文字の親文字が3つ)
  \ruby[g]{病床}{びやうしやう}に
}{%
  \ruby[<j||]{自}{みづか}ら% ルビ調整(特殊処理)(ルビ3文字の親文字が3つ)
  \ruby[<j||]{病}{びやう}
  \ruby[||j>]{床}{しやう}
  \原本頁{81-3}\改行%
  に
}%
\makeatother
\ruby{{\換字{近}}}{ちか}づきて
\ruby{問}{と}ひ
\ruby{慰}{なぐさ}めも
\ruby[g]{仕度}{し た }く、
%
\ruby[g]{看護}{せ わ }も
\ruby{仕}{し}て
\ruby{{\換字{遣}}}{や}りたき
\ruby{心}{こゝろ}の、
%
\ruby{{\換字{遣}}}{や}る
\原本頁{81-4}\改行%
\ruby{方}{かた}も
\ruby{無}{な}く
\ruby{逸}{はや}るを
\ruby{抑}{おさ}へに
\ruby{抑}{おさ}へて、
%
\ruby[g]{裏面}{う ら }にて
こそ
\ruby{力}{ちから}の
\ruby{及}{およ}ぶ
\ruby{限}{かぎ}りを
\ruby{盡}{つく}して
\ruby{駈}{か}けも
\ruby{走}{はし}りもすれ、
%
\ruby[g]{病人の}{びやうにん }
% \ruby{病人}{びやう|にん}の
\ruby{氣}{き}に
\ruby{{\換字{逆}}}{さから}はじと
\ruby{其}{そ}の
\ruby{{\換字{前}}}{まへ}には
\ruby[g]{身影}{か げ }をさへ
\ruby{見}{み}することも
\ruby{無}{な}くて、
%
たゞ
\ruby[g]{竊に}{ひそか }
\ruby{外}{そと}に
\ruby{立}{た}つて、
%
\ruby{細}{ほそ}りたる
\ruby{聲}{こゑ}の
\原本頁{81-7}\改行%
\ruby[g]{孱{\換字{弱}}}{かよわ }きを% ルビ調整(原本通り)
\ruby{聞}{き}き、
%
\ruby{或}{あるひ}は
\ruby{物}{もの}の
\ruby[g]{罅隙}{す き }より
\ruby{窶}{やつ}れたる
\ruby{其}{そ}の
\ruby[g]{面貌}{おもかげ}の
\ruby{悲}{かな}しきを
\ruby{見}{み}ては、
%
\ruby[g]{男兒}{をとこ }たる
\ruby{身}{み}の
\ruby[g]{人目}{ひとめ }はづかしくも、
%
にじみ
\ruby{來}{く}る
\ruby{涙}{なみだ}を
\ruby{止}{とゞ}め
かねて、
%
\ruby{神}{かみ}も
\ruby{我}{わ}が
\ruby[g]{誠心}{まこと }を
\ruby{憐}{あは}れませ
たまひて、
%
\ruby{此}{こ}の
\ruby{人}{ひと}の
\ruby[g]{病苦}{びやうく}を
\ruby{救}{すく}はせ
たまへ、
%
と
\ruby{何}{なん}の
\ruby{神}{かみ}に
\ruby{祈}{いの}るとも
\ruby{無}{な}く、
%
\ruby[g]{何時}{い つ }か
\ruby{我}{われ}
\ruby{知}{し}らず
\ruby{祈}{いの}り
\ruby{居}{ゐ}る、
%
\ruby{思}{おも}へば
\ruby{愚}{おろか}しき
\ruby[g]{{\換字{朝}}夕}{あさゆふ}に
\ruby{甘}{あま}んじて、
%
\ruby{{\換字{猶}}}{なほ}これより
\ruby[g]{幾日}{いくか }と
\ruby{定}{さだ}まらぬ
\原本頁{82-1}\改行%
\ruby{其}{そ}の
\ruby{間}{あひだ}を、
%
せめてもの
\ruby[g]{果敢}{は か }なき
\ruby[g]{心{\換字{遣}}}{こゝろや}りに、
%
\ruby{{\換字{猶}}}{なほ}
\ruby{其}{そ}の
おろかしき
\ruby[g]{振舞}{ふるまひ}を
\ruby{續}{つゞ}けんとは
するなり。

\原本頁{82-3}%
『
では
\ruby{汝}{おまへ}の
\ruby{云}{い}ふ
\ruby{{\換字{通}}}{とほ}りに
\ruby{仕}{し}やう。
%
\ruby[g]{一切}{いつさい}
\ruby{私}{わたし}が
\ruby[g]{受合}{うけあ }つて
\ruby{置}{お}く。
』

\原本頁{82-4}%
と、
%
\ruby[g]{決然}{けつぜん}として
\ruby[g]{水野}{みづの }は
\ruby{云}{い}へど、

\原本頁{82-5}%
『
たゞ
\ruby[g]{受合}{うけあ }つても
いけましねえ、
%
\ruby[g]{何時}{い つ }
その
\ruby[||j>]{十}{じふ}
\ruby[||j>]{兩}{りやう}は
% \ruby{十兩}{じふ|りやう}は
\ruby{渡}{わた}して
\ruby{吳}{く}れ
さつしやる。
』

\原本頁{82-7}%
と、
%
\ruby{婆}{ばゞ}は
\ruby{手}{て}に
\ruby{握}{にぎ}らぬことには
\ruby{人}{ひと}を
\ruby{信}{しん}ぜず。

\原本頁{82-8}%
『
\ruby[g]{明日}{あ す }の
\ruby{{\換字{朝}}}{あさ}
\ruby{渡}{わた}す。
』

\原本頁{82-9}%
『
\ruby{大{\換字{丈}}夫}{だい|ぢやう|ぶ}かね。
』

\原本頁{82-10}%
『
\ruby{大{\換字{丈}}夫}{だい|ぢやう|ぶ}だ。
』

\原本頁{82-11}%
『
\ruby{看病人}{かん|びやう|にん}
はエ。
』

\原本頁{83-1}%
『
\ruby[g]{矢張}{やつぱり}
\ruby{私}{わたし}が
\ruby{雇}{やと}つて
\ruby{付}{つ}ける。
%
\ruby[g]{相良}{さがら }さんに
\ruby{良}{い}いのを
\ruby[g]{世話}{せ わ }をして
\ruby{貰}{もら}ふ。
%
\換字{志}かし
\ruby[g]{一切}{いつさい}
かういふ
\ruby{事}{こと}を、
%
\ruby{私}{わたし}が
\ruby{爲}{し}たのだと
\ruby[||j>]{病}{びやう}
\ruby[||j>]{人}{ にん}に
% \ruby{病人}{びやう|にん}に
\ruby{云}{い}つては
ならぬ。
%
\ruby[||j>]{病}{びやう}
\ruby[||j>]{人}{ にん}が
% \ruby{病人}{びやう|にん}が
\ruby[g]{私の}{わたし }
\ruby[g]{世話}{せ わ }になるのを
\ruby{厭}{いや}がつて
\ruby{居}{ゐ}るから、
%
たゞ
\ruby[g]{學校}{がくかう}の
\ruby[g]{人{\換字{達}}}{ひとたち}が
\ruby{爲}{す}るのだと
\ruby{云}{い}つて
\ruby{置}{おい}てくれ。
』

\原本頁{83-5}%
『
はア、ようがす、
%
それは
\ruby[g]{無益}{む だ }な
\ruby{口}{くち}きく
\ruby{婆}{ばゞあ}でない
でがあす。
%
\換字{志}かし
\ruby{甚}{えら}い
\ruby{金}{かね}が
かゝりませうに、
%
\ruby[g]{親切}{しんせつ}な
\ruby{事}{こと}だネ。
』

\原本頁{83-7}%
と、
%
\ruby{冷}{ひや}やかに
\ruby{笑}{わら}ふ
\ruby{口}{くち}の
\ruby[g]{左右}{さ いう}に、
%
\ruby{深}{ふか}き
\ruby{皺}{しわ}
あらはれて
\ruby[||j>]{物}{もの}
\ruby[||j>]{凄}{すさま}じく、
% \ruby{物凄}{もの|すさま}じく、
%
さも〳〵
\ruby[g]{水野}{みづの }が
\ruby{爲}{な}す
\ruby[g]{一切}{いつさい}の
\ruby{事}{こと}の、
%
やがては
\ruby{{\換字{朝}}}{あした}の
\ruby{霜}{しも}の
\ruby{柱}{はしら}を
\ruby[g]{{\換字{彩}}色}{いろど }り
\ruby[<j||]{夕}{ゆふべ}% 行末行頭の境界付近なので特例処置を施す
\原本頁{83-9}\改行%
の
\ruby{露}{つゆ}の
\ruby{珠}{たま}を
\ruby{綴}{つゞ}らんとする
\ruby{痴}{おろか}なる
\ruby[g]{企畫}{くはだて}の
\ruby{如}{ごと}く
\ruby[g]{甲{\換字{斐}}}{か ひ }
\ruby{無}{な}く
\ruby{{\換字{終}}}{おは}らんを
\ruby{見}{み}
\原本頁{83-10}\改行%
\ruby{徹}{ぬ}きて
\ruby{知}{し}りたりと
\ruby{云}{い}はぬばかりの
\ruby[g]{面色}{かほつき}したり。

\原本頁{83-11}%
『
\ruby{快}{よ}くなるまで
みんな
\ruby{御{\換字{前}}樣}{お|まへ|さま}が
\ruby[g]{一人}{ひとり }で
\ruby{爲}{さ}つしやるかネ。
』

\原本頁{84-1}%
『
ム。
』

\原本頁{84-2}%
『
\ruby[<j||]{百}{ひやく}% ルビ調整(特殊処理)親字毎にルビ3文字
\ruby[||j>]{兩}{りやう}
\ruby[||g>]{位で}{ ぐらゐ}は
\ruby[g]{{\換字{追}}付}{おつつ }きましねえかも
\ruby{知}{し}れましねえヨ。
』

\原本頁{84-3}%
『
ム。
』

\原本頁{84-4}%
『
ホー
\ruby{御{\換字{前}}樣}{お|まへ|さま}は
\ruby[g]{學校}{がくかう}の
\ruby[g]{敎員}{けうゐん}でもつて、
%
\ruby[g]{其樣}{そんな }に
\ruby[g]{御金}{お かね}
\ruby{有}{も}つてるだか
\原本頁{84-5}\改行%
ネ。
』

\原本頁{84-6}%
\ruby[g]{水野}{みづの }は
\ruby{苦}{にが}りきつて
\ruby{答}{こたへ}をもせず、

\原本頁{84-7}%
『
\ruby{何}{なん}でも
\ruby{可}{よ}い、
%
\ruby[g]{其樣}{そ ん }なことを
\ruby{云}{い}つて
\ruby{居}{ゐ}る
\ruby{暇}{ひま}は
\ruby{無}{な}い。
%
わたしは
これから
\ruby[g]{尾竹}{を だけ}の
ところへ
\ruby{行}{ゆ}く。
』

\原本頁{84-9}%
\ruby{突}{つ}と
\ruby[g]{立上}{たちあが}つたる
\ruby[g]{水野}{みづの }は
\ruby[g]{此處}{こ ゝ }を
\ruby{出}{い}でゝ、
%
\ruby{村}{むら}の
\ruby{醫}{い}を
\ruby{問}{と}ひて
\ruby[g]{相良}{さがら }の
\ruby[<j||]{言}{ことば}を% 行末行頭の境界付近なので特例処置を施す
\ruby{傳}{つた}へ、
%
\ruby{手}{て}ぬかり
\ruby{無}{な}きやう
\ruby[g]{十{\換字{分}}}{じふぶん}に
\ruby{其}{そ}の
\ruby[g]{職{\換字{分}}}{つとめ }を
\ruby{盡}{つく}さんことを
\ruby{乞}{こ}ひ
\原本頁{84-11}\改行%
\ruby{求}{もと}め、
%
これより
\ruby{直}{すぐ}にも
\ruby[g]{見舞}{み ま }はん
といふ
\ruby[g]{親切}{しんせつ}
\ruby{籠}{こも}れる
\ruby{答}{こたへ}を
\ruby{聞}{き}きて、
%
はじめて
\ruby{我}{わ}が
\ruby{宿}{やど}とせる
\ruby[g]{山路}{やまぢ }が
\ruby{方}{かた}に
\ruby{歸}{かへ}りぬ。

\原本頁{85-2}%
\ruby{物}{もの}の
\ruby{味}{あぢ}さへ
\ruby{知}{し}るや
\ruby{知}{し}らずや、
%
\ruby[g]{湯漬}{ゆ づ }け
\ruby{飯}{めし}
\ruby{忙}{せは}しく
\ruby[g]{夜食}{やしよく}を
\ruby{濟}{す}ませて、
%
\ruby{長}{なが}き
\ruby{夜}{よ}も
\ruby{既}{はや}
\ruby{{\換字{更}}}{ふ}けて
\ruby[g]{何時}{なんじ }かを
\ruby{打}{う}つ
\ruby[g]{時計}{と けい}の
\ruby{音}{おと}の
\ruby{折}{をり}から
\ruby{聞}{きこ}ゆるを
\ruby{數}{かぞ}へも
\ruby{敢}{あ}へず、
%
\ruby{急}{いそ}ぎ
\ruby[g]{周章}{あ は }てゝ
\ruby{{\換字{又}}}{また}
\ruby[g]{{\換字{戸}}外}{そ と }へ
\ruby{出}{い}でんと
すれば、

\原本頁{85-5}%
『
\ruby[g]{水野}{みづの }さん、
%
\ruby[g]{何處}{ど こ }へ
\ruby{今}{いま}から
\ruby[g]{御出}{お いで}に
なります?。
』

\原本頁{85-6}%
と、
%
\ruby{低}{ひく}く
\ruby{沈}{しづ}める
\ruby[g]{聲音}{こわね }の
\ruby{呼}{よ}び
\ruby{止}{と}めたり。

\Entry{其十四}

% メモ 校正終了 2024-04-05 2024-05-24 2024-06-17
\原本頁{85-8}%
\ruby{云}{い}はゞ
\ruby{我}{わ}が
\ruby{假}{かり}の
\ruby{宿{\換字{所}}}{や|ど}の
\ruby{主人}{ある|じ}なりと
\ruby{云}{い}ふまで
なれど、
%
\ruby{東京}{とう|けい}あたりに% 原本「東京」を「とうけい」としている
\ruby{黒塗}{くろ|ぬり}の
\ruby{小札}{こ|ふだ}
\ruby{懸}{か}け
ならべたる
\ruby[||j>]{商}{しやう}
\ruby[||j>]{賣}{ ばい}づくの
% \ruby{商賣}{しやう|ばい}づくの
\ruby{下宿屋}{げ|しゆく|や}と
いふにはあらで、
%
\ruby{我}{わ}が
\ruby[||j>]{校}{かう}
\ruby[||j>]{長}{ちやう}の
% \ruby{校長}{かう|ちやう}の
\ruby{高田}{たか|だ}と
\ruby{懇意}{こん|い}なる
\ruby[||j>]{間}{あひだ}
\ruby[||j>]{柄}{ がら}なるより、
% \ruby{間柄}{あひだ|がら}なるより、
%
\ruby{其}{そ}の
\ruby{云}{い}ひ
\ruby{入}{いれ}に
よりて、
%
\原本頁{86-1}%
\ruby{唯}{たゞ}
\ruby{我}{われ}
\ruby{一人}{ひと|り}を
\ruby{賓客}{きや|く}
\ruby{同樣}{どう|やう}に、
%
\ruby{萬般}{よろ|づ}
\ruby{親切}{しん|せつ}に
\ruby{世話}{せ|わ}し
\ruby{吳}{く}るゝ
\ruby{此家}{こ|ゝ}の
\原本頁{86-2}\改行%
\ruby{老夫}{おや|ぢ}の
\ruby{吉右衛門}{きち||ゑ|もん}に
\ruby{呼}{よ}び
\ruby{{\換字{留}}}{と}められては、
%
\ruby{心}{こゝろ}の
\ruby{急}{せ}いたる
\ruby{折}{をり}からとて
\改行% 校正作業の簡略化のため
、
%
\原本頁{86-3}\改行%
あらずもがなには
\ruby{思}{おも}ひながら、
%
\ruby{後}{あと}
\ruby{振}{ふ}り
\ruby{反}{かへ}りて
\ruby{立停}{たち|とゞ}まり、

\原本頁{86-4}%
『
ア、
%
\ruby{一寸}{ちよ|いと}
\ruby[||j>]{濱}{はま}
\ruby[||j>]{町}{ちやう}まで
% \ruby{濱町}{はま|ちやう}まで
\ruby{行}{い}つて
\ruby{來}{き}ます。
%
\ruby{何程}{いく|ら}
\ruby{急}{いそ}いでも
\ruby{遲}{おそ}くは
ならうが、
%
\ruby{歸}{かへ}ることは
\ruby{屹度}{きつ|と}
\ruby{歸}{かへ}ります。
%
\ruby{濟}{す}まんけれど
\ruby{敲}{たゝ}きますから、
%
\ruby{關}{かま}はず
\ruby{{\換字{戸}}締}{し|ま}りを
\ruby{仕}{し}て
\ruby{仕舞}{し|ま}つて
\ruby{寢}{やす}んで
\ruby{下}{くだ}さい。
』

\原本頁{86-7}%
と
\ruby{云}{い}ひつゝ
\ruby{燈火}{あか|り}さす
\ruby{茶}{ちや}の
\ruby{室}{ま}を
\ruby{覗}{うかゞ}へば、
%
\ruby{讀}{よ}みさしたる
\ruby{新聞}{しん|ぶん}を
\ruby{傍}{かたへ}に
\ruby{置}{お}きて、
%
\ruby{兀}{は}げたる
\ruby{頭}{かしら}の
\ruby[g]{澤々}{つや〳〵}と
\ruby{光}{ひか}れる
\ruby{吉右衛門}{きち||ゑ|もん}は、
%
\ruby[||j>]{眞}{しん}
\ruby[||j>]{鍮}{ちゆう}
\ruby[||j>]{緣}{ ぶち}の
\ruby{鏡玉}{た|ま}
\原本頁{86-9}\改行%
\ruby{圓}{まろ}き
\ruby{{\換字{古}}風}{むか|し}
\ruby{眼鏡}{め|がね}を
\ruby{掛}{か}けたる、
%
\ruby{淸}{きよ}らなる
\ruby{赤}{あか}ら
\ruby{顏}{がほ}を
\ruby{此方}{こな|た}に% ルビ調整(原本通り)
\ruby{向}{む}けたる
\ruby{其}{そ}の
\ruby{右}{みぎ}の
\ruby{方}{かた}には、
%
\ruby[||j>]{孫}{まご}
\ruby[||j>]{娘}{むすめ}の
% \ruby{孫娘}{まご|むすめ}の
\ruby{一昨年}{をと|ゝ|し}% ルビ調整(原本通り)
\ruby{小學}{せう|がく}を
\ruby{卒}{を}へたる
ばかりなるが、
%
\ruby{何}{なに}を
\ruby{讀}{よ}める
ならんか
\ruby{燈火}{とも|しび}の
\ruby{下}{した}に
\ruby{身}{み}を
\ruby{低}{ひく}く
\ruby{俯}{ふ}して、
%
\ruby{疊}{たゝみ}に
\ruby{置}{お}ける
\ruby{書}{しよ}に
\原本頁{87-1}\改行%
\ruby{餘念}{よ|ねん}
\ruby{無}{な}く
\ruby{讀}{よ}み
\ruby{入}{い}つたる、
%
\ruby{其}{そ}の
\ruby{黑}{くろ}き
\ruby{頭髮}{かし|ら}に
\ruby{何}{なに}やら
\ruby{紅}{あか}き
\ruby{巾}{きれ}
\ruby{美}{うつく}しく、
%
\ruby{一幅}{いつ|ぷく}の
\ruby{{\換字{平}}和}{へい|わ}の
\ruby{夜}{よる}の
\ruby{圖}{づ}は
\ruby{眼}{め}の
\ruby{{\換字{前}}}{まへ}に
\ruby{現}{あら}はれて、
%
\ruby{身}{み}の
\ruby{疲}{つか}れ
\ruby{心}{こゝろ}の
\ruby{勞}{つか}れを
\原本頁{87-3}\改行%
\ruby{休}{やす}むる
\ruby{間}{ま}も
\ruby{無}{な}き
\ruby{水野}{みづ|の}をして、
%
\ruby{人}{ひと}は
\ruby{斯}{か}く
\ruby{無邪氣}{む|じや|き}に
\ruby{世}{よ}を
\ruby{{\換字{送}}}{おく}るもあるをと、
%
そゞろに
\ruby{其}{そ}の
\ruby{無事}{ぶ|じ}の
\ruby{淸福}{せい|ふく}の
\ruby{價値}{あた|ひ}
\ruby[<j||]{貴}{たつと}きを% ルビ調整(原本通り)
\ruby{思}{おも}はしめぬ。

\原本頁{87-5}%
『
ハア、
%
\ruby{左樣}{そ|う}でございますか、
%
\ruby{宜}{よろ}しうございますとも。
%
\換字{志}かし
\原本頁{87-6}\改行%
\ruby{大變}{たい|へん}
せか〳〵して
いらつしやいますが、
%
\ruby{氣}{き}を
\ruby{御付}{お|つ}けなさいまし
\改行% 校正作業の簡略化のため
、
%
\原本頁{87-7}\改行%
\ruby{爭}{あらそ}ひなんぞ
\ruby{爲}{な}すつては
いけませんぜ。
%
\ruby{{\換字{平}}井}{ひら|ゐ}の
お
\ruby{澤}{さは}
\ruby{婆}{ばゞあ}の
ところへ
\原本頁{87-8}\改行%
\ruby{御出}{お|いで}なすつたと
\ruby{聞}{き}きましたが、
%
あの
\ruby{婆}{ばゞあ}と
\ruby{物言}{もの|いひ}なんぞ
\ruby{爲}{な}さりやあ
\ruby{仕}{し}ますまいネ、
%
\ruby{彼奴}{あい|つ}はどうせ
\ruby{人}{ひと}ぢやあ
\ruby{無}{な}いのですから。
%
それは
\ruby{左樣}{そ|う}と
\ruby{岩崎}{いは|ざき}さんは
\ruby{何樣}{ど|う}でございます?。
』

\原本頁{87-11}%
『
\ruby{岩崎}{いは|ざき}は
どうも
いよ〳〵
\ruby{惡}{わる}い。
%
ナーニ
お
\ruby{澤}{さは}
\ruby{婆}{ばあ}さんには
\ruby{此方}{こつ|ち}で% ルビ調整(原本通り)
\ruby{負}{ま}けて
\ruby{居}{ゐ}るから
\ruby{論}{ろん}は
\ruby{無}{な}いよ、
%
\ruby{爭}{あらそ}ひ
なんぞ
\ruby{仕}{し}て
\ruby{來}{き}たのでは
\ruby{無}{な}い。
%
た
\原本頁{88−2}\改行%
ゞ
\ruby{早}{はや}く
\ruby[||j>]{濱}{はま}
\ruby[||j>]{町}{ちやう}へ
% \ruby{濱町}{はま|ちやう}へ
\ruby{行}{ゆ}かうと
\ruby{思}{おも}つて
\ruby{居}{ゐ}るので
\ruby{急}{いそ}いで
\ruby{居}{ゐ}るので。
』

\原本頁{88-3}%
『
\ruby[||j>]{濱}{はま}
\ruby[||j>]{町}{ちやう}は
% \ruby{濱町}{はま|ちやう}は
\ruby{島木}{しま|き}さんの
ところへで
\ruby{御座}{ご|ざ}いますか。
』

\原本頁{88-4}%
『
アヽ
\ruby{左樣}{そ|う}、
%
\ruby{島木}{しま|き}の
ところへだ。
』

\原本頁{88-5}%
『
それぢやあ
\ruby{路}{みち}は
\ruby{{\換字{遠}}}{とほ}いし、
%
\ruby{御會話}{お|はな|し}は
\ruby{長}{なが}くなりませうし、
%
\ruby{御歸}{お|かへ}りは
\ruby{大變}{たい|へん}
\ruby{遲}{おそ}くなりましやうが、
%
なんなら
\ruby{明日}{あ|す}に
なすつては
\ruby{何樣}{ど|う}で
ございます?。
』

\原本頁{88-8}%
『
\ruby{明日}{あ|す}と
\ruby{云}{い}つて
\ruby{居}{ゐ}るわけには
\ruby{行}{い}かないのだから。
』

\原本頁{88-9}%
\ruby{此時}{この|とき}
\ruby{娘}{むすめ}は
\ruby{書}{しよ}を
\ruby{棄}{す}てゝ、
%
\ruby{急}{きふ}に
\ruby{頭}{かうべ}を
\ruby{擡}{もた}げたるが、
%
さつと
\ruby{燈火}{あか|り}を
\ruby{{\換字{浴}}}{あ}びたる
\ruby{面}{おもて}の、
%
\ruby{色}{いろ}は
\ruby{初花}{はつ|はな}の
\ruby{日}{ひ}に
\ruby{匂}{にほ}ふかと
\ruby{麗}{うる}はしく、
%
\ruby{細}{ほそ}けれど
\ruby{鮮}{あざ}やかなる
\ruby{眉}{まゆ}、
%
\ruby{小}{ちひさ}けれども
はつきりと
\ruby{仕}{し}たる
\ruby{眼}{め}つき、
%
まだ
\ruby{罪}{つみ}も
\ruby{無}{な}く
\ruby{慾}{よく}
\原本頁{89-1}\改行%
も
\ruby{無}{な}く、
%
たゞ
\ruby[g]{生々}{いき〳〵}と
\ruby{愛度}{あ|ど}なく
\ruby{美}{うつく}しきが、
%
\ruby{突}{つ}と
\ruby{立上}{たち|あが}りて
\ruby{走}{はし}り
\ruby{出}{い}て
\改行% 校正作業の簡略化のため
、

\原本頁{89-2}%
『
なぜ
\ruby{其樣}{そん|な}に
\ruby{他{\換字{所}}}{よ|そ}へばかし
\ruby{入}{い}らつしやるの!。
%
\ruby{{\換字{戸}}外}{そ|と}は
もう
\ruby{眞闇}{まつ|くら}で、
%
いけませんわ。
%
\ruby[||j>]{妾}{わたし}
\ruby{御願}{ お|ねが}ひだから% ルビ調整(原本通り)
\ruby{御止}{お|よ}しなさいよ。
』

\原本頁{89-4}%
と、
%
\ruby{甘}{あま}へたる
\ruby{調子}{てう|し}に
\ruby{云}{い}ひ〳〵
\ruby{水野}{みづ|の}を
\ruby{扯}{ひ}きて、
%
はや
\ruby{女}{をんな}づくるべき
\原本頁{89-5}\改行%
\ruby{齡}{とし}なれど
\ruby{{\換字{猶}}}{なほ}
\ruby{兒童}{こ|ども}くさく、
%
\ruby{{\換字{遠}}慮}{ゑん|りよ}も
\ruby{無}{な}く
\ruby{此方}{こ|なた}へ% ルビ調整(原本通り)
\ruby{扯}{ひ}き
\ruby{入}{い}れんとすれば
\改行% 校正作業の簡略化のため
、
%
\原本頁{89-6}\改行%
\ruby{水野}{みづ|の}は
おのづと
\ruby{催}{もよほ}さるゝ
\ruby{笑}{わら}ひの
\ruby{顏}{かほ}を
\ruby{顰}{しか}めながら、
%
そつと
\ruby{其}{その}
\ruby{手}{て}をはづして、

\原本頁{89-8}%
『
マア
お
\ruby{濱}{はま}ちやん、
%
\ruby{堪{\換字{忍}}}{か|に}して% 原文通り「堪忍」
お
\ruby{吳}{く}れ、
%
どうしても
\ruby{行}{い}つて
\ruby{來}{こ}なくてはならない
\ruby{事}{こと}だから。
』

\原本頁{89-10}%
と、
%
\ruby{周章}{あ|は}てゝ
\ruby{土間}{ど|ま}へ
\ruby{下}{お}りて
\ruby{出}{い}でかゝるに、
%
\ruby{媚}{なまめ}ける
\ruby{笑}{わら}ひを
\ruby{帶}{お}びたる
\ruby{聲}{こゑ}
\ruby{美}{うつく}しく
\ruby{我}{わ}が
\ruby{背後}{うし|ろ}に
\ruby{當}{あた}つて、

\原本頁{90-1}%
『
あら、
%
いやな
\ruby{人}{ひと}!、
%
きつと
\ruby{{\換字{又}}}{また}
\ruby{五十子}{い|そ|こ}さんの
\ruby{事}{こと}で
\ruby{心配}{しん|ぱい}して
\ruby{居}{ゐ}るのよ!。
』

\原本頁{90-3}%
と、
%
\ruby{{\換字{婦}}人}{をん|な}は
\ruby{口頭}{くち|さき}より
\ruby{先}{ま}づませて、
%
\ruby{戀}{こひ}
\ruby{知}{し}り
\ruby{顏}{がほ}に
\ruby{獨語}{ひとり|ご}つが
\ruby{聞}{きこ}えぬ。
%
\ruby{心}{こゝろ}も
こゝに
あらず
\ruby{思}{おもひ}の
\ruby{忙}{せは}しければ、
%
\ruby{{\換字{平}}生}{ひご|ろ}は% ルビ調整(原本通り)
いと
\ruby{可愛}{か|はゆ}しと
\ruby{思}{おも}へる
\ruby{濱子}{はま|こ}が
\ruby{言葉}{こと|ば}をも、
%
\ruby{我}{わ}が
\ruby{胸}{むね}の
\ruby{中}{うち}に
\ruby{{\換字{留}}}{とゞ}むる
\ruby{暇無}{いとま|な}くて、
%
\ruby{急}{きふ}に
\ruby{村徑}{むら|みち}の
\ruby{闇}{やみ}
\原本頁{90-6}\改行%
を
\ruby{衝}{つ}いて
\ruby{歩}{ある}き
\ruby{出}{いだ}せば、
%
\ruby{門}{かど}を
\ruby{出}{い}づるや
\ruby{否}{いな}や
\ruby{足元}{あし|もと}
\ruby{{\換字{近}}}{ちか}き
\ruby{蓮田}{はす|だ}の% 「蓮 uf999」(参考「蓮 u84ee」)
\ruby{中}{うち}より
\改行% 校正作業の簡略化のため
、
%
\原本頁{90-7}\改行%
\ruby{人}{ひと}に
\ruby{驚}{おどろ}ける
\ruby{五位鷺}{ご|ゐ|さぎ}の
\ruby{其}{その}
\ruby{聲}{こゑ}
\ruby{淋}{さび}しく
\ruby{人}{ひと}を
\ruby{驚}{おどろ}かして、
%
ぎやあと
\ruby{鳴}{な}きつつ
\ruby{立}{た}つて
\ruby{去}{さ}りたり。

\Entry{其十五}

\ruby{勞}{らう}を
\ruby{厭}{いと}ひてにはあらず、
\ruby{時}{とき}を
\ruby{惜}{をし}みて、
\ruby{勸}{すゝ}むる
\ruby{人力車}{く|る|ま}のありしまま、さるところより
\ruby{其車}{そ|れ}には
\ruby{乘}{の}りしが、やうやく
\ruby{濱町}{はま|ちやう}に
\ruby{着}{つ}きしときには、
\ruby{流石}{さす|が}に
\ruby{人}{ひと}の
\ruby{家}{いへ}を
\ruby{音}{おと}づれんは
\ruby{後目痛}{うしろ|め|た}きほど
\ruby{{\換字{更}}}{ふ}けに
\ruby{{\換字{更}}}{ふ}けたり。
\ruby{日頃}{ひ|ごろ}
\ruby{心{\換字{遣}}}{こゝろ|づか}ひの
\ruby{鹵莾}{おろ|か}ならぬ
\ruby{水野}{みづ|の}は、
\ruby{{\換字{鎖}}}{とざ}し
\ruby{固}{かた}めたる
\ruby{{\換字{戸}}}{と}を
\ruby{思}{おも}ひ
\ruby{{\換字{遣}}}{や}りなく
\ruby{打敲}{うち|たゝ}きて、
\ruby{{\換字{近}}隣}{あた|り}の
\ruby{寢耳}{ね|みゝ}をまで
\ruby{驚}{おどろ}かさんことを
\ruby{憚}{はゞか}り、% 「憚 は(ゞ)か」
\ruby{聊}{いさゝ}か
\ruby{自}{みづか}ら
\ruby{躊躇}{ため|ら}ひしが、
\ruby{愚}{おろか}なり、
\ruby{臆}{おく}して
\ruby{已}{や}むべきにはあらぬものをと、
\ruby{手}{て}を
\ruby{擧}{あ}げてほと〳〵と
\ruby{星}{ほし}の
\ruby{下}{した}に
\ruby{敲}{たゝ}きぬ。

\ruby{心}{こゝろ}の
\ruby{優}{やさ}しさにおのづから
\ruby{手}{て}も
\ruby{柔軟}{やはら|か}に
\ruby{當}{あた}りて、
\ruby{其}{そ}の
\ruby{音}{おと}は
\ruby{左}{さ}まで
\ruby{{\換字{強}}}{つよ}からざりしが、
\ruby{幸}{さいはひ}にして
\ruby{未}{ま}だ
\ruby{睡}{ねむ}らざりし
\ruby{女}{をんな}のありけむ、ハイと
\ruby{明}{あき}らかに
\ruby{答}{こた}ふる
\ruby{聲}{こゑ}して、

『
\ruby{誰樣}{どな|た}?。
\ruby{伊東}{い|とう}さん?。
』

と、
\ruby{云}{い}ひながら
\ruby{開}{あ}けにかゝりたり。

\ruby{伊東}{い|とう}とは
\ruby{島木}{しま|き}を
\ruby{外}{ほか}にして
\ruby{唯}{たゞ}
\ruby{一人}{ひと|り}の
\ruby{此}{こ}の
\ruby{家}{や}の
\ruby{止宿者}{き|や|く}にて、
\ruby{無類}{む|るゐ}の
\ruby{極樂蜻蛉}{ごく|らく|とん|ぼ}なるよしを
\ruby{島木}{しま|き}より
\ruby{聞}{き}きしが、さては
\ruby{今{\換字{宵}}}{こ|よひ}はその
\ruby{男}{をとこ}の、
\ruby{何處}{いづ|く}の
\ruby{花}{はな}の
\ruby{陰}{かげ}にか
\ruby{憩}{いこ}ひて、
\ruby{{\換字{更}}}{ふ}けて
\ruby{{\換字{猶}}}{なほ}
\ruby{今}{いま}に
\ruby{歸}{かへ}り
\ruby{來}{きた}らざるを、
\ruby{婢}{をんな}の
\ruby{待}{ま}ち
\ruby{居}{ゐ}たりしならんと
\ruby{早}{はや}くも
\ruby{猜}{すゐ}しぬ。

『イヽエ、
\ruby{島木}{しま|き}さんを
\ruby{急用}{きふ|よう}で
\ruby{{\換字{尋}}}{たづ}ねて
\ruby{來}{き}ました。
わたしは
\ruby{水野}{みづ|の}といふものです。
』

と、
\ruby{云}{い}ふ
\ruby{間}{ま}に
\ruby{雨{\換字{戸}}}{あま|ど}は
\ruby{一枚}{いち|まい}
\ruby{繰}{く}り
\ruby{明}{あ}けられて、
\ruby{細帶姿}{ほそ|おび|すがた}の\換字{志}どけ
\ruby{無}{な}く
\ruby{背後}{うし|ろ}の
\ruby{上}{あが}り
\ruby{端}{はな}に
\ruby{置}{お}きたる
\ruby{小洋燈}{こ|らん|ぷ}の
\ruby{光}{ひかり}の
\ruby{中}{うち}に
\ruby{現}{あらは}れたるは、
\ruby{丸顏}{まる|がほ}の
\ruby{色白}{いろ|じろ}の
\ruby{氣}{き}さくものゝ、
\ruby{名}{な}は
\ruby{忘}{わす}れたれど
\ruby{見記臆}{み|おぼ|ゑ}ある
\ruby{女}{をんな}なり。

『オヤ、
\ruby{水野}{みづ|の}さんでしたか。
\ruby{存}{ぞん}じてましたよ。
たしか
\ruby{彼}{あ}の
\ruby{菖蒲}{しや|うぶ}のある
\ruby{四}{よ}ツ
\ruby{木}{ぎ}とかの。
\ruby{能}{よ}くおぼえて
\ruby{居}{ゐ}たでしやう。
\ruby{褒}{ほ}めて
\ruby{頂戴}{ちやう|だい}な、ホヽヽ、まあ
\ruby{御入}{お|はい}んなさい。
\ruby{大層}{たい|そう}
\ruby{遲}{おそ}く
\ruby{入}{い}らした
\ruby{事}{こと}ネ。
エエ、
\ruby{居}{ゐ}らつしやいますとも
\ruby{島木}{しま|き}さんは。
ハア、イエ
\ruby{未}{ま}だ
\ruby{御睡}{お|よ}り
\ruby{就}{つ}きやなさりますまい、
\ruby{今}{いま}しがた
\ruby{他{\換字{所}}}{よ|そ}から
\ruby{御歸}{お|かへ}りになつたばかりなんですから。
』

と、
\ruby{一人}{ひと|り}で
\ruby{饒舌}{しや|べ}りながら
\ruby{後}{あと}を
\ruby{{\換字{鎖}}}{し}めて、やがて、

『ホヽ、
\ruby{此樣}{こ|ん}な
\ruby{姿}{なり}を
\ruby{仕}{し}て
\ruby{居}{ゐ}て、
\ruby{御免}{ご|めん}なさいましよ。
』

と、
\ruby{云}{い}ひ〳〵
\ruby{先}{さき}に
\ruby{立}{た}つて
\ruby{二階}{に|かい}へ
\ruby{導}{みち}びき、

『
\ruby{島木}{しま|き}さん、さあ
\ruby{御起}{お|お}きなさいまし。
\ruby{貴下}{あな|た}の
\ruby{好}{す}きな
\ruby{水野}{みづ|の}さんが
\ruby{御來臨}{お|い|で}なすつてよ。
\ruby{明日}{あし|た}は
\ruby{驕}{おご}つて
\ruby{下}{くだ}さるでしようネ。
』

と、
\ruby{其室}{その|へや}に
\ruby{入}{い}つて
\ruby{{\換字{遠}}慮無}{ゑん|りよ|な}く
\ruby{洋燈}{らん|ぷ}の
\ruby{火}{ひ}を
\ruby{明}{あか}るくしたり。

『
\ruby{何}{なん}だ
\ruby{驕}{おご}つて
\ruby{下}{くだ}さるで\換字{志}やうも
\ruby{無}{な}いもんだ。
\ruby{自{\換字{分}}}{じ|ぶん}が
\ruby{岡惚}{をか|ぼ}れて
\ruby{居}{ゐ}やがるんだ
\ruby{癖}{くせ}に。
』

と、
\ruby{輕}{かる}く
\ruby{罵}{のゝし}りながら
\ruby{島木}{しま|き}は
\ruby{起}{お}き
\ruby{出}{い}でしが、
\ruby{既}{はや}
\ruby{水野}{みづ|の}の
\ruby{{\換字{近}}々}{ちか|〴〵}と
\ruby{入}{い}り
\ruby{來}{きた}り
\ruby{居}{を}りて、
\ruby{今}{いま}の
\ruby{戯言}{たは|むれ}を
\ruby{聞}{き}きしや
\ruby{苦虫}{にが|むし}を
\ruby{噛}{か}みたる
\ruby{如}{ごと}き
\ruby{顏色}{かほ|つき}なせるを
\ruby{見}{み}て、

『ヤ、
\ruby{失敬}{しつ|けい}
\g詰めruby{々々}{〳〵}。
\ruby{戯言}{じやう|だん}だよ。
\ruby{大層}{たい|そう}
\ruby{遲}{おそ}く
\ruby{來}{き}たぢや
\ruby{無}{な}いか。
さあまあ
\ruby{此上}{こ|れ}に
\ruby{坐}{すわ}つて
\ruby{吳}{く}れたまへ。
』

と、
\ruby{慌}{あわ}てゝ
\ruby{敷物}{しき|もの}を
\ruby{出}{いだ}し、
\ruby{自己}{おの|れ}は
\ruby{手早}{て|ばや}く
\ruby{衣}{い}を
\ruby{改}{あらた}めたり。

『オイお
\ruby{作}{さく}さん、
\ruby{此處}{こ|ゝ}は
\ruby{乃公}{お|れ}が
\ruby{片}{かた}づけて
\ruby{仕舞}{し|ま}ふがネ、もう
\ruby{火}{ひ}は
\ruby[<h||]{皆}{みんな}
\ruby{{\換字{消}}}{き}えて
\ruby{仕舞}{し|ま}つたかエ、せめて
\ruby{御茶}{お|ちや}だけ
\ruby{欲}{ほし}いのだが。
』

『ハア、もう
\ruby{樓下}{し|た}にもありませんが
\ruby{打火}{お|こ}してあげましやう。
ナアニ
\ruby{別段}{べつ|だん}
\ruby{譯}{わけ}はありませんから。
』

\ruby{此家}{こ|ゝ}は
\ruby{家作}{や|づく}りも
\ruby{什器}{だう|ぐ}も
\ruby{淸潔}{き|れい}に、
\ruby{四十五六}{し|じう|ご|ろく}の
\ruby{女主}{をんな|あるじ}人と、
\ruby{此女}{こ|れ}と、
\ruby{下働}{した|ばたら}きの
\ruby{婢}{をんな}と
\ruby{三人}{さん|にん}して、
\ruby{客}{きやく}はたゞ
\ruby{二人}{ふた|り}の
\ruby{島木}{しま|き}
\ruby{伊東}{い|とう}をかしづく
\ruby{下宿屋}{げ|しゆく|や}めかさぬ
\ruby{品}{ひん}の
\ruby{良}{よ}き
\ruby{家}{いへ}なれど、
\ruby{{\換字{又}}}{また}
\ruby{折々}{をり|〳〵}は
\ruby{骨牌}{は|な}に
\ruby{貸}{か}す
\ruby{窩}{あな}ともなり
\ruby{{\換字{兼}}}{か}ねぬほど、
\ruby{一切}{すべ|て}を
\ruby{金錢}{か|ね}の
\ruby{光}{ひかり}に
\ruby{美}{うつく}しく
\ruby{仕}{し}こなして
\ruby{見}{す}するところとは
\ruby{知}{し}りながら、
\ruby{深夜}{しん|や}に
\ruby{人}{ひと}を
\ruby{煩}{わずら}はすことの
\ruby{氣}{き}の
\ruby{毒}{どく}さに
\ruby{耐}{た}へかねて、

『マアいゝさ
\ruby{島木君}{しま|き|くん}、
\ruby{茶}{ちや}なぞは
\ruby{要}{い}らんよ、
お
\ruby{作}{さく}さんはもう
\ruby{寢}{やす}んで
\ruby{吳}{く}れたまへ。
』

と、
\ruby{水野}{みづ|の}は
\ruby{言葉}{こと|ば}を
\ruby{挿}{さしはさ}まざるを
\ruby{得}{え}ざりき。

\ruby{島木}{しま|き}は
\ruby{物}{もの}に
\ruby{滞}{とゞこほ}らずして、
\ruby{心}{こゝろ}の
\ruby{動}{うご}きの
\ruby{早}{はや}き
\ruby{男}{をとこ}なれば、

『ン、それも
\ruby{左樣}{さ|う}だ。
ぢやあお
\ruby{作}{さく}さん
\ruby{茶}{ちや}はいゝからね、そら
\ruby{彼}{あ}の
\ruby{葡萄酒}{ぶ|だう|しゆ}と
\ruby{乾燥牛肉}{ドラ|イド|ビー|フ}とを
\ruby{持}{も}つて
\ruby{來}{き}て
お
\ruby{吳}{く}れ。
』

と
\ruby{云}{い}へば、

『ハア、
\ruby{其}{そ}の
\ruby{方}{はう}が% 原本では「方」のルビが欠けているが他と合わせて「はう」
\ruby{却}{かへ}つて
\ruby{宜}{よろ}しう
\ruby{御座}{ご|ざ}んしやう。
』

と、
\ruby{婢}{をんな}は
\ruby{下}{した}に
\ruby{降}{お}り
\ruby{行}{ゆ}きしが、
\ruby{忽地}{たちま|ち}にして
\ruby{一}{ひと}つの
\ruby{廣}{ひろ}き
\ruby{{\換字{盆}}}{ぼん}に、
\ruby{燈}{ひ}を
\ruby{受}{う}けて
\ruby{美}{うつく}しきポカラの
\ruby{玻璃盞}{コ|ツ|プ}
\ruby{二}{ふた}つ、
\ruby{薄手}{うす|で}の
\ruby{白皿}{しろ|ざら}
\ruby{二}{ふた}つ、ニツケルの
\ruby{栓拔器}{せん|ぬ|き}、まだ
\ruby{開}{あ}けぬ
\ruby{薄}{うす}き
\ruby{罐詰}{くわん|づめ}、
\ruby{利休箸}{り|きう|ばし}を
\ruby{載}{の}せて、
\ruby{片手}{かた|て}に
\ruby{葡萄酒}{ぶ|だう|しゆ}の
\ruby{罎}{びん}を
\ruby{提}{ひつさ}げて
\ruby{來}{きた}りぬ。

『よし〳〵。
もうこれで
\ruby{好}{い}いから
\ruby{樓下}{し|た}へ
\ruby{行}{い}つて
\ruby{御就眠}{お|や|す}み。
\ruby{御客樣}{お|きやく|さま}が
\ruby{氣}{き}の
\ruby{{\換字{通}}}{とほ}つた
\ruby{方}{かた}だから
\ruby{御{\換字{酌}}}{お|しやく}には
\ruby{及}{およ}ばない。
\ruby{{\換字{勝}}手}{かつ|て}に
\ruby{御免}{ご|めん}を
\ruby{蒙}{かうむ}るさ。
』

『それぢやあ、
\ruby{御二人}{お|ふ|たり}で
\ruby{水入}{みづ|い}らずに
\ruby{御話}{お|はなし}なさいまし、まあ
\ruby{御睦}{お|むつ}まじいこと、
\ruby{些}{ちと}
\ruby{妬}{や}けますネ。
ホヽヽヽ。
ですけれど
\ruby{島木}{しま|き}さん
\ruby{御用}{ご|よう}がありましたなら
\ruby{構}{かま}はないで
\ruby{呼}{よ}んで
\ruby{下}{くだ}さいましよ。
』

\ruby{婢}{をんな}は
\ruby{樓下}{し|た}に
\ruby{去}{さ}つて
\ruby{行}{ゆ}きたり。
\ruby{手早}{て|ばや}く
\ruby{片}{かた}づけられたる
\ruby{座敷}{ざ|しき}の
\ruby{好}{よ}き
\ruby{程}{ほど}に
\ruby{坐}{すわ}りて、
\ruby{島木}{しま|き}は
\ruby{葡萄酒}{ぶ|だう|しゆ}の
\ruby{栓}{せん}を
\ruby{拔}{ぬ}きながら
\ruby{水野}{みづ|の}の
\ruby{面}{おもて}を
\ruby{見}{み}て、

『
\ruby{君}{きみ}、
\ruby{大層}{たい|そう}
\ruby{顏色}{かほ|いろ}が
\ruby{惡}{わる}いぢや
\ruby{無}{な}いか。
\ruby{何樣}{ど|う}か
\ruby{仕}{し}はせんか、
\ruby{氣}{き}になるネ。
さあ、まあ、
\ruby{飮}{や}つて
\ruby{吳}{く}れたまへナ。
』

と、
\ruby{詞}{ことば}の
\ruby{調子}{てう|し}こそ
\ruby{{\換字{猶}}}{なほ}
\ruby{冴}{さ}えたれ、
\ruby{顏}{かほ}には
\ruby{憂愁}{うれ|ひ}の
\ruby{曇}{くも}りを
\ruby{上}{のぼ}せて、
\ruby{友}{とも}を
\ruby{思}{おも}ふ
\ruby{{\換字{情}}}{こゝろ}の
\ruby{溫}{あたゝ}かくも
\ruby{溫}{あたゝ}かく、
\ruby{{\換字{強}}}{し}ひて
\ruby{玻璃盞}{コ|ツ|プ}を
\ruby{執}{と}らせて
\ruby{注}{つ}ぎて
\ruby{{\換字{遣}}}{や}りたる
\ruby{酒}{さけ}はいつはり
\ruby{無}{な}き
\ruby{血}{ち}の
\ruby{色}{いろ}をなしたり。

\Entry{其十六}

いつもながらの
\ruby{島木}{しま|き}が
\ruby{親切}{しん|せつ}の、
\ruby{今{\換字{宵}}}{こ|よひ}は
\ruby{別}{わ}けて
\ruby{身}{み}に
\ruby{染}{し}む
\ruby{心地}{こゝ|ち}して、
\ruby{今}{いま}までには
\ruby{經驗}{おぼ|え}
\ruby{無}{な}き
\ruby{事}{こと}なるが、おのずと
\ruby{脆}{もろ}くも
\ruby{涙}{なみだ}の
\ruby{湧}{わ}き
\ruby{上}{あが}るを、
\ruby{水野}{みづ|の}は
\ruby{怪}{あやし}まれやせんと
\ruby{竊}{そつ}と
\ruby{拭}{ぬぐ}ひて、わざと
\ruby{眼}{め}の
\ruby{行}{ゆ}く
\ruby{方}{かた}を
\ruby{{\換字{逸}}}{そ}らして
\ruby{床}{とこ}の
\ruby{間}{ま}を
\ruby{見}{み}つ、
\ruby{其處}{そ|こ}に
\ruby{掛}{かゝ}れる
\ruby{狩野風}{かの|う|ふう}の
\ruby{{\換字{達}}磨}{だる|ま}を、たゞ
\ruby{譯}{わけ}も
\ruby{無}{な}く
\ruby{見}{み}つめながら、

『ナニ
\ruby{何樣}{ど|う}も
\ruby{仕}{し}は
\ruby{仕無}{し|な}いよ、
\ruby{心配}{しん|ぱい}して
\ruby{吳}{く}れたまふな。
』

と、
\ruby{然}{さ}ばかり
\ruby{我}{わ}が
\ruby{胸}{むね}の
\ruby{中}{うち}の
\ruby{苦惱}{くる|しみ}の
\ruby{色}{いろ}に
\ruby{出}{い}でゝ、
\ruby{人目}{ひと|め}に
\ruby{著}{しる}く
\ruby{現}{あら}はるゝかと
\ruby{驚}{おどろ}かるゝ
\ruby{心}{こゝろ}を
\ruby{押}{お}し
\ruby{隱}{かく}して
\ruby{答}{こた}へぬ。

『
\ruby{左樣}{さ|う}かエ。
それなら
\ruby{好}{い}いが
\ruby{餘}{あんま}り
\ruby{氣}{き}を
\ruby{使}{つか}つちやあいけないぜ、
\ruby{今日}{け|ふ} --- イヤ
\ruby{今日}{け|ふ}と
\ruby{云}{い}つちやあ
\ruby{既}{もう}
\ruby{十二時}{じう|に|じ}
\ruby{{\換字{過}}}{す}ぎだからをかしい。
\ruby{昨{\換字{宵}}}{ゆふ|べ}の
\ruby{會}{くわい}にも、
\ruby{君}{きみ}は
\ruby{幹事}{かん|じ}の
\ruby{山瀬}{やま|せ}のところへ、
\ruby{君}{きみ}の
\ruby{友人}{いう|じん}が
\ruby{大病}{たい|びやう}で、
\ruby{介抱}{かい|はう}の
\ruby{仕手}{し|て}も
\ruby{無}{な}いから
\ruby{其}{そ}の
\ruby{爲}{ため}に
\ruby{出}{で}ぬ、と
\ruby{云}{い}つて
\ruby{{\換字{遣}}}{や}つたさうだが、
\ruby{君}{きみ}は
\ruby{一體}{いつ|たい}
\ruby{{\換字{情}}}{じやう}が
\ruby{深}{ふ}か
\ruby{{\換字{過}}}{す}ぎるから、
\ruby{餘計}{よ|けい}にそれで
\ruby{心勞}{しん|らう}でも
\ruby{仕}{し}や
\ruby{仕無}{し|な}いかと、
\ruby{一同}{みん|な}が
\ruby{君}{きみ}の
\ruby{爲}{ため}に
\ruby{心配}{しん|ぱい}してゐたよ。
』

\ruby{島木}{しま|き}が
\ruby{言葉}{こと|ば}には
\ruby{何}{なん}の
\ruby{事}{こと}も
\ruby{無}{な}けれど、
\ruby{水野}{みづ|の}が
\ruby{胸}{むね}には
\ruby{響}{ひび}くところあり。

『ムヽ、
\ruby{昨{\換字{宵}}}{ゆふ|べ}の
\ruby{羽{\換字{勝}}}{は|がち}の
\ruby{君}{くん}の
\ruby{會}{くわい}に
\ruby{出無}{で|な}かつたのは、
\ruby{眞誠}{ほん|と}に
\ruby{諸君}{しよ|くん}に
\ruby{濟}{す}まなかつたが、
\ruby{實}{じつ}は
\ruby{如是}{か|う}してまご〳〵して
\ruby{居}{ゐ}て、
\ruby{今}{いま}
\ruby{頃君}{ごろ|きみ}のところへ
\ruby{來}{く}る
\ruby{位}{くらゐ}だから、
\ruby{何樣}{ど|う}か
\ruby{察}{さつ}して
\ruby{赦}{ゆる}して
\ruby{吳}{く}れたまへ。
』

『ナアニ
\ruby{赦}{ゆる}すも
\ruby{赦}{ゆる}さないも
\ruby{有}{あ}りあ
\ruby{仕無}{し|な}いが
\ruby{君}{きみ}のその
\ruby{友人}{いう|じん}の
\ruby{上}{うへ}は
\ruby{兎}{と}に
\ruby{角}{かく}、
\ruby{一同}{みん|な}は
\ruby{眞誠}{ほん|と}にたゞ
\ruby{君}{きみ}の
\ruby{上}{うへ}をいろ〳〵に
\ruby{心配}{しん|ぱい}してゐたよ。
』

『ヤ、
\ruby{眞}{しん}に
\ruby{諸君}{みん|な}の
\ruby{厚意}{こう|い}は
\ruby{深}{ふか}く
\ruby{謝}{しや}する。
\ruby{誰}{たれ}も
\ruby{僕}{ぼく}の
\ruby{不參}{ふ|さん}を
\ruby{怒}{おこ}りは
\ruby{仕無}{し|な}かつたかね。
\ruby{日方}{ひ|かた}
\ruby{君}{くん}は
\ruby{何}{なん}とも
\ruby{云}{い}は
\ruby{無}{な}かつたかね。
』

『ムヽ、
\ruby{日方}{ひ|かた}は
\ruby{何}{なに}を
\ruby{言}{い}つたつて
\ruby{管}{かま}や
\ruby{仕無}{し|な}いがね、
\ruby{羽{\換字{勝}}}{は|がち}は
\ruby{君}{きみ}に
\ruby{會}{あ}へなかつたのを、
\ruby{口}{くち}には
\ruby{出}{だ}さなかつたが
\ruby{酷}{ひど}く
\ruby{殘念}{ざん|ねん}がつて
\ruby{居}{ゐ}たよ。
』

『アヽ、
\ruby{羽{\換字{勝}}}{は|がち}
\ruby{君}{くん}には
\ruby{僕}{ぼく}も
\ruby{會}{あ}ひたがつたが、
\ruby{何}{なん}にしろ
\ruby{一方}{いつ|ぱう}の
\ruby{事}{こと}があつたので、
\ruby{懷}{なつ}かしくは
\ruby{思}{おも}ひながら
\ruby{意}{い}に
\ruby{任}{まか}せ
\ruby{無}{な}かつた。
アヽ
\ruby{僕}{ぼく}は
\ruby{羽{\換字{勝}}}{は|がち}
\ruby{君}{くん}に
\ruby{負}{そむ}いた、
\ruby{濟}{す}まなかつた。
』

\ruby{水野}{みづ|の}は
\ruby{{\換字{情}}}{じやう}に
\ruby{堪}{た}へざる
\ruby{如}{ごと}く、\換字{志}つと
\ruby{俯首}{うつ|む}きて
\ruby{眼}{め}を
\ruby{瞑}{ふさ}ぎつゝ、
\ruby{獨語}{ひとり|ごと}のやうに
\ruby{{\換字{又}}}{また}
\ruby{再度}{ふた|ゝび}、

『アヽ、
\ruby{濟}{す}まなかつた。
』

と、
\ruby{繰}{く}り
\ruby{{\換字{返}}}{かへ}しぬ。
\ruby{島木}{しま|き}は
\ruby{其}{そ}のいぢらしき
\ruby{樣子}{やう|す}を
\ruby{見}{み}て、
\ruby{此}{こ}の
\ruby{{\換字{猶}}}{なほ}
\ruby{心}{こゝろ}の
\ruby{醇}{じゆん}なる
\ruby{年{\換字{若}}}{とし|わか}き
\ruby{友}{とも}を
\ruby{愛憐}{いと|ほし}む
\ruby{{\換字{情}}}{こゝろ}を
\ruby{起}{おこ}さゞるを
\ruby{得}{え}ざりき。

『マア
\ruby{其}{そ}りやあ
\ruby{其}{そ}れで
\ruby{濟}{す}んだ
\ruby{事}{こと}として、また
\ruby{羽{\換字{勝}}}{は|がち}に
\ruby{{\換字{遇}}}{あ}う
\ruby{時}{とき}も
\ruby{有}{あ}らうから
\ruby{好}{い}いぢやあ
\ruby{無}{な}いか。
さうして
\ruby{君}{きみ}のわざ〳〵
\ruby{來}{き}た
\ruby{用事}{よう|じ}といふのは?。
』

\ruby{問}{と}はれて
\ruby{水野}{みづ|の}は
\ruby{猛然}{まう|ぜん}と
\ruby{我}{われ}に
\ruby{復}{かへ}り、
\ruby{夜}{よ}を
\ruby{冒}{をか}し
\ruby{{\換字{遠}}}{とほき}を
\ruby{歩}{あゆ}みて
\ruby{此處}{こ|ゝ}に
\ruby{來}{きた}れるも、たゞ
\ruby{此}{こ}の
\ruby{一}{ひと}つの
\ruby{事}{こと}のためなるをやと、
\ruby{津}{わたり}に
\ruby{舟}{ふね}を
\ruby{得}{え}し
\ruby{心地}{こゝ|ち}して、
\ruby{自}{みづか}ら
\ruby{奮}{ふる}つて
\ruby{面}{おもて}を
\ruby{擡}{あ}げしが、
\ruby{慚}{は}づるところの
\ruby{有}{あ}ればにや
\ruby{直}{すぐ}に
\ruby{崩折}{くづ|を}れて、
\ruby{甲{\換字{斐}}無}{か|ひ|な}くも
\ruby{伏目}{ふし|め}になりて
\ruby{我}{わ}が
\ruby{膝}{ひざ}を
\ruby{見}{み}たり。

されど
\ruby{云}{い}はでは
\ruby{叶}{かな}はざることゝて、

『
\ruby{深夜}{しん|や}に
\ruby{君}{きみ}を
\ruby{驚}{おどろ}かしたのは
\ruby{濟}{す}ま
\ruby{無}{な}かつたが、かういふ
\ruby{譯}{わけ}だから
\ruby{聞}{き}いて
\ruby{吳}{く}れたまへ。
\ruby{實}{じつ}は
\ruby{僕}{ぼく}の
\ruby{出}{で}て
\ruby{居}{ゐ}る
\ruby{學校}{がく|かう}で、
\ruby{同}{おな}じ
\ruby{職}{しよく}を
\ruby{取}{と}つて
\ruby{居}{ゐ}るものに、
\ruby{僕}{ぼく}の
\ruby{新}{あたら}しい
\ruby{友人}{いう|じん}がある。
\ruby{其人}{そ|れ}は
\ruby{物}{もの}も
\ruby{出來}{で|き}れば
\ruby{氣立}{き|だて}も
\ruby{立派}{りつ|ぱ}な、まことに
\ruby{得難}{え|がた}い
\ruby{人物}{じん|ぶつ}なので、
\ruby{僕}{ぼく}は
\ruby{非常}{ひ|じやう}に
\ruby{大切}{たい|せつ}に
\ruby{思}{おも}つて
\ruby{居}{ゐ}る、ところが
\ruby{其人}{そ|れ}が
\ruby{大病}{たい|びやう}に
\ruby{罹}{かゝ}つた。
\ruby{一體}{いつ|たい}
\ruby{愍然}{あは|れ}な
\ruby{不幸}{ふ|かう}な
\ruby{人}{ひと}で、
\ruby{母}{はゝ}は
\ruby{有}{ゐ}るけれども
\ruby{継}{まゝ}しい
\ruby{中}{なか}で、
\ruby{病氣}{びやう|き}を
\ruby{知}{し}らせて
\ruby{{\換字{遣}}}{や}つても
\ruby{振}{ふ}り
\ruby{顧}{かへ}つても
\ruby{見無}{み|な}い
\ruby{位}{くらゐ}、それにまた
\ruby{家}{いへ}を
\ruby{貸}{か}して
\ruby{居}{ゐ}る
\ruby{婆}{ばゞあ}が
\ruby{殘酷}{ざん|こく}な
\ruby{奴}{やつ}で、
\ruby{病}{や}み
\ruby{惱}{なや}んで
\ruby{居}{ゐ}るものを
\ruby{{\換字{逐}}}{お}ひ
\ruby{出}{だ}さうといふ
\ruby{位}{くらゐ}な
\ruby{非{\換字{道}}}{ひ|だう}さ。
\ruby{左樣}{さ|う}いふ
\ruby{中}{なか}に
\ruby{悶臥}{もん|ぐわ}して
\ruby{居}{ゐ}て、
\ruby{誰}{たれ}に
\ruby{世話}{せ|わ}をされるといふ
\ruby{事}{こと}も
\ruby{無}{な}いので、
\ruby{可哀}{か|あい}さうに
\ruby{病人}{びやう|にん}は
\ruby{死}{し}を
\ruby{待}{ま}つばかりになつて
\ruby{居}{ゐ}るのだ。
そこで
\ruby{何樣}{だ|う}しても
\ruby{餘{\換字{所}}}{よ|そ}に
\ruby{見{\換字{兼}}}{み|か}ねるから、
\ruby{僕}{ぼく}が
\ruby{奔走}{ほん|そう}して
\ruby{良}{い}い
\ruby{醫者}{い|しや}に
\ruby{見}{み}せて
\ruby{{\換字{遣}}}{や}ると、
\ruby{病}{やまひ}は
\ruby{腸窒扶斯}{ちやう|ち|ぷ|す}だといふ
\ruby{事}{こと}で、
\ruby{看護}{かん|ご}が
\ruby{行屆}{ゆき|とゞ}か
\ruby{無}{な}けりやあ
\ruby{無}{な}い
\ruby{生命}{いの|ち}だといふ。
\ruby{僕}{ぼく}は
\ruby{自{\換字{分}}}{じ|ぶん}の
\ruby{肉}{にく}を
\ruby{{\換字{削}}}{そ}いで
\ruby{食}{く}はせてなりと、
\ruby{何樣}{ど|う}かして
\ruby{助}{たす}けて
\ruby{{\換字{遣}}}{や}りたいと
\ruby{思}{おも}ふのだが、……』

と、
\ruby{虛言}{う|そ}は
\ruby{少}{すこし}も
\ruby{無}{な}けれど
\ruby{忌}{い}むことは
\ruby{忌}{い}みて、
\ruby{此處}{こ|ゝ}までは
\ruby{云}{い}ひたりしが
\ruby{後}{あと}は
\ruby{言}{い}ひ
\ruby{澱}{よど}むを、
\ruby{其}{そ}の
\ruby{聲}{こゑ}の
\ruby{微}{かすか}に
\ruby{顫}{ふる}ふを
\ruby{聞}{き}き、
\ruby{其}{そ}の
\ruby{眼}{め}の
\ruby{濕}{ぬ}れ
\ruby{色}{いろ}なせるを
\ruby{見}{み}て、

『アヽ、
\ruby{解}{わか}つたよ、もう
\ruby{可}{い}いさ、
\ruby{君}{きみ}。
\ruby{金子}{か|ね}が
\ruby{先}{さき}に
\ruby{立}{た}つからと
\ruby{云}{い}ふのだらう。
\換字{志}て
\ruby{何}{ど}の
\ruby[<h||]{位}{くらゐ}
\ruby[||h>]{用立}{よう|だ}てやうかエ。
』

と、
\ruby{輕々}{かろ|〴〵}と
\ruby{事}{こと}も
\ruby{無}{な}げに
\ruby{引取}{ひつ|と}つて
\ruby{云}{い}つて、
\ruby{云}{い}ひ
\ruby{難}{にく}き
\ruby{口數}{くち|かず}を
\ruby{多}{おほ}くはきかせぬ
\ruby{同{\換字{情}}}{おも|ひやり}の
\ruby{骨}{ほね}に
\ruby{徹}{てつ}するほど
\ruby{嬉}{うれ}し
\ruby{悲}{かな}しく、

『
\ruby{濟}{す}まないけれども
\ruby{一時}{いち|どき}で
\ruby{無}{な}くとも
\ruby{可}{い}いから
\ruby{百圓}{ひやく|ゑん}ばかり、』

と、
\ruby{纔}{わづか}に
\ruby{口}{くち}を
\ruby{洩}{も}らせし
\ruby{限}{き}り、あとは
\ruby{無言}{む|ごん}の
\ruby{頭}{かうべ}を
\ruby{低}{た}れて、
\ruby{深々}{ふか|〴〵}と
\ruby{頼}{たの}み
\ruby{入}{い}りたりしが、
\ruby{何時}{い|つ}より
\ruby{出}{い}で
\ruby{居}{ゐ}し
\ruby{涙}{なみだ}なりけん、
\ruby{人}{ひと}の
\ruby{{\換字{情}}}{こゝろ}の
\ruby{凝}{こ}りて
\ruby{滴}{したゝ}る
\ruby{露}{つゆ}の
\ruby{眞玉}{ま|だま}はぱらりと
\ruby{墜}{お}ちたり。

\ruby{誠}{まこと}せめて
\ruby{人}{ひと}を
\ruby{頼}{たの}む
\ruby{心}{こゝろ}のいぢらしさも、
\ruby{何時}{い|つ}の
\ruby{間}{ま}にか
\ruby{謹}{つゝし}みて
\ruby{律義}{りち|ぎ}に
\ruby{端座}{す|わ}り
\ruby{居}{ゐ}たる、
\ruby{水野}{みづ|の}が
\ruby{身}{み}を
\ruby{窄}{すぼ}めし
\ruby{姿}{すがた}の
\ruby{{\換字{寒}}}{さむ}げなるを
\ruby{見}{み}て、
\ruby{島木}{しま|き}は
\ruby{思}{おも}はず
\ruby{慨然}{がい|ぜん}として、

『ナアニ
\ruby{可}{い}いさ。
\ruby{君}{きみ}、それんばかりの
\ruby{事}{こと}を。
\ruby{宜}{よろ}しい、
\ruby{承知}{しよう|ち}した。
\ruby{今}{いま}
\ruby{直}{すぐ}
\ruby{献}{あ}げる。
』

と、
\ruby{確然}{しつ|かり}と
\ruby{明}{あき}らかに
\ruby{先}{ま}づ
\ruby{答}{こた}へつ、
\ruby{少時}{しば|し}
\ruby{間}{あひだ}を
\ruby{置}{お}きて、

『\換字{志}かし、
\ruby{君}{きみ}、
\ruby{僕}{ぼく}は
\ruby{何}{なに}も
\ruby{君}{きみ}に
\ruby{恨}{うら}みを
\ruby{云}{い}ふのでは
\ruby{無}{な}いが、
\ruby{何故}{な|ぜ}
\ruby{君}{きみ}は
\ruby{僕}{ぼく}に
\ruby{其}{そ}の
\ruby{友人}{いう|じん}の
\ruby{名}{な}を、
\ruby{岩崎五十子}{いは|ざき|い|そ|こ}といふものだとは
\ruby{云}{い}つて
\ruby{吳}{く}れぬ?。
イヤ、
\ruby{吃驚}{びつ|くり}しないでも
\ruby{宜}{よ}い、
\ruby{意見}{い|けん}は
\ruby{云}{い}は
\ruby{無}{な}いが、』

と、
\ruby{何事}{なに|ごと}をか
\ruby{徐}{しづか}に
\ruby{云}{い}ひ
\ruby{出}{だ}さんとすれば、
\ruby{水野}{みづ|の}が
\ruby{面}{おもて}はたゞ
\ruby{火}{ひ}となつたり。

\Entry{其十七}

% メモ 校正 2024-04-06 2024-05-25 2024-06-17
\原本頁{105-2}%
\ruby[g]{自信}{じ しん}は
\ruby{{\換字{強}}}{つよ}くとも、
%
\ruby[g]{學問}{がくもん}は
\ruby{博}{ひろ}くとも、
%
\ruby{氣}{き}の
\ruby{働}{はたら}きは
\ruby[g]{八方}{はつぱう}に
\ruby{{\換字{銳}}}{するど}くとも
\改行% 校正作業の簡略化のため
、
%
\原本頁{105-3}\改行%
\ruby{未}{ま}だ
\ruby{世}{よ}に
\ruby{老}{お}いぬ
\ruby{心}{こゝろ}の
\ruby[g]{柔輭}{やはらか}に
\ruby{嫩}{わか}ければ、
%
\ruby{人}{ひと}には
\ruby{知}{し}らさず
\ruby{祕}{ひ}め
\ruby{置}{お}きたることを、
%
つけ〳〵と
\ruby[g]{覿面}{てきめん}に
\ruby{云}{い}ひ
\ruby{出}{いだ}されては、
%
\ruby{胸}{むね}の
\ruby{眞正中}{まつ|たゞ|なか}を
\原本頁{105-5}\改行%
\換字{志}たゝかなる
\ruby{箭}{や}に、
%
\ruby[g]{羽中}{は なか}の
\ruby{{\換字{節}}}{ふし}せめて
\ruby[g]{射{\換字{込}}}{い こ }まれたる
\ruby{思}{おも}ひして、
%
ハツと
\ruby{驚}{おどろ}き
\ruby{惑}{まど}ひしが、
%
\ruby[g]{元來}{も と }
\ruby{底}{そこ}の
\ruby{{\換字{弱}}}{よわ}からぬ
\ruby{男}{をとこ}なり、
%
\ruby{忽}{たちま}ち
\ruby{我}{われ}に
\ruby{{\換字{返}}}{かへ}つて
\原本頁{105-7}\改行%
\ruby{惡}{わる}びれず、
%
\ruby{靜}{しづか}に
\ruby{我}{わ}が
\ruby[g]{腔内}{む ね }の
\ruby{血}{ち}の
\ruby{跳}{をど}りの
\ruby{鎭}{しづ}まるを
\ruby{待}{ま}ちながら、
%
\ruby{身}{み}
\ruby{動}{うご}きだに
せずして
\ruby[g]{大人}{おとな }しく、
%
\ruby[g]{島木}{しまき }の
いふところを
\ruby{聞}{き}かんと
\ruby{仕}{し}た
\改行% 校正作業の簡略化のため
り。

\原本頁{105-10}%
\ruby[g]{島木}{しまき }は
\ruby{人}{ひと}の
\ruby{{\換字{情}}}{こゝろ}の
\ruby{流}{なが}れの
\ruby{瀬}{せ}に、
%
\ruby{慣}{な}れきつたる
\ruby{鵜}{う}の
\ruby{目}{め}の
\ruby{働}{はたら}き
\ruby[g]{敏捷}{す ばや}く
\改行% 校正作業の簡略化のため
、
%
\原本頁{106-1}\改行%
\ruby{日}{ひ}の
\ruby{光}{ひかり}の
\ruby{明}{あき}らかなるに
\ruby{我}{わ}が
\ruby{影}{かげ}を
\ruby{怯}{お}づる
\ruby[g]{{\換字{若}}鮎}{わかあゆ}の
\ruby[g]{振舞}{ふるまひ}の、
%
\ruby{優}{やさ}しくも
\原本頁{106-2}\改行%
\換字{志}ほらしき
\ruby[g]{水野}{みづの }が
\ruby[g]{樣子}{やうす }を
\ruby{見}{み}て
\ruby{取}{と}つて、
%
\ruby{曾}{かつ}て
\ruby{吉右衛門}{きち||ゑ|もん}より
\ruby{聞}{き}きしと、
%
\ruby{今}{いま}
\ruby[g]{直接}{ぢ か }に
\ruby{聞}{き}きしとの
\ruby{二}{ふた}つの
\ruby[g]{談話}{はなし }に
\ruby{照}{て}らし
\ruby{合}{あ}はせて、
%
\ruby[g]{大槪}{おほよそ}の
\原本頁{106-4}\改行%
\ruby{事}{こと}は
\ruby{曉}{さと}り
\ruby{盡}{つく}しつ、
%
\ruby{今}{いま}
\ruby{{\換字{更}}}{さら}に
また
\ruby[g]{油然}{ゆうぜん}として
\ruby[g]{愛憐}{いとほし}む
\ruby{心}{こゝろ}の
\ruby{起}{おこ}るに
\ruby{堪}{た}へぬが
\ruby{如}{ごと}く、
%
\ruby[g]{言葉}{ことば }づかひも
\ruby{砕}{くだ}けて
\ruby{露}{つゆ}
\ruby[g]{隔氣}{へだて }なく、
%
いと
\ruby{親}{した}しくも
\ruby{說}{と}き
\原本頁{106-6}\改行%
\ruby{出}{いだ}したり。

\原本頁{106-7}%
『% 原本ではこの島木の語りの終わりである「』」は欠落している
ねえ
\ruby{君}{きみ}、
%
\ruby[g]{可厭}{い や }なものは、
%
\ruby[g]{無心}{む しん}を
\ruby{聽}{き}いた
\ruby{後}{あと}で
\ruby[g]{意見}{い けん}
\ruby{云}{い}ふ
\ruby{奴}{やつ}だと、
%
\ruby{{\換字{古}}}{むかし}から
\ruby{云}{い}つてあるぢや
\ruby{無}{な}いか。
%
ハヽヽ
まさかに
\ruby{僕}{ぼく}だつて
\ruby[<j||]{其}{その }% 行末行頭の境界付近なので特例処置を施す
\ruby[<j||]{位}{くらゐ}な
% \ruby{其位}{その|くらゐ}な
\原本頁{106-9}\改行%
\ruby{事}{こと}は
\ruby{知}{し}つて
\ruby{居}{ゐ}るから、
%
\ruby[g]{此處}{こ ゝ }で
\ruby[g]{下手}{へ た }な
\ruby[g]{叔{\換字{父}}}{を ぢ }さんの
\ruby{役}{やく}を
\ruby{{\換字{勤}}}{つと}めて、
%
\ruby{何}{なん}の
\ruby{彼}{か}のと
\ruby{{\換字{難}}}{むづ}かしい
\ruby{事}{こと}を
\ruby{云}{い}ふなあ
\ruby[g]{自{\換字{分}}}{じ ぶん}で
\ruby{願}{ねが}ひ
\ruby{下}{さ}げるし、
%
\ruby{{\換字{又}}}{また}
\ruby[g]{理屈}{り くつ}な
\原本頁{106-11}\改行%
んぞといふ
\ruby[g]{野暮}{や ぼ }なものを、
%
\ruby{餘}{あま}り
\ruby{有}{あ}り
\ruby{{\換字{難}}}{がた}いと
\ruby{思}{おも}つてゐる
\ruby{僕}{ぼく}でも
\ruby{無}{な}いから、
%
\ruby{君}{きみ}が
\ruby[g]{何樣}{ど う }
\ruby{仕}{し}やうと、
%
\ruby[g]{斯樣}{か う }
\ruby{仕}{し}やうと、
%
それを
\ruby{兎}{と}や
\ruby{角}{かく}いふ
\原本頁{107-2}\改行%
\ruby{僕}{ぼく}ぢやあ
\ruby{無}{な}い。
%
\ruby{惡}{わる}い
\ruby{事}{こと}さへ
\ruby[g]{仕無}{し な }けりやあ、
%
\ruby{好}{す}きな
\ruby{事}{こと}を
\ruby{仕}{し}て
\ruby[g]{面白}{おもしろ}
\原本頁{107-3}\改行%
く
\ruby{世}{よ}を
\ruby{渡}{わた}るのが、
%
\ruby{可}{い}いぢやあ
\ruby{無}{な}いかと
いふのが
\ruby{僕}{ぼく}の
\ruby[g]{宗旨}{しゆうし}なのは
\改行% 校正作業の簡略化のため
、
%
\原本頁{107-4}\改行%
\ruby{君}{きみ}も
\ruby{知}{し}つて
\ruby{居}{ゐ}る
\ruby{{\換字{通}}}{とほ}りの
\ruby{事}{こと}だ。
%
だから
\ruby[g]{意見}{い けん}と
\ruby{思}{おも}つて
\ruby{聞}{き}いて
\ruby{吳}{く}れちやあ
\ruby{困}{こま}るが、
%
たつた
\ruby{一}{ひと}つ
\ruby{君}{きみ}に
\ruby{聞}{き}いて
\ruby{置}{お}いて
\ruby{貰}{もら}ひたい
\ruby{事}{こと}がある。
%
\原本頁{107-6}\改行%
\ruby{下}{くだ}らない
\ruby{事}{こと}では
\ruby{有}{あ}らうが、
%
\ruby{聞}{き}いて
\ruby{吳}{く}れたまへ。
%
\ruby{僕}{ぼく}は
\ruby[g]{隨{\換字{分}}}{ずゐぶん}
\ruby{今}{いま}までの
\ruby[g]{品行}{み もち}が、
%
\ruby[g]{疵瑕}{き ず }だらけの
\ruby{大馬鹿}{おほ|ば|か}な
\ruby{奴}{やつ}なんだから、
%
\ruby[g]{當世}{たうせい}で
よく
\ruby{云}{い}ふ
\ruby[g]{神聖}{しんせい}な
\ruby[g]{戀愛}{れんあい}、%
{---}{---}%
そんな
\ruby[||j>]{上}{じやう}
\ruby[||j>]{品}{ ひん}なものあ
% \ruby{上品}{じやう|ひん}なものあ
\ruby{知}{し}らないが、
%
\ruby[g]{戀愛}{れんあい}も
\ruby{惚}{ほ}
\原本頁{107-9}\改行%
れた
はれたも
\ruby{同}{おな}じ
\ruby{事}{こと}として、
%
マア
\ruby{僕}{ぼく}だけで
\ruby{云}{い}つて
\ruby{見}{み}りやあ、
%
\ruby[g]{戀愛}{れんあい}は
\ruby[g]{可怖}{こ は }いものぢやあ
\ruby{無}{な}いが、
%
\ruby[g]{戀愛}{れんあい}に
\ruby{隨}{つ}いて
\ruby{來}{く}る
\ruby{隨{\換字{伴}}者}{お|と|も}は
\ruby{怖}{こは}い
\改行% 校正作業の簡略化のため
、
%
\原本頁{107-11}\改行%
と
つく〴〵
\ruby{身}{み}に
\ruby{染}{し}みて
\ruby{覺}{おぼ}えて
\ruby{居}{ゐ}るんだ。
%
そこで
\ruby{君}{きみ}に
\ruby{其}{そ}の
\ruby{隨{\換字{伴}}者}{お|と|も}
\原本頁{108-1}\改行%
だけにやあ
\ruby[g]{戒愼}{ようじん}して
\ruby{貰}{もら}ひたいと
\ruby{思}{おも}ふ。
%
\ruby{云}{い}つて
\ruby{置}{お}きたいと
\ruby{云}{い}ふのは
\ruby{只}{たゞ}これ
\ruby{一}{ひと}つだ。
%
いゝカエ、
%
\ruby{惚}{ほ}れた
はれたの
\ruby{其}{そ}の
\ruby{{\換字{迷}}}{まよ}ひは、
%
\ruby{些}{ちつと}も
\原本頁{108-3}\改行%
\ruby[g]{可怖}{こ は }い
\ruby{事}{こと}は
\ruby{無}{な}いが、
%
それに
\ruby{付}{つ}いて
\ruby{來}{く}る
\ruby{隨{\換字{伴}}者}{お|と|も}は
\ruby{怖}{こは}い
\ruby[g]{危險}{あぶない}ものだといふのだよ。
』% 原本では島木の語りの終わりである「』」は欠落している

\Entry{其十八}

\原本頁{}
\ruby{僕}{ぼく}は
\ruby{元}{もと}から
\ruby{學問}{がく|もん}は
\ruby{{\換字{嫌}}}{きら}ひだし、
%
\ruby{身}{み}に
\ruby{浸}{し}みて
\ruby{書}{ほん}を
\ruby{讀}{よん}んだ
\ruby{事}{こと}も
\ruby{無}{な}いから、
%
どうせ
\ruby{僕}{ぼく}の
\ruby{云}{い}ふ
\ruby{事}{こと}なぞは
\ruby{下}{くだ}ら
\ruby{無}{な}からうが、
%
まんざら
\ruby{正中}{つ|ぼ}に
\ruby{外}{はづ}れたことも
\ruby{云}{い}は
\ruby{無}{な}いつもりだ。
%
かういふ
\ruby{理屈}{り|くつ}だ、
%
\ruby{聞}{き}いて
\ruby{吳}{く}れたまへ。
%
\ruby{僕}{ぼく}に
\ruby{云}{い}はせりやあ
\ruby{色戀}{いろ|こひ}といふ
\ruby{奴}{やつ}あ、
%
\ruby{人間}{にん|げん}が
\ruby{一人並}{いち|にん|なみ}に
\ruby{成熟}{でき|あが}ると、
%
\ruby{一度}{いち|ど}は
\ruby{屹度}{きつ|と}
\ruby{發}{はつ}しる
\ruby{熱病}{ね|つ}なので、
%
\ruby{身體}{から|だ}の
\ruby{中}{なか}から
\ruby{自然}{ひと|りで}に
\ruby{湧}{わ}く
\ruby{奴}{やつ}だ、
%
\ruby{各自}{めい|〳〵}の
\ruby{料簡}{れう|けん}から
\ruby{出}{で}て
\ruby{來}{く}るんぢやあ
\ruby{無}{な}い。
%
そりやあ
\ruby{其}{そ}の
\ruby{當人}{たう|にん}から
\ruby{云}{い}つて
\ruby{見}{み}りやあ、
%
\ruby{彼處}{あそ|こ}が
\ruby{好}{い}いとか、
%
\ruby{此處}{こ|ゝ}が
\ruby{好}{い}いとか、
%
それ〴〵に
\ruby{理由}{わ|け}が
\ruby{有}{あ}つて
\ruby{惚}{ほ}れるのでも
\ruby{有}{あ}らうが、
%
ナアニ
\ruby{年齡}{と|し}が
\ruby{爲}{さ}せるんだよ、
%
\ruby{年齡}{と|し}が
\ruby{爲}{さ}せるんだよ。
%
\ruby{彼}{あ}の
\ruby{女}{をんな}あ
\ruby{好}{い}いからサア
\ruby{惚}{ほ}れて
\ruby{{\換字{遣}}}{や}らうと、
%
\ruby{{\換字{分}}別}{ふん|べつ}をつけてから
\ruby{惚}{ほ}れる
\ruby{奴}{やつ}は
\ruby{無}{な}い。
%
\ruby{誰}{たれ}の
\ruby{戀路}{こひ|ぢ}も
\ruby{同}{おんな}じ
\ruby{事}{こと}で、
%
\ruby{其}{そ}の
\ruby{眞實}{ほん|たう}のところを
\ruby{云}{い}やあ、
%
\ruby{自{\換字{分}}}{じ|ぶん}にも
\ruby{理由}{わ|け}は
\ruby{{\換字{分}}}{わか}らないけれど、
%
\ruby{何}{なん}だか
\ruby{知}{し}ら
\ruby{無}{な}いが
\ruby{自然}{ひと|りで}に
\ruby{好}{す}く、
%
それが
\ruby{抑々}{そも|〳〵}の
\ruby{發端}{はじ|まり}で、
%
\ruby{其}{そ}の
\ruby{人}{ひと}の
\ruby{笑顏}{ゑ|がほ}なんぞが
\ruby{何時}{い|つ}の
\ruby{間}{ま}にか
\ruby{眼}{め}に
\ruby{染}{し}み
\ruby{付}{つ}いて
\ruby{{\換字{遺}}}{のこ}つたり、
%
\ruby{物}{もの}を
\ruby{云}{い}つた
\ruby{聲}{こゑ}の
\ruby{色}{いろ}が
\ruby{耳}{みゝ}に
\ruby{{\換字{遺}}}{のこ}つたりして、
%
\ruby{{\換字{終}}}{しまひ}にはすつかり
\ruby{其人}{その|ひと}が
\ruby{自{\換字{分}}}{じ|ぶん}の
\ruby{胸}{むね}の
\ruby{中}{うち}に
\ruby{在}{あ}るやうになる、
%
サア
\ruby{忘}{わす}れやうと
\ruby{思}{おも}つても
\ruby{忘}{わす}れられない、
%
\ruby[g]{始{\換字{終}}}{しじゆう}
\ruby{其人}{その|ひと}の
\ruby{傍}{そば}に
\ruby{居}{ゐ}て
\ruby{見}{み}たくなる、
%
\ruby{離}{はな}れて
\ruby{居}{ゐ}ちやあ
\ruby{物悲}{もの|がな}しくつて、
%
\ruby{何}{なん}と
\ruby{無}{な}く
\ruby{氣}{き}が
\ruby{濟}{す}まないやうな
\ruby{心持}{こゝろ|もち}ちがする、
%
\ruby{自{\換字{分}}}{じ|ぶん}が
\ruby{其人}{その|ひと}を
\ruby{思}{おも}ふやうに、
%
\ruby{其人}{その|ひと}にも
\ruby{自{\換字{分}}}{じ|ぶん}を
\ruby{思}{おも}つて
\ruby{貰}{もら}ひたくなる、
%
それから
\ruby{段々}{だん|〴〵}と
\ruby{泣}{な}いたり
\ruby{笑}{わら}つたりが
\ruby{始}{はじ}まる、
%
まあ
\ruby{斯樣}{か|う}
\ruby{云}{い}つた
\ruby{順立}{じゆん|だて}ぢやあ
\ruby{無}{な}いか。
%
\換字{志}て
\ruby{見}{み}りやあ
\ruby{自然}{ひと|りで}に
\ruby{好}{す}くといふのが
\ruby{戀}{こひ}の
\ruby{水上}{みな|かみ}だが、
%
\ruby{自然}{ひと|りで}の
\ruby{好惡}{すき|きらひ}だもの、
%
\ruby{理屈}{り|くつ}は
\ruby{有}{あ}りや
\ruby{仕無}{し|な}い、
%
みんな
\ruby{年齡}{と|し}が
\ruby{爲}{さ}せるんだ。
%
\ruby{懷姙者}{み|も|ち}は
\ruby{酸}{す}いものを
\ruby{自然}{ひと|りで}に
\ruby{好}{す}く、
%
\ruby{溜飮}{りう|いん}
\ruby{持}{もち}は
\ruby{香物}{かう|〳〵}で
\ruby{茶濱飯}{ちや|づ|け}を
\ruby{自然}{ひと|りで}に
\ruby{好}{す}く、
%
\ruby{其}{そ}の
\ruby{自然}{ひと|りで}に
\ruby{好}{す}くのは
\ruby{誰}{たれ}がさせる?、
%
\ruby{惡阻}{つは|り}が
\ruby{爲}{さ}せるんだ、
%
\ruby{溜飮}{りう|いん}が
\ruby{爲}{さ}せるんだ、
%
\ruby{戀路}{こひ|ぢ}の
\ruby{{\換字{迷}}惑}{まよ|ひ}は
\ruby{年齡}{と|し}が
\ruby{爲}{さ}せるんだ。
%
\ruby{男兒}{をと|こ}が
\ruby{男兒}{をと|こ}づくる
\ruby{頃}{ころ}にやあ
\ruby{髭鬚}{ひ|げ}が
\ruby{生}{は}えて
\ruby{來}{く}る、
%
\ruby{髭鬚}{ひ|げ}の
\ruby{生}{は}えるのは
\ruby{年齡}{と|し}が
\ruby{爲}{さ}せるんだもの、
%
それに
\ruby{善}{よ}いも
\ruby{惡}{わる}いも
\ruby{有}{あ}りやうは
\ruby{無}{な}い、
%
\ruby{口}{くち}の
\ruby{周圍}{まは|り}に
\ruby{出}{で}て
\ruby{來}{く}る
\ruby{髭鬚}{ひ|げ}も、
%
\ruby{心}{こゝろ}の
\ruby{上}{うへ}に
\ruby{萌}{めぐ}む
\ruby{戀}{こひ}も、
%
\ruby{年端}{と|し}が
\ruby{爲}{さ}せるに
\ruby{差異}{ちが|ひ}は
\ruby{無}{な}い、
%
\ruby{丁度}{ちやう|ど}
\ruby{同}{おんな}じ
\ruby{事}{こと}だもの、
%
ナニ
\ruby{戀愛}{こ|ひ}を
\ruby{善}{い}いとも
\ruby{惡}{わる}いとも
\ruby{云}{い}はう
\ruby{譯}{わけ}は
\ruby{無}{な}い。
%
たゞ
\ruby{年齡}{と|し}が
\ruby{爲}{さ}せる
\ruby{熱病}{ね|つ}をすらりと
\ruby{濟}{すま}せて
\ruby{仕舞}{し|ま}へば、
%
\ruby{疱瘡}{はう|さう}や
\ruby{痲疹}{はし|か}が
\ruby{濟}{す}んだと
\ruby{同}{おな}じに、
%
つまり
\ruby{芽出度}{め|で|たい}と
\ruby{云}{い}へば
\ruby{云}{い}へるので、
%
\ruby{戀}{こひ}は
\ruby{怖}{おそ}ろしいものでも
\ruby{何}{なん}でも
\ruby{無}{な}い。
%
\ruby{併}{しか}し
\ruby{{\換字{又}}}{また}、
%
\ruby{君}{きみ}は
\ruby{學問}{がく|もん}もあり
\ruby{思慮}{し|りよ}もあるから、
%
\ruby{萬々}{ばん|〳〵}
\ruby{承知}{しよう|ち}
\ruby{仕}{し}て
\ruby{居}{ゐ}やうが、
%
お
\ruby{互}{たがひ}いに
\ruby{男兒}{をと|こ}といふ
\ruby{奴}{やつ}は、
%
\ruby{戀愛}{こ|ひ}の
\ruby{奴隷}{け|らい}に
\ruby{生}{う}まれて
\ruby{居}{ゐ}るものでも
\ruby{何}{なん}でも
\ruby{無}{な}い、
%
それ〴〵
\ruby{男子}{をと|こ}
\ruby{一匹}{いつ|ぴき}
\ruby{{\換字{前}}}{まへ}の
\ruby{目的}{もく|てき}のために
\ruby{意氣}{い|き}
\ruby{地}{ぢ}を
\ruby{磨}{みが}いて
\ruby{一生}{いつ|しやう}を
\ruby{働}{はたら}いて
\ruby{行}{ゆ}かうといふ
\ruby{身}{み}、
%
\ruby{戀}{こひ}に
\ruby{捲}{ま}き
\ruby{倒}{たふ}されちやあならねえ
\ruby{身體}{から|だ}だ、
%
\ruby{其}{そ}の
\ruby{熱病}{ね|つ}に
\ruby{身體}{から|だ}を
\ruby{{\換字{遣}}}{や}る
\ruby{譯}{わけ}にやあいかねえ
\ruby{約束}{やく|そく}がある。
%
\ruby{病}{やまひ}にも
\ruby{輕}{かる}い
\ruby{重}{おも}いはあり、
%
\ruby{戀}{こひ}にも
\ruby{深}{ふか}い
\ruby{淺}{あさ}いは
\ruby{有}{あ}らうが、
%
\ruby{如何}{い|か}に
\ruby{戀}{こひ}に
\ruby{惱}{なや}んでも
\ruby{苦}{くる}しんでも、
%
\ruby{吐}{つ}く
\ruby{息}{いき}が
\ruby{火}{ひ}になつて
\ruby{燃}{も}えるほどに
\ruby{狂}{くる}はうとも、
%
\ruby{戀}{こひ}に
\ruby{負}{ま}けて
\ruby{死}{し}んぢやあ
\ruby{男子}{をと|こ}たる
\ruby{身}{み}の、
%
\ruby{眼}{め}が
\ruby{瞑}{ふさ}げねえ
\ruby{筈}{はず}だ。
%
イヤ
\ruby{瞑}{ふさ}げねえ、
%
どうしても
\ruby{死}{しに}きれねえ、
%
\ruby{死}{し}ね
\ruby{無}{ね}え
\ruby{筈}{はず}だ。
%
\ruby{乃公}{お|ら}あ
\ruby{死}{し}な
\ruby{無}{ね}え、
%
\ruby{死}{し}にも
\ruby{仕無}{し|ね}えが、
%
\ruby{汝}{おめへ}も
\ruby{死}{し}ねめえ、
%
\ruby{死}{し}にもすめえナ。
%
\ruby{知}{し}れ
\ruby{切}{き}つた
\ruby{事}{こと}だが、
%
ナア
\ruby[g]{水野}{みづの}、
%
お
\ruby{互}{たがひ}いに
\ruby{幾干{\換字{若}}干}{いく|そ|ばく|そ}の
\ruby{苦勞}{く|らう}を
\ruby{仕}{し}て、
%
\ruby{今日}{け|ふ}まで
\ruby{{\換字{遣}}}{や}つて
\ruby{來}{き}たなあ
\ruby{何}{なん}の
\ruby{爲}{ため}だ?。
%
\ruby{志}{こゝろざし}こそ
\ruby{異}{ちが}ふけれど、
%
\ruby{男兒}{をと|こ}と
\ruby{生}{うま}れた
\ruby{生}{うま}れ
\ruby{甲{\換字{斐}}}{が|ひ}にやあ、
%
\ruby{各自}{めい|〳〵}の
\ruby{念願}{おも|ひ}を
\ruby{{\換字{遂}}}{と}げやうと、
%
そればつかりの
\ruby{爲}{ため}ぢやあ
\ruby{無}{ね}えか。
%
\ruby{特}{こと}さら
\ruby{汝}{おめへ}は
\ruby{乃公}{お|れ}から
\ruby{云}{い}やあ、
%
マア
\ruby{慾}{よく}の
\ruby{無}{な}さすぎる
\ruby{偏人}{へん|じん}で、
%
\ruby{取}{と}れる
\ruby{錢}{ぜに}も
\ruby{取}{と}らず
\ruby{出世}{しゆつ|せ}も
\ruby{望}{のぞ}まず、
%
\ruby{大根}{だい|こん}
\ruby{人參}{にん|じん}の
\ruby{尻尾}{しつ|ぽ}を
\ruby{咬}{かじ}つて、
%
それで
\ruby{濟}{す}まして
\ruby{居}{ゐ}るやうな
\ruby{{\換字{遣}}}{や}り
\ruby{方}{かた}。
%
アヽ
\ruby{世}{よ}の
\ruby{中}{なか}はいろ〳〵のもんだ、
%
\ruby[g]{水野}{みづの}だつて
\ruby{不味}{ま|づ}いものあ
\ruby{不味}{ま|づ}く、
%
\ruby{美味}{う|ま}いものは
\ruby{旨}{うま}からうが、
%
\ruby{其}{それ}にも
\ruby{此}{これ}にも
\ruby{頓着無}{とん|ぢやく|な}く、
%
\ruby{{\換字{若}}}{わか}い
\ruby{身天}{み|そら}で
\ruby{色氣}{いろ|け}も
\ruby{無}{な}く、
%
\ruby{下手}{へ|た}な
\ruby{律僧}{りつ|そう}は
\ruby{及}{およ}ばぬ
\ruby{身持}{み|もち}で、
%
たゞ
\ruby{學問}{がく|もん}に
\ruby{凝}{こ}つて
\ruby{居}{ゐ}る、
%
アヽ
\ruby{聖人}{せい|じん}と
\ruby{云}{い}ふなあ
\ruby{彼樣}{あ|ん}な
\ruby{男}{をとこ}の
\ruby{事}{こと}か
\ruby{知}{し}らん、
%
\ruby{餘{\換字{所}}目}{よ|そ|め}から
\ruby{見}{み}ては
\ruby{氣}{き}が
\ruby{竭}{つ}きて、
%
\ruby{何}{なん}だか
\ruby{憫然}{かあ|いさう}なやうな% 「憫然 か(あ)いさう」
\ruby{氣}{き}がすると、
%
\ruby{思}{おも}つた
\ruby{位}{くらゐ}に
\ruby{月日}{つき|ひ}を
\ruby{經}{へ}て
\ruby{來}{き}た、
%
\ruby{其}{そ}の
\ruby{汝}{おめへ}の
\ruby{難行苦行}{なん|ぎやう|く|ぎやう}も
\ruby{何}{なん}の
\ruby{爲}{ため}だ。
%
やつぱり
\ruby{何時}{い|つ}か
\ruby{一度}{いち|ど}は
\ruby{汝}{おめへ}は
\ruby{汝}{おめへ}で、
%
\ruby{男兒}{をと|こ}
\ruby{甲{\換字{斐}}}{が|ひ}のある
\ruby{仕事}{し|ごと}を
\ruby{仕}{し}やうためばかりの
\ruby{事}{こと}ぢやあ
\ruby{無}{な}いか。
%
その
\ruby{木食坊主}{もく|じき|ばう|ず}かなんぞのやうな、
%
\ruby{味}{あぢ}の
\ruby{無}{な}い
\ruby{長}{なが}い
\ruby{月日}{つき|ひ}の
\ruby{生活}{くら|し}さへも、
%
\ruby{笑}{わら}つて
\ruby{仕}{し}て
\ruby{來}{き}た
\ruby{汝}{おめへ}だもの、
%
\ruby{何樣}{ど|ん}な
\ruby{苦}{くる}しい
\ruby{戀}{こひ}に
\ruby{落}{お}ちても、
%
よもや
\ruby{本心}{ほん|しん}を
\ruby{失}{うしな}つて、
%
\ruby{熱病}{ね|つ}に
\ruby{負}{ま}けて
\ruby{仕舞}{し|ま}ふやうなことは
\ruby{有}{あ}るめえが、
%
さあ、
%
\ruby{戀愛}{こ|ひ}は
\ruby{怖}{こは}かあ
\ruby{無}{ね}えが
\ruby{隨{\換字{伴}}者}{お|と|も}が
\ruby{怖}{こは}い、
%
\ruby{案}{あん}じられてならねえところが
\ruby{其處}{そ|こ}にある!。

\Entry{其十九}

% メモ 校正 2024-04-07
\原本頁{114-2}%
『
\ruby{隨{\換字{伴}}者}{お|と|も}と
\ruby{云}{い}ふなあ
\ruby{他}{ほか}ぢやあ
\ruby{無}{ね}えが、
%
\ruby{戀}{こひ}に
\ruby{隨}{つ}いて
\ruby{來}{く}る
\ruby[g]{心氣}{しん}の
\ruby{疲勞}{つか|れ}だ。
%
お
\ruby{互}{たがひ}に
\ruby{覺}{おぼ}えのある
\ruby{事}{こと}だが、
%
\ruby{男}{をとこ}の
\ruby{兒}{こ}といふ
\ruby{奴}{やつ}あ
\ruby{十三四}{じう|さん|し}から、
%
\原本頁{114-4}\改行%%
そろ〳〵
\ruby{野心}{や|しん}が
\ruby{燃}{も}えたつて
\ruby{來}{き}て、
%
\ruby{威張}{ゐ|ば}つて
\ruby{見}{み}たい、
%
\ruby{人}{ひと}に
\ruby{{\換字{勝}}}{か}ちたい、
%
\ruby{功}{てがら}が
\ruby{立}{た}てたい、
%
\ruby{名}{な}が
\ruby{立}{た}てたい、
%
\ruby{天下}{てん|か}が
\ruby{取}{と}りたい、
%
と
\ruby{氣象}{き|しやう}
\ruby{相應}{さう|おう}の
\ruby{望}{のぞ}みを
\ruby{起}{おこ}すが、
%
それでも
\ruby{其}{そ}の
\ruby{時{\換字{分}}}{じ|ぶん}の
\ruby{腹中}{はらん|なか}は
\ruby[g]{淸潔}{きれい}なもので、
%
たゞ
\ruby[g]{醇醉}{いつぽんぎ}の
\ruby[g]{大望心}{たいまう}が
あるばかり、
%
\ruby{乃公}{お|ら}あ
\ruby{太閤}{たい|かふ}だぞ、
%
\ruby[g]{拿破崙}{なぽれおん}だぞと、
%
\原本頁{114-8}\改行%
\ruby[g]{各自}{てん〴〵}に
\ruby{力}{りき}む
\ruby{其勢}{その|いきほひ}で、
%
\ruby{伸}{の}びも
\ruby{育}{そだ}ちも
\ruby{仕}{し}て
\ruby{來}{く}るが、
%
\ruby{遲}{おそ}かれ
\ruby{{\換字{速}}}{はや}かれ
\ruby{時{\換字{節}}}{と|き}が
\ruby{來}{き}て、
%
\ruby{戀}{こひ}という
\ruby{奴}{やつ}に
\ruby{魅入}{み|い}られちやあ、
%
さあ
\ruby{腹}{はら}の
\ruby{中}{なか}が
\ruby{揉}{も}めて
\原本頁{114-10}\改行%
\ruby{來}{く}る。
%
\ruby[g]{大望心}{たいまう}は
\ruby[g]{大望心}{たいまう}で
\ruby{居}{ゐ}しかつて
\ruby{居}{ゐ}る、
%
\ruby{戀}{こひ}の
\ruby{心}{こゝろ}は
\ruby{戀}{こひ}の
\ruby{心}{こゝろ}で
\ruby[g]{自由}{まま}に
\ruby{働}{はたら}く。
%
\原本頁{115-1}%
\ruby{双方}{さう|はう}が
\ruby{頭}{かしら}は
\ruby{下}{さ}げないから、
%
\ruby[g]{衝突}{ぶつか}りやあ
\ruby{何樣}{ど|う}しても
\ruby{忽}{たちま}ち
\ruby[g]{戰爭}{たゝかひ}で、
%
\ruby{那方}{どつ|ち}が
\ruby{{\換字{勝}}}{か}つにしても
\ruby{負}{ま}けるにしても、
%
なか〳〵
\ruby{樂}{らく}な
\原本頁{115-3}\改行%
\ruby{爭鬩}{せり|あひ}ぢやあ
\ruby{無}{な}い。
%
\ruby{戀}{こひ}が
\ruby{負}{ま}けて
\ruby{倒}{たふ}れりやあ
\ruby{其}{そ}の
\ruby{傷口}{きず|ぐち}から、
%
\ruby{溢}{こぼ}れる
\原本頁{115-4}\改行%
\ruby{血潮}{ち|しほ}が
\ruby{急}{きふ}にやあ
\ruby{止}{と}まらず、
%
\ruby[g]{大望心}{たいまう}が
\ruby{負}{ま}けりやあ
\ruby{其}{そ}の
\ruby{英氣}{えい|き}は、
%
\ruby{未練氣}{み|れん|げ}
\ruby{無}{な}く
\ruby{去}{さ}つて
\ruby{仕舞}{し|ま}つて
\ruby{呼}{よ}んでも
\ruby{{\換字{還}}}{かへ}らねえ。
%
つまり
\ruby{何樣}{ど|う}なつても
\ruby{根}{ね}が
\ruby{同士討}{どう|し|うち}の、
%
\ruby{酷}{ひど}い
\ruby[g]{戰爭}{たゝかひ}に
\ruby[g]{國土}{くに}は
\ruby{荒}{あ}れて、
%
\ruby{{\換字{遺}}}{のこ}るものは
\ruby{怖}{おそ}ろしい
\ruby[g]{心氣}{しん}の
\ruby{疲勞}{つか|れ}!。
%
\ruby{櫻色}{さくら|いろ}して
\ruby{居}{ゐ}た
\ruby{面}{かほ}は
\ruby{白}{しら}けて、
%
\ruby{葛}{くず}の
\ruby{葉裏}{は|うら}を
\ruby{見}{み}るやうになり、
%
\ruby{眼}{め}は
\ruby{冴}{さ}えなくなる、
%
\ruby{白髮}{しら|が}は
さす、
%
\ruby{{\換字{強}}}{つよ}い
\ruby{奴}{やつ}は
\ruby{癇癪持}{かん|しやく|もち}になる。
%
\ruby{{\換字{弱}}}{よわ}い
\ruby{奴}{やつ}は
\ruby{萎縮{\換字{漢}}}{いぢ|け|もの}になる。
%
\ruby{筋骨}{すぢ|ぼね}は
\ruby{弛}{ゆる}んで
\ruby{仕舞}{し|ま}ふ、
%
\ruby{勞苦{\換字{嫌}}}{ほね|をり|ぎら}ひになる。
%
\ruby{其}{そ}の
\ruby{位}{くらゐ}なのは
\ruby{未}{ま}だ
\ruby{可}{い}い
\ruby{{\換字{分}}}{ぶん}で、
%
\ruby{隨{\換字{分}}}{ずゐ|ぶん}
\ruby{怖}{おそ}ろしい
\ruby{病氣}{びやう|き}さへも
\原本頁{115-11}\改行%
\ruby{引出}{ひき|だ}す。
%
よしんば
\ruby[g]{大望心}{たいまう}と
\ruby{戀愛}{こ|ひ}とが
\ruby[g]{衝突}{ぶつか}らないで、
%
\ruby{腹}{はら}の
\ruby{中}{なか}が
それほどには
\ruby{揉}{も}め
\ruby{無}{な}いでも、
%
\ruby{向}{むこ}ふに
\ruby{的}{まと}の
\ruby{無}{な}い
\ruby{戀}{こひ}は
\ruby{無}{な}いから、
%
\ruby{星}{ほし}に
\原本頁{116-2}\改行%
\ruby{中}{あた}る
\ruby{中}{あた}らぬは
\ruby{時}{とき}の
\ruby{{\換字{運}}}{うん}
\ruby{身}{み}の
\ruby{{\換字{運}}}{うん}!。
%
\ruby{相手}{あひ|て}と
\ruby{馬}{うま}が
\ruby{合}{あ}ふ
\ruby{合}{あ}はぬもあるし、
%
\原本頁{116-3}\改行%
\ruby{相手}{あひ|て}とは
\ruby{死}{し}ぬほどに
\ruby{好}{す}き
\ruby{合}{あ}つても、
%
\ruby{自{\換字{分}}}{じ|ぶん}たち
ばかりのために
\ruby{出來}{で|き}て
\ruby{居}{ゐ}る
\ruby{世界}{せ|かい}ぢやあ
\ruby{無}{な}いもの、
%
\ruby{何}{なに}がさて
\ruby{外{\換字{道}}}{げ|だう}も
\ruby{居}{ゐ}る、
%
\ruby{惡{\換字{魔}}}{あく|ま}も
\ruby{居}{ゐ}る、
%
\ruby{敵}{てき}も
\ruby{居}{ゐ}る、
%
おせつかいも
\ruby{居}{ゐ}る、
%
\ruby{義理}{ぎ|り}もある、
%
\ruby{人{\換字{情}}}{にん|じやう}もある、
%
\ruby{時}{とき}もある、
%
\ruby{場合}{ば|あひ}もあつて、% 原文通り「場」
%
\ruby[g]{隨意}{まゝ}ならぬ
\ruby{憂}{う}き
\ruby{世}{よ}を
\ruby{泣}{な}くものが
\ruby{多}{おほ}い。
%
\原本頁{116-7}\改行%
\ruby{左樣}{さ|う}で
\ruby{無}{な}くつてさへ
\ruby{戀}{こひ}を
\ruby{知}{し}るなあ
\ruby{涙}{なみだ}を
\ruby{知}{し}る
\ruby{始}{はじめ}で、
%
\ruby{氣}{き}が
\ruby{優}{やさ}しくなる、
%
\ruby{脆}{もろ}くなる、
%
\ruby{感}{かん}じが
\ruby{早}{はや}くなる、
%
\ruby{深}{ふか}くなる、
%
\ruby{何}{なん}でも
\ruby{無}{な}い
\ruby{事}{こと}に
ハツと
\ruby{思}{おも}つたり、
%
\ruby{小}{ちひさ}な
\ruby{事}{こと}を
くよ〳〵と
\ruby{案}{あん}じたり、
%
\ruby{{\換字{前}}表}{ぜん|ぺう}といふやうな
\ruby{事}{こと}を
\ruby{氣}{き}にしたり、
%
\ruby{何}{なに}かにつけて
\ruby{思}{おも}ひ
\ruby{{\換字{過}}}{すご}しを
\ruby{仕}{し}たり、
%
\ruby{寢}{ね}るべき
\原本頁{116-11}\改行%
\ruby{時}{とき}に
\ruby{寢}{ね}られなかつたりする。
%
そこで
\ruby{段々}{だん|〴〵}と
\ruby[g]{心氣}{しん}が
\ruby{{\換字{弱}}}{よわ}る。
%
\ruby[g]{心氣}{しん}が
\原本頁{117-1}\改行%
\ruby{{\換字{弱}}}{よわ}りやあ
\ruby{愈々}{いよ|〳〵}
\ruby{氣}{き}が
\ruby{脆}{もろ}くなる、
%
\ruby{感}{かん}じが
\ruby{{\換字{強}}}{つよ}くなる。
%
\ruby{氣}{き}が
\ruby{脆}{もろ}く、
%
\ruby{感}{かん}じが
\ruby{{\換字{強}}}{つよ}くなりやあ
\ruby{{\換字{又}}}{また}
\ruby[g]{心氣}{しん}が
\ruby{{\換字{弱}}}{よわ}る。
%
\ruby{雁齒鑢}{がん|ぎ|やすり}が
かゝるやうなものだから
\ruby{堪}{たま}らう
\ruby{譯}{わけ}は
\ruby{無}{な}い。
%
\ruby{一日}{いち|にち}
\ruby{一日}{いち|にち}に
\ruby{{\換字{弱}}}{よわ}つた
\ruby{擧句}{あげ|く}は、
%
\ruby{魂魄}{たま|しひ}が
\ruby{薄手}{うす|で}に
なりきつて、
%
\ruby{觸}{さは}るものさへ
あれば
\ruby{砕}{くだ}けたがる
\ruby[g]{玻璃}{びいどろ}か
なんぞのやうになつて
\ruby{仕舞}{し|ま}ふ。
%
よく
\ruby{世間}{せ|けん}にある
\ruby{戀路}{こひ|ぢ}の
\ruby{果}{はて}の、
%
\ruby{飛}{と}んでも
\ruby{無}{な}い
\ruby{不幸福}{ふ|しあ|はせ}は%「幸福」 ここは「は」
\ruby{皆}{みな}
\ruby{其處}{そ|こ}で
\ruby{出來}{で|き}る。
%
たとひ
\ruby{{\換字{嫌}}}{きら}はれても
\ruby{{\換字{嫌}}}{きら}はれても、
%
\ruby{好}{す}かれたいのが
\ruby{戀}{こひ}の
\ruby{慾}{よく}で、
%
\ruby{{\換字{又}}}{また}
\ruby{憂}{う}いも
\ruby{辛}{つら}いも
\ruby{堪{\換字{忍}}}{しん|ばう}して% 原文通り「堪忍」
\ruby{添}{そ}ひ
\ruby{{\換字{遂}}}{と}げたいのが
\原本頁{117-8}\改行%
\ruby{戀}{こひ}の
\ruby{意地}{い|ぢ}だ。
%
\換字{志}て
\ruby{見}{み}りやあ
\ruby{戀}{こひ}に
\ruby{生命}{いの|ち}の
\ruby{捨}{す}てやうは
\ruby{無}{な}い、
%
\ruby{戀}{こひ}は
\ruby{生々}{いき|いき}と
\ruby{美}{うつく}しいものだ。
%
\ruby{世}{よ}の
\ruby{不幸福}{ふ|しあ|せ}な%「幸福」 ここは「は」欠落
\ruby{人}{ひと}を
\ruby{見}{み}りやあ、
%
\ruby{戀}{こひ}で
\ruby{死}{し}ぬものは
\ruby{一人}{ひと|り}も
\ruby{無}{な}く、
%
\ruby[<j|]{皆}{みんな}
\ruby[g]{心氣}{しん}の
\ruby{疲勞}{つか|れ}に
\ruby{堪}{こら}へ
\ruby{切}{き}れ
\ruby{無}{な}くなつて、
%
おのが
\ruby{魂魄}{たま|しひ}を
\ruby{碎}{くだ}いて
\ruby{仕舞}{し|ま}うのだが、
%
\ruby{{\換字{避}}}{さ}けやうにも
\ruby{{\換字{避}}}{さ}け
\ruby{{\換字{難}}}{にく}いのは
\ruby{此}{こ}の
\ruby{隨{\換字{伴}}者}{お|と|も}だから、
%
\ruby{戀}{こひ}は
\ruby{毫末}{ちつ|と}も
\ruby{怖}{こは}かあ
\ruby{無}{な}いが、
%
\ruby{其}{そ}の
\ruby{隨{\換字{伴}}者}{お|と|も}の
\ruby[g]{心氣}{しん}の
\ruby{疲勞}{つか|れ}は
\ruby{恐}{おそ}ろしい。
%
\ruby{實}{じつ}を
\ruby{云}{い}やあ
\ruby{僕}{ぼく}が
\ruby{君}{きみ}の
\ruby{事}{こと}を
\ruby{素破拔}{すつ|ぱ|ぬ}いて
\ruby{饒舌}{しや|べ}つたから、
%
\原本頁{118-3}\改行%
\ruby[g]{羽{\換字{勝}}}{はがち}も
\ruby[g]{日方}{ひかた}も
\ruby{君}{きみ}のために、
%
\ruby{二人}{ふた|り}とも
\ruby{甚}{ひど}く
\ruby{心配}{しん|ぱい}して
\ruby{居}{ゐ}る。
%
\ruby{特}{こと}に
\ruby[g]{日方}{ひかた}は
\ruby{彼}{あ}の
\ruby{氣性}{き|しやう}だから、
%
\ruby{{\換字{強}}}{きつ}い
\ruby{意見}{い|けん}を
\ruby{云}{い}ひに
\ruby{行}{ゆ}かうかも
\ruby{知}{し}れないが、
%
\原本頁{118-5}\改行%
\ruby{乃公}{お|ら}あ
\ruby{何}{なんに}も
\ruby{意見}{い|けん}は
\ruby{云}{い}はない。
%
\ruby{何}{なん}も
\ruby{彼}{か}も
\ruby{解}{わか}つて
\ruby{居}{ゐ}る
\ruby{君}{きみ}の
\ruby{事}{こと}だもの、
%
\原本頁{118-6}\改行%
\ruby{君}{きみ}が
\ruby{詰}{つま}ら
\ruby{無}{な}い
\ruby{事}{こと}を
\ruby{仕}{し}やう
\ruby{氣{\換字{遣}}}{き|づか}ひは
\ruby{無}{な}いが、
%
たゞ
\ruby[g]{心氣}{しん}の
\ruby{疲勞}{つか|れ}に
\ruby{負}{ま}けぬやうにと、
%
これだけを
\ruby{君}{きみ}に
\ruby{頼}{たの}んで
\ruby{置}{お}く。
%
\ruby{見}{み}りやあ
\ruby{顏色}{かほ|つき}と
\ruby{云}{い}ひ
\ruby{容態}{よう|す}といひ、
%
\ruby[g]{心氣}{しん}が
\ruby{疲}{つか}れて
\ruby{居}{ゐ}ないやうでも
\ruby{無}{な}い、
%
\ruby{氣}{き}をつけて
\原本頁{118-9}\改行%
\ruby{吳}{く}れ
\ruby{無}{な}くちやあ
いけないぜ。
%
\ruby{何時}{い|つ}かは
\ruby{云}{い}はう〳〵と
\ruby{思}{おも}つて
\ruby{居}{ゐ}たので、
%
つい
\ruby{圖}{づ}に
\ruby{乘}{の}つて
\ruby{長}{なが}く
\ruby{饒舌}{しや|べ}つて、
%
\ruby{言葉}{こと|ば}さへ
\ruby{亂暴}{らん|ばう}に
\ruby{言}{い}ひ
\ruby{{\換字{過}}}{す}ごしたが、
%
\ruby{意}{こゝろ}だけは
\ruby{是非}{ぜ|ひ}とも
\ruby{汲}{く}んで
\ruby{吳}{く}れたまへ。
%
\ruby{千言萬言}{せん|げん|ばん|げん}
\ruby{饒舌}{しや|べ}つても、
%
\ruby{身體}{から|だ}を
\ruby{大切}{たい|せつ}に
\ruby{仕}{し}て
\ruby{吳}{く}れろといふ、
%
たゞの
\ruby{一句}{いつ|く}に
\ruby{止}{とゞ}まるのだ。
%
\ruby{飯}{めし}の
\ruby{不味}{ま|づ}い
\ruby{時}{とき}も
\ruby[g]{堪{\換字{忍}}}{がまん}して% 原文通り「堪忍」
\ruby{食}{く}つて、
%
\ruby{成}{な}るたけ
\ruby{精々}{せい|〴〵}
\ruby{身體}{から|だ}を
\ruby{使}{つか}つて、
%
\ruby{寢}{ね}るべき
\ruby{時}{とき}にやあ
\ruby{整然}{ちや|ん}と
\ruby{寢}{ね}て、
%
\ruby[<j|]{力}{ちから}
\ruby{足}{あし}を
\ruby{踏}{ふ}んで
\ruby{確乎}{しつ|かり}と、
%
\ruby[g]{快活}{きさく}に
\ruby{日}{ひ}を
\ruby{{\換字{送}}}{おく}つて
\ruby{貰}{もら}ひたいのだ。
%
\ruby{君}{きみ}の
\ruby{氣}{き}に
\ruby{入}{い}つたほどの
\ruby{人}{ひと}だもの、
%
\原本頁{119-5}\改行%
\ruby{僕}{ぼく}は
\ruby{其}{そ}の
\ruby{人}{ひと}を
\ruby{知}{し}らないが、
%
\ruby{屹度}{きつ|と}
\ruby{好}{い}い
\ruby{人}{ひと}だらうと
\ruby{思}{おも}つて
\ruby{居}{ゐ}て、
%
\ruby{君}{きみ}の
\ruby{{\換字{運}}命}{う|ん}の
\ruby{好}{い}いやうにと
ばかり
\ruby{祈}{いの}つて
\ruby{居}{ゐ}る。
%
\ruby{僕}{ぼく}の
\ruby{力}{ちから}の
\ruby{要}{い}る
\ruby{事}{こと}が
あらば、
%
\ruby{何}{なん}なりと
\ruby{{\換字{遠}}慮無}{ゑん|りよ|な}く
\ruby{云}{い}つて
\ruby{吳}{く}れたまへ、
%
\ruby{君}{きみ}のために
\ruby{幸福}{しあ|はせ}になる%「幸福」ここは「は」
\ruby{事}{こと}ならば、
%
\ruby{何樣}{ど|ん}な
\ruby{事}{こと}を
\ruby{仕}{し}ても
\ruby{僕}{ぼく}は
\ruby{厭}{いと}はない。
%
\ruby{馬}{うま}にでも
\ruby{牛}{うし}にでもなつて
\ruby{働}{はたら}かうが、
%
\ruby{其}{そ}の
\ruby{代}{かは}り
\ruby{今}{いま}
\ruby{言}{い}つた
\ruby{戀}{こひ}の
\ruby{隨{\換字{伴}}者}{お|と|も}にやあ
\ruby{必}{かなら}ず
\原本頁{119-10}\改行%
\ruby{負}{ま}けて
\ruby{吳}{く}れたまふな。
%
\ruby{世界}{せ|かい}に
\ruby{人間}{ひ|と}は
\ruby{多}{おほ}いけれど、
{---}{---}
そりやあ
\原本頁{119-11}\改行%
\ruby{偉}{えら}い
\ruby{人}{ひと}も
\ruby{多}{おほ}からうが、
%
\ruby{此}{こ}の
\ruby{何年}{なん|ねん}を
\ruby{{\換字{過}}}{すご}して
\ruby{來}{き}た、
%
\ruby{君}{きみ}の
\ruby{行狀}{おこ|なひ}の
\ruby{殊{\換字{勝}}}{しゆ|しよう}さを
\ruby{見}{み}ては、
%
アヽ、
%
\ruby{眞似}{ま|ね}たつて
\ruby{眞似}{ま|ね}られない
\ruby{事}{こと}だ、
%
あゝいふ
\ruby{男}{をとこ}は
\ruby{今}{いま}の
\ruby{世}{よ}には、
%
\ruby{中々}{なか|〳〵}
\ruby{二人}{ふ|たり}とは
\ruby{有}{あ}りはすまい、
%
\ruby[g]{島木}{しまき}
\ruby{萬五郎}{まん|ご|らう}は
\ruby{俗物}{ぞく|ぶつ}だが、
%
\ruby{朋友}{とも|だち}にやあ
\ruby{幸福}{しあ|はせ}にも%「幸福」ここは「は」
\ruby{心}{こゝろ}の
\ruby{氣高}{け|だか}い
\ruby[g]{水野}{みづの}のやうな
\ruby{人}{ひと}を
\ruby{持}{も}つて
\原本頁{120-4}\改行%
\ruby{居}{ゐ}ると、
%
\ruby{天}{てん}にも
\ruby{地}{ち}にも
\ruby[<j|]{唯}{たつた}
\ruby{一人}{ひと|り}の
\ruby{大切}{たい|せつ}な
\ruby{朋友}{とも|だち}に
\ruby{思}{おも}つて
\ruby{居}{ゐ}る
\ruby{君}{きみ}の
\ruby{事}{こと}だから、
%
どうか
\ruby{身體}{から|だ}を
\ruby{大切}{たい|せつ}に
\ruby{仕}{し}て
\ruby{吳}{く}れたまへ、
%
\ruby{君}{きみ}の
\ruby{其}{そ}の
\ruby{顏}{かほ}つきを
\ruby{見}{み}ちやあ
\ruby{案}{あん}じられて
ならない。
%
くどいやうだが
\ruby{今}{いま}
\ruby{言}{い}つた
\ruby{事}{こと}を
\ruby{能}{よ}く
\ruby{聽}{き}いて
\ruby{置}{お}いて
\ruby{吳}{く}れたまへ。
』

\原本頁{120-8}%
と、
%
\ruby{眞{\換字{情}}}{ま|ごゝろ}こめて
\ruby{云}{い}ひ
\ruby{{\換字{終}}}{をは}りたり。

\Entry{其二十}

% メモ 校正終了 2024-04-08
\原本頁{120-10}%
\ruby{磊落}{らい|らく}なれども
\ruby{思{\換字{遣}}}{おもひ|や}りあり、
%
\ruby{粗}{あら}きが
\ruby{如}{ごと}くなれども
\ruby{精細}{こま|か}なるところある
\ruby{島木}{しま|き}が
\ruby{長々}{なが|〳〵}しき
\ruby{物語}{もの|がたり}は、
%
わざと
\ruby{我}{わ}が
\ruby{上}{うへ}には
\ruby{貼}{つ}かぬように
\ruby{云}{い}ひたりとは
\ruby{聞}{きこ}えたれど、
%
その
\ruby{言葉}{こと|ば}の
\ruby{中}{うち}の
\ruby{{\換字{節}}々}{ふし|〴〵}には、
%
\ruby{既}{はや}
\ruby{全然}{すつ|かり}と
\ruby{我}{わ}が
\ruby{{\換字{近}}來}{ちか|ごろ}の
\ruby{狀態}{あり|さま}を
\ruby{知}{し}り
\ruby{盡}{つく}くして
\ruby{言}{い}ふと
\ruby{思}{おぼ}しくて、
%
ひし〳〵と
\ruby{身}{み}に
\ruby{徹}{こた}ふる
ところの
\ruby{少}{すくな}からぬに、
%
\ruby{氣息}{い|き}を
さへ
\ruby{潜}{ひそ}めて% 【潛 u6f5b 「先先」】【潜 u6f5c 「夫夫」】併用されている
\ruby{聞}{き}き
\ruby{居}{ゐ}たりし
\ruby{水野}{みづ|の}は、
%
\ruby{胸}{むね}の
\ruby{中}{うち}は
\ruby{石川}{いし|かは}の
\ruby{淸}{きよ}き
\ruby{瀬}{せ}を
\ruby{流}{なが}るゝ
\ruby{水}{みづ}と
\ruby{爽快}{さわや|か}にして、
%
\ruby{底}{そこ}の
\ruby{心}{こゝろ}は
\原本頁{121-6}\改行%
\ruby{春}{はる}と
\ruby{溫}{あたゝか}き
\ruby{我}{わ}が
\ruby{友}{とも}が、
%
\ruby{虛僞}{いつ|はり}ならず
\ruby{我}{われ}を
\ruby{思}{おも}ひ
\ruby{吳}{く}るゝ
\ruby{其}{そ}の
\ruby{眞{\換字{情}}}{ま|ごゝろ}に、
%
\ruby{其}{それ}と
\ruby{指}{さ}しては
\ruby{捉}{とら}へ
\ruby{{\換字{難}}}{がた}き
\ruby{香氣}{にほ|ひ}の
\ruby{物}{もの}を
\ruby{罩}{こ}むるが
\ruby{如}{ごと}くに
\ruby{我}{わ}が
\ruby{身心}{しん|〴〵}の
\ruby{全部}{すべ|て}が
\ruby{引}{ひ}き
\ruby{包}{つゝ}まれたるを
\ruby{覺}{おぼ}えて、
%
\ruby{嗚呼}{あ|ゝ}
\ruby{我}{われ}
\ruby{不幸福}{ふ|しあ|はせ}の%「幸福」ここは「は」
\ruby{月日}{つき|ひ}の
\ruby{下}{した}に
\ruby{生}{うま}れて、
%
\原本頁{121-9}\改行%
\ruby{物}{もの}の
\ruby{心}{こゝろ}も
\ruby{知}{し}らぬ
\ruby{頃}{ころ}より、
%
\ruby{{\換字{父}}}{ちゝ}をも
\ruby{母}{はゝ}をも
\ruby{失}{うしな}ひて、
%
\ruby{兄}{あに}も
\ruby{無}{な}ければ
\ruby{姊}{あね}も
\ruby{無}{な}く、
%
\ruby{世}{よ}の
\ruby{剩}{あま}され
\ruby{物}{もの}と
なつて
\ruby{生長}{そ|だ}ちし
まゝ、
%
\ruby{幼}{をさな}き
\ruby{時}{とき}の
\ruby{心}{こゝろ}にも、
%
\原本頁{121-11}\改行%
\ruby{丁稚奉公}{でつ|ち|ぼう|こう}せし
\ruby{家}{いへ}に、
%
\ruby{巢}{す}くひし
\ruby{燕}{つばめ}の
\ruby{親鳥}{おや|どり}の、
%
\ruby{日}{ひ}に
\ruby{百度}{もゝ|たび}も
\ruby{千度}{ち|たび}も
\ruby{飛}{と}んで
\ruby{去}{さ}つては
\ruby{飛}{と}んで
\ruby{{\換字{返}}}{かへ}つて、
%
まだ
\ruby{{\換字{弱}}}{よわ}き
\ruby{雛}{ひな}に
\ruby{餌}{ゑ}を
\ruby{{\換字{運}}}{はこ}ぶを
\ruby{見}{み}て、
%
\ruby{顏}{かほ}も
おぼえぬ
\ruby{吾}{わ}が
\ruby{母}{はゝ}
\ruby{戀}{こひ}しく、
%
\ruby{親}{おや}のある
\ruby{子}{こ}の
\ruby{羨}{うらや}ましさに、
%
\換字{志}く〳〵
\原本頁{122-3}\改行%
\ruby{泣}{な}いたる
\ruby{事}{こと}の
\ruby{記臆}{おぼ|え}さへ、% 原本通り「おぼえ」
%
まざ〳〵と
\ruby{今}{いま}に
\ruby{{\換字{遺}}}{のこ}れるなるが、
%
それには
\原本頁{122-4}\改行%
\ruby{引換}{ひき|か}へて
\ruby{幸{\換字{運}}}{しあ|はせ}にも、
%
アヽ%「幸運」ここは「は」
\ruby{我}{われ}
\ruby{何}{なん}の
\ruby{福}{ふく}のあつてか、
%
\ruby{自然}{し|ぜん}
\g詰めruby{々々}{〳〵}に
\ruby{知}{し}り
\原本頁{122-5}\改行%
\ruby{合}{あ}つたる
\ruby{六人}{ろく|にん}の
\ruby{良}{よ}き
\ruby{友}{とも}の
\ruby{其}{そ}の
\ruby{中}{うち}にも、
%
\ruby{{\換字{分}}}{わ}けて
\ruby{親}{した}しき
\ruby{羽{\換字{勝}}}{は|がち}
\ruby{島木}{しま|き}、
%
\原本頁{122-6}\改行%
\ruby{特}{こと}に
\ruby{島木}{しま|き}が
\ruby{眼}{ま}の
\ruby{{\換字{前}}}{あたり}の
\ruby{友{\換字{情}}}{なさ|け}!。
%
お
\ruby{澤}{さは}
\ruby[||j>]{婆}{ばゞあ}の
\ruby{言葉}{こと|ば}の
\ruby{{\換字{通}}}{とほ}り、
%
\ruby{手}{て}をついて
\原本頁{122-7}\改行%
\ruby{頼}{たの}んだつて
\ruby{芋塊}{い|も}
\ruby{一}{ひと}つも、
%
\ruby{自然}{ひと|りで}には
\ruby{出}{で}て
\ruby{來}{こ}ない
\ruby{此}{こ}の
\ruby{世}{よ}の
\ruby{中}{なか}に、
%
いづれ
\ruby{身}{み}の
\ruby{油汗}{あぶら|あせ}が
\ruby{化}{ば}けたに
\ruby{{\換字{違}}}{ちが}ひ
\ruby{無}{な}い
\ruby{多額}{おほ|く}の
\ruby{金子}{か|ね}をも、
%
\ruby{紙}{かみ}の
\ruby{一枚}{いち|まい}でも
\ruby{吳}{く}れるやうに、
%
\ruby{惜}{をし}む
\ruby{色}{いろ}さへ
\ruby{無}{な}く
\ruby{快}{こゝろよ}く
\ruby{吳}{く}れて、
\換字{志}かも
\ruby{君}{きみ}の
ためになる
\ruby{事}{こと}ならば、
%
\ruby{馬}{うま}にでも
\ruby{牛}{うし}にでもなつて
\ruby{働}{はたら}いて
\ruby{{\換字{遣}}}{や}らうと、
%
\ruby{身}{み}を
\ruby{入}{い}れて
\ruby{吳}{く}れる
\ruby{其}{そ}の
\ruby{俠氣}{をとこ|ぎ}!。
%
\ruby{人世}{うき|よ}の
\ruby{場数}{ば|かず}を% 原文通り「場」
\ruby{踏}{ふ}んで
\ruby{來}{き}た
\ruby{人}{ひと}には、
%
\原本頁{123-1}\改行%
\ruby{隨{\換字{分}}}{ずゐ|ぶん}
\ruby{幼稚}{こ|ども}にも
\ruby{{\換字{若}}輩}{じやく|はい}にも
\ruby{思}{おも}はれようか
\ruby{知}{し}れぬ
\ruby{事}{こと}なるに、
%
\ruby{我}{わ}が
\ruby{{\換字{情}}緖}{おも|ひ}の
\ruby{上}{うへ}に
\ruby{就}{つ}いては
\ruby{咎}{とが}め
\ruby{立}{だ}てもせず、
%
\ruby{年齡}{と|し}の
\ruby{{\換字{所}}爲}{せ|ゐ}にして
\ruby{仕舞}{し|ま}つて
\ruby{一}{ひ}
ト
\原本頁{123-3}\改行%
\ruby{言}{こと}も
\ruby{云}{い}はぬ
\ruby{寛大}{おほ|やう}さ!。
%
たゞ
\ruby{身體}{から|だ}を
\ruby{大切}{だい|じ}に
\ruby{仕}{し}て
\ruby{吳}{く}れろと
\ruby{云}{い}つて
\ruby{吳}{く}れる
\ruby{其}{そ}の
\ruby{親切}{しん|せつ}!。
%
\ruby{嗚呼}{あ|ゝ}
\ruby{兄}{あに}と
\ruby{云}{い}はうか、
%
\ruby{姊}{あね}と
\ruby{云}{い}はうか、
%
\ruby{兄}{あに}も
\ruby{姊}{あね}も
\原本頁{123-5}\改行%
\ruby{中々}{なか|〳〵}
かうばかりはあるまい。
%
まして
\ruby{朋友}{とも|だち}と
\ruby{云}{い}はうには
\ruby{勿體無}{もつ|たい|な}いほど。
%
\ruby{人}{ひと}に
\ruby{云}{い}はれぬ
\ruby{苦悶}{く|るし}みを
\ruby{抱}{いだ}けば、
%
\ruby{何}{なに}につけ
\ruby{彼}{か}につけて
\ruby{此}{こ}の
\原本頁{123-7}\改行%
\ruby{世}{よ}の
\ruby{中}{なか}を、
%
\ruby{味氣}{あぢ|き}
\ruby{無}{な}く
\ruby{思}{おも}ふ
\ruby{時}{とき}のみ
\ruby{此頃}{この|ごろ}は
\ruby{多}{おほ}かりしが、
%
あゝ
\ruby{有}{あ}り
\ruby{{\換字{難}}}{がた}き
\ruby{天}{てん}の
\ruby{恩惠}{めぐ|み}、
%
\ruby{水野}{みづ|の}
\ruby{靜十郎}{せい|じう|らう}
\ruby{幸福}{さい|はひ}にして、%「幸福」ここは「は」
%
かゝる
\ruby{信義}{しん|ぎ}の
\ruby{友}{とも}にも
\ruby{未}{ま}だ
\原本頁{123-9}\改行%
\ruby{棄}{す}てられねば、
%
アヽ
\ruby{思}{おも}へば
\ruby{我}{われ}は
\ruby{世}{よ}にも
\ruby{稀}{まれ}なる
\ruby{幸{\換字{運}}}{しあ|はせ}を%「幸運」ここは「は」
\ruby{受}{う}け
\ruby{得}{え}たる
\原本頁{123-10}\改行%
\ruby{身}{み}なるかな、
%
\ruby{我}{わ}が
\ruby{行末}{ゆく|すゑ}も
\ruby{光}{ひかり}ありて、
%
\ruby{{\換字{強}}}{あなが}ち
\ruby{黑闇}{や|み}のみならず
\ruby{見}{み}ゆ、
%
と
\ruby{悅}{よろこ}ぶにも
\ruby{先}{ま}づ
\ruby{涙}{なみだ}にて、
%
\ruby{謝}{しや}する
\ruby{言葉}{こと|ば}も
たど〳〵しく、

\原本頁{124-1}%
『アヽ
\ruby{島木}{しま|き}
\ruby{君}{くん}、
%
\ruby{感謝}{かん|しや}する。
%
\ruby{免}{ゆる}して
\ruby{吳}{く}れたまへ、
%
\ruby{僕}{ぼく}は
\ruby{何}{なん}にも
\ruby{言}{い}ふことが
\ruby{出來無}{で|き|な}い。
%
\ruby{言}{い}ひたい
\ruby{{\換字{情}}懷}{こゝろ|もち}は
\ruby{澤山}{たん|と}あるが
\ruby{胸}{むね}が
\ruby{張}{は}つて
\ruby{居}{ゐ}て
\原本頁{124-3}\改行%
\ruby{何}{なん}にも
\ruby{言}{い}へない。
%
\ruby{實}{じつ}に
\g詰めruby{々々}{〳〵}
\ruby{君}{きみ}の
\ruby{親切}{しん|せつ}は
\ruby{深}{ふか}く
\ruby{謝}{しや}する。
%
\ruby{君}{きみ}の
\ruby{談}{はなし}は
\ruby{骨}{ほね}に
\ruby{浸}{し}みて
\ruby{解}{わか}つた。
%
\ruby{決}{けつ}して
\ruby{忘}{わす}れ
\ruby{無}{な}い、
%
\ruby{忘}{わす}れ
\ruby{無}{な}い!。
%
\ruby{成程}{なる|ほど}
\ruby{何}{なん}に
\ruby{卷}{ま}き
\原本頁{124-5}\改行%
\ruby{倒}{たふ}されては
\ruby{濟}{す}まない
\ruby{身體}{から|だ}だ!。
%
\ruby{僕}{ぼく}も
\ruby{果敢}{は|か}ない
\ruby{思}{おもひ}に
\ruby{死}{し}にたかあ
\ruby{無}{な}い!。
%
いや
\ruby{僕}{ぼく}は
\ruby{何樣}{ど|う}
まかり
\ruby{間{\換字{違}}}{ま|ちが}つても
\ruby{脆}{もろ}くは
\ruby{死}{し}なゝい!。
%
\ruby{戀{\換字{情}}}{じよ|う}は
\ruby{戀{\換字{情}}}{じよ|う}だけれど、
%
\ruby{大望心}{たい|ま|う}は
\ruby{大望心}{たい|ま|う}だ!。
%
\ruby{身體}{から|だ}も
\ruby{必}{かなら}ず
\ruby{大切}{たい|せつ}にする。
』

\原本頁{124-8}%
と、
%
\ruby{{\換字{強}}}{しひ}て
\ruby{勉}{つと}めて
\ruby{答}{こた}へたり。

\原本頁{124-9}%
\ruby{夜}{よ}は
\ruby{彼}{かれ}
\ruby{一句}{いつ|く}
\ruby{此}{これ}
\ruby{一句}{いつ|く}の
\ruby{二人}{ふた|り}が
\ruby{親}{した}しき
\ruby{物語}{もの|がたり}に
\ruby{漸}{やうや}く
\ruby{盡}{つ}きて、
%
\ruby{早}{はや}くも
\ruby{暁天}{あ|け}
\ruby{{\換字{近}}}{ちか}く
ならんとすれば、
%
\ruby{水野}{みづ|の}は
\ruby{{\換字{終}}}{つひ}に
\ruby{島木}{しま|き}が
\ruby{許}{もと}を
\ruby{辭}{じ}して、
%
\ruby{{\換字{情}}中}{ふと|ころ}に
\ruby{阿堵物}{も||の}あるに% 「阿堵物(あとぶつ)」お金のこと
\ruby{勢}{いきほ}ひ
\ruby{好}{よ}く、
%
\ruby{紫色}{むら|さき}
\ruby{立}{だ}てる
\ruby{天}{そら}の
\ruby{星}{ほし}
\ruby{薄}{うす}れ
\ruby{行}{ゆ}きて
\ruby{{\換字{朝}}風}{あさ|かぜ}の
\ruby{徐々}{おも|むろ}に
\ruby{吹}{ふ}き
\ruby{出}{だ}す
\ruby{頃}{ころ}、
%
\ruby{相良}{さが|ら}が
\ruby{家}{いへ}を
\ruby{敲}{たゝ}き
\ruby{起}{おこ}して
\ruby{昨日}{きの|ふ}の
\ruby{恩}{おん}を
\ruby{謝}{しや}し、
%
\ruby{{\換字{猶}}}{なほ}
\ruby{信頼}{た|の}むに
\ruby{足}{た}るべき
\ruby{看護{\換字{婦}}}{かん|ご|ふ}を
\ruby{世話}{せ|わ}せん
ことを
\ruby{乞}{こ}ひ
\ruby{求}{もと}めて、
%
\ruby{其}{そ}の
\ruby{快}{こゝろよ}く
\ruby{諾}{うけが}ひ
\ruby{吳}{く}れたるに
\ruby{心勇}{こゝろ|いさ}み、
%
\ruby{足}{あし}
\ruby{輕}{かろ}く
\ruby{歸路}{かへ|り}を
\ruby{急}{いそ}ぎて、
%
\ruby{淺草}{あさ|くさ}の
\ruby{雷神門{\換字{前}}}{かみ|なり|もん|まへ}に
さしかゝりぬ。

\Entry{其二十一}

% メモ 校正終了 2024-04-08 2024-05-25 2024-06-18
\原本頁{125-6}%
おもふ
\ruby{人}{ひと}の
\ruby{病}{やまひ}は
\ruby{篤}{あつ}けれども、
%
\ruby{思}{おも}ひし
\ruby{事}{こと}は
\ruby{皆}{みな}
\ruby{爲}{な}し
\ruby{得}{え}たり、
%
\ruby[g]{相良}{さがら }も
\原本頁{125-7}\改行%
\ruby{今}{いま}
\ruby[g]{一度}{いちど }
\ruby[g]{見舞}{み ま }ひて
\ruby[g]{尾竹}{を たけ}に
あひて
\ruby[g]{種々}{くさ〴〵}の
\ruby[||j>]{心}{こゝろ}
\ruby[||j>]{添}{ ぞへ}をも
% \ruby{心添}{こゝろ|ぞへ}をも
なし
\ruby{置}{お}かんと
\ruby{云}{い}ひ
\改行% 校正作業の簡略化のため
、
%
\原本頁{125-8}\改行%
\ruby{良}{よ}き
\ruby{看護{\換字{婦}}}{かん|ご|ふ}をも
\ruby{晝}{ひる}までとは
\ruby{{\換字{過}}}{すご}さず
\ruby{四ツ木}{よ| |ぎ}に
\ruby{{\換字{遣}}}{や}り
\ruby{吳}{く}るゝ
\ruby[g]{手筈}{て はず}に
\ruby{定}{さだ}まりたり、
%
この
\ruby{上}{うへ}は
たゞ
\ruby{健}{まめ}やかなる
\ruby[||j>]{婢}{をんな}
\ruby[||j>]{一人}{ ひと|り}を
\ruby{看護{\換字{婦}}}{かん|ご|ふ}の
\ruby[g]{指揮}{さしづ }の
\ruby{下}{しも}につけて
\ruby[g]{雜事}{ざつじ }に
\ruby{當}{あた}らすれば、
%
もとより
\ruby[g]{介抱}{かいはう}の
\ruby[g]{此上}{このうへ}
\ruby{無}{な}く
\ruby[g]{行屆}{ゆきとゞ}きて% 「屆」「届」 原本通り「屆」
\原本頁{126-1}\改行%
\ruby{善}{ぜん}を
\ruby{盡}{つく}したりと
\ruby{云}{い}ふべきには
あらねど、
%
\ruby{今}{いま}の
\ruby{身}{み}にての
\ruby{我}{わ}が
\ruby{心}{こゝろ}の
\原本頁{126-2}\改行%
\ruby{及}{およ}ぶほどだけは
\ruby{盡}{つく}したるなり、
%
と
\ruby{思}{おも}ふにつけて
\ruby{人}{ひと}
\ruby{知}{し}らず
\ruby{樂}{たの}しく
\改行% 校正作業の簡略化のため
、
%
\原本頁{126-3}\改行%
\ruby[||j>]{愁}{うれひ}の
\ruby{中}{なか}にも
\ruby{幽}{かすか}なる
\ruby{笑}{ゑみ}の
\ruby{催}{もよほ}さるゝ
\ruby[g]{心地}{こゝち }して、
%
\ruby{願}{ねが}はくは
\ruby{我}{わ}が
\ruby{五十子}{い|そ|こ}の
\ruby{病}{やまひ}の
\ruby{漸}{やうや}く
\ruby{痊}{おこた}りて、
%
\ruby[||j>]{心}{こゝろ}
\ruby[||j>]{盡}{ づく}しの
% \ruby{心盡}{こゝろ|づく}しの
\ruby[g]{甲{\換字{斐}}}{か ひ }も
あれかし、
%
\ruby{暴}{あら}き
\ruby{雨}{あめ}
\ruby{風}{かぜ}に
\ruby{根}{ね}を
\原本頁{126-5}\改行%
\ruby{搖}{ゆら}がされて
\ruby[g]{敢無}{あへな }くも
\ruby[g]{天壽}{いのち }ならず
\ruby{枯}{か}れんとする
\ruby{樹}{き}を、
%
おぼつか
\ruby{無}{な}きながら
\ruby{支}{さゝ}へ
\ruby{培}{つちか}ひて、
%
\ruby{復}{ふたゝ}び
\ruby{花}{はな}
\ruby{{\換字{咲}}}{さ}く
\ruby{春}{はる}の
\ruby{曉}{あした}に、
%
\ruby[g]{丹誠}{たんせい}の
\ruby[g]{甲{\換字{斐}}}{か ひ }ありて
\原本頁{126-7}\改行%
\ruby[||j>]{美}{うつく}しく
\ruby{日}{ひ}に
\ruby{匂}{にほ}ふを
\ruby{見}{み}ば、
%
\ruby[g]{如何}{い か }ばかりか
\ruby{心}{こゝろ}の
\ruby{嬉}{うれ}しからん、
%
それにつけても
\ruby[g]{昨日}{きのふ }よりの
\ruby{長}{なが}き
\ruby{夜}{よ}
\ruby[g]{一夜}{ひとよ }を、
%
\ruby{我}{わ}が
\ruby{五十子}{い|そ|こ}は
\ruby[g]{如何}{い か }なる
\ruby[g]{狀態}{やうす }に
\ruby{{\換字{送}}}{おく}りたらん、
%
\ruby{熱}{ねつ}の
\ruby{烈}{はげ}しく
\ruby{{\換字{進}}}{さ}すことは
\ruby{無}{な}かりしか、
%
\ruby{{\換字{強}}}{つよ}く
\ruby{苦}{くるし}む
\ruby{事}{こと}は
\ruby{無}{な}かりしか、
%
ともすれば
\ruby[g]{心臓}{しんざう}
\ruby[g]{肺臓}{はいざう}の
\ruby{此}{こ}の
\ruby{病}{やまひ}には
\ruby{惡}{あし}くなるものと
\ruby{聞}{き}きたるが、
\ruby[g]{其等}{それら }の
\ruby{凶}{あし}きことは
\ruby{無}{な}かりし
\ruby{歟}{か}、
%
\ruby[g]{尾竹}{を たけ}も
\ruby[<g>]{親切}{しんせつ}の
\makeatletter
\@ifundefined{デバッグ@ビルド}{%
  \ruby{男}{をとこ}
}{%
  \ruby[<j||]{男}{をとこ}% 行末行頭の境界付近なので特例処置を施す
  \原本頁{127-1}\改行%
}%
\makeatother
なれば、
%
\ruby[g]{容態}{ようだい}
\ruby{惡}{あし}くば
\ruby{附}{つ}きゝりに
\ruby{附}{つ}きても
\ruby{居}{ゐ}ては
\ruby{吳}{く}れたるべけれど、
%
\ruby{氷}{こほり}より
\ruby{冷}{つめた}い
\ruby{心}{こゝろ}の
\ruby{彼}{あ}の
お
\ruby{澤}{さは}
\ruby{婆}{ばゞ}、
%
くれ〴〵も
\ruby{頼}{たの}み
\ruby{置}{お}きたる
\makeatletter
\@ifundefined{デバッグ@ビルド}{%
  \ruby[||j>]{氷}{ひよう}
  \ruby[||j>]{嚢}{ なう}
}{%
  \ruby[<j||]{氷}{ひよう}
  \ruby[<j||]{嚢}{なう}
  \原本頁{127-3}\改行%
}%
\makeatother
の
% \ruby{氷嚢}{ひよう|なう}の
\ruby[g]{世話}{せ わ }さへ、
%
\ruby{既}{すで}に
\ruby[|g|]{一昨日}{をとゝひ}
といひ
\ruby[g]{昨日}{きのふ }と
\ruby{云}{い}ひ、
%
\ruby{碌}{ろく}に
\ruby{身}{み}に
\ruby{染}{し}みても
\原本頁{127-4}\改行%
\ruby{爲}{し}て
\ruby{吳}{く}れざりし、
%
あゝいふ
\ruby{不}{ふ}
\ruby[<j>]{幸}{しあはせ}の%「不幸」ここは「は」
\ruby{處}{ところ}に
\ruby[g]{居合}{ゐ あ }はせたる
\makeatletter
\@ifundefined{デバッグ@ビルド}{%
  \ruby[<j||]{病}{びやう}
  \ruby[<j||]{人}{にん}の、
}{%
  \ruby[||j>]{病}{びやう}
  \ruby[||j>]{人}{ にん}の、
}%
\makeatother
% \ruby{病人}{びやう|にん}の、
%
\ruby{思}{おも}へば
\ruby[g]{一夜}{ひとよ }が
\ruby[g]{氣{\換字{遣}}}{き づか}はるゝ、
%
と
\ruby[g]{偶然}{ふ と }
\ruby[g]{思念}{おもひ }の
\ruby[g]{其處}{そ こ }に
\ruby[g]{片荷}{かたに }づゝては
\ruby{矢}{や}も
\原本頁{127-6}\改行%
\ruby{楯}{たて}も
\ruby{堪}{たま}らず、
%
\ruby[g]{物淋}{ものさび}しく
\ruby[g]{薄暗}{うすくら}き
\ruby{離}{はな}れ
\ruby{屋}{や}の
\ruby{中}{うち}の、
%
\ruby[g]{孤燈}{こ とう}
\ruby[||j>]{力}{ちから}
\ruby[||j>]{無}{ な}く% ルビ調整(原本通り)
\ruby{照}{て}らす
\原本頁{127-7}\改行%
\ruby[||j>]{光}{ひかり}の
\ruby{下}{もと}に、
%
\ruby[g]{頭髮}{か み }は
\ruby[g]{亂菊}{らんぎく}の
\ruby[g]{花瓣}{はなびら}の% 弁 (瓣) 辦 辧 辨 辩 辯
\ruby{霜}{しも}に
\ruby{傷}{いた}める
\ruby{姿}{すがた}と
\ruby{崩}{くづ}れて、
%
\ruby{悶}{もだ}え
\ruby{悶}{もだ}えつゝ
\ruby[g]{埒無}{らちな }く
\ruby{病}{や}み
\ruby{臥}{ふ}せる
\ruby{態}{さま}の、
%
\ruby{眼}{め}の
\ruby{{\換字{前}}}{まへ}に
あり〳〵と
\ruby{{\換字{浮}}}{うか}み
\ruby{來}{く}るやう
\ruby{覺}{おぼ}えて、
%
\ruby[g]{島木}{しまき }が
\ruby{寓}{やど}を
\ruby{敲}{たゝ}きたりし
\ruby{折}{をり}、
%
\ruby{頭}{かうべ}を
\ruby{反}{かへ}して
\ruby[g]{偶然}{ふ と }
\ruby{見}{み}し
\ruby{北}{きた}の
\原本頁{127-10}\改行%
\ruby{{\換字{空}}}{そら}に、
%
\ruby{大}{おほき}なる
\ruby{美}{うつく}しき
\ruby{星}{ほし}の
\ruby[g]{長々}{なが〳〵}と
\ruby{光}{ひかり}を
\ruby{曳}{ひ}いて
\ruby{流}{なが}れて
\ruby{{\換字{消}}}{き}えしも、
%
\ruby{思}{おも}ひ
\ruby{合}{あは}されて
\ruby[g]{今{\換字{更}}}{いまさら}
\ruby{急}{きふ}に
\ruby{何}{なん}と
\ruby{無}{な}く
\ruby{忌}{いま}はしく、
%
おもはず
\ruby[g]{慄然}{りつぜん}として
\ruby{天}{てん}を
\ruby{偸}{ぬす}み
\ruby{見}{み}たり。

\原本頁{128-2}%
\ruby{天}{そら}は
\ruby{今}{いま}
\ruby{白}{しら}み
わたりて
\ruby{靜}{しづか}に、
%
\ruby[g]{星辰}{ほ し }は
\ruby{潛}{ひそ}みつ、% 【潛 u6f5b 「先先」】【潜 u6f5c 「夫夫」】併用されている
%
\ruby[g]{瑠璃}{る り }の
\ruby[||j>]{盤}{ばん}
\ruby[||j>]{上}{じやう}に
% \ruby{盤上}{ばん|じやう}に
\ruby[g]{金砂}{きんしや}を
\原本頁{128-3}\改行%
\ruby{撒}{ま}きし
\ruby{數時間{\換字{前}}}{すう|じ|かん|まへ}の
\ruby[g]{光景}{ありさま}は
\ruby{痕}{あと}も
\ruby{無}{な}く
\ruby{{\換字{消}}}{き}え
\ruby{去}{さ}つて、
%
また
ありし
おもかげを
\ruby{{\換字{忍}}}{しの}ぶべくも
あらぬ
\ruby{狀}{さま}なるに、
%
おのづと
\ruby{新}{あたら}しき
\ruby[g]{淸旦}{あした }の
\ruby{氣}{き}を
\原本頁{128-5}\改行%
\ruby{受}{う}けて
\ruby{胸}{むね}も
\ruby{開}{ひら}き、
%
アヽ
\ruby[g]{{\換字{前}}表}{ぜんぺう}と
いふやうなる
\ruby{事}{こと}を
\ruby{氣}{き}に
\ruby{仕}{し}たる
\makeatletter
\@ifundefined{デバッグ@ビルド}{%
  \ruby[<g>]{愚さ}{おろか }、
}{%
  \ruby{愚}{おろか}さ
  \改行% 校正作業の簡略化のため
  、
}%
\makeatother
%
\原本頁{128-6}\改行%
\ruby[g]{島木}{しまき }の
\ruby[g]{言葉}{ことば }にも
\ruby{羞}{はづ}か
\換字{志}かりし、
%
と
\ruby{私}{ひそか}に
\ruby{自}{みづか}ら
\ruby[g]{女々}{め ゝ }しきを
\ruby{慚}{は}ぢたり
\改行% 校正作業の簡略化のため
。
%
\原本頁{128-8}\改行%
されど
\ruby{心}{こゝろ}は
\ruby[g]{一度}{ひとたび}
\ruby{動}{うご}きて
\ruby{復}{また}
\ruby{安}{やす}まらず。
%
\ruby{曉}{あした}に
\ruby{{\換字{消}}}{き}えし
\ruby{星}{ほし}は
\ruby[g]{再度}{ふたゝび}
\ruby[||j>]{夕}{ゆふべ}に
\ruby{見}{み}るべけれども、
%
\ruby[g]{一度}{ひとたび}
\ruby{去}{さ}つては
\ruby{行}{ゆ}く
\ruby{方}{かた}
\ruby{知}{し}れぬ
\ruby{人}{ひと}の
\ruby{身}{み}の、
%
\ruby[g]{死生}{し せい}の
\ruby[g]{抑々}{そも〳〵}
\ruby{何}{なに}に
\ruby{繫}{かゝ}りて、
%
\ruby[g]{禍福}{くわふく}の
\ruby[g]{將{\換字{又}}}{はたまた}
\ruby{何}{なに}に
\ruby{本}{もと}づくかも
\ruby{{\換字{分}}}{わか}らぬ
\ruby[g]{茫々}{ばう〳〵}たる
\ruby[g]{劫{\換字{運}}}{ごふうん}の
\ruby{測}{はか}り
\ruby{{\換字{難}}}{がた}く
\ruby{窺}{うかゞ}ひ
\ruby{{\換字{難}}}{がた}きに
\ruby{思}{おも}ひ
\ruby{到}{いた}りては、
%
あゝ
\ruby{頼}{たの}まれぬ
\ruby{人}{ひと}の
\ruby{世}{よ}なるかな、
%
\ruby{我}{わ}が
\ruby{心}{こゝろ}の
\ruby{膏}{あぶら}を
\ruby{燃}{も}やし、
%
\ruby{骨}{ほね}の
\ruby{髓}{ずゐ}を% u9ad3 骨 左 月 辶
\ruby{焚}{た}きて、
%
\ruby[g]{願望}{ねがひ }は
\ruby{大}{おほい}ならぬ
\原本頁{129-1}\改行%
\ruby{我}{わ}が
\ruby{身}{み}の
\ruby[g]{周圍}{まはり }に、
%
\ruby{聊}{いさゝ}かの
\ruby[g]{光明}{ひかり }を
\ruby{得}{え}んと
\ruby{願}{ねが}ふも、
%
\ruby[g]{{\換字{運}}命}{うんめい}の
\ruby{風}{かぜ}の
\ruby[g]{容赦}{ようしや}
\原本頁{129-2}\改行%
\ruby{無}{な}く
\ruby{吹}{ふ}き
\ruby{荒}{すさ}まんには、
%
\ruby{頼}{たの}む
\ruby{影}{かげ}なき
\ruby[g]{裸火}{はだかび}の、
%
\ruby{脆}{もろ}くも
\ruby{忽}{たちま}ち
\ruby{吹}{ふ}き
\ruby{滅}{け}さ
\原本頁{129-3}\改行%
れて、
%
\ruby[g]{天地}{てんち }は
\ruby[||j>]{{\換字{情}}}{なさけ}
\ruby[||j>]{無}{ な}き
\ruby{闇}{やみ}と
なるべし。
%
おもへば
\ruby{小}{ちひさ}きは
\ruby{人}{ひと}の
\ruby{力}{ちから}なり
\改行% 校正作業の簡略化のため
。
%
\原本頁{129-4}\改行%
かほどに
\ruby{身}{み}を
\ruby{勞}{つか}らせ
\ruby{心}{こゝろ}を
\ruby{盡}{つく}して、
%
\ruby{我}{わ}が
\ruby{思}{おも}ふ
\ruby{人}{ひと}
\ruby{好}{よ}かれと
\ruby{我}{われ}は
\ruby{願}{ねが}へど、
%
\ruby[g]{慈悲}{なさけ }
\ruby{有}{あ}りや
\ruby{無}{な}しやも
おぼつかなき、
%
\ruby[g]{{\換字{運}}命}{うんめい}と
いふものゝ
\makeatletter
\@ifundefined{デバッグ@ビルド}{%
  \ruby[||j>]{意}{こゝろ}
  \ruby[||j>]{任}{ まか}せ!、
}{%
  \ruby[<j||]{意}{こゝろ}
  \ruby{任}{まか}せ!、
}%
\makeatother
%
\ruby{其}{そ}の
\ruby[g]{意が}{こゝろ }
\ruby[g]{人{\換字{情}}}{なさけ }を
\ruby{知}{し}つて
\ruby{吳}{く}れうでも
\ruby{無}{な}ければ、
%
\ruby{思}{おも}へば〳〵
\原本頁{129-7}\改行%
\ruby{悲}{かな}しきは
\ruby{人}{ひと}の
\ruby{世}{よ}!。
%
\ruby[g]{{\換字{平}}生}{ひごろ }は% ルビ調整(原本通り)
\ruby{天}{そら}
\ruby{{\換字{翔}}}{か}ける
\ruby{事}{こと}も
\ruby{爲}{な}さば
\ruby{爲}{な}すべき
\ruby[g]{雄心}{をごゝろ}
\ruby{持}{も}
\原本頁{129-8}\改行%
ちし
\ruby{我}{われ}なりしが、
%
\ruby{身}{み}に
\ruby{染}{し}みて
\ruby{今}{いま}ぞ
\ruby[g]{人間}{にんげん}の
\ruby[g]{甲{\換字{斐}}}{か ひ }
\ruby{無}{な}きを
\ruby{知}{し}りつる!
\改行% 校正作業の簡略化のため
。
%
\原本頁{129-9}\改行%
\ruby{天}{てん}は
\ruby{限}{かぎ}り
\ruby{無}{な}く
\ruby{大}{おほい}なるに、
%
\ruby{我}{われ}は
\ruby[g]{糠星}{ぬかぼし}の
\ruby{其}{それ}より
\ruby{微}{かす}けく、
%
\ruby{地}{ち}は
\ruby{涯}{はて}も
\ruby{無}{な}く
\ruby{廣}{ひろ}やかなるに、
%
\ruby{身}{み}は
\ruby{塵}{ちり}
\ruby{土}{ひぢ}と
\ruby{小}{ちひさ}なる、
%
\ruby{此}{こ}の
\ruby[<g>]{某甲}{なにがし}が
\ruby{懷}{いだ}ける
\ruby{念}{おもひ}の、
%
\原本頁{129-11}\改行%
\ruby[g]{{\換字{運}}命}{うんめい}に
\ruby{對}{むか}へる
\ruby{其}{そ}の
\ruby[g]{眞態}{ありさま}は、
%
\ruby{譬}{たと}へば
\ruby[g]{一縷}{いちる }の
\ruby{細}{ほそ}き〳〵、
%
\ruby{毛}{け}の
\ruby{如}{ごと}く
\ruby[g]{蜘蛛}{く も }の
\ruby{圍}{ゐ}の
ごとき
\ruby{絲}{いと}を、
%
\ruby{千萬馬力}{せん|まん|ば|りき}もて
\makeatletter
\@ifundefined{デバッグ@ビルド}{%
  \ruby[<g>]{轟き}{とゞろ }
}{%
  \ruby{轟}{とゞろ}き
}%
\makeatother
\ruby{{\換字{廻}}}{まは}れる
\ruby{大車輪}{だい|しや|りん}に
\ruby{繫}{か}けて
\改行% 校正作業の簡略化のため
、
%
\原本頁{130-2}\改行%
\ruby{其}{そ}の
\ruby[g]{車輪}{しやりん}の
\ruby{我}{わ}が
\ruby{願}{ねが}ふ
\ruby{方}{かた}に
\ruby{{\換字{廻}}}{まは}らんことを、
%
\ruby{竊}{ひそか}に
\ruby{願}{ねが}ひ
\ruby{求}{もと}むるが
\ruby{如}{ごと}し
\改行% 校正作業の簡略化のため
。
%
\原本頁{130-3}\改行%
\ruby[g]{嗚呼}{あ ゝ }、
%
\ruby{我}{わ}が
\ruby{願}{ねが}ひの
\ruby{聽}{き}かるべき
\ruby[<g>]{や\換字{?!}}{}。% 「?!」が行頭にならないよう特殊処理
%
\ruby[||j>]{心}{こゝろ}
\ruby[||j>]{細}{ ぼそ}くも
% \ruby{心細}{こゝろ|ぼそ}くも
また
\ruby[||j>]{心}{こゝろ}
\ruby[||j>]{細}{ ぼそ}くて、
% \ruby{心細}{こゝろ|ぼそ}くて、
%
\makeatletter
\@ifundefined{デバッグ@ビルド}{%
  \ruby[g]{{\換字{情}}無}{なさけな}くも
}{%
  \ruby[<j||]{{\換字{情}}}{なさけ}% 行末行頭の境界付近なので特例処置を施す
  \原本頁{130-4}\改行%
  \ruby{無}{な}くも
}%
\makeatother
\ruby{物}{もの}のみの
\ruby{思}{おも}はるゝ
\ruby{世}{よ}かな!。
%
\ruby{我}{わ}が
\ruby[g]{智慧}{ち ゑ }の
\ruby{今}{いま}
\ruby{効}{かひ}
\ruby{無}{な}きを
\ruby{知}{し}り
\改行% 校正作業の簡略化のため
、
%
\原本頁{130-5}\改行%
\ruby{我}{わ}が
\ruby[g]{意念}{おもひ }の
\ruby{今}{いま}
\ruby[g]{孱{\換字{弱}}}{か よわ}きを
\ruby{知}{し}り、
%
\ruby{斷}{た}えぬ
\ruby{泉}{いづみ}と
\ruby{湧}{わ}き
\ruby{上}{あが}る
\ruby{戀}{こひ}の
\ruby{誠}{まこと}に
\ruby{洗}{あら}はれて、
%
\ruby[g]{心は}{こゝろ }
\ruby[g]{無垢}{む く }の
\ruby[g]{往時}{むかし }に
\ruby{{\換字{返}}}{かへ}りぬ。
%
アヽ
\ruby{今}{いま}
\ruby{我}{われ}は
\ruby[g]{嬰兒}{みどりご}なり!。
%
\ruby[g]{天地}{てんち }の
\ruby[g]{那處}{いづく }に
\ruby[g]{慈母}{は ゝ }の
\ruby[g]{御坐}{お は }
\ruby[<g>]{す\換字{?!}}{}。% 「?!」が行頭にならないよう特殊処理
%
\ruby{泣}{な}きて
\ruby{呼}{よ}び
\ruby{度}{た}き
\ruby[g]{心地}{こゝち }ぞする。
%
と
\ruby[g]{曉天}{あかつき}の
\ruby{{\換字{猶}}}{なほ}
\ruby[g]{靜寂}{しづか }にして
\ruby{人}{ひと}の
\ruby{{\換字{通}}}{とほ}りも
\ruby[g]{稀少}{まばら }なるに、
%
\ruby{深}{ふか}くも
\ruby{心}{こゝろ}の
\ruby{奧}{おく}に
\ruby{思}{おも}ひ
\ruby{入}{い}つたる
\ruby[g]{水野}{みづの }は、
%
ふつと
\ruby{我}{われ}に
\ruby{{\換字{返}}}{かへ}つて
\ruby{頭}{かうべ}を
\ruby{擡}{あ}ぐれば、
%
\ruby{身}{み}は
\ruby[g]{何時}{い つ }の
\ruby{程}{ほど}にか
\ruby{來}{きた}りけん、
%
\ruby[g]{塵埃}{ち り }
\ruby{無}{な}き
\ruby{{\換字{朝}}}{あした}の
\ruby{露}{つゆ}けき
\ruby[g]{石路}{せきろ }の、
%
\ruby[g]{長々}{なが〳〵}しきを
\ruby{知}{し}らぬ
\原本頁{130-11}\改行%
\ruby{間}{ま}に
\ruby{{\換字{過}}}{す}ぎて、
%
\ruby{今}{いま}や
\makeatletter
\@ifundefined{デバッグ@ビルド}{%
  \ruby[<g>]{淺草寺}{せんさうじ }の
}{%
  \ruby{淺草寺}{せん|さう|じ}の
}%
\makeatother
\ruby[g]{山門}{さんもん}を、
%
\ruby{既}{すで}に
\ruby[g]{{\換字{半}}は}{なかば }
\ruby{潛}{くゞ}り% 【潛 u6f5b 「先先」】【潜 u6f5c 「夫夫」】併用されている
\ruby{居}{ゐ}たり。

\原本頁{131-1}%
\ruby[g]{晝間}{ひ る }は
\ruby{賑}{にぎ}やかなる
\ruby[g]{中店}{なかみせ}も、
%
\ruby{{\換字{猶}}}{なほ}
\ruby{寂々}{じやく|〳〵}として
\ruby{物}{もの}の
\ruby{響}{ひゞき}を
\ruby{傳}{つた}へず、
%
\ruby[g]{御{\換字{扉}}}{みとびら}を
\ruby{今}{いま}
\ruby{開}{ひら}きしばかりの、
%
\ruby[g]{御堂}{み だう}の
\ruby{内}{うち}は
\ruby[g]{仄暗}{ほのぐら}きに、
%
\ruby{御燈明}{み|あか|し}の
\ruby[g]{煌々}{きら〳〵}と
\makeatletter
\@ifundefined{デバッグ@ビルド}{%
  \ruby[<g>]{黄金色}{こがねいろ}
}{%
  \ruby{黄}{こ}
  \原本頁{131-3}\改行%
  \ruby{金}{がね}
  \ruby{色}{いろ}
}%
\makeatother
に
\ruby{見}{み}えて、
%
\ruby[g]{{\換字{朝}}{\換字{勤}}}{あさづと}めの
\ruby[g]{讀經}{どきやう}の
\ruby{聲}{こゑ}は
\ruby[||j>]{殊}{しゆ}
\ruby[||j>]{{\換字{勝}}}{しよう}に
% \ruby{殊{\換字{勝}}}{しゆ|しよう}に
\ruby{澄}{す}み
\ruby{渡}{わた}り、
%
\ruby[g]{御堂}{み だう}の
\ruby[<j||]{甍}{いらか}
\原本頁{131-4}\改行%
は
\ruby{天}{そら}に
\ruby{聳}{そび}えて、
%
そこ
\ruby[g]{此處}{こ ゝ }に
\ruby{立}{た}てる
\ruby[g]{老樹}{おいき }の
\ruby[g]{銀杏}{い てふ}は、
%
まだ
\ruby{下}{お}り
\ruby{立}{た}たぬ
\ruby[g]{鳩雞}{はととり}を
\ruby{宿}{やど}して、
%
\ruby{睡}{ねむ}れるが
\ruby{如}{ごと}く
\ruby{靜}{しづか}に
\ruby{秋}{あき}の
\ruby{曙}{あした}の
\ruby{色}{いろ}を
\ruby{見}{み}せたり。

\原本頁{131-6}%
\ruby[g]{水野}{みづの }は
あはれにも
\ruby{頭}{かうべ}を
\ruby{下}{さ}げて、
%
かつて
\ruby{拜}{をが}みしことなき
\ruby{觀世音菩薩}{くわ|んぜ|おん|ぼ|さつ}を、
%
\ruby[g]{此日}{このひ }
はじめて
\ruby{涙}{なみだ}の
\ruby{眼}{め}を
\ruby{閉}{と}ぢ、
%
\ruby[g]{一心}{いつしん}に
\ruby{拜}{をが}み
\ruby[<j>]{奉}{たてまつ}りたり。

\Entry{其二十二}

% メモ 校正終了 2024-04-09 2024-05-25 2024-06-18
\原本頁{131-9}%
\ruby{我}{わ}が
\ruby{戀}{こひ}
\ruby{叶}{かな}へかしとも
\ruby{祈}{いの}らばこそ、
%
たゞ
\ruby{人}{ひと}の
\ruby{命}{いのち}の
\ruby{暴風雨}{あ|ら|し}に
\ruby{揉}{も}まるる
\ruby{芭蕉葉}{ば|せう|ば}と
\ruby{危}{あやふ}きを
\ruby{悲}{かなし}みて、
%
\ruby{只}{ひた}
\ruby{管}{すら}に
\ruby{我}{わ}が
\ruby{五十子}{い|そ|こ}
\ruby[g]{禍災}{わざはひ}
\ruby{無}{な}かれとのみ
\改行% 校正作業の簡略化のため
、
%
\原本頁{132-1}\改行%
\ruby{堪}{た}へがたき
\ruby{思}{おもひ}の
\ruby{誠}{まこと}を
\ruby{致}{いた}して、
%
\ruby[g]{他念}{た ねん}も
\ruby{無}{な}く
\ruby[g]{水野}{みづの }の
\ruby{願}{ねが}ひ
\ruby[<j>]{奉}{たてまつ}れる
\ruby{折}{をり}から、
%
\ruby[||j>]{我}{わが}
\ruby[||j>]{傍}{かたへ}にも
% \ruby{我傍}{わが|かたへ}にも
\ruby{人}{ひと}ありて、
%
\ruby[g]{先刻}{さ き }より
\ruby{普門品}{ふ|もん|ぼん}を
ほそ〴〵と
\ruby{唱}{とな}へ
\ruby{居}{ゐ}けるが、
%
\ruby{既}{すで}に
\ruby{偈}{げ}の
ところにかゝりて
\ruby{漸}{やうや}く
\ruby[<j>]{勢}{いきほひ}づき、
%
\ruby{弘誓深如海}{ぐ|ぜい|じん|によ|かい}の
\ruby{句}{く}
\原本頁{132-4}\改行%
あたりより
\ruby{嗄}{しはが}れたる
\ruby{聲}{こゑ}も
おのづから
\ruby{張}{は}り
\ruby{來}{きた}りて、
%
いま、
%
\ruby{或漂流巨海}{わく|へう|る|こ|かい}、
%
\ruby{龍魚諸鬼{\換字{難}}}{りゆう|ぎよ|しよ|き|なん}、
%
\ruby{念彼觀音力}{ねん|ぴ|くわん|のん|りき}、% 「觀音」の読みは原本通り「くわん(の)ん」
%
\ruby{波浪不能沒}{は|らう|ふ|のう|もつ}と
\ruby[g]{調子}{てうし }に
\ruby{乘}{の}りて
\ruby[g]{打誦}{うちじゆ}せるを
\ruby{見}{み}たり。

\原本頁{132-7}%
たゞ
\ruby{一}{ひ}ト
\ruby{筋}{すぢ}に
\ruby{頼}{たの}み
\ruby[<j>]{奉}{たてまつ}る
\ruby{思}{おもひ}は
\ruby{聲}{こゑ}の
\ruby{色}{いろ}にも
\ruby{現}{あらは}はれて% 原本通りのルビと送り仮名
\ruby{願}{ねが}ひ
\ruby{求}{もと}むる
\ruby{態}{さま}の
\ruby[<j>]{僞}{いつはり}ならず
\ruby{聞}{きこ}ゆるは、
%
\ruby[g]{如何}{い か }なる
\ruby[g]{苦惱}{くるしみ}のある
\ruby{人}{ひと}なるか、
%
と
\ruby{我}{わ}が
\ruby{胸}{むね}に
\ruby[g]{疼痛}{いたみ }あれば
\ruby{他}{ひと}の
\ruby{胸}{むね}の
\ruby[g]{疼痛}{いたみ }も
\ruby[g]{餘{\換字{所}}}{よ そ }ならず
\ruby{覺}{おぼ}えて
\ruby[g]{自己}{おのれ }
\ruby{念}{ねん}じ
\ruby{{\換字{終}}}{おは}りたる
\ruby[g]{水野}{みづの }は
\ruby{其}{その}
\ruby{人}{ひと}を
\ruby{見}{み}るに、
%
\ruby[g]{衣服}{な り }こそは
\ruby[g]{見苦}{み ぐる}しからね、
%
がりゝと
\ruby{痩}{や}せて
\ruby[g]{手足}{て あし}のみ
\ruby[<j>]{徒}{いたづら}に
\ruby{長}{なが}う
\ruby{見}{み}えたる、
%
\ruby{髮}{かみ}は
\ruby{既}{すで}に
\ruby{薄}{うす}くして
\ruby[g]{光澤}{つ や }
\ruby{無}{な}き
\ruby{猫}{ねこ}
\原本頁{133-1}\改行%
\ruby{毛}{げ}の
ほや〳〵と
\ruby{烟}{けむり}のやうに
\ruby{殘}{のこ}れる、
%
\ruby{脫}{ぬ}け
\ruby{上}{あが}りたる
\ruby{額}{ひたひ}の
\ruby{特}{こと}に
\ruby{廣}{ひろ}く
\改行% 校正作業の簡略化のため
、
%
\原本頁{133-2}\改行%
\ruby{下}{くだ}り
\ruby{長}{なが}き
\ruby{鼻}{はな}の
\ruby{細}{ほそ}くして
\ruby{淋}{さび}しき、
%
\ruby[g]{下作}{げ さく}には
あらねど
\ruby{甚}{いた}く
\ruby[g]{{\換字{貧}}相}{ひんさう}なる
\原本頁{133-3}\改行%
\ruby{男}{をとこ}の、
%
\ruby[g]{眉間}{み けん}に
\ruby{苦}{くる}しげなる
\ruby[g]{八字}{はちのじ}の
\ruby{皺}{しわ}を
\ruby[g]{深々}{ふか〴〵}と
\ruby{疊}{たゝ}みて、
%
\ruby{{\換字{猶}}}{なほ}
しきりに
\ruby{念彼觀音力}{ねん|ぴ|くわん|のん|りき}、% 「觀音」の読みは原本通り「くわん(の)ん」
%
\ruby{應時得{\換字{消}}散}{おう|じ|とく|せう|さん}
などゝ
\ruby{誦}{じゆ}し
つゞけたる
\ruby[g]{狀態}{ありさま}の、
%
\ruby{老}{お}いたる
\ruby{人}{ひと}だけに
\ruby[g]{愍然}{あはれ }さ
\ruby{{\換字{勝}}}{まさ}るのみならず、
%
\ruby[g]{時々}{とき〴〵}の
\ruby{聲}{こゑ}の
\ruby{曇}{くも}りて
\ruby{顫}{ふる}ふに、
%
\原本頁{133-6}\改行%
\ruby{其}{そ}の
\ruby{胸}{むね}の
\ruby{中}{うち}も
\ruby[g]{推測}{おしはか}られて
\ruby[g]{物悲}{ものがな}しく、
%
あゝ
\ruby{憂}{うれひ}を
\ruby{懷}{いだ}くものは
\ruby{我}{われ}ばかりには
あらざりけり、
%
\ruby{心}{こゝろ}の
\ruby[g]{痛苦}{くるしみ}に
\ruby{堪}{た}へかねて、
%
\ruby[g]{此人}{このひと}も
\ruby[g]{御佛}{みほとけ}を
\ruby{頼}{たの}むなるべし、
%
\ruby{妻}{つま}や
\ruby{病}{や}み
\ruby{臥}{ふ}せる、
%
\ruby{子}{こ}や
\ruby{患}{わづら}へる、
%
\ruby{或}{あるひ}は
\ruby{老}{お}いて
\ruby{子}{こ}の
\ruby{無}{な}き
\ruby{歟}{か}、
%
\ruby{子}{こ}ありて
\ruby{或}{あるひ}は
\ruby[g]{不孝}{ふ かう}なる
\ruby{歟}{か}、
%
いづれ
\ruby{悲}{かな}しき
\ruby[g]{事{\換字{情}}}{わ け }あらんと、
%
\原本頁{133-10}\改行%
そゞろに
\ruby[g]{心惹}{こゝろひ}かれて
\ruby{直}{すぐ}には
\ruby[g]{見棄}{み す }てかぬる
\ruby{思}{おもひ}したり。

\原本頁{133-11}%
\ruby[||j>]{旱}{ひでり}
\ruby[||j>]{歳}{ どし}の
\ruby[g]{冷氣}{ひ え }
\ruby{早}{はや}き
\ruby{秋}{あき}の
\ruby[g]{曉天}{あかつき}の
\ruby{事}{こと}とて、
%
\ruby{{\換字{寒}}}{さむ}きやうに
\ruby[g]{廣々}{ひろ〴〵}としたる
\ruby[g]{御堂}{み だう}の
\ruby{中}{うち}は、
%
\ruby[g]{此人}{このひと}と
\ruby{我}{われ}との
ほかに
\ruby{人}{ひと}も
\ruby{見}{み}えず、
%
\ruby{香}{かう}の
\ruby{氣}{き}
しづかに
\ruby{薫}{くん}じて
\ruby[||j>]{殊}{しゆ}
\ruby[||j>]{{\換字{勝}}}{しよう}さ
% \ruby{殊{\換字{勝}}}{しゆ|しよう}さ
\ruby{身}{み}に
\ruby{浸}{し}み
\ruby{渡}{わた}り、
%
\ruby[g]{見上}{み あ }ぐる
\ruby{眼}{め}を
\ruby{照}{て}らす
\ruby{施無畏}{せ|む|ゐ}% 無畏を施すこと。 すなわち相手に危害を加えず恐れをいだかせないこと。
の
\ruby{三大字}{さん|だい|じ}は、
% 浅草寺の扁額 1945年3月の東京大空襲で焼失
% 正面額: 豊道春海原書 南部白雲工房作〔令和2年(2020)6月奉納〕
% 観音さまは経典において、「施無畏者」とも呼ばれ、人々の不安や恐怖を取り除き、「畏れ無きを施して」下さる。
% 「施無畏」とは、観音さまのおはたらきそのものを意味する。
%
\ruby[g]{一世}{いつせ }に
\ruby{秀}{ひいで}し
\ruby{佐{\換字{文}}山}{さ|ぶん|ざん}が、
% 江戸の能書家佐々木文山が揮毫
\ruby{長櫃三個}{なが|もち|さん|さを}の
\ruby[g]{反故}{ほ ご }を
つくつて
\ruby{纔}{わづか}に
\ruby{書}{か}きしといふ
\ruby[||j>]{傳}{いひ}
\ruby[||j>]{說}{つたへ}さへ、
% \ruby{傳說}{いひ|つたへ}さへ、
%
おのづと
\ruby{想}{おも}ひ
\ruby{起}{おこ}さるゝばかり
\ruby{筆勢遒麗}{ひつ|せい|しう|れい}に、
%
\原本頁{134-5}\改行%
\ruby[||j>]{金}{きん}
\ruby[||j>]{光}{くわう}
\ruby[||j>]{美し}{ う|つく}% ルビ調整(長いルビ対策)3文字ルビが続く
く
\ruby{高}{たか}く
\ruby{懸}{かゝ}りて、
%
まことに
\ruby{人}{ひと}をして
\ruby{慈眼視衆生}{じ |げん|じ|しゆ|じやう}の% ルビ調整(原本通り)
\ruby[g]{菩薩}{ぼ さつ}の
\原本頁{134-6}\改行%
\ruby[g]{威力}{ゐ りき}を
\ruby{仰}{あふ}がんとする
\ruby{心}{こゝろ}を
\ruby{發}{おこ}さしめ、
%
\ruby[g]{たま〳〵}{}に% 踊り字に関係するところが行別れにならないように
\ruby{鳩}{はと}の
はた〳〵と
\原本頁{134-7}\改行%
\ruby{飛}{と}んでは
\ruby{靜}{しづ}かさを
\ruby{破}{やぶ}るも
\ruby{却}{かへ}つて
\ruby{寂}{さ}びて、
%
\ruby[g]{{\換字{平}}生}{ひ ごろ}の% ルビ調整(原本通り)
\ruby{賑}{にぎ}はしさに
\ruby[g]{引反}{ひきか }へて
\ruby[g]{今{\換字{朝}}}{け さ }の
\ruby{此}{こ}の
\ruby[g]{御堂}{み だう}の
\ruby[g]{神々}{かう〴〵}しく
\ruby{{\換字{尊}}}{たつと}さに、
%
\ruby[g]{水野}{みづの }は
\ruby{今}{いま}まで
\ruby{知}{し}らざりし
\ruby[g]{趣味}{おもむき}を
おぼえたり。

\原本頁{134-10}%
\ruby{妙音觀世音}{めう|おん|くわん|ぜ|おん}、
%
\ruby{梵音海潮音}{ぼん|おん|かい|てう|おん}、
%
\ruby{{\換字{勝}}彼世間音}{しよう|ひ|せ |けん|おん}、
%
と
\ruby{老}{お}いたる
\ruby{人}{ひと}の
\ruby{誦}{じゆ}する
\原本頁{134-11}\改行%
\ruby{聲}{こゑ}は、
%
いよ〳〵
\ruby[g]{眞心}{まごゝろ}
\ruby{籠}{こも}りて
\ruby{澄}{す}み
\ruby{行}{ゆ}き、
%
\ruby{普門品}{ふ|もん|ぼん}は
\ruby{今}{いま}や
\ruby{{\換字{終}}}{をは}るに
\ruby{{\換字{近}}}{ちか}からんとす。

\原本頁{135-2}%
\ruby{時}{とき}に
\ruby[g]{御堂}{み だう}の
\ruby{内}{うち}
\ruby{俄}{にはか}に
\ruby{騷}{さわ}がしく、
%
がたごとゝ
\ruby{薩{\換字{摩}}下駄}{さつ|ま|げ|た}
\ruby{踏}{ふ}み
\ruby{鳴}{な}らす
\ruby{音}{おと}を
\ruby{憚}{はゞか}り% 「憚 は(ゞ)か」
\ruby{氣}{げ}
\ruby{無}{な}く
\ruby[g]{伽藍}{が らん}に
\ruby{響}{ひゞ}かせて、
%
\ruby{太}{ふと}き〳〵
\ruby[g]{洋{\換字{杖}}}{すてつき}
もて
\ruby{益}{えき}も
\ruby{無}{な}く
\ruby[g]{床板}{ゆかいた}を
\ruby{突}{つ}きちらし
\ruby{撲}{たゝ}きちらしながら、
%
\ruby{入}{い}り
\ruby{來}{きた}れる
\ruby[g]{二人}{ふたり }の
\ruby[g]{書生}{しよせい}あり。
%
\原本頁{135-5}\改行%
\ruby{醉}{ゑひ}を% 「醉」は原本通り「ゑ」で調整
\ruby{帶}{お}びたりとは
\ruby{見}{み}えねど
\ruby[g]{反響}{こ だま}の
\ruby{起}{おこ}るほどの
\ruby[g]{馬鹿}{ば か }
\ruby{聲}{ごゑ}を
あげて、

\原本頁{135-6}%
『
ハツ、
オイ、
%
まだ
\ruby[g]{此樣}{こ ん }なものを
\ruby[g]{本氣}{ほんき }で
\ruby[g]{禮拜}{らいはい}して
\ruby{居}{ゐ}るものがあるぜ!。
』

\原本頁{135-8}%
と、
%
\ruby[||j>]{紺}{こん}
\ruby[||j>]{絞}{しぼり}の
% \ruby{紺絞}{こん|しぼり}の
\ruby{兵兒帶}{へ|こ|おび}を
\ruby{締}{し}めたるが
\ruby{云}{い}へば、

\原本頁{135-9}%
『
ウン、
%
\ruby[g]{可愍}{ふ びん}なものさ、
%
\ruby{五六世紀}{ご|ろく|せい|き}も
\ruby{{\換字{前}}}{まへ}の
\ruby[g]{思想}{し さう}に
\ruby{養}{やしな}はれて
\ruby{居}{ゐ}るのだからナ。
』

\原本頁{135-11}%
と、
%
\ruby{白金巾}{しろ|がな|きん}の
% ( カナキンは[ポルトガル語] canequim 「しろガナキン」とも ) 白色のカナキン。 目の細かい薄地の綿布で白色のもの。
%  canequim → カナキン → 金巾
% 固くよった綿糸で織った薄い織物のこと
\ruby{帶}{おび}したるが
\ruby{答}{こた}へたり。

\原本頁{136-1}%
『
\ruby[g]{我輩}{わがはい}の
\ruby[g]{親{\換字{分}}}{おやぶん}は、
%
\ruby[g]{基督}{くりすと}が
\ruby[g]{代表}{だいへう}した
\ruby[g]{馬鹿}{ば か }
\ruby[g]{思想}{し さう}を
\ruby[g]{奴隷}{ど れい}
\ruby[g]{{\換字{道}}徳}{だうとく}と
\ruby{罵}{のゝし}つたが、
%
\ruby[g]{我輩}{わがはい}は
\ruby{法然日蓮}{はふ|ねん|にち|れん}の% 「蓮 uf999」(参考「蓮 u84ee」)
\ruby[g]{代表}{だいへう}した
\ruby[g]{馬鹿}{ば か }
\ruby[g]{思想}{し さう}を
\ruby[g]{乞食}{こ じき}
\ruby[g]{{\換字{道}}徳}{だうとく}と
\ruby[g]{斷言}{だんげん}するが、
%
\ruby[g]{何樣}{ど う }だ、
%
\ruby{可}{よ}からう。
』

\原本頁{136-4}%
『
ウン、
%
\ruby{偉}{えら}い!。
%
\ruby[g]{釋迦}{しやか }が
\ruby[g]{事實}{じ ゞつ}
\ruby[||j>]{上}{じやう}
\ruby[||j>]{乞食}{ こ|じき}だから
\ruby{{\換字{猶}}}{なほ}
\ruby[g]{可笑}{を か }しい。
%
それだのに
%
\ruby{木佛金佛}{き|ぶつ|かな|ぶつ}を
\ruby{拜}{をが}む
\ruby{奴}{やつ}さへ
あるのだからナ。
%
ほんとに
\ruby{本能主義}{ほん|のう|しゆ|ぎ}
の
\ruby{有}{あ}り
\ruby{{\換字{難}}}{がた}い、
%
\ruby{大}{おほ}もての
\ruby{美的境界}{び|てき|きやう|がい}でも
\ruby{敎}{をし}へて
\ruby{{\換字{遣}}}{や}りたいナ。
%
ハヽヽ
\改行% 校正作業の簡略化のため
ハヽ。
』

\原本頁{136-8}%
『
ヤ、
%
\ruby{酷}{ひど}いところで
\ruby[g]{自惚}{のろけ }る
\ruby{奴}{やつ}だナ。
%
ハヽハヽ。
』

\原本頁{136-9}%
\ruby{十八間四面}{じふ|はつ|けん|し|めん}の
\ruby[g]{御堂}{み だう}も
\ruby{動}{ゆら}ぐばかりに
\ruby[g]{高笑}{たかわら}ひして、
%
\ruby{繪}{ゑ}に
\ruby{見}{み}る
\ruby[g]{惡鬼}{あくき }
\ruby[g]{羅刹}{ら せつ}が
\ruby{持}{も}てる
\ruby[||j>]{銕}{てつ}
\ruby[||j>]{{\換字{杖}}}{ぢやう}の
% \ruby{銕{\換字{杖}}}{てつ|ぢやう}の
\ruby{如}{ごと}き
\ruby{恐}{おそ}ろしき
\ruby{重}{おも}げなる
\ruby{{\換字{杖}}}{つゑ}もて、
%
\ruby{我}{わ}が
\ruby{踏}{ふ}める
\ruby{床}{ゆか}を、
%
\ruby{我}{わ}が
\ruby[g]{威風}{ゐ ふう}を
\ruby{見}{み}よと
ばかりに、
%
どしんと
\ruby{突}{つ}きたり。

\原本頁{137-1}%
\ruby{皆發無等々}{かい|ほつ|む |とう|〴〵}
\ruby{阿耨多羅三藐三菩提心}{あ |のく|た |ら |さん|みやく|さん|ぼ|だい|しん}と、
%
\ruby{念}{ねん}じ
\ruby{{\換字{終}}}{をは}りて
\ruby[g]{禮拜}{らいはい}し
\ruby{濟}{すま}したる
\ruby{老}{お}いたる
\ruby{男}{をとこ}は、
%
\ruby{頭}{かうべ}を
\ruby{擡}{もた}げて
\ruby[g]{水野}{みづの }と
\ruby{顏}{かほ}
\ruby[g]{見合}{み あ }はせて、
%
おもはず
\ruby[<j||]{互}{たがひ}に% 行末行頭の境界付近なので特例処置を施す
\ruby{眉}{まゆ}を
\ruby{顰}{ひそ}めざるを
\ruby{得}{え}ざりき。

\Entry{其二十三}

\ruby{天}{そら}の
\ruby{彼方}{あな|た}に
\ruby{颶風}{つむじ|かぜ}を
\ruby{起}{おこ}しゝニイチエが
\ruby{眞趣}{おも|むき}を
\ruby{實}{まこと}に
\ruby{知}{し}れりや、それも
\ruby{覺束無}{おぼ|つか|な}げなる
\ruby{書生}{しよ|せい}の
\ruby{放言}{はう|げん}の、
\ruby{餘}{あま}りの
\ruby{事}{こと}に
\ruby{傍痛}{かたは|らいた}くはおぼえたれど、
\ruby{意}{こゝろ}を
\ruby{動}{うご}かすほどにも
\ruby{至}{いた}らざりければ、
\ruby{他}{ひと}は
\ruby{他}{ひと}なり、
\ruby{我}{われ}は
\ruby{我}{われ}なり、
\ruby{關係無}{かけ|かまひ|な}き
\ruby{禽}{とり}の
\ruby{聲}{こゑ}の、それまでの
\ruby{事}{こと}なりと
\ruby{聞}{き}き
\ruby{捨}{す}てゝ、
\ruby{既}{すで}に『ツアラツウストラ
\ruby{如是說}{によ|ぜ|せつ}』をも
\ruby{窺}{うかゞ}ひ
\ruby{讀}{よ}まぬにあらざりし
\ruby{水野}{みづ|の}は、\換字{志}ろりと
\ruby{冷}{ひや}やかに
\ruby{彼}{か}の
\ruby{二人}{ふた|り}をば
\ruby{一瞥}{いち|べつ}せしのみに
\ruby{止}{とゞ}まりて、
\ruby{徐々}{やお|ら}
\ruby{此處}{こ|ゝ}を
\ruby{去}{さ}らんと
\ruby{歩}{あゆ}み
\ruby{出}{いだ}せば、
\ruby{彼}{か}の
\ruby{老}{お}いたる
\ruby{男}{をとこ}も
\ruby{一齊}{と|も}にと
\ruby{隨}{したが}へり。

\ruby{世}{よ}の
\ruby{態人}{さま|ひと}の
\ruby[<h||]{{\換字{情}}}{こゝろ}
\ruby[||h>]{漸}{やうや}く
\ruby{移}{うつ}りて、
\ruby{礎}{いしずゑ}は
\ruby{舊}{きう}に
\ruby{依}{よ}りて
\ruby{固}{かた}く、
\ruby{棟}{むね}は
\ruby{舊}{きう}に
\ruby{依}{よ}りて
\ruby{高}{たか}けれども、
\ruby{今}{いま}は
\ruby{此}{こ}の
\ruby{莊嚴}{さう|ごん}なる
\ruby{御堂}{み|だう}の
\ruby{内}{うち}にさへも、
\ruby{謗法}{はう|ばふ}
\ruby{毀佛}{き|ぶつ}の
\ruby{暴}{あば}れ
\ruby{聲起}{ごゑ|おこ}りて、
\ruby{譬喩}{たと|へ}を
\ruby{取}{と}りて
\ruby{云}{い}はゞ
\ruby{月黑}{つき|くろ}き
\ruby{夜}{よ}の
\ruby{大潮}{おほ|しほ}の、
\ruby{洲}{す}を
\ruby{吞}{の}み
\ruby{岩}{いは}を
\ruby{嚙}{か}みて
\ruby{漸}{やうや}く
\ruby{大地}{だい|ち}を
\ruby{犯}{をか}さんとするが
\ruby{如}{ごと}くに、
\ruby{何時}{い|つ}となく
\ruby{破壞}{は|くわい}の
\ruby{吶喊}{さけ|び}の
\ruby{押寄}{おし|よ}するは、
\ruby{{\換字{所}}謂}{いは|ゆる}
\ruby{末法澆季}{まつ|ぱふ|げう|き}の
\ruby{是非}{ぜ|ひ}も
\ruby{無}{な}き
\ruby{當時}{い|ま}の
\ruby{大勢}{いき|ほい}なり。
\ruby{書生}{しよ|せい}は
\ruby{{\換字{猶}}}{なほ}がたりごとりと、
\ruby{力}{ちから}
\ruby{足}{あし}を
\ruby{踏}{ふ}み
\ruby{{\換字{杖}}}{つゑ}を
\ruby{突}{つ}き
\ruby{立}{た}てゝて
\ruby{歩}{ある}き
\ruby{居}{ゐ}しが、
\ruby{紺絞}{こん|しぼり}の
\ruby{帶}{おび}したるは、
\ruby{急}{きふ}に
\ruby{虛{\換字{空}}}{こ|くう}に
\ruby{{\換字{杖}}}{つゑ}を
\ruby{擧}{あ}げて、
\ruby{揭}{かゝ}げられたる
\ruby{額}{がく}の
\ruby{一}{ひと}つを
\ruby{指}{さ}しながら、

『
\ruby{面白}{おも|しろ}いナア、
\ruby{此}{こ}の
\ruby{一}{ひと}つ
\ruby{家}{や}の
\ruby{畫}{ゑ}は!、どうも
\ruby{巧}{よ}く
\ruby{出來}{で|き}て
\ruby{居}{ゐ}るナ、
\ruby{氣}{き}に
\ruby{入}{い}つたナア!。
』

と
\ruby{云}{い}へば、

『ムヽ、』

と、
\ruby{白}{しろ}き
\ruby{帶}{おび}したるは
\ruby{其意}{その|い}を
\ruby{得}{え}ぬげに
\ruby{應}{こた}へつ、

『\換字{志}かし
\ruby{御厩}{み|うまや}の
\ruby{喜三太}{き|さん|だ}も
\ruby{好}{い}いぢやあ
\ruby{無}{な}いか。
』

と
\ruby{附加}{つけ|くは}へたり。

『
\ruby{馬鹿}{ば|か}ツ!。
そりやあ
\ruby{{\換字{技}}{\換字{術}}}{ぎじ|ゆつ}だけの
\ruby{論}{ろん}だ。
\ruby{云}{い}ふなあ
\ruby{其處}{そ|こ}ぢやあ
\ruby{無}{な}い。
よく
\ruby{見}{み}ろ!
\ruby{吾輩}{わが|はい}の
\ruby{此}{こ}の
\ruby{一}{ひと}つ
\ruby{家}{や}の
\ruby{圖}{づ}を!。
\ruby{何樣}{ど|う}だ
\ruby{彼}{あ}の
\ruby{婆}{ばあ}さんの
\ruby{顏}{かほ}の
\ruby{立派}{りつ|ぱ}なこと!。
\ruby{實}{じつ}に
\ruby{立派}{りつ|ぱ}ぢやあ
\ruby{無}{な}いか、
\ruby{立派}{りつ|ぱ}ぢやあ
\ruby{無}{な}いか!。
\ruby{國家}{こく|か}の
\ruby{法律}{はふ|りつ}なんぞといふ
\ruby{奴}{やつ}ア
\ruby{踏}{ふ}み
\ruby{付}{つ}け
\ruby{切}{き}つた
\ruby{彼}{あ}の
\ruby{顏}{かほ}つき!。
\ruby{世間}{せ|けん}の
\ruby{善惡}{ぜん|あく}の
\ruby{沙汰}{さ|た}なんぞを
\ruby{寄}{よ}せつけも
\ruby{仕無}{し|な}い
\ruby{彼}{あ}の
\ruby{顏付}{かほ|つき}!。
\ruby{戀}{こひ}も
\ruby{人{\換字{情}}}{にん|じやう}も
\ruby{無}{な}い
\ruby{彼}{あ}の
\ruby{顏}{かほ}つき!。
\ruby{邪}{じや}でも
\ruby{非}{ひ}でもかまはない
\ruby{彼}{あ}の
\ruby{顏}{かほ}つき!。
おれが
\ruby{{\換字{勝}}手}{かつ|て}だぞといふ
\ruby{彼}{あ}の
\ruby{顏}{かほ}つき!。
\ruby{神}{かみ}でも
\ruby{佛}{ほとけ}でも
\ruby{對面}{むか|ふ}へまはつたら
\ruby{斫殺}{たゝ|つき}つて
\ruby{{\換字{遣}}}{や}らうといふ
\ruby{彼}{あ}の
\ruby{顏}{かほ}つき!。
あゝ
\ruby{何}{なん}と
\ruby{立派}{りつ|ぱ}な
\ruby{顏}{かほ}に
\ruby{書}{か}いてあるでは
\ruby{無}{な}いか。
\ruby{十{\換字{分}}}{じう|ぶん}に
\ruby{惡人}{あく|にん}の
\ruby{偉大}{ゐ|だい}な
\ruby{精神}{せい|しん}が
\ruby{發揮}{はつ|き}してある!。
\ruby{誰}{たれ}だつて
\ruby{此}{こ}の
\ruby{繪}{ゑ}を
\ruby{能}{よ}く
\ruby{見}{み}たらば、
\ruby{{\換字{強}}惡}{がう|あく}が
\ruby{美}{い}いものだといふ
\ruby{事}{こと}に
\ruby{氣}{き}が
\ruby{付}{つ}くだらう!。
\ruby{見}{み}ろ、
\ruby{彼}{あ}の
\ruby{娘}{むすめ}が
\ruby{卑小}{け|ち}な
\ruby{惡}{わる}びれた
\ruby{樣子}{やう|す}を!。
\ruby{人}{ひと}に
\ruby{縋}{すが}りたがるやうな、
\ruby{哀愍}{あは|れみ}を
\ruby{乞}{こ}うやうな、
\ruby{泣}{な}き
\ruby{出}{だ}しさうな、
\ruby{切}{せつ}なさうな、
\ruby{善惡}{ぜん|あく}の
\ruby{{\換字{道}}理}{だう|り}を
\ruby{怖}{こは}がつて
\ruby{居}{ゐ}るやうな、
\ruby{國家}{こく|か}の
\ruby{規律}{おき|て}なんぞにびく〳〵して
\ruby{居}{ゐ}るやうな、
\ruby{神佛}{しん|ぶつ}なんぞにおど〳〵して
\ruby{居}{ゐ}る、\換字{志}みつたれた、
\ruby{見}{み}つとも
\ruby{無}{な}い
\ruby{醜態}{ざ|ま}が、すつかり
\ruby{見}{み}えて
\ruby{居}{ゐ}る!。
\ruby{{\換字{所}}謂}{いは|ゆる}
\ruby{善人}{ぜん|にん}といふ
\ruby{奴}{やつ}が
\ruby{卑劣}{け|ち}なもので、
\ruby{下}{くだ}らないものだといふ
\ruby{事}{こと}は、
\ruby{何樣}{ど|ん}な
\ruby{馬鹿}{ば|か}な
\ruby{奴}{やつ}の
\ruby{眼}{め}にも
\ruby{暎}{うつ}るだらう!。% 草冠付きの「暎」
\ruby{何樣}{ど|う}だ、
\ruby{好}{い}いぢやあ
\ruby{無}{な}いか
\ruby{好}{い}い
\ruby{畫}{ゑ}ぢやあ
\ruby{無}{な}いか。
、
\ruby{何樣}{ど|う}だ、
\ruby{{\換字{分}}}{わ}かつたか、
\ruby{好}{い}いか、オイ、
\ruby{君}{きみ}!。
\ruby{此}{こ}の
\ruby{一}{ひと}つ
\ruby{家}{や}の
\ruby{御婆}{お|ばあ}さんが
\ruby{國王}{こく|わう}になりやあ、
\ruby{世界中}{せ|かい|ぢゆう}を
\ruby{斬}{き}り
\ruby{伏}{ふ}せて
\ruby{寢酒}{ね|ざけ}の
\ruby{下物}{さか|な}に
\ruby{仕}{し}て
\ruby{{\換字{遣}}}{や}らうと、
\ruby{手}{て}に
\ruby{持}{も}つた
\ruby{利器}{え|もの}を
\ruby{振}{ふ}り
\ruby{舞}{ま}はすんだ!。
もし
\ruby{此}{こ}の
\ruby{娘}{むすめ}が
\ruby{國王}{こく|わう}になりやあ、
\ruby{彼方}{あつ|ち}へも
\ruby{此方}{こつ|ち}へも
\ruby{氣}{き}がねを
\ruby{仕}{し}て、
\ruby{一年中}{いち|ねん|ぢゆう}べそをかいて
\ruby{居}{ゐ}なけりやあならないんだ!。
\ruby{何樣}{ど|う}だ、
\ruby{{\換字{強}}惡}{がう|あく}に
\ruby{限}{かぎ}るだらう!。
\ruby{一體}{いつ|たい}
\ruby{眞實}{ほん|たう}の
\ruby{理屈}{り|くつ}から
\ruby{云}{い}やあ、
\ruby{此}{こ}の
\ruby{娘}{むすめ}の
\ruby{方}{はう}が
\ruby{惡}{あく}で
\ruby{婆}{ばあ}さんの
\ruby{方}{はう}が
\ruby{善}{ぜん}なのだからナア。
』

『ウン、
\ruby{成程}{なる|ほど}
\g詰めruby{々々}{〳〵}。
\ruby{{\換字{強}}惡}{がう|あく}は
\ruby{眞實}{ほん|と}に
\ruby{偉}{えら}いナア!。
だけれど
\ruby{憫然}{かは|いさう}に% 「憫然 か(は)いさう」
\ruby{今}{いま}の
\ruby{世界}{せ|かい}ぢやあ、
\ruby{男子}{をの|こ}でも
\ruby{此}{こ}の
\ruby{娘}{むすめ}のやうな
\ruby{奴}{やつ}ばかり
\ruby{多}{おほ}いぜ!。
ハヽヽ。
』

『ハヽハヽヽ、
\ruby{左樣}{さ|う}だ、〳〵、
\ruby{笑}{わら}つて
\ruby{{\換字{遣}}}{や}れ、
\ruby{笑}{わら}つて
\ruby{{\換字{遣}}}{や}れ。
アツハツハツハヽヽヽ。
』

『アツハツハツハヽヽヽ。
』

\ruby{{\換字{朝}}詣}{あさ|まゐ}りする
\ruby{人}{ひと}のちらほらとは
\ruby{見}{み}え
\ruby{初}{そ}めたれど、
\ruby{{\換字{猶}}}{なほ}
\ruby{極}{きは}めて
\ruby{四邊}{あた|り}の
\ruby{物靜}{もの|しづ}かなれば、
\ruby{聞}{き}けよがしに
\ruby{聲}{こゑ}
\ruby{大}{おほき}く
\ruby{語}{かた}らふ
\ruby{二人}{ふた|り}の
\ruby{談}{はなし}は、
\ruby{既}{すで}に
\ruby{御堂}{み|だう}を
\ruby{離}{はな}れて
\ruby{石路}{せき|ろ}を
\ruby{歩}{あゆ}める
\ruby{水野}{みづ|の}と
\ruby{彼}{か}の
\ruby{老}{お}いたる
\ruby{男}{をとこ}との
\ruby{背後}{うし|ろ}より
\ruby{響}{ひゞ}きて、
\ruby{態}{わざ}とらしき
\ruby{其}{そ}の
\ruby{嘲}{あざけ}り
\ruby{笑}{わら}ひも
\ruby{一々}{いち|〳〵}
\ruby{聞}{きこ}こえたり。
\ruby{今}{いま}しも
\ruby{水野}{みづ|の}と
\ruby{並}{なら}びて
\ruby{歩}{ある}ける
\ruby{彼}{か}の
\ruby{男}{をとこ}は
\ruby{再}{ふたゝ}び
\ruby{水野}{みづ|の}と
\ruby{面}{おもて}を
\ruby{見合}{み|あ}はせつ、
\ruby{{\換字{終}}}{つひ}に
\ruby{堪}{た}へ
\ruby{{\換字{兼}}}{か}ねてか
\ruby{口}{くち}を
\ruby{開}{ひら}き、

『
\ruby{大變}{たい|へん}な
\ruby{世}{よ}の
\ruby{中}{なか}になつてまゐりました!。
\ruby{私共}{わたし|ども}の
\ruby{倅}{せがれ}なんぞも
\ruby{學校}{がく|かう}へ
\ruby{{\換字{遣}}}{や}つて
\ruby{置}{お}きましたら、まあ
\ruby{矢張}{や|は}り
\ruby{彼樣}{あ|ゝ}いつた
\ruby{調子}{てう|し}になりまして、
\ruby{人}{ひと}に
\ruby{苦勞}{く|らう}ばかりいたさせます。
\ruby{御参}{お|まゐり}を
\ruby{致}{いた}しますのも、
\ruby{實}{じつ}を
\ruby{申}{まを}しますと、つまりは
\ruby{其樣}{そ|ん}な
\ruby{譯}{わけ}から
\ruby{起}{おこ}つた
\ruby{事}{こと}のためでございますが、……』

と、
\ruby{思}{おも}ひ
\ruby{餘}{あま}つたる
\ruby{憂}{う}さを
\ruby{漏}{も}らしかけしが、
\ruby{流石}{さす|が}に
\ruby{心}{こゝろ}づきて、
\ruby{馴染}{な|じみ}
\ruby{無}{な}き
\ruby{人}{ひと}に
\ruby{吾}{わ}が
\ruby{家内}{い|へ}の
\ruby{事}{こと}を
\ruby{言}{い}はんもはしたなしとてや、

『
\ruby{御利生}{ご|り|しやう}を
\ruby{現}{あら}はさうとして
\ruby{書}{か}きました
\ruby{額}{がく}を
\ruby{見}{み}て、
\ruby{一}{ひと}つ
\ruby{家}{や}の
\ruby{婆}{ばあ}さんの
\ruby{方}{はう}を
\ruby{褒}{ほ}めますなんて、ほんに
\ruby{淺草寺}{せん|さう|じ}はじまつて
\ruby{以來無}{この|かた|な}い
\ruby{事}{こと}でございましやう!。
まあ
\ruby{何}{なん}といふ
\ruby{間{\換字{違}}}{ま|ちが}つた
\ruby{事}{こと}で!。
』

と、
\ruby{談}{はなし}を
\ruby{横}{よこ}に
\ruby{逸}{そ}らしたり。
\ruby{水野}{みづ|の}は
\ruby{當}{あ}たり
\ruby{障}{さは}らずに、

『まことに
\ruby{左樣}{さ|やう}でござります。
』

と、
\ruby{穩}{おだ}やかに
\ruby{答}{こた}へて
\ruby{多}{おほ}くは
\ruby{言}{ものい}はず、たゞ
\ruby{人}{ひと}の
\ruby{親}{おや}には
\ruby[<h||]{{\換字{情}}}{なさけ}
\ruby{篤}{あつ}きが
\ruby{多}{おほ}きに、
\ruby{人}{ひと}の
\ruby{子}{こ}にはまた
\ruby{彼等}{かれ|ら}
\ruby{二人}{ふた|り}の
\ruby{如}{ごと}く
\ruby{心放縱}{こゝろ|ほしい|まゝ}なるが
\ruby{多}{おほ}き
\ruby{世}{よ}の
\ruby{相}{すがた}の、さま〴〵なるを
\ruby{思}{おも}ひて
\ruby{歎}{たん}じながらも、
\ruby{今}{いま}の
\ruby{書生}{しよ|せい}の
\ruby{笑}{わら}ひ
\ruby{聲}{ごゑ}には、
\ruby{少}{すくな}からず
\ruby{不快}{ふ|くわい}を
\ruby{覺}{おぼ}えたり。

\ruby{自}{みづか}ら
\ruby{知}{し}る
\ruby{我}{わ}が
\ruby{昨夕}{ゆふ|べ}のありさまは、
\ruby{取}{と}りも
\ruby{直}{なほ}さず
\ruby{旅}{たび}の
\ruby{人}{ひと}を
\ruby{護}{かば}へる
\ruby{彼}{か}の
\ruby{娘}{むすめ}にも
\ruby{似}{に}て、
\ruby{病}{や}める
\ruby{五十子}{い|そ|こ}を
\ruby{恤}{いたは}らんがためとて、
\ruby{一}{ひと}つ
\ruby{家}{や}の
\ruby{婆}{ばゞ}にも
\ruby{似}{に}たらん
\ruby{彼}{か}の
お
\ruby{澤}{さは}
\ruby{婆}{ばゞあ}に、
\ruby{下}{さ}げがたき
\ruby{頭}{かしら}を
\ruby{幾度}{いく|たび}も
\ruby{益無}{えき|な}く
\ruby{下}{さ}げて、\換字{志}かも
\ruby{益無}{えき|な}く
\ruby{云}{い}ひ
\ruby{斥}{しりぞ}けられたる
\ruby{其事}{その|こと}の
\ruby{今}{いま}さら
\ruby{胸}{むね}に
\ruby{{\換字{浮}}}{うか}み
\ruby{來}{く}れば、
\ruby{當無}{あて|な}く
\ruby{放}{はな}ちたるには
\ruby{疑}{うたが}ひ
\ruby{無}{な}き
\ruby{嘲笑}{わら|ひ}の
\ruby{矢}{や}も、\換字{志}たゝかに
\ruby{我}{わ}が
\ruby{背}{そびら}に
\ruby{立}{た}てる
\ruby{心地}{こゝ|ち}して、
\ruby{厭}{いと}はしき
\ruby{思}{おもひ}の
\ruby{比}{たと}ふるに
\ruby{物無}{もの|な}く、
\ruby{身}{み}の
\ruby{内}{うち}を
\ruby{掻}{か}き
\ruby{挘}{むし}りたきやうなる
\ruby{感}{かん}じを
\ruby{懷}{いだ}きつゝ、
\ruby{夢路}{ゆめ|ぢ}を
\ruby{辿}{たど}るが
\ruby{如}{ごと}く
\ruby{中店}{なか|みせ}を
\ruby{出}{で}はづるれば、

『ヤ、
\ruby{水野}{みづ|の}さん。
』

と、
\ruby{凉}{すゞ}しき
\ruby{聲}{こゑ}の
\ruby{玉}{たま}を
\ruby{轉}{まろ}ばすが
\ruby{如}{ごと}くに
\ruby{呼}{よ}びかけて、
\ruby{黑革}{くろ|かは}の
\ruby{眉庇付}{ま|びさし|つ}きたる
\ruby{帽}{ばう}を
\ruby{傾}{かたぶ}けつゝ、
\ruby{身}{み}を
\ruby{{\換字{前}}}{まへ}
\ruby{屈}{かゞ}みにして
\ruby{走}{はし}り
\ruby{來}{きた}れる
\ruby{美少年}{び|せう|ねん}あり。
\ruby{彼}{か}の
\ruby{老}{お}いたる
\ruby{男}{をとこ}は
\ruby{既}{すで}に
\ruby{去}{さ}つて
\ruby{在}{あ}らず。

\Entry{其二十四}

% メモ 校正終了 2024-04-09
\原本頁{145-7}%
\ruby{{\換字{近}}}{ちか}づくや
\ruby{否}{いな}や
\ruby{帽}{ばう}を
\ruby{脫}{と}りて、
%
\ruby{眞{\換字{率}}}{しん|そつ}に
\ruby{頭}{かうべ}を
\ruby{下}{さ}げて
\ruby{挨拶}{あい|さつ}するは、
%
\ruby{林檎}{りん|ご}の
\ruby{如}{ごと}く
\ruby{美}{うつく}しき
\ruby{色澤}{いろ|つや}、
%
\ruby{人形}{にん|ぎやう}の
\ruby{如}{ごと}き
\ruby[g]{端正}{たゞ}しき
\ruby{眼鼻立}{め|はな|だち}、
%
\ruby{姊}{あね}の
\ruby{男}{をとこ}にしても
\ruby{見}{み}まはしく
\ruby{立派}{りつ|ぱ}なるには
\ruby{異}{かは}りて、
%
\ruby{此}{これ}は
\ruby{女}{をんな}にしても
\ruby{見}{み}たく
\ruby{可愛}{か|はい}らしと
\ruby{人}{ひと}に
\ruby{云}{い}はれたる
\ruby[g]{五十子}{いそこ}が
\ruby{弟}{おとゝ}の
\ruby[g]{松之助}{まつのすけ}なり。

\原本頁{146-1}%
\ruby{母}{はゝ}は
\ruby{有}{あ}りても
\ruby{繼}{まゝ}しき
\ruby{中}{なか}なり、
%
\ruby{財產}{ざい|さん}は
\ruby{繼母}{は|ゝ}に
\ruby{皆}{みな}
\ruby{奪}{と}られたり、
%
\ruby{姊}{あね}より
ほかに
\ruby{頼}{たの}むべき
\ruby{人}{ひと}を
\ruby{有}{も}たぬ
\ruby[g]{松之助}{まつのすけ}は、
%
\ruby{往時}{むか|し}の
\ruby{{\換字{乳}}母}{う|ば}なりしが
\ruby{今}{いま}は
\ruby{下谷}{した|や}の
\ruby{廣小路}{ひろ|こう|ぢ}
\ruby{{\換字{近}}}{ちか}くに、
%
\ruby{下梳}{した|すき}の
\ruby{二人}{ふた|り}も
\ruby{使}{つか}ふほどの
\ruby{女髮結}{か|み|ゆひ}となりて、
%
\ruby{堅}{かた}く
\ruby{身}{み}を
\ruby{持}{も}てる
\ruby{幸福}{しあ|はせ}には%「幸福」ここは「は」
\ruby{苦}{くる}しげ
\ruby{無}{な}く
\ruby{日}{ひ}を
\ruby{{\換字{送}}}{おく}れるが
\ruby{許}{もと}に、
%
\ruby{{\換字{留}}守番}{る|す|ばん}を
\ruby{{\換字{兼}}}{か}ねたる
\ruby{客寓人}{かゝ|り|びと}となりつ、
%
\ruby{月々}{つき|〴〵}
\ruby{姊}{あね}が
\ruby{取}{と}る
\ruby{僅少}{わづ|か}なる
\ruby{給料}{きふ|れう}の
\原本頁{146-6}\改行%
\ruby{内}{うち}より、
%
\ruby{{\換字{分}}}{わ}けて
\ruby{貰}{もら}ふ
\ruby{財布}{さい|ふ}の
\ruby{塵芥}{ご|み}ほどの
\ruby[g]{金子}{かね}を、
%
\ruby{一{\換字{半}}}{なか|ば}は
\ruby{形式}{か|た}ばかりの
\ruby{食料}{しよく|れう}として
\ruby{入}{い}れ、
%
\ruby{一{\換字{半}}}{なか|ば}は
おのれの
\ruby{學資}{がく|し}として、
%
\ruby{責}{せ}めて
\ruby{某}{それ}の
\原本頁{146-8}\改行%
\ruby{學校}{がく|かう}の
\ruby{官費生}{くわん|ぴ|せい}となりて
\ruby{世}{よ}に
\ruby{立}{た}つ
\ruby{{\換字{道}}}{みち}の
\ruby{緖}{いとぐち}を
\ruby{得}{う}る
\ruby{迄}{まで}と、
%
\ruby{足}{た}らぬ
\ruby{{\換字{勝}}}{がち}なる
\ruby{中}{なか}にも
\ruby{心}{こゝろ}を
\ruby{勵}{はげ}まして、
%
\ruby{夜學}{や|がく}の
\ruby{歸路}{かへ|り}は
\ruby{辛}{つら}き
\ruby{{\換字{冬}}}{ふゆ}の
\ruby{{\換字{雪}}}{ゆき}、
%
\ruby{籠}{こも}り
\ruby{居}{ゐ}の
\ruby{夏}{なつ}は
\ruby{堪}{た}へ
\ruby{{\換字{難}}}{がた}き
\ruby{陋巷}{ろ|じ}の
\ruby{奧}{おく}の
\ruby{矮屋}{こ|いへ}の
\ruby{暑熱}{あつ|さ}にも、
%
\ruby{萎}{め}げず
\ruby{怯}{ひる}まずして
\ruby{勉{\換字{強}}}{べん|きやう}すれば、
%
\ruby{齡}{とし}は
\ruby{{\換字{猶}}}{なほ}
\ruby{數}{かぞ}へ
\ruby{年}{どし}の
\ruby{一七}{じう|しち}にして、
%
\ruby{思想}{かん|がへ}こそは
\ruby{世}{よ}に
\ruby{磨}{す}れざれ、
%
\原本頁{147-1}\改行%
\ruby{學問}{がく|もん}の
\ruby{出來}{で|き}は
いと
\ruby{佳}{よ}くして、
%
\ruby{行末}{ゆく|すゑ}
\ruby{發{\換字{達}}}{なり|い}づべく
\ruby{見}{み}ゆる
\ruby{少年}{せう|ねん}なり。
%
\原本頁{147-2}\改行%
\ruby{繼母}{は|ゝ}は
\ruby{不品行}{ふ|み|もち}にして
\ruby{心}{こゝろ}
\ruby{曲}{ゆが}み、
%
\ruby{有}{あ}りても
\ruby{却}{かへ}つて
\ruby{無}{な}きに
\ruby{劣}{おと}れば、
%
\ruby{天}{てん}にも
\ruby{地}{ち}にも
\ruby{頼}{たの}み
\ruby{頼}{たの}まるべきは
\ruby{只}{たゞ}
\ruby{姊弟}{きよう|だい}と、
%
\ruby{深}{ふか}くも
\ruby{此}{こ}の
\ruby{弟}{おとゝ}の
\ruby{上}{うへ}を
のみ
\ruby{思}{おも}ひて、
%
\ruby{自己}{おの|れ}の
\ruby{今}{いま}の
\ruby{身}{み}は
\ruby{差}{さ}し
\ruby{當}{あた}りて
\ruby{田舎}{ゐな|か}の
\ruby{草萊}{く|さ}の
\ruby{間}{あひだ}に
\ruby{埋}{うづ}もれ
\原本頁{147-5}\改行%
\ruby{沒}{かく}るゝとも、
%
\ruby{如何}{い|か}にもして
\ruby{弟}{おとゝ}の
\ruby{{\換字{若}}}{わか}き
\ruby{時}{とき}を
\ruby{徒}{あだ}に
\ruby{{\換字{過}}}{すご}さしめず、
%
\ruby{出來}{で|き}ぬながらも
\ruby{人}{ひと}の
\ruby{後}{のち}に
\ruby{落}{お}ちぬほどには
\ruby{物學}{もの|まな}びを
させて、
%
\ruby{男兒}{をと|こ}
\ruby{一人{\換字{前}}}{いち|にん|まへ}には
\ruby{生}{おふ}し
\ruby{立}{た}て、
%
\ruby{我}{わ}が
\ruby{家}{いへ}の
\ruby{名}{な}をも
\ruby{擧}{あ}げさせん、
%
\ruby{弟}{おとゝ}のためには
\ruby{挿}{さ}したる
\ruby{掻頭}{かん|ざし}を
\ruby{賣}{う}り、
%
\ruby{着}{き}たる
\ruby{衣}{もの}を
\ruby{脫}{ぬ}ぐとも
\ruby{惜}{をし}まじとは、
%
\ruby[g]{五十子}{いそこ}が
\ruby{日頃}{ひ|ごろ}の
\ruby{念慮}{おも|ひ}なりき。

\原本頁{147-10}%
\ruby{秋風}{あき|かぜ}の
\ruby{中}{なか}に
\ruby{嬰兒}{あか|ご}の
\ruby{泣}{な}きても、
%
\ruby{拾}{ひろ}ふ
\ruby{人}{ひと}は
\ruby{少}{すくな}き
\ruby{此}{こ}の
\ruby{冷}{つめた}き
\ruby{世}{よ}に、
%
\ruby{女}{をんな}なり、
%
\ruby{少年}{せう|ねん}なりの、
%
\ruby{孱{\換字{弱}}}{か|よわ}き
\ruby{身}{み}をもて、
%
\ruby{屈}{くつ}すること
\ruby{無}{な}く
\ruby{凜々}{り|ゝ}しくも
\原本頁{148-1}\改行%
\ruby{立}{た}てる、
%
\ruby{此}{こ}の
\ruby{姊}{あね}
%\ %隙間調整 % TODO このフレーズパターンに似たものをどうするか
\ruby{此}{こ}の
\ruby{弟}{おとゝ}の
\ruby{潔}{いさぎよ}くも
\ruby{健}{けなげ}なる
\ruby{心掛}{こゝろ|がけ}は、
%
\ruby{同}{おな}じく
\ruby{{\換字{貧}}苦}{ひん|く}と
\ruby{戰}{たゝか}ひ
\ruby{來}{きた}れる
\ruby[g]{水野}{みづの}が
\ruby{心}{こゝろ}を
\ruby{少}{すくな}からず
\ruby{動}{うご}かして、
%
\ruby{深}{ふか}くも
\ruby[g]{五十子}{いそこ}を
\ruby{思}{おも}ひ
\ruby{思}{おも}ひて
\ruby{忘}{わす}るゝ
\ruby{能}{あた}はざるに
\ruby{至}{いた}りし
\ruby[g]{原因}{いはれ}の
\ruby{中}{うち}の、
%
\ruby{力{\換字{強}}}{ちから|づよ}き
\ruby{一}{ひと}つの
\ruby{個條}{か|でう}とはなりぬ。

\原本頁{148-5}%
されば
\ruby{我}{わ}が
\ruby[g]{五十子}{いそこ}が
\ruby{身}{み}にも
\ruby{代}{か}へじと
\ruby{深}{ふか}くも
\ruby{愛}{いつく}しめりと
\ruby{思}{おも}ふにつけて、
%
\ruby[g]{水野}{みづの}も
\ruby[<j||]{自然}{おのづ|から}
\ruby[g]{松之助}{まつのすけ}を
\ruby{他}{よそ}ならず
おもへば、
%
\ruby[g]{松之助}{まつのすけ}も
また
\ruby[g]{水野}{みづの}を
\ruby{他}{よそ}ならず
\ruby{思}{おも}ひ、
%
\ruby[g]{五十子}{いそこ}が
\ruby{許}{もと}にて
\ruby{相}{あひ}
\ruby{識}{し}りてより、
%
\ruby{四五度}{し|ご|たび}も
\ruby{面}{おもて}を
\ruby{會}{あ}はせたるには
\ruby{{\換字{過}}}{す}ぎねど、
%
\ruby{姊}{あね}の
\ruby{如何}{い|か}なる
\ruby{故}{ゆゑ}にか
\ruby{我}{われ}を
\ruby{好}{この}まざるに
\ruby{似}{に}ず、
%
\ruby{此兒}{こ|れ}は
\ruby{可愛}{か|はい}くも
\ruby{我}{われ}に
\ruby{睦}{むつ}みて、
%
\ruby{我}{われ}を
\ruby{眞}{まこと}の
\ruby{兄}{あに}
なんどの
\ruby{如}{ごと}くに
あしらひ、
%
\ruby{隔意}{へだて|ぎ}も
\ruby{無}{な}く
\ruby{打解}{うち|と}けて
\ruby{語}{かた}らふなり。

\原本頁{148-11}%
\ruby{我}{わ}が
\ruby{思}{おも}ふ
\ruby{人}{ひと}の
\ruby{弟}{おとゝ}と
\ruby{思}{おも}はんには、
%
たとひ
\ruby{色}{いろ}
\ruby{黑}{くろ}く
\ruby{醜}{みにく}くとも、
%
\ruby{{\換字{猶}}}{なほ}
\ruby{厭}{いと}はしき
\ruby{兒}{こ}とは
\ruby{見棄}{み|す}てざらんに、
%
まして
これは
\ruby{玉}{たま}の
\ruby{如}{ごと}く
\ruby{美}{うつく}しくして、
%
\原本頁{149-2}\改行%
\ruby[g]{加之}{しかも}
\ruby{我}{われ}に
\ruby{親}{したし}めるなり、
%
\ruby{今}{いま}
\ruby{其}{そ}の
\ruby{淸}{すゞ}しき
\ruby{眼}{め}を
\ruby{見張}{み|は}りて
\ruby{懷}{なつか}しげに
\ruby{我}{われ}を
\ruby{見}{み}ながら、

\原本頁{149-4}%
『
\ruby{君}{きみ}!、
%
\ruby{書狀}{てが|み}を
\ruby{有}{あ}り
\ruby{{\換字{難}}}{がた}う!。
%
\ruby{毫}{ちつと}も
\ruby{知}{し}らなかつた。
%
\ruby{僕}{ぼく}あ
\ruby{彼狀}{あ|れ}を
\ruby{見}{み}て
\ruby{吃驚}{びつ|くり}した!。
%
\ruby{郵便}{いう|びん}が
\ruby{昨夜}{ゆふ|べ}
\ruby{夜中}{よ|なか}に
\ruby{着}{つ}いたから、
%
それから
\ruby{今{\換字{朝}}}{け|さ}
\ruby{暗}{くら}い
\ruby{中}{うち}に
\ruby{飛}{とん}で
\ruby{出}{で}て
\ruby{來}{き}たんだ。
%
\ruby{姊}{ねえ}さんは
\ruby{何樣}{ど|ん}なだね、
%
エ、
%
\ruby{惡}{わる}いか?、
%
エヽエ。
』

\原本頁{149-8}%
と、
%
\ruby{我}{われ}を
\ruby{一家}{いつ|け}の
\ruby{人}{ひと}か
なんぞのやうに
\ruby{心易}{こゝろ|やす}く
\ruby{思}{おも}へる
\ruby{言葉}{こと|ば}つきの
\ruby{修{\換字{飾}}無}{かざ|り|な}く、
%
\ruby{姊}{あね}を
\ruby{思}{おも}へる
\ruby{{\換字{情}}}{こゝろ}の
\ruby{溢}{あふ}るゝ
ばかりに、
%
\ruby{取}{と}り
\ruby{繕}{つくろ}ひ
\ruby{氣}{げ}
\ruby{無}{な}く
\ruby{忙}{せは}しく
\ruby{問}{と}ふを
\ruby{見}{み}ては、
%
\ruby{今}{いま}まで
\ruby{胸}{むね}の
\ruby{中}{うち}に
もや〳〵としたる
\ruby{一切}{いつ|さい}の
\ruby{不快}{ふ|くわい}さ
\ruby{忌}{いま}はしさも、
%
\ruby{{\換字{朝}}日}{あさ|ひ}に
あひて
\ruby{霜柱}{しも|ばしら}の
\ruby{嵯牙}{さ|が}として
\ruby{立}{た}てるも
\ruby{忽}{たちま}ちに
\原本頁{150-1}\改行%
\ruby{摧}{くだ}き
\ruby{融}{と}かさるゝ
\ruby{心地}{こゝ|ち}して、
%
\ruby[g]{水野}{みづの}は
\ruby{思}{おも}はずも
\ruby{其}{その}
\ruby{手}{て}を
\ruby{執}{と}りて、
%
\ruby{正}{たゞ}しく
\ruby{答}{こた}ふるよりは
\ruby{先一句}{まづ|いつ|く}、

\原本頁{150-3}%
『マア
\ruby{安心}{あん|しん}したまへ。
』

\原本頁{150-4}%
と
\ruby{慰}{なぐさ}めたり。

\Entry{其二十五}

% メモ 校正 2024-04-09 2024-06-19
\原本頁{150-6}%
\ruby[g]{氣{\換字{遣}}}{き づか}はしさに
\ruby{堪}{た}へねばこそ
\ruby{知}{し}らず
\ruby{識}{し}らず
\ruby[g]{大悲}{だいひ }の
\ruby{御誓願}{おん|ち|かひ}を% ルビ調整(原本通り)
\ruby{頼}{たの}みて
\改行% 校正作業の簡略化のため
、
%
\原本頁{150-7}\改行%
その
\ruby{爲}{ため}に
\ruby[g]{書生}{しよせい}の
\ruby[g]{嘲笑}{あざけり}をも
\ruby{受}{う}くるに
\ruby{至}{いた}りたるなれ、
%
それを
\ruby{今}{いま}
\ruby{此}{こ}の
\原本頁{150-8}\改行%
\ruby[g]{少年}{せうねん}の
\ruby{姊}{あね}を
\ruby{思}{おも}ふ
\ruby[g]{心根}{こゝろね}の
いぢらしきとて、
%
\ruby{先}{ま}づ
\ruby[g]{安心}{あんしん}したまへと
\ruby[g]{眞實}{まこと }にもあらぬ
\ruby[g]{氣休}{き やす}めを
\ruby{云}{い}ひたるは
\ruby{何}{なん}の
\ruby{心}{こゝろ}ぞや、
%
\ruby{自}{みづか}ら
\ruby{欺}{あざむ}き
\ruby{人}{ひと}を
\ruby[<j||]{欺}{あざむ}くとは
\ruby{此}{こ}の
\ruby{事}{こと}なりと、
%
\ruby[g]{水野}{みづの }は
はツと% ルビ調整(原本通り)非踊り字表記
\ruby{思}{おも}ひしかど、
%
\ruby{既}{すで}に
\ruby{口}{くち}を
すべらせたれば
\ruby{駟}{し}も
\ruby{及}{およ}ばず、
%
たゞ
\ruby{四ツ木}{よ| |ぎ}に
\ruby{着}{つ}きても
\ruby{松之助}{まつ|の|すけ}が
\ruby{驚}{おどろ}く
\ruby{事}{こと}などの
\ruby{無}{な}からんをば、
%
\ruby[g]{今{\換字{更}}}{いまさら}
\ruby{{\換字{又}}}{また}
ひそかに
\ruby{切}{せつ}に
\ruby{念}{ねん}じたり。

\原本頁{151-3}%
\ruby{松之助}{まつ|の|すけ}は
\ruby{嬉}{うれ}しげに
\ruby[g]{水野}{みづの }を
\ruby{見}{み}て、

\原本頁{151-4}%
『
では
\ruby[g]{其樣}{そんな }に
\ruby{甚}{ひど}くは
\ruby{無}{な}いの?、
%
あゝ
\ruby[g]{有{\換字{難}}}{ありがた}かつた!。
%
\ruby{僕}{ぼく}は
\ruby{何}{ど}の
\ruby[<j||]{位}{くらゐ}
\ruby[g]{心配}{しんぱい}したか
\ruby{知}{し}れない。
%
\ruby{併}{\換字{志}か}し
\ruby[g]{{\換字{平}}常}{た ゞ }の
\ruby[g]{風邪}{か ぜ }では
\ruby{無}{な}いやうだつて、
%
\ruby[||j>]{何}{なに}
\ruby[||j>]{病}{びやう}だつたの?。
% \ruby{何病}{なに|びやう}だつたの?。
』

\原本頁{151-7}%
と、
%
\ruby{人}{ひと}の
\ruby[g]{一句}{いつく }を
\ruby{直}{たゞち}に
\ruby{信}{しん}じて
\ruby{無邪氣}{む|じや|き}に
\ruby{悅}{よろこ}べる
さまの
\ruby{罪}{つみ}なさは、
%
\ruby{却}{かへ}つて
\ruby[g]{水野}{みづの }の
\ruby{眼}{め}に
\ruby[g]{憫然}{あはれ }に
\ruby{見}{み}えたり。

\原本頁{151-9}%
『
\ruby[g]{病氣}{びやうき}は
\ruby[||j>]{腸}{ちやう}
\ruby{窒扶斯}{ ち|ぶ|す}といふ
\ruby{事}{こと}で、
%
なか〳〵
\ruby{輕}{かる}くは
\ruby{無}{な}い
\ruby[g]{病患}{わづらひ}なのだよ。
%
\換字{志}かし
\ruby[g]{醫師}{い し }も
\ruby[g]{信用}{しんよう}の
\ruby[g]{出來}{で き }る
\ruby{人}{ひと}を
\ruby{頼}{たの}み、
%
\ruby{看護{\換字{婦}}}{かん|ご|ふ}も
\ruby[g]{今日}{け ふ }から
\ruby{來}{く}る
\ruby[g]{手筈}{て はず}に
なつて
\ruby{居}{ゐ}るから、
%
\ruby{決}{けつ}して
\ruby[g]{無益}{む えき}の
\ruby[g]{心配}{しんぱい}は
\ruby[g]{仕玉}{し たま}ふな。
%
まあ
\ruby{大{\換字{丈}}夫}{だい|ぢやう|ぶ}だと
\ruby{僕}{ぼく}はおもふ。
』

\原本頁{152-2}%
『
ナニ
\ruby{窒扶斯}{ち|ぶ|す}だつて!。
%
\ruby{困}{こま}つたナア、
%
アヽ
\ruby{其}{そ}りやあ
\ruby[g]{大變}{たいへん}だ、
%
\ruby[g]{大變}{たいへん}だ!。
%
アヽ
\ruby{僕}{ぼく}あ
\ruby[g]{何樣}{ど う }したら
\ruby{好}{い}いんだらう!。
%
\ruby[g]{左樣}{そ う }して
\ruby[g]{醫者}{い しや}だの
\ruby{何}{なん}ぞは
\ruby{誰}{だれ}が
\ruby{仕}{し}て
\ruby{吳}{く}れたの?。
%
\ruby{姊}{ねえ}さんに
\ruby{其}{それ}だけの
\ruby{事}{こと}が
\ruby[g]{自{\換字{分}}}{じ ぶん}で
\ruby[g]{出來}{で き }たの?。
%
\ruby{姊}{ねえ}さんにやあ
\ruby[g]{其樣}{そ ん }な
\ruby{事}{こと}の
\ruby[g]{出來}{で き }さうも
\ruby{無}{な}いナア
\ruby{僕}{ぼく}が
\ruby{知}{し}つて
\ruby{居}{ゐ}る。
%
\ruby{誰}{だれ}が
\ruby{仕}{し}て
\ruby{吳}{く}れたの?。
%
\ruby{君}{きみ}が
\ruby[g]{親切}{しんせつ}に?。
』

\原本頁{152-7}%
\ruby{何}{なに}と
\ruby{無}{な}く
\ruby{{\換字{感}}}{かん}じて
\ruby{知}{し}れる
\ruby{歟}{か}
\ruby{兒童心}{こ|ども|ごゝろ}の
\ruby{敏}{さと}くも、
%
はや
\ruby{眼}{め}の
\ruby{中}{うち}は
\ruby{涙}{なみだ}ぐみて、
%
\ruby{泣}{な}き
\ruby{出}{だ}さん
ばかりの
\ruby{顏}{かほ}つきの
\ruby[||j>]{正}{しやう}
\ruby[||j>]{直}{ ぢき}にも、
% \ruby{正直}{しやう|ぢき}にも、
%
\ruby{其}{そ}の
\ruby{然}{しか}りとの
\ruby[g]{一語}{いちご }
\原本頁{152-9}\改行%
を
\ruby{聞}{き}きて
\ruby{直}{たゞち}に
\ruby{謝}{しや}せんと、
%
\ruby{待}{ま}ち
\ruby{設}{まう}けたる
\ruby{意}{い}
\ruby{中}{ちゆう}は
あり〳〵と
\ruby{見}{み}えぬ
\改行% 校正作業の簡略化のため
。
%
\原本頁{152-10}\改行%
\ruby[g]{水野}{みづの }は
\ruby[g]{自己}{お の }が
\ruby[g]{此度}{こ たび}の
\ruby[g]{振舞}{ふるまひ}の、
%
\ruby{恩}{おん}を
\ruby{賣}{う}るやうに
\ruby{取}{と}られん
\ruby{事}{こと}を
\makeatletter
\@ifundefined{デバッグ@ビルド}{%
  \ruby[||j>]{心}{こゝろ}
  \ruby[||j>]{苦}{ ぐる}
  しく
}{%
  \ruby[<j||]{心}{こゝろ}% 行末行頭の境界付近なので特例処置を施す
  \ruby{苦}{ぐる}
  \原本頁{152-11}\改行%
  しく
}%
\makeatother
\ruby{思}{おも}ひ
\ruby{居}{ゐ}たれば、
%
\ruby{彼}{か}の
お
\ruby{澤}{さは}
\ruby{婆}{ばゞ}に
\ruby{對}{むか}ひて
\ruby{云}{い}ひ
\ruby{置}{お}ける
おもむきを
\改行% 校正作業の簡略化のため
、
%
\原本頁{153-1}\改行%
\ruby{{\換字{飽}}}{あ}くまで
\ruby{徹}{とほ}さんと
\ruby{思}{おも}へるなり。

\原本頁{153-2}%
『
イヽエ。
』

\原本頁{153-3}%
\ruby{思}{おも}ひの
\ruby{外}{ほか}なる
\ruby[g]{水野}{みづの }が
\ruby{答}{こたへ}に
\ruby{松之助}{まつ|の|すけ}は
\ruby[g]{合點}{が てん}
\ruby{行}{ゆ}かぬ
ところあり。

\原本頁{153-4}%
『
ぢやあ
\ruby{誰}{だれ}が
\ruby{仕}{し}て
\ruby{吳}{く}れたの?。
』

\原本頁{153-5}%
『
\ruby[g]{學校}{がくかう}の
\ruby{人}{ひと}たちが。
』

\原本頁{153-6}%
『
\ruby{君}{きみ}だの
\ruby[||j>]{校}{かう}
\ruby[||j>]{長}{ちやう}さんだのが?。
% \ruby{校長}{かう|ちやう}さんだのが?。
』

\原本頁{153-7}%
『
マアそんなものだと
\ruby{思}{おも}つて
\ruby{居}{ゐ}たまへ。
』

\原本頁{153-8}%
『
ア、
%
それぢやあ
\ruby[g]{矢張}{やつぱ }り
\ruby{君}{きみ}の
\ruby[g]{親切}{しんせつ}なんだ、
%
きつと
\ruby[g]{左樣}{さ う }に
\ruby[g]{{\換字{違}}無}{ちがひな}い、
%
\ruby{僕}{ぼく}は
\ruby{知}{し}つてゐる!。
%
ほんたうに
\ruby{君}{きみ}
\ruby{有}{あ}り
\ruby{{\換字{難}}}{がた}う!。
%
\ruby{僕}{ぼく}あ
\ruby[||j>]{一}{いつ}
\ruby[||j>]{生}{しやう}
% \ruby{一生}{いつ|しやう}
おぼえて
\ruby{居}{ゐ}る!。
』

\原本頁{153-11}%
\ruby[g]{淡泊}{たんぱく}にも
\ruby{頭}{かうべ}を
\ruby{下}{さ}けて
\換字{志}み〴〵と
\ruby{恩}{おん}を
\ruby{謝}{しや}せる
\ruby{松之助}{まつ|の|すけ}が
\ruby{心}{こゝろ}は
\ruby{其}{そ}の
\ruby{手}{て}に
\ruby{籠}{こも}りて、
%
\ruby[g]{水野}{みづの }は
\ruby{我}{わ}が
\ruby{手}{て}の
\ruby{緊}{きび}しく
\ruby{握}{にぎ}られたるを
\ruby{{\換字{感}}}{かん}じぬ。
%
\ruby[g]{談話}{はなし }は
\原本頁{154-2}\改行%
\ruby{一}{ひ}ト
\ruby{先}{まづ}
\ruby{{\換字{終}}}{をは}りけるが、
%
\ruby[g]{問答}{もんだふ}は
\ruby{{\換字{又}}}{また}
\ruby{突}{とつ}として
\ruby{起}{おこ}りぬ。

\原本頁{154-3}%
『
\ruby{君}{きみ}は
こんなに
\ruby{夙}{はや}く
\ruby[g]{何處}{ど こ }へ
\ruby{行}{い}つたの?。
』

\原本頁{154-4}%
『
\ruby{少}{すこ}しばかり
\ruby{用}{よう}があつて
\ruby{出}{で}たんだが、
%
もう
\ruby[g]{歸路}{かへり }なのだ。
』

\原本頁{154-5}%
『
\ruby{其}{そ}の
\ruby{次}{ついで}に
\ruby[||j>]{觀}{くわん}% 「觀音」の読みは原本通り「くわん(の)ん」
\ruby[||j>]{音}{ のん}
\ruby[||j>]{樣}{ さま}へ
\ruby{詣}{まゐ}つたのかエ?。
』

\原本頁{154-6}%
『
ムヽ。
』

\原本頁{154-7}%
『
\ruby[||j>]{觀}{くわん}% 「觀音」の読みは原本通り「くわん(の)ん」
\ruby[||j>]{音}{ のん}
\ruby[||j>]{樣}{ さま}に
\ruby{何}{なん}の
\ruby{用}{よう}があつて?。
もし
\ruby{願}{ねが}ひ
\ruby{事}{ごと}でも
\ruby{爲}{し}て?。
』

\原本頁{154-8}%
『
ムヽ。
』

\原本頁{154-9}%
『
\ruby[g]{虛言}{う そ }だらう。
%
そりやあ
\ruby[g]{可笑}{をかし }いナア、
%
ハヽ。
』

\原本頁{154-10}%
『
\ruby[g]{何故}{な ぜ }そんなに
\ruby{君}{きみ}にやあ
\ruby[g]{可笑}{をかし }いのかね?。
』

\原本頁{154-11}%
『
だつて
\ruby{君}{きみ}、
%
\ruby{君}{きみ}は
いつか
\ruby{僕}{ぼく}に
\ruby{敎}{をし}へたぢやあ
\ruby{無}{な}いか。
%
ホラ、
%
\ruby{此}{こ}の
\ruby[||j>]{觀}{くわん}% 「觀音」の読みは原本通り「くわん(の)ん」
\ruby[||j>]{音}{ のん}
% \ruby{觀音}{くわん|のん}% 「觀音」の読みは原本通り「くわん(の)ん」
といふ
\ruby{人}{ひと}は、
%
\ruby{聞}{き}いて
\ruby{思}{おも}つて
\ruby{修}{をさ}めるといふ
\ruby{三}{み}つの
\ruby[g]{學問}{がくもん}の
\ruby[g]{法則}{はふそく}を、
%
\ruby{敎}{をし}へて
\ruby{{\換字{遺}}}{のこ}した
\ruby{人}{ひと}なので、
%
\ruby{敬}{けい}すべき
\ruby{人}{ひと}には
\ruby[g]{{\換字{違}}無}{ちがひな}いが、
%
\ruby{福}{ふく}を
\ruby{與}{あた}
\原本頁{155-3}\改行%
へるもの
なんぞ
として
\ruby{拜}{をが}むのは、
%
\ruby[g]{{\換字{感}}心}{かんしん}の
\ruby[g]{出來}{で き }ない
\ruby{卑}{いや}しい
\ruby{事}{こと}だと
\改行% 校正作業の簡略化のため
、
%
\原本頁{155-4}\改行%
\ruby{僕}{ぼく}が
\ruby[g]{{\換字{習}}慣}{く せ }でもつて
\ruby{拜}{をが}まうとしたら、
%
\ruby{敎}{をし}へて
\ruby{吳}{く}れた
\ruby{事}{こと}が
あつたもの!。
%
その
\ruby{君}{きみ}が
\ruby{願}{ねが}ひ
\ruby{事}{ごと}なんぞ
\ruby{仕}{し}やう
\ruby{譯}{わけ}は
\ruby{無}{な}いもの!。
』

\原本頁{155-6}%
\ruby{實}{げ}に
\ruby{嘗}{かつ}て
\ruby{此}{こ}の
\ruby[g]{少年}{せうねん}が
\ruby{四ツ木}{よ| |ぎ}よりの
\ruby{歸}{かへ}るさを
\ruby{{\換字{送}}}{おく}りがてら、
%
\ruby{共}{とも}に
\ruby[<j||]{心}{こゝろ}% 行末行頭の境界付近なので特例処置を施す
だのしく
\ruby{{\換字{遊}}}{あそ}び
あるき
つゝ
\ruby[g]{此處}{こ ゝ }に
\ruby{來}{きた}りし
\ruby{時}{とき}、
%
\ruby{生}{なま}さかしくも
\ruby{然}{さ}る
\ruby{事}{こと}
\原本頁{155-8}\改行%
を
\ruby{說}{と}きて、
%
\ruby[g]{幸福}{しあはせ}を%「幸福」ここは「は」
\ruby{得}{え}んとて
\ruby{佛}{ほとけ}を
\ruby{拜}{をが}む
\ruby{世}{よ}の
\ruby{人}{ひと}の
\ruby{心}{こゝろ}の
\ruby{卑}{いや}しさを
\ruby{笑}{わら}ひし
\ruby{事}{こと}ありしを、
%
\ruby{端}{はし}
\ruby{無}{な}くも
\ruby{今}{いま}
\ruby{云}{い}ひ
\ruby{出}{いだ}されて
\ruby{想}{おも}ひ
\ruby{起}{おこ}せば、
%
\ruby{{\換字{又}}}{また}
\ruby{新}{あらた}に
\ruby{毒}{どく}
\原本頁{155-10}\改行%
\ruby{箭}{や}を
\ruby[g]{胸板}{むないた}に
\ruby{射}{い}
\ruby{立}{た}てられし
\ruby[g]{心地}{こゝち }して、
%
\ruby{堪}{た}へがたき
\ruby[g]{不快}{ふくわい}さを
\ruby[g]{再度}{ふたゝび}
\ruby{覺}{おぼ}えつ。
%
おもへば
\ruby{其}{それ}のみには
あらざりし、
%
はじめて
\ruby[||j>]{東}{とう}
\ruby[||j>]{京}{きやう}にて
% \ruby{東京}{とう|きやう}にて
\ruby[g]{羽{\換字{勝}}}{は がち}
\原本頁{156-1}\改行%
\ruby[g]{島木}{しまき }
\ruby{等}{ら}
\ruby[g]{七人}{しちにん}% 原本には漢数字「七」のルビ無し
\ruby[g]{打揃}{うちそろ}ひて、
%
\ruby{詣}{まゐ}るとも
\ruby{無}{な}く
\ruby{此}{こ}の
\ruby[g]{御堂}{み だう}に
\ruby{參}{まゐ}りし
\ruby{折}{をり}、
%
\ruby[g]{島木}{しまき }と
\ruby[g]{楢井}{ならい }と
\ruby[g]{羽{\換字{勝}}}{は がち}とは
\ruby{手}{て}を
\ruby{合}{あは}せて
\ruby{拜}{をが}み、
%
\ruby[g]{日方}{ひ かた}と
\ruby[g]{山瀬}{やませ }と
\ruby[g]{名倉}{な ぐら}とは
\ruby[g]{三人}{さんにん}を
\ruby[g]{冷笑}{あざわら}ひしに、
%
おのれは
\ruby{拜}{をが}みもせねば
\ruby[g]{冷笑}{あざわら}ひもせで、
%
\ruby{我}{われ}は
たゞ
\makeatletter
\@ifundefined{デバッグ@ビルド}{%
  \ruby[<g>]{{\換字{古}}}{いにしへ}
  の
}{%
  \ruby[<g>]{{\換字{古}}の}{いにしへ}% 行頭なので[<j>] を使えない故「古の(いにしへ)」
}%
\makeatother
\ruby[g]{賢人}{けんじん}として
\ruby[g]{大士}{だいし }を
\ruby{待}{ま}たんと
\ruby{思}{おも}ふなりとて、
%
たゞ
\ruby{帽}{ばう}を
\ruby{脫}{ぬ}ぎて
\ruby[g]{一禮}{いちれい}したりし
\ruby{{\換字{古}}}{ふる}き
\ruby{事}{こと}まで
\ruby{心}{こゝろ}に
\ruby{{\換字{浮}}}{うか}べば、
%
\ruby[g]{一腔}{いつこう}の
\ruby{中}{うち}は
\ruby{火}{ひ}の
\ruby{散}{ち}る
\ruby{如}{ごと}くに
\ruby[g]{羞惡}{しうを }の
\ruby[||j>]{{\換字{情}}}{こゝろ}
\ruby[||j>]{燃}{ も }え
\ruby{立}{た}つて、
%
\ruby[g]{菩薩}{ぼ さつ}の
\ruby{大威力}{だい|ゐ|りき}を
\ruby{假}{か}りたき
\ruby{念}{おもひ}は
\ruby{今}{いま}
\ruby{{\換字{猶}}}{なほ}
こゝに
ありながら、
%
\ruby{今}{いま}
こゝに
\ruby{我}{われ}を
\ruby{卑}{いや}しくして、
%
\ruby{世}{よ}の
\ruby[g]{人並}{ひとなみ}に
\ruby[g]{菩薩}{ぼ さつ}を
\ruby{拜}{をが}みしを
\ruby[g]{口惜}{くちをし}く
おもふが
\ruby{如}{ごと}き
\ruby{{\換字{感}}}{かん}じも
\ruby{起}{おこ}りて、
%
\ruby[g]{不安}{ふ あん}の
\ruby{色}{いろ}の
\ruby{面}{おもて}に
\ruby{出}{い}づらんを
\ruby{制}{せん}せんとして% 原本通り「制」を「せん」とした
\ruby{制}{せい}しがたきを
\ruby{覺}{おぼ}えたり。

\原本頁{156-10}%
『
ハヽヽ、
%
そんな
\ruby{事}{こと}を
\ruby{云}{い}つた
\ruby{事}{こと}も
\ruby[g]{成程}{なるほど}
\ruby{有}{あ}つた。
』

\原本頁{156-11}%
\ruby{辛}{から}くも
\ruby{自}{みづか}ら
\ruby{克}{か}つて
\ruby{塞}{ふさ}がる
\ruby{胸}{むね}より
\ruby{答}{こた}へ
\ruby{得}{え}たるは、
%
\ruby{全}{まつた}く
\ruby[g]{意味}{い み }も
\ruby{無}{な}き
\原本頁{157-1}\改行%
\ruby[g]{言葉}{ことば }なり。

\原本頁{157-2}%
『
さうして
\ruby{君}{きみ}は
\ruby{何}{なに}を
\ruby{願}{ねが}つたの?。
』

\原本頁{157-3}%
\ruby[||j>]{心}{こゝろ}
\ruby[||j>]{無}{ な }く
\ruby{放}{はな}つ
\ruby[g]{少年}{せうねん}の
\ruby{箭}{や}は、
%
またもや
\ruby[g]{水野}{みづの }が
\ruby[g]{心窩}{む ね }の
\ruby{眞正中}{まつ|たゞ|なか}に
\ruby{立}{た}ちぬ
\改行% 校正作業の簡略化のため
。
%
\原本頁{157-4}\改行%
されど
\ruby[g]{水野}{みづの }は
\ruby[g]{痛手}{いたで }を
\ruby{外}{よそ}にして、

\原本頁{157-5}%
『
\ruby{何}{なん}でも
\ruby{可}{い}いから
\ruby{急}{いそ}いで
\ruby{行}{ゆ}かう。
』

\原本頁{157-6}%
と、
%
\ruby{松之助}{まつ|の|すけ}と
\ruby{共}{とも}に
\ruby{四ツ木}{よ| |ぎ}へと
\ruby[<j>]{志}{こゝろざ}し、
%
\ruby{人}{ひと}の
\ruby[g]{{\換字{運}}命}{う ん }、
%
\ruby{我}{わ}が
\ruby[g]{{\換字{運}}命}{う ん }の
\ruby{測}{はか}り
\原本頁{157-7}\改行%
\ruby{{\換字{難}}}{がた}き
\ruby[g]{{\換字{前}}{\換字{途}}}{ゆくて }を
\ruby{見}{み}んと、
%
\ruby{心}{こゝろ}に
\ruby[g]{幾枝}{いくし }の
\ruby{箭}{や}を
\ruby{負}{お}ひながら、
%
\ruby{路}{みち}を
\ruby{急}{いそ}ぎて
\ruby{歩}{あゆ}み
\ruby{出}{いだ}しぬ。

\原本頁{157-9}%
\ruby{此}{こ}の
\ruby{時}{とき}
\ruby{日}{ひ}は
\ruby{漸}{やうや}く
\ruby{昇}{のぼ}ると
\ruby{共}{とも}に、
%
\ruby{狂風滾々}{きやう|ふう|こん|〳〵}と
\ruby{吹}{ふ}き
\ruby{出}{いだ}して、
%
\ruby{美}{うるは}しかりし
\ruby{{\換字{空}}}{そら}は
\ruby[g]{何時}{い つ }と
\ruby{無}{な}く
\ruby{黄}{き}ばみ、
%
\ruby{暴風雨日}{あ| れ||び}% ルビ調整(原本通り)親文字4に対してルビ3
\ruby{{\換字{近}}}{ちか}き
\ruby{天}{てん}に
\ruby[g]{氣味}{き み }
あしき
\ruby{雲}{くも}の
おだやかならず
\ruby{湧}{わ}き
ひろごりて、
%
\ruby[g]{昨夜}{ゆふべ }に
\ruby{變}{かは}れる
\ruby[g]{今日}{け ふ }の
\ruby[g]{狀態}{やうす }の、
%
そぞろに
\ruby{定}{さだ}め
\ruby{無}{な}き
\ruby[g]{人間}{ひ と }の
\ruby{上}{うへ}を
\ruby{示}{しめ}すが
\ruby{如}{ごと}く、
%
\ruby{首}{かうべ}を
\ruby{傾}{かたむ}けて
\ruby{{\換字{進}}}{すゝ}む
\ruby[g]{水野}{みづの }と
\原本頁{158-2}\改行%
\ruby{松之助}{まつ|の|すけ}との
\ruby[g]{眞向}{まつかう}に
\ruby{烈}{はげ}しく
\ruby{當}{あた}る
\ruby{風}{かぜ}は、
%
\ruby[g]{二人}{ふたり }が
\ruby[g]{心臓}{む ね }をして
\ruby{騷}{さわ}ぎに
\ruby{騷}{さわ}がしめぬ。

\Entry{其二十六}

% メモ 校正終了 2024-04-09
\原本頁{158-5}%
\ruby{語}{かた}り
つゞけたる
\ruby{談話}{はな|し}の
\ruby{間}{うち}、
%
\ruby{息}{いき}
つぎ〳〵に
われ
\ruby{知}{し}らず
\ruby{飮}{の}みし
\ruby{葡萄酒}{ぶ|だう|しゆ}の
\ruby{量}{りやう}の
\ruby{少}{すくな}からで、
%
\ruby{既}{すで}に
\ruby{其}{そ}の
\ruby{六七{\換字{分}}}{ろく|しち|ぶ}を
\ruby{盡}{つく}したれば、
%
\ruby[||j>]{醉}{すゐ}
\ruby[||j>]{興}{きよう}% 「醉」は原本通り「ゐ」で調整
% \ruby{醉興}{すゐ|きよう}% 「醉」は原本通り「ゐ」で調整
おのづから
\ruby{發}{はつ}して
\ruby{獨}{ひと}り
\ruby{機{\換字{嫌}}}{き|げん}よく、
%
\ruby{不規律}{ふ|き|りつ}の
\ruby[||j>]{大}{たい}
\ruby[||j>]{將}{しやう}を
% \ruby{大將}{たい|しやう}を
もて
\ruby{自}{みづか}ら
\ruby{許}{ゆる}せるほど
ありて、
%
ふたゝび
\ruby{睡}{ねむ}りには
\ruby{就}{つ}かんともせず、
%
\ruby{島木}{しま|き}は
\ruby{{\換字{猶}}}{なほ}
ぐびりぐびりと% 原本は行末禁則箇所なので踊り字にはなっていない
\ruby[||j>]{獨}{どく}
\ruby[||j>]{{\換字{酌}}}{しやく}を
% \ruby{獨{\換字{酌}}}{どく|しやく}を
\ruby{續}{つゞ}けたり。

\原本頁{158-10}%
むつくりと
\ruby{肥}{こ}えたる
\ruby{身體}{から|だ}
ゆたかに
\ruby{胡坐}{あぐ|ら}
かきて、
%
\ruby{土}{つち}
\ruby{多}{おほ}き
\ruby{山}{やま}の
\ruby{岩}{いは}を
\原本頁{159-1}\改行%
\ruby{隱}{かく}せるが
\ruby{如}{ごと}くに、
%
\ruby{肉}{にく}
ふくらかにして
\ruby{骨}{ほね}を
\ruby{見}{み}せぬ
\ruby{丸々}{まる|〳〵}としたる
\ruby{顏}{かほ}の、
%
\ruby{其}{そ}の
\ruby{小}{ちひ}さなる
\ruby{眼}{め}の
あたりに
\ruby{笑}{ゑみ}を
\ruby{含}{ふく}み、
%
\ruby{今}{いま}しも
ぐつと
\ruby{一盞}{いつ|さん}を
\ruby{仰}{あふ}ぎたるが、

\原本頁{159-4}%
『もう
\ruby{出}{で}て
\ruby{來}{き}さうなものだがナ、
%
\ruby[||j>]{畜}{ちく}
\ruby[||j>]{生}{しやう}!、
% \ruby{畜生}{ちく|しやう}!、
%
まだかナ。
』

\原本頁{159-5}%
と、
%
\ruby{誰}{たれ}に
\ruby{云}{い}へるともなく
\ruby{自}{みづか}ら
\ruby{語}{かた}れり。

\原本頁{159-6}%
\ruby{島木}{しま|き}は
\ruby{水野}{みづ|の}が
\ruby{胸中}{む|ね}を
\ruby{知}{し}りたれど、
%
\ruby{水野}{みづ|の}は
\ruby{島木}{しま|き}が
\ruby{肚裏}{は|ら}を
\ruby{知}{し}らざりき。
%
\ruby{妻子}{さい|し}
\ruby[||j>]{兄}{きやう}
\ruby[||j>]{弟}{ だい}も
% \ruby{兄弟}{きやう|だい}も
\ruby{無}{な}く
\ruby{親}{おや}も
\ruby{無}{な}ければ、
%
\ruby{氣}{き}まゝなる
\ruby{寄寓}{かり|ずみ}の
\ruby{面倒}{めん|だう}
\ruby{無}{な}きを
\ruby{悅}{よろこ}びて、
%
\ruby{一家}{いつ|か}を
こそは
\ruby{{\換字{猶}}}{なほ}
\ruby{構}{かま}へざれ、
%
\ruby{幾度}{いく|たび}か
\ruby{{\換字{浮}}}{う}き
\ruby{幾度}{いく|たび}か
\ruby{沈}{しづ}みし
\原本頁{159-9}\改行%
\ruby{末}{すゑ}に、
%
\ruby{漸}{やうや}く
\ruby[||j>]{合}{がふ}
\ruby[||j>]{百}{ひやく}の% 「合百」証拠金を納めないで相場の上げ下げにて賭けをする一種の賭博。
% \ruby{合百}{がふ|ひやく}の% 「合百」証拠金を納めないで相場の上げ下げにて賭けをする一種の賭博。
\ruby{果敢無}{は|か|な}きより、
%
\ruby{今}{いま}は
\ruby{人}{ひと}の
\ruby{噂}{うはさ}にも
\ruby{上}{のぼ}るほどの
\ruby[||j>]{玉}{ぎよく}
\ruby[||j>]{高}{ だか}を
% \ruby{玉高}{ぎよく|だか}を
\ruby{動}{うご}かすに
\ruby{至}{いた}りし
\ruby{島木}{しま|き}も、
%
もとより
\ruby{右}{みぎ}は
\ruby{地獄}{ぢ|ごく}
\ruby{左}{ひだり}は
\ruby{極樂}{ごく|らく}の
\ruby{間}{あひだ}の
\ruby{綱}{つな}を
\ruby{渡}{わた}つて
\ruby{日}{ひ}を
\ruby{{\換字{送}}}{おく}る
\ruby{投機師}{とう|き|し}の
\ruby{身}{み}の
\ruby{上}{うへ}は、
%
\ruby{貨物}{くわ|ぶつ}を
\ruby{積}{つ}み
\ruby{問屋}{とひ|や}を
\ruby{控}{ひか}へて
\ruby{十}{じう}の
\ruby{一}{いち}
\ruby{十}{じう}の
\ruby{二}{に}の
\ruby{利}{り}を
\ruby{征}{と}りて
\ruby{行}{ゆ}く
\ruby{堅氣}{かた|ぎ}の
\ruby{商人}{あき|うど}とは
\ruby{異}{こと}なれば、
%
\ruby{此處}{こ|ゝ}
\ruby{一}{ひ}ト
\ruby{伸}{のし}と
\ruby{有}{あ}らん
\ruby{限}{かぎ}りの
\ruby[<j||]{力}{ちから}
\ruby{瘤}{こぶ}を
\ruby{入}{い}れて
\ruby{蒐}{かゝ}れる
\ruby{此}{こ}の
\ruby{秋}{あき}の、
%
\ruby{天候}{てん|こう}を
\原本頁{160-3}\改行%
\ruby{重}{おも}なる
\ruby{相場}{さう|ば}の% 原文通り「場」
\ruby{時季}{と|き}に、
%
\ruby{捉}{とら}へ
かねたる
\ruby{雲}{くも}の
\ruby[||j>]{心}{こゝろ}
\ruby[||j>]{風}{ かぜ}の
% \ruby{心風}{こゝろ|かぜ}の
\ruby{料簡}{れう|けん}は
\ruby{我}{わ}が
\ruby{思}{おも}はくと
\ruby{{\換字{違}}}{ちが}ひて、
%
\ruby{{\換字{追}}敷}{おひ|じき}
\g詰めruby{々々}{〳〵}と
\ruby{取}{と}り
\ruby{立}{た}てらるゝに
\ruby{懷中}{ふと|ころ}
\ruby{危}{あやふ}く、
%
\ruby{既}{すで}に
\ruby{其}{そ}の
\原本頁{160-5}\改行%
\ruby{剩}{あま}すところは
\ruby{幾何}{いく|ばく}も
あらぬ
\ruby[||j>]{端}{はした}
\ruby[||j>]{錢}{ がね}と
% \ruby{端錢}{はした|がね}と
なりて、
%
\ruby{{\換字{運}}}{うん}と
\ruby[<j>]{志}{こゝろざし}との
\ruby{今}{いま}
\ruby{少時}{しば|し}
\ruby{反}{そむ}かば、
%
またもや
\ruby{身}{み}の
\ruby{皮}{かは}も% 原本通り「皮 か(は)」
\ruby{無}{な}き
\ruby{赤裸々}{あか|はだ|か}となりて、
%
\ruby{賽}{さい}の
\ruby{河原}{か|はら}に
\ruby{積}{つ}める
\ruby{石}{いし}の
\ruby{{\換字{瓦}}落離}{ぐわ|ら|り}と
\ruby{崩}{くづ}れたる
\ruby{{\換字{情}}無}{なさけ|な}さを
\ruby{見}{み}るべしと、
%
\ruby{流石}{さす|が}に
\ruby{心}{こゝろ}も
おちつき
かぬるところへ、
%
\ruby{折}{をり}も
\ruby{折}{をり}とて
\ruby{水野}{みづ|の}の
\ruby{無心}{む|しん}なり。
%
\ruby{{\換字{運}}}{うん}を
\ruby{背負}{せ|お}へる
\ruby{時}{とき}には
\ruby{其}{そ}の
\ruby{二倍}{に|ばい}
\ruby{三倍}{さん|ばい}も
\ruby{與}{あた}ふるに
\ruby{易}{やす}けれど、
%
\ruby{夜明}{よ|あ}けての
\ruby{天地}{てん|ち}の
\原本頁{160-10}\改行%
\ruby{狀態}{やう|す}
\ruby{次第}{し|だい}にて
\ruby{我}{わ}が
\ruby{生命}{いの|ち}はと
さへ
\ruby{思}{おも}へる
\ruby{矢先}{や|さき}に
\ruby{云}{い}ひかけられては、
%
\原本頁{160-11}\改行%
\ruby[||j>]{敗}{まけ}
\ruby[||j>]{軍}{いくさ}の
% \ruby{敗軍}{まけ|いくさ}の
\ruby{{\換字{退}}}{ひ}き
\ruby{際}{ぎは}に
\ruby{頼}{たの}みきつたる
\ruby{持鎗}{もち|やり}を
\ruby{{\換字{所}}望}{しよ|まう}されたる
\ruby{心地}{こゝ|ち}して、
%
\ruby{流石}{さす|が}の
\ruby{島木}{しま|き}も
\ruby{行}{ゆ}き
\ruby{詰}{つま}りしが、
%
\ruby{竹}{たけ}を
\ruby{割}{わ}つたる
\ruby{如}{ごと}き
\ruby{持{\換字{前}}}{もち|まへ}の
\ruby{氣象}{き|しやう}は
\ruby{義}{ぎ}を
\原本頁{161-2}\改行%
\ruby{見}{み}て
\ruby{勇}{いさ}んで、
%
エヽ
どうせ
\ruby{曲}{まが}つて
\ruby{仕舞}{し|ま}えば
\ruby{無}{な}くなる
\ruby{金}{かね}を、
%
\ruby{今}{いま}
\ruby{{\換字{遣}}}{や}つて
\ruby{仕舞}{し|ま}へば
\ruby{友{\換字{達}}}{とも|だち}の
\ruby{利益}{た|め}!、
%
\ruby{踏張}{ふん|ば}れ〳〵
\ruby{男}{をとこ}の
\ruby{兒}{がき}だ、
%
\ruby{裸々}{はだ|か}になつても
\ruby{怖}{こは}くは
\ruby{無}{な}い、
%
\ruby{百兩}{ひやく|りやう}ばかりの
\ruby{鼻糞金}{はな|くそ|がね}を
\ruby{出}{だ}し
\ruby{悋}{をし}んでは、
%
\ruby{萬五郎}{まん|ご|らう}の
\原本頁{161-5}\改行%
\ruby{男}{をとこ}が
\ruby{廢}{す}たる!、
%
\ruby{{\換字{情}}無}{なさけ|な}い!、
%
\ruby{行末}{ゆく|すゑ}が
\ruby{見}{み}える!、
%
\ruby{百萬兩{\換字{分}}限}{ひやく|まん|りやう|ぶ|げん}になつた
\ruby{時}{とき}の
\ruby[||j>]{額}{むかふ}
\ruby[||j>]{疵}{ きず}になる!、
% \ruby{額疵}{むかふ|きず}になる!、
%
\ruby{握}{にぎ}つた
\ruby{錢}{ぜに}から
\ruby{{\換字{煙}}}{けむ}を
\ruby{出}{だ}すのは
\ruby{三{\換字{文}}野郎}{さん|もん|や|らう}のする
\ruby{事}{こと}だ、
%
と
\ruby{早}{はや}くも
\ruby[||j>]{決}{けつ}
\ruby[||j>]{着}{ちやく}して
% \ruby{決着}{けつ|ちやく}して
\ruby{臓腑}{ざう|ふ}を
\ruby{見}{み}せずに、
%
\ruby{奇麗}{き|れい}に
\ruby[<j>]{快}{こゝろよ}く
\ruby{用立}{よう|だ}てて% TODO 原本通り行末禁則で踊り字未対応
\ruby{歸}{かへ}しやりつ、
%
さて
\ruby{其}{それ}が
ためとにも
あらざるべけれど、
%
\ruby{何}{なん}と
\ruby{無}{な}く
\ruby{心}{こゝろ}に
\ruby{怡悅}{よろ|こび}を
\ruby{覺}{おぼ}えて、
%
\ruby{今}{いま}は
\ruby{氣}{き}も
\ruby{冴}{さ}え〴〵と
\ruby{飮}{の}み
\ruby{居}{を}れるなり。

\原本頁{161-10}%
『もう
\ruby{出}{で}て
\ruby{來}{き}さうなものだがナ、
%
まだかナ、
%
\ruby[||j>]{畜}{ちく}
\ruby[||j>]{生}{しやう}!。
% \ruby{畜生}{ちく|しやう}!。
』

\原本頁{161-11}%
ふたゝび
\ruby{獨}{ひと}り
ごちて
\ruby{酒盞}{さか|づき}を
\ruby{取}{と}りぬ。

\原本頁{162-1}%
『まだ
\ruby{出}{で}て
\ruby{來}{こ}ないかナ、
%
\ruby[||j>]{畜}{ちく}
\ruby[||j>]{生}{しやう}めツ!。
% \ruby{畜生}{ちく|しやう}めツ!。
』

\原本頁{162-2}%
\ruby{何}{なに}を
\ruby{待}{ま}てるにか
\ruby{三度}{み|たび}
\ruby{獨語}{ひとり|ご}ちしが、
%
\ruby{答}{こた}ふるものは
\ruby{有}{あ}るべくも
\ruby{無}{な}く、
%
\原本頁{162-3}\改行%
\ruby{室}{しつ}の
\ruby{一隅}{いち|ぐう}の
\ruby{小机}{こ|づくゑ}の
\ruby{上}{うへ}の
\ruby{懷中時計}{くわい|ちゆう|ど|けい}の% 「懷中(くわいちゆう)」「ゆ」有り
\ruby{音}{おと}のみの
\ruby{有}{あ}るか
\ruby{無}{な}きかに
\ruby{響}{ひゞ}けり。

\原本頁{162-5}%
\ruby{相手}{あひ|て}
\ruby{無}{な}き
\ruby{淋}{さび}しさに
\ruby{堪}{た}へかねてか、

\原本頁{162-6}%
『
\ruby[||j>]{畜}{ちく}
\ruby[||j>]{生}{しやう}ツ、
% \ruby{畜生}{ちく|しやう}ツ、
%
\ruby{出}{で}て
\ruby{來}{き}やがらなくつても
\ruby{仕方}{し|かた}が
\ruby{無}{な}いかナ。
%
ハヽヽ、
%
\原本頁{162-7}\改行%
\ruby{怒}{おこ}るほど
\ruby{乃公}{お|れ}も
\ruby{野暮}{や|ぼ}ぢやあ
いけねえ。
%
それは
さうと
\ruby{水野}{みづ|の}は
もう
\原本頁{162-8}\改行%
\ruby{大{\換字{分}}}{だい|ぶ}
\ruby{行}{い}つたらう。
%
\ruby[||j>]{愍}{かあ}
\ruby[||j>]{然}{いさう}に、% 「愍然 か(あ)いさう」
% \ruby{愍然}{かあ|いさう}に、% 「愍然 か(あ)いさう」
%
\ruby{堅}{かた}い
\ruby[||j>]{正}{しやう}
\ruby[||j>]{直}{ ぢき}な
% \ruby{正直}{しやう|ぢき}な
\ruby{男}{をとこ}だから、
%
\ruby{人一倍}{ひと|いち|ばい}
\ruby{何彼}{なに|か}につけて
\ruby[||j>]{物}{もの}
\ruby[||j>]{思}{おもひ}を
% \ruby{物思}{もの|おもひ}を
\ruby{仕}{し}て
\ruby{居}{ゐ}る!。

\原本頁{162-10}%
\換字{庵点}
\ruby{粋}{すゐ}な
\ruby{{\換字{浮}}世}{うき|よ}を
\ruby{戀}{こひ}
\ruby{故}{ゆゑ}に、
%
\ruby{野暮}{や|ぼ}に
\ruby{暮}{くら}すも
\ruby{心}{こゝろ}がら。
%
あゝ
\ruby{端唄}{は|うた}の
\ruby{{\換字{文}}句}{もん|く}ぢやあ
\ruby{無}{な}いが
\ruby{{\換字{迷}}}{まよ}つちやあ
\ruby{野暮}{や|ぼ}になる!。
%
フン、
%
ナンダ
\ruby{此方}{こつ|ち}やあ
\ruby{戀}{こひ}
\ruby{故}{ゆゑ}ぢやあ
\ruby{無}{ね}えで、
%
\ruby{慾}{よく}
\ruby{故}{ゆゑ}に
\ruby{野暮}{や|ぼ}になり
\ruby{切}{き}つて
\ruby{居}{ゐ}やがる!。
%
アヽ
もう
そろ〳〵
\ruby{出}{で}て
\ruby{來}{き}て
\ruby{吳}{く}れても
\ruby{好}{よ}さゝうなものだが、
%
チヨツ
\ruby{忌々}{いま|〳〵}しい、
\GWI{u1b048-u3099}れつたいナア。% 「志」+「濁点」
%
ア、
%
\ruby{豪氣}{がう|ぎ}に
\ruby{醉}{よ}つて% 「醉」は原本通り「よ」で調整
\ruby{來}{き}た、
%
\ruby{好}{い}い
\ruby[||j>]{心}{こゝろ}
\ruby[||j>]{持}{ もち}だ!。
% \ruby{心持}{こゝろ|もち}だ!。
%
\ruby{何}{なん}だか
もう
\ruby{出}{で}て
\ruby{來}{き}さうな
\ruby[||j>]{心}{こゝろ}
\ruby[||j>]{持}{ もち}がする!。
% \ruby{心持}{こゝろ|もち}がする!。
%
ヱヽト、

\原本頁{163-5}%
\換字{庵点}
\ruby{起}{お}きて
\ruby{見}{み}つ、
%
\ruby{寢}{ね}て
\ruby{見}{み}つ
\ruby{待}{ま}てど、
%
たより
\ruby{無}{な}く、
%
チン〳〵
チンチン、
%
\ruby{蚊屋}{か|や}の
\ruby{廣}{ひろ}さに
たゞ
\ruby{獨}{ひと}り、
%
ツンテン、
%
\ruby{蚊}{か}を
\ruby{焼}{や}く
\ruby{火}{ひ}より
\ruby{胸}{むね}の
\ruby{火}{ひ}の、
%
\ruby{燃}{も}ゆる
おもひを
\ruby{察}{さつ}しやんせカナ。
%
ハヽヽヽ。
』

\原本頁{163-8}%
\ruby{聲}{こゑ}は
\ruby{美}{うつく}しからず
\ruby{錆}{さ}びたれど、
%
\ruby{聞}{き}き
\ruby{記臆}{おぼ|え}なるべきには% 原本通り「おぼえ」
\ruby{似合}{に|あ}はず
\ruby{我流}{が|りう}の
\ruby{{\換字{節}}{\換字{廻}}}{ふし|まは}しにも
をかしきところありて、
%
\ruby{小聲}{こ|ゞゑ}に
\ruby{唱}{うた}ひ
\ruby{仕舞}{し|ま}ひつゝ、
%
\原本頁{163-10}\改行%
\ruby{今}{いま}
\ruby{將}{まさ}に
\ruby{一壜}{ひと|びん}の
\ruby{酒}{さけ}を
\ruby{盡}{つく}し
\ruby{果}{は}たさんとして、
%
\ruby{手}{て}に
\ruby{取}{と}り
\ruby{上}{あ}げて
\ruby{自}{みづか}ら
\ruby{{\換字{酌}}}{つ}がんと、
%
\ruby{其}{そ}の
\ruby{尻下}{しり|さが}がりの
\ruby{小}{ちひさ}き
\ruby{目}{め}を
\ruby{一}{ひ}トしほ
\ruby{下}{さ}げて、
%
\ruby{莞爾}{につ|こり}と
\ruby{樂}{たの}しげに
\ruby{笑}{わら}ひしが、
%
\ruby{何}{なに}をか
\ruby{聞}{き}きつけしや
\ruby{俄然}{が|ぜん}として、

\原本頁{164-2}%
『ヤツ、
%
\ruby{來}{き}たぞ!%\inhibitglue{}% ここは「空き」があるので
\,% 原本上でのアキを再現するため「3/18 em」空ける
\ruby{來}{き}て
\ruby{吳}{く}れたぞ!、
%
おいでなすつたぞ!。
%
\ruby{占}{し}めたナ!、
%
サア
\ruby{來}{こ}いだ!。
』

\原本頁{164-4}%
と
\ruby{飛}{と}び
\ruby{立}{た}つたり。

\原本頁{164-5}%
\ruby{投}{な}げ
\ruby{出}{だ}されたる
\ruby{壜}{びん}は
\ruby{飜筋斗}{とん|ぼ|がへり}して、
%
\ruby{疊}{たゝみ}に
\ruby{溢}{こぼ}れたる
\ruby{紅色}{くれ|なゐ}の
\ruby{餘瀝}{した|ゝり}は、
%
\原本頁{164-6}\改行%
まだ
\ruby{早}{はや}き
\ruby{紅葉}{もみ|ぢ}を
こゝに
\ruby{散}{ち}らしたり。

\Entry{其二十七}

がらりと
\ruby{樓}{ろう}の
\ruby{雨{\換字{戸}}}{あま|ど}を
\ruby{繰}{く}り
\ruby{開}{あ}くれば、
\ruby{白}{しら}みわたれる
\ruby{曉}{あかつき}の
\ruby{天}{そら}より、
\ruby{蓬々然}{ほう|〳〵|ぜん}として
\ruby{下}{おろ}し
\ruby{來}{く}る
\ruby{風}{かぜ}は、おもむろに
\ruby{面}{おもて}を
\ruby{撲}{う}ち
\ruby{胸}{むね}を
\ruby{撲}{う}つて、
\ruby{昨日}{きの|ふ}の
\ruby{夜}{よ}の
\ruby{靜穩}{おだ|やか}なりし
\ruby{俤}{おもかげ}は
\ruby{{\換字{猶}}}{なほ}
\ruby{{\換字{遺}}}{のこ}れども、
\ruby{日}{ひ}の
\ruby{將}{まさ}に
\ruby{出}{い}でんとする
\ruby{方}{かた}の
\ruby{雲}{くも}の
\ruby{色}{いろ}
\ruby{峻}{けは}しく、
\ruby{何}{なん}と
\ruby{無}{な}く
\ruby{物凄}{もの|すさ}まじき
\ruby{景象}{やう|す}は
\ruby{見}{み}る〳〵
\ruby{動}{うご}き
\ruby{展}{の}びて、やがて
\ruby{恐}{おそ}ろしくも
\ruby{一}{ひ}ト
\ruby{暴風}{あ|れ}の、
\ruby{暴}{あ}れ
\ruby{立}{た}たんとする
\ruby{勢}{いきほひ}は
\ruby{現}{あら}はれたり。

\ruby{昔語}{むかし|がたり}の
\ruby{海坊主}{うみ|ばう|ず}の
\ruby{如}{ごと}く、ヌツと
\ruby{突立}{つゝ|た}つたるまゝ
\ruby{四邊}{あ|たり}を
\ruby{見{\換字{廻}}}{み|まは}せる
\ruby{島木}{しま|き}は、
\ruby{刻一刻}{こく|いつ|こく}に
\ruby{吹募}{ふき|つの}る
\ruby{風}{かぜ}の、
\ruby{袂}{たもと}を
\ruby{揚}{あ}げ
\ruby{裾}{すそ}を
\ruby{{\換字{扇}}}{あふ}るをも
\ruby{知}{し}らぬやうに、
\ruby{身}{み}じろぎもせずして
\ruby{居}{い}たりしが、
\ruby{{\換字{終}}}{つひ}には
\ruby{此}{こ}の
\ruby{風}{かぜ}の
\ruby{高}{こう}じに
\ruby{高}{こう}じて、
\ruby{老木}{おい|き}の
\ruby{枝}{えだ}を
\ruby{裂}{さ}き、
\ruby{若樹}{わか|ぎ}の
\ruby{根}{ね}を
\ruby{拔}{ぬ}き、
\ruby{沙}{すな}を
\ruby{舞}{ま}はせ
\ruby{石}{いし}を
\ruby{躍}{をど}らすに
\ruby{至}{いた}るべきさまの、
\ruby{十{\換字{分}}}{じう|ぶん}に
\ruby{想}{おも}ひやらるゝに
\ruby{及}{およ}びて、
\ruby{大浪}{おほ|なみ}のうねりて
\ruby{寄}{よ}するが
\ruby{如}{ごと}くに、
\ruby{肥}{ふと}つたる
\ruby{顏中}{かほ|ぢゆう}を
\ruby{笑}{わらひ}に
\ruby{動}{うご}かして、

『ウフ、ウフ、ウアツハツハヽハヽ。
とう〳〵
\ruby{來}{き}やがつたナ!ヤイ
\ruby{風}{かぜ}の
\ruby{神}{かみ}!。
\ruby{男振}{をとこ|ぶり}が
\ruby{好}{い}いぞ!\換字{志}つかり
\ruby{{\換字{遺}}}{や}れ。
\ruby{雨}{あめ}の
\ruby{{\換字{随}}}{つ}いて
\ruby{來}{き}やがらねえのは
\ruby{忌々}{いま|〳〵}しいが、
\ruby{仕方}{し|かた}が
\ruby{無}{ね}え、
\ruby{汝}{てめへ}だけでウンと
\ruby{働}{はたら}け。
\ruby{男振}{をとこ|ぶり}が
\ruby{好}{い}いぞ、〳〵!。
』

と、
\ruby{打{\換字{戱}}}{うち|たはむ}れて
\ruby{引{\換字{返}}}{ひつ|かへ}せば、
\ruby{洋燈}{らん|ぷ}は
\ruby{既}{すで}に
\ruby{風}{かぜ}に
\ruby{{\換字{消}}}{け}されて、
\ruby{室}{しつ}に
\ruby{滿}{み}てる
\ruby{曙色}{しよ|しよく}は
\ruby{其光}{その|ひかり}に
\ruby{代}{かは}り
\ruby{居}{ゐ}たり。

\ruby{島木}{しま|き}が
\ruby{待}{ま}ちに
\ruby{待}{ま}つたるは、
\ruby{我}{われ}を
\ruby{訪}{と}ひ
\ruby{來}{こ}ん
\ruby{{\換字{婦}}}{をんな}にもあらねば、
\ruby{他{\換字{所}}}{よ|そ}より
\ruby{入}{い}るべき
\ruby{金}{かね}にもあらず、
\ruby{唯此}{たゞ|こ}の
\ruby{野}{の}を
\ruby{拂}{はら}ひ
\ruby{禾}{くわ}を
\ruby{偃}{ふ}すの
\ruby{風}{かぜ}なりけり。
\ruby{數日{\換字{前}}}{すう|じつ|ぜん}より、
\ruby{乾坤一擲}{けん|こん|いつ|てき}と
\ruby{試}{こゝろ}みたる
\ruby{丁{\換字{半}}}{ちやう|はん}の、
\ruby{時}{とき}、
\ruby{利}{り}あらずして
\ruby{思}{おも}ふ
\ruby{目}{め}は
\ruby{出}{い}でず、\換字{志}きりに
\ruby{敵}{てき}に
\ruby{切}{き}り
\ruby{捲}{まく}られて、
\ruby{踏}{ふ}み
\ruby{耐}{こた}へ
\ruby{踏}{ふ}み
\ruby{耐}{こた}へては
\ruby{戰}{たゝか}ふものゝ、
\ruby{既}{すで}に
\ruby{味方}{み|かた}は
\ruby{崩}{くづ}れ
\ruby{立}{た}つて
\ruby{討死手負}{うち|じに|て|おひ}の
\ruby{數}{かず}を
\ruby{知}{し}らず、
\ruby{大勢}{たい|せい}のほゞ
\ruby{定}{さだ}まりたるに、
\ruby{無念}{む|ねん}の
\ruby{牙}{きば}を
\ruby{咬}{か}み
\ruby{血眼}{ちま|なこ}を
\ruby{瞋}{いか}らして、
\ruby{大童}{おほ|わらは}になつて
\ruby{奮闘}{ふん|とう}すれども、
\ruby{疲}{つか}れきつたる
\ruby{身}{み}の
\ruby{思}{おも}ふに
\ruby{任}{まか}せねば、
\ruby{天{\換字{運}}}{てん|うん}いよ〳〵
\ruby{我}{われ}に
\ruby{恵}{めぐ}まずば
\ruby{屍}{かばね}を
\ruby{原頭}{げん|とう}に
\ruby{曝}{さら}すも
\ruby{今}{いま}の
\ruby{間}{ま}ならんと、
\ruby{覺悟}{かく|ご}の
\ruby{臍}{ほぞ}を
\ruby{固}{かた}めつゝも、あはれ
\ruby{一}{ひ}ト
\ruby{暴風}{あ|れ}もあれかしと
\ruby{祈}{いの}り
\ruby{居}{ゐ}けるに、
\ruby{昨日}{きの|ふ}の
\ruby{芝浦}{しば|うら}の
\ruby{會}{くわい}の
\ruby{席上}{せき|じやう}より、
\ruby{羽{\換字{勝}}}{は|がち}の
\ruby{豫言}{こと|ば}といひ
\ruby{星}{ほし}の
\ruby{光}{ひかり}と
\ruby{云}{い}ひ、
\ruby{頼}{たの}もしく
\ruby{思}{おも}はるゝ
\ruby{節}{ふし}の
\ruby{少}{すくな}からぬを
\ruby{知}{し}つて、
\ruby{危}{あや}ぶみながらも
\ruby{待}{ま}ち
\ruby{居}{ゐ}たりし
\ruby{其風}{その|かぜ}の、
\ruby{果}{はた}して
\ruby{獵々颯々}{れふ|〳〵|さつ|〳〵}として
\ruby{吹}{ふ}き
\ruby{出}{いだ}したるに、
\ruby{今}{いま}
\ruby{見}{み}よ
\ruby{敗}{やぶれ}を
\ruby{轉}{てん}じて
\ruby{{\換字{勝}}}{かち}となさんは
\ruby{瞬}{またゝ}く
\ruby{間}{ま}なり、
\ruby{盛}{せ}り
\ruby{{\換字{返}}}{かへ}して
\ruby{鏖殺}{みな|ごろし}にして
\ruby{吳}{く}れんと、
\ruby{駒}{こま}の
\ruby{頭}{かしら}を
\ruby{立直}{たて|なほ}して
\ruby{鞍蓋}{くら|かさ}に
\ruby{突立上}{つゝ|たち|あが}つたる
\ruby{將軍}{しやう|ぐん}の
\ruby{意氣既}{い|き|すで}に
\ruby{疾}{はや}く
\ruby{敵}{てき}を
\ruby{吞}{の}んで
\ruby{槊}{さく}を
\ruby{横}{よこた}へて
\ruby{眼}{め}も
\ruby{遙}{はる}かに
\ruby{睥睨}{へい|げい}するが
\ruby{如}{ごと}く、
\ruby{勃々}{ぼつ|〳〵}たる
\ruby{英氣}{えい|き}と
\ruby{限}{かぎ}り
\ruby{無}{な}き
\ruby{活力}{くわつ|りよく}との、
\ruby{溢}{あふ}るゝばかり
\ruby{身}{み}に
\ruby{湧}{わ}くを
\ruby{覺}{おぼ}えて、
\ruby{流石}{さす|が}の
\ruby{島木}{しま|き}も
\ruby{押包}{おし|つゝ}み
\ruby{{\換字{兼}}}{か}ねつ、
\ruby{數聲}{すう|せい}の
\ruby{笑}{わらひ}を
\ruby{漏}{も}ら\換字{志}しなりけり。

\ruby{死生存亡}{し|せい|そん|ばう}
\ruby{此}{こ}の
\ruby{一擧}{いつ|きよ}と、
\ruby{鎬}{しのぎ}を
\ruby{{\換字{削}}}{けづ}つて
\ruby{爭}{あらそ}ふべき
\ruby{戰鬪}{たゝ|かひ}は、
\ruby{今}{いま}より
\ruby{二三時間}{に|さん|じ|かん}の
\ruby{後}{のち}に
\ruby{逼}{せま}り
\ruby{居}{を}れり。
\ruby{島木}{しま|き}は
\ruby{重}{おも}げなる
\ruby{身}{み}を
\ruby{無{\換字{造}}作}{む|ざう|さ}に
\ruby{動}{うご}かして、
\ruby{自}{みづか}ら
\ruby{押入}{おし|いれ}より
\ruby{夜具取}{や|ぐ|と}り
\ruby{出}{いだ}しつ、ごろりと
\ruby{其}{そ}れにくるまりて、
\ruby{横}{よこ}になるが
\ruby{早}{はや}きか
\ruby{頓}{やが}て
\ruby{睡}{ねむ}りぬ。
\ruby{島木}{しま|き}は
\ruby{自}{みづか}ら
\ruby{敎}{をし}へ
\ruby{自}{みづか}ら
\ruby{養}{やしな}ひて、
\ruby{敎}{をし}へ
\ruby{得}{え}
\ruby{養}{やしな}ひ
\ruby{得}{え}たるところある
\ruby{男}{をとこ}なりけり。

\ruby{風}{かぜ}は
\ruby{次第}{し|だい}に
\ruby{烈}{はげ}しくなりぬ。
\ruby{鼾}{いびき}は
\ruby{漸}{やうや}く
\ruby{盛}{さかん}になりぬ。
\ruby{風}{かぜ}の
\ruby{息}{や}む
\ruby{時}{とき}、
\ruby{鼾}{いびき}の
\ruby{聲}{こゑ}あり、
\ruby{鼾}{いびき}の
\ruby{無}{な}き
\ruby{時}{とき}、
\ruby{風}{かぜ}の
\ruby{音}{おと}あり。
\ruby{開}{ひら}き
\ruby{放}{はな}されたる
\ruby{押入}{おし|いれ}、
\ruby{投出}{なげ|だ}されたる
\ruby{酒瓶}{とつ|くり}、
\ruby{{\換字{消}}}{き}えたる
\ruby{洋燈}{らん|ぷ}、
\ruby{空虛}{か|ら}の
\ruby{罐}{くわん}、
\ruby{歪}{いびつ}に
\ruby{展}{の}べられたる
\ruby{蒲團}{ふ|とん}、
\ruby{明}{あ}けかけたる
\ruby{雨{\換字{戸}}}{あま|ど}、
\ruby{雷}{らい}の
\ruby{如}{ごと}き
\ruby{鼾聲}{い|びき}、
\ruby{波濤}{な|み}と
\ruby{轟}{とどろ}く
\ruby{風}{かぜ}の
\ruby{音}{おと}、
\ruby{埒無}{らち|な}しとも
\ruby{{\換字{狼}}{\換字{藉}}}{らう|ぜき}とも
\ruby{亂暴}{らん|ばう}とも、
\ruby{云}{い}ふべき
\ruby{言葉}{こと|ば}は
\ruby{無}{な}き
\ruby{一室}{いつ|しつ}の
\ruby{狀}{さま}なり。


\Entry{其二十八}

% メモ 校正終了 2024-04-10 2024-05-27 2024-06-19
\原本頁{169-2}%
『
\ruby{甚}{ひど}く
\ruby[g]{寢{\換字{込}}}{ね こ }んで
\ruby{居}{ゐ}たぢや
\ruby{無}{な}いか。
』

\原本頁{169-3}%
と、
%
\ruby{其}{その}
\ruby[||j>]{頭}{かしら}に
\ruby{黄金細工}{き|ん|ざい|く}の
\ruby{施}{ほどこ}しある
\ruby{美}{うつく}しき
\ruby[g]{琥珀}{こ はく}の
パイプを
\ruby{口}{くち}より
\ruby{放}{はな}しさまに
\ruby{云}{い}ひたるは、
%
\ruby[g]{梅幸}{ばいかう}の% 歌舞伎役者の尾上梅幸のことと思われる
\ruby[g]{伊東}{い とう}と
\ruby[g]{渾名}{あだな }
\ruby{呼}{よ}ばるゝも
\ruby[g]{無理}{む り }ならず
\原本頁{169-5}\改行%
\ruby{見}{み}ゆる
\ruby{其}{その}
\ruby{人}{ひと}を
\ruby{其}{その}
\ruby{儘}{まゝ}の
\ruby[g]{面立}{おもだち}の、
%
\ruby{三十三四}{さん|じふ|さん|し}の
\ruby{色}{いろ}
\ruby{白}{しろ}き
\ruby{男}{をとこ}にて、
%
\ruby[g]{昨夜}{ゆふべ }を
\原本頁{169-6}\改行%
\ruby[g]{何處}{いづく }にてか
\ruby{{\換字{過}}}{すご}しての
\ruby[g]{今{\換字{朝}}}{け さ }、
%
\ruby{何}{なに}か
\ruby[g]{用事}{ようじ }ありて
\ruby[g]{此家}{こ ゝ }に
たち
\ruby{戾}{もど}りしが
\改行% 校正作業の簡略化のため
、
%
\原本頁{169-7}\改行%
\ruby{今}{いま}や
\ruby{既}{すで}に
\ruby[g]{{\換字{朝}}食}{あさげ }を
\ruby{濟}{す}ませて
\ruby{{\換字{率}}}{いざ}
これよりと、
%
\ruby[g]{戰鬪}{たゝかひ}の
\ruby{場}{ば}へ% 原文通り「場」
\ruby{赴}{おもむ}かんとする
\ruby{{\換字{前}}}{まへ}の
\ruby[g]{僅少}{わづか }の
\ruby{暇}{いとま}を、
%
\ruby[g]{{\換字{煙}}草}{たばこ }
\ruby{休}{やす}みに
\ruby[g]{島木}{しまき }の
\ruby{室}{しつ}に
\ruby{來}{き}て、
%
\ruby[g]{昨日}{きのふ }
\ruby[g]{今日}{け ふ }
こそ
\原本頁{169-10}\改行%
\ruby{敵}{てき}
\ruby[g]{味方}{み かた}と
\ruby[g]{立別}{たちわか}れて
こそあれ
\ruby{同}{おな}じ
\ruby[g]{修羅}{しゆら }の
\ruby{巷}{ちまた}の
\ruby{友}{とも}の
\ruby[g]{相語}{あひかた}らへるなり
\改行% 校正作業の簡略化のため
。

\原本頁{169-11}%
『
アヽ、
%
ちよいと
\ruby{寢}{ね}やうと
\ruby{思}{おも}つたけが、
%
つい
ぐつすりと
\ruby{寢}{ね}て
\ruby[g]{仕舞}{し ま }つた。
』

\原本頁{170-2}%
『
\ruby{{\換字{宵}}}{よひ}にやあ
\ruby{汝}{おめへ}
\ruby[||j>]{寢}{ ね }られなかつたナ。
』

\原本頁{170-3}%
\ruby{微}{かすか}に
\ruby[g]{冷笑}{あざわら}ふ
\ruby[g]{樣子}{やうす }の
\ruby[<j>]{唇}{くちびる}の
\ruby{端}{はた}に
\ruby{見}{み}ゆるを、
%
\ruby{見}{み}て
\ruby{取}{と}つたる
\ruby[g]{島木}{しまき }は
\ruby[g]{一寸}{いつすん}も
\ruby{{\換字{退}}}{ひ}けては
\ruby{居}{ゐ}ず。

\原本頁{170-5}%
『
\ruby[g]{馬鹿}{ば か }あ
\ruby{云}{い}へ。
%
\ruby{汝}{おめへ}のやうな
\ruby[g]{繊細}{か ぼそ}い
\ruby[g]{野郎}{や らう}ぢやあ
\ruby{有}{あ}るめえし、
%
そんな
\ruby[g]{卑小}{け ち }な
\ruby[||j>]{根}{こん}
\ruby[||j>]{性}{じやう}は
% \ruby{根性}{こん|じやう}は
\ruby{持}{も}たねえ
\ruby{萬五郎}{まん|ご|らう}さまだ。
%
お
\ruby{作}{さく}に
\ruby{聞}{き}いて
\ruby{見}{み}りやあ
\原本頁{170-7}\改行%
\ruby{解}{わか}る
\ruby{事}{こと}だ。
』

\原本頁{170-8}%
『
ハヽヽ、
%
\ruby[g]{豪氣}{がうぎ }に
\ruby[g]{今{\換字{朝}}}{け さ }は
\ruby{氣}{き}が
\ruby{{\換字{強}}}{つよ}いナ。
%
\ruby[g]{背後}{うしろ }から
\ruby{風}{かぜ}が
\ruby{推}{お}してるからナア。
』

\原本頁{170-10}%
『
フン、
%
\ruby[g]{{\換字{嫌}}味}{いやみ }を
\ruby{言}{い}ひやがる!。
%
よ\換字{志}にしろ、
%
\ruby{男}{をとこ}が
\ruby{下}{さが}るぜ、
%
\ruby{下}{くだ}らねえ。
』

\原本頁{171-1}%
どつと
\ruby{吹}{ふ}く
\ruby{風}{かぜ}の
\ruby{音}{おと}、
%
ひゆーと
\ruby{鳴}{な}る
\ruby{物}{もの}の
\ruby[g]{叫聲}{さけび }、
%
\ruby[g]{二人}{ふたり }が
\ruby{居}{を}れる
\ruby{此}{この}
\ruby{樓}{ろう}も、
%
ゆらりと
\ruby{今}{いま}は
\ruby{一}{ひ}ト
\ruby{搖}{ゆら}ぎして、
%
\ruby[g]{一切}{いつさい}の
\ruby{物}{もの}
\ruby{皆}{みな}
\ruby{震}{ふる}ひ
\ruby{動}{うご}けば、
%
ものこそ
\ruby{言}{い}はね
\ruby[g]{伊東}{い とう}が
\ruby{眉}{まゆ}は
ぴりゝと
\ruby{縮}{ちゞ}みて、
%
\ruby{心}{こゝろ}の
\ruby{安}{やす}からぬを
\ruby{現}{あら}はしたり。

\原本頁{171-5}%
『
\ruby[g]{何樣}{ど う }した?% \inhibitglue{}% ここは「空き」があるので
\,% 原本上でのアキを再現するため「3/18 em」空ける
\ruby[g]{梅幸}{ばいかう}!。
%
\ruby{氣}{き}が
\ruby{揉}{も}めるか?。
』

\原本頁{171-6}%
\ruby{{\換字{前}}}{まへ}の
\ruby[g]{{\換字{返}}報}{しかへし}と
\ruby[g]{島木}{しまき }が
\ruby[g]{調戲}{からか }へば、
%
\ruby{此}{これ}も
\ruby[g]{男兒}{をとこ }なり、
%
\ruby[||j>]{癇}{かん}
\ruby[||j>]{癪}{しやく}らしく
% \ruby{癇癪}{かん|しやく}らしく
\ruby{{\換字{煙}}}{けむり}を
\ruby{吐}{は}きて、

\原本頁{171-8}%
『
\ruby{高}{たか}が
\ruby[g]{此樣}{こ ん }な
\ruby{無雨之風}{か|ら|つ|かぜ}!。
%
\ruby{何}{なに}が
\ruby{怖}{こは}い。
』

\原本頁{171-9}%
と、
%
\ruby{只}{たゞ}
\ruby[g]{一言}{ひとこと}に
\ruby{云}{い}ひ
\ruby{{\換字{消}}}{け}しつ、
%
\ruby{{\換字{強}}}{しひ}て
\ruby{笑}{わら}つて、

\原本頁{171-10}%
『
\換字{志}かし
\ruby[g]{中々}{なか〳〵}
\ruby{吹}{ふ}きやがるナ。
%
\ruby{汝}{おめへ}こそ
\ruby[g]{内々}{ない〳〵}
\ruby{嬉}{うれ}しからう!。
%
\ruby{曲}{まが}り
\ruby{屋}{や}さんが
\ruby[g]{立直}{たちなほ}つて
\ruby{來}{き}さうだぜ。
』

\原本頁{172-1}%
と、
%
\ruby{云}{い}ひ
\ruby{足}{た}したり。

\原本頁{172-2}%
『
\ruby[g]{左樣}{さ う }さ、
%
いつまで
\ruby{曲}{まが}り
つゞけで
\ruby{堪}{たま}るもんか。
%
\ruby{汝}{おめへ}
ばかりに
\ruby{當}{あた}つて
\ruby{居}{ゐ}られる
\ruby[g]{世界}{せ かい}ぢやあ
\ruby{無}{ね}え。
%
たまにやあ
\ruby[g]{此樣}{こ ん }な
\ruby{風}{かぜ}も
\ruby{吹}{ふ}いて
\ruby{吳}{く}れなくつちやあ!。
』

\原本頁{172-5}%
『
\ruby[||j>]{憫}{かは}
\ruby[||j>]{然}{いさう}に、% 「憫然 か(は)いさう」
% \ruby{憫然}{かは|いさう}に、% 「憫然 か(は)いさう」
%
\ruby{{\換字{空}}}{そら}を
\ruby{見}{み}ちやあ
\ruby[<j||]{百}{ひやく}
\ruby[||j>]{姓}{しやう}
なんぞが
\ruby[g]{何程}{いくら }
\ruby{泣}{な}いてるか
\ruby{知}{し}れや
\ruby{仕}{し}ない!。
%
\ruby{汝}{おめへ}は
もつと
\ruby{吹}{ふ}け
\ruby{位}{ぐらゐ}に
\ruby{思}{おも}つて
\ruby{居}{ゐ}るだらうが。
』

\原本頁{172-7}%
『
\ruby[||j>]{當}{あたり}
\ruby[||j>]{然}{ まへ}よ。
% \ruby{當然}{あたり|まへ}よ。
%
\ruby{吹}{ふ}いて
\ruby{吹}{ふ}いて
\ruby{吹}{ふ}き
\ruby{拔}{ぬ}けと
\ruby{思}{おも}つて
\ruby{居}{ゐ}るんだ!。
%
\ruby[<j||]{農}{ひやく}
\ruby[||j>]{夫}{しやう}が
\ruby{泣}{な}いたつて
\ruby{笑}{わら}つたつて
\ruby{構}{かま}ふもんか!。
%
\ruby[g]{早稻}{わ せ }も
\ruby[g]{晩稻}{おくて }も
\ruby{吹}{ふ}き
\ruby{飛}{と}んで
\原本頁{172-9}\改行%
\ruby[g]{仕舞}{し ま }へと
\ruby{思}{おも}つて
\ruby{居}{ゐ}るんだ。
』

\原本頁{172-10}%
『
いゝ
\ruby{蟲}{むし}だナア!。
%
\ruby{酷}{ひど}い
\ruby[g]{野郎}{や らう}だぞ!。
%
\ruby{他}{ひと}に
\ruby[<j||]{百}{ひやく}
\ruby[||j>]{兩}{りやう}の
\ruby{損}{そん}を
させても、
%
\ruby[g]{自己}{う ぬ }が
\ruby[||j>]{一}{いち}
\ruby[||j>]{兩}{りやう}
% \ruby{一兩}{いち|りやう}
\ruby{儲}{ まう}けりやあ% ルビ調整(長いルビ対策)「一兩(いちりよう)」のルビが一文字突き出てくる為
\ruby{好}{い}いといふ
\ruby[g]{料簡}{れうけん}
\ruby{方}{かだ}だ。% ルビ調整(原本通り)(か(だ))
』

\原本頁{173-1}%
『
ナンダ、
%
\ruby{惡}{わる}く
\ruby[g]{素人}{しろうと}くせえ
\ruby{事}{こと}を
\ruby{吐}{ぬ}かしやがる!。
%
\ruby{今}{いま}の
\ruby[g]{世界}{せ かい}で
\ruby{金}{かね}を
\ruby{儲}{まう}けて
\ruby[g]{大顏}{おほづら}を
\ruby{仕}{し}て
\ruby{居}{ゐ}る
\ruby{奴}{やつ}に、
%
\ruby{唯}{たゞ}の
\ruby[g]{一人}{ひとり }でも
\ruby{其}{そ}の
\ruby[g]{料簡}{れうけん}で
\ruby{無}{ね}え
\ruby{奴}{やつ}が
\ruby{有}{あ}るものかい!。
%
\ruby{大}{おほき}な
\ruby[||j>]{門}{もん}
\ruby[||j>]{構}{がまへ}を
% \ruby{門構}{もん|がまへ}を
\ruby{仕}{し}て
\ruby{居}{ゐ}る
\ruby{奴}{やつ}あ、
%
\ruby[g]{悉皆}{みんな }
いゝ
\ruby{蟲}{むし}に
\ruby{羽}{はね}が
\ruby{生}{は}へたのぢや
\ruby{無}{ね}えか!。
』

\原本頁{173-5}%
『
ハヽヽ、
%
\ruby[g]{{\換字{違}}無}{ちげへね}え!。
%
\ruby{言}{い}つて
\ruby{見}{み}りやあ
まあ
\ruby[g]{其樣}{そ ん }なもんだ。
%
\ruby{併}{{\換字{志}}か}し
\ruby{汝}{おめへ}は
\ruby[g]{{\換字{平}}常}{ふだん }から、
%
\ruby[||j>]{觀}{くわん}% 「觀音」の読みは原本通り「くわん(の)ん」
\ruby[||j>]{音}{ のん}
% \ruby{觀音}{くわん|のん}% 「觀音」の読みは原本通り「くわん(の)ん」
なんぞを
\ruby[g]{信心}{しん〴〵}して
\ruby{居}{ゐ}るが
\ruby{彼}{あり}やあ
\ruby{何}{なん}だ!、
%
\ruby[g]{矢張}{やつぱ }り
\ruby[||j>]{觀}{くわん}% 「觀音」の読みは原本通り「くわん(の)ん」
\ruby[||j>]{音}{ のん}
\ruby[||j>]{樣}{ さま}
% \ruby{觀音}{くわん|のん}% 「觀音」の読みは原本通り「くわん(の)ん」
を
\ruby[g]{取捉}{とつつか}めへても、
%
\ruby[g]{其樣}{そ ん }な
あこぎな
\ruby[g]{料簡}{れうけん}で
もつて、
%
\ruby[g]{金持}{かねもち}になるやうと
\ruby{祈}{いの}つて
\ruby{居}{ゐ}るのか?。
』

\原本頁{173-9}%
『
ムヽ、
%
\ruby{他}{ほか}に
\ruby{祈}{いの}らうことは
\ruby{無}{ね}えぢやあ
\ruby{無}{ね}えか!。
』

\原本頁{173-10}%
『
ぢやあ
\ruby{惡}{わる}い
\ruby[g]{暴風}{あ れ }も
\ruby{祈}{いの}りかね
\ruby{無}{ね}えが、
%
そんな
\ruby[g]{我欲}{が よく}の
\ruby{願}{ねがひ}を
\ruby{掛}{か}けたつて、
%
\ruby[||j>]{觀}{くわん}% 「觀音」の読みは原本通り「くわん(の)ん」
\ruby[||j>]{音}{ のん}
% \ruby{觀音}{くわん|のん}% 「觀音」の読みは原本通り「くわん(の)ん」
は
\ruby[g]{正路}{しやうろ}の
\ruby{佛}{ほとけ}
ださうだぜ。
』

\原本頁{174-1}%
『
ナニ
\ruby[g]{乃公}{お ら }の
\ruby[||j>]{觀}{くわん}% 「觀音」の読みは原本通り「くわん(の)ん」
\ruby[||j>]{音}{ のん}
% \ruby{觀音}{くわん|のん}% 「觀音」の読みは原本通り「くわん(の)ん」
は
\ruby[g]{乃公}{お ら }の
\ruby[||j>]{觀}{くわん}% 「觀音」の読みは原本通り「くわん(の)ん」
\ruby[||j>]{音}{ のん}
% \ruby{觀音}{くわん|のん}% 「觀音」の読みは原本通り「くわん(の)ん」
だ!。
%
\ruby{汝}{おめへ}の
\ruby[||j>]{觀}{くわん}% 「觀音」の読みは原本通り「くわん(の)ん」
\ruby[||j>]{音}{ のん}
% \ruby{觀音}{くわん|のん}% 「觀音」の読みは原本通り「くわん(の)ん」
たあ
\ruby{異}{ちが}つたつて
\ruby{管}{かま}はねえ。
%
\ruby[g]{乃公}{お ら }あ
\ruby[g]{乃公}{お ら }で
\ruby{濟}{す}んでるんだから、
%
これで
\ruby{可}{い}いんだ。
』

\原本頁{174-3}%
『
\ruby{何}{なん}だか
\ruby[g]{{\換字{道}}理}{す ぢ }が
\ruby{{\換字{通}}}{とほ}らねえやうだが、
アツ、
%
また
\ruby{吹}{ふ}きやあがる、
%
\原本頁{174-4}\改行%
\ruby{甚}{ひど}くなつて
\ruby{來}{き}たぞ。
%
オヽ
\ruby{塀}{へい}が
\ruby{飛}{と}んだぞ、
%
\ruby[||j>]{棟}{むな}
\ruby[||j>]{{\換字{瓦}}}{がはら}が
% \ruby{棟{\換字{瓦}}}{むな|がはら}が
\ruby{落}{お}ちたぞ!。
』

\原本頁{174-5}%
『
どうだ
\ruby[g]{{\換字{情}}無}{なさけな}いか、
%
\ruby[g]{心配}{しんぱい}か!。
』

\原本頁{174-6}%
『
\ruby[g]{馬鹿}{ば か }あ
\ruby{云}{い}ふな、
%
\ruby[g]{篦棒}{べらぼう}ナ!。
%
\ruby[g]{天{\換字{運}}}{う ん }は
\ruby[g]{何樣}{ど う }
\ruby[g]{循環}{ま は }つたつて
\ruby[g]{手腕}{う で }は
\ruby[g]{手腕}{う で }だ!。
%
\ruby[||j>]{{\換字{逆}}}{むかひ}
\ruby[||j>]{風}{ かぜ}を
% \ruby{{\換字{逆}}風}{むかひ|かぜ}を
\ruby{乘}{の}つ
\ruby{切}{き}つて
\ruby[g]{腕{\換字{前}}}{うでまへ}を
\ruby{見}{み}せてやらあ。
%
\ruby[g]{此方}{こちら }あ% ルビ調整(原本通り)
\ruby[g]{昨夜}{ゆふべ }
\ruby{辨天樣}{べん|てん|さま}に、% 弁 瓣 辦 辧 (辨) 辩 辯
\換字{志}たゝか
お
\ruby[g]{賽錢}{さいせん}を
\ruby{献}{あ}げて
\ruby{來}{き}たんだ、
%
はゞかりながら
\ruby{辨天樣}{べん|てん|さま}が% 弁 瓣 辦 辧 (辨) 辩 辯
\ruby{付}{つい}て
\ruby{居}{ゐ}るんだ!。
』

\原本頁{174-10}%
\ruby{笑}{わら}ひながら
\ruby{云}{い}ひたる
\ruby{末}{すゑ}の
\ruby[g]{言葉}{ことば }は、
%
\ruby{暗}{あん}に
\ruby[g]{自己}{おのれ }が
\ruby[g]{昨夜}{きのふ }の
\ruby[g]{豪{\換字{遊}}}{がういう}を
\ruby{誇}{ほこ}つて、
%
\ruby[g]{{\換字{遊}}謔}{たはむれ}の
\ruby{中}{うち}にも
\ruby{威}{ゐ}を
\ruby{張}{は}りて、
%
\ruby{聊}{いさゝ}か
\ruby{自}{みづか}ら
\ruby{{\換字{強}}}{つよ}うせるなり。

\原本頁{175-1}%
『
\ruby{何}{なん}だ、
%
\ruby{薄}{うす}ら
\ruby[<j>]{腥}{なまぐさ}い
\ruby[g]{辨天}{べんてん}が% 弁 瓣 辦 辧 (辨) 辩 辯
\ruby{何}{なに}が
ありがたい\換字{!?}。
%
\ruby[g]{此方}{こつち }あ% ルビ調整(原本通り)
\ruby[||j>]{淸}{しやう}
\ruby[||j>]{淨}{ 〴〵}な
\ruby[g]{仙人}{せんにん}に
お
\ruby[g]{初穂}{はつほ }が
\ruby{献}{あ}げてあるんだ!。
%
\ruby[g]{今日}{け ふ }は
\ruby{此}{こ}の
\ruby[g]{乃公}{お ら }が
\ruby[g]{大當}{おほあた}りだ!。
』

\原本頁{175-3}%
これも
\ruby{私}{ひそか}に
\ruby{自}{みづか}ら
\ruby{快}{こゝろ}よし
とするところあるなり。

\原本頁{175-4}%
『
ナニ
\ruby{此}{こ}の
\ruby{鼻}{はな}が
\ruby[g]{矢張}{やつぱ }り
\ruby{當}{あた}る!。
』

\原本頁{175-5}%
『
ナニ
\ruby{此}{こ}の
\ruby{乃公樣}{お|れ|さま}が
\ruby[g]{屹度}{きつと }
\ruby{當}{あた}る!。
』

\原本頁{175-6}%
『
おれが、
』

\原本頁{175-7}%
『
おれが、
』

\原本頁{175-8}%
『
ハヽハヽ、
』

\原本頁{175-9}%
『
ハヽハヽ、
』

\原本頁{175-10}%
『
なにも
\ruby[g]{此處}{こ ゝ }で
\ruby[g]{喧嘩}{けんくわ}あ
\ruby{爲}{す}る
\ruby{事}{こと}も
\ruby{無}{ね}え。
』

\原本頁{175-11}%
『
\ruby[g]{二人}{ふたり }とも
\ruby{當}{あた}らう!。
』

\原本頁{176-1}%
『
\ruby[g]{夕方}{ゆふがた}までだ!。
』

\原本頁{176-2}%
\ruby{風}{かぜ}は
いよ〳〵
\ruby{狂}{くる}へる
\ruby{中}{なか}を、
%
\ruby[g]{二人}{ふたり }は
おのれおのれが% 原本でも通り字になっていないのでそのままとする
\ruby[g]{本陣}{ほんじん}へと、
%
\ruby[g]{勇威}{いきほひ}を
\ruby{含}{ふく}んで
\ruby[g]{立出}{たちい }でたり。
%
\ruby{風}{かぜ}も
\ruby{慾}{よく}に
\ruby{使}{つか}はるゝ
\ruby{人}{ひと}の
\ruby{世}{よ}の
\ruby{中}{なか}や。

\Entry{其二十九}

% メモ 校正終了 2024-04-10
\原本頁{176-5}%
\ruby{其}{そ}の
\ruby{日}{ひ}
\ruby{晝}{ひる}を
\ruby{{\換字{過}}}{す}ぎて
\ruby{風}{かぜ}
いよ〳〵
\ruby{烈}{はげ}しく、
%
\ruby{天}{そら}は
\ruby{塵埃}{ぢん|あい}に
\ruby{濁}{にご}れるが
\ruby{如}{ごと}くに
\ruby{一面}{いち|めん}の
\ruby{黄雲}{くわう|ゝん}に
\ruby{包}{つゝ}まれて、
%
\ruby{常}{つね}
ならぬ
\ruby[<g>]{{\換字{暖}}氣}{あたゝかさ}の
\ruby{氣味}{き|み}
\ruby{惡}{あし}ければ、
%
\ruby{人}{ひと}
\ruby{皆}{みな}
\原本頁{176-7}\改行%
\ruby{安}{やす}き
\ruby{心}{こゝろ}も
\ruby{無}{な}くて、
%
\ruby{{\換字{若}}}{も}し
\ruby{此}{この}
\ruby{上}{うへ}に
\ruby{雨}{あめ}も
\ruby{混}{まじ}らばと
\ruby{氣{\換字{遣}}}{き|づか}ふ
\ruby{折}{をり}しも、
%
\ruby{頭上}{づ|じやう}の
\ruby{雲}{くも}
やうやく
\ruby{墨色}{すみ|いろ}
さして、
%
\ruby{蔽}{おほ}ひかぶさる
\ruby{樣}{よう}に
\ruby{昏}{くら}くなれば、
%
\ruby{如何}{い|か}になり
\ruby{行}{ゆ}く
\ruby{{\換字{魔}}日}{ま|び}ぞと
\ruby{誰}{たれ}しも
\ruby{恐}{おそ}れあひぬ。
%
\ruby{事}{こと}
\ruby{無}{な}くて
\ruby{家}{いへ}にある
\ruby{爺}{ぢゞ}
\ruby{媼}{ばゞ}さへ
\ruby{是}{かく}の
\ruby{如}{ごと}くなれば、
%
まして、
%
\ruby{{\換字{遣}}}{や}らん
\ruby{買}{か}はんの
\ruby{呼}{よ}び
\ruby{聲}{ごゑ}は
\ruby{戰場}{せん|じやう}の% 原文通り「場」
\原本頁{177-1}\改行%
\ruby{矢叫}{や|さけ}びと
\ruby{入}{い}り
\ruby{亂}{みだ}れて、
%
\ruby{打振}{うち|ふ}る
\ruby{兩手}{りやう|て}は
\ruby{浪}{なみ}
\ruby{寄}{よ}る
\ruby{尾花}{を|ばな}と
\ruby{{\換字{空}}}{そら}に
\ruby{揉}{も}まるゝ
\原本頁{177-2}\改行%
\ruby{其}{その}
\ruby{場}{ば}の% 原本通り「場」
\ruby{混亂}{こん|らん}は、
%
\ruby{猜}{すゐ}するにも
\ruby{{\換字{猶}}}{なほ}
\ruby{餘}{あまり}あり。

\原本頁{177-3}%
\ruby{伊東}{い|とう}は
いづれへ
\ruby{逸}{そ}れしにや
\ruby{歸}{かへ}り
\ruby{來}{きた}らねど、
%
\ruby{雨}{あめ}
\ruby{下}{お}りんとして
\ruby{下}{お}りず
\ruby{風}{かぜ}
\ruby[||j>]{衰}{おとろ}へんとして
\ruby{衰}{おとろ}へぬ
\ruby{夕{\換字{近}}}{ゆうべ|ちか}く、
%
\ruby{島木}{しま|き}は
\ruby{悠然}{いう|ぜん}として
\ruby{歸}{かへ}り
\ruby{來}{きた}りぬ。
%
\原本頁{177-5}\改行%
\ruby{島木}{しま|き}に
つゞきて
\ruby{上}{あが}り
\ruby{來}{きた}れる
\ruby{婢}{をんな}は、
%
\ruby{例}{れい}となり
\ruby{居}{を}れると
\ruby{見}{み}えて
\ruby{茶}{ちや}を
\原本頁{177-6}\改行%
\ruby{入}{い}れて
\ruby{薦}{すゝ}めつ。

\原本頁{177-7}%
『
\ruby{伊東}{い|とう}さんは?、
%
\ruby{御存知}{ご|ぞん|ぢ}
\ruby{無}{な}くつて?。
』

\原本頁{177-8}%
『
\ruby{知}{し}らねえよ、
%
\ruby{一{\換字{所}}}{いつ|しよ}ぢやあ
\ruby{無}{ね}えから。
%
\換字{志}かし
おほかた
\ruby{彼女}{あ|れ}の
ところだらう。
』

\原本頁{177-10}%
『ほんとに
\ruby{凝}{こ}つて
\ruby{行}{い}らつしやるのネ!。
%
\ruby{幸{\換字{運}}}{い|ゝ}に
つけても、
%
\ruby{惡{\換字{運}}}{わる|い}に
\ruby{付}{つ}けてもネエ!。
』

\原本頁{178-1}%
『ウン。
%
ハヽ、
%
\ruby{今日}{け|ふ}は
\ruby{幸{\換字{運}}}{い|ゝ}に
つけてもぢやあ
\ruby{無}{な}さゝうだ!。
%
でも
\ruby{彼女}{あ|れ}の
\ruby{方}{はう}でも
\ruby{招}{よ}ぶやうだから
\ruby{堪}{たま}らねえや。
%
\ruby{汝}{おめへ}も
\ruby{女}{をんな}の
\ruby{端}{はし}くれだ、
%
\原本頁{178-3}\改行%
どうだ、
%
\ruby{些}{ちつと}あ
\ruby{妬}{や}けるかい?。
』

\原本頁{178-4}%
『
\ruby{何}{なん}ですつて、
%
\ruby{端}{はし}くれですつて?。
%
あんまり
\ruby{酷}{ひど}い
\ruby{事}{こと}ネ。
%
ようござんすよ、
%
たんと
\ruby{惡口}{わる|くち}を
\ruby{仰}{おつしや}いまし、
%
\ruby{告訴}{い|つけ}て
\ruby{{\換字{遣}}}{や}るとこを
\ruby{知}{し}つてますから。
%
ア、
%
そりやあ
\ruby{左樣}{そ|う}と
\ruby{貴君}{あな|た}は
\ruby{今日}{け|ふ}は
\ruby{大當}{おほ|あた}りでしやう。
%
あなたも
\ruby{男兒}{をと|こ}の
\ruby{端}{はし}くれだ、
%
\ruby{些}{ちつと}あ
\ruby{氣{\換字{前}}}{き|まへ}を
\ruby{見}{み}せて
\ruby{御奢}{お|おご}んなさいな。
%
\ruby{風}{かぜ}の
\ruby{音}{おと}を
\ruby{聞}{き}いちやあ
\ruby{主{\換字{婦}}}{おか|み}さんと
\ruby{一日}{いち|にち}
\ruby{云}{い}ひ
\ruby{暮}{く}らして
\ruby{居}{ゐ}ましたよ。
』

\原本頁{178-9}%
『
\ruby{左樣}{さ|う}かい、
%
\ruby{其奴}{そ|いつ}あ
\ruby{頼}{たの}もしかつた!。
%
\ruby{奢}{おご}つて
\ruby{{\換字{遣}}}{や}らう。
』

\原本頁{178-10}%
『オヤ、
%
\ruby{其}{それ}あ
\ruby{早{\換字{速}}}{さつ|そく}に
\ruby{有}{あ}り
\ruby{{\換字{難}}}{がた}う!。
%
さうして
\ruby{何}{なに}を
\ruby{奢}{おご}つて
\ruby{下}{くだ}さる?。
』

\原本頁{179-1}%
『
\ruby{生憎}{あひ|にく}
\ruby{劇塲}{しば|ゐ}は% 原文通り「塲」
\ruby{好}{い}いところが
\ruby{開}{あ}いて
\ruby{居}{ゐ}ねえナ。
』

\原本頁{179-2}%
『さうネエ。
』

\原本頁{179-3}%
『
\ruby{秋草}{あき|くさ}も
\ruby{今日}{け|ふ}の
\ruby{此}{こ}の
\ruby{風}{かぜ}ぢやあもう。
』

\原本頁{179-4}%
『さうネエ。
』

\原本頁{179-5}%
『
\ruby{矢張}{やつ|ぱ}り
\ruby{下卑}{げ|び}でも
\ruby{甘}{あま}い
\ruby{物}{もの}と
いふところで
\ruby{堪{\換字{忍}}}{かん|にん}して% 原文通り「堪忍」
\ruby{貰}{もら}はう。
』

\原本頁{179-6}%
『さうねエ。
%
それぢやあ、
%
あの、
%
\ruby{何}{なに}を?。
』

\原本頁{179-7}%
『
\ruby{今川燒}{いま|がは|やき}の
\ruby{皮}{かは}の% 原本通り「皮 か(は)」
\ruby{厚}{あつ}い
\ruby{冷}{つめ}たいのでも。
%
ハヽハヽ。
』

\原本頁{179-8}%
『エヽ
\ruby{悔}{くや}しいヨ、
%
おぼえて
\ruby{居}{ゐ}らつしやい。
%
もう
\ruby{貴君}{あな|た}の
\ruby{云}{い}ふ
\ruby{事}{こと}は
\ruby{當}{あて}に
\ruby{仕}{し}やしない。
』

\原本頁{179-10}%
『オイ〳〵
\ruby{左樣}{さ|う}
ぷり〳〵しちやあ
\ruby{困}{こま}る。
%
\ruby{頼}{たの}む
\ruby{事}{こと}が
あるんだ、
%
\ruby{大}{おほ}まじめだ。
』

\原本頁{180-1}%
『ヘイ〳〵、
%
\ruby[g]{澤山}{たんと}
お
\ruby{使}{つか}ひなさいまし!。
%
\ruby{何}{なん}の
\ruby{御用}{ご|よう}?。
』

\原本頁{180-2}%
『
\ruby{惡}{わる}く
\ruby{角}{かく}ばるナ、
%
\ruby{怒}{おこ}つちやあ
いけねえ。
%
\ruby{好}{い}いかエ、
%
\ruby{客}{きやく}が
\ruby{一人}{ひと|り}
\ruby{來}{く}る
\ruby{筈}{はず}に
\ruby{招}{よ}んで
あるんだ。
%
\ruby{汝}{おめへ}の
\ruby{見}{み}はからひで、
%
\ruby{例}{いつも}の
\ruby{家}{うち}へでも
\ruby{電話}{でん|わ}を
かけて、
%
\ruby{手一杯}{て|いつ|ぱい}に
\ruby{御馳走}{ご|ち|そう}を
\ruby{仕}{し}て
\ruby{貰}{もら}ひてえのだ。
%
\ruby{他家}{わ|き}へ
\ruby{行}{い}くなあ
\ruby{不妙}{ま|づ}いのだから。
%
ヨ、
%
\ruby{頼}{たの}むよ。
%
\ruby{客}{きやく}が
\ruby{堅人}{かた|じん}で、
%
\ruby{話}{はなし}が
\ruby{堅}{かた}いと
\ruby{來}{き}て
\原本頁{180-6}\改行%
\ruby{居}{ゐ}るんだから。
』

\原本頁{180-7}%
『ハア、
%
\ruby{左樣}{さ|う}。
%
ようござんす。
%
\ruby{御酒}{ご|しゆ}は?。
%
\ruby[g]{麦酒}{びーる}?。
%
\ruby[g]{葡萄酒}{いつもの}?。
%
\原本頁{180-8}\改行%
さうして
\ruby{直}{ぢき}に
\ruby{御入來}{お|い|で}ですか。% 国会図書館では「おいで」、国文学研究資料館では「おい 」
』

\原本頁{180-9}%
『ウン、
%
もう
そろ〳〵
\ruby{來}{く}る
\ruby{時{\換字{分}}}{じ|ぶん}だから
\ruby{急}{いそ}いでネ。
』

\原本頁{180-10}%
『あの
\ruby{水野}{みづ|の}さんとか
\ruby{仰}{おつし}ある
\ruby{方}{かた}?。
』

\原本頁{180-11}%
『ソラ
\ruby{惚}{ほ}れて
やがるもんだから
\ruby{兎角}{と|かく}
\ruby{名}{な}を
いふ!。
%
お
\ruby{生憎樣}{あひ|にく|さま}!。
%
\原本頁{181-1}\改行%
\ruby{水野}{みづ|の}ぢやあ
\ruby{無}{ね}え、
%
\ruby{羽{\換字{勝}}}{は|がち}と
いふんだ。
%
\換字{志}かし
\ruby{色}{いろ}の
\ruby{白}{しろ}い、
%
\ruby{眼}{め}の
\ruby{優}{やさ}しい、
%
\ruby{滅法}{めつ|ぱふ}に
\ruby{好}{い}い
\ruby{男}{をとこ}だから、
%
\ruby{{\換字{又}}}{また}
\ruby[||j>]{汝}{おめへ}は
\ruby{直}{すぐ}と
\ruby{惚}{ほ}れるだらう。
』

\原本頁{181-3}%
『
\ruby{他聞}{ひと|ぎゝ}の
\ruby{惡}{わる}い!。
%
よしても
\ruby{下}{くだ}さいよ。
%
\ruby{妾}{わたし}や
\ruby{男}{をとこ}の
\ruby{美}{い}いのに
\ruby{惚}{ほ}れるやうな
\ruby{耄碌}{まう|ろく}ぢやあ
\ruby{有}{あ}りませんよ。
%
ホヽホヽ。
』

\原本頁{181-5}%
『オヤ
\ruby{異}{おつ}な
たんかを
\ruby{切}{き}りやあがる。
%
それぢやあ
\ruby{何樣}{ど|ん}な
\ruby{男}{をとこ}に
\ruby{惚}{ほ}れるんだ?。
』

\原本頁{181-7}%
『
\ruby{知}{し}れた
\ruby{事}{こと}でさアネ、
%
\ruby{明治}{めい|じ}ツ
\ruby{子}{こ}ですよ。
%
\ruby[<g>]{成功者}{あたりや}さん
ばつかりに
\原本頁{181-8}\改行%
\ruby{惚}{ほ}れるんですわネ。
』

\原本頁{181-9}%
『
\ruby{畜生}{ちき|しやう}ツ、
%
\ruby{甚}{ひど}く
\ruby{當世}{たう|せい}なことを
\ruby{吐}{ぬか}しやあがる。
%
\ruby{此奴}{こい|つ}は
\ruby{今川燒}{いま|がは|やき}の
\ruby{讐}{かたき}を
\ruby{打}{う}たれた。
%
ハヽハヽ。
』

\原本頁{181-11}%
『ホヽホヽ。
』

\原本頁{182-1}%
お
\ruby{作}{さく}の
\ruby{笑}{わら}つて
\ruby{樓}{にかい}を
\ruby{下}{お}りきつたる
\ruby{時}{とき}、
%
がらりと
\ruby{格子}{かう|し}の
\ruby{明}{あ}く
\ruby{音}{おと}して、
%
\原本頁{182-2}\改行%
\ruby{頼}{たの}む
といふ
\ruby{聲}{こゑ}の
\ruby{此家}{こ|ゝ}の
\ruby{客}{きやく}には
\ruby{似合}{に|あ}はしからず
\ruby{堅}{かた}く、
%
\ruby{洋服}{やう|ふく}
\ruby{姿}{すがた}の
きりゝとしたる、
%
\ruby{日}{ひ}に
\ruby{焦}{や}けきつたる
\ruby{顏}{かほ}の
\ruby{恐}{おそ}ろしく
\ruby{赭}{あか}く、
%
\ruby{潮風}{しほ|かぜ}に
\ruby{晒}{さ}らされてか
\ruby{眼}{め}さへ
\ruby[g]{赤色}{あかみ}を
\ruby{帶}{お}びたる
\ruby{鐵}{てつ}づくりの
\ruby{如}{ごと}き
\ruby{男}{をとこ}は
\ruby{入}{い}り
\ruby{來}{きた}りぬ。

\原本頁{182-6}%
お
\ruby{作}{さく}は
\ruby{受}{う}け
\ruby{取}{と}りたる
\ruby{名刺}{めい|し}の
\ruby{表}{おもて}に
\ruby{羽{\換字{勝}}}{は|がち}
\ruby{千{\換字{造}}}{せん|ざう}といふ
\ruby{四{\換字{文}}字}{よん|も|じ}の
\ruby{記}{しる}されたるを
\ruby{見}{み}ぬ。

\Entry{其三十}

\ruby{酒}{さけ}は
\ruby{舊友}{きう|いう}と
\ruby{飮}{の}むより
\ruby{甘}{うま}きは
\ruby{無}{な}く、
\ruby{談}{だん}は
\ruby{{\換字{半}}醉}{はん|すゐ}の
\ruby{時}{とき}より
\ruby{熱}{ねつ}するは
\ruby{無}{な}し、
\ruby{雞黍}{けい|しよ}の
\ruby{設}{まう}け
\ruby{粗薄}{そ|はく}なりとも、
\ruby{膠漆}{かう|しつ}の
\ruby{{\換字{情}}}{じやう}の
\ruby{殷厚}{いん|こう}ならんには、
\ruby{杯}{さかづき}を
\ruby{手}{て}にして
\ruby{相見}{あひ|み}て
\ruby{笑}{わら}ふ
\ruby{一眄}{いち|べん}の
\ruby{中}{うち}にも
\ruby{限無}{かぎり|な}き
\ruby{味}{あぢはひ}は
\ruby{有}{あ}るべきを、ましてこれは
\ruby{范張陳雷}{はん|ちやう|ちん|らい}の
\ruby{語}{かた}らひのみならで、
\ruby{野心}{や|しん}に
\ruby{燃}{も}ゆる
\ruby{{\換字{若}}}{わか}き
\ruby{男}{をとこ}の、
\ruby{志}{こゝろざし}は
\ruby{各々異}{おの|〳〵|こと}なれども
\ruby{事}{こと}を
\ruby{一}{いつ}にして
\ruby{功}{こう}を
\ruby{擧}{あ}げんとする
\ruby{相談}{さう|だん}に、
\ruby{意氣}{い|き}は
\ruby{齊}{ひと}しく
\ruby{昻}{あが}りて
\ruby{興}{きよう}は
\ruby{湧}{わ}くが
\ruby{如}{ごと}し。

\ruby{亭主八杯}{てい|しゆ|はち|はい}の
\ruby{諺}{ことわざ}に
\ruby{洩}{も}れず、
\ruby{羽{\換字{勝}}}{は|がち}より
\ruby{先}{ま}ず
\ruby{島木}{しま|き}は
\ruby{醉}{よ}ひて、
\ruby{其}{そ}の
\ruby{肥}{ふと}つたる
\ruby{身體}{から|だ}を
\ruby{柱}{はしら}に
\ruby{靠}{もた}せながら、
\ruby{腫}{は}れたるが
\ruby{如}{ごと}き
\ruby{顏}{かほ}に
\ruby{笑}{ゑみ}を
\ruby{{\換字{浮}}}{うか}めつゝ、

『
\ruby{兎}{と}も
\ruby{角}{かく}も
\ruby{其}{それ}ぢやあ
\ruby{一萬二千圓}{いち|まん|に|せん|ゑん}だけは
\ruby{君}{きみ}の
\ruby{權利}{けん|り}の
\ruby{内}{うち}に
\ruby{置}{お}くと
\ruby{決}{き}めた。
\ruby{{\換字{船}}}{ふね}も
\ruby{借}{か}りるなら
\ruby{借}{か}りるが
\ruby{好}{い}い、
\ruby{買}{か}ふならばまた
\ruby{買}{か}ふが
\ruby{好}{い}い。
\ruby{一切}{いつ|さい}
\ruby{君}{きみ}の
\ruby{考次第}{かんがへ|し|だい}に
\ruby{任}{まか}せる。
\ruby{一艘}{いつ|ぱい}
\ruby{仕立}{し|た}てるとも
\ruby{二艘三艘}{に|はい|さん|ばい}
\ruby{仕立}{し|た}てるとも、それも
\ruby{君次第}{きみ|し|だい}で
\ruby{論}{ろん}は
\ruby{無}{な}い。
\ruby{乃公}{お|ら}あ
\ruby{素人}{しろう|と}だ、
\ruby{君}{きみ}は
\ruby{黑人}{くろ|うと}だ。
\ruby{乃公}{お|ら}あ
\ruby{何}{なに}も
\ruby{彼}{か}も
\ruby{{\換字{分}}}{わか}らないんだ。
おらあたゞ
\ruby{焔{\換字{硝}}}{えん|せう}と
\ruby{彈丸}{た|ま}とを
\ruby{出}{だ}すんだ。
\ruby{狙}{ねら}つて
\ruby{撃}{う}つて
\ruby{鳥}{とり}を
\ruby{穫}{と}るなあ
\ruby{君}{きみ}の
\ruby{手腕}{う|で}
\ruby{一}{いつ}ぱいに
\ruby{仕}{し}て
\ruby{貰}{もら}うんだ。
\ruby{後}{うしろ}から
\ruby{臂}{ひぢ}に
\ruby{觸}{さは}るやうな
\ruby{野暮}{や|ぼ}は
\ruby{仕}{し}ねえ。
\ruby{乃公}{お|ら}あ
\ruby{資金}{か|ね}を
\ruby{出}{だ}す、
\ruby{君}{きみ}は
\ruby{手腕}{う|で}を
\ruby{貸}{か}す。
\ruby{利益}{まう|け}は
\ruby{笑}{わら}つて
\ruby{山{\換字{分}}}{やま|わけ}に
\ruby{仕}{し}やうが、
\ruby{損}{そん}は
\ruby{泣言}{なき|ごと}を
\ruby{云}{い}ひつこ
\ruby{無}{な}しで、
\ruby{氣持}{き|もち}
\ruby{好}{よ}く
\ruby{骰子}{さ|い}を
\ruby{轉}{ころ}がして
\ruby{見}{み}やうと
\ruby{云}{い}ふんだ。
\換字{志}かし
\ruby{僕}{ぼく}も
\ruby{商人}{あき|んど}だ、
\ruby{算盤}{そろ|ばん}だけは
\ruby{合點}{が|てん}の
\ruby{行}{ゆ}く
\ruby{男}{をとこ}だから、
\ruby{大}{おほ}づもりのところだけは
\ruby{都度}{つ|ど}
\g詰めruby{々々}{〳〵}
\ruby{聞}{き}きたい。
\ruby{其他}{その|ほか}にやあ
\ruby{何}{なに}も
\ruby{注{\換字{文}}}{ちゆう|もん}は
\ruby{無}{な}いんだ。
\ruby{全}{まつた}く
\ruby{君}{きみ}の
\ruby{料簡}{れう|けん}
\ruby{次第}{し|だい}だ。
なあに
\ruby{一}{ぴん}と
\ruby{出}{で}やうと
\ruby{六}{ろく}と
\ruby{出}{で}やうと
\ruby{口惜}{く|やし}かあ
\ruby{無}{ね}え、
\ruby{事業}{し|ごと}の
\ruby{巧}{うま}く
\ruby{行}{い}くのと
\ruby{行}{い}かないのは、
\ruby{{\換字{半}}{\換字{分}}}{はん|ぶん}は
\ruby{手腕}{う|で}で
\ruby{{\換字{半}}{\換字{分}}}{はん|ぶん}は
\ruby{耳朶}{みゝつ|たぶ}だ!。
\ruby{{\換字{遣}}付}{やつ|つ}けるだけ
\ruby{{\換字{遣}}付}{やつ|つ}けて
\ruby{貰}{もら}やあ、
\ruby{何樣}{ど|う}なつたつて
\ruby{驚}{おどろ}かねえんだから、
\ruby{斟{\換字{酌}}}{しん|しやく}
\ruby{無}{な}く
\ruby{存{\換字{分}}}{ぞん|ぶん}に
\ruby{{\換字{遣}}}{や}つて
\ruby{吳}{く}れたまへ。
\ruby{今}{いま}も
\ruby{話}{はな}した
\ruby{{\換字{通}}}{とほ}り
\ruby{此}{こ}の
\ruby{風}{かぜ}が
\ruby{出無}{で|な}かつたら、
\ruby{擴}{ひろ}げられるだけ
\ruby{戰線}{せん|〳〵}を
\ruby{擴}{ひろ}げて
\ruby{置}{お}いた
\ruby{此}{こ}の
\ruby{萬五郎}{まん|ご|らう}は、
\ruby{今}{いま}ごろは
\ruby{何處}{ど|こ}へケシ
\ruby{飛}{と}んでるか
\ruby{{\換字{分}}}{わか}からないんだが、
\ruby{其}{そ}の
\ruby{危}{あぶ}ない
\ruby{瀬}{せ}を
\ruby{渡}{わた}つて
\ruby{揉}{も}み
\ruby{合}{あ}つたゞけに、とう〳〵
\ruby{切}{き}り
\ruby{{\換字{勝}}}{か}つて
\ruby{一}{ひ}ト
\ruby{伸}{のし}
\ruby{伸}{の}して、
\ruby{如是}{こ|う}した
\ruby{話}{はなし}も
\ruby{出來}{で|き}るんだもの!。
お
\ruby{互}{たがひ}に
\ruby{度胸}{ど|きよう}と
\ruby{腕}{うで}とに
\ruby{掛}{か}けて
\ruby{敗}{ひけ}を
\ruby{取}{と}ら
\ruby{無}{な}きやあ、
\ruby{少}{すこ}し
\ruby{{\換字{運}}}{うん}さへ
\ruby{添}{そ}やあ
\ruby{{\換字{造}}作}{ざう|さ}は
\ruby{無}{ね}え。
\ruby{三井}{みつ|ゐ}や
\ruby{岩崎}{いは|はき}を% 原本のこの部分は「いわさき」
\ruby{尻目}{しり|め}に
\ruby{見}{み}て、
\ruby{笑}{わら}つて
\ruby{一杯}{いつ|ぱい}
\ruby{飮}{の}ま
\ruby{無}{な}くつちやあ!。
\ruby{米}{こめ}や
\ruby{株}{かぶ}ばかり
\ruby{打}{たゝ}いて
\ruby{居}{ゐ}るのも
\ruby{智慧}{ち|ゑ}が
\ruby{足}{た}り
\ruby{無}{ね}えから、
\ruby{乃公}{お|ら}あ
\ruby{大蛸}{おほ|だこ}になつて
\ruby{八方}{はつ|ぽう}へ
\ruby{手}{て}を
\ruby{出}{だ}す!。
\ruby{五{\換字{分}}}{ご|ぶ}や
\ruby{七{\換字{分}}}{しち|ぶ}の
\ruby{口錢}{こう|せん}にヘイコラヘイコラと
\ruby{頭}{あたま}を
\ruby{下}{さ}げてこしらへた
\ruby{身上}{しん|しやう}ぢやあ
\ruby{無}{な}し、
\ruby{根}{ね}が
\ruby{泡沫錢}{あぶ|く|ぜに}だもの、
\ruby{{\換字{消}}}{き}えたつて
\ruby{未練}{み|れん}は
\ruby{無}{ね}えが、
\ruby{何}{なに}か
\ruby{知}{し}ら
\ruby{那方}{どつ|ち}かの
\ruby{手}{て}で
\ruby{攫}{つか}むつもりだ。
\ruby{思}{おも}ひ
\ruby{出}{だ}しやあソレ
\ruby{四五年}{し|ご|ねん}
\ruby{{\換字{前}}}{まへ}の
\ruby{事}{こと}だつけ、
\ruby{七人}{しち|にん}
\ruby{揃}{そろ}つた
\ruby{其時}{その|とき}に、おれが
\ruby{例}{いつも}の
\ruby{法螺話}{ほ|ら|ばなし}の
\ruby{末}{すゑ}、
お
\ruby{互}{たがひ}に
\ruby{那}{ど}の
\ruby{路}{みち}にせよ
\ruby{世}{よ}を
\ruby{渡}{わた}るにやあ、
\ruby{跣足}{は|だし}ぢやあ
\ruby{歩}{ある}けねえ、
\ruby{草鞋}{わら|ぢ}が
\ruby{要}{い}る。
おれが
\ruby{一番}{いち|ばん}
\ruby{巧}{うま}く
\ruby{當}{あた}りやあ、
\ruby{一同}{みん|な}に
\ruby{一萬兩}{いち|まん|りやう}づゝの
\ruby{草鞋}{わら|ぢ}を
\ruby{穿}{は}かせて、
\ruby{世}{よ}の
\ruby{石高路}{いし|だか|みち}を
\ruby{歩}{ある}かせて
\ruby{{\換字{遣}}}{や}ると
\ruby{云}{い}つたら、
\ruby{馬鹿}{ば|か}に
\ruby{誰}{たれ}も
\ruby{彼}{かれ}も
\ruby{怒}{おこ}りやあがつて、あの
\ruby{溫和}{おと|な}しい
\ruby{水野}{みづ|の}までが、
\ruby{僕}{ぼく}は
\ruby{踏}{ふ}み
\ruby{拔}{ぬ}きを
\ruby{仕}{し}たつて
\ruby{其樣}{そ|ん}な
\ruby{草鞋}{わら|ぢ}は
\ruby{貰}{もら}はないと
\ruby{云}{い}ふし、
\ruby{日方}{ひ|かた}はおらが
\ruby{背中}{せ|なか}を
\ruby{擲}{なぐ}りやがるし、
\ruby{楢井}{なら|い}や
\ruby{山瀬}{やま|せ}や
\ruby{名倉}{な|ぐら}までが、
\ruby{失敬}{しつ|けい}だ〳〵と
\ruby{腹}{はら}を
\ruby{立}{た}つたが、
\ruby{其時}{その|とき}
\ruby{君}{きみ}はたつた
\ruby{一人}{ひと|り}、なあに
\ruby{島木}{しま|き}が
\ruby{親切}{しん|せつ}で
\ruby{吳}{く}れやうといふなら
\ruby{貰}{もら}ふが
\ruby{好}{い}いぢや
\ruby{無}{な}いか、
\ruby{氣}{き}が
\ruby{狭}{せま}い!、
\ruby{成程}{なる|ほど}
\ruby{世}{よ}を
\ruby{渡}{わた}るにやあ
\ruby{草鞋}{わら|ぢ}が
\ruby{要}{い}る、と
\ruby{沈着}{おち|つ}いて
\ruby{云}{い}つて
\ruby{吳}{く}れた
\ruby{時}{とき}あ
\ruby{嬉}{うれ}しかつたよ。
それでと
\ruby{云}{い}ふ
\ruby{譯}{わけ}ぢやあ
\ruby{{\換字{更}}}{さら}に
\ruby{無}{ね}えが、
\ruby{云}{い}はゞ
\ruby{其時}{その|とき}
\ruby{云}{い}つた
\ruby{其}{その}
\ruby{草鞋}{わら|ぢ}を、
\ruby{今日}{け|ふ}から
\ruby{君}{きみ}に
\ruby{穿}{は}いて
\ruby{貰}{もら}つて、
\ruby{君}{きみ}だけに
\ruby{歩}{ある}いて
\ruby{貰}{もら}ふやうになつたなあア、
\ruby{嬉}{うれ}しい!。
サア
\ruby{羽{\換字{勝}}}{は|がち}
\ruby{君}{くん}!、これからだ。
ウンと
\ruby{大股}{おほ|また}に
\ruby{踏張}{ふん|ば}つてくれ!。
\ruby{君}{きみ}の
\ruby{腿骨}{すねつ|ぽね}の
\ruby{{\換字{達}}者}{たつ|しや}なところと、
\ruby{男兒}{をと|こ}
\ruby{振}{ぶ}りの
\ruby{好}{い}いところを
\ruby{見}{み}せて
\ruby{吳}{く}れたまへ。
ナア
\ruby{羽{\換字{勝}}}{は|がち}
\ruby{君}{くん}!。
』

と、これは
\ruby{{\換字{飽}}}{あく}まで
\ruby{醉}{よひ}に
\ruby{乘}{じよう}じて
\ruby{碎}{くだ}けて
\ruby{云}{い}へど、
\ruby{羽{\換字{勝}}}{は|がち}は
\ruby{醉}{よ}うて
\ruby{醉}{よ}はぬ
\ruby{姿勢}{し|せい}さへ
\ruby{正}{たゞ}しく、
\ruby{堅固}{けん|ご}の
\ruby{言葉}{こと|ば}つき
\ruby{力{\換字{強}}}{ちから|づよ}く、

『ム。
\ruby{悉皆}{しつ|かい}
\ruby{了解}{れう|かい}した。
\ruby{確}{たしか}に
\ruby{承諾}{しよう|だく}した。
\ruby{面白}{おも|しろ}い。
\ruby{行}{や}れるだけは
\ruby{屹}{きつ}と
\ruby{行}{や}る
\ruby{羽{\換字{勝}}}{は|がち}だ!。
\ruby{{\換字{運}}}{うん}が
\ruby{{\換字{逃}}}{に}げれば
\ruby{{\換字{運}}}{うん}を
\ruby{{\換字{追}}尾}{おつ|か}ける!。
たとひ
\ruby{草鞋}{わら|ぢ}は
\ruby{穿}{は}き
\ruby{切}{き}つても、
\ruby{歩}{ある}きだしたら
\ruby{必}{かなら}ず
\ruby{歩}{ある}く。
\ruby{中{\換字{途}}}{ちゆう|と}では
\ruby{休}{やす}まぬ、
\ruby{{\換字{運}}}{うん}は
\ruby{摑}{つか}む!。
\ruby{其代}{その|かは}り
\ruby{悉皆}{みん|な}
\ruby{屹度}{きつ|と}
\ruby{任}{まか}せて
\ruby{吳}{く}れ。
』

と、
\ruby{云}{い}ひも
\ruby{{\換字{終}}}{おは}らぬに
\ruby{島木}{しま|き}は
\ruby{烈}{はげ}しく、

『オヽ、
\ruby{任}{まか}せないで
\ruby{何}{なん}とするもんだ。
\ruby{屹度}{きつ|と}
\ruby{頼}{たの}んだぞ!。
』

と、
\ruby{口}{くち}を
\ruby{衝}{つ}いて
\ruby{答}{こた}へたり。

『ムツ、
\ruby{頼}{たの}まれたぞ。
』

『オヽ、
\ruby{頼}{たの}んだぞ。
』

『さあ
\ruby{始}{はじ}まつたぞ!。
』

『
\ruby{双六}{すご|ろく}が』

『ハヽハヽ。
』

『ハヽハヽ。
』

\ruby{玻璃盞}{こ|つ|ぷ}は
\ruby{玻璃盞}{こ|つ|ぷ}とカチリと
\ruby{觸}{あた}つて、
\ruby{酒}{さけ}は
\ruby{二人}{ふた|り}に
\ruby{一時}{いち|じ}に
\ruby{仰}{あふ}がれたり。


\Entry{其三十一}

% メモ 校正終了 2024-04-11 2024-05-27 2024-06-19
\原本頁{189-1}%
\ruby[g]{授業}{じゆげふ}も
\ruby{爲}{な}し
\ruby{{\換字{難}}}{がた}く
\ruby{見}{み}えたるほどの
\ruby[g]{暴風}{あ れ }の
\ruby[g]{一日}{いちにち}の
\ruby[<j||]{生}{なま }
\ruby[<j>]{{\換字{暖}}}{あたゝか}きに、
%
\ruby[g]{{\換字{平}}生}{いつも }の% ルビ調整(原本通り)
\ruby{如}{ごと}く
\ruby[g]{敎鞭}{けうべん}を
\ruby{執}{と}りて
\ruby[g]{太郎}{た らう}
\ruby[g]{次郎}{じ らう}を
\ruby[g]{相手}{あひて }に
\ruby{仕}{し}たりし
\ruby[g]{水野}{みづの }は、
%
\ruby{我}{が}の
\ruby{{\換字{強}}}{つよ}き
ところあれば
\ruby[g]{職務}{つとめ }を
\ruby{怠}{おこた}りこそは
\ruby{爲}{せ}ざりつれ、
%
\ruby[g]{{\換字{前}}日}{ぜんじつ}よりの
\ruby[g]{身心}{しん〴〵}の
\ruby{疲}{つか}れに、
%
\ruby[g]{五體}{ご たい}の
\ruby{綿}{わた}の
\ruby{如}{ごと}くなれるを
\ruby{我}{われ}と
\ruby{覺}{おぼ}えつゝ、
%
やうやく
\ruby[g]{午後}{ご ゞ }
\原本頁{189-6}\改行%
\ruby[g]{何時}{なんじ }の
\ruby{今}{いま}、
%
\ruby{始}{はじ}めて
\ruby{我}{わ}が
\ruby{身}{み}の
\ruby{我}{わ}が
\ruby{物}{もの}と
なりたる
\ruby[g]{心地}{こゝち }する
\ruby{氣}{き}の
\ruby{{\換字{緩}}}{ゆる}みに、
%
\ruby[||j>]{歩}{あるき}
\ruby[||j>]{調}{ つき}さへ
% \ruby{歩調}{あるき|つき}さへ
\ruby[g]{遲々}{ち ゝ }として、
%
\ruby[g]{脫力}{がつかり}して
\ruby{歸}{かへ}り
\ruby{來}{きた}りれり。

\原本頁{189-7}%
\ruby{手}{て}を
\ruby{掛}{か}けたるには
あらねど
\ruby{小}{ちひ}さき
\ruby{樹}{き}
\ruby{草}{くさ}など
\ruby{好}{よ}きほどに
\ruby{生}{は}えたれば
おのづからの
\ruby{庭}{には}と
なりたる
\ruby[g]{{\換字{空}}地}{あきち }を
\ruby{{\換字{前}}}{まへ}に、
%
\ruby{南}{みなみ}を
\ruby{受}{う}けたる
\ruby{長}{なが}き
\ruby[g]{一棟}{ひとむね}の、
%
\ruby{其}{そ}の
\ruby{奧}{おく}の
\ruby[g]{一間}{ひとま }は
\ruby{我}{わ}が
\ruby{起}{おき}
\ruby{臥}{ふし}の
ところと
\ruby{定}{さだ}まり
たるなり。
%
\ruby[g]{水野}{みづの }は
\ruby{常}{つね}の
\ruby{如}{ごと}く
\ruby[g]{庭先}{にはさき}を
\ruby{家}{いへ}に
\ruby{{\換字{沿}}}{そ}ひて
\ruby{{\換字{廻}}}{まは}りて、
%
\ruby{椽}{えん}より
\ruby{直}{たゞち}に
\ruby[g]{座敷}{ざ しき}に
\ruby{上}{あが}らんと
するに、
%
\ruby[g]{今日}{け ふ }は
\ruby{烈}{はげ}しき
\ruby{風}{かぜ}を
\ruby{厭}{いと}ひて、
%
\ruby[g]{雨{\換字{戸}}}{あまど }さへ
\ruby[g]{幾枚}{いくまい}か
\ruby{引}{ひ}かれ
\原本頁{190-1}\改行%
\ruby{居}{ゐ}たり。

\原本頁{190-2}%
『
ア、
%
\ruby{風}{かぜ}が
\ruby{甚}{ひど}いので
\ruby[g]{雨{\換字{戸}}}{あまど }を
\ruby{引}{ひ}いて
\ruby{置}{お}きました。
%
\ruby[||j>]{薄}{うすつ}
\ruby[||j>]{暗}{ くら}くつて
% \ruby{薄暗}{うすつ|くら}くつて
お
\ruby{{\換字{嫌}}}{いや}なら
\ruby{明}{あ}けて
あげませう。
%
\ruby[g]{方向}{む き }が
\ruby{好}{い}いので
\ruby[g]{此家}{こ ゝ }は
\ruby[g]{其程}{それほど}ぢやあ
\ruby{有}{あ}りませんが、
%
\ruby{何}{なに}にしろ
\ruby{甚}{ひど}い
\ruby{{\換字{嫌}}}{いや}な
\ruby{風}{かぜ}です。
』

\原本頁{190-5}%
\ruby{我}{わ}が
\ruby[g]{跫音}{あしおと}を
\ruby{聞}{きゝ}つけての
\ruby{吉右衛門}{きち||ゑ|もん}が
\ruby[g]{言葉}{ことば }に、

\原本頁{190-6}%
『
なあに
\ruby[g]{今日}{け ふ }は
\ruby{別}{べつ}に
\ruby[g]{細字}{こまかい}
\ruby{書}{ほん}を
\ruby{讀}{よ}まうとも
\ruby{思}{おも}はないから、
%
\ruby[g]{矢張}{やつぱ }り
\ruby[g]{此儘}{このまゝ}にして!。
』

\原本頁{190-8}%
と
\ruby{云}{い}ひながら
\ruby[g]{水野}{みづの }は
\ruby{身}{み}を
\ruby{側}{そば}めて、
%
\ruby{隙}{す}かして
\ruby{引}{ひ}かれたる
\ruby{{\換字{戸}}}{と}の
\ruby{間}{すき}より
\ruby{上}{あが}り、

\原本頁{190-10}%
『
ほんとに
\ruby[g]{氣持}{き もち}の
\ruby{惡}{わる}い、
%
\ruby{頭}{あたま}の
\ruby{痛}{いた}くなるやうな
\ruby{風}{かぜ}で、
‥‥
\ruby{早}{はや}く
\ruby{止}{や}んで
\ruby{吳}{く}れなくちやあ
\ruby[g]{仕方}{し かた}が
\ruby{無}{な}い。
』

\原本頁{191-1}%
と
\ruby[g]{座敷}{ざ しき}に
\ruby{入}{い}りつゝ
\ruby[g]{言葉}{ことば }を
\ruby{足}{た}せば、

\原本頁{191-2}%
『
\ruby[g]{左樣}{さ う }で
ございます。
%
\ruby{雨}{あめ}が
\ruby{隨}{つ}いて
\ruby{來}{こ}ないで
\ruby[g]{先々}{まあ〳〵}ですが、
%
\ruby[g]{土地}{ところ }に
よつちやあ
\ruby[g]{餘程}{よ ほど}の
\ruby[g]{損{\換字{害}}}{いたみ }です。
%
この
\ruby{{\換字{嫌}}}{いや}に
\ruby[<j>]{{\換字{暖}}}{あたゝか}い
\ruby{事}{こと}は
\ruby[g]{何樣}{ど う }でしやう。
%
% \原本頁{191-4}\改行%
\ruby[||j>]{病}{びやう}
\ruby[||j>]{人}{ にん}
% \ruby{病人}{びやう|にん}
なんぞにやあ
\ruby{{\換字{感}}}{き}きますネ。
%
オ、
%
\ruby[||j>]{病}{びやう}
\ruby[||j>]{人}{ にん}と
% \ruby{病人}{びやう|にん}と
\ruby{云}{い}やあ
\ruby[g]{今{\換字{朝}}}{け さ }
お
\ruby{頼}{たの}みの
\ruby{婢}{をんな}は、
%
\ruby{私}{わたし}の
\ruby[g]{本家}{う ち }の
\ruby{方}{はう}の
\ruby{小作人}{こ|さく|にん}の
\ruby{娘}{むすめ}で、
%
がせいに
\ruby{能}{よ}く
\ruby{働}{はたら}くのが
ありましたから、
%
\ruby{能}{よ}く
\ruby{云}{い}ひつけて
\ruby{其}{それ}を
\ruby{{\換字{遣}}}{や}つて
\ruby{置}{お}きました。
%
\ruby{看護{\換字{婦}}}{かん|ご|ふ}さんも
\ruby{來}{き}たさうです。
』

\原本頁{191-8}%
と、
%
\ruby{間}{あひ}の
\ruby{襖}{ふすま}は
\ruby{開}{ひら}き
\ruby{居}{ゐ}たる
\ruby{中}{なか}の
\ruby{間}{ま}に
ありて
\ruby[g]{敷居}{しきゐ }
\ruby{越}{ご}しの
\ruby[g]{挨拶}{あいさつ}なり。
%
\ruby[g]{水野}{みづの }は
\ruby[g]{床{\換字{近}}}{とこちか}く
\ruby{置}{お}きたる
\ruby{机}{つくゑ}の
\ruby{{\換字{前}}}{まへ}に
\ruby{坐}{すわ}りて、
%
\ruby{始}{はじ}めて
\ruby[g]{昨日}{きのふ }
\ruby[g]{以來}{い らい}の
\ruby[g]{疲勞}{つかれ }を
\ruby{息}{やす}めつゝ、

\原本頁{191-11}%
『
アヽ、
%
\ruby{今}{いま}
\ruby[g]{一寸}{ちよつと}
\ruby[g]{歸路}{かへり }に
\ruby[g]{立寄}{たちよ }つて
\ruby{來}{き}ました。
%
いろ〳〵
お
\ruby[g]{世話}{せ わ }を
\ruby{有}{あ}り
\ruby{{\換字{難}}}{がた}かつた。
%
\ruby{先}{まあ}
これで
\ruby[g]{一切}{いつさい}
\ruby{思}{おも}ふやうになつた。
』

\原本頁{191-2}%
と、
%
\ruby[g]{重荷}{おもに }を
\ruby{卸}{おろ}したるが
\ruby{如}{ごと}き
\ruby[g]{顏色}{かほつき}すれば、
%
\ruby{例}{れい}の
\ruby[g]{眼鏡}{め がね}の
\ruby{中}{うち}より
\ruby[g]{一寸}{ちよつと}
\ruby{見}{み}て、

\原本頁{192-4}%
『
\ruby[g]{昨夜}{ゆふべ }は
\ruby{碌}{ろく}に
\ruby{御睡眠}{お|よ|り}は
なさりますまいのに、
%
\ruby[g]{今日}{け ふ }は
\ruby{{\換字{又}}}{また}
\ruby[g]{{\換字{平}}生}{いつも }の% ルビ調整(原本通り)
\ruby{{\換字{通}}}{とほ}り
\ruby{御{\換字{勤}}務}{お|つ|とめ}では、
%
\ruby[g]{大抵}{たいてい}な
\ruby{御疲勞}{お|くた|びれ}では
ありますまい。
%
\ruby[g]{今夜}{こんや }は
まあ
\ruby{早}{はや}く
\ruby{御睡眠}{お|やす|み}なさいまし。
』

\原本頁{192-7}%
と、
%
\ruby{云}{い}ひさして
\ruby{茶}{ちや}の
\ruby{間}{ま}の
\ruby{方}{かた}を
\ruby{顧}{かへり}みて
\ruby{聲}{こゑ}
\ruby[||j>]{大}{おほき}く、

\原本頁{192-8}%
『
お
\ruby{濱}{はま}や。
%
また
\ruby[g]{其樣}{そ ん }なに
\ruby{書}{ほん}に
ばかり
\ruby[g]{取付}{とつつ }いて
\ruby{居}{ゐ}ちやあいけない。
%
\ruby[g]{先生}{せんせい}が
お
\ruby{歸}{かへ}り
なすつたぢやあ
\ruby{無}{な}いか、
%
\ruby[g]{御茶}{お ちや}を
\ruby{持}{も}つて
\ruby{來}{こ}ないか。
』

\原本頁{192-10}%
と、
%
\ruby[g]{悠然}{ゆつくり}と
したる
\ruby[g]{調子}{てうし }に
\ruby{呼}{よ}ばゝつたるは、
%
\ruby[g]{言葉}{ことば }つきなども
\ruby{異}{をか}しからぬほど
\ruby[g]{江{\換字{戸}}}{え ど }の
\ruby{水}{みづ}も
\ruby{飮}{の}んだる
\ruby{果}{はて}の
\ruby[g]{老夫}{おやぢ }なれど、
%
\ruby[g]{流石}{さすが }は
\ruby{根}{ね}が
\ruby{此}{こ}の
\ruby{邊}{あたり}の
\ruby[g]{田舎}{ゐ なか}% ルビ調整(原本通り)
\ruby{風}{ふう}なり。

\原本頁{193-2}%
\ruby{小}{ちひ}さき
\ruby[g]{刳{\換字{盆}}}{くりぼん}に
\ruby{大}{おほき}なる
\ruby{筒茶碗}{つゝ|ぢあ|わん}
\ruby{載}{の}せて、
%
\ruby[g]{嫣然}{につこり}と
\ruby{笑}{ゑ}みて
\ruby[g]{持出}{もちい }でたる
お
\ruby{濱}{はま}は、
%
\ruby[g]{水野}{みづの }が
\ruby[g]{膝{\換字{近}}}{ひざちか}く
それを
\ruby{置}{お}きて、
%
おのれは
\ruby[g]{祖{\換字{父}}}{ぢ ゞ }の
\ruby{傍}{かたへ}に
\ruby{甘}{あま}えるやうに
\ruby{坐}{すわ}り。

\原本頁{193-5}%
『
\ruby[g]{昨夜}{ゆふべ }は
\ruby{怖}{こは}かつた
でしようねえ、
%
\ruby[g]{眞闇}{まつくら}で!。
%
あれから
\ruby[||j>]{妾}{わたし}
\ruby[||j>]{床}{ とこ}へ
\ruby{入}{はい}つたら、
%
\ruby[g]{先生}{せんせい}の
\ruby{行}{いら}しつた
\ruby{方}{はう}の、
%
\ruby{{\換字{遠}}}{とほ}くの
\ruby{{\換字{遠}}}{とほ}くから、
%
\ruby{狗}{いぬ}の
\ruby{鳴}{な}く
\ruby{聲}{こゑ}が
\原本頁{193-7}\改行%
\ruby{聞}{きこ}えて
\ruby{來}{き}て、
%
\ruby{淋}{さび}しかつたわ!。
』

\原本頁{193-8}%
と
\ruby{云}{い}ひ
\ruby{出}{だ}せば、

\原本頁{193-9}%
『
ハヽヽ、
%
\ruby{何}{な}んだ
\ruby{下}{くだ}らない、
%
\ruby[g]{叩頭}{おじぎ }も
\ruby{仕}{し}ないで!。
%
\ruby[g]{突然}{いきなり}と
\ruby[g]{其樣}{そ ん }な
\ruby{事}{こと}を
\ruby{云}{い}ひ
\ruby{出}{だ}すよ。
%
\ruby{狗}{いぬ}が
\ruby{鳴}{な}いたつて
\ruby{何}{なに}
\ruby{淋}{さみ}しい
\ruby{奴}{やつ}が
あるもんか。
』

\原本頁{193-11}%
と
\ruby{笑}{わらひ}を
\ruby{帶}{お}びて
\ruby{吉右衛門}{きち||ゑ|もん}は
\ruby{叱}{しか}るを、
%
\ruby[g]{眞赤}{まつか }なる
\ruby[g]{番茶}{ばんちや}の
\ruby{味}{あぢ}も
\ruby{無}{な}く
\ruby{香}{か}も
\ruby{無}{な}けれど、
%
\ruby{熱}{あつ}き
のみに
\ruby{人}{ひと}の
\ruby{{\換字{情}}}{なさけ}は
\ruby{有}{あ}るを
\ruby{啜}{すゝ}れる
\ruby[g]{水野}{みづの }は、

\原本頁{194-2}%
『
ハヽヽ、
%
お
\ruby{濱}{はま}ちやんは
いつでも
\ruby[g]{面白}{おもしろ}い
\ruby{事}{こと}を
\ruby{云}{い}ふ!。
%
そして
\ruby[g]{昨夜}{ゆふべ }は
\makeatletter
\@ifundefined{デバッグ@ビルド}{%
  \ruby[<g||]{一生}{いつしやう}
  \ruby[g]{懸命}{けんめい}に
}{%
  \ruby[||j>]{一}{いつ }
  \ruby[||j>]{生}{しやう}
  \ruby[||j>]{懸}{ けん}
  \ruby[||j>]{命}{ めい}に
}%
\ruby{書}{ほん}を
\ruby{讀}{よ}んで
\ruby{居}{ゐ}たぢや
\ruby{無}{な}いか、
%
あれは
\ruby[g]{一體}{いつたい}
\ruby{何}{なん}の
\ruby{本}{ほん}
\原本頁{194-4}\改行%
だえ。
』

\原本頁{194-5}%
と
\ruby{問}{と}ふに、
%
お
\ruby{濱}{はま}は
\ruby{忽}{たちま}ち
\ruby[g]{不足}{ふ そく}らしき
\ruby{恨}{うら}みを
\ruby{其}{その}
\ruby{色}{いろ}に
\ruby{現}{あら}はしたり。

\原本頁{194-6}%
『
だつて
\ruby{先}{せん}の
\ruby{中}{うち}は
\ruby[g]{毎晩}{まいばん}
\ruby[g]{々々}{ 〳〵 }
いろんな
\ruby[g]{面白}{おもしろ}い
お
\ruby{譚}{はなし}を
\ruby{仕}{し}て
\ruby{聞}{き}かして
\ruby{下}{くだ}すつたのに、
%
\ruby{此}{この}
\ruby{{\換字{節}}}{せつ}は
\ruby{毫}{ちつと}も
\ruby{御談話}{お|はな|し}
なんぞして
\ruby{下}{くだ}さらないんだもの。
%
\ruby{妾}{わたし}は
ほんとに
\ruby{詰}{つま}らなくつて、
%
\ruby[g]{仕方}{し かた}が
ないから
\ruby[g]{本家}{う ち }から
\ruby{書}{ほん}を
\ruby{持}{も}つて
\ruby{來}{き}て
\ruby{讀}{よ}んで
\ruby{居}{ゐ}るのよ。
』

\原本頁{194-10}%
『
でも
\ruby{書}{ほん}が
おもしろけりやあ
\ruby{可}{いゝ}ぢやあ
\ruby{無}{な}いか、
%
\ruby{私}{わたし}の
\ruby{無器用}{ぶ|き|よう}な
\ruby[g]{談話}{はなし }
なんぞより。
』

\原本頁{195-1}%
\ruby[g]{頭髮}{か み }も
ゆら〳〵と
\ruby{頭}{かうべ}を
\ruby{振}{ふ}つて、

\原本頁{195-2}%
『
イヽエ、
%
\ruby[g]{矢張}{やつぱ }り
\ruby{御談話}{お|はな|し}の
\ruby{方}{はう}が
\ruby[||j>]{妾}{わたし}
\ruby[||j>]{好}{ す}きなのよ。
%
あの
\ruby{本}{ほん}は
\ruby[g]{面白}{おもしろ}い
\ruby{事}{こと}は
\ruby[g]{面白}{おもしろ}いけれど、
%
むづかしくつて
いけないところが
\ruby{有}{あ}るんですもの!。
%
\ruby[g]{今夜}{こんや }は
\ruby[g]{何處}{どつこ }へも
\ruby{行}{い}かないで
\ruby[g]{御話}{おはなし}を
\ruby{仕}{し}て。
%
ネ、
%
\ruby[g]{御願}{おねがひ}
ですから
\ruby{泣}{な}くやうなのを!。
%
\ruby[||j>]{妾}{わたし}
\ruby[||j>]{泣}{ な}くやうな
\ruby[g]{御話}{おはなし}が
\ruby[g]{大好}{だいす }きなのよ。
』

\原本頁{195-6}%
と
\ruby[g]{{\換字{遠}}慮}{ゑんりよ}も
\ruby{無}{な}く
\ruby[g]{{\換字{強}}{\換字{請}}}{ね だ }れば
\ruby{吉右衛門}{きち||ゑ|もん}は
\ruby{苦}{にが}りて、

\原本頁{195-7}%
『
また
\ruby[g]{其樣}{そ ん }なに
\ruby{直}{ぢき}
\ruby[||j>]{汝}{おまへ}は
\ruby{甘}{あま}つたれるよ!。
%
そんな
\ruby[g]{氣樂}{き らく}な
\ruby{事}{こと}
どころぢやあ
\ruby{無}{な}くつて
ゐらつしやるのだ。
』

\原本頁{195-9}%
と、
%
\ruby{少}{すこ}し
\ruby{叱}{しか}り
\ruby[g]{氣味}{ぎ み }に
\ruby{{\換字{遮}}}{さへぎ}り
\ruby{止}{とゞ}むるに、

\原本頁{195-10}%
『
アヽ
\ruby[||j>]{妾}{わたし}
\ruby[||j>]{知}{ し}つてますよ、
%
\ruby[g]{五十子}{いそこ}さんが
\ruby{惡}{わる}いから\換字{!?}。
%
\ruby[||j>]{妾}{わたし}
\ruby[||j>]{今日}{ け|ふ}
\ruby{見}{み}て
\ruby{來}{き}てよ
\ruby[g]{五十子}{いそこ}さんを。
%
ほんとに
\ruby[||j>]{憫}{かは}
\ruby[||j>]{然}{いさう}に% 「憫然 か(は)いさう」
% \ruby{憫然}{かは|いさう}に% 「憫然 か(は)いさう」
\ruby[g]{病重}{わ る }いのねえ。
』

\原本頁{196-1}%
と、
%
\ruby{然}{さ}も
\ruby[g]{心配}{しんぱい}
\ruby{氣}{げ}に
\ruby{艶}{つや}やかなる
\ruby{面}{おもて}の
\ruby{美}{うつく}しき
\ruby{眉}{まゆ}を
\ruby[g]{打顰}{うちひそ}めたる、
%
\ruby{云}{い}ふに
\ruby{云}{い}はれぬ
\ruby[g]{可愛}{か はい}さ
ありて、
%
\ruby[g]{此室}{こ ゝ }ばかりには
\ruby{騷}{さわ}がしき
\ruby{風}{かぜ}も
\ruby{吹}{ふ}かぬが
\ruby{如}{ごと}し。

\Entry{其三十二}

\原本頁{}
『
\ruby{人}{ひと}にも
\ruby{云}{い}はないで
\ruby{何時}{い|つ}の
\ruby{間}{ま}に
\ruby{岩崎}{いは|ざき}さんのところへ
\ruby{行}{い}つて
\ruby{見}{み}たのだ。
%
\ruby{彼方}{あち|ら}ぢやあ
\ruby{御煩}{お|うるさ}く
\ruby{御思}{お|おも}ひだらうに!。
』

\原本頁{}
『いゝえ
\ruby{祖{\換字{父}}}{お|ぢい}さん、
%
\ruby{一寸}{ちよ|つと}
\ruby{行}{い}つたばかしで、
%
\ruby{上}{あが}りも
\ruby{何}{なに}も
\ruby{仕}{し}やあ
\ruby{仕}{し}ないのよ!。
%
たゞそ{---}{---}つと
\ruby{外}{そと}から
\ruby{見}{み}たばかりなの。
%
だけれども
\ruby{肥}{ふと}つた
\ruby{看護{\換字{婦}}}{かん|ご|ふ}さんも
\ruby{見}{み}たし、
%
\ruby{丁度}{ちやう|ど}
\ruby{松}{まつ}ちやんにも
\ruby{會}{あ}つて
\ruby{話}{はなし}を
\ruby{仕}{し}て
\ruby{來}{き}たのよ。
%
\ruby{松}{まつ}ちやんは
\ruby{曩日}{いつ|か}
\ruby{吾家}{う|ち}で
\ruby{一{\換字{所}}}{いつ|しよ}に
\ruby{{\換字{遊}}}{あそ}んだ
\ruby{時}{とき}なんかとは
\ruby{{\換字{違}}}{ちが}つて、
%
\ruby{泣}{な}きさうな
\ruby{顏}{かほ}を
\ruby{仕}{し}て
\ruby{居}{ゐ}るんだもの、
%
\ruby{妾}{わたし}ほんとに
\ruby{憫然}{かは|いさう}になつちまつたの!。% 「憫然 か(は)いさう」
%
だもんだから
\ruby{彼}{あ}の
\ruby{椎}{しひ}の
\ruby{樹}{き}の
\ruby{傍}{そば}で、
%
\ruby{二人}{ふた|り}でつい
\ruby{泣}{な}いて
\ruby{話}{はなし}を
\ruby{仕}{し}て
\ruby{居}{ゐ}たら、
%
\ruby{彼家}{あす|こ}の
お
\ruby{澤}{さは}
\ruby{婆}{ばゞあ}つたら
\ruby{眞箇}{ほん|と}に
\ruby{憎}{にく}らしい!、
%
お
\ruby{濱子}{はま|つこ}!、
%
\ruby{汝}{おめへ}まで
\ruby{心配}{しん|ぱい}して
\ruby{居}{ゐ}るだけえ?、
%
だけれど
\ruby{泣}{な}いたつて
\ruby{無益}{だ|め}なこんだ!。
%
\ruby{心配}{しん|ぱい}で
\ruby{癒}{なほ}る
\ruby{病氣}{びやう|き}あ
\ruby{無}{ね}えだから、
%
つて
\ruby{{\換字{菜}}圃}{はた|け}の
\ruby{對}{むかふ}から
\ruby{大}{おほき}な
\ruby{聲}{こゑ}をして
\ruby{怒鳴}{ど|な}るんだもの!。
%
\ruby{妾}{わたし}ほんとに
\ruby{口惜}{く|や}しくつて
\ruby{口惜}{く|や}しくつて、
%
\ruby{風}{かぜ}の
\ruby{中}{なか}を
\ruby{駈}{か}け
\ruby{出}{だ}して
\ruby{歸}{かへ}つて
\ruby{來}{き}て
\ruby{一人}{ひと|り}で
\ruby{怒}{おこ}つて
\ruby{泣}{な}いたわ。
%
ほんとに
\ruby{彼樣}{あ|ん}な
\ruby{意地惡}{い|ぢ|わる}な
\ruby{婆}{ばゞあ}つたら
\ruby{有}{あ}りや
\ruby{仕}{し}ない!。
%
\ruby{今度}{こん|ど}また
\ruby{彼樣}{あ|ん}な
\ruby{事}{こと}を
\ruby{云}{い}つたら
\ruby{引爬}{ひつ|か}いて
\ruby{{\換字{遣}}}{や}らなくつちやあ。
』

\原本頁{}
『ハヽヽ、
%
また
\ruby{其樣}{そ|ん}な
お
\ruby{轉婆}{てん|ば}な
\ruby{事}{こと}をいふよ!。
%
\ruby{何樣}{ど|う}して〳〵
\ruby{彼}{あ}の
\ruby{婆}{ばあ}さんにやあ
\ruby{汝}{おまへ}なんぞの
\ruby{爪}{つめ}も
\ruby{立}{た}つもんぢやあ
\ruby{無}{な}い。
%
\ruby{婆}{ばあ}さんを
\ruby{引爬}{ひつ|か}きやあ
\ruby{汝}{おまへ}の
\ruby{爪}{つめ}は
\ruby{悉皆}{みん|な}
\ruby{脫}{と}れたつて、
%
\ruby{彼方}{むか|ふ}にやあ
\ruby{蚯蚓脹}{みゝ|ず|ばれ}も
\ruby{出來}{で|き}や
\ruby{仕}{し}ない。
%
そんな
\ruby{事}{こと}はまあ
\ruby{何樣}{ど|う}でも
\ruby{可}{い}いが、
%
もうそろ〳〵と
\ruby{日}{ひ}が
\ruby{暮}{く}れかゝる、
%
お
\ruby{鍋}{なべ}が
\ruby{何}{なに}かことつかせて
\ruby{居}{ゐ}る、
%
\ruby{汝}{おまへ}も
\ruby{彼方}{あつ|ち}へ
\ruby{行}{い}つて
\ruby{夕方}{ゆふ|がた}の
\ruby{事}{こと}を、
%
\ruby{些}{ちつと}は
\ruby{傍}{そば}から
\ruby{手傳}{て|つだ}つて
\ruby{{\換字{遣}}}{や}りナ。
』

\原本頁{}
『
\ruby{先生}{せん|せい}が
\ruby{今夜}{こん|や}
\ruby{面白}{おも|しろ}い
\ruby{御話}{お|はなし}を
\ruby{仕}{し}て
\ruby{下}{くだ}さるなら。
』

\原本頁{}
『
\ruby{祖{\換字{父}}}{お|ぢい}さんが
\ruby{命令}{いひ|つけ}るのに
\ruby{先生}{せん|せい}のところへ
\ruby{掛}{かゝ}つて
\ruby{行}{い}くとは、
%
\ruby{何}{なん}だか
\ruby{理由}{わ|け}の
\ruby{{\換字{分}}}{わか}らない
\ruby{理屈合}{り|くつ|あひ}だナ。
%
サアマア
\ruby{何}{なん}でも
\ruby{可}{い}いから
\ruby{御働}{お|はたら}き、
%
お
\ruby{働}{はたら}き!』

\原本頁{}
『ハイ。
%
ぢやあ
\ruby{先生}{せん|せい}
\ruby{屹度}{きつ|と}
\ruby{後刻}{の|ち}に
\ruby{先日}{この|あひだ}の
\ruby{御話}{お|はなし}の
\ruby{續}{つゞ}きをネ。
』

\原本頁{}
\ruby{頭}{くび}を
\ruby{曲}{ま}げて
\ruby{水野}{みづ|の}の
\ruby{顏}{かほ}を
\ruby{覗}{のぞ}き
\ruby{{\換字{込}}}{こ}むやうにして
\ruby{自己}{お|の}が
\ruby{{\換字{勝}}手}{かつ|て}を
\ruby{云}{い}ひつつ
お
\ruby{濱}{はま}は
\ruby{纔}{わづか}に
\ruby{彼方}{かな|た}に
\ruby{去}{さ}りたり。

\原本頁{}
\ruby{祖{\換字{父}}}{ぢ|ゞ}は
\ruby{孫娘}{まご|むすめ}の
\ruby{背姿}{うしろ|すがた}を
\ruby{見}{み}おくりながら、

\原本頁{}
『
\ruby{身長}{せ|い}ばかり
\ruby{彼樣}{あ|ん}なに
\ruby{大}{おほ}きくなつて、
%
いつまで
\ruby{彼樣}{あ|ん}な
\ruby{調子}{てう|し}で
\ruby{居}{ゐ}るのでしやう!。
%
もう
\ruby{少}{すこ}しは
\ruby{女}{をんな}らしくなりさうなものですのに、
%
あゝやんちやんでは
\ruby{仕方}{し|かた}が
\ruby{有}{あ}りません。
%
いくらお
\ruby{澤}{さは}
\ruby{婆}{ばあ}さんが
\ruby{憎}{にく}いと
\ruby{云}{い}つたつて、
%
\ruby{引爬}{ひつ|か}いて
\ruby{{\換字{遣}}}{や}らうなんて、
%
ハヽハヽハヽ。
』

\原本頁{}
と
\ruby{獨語}{ひとり|ごと}の
\ruby{如}{ごと}く
\ruby{{\換字{又}}}{また}
\ruby{辯護}{べん|ご}の% 弁 瓣 辦 辧 辨 辩 (辯)
\ruby{如}{ごと}く
\ruby{云}{い}へば、
%
\ruby{其}{そ}の
\ruby{語}{ことば}に
\ruby{隨}{つ}いて、

\原本頁{}
『\換字{志}かしお
\ruby{澤}{さは}といふ
\ruby{婆}{ばあ}さんは
\ruby{眞箇}{ほん|と}に
\ruby{甚}{ひど}い!。
%
\ruby{何樣}{ど|う}した
\ruby{人}{ひと}だか
\ruby{知}{し}らないが、
%
\ruby{全}{まる}で
\ruby{普{\換字{通}}}{ひと|なみ}ぢやあ
\ruby{無}{な}い、
%
\ruby{先}{まあ}
\ruby{鬼婆}{おに|ばゞあ}だから、
%
\ruby{誰}{たれ}だつて
\ruby{何樣}{ど|う}か
\ruby{仕}{し}て
\ruby{{\換字{遣}}}{や}りたい
\ruby{位}{ぐらゐ}には
\ruby{思}{おも}はうぢやあ
\ruby{無}{な}いか。
』

\原本頁{}
と、
%
\ruby{水野}{みづ|の}は
\ruby{我}{わ}が
\ruby{思}{おも}へるところを
\ruby{打}{う}ち
\ruby{出}{いだ}したり。

\原本頁{}
『
\ruby{貴君}{あな|た}も
\ruby{何}{なに}かで
\ruby{御腹立}{お|はら|だち}でしたネ。
%
\ruby{其}{そり}やあ
\ruby{左樣}{そ|う}でございますとも、
%
\ruby{普{\換字{通}}}{な|み}ぢやあ
\ruby{有}{あ}りません!。
%
\ruby{仰}{おつし}ある
\ruby{{\換字{通}}}{とほ}り
\ruby{鬼}{おに}になつて
\ruby{居}{ゐ}るのですから!。
%
あれでも
\ruby{舊}{もと}は
\ruby{人}{ひと}の
\ruby{好}{い}い
\ruby{婆}{ばあ}さんでしたが、
%
\ruby{親一人}{おや|ひと|り}
\ruby{娘一人}{こ|ひと|り}の
\ruby{秘蔵娘}{ひ|ざう|むすめ}の、
%
お
\ruby{里}{さと}といふのに
\ruby{婿}{むこ}を
\ruby{取}{と}つた{---}{---}
\ruby{其婿}{その|むこ}が
\ruby{惡}{わる}かつたところから
\ruby{彼樣}{あ|ゝ}なつたのです。
』

\原本頁{}
『フーン。
』

\原本頁{}
『
\ruby{婿}{むこ}は
\ruby{兵作}{ひやう|さく}といふ
\ruby{惡}{わる}い
\ruby{奴}{やつ}で、
%
\ruby{今}{いま}は
\ruby{東京}{とう|きやう}の
\ruby{牛{\換字{込}}}{うし|ごめ}あたりに、
%
\ruby{樂}{らく}な
\ruby{生活}{くら|し}を
\ruby{仕}{し}て
\ruby{居}{ゐ}るさうですが、
%
\ruby{出}{で}は
\ruby{二合{\換字{半}}領}{に|がふ|はん|りやう}の
\ruby{可成}{か|なり}な
\ruby{大盡}{だい|じん}の
\ruby{二番生}{に|ばん|ばえ}で、
%
\ruby[<h||]{男}{をとこ}
\ruby{振}{ぶり}の
\ruby{惡}{わる}くない
\ruby{應對}{おう|たい}の
\ruby{上手}{じやう|ず}な
\ruby{男}{をとこ}です。
%
\ruby{婆}{ばあ}さんの
\ruby{家}{うち}は
\ruby{村}{むら}でも
\ruby{指折}{ゆび|をり}の
\ruby{物持}{もの|もち}でしたが、
%
\ruby{其}{そ}の
\ruby{兵作}{ひやう|さく}といふのが
\ruby{猫}{ねこ}を
\ruby{被}{かぶ}つた
\ruby{{\換字{狼}}}{おほかみ}でして、
%
\ruby{何}{なに}を
\ruby{爲}{す}る、
%
\ruby{彼}{か}を
\ruby{爲}{す}ると
\ruby{云}{い}つては
\ruby{金}{かね}を
\ruby{持出}{もち|だ}し、
%
\ruby{{\換字{終}}}{しまひ}には
\ruby{家屋敷}{いへ|や|しき}まで
\ruby{抵當}{てい|たう}に
\ruby{打{\換字{込}}}{ぶち|こ}んだのです。
%
\換字{志}かし
\ruby{其}{それ}が
\ruby[g]{眞實}{ほんと}に
\ruby{商賣事}{しやう|ばい|ごと}で
\ruby{損}{そん}を
\ruby{仕}{し}たといふなら
\ruby{未}{ま}だ
\ruby{好}{よ}うございますが、
%
\ruby{實}{じつ}は
\ruby{婿}{むこ}になる
\ruby{{\換字{前}}}{まへ}から
\ruby{他}{ほか}に
\ruby{{\換字{情}}{\換字{婦}}}{をん|な}が
\ruby{有}{あ}つて、
%
\ruby{其方}{その|はう}に
\ruby{悉皆}{みん|な}こかしたのです。
%
\ruby{左樣}{そ|う}して
\ruby{置}{お}いて
\ruby{{\換字{平}}井}{ひら|ゐ}の
\ruby{家}{うち}に
\ruby{塵}{ちり}ツ
\ruby{葉}{ぱ}
\ruby{一}{ひと}つ
\ruby{無}{な}くなつた
\ruby{時{\換字{分}}}{じ|ぶん}に、
%
さあ
\ruby{自{\換字{分}}}{じ|ぶん}が
\ruby{{\換字{逐}}出}{おひ|だ}されて
\ruby{仕舞}{し|ま}ふ
\ruby{心算}{つも|り}で、
%
\ruby{彼}{あ}の
\ruby{婆}{ばあ}さん
\ruby{親子}{おや|こ}に
\ruby{無理}{む|り}ばかり
\ruby{云}{い}つて、
%
\ruby{打}{ぶ}ちます、
%
\ruby{蹴}{け}ます、
%
\ruby{暴}{あば}れます、
%
\ruby{散々}{さん|〴〵}に
\ruby{酷}{ひど}い
\ruby{事}{こと}を
\ruby{致}{いた}しました。
%
それが
\ruby{爲}{ため}に
お
\ruby{里}{さと}が
\ruby{癆瘵氣質}{らう|さい|かた|ぎ}になつて、
%
\ruby{氣}{き}は
\ruby{異}{をか}\換字{志}くなるし、
%
\ruby{生}{い}きながら
\ruby{幽靈}{いう|れい}のやうに
\ruby{痩}{や}せて、
%
\ruby{苦}{くる}しんで〳〵
\ruby{居}{を}りましたが、
%
\ruby{其中}{その|なか}を
\ruby{畢竟}{とう|〳〵}
\ruby{別}{わか}れ
\ruby{話}{ばなし}を
\ruby{仕}{し}て、
%
\ruby{兵作}{ひやう|さく}は
\ruby{身}{み}を
\ruby{{\換字{退}}}{の}いて
\ruby{仕舞}{し|ま}ひました。
』

\原本頁{}
『ヤ、
%
それは
\ruby{恐}{おそ}ろしい
\ruby{酷}{むご}い
\ruby{談}{はなし}で。
』

\原本頁{}
『それこれでお
\ruby{里}{さと}は
\ruby{死}{し}んで
\ruby{仕舞}{し|ま}ひます。
%
\ruby{婆}{ばあ}さんは
\ruby{住}{す}んで
\ruby{居}{ゐ}た
\ruby{家}{うち}も
\ruby{{\換字{逐}}出}{おつ|た}てられて、
%
\ruby{他人}{ひ|と}の
\ruby{物置小屋}{もの|おき|ご|や}を
\ruby{假}{か}りて
\ruby{入}{はい}るやうな
\ruby{始末}{し|まつ}にもなりましたが、
%
それから
\ruby{彼}{あ}の
\ruby{婆}{ばあ}さんは
\ruby{鬼}{おに}のやうになりまして、
%
\ruby{誰}{たれ}
\ruby{彼}{かれ}の
\ruby{見}{み}さかひ
\ruby{無}{な}く
\ruby{人}{ひと}を
\ruby{疑}{うたが}ひ、
%
\ruby{一生懸命}{いつ|しやう|けん|めい}に
\ruby{挊}{かせ}いでは
\ruby{一{\換字{文}}二{\換字{文}}}{いち|もん|に|もん}を
\ruby{溜}{た}めて、
%
\ruby{其錢}{その|ぜに}を
\ruby{苛}{ひど}い
\ruby{高利}{かう|り}で
\ruby{貸}{か}し
\ruby{出}{だ}しました。
%
\ruby{左樣}{さ|う}して
\ruby{五年六年}{ご|ねん|ろく|ねん}と
\ruby{立}{た}つ
\ruby{内}{うち}に
\ruby{段々}{だん|〴〵}
\ruby{太}{ふと}りまして、
%
\ruby{舊}{もと}の
\ruby{自{\換字{分}}}{じ|ぶん}の
\ruby{家}{うち}を
\ruby{取}{と}り
\ruby{{\換字{返}}}{かへ}して
\ruby{手}{て}に
\ruby{入}{い}れたのです。
%
\ruby{他手}{ひと|で}に
\ruby{渡}{わた}つて
\ruby{居}{ゐ}る
\ruby{中}{うち}に
\ruby{焼}{や}けましたので、
%
\ruby{母屋}{おも|や}や
\ruby{藏}{くら}は
\ruby{殘}{のこ}つて
\ruby{居}{ゐ}ませんが、
%
\ruby{丁度}{ちやう|ど}
\ruby{今}{いま}
\ruby{岩崎}{いは|ざき}さんの
\ruby{借}{か}りて
\ruby{居}{ゐ}る
\ruby{室}{へや}が、
%
\ruby{兵作}{ひやう|さく}を
\ruby{婿}{むこ}に
\ruby{取}{と}つた
\ruby{其初}{その|はじめ}に、
%
\ruby{老人}{とし|より}は
\ruby{{\換字{若}}}{わか}い
\ruby{夫{\換字{婦}}}{ふう|ふ}に
\ruby{香}{かう}ばしく
\ruby{有}{あ}るまいからつて、
%
\ruby{自{\換字{分}}}{じ|ぶん}の
\ruby{隱居{\換字{所}}}{いん|きよ|じよ}にと
\ruby{建}{た}てた
\ruby{別室}{はな|れ}で、
%
\ruby{今}{いま}
\ruby{自{\換字{分}}}{じ|ぶん}の
\ruby{入}{はい}つて
\ruby{居}{ゐ}る
\ruby{汚}{きたな}い
\ruby{家}{うち}は、
%
\ruby{{\換字{平}}井}{ひら|ゐ}の
\ruby{家}{うち}の
\ruby{榮}{さか}えて
\ruby{居}{ゐ}た
\ruby{頃}{ころ}の
\ruby{雜物小屋}{ざふ|もつ|ご|や}です。
%
\ruby{左樣}{さ|う}いふ
\ruby{婆}{ばあ}さんですから、
%
\ruby{今}{いま}ぢやあたゞ、
%
\ruby{金}{かね}より
\ruby{外}{ほか}に
\ruby{味方}{み|かた}は
\ruby{無}{な}いと
\ruby{思}{おも}つて、
%
まるで
\ruby{鬼}{おに}のやうになり
\ruby{切}{き}つて
\ruby{居}{ゐ}て、
%
\ruby{村}{むら}の
\ruby{者}{もの}にも
\ruby{憎}{にく}がられりやあ、
%
\ruby{自{\換字{分}}}{じ|ぶん}も
\ruby{村}{むら}の
\ruby{者}{もの}を
\ruby{對敵}{むか|ふ}にして
\ruby{居}{ゐ}るので
\ruby{云}{い}つて
\ruby{見}{み}りやあ
\ruby{愍然}{かは|いさう}な% 「愍然 か(は)いさう」
\ruby{筋}{すぢ}もあるのです。
』

\原本頁{}
『
\ruby{大}{おほ}きに、
%
\ruby{成程}{なる|ほど}!。
』

\原本頁{}
\ruby{水野}{みづ|の}は
\ruby{此談}{この|はなし}を
\ruby{聞}{き}きて
\ruby{黯然}{あん|ぜん}として、
%
\ruby{{\換字{情}}}{こゝろ}の
\ruby{傷}{きずつ}ける
\ruby{人}{ひと}の
\ruby{末路}{す|ゑ}の
\ruby{恐}{おそ}ろしさを
\ruby{思}{おも}ひつゝ
\ruby{歎}{たん}ずるところへ、
%
\ruby{忙}{あはた}だしく
\ruby{人}{ひと}の
\ruby{駈}{か}け
\ruby{來}{く}る
\ruby{跫音}{あし|おと}して、
%
\ruby{椽{\換字{前}}}{えん|さき}より、

\原本頁{}
『
\ruby{水野}{みづ|の}さん!
\ruby{水野}{みづ|の}さん!。
』

\原本頁{}
と
\ruby{呼}{よ}ぶは
\ruby{他人}{ほ|か}ならず
\ruby{松之助}{まつ|の|すけ}なり。

\原本頁{}
\ruby{其}{その}おろ〳〵したる
\ruby{悲}{かな}しき
\ruby{聲音}{こわ|ね}を
\ruby{聞}{き}くより、
%
\ruby{何}{なん}とは
\ruby{無}{な}しに
\ruby{胸潰}{むね|つぶ}れて、

\原本頁{}
『ど、
%
\ruby{何樣}{ど|う}かしたか?、
%
\ruby{惡}{わる}いのかえ?、
%
\ruby{姊}{ねえ}さんが。
』

\原本頁{}
と、
%
サツと
\ruby{障子}{しやう|じ}を
\ruby{開}{ひら}けば、
%
\ruby{{\換字{暖}}}{あたゝか}き
\ruby{不快}{ふ|くわい}の
\ruby{風}{かぜ}はムツと
\ruby{吹}{ふ}きて、
%
\ruby{黄昏}{たそ|がれ}の
\ruby{{\換字{空}}}{そら}の
\ruby{光線}{ひか|り}の
\ruby{{\換字{弱}}}{よわ}きに、
%
\ruby{恐怖}{おそ|れ}を
\ruby{懷}{いだ}ける
\ruby{松之助}{まつ|の|すけ}の
\ruby{顏}{かほ}は
\ruby{影}{かげ}さへ
\ruby{淋}{さみ}しく
\ruby{薄々}{うす|〳〵}と
\ruby{白}{しら}みて
\ruby{見}{み}えたり。

\原本頁{}
『
\ruby{大變}{たい|へん}に
\ruby{惡}{わる}い!。
%
いけないかも
\ruby{知}{し}れ……。
%
アヽ、
%
\ruby{僕}{ぼく}あ
\ruby{何樣}{ど|う}したら
\ruby{宜}{よ}からう!。
』

\原本頁{}
\ruby{既}{はや}
\ruby{泣}{な}き
\ruby{聲}{ごゑ}の、\換字{志}どろもどろの
\ruby{其}{その}
\ruby{言葉}{こと|ば}を
\ruby{聞}{き}くや
\ruby{聞}{き}かずや、
%
\ruby{水野}{みづ|の}は
\ruby{忽}{たちま}ち
\ruby{全身}{ぜん|しん}に
\ruby{氷}{こほり}の
\ruby{水}{みづ}を
\ruby{{\換字{浴}}}{あ}びし
\ruby{心地}{こゝ|ち}して、
%
アツとばかりに
\ruby{仆}{たふ}れんとしけるが、
%
\ruby{辛}{から}くも
\ruby{堪}{た}へて
\ruby{自}{みづか}ら
\ruby{保}{たも}ち、
%
\ruby{次}{つ}いで
\ruby{烈}{はげ}しき
\ruby{戰慄}{ふる|ひ}の
\ruby{止}{と}めても
\ruby{止}{と}まらず
\ruby{起}{おこ}り
\ruby{來}{く}るを
\ruby{{\換字{強}}}{し}ひて
\ruby{制}{せい}しつ、

\原本頁{}
『ナニ、
%
そんな
\ruby{事}{こと}が……、
%
\ruby{大{\換字{丈}}夫}{だい|ぢやう|ぶ}だ!。
』

\原本頁{}
と、
%
\ruby{我}{わ}が
\ruby{耳}{みゝ}にも
\ruby{知}{し}るゝ
\ruby{顫聲}{ふるひ|ごゑ}に
\ruby{云}{い}いさま、
%
\ruby{我}{われ}
\ruby{知}{し}らず
\ruby{我}{わ}が
\ruby{座}{ざ}より
\ruby{飛}{と}び
\ruby{立}{た}つて、
%
\ruby{踵}{かゝと}も
\ruby{地}{ち}に
\ruby{着}{つ}かぬ
\ruby{跣足}{は|だし}の
\ruby{危}{あやふ}く、
%
\ruby{轉}{まろ}ぶが
\ruby{如}{ごと}くに
\ruby{走去}{はせ|さ}つたり。

\Entry{其三十三}

% メモ 校正終了 2024-04-11
\原本頁{205-4}%
\ruby{槇籬}{まき|がき}
\ruby{隣}{とな}る
\ruby{木槿籬}{むく|げ|がき}、
%
\ruby{杉籬}{すぎ|がき}
つゞく
\ruby{藪疊}{やぶ|だゝみ}の、
%
\ruby{村徑}{むら|みち}の
\ruby{黄昏}{たそ|がれ}を
\ruby{息急}{いき|せは}しく
\ruby{走}{はし}る
\ruby{水野}{みづ|の}は、
%
\ruby{後}{あと}より
\ruby{{\換字{追}}}{お}ひ
\ruby{縋}{すが}れる
\ruby{松之助}{まつ|の|すけ}の
\ruby{手}{て}を
\ruby{引立}{ひつ|た}てゝ、
%
\ruby{夢}{ゆめ}に
\ruby{高}{たか}き
ところより
\ruby{落}{お}つるが
\ruby{如}{ごと}き
\ruby{膽}{きも}
\ruby{縮}{すく}む
\ruby{思}{おも}ひに、
%
\ruby{何}{なん}の
\ruby{{\換字{分}}別}{ふん|べつ}も
\ruby{無}{な}く
\ruby{駈}{か}けに
\ruby{駈}{か}けたり。

\原本頁{20}%
\ruby{今{\換字{朝}}}{け|さ}よりの
\ruby{風}{かぜ}に
\ruby{葉}{は}は
\ruby{裂}{さ}け
\ruby{茎}{くき}は
\ruby{折}{を}れ
\ruby{伏}{ふ}して、
%
\ruby{滿目}{まん|もく}の
\ruby{光景}{あり|さま}
\ruby{忌}{いま}はしく
\ruby{{\換字{狼}}{\換字{藉}}}{らう|ぜき}たる
\ruby{芋圃}{いも|ばた}の
\ruby{間}{あひだ}を、
%
\ruby{突}{つ}と
\ruby{行}{ゆ}き
\ruby{拔}{ぬ}けて、
%
\ruby{例}{れい}の
\ruby{婆}{ばゞ}が
\ruby{家}{いへ}の
\ruby{横}{よこ}を
\ruby{奧}{おく}へと
\ruby{{\換字{通}}}{とほ}らんとすれば、
%
\ruby{折}{をり}しも
\ruby{例}{れい}の
お
\ruby{澤}{さは}
\ruby{婆}{ばゞ}は、
%
\ruby{風}{かぜ}に
\ruby{捥}{も}がれたる
\ruby{柹}{かき}の
\原本頁{206-1}\改行%
\ruby{實}{み}の、
%
\ruby{或}{あるひ}は
\ruby{{\換字{猶}}}{なほ}
\ruby{靑}{あを}く、
%
\ruby{或}{あるひ}は
\ruby[<j|]{{\換字{半}}}{なかば}
\ruby{黄}{き}ばみ
\ruby{赤}{あか}らめるを、
\換字{志}たゝかに
\ruby{取}{と}り
\ruby{入}{い}れたる
\ruby{重}{おも}げなる
\ruby{箕}{み}に、
%
\ruby{枯柴}{かれ|しば}の
\ruby{如}{ごと}く
\ruby{骨立}{ほね|だ}つたる
\ruby{兩腕}{りやう|うで}を% 原本通り踊り字表記はしない
\ruby{長}{なが}く
\ruby{露}{あらは}して
\ruby{掛}{か}けつ、
%
\ruby{一}{ひ}
ト
\ruby{歩}{あし}
\ruby{一}{ひ}
ト
\ruby{歩}{あし}に
\ruby{{\換字{強}}欲}{がう|よく}の
\ruby{力}{ちから}を
\ruby{入}{い}れて
\ruby{辛}{から}くも
\ruby{吾家}{わが|や}に
\ruby{{\換字{運}}}{はこ}ばんと、
%
\ruby{未}{ま}だ
\ruby{止}{や}まぬ
\ruby{風}{かぜ}に
\ruby{霜}{しも}の
\ruby{薄}{すゝき}と
\ruby{騷立}{さわ|だ}つ
\ruby{白髮}{しら|が}を
\ruby{吹}{ふ}き
\ruby{立}{た}たせながら
\原本頁{206-5}\改行%
\ruby{此方}{こな|た}へ
\ruby{來}{き}かゝりしが、
%
\ruby{水野}{みづ|の}が
\ruby{慌}{あわ}て
\ruby{{\換字{狼}}狽}{うろ|た}へて
\ruby{入}{い}り
\ruby{來}{きた}れる
\ruby{態}{さま}を、
%
\ruby{圓}{つぶら}なる
\ruby{眼}{め}に
ぎろりと
\ruby{見}{み}て、
%
さも
\ruby{心地}{こゝ|ち}よげに
\ruby{冷笑}{あざ|わら}ひ、

\原本頁{206-7}%
『とう〳〵
\ruby{廿兩}{にじう|りやう}になつて
\ruby{來}{き}たゞかネ?。
』

\原本頁{206-8}%
と、
%
\ruby{恰}{あだか}も% 恰も「あ(だ)かも」
\ruby{病}{や}める
\ruby{人}{ひと}の
\ruby{疾}{と}く
\ruby{死}{し}なんことを
\ruby{待設}{まち|まう}け
\ruby{居}{を}りし
\ruby{其}{そ}の
\ruby{甲{\換字{斐}}}{か|ひ}ありて、
%
\ruby{今}{いま}や
\ruby{我}{わ}が
\ruby{望}{のぞ}める
\ruby{時機}{と|き}の
\ruby{至}{いた}らんとするに、
%
\ruby{自}{みづか}ら
\ruby{先}{ま}づ
\ruby{聲}{こゑ}を
\ruby{揚}{あ}げて
\ruby{祝}{しゆく}し
\ruby{悅}{よろこ}べるが
\ruby{如}{ごと}く
\ruby{云}{い}ひぬ。

\原本頁{206-11}%
おのが
\ruby{手}{て}に
\ruby{些少}{すこ|し}ばかりの
\ruby{金子}{か|ね}の
\ruby{落}{お}ちんことを
\ruby{希}{ねが}ふ
\ruby{意}{こゝろ}より、
%
\ruby{他}{ひと}の
\ruby{生命}{いの|ち}
\ruby{掛}{か}けて
\ruby{思}{おも}へる
\ruby{人}{ひと}をも
\ruby{死}{し}ねがしに
\ruby{云}{い}ひなしたる
\ruby{此}{こ}の
\ruby{老婆}{ば|ゞ}の
\ruby{面}{つら}の
\ruby{憎}{にく}さ!。
%
\ruby{人}{ひと}には
あらずと
\ruby{豫}{かね}てより
\ruby{思}{おも}ひ
\ruby{居}{ゐ}たれど、
%
まのあたりに
\ruby{骨}{ほね}を
\ruby{刺}{さ}す
\ruby{此}{こ}の
\ruby{酷毒}{こく|どく}の
\ruby{語}{ことば}を
\ruby{{\換字{浴}}}{あび}せられては、
%
\ruby{頭脳}{あた|ま}の
\ruby{眞中}{まん|なか}より
\ruby{烈火}{れつ|くわ}の
\ruby{奔}{はし}る
\ruby{心地}{こゝ|ち}して、
%
おのれ
\ruby{憎}{につく}き
\ruby{獸畜}{けだ|もの}め、
%
たゞ
\ruby{一}{ひ}
ト
\ruby{攫}{つかみ}に
\ruby{引攫}{ひつ|つか}んで、
%
\原本頁{207-5}\改行%
\ruby{天狗裂}{てん|ぐ|ざ}きに
\ruby{裂}{さ}きて
\ruby{木}{き}の
\ruby{股}{また}
\ruby{高}{たか}く
\ruby{掛}{か}けて
\ruby{吳}{く}れんと、
%
むら〳〵と
\ruby{恐}{おそ}ろしき
\ruby{忿怒}{いか|り}の
\ruby{衝}{つ}き
\ruby{上}{あが}り
\ruby{來}{き}て、
%
\ruby{流石}{さす|が}に
\ruby{堪}{こら}へ
\ruby{{\換字{情}}}{じやう}
\ruby{{\換字{強}}}{つよ}き
\ruby{水野}{みづ|の}も
\ruby{眞靑}{まつ|さを}に
なりたり。

\Entry{其三十四}

% メモ 校正終了 2024-04-11 2024-05-28 2024-06-23 2024-06-25
\原本頁{207-9}%
\ruby{吉右衛門}{きち||ゑ|もん}が
\ruby[||j>]{物}{もの}
\ruby[||j>]{語}{がたり}に
% \ruby{物語}{もの|がたり}に
よりて
\ruby{此}{こ}の
\ruby{婆}{ばゞ}が
\ruby{身}{み}の
\ruby{上}{うへ}を
\ruby{聞}{き}かざりせば、
%
\ruby{或}{あるひ}は
\ruby{走}{はし}り
かゝりて
\ruby{一}{ひ}ト
\ruby{踢}{け}に
\ruby{踢}{け}
\ruby{倒}{たふ}すか、
%
\ruby{左}{さ}なくば
\ruby{其}{その}
\ruby{面}{おもて}に
\ruby{唾}{つばき}して
\ruby{罵}{のゝし}る
ほどの
\ruby{事}{こと}は
\ruby{爲}{し}たらんを、
%
\ruby{其}{そ}の
\ruby[g]{如是}{か く }
\ruby[g]{鬼々}{おに〳〵}しくなれる
\ruby[g]{{\換字{所}}以}{ゆ ゑん}を
\ruby{思}{おも}ひ
\ruby{{\換字{浮}}}{うか}むると、
%
\ruby{且}{かつ}は
\ruby[g]{如是}{かゝる }
\ruby[g]{老婆}{ば ゞ }を
\ruby[g]{相手}{あひて }に
\ruby{取}{と}りて
\ruby{何}{なに}と
なすべき、
%
\ruby{田}{た}に
\ruby{生}{うま}れて
\ruby{田}{た}に
\ruby{死}{し}する
\ruby{蟲}{むし}にも
\ruby{等}{ひと}しき
\ruby[g]{田舎}{ゐ なか}% ルビ調整(原本通り)
\ruby{婆}{ばゞ}の
\ruby[g]{一言}{ひとこと}に、
%
\ruby{氣}{き}を
\ruby{動}{うご}かして
\ruby{我}{われ}を
\ruby{忘}{わす}れん
としたるは
\ruby{愚}{おろか}なりと、
%
\ruby{{\換字{飽}}}{あく}まで
\ruby{{\換字{強}}}{つよ}く
\ruby[g]{見下}{み さ }げたるとに、
%
おのづと
\ruby{心}{こゝろ}も
\ruby{{\換字{緩}}}{ゆる}み
\ruby{和}{やはら}ぎて、
%
\ruby[g]{水野}{みづの }は
\ruby[g]{滿腔}{まんこう}の
% 満腔(まんこう)
% 全身。体中。または、心底。心から。
% 「満腔の思い」「満腔の怒り」などのように用い、情念が全身全霊を包むさまを表現する。
% 「満腔の謝意」など、心が込もっている意にも用いられる。
\ruby{燃}{も}ゆる
\ruby[g]{忿恚}{いかり }を
\ruby{僅}{わづか}に
\ruby{怪}{あや}しき
\ruby[g]{侮蔑}{いやしみ}の
\ruby{笑}{わらひ}に
\ruby{洩}{も}らして、
%
\ruby[g]{言葉}{ことば }も
\ruby{無}{な}く
\ruby{突}{つ}と
\ruby{擦}{す}れ
\ruby{{\換字{違}}}{ちが}つて
\ruby{去}{さ}り
\ruby{行}{ゆ}けば、
%
\ruby{婆}{ばゞ}は
\ruby{{\換字{猶}}}{なほ}
\ruby{其}{そ}の
\ruby[<j||]{後}{うしろ}
\ruby[||j>]{姿}{すがた}を
\ruby[g]{見{\換字{送}}}{み おく}つて、

\原本頁{208-8}%
『
\ruby[<j>]{怖}{おつかな}い
\ruby{顏}{かほ}して
\ruby{怒}{おこ}つたつて
\ruby[g]{無益}{だ め }な
\ruby{事}{こん}だ。
%
そんなに
\ruby{怒}{おこ}つて
\ruby{歩}{ある}いて
\ruby[g]{柹實}{か き }を
\ruby{踏}{ふ}み
\ruby{潰}{つぶ}しては
ならねえだよ。
%
ハヽハヽ。
』

\原本頁{208-10}%
と、
%
\ruby{侮}{あなど}り
\ruby{笑}{わら}ひぬ。

\原本頁{208-11}%
\ruby[||j>]{面}{おもて}
を
\ruby{對}{あは}せたる
\ruby{時}{とき}にだに
\ruby{既}{すで}に
\ruby{{\換字{忍}}}{しの}びたれば、
%
\ruby[g]{背後}{はいご }の
\ruby{笑}{わらひ}には
\ruby{耳}{みゝ}をも
\ruby{假}{か}さず、
%
\ruby{柹}{かき}の
\ruby{樹}{き}
\ruby[g]{幾本}{いくもと}の
\ruby{下}{した}を
\ruby{潜}{くぐ}りて、% 【潛 u6f5b 「先先」】【潜 u6f5c 「夫夫」】併用されている
%
\ruby{我}{わ}が
\ruby{五十子}{い|そ|こ}の
\ruby{病}{や}みて
\ruby{臥}{ふ}せる
\ruby[g]{別室}{はなれ }
\ruby{{\換字{近}}}{ちか}く
\ruby{到}{いた}れば、
%
\ruby{風}{かぜ}の
\ruby{騷}{さわ}がしきを
\ruby{厭}{いと}ひたりと
\ruby{見}{み}えて、
%
はや
\ruby{{\換字{戸}}}{と}を
\ruby{引}{ひ}きたるが、
%
\ruby{中}{なか}には
\ruby{燈}{ひ}の
\ruby[||j>]{光}{ひかり}
\ruby[||j>]{{\換字{弱}}}{ よわ}く
\ruby{籠}{こも}りて、
%
\ruby{人}{ひと}の
\ruby{動}{うご}ける
\ruby{影}{かげ}の
ちら〳〵としたり。

\原本頁{209-5}%
\ruby{今}{いま}までは
\ruby{先}{さき}に
\ruby{立}{た}ちて
\ruby{來}{きた}れる
\ruby[g]{水野}{みづの }の、
%
\ruby[g]{此處}{こ ゝ }に
\ruby{至}{いた}りて
\ruby{俄}{にはか}に
\ruby{歩}{あゆ}み
\ruby{鈍}{にぶ}れば、
%
\ruby{松之助}{まつ|の|すけ}の
\ruby{方}{かた}、
%
\ruby{先}{さき}に
なりて、
%
\ruby{既}{すで}に
\ruby[g]{沓脫}{くつぬぎ}に
\ruby{一}{ひ}ト
\ruby{足}{あし}
\ruby{踏}{ふ}み
\ruby{入}{い}るゝに、
%
\ruby[g]{水野}{みづの }は
\ruby{其}{そ}の
\ruby{執}{と}りたる
\ruby{手}{て}を
\ruby[||j>]{力}{ちから}
\ruby[||j>]{無}{ な}く
\ruby{放}{はな}して、
%
\ruby{續}{つゞ}いて
\ruby{入}{い}らん
ともせず
\原本頁{209-8}\改行%
\ruby[g]{立{\換字{迷}}}{たちまよ}ひ
\ruby{居}{ゐ}たり。

\原本頁{209-9}%
\ruby{此}{こ}の
\ruby{心得{\換字{難}}}{こゝろ|え|がた}き
\ruby[g]{擧動}{ふるまひ}の
\ruby{意}{こゝろ}を、
%
\ruby{松之助}{まつ|の|すけ}は
\ruby{{\換字{更}}}{さら}に
\ruby{解}{と}く
\ruby[g]{由無}{よしな }ければ、
%
\ruby[g]{振顧}{ふりかへ}りて
\ruby[g]{此度}{こ たび}は
\ruby{我}{わ}が
\ruby{手}{て}に
\ruby[g]{水野}{みづの }の
\ruby{手}{て}を
\ruby{執}{と}り、
%
\ruby{疾}{と}く
\ruby[g]{此方}{こなた }へ% ルビ調整(原本通り)
\ruby{上}{あが}れよと
\ruby{眼}{め}に
\ruby{云}{い}はせて
\ruby[g]{引張}{ひつぱ }つたり。

\原本頁{210-1}%
\ruby{言}{い}はず
\ruby{語}{かた}らずの
\ruby{我}{わ}が
\ruby{誠}{まこと}の
\ruby{{\換字{情}}}{こゝろ}は、
%
\ruby{知}{し}らず
\ruby{識}{し}らずに
\ruby{他}{ひと}の
\ruby{優}{やさ}しき
\ruby{胸}{むね}に
\ruby{響}{ひゞ}きては、
%
\ruby[g]{可憐}{か はゆ}き
\ruby{我}{わ}が
\ruby{松之助}{まつ|の|すけ}は
\ruby{我}{われ}を
\ruby{兄}{あに}などの
やうに
\ruby{思}{おも}ひ
\ruby{做}{な}し
\ruby{取}{と}
\原本頁{210-3}\改行%
り
\ruby{做}{な}して、
%
\ruby{泣}{な}き
\ruby{顏}{がほ}に
\ruby{姊}{あね}が
\ruby{急}{きふ}を
\ruby{訴}{うつた}へに
\ruby{來}{きた}りし
それに
\ruby{釣}{つ}り
\ruby{{\換字{込}}}{こ}まれて
\改行% 校正作業の簡略化のため
、
%
\原本頁{210-4}\改行%
ハツと
\ruby{驚}{おどろ}きし
\ruby{餘}{あま}りに
\ruby{何}{なに}といふ
\ruby{考}{かんが}へも
\ruby{無}{な}く、
%
\ruby{走}{はし}り
\ruby{出}{い}でゝ
\ruby[g]{此處}{こ ゝ }へは
\ruby{來}{きた}りしものゝ、
%
\ruby[g]{如何}{い か }なる
\ruby[g]{宿世}{しゆくせ}の
\ruby{仇}{あだ}の
ありてか、
%
\ruby{我}{わ}が
\ruby{五十子}{い|そ|こ}を
\ruby{思}{おも}ふ
\ruby{心}{こゝろ}の
\ruby{募}{つの}るだけに、
%
\ruby{五十子}{い|そ|こ}の
\ruby{我}{われ}を
\ruby{厭}{いと}ふ
\ruby{{\換字{情}}}{こゝろ}も
\ruby{漸}{やうや}く
\ruby{募}{つの}りて、
%
\ruby{特}{こと}に
\ruby[<j||]{病}{びやう}% 行末行頭の境界付近なので特例処置を施す
\ruby[||j>]{氣}{き}の
\ruby{爲}{さ}する
\ruby{癇}{かん}の
\ruby[g]{{\換字{所}}爲}{わ ざ }とは
\ruby{云}{い}へ、
%
\ruby{此}{こ}の
\ruby{頃}{ごろ}は
\ruby{我}{わ}が
\ruby{面}{おもて}を
\ruby{見}{み}るをさ
\ruby[<g>]{へ甚}{はなはだ}しく% 行末行頭の境界付近なので特例処置を施す
\ruby{忌}{い}み
\ruby{{\換字{嫌}}}{きら}ふやうになり
\ruby{居}{を}れるなれば、
%
\ruby{我}{われ}は
こそ
\ruby{其}{そ}の
\ruby{人}{ひと}の
\ruby{傍}{そば}に
\ruby{在}{あ}りて
\ruby{兎}{と}も
\ruby{角}{かく}も
なるを
\ruby[g]{見果}{み はて}んと
\ruby{願}{ねが}へ、
%
\ruby{今}{いま}
その
\ruby[g]{病狀}{やうす }の
\ruby{凶}{あし}き
\ruby{盛}{さか}りに
\原本頁{210-10}\改行%
\ruby{我}{わ}が
\ruby{面}{おもて}を
\ruby{見}{み}せて、
%
その
\ruby{人}{ひと}に
\ruby[<j>]{快}{こゝろよ}からぬ
\ruby{思}{おもひ}させんことは、
%
たとへば
また
\ruby{復}{ふたゝ}び
\ruby{戀}{こひ}しき
\ruby{人}{ひと}の
\ruby{此}{こ}の
\ruby{世}{よ}の
\ruby{顏}{かほ}を
\ruby{見}{み}るを
\ruby{得}{え}ざるに
\ruby{至}{いた}らん
\ruby{其}{そ}の
\ruby{悲}{かな}しさは、
%
\ruby{能}{よ}く
\ruby{{\換字{忍}}}{しの}ぶべし
とするも、
%
これは
\ruby{{\換字{忍}}}{しの}ぶに
\ruby{{\換字{忍}}}{しの}びがたき
とこ
\原本頁{211-2}\改行%
ろなり。
%
\ruby{特}{こと}に
われは
\ruby{死}{し}を
\ruby{起}{おこ}し
\ruby{生}{せい}を
\ruby{囘}{かへ}すの% 原本通り「囘」
\ruby{{\換字{道}}}{みち}を
\ruby{知}{し}れるにも
あらず
\改行% 校正作業の簡略化のため
、
%
\原本頁{211-3}\改行%
また
\ruby{我}{わ}が
\ruby[g]{岩崎}{いはさき}% 原本のこの部分は「いわさき」
\ruby{氏}{うぢ}に
\ruby{何}{なん}の
\ruby[g]{因緣}{ゆ かり}もあるにもあらず、
%
\ruby{云}{い}はゞ
\ruby{赤}{あか}の
\ruby[g]{他人}{た にん}
の
\ruby{身}{み}をもて、
%
\ruby{然}{さ}らぬだに
\ruby{生}{い}くる
\ruby{死}{し}ぬるの
\ruby{境}{さかひ}に
\ruby{惱}{なや}める
\ruby{人}{ひと}の
\ruby[g]{枕頭}{まくらべ}に
\原本頁{211-5}\改行%
\ruby{見}{あらは}れて、
%
\ruby{其}{そ}の
\ruby{人}{ひと}に
\ruby{忌}{いま}はしき
\ruby{思}{おもひ}を
さするほかには
\ruby{何}{なん}の
\ruby{能}{のう}も
\ruby{無}{な}き
\ruby[<j||]{面}{おもて}
\原本頁{211-6}\改行%
を
\ruby{差}{さ}し
\ruby{出}{だ}さん
\ruby[g]{心無}{こゝろな}さは、
%
\ruby{我}{わが}
\ruby{爲}{な}し
\ruby{得}{う}べきところならんや。
%
\ruby{痩}{や}せたる
\ruby{其}{そ}の
\ruby{人}{ひと}の
\ruby{手}{て}をも
\ruby{執}{と}り、
%
\ruby{冷}{ひ}えんとする
\ruby{其}{その}
\ruby{人}{ひと}の
\ruby{身}{み}をも
\ruby{溫}{あたゝ}めて、
%
\ruby{及}{およ}ばぬまでも
\ruby[||j>]{心}{こゝろ}
\ruby[||j>]{限}{ かぎ}りの
% \ruby{心限}{こゝろ|かぎ}りの
\ruby[g]{介抱}{かいはう}を
\ruby{仕}{し}たき
\ruby{望}{のぞみ}は
\ruby[g]{熾盛}{さかん }なれども、
%
\ruby[g]{因緣}{いんねん}の
\ruby{恨}{うら}めしくも
\ruby{悲}{かな}しくも
\ruby{厭}{いと}ひ
\ruby{{\換字{嫌}}}{きら}はれたる
\ruby{身}{み}の
\ruby{其}{それ}も
\ruby{叶}{かな}はず、
%
たゞ
\ruby{{\換字{戸}}}{と}の
\ruby{外}{そと}に
\ruby{泣}{な}き
\ruby{惑}{まど}ひて、
%
あだに
\ruby{物}{もの}を
\ruby{思}{おも}ひ
\ruby{心}{こゝろ}を
\ruby{苦}{くるし}めん
ためばかりに
\ruby[g]{此處}{こ ゝ }に
\原本頁{211-11}\改行%
\ruby{來}{きた}りし
\ruby[g]{冥利}{みやうり}の
\ruby{拙}{つたな}さ!、
%
\ruby{我}{わ}が
\ruby{愚}{おろか}さ!。
%
\ruby{思}{おも}へば
\ruby{何}{なん}とせん
\ruby{意}{こゝろ}にて
\ruby[g]{此處}{こ ゝ }に
\ruby{走}{はし}りては
\ruby{來}{きた}りしぞや。
%
\ruby[g]{甲{\換字{斐}}}{か ひ }なくも
\ruby[g]{甲{\換字{斐}}}{か ひ }
\ruby{無}{な}く
\ruby{氣}{き}を
\ruby{揉}{も}みて、
%
たゞ% ルビ調整(原本通り)踊り字表記(行末行頭の境界付近)
たゞ
\ruby{亂}{みだ}れて
\ruby{絲}{いと}の
\ruby{如}{ごと}き
\ruby{思}{おもひ}に、
%
\ruby{獨}{ひと}り
\ruby{泣}{な}くより
ほかには
\ruby{爲}{な}すべき
\ruby{我}{わ}が
\ruby{事}{こと}もあらざる
\ruby[||j>]{{\換字{情}}}{なさけ}
\ruby[||j>]{無}{ な}さを
\ruby[g]{如何}{い か }にせん。

\原本頁{212-4}%
と
\ruby{松之助}{まつ|の|すけ}の
\ruby{手}{て}を
そつと
\ruby{拂}{はら}つて、
%
\ruby{面}{おもて}を
かくしつゝ
\ruby{逸}{そ}れたる
\ruby[g]{水野}{みづの }は
\改行% 校正作業の簡略化のため
、
%
\原本頁{212-5}\改行%
\ruby{家}{いへ}の
\ruby[g]{背後}{うしろ }の
\ruby{椎}{しひ}の
\ruby[g]{老樹}{おいき }の
\ruby{幹}{みき}に
\ruby{頭}{かうべ}を
\ruby{埋}{うづ}めて、
%
こんもりとしたる
\ruby{其}{その}
\ruby{陰}{かげ}には、
%
はや
\ruby[g]{夕闇}{ゆふやみ}の
\ruby{逼}{せま}りて
\ruby{昏}{くら}くなれるが
\ruby{中}{なか}に
\ruby[g]{立盡}{たちつく}せり。

\原本頁{212-7}%
\ruby{風}{かぜ}は
\ruby{{\換字{猶}}}{なほ}
\ruby{吹}{ふ}けど
やゝ
\ruby{衰}{おとろ}へて、
『
\ruby{四十七士}{し|じふ|しち|し}の% 原本には漢数字「七」のルビ無し
\ruby{墓}{はか}どころ、
%
\ruby{{\換字{雪}}}{ゆき}は
\ruby{{\換字{消}}}{き}えても
%
\原本頁{212-9}\改行%
\ruby{名}{な}は
\ruby{殘}{のこ}る、
』
% 鉄道唱歌 東海道篇 二 の一部から
% 右は高輪泉岳寺           / 四十七士の墓どころ       / 雪は消えても消えのこる   / 名は千載の後までも
% みぎはたかなわせんがくじ / しじふしちしのはかどころ / ゆきはきえてもきえのこる / なはせんざいののちまでも
と、
%
\ruby{村}{むら}の
\ruby{兒}{こ}が
\ruby[g]{{\換字{遠}}方}{とほく }にて
\ruby{唱}{うた}ふ
\ruby{金切聲}{かな|きり|ごゑ}の
\ruby{幽}{かすか}に
\ruby{聞}{きこ}えくるも
\原本頁{212-10}\改行%
\ruby{時}{とき}に
\ruby{取}{と}りて
\ruby{忌}{いま}はしく、
%
\ruby{塒}{ねぐら}に
\ruby{急}{いそ}ぐ
\ruby{歸}{かへ}り
\ruby{鴉}{がらす}の
\ruby{二三羽}{に|さん|ば}
\ruby{鳴}{な}きつれたるも
\ruby[g]{耳立}{みゝだ }つて
\ruby{淋}{さび}しく、
%
\ruby{其}{その}
\ruby{後}{ゝち}は
\ruby[g]{物音}{ものおと}も
\ruby{無}{な}く
\ruby{日}{ひ}は
\ruby{暮}{く}れんとす。

\Entry{其三十五}

% メモ 校正終了 2024-04-11 2024-05-28 2024-06-25
\原本頁{213-2}%
\ruby{我}{われ}は
\ruby{今}{いま}
\ruby{何}{なに}として
\ruby{來}{きた}りけん
\ruby{我}{われ}
\ruby{知}{し}らず、
%
\ruby{我}{われ}は
\ruby{今}{いま}
\ruby{何}{なに}となさば
\ruby{宜}{よ}からん
\ruby{我}{われ}
\ruby{知}{し}らず、
%
\ruby{我}{われ}は
たゞ
\ruby[g]{此處}{こ ゝ }に
\ruby{來}{こ}では
\ruby{叶}{かな}はざるやう
\ruby{思}{おも}ひて
\ruby[g]{此處}{こ ゝ }に
\ruby{來}{きた}り、
%
\ruby{我}{われ}は
たゞ
\ruby[g]{此處}{こ ゝ }を
\ruby{去}{さ}りがたき
\ruby[g]{心地}{こゝち }するばかりに
\ruby[g]{此處}{こ ゝ }に
\ruby{在}{あ}るなり、
%
\ruby{來}{きた}れるが
\ruby{他}{ひと}の
\ruby{益}{やく}にも
\ruby{立}{た}たず、
%
\ruby{在}{あ}るが
\ruby{思}{おも}ひの
\ruby{晴}{は}るゝ
\ruby{業}{わざ}にも
あらざるを、
%
\ruby[g]{女々}{め ゝ }しくも
\ruby[g]{男兒}{をとこ }らしからぬ
\ruby[g]{振舞}{ふるまひ}をするかな!。
%
\ruby[<j||]{愚}{おろか}な
\原本頁{213-7}\改行%
りとも
\ruby[g]{日頃}{ひ ごろ}の
\ruby{我}{われ}は
\ruby[g]{如是}{か く }は
あらざりしものを、
%
\ruby{意氣地}{い|く|ぢ}
\ruby{無}{な}くも
\ruby[g]{崩折}{くづを }れたる
\ruby{心}{こゝろ}の
\ruby{何}{なに}を
\ruby{待}{ま}てるぞや!。
%
\ruby[g]{醫藥}{い やく}の
\ruby[g]{力は}{ちから }
\ruby{限}{かぎり}あり、
%
\ruby[<j||]{定}{ぢやう}
\ruby[||j>]{命}{みやう}
は
\ruby[g]{如何}{いかん }とも
\ruby{爲}{な}しがたければ、
%
その
\ruby{人}{ひと}の
\ruby[g]{魂魄}{た ま }の
\@ifundefined{デバッグ@ビルド}{%
  \ruby[g]{{\換字{情}}無}{なさけな}くも
}{%
  \ruby[||j>]{{\換字{情}}}{なさけ}
  \ruby[||j>]{無}{ な }くも
}%
\ruby{天}{そら}に
\ruby{去}{さ}つて、
%
\ruby{松之助}{まつ|の|すけ}の
\ruby[g]{泣聲}{なきごゑ}の
わつと
\ruby{起}{おこ}らん
\ruby{時}{とき}、
%
\ruby{我}{われ}は
\ruby{其}{そ}の
\ruby{聲}{こゑ}を
\ruby{聞}{き}いて
\ruby{世}{よ}を
\ruby{思}{おも}い
\ruby{切}{き}り、
%
\ruby{此}{こ}の
\ruby{椎}{しひ}の
\ruby{幹}{みき}の
\ruby{岩}{いは}の
ごときに、
%
\ruby{額}{ひたひ}を
\ruby[g]{打付}{うちつ }け
\ruby[g]{頭顱}{なづき }を
\ruby{破}{わ}つて、
%
よしや
\ruby{身}{み}は
\ruby[g]{輪{\換字{廻}}}{りんね }の
\ruby{闇}{やみ}に
\ruby{{\換字{迷}}}{まよ}ひ
\ruby{入}{い}るとも、
%
\ruby[g]{一念}{おもひ}は
\ruby[g]{芳魂}{はうこん}の
\ruby[g]{行方}{ゆくへ }を
\ruby{{\換字{追}}}{お}ひて
\改行% 校正作業の簡略化のため
、
%
\原本頁{214-3}\改行%
\ruby[g]{紫雲}{し うん}の
\ruby{{\換字{空}}}{そら}の
\ruby{遙}{はる}けくも
あれ、
%
\ruby[||j>]{黄}{くわう}
\ruby[||j>]{泉}{ せん}の
% \ruby{黄泉}{くわう|せん}の
\ruby{涯}{はて}の
\ruby{{\換字{遠}}}{とほ}くも
あれ、
%
つれなき
\ruby{風}{かぜ}
の
\ruby{持}{も}て
\ruby{去}{さ}れる
\ruby{花}{はな}の
\ruby{香}{かをり}に
\ruby{引}{ひ}かされて、
%
あくがれ
\ruby[g]{漂泊}{さまよ }ふ
\ruby{蝶}{てふ}の
\ruby{如}{ごと}くに
\改行% 校正作業の簡略化のため
、
%
\原本頁{214-5}\改行%
\ruby{{\換字{飽}}}{あく}まで
\ruby{戀}{こひ}しき
\ruby{人}{ひと}に
\ruby{{\換字{伴}}}{ともな}はんとて、
%
こゝには
\ruby{{\換字{空}}}{むな}しく
\ruby{佇}{たゝず}める
\ruby{歟}{か}。
%
\ruby{或}{ある}は
\原本頁{214-6}\改行%
\ruby{{\換字{又}}}{また}
\ruby{{\換字{強}}}{つよ}く
\ruby{忌}{い}み
\ruby{{\換字{嫌}}}{きら}はれたるより、
%
\ruby{堪}{た}へがたき
\ruby[g]{苦悶}{も だえ}に
\ruby{自}{みづか}ら
\ruby{堪}{た}へて、
%
\ruby{其}{その}
\ruby{人}{ひと}に
\ruby{{\換字{近}}}{ちか}づきもせず
\ruby{{\換字{過}}}{すご}し% 国会図書館では「すご」、国文学研究資料館では「 ご」
\ruby{居}{ゐ}けるが、
%
\ruby{{\換字{若}}}{も}し
\ruby[g]{不幸}{ふ かう}にして
\ruby{其}{そ}の
\ruby[g]{{\換字{遠}}慮}{ゑんりよ}の
\makeatletter
\@ifundefined{デバッグ@ビルド}{%
  \ruby[||j>]{俄}{にはか}に
}{%
  \ruby[<j||]{俄}{にはか}に% 行末行頭の境界付近なので特例処置を施す
}%
\makeatother
\ruby{失}{う}すべき
\ruby{時}{とき}にも
\ruby{至}{いた}らば、
%
\ruby{先}{ま}ず
\ruby{枕}{まくら}の
\ruby{邊}{ほとり}に
\ruby{走}{はし}り
\ruby{寄}{よ}つて、
%
\ruby{我}{わ}が
\ruby{火}{ひ}と
\原本頁{214-9}\改行%
\ruby{熱}{あつ}き
\ruby[g]{萬石}{ばんこく}の
\ruby{涙}{なみだ}を、
%
せめては
\ruby{其}{そ}の
\ruby{冷}{つめた}き
\ruby[g]{骸に}{かばね }
\ruby{親}{した}しく
\ruby{濺}{そゝ}ぎ、% 国文学研究資料館のは印字不鮮明で判読できず、国会図書館のを採用。
%
\ruby{{\換字{情}}}{つれ}
\ruby{無}{な}かりし
\ruby{其}{そ}の
\ruby{人}{ひと}の
\ruby{手}{て}を
\ruby{執}{と}り
\ruby{搖}{ゆさ}ぶりて、
%
\ruby{心}{こゝろ}ゆく
ばかり
\ruby[g]{號哭}{がうこく}せんとて
\ruby[g]{此處}{こ ゝ }には
\ruby{居}{ゐ}るにや。
%
それにもあらねば、
%
これにもあらず、
%
\ruby{何}{なに}せん
\makeatletter
\@ifundefined{デバッグ@ビルド}{%
  \ruby[||j>]{心}{こゝろ}は
}{%
  \ruby[<j||]{心}{こゝろ}は% 行末行頭の境界付近なので特例処置を施す
}%
\makeatother
\ruby{{\換字{更}}}{さら}に
\ruby{無}{な}くして、
%
\ruby{我}{われ}にも
\ruby{我}{われ}の
\ruby{解}{わか}らぬ
\ruby[g]{{\換字{感}}想}{おもひ }に、
%
たゞ
\ruby[g]{此處}{こ ゝ }を
\ruby{去}{さ}りかねて
\ruby[g]{水野}{みづの }は
\ruby{{\換字{猶}}}{なほ}
\ruby{立}{た}てり。

\原本頁{215-3}%
\ruby{暮}{く}るゝに
\ruby{{\換字{連}}}{つ}れて
\ruby{風}{かぜ}は
\ruby{收}{をさ}まり、
%
\ruby{闇}{やみ}は
\ruby{葉}{は}の
\ruby{密}{こ}みたる
\ruby{椎}{しひ}の
\ruby{{\換字{梢}}}{こずゑ}より
\ruby{廣}{ひろ}がつて、
%
\ruby{{\換字{終}}}{つひ}に
\ruby{其}{その}
\ruby{黑}{くろ}き
\ruby[<j>]{懷}{ふところ}の
\ruby{中}{うち}に
\ruby[g]{四邊}{あたり }を
\ruby{包}{つゝ}みぬ。

\原本頁{215-5}%
\ruby[g]{森々}{しん〳〵}と
\ruby{靜}{しづか}なる
\ruby{此}{こ}の
\ruby{日}{ひ}
\ruby{此}{こ}の
\ruby[||j>]{{\換字{宵}}}{ゆふべ}
\ruby[||j>]{天}{ てん}に
\ruby{星}{ほし}
\ruby{無}{な}し、
%
\ruby{星}{ほし}は
\ruby{死}{し}したるならん、
%
\ruby{地}{ち}に
\ruby{風}{かぜ}は
\ruby{{\換字{弱}}}{よわ}りぬ、
%
\ruby{風}{かぜ}は
\ruby{今}{いま}
おのが
\ruby[g]{墓穴}{はかあな}を
\ruby{{\換字{尋}}}{たづ}ねて
\ruby{永}{なが}く
\ruby{休}{やす}まんとせり
\改行% 校正作業の簡略化のため
。
%
\原本頁{215-7}\改行%
\ruby{{\換字{古}}}{ふ}りたる
\ruby{椎}{しひ}の
\ruby{木}{き}は
\ruby[g]{忽然}{こつぜん}として
\ruby{人}{ひと}の
\ruby{聲}{こゑ}をなし、

\原本頁{215-8}%
『
\ruby{衆生被困厄}{しゆ|じやう|び|こん|やく}、
%
\ruby{無量苦逼身}{む |りや|うく|ひつ|しん}、% ルビ調整(原本の空きを再現)
%
\ruby{觀音妙智力}{くわん|のん|めう|ち|りき}、% 「觀音」の読みは原本通り「くわん(の)ん」
%
\ruby{能救世間苦}{のう|ぐ|せ|けん|く}、
』

\原本頁{215-9}%
と
\ruby{囁}{さゝや}くが
\ruby{如}{ごと}くに
\ruby{誦}{じゆ}し
\ruby{出}{いだ}せり。

\原本頁{215-10}%
\ruby{椎}{しひ}の
\ruby[g]{那處}{いづく }に
\ruby{彼}{か}の
\ruby[||j>]{額}{ひたひ}
\ruby[||j>]{廣}{ ひろ}く
% \ruby{額廣}{ひたひ|ひろ}く
\ruby[g]{鼻細}{はなほそ}き
\ruby{老}{お}いたる
\ruby{男}{をとこ}の
\ruby{潛}{ひそ}み% 【潛 u6f5b 「先先」】【潜 u6f5c 「夫夫」】併用されている
\ruby{居}{を}れりや、
%
\ruby{聲}{こゑ}は
\ruby{全}{まつた}く
\ruby{其}{そ}の
\ruby{聲}{こゑ}なりけり。

\原本頁{216-1}%
\ruby{愚}{おろか}なり!、
%
こは
\ruby{我}{わ}が
\ruby{招}{よ}ばずして
\ruby{我}{わ}が
\ruby[g]{記臆}{き おく}の
\ruby{現}{あらは}れ
\ruby{來}{きた}れるには
\ruby{{\換字{過}}}{す}ぎざるものをと
\ruby[g]{水野}{みづの }が
\ruby{冷}{ひや}やかに
\ruby{聞}{き}きし
\ruby{時}{とき}は、
%
\ruby{其}{その}
\ruby{聲}{こゑ}は
\ruby{既}{はや}
\ruby{失}{う}せて
\ruby[g]{{\換字{遺}}響}{ひゞき }も
\ruby{無}{な}かりしが、
%
\ruby[g]{當時}{そのとき}
\ruby{椎}{しひ}の
\ruby[g]{大木}{おほき }は
\ruby{忽}{たちま}ち
\ruby{二}{ふた}つに
\ruby{裂}{さ}けて、
%
\ruby[g]{其處}{そ こ }に
\ruby{明}{あき}らかなる
\ruby[g]{世界}{せ かい}の
\ruby{朗}{ほが}らかに
\ruby{現}{あらは}れたるが
\ruby{中}{うち}に、
%
\ruby[g]{年齡}{と し }は
\ruby{二十四五}{に|じふ|し|ご}なる
\ruby[<j||]{男}{をとこ}の% 行末行頭の境界付近なので特例処置を施す
\ruby{戀}{こひ}に
\ruby{窶}{やつ}れたる
\ruby{顏}{かほ}の
\ruby[g]{勇威}{いきほひ}
\ruby{無}{な}く
\ruby[g]{光釆}{ひかり }
\ruby{無}{な}く、
%
\ruby{五月雨}{さ|み|だれ}の
\ruby{檐}{のき}の
\ruby{雫}{しづく}と
\ruby{涙}{なみだ}を
\原本頁{216-6}\改行%
\ruby{放}{はふ}らし
\ruby{落}{おと}し
\ruby{居}{を}れるさまの
\ruby{醜}{みにく}くも
\ruby{醜}{みにく}きを、
%
\ruby{右}{みぎ}の
\ruby{肩}{かた}には
\ruby{恐}{おそ}ろしき
\ruby[g]{猛鷲}{あらわし}を
\ruby{宿}{と}まらしめ、
%
\ruby{後}{うしろ}には
\ruby{凄}{すさま}じき
\ruby[g]{大蛇}{だいじや}を
\ruby{隨}{したが}へたる
\ruby[g]{氣味}{き み }
\ruby{惡}{あ}しき
\ruby[<j||]{大}{おほ }
\ruby[<j||]{男}{をとこ}
% \ruby{大男}{おほ|をとこ}
\原本頁{216-8}\改行%
の、
%
\ruby{神}{かみ}に
\ruby{似}{に}て
\ruby{神}{かみ}の
\ruby{威}{ゐ}
\ruby{無}{な}く、
%
\ruby{人}{ひと}かと
\ruby{見}{み}れば
\ruby{人}{ひと}らしからぬが、
%
\ruby{憐}{あはれ}む
\原本頁{216-9}\改行%
が
\ruby{如}{ごと}く
\ruby{侮}{あなど}るが
\ruby{如}{ごと}き
\ruby{眼}{め}して
\ruby[g]{見詰}{み つ }め
\ruby{居}{ゐ}たるが
\ruby[g]{{\換字{分}}明}{あり〳〵}と
\ruby{見}{み}えぬ。

\Entry{其三十六}

\原本頁{217-1}%
\ruby{落葉}{おち|ば}を
\ruby{誘}{さそ}ふ
\ruby{山}{やま}
\ruby{下}{おろ}しの
\ruby{風}{かぜ}を
\ruby{其}{その}
\ruby{儘}{まゝ}なる
\ruby{猛鷲}{あら|わし}の
\ruby{打翥}{うち|はぶ}く
\ruby{音}{おと}の
\ruby{中}{うち}には、
%
『
\ruby{神明}{か|み}は
\ruby{殪}{たふ}れたり、
』
『
\ruby{佛陀}{ほと|け}は
\ruby{死}{し}したり、
』といふ
\ruby{響}{ひゞき}の
\ruby{聞}{きこ}え、
%
\ruby{首}{かうべ}を
\ruby{擡}{あ}げて
\ruby{蜿蜒}{う|ね}る
\ruby{大蛇}{だい|じや}の
ざわ〳〵と
\ruby{木茅}{き|かや}を
\ruby{倒}{たふ}し
\ruby{行}{ゆ}く
\ruby{音}{おと}の
\ruby{中}{うち}には、
%
『
\ruby{神明}{か|み}は
\ruby{想像}{さう|ざう}のみ、
』『
\ruby{佛陀}{ほと|け}は
\ruby{假說}{か|せつ}のみ、
』といふ
\ruby{聲}{こゑ}あり。

\原本頁{217-5}%
\ruby{水野}{みづ|の}は
\ruby{自}{みづ}から
\ruby{思}{おも}はずして
\ruby{自}{おのづ}から
\ruby{如是}{か|く}
\ruby{想}{おも}ひ、
%
\ruby{外}{そと}に
\ruby{見聞}{み|きゝ}せずして
\ruby{内}{うち}に
\ruby{如是}{か|く}
\ruby{見聞}{み|き}きせる
\ruby{時}{とき}、
%
\ruby{靜}{しづか}なる
\ruby{五十子}{い|そ|こ}が
\ruby{家}{いへ}の
\ruby{方}{かた}にて、
%
かたりと
\ruby{微}{かすか}の
\ruby{物音}{もの|おと}の
\ruby{仕}{し}たるを
\ruby{聞}{き}きつけ、
%
\ruby[||j>]{豁}{くわつ}
\ruby[||j>]{然}{ ぜん}として
% \ruby{豁然}{くわつ|ぜん}として
われに
\ruby{{\換字{返}}}{かへ}れば、
%
\ruby{我}{わ}が
\ruby{止}{と}め
\ruby{{\換字{途}}}{ど}
\ruby{無}{な}かりし
\ruby{涙}{なみだ}の
\ruby{何時}{い|つ}か
\ruby{乾}{かわ}き、
%
\ruby{我}{わ}が
\ruby{疲}{つか}れたる
\ruby{心}{こゝろ}の
\ruby{何時}{い|つ}か
\ruby{奮}{ふる}ひて、
%
\原本頁{217-9}\改行%
\ruby{倚}{よ}りかゝりたる
\ruby{椎}{しひ}の
\ruby{幹}{みき}を
\ruby{離}{はな}れ、
%
そを
\ruby{背向}{そ|がひ}にして
\ruby{挺然}{てい|ぜん}と
\ruby{獨}{ひと}り
\ruby{樹蔭}{こ|かげ}の
\ruby{闇}{やみ}に
\ruby{立}{た}ちつ、
%
\ruby{{\換字{魔}}}{ま}の
\ruby{如}{ごと}くに
\ruby{來}{きた}り
\ruby{{\換字{魔}}}{ま}の
\ruby{如}{ごと}くに
\ruby{去}{さ}る
\ruby{蝙蝠}{かは|ほり}の、
%
ひらひらと% 原本では行末禁則により非踊り字
\ruby{{\換字{梢}}}{こずゑ}の
\ruby{盡頭}{はず|れ}を
\ruby{飛}{と}びかへれるを、
%
\ruby{雲{\換字{透}}}{くも|ずき}に
\GWI{u1b048-u3099}つと% 「志」+「濁点」
\ruby{打見}{うち|み}やりたり。

\原本頁{218-1}%
\ruby{有}{あ}りや
\ruby[||j>]{神}{かみ}
\ruby[||j>]{佛}{ほとけ}の?、
% \ruby{神佛}{かみ|ほとけ}の?、
%
\ruby{有}{あ}るにも
\ruby{似}{に}たるかな!。
%
\ruby{無}{な}しや
\ruby[||j>]{神}{かみ}
\ruby[||j>]{佛}{ほとけ}の?、
% \ruby{神佛}{かみ|ほとけ}の?、
%
\ruby{無}{な}きにも
\ruby{似}{に}たるかな!。

\原本頁{218-3}%
\ruby{有}{あ}るには
\ruby{無}{な}きの
\ruby[<j>]{疑}{うたがひ}あり、
%
\ruby{無}{な}きには
\ruby{有}{あ}るの
\ruby[<j>]{疑}{うたがひ}あり、
%
\ruby{有}{あ}りとも
\ruby{爲{\換字{難}}}{し|がた}く、
%
\ruby{無}{な}しとも
\ruby{爲{\換字{難}}}{し|がた}し。
%
\ruby{有無}{う|む}の
いづれは
\ruby{今}{いま}
\ruby{知}{し}らねども、
%
\ruby{世}{よ}に
\ruby{無}{な}き
\ruby{方}{かた}の
\ruby{眞實}{まこ|と}ならば、% ここの「まこと」は「眞實」
%
\ruby{男兒}{をと|こ}の
\ruby{頭}{かうべ}を
\ruby{下}{さ}げて
\ruby{祈願}{き|ぐわん}を
\ruby{捧}{さゝ}げんことの
\ruby{羞}{はづか}しくも
\ruby{口惜}{くち|を}しく、
%
\ruby{{\換字{若}}}{も}し
\ruby{世}{よ}に
\ruby{在}{おは}す
\ruby{事}{こと}の
\ruby{定}{ぢやう}ならば、
%
\ruby{身}{み}をも
\ruby{魂魄}{たま|しひ}をも
\ruby{犠牲}{いけ|にへ}にして、
%
\ruby[||j>]{廣}{くわう}
\ruby[||j>]{大}{ だい}の
% \ruby{廣大}{くわう|だい}の
\ruby{御慈悲}{おん|じ|ひ}を
\ruby{頼}{たの}み
\ruby[<j>]{奉}{たてまつ}らんと
\ruby{思}{おも}ふ
\ruby{此}{こ}の
\ruby{人間}{ひ|と}の
\ruby{心}{こゝろ}のみぞ
\原本頁{218-8}\改行%
\ruby{僞}{いつは}り
\ruby{無}{な}き
\ruby{眞實}{まこ|と}なる!。% ここの「まこと」は「眞實」
%
\ruby{二}{ふ}タ
\ruby{路}{みち}かけて
\ruby{取舎}{しゆ|しや}し
わづらひつゝ、
%
\ruby{利}{よき}に
\ruby{就}{つ}かんとする
\ruby{此}{こ}の
\ruby{{\換字{分}}別}{ふん|べつ}の
\ruby{醜}{みにく}さよ、
%
\ruby{智慧}{ち|ゑ}の
\ruby{狡猾}{かし|こ}さよ!。
%
あゝ
\ruby{人間}{ひ|と}は
\ruby{卑劣}{さ|も}しくも
\ruby{怯}{きたな}き
\ruby{心}{こゝろ}を
\ruby{有}{も}てるかな!。
%
されど
\ruby{此}{こ}の
\ruby{疑}{うたが}ひ
\ruby{惑}{まど}ひて
\ruby{苦}{くるし}めるこそは、
%
\ruby{人間}{ひ|と}の
\ruby{僞}{いつは}り
\ruby{無}{な}き
\ruby{眞實}{まこ|と}の% ここの「まこと」は「眞實」
\ruby{{\換字{情}}狀}{さ|ま}
なるべけれ。
%
\ruby{我}{われ}
こゝに
\原本頁{219-1}\改行%
\ruby{在}{あ}り、
%
われ
こゝに
\ruby{思}{おも}ふ。
%
\ruby{思}{おも}はるゝものゝ
\ruby{有}{あ}り
\ruby{無}{な}しは
\ruby{定}{さだ}かならず、
%
\ruby{思}{おも}ふ
\ruby{我}{われ}の
\ruby{在}{あ}る
\ruby{事}{こと}が
\ruby{眞實}{まこ|と}なるのみ。% ここの「まこと」は「眞實」
%
\ruby{菩薩}{ぼ|さつ}の
\ruby{言葉}{こと|ば}、
%
\ruby{鷲}{わし}の
\ruby{言葉}{こと|ば}、
%
\ruby{妙典}{めう|てん}の
\ruby{敎}{をしへ}、
%
\ruby{大蛇}{だい|じや}の
\ruby{敎}{をしへ}、
%
\ruby{我}{われ}に
いづれを
\ruby{取}{と}り
\ruby{那方}{いづ|れ}を
\ruby{捨}{す}つる
\ruby[<j||]{力}{ちから}
\ruby{無}{な}し、
%
たゞ
\ruby{那方}{いづ|れ}をも
\ruby{取}{と}り
\ruby{惱}{なや}み、
%
また
いづれをも
\ruby{捨}{す}て
\ruby{惱}{なや}む
\ruby{其}{その}
\ruby{事}{こと}のみぞ
\ruby{我}{わ}が
\ruby{眞{\換字{情}}}{まこ|と}なる!。% ここの「まこと」は「眞情」
%
\ruby{神明}{か|み}
\ruby{佛陀}{ほと|け}を
\ruby{頼}{たの}み
\ruby[<j>]{奉}{たてまつ}りたき
\ruby{心地}{こゝ|ち}のするも、
\ruby{我}{わ}が
\ruby{欺}{あざむ}かぬ
\原本頁{219-6}\改行%
\ruby{眞{\換字{情}}}{まこ|と}なり、% ここの「まこと」は「眞情」
%
\ruby{神明}{か|み}
\ruby{佛陀}{ほと|け}をも
\ruby{肯}{うけが}はずして、
%
\ruby{智慧}{ち|ゑ}の
\ruby{鋼鐵}{はが|ね}の
\ruby{{\換字{杖}}}{つゑ}に
\ruby{頼}{よ}つて
\ruby{此}{こ}の
\ruby{戰鬪}{たゝ|かひ}の
\ruby{世}{よ}に
\ruby{立}{た}たんとするも
\ruby{我}{わ}が
\ruby{欺}{あざむ}かぬ
\ruby{眞{\換字{情}}}{まこ|と}なり、% ここの「まこと」は「眞情」
%
\ruby{獸}{けもの}にもあらず
\ruby{鳥}{とり}にもあらで、
%
\ruby{光明}{ひか|り}の
\ruby{國}{くに}
\ruby{黑闇}{や|み}の
\ruby{國}{くに}の
\ruby{境}{さかひ}を
\ruby{飛}{と}ぶ
\ruby{彼}{あ}の
\ruby{{\換字{魔}}魅}{ま|もの}の
\ruby{如}{ごと}き
\ruby{蝙蝠}{かは|ほり}の、
%
\ruby{世}{よ}にも
\ruby{厭}{いと}はしく
\ruby{醜}{みにく}きは、
%
\ruby{我}{わ}が
\ruby{胸}{むね}の
\ruby{中}{うち}の
\ruby[||j>]{怪}{くわい}
\ruby[||j>]{物}{ ぶつ}の、
% \ruby{怪物}{くわい|ぶつ}の、
%
\ruby{化}{な}りて
\ruby{出}{い}でしかとも
\ruby{思}{おも}はれて、
%
\ruby{何}{なに}とも
\ruby{云}{い}へぬ
\ruby{忌}{いま}はしき
\ruby{氣}{き}のする!。
%
\原本頁{219-11}\改行%
されど、
%
されど、
%
\ruby{是}{こ}は
\ruby{眞實}{まこ|と}なり、% ここの「まこと」は「眞實」
%
\ruby{我}{われ}は
\ruby{僞}{いつは}らず、
%
\ruby{我}{われ}は
\ruby{矯}{た}めず、
%
\ruby{我}{われ}は
\ruby{{\換字{飾}}}{かざ}らず、
\原本頁{220-1}%% 120 ページになっているが
%
\ruby{恐}{おそ}るゝ
ところ
\ruby{無}{な}し。
%
われ
こゝに
\ruby{思}{おも}ふ!。
%
\ruby{我}{われ}
こゝに
\ruby{在}{あ}り!。
%
\ruby{天}{てん}
\ruby{我}{わ}が
\ruby{戀}{おも}へる
\ruby{人}{ひと}を
\ruby{何}{なに}と
せんとはする\換字{!?}、
%
\ruby{天}{てん}
そも〳〵
\ruby{我}{われ}を
\ruby{何}{なに}と
なれとかする\換字{!?}。

\原本頁{220-4}%% 120 ページになっているが
と
\ruby{淺草}{あさ|くさ}の
\ruby{御堂}{み|だう}に
\ruby{身}{み}を
\ruby{投}{な}げ
\ruby{伏}{ふ}して
\ruby{涙}{なみだ}に
くれし
\ruby[<j>]{曉}{あかつき}には
\ruby{引}{ひき}かへ、
%
\ruby{一{\換字{文}}字口}{いち|もん|じ|ぐち}
\ruby{緊}{きび}しく
\ruby{引締}{ひき|し}めて、
%
\ruby{{\換字{猶}}}{なほ}
\ruby{石人}{せき|じん}の
\ruby{如}{ごと}く
\ruby{突立}{つゝ|た}てる
\ruby{時}{とき}、
%
\ruby{尾竹}{を|たけ}と
\ruby{松之助}{まつ|の|すけ}とは
\ruby{家}{いへ}の
\ruby{中}{うち}より
\ruby{現}{あらは}れ
\ruby{出}{い}でゝ、

\原本頁{220-7}%
『そこに
\ruby{居}{ゐ}らつしやるのは
\ruby{水野}{みづ|の}さんで?。
%
ア、
%
\ruby{御入}{お|はい}んなされば
\ruby{宜}{よろ}しかつたものを。
』

\原本頁{220-9}%
と
\ruby{尾竹}{を|たけ}の
\ruby{云}{い}ふに
\ruby{續}{つゞ}いて
\ruby{松之助}{まつ|の|すけ}は、

\原本頁{220-10}%
『そこに
\ruby{居}{ゐ}たの?。
%
\ruby{僕}{ぼく}は
\ruby{君}{きみ}は
\ruby{何}{なに}か
\ruby{思}{おも}ひ
\ruby{出}{だ}して
\ruby{歸}{かへ}つたのかと
\ruby{思}{おも}つた!。
%
\ruby{水野}{みづ|の}
\ruby{君}{くん}、
%
\ruby{君}{きみ}は
\ruby{變}{へん}な
\ruby{人}{ひと}だネ。
』

\原本頁{221-1}%
と、
%
\ruby{我}{わ}が
\ruby{姊}{あね}の
\ruby{水野}{みづ|の}を
\ruby{{\換字{嫌}}}{きら}へる
\ruby{事}{こと}の
\ruby{如何}{い|か}
ばかり
\ruby{其}{そ}の
\ruby{人}{ひと}を
\ruby{苦}{くるし}め
\ruby{居}{を}るかをも
\ruby{知}{し}らずして
\ruby{云}{い}ふ。

\原本頁{221-3}%
\ruby{尾竹}{を|たけ}は
また
\ruby{直}{たゞち}に
\ruby{引取}{ひつ|と}つて、

\原本頁{221-4}%
『
\ruby{定}{さだ}めし
\ruby{案}{あん}じて
\ruby{居}{ゐ}て
\ruby{下}{くだ}さる
だらうと
いふので、
%
\ruby{今}{いま}
\ruby{御宅}{お|たく}へ
\ruby{一寸}{ちよ|つと}
\ruby{樣子}{やう|す}を
\ruby{申}{まを}しに
\ruby{上}{あが}らうとした
ところで
ござりました。
%
\ruby{熱}{ねつ}が
\ruby{甚}{ひど}く
\ruby{發}{はつ}して
\ruby{譫語}{せん|ご}が
\ruby{{\換字{強}}}{つよ}かつたり
なんぞ
したので、
%
\ruby{傍}{そば}の
\ruby{人}{ひと}は
\ruby{一時}{いち|じ}
\ruby{驚}{おどろ}いたのでしたが、
%
\ruby{別}{べつ}の
\ruby{事}{こと}も
\ruby{無}{な}くつて
まあ
\ruby{濟}{す}みました。
%
\ruby{肺}{はい}も
\ruby{心臓}{しん|ざう}も
\ruby{故障}{こ|しやう}は
\ruby{無}{な}し、
%
まづ
\ruby{今}{いま}の
ところでは
\ruby{怖}{こは}くは
\ruby{無}{な}いです。
%
\換字{志}かし
\ruby{二三日}{に|さん|にち}は
まだ
\ruby{此樣}{こ|ん}な
\ruby{事}{こと}も
ありましやうよ、
%
\ruby{此處}{こ|ゝ}
\ruby{二三日}{に|さん|にち}が
\ruby{峠}{たうげ}ですから。
』

\原本頁{221-11}%
と、
%
いと
\ruby{親切}{しん|せつ}に
\ruby{語}{かた}り
\ruby{聞}{きか}せたり。

\Entry{其三十七}

% メモ 校正終了 2024-04-12 2024-05-28
\原本頁{222-2}%
『
あら
お
\ruby{止}{よし}なさいよ、
%
\ruby{頭髮}{か|み}が
\ruby{壞}{こは}れまさあネ。
%
いやですよ、
%
ほんとに、
%
\ruby{人}{ひと}を
\ruby{馬鹿}{ば|か}にしたツ!。
%
そんな
\ruby{事}{こと}は
\ruby{妾}{わたし}や
\ruby{{\換字{嫌}}}{きら}ひ
ですつてば、
%
\ruby{大}{おほ}きな
\ruby{聲}{こゑ}を
\ruby{出}{だ}しますよ。
%
ほら、
%
ほら
\ruby{御師匠}{おつ|し|よ}さんの
\ruby{下駄}{げ|た}の
\ruby{音}{おと}ぢや
ありませんか。
』

\原本頁{222-6}%
\ruby[||j>]{男}{をとこ}の
\ruby{力}{ちから}の
\ruby{{\換字{緩}}}{ゆる}む
\ruby{間}{ひま}に
\ruby{辛}{から}くも
\ruby{{\換字{逃}}}{のが}れて、
\換字{志}どけ
\ruby{無}{な}く
\ruby{亂}{みだ}れたる
\ruby{衣服}{な|り}の
\ruby{{\換字{前}}}{まへ}を
\ruby{引直}{ひき|なほ}しつ、
%
\ruby{膳}{ぜん}の
\ruby{先}{さき}に
\ruby{{\換字{遠}}}{とほ}く
\ruby{離}{はな}れて
\ruby{坐}{すわ}つたるは、
%
さして
\ruby{美}{うつく}し
といふには
あらねど、
%
\ruby{光}{ひか}り
\ruby{流}{なが}るゝが
\ruby{如}{ごと}き
\ruby{眼}{め}の
\ruby{中}{なか}に
\ruby[||j>]{{\換字{情}}}{なさけ}
\ruby[||j>]{有}{ あ}つて、
%
\ruby{世}{よ}に
いふ
\ruby[||j>]{男}{をとこ}
\ruby[||j>]{好}{ ずき}
のする
\ruby{何處}{ど|こ}と
\ruby{無}{な}く
\ruby{仇}{あだ}つぽき
\ruby{廿歳}{はた|ち}ばかりの
すらりとしたる
\原本頁{222-11}\改行%
\ruby{女}{をんな}にて、
%
\ruby{人{\換字{前}}}{ひと|まへ}は
\ruby{此家}{こ|ゝ}の
\ruby{女主人}{あ|る|じ}の
\ruby{内弟子}{うち|で|し}なり、
%
\ruby[||j>]{娘}{むすめ}
\ruby[||j>]{{\換字{分}}}{ ぶん}なり
% \ruby{娘{\換字{分}}}{むすめ|ぶん}なり
なれど、
%
\原本頁{223-1}\改行%
\ruby{人}{ひと}の
\ruby{見}{み}ぬ
\ruby{時}{とき}は
\ruby{水}{みづ}
\ruby{仕業}{し|わざ}も
\ruby{爲}{さ}せらるゝ、
%
\ruby{寄食者}{かゝ|りう|ど}ともつかず
\ruby{下婢}{はし|た}ともつかぬ
\ruby{怪}{あや}しきものなれば、
%
\ruby{置}{お}く
\ruby{方}{かた}にも
\ruby{置}{お}かるゝ
\ruby{方}{かた}にも、
%
いづれ
\ruby{一寸}{ちよ|つと}したる
\ruby{關係}{あ|や}は
\ruby{潜}{ひそ}める% 【潛 u6f5b 「先先」】【潜 u6f5c 「夫夫」】併用されている
なるべし。
%
\ruby{男}{をとこ}は
\ruby{顏}{かほ}の
\ruby{色}{いろ}
\ruby{黑}{くろ}く
\ruby{{\換字{強}}壯}{ぢやう|ぶ}さうに
\ruby[||j>]{膩}{あぶら}
\ruby[||j>]{光}{ でり}の
% \ruby{膩光}{あぶら|でり}の
したる、
%
\ruby{四十餘歳}{し|じふ|いく|つ}の
\ruby{品格}{ひ|ん}の
\ruby{無}{な}きなるが、
%
\ruby{膳}{ぜん}を
\ruby{{\換字{前}}}{まへ}にして
\ruby{胡坐}{あぐ|ら}
\ruby{組}{く}めり。

\原本頁{223-6}%
\ruby{格子{\換字{戸}}}{かう|し|ど}は
\ruby{輕}{かろ}く
からりと
\ruby{開}{あ}きて、
%
やがて
\ruby{入}{い}り
\ruby{來}{きた}れるは
\ruby{果}{はた}して
\ruby{女主人}{あ|る|じ}なり。
%
\ruby{五十}{ご|じふ}に
\ruby{{\換字{近}}}{ちか}きには
\ruby{疑}{うたが}ひ
\ruby{無}{な}けれど、
%
ぶつてりと
\ruby{肥}{ふと}つたる
\ruby{{\換字{平}}顏}{ひら|がほ}の、
%
\ruby{特}{こと}に
\ruby{今}{いま}は
\ruby{{\換字{浴}}後}{ゆ|あがり}とて
\ruby{照}{て}らつきて
\ruby{赤}{あか}きに、
%
\ruby{絲}{いと}の
\ruby{如}{ごと}く
\ruby{剃}{す}りつけたる
\ruby{眉}{まゆ}の
\ruby{{\換字{嫌}}味}{いや|み}たらしく
\ruby{細}{ほそ}く、
%
\ruby{髮際}{はえ|ぎは}
\ruby{異樣}{こと|やう}に
\ruby{濃}{こ}き
\ruby{髮}{かみ}を、
\換字{志}たゝかに
\原本頁{223-10}\改行%
\ruby{油}{あぶら}つけて
\ruby{銀杏{\換字{返}}}{い|てふ|がへ}しに
\ruby{結}{ゆ}ひたる、
%
みづからは
\ruby{未}{ま}だ
\ruby{老}{お}い
\ruby{{\換字{込}}}{こ}まぬ
\ruby{意氣}{い|き}を
\ruby{示}{しめ}したる
なるべけれど、
%
\ruby{人}{ひと}は
\ruby{見}{み}るより
\ruby{恐}{おそ}れて
\ruby{{\換字{逃}}}{にげ}
\ruby{走}{はし}るべき
\ruby{態}{さま}な
\原本頁{224-1}\改行%
り。

\原本頁{224-2}%
\ruby{女主人}{あ|る|じ}は
\ruby[||j>]{糠}{ぬか}
\ruby[||j>]{袋}{ぶくろ}の
% \ruby{糠袋}{ぬか|ぶくろ}の
\ruby{絲}{いと}を
\ruby{口}{くち}に
しつゝ、
%
\ruby{手拭}{て|ぬぐひ}を
ばたりと
\ruby{一度}{ひと|たび}
\ruby{鳴}{な}らして、
\原本頁{224-3}\改行%
\GWI{u1b048-u3099}ろりと% 「志」+「濁点」
\ruby{白}{しら}けたる
\ruby{此場}{この|ば}の% 原文通り「場」
\ruby{狀}{さま}を
\ruby{見}{み}れば、
%
\ruby{男}{をとこ}は
\ruby{何}{なに}
\ruby{喰}{く}はぬ
\ruby{顏}{かほ}して
\ruby{酒}{さけ}
\ruby{無}{な}き
\ruby{猪口}{ちよ|く}を
\ruby{吸}{す}ひ、
%
\ruby{女}{をんな}は
\ruby{徳利}{とく|り}に
\ruby{手}{て}は
\ruby{觸}{ふ}れ
ながら
\ruby{{\換字{酌}}}{しやく}を
せんとも
\ruby{爲}{せ}で
\ruby{護}{まも}り
\ruby{居}{ゐ}たる
\ruby{其}{そ}の
\ruby{呼吸}{い|き}は
\ruby{{\換字{猶}}}{なほ}
はづみて
\ruby{事實}{ま|こと}を
\ruby{語}{かた}れり。

\原本頁{224-6}%
\ruby{十{\換字{分}}}{じふ|ぶん}に
\ruby{男}{をとこ}の
\ruby{何}{なに}と
\ruby{爲}{し}たりしかを
\ruby{猜}{すゐ}したる
\ruby{女主人}{あ|る|じ}の
\ruby{顏}{かほ}は、
%
\ruby{見}{み}る〳〵
\ruby{紫色}{むら|さき}に
\ruby{脹}{は}れたるが
\ruby{如}{ごと}くなりて、

\原本頁{224-8}%
『
\ruby{何}{なに}を
\ruby{仕}{し}て
おいでだつたエ、
%
\ruby{貴郞}{おま|へ}さんは。
』

\原本頁{224-9}%
と、
%
\ruby{先}{ま}づ
\ruby{一句}{いつ|く}
\ruby[||j>]{男}{をとこ}の
\ruby{顏}{かほ}を
\ruby{見}{み}て
\ruby{詰}{なじ}りしが、

\原本頁{224-10}%
『
\ruby{先}{さき}へ
\ruby{始}{はじ}めたなあ
\ruby{惡}{わる}かつたが、
%
\ruby{飮}{や}つてた
ばかりだわナ、
%
\ruby{堪{\換字{忍}}}{か|に}しねえナ。% 原文通り「堪忍」
』

\原本頁{225-1}%
と、
%
\ruby{男}{をとこ}も
さるもの、
%
\ruby{穩}{おだ}やかに
\ruby{澱}{よど}まず
\ruby{云}{い}ひ
\ruby{流}{なが}すを
\ruby{聞}{き}きて、
%
いよいよ% ルビ調整(原本通り)非踊り字表記(行末行頭の境界付近)
\ruby{眼}{まなこ}を
\ruby{嶮}{けは}\換字{志}くし、

\原本頁{225-3}%
『
\ruby{左樣}{さ|う}かい!。
%
そりやあ
\ruby{堪{\換字{忍}}}{か|に}するも% 原文通り「堪忍」
\ruby{何}{なに}も
ありやあ
\ruby{仕}{し}ない。
』

\原本頁{225-4}%
と
\ruby{冷}{ひや}やかに
\ruby{云}{い}ひ
\ruby{切}{き}りつ、
%
\ruby{間}{あひだ}を
\ruby{隔}{お}きて、

\原本頁{225-5}%
『
だつて
\ruby{盗賊}{どろ|ばう}
\ruby{猫}{ねこ}が
\ruby{暴}{あば}れた
やうだからサ。
%
\ruby{{\換字{留}}守番}{る|す|ばん}
\ruby{甲{\換字{斐}}}{が|ひ}が
\ruby{無}{な}いと
\ruby{思}{おも}つて
\ruby{聞}{き}いたんだよ。
%
お
\ruby{龍}{りゆう}、
%
お
\ruby{{\換字{前}}}{まへ}、
%
\ruby{氣}{き}を
つけ
\ruby{無}{な}くつちやあ
いけないよ。

\原本頁{225-8}%
ほんとに
\ruby{碌}{ろく}で
\ruby{無}{な}しの
\ruby{盗賊}{どろ|ばう}
\ruby{猫}{ねこ}が
\ruby{居}{ゐ}るんだからネ。
%
\ruby{恐}{おそ}ろしい
\ruby{圖々}{づう|〴〵}しい
\ruby{奴}{やつ}なんだからネ。
%
\ruby{油斷}{ゆ|だん}も
\ruby{隙}{すき}も
なりや
\ruby{仕}{し}ない。
%
\ruby{捕}{つかま}へたら
\ruby{鼻}{はな}づらを
\ruby{引擦}{ひつ|こす}つて
\ruby{{\換字{遣}}}{や}りたいぢや
\ruby{無}{な}いか。
』

\原本頁{225-11}%
と、
%
\ruby{云}{い}ひながら
\ruby{男}{をとこ}の
\ruby{對面}{むか|ふ}へ、
%
むずと
\ruby{坐}{すわ}つたり。

\原本頁{226-1}%
\ruby{男}{をとこ}は
\ruby{困}{こう}じたる
\ruby{顏}{かほ}に
\ruby[||j>]{苦}{にが}
\ruby[||j>]{笑}{わらひ}して
% \ruby{苦笑}{にが|わらひ}して
\ruby{横}{よこ}を
\ruby{向}{む}けり。

\Entry{其三十八}

\ruby{見馴}{み|な}れ
\ruby{聞}{き}き
\ruby{馴}{な}れたるにさまでは
\ruby{感}{かん}ぜねど、
\ruby{何}{なん}と
\ruby{挨拶}{あい|さつ}すべき
\ruby{言葉}{こと|ば}を
\ruby{知}{し}らねば、
お
\ruby{龍}{りう}は
\ruby{手拭糠袋}{てぬ|ぐひ|ぬか|ぶくろ}を
\ruby{手渡}{て|わた}しされたるを
\ruby{機}{き}に、
\ruby{其}{そ}を
\ruby{臺{\換字{所}}{\換字{近}}}{だい|どころ|ちか}き
\ruby{掛竿}{かけ|ざを}に
\ruby{叮嚀}{てい|ねい}に
\ruby{懸}{か}けて、わざと
\ruby{暇取}{ひま|と}りて
\ruby{此方}{こな|た}へ
\ruby{來}{く}れば、
\ruby{膳}{ぜん}の
\ruby{上}{うへ}に
\ruby{伏}{ふ}せありたる
\ruby{我}{わ}が
\ruby{猪口}{ちよ|く}を、
\ruby{不興氣}{ふ|きよう|げ}に
\ruby{取}{と}り
\ruby{上}{あ}げたる
\ruby{主人}{ある|じ}に
\ruby{向}{むか}ひて、
\ruby{男}{をとこ}は
\ruby{自}{みづか}ら
\ruby{徳利}{とく|り}を
\ruby{手}{て}にして、
\ruby{諂}{へつら}ひ
\ruby{笑}{わらひ}を
\ruby{面}{おもて}に
\ruby{{\換字{浮}}}{うか}べつゝ、
\ruby{今}{いま}や
\ruby{酌}{しやく}して
\ruby{{\換字{遣}}}{や}らんとしたる
\ruby{其}{そ}の
\ruby{狀態}{あり|さま}の、たとへば
\ruby{女主人}{あ|る|じ}は
\ruby{怒}{いか}つたる
\ruby{蝦蟇}{ひき|がへる}の
\ruby{如}{ごと}く、
\ruby{男}{をとこ}は
\ruby{{\換字{又}}}{また}
\ruby{地}{ち}に
\ruby{下}{お}りたる
\ruby{狡猾}{わる|がしこ}き
\ruby{烏}{からす}の
\ruby{如}{ごと}くなるに、
\ruby{思}{おも}はずも
\ruby{安芝居}{やす|しば|ゐ}の
\ruby{安役者}{やす|やく|しや}が
\ruby{出}{だ}せる
\ruby{世話物}{せ|わ|もの}の、
\ruby{下卑}{げ|び}たる
\ruby{一}{ひ}ト
\ruby{場}{ば}を
\ruby{見}{み}る
\ruby{心地}{こゝ|ち}して、おのれもまた
\ruby{其}{そ}の
\ruby{同}{おな}じ
\ruby{此}{こ}の
\ruby{舞臺}{ぶ|たい}に
\ruby{{\換字{交}}}{まじ}りて
\ruby{一}{ひ}ト
\ruby{役}{やく}を
\ruby{演}{す}ることかと、
\ruby{身}{み}に
\ruby{染}{し}みてつく〴〵と
\ruby{嬉}{うれ}しからず
\ruby{思}{おも}ひしが、
\ruby{漸}{やうや}く
\ruby{二人}{ふた|り}の
\ruby{仲}{なか}の
\ruby{治}{をさ}まり
\ruby{行}{ゆ}かんとするさまなれば、
\ruby{差當}{さし|あた}り
\ruby{先}{ま}づ
\ruby{其事}{そ|れ}を
\ruby{{\換字{悅}}}{よろ}びて
\ruby{坐}{ざ}に
\ruby{戾}{もど}り、
\ruby{膳}{ぜん}の
\ruby{上}{うへ}の
\ruby{聊}{いさゝ}か
\ruby{淋}{さび}しきを
\ruby{見}{み}て、

『お
\ruby{師匠}{し|よ}さん、あの
\ruby{傳}{でん}さんの
\ruby{下}{くだ}すつたものを
\ruby{開}{あ}けましやうか。
』

と、
\ruby{機{\換字{嫌}}取}{き|げん|と}り
\ruby{顏}{がほ}に
\ruby{優}{やさし}しく
\ruby{云}{い}へば、
\ruby{主人}{ある|じ}も
\ruby{此女}{こ|れ}に
\ruby{對}{むか}つては
\ruby{言葉}{こと|ば}を
\ruby{和}{やは}らげつ。

『アヽ、たしか
\ruby{雀燒}{すゞめ|やき}だつたネ、ぢやあ
\ruby{開}{あ}けておくれ!。
オヤありやあ
\ruby{汝}{おまへ}につて
\ruby{彼人}{あの|ひと}が
\ruby{{\換字{呉}}}{く}れたんだつたのに。
』

『あらいやな、そんな
\ruby{事}{こと}を!。
どうだつて
\ruby{好}{い}いぢやあありませんか。
』

『
\ruby{左樣}{さ|う}かい。
ぢやあ、まあ、
\ruby{貰}{もら}ふよ。
\ruby{面倒}{めん|ど}くさいから
\ruby{取}{と}り
\ruby{{\換字{分}}}{わ}けずともだよ。
あゝ
\ruby{左樣}{さ|う}さ、
\ruby{其儘}{その|まゝ}で
\ruby{好}{い}いやネ、
\ruby{構}{かま}やあしないよ。
』

\ruby{大}{おほき}からぬ
\ruby{杉折}{すぎ|おり}は
\ruby{膳}{ぜん}の
\ruby{傍}{かたはら}に
\ruby{出}{いだ}されたり。

『オヤ
\ruby{此}{これ}あ
\ruby{千住}{せん|じゆ}のだよ、\換字{志}かも
\ruby{鮒}{ふな}だ、
\ruby{自轉車天狗}{じ|てん|しや|てん|ぐ}が
\ruby{物}{もの}を
\ruby{{\換字{呉}}}{く}れると、いつでも
\ruby{奇妙}{き|めう}に
\ruby{{\換字{遠}}}{とほ}い
\ruby{{\換字{所}}}{ところ}のものばかりだから
\ruby{可笑}{を|か}しいのさ、
\ruby{帝釋}{たい|しやく}さまの
お
\ruby{水}{みづ}を
\ruby{何}{なん}でも
\ruby{無}{な}い
\ruby{日}{ひ}に
\ruby{持}{も}つて
\ruby{來}{き}て
\ruby{{\換字{呉}}}{く}れたりなんぞするのは、
\ruby{自轉車乘}{じ|てん|しや|の}りで
\ruby{無}{な}くつちやあ
\ruby{出來}{で|き}ない
\ruby{事}{こと}だよ。
ン、
\ruby{中々}{なか|〳〵}おいしいよ、
\ruby{汝}{おまへ}も
お
\ruby{食}{あが}りな、
\ruby{一杯}{ひと|つ}あげやう。
』

『イヽエ
\ruby{妾}{わたし}は。
』

『ハヽヽ、ちつとも
\ruby{飮}{や}らないだけは、ほんとに
\ruby{汝}{おまへ}にも
\ruby{似合}{に|あ}はないよ。
だけれど、
\ruby{其行狀}{そ||れ}で
\ruby{飮}{や}られちやあ
\ruby{大變}{たい|へん}だからネ、
\ruby{其}{それ}も
\ruby{可}{い}いかも
\ruby{知}{し}れないよ。
』

『あらまあ
\ruby{甚}{ひど}い
\ruby{事}{こと}を。
』

『だつてお
\ruby{酒}{さけ}まで
\ruby{好}{すき}だつた
\ruby{日}{ひ}にやあ
\ruby{何樣}{ど|う}したつて
お
\ruby{前}{まへ}は、
\ruby{紀伊国屋}{き|の|くに|や}が
\ruby{演}{し}さうな
\ruby{肌}{はだ}の
\ruby{女}{をんな}になるからねえ!。
\ruby{折角妾}{せつ|かく|わたし}の
\ruby{名跡}{あ|と}を
\ruby{取}{と}つて
\ruby{貰}{もら}はうと
\ruby{思}{おも}つて
\ruby{居}{ゐ}たつて、
\ruby{何樣}{ど|ん}な
\ruby{場}{ば}を
お
\ruby{前}{まへ}が
\ruby{出}{だ}して
\ruby{仕舞}{し|ま}ふか
\ruby{知}{し}れやしないもの!。
』

『いやですよ、お
\ruby{師匠}{し|よ}さん、そんな
\ruby{事}{こと}を
\ruby{云}{い}つちやあ、
\ruby{妾}{わたし}はもう
\ruby{澤山}{たん|と}
\ruby{凝}{こ}りて
\ruby{居}{ゐ}るんですもの、いつまでもおとなしく
\ruby{仕}{し}て
\ruby{居}{ゐ}て
\ruby{一生獨身}{いつ|しやう|どく|しん}で、
お
\ruby{師匠}{し|よ}さんの
\ruby{傍}{そば}にばかり
\ruby{居}{ゐ}るつもりなんですから。
』

『
\ruby{嬉}{うれ}しいねえ。
お
\ruby{前}{まへ}が
\ruby{左樣}{さ|う}いふ
\ruby{氣}{き}で
\ruby{居}{ゐ}て
\ruby{{\換字{呉}}}{く}れりやあ
\ruby{妾}{わたし}あ
\ruby{此上無}{この|うへ|な}しさ。
いよ〳〵
\ruby{左樣}{さ|う}なら
\ruby{妾}{わたし}の
\ruby{事}{こと}をネ、これから
お
\ruby{母}{つか}さん
お
\ruby{母}{つか}さんと
\ruby{呼}{よ}んでも
\ruby{可}{い}いよ。
\ruby{妾}{わたし}の
\ruby{方}{はう}ぢやあ
\ruby{疾}{とう}から
\ruby{既實}{もう|じつ}の
\ruby{娘}{こ}のやうに
\ruby{思}{おも}つて
\ruby{居}{ゐ}るんだから。
』

『お
\ruby{師匠}{し|よ}さん、そりやあ
\ruby{本當}{ほん|たう}なの、きつと
\ruby{本當}{ほん|たう}なの?。
お
\ruby{母}{つか}さんと
\ruby{云}{い}つても
\ruby{惡}{わる}かあ
\ruby{無}{な}くつて?。
』

『あゝ
\ruby{可}{いゝ}ともさ。
\ruby{妾}{わたし}あ
\ruby{何樣}{ど|ん}なに
\ruby{嬉}{うれ}しいか
\ruby{知}{し}れやしないよ。
』

\ruby{男}{をとこ}は
\ruby{此時}{この|とき}まで
\ruby{手持無}{て|もち|な}くて、
\ruby{二人}{ふた|り}が
\ruby{對話}{はな|し}を
\ruby{聞}{き}き
\ruby{居}{ゐ}たりしが、こゝにむぐ〳〵と
\ruby{口}{くち}を
\ruby{動}{うご}かして、

『お
\ruby{母}{つか}さんにしちやあ
\ruby{變}{へん}に
\ruby{若}{わか}いナ。
』

と、
\ruby{阿諛}{あ|ゆ}に
\ruby{似}{に}たる
\ruby{語}{ご}を
\ruby{挿}{さしはさ}めば、
\ruby{女主人}{あ|る|じ}は
\ruby{忽}{たちま}ち、

『
\ruby{何}{なん}だとエ、
\ruby{餘計}{よ|けい}な
\ruby{御世話}{お|せ|わ}だよ。
\ruby{黙}{だま}つておいで!。
』

と、たしなめは
\ruby{仕}{し}たけれど
\ruby{腹}{はら}は
\ruby{立}{た}てぬ
\ruby{顏}{かほ}なり。

『
\ruby{妾}{わたし}もネ、
お
\ruby{前}{まへ}は
\ruby{知}{し}るまいが
\ruby{子}{こ}はあるけれども、 --- もつとも
\ruby{義理}{ぎ|り}だけで
\ruby{根}{ね}は
\ruby{他人}{た|にん}なのさ、だもんだから
お
\ruby{前}{まへ}、
\ruby{妾}{わたし}を
\ruby{馬鹿}{ば|か}にして、
\ruby{一人}{ひと|り}は
\ruby{女}{をんな}の
\ruby{癖}{くせ}に
\ruby{生意氣}{なま|い|き}に
\ruby{敎員}{けう|ゐん}なんぞになりやがつて、
\ruby{{\換字{近}}在}{きん|ざい}に
\ruby{一人}{ひと|り}で
\ruby{暮}{くら}して
\ruby{居}{ゐ}るし、
\ruby{其弟}{その|おとうと}は
\ruby{書生}{しよ|せい}を
\ruby{仕}{し}て
\ruby{居}{ゐ}るが、
\ruby{二人}{ふ|たり}とも
\ruby{妾}{わたし}を
\ruby{馬鹿}{ば|か}に
\ruby{仕}{し}きつて
\ruby{居}{ゐ}て、
\ruby{此家}{こ|ゝ}なんぞへは
\ruby{寄}{よ}りつきも
\ruby{仕}{し}ないんだが、ほんとにまあ
\ruby{何樣}{ど|ん}なに
\ruby{高慢}{かう|まん}な
\ruby{憎}{にく}らしい
\ruby{奴等}{やつ|ら}だらう!。
だから
\ruby{妾}{わたし}も
\ruby{其等}{そい|ら}を
\ruby{子}{こ}だとは
\ruby{思}{おも}つて
\ruby{居}{ゐ}やしないのさ。
\ruby{同}{おな}じ
\ruby{他人}{た|にん}なら
\ruby{妾}{わたし}は
お
\ruby{前}{まへ}を、ほんたうに
\ruby{妾}{わたし}の
\ruby{娘}{むすめ}にして、
\ruby{何樣}{ど|ん}なにでも
\ruby{好}{よ}くして
\ruby{{\換字{遣}}}{や}りたいよ。
なあに
\ruby{何}{なん}にも
\ruby{有}{あ}りや
\ruby{仕}{し}ないけれど、それでも
お
\ruby{前}{まへ}、
\ruby{妾}{わたし}は
\ruby{妾}{わたし}
\ruby{一人}{ひと|り}でもつて、どうやら
\ruby{斯樣}{こ|う}やら
\ruby{{\換字{遣}}}{や}つて
\ruby{來}{き}て
\ruby{居}{ゐ}るんだからネ、それだけの
\ruby{事}{こと}は
お
\ruby{前}{まへ}に
\ruby{譲}{ゆづ}るつもりなのさ。
エ、
\ruby{其}{そ}の
\ruby{娘}{むすめ}かエ、
\ruby{五十}{い|そ}と
\ruby{云}{い}つてネ、
\ruby{容貌}{きり|やう}も
\ruby{惡}{わる}かあ
\ruby{無}{な}いが、
\ruby{愛}{あい}の
\ruby{無}{な}い、
\ruby{矢張}{やつ|ぱ}りあの
\ruby{妾}{わたし}の
\ruby{大{\換字{嫌}}}{だい|きら}ひな
\ruby{海老茶}{え|び|ちや}の
\ruby{袋}{ふくろ}を
\ruby{穿}{は}いてる
\ruby{奴}{やつ}なのさ。
\ruby{男}{をとこ}の
\ruby{子}{こ}は
\ruby{松之助}{まつ|の|すけ}といつて、
\ruby{直}{ぢき}そこの
\ruby{下谷}{した|や}に
\ruby{居}{ゐ}るのだがネ、
\ruby{此}{こ}の
\ruby{方}{はう}はまだしも
\ruby{素直}{す|なほ}な
\ruby{性質}{た|ち}だから
\ruby{手}{て}なづけては
\ruby{居}{ゐ}るけれど、やつぱし
\ruby{姊}{あね}びいきだから
\ruby{妾}{わたし}の
\ruby{爲}{ため}にやあ、
\ruby{末始{\換字{終}}}{すゑ|し|ゞう}は
\ruby{爲}{な}りさうもない
\ruby{奴}{やつ}なのさ。
\ruby{此樣}{こ|ふ}いふ
\ruby{譯}{わけ}なんだから、
お
\ruby{前次第}{まへ|し|だい}で、ほんとに
お
\ruby{前}{まへ}が
\ruby{妾}{わたし}の
\ruby{後}{あと}を
\ruby{取}{と}る
\ruby{氣}{き}になつて
お
\ruby{{\換字{呉}}}{く}れなら、どんなにでも
\ruby{妾}{わたし}は
お
\ruby{前}{まへ}に
\ruby{肩}{かた}を
\ruby{入}{い}れるよ。
\ruby{其代}{その|かは}り
お
\ruby{前}{まへ}\換字{志}つかりしてネ、よその
\ruby{下}{くだ}らない
\ruby{猫}{ねこ}なんぞに
\ruby{手}{て}をかけられたりなんぞ
\ruby{仕}{し}ないやうに
\ruby{仕}{し}て
お
\ruby{{\換字{呉}}}{く}れで
\ruby{無}{な}くちやいけないよ。
ハヽヽ。
おや、
\ruby{暗}{くら}くなつて
\ruby{來}{き}たネ、
\ruby{洋燈}{らん|ぷ}さへ
\ruby{準備}{し|たく}が
\ruby{仕}{し}てあるなら
\ruby{構}{かま}はないから、
お
\ruby{湯}{ゆ}へ
\ruby{行}{い}つておいでな。
\ruby{妾}{わたし}あ
お
\ruby{前}{まへ}が
\ruby{美麗}{き|れい}だつて
\ruby{云}{い}はれると
\ruby{眞實}{ほ|んと}に
\ruby{天狗}{てん|ぐ}なんだから、いくらでも
\ruby{悠々磨}{ゆつ|くり|みが}いておいで!。
』


\Entry{其三十九}

\原本頁{}%
『あら
\ruby{虛言}{う|そ}ばつかり!。
%
いくら
\ruby{磨}{みが}いたつて、
%
どうせ
\ruby{美麗}{き|れい}になんか
\ruby{成}{な}りやあ
\ruby{仕}{し}ませんよ。
』

\原本頁{}%
とは
\ruby{云}{い}ひたれど
\ruby{師匠}{し|ゝやう}が
\ruby{言葉}{こと|ば}に
\ruby{悅}{よろこ}べるさまは、
%
\ruby{掩}{おほ}はんとして
\ruby{掩}{おほ}ひきれず、
%
\ruby{愛嬌}{あい|けう}
\ruby{溢}{こぼ}るゝ
\ruby{眼}{め}のしほに
\ruby{見}{み}えたり。
%
\ruby{女主人}{あ|る|じ}はこれを
\ruby{見}{み}て
\ruby{取}{と}りて、
%
\ruby{此}{これ}もおなじく
\ruby{笑顏}{ゑ|がほ}つくり、

\原本頁{}%
『ナニ
\ruby{妾}{わたし}が
お
\ruby{茶々羅}{ちや|〳〵|ら}を
\ruby{云}{い}ふもんかネ。
%
\ruby{傳}{でん}さんだつて
\ruby{淸}{せい}さんだつて
\ruby{{\換字{勝}}}{かつ}さんだつて、
%
みんな
お
\ruby{{\換字{前}}}{まへ}が
\ruby{美麗}{き|れい}だもんだから
\ruby{大騷}{おほ|さわ}ぎ
\ruby{{\換字{遣}}}{や}つてるんだあネ。
%
\ruby{虛言}{う|そ}だと
\ruby{思}{おも}ふなら
\ruby{聞}{き}いて
\ruby{御覧}{ご|らん}!。
』

\原本頁{}%
と、
%
\ruby{重}{かさ}ねて
\ruby{復}{また}も
\ruby{悅}{よろこ}ばせにかゝれば、

\原本頁{}%
『あら、
%
あんまりだわ
\ruby{御師匠}{お|し|よ}さん!。
%
たんと
\ruby{御嬲}{お|なぶ}りなさいよ、
%
ようござんすわ。
』

\原本頁{}%
と、
%
\ruby{此度}{こ|たび}はつんとして
\ruby{横}{よこ}を
\ruby{向}{む}きしが、
%
\ruby{媚}{なまめ}きながら
\ruby{微瞋}{やゝ|いか}れる
\ruby{顏}{かほ}は、
%
\ruby{女主人}{あ|る|じ}が
\ruby{言葉}{こと|ば}もいつはりならず
\ruby{艶}{えん}なり。% 原本通り「えん」

\原本頁{}%
やゝありて
\ruby{思}{おも}ひ
\ruby{出}{だ}したるやうに、

\原本頁{}%
『
\ruby{少}{すこ}し
\ruby{早}{はや}くつても
\ruby[g]{洋燈}{らんぷ}を
\ruby{點}{つ}けましやう。
』

\原本頁{}%
と、
%
\ruby{云}{い}ひさまに
\ruby{立}{た}つて
お
\ruby{龍}{りゆう}は
\ruby{去}{さ}りつ、
%
\ruby{何}{なに}をなせるにや
\ruby{少時}{しば|らく}
\ruby{其姿}{その|すがた}を
\ruby{見}{み}せざりしが、
%
\ruby{火}{ひ}を
\ruby{點}{てん}じたる
\ruby{釣}{つり}
\ruby[g]{洋燈}{らんぷ}を
\ruby{持}{も}ち
\ruby{來}{きた}りて、
%
\ruby{座敷}{ざ|しき}の
\ruby{中央}{ま|なか}に
\ruby{高}{たか}く
\ruby{吊}{つ}りし
\ruby{時}{とき}には、
%
\ruby{今}{いま}までのほつれかゝりたる
\ruby{髷}{まげ}のあとかたも
\ruby{無}{な}く、
%
\ruby{其}{そ}の
\ruby{頭髮}{か|み}は
\ruby{早}{はや}くも
\ruby{結}{ゆ}ひかへられて、
%
さつぱりとしたる
\ruby{束髮}{そく|はつ}の
\ruby{美}{うつく}しきが、
%
\ruby{燈}{ひ}の
\ruby{光}{ひかり}に
\ruby{鮮}{あざ}やかに
\ruby{映}{うつ}し
\ruby{出}{いだ}されたり。

\原本頁{}%
『オヤ
\ruby{早變}{はや|がは}りだネエ、
%
\ruby{吃驚}{びつ|くり}させられたよ。
%
チヨイと
\ruby{彼方}{あつ|ち}を
\ruby{向}{む}いて
\ruby{御見}{お|み}せナ、
%
ヘーエそれが
\ruby{花月卷}{か|げつ|まき}とやらかエ?。
』

\原本頁{}%
『ハア、
%
\ruby{左樣}{さ|う}ですの。
%
\ruby{似合}{に|あ}はなくつて?。
』

\原本頁{}%
『イヽエ
\ruby{似合}{に|あ}はないどころぢあ
\ruby{無}{な}いよ、
%
これは
\ruby{此}{これ}でもつて、
%
いつそ
\ruby{{\換字{又}}}{また}
\ruby{好}{い}いよ。
%
お
\ruby{{\換字{前}}}{まへ}は
\ruby{徳}{とく}な
\ruby{顏立}{かほ|だち}で、
%
\ruby{何}{なん}に
\ruby{結}{ゆ}つても
\ruby{似合}{に|あ}ふのが
\ruby{妙}{めう}だネ。
%
だが
\ruby{束髮}{そく|はつ}も
\ruby{此頃}{この|ごろ}は
\ruby{考}{かんが}へたネ、
%
\ruby{一}{ひ}ト\換字{志}きり
\ruby{人}{ひと}が
\ruby{爲}{し}た
\ruby{蝸牛}{まひ〳〵|つぶろ}の
\ruby{親方見}{おや|かた|み}たやうなのなんざあ、
%
\ruby{堪}{たま}らなく
\ruby{可厭}{い|や}なもんだつたがねえ、
%
ハヽヽ。
』

\原本頁{}%
『ホヽヽ、
%
\ruby{御師匠}{お|し|よ}さんの
\ruby{口}{くち}には
\ruby{叶}{かな}いませんわ。
%
ぢやあ
\ruby{一寸}{ちよ|つと}
\ruby{御湯}{お|ゆう}へ。
』

\原本頁{}%
『あゝ
\ruby{可}{い}いとも!。
%
さあ〳〵
\ruby{髮}{かみ}も
\ruby{出來}{で|き}たし、
%
\ruby{行}{い}つておいで、
%
\ruby{行}{い}つておいで!。
』

\原本頁{}%
『ぢやあ
\ruby{一寸}{ちよ|いと}。
』

\原本頁{}%
\ruby{云}{い}ひながら
\ruby{會釋}{ゑ|しやく}して
\ruby{身}{み}を
\ruby{起}{おこ}し、
%
やがて
\ruby{徐}{しづか}に
\ruby{出}{で}て
\ruby{行}{ゆ}きけるが
\ruby{輕}{かろ}らかなる
\ruby{下駄}{げ|た}の
\ruby{音}{おと}は
\ruby{幾程}{いく|ほど}も
\ruby{無}{な}く
\ruby{{\換字{消}}}{き}えぬ。

\原本頁{}%
『
\ruby{大{\換字{分}}}{だい|ぶ}
\ruby{念入}{ねん|い}りにあやなすぢやあ
\ruby{無}{ね}えか。
』

\原本頁{}%
\ruby{男}{をとこ}は
\ruby{女主人}{あ|る|じ}が
お
\ruby{龍}{りゆう}に
\ruby{對}{たい}する
\ruby{擧動}{ふる|まひ}を
\ruby{怪}{あや}しむやうに
\ruby{云}{い}へば、
%
やゝ
\ruby{醉}{ゑ}ひたる% 「醉」は原本通り「ゑ」で調整
\ruby{女主人}{あ|る|じ}はそれには
\ruby{關}{かま}はず、
%
\ruby{今}{いま}
\ruby{迄}{まで}は
\ruby{他}{ひと}の
\ruby{見}{み}る
\ruby{目}{め}を
\ruby{{\換字{兼}}}{か}ねて
\ruby{堪}{こら}へ
\ruby{居}{ゐ}しが、
%
\ruby{今}{いま}は
\ruby{憚}{はゞか}るところも% 「憚 は(ゞ)か」
\ruby{無}{な}きに、
%
\ruby{突然}{いき|なり}
\ruby{手}{て}あたり
\ruby{任}{まか}せに
\ruby{男}{をとこ}の
\ruby{口}{くち}の
\ruby{端}{はた}をいやといふほど
\ruby{捻}{つね}りて、

\原本頁{}%
『あやなすぢやあ
\ruby{無}{ね}えかも
\ruby{無}{な}いもんだ。
%
\ruby{人}{ひと}の
\ruby{居}{ゐ}ない
\ruby{中}{うち}
\ruby{何}{なに}を
\ruby{爲}{し}やうと
\ruby{仕}{し}たんだエ。
』

\原本頁{}%
と、
%
\ruby{新}{あらた}に
\ruby{罪}{つみ}を
\ruby{糺}{たゞ}さんとする
\ruby{其勢}{その|いきほひ}なか〳〵
\ruby{當}{あた}りがたければ
\ruby{男}{をとこ}はこれに
\ruby{辟易}{へき|えき}して
\ruby{聊}{いさゝ}か
\ruby{身}{み}を
\ruby{{\換字{退}}}{ひ}きぬ。

\原本頁{}%
『ナニたゞ
\ruby{調戲}{から|か}つたばかりだよ、
%
\ruby{戲談}{じやう|だん}だわナ。
』

\原本頁{}%
『フン、
%
\ruby{戲談}{じやう|だん}から
\ruby{駒}{こま}が
\ruby{出無}{で|な}くつて
\ruby{御仕合}{お|し|あはせ}さ。
』

\原本頁{}%
\ruby{長{\換字{煙}}管}{なが|ぎせ|る}は
\ruby{忽}{たちま}ち
\ruby{烈}{はげ}しく
\ruby{膝頭}{ひざ|がしら}を
\ruby{突}{つ}きぬ。
%
\ruby{男}{をとこ}はいよ〳〵
\ruby{後}{あと}じさりするのみ。

\原本頁{}%
『あやまつた〳〵。
%
いゝ
\ruby{加減}{か|げん}にして
\ruby{吳}{く}れ、
%
\ruby{痛}{いて}えやナ。
』

\原本頁{}%
『
\ruby{痛}{いた}くつても
\ruby{關}{かま}ふもんか、
%
\ruby{碌}{ろく}で
\ruby{無}{な}しめ。
』

\原本頁{}%
『あやまつたと
\ruby{云}{い}ふに
\ruby{執念深}{しふ|ねん|ぶか}いなあ。
』

\原本頁{}%
『
\ruby{執念深}{しふ|ねん|ぶか}いなあ
\ruby{妾}{わたし}の
\ruby{性}{しやう}だよ。
%
ほんとに
\ruby{彼女}{あ|れ}なんぞに
\ruby{指}{ゆび}でもさして
\ruby{御覧}{ご|らん}、
%
\ruby{今度}{こん|ど}からたゞ
\ruby{置}{お}きやあ
\ruby{仕無}{し|な}いから。
%
\ruby{彼女}{あ|れ}あ
\ruby{妾}{わたし}が
\ruby{大事}{だい|じ}にかけてるんだもの。
』

\原本頁{}%
『だから
\ruby{彼樣}{あ|ん}なに
\ruby{味}{あぢ}に
\ruby{{\換字{文}}}{あや}なして
\ruby{何樣}{ど|う}するんだと
\ruby{聞}{き}くのだ!。
』

\原本頁{}%
『どうしたつて
\ruby{宜}{い}いよ、
%
\ruby{汝}{おまへ}の
\ruby{御世話}{お|せ|わ}にやあならない。
%
\ruby{妾}{わたし}も
\ruby{取}{と}る
\ruby{年}{とし}だし、
%
\ruby{子}{こ}は
\ruby{無}{な}いし、
%
どうせ
\ruby{汝}{おまへ}はちつとも
\ruby{當}{あて}にやあならないしするから、
%
\ruby{彼女}{あ|れ}に
\ruby{後}{あと}を
\ruby{{\換字{遣}}}{や}つて
\ruby{彼女}{あ|れ}にかゝるんだよ。
』

\原本頁{}%
『フーム、
%
\ruby{{\換字{強}}氣}{がう|ぎ}に
\ruby{彼岸詣}{ひ|がん|まひ}りでも
\ruby{仕}{し}さうな
\ruby{風}{ふう}な
\ruby{事}{こと}をいふナ。
%
そりやあ
\ruby{眞實}{ほん|たう}かエ。
』

\原本頁{}%
『さうさ、
%
ほんたうで
\ruby{無}{な}くつてサ。
』

\原本頁{}%
『ハヽヽ、
%
\ruby{虛言}{う|そ}を
\ruby{云}{い}ひねえナ。
%
\ruby{止}{よ}しねえ〳〵!。
%
\ruby{繼子}{まゝ|こ}だつて
\ruby{何}{なん}だつて
\ruby{二人}{ふた|り}も
\ruby{子}{こ}もあるのに、
%
\ruby{其樣}{そ|ん}な
\ruby{事}{こと}がなんで
\ruby{出來}{で|き}るもんか。
』

\原本頁{}%
『
\ruby{出來無}{で|き|な}いものかネ、
%
\ruby{爲}{す}るんだもの!。
%
\ruby{無理}{む|り}でも
\ruby{左樣}{さ|う}して
\ruby{妾}{わたし}やあ
\ruby{彼女}{あ|れ}にかゝるんだよ。
%
\ruby{相続人}{さう|ぞく|にん}になつてる
\ruby[g]{五十}{いそ}は
\ruby{死}{し}ぬかも
\ruby{知}{し}れないのだから。
』

\原本頁{}%
『ハヽヽ、
%
\ruby{{\換字{強}}氣}{がう|ぎ}に
\ruby{老}{お}い
\ruby{{\換字{込}}}{こ}んだ
\ruby{事}{こと}をいふが、
%
\ruby{乃公}{お|れ}まで
\ruby{食}{く}はせやうと
\ruby{云}{い}ふなあ、
%
ちつと
\ruby{甚}{ひど}い!。
%
どうしてお
\ruby{{\換字{前}}}{めへ}が
\ruby{後}{あと}を
\ruby{案}{あん}じる
\ruby{風}{ふう}かエ。
%
\ruby{汝}{おめへ}は
\ruby{彼女}{あ|れ}をすつかり
\ruby{取}{と}り
\ruby{{\換字{込}}}{こ}んで、\換字{志}やぶつて
\ruby{{\換字{遣}}}{や}らうと
\ruby{云}{い}ふんだらう。
』

\原本頁{}%
『
\ruby{何}{なん}だとエ?。
』

\原本頁{}%
『
\ruby{知}{し}れた
\ruby{事}{こと}さ!。
%
\ruby{食物}{くひ|もの}に
\ruby{仕}{し}やうと
\ruby{云}{い}ふんだらう!。
%
\ruby{何}{なに}も
\ruby{一人}{ひと|り}で
\ruby{占}{し}めずともの
\ruby{事}{こと}だ、
%
\ruby{乃公}{お|れ}にも
\ruby{{\換字{半}}{\換字{分}}}{はん|ぶん}
\ruby{{\換字{遺}}}{よこ}しねえナ。
%
\ruby{圃}{はたけ}でこしらへたものぢやあ
\ruby{有}{あ}るまいし、
%
たゞ
\ruby{穫}{と}つた
\ruby{魚}{さかな}ぢやあ
\ruby{無}{ね}えか、
%
\ruby{吝}{おし}みなさんナ。
%
\ruby{其代}{その|かは}り
\ruby{骨}{ほね}つきの
\ruby{方}{はう}は
\ruby{其方}{そつ|ち}へ
\ruby{{\換字{遣}}}{や}らあ!。
』

\原本頁{}%
『
\ruby{畜生}{ちく|しやう}!、
%
\ruby{惡徒}{あく|とう}め!、
%
えゝ
\ruby{仕方}{し|かた}が
\ruby{無}{な}い!。
%
それぢやあ
\ruby{片身}{かた|み}はあげるからネ、
%
\ruby{要}{い}る
\ruby{時}{とき}に
\ruby{何時}{い|つ}でも
\ruby{庖丁}{はう|ちやう}を
お
\ruby{貸}{か}し!。
』

\Entry{其四十}

\ruby{互}{たがひ}の
\ruby{胸中}{む|ね}に
\ruby{塊物}{も|の}はありながら、
\ruby{相}{あひ}
\ruby{{\換字{酌}}}{じやく}の
\ruby{酒}{さけ}にいつしか
\ruby{解}{と}け
\ruby{合}{あ}つて、
\ruby{男}{をとこ}が
\ruby{{\換字{勤}}}{つと}むる
\ruby{亭主役}{てい|しゆ|やく}、
\ruby{銚子}{てう|し}のかはり
\ruby{目間}{め|ま}を
\ruby{拔}{ぬ}けさせねば、
\ruby{女主人}{あ|る|じ}は
\ruby{湯上}{ゆ|あがり}の
\ruby{早}{はや}くも
\ruby{上機{\換字{嫌}}}{じやう|き|げん}となつて、

『そりやあ
\ruby{幾干}{いく|ら}でも
\ruby{働}{はたら}かうが、
\ruby{一體}{いつ|たい}
\ruby{彼女}{あ|れ}あ
\ruby{何樣}{ど|う}した
\ruby{譯}{わけ}の
\ruby{娘}{こ}なんだ?。
いつ
\ruby{聞}{き}いても
\ruby{些}{ちと}
\ruby{仔細}{し|さい}があつてとばかしで、
\ruby{聞}{き}かされないが。
』

と
\ruby{男}{をとこ}の
\ruby{云}{い}ふを
\ruby{聞}{き}いて
\ruby{舌}{した}なめずりしつ
\ruby{低聲}{こ|ごゑ}に
\ruby{說出}{とき|いだ}したり。

『
\ruby{汝}{おまへ}は
\ruby{成程}{なる|ほど}
\ruby{知}{し}るまいがネ、
\ruby{一昨々年}{さき|を|と|ゝし}の
\ruby{春}{はる}までは
\ruby{彼女}{あ|れ}も
\ruby{矢張}{やつ|ぱ}り、
\ruby{妾}{わたし}のところへ
\ruby{稽{\換字{古}}}{けい|こ}に
\ruby{來}{き}た
\ruby{娘}{こ}さ。
』

『ウン。
』

『
\ruby{内務省}{ない|む|しやう}とかの
\ruby{小吏}{こし|べん}の
\ruby{老人}{おぢい|さん}と、
\ruby{{\換字{父}}子二人}{おや|こ|ふた|り}きりで
\ruby{暮}{くら}して
\ruby{居}{ゐ}たんだが、
お
\ruby{{\換字{父}}}{とつ}さんが
\ruby{日光羊羹}{につ|くわう|やう|かん}
\ruby{見}{み}たやうに
\ruby{變}{へん}に
\ruby{乾固}{ひ|かた}まつた
\ruby{朴實}{こく|めい}な
\ruby{人}{ひと}だつたのには
\ruby{似合}{に|あ}はないで、あの
\ruby{子}{こ}は
\ruby{蓮葉}{はす|は}でも% TODO 暫定で「蓮 uf999」とする(参考「蓮 uu84ee」)
\ruby{無}{な}いが
\ruby{妙}{めう}に
\ruby{{\換字{浮}}氣}{うは|き}つぽい、
お
\ruby{狭}{きやん}な
\ruby{面白}{おも|しろ}いところのある、
\ruby{好}{す}いた
\ruby{男}{をとこ}になら
\ruby{生命}{いの|ち}でも
\ruby{抛}{はふ}り
\ruby{出}{だ}さうツてつたやうな
\ruby{肌合}{はだ|あひ}の
\ruby{娘}{こ}で、
\ruby{同}{おな}い
\ruby{齡}{どし}ぐらゐな
\ruby{娘{\換字{達}}}{こ|たち}が
\ruby{集}{よ}つて
\ruby{談話}{はな|し}を
\ruby{仕}{し}た
\ruby{時}{とき}、
お
\ruby{七}{しち}の
\ruby{爲}{し}た
\ruby{事}{こと}が
\ruby{{\換字{道}}理}{もつ|とも}だといつて
\ruby{一同}{みん|な}に
\ruby{笑}{わら}はれたつて、
\ruby{泣}{な}いて
\ruby{口惜}{く|やし}がつて
\ruby{怒}{おこ}つた
\ruby{事}{こと}がある
\ruby{程}{ほど}なのさ。
そんな
\ruby{調子}{てう|し}だつたもんだから
\ruby{年齡}{と|し}も
\ruby{行}{ゆ}かないのに、これも
\ruby{矢張}{やつ|ぱ}り
\ruby{吾家}{う|ち}へ
\ruby{來}{き}て
\ruby{居}{ゐ}た
\ruby{建具屋}{たて|ぐ|や}の
\ruby{息子}{むす|こ}の
\ruby{源}{げん}といふいなせな
\ruby{男}{をとこ}と
\ruby{人知}{ひと|し}れず
\ruby{出來}{で|き}て
\ruby{仕舞}{し|ま}つたのさ。
』

『フーム、なある
\ruby{程}{ほど}。
お
\ruby{{\換字{前}}}{まへ}が
\ruby{撮合山}{とり|も|ち}を
\ruby{行}{や}つたんだナ。
\ruby{兩方}{りやう|はう}から
\ruby{拜}{おが}まれて
\ruby{錢}{ぜに}を
\ruby{取}{と}つたらう!。
\ruby{惡徒}{あく|とう}ツて
\ruby{云}{い}ふなあ
\ruby{其樣}{さ|う}いふのゝ
\ruby{事}{こと}だぜ。
』

『
\ruby{{\換字{交}}}{ま}ぜるなら
\ruby{後}{あと}を
\ruby{話}{はな}さないよ。
』

『あやまつた、あやまつた、それから。
』

『
\ruby{其}{そ}の
\ruby{中}{うち}に
\ruby{彼}{あ}の
\ruby{娘}{こ}の
お
\ruby{{\換字{父}}}{とつ}さんが
\ruby{病}{わづら}ひついて、
\ruby{老齡}{と|し}だから
\ruby{叶}{かな}はない、
\ruby{死}{ごね}つちまつたんだ。
すると
\ruby{駿府}{すん|ぷ}とかゝら
\ruby{叔母}{を|ば}さんが
\ruby{出}{で}て
\ruby{來}{き}て、あの
\ruby{娘}{こ}を
\ruby{田舎}{ゐな|か}へ
\ruby{{\換字{連}}}{つ}れて
\ruby{行}{い}かうといふのさ。
そら
\ruby{{\換字{情}}夫}{をと|こ}の
\ruby{一件}{いつ|けん}があるから
\ruby{行}{い}きたかあ
\ruby{無}{な}いが、まさか
\ruby{十七八}{じう|しち|はち}だから
\ruby{曝露}{さら|け}け
\ruby{出}{だ}して
\ruby{言}{い}ふことあ
\ruby{出來}{で|き}ず、
\ruby{自{\換字{分}}}{じ|ぶん}の
\ruby{家}{うち}に
\ruby{財產}{しん|だい}は
\ruby{無}{な}し、
\ruby{他}{ほか}に
\ruby{身寄}{み|より}も
\ruby{何}{なに}も
\ruby{無}{な}いから、
\ruby{楯}{たて}にして
\ruby{取}{と}る
\ruby{理屈}{り|くつ}が
\ruby{無}{な}いんで、とう〳〵
\ruby{駿府}{すん|ぷ}へ
\ruby{{\換字{連}}}{つ}れて
\ruby{行}{い}かれたアネ。
』

『だつて
\ruby{其}{それ}ぢやあ
\ruby{其}{そ}の
\ruby{建具屋}{たて|ぐ|や}の
\ruby{倅}{せがれ}が
\ruby{意氣地}{い|く|ぢ}が
\ruby{無}{な}さ
\ruby{{\換字{過}}}{す}ぎるぢやあ
\ruby{無}{ね}えか。
』

『それがお
\ruby{{\換字{前}}}{まへ}、
\ruby{理由}{わ|け}があるからなんさ。
\ruby{其}{それ}あ
\ruby{其}{そ}の
\ruby{源}{げん}といふのにやあ
\ruby{嫁}{よめ}になる
\ruby{筈}{はず}の
\ruby{娘}{こ}が、
\ruby{親類内}{しん|るゐ|うち}に
\ruby{決定}{き|ま}つて
\ruby{居}{ゐ}たんで、つまり
\ruby{源}{げん}の
\ruby{方}{はう}ぢやあ
\ruby{初手}{しよ|て}から
\ruby{當座}{たう|ざ}の
\ruby{花}{はな}にしたんだネ。
だから
\ruby{彼}{あ}の
\ruby{娘}{こ}に
\ruby{捕}{つか}まへられて
\ruby{煮}{に}え
\ruby{詰}{つま}つた
\ruby{話}{はなし}をされる
\ruby{段}{だん}になりやあ、いつでも
\ruby{間}{ま}に
\ruby{合}{あは}せを
\ruby{云}{い}つて
\ruby{巧}{うま}く
\ruby{{\換字{逃}}}{に}げて、とう〳〵
\ruby{{\換字{逃}}}{に}げて〳〵
\ruby{惡}{わる}くも
\ruby{思}{おも}はれずに
\ruby{{\換字{逃}}}{に}げおはせたんだよ。
』

『ヤ、そりやあ
\ruby{源}{げん}といふ
\ruby{奴}{やつ}あ
\ruby{酷}{むご}かつたナ、
お
\ruby{龍}{りゆう}こそ
\ruby{眞實}{ほん|と}に
\ruby{憫然}{かは|いさう}だ。% 「憫然 か(は)いさう」
』

『ひどく
\ruby{御察}{お|さつ}しがいゝネ、
\ruby{何樣}{ど|う}かして
お
\ruby{{\換字{遣}}}{や}りナ。
』

『すぐと
\ruby{左樣}{さ|う}
\ruby{皮肉}{ひ|にく}を
\ruby{云}{い}はずともだ。
ウン、それから。
』

『そこで
\ruby{生木}{なま|き}を
\ruby{引裂}{ひき|さ}かれて
\ruby{駿府}{すん|ぷ}へ
\ruby{{\換字{連}}}{つ}れて
\ruby{行}{い}かれたんだから、
お
\ruby{龍}{りゆう}は
\ruby{矢}{や}も
\ruby{楯}{たて}も
\ruby{堪}{たま}りや
\ruby{仕}{し}ない、
\ruby{雨}{あめ}の
\ruby{降}{ふ}るやうに
\ruby{手紙}{て|がみ}を
\ruby{{\換字{遣}}}{よこ}したのさ。
ところが
\ruby{源}{げん}の
\ruby{方}{はう}が
\ruby{其心}{そ|れ}なんだから
\ruby{{\換字{返}}事}{へん|じ}も
\ruby{{\換字{遣}}}{や}らない。
\ruby{斷念}{あき|らめ}させやうといふんで
\ruby{關}{かま}はずに
\ruby{置}{お}くから、
お
\ruby{龍}{りゆう}は
\ruby{餘程恨}{よつ|ぽど|うら}んだらしい。
それでも
\ruby{此方}{こつ|ち}ぢやあ
\ruby{關}{かま}はずに
\ruby{置}{お}くと、
\ruby{流石}{さす|が}は
\ruby{明治}{めい|じ}ツ
\ruby{子}{こ}だから
\ruby{氣}{き}が
\ruby{{\換字{強}}}{つよ}いネ、
\ruby{源}{げん}の
\ruby{家}{うち}へ
\ruby{押}{お}しかけやうつて
\ruby{云}{い}つて
\ruby{來}{き}たんだよ。
さあ、
\ruby{來}{こ}られちやあ
\ruby{大事}{おほ|ごと}だから
\ruby{源}{げん}は
\ruby{{\換字{弱}}}{よわ}つて、
\ruby{一{\換字{丈}}}{いち|ぢやう}もある
\ruby{手紙}{て|がみ}を
\ruby{三日}{み|つか}もかゝつて
\ruby{書}{か}いて、
\ruby{親々}{おや|〳〵}の
\ruby{壓制}{おし|つけ}で
\ruby{仕方}{し|かた}が
\ruby{無}{な}くつて、
お
\ruby{{\換字{前}}}{まへ}にやあ
\ruby{濟}{す}まないが
\ruby{實}{じつ}は
\ruby{既}{もう}
\ruby{女{\換字{房}}}{によう|ばう}を
\ruby{貰}{もら}つた。
\ruby{腹}{はら}も
\ruby{立}{た}つだらうが
\ruby{何樣}{ど|う}か
\ruby{堪{\換字{忍}}}{か|に}して% 原文通り「堪忍」
\ruby{吳}{く}れ、
\ruby{二人}{ふた|り}の
\ruby{中}{なか}は
\ruby{無}{な}い
\ruby{緣}{えん}と
\ruby{諦}{あきら}めて、
\ruby{汝}{おまへ}も
\ruby{叔母}{を|ば}さん
\ruby{次第}{し|だい}に
\ruby{好}{い}い
\ruby{婿}{むこ}を
\ruby{取}{と}つて
\ruby{榮}{さか}えてくれろ、と
\ruby{哀}{あは}れつぽく
\ruby{巧}{うま}く
\ruby{虛言}{う|そ}をついたネ。
』

『やれ〳〵!。
いよ〳〵
\ruby{酷}{むご}いナア、
\ruby{惡}{わる}い
\ruby{奴}{やつ}だ。
』

『するとお
\ruby{{\換字{前}}}{まへ}、よく〳〵だつたと
\ruby{見}{み}えて、
\ruby{怖}{こは}い
\ruby{話}{はなし}さ!、
\ruby{忘}{わす}れもしない
\ruby{去年}{きよ|ねん}の
\ruby{一月}{いち|ぐわつ}の
\ruby{十三日}{じう|さん|にち}、
\ruby{{\換字{寒}}}{かん}の
\ruby{眞中}{さ|なか}の
\ruby{{\換字{雪}}}{ゆき}のふるのに、
\ruby{安倍川}{あ|べ|かは}とかいふ
\ruby{大}{おほき}な
\ruby{川}{かは}へ
\ruby{飛}{と}び
\ruby{{\換字{込}}}{こ}まうとしたさうさ。
\ruby{幸福}{しあ|はせ}に%「幸福」ここは「は」
\ruby{助}{たす}けられたから
\ruby{可}{い}いやうなものゝ、
\ruby{死}{し}なれりやあ
\ruby{差}{さ}し
\ruby{詰}{づ}め
\ruby{源}{げん}は
\ruby{取}{と}り
\ruby{憑}{つ}かれ
\ruby{無}{な}くちやあならないんだつたのさ。
』

『フム、それから。
』

『まあお
\ruby{待}{ま}ち。
さぞ
\ruby{湯}{ゆ}の
\ruby{中}{なか}で
\ruby{噴嚏}{くし|やみ}を
\ruby{仕}{し}て
\ruby{居}{ゐ}るだらう、
\ruby{憫然}{かは|いさう}に。% 「憫然 か(は)いさう」
ハヽヽ。

\ruby{話}{はな}しながら
\ruby{飮}{や}るんで
\ruby{大層}{たい|そう}
\ruby{發}{はつ}したよ。
\ruby{駿府}{すん|ぷ}へ
\ruby{行}{い}つたのが
\ruby{一昨年}{を|とゝ|し}の
\ruby{夏}{なつ}の
\ruby{末}{すゑ}で、
\ruby{飛}{と}び
\ruby{{\換字{込}}}{こ}んだのが
\ruby{去年}{きよ|ねん}の
\ruby{一月}{いち|ぐわつ}だから、
\ruby{其間}{その|あひだ}の
\ruby{彼}{あ}の
\ruby{女}{こ}の
\ruby{事}{こと}を
\ruby{思}{おも}ふと
\ruby{實}{じつ}は
\ruby{愍然}{ふ|びん}さね。
だが
\ruby{驚}{おどろ}いたのは
\ruby{源}{げん}さ。
\ruby{離}{はな}れて
\ruby{居}{ゐ}る
\ruby{土地}{と|ち}だから
\ruby{助}{たす}かつたのか
\ruby{助}{たす}からなかつたも
\ruby{知}{し}りやうは
\ruby{無}{な}いし、とても
\ruby{生}{い}きて
\ruby{居}{ゐ}ても
\ruby{詰}{つま}らないから
\ruby{死}{し}んで
\ruby{仕舞}{し|ま}ふから
\ruby{憫然}{あは|れ}と
\ruby{思}{おも}つて、
\ruby{一片}{いつ|ぺん}の
\ruby{囘向}{ゑ|かう}でも
\ruby{仕}{し}て
\ruby{吳}{く}れろといふ
\ruby{涙}{なみだ}の
\ruby{痕}{あと}の
\ruby{一}{いつ}ぱいにある
\ruby{不氣味}{ぶ|き|み}な
\ruby{手紙}{て|がみ}を
\ruby{受取}{うけ|と}つたのだから、
\ruby{眞靑}{まつ|さを}になつて
\ruby{慄}{ふる}へて
\ruby{仕舞}{し|ま}つて、いよ〳〵
\ruby{死}{し}んで
\ruby{{\換字{終}}}{しま}つたものなら
\ruby{仕方}{し|かた}が
\ruby{無}{な}い、
\ruby{陰}{かげ}ながら
\ruby{法事}{はふ|じ}でも
\ruby{仕}{し}て
\ruby{祟}{たゝ}りの
\ruby{來}{こ}ないやうに
\ruby{仕}{し}やうと、
\ruby{彼地}{あつ|ち}の
\ruby{新聞}{しん|ぶん}を
\ruby{取}{と}つて
\ruby{調}{しら}べて
\ruby{見}{み}ると、
\ruby{丁度}{ちやう|ど}
\ruby{其}{そ}の
\ruby{手紙}{て|がみ}の
\ruby{日付}{ひ|づけ}の
\ruby{{\換字{翌}}日}{よく|じつ}の
\ruby{新聞}{しん|ぶん}に、
\ruby{美人}{び|じん}の
\ruby{投身}{みな|げ}といふ
\ruby{標題}{み|だし}があつて、
\ruby{彼}{あ}の
\ruby{名}{な}が
\ruby{見}{み}えたから
\ruby{捼然}{ぎよ|つ}としたが、
\ruby{助}{たす}かつて
\ruby{叔母}{を|ば}の
\ruby{家}{うち}へ
\ruby{引渡}{ひき|わた}された、
\ruby{仔細}{し|さい}は
\ruby{解}{わか}らないが
\ruby{發狂}{はつ|きやう}した
\ruby{{\換字{所}}爲}{せ|ゐ}だらう、と
\ruby{書}{か}いてあつたのでホツと
\ruby{氣息}{い|き}を
\ruby{吐}{つ}いたネ。
』

『ン、そこで
\ruby{源}{げん}といふ
\ruby{奴}{やつ}は
\ruby{何樣}{ど|う}したエ。
』

\makeatletter
\@ifundefined{全三巻@一括ビルド}{%
\vspace{4zw}
{\Large{天うつ浪 {\normalsize 第一{\換字{終}}}}}
}
\makeatother


\end{indentation}

\section*{後注}
\theendnotes

\end{document}
