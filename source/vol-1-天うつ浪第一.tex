%! ptex2pdf -l -u           --shell-escape -output-directory=myout vol-1-天うつ浪第一
%! cluttex --engine=uplatex --shell-escape -output-directory=myout vol-1-天うつ浪第一
%  ptex2pdf でビルド時に必要なディレクトリ mycache と myout
%  cluttex  でビルド時に必要なディレクトリ myout と myout/mycache
%           また --includeonly=NAMEs  を指定すると '\includeonly{NAMEs}' を仮挿入してくれる
%%
\RequirePackage{plautopatch}
%\RequirePackage{exppl2e}% 警告メッセージ削減のためコメントアウト
% upLaTeX文書
\documentclass[dvipdfmx,uplatex,tate,book,paper=a5paper,jafontsize=13pt,
    open_bracket_pos=nibu_tentsuki,hanging_punctuation]{jlreq}
\usepackage{bxpapersize}
\usepackage{pxrubrica}
\usepackage{sfkanbun}
\usepackage[deluxe,multi,jis2004]{otf}
\usepackage[directunicode*, noalphabet]{pxchfon}[2017/04/08]
\usepackage{plext}
\usepackage{graphicx}
\usepackage{bxglyphwiki}
\usepackage{indent}
\usepackage{CJK-char-convert}
\rubysetup{<hj>}% 無指定時のルビ:(<)前進入大、(h)肩付き、(j)熟語、(>)後進入大
% \rubysetup{|h>}% 無指定時のルビ:(|)前進入禁止、(h)肩付き、(>)後進入大
\title{\Huge 天うつ浪 {\Large 第一}}
\author{幸田露伴}
\date{         {\small 明治三十九年一月} 春陽{\換字{堂}}}
%\includeonly{其四十}
\newcommand{\詰めruby}[2]{\ruby[g]{#1}{{\kanjiskip=0pt plus 0pt minus 0pt #2}}}
\newcommand{\g詰めruby}[2]{\ruby[g]{#1}{{\kanjiskip=0pt plus 0pt minus 0pt #2}}}
\newcommand{\原本頁}[1]{}
\begin{document}
\maketitle
\pagestyle{myheadings}
\newcommand{\Entry}[1]{
	\section*{#1}
	\markboth{#1}{#1}
	\setcounter{equation}{0}}
\begin{indentation}{4zw}{3zw}
\parindent=0pt

\newpage
\ %全角空白
\newpage

{\huge
\ruby{天}{そら} う つ %空白有り
\ruby{浪}{なみ}}  {\normalsize 第一}
\vspace*{3zw}

\Entry{其一}

\ruby{秋}{あき}は
\ruby{海樓}{かい|ろう}の
\ruby[g]{直簾}{すだれ}に
\ruby{動}{うご}きて、ぱつと
\ruby{吹}{ふ}き
\ruby{來}{く}る
\ruby{沖}{おき}の
\ruby{風}{かぜ}は、
\ruby[g]{夕日}{ゆふひ}の
\ruby[g]{餘光}{よくわう}
\ruby{美}{うる}はしきが
\ruby{中}{なか}に、
\ruby{無限}{む|げん}の
\ruby{爽涼}{さう|りやう}の
\ruby{氣}{き}を
\ruby{齎}{もた}らせば、
\ruby{白帆}{しら|ほ}
\ruby{明}{あか}るき
\ruby[g]{{\換字{遠}}方}{とほく}の
\ruby{{\換字{船}}}{ふね}の
\ruby{數々}{かず|〳〵}も、
\ruby{{\換字{鉛}}色}{なまり|いろ}なして
\ruby{漫々}{まん|〳〵}たる
\ruby{潮}{うしほ}の
\ruby{果}{はて}に
\ruby{却}{かへ}つて
\ruby{物淋}{もの|さび}しう
\ruby{見}{み}え
\ruby{渡}{わた}りつゝ、
\ruby{竹芝}{たけ|しば}の
\ruby{浦}{うら}の
\ruby{浪靜}{なみ|しず}かに、
\ruby[g]{增上寺}{ぞうじやうじ}の
\ruby{鐘聲}{か|ね}に
\ruby{暮}{く}れ
\ruby{行}{ゆ}かんとす。

\ruby{此}{こ}の
\ruby{夕}{ゆふべ }
\ruby{此}{こ}の
\ruby{時}{とき}、『見はらし』の
\ruby{樓上}{ろう|じよう}の一
\ruby{室}{しつ}に、
\ruby{貸}{か}し
\ruby[g]{浴衣}{ゆかた}の
\ruby{胸元}{むな|もと}ゆたかにくつろげて、
\ruby{醉}{よひ}に
\ruby{嘯}{うそぶ}く
\ruby[g]{大胡坐}{おほあぐら}、たゞ
\ruby{秋}{あき}の
\ruby{{\換字{飲}}酒}{さ|け}に
\ruby{宜}{よろ}しきを
\ruby{知}{し}つてその
\ruby{他}{た}を
\ruby{知}{し}らぬ
\ruby{面構}{つら|がま}へきび〳〵と、あはれも
\ruby[g]{絲瓜}{へちま}もあるものか、
\ruby{鴫}{しぎ}が
\ruby{飛}{と}んだら
\ruby{撃}{う}つて
\ruby[g]{下物}{さかな}、と
\ruby{云}{い}はぬばかりの
\ruby{顏}{かほ}つきして、いずれも
\ruby{勇}{いさ}みを
\ruby{含}{ふく}む
\ruby{酒盃}{さか|づき}の
\ruby{遣}{や}り
\ruby{取}{と}り、
\ruby{火}{ひ}の
\ruby{珠}{たま}も
\ruby{挾}{はさ}んで
\ruby{食}{く}ふべき
\ruby{年齢}{とし|ばへ}の
\ruby{勢}{いきほ}ひに、
\ruby[g]{此方}{こなた}の
\ruby[g]{壯語}{さうご}、
\ruby[g]{彼方}{かなた}の
\ruby[g]{傲語}{がうご}、
\ruby{或}{あるひ}は
\ruby[g]{彼此哄然}{かれこれどつ}と一
\ruby{齊}{ど}の
\ruby[g]{天狗笑}{てんぐわら}ひの
\ruby[g]{響}{どよみ}の
\ruby{中}{うち}に、
\ruby{間近}{ま|ぢか}く
\ruby{{\換字{通}}}{とほ}る
\ruby{{\換字{滊}}車}{き|しや}の
\ruby{音}{おと}をも
\ruby{埋}{うづ}めて
\ruby{仕舞}{し|ま}ふまで、
\ruby{無邪氣}{む|じや|き}に
\ruby{睦}{むつ}み
\ruby{語}{かた}らへる
\ruby{四人}{よ|にん}
\ruby{{\換字{連}}}{づれ}あり。

\ruby[g]{陽氣}{やうき}の
\ruby{歡笑}{くわん|せう}は
\ruby{一}{ひ}トしきり
\ruby{濟}{す}みて、
\ruby{今}{いま}しも
\ruby[g]{談話}{はなし}は
\ruby{少}{すこ}し
\ruby{沈}{しづ}みぬ。

\ruby{手}{て}さき
\ruby{頸筋}{くび|すぢ}に
\ruby{洋服}{やう|ふく}の
\ruby{痕}{あと}
\ruby[g]{{\換字{判}}然}{はつきり}と
\ruby{知}{し}れて、
\ruby{誰}{た}が
\ruby{眼}{め}にも
\ruby{{\換字{船}}人}{ふな|のり}と
\ruby{暎}{うつ}る
\ruby{赭顏}{あから|がほ}の
\ruby{日}{ひ}に
\ruby{焦}{や}けきつたる
\ruby[g]{{\換字{羽}\換字{勝}}}{はがち}
\ruby{千{\換字{造}}}{せん|ぞう}は、
\ruby{酒盃}{さか|づき}を
\ruby{擧}{あ}げて
\ruby{一}{ひ}ト
\ruby{口}{くち}
\ruby{{\換字{飲}}}{の}みしが、
\ruby{不興氣}{ふ|きよう|げ}に
\ruby{復下}{また|した}に
\ruby{置}{お}きて、

『フーム』
とばかり
\ruby{力無}{ちから|な}く
\ruby{答}{こた}へつ、
\ruby{{\換字{猶}}}{なほ}
\ruby{其}{そ}の
\ruby{對手}{あひ|て}の
\ruby{何事}{なに|ごと}をか
\ruby{語}{かた}り
\ruby{添}{そ}ふるを
\ruby{待}{ま}つが
\ruby{如}{ごと}き
\ruby{意}{こゝろ}を
\ruby{其}{そ}の
\ruby{語氣}{ご|き}に
\ruby{現}{あらは}したり。

\ruby[g]{{\換字{羽}\換字{勝}}}{はがち}に
\ruby{對}{むか}ひて
\ruby{坐}{ざ}せる
\ruby{小男}{こを|とこ}の、
\ruby{面{\換字{清}}}{おもて|きよ}らかにして
\ruby{桃花}{とう|くわ}の
\ruby{如}{ごと}き
\ruby{山瀬荒吉}{やま|せ|あら|きち}は
\ruby{其意}{その|い}を
\ruby{悟}{さと}つて、
\ruby{果}{はた}して
\ruby{直}{たゞち}に
\ruby{言葉}{こと|ば}を
\ruby{足}{た}しぬ。

『ト
\ruby{云}{い}う
\ruby{次第}{し|だい}なので
\ruby[g]{水野}{みづの}
\ruby{君}{くん}は
\ruby{來}{こ}んのさ。
\ruby{今}{いま}
\ruby{話}{はな}した
\ruby{内{\換字{情}}}{ない|じやう}も
\ruby{解}{わか}つて
\ruby{居}{ゐ}たので、
\ruby{今日}{け|ふ}の
\ruby{會合}{くわい|がふ}の
\ruby{發起人}{ほつ|き|にん}の
\ruby{僕}{ぼく}は、十
\ruby{分}{ぶん}に
\ruby{{\換字{情}}理}{じやう|り}を
\ruby{盡}{つく}した
\ruby{手紙}{て|がみ}を
\ruby{興}{や}つて、
\ruby{是非}{ぜ|ひ}
\ruby{出}{で}て
\ruby{來}{く}るやうにと
\ruby{勸}{すゝ}めたんだが、たゞ
\ruby{差支}{さし|つかへ}があつて
\ruby{行}{ゆ}かれないといふ
\ruby{冷淡}{れい|たん}
\ruby{極}{きは}まる
\ruby{{\換字{返}}事}{へん|じ}なんで、
\ruby{仕方}{し|かた}が
\ruby{無}{な}いと
\ruby[g]{斷念}{あきら}めて
\ruby{仕舞}{し|ま}つた。
\ruby{實}{じつ}に
\ruby[g]{水野}{みづの}
\ruby{君}{くん}にも
\ruby{似合}{に|あ}はない、
\ruby[g]{全然}{まるで}
\ruby{無茶苦茶}{む|ちあ|く|ちあ}になつて
\ruby{居}{ゐ}られるのだからね。
』

\ruby{見}{み}る〳〵
\ruby[g]{{\換字{羽}\換字{勝}}}{はがち}が
\ruby{面}{おもて}には
\ruby{憂色}{いう|しよく }
\ruby{現}{ あらは}れ、その
\ruby{眼}{め}は
\ruby{沈思}{ちん|し}に
\ruby{凝然}{じ|つ}と
\ruby{動}{うご}かずなりたり。

\ruby[g]{{\換字{羽}\換字{勝}}}{はがち}が
\ruby[g]{左方}{ひだり}に
\ruby{坐}{ざ}して
\ruby{{\換字{黙}}々}{もく|〳〵}と
\ruby{{\換字{飲}}}{の}み
\ruby{居}{ゐ}し
\ruby[g]{骨太岩{\換字{畳}}}{ほねぶとがんでふ}づくりの
\ruby[g]{日方}{ひかた}八
\ruby{郞}{らう}は、
\ruby{突然}{とつ|ぜん}として
\ruby{牛}{うし}の
\ruby{吼}{ほ}ゆるが
\ruby{如}{ごと}くに
\ruby{叫}{さけ}び
\ruby{出}{だ}し、

『
\ruby{山瀬}{やま|せ}、
\ruby{貴様}{き|さま}も
\ruby{今}{いま}は
\ruby{堂々}{だう|〴〵}たる
\ruby{新聞記者}{しん|ぶん|き|しや}だ。
\ruby[g]{往時}{むかし}のやうに
\ruby{想像談}{さう|〴〵|だん}や
\ruby{法螺}{ほ|ら}
\ruby{話}{ばなし}は
\ruby{語}{かた}るまいな。
』

と、
\ruby{詰}{なじ}り
\ruby{氣味}{ぎ|み}に
\ruby{問}{と}ひ
\ruby{糺}{たゞ}せば、
\ruby{山瀬}{やま|せ}は
\ruby{聊}{いさゝ}か
\ruby{怫然}{む|つ}として、

『
\ruby[g]{日方}{ひかた}
\ruby[g]{陸軍少尉殿}{りくぐんせうゐどの}に
\ruby{伺}{うかが}ひます。
\ruby{報告}{はう|こく}は
\ruby{無責任}{む|せき|にん}を
\ruby{以}{もつ}て
\ruby[g]{作爲}{さくゐ}すべきものでござりまする
\ruby{歟}{か}。
はゝはゝはゝ。
』

と
\ruby{{\換字{遣}}}{や}り
\ruby{{\換字{返}}}{かへ}して
\ruby{笑}{わら}ふ。

\ruby[g]{日方}{ひかた}は
\ruby{山瀬}{やま|せ}の
\ruby[g]{戯言}{たはむれ}には
\ruby{頓着}{とん|ぢやく}
\ruby{無}{ な}く、
\ruby{怒}{いか}れるが
\ruby{如}{ごと}く
\ruby{眞面目}{ま|じ|め}になりて、

『ムゝ、して
\ruby{見}{み}れば
\ruby{全}{まつた}く
\ruby{事實}{じ|じつ}と
\ruby{見}{み}える。
イヤ
\ruby{怪}{け}しからん、
\ruby{實}{じつ}に
\ruby{怪}{け}しからん。
\ruby{何}{なん}だ!。
\ruby{愚劣極}{ぐ|れつ|きは}まる!。
\ruby{馬鹿}{ば|か}
\ruby[g]{々々}{〳〵}しい。
ナニ?。
\ruby{戀愛}{れん|あい}に
\ruby{陷}{おちい}って
\ruby{苦悶}{く|もん}しちよる、それで
\ruby{朋友}{ほう|いう}の
\ruby{集會}{しふ|かい}にも
\ruby{出席}{しゆつ|せき}しないと?。
たッ
\ruby{白痴野郎}{た|はけ|や|らう}め、
\ruby{何}{なん}といふ
\ruby{事}{こつ}た。
そんな
\ruby{愚}{ぐ}な
\ruby{奴}{やつ}では
\ruby{無}{な}かつたが、
\ruby{{\換字{魔}}}{ま}にでも
\ruby{憑}{つ}かれ
\ruby{居}{を}つたか、
\ruby{下}{くだ}らない。
\ruby{山瀬}{やま|せ}、
\ruby{貴様}{き|さま}も
\ruby{幹事甲斐}{かん|じ|が|ひ}がない。
\ruby[g]{其様}{そん}な
\ruby{生{\換字{温}}}{なま|ぬる}つこい
\ruby{事}{こと}を
\ruby{云}{い}はす
\ruby{法}{はふ}が
\ruby{有}{あ}るかい!。
\ruby{領上}{えり|がみ}に
\ruby{手}{て}を
\ruby{掛}{か}けて
\ruby{引摺}{ひき|ず}つて
\ruby{來}{く}りやあ、
\ruby[g]{一同}{みんな}で
\ruby{引{\換字{擲}}}{ひつ|ぱた}いて
\ruby{正氣}{しやう|き}に
\ruby{仕}{し}て
\ruby{{\換字{遣}}}{や}るのに。
』

ゑゝ、
\ruby{理由}{わ|け}を
\ruby{聞}{き}かぬ
\ruby{間}{うち}は
\ruby{知}{し}らぬが
\ruby{佛}{ほとけ}で
\ruby{腹}{はら}も
\ruby{立}{た}たなかつたが、
\ruby{聞}{き}いて
\ruby{見}{み}りやあ
\ruby{馬鹿}{ば|か}
\ruby[g]{々々}{〳〵}しくつて
\ruby{腹}{はら}が
\ruby{立}{た}つ。
\ruby{山瀬}{やま|せ}!。
一
\ruby{體}{たい}
\ruby{貴様}{き|さま}が
\ruby{薄}{うす}つぺらで
\ruby{眞底}{しん|そこ}からの
\ruby{信實氣}{しん|じつ|ぎ}が
\ruby{足}{た}らん。
\ruby{本來}{ほん|らい}
\ruby[g]{我々}{われ〳〵}七
\ruby{人}{にん}は
\ruby{何様}{ど|う}いふ
\ruby{交{\換字{情}}}{な|か}だ。
みんな
\ruby{野州}{や|しう}の
\ruby{田舎漢}{ゐ|なか|もの}、
\ruby{碌}{ろく}な
\ruby{親}{おや}を
\ruby{持}{も}つたものは
\ruby[g]{一人}{ひとり}も
\ruby{無}{な}くつて、
\ruby{役場}{やく|ば}の
\ruby{書記}{しよ|き}や
\ruby{小學{\換字{教}}師}{せう|がく|けう|し}、
\ruby{乃公}{お|ら}あ
\ruby[g]{人力車}{くるま}も
\ruby{曳}{ひつ}ぱつた
\ruby{貧書生}{ひん|しよ|せい}だが、
\ruby{自己}{う|ぬ}が
\ruby{腕臑}{うで|すね}で
\ruby{食}{く}ふ
\ruby{貧乏同士}{びん|ばふ|どう|し}、
\ruby{何時}{い|つ}と
\ruby{無}{な}く
\ruby{知}{し}り
\ruby{合}{あ}ひになつた
\ruby{七人}{しち|にん}が、
\ruby[g]{男兒}{をとこ}と
\ruby{生}{うま}れて
\ruby{此狀}{こ|れ}ぢやあ
\ruby{死}{し}ねぬ、
\ruby{志}{こヽろざ}すところは
\ruby{異}{ちが}つても
\ruby{互}{たがひ}に
\ruby{助}{たす}け
\ruby{幇}{たす}け
\ruby{合}{あ}つて、
\ruby{有時}{ある|とき}は
\ruby{兄}{あに}となつて
\ruby{學資}{がく|し}も
\ruby{貢}{みつ}ぎ、
\ruby{有時}{ある|とき}は
\ruby{弟}{おとヽ}となつて
\ruby{恩}{おん}を
\ruby{報}{はう}じ、
\ruby{勵}{はげ}み
\ruby{合}{あ}い
\ruby{擁護}{か|ば}ひ
\ruby{合}{あ}つて
\ruby{{\換字{進}}}{すヽ}んで
\ruby{行}{い}つたら、
\ruby{世}{よ}に
\ruby{立}{た}つて
\ruby{生}{い}き
\ruby{甲斐}{が|い}のある
\ruby{身}{み}ともなれやうと、
\ruby{七人}{しち|にん}
\ruby{集}{あつ}まつた
\ruby[g]{宇都宮}{うつのみや}の
\ruby{二荒山神社}{ふた|あら|やま|じん|じや}の
\ruby{廣前}{ひろ|まへ}で、
\ruby{此}{こ}の
\ruby{願}{ねがひ}
\ruby{此}{ こ}の
\ruby{心}{こヽろ}
\ruby{渝}{かは}るまじ、
\ruby{必}{かなら}ず
\ruby{信義}{しん|ぎ}を
\ruby{盡}{つく}し
\ruby{合}{あ}はんと、
\ruby{神}{かみ}に
\ruby{誓}{ちか}つた
\ruby{交{\換字{情}}}{な|か}では
\ruby{無}{な}いか。
\ruby{指折}{ゆび|を}り
\ruby{數}{かぞ}ふれば
\ruby{速}{はや}いもので
\ruby{既七年}{はや|しち|ねん}の
\ruby[g]{往時}{むかし}になるが、
\ruby{其時}{そ|れ}からといふものは
\ruby{段々}{だん|〳〵}と、
\ruby{苦}{くる}しい
\ruby{同士}{どう|し}で
\ruby{無理才覺}{む|り|さい|かく}、
\ruby{三人}{さん|にん}の
\ruby{財布}{さい|ふ}を
\ruby{揮}{ふる}つては
\ruby{一人}{いち|にん}の
\ruby{{\換字{遊}}學}{いう|がく}の
\ruby{支度}{し|たく}を
\ruby{拵}{こしら}へ、
\ruby{五人}{ご|にん}の
\ruby{着物}{き|もの}を
\ruby{賣}{う}つては
\ruby{一人}{いち|にん}の
\ruby{身}{み}の
\ruby{立}{た}つ
\ruby[g]{本錢}{もとで}とするといふ
\ruby{始末}{し|まつ}で、ポツリ〳〵と
\ruby{皆}{みな}
\ruby{東京}{とう|きやう}へ、
\ruby{漸}{やうや}く
\ruby{這}{は}ひ
\ruby{出}{だ}してそれ〴〵に、
\ruby{志}{こヽろざ}す
\ruby{{\換字{道}}}{みち}へと
\ruby{身}{み}を
\ruby{入}{い}れた、
\ruby{如是}{かう|いふ}
\ruby{{\換字{交}}{\換字{情}}}{な|か}だのに
\ruby{何}{なん}の
\ruby{事}{こつ}た!。
\ruby{胸糞}{むな|くそ}の
\ruby{惡}{わる}い
\ruby{戀愛}{れん|あい}なんぞに
\ruby[g]{水野}{みづの}が
\ruby{{\換字{迷}}}{まよ}ってるなら
\ruby{何故打棄}{な|ぜ|うつ|ちや}つて
\ruby{置}{お}く?。
{\GWI{u1b048}}かも
\ruby[g]{{\換字{羽}\換字{勝}}}{はがち}が
\ruby{始}{はじ}めて
\ruby{首尾}{しゆ|び}よく
\ruby{{\換字{遠}}洋漁業}{ゑん|やう|ぎよ|げふ}の
\ruby{長}{なが}い
\ruby{航海}{かう|かい}を、
\ruby{{\換字{終}}}{をは}つて
\ruby{來}{き}た
\ruby{今日}{け|ふ}の
\ruby{欣喜}{よろ|こび}の
\ruby{集會}{あつ|まり}に、
\ruby{自己}{お|の}が
\ruby[g]{勝手}{かつて}の
\ruby{女沙汰}{をんな|ざ|た}のために
\ruby{不參}{ふ|さん}とは、
\ruby{我々}{われ|〳〵}を
\ruby{踏}{ふ}み
\ruby{付}{つ}けた
\ruby{憎}{にく}い
\ruby{我儘}{わが|まゝ}。
\ruby{山瀬}{やま|せ}
\ruby{汝}{きさま}は
\ruby{何故}{な|ぜ}
\ruby{打棄}{うつ|ちや}つて
\ruby{置}{お}く?
\ruby{汝}{きさま}が
\ruby{新聞記者}{しん|ぶん|き|しや}になつた
\ruby{時}{とき}は、
\ruby{我我}{われ|〳〵}
\ruby{七人}{しち|にん}
\ruby{皆}{みな}
\ruby{揃}{そろ}つた。
\ruby{乃公}{お|れ}が
\ruby{士官候補生}{し|くわん|こう|ほ|せい}になつた
\ruby{時}{とき}にも
\ruby{皆}{みな}
\ruby{集}{あつ}まつて
\ruby{悦}{よろこ}んで
\ruby{{\換字{呉}}}{く}れた。
\ruby[g]{{\換字{羽}\換字{勝}}}{はがち}の
\ruby{今日}{け|ふ}の
\ruby{祝賀}{よろ|こび}の
\ruby{會}{くわい}には、
\ruby{楢井}{なら|い}は
\ruby{北海{\換字{道}}}{ほく|かい|だう}に
\ruby{行}{い}つて
\ruby{居}{を}り、
\ruby{名倉}{な|くら}は
\ruby{病氣}{びやう|き}、
\ruby[g]{二人}{ふたり}
\ruby{缺}{か}けて
\ruby{居}{ゐ}るさへ
\ruby{殘念}{ざん|ねん}なに、
\ruby[g]{水野}{みづの}まで
\ruby{來}{こ}ぬので
\ruby{只}{たつた}
\ruby{四人}{よ|にん}、
\ruby{第一}{だい|いち}
\ruby[g]{{\換字{羽}\換字{勝}}君}{はがちくん}にも
\ruby{氣}{き}の
\ruby{毒}{どく}
\ruby{千萬}{せん|ばん}だ。
\ruby{戀愛}{れん|あい}も
\ruby{糞}{くそ}もあるものか、
\ruby{世間一統}{せ|けん|いつ|とう}の
\ruby{愚物}{ぐ|ぶつ}は
\ruby{知}{し}らず、
\ruby{何時}{い|つ}でも
\ruby{現在}{げん|ざい}に
\ruby{滿足}{まん|ぞく}せいで、
\ruby{永久}{えい|きう}に
\ruby{{\換字{進}}}{すヽ}んで
\ruby{{\換字{飽}}}{あ}くこと
\ruby{知}{し}らぬを
\ruby{理想}{り|さう}と
\ruby{定}{さだ}めた
\ruby{我我}{われ|〳〵}
\ruby{七人}{しち|にん}、
\ruby{戀愛}{れん|あい}なんぞといふアタ
\ruby{{\換字{嫌}}}{いや}らしい
\ruby{濕氣}{しつ|け}の
\ruby{蠹}{むし}に、
\ruby{魂魄}{たま|しひ}を
\ruby{蝕}{くは}せて
\ruby{居}{ゐ}る
\ruby{間}{ま}は
\ruby{無}{な}い
\ruby{筈}{はず}。
\ruby{一體全體}{いつ|たい|ぜん|たい}
\ruby{癪}{しやく}に
\ruby{觸}{さは}る!。
\ruby{何}{なに}を
\ruby{讀}{よ}んでも
\ruby{何處}{ど|こ}へ
\ruby{行}{い}つても、
\ruby{此頃}{この|ごろ}は
\ruby{戀愛}{れん|あい}といふ
\ruby{奴}{やつ}ばかり
\ruby{轉}{ころ}げて
\ruby{居}{ゐ}をるが、
\ruby{戀愛}{れん|あい}たあ
\ruby{何}{なん}だ?、
\ruby{何}{なん}だ
\ruby{正體}{しやう|たい}は?。
\ruby{自己}{う|ぬ}から
\ruby{見}{み}りやあ
\ruby{貴}{い}いか
\ruby{知}{し}らぬが、
\ruby{他}{ひと}から
\ruby{見}{み}りやあ
\ruby{石{\換字{決}}明}{あは|びつ|かひ}を
\ruby{當}{あ}てがつて
\ruby{{\換字{遣}}}{や}る
\ruby{價値}{ね|うち}も
\ruby{無}{な}い
\ruby{馬糞}{ば|ふん}に
\ruby{劣}{おと}つた
\ruby{貨物}{しろ|もの}で、
\ruby{高}{たか}が
\ruby{女}{をんな}にびりつく
\ruby{事}{こと}だ!。
\ruby[g]{水野}{みづの}は
\ruby{釅}{きぶ}い
\ruby{醋}{す}のやうな
\ruby{恐}{おそ}ろしいところのある
\ruby{奴}{やつ}ぢやつたが、
\ruby{{\換字{浮}}世}{うき|よ}に
\ruby{感染}{か|ぶ}れたのは
\ruby{氣}{き}が
\ruby{緩}{ゆる}んだ
\ruby{歟}{か}。
\ruby{打棄}{うつ|ちや}つて
\ruby{置}{おい}ては
\ruby{利{\換字{益}}}{た|め}にならん。
\ruby{直}{すぐ}これから
\ruby{行}{い}つて
\ruby{引摺}{ひき|ず}って
\ruby{來}{こ}やう。
さあ
\ruby{山瀬}{やま|せ}!一
\ruby{緖}{しよ}に
\ruby{行}{ゆ}け、
\ruby{立}{た}たぬかやい。
\ruby[g]{水野}{みづの}めを
\ruby{引張}{ひつ|ぱ}つて
\ruby{來}{き}て
\ruby{此處}{こ|こ}で
\ruby{諫}{いさ}めて
\ruby{聽}{き}かずば
\ruby{擲}{たヽ}き
\ruby{撲}{なぐ}つて、
\ruby{正氣}{しやう|き}に
\ruby{{\換字{返}}}{かへ}らせて
\ruby{{\換字{呉}}}{く}れにやならぬ、さあ
\ruby{立}{た}て
\ruby{山瀬}{やま|せ}!。
』

と
\ruby{云}{い}ひざまに、
\ruby{五分}{ご|ぶ}の
\ruby{慷慨}{かう|がい}、
\ruby{五分}{ご|ぶ}の
\ruby{醉}{ゑひ}、
\ruby{山瀬}{やま|せ}が
\ruby{肩頭}{かた|さき}を
\ruby{引攫}{ひつ|つか}んで
\ruby{氣勢}{いき|ほひ}
\ruby{猛}{もう}に
\ruby{立上}{たち|あが}つたり。

\Entry{其二}

\ruby{薄墨}{うす|ずみ}の
\ruby{夕}{ゆふべ}の
\ruby{色}{いろ}は
\ruby{物蔭}{もの|かげ}より
\ruby{擴}{ひろ}まりて、
\ruby{廓然}{くわ|らり}と
\ruby{{\換字{晴}}}{は}れやかなりし
\ruby{樓}{ろう}の
\ruby{上}{うへ}も、
\ruby{手許}{て|もと}やうやく
\ruby{暗}{くら}くなり、いづくに
\ruby{歸}{かへ}る
\ruby{鵜}{う}の
\ruby{鳥}{とり}の、
\ruby{浪}{なみ}を
\ruby{{\換字{摩}}}{す}つて
\ruby{飛}{と}ぶ
\ruby{{\換字{羽}}音}{は|おと}も
\ruby{寂}{さ}びたり。
\ruby{右}{みぎ}の
\ruby{方}{かた}は
\ruby{高輪八}{たか|なわ|や}ツ
\ruby{山}{やま}
\ruby{品川}{しな|がは}の
\ruby{一}{ひ}トつゞき、
\ruby{森}{もり}も
\ruby{人家}{じん|か}もたゞ
\ruby{一}{ひ}ト
\ruby{筆}{ふで}のなすり
\ruby{書}{がき}と
\ruby{黑}{くろ}み、
\ruby{左}{ひだり}に
\ruby{低}{ひく}き
\ruby{築地月島}{つき|ぢ|つき|しま}、
\ruby{洲崎}{す|さき}は
\ruby{微}{かすか}にして
\ruby{{\換字{消}}}{き}えんとする
\ruby{時}{とき}、
\ruby{其處}{そ|こ}に
\ruby{電燈}{でん|とう}の
\ruby{白々}{しろ|〴〵}と
\ruby{輝}{かゞや}き
\ruby{出}{い}づれば、
\ruby{燈火}{とも|しび}
\ruby{華}{はな}やかに
\ruby{此家}{こ|こ}にも
\ruby{點}{つ}きて、
\ruby{室}{へや}の
\ruby{内}{うち}ぱつと
\ruby{明}{あか}るくなり、
\ruby{外}{そと}は
\ruby{全}{まつた}く
\ruby{海}{うみ}
\ruby{玄}{くろ}く
\ruby{風}{かぜ}
\ruby{睡}{ねむ}れる
\ruby{{\換字{穏}}}{おだ}やかなる
\ruby{夜}{よ}となり
\ruby{畢}{をは}んぬ。

\ruby[g]{日方}{ひかた}が
\ruby{急}{せ}き
\ruby{{\換字{込}}}{こ}み
\ruby{調子}{てう|し}に
\ruby{物言}{もの|い}ひても、
\ruby{特更}{こと|さら}に
\ruby{沈着}{おち|つき}を
\ruby{爲}{つく}れる
\ruby{山瀬荒吉}{やま|せ|あら|きち}は、
\ruby{言}{い}ひ
\ruby{爭}{あらそ}はんともせで
\ruby{良}{やゝ}
\ruby[g]{少時}{しばし}、
\ruby{何事}{なに|ごと}をか
\ruby{思}{おも}ひ
\ruby{廻}{めぐ}らし
\ruby{居}{ゐ}けるが、
\ruby{今}{いま}しも
\ruby{燈火}{とも|しび}の
\ruby{光}{ひかり}を
\ruby{得}{え}て、
\ruby{心}{こゝろ}の
\ruby{中}{うち}に
\ruby{索}{たづ}ね
\ruby{得}{え}し
\ruby[g]{言葉}{ことば}の
\ruby{緒}{いとぐち}をや
\ruby{求}{もと}め
\ruby{得}{え}けん、
\ruby{逸}{はや}りきつたる
\ruby[g]{日方}{ひかた}の
\ruby{面}{おもて}の、いさゝか
\ruby{怒}{いかり}をさへ
\ruby{{\換字{帯}}}{お}びたるを、
\ruby{愛}{あい}するが
\ruby{如}{ごと}く
\ruby{打見}{うち|み}やりて、

『マア
\ruby{坐}{すわ}つて
\ruby{{\換字{呉}}}{く}れ、
\ruby[g]{日方}{ひかた}!。
\ruby{成程}{なろ|ほど}
\ruby{打棄}{うつ|ちや}つて
\ruby{置}{おい}ては
\ruby{水野}{みづ|の}の
\ruby{不利{\換字{益}}}{ふ|た|め}になるから、
\ruby{君}{きみ}と一
\ruby{緒}{しよ}に
\ruby{尋}{たづ}ねて
\ruby{行}{い}つて、
\ruby{隨分}{ずゐ|ぶん }
\ruby{忠告}{ちゆう|こく}も
\ruby{試}{こゝろ}みやう。
\ruby{倂}{しか}し
\ruby{水野}{みづ|の}のところは
\ruby{大分}{だい|ぶん}
\ruby{{\換字{遠}}}{とほ}い。
\ruby{{\換字{連}}}{つ}れて
\ruby{來}{く}るにしても
\ruby{時間}{と|き}がかゝる。
もう
\ruby{此}{こ}の
\ruby{{\換字{通}}}{とほ}り
\ruby{夜}{よ}にも
\ruby{入}{い}つて
\ruby{居}{ゐ}る。
\ruby{{\換字{連}}}{つ}れて
\ruby{來}{き}たにしたところで
\ruby{話}{はな}す
\ruby{間}{ま}も
\ruby{無}{な}い。
\ruby{第一}{だい|いち}
\ruby{左様}{さ|う}で
\ruby{無}{な}くつてさへ、
\ruby{七人}{しち|にん}の
\ruby{中}{うち}が
\ruby{三人}{さん|にん}
\ruby{缺}{か}けて、
\ruby{四人}{よ|にん}しか
\ruby{居}{を}らぬ
\ruby{此}{こ}の
\ruby{席}{せき}を、
\ruby{君}{きみ}と
\ruby{僕}{ぼく}と
\ruby{二人}{ふた|り}
\ruby{脱}{ぬ}けて
\ruby{仕舞}{し|ま}へば
\ruby{後}{あと}は
\ruby{何様}{ど|う}だ。
\ruby[g]{{\換字{羽}\換字{勝}}君}{はがちくん}と
\ruby{島木君}{しま|き|くん}とたつた
\ruby{二人}{ふた|り}だ。
\ruby{今日}{け|ふ}の
\ruby{客}{きやく}たる
\ruby[g]{{\換字{羽}\換字{勝}}君}{はがちくん}を、
\ruby{島木君}{しま|き|くん}と
\ruby{只二人}{たつた|ふた|り}に
\ruby{仕}{し}て
\ruby{仕舞}{し|ま}つて、
\ruby{僕等}{ぼく|ら}が
\ruby{出}{で}て
\ruby{行}{い}くといふのは
\ruby{勝手}{かつ|て}
\ruby{{\換字{過}}}{す}ぎる。
それでは
\ruby{餘}{あんま}り
\ruby{無禮}{ぶ|れい}になる。
こゝを
\ruby{無理}{む|り}に
\ruby{君}{きみ}と
\ruby{二人}{ふた|り}で
\ruby{出}{で}て
\ruby{行}{い}つたら、
\ruby{水野}{みづ|の}には
\ruby{成程親切}{なる|ほど|しん|せつ}にもならう。
\ruby{倂}{しか}し
\ruby[g]{{\換字{羽}\換字{勝}}君}{はがちくん}には
\ruby{失敬}{しつ|けい}に
\ruby{當}{あた}らう。
もと〳〵
\ruby{君}{きみ}が
\ruby{怒}{おこ}り
\ruby{立}{た}つたのも、つまりは
\ruby{水野}{みづ|の}が
\ruby[g]{{\換字{羽}\換字{勝}}君}{はがちくん}に
\ruby{對}{たい}する
\ruby{仕方}{し|かた}が
\ruby{冷淡}{れい|たん}だといふのにあらう。
\ruby[g]{{\換字{羽}\換字{勝}}君}{はがちくん}に
\ruby{滿足}{まん|ぞく}を
\ruby{感}{かん}ぜしめぬ
\ruby{其事}{そ|れ}が
\ruby{惡}{にく}むべき
\ruby{我儘}{わが|まま}だといふのだ。
それだのに
\ruby{今}{いま}
\ruby{僕等}{ぼく|ら}が
\ruby{此席}{こ|こ}を
\ruby{去}{さ}つては、たゞ
\ruby{淋}{さび}しさを
\ruby{{\換字{増}}}{ま}すばかりで、
\ruby[g]{{\換字{羽}\換字{勝}}君}{はがちくん}はいよ〳〵おもしろく
\ruby{無}{な}く
\ruby{感}{かん}じやう。
\ruby{今日}{け|ふ}は
\ruby{既}{もう}十
\ruby{分}{ぶん}に
\ruby{談笑}{だん|せう}も
\ruby{仕}{し}て、
\ruby{大分}{だい|ぶ}
\ruby{醉}{よひ}さえも
\ruby{{\換字{廻}}}{まは}って
\ruby{居}{ゐ}る。
\ruby[g]{談話}{はなし}の
\ruby{序}{つひで}から
\ruby{不圖}{ふ|と}
\ruby{水野}{みづ|の}の
\ruby{事}{こと}が
\ruby{出}{で}て、
\ruby{始}{はじ}めて
\ruby{君}{きみ}は
\ruby{其}{それ}を
\ruby{聞}{き}いたところから、
\ruby{大}{おほき}に
\ruby{忌}{いま}はわしくも
\ruby{感}{かん}じたらうが、
\ruby{何}{なに}も
\ruby{今}{いま}が
\ruby{今}{いま}でなくちやならぬといふ
\ruby{事}{こと}では
\ruby{無}{な}いから、
\ruby{彼}{かれ}を
\ruby{訪}{と}ふのは
\ruby{明日}{あ|す}でも
\ruby[g]{明後日}{あさつて}でもの
\ruby{事}{こと}として、
\ruby{其時}{その|とき}
\ruby{戀愛{\換字{嫌}}}{れん|あい|ぎら}ひの
\ruby{君}{きみ}の
\ruby{存分}{ぞん|ぶん}に、
\ruby{諫}{いさ}めるとも
\ruby{擲}{なぐ}るともするが
\ruby{宜}{よ}からう。
\ruby{今日}{け|ふ}は
\ruby{先}{ま}づ
\ruby{堪{\換字{忍}}}{かん|にん}して
\ruby[g]{一同}{みんな}と
\ruby{共}{とも}に、
\ruby{{\換字{飲}}}{の}んで
\ruby{居}{ゐ}て
\ruby{{\換字{呉}}}{く}れたつて
\ruby{可}{よ}いでは
\ruby{無}{な}いか。
』

と、
\ruby{他}{ひと}の
\ruby{言}{い}ふところは
\ruby{斜}{なゝめ}に
\ruby{外}{そ}らせて、
\ruby{我}{わ}が
\ruby{言}{い}ふところは
\ruby{斜}{なゝめ}に
\ruby{徹}{とほ}す
\ruby{才士}{さい|し}の
\ruby{面}{おもて}は
\ruby{笑}{ゑみ}を
\ruby{湛}{たゝ}へて、
\ruby{巧}{たくみ}に
\ruby{粗獷}{ぶ|こつ}なる
\ruby{相手}{あひ|て}を
\ruby{制}{せい}すれば、
\ruby[g]{正直三昧}{しやうぢきざんまい}の
\ruby[g]{日方}{ひかた}は、
\ruby{脆}{もろ}くも、
\ruby{{\換字{羽}}{\換字{勝}}}{は|がち}を
\ruby{重}{おも}んずる
\ruby{{\換字{情}}}{こゝろ}より、

『ムー、
\ruby{此}{こ}の
\ruby{席}{せき}が
\ruby{淋}{さび}しくなる?。
ア、
\ruby{其處}{そ|こ}へは
\ruby{些}{ちつと}も
\ruby{氣}{き}がつかなかつた。
\ruby{成程}{なる|ほど}
\ruby{今直}{いま|すぐ}
\ruby{引張}{ひつ|ぱ}つて
\ruby{來}{こ}やうと
\ruby{云}{い}ったのは、
\ruby{乃公}{お|れ}が
\ruby{惡}{わる}かつた。
こいつは
\ruby{一番}{いち|ばん}
\ruby{山瀬}{やま|せ}にやられた。
ハヽヽ。
どうも
\ruby{山瀬}{やま|せ}は
\ruby{乃公}{お|れ}より
\ruby{怜悧}{り|こう}だ。
ハヽヽ。
』

と、
\ruby{露}{つゆ}ばかりの
\ruby{我執}{が|しふ}も
\ruby{無}{な}く
\ruby{笑}{わら}つて
\ruby{仕舞}{し|ま}つて、
\ruby{霽々}{はれ|〴〵}したる
\ruby{顏色}{かほ|つき}にも
\ruby{著}{しる}き
\ruby{胸}{むね}に
\ruby{何}{なに}も
\ruby{{\換字{遺}}}{のこ}さぬ
\ruby{有様}{あり|さま}は、
\ruby{譬}{たと}へば
\ruby{風{\換字{過}}}{かぜ|す}ぎて
\ruby{林}{はやし}おのづから
\ruby{靜}{しづか}に、
\ruby{雲}{くも}
\ruby{去}{さ}つて
\ruby{山}{やま}
\ruby{更}{さら}に
\ruby{靑}{あお}きが
\ruby{如}{ごと}くなりしが、
\ruby{例}{れい}の
\ruby{癖}{くせ}とて
\ruby{突然}{とつ|ぜん}と、

『ヤ、時に
\ruby[g]{{\換字{羽}\換字{勝}}君}{はがちくん}
\ruby{一盃}{いつ|ぱい}
\ruby{吳}{く}れたまへ。
』

と
\ruby{云}{い}ひ
\ruby{出}{いだ}したり。
\ruby{{\換字{羽}}{\換字{勝}}}{は|がち}は
\ruby{機{\換字{嫌}}}{き|げん}
\ruby{良}{よ}く
\ruby{盃}{さかづき}をさして、

『
\ruby{相變}{あひ|かは}らず
\ruby{君}{きみ}は
\ruby{君}{きみ}の
\ruby{氣風}{き|ふう}で
\ruby{押{\換字{通}}}{おし|とほ}すナ。
どうだ
\ruby{軍{\換字{隊}}}{ぐん|たい}の
\ruby{生活}{せい|くわつ}は
\ruby{{\換字{愉}}快}{ゆ|くわい}かネ。
』

と
\ruby{{\換字{懐}}}{なつ}かし
\ruby{氣}{げ}に
\ruby{問}{と}へば、

『ムヽ。
\ruby{左様}{さ|う}さ。
\ruby{快活}{くわい|くわつ}な
\ruby{事}{こと}ばかりといふ
\ruby{譯}{わけ}にも行かん。
\ruby{僕等}{ぼく|ら}の
\ruby{身分}{み|ぶん}では
\ruby{隨分}{ずゐ|ぶん}
\ruby{箱詰}{はこ|づめ}になるのを
\ruby{甘}{あま}んじなけりやならん
\ruby{事}{こと}もあるが、
\ruby{其}{それ}が
\ruby{{\換字{即}}}{すなは}ち
\ruby{規律}{き|りつ}で、
\ruby{規律}{き|りつ}が
\ruby{{\換字{即}}}{すなは}ち
\ruby{精神}{せい|しん}である、といふやうに
\ruby{考}{かんが}へて
\ruby{居}{ゐ}りやあ、
\ruby{別}{べつ}に
\ruby{窮屈}{きう|くつ}にも
\ruby{感}{かん}じない。
ホワイトシヤツを
\ruby{着慣}{き|な}れて
\ruby{見}{み}ると、
\ruby{彼}{あ}の
\ruby{硬}{こは}いものを
\ruby{身}{み}につけるのが、
\ruby{却}{かへ}つて
\ruby{好}{い}い
\ruby{心持}{こゝろ|もち}に
\ruby{思}{おも}へて
\ruby{來}{く}る。
\ruby{丁度}{ちやう|ど}それと
\ruby{同}{おな}じ
\ruby{事}{こと}で、
\ruby{慣}{な}れてみると
\ruby{嚴肅}{げん|しゆく}な
\ruby{中}{うち}には
\ruby{{\換字{愉}}快}{ゆ|くわい}があるから、
\ruby{僕}{ぼく}はまあ
\ruby{不{\換字{愉}}快}{ふ|ゆ|くわい}には
\ruby{日}{ひ}を
\ruby{{\換字{送}}}{おく}らん。
』

と
\ruby{答}{こた}へて
\ruby{其}{そ}の
\ruby{盃}{さかづき}を
\ruby{乾}{ほ}して
\ruby{洗}{あら}ふ。

『
\ruby{左様}{さ|う}だ。
\ruby{規律}{き|りつ}を
\ruby{{\換字{尊}}重}{そん|ちやう}する
\ruby{中}{うち}には
\ruby{{\換字{愉}}快}{ゆ|くわい}がある。
そして
\ruby{何}{なん}の
\ruby{方面}{はう|めん}の
\ruby{事}{こと}でも
\ruby{規律}{き|りつ}は
\ruby{大切}{たい|せつ}だ。
\ruby{{\換字{船}}}{ふね}の
\ruby{中}{うち}などは
\ruby{特}{こと}に
\ruby{然様}{さ|う}だ。
そればかりぢやあ
\ruby{無}{な}い、
\ruby{僕}{ぼく}が
\ruby{私}{ひそか}に
\ruby{思}{おも}ふには、
\ruby[g]{身體}{からだ}を
\ruby{扱}{あつか}ふのに
\ruby{規律}{き|りつ}が
\ruby{無}{な}いと
\ruby[g]{身體}{からだ}が
\ruby{衰}{おとろ}へる、
\ruby{心}{こゝろ}を
\ruby{扱}{あつか}ふにも
\ruby{規律}{き|りつ}が
\ruby{無}{な}いと
\ruby{心}{こゝろ}が
\ruby{歪}{ゆが}んで、そこで
\ruby{戀愛}{れん|あい}などゝいふものに
\ruby{取}{と}り
\ruby{憑}{つ}かれるのだ。
』

と
\ruby{云}{い}ひながら
\ruby{徐}{しづか}に
\ruby{酒盃}{さか|づき}を
\ruby{受}{う}くれば、
\ruby[g]{日方}{ひかた}は

『
\ruby{確論}{かく|ろん}、
\ruby{確論}{かく|ろん}。
』

と
\ruby{{\換字{悅}}}{よろこ}び
\ruby{叫}{さけ}んで、
\ruby{自}{みづか}ら
\ruby{{\換字{酌}}}{しやく}を
\ruby{仕}{し}て
\ruby{{\換字{遣}}}{や}らんと
\ruby{徳利}{とく|り}を
\ruby{擧}{あ}ぐれば、
\ruby{既}{はや}
\ruby{{\換字{飲}}}{の}み
\ruby{盡}{つく}して
\ruby{二三滴}{に|さん|てき}のみ。
\ruby{山瀬}{やま|せ}は
\ruby{急}{いそ}ぎ
\ruby{手}{て}を
\ruby{拍}{たゝ}き
\ruby{立}{た}つ。

\ruby{此時}{この|とき}までにや〳〵と
\ruby{笑}{わら}ひながら、
\ruby{人々}{ひと|〴〵}の
\ruby{談}{はなし}をのみ
\ruby{聞}{き}き
\ruby{居}{ゐ}たりし
\ruby{布袋肥胖}{ほ|てい|ぶ|と}りに
\ruby{肥}{ふと}つたる、
\ruby{丸{\換字{顔}}}{まる|がほ}の
\ruby{眼下}{め|さが}りなる
\ruby{島木}{しま|き}は
\ruby{笑}{わら}つて、

『ハヽヽ、
\ruby{談話}{はな|し}が
\ruby{惡}{わる}つ
\ruby{固}{かた}いから
\ruby{堪}{たま}りやあ
\ruby{仕無}{し|な}い。
\ruby{婢}{をんな}だつて
\ruby{何}{なん}だつて
\ruby{逃}{に}げたつきりだ。
\ruby{徳利}{とつ|くり}の
\ruby{番兵}{ばん|ぺい}は
\ruby{野暮}{や|ぼ}ぢやあ
\ruby{使}{つか}へ
\ruby{無}{ね}えからな。
ハヽヽ、
\ruby{何}{なん}だい。
\ruby{規律}{き|りつ}が
\ruby{無}{な}いといけ
\ruby{無}{な}いつて?
\ruby{戯談}{じやう|だん}
\ruby{言}{い}つちやあいけない、
\ruby{舞臺}{ぶ|たい}に
\ruby{障}{さは}るぜ。
\ruby{不規律}{ふ|き|りつ}の
\ruby{大將}{たい|しやう}、
\ruby[g]{實業家{\換字{兼}}{\換字{虛}}業家}{じつげふかけんきよげふか}、
\ruby{相場師}{さう|ば|し}になつたつて、
\ruby[g]{一同}{みんな}に
\ruby{怒}{おこ}られた、
\ruby{御利{\換字{益}}}{ご|り|やく}は
\ruby{未}{ま}だ
\ruby{蒙}{かうむ}ら
\ruby{無}{な}いが
\ruby{拝金宗}{はい|きん|しう}の
\ruby{信徒}{しん|と}の、
\ruby{島木萬五郎様}{しま|き|まん|ご|らう|さま}が
\ruby{此處}{こ|こ}に
\ruby{御坐}{お|いで}なさるぜ。
\ruby{憚}{はゞか}りながら
\ruby{乃公}{お|れ}が
\ruby{何時}{い|つ}
\ruby{戀愛}{れん|あい}に
\ruby{取}{と}り
\ruby{憑}{つ}かれた。
ハヽヽ、
\ruby{其}{そ}りやあ
\ruby{左様}{さ|う}と
\ruby{水野}{みづ|の}の
\ruby{談}{はなし}は
\ruby{譯}{わけ}
\ruby{有}{あ}つて
\ruby{一番}{いち|ばん}
\ruby{乃公}{お|れ}が
\ruby{知}{し}つている。
どうも
\ruby[g]{一同}{みんな}が
\ruby{氣}{き}に
\ruby{仕}{し}て
\ruby{居}{ゐ}る。
\ruby{{\換字{羽}}{\換字{勝}}}{は|がち}の
\ruby{腹}{はら}の
\ruby{中}{なか}では
\ruby{取}{と}り
\ruby{分}{わ}け
\ruby{深}{ふか}く
\ruby{心配}{しん|ぱい}して
\ruby{居}{ゐ}るやうすだから
\ruby{話}{はな}して
\ruby{聞}{き}かさうか。
』

と、
\ruby{始}{はじめ}は
\ruby{戯}{たはむ}れ、
\ruby{終}{をわり}は
\ruby{眞面目}{ま|じ|め}に
\ruby{云}{い}ひ
\ruby{出}{い}づれば、
\ruby{謹聽}{きん|ちやう}の
\ruby{聲}{こゑ}は
\ruby{異口一齊}{い|く|いつ|せい}に
\ruby{出}{い}でぬ。


\Entry{其三}

\ruby{島木}{しま|き}は
\ruby{驕}{おご}れるにもあらず
\ruby{慢}{あなど}れるにもあらず、たゞ\換字{志}たゝかなる
\ruby[g]{放肆兒}{だゞつこ}の、
\ruby{一家}{いつ|か}の
\ruby{長者}{ちやう|じや}をもはゞからずして、
\ruby{自己}{お|の}の
\ruby{勝手}{かつ|て}に
\ruby{泣}{な}きも
\ruby{笑}{わら}ひもするやうに、\換字{志}かも
\ruby{其}{そ}の
\ruby[g]{小兒}{こども}らしき
\ruby{{\換字{顔}}}{かほ}に
\ruby{微笑}{ゑ|み}をうかめて、

『ハヽヽ、
\ruby[g]{日方}{ひかた}までが
\ruby{謹聽}{きん|ちやう}と
\ruby{吐}{ぬ}かし
\ruby{居}{を}つたな!。
\ruby{一體}{いつ|たい }
\ruby{汝}{きさま}は
\ruby{人}{ひと}は
\ruby{好}{よ}いが、
\ruby{我}{が}ばかり
\ruby{{\換字{強}}}{つよ}くつて
\ruby{思}{おも}ひ
\ruby{{\換字{遣}}}{や}りが
\ruby{足}{た}らない。
\ruby{此}{こ}の
\ruby{思}{おも}ひ
\ruby{{\換字{遣}}}{や}りの
\ruby{足}{た}らない
\ruby{手合}{て|あひ}が、
\ruby{他人}{た|にん}の
\ruby{戀愛}{れん|あい}の
\ruby{談}{はなし}などには、
\ruby{兎角}{と|かく}に
\ruby{點頭}{がつ|てん}しかねるものだ。
\ruby{線}{せん}の
\ruby{無}{な}い
\ruby{家}{うち}にやあ
\ruby{電話}{でん|わ}は
\ruby{{\換字{通}}}{つう}じない、
\ruby{思}{おも}ひ
\ruby{{\換字{遣}}}{や}りの
\ruby{足}{た}らない
\ruby{奴等}{やつ|ら}にやあ
\ruby{戀愛}{れん|あい}は
\ruby{解}{げ}せない。
そこへ
\ruby{行}{い}つちやあ
\ruby{乃公}{お|れ}なんぞは、
\ruby{身}{み}に
\ruby[g]{經驗}{おぼえ}があつて
\ruby{同{\換字{情}}}{おもひ|やり}が
\ruby{{\換字{強}}}{つよ}いから、ツーと
\ruby{云}{い}やあカーと
\ruby{合點}{が|てん}がいくので、
\ruby{初心}{う|ぶ}な
\ruby{水野}{みづ|の}の
\ruby{譚}{はなし}なんざあ、
\ruby{何程}{いく|ら}
\ruby{彼}{あれ}が
\ruby{心}{こゝろ}の
\ruby{奥}{おく}に
\ruby{秘}{かく}して
\ruby{居}{お}つても、
\ruby{深}{ふか}い
\ruby{井}{いど}の
\ruby{床}{そこ}を
\ruby{鏡}{かゞみ}で
\ruby{照}{て}らして、
\ruby{見}{み}て
\ruby{取}{と}るやうに
\ruby{譯}{わけ}も
\ruby{無}{な}く
\ruby{見抜}{み|ぬ}く。
\ruby{本來}{ほん|らい}
\ruby{戀}{こひ}といふ
\ruby{事}{こと}が
\ruby{罪惡}{つ|み}ぢやあ
\ruby{有}{あ}るまいし、
\ruby[g]{日方}{ひかた}のやうな
\ruby{暴論}{ばう|ろん}の
\ruby{愚論}{ぐ|ろん}・・・』

と
\ruby{云}{い}ひかくる
\ruby{時}{とき}
\ruby[g]{日方}{ひかた}は
\ruby{堪}{こら}へず、

『
\ruby{何}{なん}だ、
\ruby{暴論}{ばう|ろん}だと!。
こりやあ
\ruby{怪}{け}しからん。
\ruby{汝}{きさま}も
\ruby{戀愛}{れん|あい}の
\ruby{奴隷}{ど|れい}
\ruby{臭}{くさ}いぞ。
\ruby{身}{み}に
\ruby[g]{經驗}{おぼえ}があつてとは
\ruby{何}{なん}たる
\ruby{囈語}{ね|ごと}だ。
\ruby{聞}{き}きぐるしいことを
\ruby{吐}{ぬか}さずとも、さつさと
\ruby{水野}{みづ|の}のことを
\ruby{話}{はな}すが
\ruby{可}{よ}い。
』

と
\ruby{怒鳴}{ど|な}りつくれば、
\ruby[g]{此方}{こなた}はいよ〳〵
\ruby{笑}{わら}い
\ruby{傾}{かたむ}き、

『
\ruby{安心}{あん|しん}しろ
\ruby[g]{日方}{ひかた}!。
\ruby{乃公}{お|ら}あ
\ruby{女}{をんな}に
\ruby{惚}{ほ}れて
\ruby{戀}{こひ}はおぼえねえ。
ヘン
\ruby{惚}{ほ}れられて
\ruby{惚}{ほ}れられて
\ruby{戀}{こひ}といふものは
\ruby{此様}{こ|ん}なものかと
\ruby{知}{し}つたんだからナ。
アハヽヽヽ、
\ruby{何様}{ど|う}だい
\ruby{奴}{やつこ}さん、
\ruby{如何}{い|かゞ}でござる!。
そこで
\ruby{惚}{ほ}れられて
\ruby{惚}{ほ}れられて
\ruby{悟}{さと}つて
\ruby{見}{み}ると、
\ruby{水野}{みづ|の}を
\ruby{辯護}{べん|ご}するといふ
\ruby{譯}{わけ}ぢやあ
\ruby{無}{な}いが、
\ruby{戀}{こひ}は
\ruby{人間}{ひ|と}の
\ruby{{\換字{情}}}{じやう}の
\ruby{自然}{し|ぜん}の
\ruby[g]{發動}{うごき}で、
\ruby{何}{なに}も
\ruby{咎}{とが}め
\ruby{立}{だ}てをすることは
\ruby{有}{あ}りやしない。
\ruby[g]{日方}{ひかた}にやあ
\ruby[g]{日方}{ひかた}だけの
\ruby{愚論}{ぐ|ろん}もあらうが、
\ruby{乃公}{お|ら}あ
\ruby{戀}{こひ}に
\ruby{{\換字{迷}}}{まよ}つた
\ruby{彼}{あ}の
\ruby{水野}{みづ|の}を、
\ruby[g]{憫然}{かはいさう}だたあ
\ruby{思}{おも}ふが
\ruby{惡}{にく}かあ
\ruby{思}{おも}はねえ。
』

と
\ruby{云}{い}はせも
\ruby{果}{は}てず
\ruby[g]{日方}{ひかた}は
\ruby{目}{め}を
\ruby{剥}{む}き、

『
\ruby{馬鹿野郎}{ば|か|や|らう}ッ。
』

と
\ruby{烈}{はげ}しく
\ruby{罵}{のゝ}しつたる
\ruby{裂帛}{れつ|ぱく}の
\ruby{一聲}{いつ|せい}に
\ruby{氣合}{き|あひ}
\ruby{籠}{こも}つて、
\ruby{人}{ひと}の
\ruby{肺腑}{はい|ふ}に
\ruby{響}{ひゞ}き
\ruby{徹}{てつ}したり。

『マァ
\ruby{待}{ま}ち
\ruby{玉}{たま}へ。
』

『
\ruby{爭}{あらそ}つちやいかん。
』

と、
\ruby{口}{くち}を
\ruby{衝}{つ}いて
\ruby{出}{い}でたる
\ruby{山瀬}{やま|せ}
\ruby[g]{{\換字{羽}\換字{勝}}}{はがち}の
\ruby{二人}{に|にん}の
\ruby{言葉}{こと|ば}は
\ruby{一句}{いつ|く}と
\ruby{一句}{いつ|く}と
\ruby{斷}{き}るゝ
\ruby{間}{ひま}
\ruby{無}{な}く
\ruby{巧}{たくみ}に
\ruby{續}{つゞ}きて、
\ruby[g]{突差}{とつさ}に
\ruby{緊}{きび}しく
\ruby{制}{せい}し
\ruby{止}{と}むれば、
\ruby[g]{流石}{さすが}に
\ruby[g]{日方}{ひかた}も
\ruby[g]{{\換字{羽}\換字{勝}}}{はがち}を
\ruby{憚}{はゞか}りて、
\ruby{言}{ものい}はんとして
\ruby{言}{い}はず
\ruby{已}{や}みけるが、
\ruby{眼}{め}には
\ruby{{\換字{猶}}}{なほ}
\ruby{稜角}{か|ど}を
\ruby{立}{た}てゝ
\ruby{島木}{しま|き}を
\ruby{睨}{にら}み、
\ruby{此}{こ}の
\ruby{時}{とき}
\ruby{遲}{おそ}く
\ruby{彼}{か}の
\ruby{時速}{とき|はや}く、

『そら
\ruby{{\換字{叉}}}{また}
\ruby{馬鹿野郎}{ば|か|や|らう}が
\ruby{御來臨}{お|い|で}なすつた。
ハヽヽ、
\ruby[g]{何程}{いくら }
\ruby{罵}{のヽし}られても
\ruby{相手}{あひ|て}にはならねえ。
\ruby{汝}{きさま}は
\ruby{乃公}{お|れ}に
\ruby{楯}{たて}をついても、
\ruby{乃公}{お|ら}あ
\ruby{汝}{きさま}を
\ruby{生呑}{まる|のみ}に
\ruby{吞}{の}んでゝ、そして
\ruby{腹}{はら}にも
\ruby{障}{さは}らねえから。
』

と、
\ruby{島木}{しま|き}の
\ruby{冷}{ひや}やかに
\ruby{一矢}{いつ|し}
\ruby{酬}{むく}ゆるに、

『
\ruby{何}{なん}だ、
\ruby{吞}{の}んで
\ruby{居}{ゐ}る。
\ruby{可矣}{よ|し}ツ、
\ruby{吞}{の}まれたつて
\ruby{鐵釘}{かな|くぎ}が
\ruby{何}{なん}となる!
\ruby{曲}{まが}りも
\ruby{仕無}{し|な}いは!。
\ruby{丸}{まる}くもならんは!。
』

と、
\ruby[g]{日方}{ひかた}は
\ruby{{\換字{叉}}}{また }
\ruby{直}{ただち}に
\ruby{熱}{ねつ}して
\ruby{答}{こた}ふ。

\ruby{悠然}{いう|ぜん}と
\ruby{笑}{ゑみ}を
\ruby{含}{ふく}める
\ruby[g]{{\換字{羽}\換字{勝}}}{はがち}は
\ruby{靜}{しづ}かに、

『
\ruby{可}{い}いさ、
\ruby[g]{二人}{ふたり}とも、もう
\ruby{可}{い}いさ。
ハヽヽ、
\ruby{互}{たがひ}に
\ruby{其}{そ}の
\ruby{位}{くらゐ}
\ruby{威張}{ ゐ|ば}つたら
\ruby{可}{い}いぢあ
\ruby{無}{な}いか。
\ruby{島木}{しま|き}は
\ruby[g]{日方}{ひかた}に
\ruby{關}{かま}はないで
\ruby{僕}{ぼく}に
\ruby{話}{はな}すつもりで
\ruby{話}{はな}して
\ruby{{\換字{呉}}}{く}れ
\ruby{玉}{たま}へ。
\ruby[g]{日方}{ひかた}はまた
\ruby{島木}{しま|き}に
\ruby{關}{かま}はないで
\ruby{僕}{ぼく}に
\ruby[g]{交際}{つきあ}つて
\ruby{聞}{きい}て
\ruby{居}{い}て
\ruby{{\換字{呉}}}{く}れ
\ruby{玉}{たま}へな。
つまりお
\ruby{互}{たがひ}に
\ruby{水野}{みづ|の}の
\ruby{上}{うへ}が
\ruby{知}{し}りたいのだからネ。
』

と、
\ruby{優}{やさ}しく
\ruby{制}{せい}すれば、

『ャ、
\ruby{濟}{す}まなかつた、
\ruby{僕}{ぼく}が
\ruby{惡}{わる}かつた。
』

『ァ、
\ruby{左様}{さ|う}
\ruby{云}{い}はれりやあ
\ruby{乃公}{お|れ}も
\ruby{下}{くだ}らなかつた。
』

と
\ruby[g]{日方}{ひかた}も
\ruby{島木}{しま|き}も
\ruby{爭}{あらそ}ひ
\ruby{止}{や}みて、
\ruby{誰}{たれ}
\ruby{勸}{すヽ}めねど
\ruby{同}{おな}じ
\ruby{思}{おも}ひの、
\ruby{双方一時}{さう|はう|いち|じ}に
\ruby[g]{酒盃}{さかづき}を
\ruby{交}{かは}して、
\ruby{笑}{わら}つて
\ruby{仕舞}{し|ま}つて
\ruby{痕跡}{あと|かた}もなし。

\ruby{島木}{しま|き}は
\ruby{此度}{こ|たび}はやゝ
\ruby{眞面目}{ま|じ|め}に、
\ruby[g]{{\換字{羽}\換字{勝}}}{はがち}の
\ruby{方}{かた}に
\ruby{向}{むか}つて
\ruby{語}{かた}り
\ruby{出}{だ}したり。

『
\ruby[g]{一同}{みんな}も
\ruby{知}{し}っている
\ruby{{\換字{通}}}{とほ}り
\ruby{彼}{あ}の
\ruby{水野}{みづ|の}は、
\ruby{我等}{おれ|たち}の
\ruby{中}{なか}では
\ruby{一番年下}{いち|ばん|とし|した}、
\ruby{乃公}{お|れ}が
\ruby{今年}{こ|とし}は
\ruby{二十七}{に|じう|しち}だから、
\ruby{七}{しち}、
\ruby{六}{ろく}、
\ruby{五}{ご}、
\ruby{四}{し}と
\ruby{四}{よつ}つ
\ruby{目}{め}で
\ruby{丁度}{ちやう|ど}
\ruby{二十四}{に|じう|し}だ。
\ruby[g]{宇都宮}{みや}から
\ruby{東京}{とう|きやう}へ
\ruby{上}{のぼ}る
\ruby{時}{とき}にも、
\ruby{一番先}{いち|ばん|さき}へ
\ruby{出}{で}たのは
\ruby[g]{{\換字{羽}\換字{勝}}}{はがち}だつたが、
\ruby{一番後}{いち|ばん|あと}へ
\ruby{殘}{のこ}つたのは
\ruby{水野}{みづ|の}だつた。
\ruby{若}{わか}いに
\ruby{似合}{に|あ}はず
\ruby{能}{よ}く
\ruby{出來}{で|き}たから、
\ruby{君}{きみ}は
\ruby{若}{わか}いけれども
\ruby{學業}{わ|ざ}が
\ruby{出來}{で|き}る、
\ruby{早}{はや}く
\ruby{東京}{とう|きやう}へ
\ruby{出}{で}て
\ruby{身}{み}を
\ruby{立}{た}てるが
\ruby{可}{い}いと、
\ruby{勸}{すゝ}めたのは
\ruby{乃公}{お|れ}
\ruby{一人}{ひと|り}で
\ruby{無}{な}かつたが、いや
\ruby{小生}{わた|くし}の
\ruby{志}{こヽろざ}すところは
\ruby{些}{ちと}
\ruby{{\換字{違}}}{ちが}ふから、
\ruby{左様}{さ|う}
\ruby{急}{いそ}がないでも
\ruby{可}{い}い
\ruby{事}{こと}だ、
\ruby{他}{ほか}の
\ruby{人}{ひと}は
\ruby{一日遲}{いち|にち|おそ}ければ
\ruby{一日損}{いち|にち|そん}、
\ruby{少}{すこ}しも
\ruby{疾}{はや}く
\ruby{上京}{じやう|きやう}するが
\ruby{可}{い}い、と
\ruby{妙}{めう}に
\ruby{片意地}{かた|い|ぢ}に
\ruby{謙遜}{けん|そん}して
\ruby{出}{で}ず。
\ruby{二番}{に|ばん}に
\ruby{出}{で}たが
\ruby[g]{日方}{ひかた}
\ruby{山瀬}{やま|せ}、それから
\ruby{名倉}{な|ぐら}、それから
\ruby{楢井}{なら|い}、それから
\ruby{乃公}{お|れ}で、
\ruby{其後}{その|あと}から
\ruby{漸}{やつ}と
\ruby{上京}{じやう|きやう}した。
\ruby{其}{そ}の
\ruby{位}{くらゐ}\ %空白有り
\ruby{異}{おつ}に
\ruby{固}{かた}いところのある
\ruby{男}{をとこ}で、
\ruby{東京}{とう|きやう}へ
\ruby{出}{で}てからも
\ruby[g]{一同}{みんな}は
\ruby{誰}{たれ}しも、
\ruby{身}{み}を
\ruby{立}{た}てる
\ruby{{\換字{道}}}{みち}に
\ruby{汲々}{きふ|〳〵}として、
\ruby{隨分}{ずゐ|ぶん}
\ruby{骨}{ほね}を
\ruby{折}{を}つてそれ〴〵に、
\ruby{辛}{から}く
\ruby{出世}{しゆ|つせ}も
\ruby{仕}{し}て
\ruby{來}{き}たに、
\ruby{彼}{あ}の
\ruby{男}{をとこ}ばかりは
\ruby{澄}{す}ましかへつて、
\ruby{今}{いま}でも
\ruby{小學{\換字{教}}師}{せう|がく|けう|し}で
\ruby{甘}{あま}んじて
\ruby{居}{お}る。
それで
\ruby{惰}{なま}けて
\ruby{居}{を}るのかと
\ruby{思}{おも}へば、
\ruby{一寸}{いつ|すん}の
\ruby{暇}{ひま}も
\ruby{惜}{をし}んで
\ruby{勉強}{べん|きやう}して、あらゆる
\ruby{方面}{はう|めん}に
\ruby{行}{ゆ}き
\ruby{渡}{わた}つて
\ruby{居}{ゐ}る。
\ruby{僕}{ぼく}は
\ruby{一生}{いつ|しやう}をかけて
\ruby{此}{こ}の
\ruby{世}{よ}の
\ruby{中}{なか}に、たゞ
\ruby{一篇}{いつ|ぺん}の
\ruby{詩}{し}を
\ruby{留}{とゞ}めれば
\ruby{可}{い}いのだ。
\ruby{今}{いま}は
\ruby{其}{そ}の
\ruby{準備}{よう|い}に
\ruby{勤}{つと}めて
\ruby{居}{ゐ}るので、
\ruby{他}{ほか}に
\ruby{慾}{よく}も
\ruby{無}{な}ければ
\ruby{望}{のぞみ}も
\ruby{無}{な}い、
\ruby{{\換字{半}}熟}{なま|にえ}なものを
\ruby{世}{よ}に
\ruby{出}{だ}して、
\ruby{今}{いま}っから
\ruby{文人顏}{ぶん|じん|がほ}するのも
\ruby{羞}{はづ}かしいから、もう
\ruby{十年}{じう|ねん}ばかりは
\ruby{小學讀本}{と|く|ほ|ん}いぢりで、たゞ〳〵
\ruby{勉{\換字{強}}}{べん|きやう}をするつもりだ、と
\ruby{隱君子氣質}{いん|くん|し|かた|ぎ}で
\ruby{日}{ひ}を
\ruby{經}{へ}て
\ruby{居}{ゐ}たのは、
\ruby[g]{{\換字{羽}\換字{勝}}}{はがち}はじめ
\ruby[g]{一同}{みんな}も
\ruby{知}{し}つて
\ruby{居}{ゐ}やう。
ところで
\ruby{此}{こ}の
\ruby{乃公}{お|れ}は
\ruby{金}{かね}まうけ
\ruby[g]{主義}{しゆぎ}、
\ruby{卑}{いや}しいと
\ruby{云}{い}つて
\ruby[g]{一同}{みんな}に
\ruby{罵}{のゝし}られた
\ruby{位}{くらゐ}だから、
\ruby{守}{まも}るところのある
\ruby{浪人肌}{らう|にん|はだ}の、
\ruby{水野}{みづ|の}と
\ruby{氣}{き}の
\ruby{合}{あ}ふ
\ruby{譯}{わけ}は
\ruby{毫}{ちつと}も
\ruby{無}{な}いが、
\ruby{他}{ほか}の
\ruby{五人}{ご|にん}は
\ruby{上京}{じやう|きやう}して、
\ruby[g]{二人}{ふたり}だけ
\ruby{宮}{みや}に
\ruby{殘}{のこ}つた
\ruby{時}{とき}、
\ruby{彼}{あれ}が
\ruby{熱}{ねつ}を
\ruby{病}{や}んだのを
\ruby{介抱}{かい|はう}して、
\ruby{長}{なが}い
\ruby[g]{看護}{みとり}を
\ruby{爲}{し}て
\ruby{{\換字{遣}}}{や}つた、
\ruby{其事}{そ|れ}が
\ruby{{\換字{鎖}}}{くさり}になつて
\ruby[g]{此地}{こつち}へ
\ruby{來}{き}ても、
\ruby{取}{と}り
\ruby{分}{わ}け
\ruby[g]{二人}{ふたり}は
\ruby{親}{した}しく
\ruby{仕}{し}て
\ruby{居}{ゐ}た。
\換字{志}かし
\ruby{乃公}{お|ら}あ
\ruby{俗物}{ぞく|ぶつ}、
\ruby{水野}{みづ|の}は
\ruby{仙骨}{せん|こつ}、
\ruby[g]{此方}{こつち}は
\ruby{飛}{と}んだり
\ruby{跳}{はね}たりして
\ruby{悶躁}{も|が}いて
\ruby{居}{ゐ}るので、
\ruby{中々}{なか|〳〵}
\ruby{往來}{ゆき|き}することも
\ruby{多}{おほ}くは
\ruby{無}{な}かつた。
さあ
\ruby{此處}{こ|こ}で
\ruby{白狀}{はく|じやう}
\ruby{仕}{し}しなけりやならないが、
\ruby{丁度}{ちやう|ど}
\ruby[g]{一昨年}{をとヽし}の
\ruby{暮}{くれ}だつた。
\ruby{實}{じつ}は
\ruby{此}{こ}の
\ruby{乃公}{お|れ}が
\ruby{山氣}{やま|ぎ}に
\ruby{{\換字{逸}}}{はや}つて、
\ruby{危}{あぶ}ない
\ruby{橋}{はし}を
\ruby{渡}{わた}る
\ruby{輕業}{かる|わざ}をやつたところ、
\ruby{{\換字{運}}}{うん}が
\ruby{惡}{わる}くつて
\ruby{可厭}{い|や}な
\ruby{目}{め}が
\ruby{出}{で}て、
\ruby{甘}{うま}く
\ruby{行}{い}きあ
\ruby{論}{ろん}はないことが
\ruby{打壞}{ぶつ|こわ}れたんで、たつた
\ruby{五十}{ご|じう}
\ruby{兩}{りやう}ばかりの
\ruby{有無}{ある|なし}で
\ruby{何様}{ど|う}にも
\ruby{仕切}{し|き}れない
\ruby{機會}{は|め}へ
\ruby{臨}{のぞ}んだ。
そも〳〵
\ruby{投機}{や|ま}を
\ruby{始}{はじ}めた
\ruby{其}{そ}の
\ruby{時}{とき}から、
\ruby{乃公}{お|ら}あ
\ruby{危}{あぶな}い
\ruby{事}{こと}をする
\ruby{代}{かは}りにやあ、
\ruby{乃公}{お|れ}が
\ruby{一六}{いち|ろく}
\ruby{沙汰}{ざ|た}を
\ruby{廢}{や}めぬ
\ruby{内}{うち}は、
\ruby{金錢}{きん|せん}に
\ruby{關}{かヽ}つた
\ruby{事}{こと}では
\ruby{決}{けつ}して
\ruby[g]{一同}{みんな}に、
\ruby{苦勞}{く|らう}は
\ruby{掛}{か}けぬと
\ruby{誓言}{ちか|ひ}を
\ruby{立}{た}つた
\ruby{表}{おもて}があるから
\ruby{誰}{だれ}にも
\ruby{云}{い}へず
\ruby{思案}{し|あん}に
\ruby{餘}{あま}つて
\ruby{獨語}{ひとり|ごと}のやうに、
\ruby{其譯}{その|わけ}を
\ruby{水野}{みづ|の}に
\ruby{話}{はな}して
\ruby{見}{み}ると、
\ruby{手箱}{て|ばこ}の
\ruby{底}{そこ}から
\ruby{書}{か}いたものを
\ruby{出}{だ}して、
\ruby{此}{これ}を
\ruby{山瀬君}{やま|せ|くん}に
\ruby{頼}{たの}んで
\ruby{賣}{う}つて
\ruby{貰}{もら}つたら、
\ruby{其位}{その|くらゐ}の
\ruby{金}{かね}は
\ruby{出來}{で|き}るか
\ruby{知}{し}れぬ、
\ruby{出來}{で|き}たら
\ruby{使}{つか}ひ
\ruby{玉}{たま}へといふ
\ruby{話}{はなし}。
\ruby{當}{あて}にはならないと
\ruby{思}{おも}つたが、
\ruby{山瀬}{やま|せ}に
\ruby{頼}{たの}むと
\ruby{其事}{そ|れ}が
\ruby{出來}{で|き}て、そこで
\ruby{大}{おほき}に
\ruby{助}{たす}かつた。
\ruby{其}{そ}の
\ruby{味}{あじ}を
\ruby{占}{し}めたといふのでは
\ruby{無}{な}いが、
\ruby{其}{そ}の
\ruby{後}{のち}も
\ruby{種子}{た|ね}を
\ruby{耗}{す}つた
\ruby{其時}{その|とき}は、三
\ruby{度}{ど}といふもの
\ruby{助}{たす}けて
\ruby{貰}{もら}つて、
\ruby{矢種}{や|だね}をつぎ〳〵
\ruby{戦}{たヽか}つた
\ruby{末}{すゑ}、どうやら
\ruby{{\換字{遣}}}{や}つて
\ruby{行}{い}かれる
\ruby[g]{身體}{からだ}になつた。
そこで
\ruby{水野}{みづ|の}に
\ruby{對}{むか}つて
\ruby{乃公}{お|れ}がいふには、
\ruby{貰}{もら}つたものを
\ruby{{\換字{返}}}{かへ}さうとは
\ruby{云}{い}はないが、
\ruby{金}{かね}が
\ruby{要}{い}る
\ruby{時}{とき}は
\ruby{何時}{い|つ}でも
\ruby{云}{い}ひたまへ、
\ruby{乃公}{お|れ}が
\ruby{懷中}{ふと|ころ}だけなら
\ruby{洗}{さら}け
\ruby{出}{だ}すから、と
\ruby{此}{こ}の
\ruby{春}{はる}
\ruby{{\換字{遇}}}{あ}つた
\ruby{時}{とき}
\ruby{云}{い}つて
\ruby{置}{お}いた。
ところが
\ruby{金}{かね}を
\ruby{使}{つか}ふ
\ruby{水野}{みづ|の}では
\ruby{無}{な}し、たゞ
\ruby{其限}{それ|ぎり}で
\ruby{濟}{す}んで
\ruby{居}{ゐ}たが、
\ruby{此}{こ}の
\ruby{夏}{なつ}になつて
\ruby{{\換字{遣}}}{や}つて
\ruby{來}{き}て、
\ruby[g]{眞赤}{まつか}な
\ruby{{\換字{顔}}}{かほ}をしてきまり
\ruby{惡}{わる}さうに、三十
\ruby{兩}{りやう}ばかり
\ruby{貸}{か}して
\ruby{{\換字{呉}}}{く}れろ、と
\ruby{云}{い}つたのが
\ruby{最初}{はじ|まり}で
\ruby{其後}{その|のち}も、ぼつり〳〵と
\ruby{持}{も}つて
\ruby{行}{ゆ}く。
\ruby{其事}{そ|れ}が
\ruby{乃公}{お|れ}が
\ruby{勘}{かん}を
\ruby{付}{つ}けたはじまりだつた。


\Entry{其四}

\ruby{考}{かんが}へて
\ruby{見}{み}りやあ
\ruby{合點}{が|てん}がいかない。
\ruby[g]{多分}{たんと}では
\ruby{無}{な}いが
\ruby{給料}{きふ|れう}も
\ruby{取}{と}るし、
\ruby{別}{べつ}に
\ruby{蕩樂}{だう|らく}の
\ruby{無}{な}い
\ruby{男}{をとこ}だから
\ruby{其金}{そ|れ}で
\ruby[g]{一人身}{ひとりみ}の
\ruby{今日}{こん|にち}を
\ruby{濟}{す}ませて、
\ruby{剩餘}{あま|り}で
\ruby{書物}{しよ|もつ}を
\ruby{買}{か}つて
\ruby{讀}{よ}む
\ruby{位}{くらひ}の
\ruby{事}{こと}。
その
\ruby{書物}{しよ|もつ}を
\ruby{買}{か}ふにもたゞは
\ruby{買}{か}はないで、
\ruby{何時}{い|つ}でも
\ruby{讀}{よ}んで
\ruby{了}{しま}つたのを
\ruby{下}{した}に
\ruby{{\換字{遣}}}{や}つて、まだ
\ruby{讀}{よ}まぬものと
\ruby{取換}{とり|か}へる。
それを
\ruby{自分}{じ|ぶん}でも
\ruby[g]{可笑}{をかし}がつて、
\ruby{何}{なん}の
\ruby{事}{こと}は
\ruby{無}{な}い
\ruby{僕}{ぼく}の
\ruby{爲}{す}ることは
\ruby{書肆}{ほん|や}のために、一
\ruby{枚}{まい}一
\ruby{枚}{まい}
\ruby{蠹拂}{むし|はら}ひを
\ruby{叮嚀}{てい|ねい}に
\ruby{仕}{し}て
\ruby{{\換字{遣}}}{や}るやうなものだと
\ruby{云}{い}つて
\ruby{居}{ゐ}た
\ruby{程}{ほど}。
\ruby{併}{しか}し
\ruby{左樣}{さ|う}いふ
\ruby{{\換字{遣}}}{や}り
\ruby{方}{かた}をして
\ruby{少}{すくな}い
\ruby{錢}{ぜに}で
\ruby{多}{おほ}く
\ruby{讀}{よ}む、それだけ
\ruby{始末}{し|まつ}の
\ruby{好}{よ}い
\ruby{賢}{かしこ}い
\ruby{水野}{みづ|の}が、
\ruby{何}{なん}の
\ruby{彼}{か}のと
\ruby{云}{い}つては
\ruby{金}{かね}を
\ruby{持}{も}つて
\ruby{行}{ゆ}く。
ハテ
\ruby{是}{これ}にやあ
\ruby{何}{なん}ぞ
\ruby{仔細}{し|さい}があらう、
\ruby{譯}{わけ}が
\ruby{無}{な}くちやあ
\ruby{要}{い}らない
\ruby{金}{かね}だ。
いくら
\ruby{表面}{うは|べ}は
\ruby{物柔}{もの|やは}らかな
\ruby{君子風}{くん|し|ふう}で、
\ruby{腹}{はら}の
\ruby{底}{そこ}の
\ruby{底}{そこ}にやあ
\ruby{恐}{おそ}ろしい
\ruby{高慢}{かう|まん}、
\ruby{世界中}{せ|かい|ぢゆう}の
\ruby{奴}{やつ}を
\ruby{相手}{あひ|て}にしても、
\ruby{鼻}{はな}の
\ruby{頭}{さき}で
\ruby{笑}{わら}つて
\ruby{居}{ゐ}やうといふ
\ruby{沈毅漢}{しつ|かり|もの}の、
\ruby{彼}{あ}の
\ruby{水野}{みづ|の}でも、
\ruby{年齢}{と|し}やあ
\ruby{年齢}{と|し}だ。
\ruby{桃}{もゝ}の
\ruby{速}{はや}いのも
\ruby{柹}{かき}の
\ruby{遲}{おそ}いのも、いづれ
\ruby{時}{とき}が
\ruby{來}{く}りやあ
\ruby{花}{はな}は
\ruby{{\換字{咲}}}{さ}き出す。
\ruby{才}{さい}はじけたも
\ruby{謹}{つゝし}まやかなも、
\ruby[g]{時{\換字{節}}因緣}{じせついんねん}で
\ruby{{\換字{情}}}{こゝろ}が
\ruby{萌}{も}える。
\ruby{乃公}{お|れ}のやうな
\ruby{早熟}{はや|なり}やあ
\ruby{十七八}{じう|しち|はち}から、
\ruby{白粉}{おし|ろい}や
\ruby{油}{あぶら}の
\ruby{香}{にほひ}に
\ruby{鼻}{はな}もひこつかせたが、
\ruby{其代}{その|かは}り
\ruby{{\換字{浮}}氣}{うは|き}の
\ruby{掛}{か}け
\ruby{流}{なが}しで、
\ruby{笑}{わら}ふのも
\ruby{泣}{な}くのも
\ruby[g]{二日}{ふつか}か
\ruby[g]{三日}{みつか}
\ruby{限}{き}り、
\ruby{思}{おも}ふも
\ruby{思}{おも}はれるも
\ruby{實}{じつ}は
\ruby{無}{な}くつて、のほゝんで
\ruby{今日}{け|ふ}まで
\ruby{無事}{ぶ|じ}に
\ruby{來}{き}たが、
\ruby{水野}{みづ|の}のやうな
\ruby{彼樣}{あ|ん}な
\ruby{男}{をとこ}が、
\ruby{惡}{わる}くすると
\ruby{唯一{\換字{途}}}{たゞ|いち|づ}に
\ruby[g]{純粹}{いつぽんぎ}の、
\ruby[g]{眞正直}{まつしやうぢき}な
\ruby{戀}{こひ}に
\ruby{落}{お}ちて、
\ruby{人}{ひと}にも
\ruby{知}{し}らさず
\ruby{獨}{ひと}り
\ruby{苦}{くる}しみ、
\ruby{思}{おも}ひ
\ruby{詰}{つ}め
\ruby{思}{おも}ひ
\ruby{詰}{つ}めて
\ruby{忘}{わす}れる
\ruby{間}{ま}も
\ruby{無}{な}く、
\ruby{胸}{むね}に
\ruby{解}{と}けかねる
\ruby{凝塊}{し|こり}を
\ruby{出}{で}かして、
\ruby{長}{なが}く〳〵
\ruby{悶}{もだ}へて
\ruby{惱}{なや}むともあるもの。
\ruby{若}{もし}や
\ruby{其樣}{そ|ん}な
\ruby{事}{こと}でゞもあるならば、
\ruby{朋友}{とも|だち}のよしみ、
\ruby{年上}{とし|うへ}の
\ruby{甲斐}{か|ひ}、
\ruby{特}{こと}には
\ruby{誰}{たれ}にも
\ruby{知}{し}らさず
\ruby{内々}{ない|〳〵}で、
\ruby{恩}{おん}を
\ruby{受}{う}けて
\ruby{居}{ゐ}る
\ruby{譯合}{わけ|あひ}もあり、
\ruby{一}{ひ}ト
\ruby{心配}{しん|ぱい}
\ruby{仕無}{し|な}けりあならぬと
\ruby{意}{こゝろ}を
\ruby{定}{さだ}めて、さて
\ruby{其時}{そ|れ}から
\ruby{水野}{みづ|の}の
\ruby{樣子}{やう|す}を
\ruby{見}{み}ると
\ruby{推量}{すゐ|りやう}の
\ruby{{\換字{通}}}{とほ}り。
\ruby{何}{なん}と
\ruby{無}{な}く
\ruby{人}{ひと}に
\ruby[g]{隔心}{へだてごゝろ}がある。
\ruby{何}{なん}と
\ruby{無}{な}くそは〳〵としたところがある。
\ruby[g]{此方}{こつち}から
\ruby{話}{はな}す
\ruby{談}{はなし}には
\ruby{身}{み}を
\ruby{入}{い}れて
\ruby{聞}{き}かぬ。
\ruby{彼}{あれ}が
\ruby{話}{はな}す
\ruby{談}{はなし}には
\ruby{氣焰}{いき|ほひ}が
\ruby{足}{た}らぬ。
\ruby{人}{ひと}と
\ruby{對}{むか}ひあつて
\ruby{坐}{すわ}つて
\ruby{居}{ゐ}ながら、
\ruby{談話}{はな|し}が
\ruby[g]{一寸}{ちよつと}
\ruby{斷}{た}えれば
\ruby{胸}{むね}の
\ruby{中}{なか}では、
\ruby{既}{もう}
\ruby{他方}{よ|そ}の
\ruby{事}{こと}を
\ruby{思}{おも}つて
\ruby{居}{ゐ}る
\ruby{樣子}{やう|す}。
\ruby{將來}{ゆく|すゑ}の
\ruby{希望}{の|ぞみ}は
\ruby{餘}{あま}り
\ruby{言}{い}はずに、やゝもすると
\ruby{{\換字{過}}}{す}ぎた
\ruby{事}{こと}を
\ruby{云}{い}ひ
\ruby{出}{だ}しては、
\ruby{無邪氣}{む|じや|き}だつた
\ruby{往時}{むか|し}をなつかしがる。
\ruby{試}{こゝろ}みに
\ruby{{\換字{浮}}世話}{うき|よ|ばなし}を
\ruby[g]{三種}{みいろ}
\ruby[g]{四種}{よいろ}
\ruby{爲}{し}て、
\ruby{何}{ど}の
\ruby{話}{はなし}が
\ruby{彼}{あれ}の
\ruby{胸}{むね}の
\ruby{中}{うち}と
\ruby{響}{ひゞ}き
\ruby{合}{あ}ふかと、
\ruby{探}{さぐ}つて
\ruby{見}{み}れば
\ruby{全然}{すつ|かり}
\ruby{分}{わか}つて、
\ruby{此}{こ}の
\ruby{絃}{いと}に
\ruby{和}{あ}つて
\ruby{鳴}{な}るのは
\ruby{其}{そ}の
\ruby{絃}{いと}と、
\ruby[g]{{\換字{判}}然}{ちやん}と
\ruby{正體}{しやう|たい}の
\ruby{合點}{が|てん}がいつた。
さあ
\ruby{打棄}{うつ|ちや}つて
\ruby{置}{お}く
\ruby{譯}{わけ}にやあ
\ruby{行}{い}かない。
\ruby{相手}{あひ|て}さへ
\ruby{好}{よ}けりやあ
\ruby{仔細}{し|さい}は
\ruby{無}{な}いこと。
\ruby{南方}{み|なみ}へ
\ruby{枝}{えだ}がさして
\ruby{花}{はな}が
\ruby{{\換字{咲}}}{さ}くに
\ruby{何}{なん}の
\ruby{罪}{つみ}!。
\ruby{人{\換字{情}}}{にん|じやう}の
\ruby{{\換字{温}}{\換字{暖}}}{あつた|かみ}を
\ruby{得}{え}やうとおもつて、
\ruby{若}{わか}い
\ruby{心}{こゝろ}の
\ruby{動}{うご}き
\ruby{出}{だ}すのが
\ruby{何無理}{なに|む|り}だらう!。
\ruby{年齢}{と|し}が
\ruby{年齢}{と|し}だもの、
\ruby{有}{あ}り
\ruby{内}{うち}の
\ruby{事}{こと}だ。
\ruby{然}{しか}し
\ruby{緣}{えん}は
\ruby{異}{い}なもの
\ruby{危}{あぶな}いもの、よもやとは
\ruby{思}{おも}ふけれど、
\ruby{萬}{まん}が一にも、
\ruby{素性}{す|じやう}や
\ruby{筋}{すぢ}の
\ruby{惡}{わる}い
\ruby{女}{をんな}が
\ruby{相手}{あひ|て}だつた
\ruby{日}{ひ}には
\ruby{水野}{みづ|の}の
\ruby{不幸}{ふ|かう}、
\ruby{止}{と}め
\ruby{立}{だて}も
\ruby{爭}{あらそ}ひ
\ruby{立}{だて}も
\ruby{仕無}{し|な}けりやならぬ。
\ruby{金}{かね}の
\ruby{要}{い}るだけに
\ruby{氣}{き}がゝりなところがある。
と
\ruby{思}{おも}つたので
\ruby{乃公}{お|れ}の
\ruby[g]{身體}{からだ}にやあ
\ruby{暇}{ひま}も
\ruby{無}{な}かつたが、
\ruby{或日}{ある|ひ}
\ruby{水野}{みづ|の}の
\ruby{不在}{る|す}を
\ruby{覗}{ねら}つて、
\ruby{水野}{みづ|の}を
\ruby{置}{お}いて
\ruby{世話}{せ|わ}をしている
\ruby{山路}{やま|ぢ}の
\ruby{老夫}{おや|ぢ}を
\ruby{捕}{つかま}へて
\ruby{糺}{ただ}しかけると、
\ruby{彼}{あ}の
\ruby{老夫}{おや|ぢ}も
\ruby{中々}{なか|〳〵}の
\ruby{親切者}{しん|せつ|もの}で、
\ruby{特}{こと}さら
\ruby{水野}{みづ|の}の
\ruby{平生}{ひ|ごろ}の
\ruby{品行}{み|もち}に
\ruby{惚}{ほ}れて
\ruby{居}{ゐ}るので、
\ruby{實}{じつ}は
\ruby[g]{水野樣}{みづのさん}の
\ruby{御利益}{お|た|め}を
\ruby{思}{おも}つて、
\ruby[g]{貴下}{あなた}でも
\ruby{御来臨}{お|い|で}になつたら
\ruby{申}{まを}し
\ruby{上}{あ}げたいと、
\ruby{内々}{ない|〳〵}
\ruby{願}{ねが}つて
\ruby{居}{ゐ}たところでござりました、といふので
\ruby{一切}{いつ|さい}の
\ruby{事{\換字{情}}}{じ|じやう}は
\ruby{老夫}{おや|ぢ}の
\ruby{口}{くち}から
\ruby{知}{し}れた。


\Entry{其五}

\ruby[g]{老夫}{おやぢ}の
\ruby[g]{談話}{はなし}を
\ruby{聞}{き}いて
\ruby{見}{み}りやあ
\ruby{水野}{みづ|の}は
\ruby{實}{じつ}に
\ruby[g]{憫然}{かはいさう}だ。
\ruby{勿論}{もち|ろん}
\ruby{其}{そ}の
\ruby[g]{老夫}{おやぢ}の
\ruby{云}{い}ったことが
\ruby{一}{いち}から
\ruby{十}{じう}まで
\ruby[g]{眞實}{ほんと}とも
\ruby{限}{かぎ}るまいが、
\ruby{岡目}{をか|め}の
\ruby{{\換字{評}}{\換字{判}}}{ひやう|ばん}なり
\ruby[g]{老夫}{としより}の
\ruby{言葉}{こと|ば}なり、
\ruby{大體}{だい|たい}は
\ruby{{\換字{違}}}{ちが}ふ
\ruby{氣{\換字{遣}}}{きづ|かひ}はあるまい。
そも〳〵は
\ruby[g]{今年}{ことし}の
\ruby{春}{はる}の
\ruby{始}{はじめ}、
\ruby{水野}{みづ|の}の
\ruby{出}{で}て
\ruby{居}{ゐ}る
\ruby{學校}{がく|かう}の
\ruby{女{\換字{教}}師}{ぢよ|けう|し}が
\ruby[g]{一人}{ひとり}
\ruby{故鄕}{く|に}へ
\ruby{歸}{かへ}つたので
\ruby{闕員}{けつ|いん}が
\ruby{出來}{で|き}た、
\ruby{其}{そ}の
\ruby{補闕}{ほ|けつ}として
\ruby{新}{あらた}に
\ruby{來}{き}たのが、まだ
\ruby{{\換字{教}}員}{けう|いん}になりたての、
\ruby{年}{とし}の
\ruby{若}{わか}い
\ruby{岩崎五十子}{いわ|さき|い|そ|こ}といふ
\ruby{女}{をんな}だつた。
\ruby[g]{老夫}{おやぢ}も
\ruby{度々}{たび|たび}
\ruby{見}{み}て
\ruby{知}{し}つているさうだが、
\ruby{極}{ごく}
\ruby{可愛}{か|あい}らしい
\ruby{惚}{ほ}れ〴〵するといふやうな
\ruby{{\換字{顔}}立}{かほ|だち}では
\ruby{無}{な}いけれど、
\ruby{眼}{め}の
\ruby{{\換字{清}}}{すゞ}しい
\ruby{鼻}{はな}の
\ruby{高}{たか}い
\ruby[g]{端然}{しやん}とした
\ruby{女}{をんな}で、まあ
\ruby{當世}{たう|せい}の
\ruby[g]{下司根性}{げすこんじやう}から
\ruby{云}{い}へば、あれだけの
\ruby[g]{容貌}{きりやう}をもつて
\ruby{居}{ゐ}ながら、
\ruby{何}{なん}だつて
\ruby{{\換字{教}}師}{けう|し}なんぞになつて
\ruby{居}{ゐ}るだらう、と
\ruby{蔭口}{かげ|ぐち}も
\ruby{云}{い}はれ
\ruby{{\換字{兼}}}{かね}ない
\ruby{女}{をんな}ぶりださうさ。
\換字{志}かも、
\ruby[g]{容貌}{きりやう}の
\ruby{佳}{よ}い
\ruby{奴}{やつ}は
\ruby[g]{十人}{じうにん}が
\ruby[g]{八人}{はちにん}まで、
\ruby{兎角}{と|かく}
\ruby{他人}{た|にん}に
\ruby{甘}{あま}つたれるやうな
\ruby{調子}{てふ|し}があつて、
\ruby{學問}{がく|もん}なぞは
\ruby{得}{え}て
\ruby{出來}{で|き}ないが、
\ruby{中々}{なか|〳〵}
\ruby{其女}{その|をんな}は
\ruby{能}{よ}く
\ruby{出來}{で|き}る
\ruby{上}{うへ}、それこそ
\ruby[g]{日方}{ひかた}の
\ruby{云}{い}ひ
\ruby{草}{ぐさ}ぢやあ
\ruby{無}{な}いが、いつでも
\ruby{現在}{げん|ざい}に
\ruby{滿足}{まん|ぞく}しないで、
\ruby{永久}{えい|きう}に
\ruby{{\換字{進}}}{すゝ}んで
\ruby{{\換字{飽}}}{あ}くことを
\ruby{知}{し}らぬ
\ruby{歟}{か}、
\ruby{感心}{かん|しん}に
\ruby{自分}{じ|ぶん}は
\ruby{自分}{じ|ぶん}の
\ruby{勉{\換字{強}}}{べん|きやう}を
\ruby{仕}{し}て
\ruby{居}{ゐ}るさうだ。
\換字{志}て
\ruby{見}{み}りやあ
\ruby[g]{容貌}{きりやう}も
\ruby{佳}{よ}いし、
\ruby[g]{心掛}{こゝろがけ}も
\ruby{可}{よ}いし、
\ruby{別}{べつ}に
\ruby{難}{なん}はない
\ruby{女}{をんな}なんだ。
\ruby{左様}{さ|う}いふ
\ruby{女}{をんな}が
\ruby{現}{あらは}れたので、
\ruby{學校}{がく|かう}の
\ruby{内}{うち}でも
\ruby{外}{そと}でも
\ruby{珍}{めづ}らしがつて、
\ruby{何}{なん}とか
\ruby{彼}{か}とか
\ruby{{\換字{評}}{\換字{判}}}{ひよう|ばん}が
\ruby{立}{た}つて
\ruby{居}{ゐ}たが、
\ruby{其内}{その|うち}に
\ruby{水野}{みづ|の}が
\ruby{{\換字{迷}}}{まよ}ひ
\ruby{出}{だ}した。
\ruby{何様}{ど|う}いふ
\ruby{機會}{は|め}から
\ruby{水野}{みづ|の}の
\ruby{心}{こゝろ}が
\ruby{其女}{その|をんな}に
\ruby{傾}{かたむ}いたかは
\ruby{解}{わか}らないが、
\ruby{乃公}{お|れ}が
\ruby{思}{おも}ふにやあ
\ruby{別}{べつ}な
\ruby{事}{こと}はない。
\ruby{{\換字{浄}}瑠璃}{じやう|る|り}の
\ruby{{\換字{文}}句}{もん|く}にある
\ruby{{\換字{通}}}{とほ}り、
\ruby{琥珀}{こ|はく}の
\ruby{塵}{ちり}や
\ruby[g]{磁石}{じしやく}の
\ruby{針}{はり}で、
\ruby{眼}{め}に
\ruby{見}{み}えて
\ruby{何處}{ど|こ}が
\ruby{何様}{ど|う}といふ
\ruby{事}{こと}は
\ruby{無}{な}いが、たゞ
\ruby{譯}{わけ}も
\ruby{無}{な}く
\ruby{引}{ひ}き
\ruby{寄}{よ}せられて、
\ruby{心}{こゝろ}が
\ruby{其處}{そ|こ}へ
\ruby{行}{ゆ}くのが
\ruby{戀}{こひ}の
\ruby{{\換字{習}}}{なら}ひだ。
こりあ
\ruby{俗物}{ぞく|ぶつ}でも
\ruby{仙骨}{せん|こつ}でも
\ruby{同}{おな}じ
\ruby{事}{こと}、いくら
\ruby{水野}{みづ|の}が
\ruby{俊才}{すぐれ|もの}だつて、
\ruby{生血}{なま|ち}を
\ruby{包}{つゝ}んだ五
\ruby{尺}{しやく}の
\ruby[g]{身體}{からだ}を、
\ruby{抱}{かゝ}へて
\ruby{居}{ゐ}るのだもの
\ruby{無理}{む|り}も
\ruby{無}{な}い、
\ruby[g]{矢張}{やつぱ}り
\ruby{年齢}{と|し}が
\ruby{年齢}{と|し}だから
\ruby{{\換字{迷}}}{まよ}つたんだらう。
\換字{志}かし
\ruby{相手}{あひ|て}も
\ruby{商賣人}{しやう|ばい|にん}ぢあ
\ruby{無}{な}し、
\ruby{水野}{みづ|の}も
\ruby{獨身}{ひと|りみ}で
\ruby{居}{ゐ}なけりあならぬといふので
\ruby{無}{な}いから、
\ruby{全}{まつた}く
\ruby{深}{ふか}く
\ruby{思}{おも}ひ
\ruby{込}{こ}んだものならば、
\ruby{緣}{えん}を
\ruby{纏}{まと}めりやあ
\ruby{其}{それ}で
\ruby{可}{い}いのだが、さあ、
\ruby{水野}{みづ|の}の
\ruby{不仕合}{ふ|しあは|せ}といふのは
\ruby{其處}{そ|こ}の
\ruby{事}{こと}で、
\ruby{俗}{ぞく}にいふ
\ruby{蟲}{むし}が
\ruby{{\換字{嫌}}}{きら}ふといふものでゞもあらうか、
\ruby{其女}{その|をんな}が
\ruby{水野}{みづ|の}の
\ruby[g]{眞心}{まごゝろ}を
\ruby{受}{う}け
\ruby{納}{い}れぬので、それで
\ruby{水野}{みづ|の}は
\ruby{懊惱}{あう|なう}して
\ruby{居}{ゐ}るといふのだ。
もつとも
\ruby{水野}{みづ|の}が
\ruby{明}{あか}らさまに、
\ruby{其女}{その|をんな}に
\ruby{何事}{なに|ごと}を
\ruby{云}{い}つたでもあるまいが、これは
\ruby{世間}{せ|けん}に
\ruby{老}{お}いた
\ruby{山路}{やま|ぢ}の
\ruby{老夫}{ぢゞ|い}が、
\ruby{水野}{みづ|の}の
\ruby{様子}{やう|す}を
\ruby{見}{み}て
\ruby{察}{さつ}しての
\ruby{話}{はなし}だ。
さて
\ruby{其}{それ}にしたところで
\ruby{其限}{それ|ぎ}りの
\ruby{事}{こと}なら、
\ruby{芥火}{あくた|び}の
\ruby{燃}{も}えるやうにぶすりぶすりと、
\ruby{水野}{みづ|の}が
\ruby{物}{もの}を
\ruby{思}{おも}つて
\ruby{居}{ゐ}るだけで
\ruby{濟}{す}むのだが、こゝに
\ruby{其}{そ}の
\ruby{五十子}{い|そ|こ}の
\ruby{親}{おや}にお
\ruby{關}{せき}といふ、
\ruby{可憎}{い|や}な
\ruby{{\換字{強}}欲}{がう|よく}な
\ruby{惡婆}{あく|ば}がある。
\ruby{勿論}{もち|ろん}
\ruby{生}{うみ}の
\ruby{母}{はゝ}では
\ruby{無}{な}くつて、
\ruby{五十子}{い|そ|こ}とは
\ruby{別々}{べつ|〳〵}に
\ruby{住}{す}んで
\ruby{居}{ゐ}るほど、
\ruby{氣性}{きし|やう}も
\ruby{合}{あ}はねば
\ruby{仲}{なか}も
\ruby{惡}{わる}いのだが、
\ruby{時々}{とき|〴〵}
\ruby{五十子}{い|そ|こ}のところへ
\ruby{來}{き}ては
\ruby{無理}{む|り}を
\ruby{云}{い}つて、
\ruby{無}{な}け
\ruby{無}{な}しの
\ruby{金}{かね}を
\ruby{絞}{しぼ}って
\ruby{行}{ゆ}く。
\ruby[g]{其奴}{そいつ}が
\ruby{水野}{みづ|の}の
\ruby{腹}{はら}を
\ruby{見}{み}て
\ruby{取}{と}つて、
\ruby{其}{そ}の
\ruby{初心}{う|ぶ}なところに
\ruby{付}{つ}け
\ruby{{\換字{込}}}{こ}んで、いろいろさまざまな
\ruby{事}{こと}を
\ruby{云}{い}ひ
\ruby{散}{ち}らしちやあ、つまり
\ruby[g]{幾干}{いくら}かづゝ
\ruby{捲}{ま}き
\ruby{上}{あ}げるさうだ。
\ruby{金}{かね}は
\ruby{些少}{わづ|か}の
\ruby{事}{こと}だから
\ruby{仔細}{し|さい}は
\ruby{無}{な}いが、
\ruby{金}{かね}を
\ruby{取}{と}らう
\ruby{爲}{ため}ばつかりに
\ruby{其}{その}
\ruby{婆}{ばゞあ}めが、
\ruby{好}{い}い
\ruby{加減}{か|げん}な
\ruby{事}{こと}を
\ruby{云}{い}つて
\ruby{煽}{あふ}り
\ruby{立}{た}つて
\ruby{燃}{も}え
\ruby{立}{た}たする。
ところが
\ruby{一方}{いつ|ぱう}ぢやあ
\ruby{{\換字{叉}}}{また}、
\ruby{肝心}{かん|じん}の
\ruby{人}{ひと}によそ〳〵しく
\ruby{冷}{ひや}つこく
\ruby[g]{待{\換字{遇}}}{あしら}はれる。
\ruby{火}{ひ}にあひ
\ruby{水}{みづ}にあふのだから
\ruby{敵}{かな}はない、
\ruby{水野}{みづ|の}の
\ruby{心}{こゝろ}の
\ruby[g]{{\換字{静}}{\換字{穏}}}{しづか}なことは、
\ruby{今}{いま}は
\ruby{一時}{いつ|とき}でも
\ruby{有}{あ}りさうも
\ruby{無}{な}い
\ruby{譯}{わけ}。
そこで
\ruby{今}{いま}までの
\ruby[g]{行状}{みもち}とは
\ruby{打}{う}つて
\ruby{變}{かは}つて、
\ruby{家}{うち}に
\ruby{居}{ゐ}る
\ruby{時}{とき}は
\ruby{鬱々}{うつ|〳〵}として、たゞ
\ruby{沈}{しづ}みきつて
\ruby{物}{もの}も
\ruby{言}{い}はず、
\ruby{机}{つくえ}に
\ruby{對}{むか}つても
\ruby{書}{ほん}は
\ruby{讀}{よ}まずに、
\ruby[g]{長太息}{ためいき}を
\ruby{吐}{つ}く
\ruby{時}{とき}のみ
\ruby{多}{おほ}く、
\ruby{朝}{あさ}は
\ruby{心}{こゝろ}よく
\ruby{起}{お}きる
\ruby{日}{ひ}も
\ruby{無}{な}く、
\ruby{夜}{よ}も
\ruby{寐苦}{ね|ぐる}しく
\ruby{{\換字{過}}}{すご}すさうだ。
これは
\ruby{乃公}{お|れ}が
\ruby{老夫}{おや|ぢ}から
\ruby{聞}{き}いたゞけで、
\ruby[g]{無論}{むろん}
\ruby{山路}{やま|ぢ}の
\ruby[g]{老夫}{おやぢ}のつもりでは、
\ruby{乃公}{お|ら}に
\ruby{意見}{い|けん}して
\ruby{{\換字{遣}}}{や}れといふのだつた。
\換字{志}かし
\ruby{乃公}{お|れ}は
\ruby{乃公}{お|れ}の
\ruby{考}{かんがへ}で、
\ruby{水野}{みづ|の}のためには
\ruby{幾干}{いく|ら}でも、
\ruby{盡力}{つ|く}したいと
\ruby{思}{おも}つて
\ruby{居}{ゐ}ることは
\ruby{思}{おも}つて
\ruby{居}{ゐ}るが、
\ruby{意見}{い|けん}を
\ruby{仕}{し}て
\ruby{利{\換字{益}}}{た|め}になりさうな
\ruby{筋}{すぢ}では
\ruby{無}{な}いと、
\ruby{見切}{み|き}つてつい
\ruby{其儘}{その|まゝ}に
\ruby{{\換字{過}}}{す}ごして
\ruby{來}{き}たのだ。
』

\ruby{辛}{から}くも
\ruby{此時}{こ|ゝ}まで
\ruby{堪}{こら}へたりし
\ruby[g]{日方}{ひかた}は
\ruby{再}{ふたゝ}び
\ruby{叫}{さけ}び
\ruby{出}{いだ}しぬ。

『
\ruby{何故}{な|ぜ}
\ruby{意見}{い|けん}を
\ruby{仕}{し}ても
\ruby{利{\換字{益}}}{た|め}にならん?。
\ruby{意見}{い|けん}を
\ruby{仕無}{し|な}いで
\ruby{何}{なん}と
\ruby{爲}{す}るんだ?。
\ruby{何様}{ど|う}して
\ruby{水野}{みづ|の}の
\ruby{爲}{ため}に
\ruby{盡力}{つ|く}す?。
』

『
\ruby{乃公}{お|ら}あ
\ruby{出來}{で|き}る
\ruby{事}{こと}なら
\ruby{水野}{みづ|の}の
\ruby{思}{おも}ひの、
\ruby{徹}{とほ}るやうに
\ruby{爲}{し}て
\ruby{{\換字{遣}}}{や}らうと
\ruby{思}{おも}つて
\ruby{居}{ゐ}るのだ。
』

『
\ruby{何}{なん}だと、
\ruby{馬鹿野郎}{ば|か|や|らう}ツ!、
\ruby{愚}{ぐ}にもつかん!。
そんな
\ruby{下}{くだ}らんことがあるものか、
\ruby{貴様}{き|さま}は
\ruby{一體}{いつ|たい}
\ruby{腐敗}{ふ|はい}して
\ruby{居}{ゐ}る!。
』

『また
\ruby{馬鹿}{ば|か}
\ruby{呼}{よば}はりをするナ!。
\ruby{汝}{きさま}こそ
\ruby{馬鹿}{ば|か}だ。
\ruby{意見}{い|けん}して
\ruby{役}{やく}に
\ruby{立}{た}つ
\ruby{位}{くらいゐ}なら
\ruby{乃公}{お|れ}が
\ruby{爲}{す}るは。
\ruby{人}{ひと}は
\ruby{銘々}{めい|〳〵}に
\ruby[g]{所考}{かんがへ}がある。
\ruby{乃公}{お|れ}は
\ruby{乃公}{お|れ}、
\ruby{汝}{きさま}は
\ruby{汝}{きさま}で
\ruby{可矣}{い|ゝ}ぢやあ
\ruby{無}{ね}えか。
\ruby{意見}{い|けん}が
\ruby{仕}{し}たけりやあ
\ruby{汝}{きさま}
\ruby{爲}{ し}ろ。
』

『
\ruby{勿論}{もち|ろん}だ。
\ruby{諫}{いさ}めて
\ruby{{\換字{遣}}}{や}らないで
\ruby{何様}{ど|う}するものか。
\ruby{女}{をんな}が
\ruby{美}{よ}くつても
\ruby{惡}{わる}くつても、
\ruby{何}{なん}だ!、
\ruby{女}{をんな}が!。
\ruby{苟}{いやし}くも
\ruby{大{\換字{丈}}夫}{だい|ぢやう|ぶ}たるものが
\ruby{高}{たか}が
\ruby[g]{一婦人}{いちふじん}に、
\ruby{志}{こゝろざし}を
\ruby{喪}{うしな}ふとは
\ruby{何}{なん}たる
\ruby{事}{こつ}た。
\ruby{實}{じつ}に
\ruby{怪}{け}しからん、はがゆい
\ruby{奴}{やつ}だ。
\ruby{是非}{ぜ|ひ}
\ruby{尋}{たづ}ねて
\ruby{行}{い}つて
\ruby{大}{おほい}に
\ruby{諫}{いさ}める。
』

\ruby[g]{二人}{ふたり}の
\ruby{問答}{もん|だふ}はこゝに
\ruby{已}{や}んで、
\ruby{山瀬}{やま|せ}は
\ruby{爽}{さわ}やかに
\ruby{口}{くち}を
\ruby{開}{ひら}きぬ。

『
\ruby{僕}{ぼく}は
\ruby{他人}{ひ|と}の
\ruby[g]{意思感{\換字{情}}}{いしかんじやう}の
\ruby{自由}{じ|いう}を
\ruby{{\換字{尊}}重}{そん|ちよう}するから、
\ruby{立入}{たち|い}つては
\ruby{敢}{あへ}て
\ruby{兎角}{と|かく}を
\ruby{言}{い}はぬ。
\換字{志}かしこれは
\ruby[g]{水野君}{みづのくん}のために
\ruby{不利{\換字{益}}}{ふ|り|えき}と
\ruby{思}{おも}ふから、
\ruby{一應}{いち|おう}は
\ruby{忠告}{ちゆう|こく}を
\ruby{試}{こゝろ}みるつもりだ。
』

\ruby{人皆}{ひと|みな}
\ruby{語}{かた}れども
\ruby{{\換字{羽}}{\換字{勝}}}{は|がち}は
\ruby{語}{かた}らず、たゞ
\ruby{僅}{わづか}に
\ruby{吁然}{ほ|つ}と
\ruby{息}{いき}つけば、
\ruby{手}{て}にせし
\ruby{巻{\換字{煙}}草}{た|ば|こ}の
\ruby{{\換字{灰}}}{はい}の
\ruby{長}{なが}く
\ruby{續}{つゞ}けるが、ぼたりと
\ruby{膝}{ひざ}の
\ruby{上}{うへ}に
\ruby{落}{お}ちて
\ruby{脆}{もろ}く
\ruby{散}{ち}つたり。

\ruby{夜色}{や|しよく}は
\ruby{樓外}{ろう|ぐわい}に
\ruby{沈々}{ちん|〳〵}として、
\ruby{澄}{す}みわたりたる
\ruby{天}{そら}にかゝれる
\ruby{星斗}{ほ|し}は
\ruby{爛然}{らん|ぜん}と
\ruby{明}{あき}らかに、
\ruby{明日}{あ|す}は
\ruby{風}{かぜ}にや
\ruby{其}{そ}の
\ruby{大}{おほき}なるは、いづれ
\ruby{煌々}{ひか|〳〵}と
\ruby{瞬目}{めは|じき}して、
\ruby{光}{ひかり}の
\ruby{芒}{のぎ}は
\ruby{搖}{ゆら}ぎに
\ruby{搖}{ゆら}げり。


\Entry{其六}

\ruby{山瀬}{やま|せ}が
\ruby{催}{もよほ}せし
\ruby{小集}{せう|しふ}の、
\ruby{竹芝}{たけ|しば}の
\ruby{浦}{うら}に
\ruby{開}{ひら}かれし
\ruby{日}{ひ}なり、これは
\ruby{東京}{とう|きやう}を
\ruby{丑寅}{うし|とら}に
\ruby{離}{はな}れし
\ruby{東武線}{とう|ぶ|せん}の
\ruby{鐘淵}{かねが|ふち}の
\ruby{停車場}{てい|しや|じやう}より、% 原文通り「場」
\ruby{上}{のぼ}り
\ruby{滊車}{き|しや}の
\ruby{今}{いま}や
\ruby{出}{い}でんとするに
\ruby{駈}{か}け
\ruby{付}{つ}けて、
\ruby{辛}{から}くも
\ruby{乘}{の}り
\ruby{込}{こ}みし
\ruby{水野靜十郎}{みづ|の|せい|じう|らう}は、
\ruby{車室}{しや|しつ}の
\ruby{一隅}{いち|ぐう}に
\ruby{身}{み}をおちつけて、
\ruby{煎}{い}りつくが
\ruby{如}{ごと}き
\ruby{急}{せ}き
\ruby{心}{ごゝろ}に
\ruby{少}{すくな}からぬ
\ruby{路程}{みち|のり}を
\ruby{走}{はし}り
\ruby{來}{きた}りし
\ruby{胸}{むね}の
\ruby{轟}{とゞろ}きを
\ruby{纔}{わづか}に
\ruby{息}{やす}めぬ。

\ruby{車窓}{しや|そう}の
\ruby{外}{そと}は、
\ruby{目}{め}に
\ruby{障}{さは}るものも
\ruby{無}{な}く
\ruby{廣々}{ひろ|〴〵}としたる
\ruby{葛{\換字{飾}}}{かつ|しか}の
\ruby{秋}{あき}の
\ruby{稻田}{いな|だ}に、
\ruby{黄金色}{こ|がね|いろ}の
\ruby{夕陽}{ゆふ|ひ}の
\ruby{光線}{ひか|り}
\ruby{明}{あか}るく
\ruby{斜}{なゝめ}に
\ruby{落}{お}ちて、
\ruby{折々}{をり|〳〵}ぱつと
\ruby{立}{た}つ
\ruby{群雀}{ぐん|じやく}の
\ruby{{\換字{空}}}{そら}に
\ruby{散}{ち}る
\ruby{景色}{け|しき}も、
\ruby{土用}{ど|よう}の
\ruby{旱}{てり}の
\ruby{足}{た}りて
\ruby{豊}{ゆたか}なる
\ruby{年}{とし}の\換字{志}るしと
\ruby{好}{この}もしく、
\ruby{暑}{あつ}かりし
\ruby{夏}{なつ}の
\ruby{日}{ひ}の
\ruby{汗}{あせ}の
\ruby{滴}{しづく}は、
\ruby{今}{いま}
\ruby{皆}{みな}やがて
\ruby{粒々}{りふ|〳〵}の
\ruby{實}{み}となつて
\ruby{現}{あらは}るべき
\ruby{快}{こゝろよ}き
\ruby{眺望}{なが|め}なり。

されば
\ruby{乗}{の}り
\ruby{合}{あ}はせし
\ruby{人々}{ひと|〴〵}も
\ruby{悅}{よろこ}び
\ruby{顏}{がほ}して、

『
\ruby{先}{ま}づ
\ruby{此}{こ}の
\ruby{{\換字{分}}}{ぶん}に
\ruby{行}{ゆ}きやあ
\ruby{豐年}{ほう|ねん}でがす。
』

と
\ruby{股引}{もゝ|ひき}に
\ruby{草鞋穿}{わら|ぢ|ば}きの
\ruby{農夫}{ひやく|しやう}らしきが
\ruby{眞先}{まつ|さき}に
\ruby{云}{い}い
\ruby{出}{だ}せば、

『さうです、
\ruby{風}{かぜ}さへ
\ruby{無}{な}きやあ
\ruby{既}{もう}
\ruby{大{\換字{丈}}夫}{だい|ぢやう|ぶ}です。
おほかた
\ruby{不景氣}{ふ|けい|き}も
\ruby{直}{なほ}るでがせう。
』

と
\ruby{同}{おな}じ
\ruby{風}{ふう}の
\ruby{男}{をとこ}が
\ruby{云}{い}ふ。
その
\ruby{後}{あと}より
\ruby{髮}{かみ}の
\ruby{毛}{け}を
\ruby{綺麗}{き|れい}に
\ruby{{\換字{分}}}{わ}けたる
\ruby{生意氣}{なま|い|き}の
\ruby{{\換字{若}}}{わか}き
\ruby{男}{をとこ}の、これは
\ruby{商人}{しやう|にん}と
\ruby{見}{み}えたるが、

『
\ruby{何}{なに}にしろ
\ruby{此夏}{この|なつ}の
\ruby{暑氣}{あつ|さ}のおかげですもの、
\ruby{此位}{この|ぐらゐ}の
\ruby{事}{こと}あ
\ruby{無}{な}くちやあなりませんや。
\ruby{暑}{あつ}かつた
\ruby{事}{こと}あ
\ruby{無法}{む|はふ}に
\ruby{暑}{あつ}うございましたが、
\ruby{何樣}{ど|う}でしやう
\ruby{全國}{ぜん|こく}ぢやあ
\ruby{其}{それ}がために、
\ruby{去年}{きよ|ねん}に
\ruby{比}{くら}べりやあ
\ruby{一千萬石}{いつ|せん|まん|ごく}も
\ruby{餘計}{よ|けい}に
\ruby{穫}{と}れる
\ruby{算盤}{そろ|ばん}だつて
\ruby{云}{い}ふんですからなア!。
\ruby{一石十圓}{いつ|こく|じう|ゑん}としても
\ruby{一億圓}{いち|おく|ゑん}、
\ruby{四千萬人}{よん|せん|まん|にん}に
\ruby{割}{わ}つてみると、
\ruby{一人{\換字{前}}}{いち|にん|まへ}が
\ruby{二圓五十錢}{に|ゑん|ご|じう|せん}
\ruby{宛}{づゝ}、
\ruby{畢竟}{つま|り}それだけ
\ruby{宛}{づゝ}
\ruby{暑氣}{あつ|さ}の
\ruby{我慢賃}{が|まん|ちん}に
\ruby{貰}{もら}つたやうな
\ruby{譯}{わけ}に
\ruby{當}{あ}たりますから、
\ruby{隨{\換字{分}}}{ずゐ|ぶん}
\ruby{暑}{あつ}かつたのも
\ruby{無理}{む|り}は
\ruby{有}{あ}りません。
\ruby{併}{しか}し
\ruby{如是}{か|う}なつて
\ruby{見}{み}りやあ
\ruby{有}{あ}り
\ruby{難}{がた}いもんで、
\ruby{屹度}{きつ|と}
\ruby{景氣}{けい|き}も
\ruby{好}{よ}くなりまさあネ。
』

などゝ
\ruby{口々}{くち|〴〵}に
\ruby{語}{かた}りあへど、
\ruby{思}{おもひ}
\ruby{有}{あ}る
\ruby{身}{み}の
\ruby{水野}{みづ|の}
\ruby{一人}{ひと|り}は、
\ruby{景色}{け|しき}も
\ruby{眼}{め}に
\ruby{{\換字{更}}}{さら}に
\ruby{見}{み}ざるがごとく、
\ruby{談話}{はな|し}も
\ruby{耳}{みゝ}に
\ruby{{\換字{更}}}{さら}に
\ruby{聞}{き}かぬが
\ruby{如}{ごと}く、
\ruby{身}{み}じろぎも
\ruby{多}{おほ}くはせで
\ruby{寂然}{じやく|ねん}と
\ruby{坐}{すわ}りつ、たゞ
\ruby{帶}{おび}の
\ruby{間}{あひだ}より
\ruby{時計}{と|けい}を
\ruby{出}{いだ}して、
\ruby{恰}{あだか}も
\ruby{滊車}{き|しや}の
\ruby{{\換字{速}}力}{はや|さ}を
\ruby{疑}{うたが}ふやうに、
\ruby{幾度}{いく|たび}か
\ruby{其}{そ}の
\ruby{鍼}{はり}を
\ruby{甲{\換字{斐}}無}{か|ひ|な}く
\ruby{視詰}{み|つ}めぬ。
\ruby{淺黑}{あさ|ぐろ}き
\ruby{其}{そ}の
\ruby{面}{おもて}は
\ruby{底}{そこ}に
\ruby{蒼色}{あを|み}を
\ruby{帶}{お}びて、
\ruby{鳳眼}{ほう|がん}とやらん
\ruby{人}{ひと}のいふ
\ruby{魚尾上}{し|り|あが}りの
\ruby{眼}{め}は、どんよりと
\ruby{曇}{くも}りて
\ruby{光}{ひか}り
\ruby{澱}{よど}み、やゝ
\ruby{狭}{せま}き
\ruby{鼻}{はな}はつんと
\ruby{高}{たか}くして、
\ruby{血}{ち}の
\ruby{色薄}{いろ|うす}き
\ruby{一}{いち}の
\ruby{字口}{じ|ぐち}の
\ruby{唇}{くちびる}は、
\ruby{復}{ふたゝ}び
\ruby{開}{ひら}かるゝ
\ruby{時}{とき}の
\ruby{無}{な}からん
\ruby{如}{ごと}くに
\ruby{{\換字{飽}}}{あく}まで
\ruby{緊}{きび}しく
\ruby{閉}{とぢ}られたり。
\ruby{眼鼻立}{め|はな|だち}は
\ruby{醜}{あし}きにあらぬ
\ruby{男}{をとこ}ながら、
\ruby{水野}{みづ|の}が
\ruby{今}{いま}の
\ruby{顏}{かほ}の
\ruby{氣色}{やう|す}は、
\ruby{稚兒}{をさな|ご}は
\ruby{之}{これ}を
\ruby{望}{のぞ}まば
\ruby{怖}{おそ}れて
\ruby{泣}{な}くべし。

\ruby{滊車}{き|しや}のやがて
\ruby{吾妻橋}{あ|づま|ばし}
\ruby{停車場}{てい|しや|じやう}に% 原文通り「場」
\ruby{着}{つ}きし
\ruby{時}{とき}には、
\ruby{暮}{く}れやすき
\ruby{秋}{あき}の
\ruby{日}{ひ}は
\ruby{既}{はや}
\ruby{沒}{い}りて、
\ruby{千點萬點}{せん|てん|ばん|てん}の
\ruby{燈火}{とも|しび}に
\ruby{{\換字{飾}}}{かざ}られたる
\ruby{夜}{よる}の
\ruby{東京}{とう|きやう}は
\ruby{眼}{め}の
\ruby{{\換字{前}}}{まへ}に
\ruby{現}{あら}はれぬ。

\ruby{水野}{みづ|の}は
\ruby{人}{ひと}を
\ruby{突}{つ}き
\ruby{{\換字{退}}}{の}くるまでに
\ruby{忙}{いそ}がはしく
\ruby{歩}{あゆ}みて、
\ruby{忽}{たちま}ち
\ruby{停車場}{てい|しや|じやう}を% 原文通り「場」
\ruby{出}{い}で、
\ruby{忽}{たちま}ち
\ruby{吾妻橋}{あ|づま|ばし}を
\ruby{越}{こ}え、
\ruby{忽}{たちま}ち
\ruby{茶屋町}{ちや|ゝ|まち}を
\ruby{{\換字{過}}}{す}ぎ、
\ruby{忽}{たちま}ち
\ruby{並木}{なみ|き}を
\ruby{經}{へ}て、
\ruby{忽}{たちま}ち
\ruby{藏{\換字{前}}}{くら|まへ}に
\ruby{至}{いた}り、
\ruby{其處}{そ|こ}に
\ruby{住}{すま}へる
\ruby{月日}{つき|ひ}は
\ruby{未}{いま}だ
\ruby{長}{なが}からねど、
\ruby{淺草}{あさ|くさ}
\ruby{一}{いち}との
\ruby{噂}{うはさ}を
\ruby{得}{え}たる
\ruby{醫學士}{い|がく|し}
\ruby{相良公{\換字{平}}}{さが|ら|こう|へい}の
\ruby{玄關}{げん|くわん}に
\ruby{至}{いた}り、

『
\ruby{頼}{たの}む。
』

と
\ruby{一聲}{いつ|せい}
\ruby{音}{おと}づれたり。


\Entry{其七}

\ruby{應}{おう}と
\ruby{答}{こた}へて
\ruby{出}{い}で
\ruby{來}{きた}れるは、
\ruby{盤臺面}{ばん|だい|づら}の
\ruby{鼻}{はな}の
\ruby{下}{した}に
\ruby{薄髭}{うす|ひげ}しよぼ〳〵と
\ruby{{\換字{煙}}}{けむり}の
\ruby{如}{ごと}く
\ruby{生}{は}えたる、
\ruby{二十七八}{に|じう|しち|はち}の
\ruby{物體}{もつ|たい}ぶつた
\ruby{男}{をとこ}なり。
\ruby{水野}{みづ|の}が
\ruby{紺飛白}{こん|が|すり}の
\ruby{單衣}{ひとへ|もの}に、
\ruby{着皺}{き|じわ}も
\ruby{見}{み}ゆる
\ruby{薄羽織}{うす|ば|おり}といふ
\ruby{身}{み}の
\ruby[g]{周圍}{まはり}を
\ruby{見}{み}て、
\ruby{突立}{つゝ|た}ちたるまゝ
\ruby{{\換字{尊}}大}{おほ|ふう}に、

『もう
\ruby{診察}{しん|さつ}の
\ruby{時間}{じ|かん}は
\ruby{濟}{す}んだが。
』

と
\ruby{云}{い}ひかけしが、また
\ruby{其}{そ}の
\ruby{{\換字{顔}}色}{かほ|いろ}の
\ruby{好}{よ}からぬを
\ruby{見}{み}て、

『お
\ruby{{\換字{前}}}{まへ}さんかネ。
』

と
\ruby{僅}{わずか}に
\ruby[g]{愛想}{あいそ}あり。

\ruby{水野}{みづ|の}は
\ruby{叮嚀}{てい|ねい}に
\ruby{會釋}{ゑ|しやく}して、

『イヤ
\ruby{私}{わたくし}ではございません。
\ruby{御書{\換字{留}}}{お|かき|とめ}
\ruby{置}{お}き
\ruby{下}{くだ}すつたといふ
\ruby{事}{こと}ですが、
\ruby{昨日}{さく|じつ}
\ruby[g]{使丁}{つかひ}を
\ruby{以}{も}つて
\ruby{願}{ねが}ひました
\ruby{四木村}{よ|つ|ぎ}の
\ruby{{\換字{平}}井}{ひら|ゐ}と
\ruby{申}{まを}す
\ruby{者}{もの}の
\ruby{方}{かた}の
\ruby{病人}{びやう|にん}、
\ruby{岩崎五十}{いは|さき|い|そ}といふものを
\ruby{御來診}{ご|らい|しん}
\ruby{願}{ねが}ひたいので
\ruby{出}{で}ましたのです。
』

と
\ruby{云}{い}へば、

『アヽ、
\ruby{其}{そ}の\ %空白有り
\ruby{四}{よ}ツ
\ruby{木}{ぎ}とかいふところは、
\ruby{非常}{ひ|じやう}に
\ruby{{\換字{遠}}}{とほ}いところぢやさうだナ。
\ruby{知}{し}らんものだから
\ruby{仕方}{し|かた}が
\ruby{無}{な}い、
\ruby{小梅}{こ|うめ}か
\ruby{{\換字{請}}地}{うけ|ぢ}の
\ruby[g]{{\換字{近}}傍}{ちかく}かと
\ruby{思}{おも}うて、ムヽ
\ruby{可矣}{よ|し}
\ruby{願}{ねが}つて
\ruby{置}{お}いて
\ruby{{\換字{遣}}}{や}ると
\ruby{僕}{ぼく}が
\ruby{{\換字{受}}合}{うけ|あ}つたが、
\ruby{後}{あと}で
\ruby{先生}{せん|せい}に
\ruby{酷}{ひど}く
\ruby{叱}{しか}られた!。
\ruby{重病人}{ぢゆう|びやう|にん}や
\ruby{長病人}{ちやう|びやう|にん}を
\ruby{澤山}{たく|さん}に
\ruby{扣}{ひか}へて
\ruby{居}{ゐ}られるから、
\ruby{中々}{なか|〳〵}
\ruby{其樣}{そ|ん}な
\ruby{{\換字{遠}}}{とほ}いところへ
\ruby{御往診}{お|い|で}にはなりかねるといふことだ。
どうか
\ruby{他家}{よ|そ}へ
\ruby{行}{い}つて
\ruby{頼}{たの}んで
\ruby{見}{み}てくれ。
』

と、
\ruby{實}{じつ}に
\ruby{酷}{ひど}く
\ruby{叱}{しか}られや
\ruby{仕}{し}けむ、
\ruby{其}{そ}の
\ruby{時}{とき}の
\ruby[g]{不{\換字{平}}}{ふへい}は
\ruby{今}{いま}の
\ruby{{\換字{顔}}}{かほ}に
\ruby{膨}{ふく}れ
\ruby{出}{だ}して、
\ruby{{\換字{逐}}拂}{おつ|ぱら}つて
\ruby{仕舞}{し|ま}ふつもりの
\ruby{物言}{もの|い}ひ
\ruby[g]{仁慈}{なさけ}
\ruby{無}{な}し。

\ruby[g]{二三度}{にさんど}
\ruby[g]{四五度}{しごど}
\ruby{呼}{よ}びに
\ruby{{\換字{遣}}}{や}りける、といふ
\ruby{前句}{まへ|く}に、
\ruby{引}{ひ}く
\ruby{息}{いき}の
\ruby{{\換字{絕}}}{た}ゆるに
\ruby{醫者}{い|しや}のおどろかず、と
\ruby{付}{つ}けたるを、
\ruby{西鶴}{さい|くわく}が
\ruby{撰}{えら}みし
\ruby{其}{そ}の
\ruby[g]{疇昔}{むかし}より、
\ruby{世}{よ}に
\ruby{勢威}{いき|ほひ}ある
\ruby{醫者}{い|しや}を、
\ruby{富}{とみ}も
\ruby{無}{な}く
\ruby{名}{な}も
\ruby{無}{な}き
\ruby{賤人}{し|づ}が
\ruby{伏屋}{ふせ|や}に
\ruby{{\換字{請}}}{しやう}じ
\ruby{入}{い}れんとするほど、
\ruby{心}{こゝろ}に
\ruby{任}{まか}せで
\ruby{口惜}{くち|をし}きは
\ruby{無}{な}し。
\ruby[g]{相良}{さがら}が
\ruby{書生}{しよ|せい}の
\ruby{冷}{ひや}やかなる
\ruby{言葉}{こと|ば}も、
\ruby{今}{いま}さら
\ruby{珍}{めづ}しからぬ
\ruby{{\換字{浮}}世}{うき|よ}の
\ruby{態}{さま}なれば、
\ruby{腹}{はら}は
\ruby{立}{た}てねども
\ruby{差當}{さし|あた}つて
\ruby{恨}{うら}めしく
\ruby{悲}{かな}しく、
\ruby{水野}{みづ|の}は

『
\ruby{左樣}{さ|う}
\ruby{仰}{おつし}あつては
\ruby{當惑}{たう|わく}いたします。
\ruby{實}{じつ}は
\ruby{昨日}{さく|じつ}から
\ruby{今}{いま}
\ruby{御來臨}{お|い|で}か
\ruby{今}{いま}
\ruby{御來臨}{お|い|で}かと
\ruby{御待}{お|ま}ち
\ruby{申}{まを}して
\ruby{居}{をり}ました
\ruby{樣}{やう}な
\ruby{譯}{わけ}でございますから。
』

と
\ruby{云}{い}ひかくるを、
\ruby{書生}{しよ|せい}は
\ruby{面倒}{めん|だう}と
\ruby{云}{い}はぬばかりに、

『だから、うつかり
\ruby{受合}{うけ|あ}つた
\ruby{段}{だん}は
\ruby{僕}{ぼく}が
\ruby[g]{謝罪}{あやま}る。
たゞし
\ruby{先生}{せん|せい}は
\ruby{御忙}{お|いそ}がしくつて
\ruby{御來診}{お|い|で}になられんといふのぢやから
\ruby{仕方}{し|かた}が
\ruby{無}{な}いぢや
\ruby{無}{な}いか。
』

と
\ruby{後}{あと}を
\ruby{言}{い}はせぬやうに
\ruby{壓}{お}し
\ruby{被}{かぶ}せて
\ruby{云}{い}ふ。
それを
\ruby[g]{此方}{こなた}は
\ruby{押返}{おし|かへ}して、

『では
\ruby{御座}{ご|ざ}いませうが
\ruby{其處}{そ|こ}を
\ruby[g]{何卒}{どうぞ}
、もう
\ruby{一度}{いち|ど}
\ruby{御願}{お|ねが}ひ
\ruby{下}{くだ}すつて
\ruby{見}{み}て
\ruby{頂}{いたゞ}きたいのです。
\ruby{先生}{せん|せい}より
\ruby{他}{ほか}の
\ruby{方}{かた}を
\ruby{願}{ねが}はう
\ruby{氣}{き}は
\ruby{無}{な}くつて、かうして
\ruby{態々}{わざ|〳〵}
\ruby{四}{よ}ツ
\ruby{木}{ぎ}から、
\ruby{御願}{お|ねが}ひに
\ruby{出}{で}たのでございますから。
』

と、
\ruby{低}{ひく}き
\ruby[g]{聲音}{こわね}に
\ruby[g]{顫動}{ふるひ}をさへ
\ruby{帶}{お}びて、
\ruby{思}{おも}ひ
\ruby{入}{い}つて
\ruby{頭}{かうべ}を
\ruby{下}{さ}げて{\換字{志}}み〴〵と
\ruby{頼}{たの}み
\ruby{聞}{きこ}えぬ。
\ruby{見}{み}れば
\ruby{其面}{その|おもて}は
\ruby{深}{ふか}き
\ruby{憂愁}{うれ|ひ}の
\ruby{陰雲}{く|も}に
\ruby{生氣}{せい|き}を
\ruby{{\換字{鎻}}}{とざ}されて、
\ruby{疑懼}{ぎ|く}に
\ruby{潤}{うる}める
\ruby{眼}{め}の
\ruby{中}{うち}には、
\ruby{限無}{かぎり|な}き
\ruby{悲痛}{ひ|つう}の
\ruby{色}{いろ}を
\ruby{{\換字{浮}}}{うか}めたり。
\ruby[g]{至誠}{まこと}に
\ruby{動}{うご}かされて
\ruby{爭}{あらそ}ひかねたる
\ruby{書生}{しよ|せい}は
\ruby{是非}{ぜ|ひ}
\ruby{無}{な}く
\ruby{立}{な}ち
\ruby{上}{あが}がつて、

『それぢやあ
\ruby{先}{まあ }
\ruby{伺}{うかが}つて
\ruby{見}{み}て
\ruby{上}{あ}げやうから、
\ruby{其處}{そ|こ}へ
\ruby{上}{あが}がつて
\ruby{待}{ま}つて
\ruby{居}{ゐ}なさい。
』

と、
\ruby{{\換字{猶}}}{なほ}
\ruby{水野}{みづ|の}を
\ruby{田舎漢}{ゐな|か|もの}あしらひにして
\ruby{奥}{おく}へ
\ruby{行}{ゆ}きぬ。

\ruby{丁度}{ちやう|ど}
\ruby{人}{ひと}の
\ruby{{\換字{途}}{\換字{絕}}}{と|だ}えし
\ruby{夜食}{や|しよく}の
\ruby{頃}{ころ}とて、
\ruby{人}{ひと}も
\ruby{無}{な}き
\ruby{玄關}{げん|くわん}にたゞ
\ruby{我}{われ}ひとり、
\ruby{兀然}{つゝ|くり}として
\ruby{坐}{すわ}り
\ruby{居}{を}れば、
\ruby{我}{わ}が
\ruby{影子}{か|げ}
\ruby{淋}{さび}しく
\ruby{古畳}{ふる|だゝみ}に
\ruby{浸}{し}みて、
\ruby{偶然}{ふ|と}
\ruby{見}{み}れば
\ruby{低}{ひく}く
\ruby{吊}{つ}りたる
\ruby{電燈}{でん|とう}の
\ruby{蓋裏}{かさ|うら}に、
\ruby{{\換字{弱}}々}{よわ|〳〵}としたる
\ruby{白}{しろ}き
\ruby{蛾}{が}の、
\ruby{蝶}{てふ}といふほども
\ruby{無}{な}く
\ruby{小}{ちひさ}なるが、やがて
\ruby{力盡}{ちから|つ}きての
\ruby{身}{み}の
\ruby{果}{はて}をも
\ruby{思}{おも}はず、
\ruby{飛}{と}んでは
\ruby{止}{と}まり、
\ruby{止}{と}まつては
\ruby{飛}{と}びて
\ruby{狂}{くる}ひ
\ruby{居}{を}れり。

\ruby{待}{ま}つこと
\ruby[g]{少時}{しばし}して
\ruby{間}{あひ}の
\ruby{劃}{しきり}の
\ruby{唐紙}{から|かみ}をがらりと
\ruby{明}{あ}けて、
\ruby{書生}{しよ|せい}は
\ruby{復}{ふたゝ}び
\ruby{入}{い}り
\ruby{來}{きた}りぬ。

『
\ruby{何樣}{ど|う}も
\ruby{他}{ほか}の
\ruby{病家}{びよう|か}の
\ruby{都合}{つ|がふ}もあつて出られぬと
\ruby{仰}{おつし}ある。
\ruby{氣}{き}の
\ruby{毒}{どく}だけれども
\ruby{他}{ほか}へ
\ruby{行}{い}つて
\ruby{下}{くだ}さい。
』

\ruby{言葉}{こと|ば}の
\ruby{柔}{やさ}しくなりたるだけに
\ruby{拒{\換字{絕}}}{きよ|ぜつ}の
\ruby{意}{こゝろ}はいよ〳〵
\ruby{堅}{かた}し。
さりとて
\ruby{病}{や}める
\ruby{五十子}{い|そ|こ}が
\ruby{曾}{かつ}てより
\ruby{信}{しん}じて、
\ruby{苦悶}{く|もん}の
\ruby{床}{とこ}の
\ruby{上}{うへ}の
\ruby{獨語}{ひとり|ごと}に
\ruby{頼}{たの}みたしといひしは、たゞ
\ruby{此}{こ}の
\ruby{家}{いへ}の
\ruby[g]{主人}{あるじ}なるを、いづくにか
\ruby{行}{ゆ}き
\ruby{他人}{ひ|と}を
\ruby{頼}{たの}まん。
\ruby{水野}{みづ|の}はほとほと
\ruby{行}{ゆ}き
\ruby{詰}{つ}まりて、
\ruby[g]{言葉}{ことば}も
\ruby{無}{な}く
\ruby{力}{ちから}も
\ruby{無}{な}く
\ruby{首}{かうべ}を
\ruby{垂}{た}れしが、
\ruby{搏}{はたゝ}き
\ruby{已}{や}めぬ
\ruby{彼}{か}の
\ruby{白}{しろ}き
\ruby{蛾}{が}の、
\ruby[g]{電燈}{あかり}の
\ruby[g]{周圍}{まはり}を
\ruby{飛}{と}び
\ruby{廻}{めぐ}る
\ruby{其}{そ}の
\ruby{陰翳眼}{か|げ|め}の
\ruby{前}{まへ}にちら〳〵と
\ruby{落}{お}つれば、
\ruby{噫}{あゝ}、
\ruby{我}{われ}も
\ruby{取}{と}りかぬる
\ruby{燈}{ひ}の
\ruby{{\換字{近}}傍}{かた|はら}を、
\ruby{{\換字{猶}}}{なほ}
\ruby{去}{さ}らぬ
\ruby{蟲}{むし}と
\ruby{愚}{ぐ}にも
\ruby{愚}{ぐ}なれど、
\ruby{甲斐無}{か|ひ|な}くも
\ruby{飛}{と}び
\ruby{直}{なほ}し〳〵するごとく、
\ruby[g]{言葉}{ことば}を
\ruby{換}{か}へて
\ruby{頼}{たの}みて
\ruby{見}{み}んと、
\ruby{其場}{その|ば}は
\ruby{立}{た}たんともせざる
\ruby{折}{をり}から、
\ruby{奥}{おく}の
\ruby{方}{はう}より
\ruby{丁}{ちやう}といふ
\ruby{石子}{い|し}の
\ruby{響}{ひゞ}き、
\ruby{確}{たしか}に
\ruby{人}{ひと}の
\ruby{碁}{ご}を
\ruby{打}{う}てる
\ruby{音}{おと}の、
\ruby{幽}{かすか}に
\ruby[g]{此方}{こなた}に
\ruby{聞}{きこ}えたり。


\Entry{其八}

\ruby{人}{ひと}おの〳〵
\ruby{我}{わ}が
\ruby{娯樂}{たの|しみ}に
\ruby{使}{つか}はれるは
\ruby{無}{な}し。
\ruby{中}{なか}にも
\ruby{碁好}{ご|ずき}は
\ruby{聖}{せい}に
\ruby{{\GWI{u8fd1-k}}}{ちか}く
\ruby{愚}{ぐ}に
\ruby{{\GWI{u8fd1-k}}}{ちか}く、
\ruby{假}{かり}の
\ruby{與奪}{やり|とり}の
\ruby{白黑}{しろ|くろ}の
\ruby{石}{いし}に、
\ruby{氣}{き}を
\ruby{{\GWI{u9063-k}}}{つか}ひ
\ruby{心}{こヽろ}を
\ruby{苦}{くるし}めて
\ruby{一切}{いつ|さい}を
\ruby{忘}{わす}れ
\ruby{果}{は}て、
\ruby{一寸}{いつ|すん}の
\ruby{暇}{ひま}を
\ruby{偸}{ぬす}んで
\ruby{始}{はじ}めし
\ruby{爭戰}{あら|そひ}にも、
\ruby{思}{おも}はず
\ruby{{\換字{半}}日}{はん|にち}の
\ruby{尻}{しり}を
\ruby{腐}{くさ}らせて
\ruby{悔}{くや}まぬが
\ruby{常}{つね}なり。
されば
\ruby{殆}{ほとん}ど
\ruby{一日}{いち|にち}の
\ruby{忙}{せは}しき
\ruby[g]{業務}{つとめ}を
\ruby{{\換字{終}}}{を}へし
\ruby{擧句}{あげ|く}、
\ruby{心}{こヽろ }
\ruby{蘇生}{よみ|が}へる
\ruby{晩餐}{ばん|さん}の
\ruby{小酌}{せう|しやく}の
\ruby{後}{のち}に、
\ruby{憎}{にく}くも
\ruby{可愛}{かは|ゆ}くもある
\ruby{其敵}{その|てき}を
\ruby{得}{え}て、
\ruby{罪無}{つみ|な}き
\ruby{樂}{たのし}みを
\ruby{取}{と}る
\ruby{一手}{いつ|て}々々の、
\ruby{興}{きよう}の
\ruby{極}{きは}めて
\ruby{旺}{さかん}なるところへ、
\ruby[g]{熟知}{なじみ}にもあらぬ
\ruby{病家}{びよ|うか}の、\GWI{u1b048}かも
\ruby{普通}{な|み}
\ruby{外}{はづ}れて
\ruby{{\GWI{u9060-k}}}{とほ}きより、
\ruby{夜陰}{や|いん}に
\ruby{及}{およ}びて
\ruby{呼}{よ}び
\ruby{{\GWI{u8fce-k}}}{むか}へんとするとも、
\ruby{門前}{もん|ぜん}の
\ruby{雀羅}{じや|くら}、
\ruby{藥局}{やく|きよく}の
\ruby{蛛網}{しゆ|まう}、
\ruby{客}{きやく}に
\ruby{饑}{う}ゑきつたる
\ruby{庸醫}{よう|い}はいざ
\ruby{知}{し}らず、
\ruby{苟}{いやし}くも
\ruby{名}{な}の
\ruby{{\GWI{u901a-k}}}{とほ}つたるほどの
\ruby{人}{ひと}の
\ruby{應}{おう}ぜざるべきは、
\ruby{思}{おも}へば
\ruby{無理}{む|り}も
\ruby{無}{な}き
\ruby{事{\換字{情}}}{わ|け}なりと、
\ruby{鈍}{にぶ}からぬ
\ruby[g]{水野}{みづの}は
\ruby{早}{はや}くも
\ruby{悟}{さと}りしが、
\ruby{物}{もの}に
\ruby{脆}{もろ}からぬ
\ruby{性質}{せい|しつ}の
\ruby{{\換字{猶}}}{なほ}
\ruby{思}{おも}ひ
\ruby{棄}{す}てず、
\ruby{何}{なに}をか
\ruby{考}{かんが}へ
\ruby{得}{え}しや
\ruby{此度}{こ|たび}は
\ruby{氣輕}{き|がる}く、

『ヤ、たび〳〵
\ruby{御面倒}{ご|めん|だう}を
\ruby{願}{ねが}ひまして、
\ruby{有}{あ}り
\ruby{難}{がた}うございました。
』

と、
\ruby{云}{い}ひながら
\ruby[g]{多少錢}{いくらか}を
\ruby{手早}{て|ばや}く
\ruby[g]{白色包}{かみづヽみ}にして、

『
\ruby{煙草}{た|ばこ}でも
\ruby{購}{と}つて
\ruby{參}{まゐ}つて
\ruby{獻}{あ}げるべきですが。
』

と、
\ruby{言葉}{こと|ば}を
\ruby{飾}{かざ}って
\ruby{取}{と}りつくろひ、
\ruby[g]{流石}{さすが}
\ruby{手}{て}を
\ruby{出}{いだ}しては
\ruby{取}{と}りかぬるを
\ruby{無理}{む|り}やりに
\ruby{握}{にぎ}らすれば、まさかに
\ruby{投}{な}げ
\ruby{{\GWI{u8fd4-k}}}{かへ}すこともせず、

『どうも
\ruby{御氣}{お|き}の
\ruby{毒}{どく}で、』

と、
\ruby{我}{わ}が
\ruby{師}{し}の
\ruby{{\GWI{u8fce-k}}}{むかへ}に
\ruby{應}{おう}ぜぬが
\ruby{氣}{き}の
\ruby{毒}{どく}なやら、
\ruby{我}{わ}が
\ruby{錢}{ぜに}
\ruby{使}{つか}はせしが
\ruby{氣}{き}の
\ruby{毒}{どく}なやら、どちら
\ruby{付}{つ}かぬ
\ruby{挨拶}{あい|さつ}して、うぢ〳〵と
\ruby{取}{と}りぬ。

\ruby{印}{いん}を
\ruby{結}{むす}び、
\ruby{呪}{じゆ}を
\ruby{誦}{じゆ}すること、
\ruby{今}{いま}は
\ruby{流行}{は|や}らず、
\ruby{世}{よ}にたゞ
\ruby{錢術}{せん|じゆつ}ありて
\ruby{神}{かみ}に
\ruby{{\GWI{u901a-k}}}{つう}ずるを、
\ruby{知}{し}らぬほど
\ruby{迂闊}{うく|わつ}にはあらざりし
\ruby[g]{水野}{みづの}は、
\ruby{書生}{しよ|せい}が
\ruby{我}{わ}が
\ruby[g]{人{\換字{情}}錢}{こヽろづけ}を
\ruby{収}{おさ}めしを
\ruby{見}{み}て、

『
\ruby{何様}{ど|う}いふものでございましやう?
\ruby{病人}{びやう|にん}が
\ruby{思}{おも}ひ
\ruby{込}{こ}んで
\ruby{居}{を}るのでございますから、
\ruby{一度}{いち|ど}だけなりと
\ruby{診}{み}て
\ruby{戴}{いたゞ}く
\ruby{譯}{わけ}には
\ruby{參}{まゐ}りますまいか。
こちらの
\ruby{先生}{せん|せい}の
\ruby{事}{こと}でございますから、
\ruby{澤山}{たく|さん}の
\ruby{御病家}{ご|びやう|か}の
\ruby{御都合}{ご|つ|がふ}もあつて、
\ruby{御暇}{お|ひま}の
\ruby{少}{すく}ないのは
\ruby{承知}{しよ|うち}して
\ruby{居}{を}りますから、
\ruby{始{\換字{終}}}{し|じゆう}
\ruby{來}{き}て
\ruby{戴}{いたゞ}きたいとは
\ruby{申}{まを}しますまいが、
\ruby{只一度}{たつた|いち|ど}おいでなすつて
\ruby{下}{くだ}さるほどの
\ruby{事}{こと}なら、
\ruby{然程}{さ|ほど}
\ruby{御暇}{お|ひま}の
\ruby{取}{と}れるでは
\ruby{無}{な}し、
\ruby{御都合}{ご|つ|がふ}の
\ruby{出來}{で|き}ぬでも
\ruby{無}{な}からうと
\ruby{存}{ぞん}じます。
\ruby{一度}{いち|ど}でも
\ruby{御診察}{ご|しん|さつ}
\ruby{下}{くだ}すつて、そして
\ruby{御指揮}{おさ|し|づ}を
\ruby{仕}{し}て
\ruby{戴}{いたゞ}いたら、あとは
\ruby{村醫}{そん|い}でも
\ruby{間}{ま}に
\ruby{合}{あ}はうかと
\ruby{存}{ぞん}じますが、
\ruby{病人}{びやう|にん}も
\ruby{信}{しん}じて
\ruby{居}{を}りませぬ
\ruby{村醫}{そん|い}ばかりでは、
\ruby{實以}{じつ|もつ}て
\ruby{傍観}{わき|め}にも
\ruby{案}{あん}じられまして、
\ruby{癒}{なほ}るものも
\ruby{癒}{なほ}るまいかと
\ruby{心配致}{しん|ぱい|いた}します。
\ruby{貴君}{あな|た}には
\ruby{御無理}{ご|む|り}を
\ruby{申}{まを}して
\ruby{濟}{す}みませんが、
\ruby{折}{を}り
\ruby{入}{い}つて
\ruby{一}{ひと}つ
\ruby{此}{こ}の
\ruby{譯}{わけ}を
\ruby{仰}{おつし}あつて、も
\ruby{一度}{いち|ど}
\ruby[g]{何卒}{どうぞ}
\ruby{御願}{お|ねが}ひなすつて
\ruby{見}{み}てはいたゞけますまいか。
』

と
\ruby{泣}{な}かぬばかりに
\ruby{掻口{\GWI{u8aaa-jv}}}{かき|く|ど}けば、
\ruby{書生}{しよ|せい}の
\ruby{面}{おもて}には
\ruby{難色}{なん|しよく}
\ruby{見}{み}えしが、
\ruby{既}{すで}に
\ruby{毒}{どく}を
\ruby{盛}{も}られたれば
\ruby{爭}{あらそ}ひ
\ruby{難}{がた}く、
\ruby{無下}{む|げ}に
\ruby{酷}{むご}くは
\ruby{斥}{しりぞ}けかねて、

『では
\ruby{始{\換字{終}}}{し|じゆう}
\ruby{病人}{びやう|にん}を
\ruby{受合}{うけ|あ}つて
\ruby{{\換字{呉}}}{く}れといふのでは
\ruby{無}{な}くつて、
\ruby{診斷}{しん|だん}だけで
\ruby{好}{よ}いからといふのぢやネ。
』

『ハイ、それで
\ruby{滿足}{まん|ぞく}
\ruby{致}{いた}しませうと
\ruby{申}{まを}しますのですから、
\ruby[g]{何卒}{どうか}
\ruby{枉}{ま}げて
\ruby{御聞入}{お|きヽ|い}れ
\ruby{下}{くだ}さるやうに
\ruby{御願}{お|ねが}ひなすつて。
』

と
\ruby{一問一答}{いち|もん|いつ|たふ}の
\ruby{果}{は}てし
\ruby{後}{のち}、
\ruby{澁}{しぶ}る〳〵
\ruby{{\換字{弱}}}{よわ}つた
\ruby{氣色}{け|しき}して
\ruby{奥}{おく}へ
\ruby{行}{ゆ}きぬ。
\ruby[g]{水野}{みづの}は
\ruby{病}{や}める
\ruby{我}{わ}が
\ruby{五十子}{い|そ|こ}が
\ruby{物憂}{もの|う}げに、
\ruby{此}{こ}の
\ruby{廣}{ひろ}き
\ruby{世}{よ}に
\ruby{只一人}{たゞ|ひ|とり}の
\ruby[g]{誠意}{まこと}ある
\ruby{介抱者}{かい|はう|しや}をも
\ruby{有}{も}たずして、
\ruby{頼}{たの}み
\ruby{少}{すくな}き
\ruby[g]{村醫}{そんい}の
\ruby{怪}{あや}しき
\ruby{藥}{くすり}をのみ
\ruby{力}{ちから}としつゝ、
\ruby{心淋}{こヽろ|さび}しくも
\ruby{秋}{あき}の
\ruby{夜}{よ}
\ruby{悲}{かな}しき
\ruby{田舎家}{ゐ|な|か}の
\ruby{一室}{ひと|ま}の
\ruby{内}{うち}に
\ruby{横}{よこた}はれる
\ruby[g]{光景}{ありさま}を
\ruby{胸}{むね}のうちに
\ruby{描}{ゑが}きながら、こたびの
\ruby{{\GWI{u8fd4-k}}事}{へん|じ}は
\ruby{如何}{い|か}にぞと、
\ruby{聞}{き}く
\ruby{耳}{みヽ}
\ruby{立}{た}てゝ
\ruby{意}{こヽろ}を
\ruby{注}{つ}くれば、

『うるさい!。
\GWI{u1b048}つゝこい!。
』

と
\ruby{叱}{しか}る
\ruby{聲}{こゑ}に
\ruby{次}{つ}いで、
\ruby{負}{ま}けかゝりたるに
\ruby{怒}{いかり}をや
\ruby{含}{ふく}みけん、パチリと
\ruby{{\換字{強}}}{つよ}く
\ruby{石}{いし}を
\ruby{下}{くだ}す
\ruby{音}{おと}して、やがて
\ruby{書生}{しよ|せい}は
\ruby{膨}{ふく}れかへつて
\ruby{出}{い}で
\ruby{來}{きた}りぬ。
\ruby{挨拶}{あい|さつ}は
\ruby{聞}{き}かずとも
\ruby{既}{はや}
\ruby{解}{わか}りたり。
されど
\ruby{如是}{か|く}ても
\ruby[g]{水野}{みづの}は
\ruby{屈}{くつ}せず、
\ruby{書生}{しよ|せい}が
\ruby{何}{なに}を
\ruby{云}{い}ひしやらも
\ruby{知}{し}らずに、
\ruby{如何}{い|か}にしてか
\ruby{我}{わ}が
\ruby{念}{おもひ}を
\ruby{{\GWI{u9042-k}}}{と}げんと
\ruby{考}{かんが}へ
\ruby{沈}{しづ}みし
\ruby{後}{のち}、
\ruby{思}{おも}ひ
\ruby{得}{え}しところやありけん
\ruby{頭}{かうべ}を
\ruby{擡}{あ}げしが、
\ruby{其}{そ}の
\ruby{面}{おもて}は
\ruby{何時}{い|つ}か
\ruby{聊}{いさヽ}か
\ruby{色}{いろ}ざし
\ruby{來}{きた}り、
\ruby{其}{そ}の
\ruby{眼}{め}よりは
\ruby{今}{いま}まで
\ruby{潛}{ひそ}み
\ruby{居}{ゐ}たりし
\ruby{烱々}{けい|〳〵}たる
\ruby{光}{ひかり}の
\ruby{閃}{ひらめ}き
\ruby{出}{い}でゝ、
\ruby{見}{み}る〳〵
\ruby{如何}{い|か}なる
\ruby[g]{任務}{つとめ}にも
\ruby{堪}{た}ふべく、
\ruby{如何}{い|か}なる
\ruby{人}{ひと}にも
\ruby{爭}{あらそ}つて
\ruby{勝}{か}つべき
\ruby{峻烈}{しゆん|れつ}の
\ruby{氣象}{き|しやう}を
\ruby{現}{あらは}し
\ruby{出}{いだ}しぬ。
\ruby{折}{をり}から
\ruby{一}{ひと}つの
\ruby{彼}{か}の
\ruby{小}{ちひ}さき
\ruby{蛾}{が}は、
\ruby{力盡}{ちから|つ}き
\ruby{翼傷}{つばさ|きず}つきて
\ruby{翩々}{ひら|〳〵}として、
\ruby{落花}{らく|くわ}の
\ruby{枝}{えだ}を
\ruby{辭}{じ}せしが
\ruby{如}{ごと}くに、あはれにも
\ruby[g]{水野}{みづの}が
\ruby{膝}{ひざ}の
\ruby{前}{まへ}に
\ruby{墜}{お}ちぬ。


\Entry{其九}

\ruby{頭}{かうべ}を
\ruby{下}{さ}げ
\ruby{言葉}{こと|ば}を
\ruby{低}{ひく}くして、
\ruby{頼}{たの}むほどは
\ruby{頼}{たの}み
\ruby{盡}{つく}せしを、
\ruby{膠無}{にべ|な}く
\ruby{色}{いろ}なく
\ruby{斷}{ことわ}りに
\ruby{斷}{ことわ}られたり。
\ruby{今}{いま}は
\ruby{復言}{また|い}うべき
\ruby{餘地}{よ|ち}も
\ruby{無}{な}からんを、
\ruby{水野}{みづ|の}はそも〳〵
\ruby{何}{なん}とせんとかする。

\ruby{水}{みづ}をもて
\ruby{解}{と}くべからざるものは
\ruby{火}{ひ}をもて
\ruby{熔}{と}かすべし、
\ruby{刀}{たう}をもて
\ruby{截}{き}り
\ruby{難}{がた}きものは
\ruby{槌}{つち}をもて
\ruby{碎}{くだ}き
\ruby{得}{え}ん。
\ruby{求}{もと}めて
\ruby{已}{や}まぬ
\ruby[g]{願望}{ねがひ}の
\ruby{心}{こゝろ}あれば、おのづと
\ruby{働}{はたら}く
\ruby{智慧}{ち|ゑ}の
\ruby{眼}{まなこ}は、
\ruby{我}{わ}が
\ruby{思}{おも}へる
\ruby{地}{ち}に
\ruby{到}{いた}らんとするに
\ruby{{\換字{平}}和}{おだ|やか}なる
\ruby{路}{みち}を
\ruby{取}{と}ることの
\ruby{甲斐無}{か|ひ|な}きを
\ruby{悟}{さと}りたらん
\ruby{曉}{あかつき}、いかで
\ruby{{\換字{猶}}}{なほ}
\ruby{別}{べつ}に
\ruby{峻}{さか}しき
\ruby{一}{ひ}ト
\ruby{條}{すぢ}の
\ruby{徑}{こみち}ありて
\ruby{其處}{そ|こ}に
\ruby{{\換字{通}}}{つう}ずるを
\ruby{見出}{み|いだ}さゞらんや。
\ruby{水野}{みづ|の}は
\ruby{今}{いま}その
\ruby{峻}{さか}しきを
\ruby{見出}{み|いだ}して
\ruby{攀}{よ}ぢ
\ruby{上}{のぼ}らんとするなり。
\ruby{火}{ひ}の
\ruby{力}{ちから}、
\ruby{槌}{つち}の
\ruby{力}{ちから}を
\ruby{試}{こゝろ}みんとするなり。

\ruby{其}{そ}の
\ruby{顏}{かほ}つきの
\ruby{變}{かは}れる
\ruby{如}{ごと}くに、
\ruby{言葉}{こと|ば}の
\ruby{調子}{てう|し}も
\ruby{俄}{にはか}に
\ruby{變}{かは}り、
\ruby{聲}{こゑ}も\換字{志}たたかに
\ruby{大}{おおき}くなりぬ。

『いよ〳〵
\ruby{先生}{せん|せい}は
\ruby{御來臨}{お|い|で}
\ruby{下}{くだ}さらんと
\ruby{仰}{おつし}あるのですか。
イヤ、それは
\ruby{失禮}{しつ|れい}ながら
\ruby{左樣}{さ|う}ではございますまい、
\ruby{御取次}{お|とり|つぎ}の
\ruby{御言葉}{お|こと|ば}が
\ruby{足}{た}らんので、
\ruby{先生}{せん|せい}に
\ruby[g]{御理解}{おわかり}が
\ruby{無}{な}いのでしやう。
\ruby{{\換字{遠}}方}{えん|ぱう}だから
\ruby{行}{い}つて
\ruby{{\換字{遣}}}{や}らぬと、そんな
\ruby{事}{こと}を
\ruby{仰}{おつし}ある
\ruby{先生}{せん|せい}では
\ruby{無}{な}い、そんな
\ruby{無慈悲}{む|じ|ひ}な
\ruby{先生}{せん|せい}では
\ruby{無}{な}い。
…… 』

と、
\ruby{今}{いま}までは
\ruby{頭}{あたま}の
\ruby{低}{ひく}かりし
\ruby{男}{をとこ}の、
\ruby[g]{居丈高}{ゐたけだか}になつて、
\ruby{思}{おも}ひの
\ruby{外}{ほか}なる
\ruby{{\換字{強}}言}{しひ|ごと}を
\ruby{云}{い}い
\ruby{出}{いだ}せば、
\ruby{書生}{しよ|せい}は
\ruby{其}{そ}の
\ruby{意外}{いぐ|わい}なるに
\ruby{度}{ど}を
\ruby{失}{うしな}つて、
\ruby[g]{狼狽}{うろた}へながらも
\ruby{怫然}{ふつ|ぜん}として、
\ruby{急}{きふ}に
\ruby{遮}{さへぎ}り
\ruby{止}{とゞ}めんと、

『バ、バ、
\ruby{馬鹿}{ば|か}な
\ruby{事}{こと}を、』

と、
\ruby[g]{眞赤}{まつか}になりて
\ruby[g]{抗辯}{あらが}はんとしけるが、
\ruby{紫電閃}{し|でん|ひら}めきて
\ruby{出}{い}づるが
\ruby{如}{ごと}き
\ruby{水野}{みづ|の}の
\ruby{恐}{おそ}ろしき
\ruby{眼}{め}に
\ruby{眼}{め}を
\ruby{見合}{み|あは}せて、
\ruby{睨}{にら}み
\ruby{殺}{ころ}さんばかりに
\ruby{我}{われ}を
\ruby{見据}{み|す}ゑたる
\ruby{其}{そ}の
\ruby{異}{あや}しき
\ruby{力}{ちから}に
\ruby[g]{所以無}{いはれな}くも
\ruby{氣壓}{け|お}され、
\ruby{云}{い}ひ
\ruby{甲斐無}{か|ひ|な}くも
\ruby{當}{あた}り
\ruby{難}{がた}くおぼえて、
\ruby{我知}{われ|し}らず
\ruby{面}{おもて}を
\ruby{背向}{そ|む}け
\ruby{言葉}{こと|ば}を
\ruby{吞}{の}みたり。
\ruby{水野}{みづ|の}は
\ruby{相手}{あひ|て}のたぢろぎしに
\ruby{{\換字{緩}}}{ゆる}みを
\ruby{{\換字{呉}}}{く}れず、
\ruby{往來}{わう|らい}にも
\ruby{鳴}{な}り
\ruby{渡}{わた}れ、
\ruby{奥}{おく}にも
\ruby{響}{ひび}けと、いよ〳〵
\ruby{聲}{こゑ}を
\ruby{高}{たか}め、
\ruby{言葉}{こと|ば}を
\ruby{荒}{あら}くして、

『
\ruby{御當家}{こ|ち|ら}の
\ruby{先生}{せん|せい}は
\ruby{仁慈深}{な|さけ|ぶか}い
\ruby{先生}{せん|せい}だ、
\ruby{取次}{とり|つぎ}の
\ruby{君}{きみ}がまだ
\ruby{新參}{しん|ざん}で、
\ruby{御當家}{こ|ち|ら}の
\ruby{御風儀}{ご|ふう|ぎ}を
\ruby{知}{し}らんので、
\ruby{中{\換字{途}}}{ちゆう|と}で
\ruby{間{\換字{違}}}{ま|ちが}つた
\ruby{忠義立}{ちゆう|ぎ|だて}で
\ruby{計}{はか}らつて、
\ruby{其樣}{そ|ん}な
\ruby{好}{い}い
\ruby{加減}{か|げん}な
\ruby{事}{こと}を
\ruby{御言}{お|い}ひのだ。
\ruby{御慈悲深}{お|じ|ひ|ぶか}い
\ruby{此方}{こち|ら}の
\ruby{先生}{せん|せい}だもの、
\ruby{{\換字{遠}}方}{ゑん|ぽう}だつて
\ruby{來}{き}て
\ruby{下}{くだ}さるのだ。
\ruby{世間}{せ|けん}にも
\ruby{有}{あ}り
\ruby{觸}{ふ}れた
\ruby{藥賣}{くす|りう}り
\ruby{坊主}{ばう|ず}と、
\ruby{此方}{こち|ら}の
\ruby{先生}{せん|せい}とは
\ruby{譯}{わけ}が
\ruby{{\換字{違}}}{ちが}ふ。
\ruby{商賣}{しやう|ばい}づくばかりで
\ruby{病人}{びやう|にん}をいぢる、
\ruby{其樣}{そ|ん}な
\ruby{卑劣}{ひ|れつ}くさい
\ruby{先生}{せん|せい}では
\ruby{無}{な}いのだ、
\ruby{先生}{せん|せい}の
\ruby{御性{\換字{分}}}{ご|しやう|ぶん}の
\ruby{美}{うつく}しい
\ruby{御慈悲深}{お|じ|ひ|ぶか}いのは
\ruby{誰}{たれ}だつて
\ruby{知}{し}つて
\ruby{居}{ゐ}る。
\ruby{他人}{ひ|と}も
\ruby{知}{し}つて
\ruby{居}{ゐ}る、
\ruby{自{\換字{分}}}{じ|ぶん}も
\ruby{知}{し}つて
\ruby{居}{ゐ}る。
\ruby{先生}{せん|せい}で
\ruby{無}{な}くちやあならんと
\ruby{云}{い}つて、
\ruby{御願}{お|ねが}ひ
\ruby{申}{まを}すのに
\ruby{來}{き}て
\ruby{下}{くだ}さらん、そんな
\ruby{仁慈}{な|さけ}の
\ruby{無}{な}い
\ruby{先生}{せん|せい}では
\ruby{無}{な}い。
\ruby{先生}{せん|せい}の
\ruby{御氣性}{ご|き|しやう}も
\ruby{知}{し}らないで、
\ruby{何}{なに}を
\ruby[g]{{\換字{寝}}惚}{ねとぼ}けた
\ruby{挨拶}{あい|さつ}をするのだ。
』

と、
\ruby{口}{くち}も
\ruby{開}{あ}かせず
\ruby{疊}{たゝ}みかけて、
\ruby{{\換字{猶}}}{なほ}も
\ruby{止}{と}め
\ruby{度}{ど}
\ruby{無}{な}く
\ruby{罵}{のゝし}らんとす。
\ruby{此}{こ}の
\ruby{時}{とき}
\ruby{藥局}{やく|きよく}の
\ruby{内}{うち}こと〳〵と
\ruby{音}{おと}して、
\ruby{物騒}{もの|さわ}がしき
\ruby{此場}{この|ば}の
\ruby{樣子}{やう|す}を、
\ruby{何事}{なに|ごと}かと
\ruby{他}{た}の
\ruby{書生}{しよ|せい}の
\ruby{覗}{うかゞ}ひに
\ruby{來}{き}しとおぼしく、
\ruby{{\換字{又}}}{また}
\ruby{今}{いま}の
\ruby{間}{ま}に
\ruby{來}{き}し
\ruby{二三人}{に|さん|にん}の
\ruby{藥取}{くすり|と}りは、こそ〳〵と
\ruby{隅}{すみ}の
\ruby{方}{かた}に
\ruby{潛}{ひそ}み
\ruby{居}{ゐ}て
\ruby{成行}{なり|ゆき}を
\ruby{見}{み}、はや
\ruby{門}{もん}の
\ruby{外}{そと}にはちらりほらりと、
\ruby{人}{ひと}さへ
\ruby{立}{た}ちて
\ruby{見居}{み|ゐ}るさまなり。

\ruby{書生}{しよ|せい}は
\ruby{心}{こゝろ}も
\ruby{心}{こゝろ}ならず、

『マア
\ruby{左樣}{そ|う}
\ruby{大}{おほき}な
\ruby{聲}{こゑ}を
\ruby{立}{た}てゝは
\ruby{困}{こま}るぢや
\ruby{無}{な}いか。
』

と
\ruby{制}{せい}すれども
\ruby{耳}{みゝ}にも
\ruby{入}{い}るればこそ、

『つまり
\ruby{君}{きみ}のやうな
\ruby{取次}{とり|つぎ}は
\ruby{先生}{せん|せい}の
\ruby{不利{\換字{益}}}{ふ|た|め}だ、
\ruby{先生}{せん|せい}の
\ruby{{\換字{評}}{\換字{判}}}{ひやう|ばん}を
\ruby{惡}{わる}くする。
\ruby{{\換字{技}}{\換字{術}}}{わ|ざ}ばかり
\ruby{良}{よ}い
\ruby{先生}{せん|せい}では
\ruby{無}{な}い、
\ruby{御優}{お|やさ}しいので
\ruby{人徳}{にん|とく}のある
\ruby{先生}{せん|せい}をそれぢやあ
\ruby{臺無}{だい|な}しに
\ruby{仕}{し}て
\ruby{仕舞}{し|ま}ふでは
\ruby{無}{な}いか。
さつさと
\ruby{{\換字{猶}}一度}{も|いち|ど}
\ruby{奧}{おく}へ
\ruby{行}{い}つて
\ruby{願}{ねが}つて
\ruby{來}{き}てくれ。
\ruby{願}{ねが}ひ
\ruby{直}{なほ}して
\ruby{{\換字{呉}}}{く}れなければ
\ruby{此處}{こ|ゝ}は
\ruby{動}{うご}かん。
\ruby{病人}{びやう|にん}が
\ruby{先生}{せん|せい}で
\ruby{無}{な}ければと
\ruby{云}{い}つて
\ruby{首}{くび}を
\ruby{{\換字{延}}}{の}ばして
\ruby{待}{ま}つて
\ruby{居}{ゐ}るのだ、
\ruby{先生}{せん|せい}の
お
\ruby{供}{とも}を
\ruby{仕}{し}て
\ruby{歸}{かへ}らなけりやあ
\ruby{此處}{こ|ゝ}は
\ruby{動}{うご}かん。
\ruby{書生}{しよ|せい}の
\ruby{癖}{くせ}に
\ruby{有}{あ}る
\ruby{間敷事}{ま|じき|こと}だ。
\ruby{碁}{ご}なぞに
\ruby{凝}{こ}つて
\ruby{居}{ゐ}るやうだから
\ruby{取次}{とり|つぎ}が
\ruby{間{\換字{違}}}{ま|ちが}ふのだ。
さあ
\ruby{確乎}{しつ|かり}として
\ruby{先生}{せん|せい}に
\ruby{願}{ねが}つて
\ruby{見}{み}て
\ruby{{\換字{呉}}}{く}れ。
うるさい、\換字{志}つゝこい、とは
\ruby{何}{なん}の
\ruby{事}{こと}だ。
\換字{志}つゝこい
\ruby{人間}{にん|げん}に
\ruby{恨}{うら}まれたら、
\ruby{先生}{せん|せい}に
\ruby{飛}{と}んだ
\ruby{御{\換字{迷}}惑}{ご|めい|わく}が
\ruby{掛}{かゝ}らう、
\ruby{祟}{たゝ}りかね
\ruby{無}{な}いものだと
\ruby{思}{おも}ふか。
』

と、
\ruby[g]{次第}{しだい}〳〵に
\ruby{聲高}{こわ|だか}に
\ruby{云}{い}へば、
\ruby{門外}{もん|ぐわい}に
\ruby{人}{ひと}は
\ruby{愈々}{いよ|〳〵}
\ruby{嵩}{かさ}みて、
\ruby{奧}{おく}の
\ruby{方}{かた}は
\ruby{人}{ひと}の
\ruby{氣}{け}もせず
\ruby[g]{靜謐}{しづか}になりぬ。

\ruby{時}{とき}に
\ruby{此室}{こ|ゝ}と
\ruby{奧}{おく}との
\ruby{劃域}{し|きり}はするりと
\ruby{開}{あ}いて、
\ruby{立出}{たち|いで}でたる
\ruby{{\換字{猶}}}{なほ}
\ruby{若}{わか}き
\ruby{此家}{この|や}の
\ruby{主人}{しゆ|じん}は、
\ruby{福々}{ふく|〴〵}しく
\ruby{肥}{ふと}りたる
\ruby{其顔}{その|かほ}に、
\ruby{莞爾}{にこ|やか}なる
\ruby{笑}{えみ}をつくりて、

『ヤ、
\ruby{取次}{とり|つぎ}のものを
\ruby{御叱}{お|しか}りでは
\ruby{恐}{おそ}れ
\ruby{入}{い}る。
\ruby{直}{すぐ}と
\ruby{今}{いま}から
\ruby{出}{で}ますから、さあ\kundoku{一}{ひ}{ト}{}\kundoku{足}{あし}{}{}
\ruby{御先}{お|さき}へ。
\ruby{相田}{あひ|だ}!、
\ruby{所}{ところ}は
\ruby{{\換字{分}}}{わか}かつて
\ruby{居}{ゐ}るだらうな、ムヽ
\ruby{左樣}{さ|う}か、
\ruby{直}{すぐ}と
\ruby{車}{くるま}の
\ruby{{\換字{支}}度}{し|たく}をさせろ。
』

と、
\ruby{卒直}{そつ|ちよく}に
\ruby{水野}{みづ|の}に
\ruby{滿足}{まん|ぞく}を
\ruby{與}{あた}へぬ。

\ruby{水野}{みづ|の}は、
\ruby{此}{こ}の
\ruby{己}{おのれ}に
\ruby{克}{か}つことを
\ruby{知}{し}つて
\ruby{非}{ひ}を
\ruby{{\換字{遂}}}{と}げんともせざる
\ruby{良醫}{りやう|い}の
\ruby{前}{まへ}に、
\ruby{心}{こゝろ}よりの
\ruby{感謝}{かん|しや}の
\ruby{禮}{れい}を
\ruby{深々}{ふか|〴〵}と
\ruby{施}{ほどこ}して、
\ruby{欣}{よろこ}び
\ruby{勇}{いさ}んで
\ruby[g]{室外}{おもて}に
\ruby{出}{い}でぬ。

\ruby{惡}{あ}しき
\ruby{兆}{しるし}かと
\ruby{忌}{いま}はしかりし
\ruby{彼}{か}の
\ruby{蛾}{が}の
\ruby{弄}{なぶ}りし
\ruby{電燈}{でん|とう}の
\ruby{下}{した}は
\ruby{去}{さ}つて、
\ruby{藍色滴}{らん|しよく|したゝ}るが
\ruby{如}{ごと}き
\ruby{澄}{す}みたる
\ruby{天}{そら}に、
\ruby{星}{ほし}は
\ruby{梨子地}{な|し|じ}を
\ruby{描}{か}きたらんやうに
\ruby{光}{ひか}り
\ruby{輝}{かがや}けるを、
\ruby{振}{ふ}り
\ruby{仰}{あお}ぎて
\ruby{眺}{なが}めたる
\ruby{可憐}{か|れん}の
\ruby{水野}{みづ|の}は、
\ruby{我}{わ}が
\ruby{意}{こゝろ}の
\ruby{中}{うち}の
\ruby{其人}{その|ひと}のために、
\ruby{思}{おも}ふ
\ruby{事}{こと}
\ruby{{\換字{遂}}}{と}げたる
\ruby{嬉}{うれ}しさに
\ruby{頭}{かしら}
\ruby{高}{ たか}き
\ruby{心地}{こゝ|ち}して、
\ruby{水色}{みず|いろ}の
\ruby{光}{ひか}り
\ruby{特}{こと}に
\ruby{優}{すぐ}れたる
\ruby{一}{ひと}つの
\ruby{星}{ほし}に
\ruby{眼}{まなこ}を
\ruby{止}{とど}めて、
\ruby{少時}{しば|し}は
\ruby{人知}{ひと|し}らぬ
\ruby{胸}{むね}の
\ruby{{\換字{凉}}}{すゞ}しさを
\ruby{味}{あじは}ひたり。


\Entry{其十}

\ruby{先挽後推}{さき|びき|あと|おし}の
\ruby{勢}{いきほひ}よく、
\ruby{矢}{や}を
\ruby{射}{い}る
\ruby{如}{ごと}くに
\ruby{走}{はし}れる
\ruby{相良}{さが|ら}の
\ruby{車}{くるま}は、
\ruby{長橋}{ちやう|けう}を
\ruby{東}{ひがし}に
\ruby{渡}{わた}つて
\ruby{小梅}{こ|うめ}にかゝり、
\ruby{引舟{\換字{通}}}{ひき|ふね|どほ}りを
\ruby{眞直}{まつ|すぐ}に
\ruby{北}{きた}へと、
\ruby{夜風}{よ|かぜ}のやや
\ruby{{\換字{寒}}}{さむ}きを
\ruby{衝}{つ}いて
\ruby{{\換字{進}}}{すゝ}みに
\ruby{{\換字{進}}}{すゝ}みぬ。
\ruby{{\換字{道}}}{みち}は
\ruby{砥}{と}の
\ruby{如}{ごと}し、
\ruby{人}{ひと}の
\ruby{往來}{ゆき|き}は
\ruby{無}{な}し、
\ruby{車夫}{しや|ふ}は
\ruby{脚一杯}{あし|いつ|ぱい}に
\ruby{駈}{か}くるほどに、おほよその
\ruby{二里}{に|り}を
\ruby{瞬}{またゝ}く
\ruby{間}{ま}に
\ruby{{\換字{過}}}{す}ぎて、
\ruby{忽地}{たち|まち}にして
\ruby{目}{め}ざす
\ruby{四}{よ}ツ
\ruby{木}{ぎ}へと
\ruby{着}{つ}きぬ。

\ruby{病人}{びやう|にん}の
\ruby{大切}{たい|せつ}さは
\ruby{貧富}{ひん|ぷ}に
\ruby{關}{かゝ}はらぬ
\ruby{事}{こと}ながら、
\ruby{市街}{ま|ち}
\ruby{離}{はな}れたる
\ruby{{\換字{遠}}}{とほ}きところより、
\ruby{夜}{よ}にさへ
\ruby{入}{い}りたるに
\ruby{無理{\換字{強}}}{む|り|じひ}に
\ruby{{\換字{強}}}{し}ひて、
\ruby{我}{わ}が
\ruby{先生}{せん|せい}を
\ruby{{\換字{迎}}}{むか}へたるは、
\ruby{田舎}{ゐな|か}とは
\ruby{云}{い}へ、
\ruby{定}{さだ}めし
\ruby{門構}{もん|がま}への
\ruby{立派}{りつ|ぱ}に、
\ruby{庭{\換字{前}}}{には|さき}
\ruby{廣}{ひろ}く、がつしりとしたる
\ruby{槻柱}{けやき|ばしら}の
\ruby{太}{ふと}きが、
\ruby{二尺}{に|しやく}も
\ruby{厚}{あつ}さのある
\ruby{茅葺屋根}{かや|ぶき|や|ね}のいと
\ruby{高}{たか}く
\ruby{大}{おほき}なるを
\ruby{支}{さゝ}へたるやうの
\ruby{家}{いへ}ならんと、
\ruby{車夫}{しや|ふ}は
\ruby{心}{こゝろ}の
\ruby{中}{うち}に
\ruby{算}{つも}り
\ruby{居}{ゐ}けるが、
\ruby{{\換字{分}}}{わか}り
\ruby{{\換字{兼}}}{か}ぬる
\ruby{闇}{やみ}の
\ruby{村逕}{むら|みち}を
\ruby{{\換字{迷}}}{まよ}ひ〳〵て、やうやくに
\ruby{{\換字{尋}}}{たづ}ね
\ruby{當}{あ}てたるは
\ruby{是}{これ}は
\ruby{如何}{い|か}な
\ruby{事}{こと}、
\ruby{{\換字{寒}}竹}{かん|ちく}の
\ruby{藪疊}{やぶ|だゝみ}の
\ruby{不體裁}{ぶ|ざ|ま}に
\ruby{歪}{ゆが}みたる
\ruby{其}{そ}の
\ruby{構}{かまへ}の
\ruby{中}{うち}こそは
\ruby{意外}{い|ぐわい}に
\ruby{濶}{ひろ}けれ、
\ruby{{\換字{空}}}{むな}しく
\ruby{明}{あ}け
\ruby{置}{お}く
\ruby{地}{ち}を
\ruby{惜}{をし}んでか、
\ruby{{\換字{通}}}{かよ}ひ
\ruby{路}{ぢ}をも
\ruby{埋}{うづ}むるまでに
\ruby{作}{つく}りたる
\ruby{芋}{いも}の
\ruby{圃}{はたけ}の
\ruby{奧}{おく}に、
\ruby{微}{かす}けき
\ruby{星}{ほし}のひかりを
\ruby{{\換字{浴}}}{あ}びて
\ruby{黑}{くろ}みて
\ruby{立}{た}てる、
\ruby{見}{み}るからが
\ruby{悲}{かな}しき
\ruby{草}{くさ}の
\ruby{屋}{や}なり。

\ruby{餘}{あま}りの
\ruby{思}{おも}はくの
\ruby{{\換字{違}}}{ちが}ひの
\ruby{忌々}{いま|〳〵}しくてや、
\ruby{車夫}{しや|ふ}は
\ruby{憚}{はゞか}り
\ruby{氣}{げ}
\ruby{無}{な}く
\ruby{人力車}{く|る|ま}を
\ruby{挽}{ひ}き
\ruby{入}{い}るれば、
\ruby{車輪}{しや|りん}に
\ruby{觸}{ふ}るゝ
\ruby{芋}{いも}の
\ruby{葉}{は}は
\ruby{左右}{さ|いう}に
\ruby{開}{ひら}けて、
\ruby{湛}{たゝ}へられし
\ruby{露}{つゆ}の
\ruby{珠}{たま}は
\ruby{墜}{お}ちて
\ruby{聲}{こゑ}あり。

\ruby{人}{ひと}ありや
\ruby{無}{な}しや
\ruby{岑閑}{しん|かん}として、たゞ
\ruby{燈}{ひ}のみ
\ruby{洩}{も}るゝ
\ruby{板{\換字{戸}}}{いた|ど}を
\ruby{敲}{たゝ}き
\ruby{驚}{おどろ}かしつゝ
\ruby{車夫}{しや|ふ}は
\ruby{聲明}{こゑ|あき}らかにそれと
\ruby{云}{い}ひ
\ruby{入}{い}るれば、
\ruby{何}{なに}を
\ruby{擱}{さしお}きても
\ruby{飛}{と}んで
\ruby{出}{い}でゝ、
\ruby{喜}{よろこ}び〳〵て
\ruby{{\換字{迎}}}{むか}へ
\ruby{入}{い}るべきを、
\ruby{是}{これ}はまた
\ruby{何}{なん}たる
\ruby{事}{こと}ぞ
\ruby{沈着}{おち|つ}き
\ruby{拂}{はら}つて、

『ハア、
\ruby{左樣}{さ|う}ですかい!。
』

と、
\ruby{田舎}{ゐな|か}
\ruby{詞}{ことば}の
\ruby{素氣無}{す|げ|な}く
\ruby{答}{こた}へたるのみにて
\ruby{嬉}{うれ}しき
\ruby{顏}{かほ}もせねば、
\ruby{{\換字{請}}}{しやう}じ
\ruby{入}{い}れんともせず、
\ruby{折}{をり}から
\ruby{自裂}{は|じ}け
\ruby{{\換字{兼}}}{か}ねたる
\ruby{大豆}{ま|め}の
\ruby{莢}{さや}を
\ruby{取}{と}るにやあらん、
\ruby{箕}{み}を
\ruby{{\換字{前}}}{まへ}にして
\ruby{乾}{かは}きたる
\ruby{豆}{まめ}を
\ruby{弄}{いぢ}り
\ruby{居}{ゐ}し
\ruby{婆}{ばゞ}の、
\ruby{面}{おもて}は
\ruby{赭黄色}{あか|き|いろ}く
\ruby{焦}{や}け
\ruby{皺}{しわ}びて、
\ruby{髪}{かみ}は
\ruby{天蠶糸屑}{て|ぐ|す|くず}の
\ruby{如}{ごと}く
\ruby{白}{しろ}く
\ruby{光}{ひか}るが
\ruby{{\換字{交}}}{まじ}れる、
\ruby{年}{とし}の
\ruby{頃}{ころ}は
\ruby{六十}{ろく|じう}ばかりなるが、
\ruby{不承不承}{ふ|しよう|ぶ|しよう}に
\ruby{身}{み}を
\ruby{起}{おこ}して
\ruby{{\換字{戸}}口}{と|ぐち}に
\ruby{立塞}{たち|ふさ}がり、

『
\ruby{病人}{びやう|にん}は
\ruby{此處}{こ|ゝ}には
\ruby{居}{を}りましねえ。
\ruby{別室}{はな|れ}の
\ruby{方}{はう}に
\ruby{寢}{ね}て
\ruby{居}{を}りますから、
\ruby{直}{すぐ}とそつちらへ
\ruby{御座}{ご|ざ}らしつて
\ruby{下}{くだ}さい。
\ruby{暗}{くら}くつて
\ruby{{\換字{分}}}{わか}りますまいが
\ruby{足元}{あし|もと}は
\ruby{好}{い}いでがす。
\ruby{家}{うち}へさへ
\ruby{付}{つ}いて
\ruby{{\換字{廻}}}{まは}れば
\ruby{直}{ぢき}でがすよ。
あ、\換字{志}かし
\ruby{{\換字{菜}}}{な}
\ruby{圃}{ばたけ}へでも
\ruby{轉}{ころ}げられると
\ruby{詰}{つま}らない。
\ruby{水野}{みづ|の}さんが
\ruby{後}{あと}になつたゞから
\ruby{仕方}{し|かた}が
\ruby{無}{な}い、
\ruby{妾}{わし}が
\ruby{案内}{あん|ない}を
\ruby{仕}{し}てあげやう。
ヤ、
\ruby{車夫}{くる|まや}さん、
\ruby{提灯}{ちやう|ちん}があるの、
\ruby{其}{そ}の
\ruby{提灯}{ちやう|ちん}を
\ruby{妾}{わし}に
\ruby{貸}{か}さつせえ。
さあ
\ruby{先生}{せん|せい}さん、
\ruby{妾}{わし}に
\ruby{隨}{つ}いて
\ruby{御坐}{ご|ざ}らつせえ。
』

と、
\ruby{藁草履}{わら|ざう|り}つゝかけて
\ruby{先}{さき}に
\ruby{立}{た}つたり。
\ruby{相良}{さが|ら}は
\ruby{是非無}{ぜ|ひ|な}く
\ruby{後}{あと}に
\ruby{隨}{つ}きて、
\ruby{家}{いへ}の
\ruby{横手}{よこ|て}を
\ruby{斜}{なゝめ}に
\ruby{奧}{おく}へ、
\ruby{此方}{こな|た}には
\ruby{燃料}{たき|れう}の
\ruby{柴木}{しば|き}の
\ruby{積}{つ}まれ、
\ruby{彼方}{かな|た}には
\ruby{玉蜀黍幹}{たう|もろ|こし|がら}の
\ruby{埒無}{らち|な}く
\ruby{置}{お}かれなどしたる
\ruby{間}{あひだ}を
\ruby{縫}{ぬ}ひて、さて、
\ruby{下}{した}は
\ruby{夏蒔}{なつ|まき}の
\ruby{{\換字{菜}}}{な}の
\ruby{圃}{はたけ}の
\ruby{細徑}{ほそ|みち}の
\ruby{滑}{すべ}り
\ruby{易}{やす}く、
\ruby{上}{うへ}は
\ruby{{\換字{柿}}}{かき}の
\ruby{樹}{き}の
\ruby{幾本}{いく|ほん}の
\ruby{枝低}{えだ|ひく}くして
\ruby{帽子}{ばう|し}
\ruby{危}{あやふ}きところを
\ruby{{\換字{過}}}{す}ぐれば、
\ruby{{\換字{前}}}{まへ}の
\ruby{家}{いへ}よりは
\ruby{彼是}{かれ|これ}
\ruby{二十間餘}{に|じう|けん|あま}りも
\ruby{距}{はな}れたりとおぼしきところに、
\ruby{椎}{しひ}の
\ruby{樹}{き}ならん
\ruby{眞黑}{まつ|くろ}に
\ruby{見}{み}ゆる
\ruby{{\換字{丈}}矮}{たけ|ひく}き
\ruby{樹}{き}のいと
\ruby{大}{おほい}なるを
\ruby{後楯}{うしろ|だて}に
\ruby{取}{と}りて、
\ruby{僅}{わづか}に
\ruby{二}{ふ}タ
\ruby{室}{ま}ほどなるべき
\ruby{離屋}{はな|れや}
\ruby{立}{た}てり。

『さあ
\ruby{此處}{こ|ゝ}でがあす、
\ruby{上}{あが}つて
\ruby{下}{くだ}さい。
』

と、
\ruby{婆}{ばゞ}は
\ruby{{\換字{戸}}}{と}を
\ruby{引}{ひ}き
\ruby{明}{あ}けてつか〳〵と
\ruby{上}{あが}りぬ。

『お
\ruby{{\換字{前}}}{まへ}さまが
\ruby{頼}{たの}み
\ruby{度}{た}いと
\ruby{云}{い}つた
\ruby{先生}{せん|せい}がござらしつた。
』

と、
\ruby{云}{い}ひながら
\ruby{次}{つぎ}の
\ruby{室}{ま}の
\ruby{長四疊}{なが|よ|でふ}を
\ruby{{\換字{過}}}{す}ぎて、
\ruby{六疊}{ろく|でふ}の
\ruby{其}{そ}の
\ruby{室}{ま}に
\ruby{至}{いた}りたれど、
\ruby{熱}{ねつ}の
\ruby{一}{ひ}ト
\ruby{退}{ひき}
\ruby{退}{ひ}きし
\ruby{汐合}{しほ|あひ}の
\ruby{時}{とき}にや、
\ruby{病人}{びやう|にん}は
\ruby{答}{こた}へも
\ruby{無}{な}く
\ruby{音}{おと}も
\ruby{無}{な}く
\ruby{眠}{ねむ}り
\ruby{居}{を}れり。

\ruby{醫師}{い|し}は
\ruby{婆}{ばゞ}につゞきて
\ruby{上}{あが}りけるが、
\ruby{先}{ま}ず
\ruby{此}{こ}の
\ruby{室}{ま}に
\ruby{籠}{こも}りたる
\ruby{不快}{ふ|くわい}の
\ruby{臭氣}{にほ|い}に、
\ruby{不審}{ふ|しん}の
\ruby{眉}{まゆ}を
\ruby{顰}{ひそ}めて\換字{志}ろりと
\ruby{見渡}{み|わた}せば、
\ruby{廣}{ひろ}からぬ
\ruby{一室}{ひと|ま}の
\ruby{内}{うち}
\ruby{法外}{はふ|ぐわい}に
\ruby{明}{あか}るく、
\ruby{病人}{びやう|にん}が
\ruby{枕上}{まくら|もと}の
\ruby{洋燈}{らん|ぷ}は
\ruby{何時}{いつ|か}か
\ruby{燃}{も}え
\ruby{高}{かう}じて、
\ruby{其}{そ}の
\ruby{火屋}{ほ|や}の
\ruby{上}{うへ}の
\ruby{方}{かた}は
\ruby{眞黑}{まつ|くろ}に
\ruby{煤}{すゝ}け、
\ruby{毒々}{どく|〴〵}しき
\ruby{黑}{くろ}き
\ruby{油{\換字{煙}}}{ゆ|えん}は
\ruby{今}{いま}やしたゝかに
\ruby{舞}{ま}ひ
\ruby{上}{あが}り
\ruby{居}{を}れり。

『オーヤ、
\ruby{洋燈}{らん|ぷ}が
\ruby{出{\換字{過}}}{で|す}ぎて
\ruby{居}{ゐ}る!。
\ruby{何}{なん}とマア
\ruby{危}{あぶな}い
\ruby{事}{こと}だつた!。
いくら
\ruby{病人}{びやう|にん}だつて、
\ruby{意氣地}{い|く|ぢ}が
\ruby{無}{な}いつて、ハア、
\ruby{此樣}{こ|ん}な
\ruby{事}{こと}つて
\ruby{有}{あ}る
\ruby{譯}{わけ}で
\ruby{無}{な}い。
』

と
\ruby{婆}{ばゞ}は
\ruby{獨語}{ひとり|ごと}して
\ruby{其}{そ}の
\ruby{心}{しん}を
\ruby{引{\換字{込}}}{ひつ|こ}ませぬ。

\ruby{臭氣}{にほ|い}の
\ruby{源}{もと}は
\ruby{仔細無}{し|さい|な}き
\ruby{事}{こと}なりけるが、
\ruby{惱}{なや}み
\ruby{疲}{つか}れし
\ruby{後}{のち}の
\ruby{睡}{ねむ}りたる
\ruby{間}{ま}に、
\ruby{洋燈}{らん|ぷ}はおのづと
\ruby{燃}{も}え
\ruby{高}{かう}じて、\換字{志}たゝかに
\ruby{憫然}{あは|れ}なる
\ruby{人}{ひと}に
\ruby{惡氣}{あく|き}をや
\ruby{吸}{す}はせけん。
\ruby{相良}{さが|ら}は
\ruby{眼}{ま}のあたりに
\ruby{見}{み}たる
\ruby{此}{こ}の
\ruby{一事}{ひと|こと}と、
\ruby{婆}{ばゞ}が
\ruby{今}{いま}
\ruby{洩}{も}らしたる
\ruby{其}{そ}の
\ruby{一語}{ひと|こと}とに、
\ruby{誰看護}{たれ|み|まも}るものも
\ruby{無}{な}き
\ruby{此}{こ}の
\ruby{病人}{びやう|にん}の、
\ruby{何病}{なに|びやう}に
\ruby{惱}{なや}めるかはいざ
\ruby{知}{し}らず、
\ruby{萬般}{よろ|づ}のあはれさ
\ruby{推}{お}し
\ruby{測}{はか}り
\ruby{知}{し}られて、
\ruby{他}{ひと}の
\ruby{憂}{うき}を
\ruby{見}{み}るには
\ruby{馴}{な}れたる
\ruby{身}{み}も、
\ruby{先}{ま}づ
\ruby{惻然}{そく|ぜん}として
\ruby{心}{こゝろ}を
\ruby{動}{うご}かしぬ。

\Entry{其十一}

\ruby{思}{おも}ふまゝに
\ruby{世}{よ}を
\ruby{振舞}{ふる|ま}ふは
\ruby{下人}{げ|にん}の
\ruby{常}{つね}なり。
\ruby{相良}{さが|ら}の
\ruby{車夫等}{しや|ふ|ら}は
\ruby{此}{こ}の
\ruby{狀態}{やう|す}に
\ruby{呆}{あき}れ
\ruby{果}{は}てゝ、せめては
\ruby{番茶}{ばん|ちや}なりと
\ruby{飮}{の}んで
\ruby{寢}{ね}ころんで
\ruby{寛}{くつろ}がんと、
\ruby{母家}{おも|や}をさして
\ruby{戾}{もど}りけるが、

『
\ruby{何}{なん}と
\ruby{飛}{と}んだところへ
\ruby{來}{き}たぢや
\ruby{無}{ね}えか。
とても
\ruby{眼}{め}も
\ruby{鼻}{はな}も
\ruby{明}{あ}きさうぢや
\ruby{無}{ね}えぜ。
』

『ハヽヽ、あの
\ruby{婆}{ばあ}さんは
\ruby{大方}{おほ|かた}、
\ruby{御醫者}{お|い|しや}さんの
\ruby{御抱}{お|かゝへ}は
\ruby{澤山}{たく|さん}
\ruby{給金}{きふ|きん}を
\ruby{取}{と}るだらう
\ruby{位}{ぐらゐ}に
\ruby{思}{おも}つて
\ruby{居}{ゐ}るだらうよ。
』

『ウツ、
\ruby{{\換字{違}}}{ちげ}へ
\ruby{無}{ね}え。
\ruby{一體}{いつ|たい}
\ruby{吾家}{う|ち}の
\ruby{先生}{せん|せい}は
\ruby{人}{ひと}が
\ruby{好{\換字{過}}}{よ|す}ぎるからナア。
\ruby[g]{此方等}{こちとら}あ
\ruby{何樣}{ど|う}したつて
\ruby{取}{と}るものあ
\ruby{取}{と}るが、
\ruby{先生}{せん|せい}が
\ruby{第一}{だい|いち}
\ruby{馬鹿}{ば|か}を
\ruby{見}{み}らあ。
』

と
\ruby{闇}{やみ}にはびこる
\ruby{胴間聲}{どう|ま|ごゑ}
\ruby{太}{ふと}く、
\ruby{{\換字{遠}}慮}{ゑん|りよ}も
\ruby{無}{な}く
\ruby{二人}{ふた|り}して
\ruby{喚}{わめ}き
\ruby{散}{ち}らしたり。

\ruby{婆}{ばゞ}は
\ruby{此等}{これ|ら}の
\ruby{聲}{こゑ}を
\ruby{聞}{き}かざりしと
\ruby{見}{み}ゆ。
いたはり
\ruby{氣}{げ}も
\ruby{無}{な}く
\ruby{病人}{びやう|にん}を
\ruby{搖}{ゆ}り
\ruby{起}{おこ}こして、

『お
\ruby{前}{めへ}さまが
\ruby{頼}{たの}みたいと
\ruby{云}{い}つた
\ruby{先生}{せん|せい}が
\ruby{御坐}{ご|ざ}らしつたよ。
』

と、
\ruby{同}{おな}じ
\ruby{言葉}{こと|ば}を
\ruby{冷}{ひや}やかに
\ruby{繰}{く}り
\ruby{{\換字{返}}}{かへ}しつ、
\ruby{重}{おも}き
\ruby{眶}{まぶた}を
\ruby{力無}{ちから|な}く
\ruby{擧}{あ}げて、

\ruby{微}{かすか}に
\ruby{點頭}{うな|づく}を
\ruby{見}{み}るより、

『さあ
\ruby{先生樣}{せん|せい|さん}、
\ruby{見}{み}てやつて
\ruby{下}{くだ}さい。
\ruby{濟}{す}んだらば
\ruby{別}{べつ}に
\ruby{水}{みづ}はあげ
\ruby{無}{な}いから、
\ruby{其處}{そ|こ}の
\ruby{椽先}{えん|さき}の
\ruby{手水鉢}{てう|づ|ばち}で、
\ruby{勝手}{かつ|て}に
\ruby{手}{て}を
\ruby{洗}{あら}ふが
\ruby{可}{い}いでがあす。
ナアニ
\ruby{一昨日}{を|とゝ|ひ}
\ruby{汲}{く}んだばかりで、
\ruby{誰}{たれ}も
\ruby{使}{つか}はないから
\ruby{奇麗}{き|れい}でがあすよ。
そして
\ruby{彼方}{あつ|ち}へ
\ruby{寄}{よ}つて
\ruby{溫茶}{ぬる|ちや}でも
\ruby{上}{あが}らつしやい。
どれ
\ruby{妾}{わたし}は
\ruby{先}{さき}へ
\ruby{行}{い}つて
\ruby{火}{ひ}でも
\ruby{燃}{た}きましやう。
』

と、
\ruby{他人同士}{た|にん|どう|し}とは
\ruby{本}{もと}より
\ruby{一目}{ひと|め}にも
\ruby{知}{し}れわたりたれど、さりとては
\ruby{乾}{かわ}き
\ruby{切}{き}つたる
\ruby{心}{こゝろ}の
\ruby{鬼々}{おに|〳〵}しくも
\ruby[g]{人{\換字{情}}}{なさけ}
\ruby{無}{な}き
\ruby{婆}{ばゞ}かな、と
\ruby{竊}{ひそか}に
\ruby{驚}{おどろ}ける
\ruby{相良}{さが|ら}を
\ruby{後}{あと}にして、
\ruby{恰}{あたか}も
\ruby{機關仕掛}{ぜん|まい|じ|かけ}の
\ruby{人形}{にん|ぎやう}かなんぞの
\ruby{動}{うご}くやうに、
\ruby[g]{四圍}{あたり}への
\ruby{斟酌}{しん|しやく}も
\ruby{氣{\換字{兼}}}{き|がね}も
\ruby{無}{な}く、
\ruby{我}{わ}が
\ruby{行}{ゆ}かんとする
\ruby{方}{かた}へ
\ruby[g]{早速}{さつさ}と
\ruby{行}{ゆ}きぬ。

『
\ruby{水野}{みづ|の}さんが
\ruby{居}{ゐ}ないで、ハア
\ruby{餘計}{よ|けい}な
\ruby{暇潰}{ひまつ|ゝぶ}しな。
アヽ
\ruby{江戸}{え|ど}の
\ruby{人}{ひと}と
\ruby{挨拶}{あい|さつ}するのは
\ruby{面倒}{めん|だう}な。
』

と、つぶやきながら
\ruby{婆}{ばゞ}は
\ruby{火}{ひ}を
\ruby{焚}{た}きはじめたり。

\ruby{急}{いそ}ぎに
\ruby{急}{いそ}ぎて
\ruby{今}{いま}
\ruby{歸}{かへ}り
\ruby{來}{きた}れる
\ruby{水野}{みづ|の}は、
\ruby{額}{ひたひ}に
\ruby{汗}{あせ}の
\ruby{玉}{たま}を
\ruby{散}{ち}らして、
\ruby{蒸}{む}されたるが
\ruby{如}{ごと}くになりたる
\ruby{面}{おもて}は、
\ruby{薄紅}{うす|くれなひ}に
\ruby{血}{ち}の
\ruby{色}{いろ}
\ruby{潮}{さ}したれば、
\ruby{引}{ひ}き
\ruby{立}{た}つて
\ruby{見}{み}ゆる
\ruby{眉目}{び|もく}のあたりに
\ruby{淸秀}{せい|しう}の
\ruby{氣}{き}
\ruby{滿}{み}ち
\ruby{溢}{あふ}れて、これこそ
\ruby{水野}{みづ|の}が
\ruby[g]{往時}{むかし}の
\ruby[g]{面貌}{おもわ}かと、
\ruby{天晴}{あつ|ぱ}れ
\ruby{美}{うつく}しく
\ruby{生々}{いき|〳〵}としたり。
\ruby{早}{はや}くも
\ruby{既}{すで}に
\ruby{相良}{さが|ら}の
\ruby{見}{み}えたるに
\ruby{欣}{よろこ}び
\ruby{悅}{よろこ}び、
\ruby{取}{と}り
\ruby{敢}{あ}へず
\ruby{先}{ま}ず
\ruby{車夫}{しや|ふ}を
\ruby{犒}{ねぎら}ひて
\ruby{手當}{て|あて}を
\ruby{與}{あた}へ、
\ruby{更}{さら}に
\ruby{病室}{びやう|しつ}には
\ruby{行}{ゆ}かんともせずして、こゝに
\ruby{數分間}{すう|ふん|かん}の
\ruby{後}{のち}
\ruby{我}{わ}が
\ruby{受}{う}くべき
\ruby{吉凶}{きつ|きよう}いづれかの
\ruby[g]{報告}{しらせ}の、
\ruby{醫}{い}によつて
\ruby{齎}{もた}らさるべきを
\ruby{恐}{おそ}る
\ruby{懼}{おそ}る
\ruby{待}{ま}ちたり。

\ruby{程經}{ほど|へ}て
\ruby{相良}{さが|ら}は
\ruby{歸}{かへ}り
\ruby{來}{きた}りぬ。
むさくろしき
\ruby{此}{こ}の
\ruby{婆}{ばゞ}が
\ruby{茶}{ちや}の
\ruby{間}{ま}の
\ruby{中}{うち}にて、
\ruby{水野}{みづ|の}と
\ruby{互}{たがひ}に
\ruby{挨拶}{あい|さつ}して、さて
\ruby{婆}{ばゞ}と
\ruby{水野}{みづ|の}とに
\ruby{向}{むか}つて
\ruby{徐}{おもむ}ろに、
\ruby{病人}{びやう|にん}の
\ruby{中々}{なか|〳〵}に
\ruby{重體}{ぢゆう|たい}なる
\ruby{事}{こと}、
\ruby{徴候}{ちよう|こう}の
\ruby{不完全}{ふ|くわん|ぜん}なるをもて
\ruby{今}{いま}までの
\ruby{醫}{い}は、
\ruby{何}{なん}と
\ruby{診斷}{しん|だん}したるか
\ruby{知}{し}らざれども、
\ruby{病氣}{やま|ひ}は
\ruby{全}{まつた}く
\ruby{腸窒扶斯}{ちやう|ち|ぷ|す}なる
\ruby{事}{こと}、
\ruby{傳染}{でん|せん}の
\ruby{{\換字{虞}}}{おそれ}ある
\ruby{病氣}{やま|ひ}なれば
\ruby{其}{そ}の
\ruby{心}{こゝろ}すべき
\ruby{事}{こと}、
\ruby{患者}{くわん|じや}のためには
\ruby{設備}{せつ|び}
\ruby{宜}{よろ}しき
\ruby{病院}{びやう|ゐん}に
\ruby{入}{い}らしむるを
\ruby{良}{よ}しとする
\ruby{事}{こと}、されども
\ruby{{\換字{遠}}路}{ゑん|ろ}を
\ruby{{\換字{伴}}}{ともな}ひ
\ruby{行}{ゆ}かんも
\ruby{難儀}{なん|ぎ}にして、
\ruby{聊}{いさゝ}か
\ruby{懸念}{け|ねん}も
\ruby{無}{な}きにあらねば、
\ruby{一軒建}{いつ|けん|だち}の
\ruby{離}{はな}れ
\ruby{家}{や}なるを
\ruby{幸}{さいは}ひ、
\ruby{彼處}{かし|こ}にて
\ruby{療養}{れう|やう}さするも
\ruby{惡}{あし}からぬ
\ruby{事}{こと}、たゞし
\ruby{此}{こ}の
\ruby{病}{やまひ}は
\ruby[g]{藥劑}{くすり}よりも
\ruby{寧}{むし}ろ
\ruby[g]{看護}{かんご}の
\ruby{良否}{よし|あし}によりて、
\ruby{{\換字{回}}復}{くわい|ふく}すると
\ruby{爲}{せ}ざるとも
\ruby{生}{しやう}ずるもの
\ruby{故}{ゆゑ}、
\ruby{今}{いま}の
\ruby{如}{ごと}き
\ruby{狀}{さま}にては
\ruby{宜}{よろ}しからぬ
\ruby{事}{こと}、
\ruby[g]{彼處}{かしこ}にて
\ruby{其儘療養}{その|まゝ|れう|やう}せんには
\ruby{是非}{ぜ|ひ}とも
\ruby{智識經驗}{ち|しき|けい|けん}の
\ruby{十分}{じう|ぶん}なる
\ruby{良看護婦}{りやう|かん|ご|ふ}を
\ruby{添}{そ}ふべき
\ruby{事}{こと}、くれ〴〵も
\ruby{患者}{くわん|じや}をして
\ruby{{\換字{強}}}{つよ}き
\ruby{身動}{み|うご}きなど
\ruby{爲}{せ}しめざるやう、
\ruby{取}{と}り
\ruby{扱}{あつか}ひも
\ruby{極}{きは}めて
\ruby{手柔}{て|やはら}かにすべき
\ruby{事}{こと}、
\ruby{看護}{かん|ご}の
\ruby{力}{ちから}
\ruby{足}{ た}らねば
\ruby{危}{あやふ}き
\ruby{事}{こと}、
\ruby{今}{いま}まで
\ruby{投劑}{とう|ざい}し
\ruby{居}{を}れる
\ruby{醫}{い}に
\ruby{此由}{この|よし}を
\ruby{語}{かた}りて、
\ruby{其}{そ}のつもりの
\ruby{處方}{しよ|はう}を
\ruby{乞}{こ}ひ、
\ruby{且}{か}つ
\ruby{種々}{いろ|〳〵}の
\ruby{注意}{ちゆう|い}を
\ruby{受}{う}くべき
\ruby{事}{こと}、
\ruby{其他}{その|た}さし
\ruby{當}{あた}つての
\ruby{樣々}{さま|〴〵}の
\ruby{處置}{しよ|ち}など、
\ruby{我}{わ}が
\ruby[g]{職分}{つとめ}の
\ruby{上}{うへ}より
\ruby{云}{い}ふべきほどの
\ruby{事}{こと}は、
\ruby{一々物柔}{いち|〳〵|もの|やは}らかに
\ruby{言}{い}ひ
\ruby{盡}{つく}して、
\ruby{御大切}{ご|たい|せつ}にと
\ruby{靜々}{しづ|〳〵}と
\ruby{歸}{かへ}りぬ。

\ruby{醫師}{い|し}が
\ruby{親切}{しん|せつ}の
\ruby{長々}{なが|〳〵}しき
\ruby{物語}{もの|がた}りの
\ruby{間}{あひだ}に、
\ruby{雲間}{くも|ま}の
\ruby{月}{つき}の
\ruby{如唯僅}{ごと|たゞ|わづか}の
\ruby{間}{あひだ}だけ
\ruby{美}{うつく}\換字{志}かりし
\ruby{水野}{みづ|の}は、
\ruby{其}{そ}の
\ruby[g]{往時}{むかし}の
\ruby{俤}{おもかげ}もいづくへやら、
\ruby{唇}{くちびる}は
\ruby{微}{かすか}かに
\ruby{顚}{ふる}へて
\ruby{自然}{ひと|りで}に
\ruby{戰}{おのゝ}き、
\ruby{眼}{め}は
\ruby{洞然}{どう|ぜん}として
\ruby[g]{何處}{いづく}を
\ruby{見}{み}るとも
\ruby{無}{な}く
\ruby{据}{すわ}りたるに、
\ruby{引}{ひ}きかへて
\ruby{冷酷}{れい|こく}なる
\ruby[g]{主人}{あるじ}の
\ruby{婆}{ばゞ}は、
\ruby{哂}{しや}れ
\ruby{古}{ふる}したる
\ruby{木彫}{き|ぼり}の
\ruby{假面}{め|ん}の、いづくにも
\ruby{潤}{うるほ}ひの
\ruby{無}{な}きが
\ruby{如}{ごと}き
\ruby{顏}{かほ}して、

『
\ruby{傳染病}{うつ|りや|まひ}ぢやあハア
\ruby{大變}{たい|へん}な
\ruby{事}{こと}だ。
\ruby{死}{し}なれでも
\ruby{仕}{し}たらまあ、オヽ
\ruby{厭}{いや}な
\ruby{事}{こと}だ。
\ruby{早{\換字{速}}}{さつ|そく}に
\ruby{{\換字{逐}}}{ぼ}ひ
\ruby{出}{だ}して
\ruby{仕舞}{し|ま}は
\ruby{無}{な}けりやあ。
』

と、
\ruby{慈悲}{じ|ひ}も
\ruby[g]{人{\換字{情}}}{なさけ}も
\ruby{無}{な}く
\ruby{云}{い}ひ
\ruby{出}{いで}しさまは、たゞ
\ruby{地獄物語}{ぢ|ごく|もの|がたり}の
\ruby{奪衣婆}{だつ|え|ば}を、
\ruby{今眼}{いま|め}の
\ruby{前}{まへ}に
\ruby{見}{み}るが
\ruby{如}{ごと}し。


\Entry{其十二}

\ruby{重}{おも}き
\ruby{風邪}{か|ぜ}なりと
\ruby{村}{むら}の
\ruby{醫}{い}の
\ruby{尾竹}{を|だけ}の
\ruby{云}{い}ひし
\ruby{時}{とき}だに、
\ruby{其}{そ}の
\ruby{容態}{よう|だい}の
\ruby{傍觀}{わき|め}にもたゞならぬに、
\ruby{淺}{あさ}からず
\ruby{心}{こゝろ}をも
\ruby{使}{つか}ひ
\ruby{氣}{き}を
\ruby{揉}{も}みしものを、
\ruby{淺草以北}{あさ|くさ|い|ほく}にては
\ruby{上無}{うへ|な}き
\ruby{人}{ひと}に
\ruby{頼}{たの}みおもへる
\ruby{相良}{さが|ら}に
\ruby{今}{いま}、
\ruby{病}{やまひ}はこれこれなり、
\ruby{看護}{かん|ご}
\ruby{行}{ゆ}き
\ruby{届}{とど}かずば
\ruby{危}{あやふ}からんと
\ruby{云}{い}はれては、
\ruby{愕然}{がく|ぜん}として
\ruby{打驚}{うち|おど}きつ、
\ruby{胸}{むね}のたゞ
\ruby{中}{なか}に
\ruby{鐵槌}{てつ|ゝゐ}の
\ruby{一撃}{いち|げき}を
\ruby{受}{う}けたるやうにおぼえて、
\ruby{我}{われ}
\ruby{先}{ま}づ
\ruby{死}{し}にもすべく
\ruby{惱}{なや}ましきに、
\ruby{垂死}{すゐ|し}の
\ruby{人}{ひと}を
\ruby{{\換字{逐}}}{お}ひ
\ruby{出}{いだ}さんといふ
\ruby{苛酷}{いら|ひど}き
\ruby{婆}{ばゞ}の
\ruby{言葉}{こと|ば}を
\ruby{聞}{き}きては、
\ruby{怒火心頭}{いか|り|しん|とう}に
\ruby{起}{おこ}つて
\ruby{堪}{た}ふるにも
\ruby{堪}{た}へられず、
\ruby{思}{おも}はず
\ruby{目}{め}に
\ruby{稜角立}{か|ど|た}てゝ
\ruby{峻}{けは}しく
\ruby{睨}{にら}みしが、ハツト
\ruby{心}{こゝろ}づきて
\ruby{自}{みづか}ら
\ruby{警}{いまし}め、
\ruby{燃}{も}え
\ruby{立}{た}つ
\ruby[g]{瞋恚}{いかり}を
\ruby{押{\換字{鎮}}}{おし|しづ}め
\ruby{押{\換字{鎮}}}{おし|しづ}めて、わざと
\ruby{何氣}{なに|げ}なく
\ruby{粧}{よそほ}ふ
\ruby{言葉}{こと|ば}つき
\ruby{{\換字{平}}{\換字{穏}}}{なだ|らか}に、

『そんな
\ruby{酷}{むご}らしいことを
\ruby{云}{い}つたつて
\ruby{仕樣}{し|やう}が
\ruby{無}{な}いぢや
\ruby{無}{な}いか、
\ruby{歩}{ある}けも
\ruby{仕無}{し|な}い
\ruby{病人}{びやう|にん}を
\ruby{{\換字{逐}}}{お}ひ
\ruby{出}{だ}すなんて。
』

と、
\ruby{打碎}{うち|くだ}けて
\ruby{云}{い}へど
\ruby{婆}{ばゞ}は
\ruby{應}{おう}ぜず、

『
\ruby{歩}{ある}けても
\ruby{歩}{ある}けないでも
\ruby{構}{かま}ひは
\ruby{有}{あ}りましねえ。
そんな
\ruby{病}{やまひ}で
\ruby{死}{し}なれた
\ruby{日}{ひ}には、
\ruby{彼}{あ}の
\ruby{家}{うち}へ
\ruby{入}{はい}る
\ruby{人}{ひと}は
\ruby{無}{な}くなつて、
\ruby{後}{あと}が
\ruby{廃物}{だ|め}になつて
\ruby{仕舞}{し|ま}ひます。
\ruby{早{\換字{速}}}{さつ|さ}と
\ruby{出}{で}て
\ruby{貰}{もら}つて
\ruby{掃除}{さう|ぢ}を
\ruby{仕}{し}て、
\ruby{行者}{ぎやう|じや}さんにでも
\ruby{淸}{きよ}めて
\ruby{貰}{もら}ひます。
\ruby{行者}{ぎやう|じや}さんを
\ruby{喚}{よ}ぶだけは
\ruby{痛}{いた}みになるが、それだけは
\ruby{時}{とき}の
\ruby{不祥}{ふ|しやう}と
\ruby{勘辯}{かん|べん}するでがあす。
』

と、
\ruby{{\換字{飽}}}{あ}くまで
\ruby{我欲}{が|よく}の
\ruby{云}{い}ひ
\ruby{草}{ぐさ}なり。

『だつて
\ruby{病人}{びやう|にん}が
\ruby{自{\換字{分}}}{じ|ぶん}で
\ruby{出}{で}て
\ruby{行}{ゆ}きやうは
\ruby{無}{な}し、
\ruby{{\換字{又}}}{また}
\ruby{五十子}{い|そ|こ}さんの
お
\ruby{母}{つか}さんは、
\ruby{汝}{おまへ}の
\ruby{知}{し}つて
\ruby{居}{ゐ}る
\ruby{通}{とほ}りの
\ruby{自{\換字{分}}{\換字{勝}}手}{じ|ぶん|かつ|て}ばかりの
\ruby{繼母}{まゝ|はゝ}さんで、
\ruby{{\換字{平}}常}{ふだ|ん}から
\ruby{五十子}{い|そ|こ}さんには
\ruby{無理}{む|り}を
\ruby{云}{い}ふけれど、
\ruby{五十子}{い|そ|こ}さんの
\ruby{世話}{せ|わ}は
\ruby{毫末}{ちつ|と}も
\ruby{仕無}{し|な}い、
\ruby{酷}{ひど}い〳〵
\ruby{人{\換字{情}}}{にん|じやう}の
\ruby{無}{な}い
\ruby{人}{ひと}ぢあ
\ruby{無}{な}いか。
\ruby{今度}{こん|ど}の
\ruby{病氣}{びやう|き}を
\ruby{知}{し}らせて
\ruby{{\換字{遣}}}{や}つても、
\ruby{顏}{かほ}も
\ruby{出}{だ}さ
\ruby{無}{な}けりあ、
\ruby{手紙一}{て|がみ|ひと}つ
\ruby{{\換字{遣}}}{よこ}さない
\ruby{位}{くらゐ}の
\ruby{人}{ひと}だもの、
\ruby{病人}{びやう|にん}を
\ruby{引取}{ひき|と}らうとは
\ruby{云}{い}ふまいぢあ
\ruby{無}{な}いか。
』

『けれども
\ruby{親}{おや}は
\ruby{親}{おや}でがあす。
\ruby{引}{ひ}き
\ruby{取}{と}らないとは
\ruby{云}{い}はせましねえ。
\ruby{親}{おや}が
\ruby{引取}{ひき|と}らないほどの
\ruby{厄介者}{やく|かい|もの}を、
\ruby{他人}{た|にん}の
\ruby{婆}{ばゞあ}がハア
\ruby{擔}{かつ}がう
\ruby{理由}{わ|け}は
\ruby{有}{あ}りましねえ。
たつて
\ruby{引取}{ひき|と}ら
\ruby{無}{な}けりやあ、ナアニ
\ruby{譯}{わけ}は
\ruby{無}{な}い、
\ruby{{\換字{巡}}査}{じゆん|さ}さん
\ruby{頼}{たの}んで
\ruby{引取}{ひき|と}らせるだ。
ハア、
\ruby{{\換字{道}}理}{す|ぢ}の
\ruby{{\換字{違}}}{ちが}つた
\ruby{事}{こと}
\ruby{云}{ い}はない
\ruby{婆}{ばゞ}だよ。
\ruby{婆}{ばゞ}は
\ruby{他人}{た|にん}だよ、
\ruby{身寄}{み|より}で
\ruby{無}{な}いだよ、
\ruby{錢金}{ぜに|かね}づくで
\ruby{彼}{あ}の
\ruby{家}{うち}に
\ruby{置}{お}いたばかりだよ。
\ruby{貯金}{たく|はへ}も
\ruby{有}{あ}るか
\ruby{無}{な}いか
\ruby{知}{し}れない
\ruby{病人}{びやう|にん}を
\ruby{預}{あづ}かる、---\ \換字{志}かも
\ruby{傳染病}{うつ|りや|まひ}の
\ruby{大病人}{たい|びやう|にん}を
\ruby{預}{あづ}かる、
\ruby{其樣}{そ|ん}な
\ruby{鈍}{どん}くさい
\ruby{事}{こと}
\ruby{出來}{で|き}ないだよ。
お
\ruby{前樣}{めへ|さま}も
\ruby{病人}{びやう|にん}には
\ruby{他人}{た|にん}で
\ruby{無}{な}いか、
\ruby{恨}{うら}みつぽい
\ruby{其樣}{そ|ん}な
\ruby{眼}{め}つきをし
\ruby{何}{なに}も
\ruby{此}{この }
\ruby{婆}{ばゞあ}を
\ruby{視}{み}さつしやることは
\ruby{無}{な}い。
』

『なるほど
\ruby{其}{それ}は
\ruby{左樣}{さ|う}でもあらうが、いくら
\ruby{他人}{た|にん}でも
\ruby{病人}{びやう|にん}を
\ruby{突出}{つき|だ}さうといふのは、それは
\ruby{餘}{あま}り
\ruby{酷}{ひど}いぢやあ
\ruby{無}{な}いか。
』

『
\ruby{病人}{びやう|にん}だから
\ruby{{\換字{逐}}}{お}ひ
\ruby{出}{だ}さうといふので、
\ruby{酷}{ひど}かあ
\ruby{酷}{ひど}いに
\ruby{仕}{し}て
\ruby{置}{お}かつしやい。
』

『お
\ruby{婆}{ばあ}さん、
お
\ruby{前}{まへ}、そんな
\ruby{事}{こと}を
\ruby{云}{い}つたつて、
\ruby{人間}{ひ|と}には
\ruby{人{\換字{道}}}{み|ち}といふものがある。
\ruby{動}{うご}かしてさへ
\ruby{惡}{わる}いと
\ruby{醫者}{い|しや}の
\ruby{云}{い}つた
\ruby{病人}{びやう|にん}を
\ruby{{\換字{逐}}}{お}ひ
\ruby{出}{だ}さうとは
\ruby{非{\換字{道}}}{ひ|だう}では
\ruby{無}{な}いか。
』

『
\ruby{非{\換字{道}}}{ひ|だう}なら
\ruby{非{\換字{道}}}{ひ|だう}に
\ruby{仕}{し}て
\ruby{置}{お}かつしやい。
\ruby{金}{かね}の
\ruby{出處}{で|どこ}の
\ruby{覺束無}{おぼ|つか|な}い
\ruby{其樣}{そ|ん}な
\ruby{大病人}{たい|びやう|にん}を、
\ruby{世話}{せ|わ}をして
\ruby{損}{そん}をするのは
\ruby{婆}{ばゞあ}は
\ruby{{\換字{嫌}}}{きら}ひだ。
』

『でも
\ruby{有}{あ}らうが
\ruby{汝}{おまへ}が
\ruby{今}{いま}
\ruby{{\換字{逐}}}{お}ひ
\ruby{出}{だ}して
\ruby{仕舞}{し|ま}へば、よしんば
\ruby{繼母}{おつ|か}さんが
\ruby{引}{ひ}き
\ruby{取}{と}るにしても、あちこち
\ruby{持}{も}ち
\ruby{{\換字{廻}}}{まは}られるのは
\ruby{病人}{びやう|にん}の
\ruby{不利{\換字{益}}}{ふ|た|め}、\換字{志}かも
\ruby{何樣}{ど|う}して
\ruby{彼}{あ}の
\ruby{繼母}{おつ|か}さんが、
\ruby{碌}{ろく}な
\ruby{世話}{せ|わ}をする
\ruby{事}{こと}では
\ruby{無}{な}い。
\ruby{仕}{し}て
\ruby{見}{み}れば
\ruby{看護}{かん|ご}が
\ruby{惡}{わる}けりやあ
\ruby{危}{あぶな}いといふ
\ruby{病氣}{びや|うき}だもの、
\ruby{十}{とほ}に
\ruby{一}{ひと}つも
\ruby{助}{たすか}る
\ruby{瀬}{せ}は
\ruby{無}{な}い、
\ruby{見}{み}す〳〵
\ruby{病人}{びやう|にん}は
\ruby{殺}{ころ}されるやうなもの!。
\ruby{譯}{わけ}の
\ruby{{\換字{分}}}{わか}らない
\ruby{汝}{おまへ}でも
\ruby{無}{な}し、こゝのところを
\ruby{考}{かんが}へて、
\ruby{私}{わたし}が
\ruby{此}{こ}の
\ruby{{\換字{通}}}{とほ}り
\ruby{手}{て}をついて
\ruby{頼}{たの}むから、どうか
\ruby{左樣}{さ|う}あこぎな
\ruby{事}{こと}を
\ruby{云}{い}はないで、
\ruby{當{\換字{分}}}{たう|ぶん}\ ---\ 。
』

『イヽエ、あこぎな
\ruby{事}{こと}を
\ruby{云}{い}ふでがあすよ。
\ruby{手}{て}をついて
\ruby{頼}{たの}んだつて、
\ruby{芋塊}{い|も}が
\ruby{一}{ひと}つ
\ruby{自然}{ひと|りで}に
\ruby{出來}{で|き}て
\ruby{來}{く}るものぢやあござら
\ruby{無}{な}い。
\ruby{頼}{たの}むなら
\ruby{頼}{たの}むやうにして
\ruby{頼}{たの}まつしやい。
』

『
\ruby{頼}{たの}むやうに
\ruby{仕}{し}ろつて、
\ruby{何樣}{ど|う}すれば
\ruby{好}{い}いと
\ruby{云}{い}ふのかえ。
』

『
\ruby{婆}{ばゞあ}は
\ruby{年}{とし}をとつて
\ruby{氣}{き}が
\ruby{短}{みじか}い、
\ruby{打撒}{ぶち|ま}けて
\ruby{汝樣}{おまへ|さま}に
\ruby{云}{い}つて
\ruby{上}{あ}げやう。
\ruby{病人}{びやう|にん}の
\ruby{月々}{つき|〴〵}のものは
\ruby{今}{いま}まで
\ruby{{\換字{通}}}{どほ}りに
\ruby{屹}{きつ}と
お
\ruby{前樣}{めへ|さま}が
\ruby{受合}{うけ|あ}つて、それから
\ruby{病人}{びやう|にん}がいけなかつたら、
\ruby{後}{あと}の
\ruby{始末}{し|まつ}は
\ruby{皆此}{みな|この }
\ruby{婆}{ばゞあ}に
\ruby{{\換字{迷}}惑}{めい|わく}を
\ruby{掛}{か}けないで、そして
\ruby{座敷}{ざ|しき}に
\ruby{死穢}{け|がれ}を
\ruby{付}{つ}けた
\ruby{謝罪}{あや|まり}に
\ruby{二十兩}{に|じう|りやう}、
\ruby{癒}{なほ}つたら
\ruby{祝}{いはい}に
\ruby{十兩}{じう|りやう}
\ruby{{\換字{遣}}}{よこ}すと、
\ruby{確乎}{しつ|かり}
\ruby{御前樣}{お|めへ|さま}が
\ruby{呑込}{のみ|こ}んで、
\ruby{先}{ま}づ
\ruby{十兩}{じう|りやう}だけ
\ruby{渡}{わた}して
\ruby{置}{お}かつしやい。
その
\ruby{代}{かは}り
\ruby{病人}{びやう|にん}には
\ruby{構}{かま}はないから、どうなりと
\ruby{{\換字{勝}}手}{かつ|て}に
\ruby{介抱}{かい|はう}さつしやい。
さ、お
\ruby{前樣}{めへ|さま}もあかの
\ruby{他人}{た|にん}、これだけ
\ruby{踏込}{ふみ|こ}んで
\ruby{世話}{せ|わ}もなるまい。
それとも
\ruby{病人}{びやう|にん}が
\ruby{愍然}{かは|いさう}で、
\ruby{金}{かね}を
\ruby{出}{だ}してもと
\ruby{云}{い}はつしやるか、どつちでも
お
\ruby{前樣}{めへ|さま}の
\ruby{好}{すき}にさつしやい。
』

『ムヽ』

\ruby{手}{て}に
\ruby{在}{あ}らば
\ruby{千金萬金}{せん|きん|ばん|きん}も
\ruby{何悋}{なに|をし}かるべきを、
\ruby{及}{およ}ぶことの
\ruby{及}{およ}ばぬに
\ruby{口惜}{くち|をし}きは
\ruby{金沙汰}{かね|さ|た}なり。
\ruby{水野}{みづ|の}は
\ruby{生}{うま}れてはじめて
\ruby{日頃}{ひ|ごろ}
\ruby{此}{この}
\ruby{阿堵物}{も||の}を% 「阿堵物(あとぶつ)」お金のこと
\ruby{卑}{いやし}みしを
\ruby{悔}{く}いぬ。


\Entry{其十三}

% メモ 校正終了 2024-04-04
\原本頁{78-10}%
おのづから
\ruby{横}{よこ}さまに
\ruby{降}{ふ}る
\ruby{雨}{あめ}はあらじ、
%
\ruby{風}{かぜ}の
\ruby{添}{そ}はるにこそ、
%
\ruby{音}{おと}
あらけなく
\ruby{夜}{よる}の
\ruby{窓}{まど}をも
\ruby{打}{う}つなれ、
%
と
\ruby{胸}{むね}
ゆたかなる
\ruby{{\換字{古}}}{むかし}の
\ruby{人}{ひと}の
\ruby{云}{い}ひける。
%
かゝる
\ruby{鬼}{おに}くさき
\ruby{婆}{ばゞ}も、
%
\ruby{齡}{とし}の
\ruby{十七八}{じふ|なな|はち}には、
%
\ruby{女}{をんな}の
\ruby[||j>]{本}{うまれ}
\ruby[||j>]{性}{ つき}とて、
% \ruby{本性}{うまれ|つき}とて、
%
\ruby{臙脂}{べ|に}
\ruby{白{\換字{粉}}}{おし|ろい}に
\ruby{色}{いろ}つくりて、
%
\ruby{人}{ひと}に
\ruby{悅}{よろこ}ばれんと
\ruby{願}{ねが}ひたる
\ruby{日}{ひ}もあるべきに、
%
\ruby{其}{そ}の
\ruby{後}{のち}
\ruby{如何}{い|か}なる
\ruby{世}{よ}の
\ruby{風}{かぜ}に
\ruby{吹}{ふ}き
\ruby{曲}{ゆが}められてか、
%
\ruby{今}{いま}は
\ruby{如是}{か|く}
\ruby{直}{すぐ}ならず
\ruby{人}{ひと}には
\ruby{當}{あた}るならん。
%
\ruby{水野}{みづ|の}は
\ruby{一度}{ひと|たび}は
\ruby{此}{こ}の
\ruby{婆}{ばゞ}を
\ruby{憎}{にく}しと
\ruby{見}{み}しかど、
%
\ruby{憎}{にく}む
\ruby{心}{こゝろ}は
\ruby{忽}{たちま}ちに
\ruby{失}{う}せて、
%
\ruby{且}{かつ}は
\ruby{其}{そ}の
\ruby{欲深}{よく|ふか}きに
\ruby{呆}{あき}れ、
%
\ruby{且}{かつ}は
\ruby{其}{そ}の
\ruby[||j>]{意}{こゝろ}
\ruby{剛}{ たけ}きを% 全角空白は「意(こゝろ)」の突出対策
\ruby{怪}{あやし}み、
%
\ruby{且}{かつ}は
\ruby{其}{そ}の
\ruby{人}{ひと}
らしからぬまでに
\ruby{{\換字{尊}}}{たつと}き
\ruby{愛{\換字{情}}}{こゝ|ろ}の
\ruby{既}{すで}に
\ruby{壞}{やぶ}れ
\ruby{盡}{つく}して、
%
\ruby{卑}{いや}しき
\ruby{我}{が}のみの
\ruby{殘}{のこ}りて
\ruby{高}{たか}ぶれるを、
%
\ruby{哀}{かなし}み
\ruby{愍}{あはれ}みて
\ruby{打見}{うち|み}やりたり。

\原本頁{79-10}%
されど
\ruby{今}{いま}は
\ruby{他人}{ひ|と}を
\ruby{愍}{あはれ}みて
あるべき
\ruby{時}{とき}ならねば、
%
\ruby{水野}{みづ|の}は
\ruby{直}{たゞち}に
\ruby{差}{さ}し
\ruby{當}{あた}つての
\ruby{我}{わ}が
\ruby{上}{うへ}に
\ruby{掛}{かゝ}れる
\ruby{事}{こと}に
\ruby{心}{こゝろ}を
\ruby{惱}{なや}ましめぬ。

\原本頁{80-1}%
\ruby{五十子}{い|そ|こ}を
\ruby{如是}{か|く}
\ruby{忌}{いま}はしく
\ruby{親切}{しん|せつ}
\ruby{無}{な}き
\ruby{婆}{ばゞ}の
\ruby{家}{いへ}に
\ruby{在}{あ}らせんよりは、
%
\ruby{良}{よ}き
\ruby[||j>]{病}{びやう}
\ruby[||j>]{院}{ ゐん}に
% \ruby{病院}{びやう|ゐん}に
\ruby{移}{うつ}さんかた
\ruby{萬般}{すべ|て}に
\ruby{就}{つ}けて
\ruby{心地}{こゝ|ち}よし
とは
\ruby{思}{おも}ひながらも、
%
\ruby{今{\換字{宵}}}{こ|よひ}の
\ruby{如}{ごと}く
\ruby{穩}{おだ}やかに
\ruby{晴}{は}れてのみ
あるべくは
あらぬ
\ruby{秋}{あき}の
\ruby{天候}{そ|ら}の
\ruby{{\換字{習}}}{ならひ}なれば、
%
\ruby{時}{とき}に
\ruby{臨}{のぞ}みて
\ruby{如何}{い|か}なる
\ruby{雨}{あめ}
\ruby{風}{かぜ}の
\ruby{妨{\換字{害}}}{さま|たげ}に
\ruby{{\換字{遇}}}{あ}はんも
\ruby{知}{し}るべからず、
%
\ruby{{\換字{又}}}{また}
\ruby{然無}{さ|な}きだに
\ruby{{\換字{遠}}路}{ゑん|ろ}を
\ruby{{\換字{伴}}}{ともな}ひ
\ruby{行}{ゆ}く
\ruby{{\換字{途}}上}{と|じやう}は
\ruby[||j>]{病}{びやう}
\ruby[||j>]{人}{ にん}も
% \ruby{病人}{びやう|にん}も
\ruby{特}{こと}に
\ruby[||j>]{心}{こゝろ}
\ruby[||j>]{惱}{ なや}ましかるべく、
% \ruby{心惱}{こゝろ|なや}ましかるべく、
%
それがために
\ruby{萬一}{まん|いち}
\ruby{惡}{あし}き
\ruby{事}{こと}もやとの
\ruby{懸念}{け|ねん}も
\ruby{少}{すくな}からぬに、
%
\ruby{由無}{よし|な}き
\ruby{金錢}{きん|せん}を
\ruby{婆}{ばゞ}に
\ruby{貪}{むさぼ}らるゝは
\ruby{愚}{おろか}なるに
\ruby{似}{に}たれど、
%
これも
\ruby{病}{や}める
\ruby{人}{ひと}のためと
\ruby{{\換字{忍}}}{しの}ばんには
\ruby{露}{つゆ}
\ruby{厭}{いと}はしからずと、
%
\ruby{水野}{みづ|の}は
\ruby{{\換字{終}}}{つひ}に
\ruby{意}{こゝろ}を
\ruby{決}{けつ}して、
%
\ruby{彼}{か}の
\ruby{離}{はな}れ
\ruby{室}{や}に
\ruby{置}{お}きたるまゝ
\ruby{介抱}{かい|はう}する
\ruby{事}{こと}と
\ruby{定}{さだ}めたり。
%
もとより
\ruby{一}{ひと}つには
\ruby{其}{そ}の
\ruby{奧深}{おく|ふか}き
\ruby{底}{そこ}の
\ruby{底}{そこ}の
\ruby{心}{こゝろ}に、
%
\ruby{五十子}{い|そ|こ}と
\ruby{我}{われ}との
\ruby{相距}{あひ|さ}らざらんを
\ruby{望}{のぞ}む
\ruby{思}{おもひ}の
\ruby{潜}{ひそ}めばなるべし。% 【潛 u6f5b 「先先」】【潜 u6f5c 「夫夫」】併用されている
%
たとひ
\ruby{自己}{お|の}が
\ruby{身}{み}は
\ruby{如何}{い|か}なる
\ruby{故}{ゆゑ}にか
\原本頁{81-1}%
\ruby{五十子}{い|そ|こ}に
\ruby{{\換字{嫌}}}{きら}はれて、
%
\ruby{特}{こと}に
\ruby{病}{やまひ}のため
\ruby{癇}{かん}の
\ruby{高}{たか}ぶりて
\ruby{我}{が}の
\ruby{{\換字{強}}}{つよ}くなれる
\ruby{此}{こ}の
\ruby{頃}{ごろ}の
\ruby{彼女}{か|れ}には、
%
\ruby{面}{おもて}を
\ruby{會}{あ}はすを
さへ
\ruby{厭}{いと}はるゝより、
%
\ruby{自}{みづか}ら
\ruby{病床}{びやう|しやう}に
\ruby{{\換字{近}}}{ちか}づきて
\ruby{問}{と}ひ
\ruby{慰}{なぐさ}めも
\ruby{仕度}{し|た}く、
%
\ruby{看護}{せ|わ}も
\ruby{仕}{し}て
\ruby{{\換字{遣}}}{や}りたき
\ruby{心}{こゝろ}の、
%
\ruby{{\換字{遣}}}{や}る
\ruby{方}{かた}も
\ruby{無}{な}く
\ruby{逸}{はや}るを
\ruby{抑}{おさ}へに
\ruby{抑}{おさ}へて、
%
\ruby{裏面}{う|ら}にて
こそ
\ruby{力}{ちから}の
\ruby{及}{およ}ぶ
\ruby{限}{かぎ}りを
\ruby{盡}{つく}して
\ruby{駈}{か}けも
\ruby{走}{はし}りもすれ、
%
\ruby[||j>]{病}{びやう}
\ruby[||j>]{人}{ にん}の
% \ruby{病人}{びやう|にん}の
\ruby{氣}{き}に
\ruby{{\換字{逆}}}{さから}はじと
\ruby{其}{そ}の
\ruby{{\換字{前}}}{まへ}には
\ruby{身影}{か|げ}をさへ
\ruby{見}{み}することも
\ruby{無}{な}くて、
%
たゞ
\ruby{竊}{ひそか}に
\ruby{外}{そと}に
\ruby{立}{た}つて、
%
\ruby{細}{ほそ}りたる
\ruby{聲}{こゑ}の
\ruby{孱{\換字{弱}}}{か|よわ}きを
\ruby{聞}{き}き、
%
\ruby{或}{あるひ}は
\ruby{物}{もの}の
\ruby{罅隙}{す|き}より
\ruby{窶}{やつ}れたる
\ruby{其}{そ}の
\ruby{面貌}{おも|かげ}の
\ruby{悲}{かな}しきを
\ruby{見}{み}ては、
%
\ruby{男兒}{をと|こ}たる
\ruby{身}{み}の
\ruby{人目}{ひと|め}はづかしくも、
%
にじみ
\ruby{來}{く}る
\ruby{涙}{なみだ}を
\ruby{止}{とゞ}め
かねて、
%
\ruby{神}{かみ}も
\ruby{我}{わ}が
\ruby{誠心}{まこ|と}を
\ruby{憐}{あは}れませ
たまひて、
%
\ruby{此}{こ}の
\ruby{人}{ひと}の
\ruby{病苦}{びやう|く}を
\ruby{救}{すく}はせ
たまへ、
%
と
\ruby{何}{なん}の
\ruby{神}{かみ}に
\ruby{祈}{いの}るとも
\ruby{無}{な}く、
%
\ruby{何時}{い|つ}か
\ruby{我}{われ}
\ruby{知}{し}らず
\ruby{祈}{いの}り
\ruby{居}{ゐ}る、
%
\ruby{思}{おも}へば
\ruby{愚}{おろか}しき
\ruby{{\換字{朝}}夕}{あさ|ゆふ}に
\ruby{甘}{あま}んじて、
%
\ruby{{\換字{猶}}}{なほ}これより
\ruby{幾日}{いく|か}と
\ruby{定}{さだ}まらぬ
\原本頁{82-1}%
\ruby{其}{そ}の
\ruby{間}{あひだ}を、
%
せめてもの
\ruby{果敢}{は|か}なき
\ruby{心{\換字{遣}}}{こゝろ|や}りに、
%
\ruby{{\換字{猶}}}{なほ}
\ruby{其}{そ}の
おろかしき
\ruby{振舞}{ふる|まひ}を
\ruby{續}{つゞ}けんとは
するなり。

\原本頁{82-3}%
『では
\ruby{汝}{おまへ}の
\ruby{云}{い}ふ
\ruby{{\換字{通}}}{とほ}りに
\ruby{仕}{し}やう。
%
\ruby{一切}{いつ|さい}
\ruby{私}{わたし}が
\ruby{受合}{うけ|あ}つて
\ruby{置}{お}く。
』

\原本頁{82-4}%
と、
%
\ruby{決然}{けつ|ぜん}として
\ruby{水野}{みづ|の}は
\ruby{云}{い}へど、

\原本頁{82-5}%
『たゞ
\ruby{受合}{うけ|あ}つても
いけましねえ、
%
\ruby{何時}{い|つ}
その
\ruby[||j>]{十}{じふ}
\ruby[||j>]{兩}{りやう}は
% \ruby{十兩}{じふ|りやう}は
\ruby{渡}{わた}して
\ruby{吳}{く}れ
さつしやる。
』

\原本頁{82-7}%
と、
%
\ruby{婆}{ばゞ}は
\ruby{手}{て}に
\ruby{握}{にぎ}らぬことには
\ruby{人}{ひと}を
\ruby{信}{しん}ぜず。

\原本頁{82-8}%
『
\ruby{明日}{あ|す}の
\ruby{{\換字{朝}}}{あさ}
\ruby{渡}{わた}す。
』

\原本頁{82-9}%
『
\ruby{大{\換字{丈}}夫}{だい|ぢやう|ぶ}かね。
』

\原本頁{82-10}%
『
\ruby{大{\換字{丈}}夫}{だい|ぢやう|ぶ}だ。
』

\原本頁{82-11}%
『
\ruby{看病人}{かん|びやう|にん}
はエ。
』

\原本頁{83-1}%
『
\ruby{矢張}{やつ|ぱり}
\ruby{私}{わたし}が
\ruby{雇}{やと}つて
\ruby{付}{つ}ける。
%
\ruby{相良}{さが|ら}さんに
\ruby{良}{い}いのを
\ruby{世話}{せ|わ}をして
\ruby{貰}{もら}ふ。
%
\換字{志}かし
\ruby{一切}{いつ|さい}
かういふ
\ruby{事}{こと}を、
%
\ruby{私}{わたし}が
\ruby{爲}{し}たのだと
\ruby[||j>]{病}{びやう}
\ruby[||j>]{人}{ にん}に
% \ruby{病人}{びやう|にん}に
\ruby{云}{い}つては
ならぬ。
%
\ruby[||j>]{病}{びやう}
\ruby[||j>]{人}{ にん}が
% \ruby{病人}{びやう|にん}が
\ruby{私}{わたし}の
\ruby{世話}{せ|わ}になるのを
\ruby{厭}{いや}がつて
\ruby{居}{ゐ}るから、
%
たゞ
\ruby{學校}{がく|かう}の
\ruby{人{\換字{達}}}{ひと|たち}が
\ruby{爲}{す}るのだと
\ruby{云}{い}つて
\ruby{置}{おい}てくれ。
』

\原本頁{83-5}%
『はア、ようがす、
%
それは
\ruby{無益}{む|だ}な
\ruby{口}{くち}きく
\ruby{婆}{ばゞあ}でない
でがあす。
%
\換字{志}かし
\ruby{甚}{えら}い
\ruby{金}{かね}が
かゝりませうに、
%
\ruby{親切}{しん|せつ}な
\ruby{事}{こと}だネ。
』

\原本頁{83-7}%
と、
%
\ruby{冷}{ひや}やかに
\ruby{笑}{わら}ふ
\ruby{口}{くち}の
\ruby{左右}{さ|いう}に、
%
\ruby{深}{ふか}き
\ruby{皺}{しわ}
あらはれて
\ruby[||j>]{物}{もの}
\ruby[||j>]{凄}{すさま}じく、
% \ruby{物凄}{もの|すさま}じく、
%
さも〳〵
\ruby{水野}{みづ|の}が
\ruby{爲}{な}す
\ruby{一切}{いつ|さい}の
\ruby{事}{こと}の、
%
やがては
\ruby{{\換字{朝}}}{あした}の
\ruby{霜}{しも}の
\ruby{柱}{はしら}を
\ruby{{\換字{彩}}色}{いろ|ど}り
\ruby{夕}{ゆふべ}の
\ruby{露}{つゆ}の
\ruby{珠}{たま}を
\ruby{綴}{つゞ}らんとする
\ruby{痴}{おろか}なる
\ruby{企畫}{くは|だて}の
\ruby{如}{ごと}く
\ruby{甲{\換字{斐}}}{か|ひ}
\ruby{無}{な}く
\ruby{{\換字{終}}}{おは}らんを
\ruby{見徹}{み|ぬ}きて
\ruby{知}{し}りたりと
\ruby{云}{い}はぬばかりの
\ruby{面色}{かほ|つき}したり。

\原本頁{83-11}%
『
\ruby{快}{よ}くなるまで
みんな
\ruby{御{\換字{前}}樣}{お|まへ|さま}が
\ruby{一人}{ひと|り}で
\ruby{爲}{さ}つしやるかネ。
』

\原本頁{84-1}%
『ム。
』

\原本頁{84-2}%
『
\ruby{百兩位}{ひやく|りやう|ぐらゐ}では
\ruby{{\換字{追}}付}{おつ|つ}きましねえかも
\ruby{知}{し}れましねえヨ。
』

\原本頁{84-3}%
『ム。
』

\原本頁{84-4}%
『ホー
\ruby{御{\換字{前}}樣}{お|まへ|さま}は
\ruby{學校}{がく|かう}の
\ruby{敎員}{けう|ゐん}でもつて、
%
\ruby{其樣}{そん|な}に
\ruby{御金}{お|かね}
\ruby{有}{も}つてるだかネ。
』

\原本頁{84-6}%
\ruby{水野}{みづ|の}は
\ruby{苦}{にが}りきつて
\ruby{答}{こたへ}をもせず、

\原本頁{84-7}%
『
\ruby{何}{なん}でも
\ruby{可}{よ}い、
%
\ruby{其樣}{そ|ん}なことを
\ruby{云}{い}つて
\ruby{居}{ゐ}る
\ruby{暇}{ひま}は
\ruby{無}{な}い。
%
わたしは
これから
\ruby{尾竹}{を|だけ}の
ところへ
\ruby{行}{ゆ}く。
』

\原本頁{84-9}%
\ruby{突}{つ}と
\ruby{立上}{たち|あが}つたる
\ruby{水野}{みづ|の}は
\ruby{此處}{こ|ゝ}を
\ruby{出}{い}でゝ、
%
\ruby{村}{むら}の
\ruby{醫}{い}を
\ruby{問}{と}ひて
\ruby{相良}{さが|ら}の
\ruby{言}{ことば}を
\ruby{傳}{つた}へ、
%
\ruby{手}{て}ぬかり
\ruby{無}{な}きやう
\ruby{十{\換字{分}}}{じふ|ぶん}に
\ruby{其}{そ}の
\ruby{職{\換字{分}}}{つと|め}を
\ruby{盡}{つく}さんことを
\ruby{乞}{こ}ひ
\ruby{求}{もと}め、
%
これより
\ruby{直}{すぐ}にも
\ruby{見舞}{み|ま}はん
といふ
\ruby{親切}{しん|せつ}
\ruby{籠}{こも}れる
\ruby{答}{こたへ}を
\ruby{聞}{き}きて、
%
\原本頁{85-1}%
はじめて
\ruby{我}{わ}が
\ruby{宿}{やど}とせる
\ruby{山路}{やま|ぢ}が
\ruby{方}{かた}に
\ruby{歸}{かへ}りぬ。

\原本頁{85-2}%
\ruby{物}{もの}の
\ruby{味}{あぢ}さへ
\ruby{知}{し}るや
\ruby{知}{し}らずや、
%
\ruby{湯漬}{ゆ|づ}け
\ruby{飯}{めし}
\ruby{忙}{せは}しく
\ruby{夜食}{や|しよく}を
\ruby{濟}{す}ませて、
%
\ruby{長}{なが}き
\ruby{夜}{よ}も
\ruby{既}{はや}
\ruby{{\換字{更}}}{ふ}けて
\ruby{何時}{なん|じ}かを
\ruby{打}{う}つ
\ruby{時計}{と|けい}の
\ruby{音}{おと}の
\ruby{折}{をり}から
\ruby{聞}{きこ}ゆるを
\ruby{數}{かぞ}へも
\ruby{敢}{あ}へず、
%
\ruby{急}{いそ}ぎ
\ruby{周章}{あ|は}てゝ
\ruby{{\換字{又}}}{また}
\ruby{{\換字{戸}}外}{そ|と}へ
\ruby{出}{い}でんと
すれば、

\原本頁{85-5}%
『
\ruby{水野}{みづ|の}さん、
%
\ruby{何處}{ど|こ}へ
\ruby{今}{いま}から
\ruby{御出}{お|いで}に
なります?。
』

\原本頁{85-6}%
と、
%
\ruby{低}{ひく}く
\ruby{沈}{しづ}める
\ruby{聲音}{こわ|ね}の
\ruby{呼}{よ}び
\ruby{止}{と}めたり。

\Entry{其十四}

\ruby{云}{い}はゞ
\ruby{我}{わ}が
\ruby{假}{かり}の
\ruby{宿{\換字{所}}}{や|ど}の
\ruby{主人}{ある|じ}なりと
\ruby{云}{い}ふまでなれど、
\ruby{東京}{とう|けい}あたりに
\ruby{黒塗}{くろ|ぬり}の
\ruby{小札}{こ|ふだ}
\ruby{懸}{か}けならべたる
\ruby{商賣}{しやう|ばい}づくの
\ruby{下宿屋}{げ|しゆく|や}といふにはあらで、
\ruby{我}{わ}が
\ruby{校長}{かう|ちやう}の
\ruby{高田}{たか|だ}と
\ruby{懇意}{こん|い}なる
\ruby{間柄}{あひだ|がら}なるより、
\ruby{其}{そ}の
\ruby{云}{い}ひ
\ruby{入}{いれ}によりて、
\ruby{唯}{たゞ}
\ruby{我一人}{われ|ひ|とり}を
\ruby{賓客}{きや|く}
\ruby{同樣}{どう|やう}に、
\ruby{萬般}{よろ|づ}
\ruby{親切}{しん|せつ}に
\ruby{世話}{せ|わ}し
\ruby{吳}{く}るゝ
\ruby{此家}{こ|ゝ}の
\ruby{老夫}{おや|ぢ}の
\ruby[g]{吉右衛門}{きちゑもん}に
\ruby{呼}{よ}び
\ruby{{\換字{留}}}{と}められては、
\ruby{心}{こゝろ}の
\ruby{急}{せ}いたる
\ruby{折}{をり}からとて、あらずもがなには
\ruby{思}{おも}ひながら、
\ruby{後}{あと}
\ruby{振}{ふ}り
\ruby{{\換字{返}}}{かへ}りて
\ruby{立停}{たち|とゞ}まり、

『ア、
\ruby{一寸}{ちよ|いと}
\ruby{濱町}{はま|ちやう}まで
\ruby{行}{い}つて
\ruby{來}{き}ます。
\ruby{何程}{いく|ら}
\ruby{急}{いそ}いでも
\ruby{遲}{おそ}くはならうが、
\ruby{歸}{かへ}ることは
\ruby{屹度}{きつ|と}
\ruby{歸}{かへ}ります。
\ruby{濟}{す}まんけれど
\ruby{敲}{たゝ}きますから、
\ruby{關}{かま}はず
\ruby{{\換字{戸}}締}{し|ま}りを
\ruby{仕}{し}て
\ruby{仕舞}{し|ま}つて
\ruby{寢}{やす}んで
\ruby{下}{くだ}さい。
』

と
\ruby{云}{い}ひつゝ
\ruby{燈火}{あか|り}さす
\ruby{茶}{ちや}の
\ruby{室}{ま}を
\ruby{窺}{うかゞ}へば、
\ruby{讀}{よ}みさしたる
\ruby{新聞}{しん|ぶん}を
\ruby{傍}{かたへ}に
\ruby{置}{お}きて、
\ruby{兀}{は}げたる
\ruby{頭}{かしら}の
\ruby{澤々}{つや|〳〵}と
\ruby{光}{ひか}れる
\ruby[g]{吉右衛門}{きちゑもん}は、
\ruby{眞鍮{\換字{縁}}}{しん|ちゆう|ぶち}の
\ruby{鏡玉}{た|ま}
\ruby{圓}{まろ}き
\ruby{昔風眼鏡}{むか|し|め|がね}を
\ruby{掛}{か}けたる、
\ruby{淸}{きよ}らなる
\ruby{赤}{あか}ら
\ruby{顏}{がほ}を
\ruby{此方}{こな|た}に
\ruby{向}{む}けたる
\ruby{其}{そ}の
\ruby{右}{みぎ}の
\ruby{方}{かた}には、
\ruby{孫娘}{まご|むすめ}の
\ruby{一昨年}{を|とゝ|し}
\ruby{小學}{せう|がく}を
\ruby{卒}{を}へたるばかりなるが、
\ruby{何}{なに}を
\ruby{讀}{よ}めるならんか
\ruby{燈火}{とも|しび}の
\ruby{下}{した}に
\ruby{身}{み}を
\ruby{低}{ひく}く
\ruby{俯}{ふ}して、
\ruby{疊}{たゝみ}に
\ruby{置}{お}ける
\ruby{書}{しよ}に
\ruby{餘念無}{よ|ねん|な}く
\ruby{讀}{よ}み
\ruby{入}{い}つたる、
\ruby{其}{そ}の
\ruby{黑}{くろ}き
\ruby{頭髪}{かし|ら}に
\ruby{何}{なに}やら
\ruby{紅}{あか}き
\ruby{巾}{きれ}
\ruby{美}{うつく}しく、
\ruby{一幅}{いつ|ぷく}の
\ruby{{\換字{平}}和}{へい|わ}の
\ruby{夜}{よる}の
\ruby{圖}{づ}は
\ruby{眼}{め}の
\ruby{{\換字{前}}}{まへ}に
\ruby{現}{あら}はれて、
\ruby{身}{み}の
\ruby{疲}{つか}れ
\ruby{心}{こゝろ}の
\ruby{勞}{つか}れを
\ruby{休}{やす}むる
\ruby{間}{ま}も
\ruby{無}{な}き
\ruby{水野}{みづ|の}をして、
\ruby{人}{ひと}は
\ruby{斯}{か}く
\ruby{無邪氣}{む|じや|き}に
\ruby{世}{よ}を
\ruby{{\換字{送}}}{おく}るもあるをと、そゞろに
\ruby{其}{そ}の
\ruby{無事}{ぶ|じ}の
\ruby{淸福}{せい|ふく}の
\ruby{價値}{あた|ひ}
\ruby{貴}{たつと}きを
\ruby{思}{おも}はしめぬ。

『ハア、
\ruby{左樣}{そ|う}でございますか、
\ruby{宜}{よろ}しうございますとも。
\換字{志}かし
\ruby{大變}{たい|へん}せか〳〵していらつしやいますが、
\ruby{氣}{き}を
\ruby{御付}{お|つ}けなさいまし、
\ruby{爭}{あらそ}ひなんぞ
\ruby{爲}{な}すつてはいけませんぜ。
\ruby{{\換字{平}}井}{ひら|ゐ}の
お
\ruby{澤}{さは}
\ruby{婆}{ばゞあ}のところへ
\ruby{御出}{お|いで}なすつたと
\ruby{聞}{き}きましたが、あの
\ruby{婆}{ばゞあ}と
\ruby{物言}{もの|いひ}なんぞ
\ruby{爲}{な}さりあ
\ruby{仕}{し}ますまいネ、
\ruby{彼奴}{あい|つ}はどうせ
\ruby{人}{ひと}ぢやあ
\ruby{無}{な}いのですから。
それは
\ruby{左樣}{そ|う}と
\ruby{岩崎}{いは|ざき}さんは
\ruby{何樣}{ど|う}でございます?。
』

『
\ruby{岩崎}{いは|ざき}はどうもいよ〳〵
\ruby{惡}{わる}い。
ナーニお
\ruby{澤}{さは}
\ruby{婆}{ばあ}さんには
\ruby{此方}{こつ|ち}で
\ruby{負}{ま}けて
\ruby{居}{ゐ}るから
\ruby{論}{ろん}は
\ruby{無}{な}いよ。
\ruby{爭}{あらそ}ひなんぞ
\ruby{仕}{し}て
\ruby{來}{き}たのでは
\ruby{無}{な}い。
たゞ
\ruby{早}{はや}く
\ruby{濱町}{はま|ちやう}へ
\ruby{行}{ゆ}かうと
\ruby{思}{おも}つて
\ruby{急}{いそ}いで
\ruby{居}{ゐ}るので。
』

『
\ruby{濱町}{はま|ちやう}は
\ruby{島木}{しま|き}さんのところへで
\ruby{御座}{ご|ざ}いますか。
』

『アヽ
\ruby{左樣}{そ|う}、
\ruby{島木}{しま|き}のところへだ。
』

『それぢやあ
\ruby{路}{みち}は
\ruby{{\換字{遠}}}{とほ}いし、
\ruby{御會話}{お|はな|し}は
\ruby{長}{なが}くなりませうし、
\ruby{御歸}{お|かへ}りは
\ruby{大變}{たい|へん}
\ruby{遲}{おそ}くなりましやうが、なんなら
\ruby{明日}{あ|す}になすつては
\ruby{何樣}{ど|う}でございます?。
』

『
\ruby{明日}{あ|す}と
\ruby{云}{い}つて
\ruby{居}{ゐ}るわけには
\ruby{行}{い}かないのだから。
』

\ruby{此時}{この|とき}
\ruby{娘}{むすめ}は
\ruby{書}{しよ}を
\ruby{棄}{す}てゝ、
\ruby{急}{きふ}に
\ruby{頭}{かうべ}を
\ruby{擡}{もた}げたるが、さつと
\ruby{燈火}{あか|り}を
\ruby{{\換字{浴}}}{あ}びたる
\ruby{面}{おもて}の、
\ruby{色}{いろ}は
\ruby{初花}{はつ|はな}の
\ruby{日}{ひ}に
\ruby{匂}{にほ}ふかと
\ruby{麗}{うる}はしく、
\ruby{細}{ほそ}けれど
\ruby{鮮}{あざ}やかなる
\ruby{眉}{まゆ}、
\ruby{小}{ちひさ}けれどもはつきりと
\ruby{仕}{し}たる
\ruby{眼}{め}つき、まだ
\ruby{罪}{つみ}も
\ruby{無}{な}く
\ruby{慾}{よく}も
\ruby{無}{な}く、たゞ
\ruby{生々}{いき|〳〵}と
\ruby{愛度}{あ|ど}なく
\ruby{美}{うつく}しきが、
\ruby{突}{つ}と
\ruby{立上}{たち|あが}りて
\ruby{走}{はし}り
\ruby{出}{い}で、

『なぜ
\ruby{其樣}{そん|な}に
\ruby{他{\換字{所}}}{よ|そ}へばかし
\ruby{入}{い}らつしやるの!。
\ruby{{\換字{戸}}外}{そ|と}はもう
\ruby{眞闇}{まつ|くら}で、いけませんわ。
\ruby[<h||]{妾}{わたし}
\ruby{御願}{お|ねが}ひだから
\ruby{御止}{お|よ}しなさいよ。
』

と、
\ruby{甘}{あま}へたる
\ruby{調子}{てう|し}に
\ruby{云}{い}ひ〳〵
\ruby{水野}{みづ|の}を
\ruby{扯}{ひ}きて、はや
\ruby{女}{をんな}づくるべき
\ruby{齡}{とし}なれど
\ruby{{\換字{猶}}}{なほ}
\ruby{兒童}{こ|ども}くさく、
\ruby{{\換字{遠}}慮}{ゑん|りよ}も
\ruby{無}{な}く
\ruby{此方}{こな|た}へ
\ruby{扯}{ひ}き
\ruby{入}{い}れんとすれば、
\ruby{水野}{みづ|の}はおのづと
\ruby{催}{もよほ}さるゝ
\ruby{笑}{わら}ひの
\ruby{顏}{かほ}を
\ruby{顰}{しか}めながら、そつと
\ruby{其手}{その|て}をはづして、

『マアお
\ruby{濱}{はま}ちやん、
\ruby{堪忍}{か|に}して
お
\ruby{吳}{く}れ、どうしても
\ruby{行}{い}つて
\ruby{來}{こ}なくてはならない
\ruby{事}{こと}だから。
』

と、
\ruby{周章}{あ|は}てゝ
\ruby{土間}{ど|ま}へ
\ruby{下}{お}りて
\ruby{出}{い}でかゝるに、
\ruby{媚}{なまめ}ける
\ruby{笑}{わら}ひを
\ruby{帶}{お}びたる
\ruby{聲美}{こゑ|うつく}しく
\ruby{我}{わ}が
\ruby{背後}{うし|ろ}に
\ruby{當}{あた}つて、

『あら、いやな
\ruby{人}{ひと}!、きつと
\ruby{{\換字{又}}}{また}
\ruby{五十子}{い|そ|こ}さんの
\ruby{事}{こと}で
\ruby{心配}{しん|ぱい}して
\ruby{居}{ゐ}るのよ!。
』

と、
\ruby{{\換字{婦}}人}{をん|な}は
\ruby{口頭}{くち|さき}より
\ruby{先}{ま}づませて、
\ruby{戀}{こひ}
\ruby{知}{し}り
\ruby{顏}{がほ}に
\ruby{獨語}{ひとり|ご}つが
\ruby{聞}{きこ}えぬ。

\ruby{心}{こゝろ}もこゝにあらず
\ruby{思}{おもひ}の
\ruby{忙}{せは}しければ、
\ruby{{\換字{平}}生}{ひご|ろ}はいと
\ruby{可愛}{か|はゆ}しと
\ruby{思}{おも}へる
\ruby{濱子}{はま|こ}が
\ruby{言葉}{こと|ば}をも、
\ruby{我}{わ}が
\ruby{胸}{むね}の
\ruby{中}{うち}に
\ruby{{\換字{留}}}{とゞ}むる
\ruby{暇無}{いとま|な}くて、
\ruby{急}{きふ}に
\ruby{村徑}{むら|みち}の
\ruby{闇}{やみ}を
\ruby{衝}{つ}いて
\ruby{歩}{ある}き
\ruby{出}{いだ}せば、
\ruby{門}{かど}を
\ruby{出}{い}づるや
\ruby{否}{いな}や
\ruby{足元}{あし|もと}
\ruby{{\換字{近}}}{ちか}き
\ruby{{\換字{蓮}}田}{はす|だ}の
\ruby{中}{うち}より、
\ruby{人}{ひと}に
\ruby{驚}{おどろ}ける
\ruby{五位鷺}{ご|ゐ|さぎ}の
\ruby{其聲淋}{その|こゑ|さび}しく
\ruby{人}{ひと}を
\ruby{驚}{おどろ}かして、ぎやあと
\ruby{鳴}{な}きつつ
\ruby{立}{た}つて
\ruby{去}{さ}りたり。

\Entry{其十五}

% メモ 校正終了 2024-04-06
\原本頁{90-10}%
\ruby{勞}{らう}を
\ruby{厭}{いと}ひてにはあらず、
%
\ruby{時}{とき}を
\ruby{惜}{をし}みて、
%
\ruby{勸}{すゝ}むる
\ruby{人力車}{く|る|ま}の
ありしまま、
%
\原本頁{91-1}%
さるところより
\ruby[g]{其車}{それ}には
\ruby{乘}{の}りしが、
%
やうやく
\ruby{濱町}{はま|ちやう}に
\ruby{着}{つ}きし
\ruby{時}{とき}には、
%
\ruby{流石}{さす|が}に
\ruby{人}{ひと}の
\ruby{家}{いへ}を
\ruby{音}{おと}づれんは
\ruby{後目痛}{うしろ|め|た}きほど
\ruby{{\換字{更}}}{ふ}けに
\ruby{{\換字{更}}}{ふ}けたり。
%
\ruby{日頃}{ひ|ごろ}
\ruby{心{\換字{遣}}}{こゝろ|づか}ひの
\ruby{鹵莾}{おろ|か}ならぬ
\ruby[g]{水野}{みづの}は、
%
\ruby{{\換字{鎖}}}{とざ}し
\ruby{固}{かた}めたる
\ruby{{\換字{戸}}}{と}を
\ruby{思}{おも}ひ
\ruby{{\換字{遣}}}{や}りなく
\ruby{打敲}{うち|たゝ}きて、
%
\ruby{{\換字{近}}隣}{あた|り}の
\ruby{寢耳}{ね|みゝ}をまで
\ruby{驚}{おどろ}かさんことを
\ruby{憚}{はゞか}り、% 「憚 は(ゞ)か」
%
\ruby{聊}{いさゝ}か
\ruby{自}{みづか}ら
\ruby{躊躇}{ため|ら}ひしが、
%
\ruby{愚}{おろか}なり、
%
\ruby{臆}{おく}して
\ruby{已}{や}むべきには
あらぬものをと、
%
\ruby{手}{て}を
\ruby{擧}{あ}げて
ほと〳〵と
\ruby{星}{ほし}の
\ruby{下}{した}に
\ruby{敲}{たゝ}きぬ。

\原本頁{91-7}%
\ruby{心}{こゝろ}の
\ruby{優}{やさ}しさに
おのづから
\ruby{手}{て}も
\ruby{柔軟}{やはら|か}に
\ruby{當}{あた}りて、
%
\ruby{其}{そ}の
\ruby{音}{おと}は
\ruby{左}{さ}まで
\ruby{{\換字{強}}}{つよ}からざりしが、
%
\ruby{幸}{さいはひ}にして
\ruby{未}{ま}だ
\ruby{睡}{ねむ}らざりし
\ruby{女}{をんな}の
ありけむ、
%
ハイと
\ruby{明}{あき}らかに
\ruby{答}{こた}ふる
\ruby{聲}{こゑ}して、

\原本頁{91-10}%
『
\ruby{誰樣}{どな|た}?。
%
\ruby[g]{伊東}{いとう}さん?。
』

\原本頁{91-11}%
と、
%
\ruby{云}{い}ひながら
\ruby{開}{あ}けに
かゝりたり。

\原本頁{92-1}%
\ruby[g]{伊東}{いとう}とは
\ruby[g]{島木}{しまき}を
\ruby{外}{ほか}にして
\ruby{唯}{たゞ}
\ruby{一人}{ひと|り}の
\ruby{此}{こ}の
\ruby{家}{や}の
\ruby[g]{止宿者}{きやく}にて、
%
\ruby{無類}{む|るゐ}の
\ruby{極樂蜻蛉}{ごく|らく|とん|ぼ}なるよしを
\ruby[g]{島木}{しまき}より
\ruby{聞}{き}きしが、
%
さては
\ruby{今{\換字{宵}}}{こ|よひ}は
その
\ruby{男}{をとこ}の、
%
\ruby{何處}{いづ|く}の
\ruby{花}{はな}の
\ruby{陰}{かげ}にか
\ruby{憩}{いこ}ひて、
%
\ruby{{\換字{更}}}{ふ}けて
\ruby{{\換字{猶}}}{なほ}
\ruby{今}{いま}に
\ruby{歸}{かへ}り
\ruby{來}{きた}らざるを、
%
\ruby{婢}{をんな}の
\ruby{待}{ま}ち
\ruby{居}{ゐ}たりしならんと
\ruby{早}{はや}くも
\ruby{猜}{すゐ}しぬ。

\原本頁{92-5}%
『イヽエ、
%
\ruby[g]{島木}{しまき}さんを
\ruby{急用}{きふ|よう}で
\ruby{{\換字{尋}}}{たづ}ねて
\ruby{來}{き}ました。
%
わたしは
\ruby[g]{水野}{みづの}と
いふものです。
』

\原本頁{92-7}%
と、
%
\ruby{云}{い}ふ
\ruby{間}{ま}に
\ruby{雨{\換字{戸}}}{あま|ど}は
\ruby{一枚}{いち|まい}
\ruby{繰}{く}り
\ruby{明}{あ}けられて、
%
\ruby{細帶姿}{ほそ|おび|すがた}の
\換字{志}どけ
\ruby{無}{な}く
\ruby{背後}{うし|ろ}の
\ruby{上}{あが}り
\ruby{端}{はな}に
\ruby{置}{お}きたる
\ruby{小}{こ}
\ruby[g]{洋燈}{らんぷ}の
\ruby{光}{ひかり}の
\ruby{中}{うち}に
\ruby{現}{あらは}れたるは、
%
\ruby{丸顏}{まる|がほ}の
\ruby{色白}{いろ|じろ}の
\ruby{氣}{き}さくものゝ、
%
\ruby{名}{な}は
\ruby{忘}{わす}れたれど
\ruby[g]{見記臆}{みおぼえ}ある% 原本通り「おぼえ」
\ruby{女}{をんな}なり。

\原本頁{92-10}%
『オヤ、
%
\ruby[g]{水野}{みづの}さんでしたか、
%
\ruby{存}{ぞん}じてましたよ、
%
たしか
\ruby{彼}{あ}の
\ruby{菖蒲}{しやう|ぶ}のある
\ruby[g]{四ッ木}{よ ぎ}とかの。% TODO 四ツ木
%
\ruby{能}{よ}くおぼえて
\ruby{居}{ゐ}たでしやう。
%
\ruby{褒}{ほ}めて
\ruby{頂戴}{ちやう|だい}な、
%
\原本頁{93-1}
ホヽヽ、
%
まあ
\ruby{御入}{お|はい}んなさい。
%
\ruby{大層}{たい|そう}
\ruby{遲}{おそ}く
\ruby{入}{い}らした
\ruby{事}{こと}ネ。
%
エエ、
%
\ruby{居}{ゐ}らつしやいますとも
\ruby[g]{島木}{しまき}さんは。
%
ハア、
%
イエ
\ruby{未}{ま}だ
\ruby{御睡}{お|よ}り
\ruby{就}{つ}きや
なさりますまい、
%
\ruby{今}{いま}しがた
\ruby{他{\換字{所}}}{よ|そ}から
\ruby{御歸}{お|かへ}りに
なつたばかりなんですから。
』

\原本頁{93-4}%
と、
%
\ruby{一人}{ひと|り}で
\ruby[g]{饒舌}{しやべ}りながら
\ruby{後}{あと}を
\ruby{{\換字{鎖}}}{し}めて、
%
やがて、

\原本頁{93-5}%
『ホヽ、
%
\ruby{此樣}{こ|ん}な
\ruby{姿}{なり}を
\ruby{仕}{し}て
\ruby{居}{ゐ}て、
%
\ruby{御免}{ご|めん}なさいましよ。
』

\原本頁{93-6}%
と、
%
\ruby{云}{い}ひ〳〵
\ruby{先}{さき}に
\ruby{立}{た}つて
\ruby{二階}{に|かい}へ
\ruby{導}{みち}びき、

\原本頁{93-7}%
『
\ruby[g]{島木}{しまき}さん、
%
さあ
\ruby{御起}{お|お}きなさいまし。
%
\ruby{貴下}{あな|た}の
\ruby{好}{すき}な
\ruby[g]{水野}{みづの}さんが
\ruby[g]{御來臨}{おいで}なすつてよ。
%
\ruby{明日}{あし|た}は
\ruby{驕}{おご}つて
\ruby{下}{くだ}さるでしようネ。
』

\原本頁{93-10}%
と、
%
\ruby{其室}{その|へや}に
\ruby{入}{い}つて
\ruby{{\換字{遠}}慮無}{ゑん|りよ|な}く
\ruby[g]{洋燈}{らんぷ}の
\ruby{火}{ひ}を
\ruby{明}{あか}るくしたり。

\原本頁{93-11}%
『
\ruby{何}{なん}だ
\ruby{驕}{おご}つて
\ruby{下}{くだ}さるで\換字{志}やうも
\ruby{無}{な}いもんだ、
%
\ruby{自{\換字{分}}}{じ|ぶん}が
\ruby{岡惚}{をか|ぼ}れて
\ruby{居}{ゐ}やがるんだ
\ruby{癖}{くせ}に。
』

\原本頁{94-2}%
と、
%
\ruby{輕}{かる}く
\ruby{罵}{のゝし}りながら
\ruby[g]{島木}{しまき}は
\ruby{起}{お}き
\ruby{出}{い}でしが、
%
\ruby{既}{はや}
\ruby[g]{水野}{みづの}の
\ruby{{\換字{近}}々}{ちか|〴〵}と
\ruby{入}{い}り
\ruby{來}{きた}り
\ruby{居}{を}りて、
%
\ruby{今}{いま}の
\ruby{戲言}{たは|むれ}を
\ruby{聞}{き}きしや
\ruby{苦虫}{にが|むし}を% 原本通り「虫」
\ruby{噛}{か}みたる
\ruby{如}{ごと}き
\ruby{顏色}{かほ|つき}なせるを
\ruby{見}{み}て、

\原本頁{94-5}%
『ヤ、
%
\ruby{失敬}{しつ|けい}
\g詰めruby{々々}{〳〵}。
%
\ruby{戲言}{じやう|だん}だよ。
%
\ruby{大層}{たい|そう}
\ruby{遲}{おそ}く
\ruby{來}{き}たぢや
\ruby{無}{な}いか。
%
さあ
まあ
\ruby{此上}{こ|れ}に
\ruby{坐}{すわ}つて
\ruby{吳}{く}れたまへ。
』

\原本頁{94-7}%
と、
%
\ruby{慌}{あわ}てゝ
\ruby{敷物}{しき|もの}を
\ruby{出}{いだ}し、
%
\ruby{自己}{おの|れ}は
\ruby{手早}{て|ばや}く
\ruby{衣}{い}を
\ruby{改}{あらた}めたり。

\原本頁{94-8}%
『オイ
お
\ruby{作}{さく}さん、
%
\ruby{此處}{こ|ゝ}は
\ruby{乃公}{お|れ}が
\ruby{片}{かた}づけて
\ruby{仕舞}{し|ま}ふがネ、
%
もう
\ruby{火}{ひ}は
\ruby[<j|]{皆}{みんな}
\ruby{{\換字{消}}}{き}えて
\ruby{仕舞}{し|ま}つたかエ、
%
せめて
\ruby{御茶}{お|ちや}だけ
\ruby{欲}{ほし}いのだが。
』

\原本頁{94-10}%
『ハア、
%
もう
\ruby{樓下}{し|た}にも
ありませんが
\ruby{打火}{お|こ}して
あげましやう。
%
ナアニ
\ruby{別段}{べつ|だん}
\ruby{譯}{わけ}は
ありませんから。
』

\原本頁{95-1}%
\ruby{此家}{こ|ゝ}は
\ruby{家作}{や|づく}りも
\ruby{什器}{だう|ぐ}も
\ruby[g]{淸潔}{きれい}に、
%
\ruby{四十五六}{し|じう|ご|ろく}の
\ruby{女}{をんな}
\ruby[g]{主人}{あるじ}と、
%
\ruby{此女}{こ|れ}と、
%
\ruby{下働}{した|ばたら}きの
\ruby{婢}{をんな}と
\ruby{三人}{さん|にん}して、
%
\ruby{客}{きやく}は
たゞ
\ruby{二人}{ふた|り}の
\ruby[g]{島木}{しまき}
\ruby[g]{伊東}{いとう}を
かしづく
\ruby{下宿屋}{げ|しゆく|や}
めかさぬ
\ruby{品}{ひん}の
\ruby{良}{よ}き
\ruby{家}{いへ}なれど、
%
\ruby{{\換字{又}}}{また}
\ruby{折々}{をり|〳〵}は
\ruby{骨牌}{は|な}に
\ruby{貸}{か}す
\ruby{窩}{あな}ともなり
\ruby{{\換字{兼}}}{か}ねぬほど、
%
\ruby{一切}{すべ|て}を
\ruby[g]{金錢}{かね}の
\ruby{光}{ひかり}に
\ruby{美}{うつく}しく
\ruby{仕}{し}こなして
\ruby{見}{す}するところとは
\ruby{知}{し}りながら、
%
\ruby{深夜}{しん|や}に
\ruby{人}{ひと}を
\ruby{煩}{わづら}はすことの
\ruby{氣}{き}の
\ruby{毒}{どく}さに
\ruby{耐}{た}へかねて、

\原本頁{95-7}%
『マア
いゝさ
\ruby[g]{島木}{しまき}
\ruby{君}{くん}、
%
\ruby{茶}{ちや}なぞは
\ruby{要}{い}らんよ、
%
お
\ruby{作}{さく}さんは
もう
\ruby{寢}{やす}んで
\ruby{吳}{く}れたまへ。
』

\原本頁{95-9}%
と、
%
\ruby[g]{水野}{みづの}は
\ruby{言葉}{こと|ば}を
\ruby{挿}{さしはさ}まざるを
\ruby{得}{え}ざりき。

\原本頁{95-10}%
\ruby[g]{島木}{しまき}は
\ruby{物}{もの}に
\ruby{滞}{とゞこほ}らずして、
%
\ruby{心}{こゝろ}の
\ruby{動}{うご}きの
\ruby{早}{はや}き
\ruby{男}{をとこ}なれば、

\原本頁{95-11}%
『ン、
%
それも
\ruby{左樣}{さ|う}だ。
%
ぢやあ
お
\ruby{作}{さく}さん
\ruby{茶}{ちや}は
いゝからね、
%
そら
\ruby{彼}{あ}の
\ruby{葡萄酒}{ぶ|だう|しゆ}と
\ruby[g]{乾燥牛肉}{ドライドビーフ}とを
\ruby{持}{も}つて
\ruby{來}{き}て
お
\ruby{吳}{く}れ。
』

\原本頁{96-2}%
と
\ruby{云}{い}へば、

\原本頁{96-3}%
『ハア、
%
\ruby{其}{そ}の
\ruby{方}{はう}が% 原本では「方」のルビが欠けているが他と合わせて「はう」
\ruby{却}{かへ}つて
\ruby{宜}{よろ}しう
\ruby{御座}{ご|ざ}んしやう。
』

\原本頁{96-4}%
と、
%
\ruby{婢}{をんな}は
\ruby{下}{した}に
\ruby{降}{お}り
\ruby{行}{ゆ}きしが、
%
\ruby{忽地}{たちま|ち}にして
\ruby{一}{ひと}つの
\ruby{廣}{ひろ}き
\ruby{{\換字{盆}}}{ぼん}に、
%
\ruby{燈}{ひ}を
\ruby{受}{う}けて
\ruby{美}{うつく}しき
ポカラの
\ruby[g]{玻璃盞}{コツプ}
\ruby{二}{ふた}つ、
%
\ruby{薄手}{うす|で}の
\ruby{白皿}{しろ|ざら}
\ruby{二}{ふた}つ、
%
ニツケルの
\ruby{栓拔器}{せん|ぬ|き}、
%
まだ
\ruby{開}{あ}けぬ
\ruby{薄}{うす}き
\ruby{罐詰}{くわん|づめ}、
%
\ruby{利休箸}{り|きう|ばし}を
\ruby{載}{の}せて、
%
\ruby{片手}{かた|て}に
\ruby{葡萄酒}{ぶ|だう|しゆ}の
\ruby{罎}{びん}を
\ruby{提}{ひつさ}げて
\ruby{來}{きた}りぬ。

\原本頁{96-8}%
『よし〳〵。
%
もうこれで
\ruby{好}{い}いから
\ruby{樓下}{し|た}へ
\ruby{行}{い}つて
\ruby{御就眠}{お|や|す}み。
%
\ruby{御客樣}{お|きやく|さま}が
\ruby{氣}{き}の
\ruby{{\換字{通}}}{とほ}つた
\ruby{方}{かた}だから
\ruby{御{\換字{酌}}}{お|しやく}には
\ruby{及}{およ}ばない。
%
\ruby{{\換字{勝}}手}{かつ|て}に
\ruby{御免}{ご|めん}を
\ruby{蒙}{かうむ}るさ。
』

\原本頁{96-11}%
『それぢやあ、
%
\ruby{御二人}{お|ふ|たり}で
\ruby{水入}{みづ|い}らずに
\ruby{御話}{お|はな}しなさいまし、
%
まあ
\ruby{御睦}{お|むつ}まじいこと、
%
\ruby{些}{ちと}
\ruby{妬}{や}けますネ。
%
ホヽヽヽ。
%
ですけれど
\ruby[g]{島木}{しまき}さん
\ruby{御用}{ご|よう}が
ありましたなら
\ruby{構}{かま}はないで
\ruby{呼}{よ}んで
\ruby{下}{くだ}さいましよ。
』

\原本頁{97-3}%
\ruby{婢}{をんな}は
\ruby{樓下}{し|た}に
\ruby{去}{さ}つて
\ruby{行}{ゆ}きたり。
%
\ruby{手早}{て|ばや}く
\ruby{片}{かた}づけられたる
\ruby{座敷}{ざ|しき}の
\ruby{好}{よ}き
\ruby{程}{ほど}に
\ruby{坐}{すわ}りて、
%
\ruby[g]{島木}{しまき}は
\ruby{葡萄酒}{ぶ|だう|しゆ}の
\ruby[g]{栓}{くち}を
\ruby{拔}{ぬ}きながら
\ruby[g]{水野}{みづの}の
\ruby{面}{おもて}を
\ruby{見}{み}て、

\原本頁{97-5}%
『
\ruby{君}{きみ}、
%
\ruby{大層}{たい|そう}
\ruby{顏色}{かほ|いろ}が
\ruby{惡}{わる}いぢや
\ruby{無}{な}いか。
%
\ruby{何樣}{ど|う}か
\ruby{仕}{し}はせんか、
%
\ruby{氣}{き}になるネ、
%
さあ、
%
まあ、
%
\ruby{飮}{や}つて
\ruby{吳}{く}れたまへナ。
』

\原本頁{97-7}%
と、
%
\ruby{詞}{ことば}の
\ruby{調子}{てう|し}こそ
\ruby{{\換字{猶}}}{なほ}
\ruby{冴}{さ}えたれ、
%
\ruby{顏}{かほ}には
\ruby{憂愁}{うれ|ひ}の
\ruby{曇}{くも}りを
\ruby{上}{のぼ}せて、
%
\ruby{友}{とも}を
\ruby{思}{おも}ふ
\ruby{{\換字{情}}}{こゝろ}の
\ruby{溫}{あたゝ}かくも
\ruby{溫}{あたゝ}かく、
%
\ruby{{\換字{強}}}{し}ひて
\ruby[g]{玻璃盞}{コツプ}を
\ruby{執}{と}らせて
\ruby{注}{つ}ぎて
\ruby{{\換字{遣}}}{や}りたる
\ruby{酒}{さけ}は
いつはり
\ruby{無}{な}き
\ruby{血}{ち}の
\ruby{色}{いろ}を
なしたり。

\Entry{其十六}

いつもながらの
\ruby{島木}{しま|き}が
\ruby{親切}{しん|せつ}の、
\ruby{今{\換字{宵}}}{こ|よひ}は
\ruby{別}{わ}けて
\ruby{身}{み}に
\ruby{染}{し}む
\ruby{心地}{こゝ|ち}して、
\ruby{今}{いま}までには
\ruby{經驗}{おぼ|え}
\ruby{無}{な}き
\ruby{事}{こと}なるが、おのずと
\ruby{脆}{もろ}くも
\ruby{涙}{なみだ}の
\ruby{湧}{わ}き
\ruby{上}{あが}るを、
\ruby{水野}{みづ|の}は
\ruby{怪}{あやし}まれやせんと
\ruby{竊}{そつ}と
\ruby{拭}{ぬぐ}ひて、わざと
\ruby{眼}{め}の
\ruby{行}{ゆ}く
\ruby{方}{かた}を
\ruby{{\換字{逸}}}{そ}らして
\ruby{床}{とこ}の
\ruby{間}{ま}を
\ruby{見}{み}つ、
\ruby{其處}{そ|こ}に
\ruby{掛}{かゝ}れる
\ruby{狩野風}{かの|う|ふう}の
\ruby{{\換字{達}}磨}{だる|ま}を、たゞ
\ruby{譯}{わけ}も
\ruby{無}{な}く
\ruby{見}{み}つめながら、

『ナニ
\ruby{何樣}{ど|う}も
\ruby{仕}{し}は
\ruby{仕無}{し|な}いよ、
\ruby{心配}{しん|ぱい}して
\ruby{吳}{く}れたまふな。
』

と、
\ruby{然}{さ}ばかり
\ruby{我}{わ}が
\ruby{胸}{むね}の
\ruby{中}{うち}の
\ruby{苦惱}{くる|しみ}の
\ruby{色}{いろ}に
\ruby{出}{い}でゝ、
\ruby{人目}{ひと|め}に
\ruby{著}{しる}く
\ruby{現}{あら}はるゝかと
\ruby{驚}{おどろ}かるゝ
\ruby{心}{こゝろ}を
\ruby{押}{お}し
\ruby{隱}{かく}して
\ruby{答}{こた}へぬ。

『
\ruby{左樣}{さ|う}かエ。
それなら
\ruby{好}{い}いが
\ruby{餘}{あんま}り
\ruby{氣}{き}を
\ruby{使}{つか}つちやあいけないぜ、
\ruby{今日}{け|ふ} --- イヤ
\ruby{今日}{け|ふ}と
\ruby{云}{い}つちやあ
\ruby{既}{もう}
\ruby{十二時}{じう|に|じ}
\ruby{{\換字{過}}}{す}ぎだからをかしい。
\ruby{昨{\換字{宵}}}{ゆふ|べ}の
\ruby{會}{くわい}にも、
\ruby{君}{きみ}は
\ruby{幹事}{かん|じ}の
\ruby{山瀬}{やま|せ}のところへ、
\ruby{君}{きみ}の
\ruby{友人}{いう|じん}が
\ruby{大病}{たい|びやう}で、
\ruby{介抱}{かい|はう}の
\ruby{仕手}{し|て}も
\ruby{無}{な}いから
\ruby{其}{そ}の
\ruby{爲}{ため}に
\ruby{出}{で}ぬ、と
\ruby{云}{い}つて
\ruby{{\換字{遣}}}{や}つたさうだが、
\ruby{君}{きみ}は
\ruby{一體}{いつ|たい}
\ruby{{\換字{情}}}{じやう}が
\ruby{深}{ふ}か
\ruby{{\換字{過}}}{す}ぎるから、
\ruby{餘計}{よ|けい}にそれで
\ruby{心勞}{しん|らう}でも
\ruby{仕}{し}や
\ruby{仕無}{し|な}いかと、
\ruby{一同}{みん|な}が
\ruby{君}{きみ}の
\ruby{爲}{ため}に
\ruby{心配}{しん|ぱい}してゐたよ。
』

\ruby{島木}{しま|き}が
\ruby{言葉}{こと|ば}には
\ruby{何}{なん}の
\ruby{事}{こと}も
\ruby{無}{な}けれど、
\ruby{水野}{みづ|の}が
\ruby{胸}{むね}には
\ruby{響}{ひび}くところあり。

『ムヽ、
\ruby{昨{\換字{宵}}}{ゆふ|べ}の
\ruby{羽{\換字{勝}}}{は|がち}の
\ruby{君}{くん}の
\ruby{會}{くわい}に
\ruby{出無}{で|な}かつたのは、
\ruby{眞誠}{ほん|と}に
\ruby{諸君}{しよ|くん}に
\ruby{濟}{す}まなかつたが、
\ruby{實}{じつ}は
\ruby{如是}{か|う}してまご〳〵して
\ruby{居}{ゐ}て、
\ruby{今}{いま}
\ruby{頃君}{ごろ|きみ}のところへ
\ruby{來}{く}る
\ruby{位}{くらゐ}だから、
\ruby{何樣}{ど|う}か
\ruby{察}{さつ}して
\ruby{赦}{ゆる}して
\ruby{吳}{く}れたまへ。
』

『ナアニ
\ruby{赦}{ゆる}すも
\ruby{赦}{ゆる}さないも
\ruby{有}{あ}りあ
\ruby{仕無}{し|な}いが
\ruby{君}{きみ}のその
\ruby{友人}{いう|じん}の
\ruby{上}{うへ}は
\ruby{兎}{と}に
\ruby{角}{かく}、
\ruby{一同}{みん|な}は
\ruby{眞誠}{ほん|と}にたゞ
\ruby{君}{きみ}の
\ruby{上}{うへ}をいろ〳〵に
\ruby{心配}{しん|ぱい}してゐたよ。
』

『ヤ、
\ruby{眞}{しん}に
\ruby{諸君}{みん|な}の
\ruby{厚意}{こう|い}は
\ruby{深}{ふか}く
\ruby{謝}{しや}する。
\ruby{誰}{たれ}も
\ruby{僕}{ぼく}の
\ruby{不參}{ふ|さん}を
\ruby{怒}{おこ}りは
\ruby{仕無}{し|な}かつたかね。
\ruby{日方}{ひ|かた}
\ruby{君}{くん}は
\ruby{何}{なん}とも
\ruby{云}{い}は
\ruby{無}{な}かつたかね。
』

『ムヽ、
\ruby{日方}{ひ|かた}は
\ruby{何}{なに}を
\ruby{言}{い}つたつて
\ruby{管}{かま}や
\ruby{仕無}{し|な}いがね、
\ruby{羽{\換字{勝}}}{は|がち}は
\ruby{君}{きみ}に
\ruby{會}{あ}へなかつたのを、
\ruby{口}{くち}には
\ruby{出}{だ}さなかつたが
\ruby{酷}{ひど}く
\ruby{殘念}{ざん|ねん}がつて
\ruby{居}{ゐ}たよ。
』

『アヽ、
\ruby{羽{\換字{勝}}}{は|がち}
\ruby{君}{くん}には
\ruby{僕}{ぼく}も
\ruby{會}{あ}ひたがつたが、
\ruby{何}{なん}にしろ
\ruby{一方}{いつ|ぱう}の
\ruby{事}{こと}があつたので、
\ruby{懷}{なつ}かしくは
\ruby{思}{おも}ひながら
\ruby{意}{い}に
\ruby{任}{まか}せ
\ruby{無}{な}かつた。
アヽ
\ruby{僕}{ぼく}は
\ruby{羽{\換字{勝}}}{は|がち}
\ruby{君}{くん}に
\ruby{負}{そむ}いた、
\ruby{濟}{す}まなかつた。
』

\ruby{水野}{みづ|の}は
\ruby{{\換字{情}}}{じやう}に
\ruby{堪}{た}へざる
\ruby{如}{ごと}く、\換字{志}つと
\ruby{俯首}{うつ|む}きて
\ruby{眼}{め}を
\ruby{瞑}{ふさ}ぎつゝ、
\ruby{獨語}{ひとり|ごと}のやうに
\ruby{{\換字{又}}}{また}
\ruby{再度}{ふた|ゝび}、

『アヽ、
\ruby{濟}{す}まなかつた。
』

と、
\ruby{繰}{く}り
\ruby{{\換字{返}}}{かへ}しぬ。
\ruby{島木}{しま|き}は
\ruby{其}{そ}のいぢらしき
\ruby{樣子}{やう|す}を
\ruby{見}{み}て、
\ruby{此}{こ}の
\ruby{{\換字{猶}}}{なほ}
\ruby{心}{こゝろ}の
\ruby{醇}{じゆん}なる
\ruby{年若}{とし|わか}き
\ruby{友}{とも}を
\ruby{愛憐}{いと|ほし}む
\ruby{{\換字{情}}}{こゝろ}を
\ruby{起}{おこ}さゞるを
\ruby{得}{え}ざりき。

『マア
\ruby{其}{そ}りやあ
\ruby{其}{そ}れで
\ruby{濟}{す}んだ
\ruby{事}{こと}として、また
\ruby{羽{\換字{勝}}}{は|がち}に
\ruby{{\換字{遇}}}{あ}う
\ruby{時}{とき}も
\ruby{有}{あ}らうから
\ruby{好}{い}いぢやあ
\ruby{無}{な}いか。
さうして
\ruby{君}{きみ}のわざ〳〵
\ruby{來}{き}た
\ruby{用事}{よう|じ}といふのは?。
』

\ruby{問}{と}はれて
\ruby{水野}{みづ|の}は
\ruby{猛然}{まう|ぜん}と
\ruby{我}{われ}に
\ruby{復}{かへ}り、
\ruby{夜}{よ}を
\ruby{冒}{をか}し
\ruby{{\換字{遠}}}{とほき}を
\ruby{歩}{あゆ}みて
\ruby{此處}{こ|ゝ}に
\ruby{來}{きた}れるも、たゞ
\ruby{此}{こ}の
\ruby{一}{ひと}つの
\ruby{事}{こと}のためなるをやと、
\ruby{津}{わたり}に
\ruby{舟}{ふね}を
\ruby{得}{え}し
\ruby{心地}{こゝ|ち}して、
\ruby{自}{みづか}ら
\ruby{奮}{ふる}つて
\ruby{面}{おもて}を
\ruby{擡}{あ}げしが、
\ruby{慚}{は}づるところの
\ruby{有}{あ}ればにや
\ruby{直}{すぐ}に
\ruby{崩折}{くづ|を}れて、
\ruby{甲{\換字{斐}}無}{か|ひ|な}くも
\ruby{伏目}{ふし|め}になりて
\ruby{我}{わ}が
\ruby{膝}{ひざ}を
\ruby{見}{み}たり。

されど
\ruby{云}{い}はでは
\ruby{叶}{かな}はざることゝて、

『
\ruby{深夜}{しん|や}に
\ruby{君}{きみ}を
\ruby{驚}{おどろ}かしたのは
\ruby{濟}{す}ま
\ruby{無}{な}かつたが、かういふ
\ruby{譯}{わけ}だから
\ruby{聞}{き}いて
\ruby{吳}{く}れたまへ。
\ruby{實}{じつ}は
\ruby{僕}{ぼく}の
\ruby{出}{で}て
\ruby{居}{ゐ}る
\ruby{學校}{がく|かう}で、
\ruby{同}{おな}じ
\ruby{職}{しよく}を
\ruby{取}{と}つて
\ruby{居}{ゐ}るものに、
\ruby{僕}{ぼく}の
\ruby{新}{あたら}しい
\ruby{友人}{いう|じん}がある。
\ruby{其人}{そ|れ}は
\ruby{物}{もの}も
\ruby{出來}{で|き}れば
\ruby{氣立}{き|だて}も
\ruby{立派}{りつ|ぱ}な、まことに
\ruby{得難}{え|がた}い
\ruby{人物}{じん|ぶつ}なので、
\ruby{僕}{ぼく}は
\ruby{非常}{ひ|じやう}に
\ruby{大切}{たい|せつ}に
\ruby{思}{おも}つて
\ruby{居}{ゐ}る、ところが
\ruby{其人}{そ|れ}が
\ruby{大病}{たい|びやう}に
\ruby{罹}{かゝ}つた。
\ruby{一體}{いつ|たい}
\ruby{愍然}{あは|れ}な
\ruby{不幸}{ふ|かう}な
\ruby{人}{ひと}で、
\ruby{母}{はゝ}は
\ruby{有}{ゐ}るけれども
\ruby{継}{まゝ}しい
\ruby{中}{なか}で、
\ruby{病氣}{びやう|き}を
\ruby{知}{し}らせて
\ruby{{\換字{遣}}}{や}つても
\ruby{振}{ふ}り
\ruby{顧}{かへ}つても
\ruby{見無}{み|な}い
\ruby{位}{くらゐ}、それにまた
\ruby{家}{いへ}を
\ruby{貸}{か}して
\ruby{居}{ゐ}る
\ruby{婆}{ばゞあ}が
\ruby{殘酷}{ざん|こく}な
\ruby{奴}{やつ}で、
\ruby{病}{や}み
\ruby{惱}{なや}んで
\ruby{居}{ゐ}るものを
\ruby{{\換字{逐}}}{お}ひ
\ruby{出}{だ}さうといふ
\ruby{位}{くらゐ}な
\ruby{非{\換字{道}}}{ひ|だう}さ。
\ruby{左樣}{さ|う}いふ
\ruby{中}{なか}に
\ruby{悶臥}{もん|ぐわ}して
\ruby{居}{ゐ}て、
\ruby{誰}{たれ}に
\ruby{世話}{せ|わ}をされるといふ
\ruby{事}{こと}も
\ruby{無}{な}いので、
\ruby{可哀}{か|あい}さうに
\ruby{病人}{びやう|にん}は
\ruby{死}{し}を
\ruby{待}{ま}つばかりになつて
\ruby{居}{ゐ}るのだ。
そこで
\ruby{何樣}{だ|う}しても
\ruby{餘{\換字{所}}}{よ|そ}に
\ruby{見{\換字{兼}}}{み|か}ねるから、
\ruby{僕}{ぼく}が
\ruby{奔走}{ほん|そう}して
\ruby{良}{い}い
\ruby{醫者}{い|しや}に
\ruby{見}{み}せて
\ruby{{\換字{遣}}}{や}ると、
\ruby{病}{やまひ}は
\ruby{腸窒扶斯}{ちやう|ち|ぷ|す}だといふ
\ruby{事}{こと}で、
\ruby{看護}{かん|ご}が
\ruby{行屆}{ゆき|とゞ}か
\ruby{無}{な}けりやあ
\ruby{無}{な}い
\ruby{生命}{いの|ち}だといふ。
\ruby{僕}{ぼく}は
\ruby{自{\換字{分}}}{じ|ぶん}の
\ruby{肉}{にく}を
\ruby{{\換字{削}}}{そ}いで
\ruby{食}{く}はせてなりと、
\ruby{何樣}{ど|う}かして
\ruby{助}{たす}けて
\ruby{{\換字{遣}}}{や}りたいと
\ruby{思}{おも}ふのだが、……』

と、
\ruby{虛言}{う|そ}は
\ruby{少}{すこし}も
\ruby{無}{な}けれど
\ruby{忌}{い}むことは
\ruby{忌}{い}みて、
\ruby{此處}{こ|ゝ}までは
\ruby{云}{い}ひたりしが
\ruby{後}{あと}は
\ruby{言}{い}ひ
\ruby{澱}{よど}むを、
\ruby{其}{そ}の
\ruby{聲}{こゑ}の
\ruby{微}{かすか}に
\ruby{顫}{ふる}ふを
\ruby{聞}{き}き、
\ruby{其}{そ}の
\ruby{眼}{め}の
\ruby{濕}{ぬ}れ
\ruby{色}{いろ}なせるを
\ruby{見}{み}て、

『アヽ、
\ruby{解}{わか}つたよ、もう
\ruby{可}{い}いさ、
\ruby{君}{きみ}。
\ruby{金子}{か|ね}が
\ruby{先}{さき}に
\ruby{立}{た}つからと
\ruby{云}{い}ふのだらう。
\換字{志}て
\ruby{何}{ど}の
\ruby[<h||]{位}{くらゐ}
\ruby[||h>]{用立}{よう|だ}てやうかエ。
』

と、
\ruby{輕々}{かろ|〴〵}と
\ruby{事}{こと}も
\ruby{無}{な}げに
\ruby{引取}{ひつ|と}つて
\ruby{云}{い}つて、
\ruby{云}{い}ひ
\ruby{難}{にく}き
\ruby{口數}{くち|かず}を
\ruby{多}{おほ}くはきかせぬ
\ruby{同{\換字{情}}}{おも|ひやり}の
\ruby{骨}{ほね}に
\ruby{徹}{てつ}するほど
\ruby{嬉}{うれ}し
\ruby{悲}{かな}しく、

『
\ruby{濟}{す}まないけれども
\ruby{一時}{いち|どき}で
\ruby{無}{な}くとも
\ruby{可}{い}いから
\ruby{百圓}{ひやく|ゑん}ばかり、』

と、
\ruby{纔}{わづか}に
\ruby{口}{くち}を
\ruby{洩}{も}らせし
\ruby{限}{き}り、あとは
\ruby{無言}{む|ごん}の
\ruby{頭}{かうべ}を
\ruby{低}{た}れて、
\ruby{深々}{ふか|〴〵}と
\ruby{頼}{たの}み
\ruby{入}{い}りたりしが、
\ruby{何時}{い|つ}より
\ruby{出}{い}で
\ruby{居}{ゐ}し
\ruby{涙}{なみだ}なりけん、
\ruby{人}{ひと}の
\ruby{{\換字{情}}}{こゝろ}の
\ruby{凝}{こ}りて
\ruby{滴}{したゝ}る
\ruby{露}{つゆ}の
\ruby{眞玉}{ま|だま}はぱらりと
\ruby{墜}{お}ちたり。

\ruby{誠}{まこと}せめて
\ruby{人}{ひと}を
\ruby{頼}{たの}む
\ruby{心}{こゝろ}のいぢらしさも、
\ruby{何時}{い|つ}の
\ruby{間}{ま}にか
\ruby{謹}{つゝし}みて
\ruby{律義}{りち|ぎ}に
\ruby{端座}{す|わ}り
\ruby{居}{ゐ}たる、
\ruby{水野}{みづ|の}が
\ruby{身}{み}を
\ruby{窄}{すぼ}めし
\ruby{姿}{すがた}の
\ruby{{\換字{寒}}}{さむ}げなるを
\ruby{見}{み}て、
\ruby{島木}{しま|き}は
\ruby{思}{おも}はず
\ruby{慨然}{がい|ぜん}として、

『ナアニ
\ruby{可}{い}いさ。
\ruby{君}{きみ}、それんばかりの
\ruby{事}{こと}を。
\ruby{宜}{よろ}しい、
\ruby{承知}{しよう|ち}した。
\ruby{今}{いま}
\ruby{直}{すぐ}
\ruby{献}{あ}げる。
』

と、
\ruby{確然}{しつ|かり}と
\ruby{明}{あき}らかに
\ruby{先}{ま}づ
\ruby{答}{こた}へつ、
\ruby{少時}{しば|し}
\ruby{間}{あひだ}を
\ruby{置}{お}きて、

『\換字{志}かし、
\ruby{君}{きみ}、
\ruby{僕}{ぼく}は
\ruby{何}{なに}も
\ruby{君}{きみ}に
\ruby{恨}{うら}みを
\ruby{云}{い}ふのでは
\ruby{無}{な}いが、
\ruby{何故}{な|ぜ}
\ruby{君}{きみ}は
\ruby{僕}{ぼく}に
\ruby{其}{そ}の
\ruby{友人}{いう|じん}の
\ruby{名}{な}を、
\ruby{岩崎五十子}{いは|ざき|い|そ|こ}といふものだとは
\ruby{云}{い}つて
\ruby{吳}{く}れぬ?。
イヤ、
\ruby{吃驚}{びつ|くり}しないでも
\ruby{宜}{よ}い、
\ruby{意見}{い|けん}は
\ruby{云}{い}は
\ruby{無}{な}いが、』

と、
\ruby{何事}{なに|ごと}をか
\ruby{徐}{しづか}に
\ruby{云}{い}ひ
\ruby{出}{だ}さんとすれば、
\ruby{水野}{みづ|の}が
\ruby{面}{おもて}はたゞ
\ruby{火}{ひ}となつたり。

\Entry{其十七}

\ruby{自信}{じ|しん}は
\ruby{{\換字{強}}}{つよ}くとも、
\ruby{學問}{がく|もん}は
\ruby{博}{ひろ}くとも、
\ruby{氣}{き}の
\ruby{働}{はたら}きは
\ruby{八方}{はつ|ぱう}に
\ruby{{\換字{銳}}}{するど}くとも、
\ruby{未}{ま}だ
\ruby{世}{よ}に
\ruby{老}{お}いぬ
\ruby{心}{こヽろ}の
\ruby{柔輭}{やは|らか}に
\ruby{嫩}{わか}ければ、
\ruby{人}{ひと}には
\ruby{知}{し}らさず
\ruby{祕}{ひ}め
\ruby{置}{お}きたることを、つけ〳〵と
\ruby{覿面}{てき|めん}に
\ruby{云}{い}ひ
\ruby{出}{いだ}されては、
\ruby{胸}{むね}の
\ruby{眞正中}{まつ|たゞ|なか}を\GWI{koseki-900370}たゝかなる
\ruby{箭}{や}に、
\ruby{羽中}{は|なか}の
\ruby{{\換字{節}}}{ふし}せめて
\ruby{射込}{い|こ}まれたる
\ruby{思}{おも}ひして、ハツと
\ruby{驚}{おどろ}き
\ruby{惑}{まど}ひしが、
\ruby{元來}{も|と}
\ruby{底}{そこ}の
\ruby{{\換字{弱}}}{よわ}からぬ
\ruby{男}{をとこ}なり、
\ruby{忽}{たちま}ち
\ruby{我}{われ}に
\ruby{{\換字{返}}}{かへ}つて
\ruby{惡}{わる}びれず、
\ruby{靜}{しづか}かに
\ruby{我}{わ}が
\ruby{腔内}{む|ね}の
\ruby{血}{ち}の
\ruby{跳}{おど}りの
\ruby{鎭}{しづ}まるを
\ruby{待}{ま}ちながら、
\ruby{身動}{み|うご}きだにせずして
\ruby{大人}{おと|な}しく、
\ruby{島木}{しま|き}のいふところを
\ruby{聞}{き}かんと
\ruby{仕}{し}たり。

\ruby{島木}{しま|き}は
\ruby{人}{ひと}の
\ruby{{\換字{情}}}{こヽろ}の
\ruby{流}{なが}れの
\ruby{瀬}{せ}に、
\ruby{慣}{な}れきつたる
\ruby{鵜}{う}の
\ruby{目}{め}の
\ruby{働}{はたら}き
\ruby{敏捷}{す|ばや}く、
\ruby{日}{ひ}の
\ruby{光}{ひかり}の
\ruby{明}{あき}らかなるに
\ruby{我}{わ}が
\ruby{影}{かげ}を
\ruby{怯}{お}づる
\ruby{若鮎}{わか|あゆ}の
\ruby{振舞}{ふる|まひ}の、
\ruby{優}{やさ}しくも\GWI{koseki-900370}ほらしき
\ruby[g]{水野}{みづの}が
\ruby{様子}{やう|す}を
\ruby{見}{み}て
\ruby{取}{と}つて、
\ruby{曾}{かつ}て
\ruby[g]{吉右衛門}{きちゑもん}より
\ruby{聞}{き}きしと、
\ruby{今}{いま}
\ruby{直接}{ぢ|か}に
\ruby{聞}{き}きしとの
\ruby{二}{ふた}つの
\ruby[g]{談話}{はなし}に
\ruby{照}{て}らし
\ruby{合}{あ}はせて、
\ruby{大槪}{おほ|よそ}の
\ruby{事}{こと}は
\ruby{曉}{さと}り
\ruby{盡}{つく}しつ、
\ruby{今更}{いま|さら}にまた
\ruby{油然}{ゆう|ぜん}として
\ruby{愛憐}{いと|ほし}む
\ruby{心}{こヽろ}の
\ruby{起}{おこ}るに
\ruby{堪}{た}へぬが
\ruby{如}{ごと}く、
\ruby{言葉}{こと|ば}づかひも
\ruby{砕}{くだ}けて
\ruby{露隔氣}{つゆ|へだ|て}なく、いと
\ruby{親}{した}しくも
\ruby{說}{と}き
\ruby{出}{いだ}したり。

『ねえ
\ruby{君}{きみ}、
\ruby{可厭}{い|や}なものは、
\ruby{無心}{む|しん}を
\ruby{聽}{き}いた
\ruby{後}{あと}で
\ruby{意見}{い|けん}
\ruby{云}{い}ふ
\ruby{奴}{やつ}だと、
\ruby{古}{むかし}から
\ruby{云}{い}つてあるぢやあ
\ruby{無}{な}いか。
ハヽヽまさかに
\ruby{僕}{ぼく}だつて
\ruby{其位}{その|くらひ}な
\ruby{事}{こと}は
\ruby{知}{し}つて
\ruby{居}{ゐ}るから、
\ruby{此處}{こ|ゝ}で
\ruby{下手}{へ|た}な
\ruby{叔{\換字{父}}}{を|ぢ}さんの
\ruby{役}{やく}を
\ruby{{\換字{勤}}}{つと}めて、
\ruby{何}{なん}の
\ruby{彼}{か}のと
\ruby{難}{むづ}かしい
\ruby{事}{こと}を
\ruby{云}{い}ふなあ
\ruby{自分}{じ|ぶん}で
\ruby{願}{ねが}ひ
\ruby{下}{さ}げるし、
\ruby{{\換字{叉}}理屈}{また|り|くつ}なんぞといふ
\ruby{野暮}{や|ぼ}なものを、
\ruby{餘}{あま}り
\ruby{有}{あ}り
\ruby{難}{がた}いと
\ruby{思}{おも}つてゐる
\ruby{僕}{ぼく}でも
\ruby{無}{な}いから、
\ruby{君}{きみ}が
\ruby{何様仕}{ど|う|し}やうと、それを
\ruby{兎}{と}や
\ruby{角}{かく}いふ
\ruby{僕}{ぼく}ぢやあ
\ruby{無}{な}い。
\ruby{惡}{わる}い
\ruby{事}{こと}さへ
\ruby{仕無}{し|な}けりやあ、
\ruby{好}{すき}きな
\ruby{事}{こと}を
\ruby{仕}{し}て
\ruby{面白}{おも|しろ}く
\ruby{世}{よ}を
\ruby{渡}{わた}るのが、
\ruby{可}{い}いぢやあ
\ruby{無}{な}いかといふのが
\ruby{僕}{ぼく}の
\ruby{宗旨}{しゆう|し}なのは、
\ruby{君}{きみ}も
\ruby{知}{し}つて
\ruby{居}{ゐ}る
\ruby{{\換字{通}}}{とほ}りの
\ruby{事}{こと}だ。
だから
\ruby{意見}{い|けん}と
\ruby{思}{おも}つて
\ruby{聞}{き}いて
\ruby{{\換字{呉}}}{く}れちやあ
\ruby{困}{こま}るが、たつた
\ruby{一}{ひと}つ
\ruby{君}{きみ}に
\ruby{聞}{き}いて
\ruby{置}{お}いて
\ruby{貰}{もら}ひたい
\ruby{事}{こと}がある。
\ruby{下}{くだ}らない
\ruby{事}{こと}では
\ruby{有}{あ}らうが、聞いて
\ruby{{\換字{呉}}}{く}れたまへ。
\ruby{僕}{ぼく}は
\ruby{{\GWI{u968f-var-001}}分}{ずゐ|ぶん}
\ruby{今}{いま}までの
\ruby[g]{品行}{みもち}が、
\ruby{疵瑕}{き|ず}だらけの
\ruby{大馬鹿}{おほ|ば|か}な
\ruby{奴}{やつ}なんだから、
\ruby{當世}{たう|せい}でよく
\ruby{云}{い}ふ
\ruby{神聖}{しん|せい}な
\ruby{戀愛}{れん|あい}、--- そんな
\ruby{上品}{じやう|ひん}なものあ
\ruby{知}{し}らないが、
\ruby{戀愛}{れん|あい}も
\ruby{惚}{ほ}れたはれたも
\ruby{同}{おな}じ
\ruby{事}{こと}として、マア
\ruby{僕}{ぼく}だけで
\ruby{云}{い}つて
\ruby{見}{み}りやあ、
\ruby{戀愛}{れん|あい}は
\ruby{可怖}{こ|は}いものぢやあ
\ruby{無}{な}いが、
\ruby{戀愛}{れん|あい}に
\ruby{{\GWI{u968f-var-001}}}{つ}いて
\ruby{來}{く}る
\ruby{{\GWI{u968f-var-001}}{\換字{伴}}者}{お|と|も}は
\ruby{怖}{こは}い、とつく〴〵
\ruby{身}{み}に
\ruby{染}{し}みて
\ruby{覺}{おぼ}えて
\ruby{居}{ゐ}るんだ。
そこで
\ruby{君}{きみ}に
\ruby{其}{そ}の
\ruby{{\GWI{u968f-var-001}}{\換字{伴}}者}{お|と|も}だけにやあ
\ruby{戒愼}{よう|じん}して
\ruby{貰}{もら}ひたいと
\ruby{思}{おも}ふ。
\ruby{云}{い}つて
\ruby{置}{お}きたいと
\ruby{云}{い}ふのは
\ruby{只}{たゞ}これ
\ruby{一}{ひと}つだ。
いゝかエ、
\ruby{惚}{ほ}れたはれたの
\ruby{其}{そ}の
\ruby{{\換字{迷}}}{まよ}ひは、
\ruby{些}{ちつと}も
\ruby{可怖}{こ|は}い
\ruby{事}{こと}は
\ruby{無}{な}いが、それに
\ruby{付}{つ}いて
\ruby{來}{く}る
\ruby{{\GWI{u968f-var-001}}{\換字{伴}}者}{お|と|も}は
\ruby{怖}{こは}い
\ruby{危険}{き|けん}ものだといふのだよ。
』


\Entry{其十八}

\原本頁{}
\ruby{僕}{ぼく}は
\ruby{元}{もと}から
\ruby{學問}{がく|もん}は
\ruby{{\換字{嫌}}}{きら}ひだし、
%
\ruby{身}{み}に
\ruby{浸}{し}みて
\ruby{書}{ほん}を
\ruby{讀}{よん}んだ
\ruby{事}{こと}も
\ruby{無}{な}いから、
%
どうせ
\ruby{僕}{ぼく}の
\ruby{云}{い}ふ
\ruby{事}{こと}なぞは
\ruby{下}{くだ}ら
\ruby{無}{な}からうが、
%
まんざら
\ruby{正中}{つ|ぼ}に
\ruby{外}{はづ}れたことも
\ruby{云}{い}は
\ruby{無}{な}いつもりだ。
%
かういふ
\ruby{理屈}{り|くつ}だ、
%
\ruby{聞}{き}いて
\ruby{吳}{く}れたまへ。
%
\ruby{僕}{ぼく}に
\ruby{云}{い}はせりやあ
\ruby{色戀}{いろ|こひ}といふ
\ruby{奴}{やつ}あ、
%
\ruby{人間}{にん|げん}が
\ruby{一人並}{いち|にん|なみ}に
\ruby{成熟}{でき|あが}ると、
%
\ruby{一度}{いち|ど}は
\ruby{屹度}{きつ|と}
\ruby{發}{はつ}しる
\ruby{熱病}{ね|つ}なので、
%
\ruby{身體}{から|だ}の
\ruby{中}{なか}から
\ruby{自然}{ひと|りで}に
\ruby{湧}{わ}く
\ruby{奴}{やつ}だ、
%
\ruby{各自}{めい|〳〵}の
\ruby{料簡}{れう|けん}から
\ruby{出}{で}て
\ruby{來}{く}るんぢやあ
\ruby{無}{な}い。
%
そりやあ
\ruby{其}{そ}の
\ruby{當人}{たう|にん}から
\ruby{云}{い}つて
\ruby{見}{み}りやあ、
%
\ruby{彼處}{あそ|こ}が
\ruby{好}{い}いとか、
%
\ruby{此處}{こ|ゝ}が
\ruby{好}{い}いとか、
%
それ〴〵に
\ruby{理由}{わ|け}が
\ruby{有}{あ}つて
\ruby{惚}{ほ}れるのでも
\ruby{有}{あ}らうが、
%
ナアニ
\ruby{年齡}{と|し}が
\ruby{爲}{さ}せるんだよ、
%
\ruby{年齡}{と|し}が
\ruby{爲}{さ}せるんだよ。
%
\ruby{彼}{あ}の
\ruby{女}{をんな}あ
\ruby{好}{い}いからサア
\ruby{惚}{ほ}れて
\ruby{{\換字{遣}}}{や}らうと、
%
\ruby{{\換字{分}}別}{ふん|べつ}をつけてから
\ruby{惚}{ほ}れる
\ruby{奴}{やつ}は
\ruby{無}{な}い。
%
\ruby{誰}{たれ}の
\ruby{戀路}{こひ|ぢ}も
\ruby{同}{おんな}じ
\ruby{事}{こと}で、
%
\ruby{其}{そ}の
\ruby{眞實}{ほん|たう}のところを
\ruby{云}{い}やあ、
%
\ruby{自{\換字{分}}}{じ|ぶん}にも
\ruby{理由}{わ|け}は
\ruby{{\換字{分}}}{わか}らないけれど、
%
\ruby{何}{なん}だか
\ruby{知}{し}ら
\ruby{無}{な}いが
\ruby{自然}{ひと|りで}に
\ruby{好}{す}く、
%
それが
\ruby{抑々}{そも|〳〵}の
\ruby{發端}{はじ|まり}で、
%
\ruby{其}{そ}の
\ruby{人}{ひと}の
\ruby{笑顏}{ゑ|がほ}なんぞが
\ruby{何時}{い|つ}の
\ruby{間}{ま}にか
\ruby{眼}{め}に
\ruby{染}{し}み
\ruby{付}{つ}いて
\ruby{{\換字{遺}}}{のこ}つたり、
%
\ruby{物}{もの}を
\ruby{云}{い}つた
\ruby{聲}{こゑ}の
\ruby{色}{いろ}が
\ruby{耳}{みゝ}に
\ruby{{\換字{遺}}}{のこ}つたりして、
%
\ruby{{\換字{終}}}{しまひ}にはすつかり
\ruby{其人}{その|ひと}が
\ruby{自{\換字{分}}}{じ|ぶん}の
\ruby{胸}{むね}の
\ruby{中}{うち}に
\ruby{在}{あ}るやうになる、
%
サア
\ruby{忘}{わす}れやうと
\ruby{思}{おも}つても
\ruby{忘}{わす}れられない、
%
\ruby{始{\換字{終}}}{し|ゞう}
\ruby{其人}{その|ひと}の
\ruby{傍}{そば}に
\ruby{居}{ゐ}て
\ruby{見}{み}たくなる、
%
\ruby{離}{はな}れて
\ruby{居}{ゐ}ちやあ
\ruby{物悲}{もの|がな}しくつて、
%
\ruby{何}{なん}と
\ruby{無}{な}く
\ruby{氣}{き}が
\ruby{濟}{す}まないやうな
\ruby{心持}{こゝろ|もち}ちがする、
%
\ruby{自{\換字{分}}}{じ|ぶん}が
\ruby{其人}{その|ひと}を
\ruby{思}{おも}ふやうに、
%
\ruby{其人}{その|ひと}にも
\ruby{自{\換字{分}}}{じ|ぶん}を
\ruby{思}{おも}つて
\ruby{貰}{もら}ひたくなる、
%
それから
\ruby{段々}{だん|〴〵}と
\ruby{泣}{な}いたり
\ruby{笑}{わら}つたりが
\ruby{始}{はじ}まる、
%
まあ
\ruby{斯樣}{か|う}
\ruby{云}{い}つた
\ruby{順立}{じゆん|だて}ぢやあ
\ruby{無}{な}いか。
%
\換字{志}て
\ruby{見}{み}りやあ
\ruby{自然}{ひと|りで}に
\ruby{好}{す}くといふのが
\ruby{戀}{こひ}の
\ruby{水上}{みな|かみ}だが、
%
\ruby{自然}{ひと|りで}の
\ruby{好惡}{すき|きらひ}だもの、
%
\ruby{理屈}{り|くつ}は
\ruby{有}{あ}りや
\ruby{仕無}{し|な}い、
%
みんな
\ruby{年齡}{と|し}が
\ruby{爲}{さ}せるんだ。
%
\ruby{懷姙者}{み|も|ち}は
\ruby{酸}{す}いものを
\ruby{自然}{ひと|りで}に
\ruby{好}{す}く、
%
\ruby{溜飮}{りう|いん}
\ruby{持}{もち}は
\ruby{香物}{かう|〳〵}で
\ruby{茶濱飯}{ちや|づ|け}を
\ruby{自然}{ひと|りで}に
\ruby{好}{す}く、
%
\ruby{其}{そ}の
\ruby{自然}{ひと|りで}に
\ruby{好}{す}くのは
\ruby{誰}{たれ}がさせる?、
%
\ruby{惡阻}{つは|り}が
\ruby{爲}{さ}せるんだ、
%
\ruby{溜飮}{りう|いん}が
\ruby{爲}{さ}せるんだ、
%
\ruby{戀路}{こひ|ぢ}の
\ruby{{\換字{迷}}惑}{まよ|ひ}は
\ruby{年齡}{と|し}が
\ruby{爲}{さ}せるんだ。
%
\ruby{男兒}{をと|こ}が
\ruby{男兒}{をと|こ}づくる
\ruby{頃}{ころ}にやあ
\ruby{髭鬚}{ひ|げ}が
\ruby{生}{は}えて
\ruby{來}{く}る、
%
\ruby{髭鬚}{ひ|げ}の
\ruby{生}{は}えるのは
\ruby{年齡}{と|し}が
\ruby{爲}{さ}せるんだもの、
%
それに
\ruby{善}{よ}いも
\ruby{惡}{わる}いも
\ruby{有}{あ}りやうは
\ruby{無}{な}い、
%
\ruby{口}{くち}の
\ruby{周圍}{まは|り}に
\ruby{出}{で}て
\ruby{來}{く}る
\ruby{髭鬚}{ひ|げ}も、
%
\ruby{心}{こゝろ}の
\ruby{上}{うへ}に
\ruby{萌}{めぐ}む
\ruby{戀}{こひ}も、
%
\ruby{年端}{と|し}が
\ruby{爲}{さ}せるに
\ruby{差異}{ちが|ひ}は
\ruby{無}{な}い、
%
\ruby{丁度}{ちやう|ど}
\ruby{同}{おんな}じ
\ruby{事}{こと}だもの、
%
ナニ
\ruby{戀愛}{こ|ひ}を
\ruby{善}{い}いとも
\ruby{惡}{わる}いとも
\ruby{云}{い}はう
\ruby{譯}{わけ}は
\ruby{無}{な}い。
%
たゞ
\ruby{年齡}{と|し}が
\ruby{爲}{さ}せる
\ruby{熱病}{ね|つ}をすらりと
\ruby{濟}{すま}せて
\ruby{仕舞}{し|ま}へば、
%
\ruby{疱瘡}{はう|さう}や
\ruby{痲疹}{はし|か}が
\ruby{濟}{す}んだと
\ruby{同}{おな}じに、
%
つまり
\ruby{芽出度}{め|で|たい}と
\ruby{云}{い}へば
\ruby{云}{い}へるので、
%
\ruby{戀}{こひ}は
\ruby{怖}{おそ}ろしいものでも
\ruby{何}{なん}でも
\ruby{無}{な}い。
%
\ruby{併}{しか}し
\ruby{{\換字{又}}}{また}、
%
\ruby{君}{きみ}は
\ruby{學問}{がく|もん}もあり
\ruby{思慮}{し|りよ}もあるから、
%
\ruby{萬々}{ばん|〳〵}
\ruby{承知}{しよう|ち}
\ruby{仕}{し}て
\ruby{居}{ゐ}やうが、
%
お
\ruby{互}{たがひ}いに
\ruby{男兒}{をと|こ}といふ
\ruby{奴}{やつ}は、
%
\ruby{戀愛}{こ|ひ}の
\ruby{奴隷}{け|らい}に
\ruby{生}{う}まれて
\ruby{居}{ゐ}るものでも
\ruby{何}{なん}でも
\ruby{無}{な}い、
%
それ〴〵
\ruby{男子}{をと|こ}
\ruby{一匹}{いつ|ぴき}
\ruby{{\換字{前}}}{まへ}の
\ruby{目的}{もく|てき}のために
\ruby{意氣}{い|き}
\ruby{地}{ぢ}を
\ruby{磨}{みが}いて
\ruby{一生}{いつ|しやう}を
\ruby{働}{はたら}いて
\ruby{行}{ゆ}かうといふ
\ruby{身}{み}、
%
\ruby{戀}{こひ}に
\ruby{捲}{ま}き
\ruby{倒}{たふ}されちやあならねえ
\ruby{身體}{から|だ}だ、
%
\ruby{其}{そ}の
\ruby{熱病}{ね|つ}に
\ruby{身體}{から|だ}を
\ruby{{\換字{遣}}}{や}る
\ruby{譯}{わけ}にやあいかねえ
\ruby{約束}{やく|そく}がある。
%
\ruby{病}{やまひ}にも
\ruby{輕}{かる}い
\ruby{重}{おも}いはあり、
%
\ruby{戀}{こひ}にも
\ruby{深}{ふか}い
\ruby{淺}{あさ}いは
\ruby{有}{あ}らうが、
%
\ruby{如何}{い|か}に
\ruby{戀}{こひ}に
\ruby{惱}{なや}んでも
\ruby{苦}{くる}しんでも、
%
\ruby{吐}{つ}く
\ruby{息}{いき}が
\ruby{火}{ひ}になつて
\ruby{燃}{も}えるほどに
\ruby{狂}{くる}はうとも、
%
\ruby{戀}{こひ}に
\ruby{負}{ま}けて
\ruby{死}{し}んぢやあ
\ruby{男子}{をと|こ}たる
\ruby{身}{み}の、
%
\ruby{眼}{め}が
\ruby{瞑}{ふさ}げねえ
\ruby{筈}{はず}だ。
%
イヤ
\ruby{瞑}{ふさ}げねえ、
%
どうしても
\ruby{死}{しに}きれねえ、
%
\ruby{死}{し}ね
\ruby{無}{ね}え
\ruby{筈}{はず}だ。
%
\ruby{乃公}{お|ら}あ
\ruby{死}{し}な
\ruby{無}{ね}え、
%
\ruby{死}{し}にも
\ruby{仕無}{し|ね}えが、
%
\ruby{汝}{おめへ}も
\ruby{死}{し}ねめえ、
%
\ruby{死}{し}にもすめえナ。
%
\ruby{知}{し}れ
\ruby{切}{き}つた
\ruby{事}{こと}だが、
%
ナア
\ruby[g]{水野}{みづの}、
%
お
\ruby{互}{たがひ}いに
\ruby{幾干{\換字{若}}干}{いく|そ|ばく|そ}の
\ruby{苦勞}{く|らう}を
\ruby{仕}{し}て、
%
\ruby{今日}{け|ふ}まで
\ruby{{\換字{遣}}}{や}つて
\ruby{來}{き}たなあ
\ruby{何}{なん}の
\ruby{爲}{ため}だ?。
%
\ruby{志}{こゝろざし}こそ
\ruby{異}{ちが}ふけれど、
%
\ruby{男兒}{をと|こ}と
\ruby{生}{うま}れた
\ruby{生}{うま}れ
\ruby{甲{\換字{斐}}}{が|ひ}にやあ、
%
\ruby{各自}{めい|〳〵}の
\ruby{念願}{おも|ひ}を
\ruby{{\換字{遂}}}{と}げやうと、
%
そればつかりの
\ruby{爲}{ため}ぢやあ
\ruby{無}{ね}えか。
%
\ruby{特}{こと}さら
\ruby{汝}{おめへ}は
\ruby{乃公}{お|れ}から
\ruby{云}{い}やあ、
%
マア
\ruby{慾}{よく}の
\ruby{無}{な}さすぎる
\ruby{偏人}{へん|じん}で、
%
\ruby{取}{と}れる
\ruby{錢}{ぜに}も
\ruby{取}{と}らず
\ruby{出世}{しゆつ|せ}も
\ruby{望}{のぞ}まず、
%
\ruby{大根}{だい|こん}
\ruby{人參}{にん|じん}の
\ruby{尻尾}{しつ|ぽ}を
\ruby{咬}{かじ}つて、
%
それで
\ruby{濟}{す}まして
\ruby{居}{ゐ}るやうな
\ruby{{\換字{遣}}}{や}り
\ruby{方}{かた}。
%
アヽ
\ruby{世}{よ}の
\ruby{中}{なか}はいろ〳〵のもんだ、
%
\ruby[g]{水野}{みづの}だつて
\ruby{不味}{ま|づ}いものあ
\ruby{不味}{ま|づ}く、
%
\ruby{美味}{う|ま}いものは
\ruby{旨}{うま}からうが、
%
\ruby{其}{それ}にも
\ruby{此}{これ}にも
\ruby{頓着無}{とん|ぢやく|な}く、
%
\ruby{{\換字{若}}}{わか}い
\ruby{身天}{み|そら}で
\ruby{色氣}{いろ|け}も
\ruby{無}{な}く、
%
\ruby{下手}{へ|た}な
\ruby{律僧}{りつ|そう}は
\ruby{及}{およ}ばぬ
\ruby{身持}{み|もち}で、
%
たゞ
\ruby{學問}{がく|もん}に
\ruby{凝}{こ}つて
\ruby{居}{ゐ}る、
%
アヽ
\ruby{聖人}{せい|じん}と
\ruby{云}{い}ふなあ
\ruby{彼樣}{あ|ん}な
\ruby{男}{をとこ}の
\ruby{事}{こと}か
\ruby{知}{し}らん、
%
\ruby{餘{\換字{所}}目}{よ|そ|め}から
\ruby{見}{み}ては
\ruby{氣}{き}が
\ruby{竭}{つ}きて、
%
\ruby{何}{なん}だか
\ruby{憫然}{かあ|いさう}なやうな% 「憫然 か(あ)いさう」
\ruby{氣}{き}がすると、
%
\ruby{思}{おも}つた
\ruby{位}{くらゐ}に
\ruby{月日}{つき|ひ}を
\ruby{經}{へ}て
\ruby{來}{き}た、
%
\ruby{其}{そ}の
\ruby{汝}{おめへ}の
\ruby{難行苦行}{なん|ぎやう|く|ぎやう}も
\ruby{何}{なん}の
\ruby{爲}{ため}だ。
%
やつぱり
\ruby{何時}{い|つ}か
\ruby{一度}{いち|ど}は
\ruby{汝}{おめへ}は
\ruby{汝}{おめへ}で、
%
\ruby{男兒}{をと|こ}
\ruby{甲{\換字{斐}}}{が|ひ}のある
\ruby{仕事}{し|ごと}を
\ruby{仕}{し}やうためばかりの
\ruby{事}{こと}ぢやあ
\ruby{無}{な}いか。
%
その
\ruby{木食坊主}{もく|じき|ばう|ず}かなんぞのやうな、
%
\ruby{味}{あぢ}の
\ruby{無}{な}い
\ruby{長}{なが}い
\ruby{月日}{つき|ひ}の
\ruby{生活}{くら|し}さへも、
%
\ruby{笑}{わら}つて
\ruby{仕}{し}て
\ruby{來}{き}た
\ruby{汝}{おめへ}だもの、
%
\ruby{何樣}{ど|ん}な
\ruby{苦}{くる}しい
\ruby{戀}{こひ}に
\ruby{落}{お}ちても、
%
よもや
\ruby{本心}{ほん|しん}を
\ruby{失}{うしな}つて、
%
\ruby{熱病}{ね|つ}に
\ruby{負}{ま}けて
\ruby{仕舞}{し|ま}ふやうなことは
\ruby{有}{あ}るめえが、
%
さあ、
%
\ruby{戀愛}{こ|ひ}は
\ruby{怖}{こは}かあ
\ruby{無}{ね}えが
\ruby{隨{\換字{伴}}者}{お|と|も}が
\ruby{怖}{こは}い、
%
\ruby{案}{あん}じられてならねえところが
\ruby{其處}{そ|こ}にある!。

\Entry{其十九}

『
\ruby{{\換字{随}}{\換字{伴}}者}{お|と|も}と
\ruby{云}{い}ふなあ
\ruby{他}{ほか}ぢやあ
\ruby{無}{ね}えが、
\ruby{戀}{こひ}に
\ruby{{\換字{随}}}{つ}いて
\ruby{來}{く}る
\ruby{心氣}{し|ん}の
\ruby{疲勞}{つか|れ}だ。
お
\ruby{互}{たがひ}いに
\ruby{覺}{おぼ}えのある
\ruby{事}{こと}だが、
\ruby{男}{をとこ}の
\ruby{兒}{こ}といふ
\ruby{奴}{やつ}あ
\ruby{十三四}{じう|さん|し}から、そろ〳〵
\ruby{野心}{や|しん}が
\ruby{燃}{も}え
\ruby{立}{た}つて
\ruby{來}{き}て、
\ruby{威張}{ゐ|ば}つて
\ruby{見}{み}たい、
\ruby{人}{ひと}に
\ruby{{\換字{勝}}}{か}ちたい、
\ruby{功}{てがら}が
\ruby{立}{た}てたい、
\ruby{名}{な}が
\ruby{立}{た}てたい、
\ruby{天下}{てん|か}が
\ruby{取}{と}りたい、と
\ruby{氣象相應}{き|しやう|さう|おう}の
\ruby{望}{のぞ}みを
\ruby{起}{おこ}すが、それでも
\ruby{其}{そ}の
\ruby{時{\換字{分}}}{じ|ぶん}の
\ruby{腹中}{はらん|なか}は
\ruby{淸潔}{せい|けつ}なもので、たゞ
\ruby{醇醉}{いつ|ぽんぎ}の
\ruby{大望心}{たい|ま|う}があるばかり、
\ruby{乃公}{お|ら}あ
\ruby{太閤}{たい|かふ}だぞ、
\ruby{拿破崙}{なぽ|れお|ん}だぞと、
\ruby{各自}{てん|〴〵}に
\ruby{力}{りき}む
\ruby{其勢}{その|いきほひ}で、
\ruby{伸}{の}びも
\ruby{育}{そだ}ちも
\ruby{仕}{し}て
\ruby{來}{く}るが、
\ruby{遲}{おそ}かれ
\ruby{{\換字{速}}}{はや}かれ
\ruby{時{\換字{節}}}{と|き}が
\ruby{來}{き}て、
\ruby{戀}{こひ}という
\ruby{奴}{やつ}に
\ruby{魅入}{み|い}られちやあ、さあ
\ruby{腹}{はら}の
\ruby{中}{なか}が
\ruby{揉}{も}めて
\ruby{來}{く}る。
\ruby{大望心}{たい|ま|う}は
\ruby{大望心}{たい|ま|う}で
\ruby{居}{ゐ}しかつて
\ruby{居}{ゐ}る、
\ruby{戀}{こひ}の
\ruby{心}{こゝろ}は
\ruby{戀}{こひ}の
\ruby{心}{こゝろ}で
\ruby{自由}{ま|ゝ}に
\ruby{働}{はたら}く。
\ruby{双方}{さう|はう}が
\ruby{頭}{かしら}は
\ruby{下}{さ}げないから、
\ruby{衝突}{ぶつ|か}りやあ
\ruby{何樣}{ど|う}しても
\ruby{忽}{たちま}ち
\ruby{戰爭}{たゝ|かひ}で、
\ruby{那方}{どつ|ち}が
\ruby{{\換字{勝}}}{か}つにしても
\ruby{負}{ま}けるにしても、なか〳〵
\ruby{樂}{らく}な
\ruby{爭鬩}{せり|あひ}ぢやあ
\ruby{無}{な}い。
\ruby{戀}{こひ}が
\ruby{負}{ま}けて
\ruby{倒}{たふ}れりやあ
\ruby{其}{そ}の
\ruby{傷口}{きず|ぐち}から、
\ruby{溢}{あふ}れる
\ruby{血潮}{ち|しほ}が
\ruby{急}{きふ}にやあ
\ruby{止}{と}まらず、
\ruby{大望心}{たい|ま|う}が
\ruby{負}{ま}けりやあ
\ruby{其}{そ}の
\ruby{英氣}{えい|き}は、
\ruby{未練氣}{み|れん|げ}
\ruby{無}{な}く
\ruby{去}{さ}つて
\ruby{仕舞}{し|ま}つて
\ruby{呼}{よ}んでも
\ruby{{\換字{還}}}{かへ}らねえ。
つまり
\ruby{何樣}{ど|う}なつても
\ruby{根}{ね}が
\ruby{同士討}{どう|し|うち}の、
\ruby{酷}{ひど}い
\ruby{戰爭}{たゝ|かひ}に
\ruby{國土}{く|に}は
\ruby{荒}{あ}れて、
\ruby{{\換字{遺}}}{のこ}るものは
\ruby{怖}{おそ}ろしい
\ruby{心氣}{し|ん}の
\ruby{疲勞}{つか|れ}!。
\ruby{櫻色}{さくら|いろ}して
\ruby{居}{ゐ}た
\ruby{面}{かほ}は
\ruby{白}{しら}けて、
\ruby{葛}{くず}の
\ruby{葉裏}{は|うら}を
\ruby{見}{み}るやうになり、
\ruby{眼}{め}は
\ruby{冴}{さ}えなくなる、
\ruby{白髮}{しら|が}はさす、
\ruby{{\換字{強}}}{つよ}い
\ruby{奴}{やつ}は
\ruby{癇癪持}{かん|しやく|もち}になる。
\ruby{{\換字{弱}}}{よわ}い
\ruby{奴}{やつ}は
\ruby{萎縮漢}{いぢ|け|もの}になる。
\ruby{筋骨}{すぢ|ほね}は
\ruby{弛}{ゆる}んで
\ruby{仕舞}{し|ま}ふ、
\ruby{勞苦{\換字{嫌}}}{ほね|をり|ぎら}ひになる。
\ruby{其}{そ}の
\ruby{位}{くらゐ}なのは
\ruby{未}{ま}だ
\ruby{可}{い}い
\ruby{{\換字{分}}}{ぶん}で、
\ruby{隨{\換字{分}}}{ずゐ|ぶん}
\ruby{怖}{おそ}ろしい
\ruby{病氣}{びやう|き}さへも
\ruby{引出}{ひき|だ}す。
よしんば
\ruby{大望心}{たい|ま|う}と
\ruby{戀愛}{こ|ひ}とが
\ruby{衝突}{ぶつ|か}らないで、
\ruby{腹}{はら}の
\ruby{中}{なか}がそれほどには
\ruby{揉}{も}め
\ruby{無}{な}いでも、
\ruby{向}{むこ}ふに
\ruby{的}{まと}の
\ruby{無}{な}い
\ruby{戀}{こひ}は
\ruby{無}{な}いから、
\ruby{星}{ほし}に
\ruby{中}{あた}る
\ruby{中}{あた}らぬは
\ruby{時}{とき}の
\ruby{{\換字{運}}身}{うん|み}の
\ruby{{\換字{運}}}{うん}!。
\ruby{相手}{あひ|て}と
\ruby{馬}{うま}が
\ruby{合}{あ}ふ
\ruby{合}{あ}はぬもあるし、
\ruby{相手}{あひ|て}とは
\ruby{死}{し}ぬほどに
\ruby{好}{す}き
\ruby{合}{あ}つても、
\ruby{自{\換字{分}}}{じ|ぶん}たちばかりのために
\ruby{出來}{で|き}て
\ruby{居}{ゐ}る
\ruby{世界}{せ|かい}ぢやあ
\ruby{無}{な}いもの、
\ruby{何}{なに}がさて
\ruby{外{\換字{道}}}{げ|だう}も
\ruby{居}{ゐ}る、
\ruby{惡魔}{あく|ま}も
\ruby{居}{ゐ}る、
\ruby{敵}{てき}も
\ruby{居}{ゐ}る、おせつかいも
\ruby{居}{ゐ}る、
\ruby{義理}{ぎ|り}もある、
\ruby{人{\換字{情}}}{にん|じやう}もある、
\ruby{時}{とき}もある、
\ruby{塲合}{ば|あひ}もあつて、
\ruby{{\換字{随}}意}{ま|ゝ}ならぬ
\ruby{憂}{う}き
\ruby{世}{よ}を
\ruby{泣}{な}くものが
\ruby{多}{おほ}い。
\ruby{左樣}{さ|う}で
\ruby{無}{な}くつてさへ
\ruby{戀}{こひ}を
\ruby{知}{し}るなあ
\ruby{涙}{なみだ}を
\ruby{知}{し}る
\ruby{始}{はじめ}で、
\ruby{氣}{き}が
\ruby{優}{やさ}しくなる、
\ruby{脆}{もろ}くなる、
\ruby{感}{かん}じが
\ruby{早}{はや}くなる、
\ruby{深}{ふか}くなる、
\ruby{何}{なん}でも
\ruby{無}{な}い
\ruby{事}{こと}にハツと
\ruby{思}{おも}つたり、
\ruby{小}{ちひさ}な
\ruby{事}{こと}をくよ〳〵と
\ruby{案}{あん}じたり、
\ruby{{\換字{前}}表}{ぜん|ぺう}といふやうな
\ruby{事}{こと}を
\ruby{氣}{き}にしたり、
\ruby{何}{なに}かにつけて
\ruby{思}{おも}ひ
\ruby{{\換字{過}}}{すご}しを
\ruby{仕}{し}たり、
\ruby{寢}{ね}るべき
\ruby{時}{とき}に
\ruby{寢}{ね}られなかつたりする。
そこで
\ruby{段々}{だん|〴〵}
\ruby{心氣}{し|ん}が
\ruby{{\換字{弱}}}{よわ}る。
\ruby{心氣}{し|ん}が
\ruby{{\換字{弱}}}{よわ}りやあ
\ruby{愈々}{いよ|〳〵}
\ruby{氣}{き}が
\ruby{脆}{もろ}くなる、
\ruby{感}{かん}じが
\ruby{{\換字{強}}}{つよ}くなる。
\ruby{氣}{き}が
\ruby{脆}{もろ}く、
\ruby{感}{かん}じが
\ruby{{\換字{強}}}{つよ}くなりやあ
\ruby{{\換字{又}}}{また}
\ruby{心氣}{し|ん}が
\ruby{{\換字{弱}}}{よわ}る。
\ruby{雁齒鑢}{がん|ぎ|やすり}がかゝるやうなものだから
\ruby{堪}{たま}らう
\ruby{譯}{わけ}は
\ruby{無}{な}い。
\ruby{一日}{いち|にち}
\ruby{一日}{いち|にち}に
\ruby{{\換字{弱}}}{よわ}つた
\ruby{擧句}{あげ|く}は、
\ruby{魂魄}{たま|しひ}が
\ruby{薄手}{うす|で}になりきつて、
\ruby{觸}{さは}るものさへあれば
\ruby{砕}{くだ}けたがる
\ruby{玻璃}{びい|どろ}かなんぞのやうになつて
\ruby{仕舞}{し|ま}ふ。
よく
\ruby{世間}{せ|けん}にある
\ruby{戀路}{こひ|ぢ}の
\ruby{果}{はて}の、
\ruby{飛}{と}んでも
\ruby{無}{な}い
\ruby{不幸福}{ふ|しあ|わせ}は
\ruby{皆}{みな}
\ruby{其處}{そ|こ}で
\ruby{出來}{で|き}る。
たとひ
\ruby{{\換字{嫌}}}{きら}はれても
\ruby{{\換字{嫌}}}{きら}はれても、
\ruby{好}{す}かれたいのが
\ruby{戀}{こひ}の
\ruby{慾}{よく}で、また
\ruby{憂}{う}いも
\ruby{辛}{つら}いも
\ruby{堪忍}{しん|ばう}して
\ruby{添}{そ}ひ
\ruby{{\換字{遂}}}{と}げたいのが
\ruby{戀}{こひ}の
\ruby{意地}{い|ぢ}だ。
\換字{志}て
\ruby{見}{み}りやあ
\ruby{戀}{こひ}に
\ruby{生命}{いの|ち}の
\ruby{捨}{す}てやうは
\ruby{無}{な}い、
\ruby{戀}{こひ}は
\ruby{生々}{いき|〳〵}と
\ruby{美}{うつく}しいものだ。
\ruby{世}{よ}の
\ruby{不幸福}{ふ|しあ|わせ}な
\ruby{人}{ひと}を
\ruby{見}{み}りやあ、
\ruby{戀}{こひ}で
\ruby{死}{し}ぬものは
\ruby{一人}{ひと|り}も
\ruby{無}{な}く、
\ruby[<h||]{皆}{みんな}
\ruby{心氣}{し|ん}の
\ruby{疲勞}{つか|れ}に
\ruby{堪}{こら}へ
\ruby{切}{き}れ
\ruby{無}{な}くなつて、おのが
\ruby{魂魄}{たま|しひ}を
\ruby{碎}{くだ}いて
\ruby{仕舞}{し|ま}うのだが、
\ruby{{\換字{避}}}{さ}けやうにも
\ruby{{\換字{避}}}{さ}け
\ruby{難}{にく}いのは
\ruby{此}{こ}の
\ruby{{\換字{随}}{\換字{伴}}者}{お|と|も}だから、
\ruby{戀}{こひ}は
\ruby{毫末}{ちつ|と}も
\ruby{怖}{こは}かあ
\ruby{無}{な}いが、
\ruby{其}{そ}の
\ruby{{\換字{随}}{\換字{伴}}者}{お|と|も}の
\ruby{心氣}{し|ん}の
\ruby{疲勞}{つか|れ}は
\ruby{恐}{おそ}ろしい。
\ruby{實}{じつ}を
\ruby{云}{い}やあ
\ruby{僕}{ぼく}が
\ruby{君}{きみ}の
\ruby{事}{こと}を
\ruby{素破拔}{すつ|ぱ|ぬ}いて
\ruby{饒舌}{しや|べ}つたから、
\ruby{羽{\換字{勝}}}{は|がち}も
\ruby{日方}{ひ|かた}も
\ruby{君}{きみ}のために、
\ruby{二人}{ふ|たり}とも
\ruby{甚}{ひど}く
\ruby{心配}{しん|ぱい}して
\ruby{居}{ゐ}る。
\ruby{特}{こと}に
\ruby{日方}{ひ|かた}は
\ruby{彼}{あ}の
\ruby{氣性}{き|しやう}だから、
\ruby{{\換字{強}}}{きつ}い
\ruby{意見}{い|けん}を
\ruby{云}{い}ひに
\ruby{行}{ゆ}かうかも
\ruby{知}{し}れないが、
\ruby{乃公}{お|ら}あ
\ruby{何}{なんに}も
\ruby{意見}{い|けん}は
\ruby{云}{い}はない。
\ruby{何}{なん}も
\ruby{彼}{か}も
\ruby{解}{わか}つて
\ruby{居}{ゐ}る
\ruby{君}{きみ}の
\ruby{事}{こと}だもの、
\ruby{君}{きみ}が
\ruby{詰}{つま}まら
\ruby{無}{な}い
\ruby{事}{こと}を
\ruby{仕}{し}やう
\ruby{氣{\換字{遣}}}{き|づか}ひは
\ruby{無}{な}いが、たゞ
\ruby{心氣}{し|ん}の
\ruby{疲勞}{つか|れ}に
\ruby{負}{ま}けぬやうにと、これだけを
\ruby{君}{きみ}に
\ruby{頼}{たの}んで
\ruby{置}{お}く。
\ruby{見}{み}りやあ
\ruby{顏色}{かほ|つき}と
\ruby{云}{い}ひ
\ruby{容態}{よう|す}といひ、
\ruby{心氣}{し|ん}が
\ruby{疲}{つか}れて
\ruby{居}{ゐ}ないやうでも
\ruby{無}{な}い、
\ruby{氣}{き}をつけて
\ruby{吳}{く}れ
\ruby{無}{な}くちやあいけないぜ。
\ruby{何時}{い|つ}かは
\ruby{云}{い}はう〳〵と
\ruby{思}{おも}つて
\ruby{居}{ゐ}たので、つい
\ruby{圖}{づ}に
\ruby{乘}{の}つて
\ruby{長}{なが}く
\ruby{饒舌}{しや|べ}つて、
\ruby{言葉}{こと|ば}さへ
\ruby{亂暴}{らん|ばう}に
\ruby{言}{い}ひ
\ruby{{\換字{過}}}{す}ごしたが、
\ruby{意}{こゝろ}だけは
\ruby{是非}{ぜ|ひ}とも
\ruby{汲}{く}んで
\ruby{吳}{く}れたまへ。
\ruby{千言萬言}{せん|げん|ばん|げん}
\ruby{饒舌}{しや|べ}つても、
\ruby{身體}{から|だ}を
\ruby{大切}{たい|せつ}に
\ruby{仕}{し}て
\ruby{吳}{く}れろといふ、たゞの
\ruby{一句}{いつ|く}に
\ruby{止}{とゞ}まるのだ。
\ruby{{\換字{飯}}}{めし}の
\ruby{不味}{ま|づ}い
\ruby{時}{とき}も
\ruby{堪忍}{が|まん}して
\ruby{食}{く}つて、
\ruby{成}{な}るたけ
\ruby{精々}{せい|〴〵}
\ruby{身體}{から|だ}を
\ruby{使}{つか}つて、
\ruby{寢}{ね}るべき
\ruby{時}{とき}にやあ
\ruby{整然}{ちや|ん}と
\ruby{寢}{ね}て、
\ruby{力}{ちから}
\ruby{足}{あし}を
\ruby{踏}{ふ}んで
\ruby{確乎}{しつ|かり}と、
\ruby{快活}{き|さく}に
\ruby{日}{ひ}を
\ruby{{\換字{送}}}{おく}つて
\ruby{貰}{もら}ひたいのだ。
\ruby{君}{きみ}の
\ruby{氣}{き}に
\ruby{入}{い}つたほどの
\ruby{人}{ひと}だもの、
\ruby{僕}{ぼく}は
\ruby{其}{そ}の
\ruby{人}{ひと}を
\ruby{知}{し}らないが、
\ruby{屹度}{きつ|と}
\ruby{好}{い}い
\ruby{人}{ひと}だらうと
\ruby{思}{おも}つて
\ruby{居}{ゐ}て、
\ruby{君}{きみ}の
\ruby{{\換字{運}}命}{う|ん}の
\ruby{好}{い}いやうにとばかり
\ruby{祈}{いの}つて
\ruby{居}{ゐ}る。
\ruby{僕}{ぼく}の
\ruby{力}{ちから}の
\ruby{要}{い}る
\ruby{事}{こと}があらば、
\ruby{何}{なん}なりと
\ruby{{\換字{遠}}慮無}{ゑん|りよ|な}く
\ruby{云}{い}つて
\ruby{吳}{く}れたまへ、
\ruby{君}{きみ}のために
\ruby{幸福}{しあ|はせ}になる
\ruby{事}{こと}ならば、
\ruby{何樣}{ど|ん}な
\ruby{事}{こと}を
\ruby{仕}{し}ても
\ruby{僕}{ぼく}は
\ruby{厭}{いと}はない。
\ruby{馬}{うま}にでも
\ruby{牛}{うし}にでもなつて
\ruby{働}{はたら}かうが、
\ruby{其}{そ}の
\ruby{代}{かは}り
\ruby{今}{いま}
\ruby{言}{い}つた
\ruby{戀}{こひ}の
\ruby{{\換字{随}}{\換字{伴}}者}{お|と|も}にやあ
\ruby{必}{かなら}ず
\ruby{負}{ま}けて
\ruby{吳}{く}れたまうな。
\ruby{世界}{せ|かい}に
\ruby{人間}{ひ|と}は
\ruby{多}{おほ}いけれど、--- そりやあ
\ruby{偉}{えら}い
\ruby{人}{ひと}も
\ruby{多}{おほ}からうが、
\ruby{此}{こ}の
\ruby{何年}{なん|ねん}を
\ruby{{\換字{過}}}{す}ごして
\ruby{來}{き}た、
\ruby{君}{きみ}の
\ruby{行狀}{おこ|なひ}の
\ruby{殊{\換字{勝}}}{しゆ|しよう}さを
\ruby{見}{み}ては、アヽ、
\ruby{眞似}{ま|ね}たつて
\ruby{眞似}{ま|ね}られない
\ruby{事}{こと}だ、あゝいふ
\ruby{男}{をとこ}は
\ruby{今}{いま}の
\ruby{世}{よ}には、
\ruby{中々二人}{なか|〳〵|ふ|たり}とは
\ruby{有}{あ}りはすまい、
\ruby{島木萬五郎}{しま|き|まん|ご|らう}は
\ruby{俗物}{ぞく|ぶつ}だが、
\ruby{朋友}{とも|だち}にやあ
\ruby{幸福}{しあ|はせ}にも
\ruby{心}{こゝろ}の
\ruby{氣高}{け|だか}い
\ruby{水野}{みづ|の}のやうな
\ruby{人}{ひと}を
\ruby{持}{も}つて
\ruby{居}{ゐ}ると、
\ruby{天}{てん}にも
\ruby{地}{ち}にも
\ruby[<h||]{唯}{たつた}
\ruby{一人}{ひと|り}の
\ruby{大切}{たい|せつ}な
\ruby{朋友}{とも|だち}に
\ruby{思}{おも}つて
\ruby{居}{ゐ}る
\ruby{君}{きみ}の
\ruby{事}{こと}だから、どうか
\ruby{身體}{から|だ}を
\ruby{大切}{たい|せつ}に
\ruby{仕}{し}て
\ruby{吳}{く}れたまへ、
\ruby{君}{きみ}の
\ruby{其}{そ}の
\ruby{顏}{かほ}つきを
\ruby{見}{み}ちやあ
\ruby{案}{あん}じられてならない。
くどいやうだが
\ruby{今}{いま}
\ruby{言}{い}つた
\ruby{事}{こと}を
\ruby{能}{よ}く
\ruby{聽}{き}いて
\ruby{置}{お}いて
\ruby{吳}{く}れたまへ。
』

と、
\ruby{眞{\換字{情}}}{ま|ろ}こめて
\ruby{云}{い}ひ
\ruby{{\換字{終}}}{をは}りたり。

\Entry{其二十}

\ruby{磊落}{らい|らく}なれども
\ruby{思{\換字{遣}}}{おもひ|や}りあり、
\ruby{粗}{あら}きが
\ruby{如}{ごと}くなれども
\ruby{精細}{こま|か}なるところある
\ruby{島木}{しま|き}が
\ruby{長々}{なが|〳〵}しき
\ruby{物語}{もの|がたり}は、わざと
\ruby{我}{わ}が
\ruby{上}{うへ}には
\ruby{貼}{つ}かぬように
\ruby{云}{い}いたりとは
\ruby{聞}{きこ}えたれど、その
\ruby{言葉}{こと|ば}の
\ruby{中}{うち}の
\ruby{{\換字{節}}々}{ふし|〴〵}には、
\ruby{既}{はや}
\ruby{全然}{すつ|かり}と
\ruby{我}{わ}が
\ruby{{\換字{近}}來}{ちか|ごろ}の
\ruby{狀態}{あり|さま}を
\ruby{知}{し}り
\ruby{盡}{つく}くして
\ruby{言}{い}ふと
\ruby{思}{おぼ}しくて、ひし〳〵と
\ruby{身}{み}に
\ruby{徹}{こた}ふるところの
\ruby{少}{すくな}からぬに、
\ruby{氣息}{い|き}をさへ
\ruby{潜}{ひそ}めて% 【潛 u6f5b 「先先」】【潜 u6f5c 「夫夫」】併用されている
\ruby{聞}{き}き
\ruby{居}{ゐ}たりし
\ruby{水野}{みづ|の}は、
\ruby{胸}{むね}の
\ruby{中}{うち}は
\ruby{石川}{いし|かは}の
\ruby{淸}{きよ}き
\ruby{瀬}{せ}を
\ruby{流}{なが}るゝ
\ruby{水}{みづ}と
\ruby{爽快}{さわ|やか}にして、
\ruby{底}{そこ}の
\ruby{心}{こゝろ}は
\ruby{春}{はる}と
\ruby{溫}{あたゝか}き
\ruby{我}{わ}が
\ruby{友}{とも}が、
\ruby{虛僞}{いつ|はり}ならず
\ruby{我}{われ}を
\ruby{思}{おも}ひ
\ruby{吳}{く}るゝ
\ruby{其}{そ}の
\ruby{眞{\換字{情}}}{ま|ごゝろ}に、
\ruby{其}{それ}と
\ruby{指}{さ}しては
\ruby{捉}{とら}へ
\ruby{難}{がた}き
\ruby{香氣}{に|ほひ}の
\ruby{物}{もの}を
\ruby{罩}{こ}むるが
\ruby{如}{ごと}くに
\ruby{我}{わ}が
\ruby{身心}{しん|〴〵}の
\ruby{全部}{すべ|て}が
\ruby{引}{ひ}き
\ruby{包}{つゝ}まれたるを
\ruby{覺}{おぼ}えて、
\ruby{嗚呼}{あ|ゝ}
\ruby{我}{われ}
\ruby{不幸福}{ふ|しあ|はせ}の%「幸福」ここは「は」
\ruby{月日}{つき|ひ}の
\ruby{下}{した}に
\ruby{生}{うま}れて、
\ruby{物}{もの}の
\ruby{心}{こゝろ}も
\ruby{知}{し}らぬ
\ruby{頃}{ころ}より、
\ruby{{\換字{父}}}{ちゝ}をも
\ruby{母}{はゝ}をも
\ruby{失}{うしな}ひて、
\ruby{兄}{あに}も
\ruby{無}{な}ければ
\ruby{姊}{あね}も
\ruby{無}{な}く、
\ruby{世}{よ}の
\ruby{剩}{あま}され
\ruby{物}{もの}となつて
\ruby{生長}{そ|だ}ちしまゝ、
\ruby{幼}{おさな}き
\ruby{時}{とき}の
\ruby{心}{こゝろ}にも、
\ruby{丁稚奉公}{でつ|ち|ぼう|こう}せし
\ruby{家}{いへ}に、
\ruby{巢}{す}くひし
\ruby{燕}{つばめ}の
\ruby{親鳥}{おや|どり}の、
\ruby{日}{ひ}に
\ruby{百度}{もゝ|たび}も
\ruby{千度}{ち|たび}も
\ruby{飛}{と}んで
\ruby{去}{さ}つては
\ruby{飛}{と}んで
\ruby{{\換字{返}}}{かへ}つてまだ
\ruby{{\換字{弱}}}{よわ}き
\ruby{雛}{ひな}に
\ruby{餌}{ゑ}を
\ruby{{\換字{運}}}{はこ}ぶを
\ruby{見}{み}て、
\ruby{顏}{かほ}もおぼえぬ
\ruby{吾}{わ}が
\ruby{母}{はゝ}
\ruby{戀}{こひ}しく、
\ruby{親}{おや}のある
\ruby{子}{こ}の
\ruby{羨}{うらや}ましさに、\換字{志}く〳〵
\ruby{泣}{な}いたる
\ruby{事}{こと}の
\ruby{記臆}{おぼ|ゑ}さへ、まざ〳〵と
\ruby{今}{いま}に
\ruby{{\換字{遺}}}{のこ}れるなるが、それには
\ruby{引換}{ひき|か}へて
\ruby{幸{\換字{運}}}{しあ|はせ}にも、アヽ%「幸運」ここは「は」
\ruby{我}{われ}
\ruby{何}{なん}の
\ruby{福}{ふく}のあつてか、
\ruby{自然}{し|ぜん}
\g詰めruby{々々}{〳〵}に
\ruby{知}{し}り
\ruby{合}{あ}つたる
\ruby{六人}{ろく|にん}の
\ruby{良}{よ}き
\ruby{友}{とも}の
\ruby{其}{そ}の
\ruby{中}{うち}にも、
\ruby{{\換字{分}}}{わ}けて
\ruby{親}{した}しき
\ruby{羽{\換字{勝}}}{は|がち}
\ruby{島木}{しま|き}、
\ruby{特}{こと}に
\ruby{島木}{しま|き}が
\ruby{眼}{ま}の
\ruby{{\換字{前}}}{あたり}の
\ruby{友{\換字{情}}}{なさ|け}!。
   お
\ruby{澤}{さは}
\ruby{婆}{ばゞあ}の
\ruby{言葉}{こと|ば}の
\ruby{{\換字{通}}}{とほ}り、
\ruby{手}{て}をついて
\ruby{頼}{たの}んだつて
\ruby{芋塊}{い|も}
\ruby{一}{ひと}つも、
\ruby{自然}{ひと|りで}には
\ruby{出}{で}て
\ruby{來}{こ}ない
\ruby{此}{こ}の
\ruby{世}{よ}の
\ruby{中}{なか}に、いづれ
\ruby{身}{み}の
\ruby{油汗}{あぶら|あせ}が
\ruby{化}{ば}けたに
\ruby{{\換字{違}}}{ちが}ひ
\ruby{無}{な}い
\ruby{多額}{おほ|く}の
\ruby{金子}{か|ね}をも、
\ruby{紙}{かみ}の
\ruby{一枚}{いち|まい}でも
\ruby{吳}{く}れるやうに、
\ruby{惜}{をし}む
\ruby{色}{いろ}さへ
\ruby{無}{な}く
\ruby{快}{こゝろよ}く
\ruby{吳}{く}れて、\換字{志}かも
\ruby{君}{きみ}のためになる
\ruby{事}{こと}ならば、
\ruby{馬}{うま}にでも
\ruby{牛}{うし}にでもなつて
\ruby{働}{はたら}いて
\ruby{{\換字{遣}}}{や}らうと、
\ruby{身}{み}を
\ruby{入}{い}れて
\ruby{吳}{く}れる
\ruby{其}{そ}の
\ruby{俠氣}{をとこ|ぎ}!。
\ruby{人世}{うき|よ}の
\ruby{場数}{ば|かず}を% 原文通り「場」
\ruby{踏}{ふ}んで
\ruby{來}{き}た
\ruby{人}{ひと}には、
\ruby{隨{\換字{分}}}{ずゐ|ぶん}
\ruby{幼稚}{こ|ども}にも
\ruby{{\換字{若}}輩}{じやく|はい}にも
\ruby{思}{おも}はれようか
\ruby{知}{し}れぬ
\ruby{事}{こと}なるに、
\ruby{我}{わ}が
\ruby{{\換字{情}}緖}{おも|ひ}の
\ruby{上}{うへ}に
\ruby{就}{つ}いては
\ruby{咎}{とが}め
\ruby{立}{だ}てもせず、
\ruby{年齡}{と|し}の
\ruby{{\換字{所}}爲}{せ|ゐ}にして
\ruby{仕舞}{し|ま}つて
\ruby{一}{ひ}ト
\ruby{言}{こと}も
\ruby{云}{い}はぬ
\ruby{寛大}{おほ|やう}さ!。
たゞ
\ruby{身體}{から|だ}を
\ruby{大切}{だい|じ}に
\ruby{仕}{し}て
\ruby{吳}{く}れろと
\ruby{云}{い}つて
\ruby{吳}{く}れる
\ruby{其}{そ}の
\ruby{親切}{しん|せつ}!。
\ruby{嗚呼}{あ|ゝ}
\ruby{兄}{あに}と
\ruby{云}{い}はうか、
\ruby{姊}{あね}と
\ruby{云}{い}はうか、
\ruby{兄}{あに}も
\ruby{姊}{あね}も
\ruby{中々}{なか|〳〵}かうばかりはあるまい。
まして
\ruby{朋友}{とも|だち}と
\ruby{云}{い}はうには
\ruby{勿體無}{もつ|たい|な}いほど。
\ruby{人}{ひと}に
\ruby{云}{い}はれぬ
\ruby{苦悶}{く|るし}みを
\ruby{抱}{いだ}けば、
\ruby{何}{なに}につけ
\ruby{彼}{か}につけて
\ruby{此}{こ}の
\ruby{世}{よ}の
\ruby{中}{なか}を、
\ruby{味氣無}{あぢ|き|な}く
\ruby{思}{おも}ふ
\ruby{時}{とき}のみ
\ruby{此頃}{この|ごろ}は
\ruby{多}{おほ}かりしが、あゝ
\ruby{有}{あ}り
\ruby{難}{がた}き
\ruby{天}{てん}の
\ruby{恩惠}{めぐ|み}、
\ruby{水野靜十郎}{みづ|の|せい|じう|らう}
\ruby{幸福}{さい|はひ}にして、かゝる%「幸福」ここは「は」
\ruby{信義}{しん|ぎ}の
\ruby{友}{とも}にも
\ruby{未}{ま}だ
\ruby{棄}{す}てられねば、アヽ
\ruby{思}{おも}へば
\ruby{我}{われ}は
\ruby{世}{よ}にも
\ruby{稀}{まれ}なる
\ruby{幸{\換字{運}}}{しあ|はせ}を%「幸運」ここは「は」
\ruby{受}{う}け
\ruby{得}{え}たる
\ruby{身}{み}なるかな
\ruby{我}{わ}が
\ruby{行末}{ゆく|すゑ}も
\ruby{光}{ひかり}ありて、
\ruby{{\換字{強}}}{あなが}ち
\ruby{黑闇}{や|み}のみならず
\ruby{見}{み}ゆ、と
\ruby{悅}{よろこ}ぶにも
\ruby{先}{ま}づ
\ruby{涙}{なみだ}にて、
\ruby{謝}{しや}する
\ruby{言葉}{こと|ば}もたど〳〵しく、

『アヽ
\ruby{島木君}{しま|き|くん}、
\ruby{感謝}{かん|しや}する。
\ruby{免}{ゆる}して
\ruby{吳}{く}れたまへ、
\ruby{僕}{ぼく}は
\ruby{何}{なん}にも
\ruby{言}{い}ふことが
\ruby{出來無}{で|き|な}い。
\ruby{言}{い}ひたい
\ruby{{\換字{情}}懷}{こゝろ|もち}は
\ruby{澤山}{たん|と}あるが
\ruby{胸}{むね}が
\ruby{張}{は}つて
\ruby{居}{ゐ}て
\ruby{何}{なん}にも
\ruby{言}{い}へない。
\ruby{實}{じつ}に
\g詰めruby{々々}{〳〵}
\ruby{君}{きみ}の
\ruby{親切}{しん|せつ}は
\ruby{深}{ふか}く
\ruby{謝}{しや}する。
\ruby{君}{きみ}の
\ruby{談}{はなし}は
\ruby{骨}{ほね}に
\ruby{浸}{し}みて
\ruby{解}{わか}つた。
\ruby{決}{けつ}して
\ruby{忘}{わす}れ
\ruby{無}{な}い、
\ruby{決}{けつ}して
\ruby{忘}{わす}れ
\ruby{無}{な}い!。
\ruby{成程}{なる|ほど}
\ruby{何}{なん}に
\ruby{{\換字{巻}}}{ま}き
\ruby{倒}{たふ}されては
\ruby{濟}{す}まない
\ruby{身體}{から|だ}だ!。
\ruby{僕}{ぼく}も
\ruby{果敢}{は|か}ない
\ruby{思}{おもひ}に
\ruby{死}{し}にたかあ
\ruby{無}{な}い!。
いや
\ruby{僕}{ぼく}は
\ruby{何樣}{ど|う}まかり
\ruby{間{\換字{違}}}{ま|ちが}つても
\ruby{脆}{もろ}くは
\ruby{死}{し}なゝい!。
\ruby{戀{\換字{情}}}{じよ|う}は
\ruby{戀{\換字{情}}}{じよ|う}だけれど、
\ruby{大望心}{たい|ま|う}は
\ruby{大望心}{たい|ま|う}だ!。
\ruby{身體}{から|だ}も
\ruby{必}{かなら}ず
\ruby{大切}{たい|せつ}にする。
』

と、
\ruby{{\換字{強}}}{しひ}て
\ruby{勉}{つと}めて
\ruby{答}{こた}へたり。

\ruby{夜}{よ}は
\ruby{彼}{かれ}
\ruby{一句}{いつ|く}
\ruby{此一句}{これ|いつ|く}の
\ruby{二人}{ふた|り}が
\ruby{親}{した}しき
\ruby{物語}{もの|がたり}に
\ruby{漸}{やうや}く
\ruby{盡}{つ}きて、
\ruby{早}{はや}くも
\ruby{暁天}{あ|け}
\ruby{{\換字{近}}}{ちか}くならんとすれば、
\ruby{水野}{みづ|の}は
\ruby{{\換字{終}}}{つひ}に
\ruby{島木}{しま|き}が
\ruby{許}{もと}を
\ruby{辭}{じ}して、
\ruby{{\換字{情}}中}{ふと|ころ}に
\ruby{阿堵物}{も||の}あるに% 「阿堵物(あとぶつ)」お金のこと
\ruby{勢}{いきほ}ひ
\ruby{好}{よ}く、
\ruby{紫色}{むら|さき}
\ruby{立}{だ}てる
\ruby{天}{そら}の
\ruby{星薄}{ほし|うす}れ
\ruby{行}{ゆ}きて
\ruby{{\換字{朝}}風}{あさ|かぜ}の
\ruby{徐々}{おも|むろ}に
\ruby{吹}{ふ}き
\ruby{出}{だ}す
\ruby{頃}{ころ}、
\ruby{相良}{さが|ら}が
\ruby{家}{いへ}を
\ruby{敲}{たゝ}き
\ruby{起}{おこ}して
\ruby{昨日}{きの|ふ}の
\ruby{恩}{おん}を
\ruby{謝}{しや}し、
\ruby{{\換字{猶}}}{なほ}
\ruby{信頼}{た|の}むに
\ruby{足}{た}るべき
\ruby{看護{\換字{婦}}}{かん|ご|ふ}を
\ruby{世話}{せ|わ}せんことを
\ruby{乞}{こ}ひ
\ruby{求}{もと}めて、
\ruby{其}{そ}の
\ruby{快}{こゝろよ}く
\ruby{諾}{うけが}ひ
\ruby{吳}{く}れたるに
\ruby{心勇}{こゝろ|いさ}み、
\ruby{足輕}{あし|かろ}く
\ruby{歸路}{かへ|り}を
\ruby{急}{いそ}ぎて、
\ruby{淺草}{あさ|くさ}の
\ruby{雷神門{\換字{前}}}{かみ|なり|もん|まへ}にさしかゝりぬ。


\Entry{其二十一}

おもふ
\ruby{人}{ひと}の
\ruby{病}{やまひ}は
\ruby{篤}{あつ}けれども、
\ruby{思}{おも}ひし
\ruby{事}{こと}は
\ruby{皆爲}{みな|な}し
\ruby{得}{え}たり、
\ruby{相良}{さが|ら}も
\ruby{今一度}{いま|いち|ど}
\ruby{見舞}{み|ま}ひて
\ruby{尾竹}{お|たけ}にあひて
\ruby{種々}{くさ|〴〵}の
\ruby{心添}{こヽろ|ぞへ}をもなし
\ruby{置}{お}かんと
\ruby{云}{い}ひ、
\ruby{良}{よ}き
\ruby{看護婦}{かん|ご|ふ}をも
\ruby{晝}{ひる}までとは
\ruby{{\GWI{u904e-k}}}{すご}さず
\ruby{四}{よ}つ
\ruby{木}{ぎ}に
\ruby{{\GWI{u9063-k}}}{や}り
\ruby{{\換字{呉}}}{く}るゝ
\ruby{手筈}{て|はず}に
\ruby{定}{さだ}まりたり、この
\ruby{上}{うへ}はたゞ
\ruby{健}{まめ}やかなる
\ruby[g]{婢一人}{をんなひとり}を
\ruby{看護婦}{かん|ご|ふ}の
\ruby[g]{指揮}{さしず}の
\ruby{下}{しも}につけて
\ruby{雑事}{ざつ|じ}に
\ruby{當}{あた}らすれば、もとより
\ruby{介抱}{かい|はう}の
\ruby{此上無}{この|うへ|な}く
\ruby{行届}{ゆき|とゞ}きて
\ruby{善}{ぜん}を
\ruby{盡}{つ}くしたりと
\ruby{云}{い}うべきにはあらねど、
\ruby{今}{いま}の
\ruby{身}{み}にての
\ruby{我}{わ}が
\ruby{心}{こヽろ}の
\ruby{及}{およ}ぶほどだけは
\ruby{盡}{つ}くしたるなり、と
\ruby{思}{おも}ふにつけて
\ruby{人知}{ひと|し}らず
\ruby{樂}{たの}しく、
\ruby{愁}{うれひ}の
\ruby{中}{なか}にも
\ruby{幽}{かすか}なる
\ruby{笑}{ゑみ}の
\ruby{催}{もよほ}さるゝ
\ruby[g]{心地}{こゝち}して、
\ruby{願}{ねが}はくは
\ruby{我}{わ}が
\ruby{五十子}{い|そ|こ}の
\ruby{病}{やまひ}の
\ruby{漸}{やうや}く
\ruby{痊}{おこた}りて、
\ruby{心盡}{こヽろ|づく}しの
\ruby{甲斐}{か|ひ}もあれかし、
\ruby{暴}{あら}き
\ruby{雨風}{あめ|かぜ}に
\ruby{根}{ね}を
\ruby{{\換字{揺}}}{ゆら}がされて
\ruby{敢無}{あへ|な}くも
\ruby[g]{天壽}{いのち}ならず
\ruby{枯}{か}れんとする
\ruby{樹}{き}を、おぼつか
\ruby{無}{な}きながら
\ruby{{\換字{支}}}{さゝ}へ
\ruby{培}{つちか}ひて、
\ruby{復}{ふたゝ}び
\ruby{花{\換字{咲}}}{はな|さ}く
\ruby{春}{はる}の
\ruby{曉}{あした}に、
\ruby{丹誠}{たん|せい}の
\ruby{甲斐}{か|ひ}ありて
\ruby{美}{うつく}しく
\ruby{日}{ひ}に
\ruby{匂}{にほ}ふを
\ruby{見}{み}ば、
\ruby{如何}{い|か}ばかりか
\ruby{心}{こヽろ}の
\ruby{嬉}{うれ}しからん、それにつけても
\ruby{昨日}{きの|ふ}よりの
\ruby{長}{なが}き
\ruby{夜一夜}{よ|ひと|よ}を、
\ruby{我}{わ}が
\ruby{五十子}{い|そ|こ}は
\ruby{如何}{い|か}なる
\ruby{狀態}{やう|す}に
\ruby{{\GWI{u9001-k}}}{おく}りたらん、
\ruby{熱}{ねつ}の
\ruby{烈}{はげ}しく
\ruby{{\GWI{u9032-k}}}{さ}すことは
\ruby{無}{な}かりしか、
\ruby{{\換字{強}}}{つよ}く
\ruby{苦}{くるし}む
\ruby{事}{こと}は
\ruby{無}{な}かりしか、ともすれば
\ruby{心臓肺臓}{しん|ざう|はい|ざう}の
\ruby{此}{こ}の
\ruby{病}{やまひ}には
\ruby{惡}{あし}くなるものと
\ruby{聞}{き}きたるが
\ruby[g]{其等}{それら}の
\ruby{凶}{あし}きことは
\ruby{無}{な}かりし
\ruby{歟}{か}、
\ruby{尾竹}{を|たけ}も
\ruby{親切}{しん|せつ}の
\ruby{男}{をとこ}なれば、
\ruby{容態惡}{よう|だい|あし}くば
\ruby{附}{つ}きゝりに
\ruby{附}{つ}きても
\ruby{居}{ゐ}ては
\ruby{{\換字{呉}}}{く}れたるべけれど、
\ruby{氷}{こほり}より
\ruby{冷}{つめた}い
\ruby{心}{こヽろ}の
\ruby{彼}{あ}のお
\ruby{澤婆}{さは|ばゞ}、くれ〴〵も
\ruby{頼}{たの}み
\ruby{置}{お}きたる
\ruby{氷嚢}{ひよう|なう}の
\ruby{世話}{せ|わ}さへ、
\ruby{既}{すで}に
\ruby[g]{一昨日}{をとゝひ}といひ
\ruby[g]{昨日}{きのふ}と云ひ、
\ruby{碌}{ろく}に
\ruby{身}{み}に
\ruby{染}{し}みても
\ruby{爲}{し}て
\ruby{{\換字{呉}}}{く}れざりし、あゝいふ
\ruby{不幸}{ふし|あわせ}の
\ruby{處}{ところ}に
\ruby{居合}{ゐ|あ}はせたる
\ruby{病人}{びやう|にん}の、
\ruby{思}{おも}へば
\ruby[g]{一夜}{ひとよ}が
\ruby{氣遣}{き|づか}はるゝ、と
\ruby{偶然}{ふ|と}
\ruby[g]{思念}{おもひ}の
\ruby{其處}{そ|こ}に
\ruby{片荷}{かた|に}づゝては
\ruby{矢}{や}も
\ruby{楯}{たて}も
\ruby{堪}{たま}らず、
\ruby{物淋}{もの|さび}しく
\ruby{薄暗}{うす|くら}き
\ruby{離}{はな}れ
\ruby{屋}{や}の
\ruby{中}{うち}の、
\ruby{孤燈力無}{こ|とう|ちから|な}く
\ruby{照}{て}らす
\ruby{光}{ひかり}の
\ruby{下}{もと}に、
\ruby{頭髪}{か|み}は
\ruby{亂菊}{らん|ぎく}の
\ruby{花瓣}{はな|びら}の
\ruby{霜}{しも}に
\ruby{傷}{いた}める
\ruby{姿}{すがた}と
\ruby{崩}{くづ}れて、
\ruby{悶}{もだ}え
\ruby{悶}{もだ}えつゝ
\ruby{埒無}{らち|な}く
\ruby{病}{や}み
\ruby{臥}{ふ}せる
\ruby{態}{さま}の、
\ruby{眼}{め}の
\ruby{前}{まへ}にあり〳〵と
\ruby{{\GWI{u6d6e-k}}}{うか}み
\ruby{來}{く}るやう
\ruby{覺}{おぼ}えて、
\ruby{島木}{しま|き}が
\ruby{寓}{やど}を
\ruby{敲}{たゝ}きたりし
\ruby{折}{をり}、
\ruby{頭}{かうべ}を
\ruby{反}{かへ}して、
\ruby{偶然見}{ふ|と|み}し
\ruby{北}{きた}の
\ruby{空}{そら}に、
\ruby{大}{おほき}なる
\ruby{美}{うつく}しき
\ruby{星}{ほし}の
\ruby{長々}{なが|〳〵}と
\ruby{光}{ひかり}を
\ruby{曳}{ひ}いて
\ruby{流}{なが}れて
\ruby{{\GWI{u6d88-k}}}{き}えしも、
\ruby{思}{おも}ひ
\ruby{合}{あは}されて
\ruby{今更急}{いま|さら|きふ}に
\ruby{何}{なん}と
\ruby{無}{な}く
\ruby{忌}{いま}はしく、おもはず
\ruby{慄然}{りつ|ぜん}として
\ruby{天}{てん}を
\ruby{偸}{ぬす}み
\ruby{見}{み}たり。

\ruby{天}{そら}は
\ruby{今白}{いま|しら}みわたりて
\ruby{靜}{しづか}に、
\ruby{星辰}{ほ|し}は
\ruby{潛}{ひそ}みつ、
\ruby{瑠璃}{る|り}の
\ruby{盤上}{ばん|じやう}に
\ruby{金砂}{きん|しや}を
\ruby{撒}{ま}きし
\ruby{數時間前}{すう|じ|かん|まへ}の
\ruby{光景}{あり|さま}は
\ruby{痕}{あと}も
\ruby{無}{な}く
\ruby{{\GWI{u6d88-k}}}{き}え
\ruby{去}{さ}つて、またありしおもかげを
\ruby{忍}{しの}ぶべくもあらぬ
\ruby{狀}{さま}なるに、おのづと
\ruby{新}{あたら}しき
\ruby[g]{{\GWI{u6df8-jv}}旦}{あした}の
\ruby{氣}{き}を
\ruby{受}{う}けて
\ruby{胸}{むね}も
\ruby{開}{ひら}き、アゝ
\ruby{前表}{ぜん|ぺう}といふやうなる
\ruby{事}{こと}を
\ruby{氣}{き}に
\ruby{仕}{し}たる
\ruby{愚}{おろか}さ、
\ruby{島木}{しま|き}の
\ruby{言葉}{こと|ば}にも
\ruby{羞}{はづ}か\GWI{koseki-900370}かりし、と
\ruby{私}{ひそか}に
\ruby{自}{みづか}ら
\ruby{女々}{め|ゝ}しきを
\ruby{慚}{は}ぢたり。
されど
\ruby{心}{こヽろ}は
\ruby{一度}{ひと|たび}
\ruby{動}{うご}きて
\ruby{復}{また}
\ruby{安}{やす}まらず。
\ruby{曉}{あした}に
\ruby{{\GWI{u6d88-k}}}{き}えし
\ruby{星}{ほし}は
\ruby{再度}{ふた|たび}
\ruby{夕}{ %全角空白
ゆふべ}に
\ruby{見}{み}るべけれども、
\ruby{一度}{ひと|たび}
\ruby{去}{さ}つては
\ruby{行}{ゆ}く
\ruby{方}{かた}
\ruby{知}{し}れぬ
\ruby{人}{ひと}の
\ruby{身}{み}の、
\ruby{死生}{し|せい}の
\ruby{抑々}{そも|そも}
\ruby{何}{なに}に
\ruby{繫}{かヽ}りて、
\ruby{禍福}{か|ふく}の
\ruby{將{\換字{叉}}}{はた|また}
\ruby{何}{なに}に
\ruby{本}{もと}づくかも
\ruby{分}{わか}からぬ
\ruby{茫々}{ばう|〳〵}たる
\ruby{劫{\GWI{u904b-k}}}{ごふ|うん}の
\ruby{測}{はか}り
\ruby{難}{がた}く
\ruby{窺}{うかゞ}ひ
\ruby{難}{がた}きに
\ruby{思}{おも}ひ
\ruby{到}{いた}りては、あゝ
\ruby{頼}{たの}まれぬ
\ruby{人}{ひと}の
\ruby{世}{よ}なるかな、
\ruby{我}{わ}が
\ruby{心}{こヽろ}の
\ruby{膏}{あぶら}を
\ruby{燃}{も}やし、
\ruby{骨}{ほね}の
\ruby{髄}{ずゐ}を
\ruby{焚}{た}きて、
\ruby[g]{願望}{ねがひ}は
\ruby{大}{おほい}ならぬ
\ruby{我}{わ}が
\ruby{身}{み}の
\ruby[g]{周圍}{まはり}に、
\ruby{聊}{いさゝ}かの
\ruby[g]{光明}{ひかり}を
\ruby{得}{え}んと
\ruby{願}{ねが}ふも、
\ruby{{\GWI{u904b-k}}命}{うん|めい}の
\ruby{風}{かぜ}の
\ruby{容赦無}{よう|しや|な}く
\ruby{吹}{ふ}き
\ruby{荒}{すさ}まんには、
\ruby{頼}{たの}む
\ruby{影}{かげ}なき
\ruby{裸火}{はだ|かび}の、
\ruby{脆}{もろ}くも
\ruby{忽}{たちま}ち
\ruby{吹}{ふ}き
\ruby{滅}{け}されて、
\ruby{天地}{てん|ち}は
\ruby{{\換字{情}}無}{なさ|けな}き
\ruby{闇}{やみ}となるべし。
おもへば
\ruby{小}{ちひさ}きは
\ruby{人}{ひと}の
\ruby{力}{ちから}なり。
かほどに
\ruby{身}{み}を
\ruby{勞}{つか}らせ
\ruby{心}{こヽろ}を
\ruby{盡}{つく}して、
\ruby{我}{わ}が
\ruby{思}{おも}ふ
\ruby{人}{ひと}
\ruby{好}{よ}かれと
\ruby{我}{われ}は
\ruby{願}{ねが}へど、
\ruby{慈悲有}{なさ|け|あ}りや
\ruby{無}{な}しやもおぼつかなき、
\ruby{{\GWI{u904b-k}}命}{うん|めい}といふものゝ
\ruby{意任}{こヽろ|まか}せ!、
\ruby{其}{そ}の
\ruby{心}{こヽろ}が
\ruby[g]{人{\換字{情}}}{なさけ}を
\ruby{知}{し}つて
\ruby{{\換字{呉}}}{く}れうでも
\ruby{無}{な}ければ、
\ruby{思}{おも}へば〳〵
\ruby{悲}{かな}しきは
\ruby{人}{ひと}の
\ruby{世}{よ}!。
\ruby[g]{平生}{ひごろ}は
\ruby{天{\GWI{u7fd4-k}}}{そら|か}ける
\ruby{事}{こと}も
\ruby{爲}{な}さば
\ruby{爲}{な}すべき
\ruby{雄心持}{をご|ヽろ|も}ちし
\ruby{我}{われ}なりしが、
\ruby{身}{み}に
\ruby{染}{し}みて
\ruby{今}{いま}ぞ
\ruby{人間}{にん|げん}の
\ruby{甲斐無}{か|ひ|な}きを
\ruby{知}{し}りつる!。
\ruby{天}{てん}は
\ruby{限}{かぎ}り
\ruby{無}{な}く
\ruby{大}{おほい}なるに、
\ruby{我}{われ}は
\ruby{糠星}{ぬか|ぼし}の
\ruby{其}{それ}より
\ruby{微}{かす}けく、
\ruby{地}{ち}は
\ruby{涯}{はて}も
\ruby{無}{な}く
\ruby{廣}{ひろ}やかなるに、
\ruby{身}{み}は
\ruby{塵土}{ちり|ひぢ}と
\ruby{小}{ちひさ}なる、
\ruby{此}{こ}の
\ruby{某甲}{なに|がし}が
\ruby{{\換字{懐}}}{いだ}ける
\ruby{念}{おもひ}の、
\ruby{{\GWI{u904b-k}}命}{うん|めい}に
\ruby{對}{むか}へる
\ruby{其}{そ}の
\ruby{眞態}{あり|さま}は、
\ruby{譬}{たと}へば
\ruby{一縷}{いち|る}の
\ruby{細}{ほそ}き〳〵、
\ruby{毛}{け}の
\ruby{如}{ごと}く
\ruby{蜘蛛}{く|も}の
\ruby{圍}{い}のごとき
\ruby{絲}{いと}を、
\ruby{千萬馬力}{せん|まん|ば|りき}もて
\ruby{轟}{とゞろ}き
\ruby{{\換字{廻}}}{まは}れる
\ruby{大車輪}{だい|しや|りん}に
\ruby{繫}{か}けて、
\ruby{其}{そ}の
\ruby{車輪}{しや|りん}の
\ruby{我}{わ}が
\ruby{願}{ねが}ふ
\ruby{方}{かた}に
\ruby{{\換字{廻}}}{まは}らんことを、
\ruby{竊}{ひそか}に
\ruby{願}{ねが}ひ
\ruby{求}{もと}むるが
\ruby{如}{ごと}し。
\ruby{嗚呼}{あ|ゝ}、
\ruby{我}{わ}が
\ruby{願}{ねが}ひの
\ruby{聽}{き}かるべきや?。
\ruby{心細}{こヽろ|ぼそ}くもまた
\ruby{心細}{こヽろ|ぼそ}くて、
\ruby{{\換字{情}}無}{なさけ|な}くも
\ruby{物}{もの}のみの
\ruby{思}{おも}はるゝ
\ruby{世}{よ}かな!。
\ruby{我}{わ}が
\ruby{智慧}{ち|ゑ}の
\ruby{今効無}{いま|かひ|な}きを
\ruby{知}{し}り、
\ruby{我}{わ}が
\ruby[g]{意念}{おもひ}の
\ruby{今孱{\換字{弱}}}{いま|か|よわ}きを
\ruby{知}{し}り、
\ruby{斷}{た}えぬ
\ruby{泉}{いづみ}と
\ruby{湧}{わ}き
\ruby{上}{あが}る
\ruby{戀}{こひ}の
\ruby{誠}{まこと}に
\ruby{洗}{あら}はれて、
\ruby{心}{こヽろ}は
\ruby{無垢}{む|く}の
\ruby[g]{往時}{むかし}に
\ruby{{\GWI{u8fd4-k}}}{かへ}りぬ。
アゝ
\ruby{今我}{いま|われ}は
\ruby[g]{嬰兒}{みどりご}なり!。
\ruby[g]{天地}{てんち}の
\ruby[g]{那處}{いづく}に
\ruby{慈母}{は|ヽ}の
\ruby{御坐}{お|は}す\GWI{u2048}。
\ruby{泣}{な}きて
\ruby{呼}{よ}び
\ruby{度}{た}き
\ruby[g]{心地}{こゝち}ぞする。
と
\ruby{曉天}{あか|つき}の
\ruby{{\GWI{u7336-k}}}{なほ}
\ruby[g]{靜寂}{しづか}にして
\ruby{人}{ひと}の
\ruby{{\GWI{u901a-k}}}{とほ}りも
\ruby[g]{稀少}{まばら}なるに、
\ruby{深}{ふか}くも
\ruby{心}{こヽろ}の
\ruby{奥}{おく}に
\ruby{思}{おも}ひ
\ruby{入}{い}つたる
\ruby[g]{水野}{みづの}は、ふつと
\ruby{我}{われ}に
\ruby{{\GWI{u8fd4-k}}}{かへ}つて
\ruby{頭}{かうべ}を
\ruby{擡}{あ}ぐれば、
\ruby{身}{み}は
\ruby{何時}{い|つ}の
\ruby{程}{ほど}にか
\ruby{來}{きた}りけん、
\ruby{塵埃無}{ち|り|な}き
\ruby{{\換字{朝}}}{あした}の
\ruby{露}{つゆ}けき
\ruby{石路}{せき|ろ}の、
\ruby{長々}{なが|〳〵}しきを
\ruby{知}{し}らぬ
\ruby{間}{ま}に
\ruby{{\GWI{u904e-k}}}{す}ぎて、
\ruby{今}{いま}や
\ruby{淺草寺}{せん|さう|じ}の
\ruby{山門}{さん|もん}を、
\ruby{既}{すで}に
\ruby{{\換字{半}}}{なかば}は
\ruby{潛}{くゞ}り
\ruby{居}{ゐ}たり。

\ruby{晝間}{ひ|る}は
\ruby{賑}{にぎ}やかなる
\ruby{中店}{なか|みせ}も、
\ruby{{\GWI{u7336-k}}}{なほ}
\ruby{寂々}{じやく|〳〵}として
\ruby{物}{もの}の
\ruby{響}{ひゞき}を
\ruby{傳}{つた}へず、
\ruby{御{\換字{扉}}}{みと|びら}を
\ruby{今開}{いま|ひら}きしばかりの、
\ruby{御堂}{み|だう}の
\ruby{内}{うち}は
\ruby{仄暗}{ほの|ぐら}きに、
\ruby{御燈明}{み|あか|し}の
\ruby{煌々}{きら|〳〵}と
\ruby{黄金色}{こ|がね|いろ}に
\ruby{見}{み}えて、
\ruby{{\換字{朝}}勤}{あさ|づと}めの
\ruby{讀經}{どき|やう}の
\ruby{聲}{こゑ}は
\ruby{殊{\換字{勝}}}{しゆ|しよう}に
\ruby{澄}{す}み
\ruby{渡}{わた}り、
\ruby{御堂}{み|だう}の
\ruby{甍}{いらか}は
\ruby{天}{そら}に
\ruby{聳}{そび}えて、そこ
\ruby{此處}{こ|ゝ}に
\ruby{立}{た}てる
\ruby[g]{老樹}{おいき}の
\ruby{銀杏}{い|てふ}は、まだ
\ruby{下}{お}り
\ruby{立}{た}たぬ
\ruby{鳩雞}{はと|とり}を
\ruby{宿}{やど}して、
\ruby{睡}{ねむ}れるが
\ruby{如}{ごと}く
\ruby{靜}{しづか}かに
\ruby{秋}{あき}の
\ruby{曙}{あした}の
\ruby{色}{いろ}を
\ruby{見}{み}せたり。

\ruby[g]{水野}{みづの}はあはれにも
\ruby{頭}{かうべ}を
\ruby{下}{さ}げて、かつて
\ruby{拜}{をが}みしことなき
\ruby{觀世音菩薩}{くわ|んぜ|おん|ぼ|さつ}を、
\ruby{此日}{この|ひ}はじめて
\ruby{淚}{なみだ}の
\ruby{眼}{め}を
\ruby{閉}{と}ぢ、
\ruby{一心}{いつ|しん}に
\ruby{拜}{をが}み
\ruby{奉}{たてまつ}りたり。


\Entry{其二十二}

\ruby{我}{わ}が
\ruby{戀叶}{こひ|かな}へかしとも
\ruby{祈}{いの}らばこそ、たゞ
\ruby{人}{ひと}の
\ruby{命}{いのち}の
\ruby{暴風雨}{あ|ら|し}に
\ruby{揉}{も}まるる
\ruby{芭蕉葉}{ば|せう|ば}と
\ruby{危}{あやふ}きを
\ruby{悲}{かなし}みて、
\ruby{只管}{ひた|すら}に
\ruby{我}{わ}が
\ruby{五十子}{い|そ|こ}
\ruby{禍災無}{わざ|はひ|な}かれとのみ、
\ruby{堪}{た}へがたき
\ruby{思}{おもひ}の
\ruby{誠}{まこと}を
\ruby{致}{いた}して、
\ruby{他念}{た|ねん}も
\ruby{無}{な}く
\ruby{水野}{みづ|の}の
\ruby{願}{ねが}ひ
\ruby{奉}{たてまつ}れる
\ruby{折}{をり}から、
\ruby{我傍}{わが|かたへ}にも
\ruby{人}{ひと}ありて、
\ruby{先刻}{さ|き}より
\ruby{普門品}{ふ|もん|ぽん}をほそ〴〵と
\ruby{唱}{とな}へ
\ruby{居}{ゐ}けるが、
\ruby{既}{すで}に
\ruby{偈}{げ}のところにかゝりて
\ruby{漸}{やうや}く
\ruby{勢}{いきほひ}づき、
\ruby{弘誓深如海}{ぐ|ぜい|じん|によ|かい}の
\ruby{句}{く}あたりより
\ruby{嗄}{しはが}れたる
\ruby{聲}{こゑ}もおのづから
\ruby{張}{は}り
\ruby{來}{きた}りて、いま、
\ruby{或漂流巨海}{わく|へう|る|こ|かい}、
\ruby{龍魚諸鬼難}{りゆう|ぎよ|しよ|き|なん}、
\ruby{念彼觀音力}{ねん|ぴ|くわん|のん|りき}、
\ruby{波浪不能沒}{は|らう|ふ|のう|もつ}と
\ruby{調子}{てう|し}に
\ruby{乘}{の}りて
\ruby{打誦}{うち|じゆ}せるを
\ruby{見}{み}たり。

たゞ
\ruby{一}{ひ}ト
\ruby{筋}{すじ}に
\ruby{頼}{たの}み
\ruby{奉}{たてまつ}る
\ruby{思}{おもひ}は
\ruby{聲}{こゑ}の
\ruby{色}{いろ}にも
\ruby{現}{あらは}れて
\ruby{願}{ねが}ひ
\ruby{求}{もと}むる
\ruby{態}{さま}の
\ruby{僞}{いつはり}ならず
\ruby{聞}{きこ}ゆるは、
\ruby{如何}{い|か}なる
\ruby{苦惱}{くる|しみ}のある
\ruby{人}{ひと}なるか、と
\ruby{我}{わ}が
\ruby{胸}{むね}に
\ruby{疼痛}{いた|み}あれば
\ruby{他}{ひと}の
\ruby{胸}{むね}の
\ruby{疼痛}{いた|み}も
\ruby{餘{\換字{所}}}{よ|そ}ならず
\ruby{覺}{おぼ}えて
\ruby{自己念}{おの|れ|ねん}じ
\ruby{{\換字{終}}}{おは}りたる
\ruby{水野}{みづ|の}は
\ruby{其人}{その|ひと}を
\ruby{見}{み}るに、
\ruby{衣服}{な|り}こそは
\ruby{見苦}{み|ぐる}しからね、がりゝと
\ruby{痩}{や}せて
\ruby{手足}{て|あし}のみ
\ruby{徒}{いたづら}に
\ruby{長}{なが}う
\ruby{見}{み}えたる、
\ruby{髪}{かみ}は
\ruby{既}{すで}に
\ruby{薄}{うす}くして
\ruby{光澤無}{つ|や|な}き
\ruby{猫毛}{ねこ|げ}のほや〳〵と
\ruby{烟}{けむり}のやうに
\ruby{殘}{のこ}れる、
\ruby{{\換字{脱}}}{ぬ}け
\ruby{上}{あが}りたる
\ruby{額}{ひたひ}の
\ruby{特}{こと}に
\ruby{廣}{ひろ}く、
\ruby{下}{くだ}り
\ruby{長}{なが}き
\ruby{鼻}{はな}の
\ruby{細}{ほそ}くして
\ruby{淋}{さび}しさ、
\ruby{下作}{げ|さく}にはあらねど
\ruby{甚}{いた}く
\ruby{貧相}{ひん|さう}なる
\ruby{男}{をとこ}の、
\ruby{眉間}{み|けん}に
\ruby{苦}{くる}しげなる
\ruby{八字}{はち|のじ}の
\ruby{皺}{しわ}を
\ruby{深々}{ふか|〴〵}と
\ruby{疊}{たゝ}みて、
\ruby{{\換字{猶}}}{なほ}しきりに
\ruby{念彼觀音力}{ねん|ぴ|くわん|のん|りき}、
\ruby{應時得{\換字{消}}散}{おう|じ|とく|せう|さん}などゝ
\ruby{誦}{じゆ}しつゞけたる
\ruby{狀態}{あり|さま}の、
\ruby{老}{お}いたる
\ruby{人}{ひと}だけに
\ruby{愍然}{あは|れ}さ
\ruby{{\換字{勝}}}{まさ}るのみならず、
\ruby{時々}{とき|〴〵}の
\ruby{聲}{こゑ}の
\ruby{曇}{くも}りて
\ruby{顫}{ふる}ふに、
\ruby{其}{そ}の
\ruby{胸}{むね}の
\ruby{中}{うち}も
\ruby{推測}{おし|はか}られて
\ruby{物悲}{もの|がな}しく、あゝ
\ruby{憂}{うれひ}を
\ruby{懷}{いだ}くものは
\ruby{我}{われ}ばかりにはあらざりけり、
\ruby{心}{こゝろ}の
\ruby{痛苦}{くる|しみ}に
\ruby{堪}{た}へかねて、
\ruby{此人}{この|ひと}も
\ruby{御佛}{みほ|とけ}を
\ruby{頼}{たの}むなるべし、
\ruby{妻}{つま}や
\ruby{病}{や}み
\ruby{臥}{ふ}せる、
\ruby{子}{こ}や
\ruby{患}{わづら}へる、
\ruby{或}{あるひ}は
\ruby{老}{お}いて
\ruby{子}{こ}の
\ruby{無}{な}き
\ruby{歟}{か}、
\ruby{子}{こ}ありて
\ruby{或}{あるひ}は
\ruby{不孝}{ふ|かう}なる
\ruby{歟}{か}、いづれ
\ruby{悲}{かな}しき
\ruby{事{\換字{情}}}{わ|け}あらんと、そゞろに
\ruby{心惹}{こゝろ|ひ}かれて
\ruby{直}{すぐ}には
\ruby{見棄}{み|す}てかぬる
\ruby{思}{おもひ}したり。

\ruby{旱歳}{ひでり|どし}の
\ruby{冷氣早}{ひ|え|はや}き
\ruby{秋}{あき}の
\ruby{曉天}{あか|つき}の
\ruby{事}{こと}とて、
\ruby{{\換字{寒}}}{さむ}きやうに
\ruby{廣々}{ひろ|〴〵}としたる
\ruby{御堂}{み|だう}の
\ruby{中}{うち}は、
\ruby{此人}{この|ひと}と
\ruby{我}{われ}とのほかに
\ruby{人}{ひと}も
\ruby{見}{み}えず、
\ruby{香}{かう}の
\ruby{氣}{き}しづかに
\ruby{薫}{くん}じて
\ruby{殊{\換字{勝}}}{しゆ|しよう}さ
\ruby{身}{み}に
\ruby{浸}{し}み
\ruby{渡}{わた}り、
\ruby{見上}{み|あ}ぐる
\ruby{眼}{め}を
\ruby{照}{て}らす
\ruby{施無畏}{せ|む|ゐ}の
\ruby{三大字}{さん|だい|じ}は、
\ruby{一世}{いつ|せ}に
\ruby{秀}{ひいで}し
\ruby{佐{\換字{文}}山}{さ|ぶん|ざん}が
\ruby{長櫃三個}{なが|もち|さん|さを}の
\ruby{反故}{ほ|ご}をつくつて
\ruby{纔}{わづか}に
\ruby{書}{か}きしといふ
\ruby{傳{\換字{説}}}{いひ|つたへ}さへ、おのづと
\ruby{想}{おも}ひ
\ruby{起}{おこ}さるゝばかり
\ruby{筆勢遒麗}{ひつ|せい|しう|れい}に、
\ruby{金光美}{きん|くわう|ゝつく}しく
\ruby{高}{たか}く
\ruby{懸}{かゝ}りて、まことに
\ruby{人}{ひと}をして
\ruby{慈眼視衆生}{じ|げん|じ|しゆ|じやう}の
\ruby{菩薩}{ぼ|さつ}の
\ruby{威力}{い|りき}を
\ruby{仰}{あお}がんとする
\ruby{心}{こゝろ}を
\ruby{發}{おこ}さしめ、たま〳〵に
\ruby{鳩}{はと}のはた〳〵と
\ruby{飛}{と}んでは
\ruby{靜}{しづ}かさを
\ruby{破}{やぶ}るのも
\ruby{却}{かへ}つて
\ruby{寂}{さ}びて、
\ruby{{\換字{平}}生}{ひご|ろ}の
\ruby{賑}{にぎ}はじさに
\ruby{引反}{ひき|か}へて
\ruby{今{\換字{朝}}}{け|さ}の
\ruby{此}{こ}の
\ruby{御堂}{み|だう}の
\ruby{神々}{かう|〴〵}しく
\ruby{{\換字{尊}}}{たつと}さに、
\ruby{水野}{みづ|の}は
\ruby{今}{いま}まで
\ruby{知}{し}らざりし
\ruby{趣味}{おも|むき}をおぼえたり。

\ruby{妙音觀世音}{めう|おん|くわん|ぜ|おん}、
\ruby{梵音海潮音}{ぼん|おん|かい|てう|おん}、
\ruby{{\換字{勝}}彼世間音}{しよう|ひ|せ|けん|おん}、と
\ruby{老}{お}いたる
\ruby{人}{ひと}の
\ruby{誦}{しよう}する
\ruby{聲}{こゑ}は、いよ〳〵
\ruby{眞心籠}{ま|ごゝろ|こも}りて
\ruby{澄}{す}み
\ruby{行}{ゆ}き、
\ruby{普門品}{ふ|もん|ぼん}は
\ruby{今}{いま}や
\ruby{{\換字{終}}}{をは}るに
\ruby{{\換字{近}}}{ちか}からんとす。

\ruby{時}{とき}に
\ruby{御堂}{み|だう}の
\ruby{内俄}{うち|にはか}に
\ruby{騒}{さわ}がしく、がたごとゝ
\ruby{薩摩下駄踏}{さつ|ま|げ|た|ふ}み
\ruby{鳴}{な}らす
\ruby{音}{おと}を
\ruby{憚}{はゞか}り
\ruby{氣無}{げ|な}く
\ruby{伽藍}{が|らん}に
\ruby{響}{ひび}かせて、
\ruby{太}{ふと}き〳〵
\ruby{洋杖}{すて|つき}もて
\ruby{{\換字{益}}}{えき}も
\ruby{無}{な}く
\ruby{床板}{ゆか|いた}を
\ruby{突}{つ}きちらし
\ruby{撲}{たゝ}きちらしながら、
\ruby{入}{い}り
\ruby{來}{きた}れる
\ruby{二人}{ふた|り}の
\ruby{書生}{しよ|せい}あり。
\ruby{醉}{ゑひ}を
\ruby{帶}{お}びたりとは
\ruby{見}{み}えねど
\ruby{反響}{こだ|ま}の
\ruby{起}{おこ}るほどの
\ruby{馬鹿聲}{ば|か|ごゑ}をあげて、

『ハツ、オイ、まだ
\ruby{此樣}{こ|ん}なものを
\ruby{本氣}{ほん|き}で
\ruby{禮拜}{らい|はい}して
\ruby{居}{ゐ}るものがあるぜ!。
』

と、
\ruby{紺絞}{こん|しぼり}の
\ruby{兵兒帶}{へ|こ|おび}を
\ruby{締}{し}めたるが
\ruby{云}{い}へば、

『ウン、
\ruby{可愍}{ふ|びん}なものさ、
\ruby{五六世紀}{ご|ろく|せい|き}も
\ruby{{\換字{前}}}{まへ}の
\ruby{思想}{し|さう}に
\ruby{養}{やしな}はれて
\ruby{居}{い}るのだからナ。
』

と、
\ruby{白金巾}{しろ|がな|きん}の
\ruby{帶}{おび}したるが
\ruby{答}{こた}へたり。

『
\ruby{我輩}{わが|はい}の
\ruby{親{\換字{分}}}{おや|ぶん}は、
\ruby{基督}{くり|すと}が
\ruby{代表}{だい|へう}した
\ruby{馬鹿思想}{ば|か|し|さう}を
\ruby{奴隷{\換字{道}}徳}{ど|れい|だう|とく}と
\ruby{罵}{のゝし}つたが、
\ruby{我輩}{わが|はい}は
\ruby{法然日蓮}{はふ|ねん|にち|れん}の
\ruby{代表}{だい|へう}した
\ruby{馬鹿思想}{ば|か|し|さう}を
\ruby{乞食{\換字{道}}徳}{こ|じき|だう|とく}と
\ruby{斷言}{だん|げん}するが、
\ruby{何樣}{ど|う}だ、
\ruby{可}{よ}からう。
』

『ウン、
\ruby{偉}{えら}い!。
\ruby{釋迦}{しや|か}が
\ruby{事實上乞食}{じ|ゞつ|じやう|こ|じき}だから
\ruby{{\換字{猶}}}{なほ}
\ruby{可笑}{を|か}しい。
それだのに、
\ruby{木佛金佛}{き|ぶつ|かな|ぶつ}を
\ruby{拜}{をが}む
\ruby{奴}{やつ}さへあるのだからナ。
ほんとに
\ruby{本能主義}{ほん|のう|しゆ|ぎ}の
\ruby{有}{あ}り
\ruby{難}{がた}い、
\ruby{大}{おほ}もての
\ruby{美的境界}{び|てき|きやう|がい}でも
\ruby{敎}{をし}へて
\ruby{{\換字{遣}}}{や}りたいナ。
ハヽヽハヽ。
』

『ヤ、
\ruby{酷}{ひど}いところで
\ruby{自惚}{うぬ|ぼれ}る
\ruby{奴}{やつ}だナ。
ハヽハヽ。
』

\ruby{十八間四面}{じう|はつ|けん|し|めん}の
\ruby{御堂}{み|だう}も
\ruby{動}{ゆら}ぐばかりに
\ruby{高笑}{たか|わら}ひして、
\ruby{繪}{ゑ}に
\ruby{見}{み}る
\ruby{惡鬼羅刹}{あく|き|ら|せつ}が
\ruby{持}{も}てる
\ruby{銕{\換字{杖}}}{てつ|ぢやう}の
\ruby{如}{ごと}き
\ruby{恐}{おそ}ろしき
\ruby{重}{おも}げなる
\ruby{{\換字{杖}}}{つゑ}もて、
\ruby{我}{わ}が
\ruby{踏}{ふ}める
\ruby{床}{ゆか}を、
\ruby{我}{わ}が
\ruby{威風}{ゐ|ふう}を
\ruby{見}{み}よとばかりに、どしんと
\ruby{突}{つ}きたり。

\ruby{皆發無等々阿耨多羅三藐三菩提心}{かい|ほつ|む |とう|〳〵|あ |のく|た |ら |さん|みやく|さん|ぼ|だい|しん}と、
\ruby{念}{ねん}じ
\ruby{{\換字{終}}}{をは}りて
\ruby{禮拜}{らい|はい}し
\ruby{濟}{すま}したる
\ruby{老}{お}いたる
\ruby{男}{をとこ}は、
\ruby{頭}{かうべ}を
\ruby{擡}{もた}げて
\ruby{水野}{みづ|の}と
\ruby{顏見合}{かほ|み|あ}はせて、おもはず
\ruby{互}{たがひ}に
\ruby{眉}{まゆ}を
\ruby{顰}{ひそ}めざるを
\ruby{得}{え}ざりき。


\Entry{其二十三}

\ruby{天}{そら}の
\ruby{彼方}{あな|た}に
\ruby{颶風}{つむじ|かぜ}を
\ruby{起}{おこ}しゝニイチエが
\ruby{眞趣}{おも|むき}を
\ruby{實}{まこと}に
\ruby{知}{し}れりや、それも
\ruby{覺束無}{おぼ|つか|な}げなる
\ruby{書生}{しよ|せい}の
\ruby{放言}{はう|げん}の、
\ruby{餘}{あま}りの
\ruby{事}{こと}に
\ruby{傍痛}{かたは|らいた}くはおぼえたれど、
\ruby{意}{こゝろ}を
\ruby{動}{うご}かすほどにも
\ruby{至}{いた}らざりければ、
\ruby{他}{ひと}は
\ruby{他}{ひと}なり、
\ruby{我}{われ}は
\ruby{我}{われ}なり、
\ruby{關係無}{かけ|かまひ|な}き
\ruby{禽}{とり}の
\ruby{聲}{こゑ}の、それまでの
\ruby{事}{こと}なりと
\ruby{聞}{き}き
\ruby{捨}{す}てゝ、
\ruby{既}{すで}に『ツアラツウストラ
\ruby{如是{\換字{説}}}{によ|ぜ|せつ}』をも
\ruby{窺}{うかが}ひ
\ruby{讀}{よ}まぬにあらざりし
\ruby{水野}{みづ|の}は、\換字{志}ろりと
\ruby{冷}{ひや}やかに
\ruby{彼}{か}の
\ruby{二人}{ふ|たり}をば
\ruby{一瞥}{いち|べつ}せしのみに
\ruby{止}{とゞ}まりて、
\ruby{徐々此處}{やお|ら|こ|ゝ}を
\ruby{去}{さ}らんと
\ruby{歩}{あゆ}み
\ruby{出}{いだ}せば、
\ruby{彼}{か}の
\ruby{老}{お}いたる
\ruby{男}{をとこ}も
\ruby{一齊}{と|も}にと
\ruby{{\換字{随}}}{したが}へり。

\ruby{世}{よ}の
\ruby{態人}{さま|ひと}の
\ruby{{\換字{情}}}{こゝろ }\ %空白有り
\ruby{漸}{ やうや}く
\ruby{移}{うつ}りて、
\ruby{礎}{いしずゑ}は
\ruby{舊}{きう}に
\ruby{依}{よ}りて
\ruby{固}{かた}く、
\ruby{棟}{むね}は
\ruby{舊}{きう}に
\ruby{依}{よ}りて
\ruby{高}{たか}けれども、
\ruby{今}{いま}は
\ruby{此}{こ}の
\ruby{莊嚴}{さう|ごん}なる
\ruby{御堂}{み|だう}の
\ruby{内}{うち}にさへも、
\ruby{謗法毀佛}{はう|ばふ|き|ぶつ}の
\ruby{暴}{あば}れ
\ruby{聲起}{ごゑ|おこ}りて、
\ruby{譬喩}{たと|へ}を
\ruby{取}{と}りて
\ruby{云}{い}はゞ
\ruby{月黑}{つき|くろ}き
\ruby{夜}{よ}の
\ruby{大潮}{おほ|しほ}の、
\ruby{洲}{す}を
\ruby{吞}{の}み
\ruby{岩}{いは}を
\ruby{嚙}{か}みて
\ruby{漸}{やうや}く
\ruby{大地}{だい|ち}を
\ruby{犯}{をか}さんとするが
\ruby{如}{ごと}くに、
\ruby{何時}{い|つ}となく
\ruby{破壞}{は|くわい}の
\ruby[g]{吶喊}{さけび}の
\ruby{押寄}{おし|よ}するは、
\ruby{所謂末法澆季}{いは|ゆる|まつ|ぱふ|げう|き}の
\ruby{是非}{ぜ|ひ}も
\ruby{無}{な}き
\ruby{當時}{い|ま}の
\ruby{大勢}{いき|ほい}なり。
\ruby{書生}{しよ|せい}は
\ruby{{\換字{猶}}}{なほ}がたりごとりと、
\ruby{力足}{ちから|あし}を
\ruby{踏}{ふ}み
\ruby{{\換字{杖}}}{つゑ}を
\ruby{突}{つ}き
\ruby{立}{た}てゝて
\ruby{歩}{ある}き
\ruby{居}{ゐ}しが、
\ruby{紺絞}{こん|しぼり}の
\ruby{帶}{おび}したるは、
\ruby{急}{きふ}に
\ruby{虛空}{こ|くう}に
\ruby{{\換字{杖}}}{つゑ}を
\ruby{擧}{あ}げて、
\ruby{揭}{かゝ}げられたる
\ruby{額}{がく}の
\ruby{一}{ひと}つを
\ruby{指}{さ}しながら、

『
\ruby{面白}{おも|しろ}いナア、
\ruby{此}{こ}の
\ruby{一}{ひと}つ
\ruby{家}{や}の
\ruby{畫}{ゑ}は!、どうも
\ruby{巧}{よ}く
\ruby{出來}{で|き}て
\ruby{居}{ゐ}るナ、
\ruby{氣}{き}に
\ruby{入}{い}つたナア!。
』

と
\ruby{云}{い}へば、

『ムヽ、』

と、
\ruby{白}{しろ}き
\ruby{帶}{おび}したるは
\ruby{其意}{その|い}を
\ruby{得}{え}ぬげに
\ruby{應}{こた}へつ、

『\換字{志}かし
\ruby{御厩}{みう|まや}の
\ruby{喜三太}{き|さん|だ}も
\ruby{好}{い}いぢやあ
\ruby{無}{な}いか。
』

と
\ruby{附加}{つけ|くは}へたり。

『
\ruby{馬鹿}{ば|か}ツ!。
そりやあ
\ruby{技術}{ぎじ|ゆつ}だけの
\ruby{論}{ろん}だ。
\ruby{云}{い}ふなあ
\ruby{其處}{そ|こ}ぢやあ
\ruby{無}{な}い。
よく
\ruby{見}{み}ろ!
\ruby{吾輩}{わが|はい}の
\ruby{此}{こ}の
\ruby{一}{ひと}つ
\ruby{家}{や}の
\ruby{圖}{づ}を!。
\ruby{何樣}{ど|う}だ
\ruby{彼}{あ}の
\ruby{婆}{ばあ}さんの
\ruby{顏}{かほ}の
\ruby{立派}{りつ|ぱ}なこと!。
\ruby{實}{じつ}に
\ruby{立派}{りつ|ぱ}ぢやあ
\ruby{無}{な}いか、
\ruby{立派}{りつ|ぱ}ぢやあ
\ruby{無}{な}いか!。
\ruby{國家}{こく|か}の
\ruby{法律}{はふ|りつ}なんぞといふ
\ruby{奴}{やつ}ア
\ruby{踏}{ふ}み
\ruby{付}{つ}け
\ruby{切}{き}つた
\ruby{彼}{あ}の
\ruby{顏}{かほ}つき!。
\ruby{世間}{せ|けん}の
\ruby{善惡}{ぜん|あく}の
\ruby{沙汰}{さ|た}なんぞを
\ruby{寄}{よ}せつけも
\ruby{仕無}{し|な}い
\ruby{彼}{あ}の
\ruby{顏付}{かほ|つき}!。
\ruby{戀}{こひ}も
\ruby{人{\換字{情}}}{にん|じやう}も
\ruby{無}{な}い
\ruby{彼}{あ}の
\ruby{顏}{かほ}つき!。
\ruby{邪}{じや}でも
\ruby{非}{ひ}でもまかはない
\ruby{彼}{あ}の
\ruby{顏}{かほ}つき!。
おれが
\ruby{{\換字{勝}}手}{かつ|て}だぞといふ
\ruby{彼}{あ}の
\ruby{顏}{かほ}つき!。
\ruby{神}{かみ}でも
\ruby{佛}{ほとけ}でも
\ruby{對面}{むか|ふ}へまはつたら
\ruby{斫殺}{たゝ|つき}つて
\ruby{{\換字{遣}}}{や}らうといふ
\ruby{彼}{あ}の
\ruby{顏}{かほ}つき!。
あゝ
\ruby{何}{なん}と
\ruby{立派}{りつ|ぱ}な
\ruby{顏}{かほ}に
\ruby{書}{か}いてあるでは
\ruby{無}{な}いか。
\ruby{十{\換字{分}}}{じう|ぶん}に
\ruby{惡人}{あく|にん}の
\ruby{偉大}{ゐ|だい}な
\ruby{精神}{せい|しん}が
\ruby{發揮}{はつ|き}してある!。
\ruby{誰}{たれ}だつて
\ruby{此}{こ}の
\ruby{繪}{ゑ}を
\ruby{能}{よ}く
\ruby{見}{み}たらば、
\ruby{{\換字{強}}惡}{がう|あく}が
\ruby{美}{い}いものだといふ
\ruby{事}{こと}に
\ruby{氣}{き}が
\ruby{付}{つ}くだらう!。
\ruby{見}{み}ろ、
\ruby{彼}{あ}の
\ruby{娘}{むすめ}が
\ruby{卑小}{け|ち}な
\ruby{惡}{わる}びれた
\ruby{樣子}{やう|す}を!。
\ruby{人}{ひと}に
\ruby{縋}{すが}りたがるやうな、
\ruby{哀愍}{あは|れみ}を
\ruby{乞}{こ}うやうな、
\ruby{泣}{な}き
\ruby{出}{だ}しさうな、
\ruby{切}{せつ}なさうな、
\ruby{善惡}{ぜん|あく}の
\ruby{{\換字{道}}理}{だう|り}を
\ruby{怖}{こは}がつて
\ruby{居}{ゐ}るやうな、
\ruby{國家}{こく|か}の
\ruby[g]{規律}{おきて}なんぞにびく〳〵して
\ruby{居}{ゐ}るやうな、
\ruby{神佛}{しん|ぶつ}なんぞにおど〳〵して
\ruby{居}{ゐ}る、\換字{志}みつたれた、
\ruby{見}{み}つとも
\ruby{無}{な}い
\ruby{醜態}{ざ|ま}が、すつかり
\ruby{見}{み}えて
\ruby{居}{ゐ}る!。
\ruby{所謂善人}{いは|ゆる|ぜん|にん}といふ
\ruby{奴}{やつ}が
\ruby{卑劣}{け|ち}なもので、
\ruby{下}{くだ}らないものだといふ
\ruby{事}{こと}は、
\ruby{何樣}{ど|ん}な
\ruby{馬鹿}{ば|か}な
\ruby{奴}{やつ}の
\ruby{眼}{め}にも
\ruby{暎}{うつ}るだらう!。
\ruby{何樣}{ど|う}だ、
\ruby{好}{い}いぢやあ
\ruby{無}{な}いか
\ruby{好}{い}い
\ruby{畫}{ゑ}ぢやあ
\ruby{無}{な}いか。
、
\ruby{何樣}{ど|う}だ、
\ruby{{\換字{分}}}{わ}かつたか、
\ruby{好}{い}いか、オイ、
\ruby{君}{きみ}!。
\ruby{此}{こ}の
\ruby{一}{ひと}つ
\ruby{家}{や}の
\ruby{御婆}{お|ばあ}さんが
\ruby{國王}{こく|わう}になりやあ、
\ruby{世界中}{せ|かい|ぢゆう}を
\ruby{斬}{き}り
\ruby{伏}{ふ}せて
\ruby{寝酒}{ね|ざけ}の
\ruby[g]{下物}{さかな}に
\ruby{仕}{し}て
\ruby{{\換字{遣}}}{や}らうと、
\ruby{手}{て}に
\ruby{持}{も}つた
\ruby{利器}{え|もの}を
\ruby{振}{ふ}り
\ruby{舞}{ま}はすんだ!。
もし
\ruby{此}{こ}の
\ruby{娘}{むすめ}が
\ruby{國王}{こく|わう}になりやあ、
\ruby{彼方}{あつ|ち}へも
\ruby{此方}{こつ|ち}へも
\ruby{氣}{き}がねを
\ruby{仕}{し}て、
\ruby{一年中}{いち|ねん|ぢゆう}べそをかいて
\ruby{居}{ゐ}なけりやあならないんだ!。
\ruby{何樣}{ど|う}だ、
\ruby{{\換字{強}}惡}{がう|あく}に
\ruby{限}{かぎ}るだらう!。
\ruby{一體眞實}{いつ|たい|ほん|たう}の
\ruby{理屈}{り|くつ}から
\ruby{云}{い}やあ、
\ruby{此}{こ}の
\ruby{娘}{むすめ}の
\ruby{方}{はう}が
\ruby{善}{ぜん}なのだからナア。
』

『ウン、
\ruby[g]{成程々々}{なるほど〳〵}。
\ruby{{\換字{強}}惡}{がう|あく}は
\ruby{眞實}{ほん|と}に
\ruby{偉}{えら}いナア!。
だけれど
\ruby{憫然}{かはい|さう}に
\ruby{今}{いま}の
\ruby[g]{世界}{せかい}ぢやあ、
\ruby{男子}{をの|こ}でも
\ruby{此}{こ}の
\ruby{娘}{むすめ}のやうな
\ruby{奴}{やつ}ばかり
\ruby{多}{おほ}いぜ!。
ハヽヽ。
』

『ハヽハヽヽ、
\ruby{左樣}{さ|う}だ、〳〵、
\ruby{笑}{わら}つて
\ruby{{\換字{遣}}}{や}れ、
\ruby{笑}{わら}つて
\ruby{{\換字{遣}}}{や}れ。
アツハツハツハヽヽヽ。
』

『アツハツハツハヽヽヽ。
』

\ruby{朝詣}{あさ|まゐ}りする
\ruby{人}{ひと}のちらほらとは
\ruby{見}{み}え
\ruby{初}{そ}めたれど、
\ruby{{\換字{猶}}}{なほ}
\ruby{極}{きは}めて
\ruby[g]{四邊}{あたり}の
\ruby{物靜}{もの|しづ}かなれば、
\ruby{聞}{き}けよがしに
\ruby{聲大}{こゑ|おほき}く
\ruby{語}{かた}らふ
\ruby{二人}{ふた|り}の
\ruby{談}{はなし}は、
\ruby{既}{すで}に
\ruby{御堂}{み|だう}を
\ruby{離}{はな}れて
\ruby{石路}{せき|ろ}を
\ruby{歩}{あゆ}める
\ruby{水野}{みづ|の}と
\ruby{彼}{か}の
\ruby{老}{お}いたる
\ruby{男}{をとこ}との
\ruby{背後}{うし|ろ}より
\ruby{響}{ひゞ}きて、
\ruby{態}{わざ}とらしき
\ruby{其}{そ}の
\ruby{嘲}{あざけ}り
\ruby{笑}{わら}ひも
\ruby{一々}{いち|〳〵}
\ruby{聞}{きこ}こえたり。
今しも
\ruby{水野}{みづ|の}と
\ruby{並}{なら}びて
\ruby{歩}{ある}ける
\ruby{彼}{か}の
\ruby{男}{をとこ}は
\ruby{再}{ふたた}び
\ruby{水野}{みづ|の}と
\ruby{面}{おもて}を
\ruby{見合}{み|あ}はせつ、
\ruby{{\換字{終}}}{つひ}に
\ruby{堪}{た}へ
\ruby{{\換字{兼}}}{か}ねてか
\ruby{口}{くち}を
\ruby{開}{ひら}き、

『
\ruby{大變}{たい|へん}な
\ruby{世}{よ}の
\ruby{中}{なか}になつてまゐりました!。
\ruby{私共}{わたし|ども}の
\ruby{倅}{せがれ}なんぞも
\ruby{學校}{ぐあ|かう}へ
\ruby{{\換字{遣}}}{や}つて
\ruby{置}{お}きましたら、まあ
\ruby{矢張}{や|は}り
\ruby{彼樣}{あ|ゝ}いつた
\ruby{調子}{てう|し}になりまして、
\ruby{人}{ひと}に
\ruby{苦勞}{く|らう}ばかりいたさせます。
\ruby{御参}{おま|ゐり}を
\ruby{致}{いた}しますのも、
\ruby{實}{じつ}を
\ruby{申}{まを}しますと、つまりは
\ruby{其樣}{そ|ん}な
\ruby{譯}{わけ}から
\ruby{起}{おこ}つた
\ruby{事}{こと}のためでございますが、……』

と、
\ruby{思}{おも}ひ
\ruby{餘}{あま}つたる
\ruby{憂}{う}さを
\ruby{漏}{も}らしかけしが、
\ruby{流石}{さす|が}に
\ruby{心}{こゝろ}づきて、
\ruby{馴染無}{なじ|み|な}き
\ruby{人}{ひと}に
\ruby{吾}{わ}が
\ruby{家内}{い|へ}の
\ruby{事}{こと}を
\ruby{言}{い}はんもはしたなしとてや、

『
\ruby{御利生}{ご| り |しやう}を
\ruby{現}{あら}はさうとして
\ruby{書}{か}きました
\ruby{額}{がく}を
\ruby{見}{み}て、
\ruby{一}{ひと}つ
\ruby{家}{や}の
\ruby{婆}{ばあ}さんの
\ruby{方}{はう}を
\ruby{褒}{ほ}めますなんて、ほんに
\ruby{淺草寺}{せん|さう|じ}はじまつて
\ruby{以來無}{この|かた|な}い
\ruby{事}{こと}でございましやう!。
まあ
\ruby{何}{なん}といふ
\ruby{間{\換字{違}}}{ま|ちが}つた
\ruby{事}{こと}で!。
』

と、
\ruby{談}{はなし}を
\ruby{横}{よこ}に
\ruby{{\換字{逸}}}{そ}らしたり。
\ruby{水野}{みづ|の}は
\ruby{當}{あ}たり
\ruby{障}{さは}らずに、

『まことに
\ruby{左樣}{さ|やう}でござります。
』

と、
\ruby{穩}{おだ}やかに
\ruby{答}{こた}へて
\ruby{多}{おほ}くは
\ruby{言}{ものい}はず、たゞ
\ruby{人}{ひと}の
\ruby{親}{おや}には
\ruby{{\換字{情}}}{なさけ}
\ruby{篤}{あつ}きが
\ruby{多}{おほ}きに、
\ruby{人}{ひと}の
\ruby{子}{こ}にはまた
\ruby{彼等二人}{かれ|ら|ふた|り}の
\ruby{如}{ごと}く
\ruby{心放縱}{こゝろ|ほしい|まゝ}なるが
\ruby{多}{おほ}き
\ruby{世}{よ}の
\ruby{相}{すがた}の、さま〴〵なるを
\ruby{思}{おも}ひて
\ruby{歎}{たん}じながらも、
\ruby{今}{いま}の
\ruby{書生}{しよ|せい}の
\ruby{笑}{わら}ひ
\ruby{聲}{ごゑ}には、
\ruby{少}{すくな}からず
\ruby{不快}{ふ|くわい}を
\ruby{覺}{おぼ}えたり。

\ruby{自}{みづか}ら
\ruby{知}{し}る
\ruby{我}{わ}が
\ruby[g]{昨夕}{ゆふべ}のありさまは、
\ruby{取}{と}りも
\ruby{直}{なほ}さず
\ruby{旅}{たび}の
\ruby{人}{ひと}を
\ruby{護}{かば}へる
\ruby{彼}{か}の
\ruby{娘}{むすめ}にも
\ruby{似}{に}て、
\ruby{病}{や}める
\ruby{五十子}{い|そ|こ}を
\ruby{恤}{いたは}らんがためとて、
\ruby{一}{ひと}つ
\ruby{家}{や}の
\ruby{婆}{ばゞ}にも
\ruby{似}{に}たらん
\ruby{彼}{か}の
お
\ruby{澤婆}{さは|ばゞあ}に、
\ruby{下}{さ}げがたき
\ruby{頭}{かしら}を
\ruby{幾度}{いく|たび}も
\ruby{{\換字{益}}無}{えき|な}く
\ruby{下}{さ}げて、\換字{志}かも
\ruby{{\換字{益}}無}{えき|な}く
\ruby{云}{い}ひ
\ruby{斥}{しりぞ}けられたる
\ruby{其事}{その|こと}の
\ruby{今}{いま}さら
\ruby{胸}{むね}に
\ruby{{\換字{浮}}}{うか}み
\ruby{來}{く}れば、
\ruby{當無}{あて|な}く
\ruby{放}{はな}ちたるには
\ruby{疑}{うたが}ひ
\ruby{無}{な}き
\ruby[g]{嘲笑}{わらひ}の
\ruby{矢}{や}も、\換字{志}たゝかに
\ruby{我}{わ}が
\ruby{背}{そびら}に
\ruby{立}{た}てる
\ruby{心地}{こゝ|ち}して、
\ruby{厭}{いと}はしき
\ruby{思}{おもひ}の
\ruby{比}{たと}ふるに
\ruby{物無}{もの|な}く、
\ruby{身}{み}の
\ruby{内}{うち}を
\ruby{掻}{か}き
\ruby{挘}{むし}りたきやうなる
\ruby{感}{かん}じを
\ruby{{\換字{懐}}}{いだ}きつゝ、
\ruby{夢路}{ゆめ|ぢ}を
\ruby{辿}{たど}るが
\ruby{如}{ごと}く
\ruby{中店}{なか|みせ}を
\ruby{出}{で}はづるれば、

『ヤ、
\ruby{水野}{みづ|の}さん。
』

と、
\ruby{凉}{すゞ}しき
\ruby{聲}{こゑ}の
\ruby{玉}{たま}を
\ruby{轉}{まろ}ばすが
\ruby{如}{ごと}くに
\ruby{呼}{よ}びかけて、
\ruby{黑革}{くろ|かは}の
\ruby{眉庇付}{まび|さし|つ}きたる
\ruby{帽}{ぼう}を
\ruby{傾}{かたぶ}けつゝ、
\ruby{身}{み}を
\ruby{前屈}{まへ|かゞ}みにして
\ruby{走}{はし}り
\ruby{來}{きた}れる
\ruby{美少年}{び|せう|ねん}あり。
\ruby{彼}{か}の
\ruby{老}{お}いたる
\ruby{男}{をとこ}は
\ruby{既}{すで}に
\ruby{去}{さ}つて
\ruby{在}{あ}らず。


\Entry{其二十四}

% メモ 校正終了 2024-04-09 2024-05-26 2024-06-19
\原本頁{145-7}%
\ruby{{\換字{近}}}{ちか}づくや
\ruby{否}{いな}や
\ruby{帽}{ばう}を
\ruby{脫}{と}りて、
%
\ruby{眞{\換字{率}}}{しん|そつ}に
\ruby{頭}{かうべ}を
\ruby{下}{さ}げて
\ruby{挨拶}{あい|さつ}するは、
%
\ruby{林檎}{りん|ご}の
\ruby{如}{ごと}く
\ruby{美}{うつく}しき
\ruby{色澤}{いろ|つや}、
%
\ruby[||j>]{人}{にん}
\ruby[||j>]{形}{ぎやう}の
% \ruby{人形}{にん|ぎやう}の
\ruby{如}{ごと}き
\ruby{端正}{た|ゞ}しき
\ruby{眼鼻立}{め|はな|だち}、
%
\ruby{姊}{あね}の
\ruby{男}{をとこ}にしても
\ruby{見}{み}まはしく
\ruby{立派}{りつ|ぱ}なるには
\ruby{異}{かは}りて、
%
\ruby{此}{これ}は
\ruby{女}{をんな}にしても
\ruby{見}{み}たく
\ruby{可愛}{か|はい}らしと
\ruby{人}{ひと}に
\ruby{云}{い}はれたる
\ruby{五十子}{い|そ|こ}が
\ruby{弟}{おとゝ}の
\ruby{松之助}{まつ|の|すけ}なり。

\原本頁{146-1}%
\ruby{母}{はゝ}は
\ruby{有}{あ}りても
\ruby{繼}{まゝ}しき
\ruby{中}{なか}なり、
%
\ruby{財產}{ざい|さん}は
\ruby{繼母}{は|ゝ}に
\ruby{皆}{みな}
\ruby{奪}{と}られたり、
%
\ruby{姊}{あね}より
ほかに
\ruby{頼}{たの}むべき
\ruby{人}{ひと}を
\ruby{有}{も}たぬ
\ruby{松之助}{まつ|の|すけ}は、
%
\ruby{往時}{むか|し}の
\ruby{{\換字{乳}}母}{う|ば}なりしが
\ruby{今}{いま}は
\ruby{下谷}{した|や}の
\ruby{廣小路}{ひろ|こう|ぢ}
\ruby{{\換字{近}}}{ちか}くに、
%
\ruby{下梳}{した|すき}の
\ruby{二人}{ふた|り}も
\ruby{使}{つか}ふほどの
\ruby{女髮結}{か|み|ゆひ}となりて、
%
\ruby{堅}{かた}く
\ruby{身}{み}を
\ruby{持}{も}てる
\ruby{幸福}{しあ|はせ}には%「幸福」ここは「は」
\ruby{苦}{くる}しげ
\ruby{無}{な}く
\ruby{日}{ひ}を
\ruby{{\換字{送}}}{おく}れるが
\ruby{許}{もと}に、
%
\ruby{{\換字{留}}守番}{る|す|ばん}を
\ruby{{\換字{兼}}}{か}ねたる
\ruby{客寓人}{かゝ|り|びと}となりつ、
%
\ruby{月々}{つき|〴〵}
\ruby{姊}{あね}が
\ruby{取}{と}る
\ruby{僅少}{わづ|か}なる
\ruby{給料}{きふ|れう}の
\原本頁{146-6}\改行%
\ruby{内}{うち}より、
%
\ruby{{\換字{分}}}{わ}けて
\ruby{貰}{もら}ふ
\ruby{財布}{さい|ふ}の
\ruby{塵芥}{ご|み}ほどの
\ruby{金子}{か|ね}を、
%
\ruby{一{\換字{半}}}{なか|ば}は
\ruby{形式}{か|た}ばかりの
\ruby[||j>]{食}{しよく}
\ruby[||j>]{料}{ れう}として
% \ruby{食料}{しよく|れう}として
\ruby{入}{い}れ、
%
\ruby{一{\換字{半}}}{なか|ば}は
おのれの
\ruby{學資}{がく|し}として、
%
\ruby{責}{せ}めて
\ruby{某}{それ}の
\原本頁{146-8}\改行%
\ruby{學校}{がく|かう}の
\ruby{官費生}{くわん|ぴ|せい}となりて
\ruby{世}{よ}に
\ruby{立}{た}つ
\ruby{{\換字{道}}}{みち}の
\ruby[<j>]{緖}{いとぐち}を
\ruby{得}{う}る
\ruby{迄}{まで}と、
%
\ruby{足}{た}らぬ
\ruby{{\換字{勝}}}{がち}なる
\ruby{中}{なか}にも
\ruby{心}{こゝろ}を
\ruby{勵}{はげ}まして、
%
\ruby{夜學}{や|がく}の
\ruby{歸路}{かへ|り}は
\ruby{辛}{つら}き
\ruby{{\換字{冬}}}{ふゆ}の
\ruby{{\換字{雪}}}{ゆき}、
%
\ruby{籠}{こも}り
\ruby{居}{ゐ}の
\ruby{夏}{なつ}は
\ruby{堪}{た}へ
\ruby{{\換字{難}}}{がた}き
\ruby{陋巷}{ろ|じ}の
\ruby{奧}{おく}の
\ruby{矮屋}{こ|いへ}の
\ruby{暑熱}{あつ|さ}にも、
%
\ruby{萎}{め}げず
\ruby{怯}{ひる}まずして
\ruby[<j||]{勉}{べん }% 行末行頭の境界付近なので特例処置を施す
\ruby[<j||]{{\換字{強}}}{きやう}
% \ruby{勉{\換字{強}}}{べん|きやう}
\原本頁{146-11}\改行%
すれば、
%
\ruby{齡}{とし}は
\ruby{{\換字{猶}}}{なほ}
\ruby{數}{かぞ}へ
\ruby{年}{どし}の
\ruby{十七}{じふ|なな}にして、
%
\ruby{思想}{かん|がへ}こそは
\ruby{世}{よ}に
\ruby{磨}{す}れざれ
\改行% 校正作業の簡略化のため
、
%
\原本頁{147-1}\改行%
\ruby{學問}{がく|もん}の
\ruby{出來}{で|き}は
いと
\ruby{佳}{よ}くして、
%
\ruby{行末}{ゆく|すゑ}
\ruby{發{\換字{達}}}{なり|い}づべく
\ruby{見}{み}ゆる
\ruby{少年}{せう|ねん}なり。
%
\原本頁{147-2}\改行%
\ruby{繼母}{は|ゝ}は
\ruby{不品行}{ふ|み|もち}にして
\ruby[||j>]{心}{こゝろ}
\ruby[||j>]{曲}{ ゆが}み、
%
\ruby{有}{あ}りても
\ruby{却}{かへ}つて
\ruby{無}{な}きに
\ruby{劣}{おと}れば、
%
\ruby{天}{てん}にも
\ruby{地}{ち}にも
\ruby{頼}{たの}み
\ruby{頼}{たの}まるべきは
\ruby{只}{たゞ}
\ruby[||j>]{姊}{きよう}
\ruby[||j>]{弟}{ だい}と、
% \ruby{姊弟}{きよう|だい}と、
%
\ruby{深}{ふか}くも
\ruby{此}{こ}の
\ruby{弟}{おとゝ}の
\ruby{上}{うへ}を
のみ
\ruby{思}{おも}ひて、
%
\ruby{自己}{おの|れ}の
\ruby{今}{いま}の
\ruby{身}{み}は
\ruby{差}{さ}し
\ruby{當}{あた}りて
\ruby{田舎}{ゐな|か}の
\ruby{草萊}{く|さ}の
\ruby{間}{あひだ}に
\ruby{埋}{うづ}もれ
\原本頁{147-5}\改行%
\ruby{沒}{かく}るゝとも、
%
\ruby{如何}{い|か}にもして
\ruby{弟}{おとゝ}の
\ruby{{\換字{若}}}{わか}き
\ruby{時}{とき}を
\ruby{徒}{あだ}に
\ruby{{\換字{過}}}{すご}さしめず、
%
\ruby{出來}{で|き}ぬながらも
\ruby{人}{ひと}の
\ruby{後}{のち}に
\ruby{落}{お}ちぬほどには
\ruby{物學}{もの|まな}びを
させて、
%
\ruby{男兒}{をと|こ}
\ruby{一人{\換字{前}}}{いち|にん|まへ}には
\ruby{生}{おふ}し
\ruby{立}{た}て、
%
\ruby{我}{わ}が
\ruby{家}{いへ}の
\ruby{名}{な}をも
\ruby{擧}{あ}げさせん、
%
\ruby{弟}{おとゝ}のためには
\ruby{挿}{さ}したる
\ruby{掻頭}{かん|ざし}を
\ruby{賣}{う}り、
%
\ruby{着}{き}たる
\ruby{衣}{もの}を
\ruby{脫}{ぬ}ぐとも
\ruby{惜}{をし}まじとは、
%
\ruby{五十子}{い|そ|こ}が
\ruby{日頃}{ひ|ごろ}の
\ruby{念慮}{おも|ひ}なりき。

\原本頁{147-10}%
\ruby{秋風}{あき|かぜ}の
\ruby{中}{なか}に
\ruby{嬰兒}{あか|ご}の
\ruby{泣}{な}きても、
%
\ruby{拾}{ひろ}ふ
\ruby{人}{ひと}は
\ruby{少}{すくな}き
\ruby{此}{こ}の
\ruby{冷}{つめた}き
\ruby{世}{よ}に、
%
\ruby{女}{をんな}なり、
%
\ruby{少年}{せう|ねん}なりの、
%
\ruby{孱{\換字{弱}}}{か|よわ}き
\ruby{身}{み}をもて、
%
\ruby{屈}{くつ}すること
\ruby{無}{な}く
\ruby{凜々}{り|ゝ}しくも
\原本頁{148-1}\改行%
\ruby{立}{た}てる、
%
\ruby{此}{こ}の
\ruby{姊}{あね}
% \ % 隙間調整
\ruby{此}{こ}の
\ruby[<j||]{弟}{おとゝ}の% ルビ調整(配置位置調整)ルビが重なるので離す
\ruby[<j>]{潔}{いさぎよ}くも
\ruby{健}{けなげ}なる
\ruby[||j>]{心}{こゝろ}
\ruby[||j>]{掛}{ がけ}は、
% \ruby{心掛}{こゝろ|がけ}は、
%
\ruby{同}{おな}じく
\ruby{{\換字{貧}}苦}{ひん|く}と
\ruby[<j||]{戰}{たゝか}ひ% 行末行頭の境界付近なので特例処置を施す
\ruby{來}{きた}れる
\ruby{水野}{みづ|の}が
\ruby{心}{こゝろ}を
\ruby{少}{すくな}からず
\ruby{動}{うご}かして、
%
\ruby{深}{ふか}くも
\ruby{五十子}{い|そ|こ}を
\ruby{思}{おも}ひ
\ruby{思}{おも}ひて
\ruby{忘}{わす}るゝ
\ruby{能}{あた}はざるに
\ruby{至}{いた}りし
\ruby{原因}{い|はれ}の
\ruby{中}{うち}の、
%
\ruby[||j>]{力}{ちから}
\ruby[||j>]{{\換字{強}}}{ づよ}き
% \ruby{力{\換字{強}}}{ちから|づよ}き
\ruby{一}{ひと}つの
\ruby{個條}{か|でう}とはなりぬ。

\原本頁{148-5}%
されば
\ruby{我}{わ}が
\ruby{五十子}{い|そ|こ}が
\ruby{身}{み}にも
\ruby{代}{か}へじと
\ruby{深}{ふか}くも
\ruby{愛}{いつく}しめりと
\ruby{思}{おも}ふにつけて、
%
\ruby{水野}{みづ|の}も
\ruby[<j||]{自}{おのづ}% 「松之助」が続くので、ここで前突き出ししておく
\ruby[||j>]{然}{から}
% \ruby{自然}{おのづ|から}
\ruby{松之助}{まつ|の|すけ}を
\ruby{他}{よそ}ならず
おもへば、
%
\ruby{松之助}{まつ|の|すけ}も
また
\ruby{水野}{みづ|の}を
\ruby{他}{よそ}ならず
\ruby{思}{おも}ひ、
%
\ruby{五十子}{い|そ|こ}が
\ruby{許}{もと}にて
\ruby{相}{あひ}
\ruby{識}{し}りてより、
%
\ruby{四五度}{し|ご|たび}も
\ruby[<j||]{面}{おもて}を% 行末行頭の境界付近なので特例処置を施す
\ruby{會}{あ}はせたるには
\ruby{{\換字{過}}}{す}ぎねど、
%
\ruby{姊}{あね}の
\ruby{如何}{い|か}なる
\ruby{故}{ゆゑ}にか
\ruby{我}{われ}を
\ruby{好}{この}まざるに
\ruby{似}{に}ず、
%
\ruby{此兒}{こ|れ}は
\ruby{可愛}{か|はい}くも
\ruby{我}{われ}に
\ruby{睦}{むつ}みて、
%
\ruby{我}{われ}を
\ruby{眞}{まこと}の
\ruby{兄}{あに}
なんどの
\ruby{如}{ごと}くに
あしらひ、
%
\ruby{隔意}{へだて|ぎ}も
\ruby{無}{な}く
\ruby{打解}{うち|と}けて
\ruby{語}{かた}らふなり。

\原本頁{148-11}%
\ruby{我}{わ}が
\ruby{思}{おも}ふ
\ruby{人}{ひと}の
\ruby{弟}{おとゝ}と
\ruby{思}{おも}はんには、
%
たとひ
\ruby{色}{いろ}
\ruby{黑}{くろ}く
\ruby{醜}{みにく}くとも、
%
\ruby{{\換字{猶}}}{なほ}
\ruby{厭}{いと}は
\原本頁{149-1}\改行%
しき
\ruby{兒}{こ}とは
\ruby{見棄}{み|す}てざらんに、
%
まして
これは
\ruby{玉}{たま}の
\ruby{如}{ごと}く
\ruby{美}{うつく}しくして
%
\原本頁{149-2}\改行%
\ruby{加之}{しか|も}
\ruby{我}{われ}に
\ruby{親}{したし}めるなり、
%
\ruby{今}{いま}
\ruby{其}{そ}の
\ruby{淸}{すゞ}しき
\ruby{眼}{め}を
\ruby{見張}{み|は}りて
\ruby{懷}{なつか}しげに
\ruby{我}{われ}を
\ruby{見}{み}ながら、

\原本頁{149-4}%
『
\ruby{君}{きみ}!、
%
\ruby{書狀}{てが|み}を
\ruby{有}{あ}り
\ruby{{\換字{難}}}{がた}う!。
%
\ruby{毫}{ちつと}も
\ruby{知}{し}らなかつた。
%
\ruby{僕}{ぼく}あ
\ruby{彼狀}{あ|れ}を
\ruby{見}{み}て
\ruby{吃驚}{びつ|くり}した!。
%
\ruby{郵便}{いう|びん}が
\ruby{昨夜}{ゆふ|べ}
\ruby{夜中}{よ|なか}に
\ruby{着}{つ}いたから、
%
それから
\ruby{今{\換字{朝}}}{け|さ}
\ruby{暗}{くら}
\原本頁{149-6}\改行%
い
\ruby{中}{うち}に
\ruby{飛}{とん}で
\ruby{出}{で}て
\ruby{來}{き}たんだ。
%
\ruby{姊}{ねえ}さんは
\ruby{何樣}{ど|ん}なだね、
%
エ、
%
\ruby{惡}{わる}いか?
\改行% 校正作業の簡略化のため
、
%
エヽエ。
』

\原本頁{149-8}%
と、
%
\ruby{我}{われ}を
\ruby{一家}{いつ|け}の
\ruby{人}{ひと}か
なんぞのやうに
\ruby[||j>]{心}{こゝろ}
\ruby[||j>]{易}{ やす}く
% \ruby{心易}{こゝろ|やす}く
\ruby{思}{おも}へる
\ruby{言葉}{こと|ば}つきの
\ruby{修{\換字{飾}}無}{かざ|り|な}く、
%
\ruby{姊}{あね}を
\ruby{思}{おも}へる
\ruby{{\換字{情}}}{こゝろ}の
\ruby{溢}{あふ}るゝ
ばかりに、
%
\ruby{取}{と}り
\ruby{繕}{つくろ}ひ
\ruby{氣}{げ}
\ruby{無}{な}く
\ruby{忙}{せは}しく
\ruby{問}{と}ふを
\ruby{見}{み}ては、
%
\ruby{今}{いま}まで
\ruby{胸}{むね}の
\ruby{中}{うち}に
もや〳〵としたる
\ruby{一切}{いつ|さい}の
\ruby{不快}{ふ|くわい}さ
\ruby{忌}{いま}はしさも、
%
\ruby{{\換字{朝}}日}{あさ|ひ}に
あひて
\ruby[||j>]{霜}{しも}
\ruby[||j>]{柱}{ばしら}の
% \ruby{霜柱}{しも|ばしら}の
\ruby{嵯牙}{さ|が}として
\ruby{立}{た}てるも
\ruby{忽}{たちま}ちに
\原本頁{150-1}\改行%
\ruby{摧}{くだ}き
\ruby{融}{と}かさるゝ
\ruby{心地}{こゝ|ち}して、
%
\ruby{水野}{みづ|の}は
\ruby{思}{おも}はずも
\ruby{其}{その}
\ruby{手}{て}を
\ruby{執}{と}りて、
%
\ruby{正}{たゞ}しく
\ruby{答}{こた}ふるよりは
\ruby{先}{まづ}
\ruby{一句}{いつ|く}、

\原本頁{150-3}%
『
マア
\ruby{安心}{あん|しん}したまへ。
』

\原本頁{150-4}%
と
\ruby{慰}{なぐさ}めたり。

\Entry{其二十五}

% メモ 校正 2024-04-09
\原本頁{150-6}%
\ruby{氣{\換字{遣}}}{き|づか}はしさに
\ruby{堪}{た}へねばこそ
\ruby{知}{し}らず
\ruby{識}{し}らず
\ruby{大悲}{だい|ひ}の
\ruby{御誓願}{おん|ちか|ひ}を
\ruby{頼}{たの}みて、
%
\原本頁{150-7}\改行%
その
\ruby{爲}{ため}に
\ruby{書生}{しよ|せい}の
\ruby{嘲笑}{あざ|けり}をも
\ruby{受}{う}くるに
\ruby{至}{いた}りたるなれ、
%
それを
\ruby{今}{いま}
\ruby{此}{こ}の
\原本頁{150-8}\改行%
\ruby{少年}{せう|ねん}の
\ruby{姊}{あね}を
\ruby{思}{おも}ふ
\ruby{心根}{こゝろ|ね}の
いぢらしきとて、
%
\ruby{先}{ま}づ
\ruby{安心}{あん|しん}したまへと
\ruby[g]{眞實}{まこと}にもあらぬ
\ruby{氣休}{き|やす}めを
\ruby{云}{い}ひたるは
\ruby{何}{なん}の
\ruby{心}{こゝろ}ぞや、
%
\ruby{自}{みづか}ら
\ruby{欺}{あざむ}き
\ruby{人}{ひと}を
\ruby{欺}{あざむ}くとは
\ruby{此}{こ}の
\ruby{事}{こと}なりと、
%
\ruby[g]{水野}{みづの}は
はツと
\ruby{思}{おも}ひしかど、
%
\ruby{既}{すで}に
\ruby{口}{くち}を
すべらせたれば
\ruby{駟}{し}も
\ruby{及}{およ}ばず、
%
たゞ
\ruby[g]{四ッ木}{よ ぎ}に% TODO 四ツ木
\ruby{着}{つ}きても
\ruby[g]{松之助}{まつのすけ}が
\ruby{驚}{おどろ}く
\ruby{事}{こと}などの
\ruby{無}{な}からんをば、
%
\ruby{今{\換字{更}}}{いま|さら}
\ruby{{\換字{又}}}{また}
ひそかに
\ruby{切}{せつ}に
\ruby{念}{ねん}じたり。

\原本頁{151-3}%
\ruby[g]{松之助}{まつのすけ}は
\ruby{嬉}{うれ}しげに
\ruby[g]{水野}{みづの}を
\ruby{見}{み}て、

\原本頁{151-4}%
『では
\ruby{其樣}{そん|な}に
\ruby{甚}{ひど}くは
\ruby{無}{な}いの?、
%
あゝ
\ruby{有難}{あり|がた}かつた!。
%
\ruby{僕}{ぼく}は
\ruby{何}{ど}の
\ruby[<j|]{位}{くらゐ}
\ruby{心配}{しん|ぱい}したか
\ruby{知}{し}れない。
%
\ruby{併}{\換字{志}か}し
\ruby{{\換字{平}}常}{た|ゞ}の
\ruby{風邪}{か|ぜ}では
\ruby{無}{な}いやうだつて、
%
\ruby{何病}{なに|びやう}だつたの?。
』

\原本頁{151-7}%
と、
%
\ruby{人}{ひと}の
\ruby{一句}{いつ|く}を
\ruby{直}{たゞち}に
\ruby{信}{しん}じて
\ruby{無邪氣}{む|じや|き}に
\ruby{悅}{よろこ}べる
さまの
\ruby{罪}{つみ}なさは、
%
\ruby{却}{かへ}つて
\ruby[g]{水野}{みづの}の
\ruby{眼}{め}に
\ruby{憫然}{あは|れ}に
\ruby{見}{み}えたり。

\原本頁{151-9}%
『
\ruby{病氣}{びやう|き}は
\ruby[g]{腸窒扶斯}{ちやうちぶす}といふ
\ruby{事}{こと}で、
%
なか〳〵
\ruby{輕}{かる}くは
\ruby{無}{な}い
\ruby{病患}{わづ|らひ}なのだよ。
%
\換字{志}かし
\ruby{醫師}{い|し}も
\ruby{信用}{しん|よう}の
\ruby{出來}{で|き}る
\ruby{人}{ひと}を
\ruby{頼}{たの}み、
%
\ruby{看護{\換字{婦}}}{かん|ご|ふ}も
\ruby{今日}{け|ふ}から
\ruby{來}{く}る
\ruby{手筈}{て|はず}に
なつて
\ruby{居}{ゐ}るから、
%
\ruby{決}{けつ}して
\ruby{無益}{む|えき}の
\ruby{心配}{しん|ぱい}は
\ruby{仕玉}{し|たま}ふな。
%
まあ
\ruby{大{\換字{丈}}夫}{だい|ぢやう|ぶ}だと
\ruby{僕}{ぼく}はおもふ。
』

\原本頁{152-2}%
『ナニ
\ruby[g]{窒扶斯}{ちぶす}だつて!。
%
\ruby{困}{こま}つたナア、
%
アヽ
\ruby{其}{そ}りやあ
\ruby{大變}{たい|へん}だ、
%
\ruby{大變}{たい|へん}だ!。
%
アヽ
\ruby{僕}{ぼく}あ
\ruby{何樣}{ど|う}したら
\ruby{好}{い}いんだらう!。
%
\ruby{左樣}{そ|う}して
\ruby{醫者}{い|しや}だの
\ruby{何}{なん}ぞは
\ruby{誰}{たれ}が
\ruby{仕}{し}て
\ruby{吳}{く}れたの?。
%
\ruby{姊}{ねえ}さんに
\ruby{其}{それ}だけの
\ruby{事}{こと}が
\ruby{自{\換字{分}}}{じ|ぶん}で
\ruby{出來}{で|き}たの?。
%
\ruby{姊}{ねえ}さんにやあ
\ruby{其樣}{そ|ん}な
\ruby{事}{こと}の
\ruby{出來}{で|き}さうも
\ruby{無}{な}いナア
\ruby{僕}{ぼく}が
\ruby{知}{し}つて
\ruby{居}{ゐ}る。
%
\ruby{誰}{だれ}が
\ruby{仕}{し}て
\ruby{吳}{く}れたの?。
%
\ruby{君}{きみ}が
\ruby{親切}{しん|せつ}に?。
』

\原本頁{152-7}%
\ruby{何}{なに}と
\ruby{無}{な}く
\ruby{感}{かん}じて
\ruby{知}{し}れる
\ruby{歟}{か}
\ruby{兒童心}{こ|ども|ごゝろ}の
\ruby{敏}{さと}くも、
%
はや
\ruby{眼}{め}の
\ruby{中}{うち}は
\ruby{涙}{なみだ}ぐみて、
%
\ruby{泣}{な}き
\ruby{出}{だ}さん
ばかりの
\ruby{顏}{かほ}つきの
\ruby{正直}{しやう|ぢき}にも、
%
\ruby{其}{そ}の
\ruby{然}{しか}りとの
\ruby{一語}{いち|ご}を
\ruby{聞}{き}きて
\ruby{直}{たゞち}に
\ruby{謝}{しや}せんと、
%
\ruby{待}{ま}ち
\ruby{設}{まう}けたる
\ruby{意中}{い|ちゆう}は
あり〳〵と
\ruby{見}{み}えぬ。
%
\原本頁{152-10}\改行%
\ruby[g]{水野}{みづの}は
\ruby{自己}{お|の}が
\ruby{此度}{こ|たび}の
\ruby{振舞}{ふる|まひ}の、
%
\ruby{恩}{おん}を
\ruby{賣}{う}るやうに
\ruby{取}{と}られん
\ruby{事}{こと}を
\ruby{心苦}{こゝろ|ぐる}しく
\ruby{思}{おも}ひ
\ruby{居}{ゐ}たれば、
%
\ruby{彼}{か}の
お
\ruby{澤}{さは}
\ruby{婆}{ばゞ}に
\ruby{對}{むか}ひて
\ruby{云}{い}ひ
\ruby{置}{お}ける
おもむきを、
%
\原本頁{153-1}\改行%
\ruby{{\換字{飽}}}{あ}くまで
\ruby{徹}{とほ}さんと
\ruby{思}{おも}へるなり。

\原本頁{153-2}%
『イヽエ。
』

\原本頁{153-3}%
\ruby{思}{おも}ひの
\ruby{外}{ほか}なる
\ruby[g]{水野}{みづの}が
\ruby{答}{こたへ}に
\ruby[g]{松之助}{まつのすけ}は
\ruby{合點}{が|てん}
\ruby{行}{ゆ}かぬ
ところあり。

\原本頁{153-4}%
『ぢやあ
\ruby{誰}{たれ}が
\ruby{仕}{し}て
\ruby{吳}{く}れたの?。
』

\原本頁{153-5}%
『
\ruby{學校}{がく|かう}の
\ruby{人}{ひと}たちが。
』

\原本頁{153-6}%
『
\ruby{君}{きみ}だの
\ruby{校長}{かう|ちやう}さんだのが?。
』

\原本頁{153-7}%
『マアそんなものだと
\ruby{思}{おも}つて
\ruby{居}{ゐ}たまへ。
』

\原本頁{153-8}%
『ア、
%
それぢやあ
\ruby{矢張}{やつ|ぱ}り
\ruby{君}{きみ}の
\ruby{親切}{しん|せつ}なんだ、
%
きつと
\ruby{左樣}{さ|う}に
\ruby{{\換字{違}}無}{ちがひ|な}い、
%
\原本頁{153-9}\改行%
\ruby{僕}{ぼく}は
\ruby{知}{し}つてゐる!。
%
ほんたうに
\ruby{君}{きみ}
\ruby{有}{あ}り
\ruby{難}{がた}う!。
%
\ruby{僕}{ぼく}あ
\ruby{一生}{いつ|しやう}
おぼえて
\ruby{居}{ゐ}る!。
』

\原本頁{153-11}%
\ruby{淡泊}{たん|ぱく}にも
\ruby{頭}{かうべ}を
\ruby{下}{さ}げて
\換字{志}み〴〵と
\ruby{恩}{おん}を
\ruby{謝}{しや}せる
\ruby[g]{松之助}{まつのすけ}が
\ruby{心}{こゝろ}は
\ruby{其}{そ}の
\ruby{手}{て}に
\ruby{籠}{こも}りて、
%
\ruby[g]{水野}{みづの}は
\ruby{我}{わ}が
\ruby{手}{て}の
\ruby{緊}{きび}しく
\ruby{握}{にぎ}られたるを
\ruby{感}{かん}じぬ。
%
\ruby{談話}{はな|し}は
\原本頁{154-2}\改行%
\ruby{一}{ひ}ト
\ruby{先}{まづ}
\ruby{{\換字{終}}}{をは}りけるが、
%
\ruby{問答}{もん|だふ}は
\ruby{{\換字{又}}}{また}
\ruby{突}{とつ}として
\ruby{起}{おこ}りぬ。

\原本頁{154-3}%
『
\ruby{君}{きみ}は
こんなに
\ruby{夙}{はや}く
\ruby{何處}{ど|こ}へ
\ruby{行}{い}つたの?。
』

\原本頁{154-4}%
『
\ruby{少}{すこ}しばかり
\ruby{用}{よう}があつて
\ruby{出}{で}たんだが、
%
もう
\ruby{歸路}{かへ|り}なのだ。
』

\原本頁{154-5}%
『
\ruby{其}{そ}の
\ruby{次}{ついで}に
\ruby{觀音樣}{くわん|のん|さま}へ% 「觀音」の読みは原本通り「くわん(の)ん」
\ruby{詣}{まゐ}つたのかエ?。
』

\原本頁{154-6}%
『ムヽ。
』

\原本頁{154-7}%
『
\ruby{觀音樣}{くわん|のん|さま}に% 「觀音」の読みは原本通り「くわん(の)ん」
\ruby{何}{なん}の
\ruby{用}{よう}があつて?。
もし
\ruby{願}{ねが}ひ
\ruby{事}{ごと}でも
\ruby{爲}{し}て?。
』

\原本頁{154-8}%
『ムヽ。
』

\原本頁{154-9}%
『
\ruby{虛言}{う|そ}だらう。
%
そりやあ
\ruby{可笑}{を|か}しいナア、
%
ハヽ。
』

\原本頁{154-10}%
『
\ruby{何故}{な|ぜ}そんなに
\ruby{君}{きみ}にやあ
\ruby{可笑}{を|か}しいのかね?。
』

\原本頁{154-11}%
『だつて
\ruby{君}{きみ}、
%
\ruby{君}{きみ}は
いつか
\ruby{僕}{ぼく}に
\ruby{敎}{をし}へたぢやあ
\ruby{無}{な}いか。
%
ホラ、
%
\ruby{此}{こ}の
\原本頁{155-1}\改行%
\ruby{觀音}{くわん|のん}といふ% 「觀音」の読みは原本通り「くわん(の)ん」
\ruby{人}{ひと}は
\ruby{聞}{き}いて
\ruby{思}{おも}つて
\ruby{修}{をさ}めるといふ
\ruby{三}{みつ}つの
\ruby{學問}{がく|もん}の
\ruby{法則}{はふ|そく}を、
%
\ruby{敎}{をし}へて
\ruby{{\換字{遺}}}{のこ}した
\ruby{人}{ひと}なので、
%
\ruby{敬}{けい}すべき
\ruby{人}{ひと}には
\ruby{{\換字{違}}無}{ちがひ|な}いが、
%
\ruby{福}{ふく}を
\ruby{與}{あた}へるものなんぞとして
\ruby{拜}{をが}むのは、
%
\ruby{感心}{かん|しん}の
\ruby{出來}{で|き}ない
\ruby{卑}{いや}しい
\ruby{事}{こと}だと、
%
\原本頁{155-4}\改行%
\ruby{僕}{ぼく}が
\ruby{{\換字{習}}慣}{く|せ}でもつて
\ruby{拜}{をが}まうとしたら、
%
\ruby{敎}{をし}へて
\ruby{吳}{く}れた
\ruby{事}{こと}が
あつたもの!。
%
その
\ruby{君}{きみ}が
\ruby{願}{ねが}ひ
\ruby{事}{ごと}なんぞ
\ruby{仕}{し}やう
\ruby{譯}{わけ}は
\ruby{無}{な}いもの!。
』

\原本頁{155-6}%
\ruby{實}{げ}に
\ruby{嘗}{かつ}て
\ruby{此}{こ}の
\ruby{少年}{せう|ねん}が
\ruby[g]{四ッ木}{よ ぎ}よりの% TODO 四ツ木
\ruby{歸}{かへ}るさを
\ruby{{\換字{送}}}{おく}りがてら、
%
\ruby{共}{とも}に
\ruby{心}{こゝろ}
たのしく
\ruby{{\換字{遊}}}{あそ}び
あるきつゝ
\ruby{此處}{こ|ゝ}に
\ruby{來}{きた}りし
\ruby{時}{とき}、
%
\ruby{生}{なま}さかしくも
\ruby{然}{さ}る
\ruby{事}{こと}を
\ruby{說}{と}きて、
%
\ruby{幸福}{しあ|はせ}を%「幸福」ここは「は」
\ruby{得}{え}んとて
\ruby{佛}{ほとけ}を
\ruby{拜}{をが}む
\ruby{世}{よ}の
\ruby{人}{ひと}の
\ruby{心}{こゝろ}の
\ruby{卑}{いや}しさを
\ruby{笑}{わら}ひし
\ruby{事}{こと}ありしを、
%
\ruby{端}{はし}
\ruby{無}{な}くも
\ruby{今}{いま}
\ruby{云}{い}ひ
\ruby{出}{いだ}されて
\ruby{想}{おも}ひ
\ruby{起}{おこ}せば、
%
\ruby{{\換字{又}}}{また}
\ruby{新}{あらた}に
\ruby{毒箭}{どく|や}を
\ruby{胸板}{むな|いた}に
\ruby{射}{い}
\ruby{立}{た}てられし
\ruby{心地}{こゝ|ち}して、
%
\ruby{堪}{た}へがたき
\ruby{不快}{ふ|くわい}さを
\ruby{再度}{ふた|ゝび}
\ruby{覺}{おぼ}えつ。
%
おもへば
\ruby{其}{それ}のみには
あらざりし、
%
はじめて
\ruby{東京}{とう|きやう}にて
\ruby[g]{羽{\換字{勝}}}{はがち}
\原本頁{156-1}\改行%
\ruby[g]{島木}{しまき}
\ruby{等}{ら}
\ruby{七人}{なな|にん}
\ruby{打揃}{うち|そろ}ひて、
%
\ruby{詣}{まゐ}るとも
\ruby{無}{な}く
\ruby{此}{こ}の
\ruby{御堂}{み|だう}に
\ruby{參}{まゐ}りし
\ruby{折}{をり}、
%
\ruby[g]{島木}{しまき}と
\ruby[g]{楢井}{ならい}と
\ruby[g]{羽{\換字{勝}}}{はがち}とは
\ruby{手}{て}を
\ruby{合}{あは}せて
\ruby{拜}{をが}み、
%
\ruby[g]{日方}{ひかた}と
\ruby[g]{山瀬}{やませ}と
\ruby[g]{名倉}{なぐら}とは
\ruby{三人}{さん|にん}を
\ruby{冷笑}{あざ|わら}ひしに、
%
おのれは
\ruby{拜}{をが}みもせねば
\ruby{冷笑}{あざ|わら}ひもせで、
%
\ruby{我}{われ}は
たゞ
\ruby{{\換字{古}}}{いにしへ}の
\ruby{賢人}{けん|じん}として
\ruby{大士}{だい|し}を
\ruby{待}{ま}たんと
\ruby{思}{おも}ふなりとて、
%
たゞ
\ruby{帽}{ばう}を
\ruby{脫}{ぬ}ぎて
\ruby{一禮}{いち|れい}したりし
\ruby{{\換字{古}}}{ふる}き
\ruby{事}{こと}まで
\ruby{心}{こゝろ}に
\ruby{{\換字{浮}}}{うか}べば、
%
\ruby{一腔}{いつ|こう}の
\ruby{中}{うち}は
\ruby{火}{ひ}の
\ruby{散}{ち}る
\ruby{如}{ごと}くに
\ruby{羞惡}{しう|を}の
\ruby[<j|]{{\換字{情}}}{こゝろ}
\ruby{燃}{も}え
\ruby{立}{た}つて、
%
\ruby{菩薩}{ぼ|さつ}の
\ruby{大威力}{だい|ゐ|りき}を
\ruby{假}{か}りたき
\ruby{念}{おもひ}は
\ruby{今}{いま}
\ruby{{\換字{猶}}}{なほ}
こゝに
ありながら、
%
\ruby{今}{いま}
こゝに
\ruby{我}{われ}を
\ruby{卑}{いや}しくして、
%
\ruby{世}{よ}の
\ruby{人並}{ひと|な}みに
\ruby{菩薩}{ぼ|さつ}を
\ruby{拜}{をが}みしを
\ruby{口惜}{くち|をし}く
おもふが
\ruby{如}{ごと}き
\ruby{感}{かん}じも
\ruby{起}{おこ}りて、
%
\ruby{不安}{ふ|あん}の
\ruby{色}{いろ}の
\ruby{面}{おもて}に
\ruby{出}{い}づらんを
\ruby{制}{せん}せんとして% 原本通り「制」を「せん」とした
\ruby{制}{せい}しがたきを
\ruby{覺}{おぼ}えたり。

\原本頁{156-10}%
『ハヽヽ、
%
そんな
\ruby{事}{こと}を
\ruby{云}{い}つた
\ruby{事}{こと}も
\ruby{成程}{なる|ほど}
\ruby{有}{あ}つた。
』

\原本頁{156-11}%
\ruby{辛}{から}くも
\ruby{自}{みづか}ら
\ruby{克}{か}つて
\ruby{塞}{ふさ}がる
\ruby{胸}{むね}より
\ruby{答}{こた}へ
\ruby{得}{え}たるは、
%
\ruby{全}{まつた}き
\ruby{意味}{い|み}も
\ruby{無}{な}き
\原本頁{157-1}\改行%
\ruby{言葉}{こと|ば}なり。

\原本頁{157-2}%
『さうして
\ruby{君}{きみ}は
\ruby{何}{なに}を
\ruby{願}{ねが}つたの?。
』

\原本頁{157-3}%
\ruby{心}{こゝろ}
\ruby{無}{な}く
\ruby{放}{はな}つ
\ruby{少年}{せう|ねん}の
\ruby{箭}{や}は、
%
またもや
\ruby[g]{水野}{みづの}が
\ruby{心窩}{む|ね}の
\ruby{眞正中}{まつ|たゞ|なか}に
\ruby{立}{た}ちぬ。
%
\原本頁{157-4}\改行%
されど
\ruby[g]{水野}{みづの}は
\ruby{痛手}{いた|で}を
\ruby{外}{よそ}にして、

\原本頁{157-5}%
『
\ruby{何}{なん}でも
\ruby{可}{い}いから
\ruby{急}{いそ}いで
\ruby{行}{ゆ}かう。
』

\原本頁{157-6}%
と、
%
\ruby[g]{松之助}{まつのすけ}と
\ruby{共}{とも}に
\ruby[g]{四ッ木}{よ ぎ}へと% TODO 四ツ木
\ruby{志}{こゝろざ}し、
%
\ruby{人}{ひと}の
\ruby{{\換字{運}}命}{う|ん}、
%
\ruby{我}{わ}が
\ruby{{\換字{運}}命}{う|ん}の
\ruby{測}{はか}り
\原本頁{157-7}\改行%
\ruby{難}{がた}き
\ruby{{\換字{前}}{\換字{途}}}{ゆく|て}を
\ruby{見}{み}んと、
%
\ruby{心}{こゝろ}に
\ruby{幾枝}{いく|し}の
\ruby{箭}{や}を
\ruby{負}{お}ひながら、
%
\ruby{路}{みち}を
\ruby{急}{いそ}ぎて
\ruby{歩}{あゆ}み
\ruby{出}{いだ}しぬ。

\原本頁{157-9}%
\ruby{此}{こ}の
\ruby{時}{とき}
\ruby{日}{ひ}は
\ruby{漸}{やうや}く
\ruby{昇}{のぼ}ると
\ruby{共}{とも}に、
%
\ruby{狂風滾々}{きやう|ふう|こん|〳〵}と
\ruby{吹}{ふ}き
\ruby{出}{いだ}して、
%
\ruby{美}{うるは}しかりし
\ruby{{\換字{空}}}{そら}は
\ruby{何時}{い|つ}と
\ruby{無}{な}く
\ruby{黄}{き}ばみ、
%
\ruby[g]{暴風雨日}{あれび}
\ruby{{\換字{近}}}{ちか}き
\ruby{天}{てん}に
\ruby{氣味}{き|み}
あしき
\ruby{雲}{くも}の
おだやかならず
\ruby{湧}{わ}き
ひろごりて、
%
\ruby{昨夜}{ゆふ|べ}に
\ruby{變}{かは}れる
\ruby{今日}{け|ふ}の
\ruby{狀態}{やう|す}の、
%
そぞろに
\ruby{定}{さだ}め
\ruby{無}{な}き
\ruby{人間}{ひ|と}の
\ruby{上}{うへ}を
\ruby{示}{しめ}すが
\ruby{如}{ごと}く、
%
\ruby{首}{かうべ}を
\ruby{傾}{かたむ}けて
\ruby{{\換字{進}}}{すゝ}む
\ruby[g]{水野}{みづの}と
\原本頁{158-2}\改行%
\ruby[g]{松之助}{まつのすけ}との
\ruby{眞向}{まつ|かう}に
\ruby{烈}{はげ}しく
\ruby{當}{あた}る
\ruby{風}{かぜ}は、
%
\ruby{二人}{ふた|り}が
\ruby{心臓}{む|ね}をして
\ruby{騷}{さわ}ぎに
\ruby{騷}{さわ}がしめぬ。

\Entry{其二十六}

\ruby{語}{かた}りつゞけたる
\ruby[g]{談話}{はなし}の
\ruby{間}{うち}、
\ruby{息}{いき}つぎ〳〵にわれ
\ruby{知}{し}らず
\ruby{{\換字{飲}}}{の}みし
\ruby{葡萄酒}{ぶ|だう|しゆ}の
\ruby{量}{りやう}の
\ruby{少}{すくな}からで、
\ruby{既}{すで}に
\ruby{其}{そ}の
\ruby{六七分}{ろく|しち|ぶ}を
\ruby{盡}{つく}したれば、
\ruby{醉興}{すゐ|きよう}おのづから
\ruby{發}{はつ}して
\ruby{獨}{ひと}り
\ruby{機嫌}{き|げん}よく、
\ruby{不規律}{ふ|き|りつ}の
\ruby{大將}{たい|しやう}をもて
\ruby{自}{みづか}ら
\ruby{許}{ゆる}せるほどありて、ふたゝび
\ruby{睡}{ねむ}りには
\ruby{就}{つ}かんともせず、
\ruby{島木}{しま|き}は
\ruby{{\換字{猶}}}{なほ}ぐびりぐびりと
\ruby{獨酌}{どく|しやく}を
\ruby{續}{つゞ}けたり。

むつくりと
\ruby{肥}{こ}えたる
\ruby[g]{身體}{からだ}ゆたかに
\ruby[g]{胡坐}{あぐら}をかきて、
\ruby{土多}{つち|おほ}き
\ruby{山}{やま}の
\ruby{岩}{いは}を
\ruby{{\換字{隠}}}{かく}せるが
\ruby{如}{ごと}くに、
\ruby{肉}{にく}ふくらかにして
\ruby{骨}{ほね}を
\ruby{見}{み}せぬ
\ruby{丸々}{まる|〳〵}としたる
\ruby{顏}{かほ}の、
\ruby{其}{そ}の
\ruby{小}{ちい}さなる
\ruby{眼}{め}のあたりに
\ruby{笑}{えみ}を
\ruby{含}{ふく}み、
\ruby{今}{いま}しもぐつと
\ruby{一盞}{いつ|さん}を
\ruby{仰}{あふ}ぎたるが、

『もう
\ruby{出}{で}て
\ruby{來}{き}さうなものだがナ、
\ruby{畜生}{ちく|しやう}!、まだかナ。
』

と、
\ruby{誰}{たれ}に
\ruby{云}{い}へるともなく
\ruby{自}{みづか}ら
\ruby{語}{かた}れり。

\ruby{島木}{しま|き}は
\ruby{水野}{みづ|の}が
\ruby{胸中}{む|ね}を
\ruby{知}{し}りたれど、
\ruby{水野}{みづ|の}は
\ruby{島木}{しま|き}が
\ruby{肚裏}{は|ら}を
\ruby{知}{し}らざりき。
\ruby{妻子兄弟}{さい|し|きやう|だい}も
\ruby{無}{な}く
\ruby{親}{おや}も
\ruby{無}{な}ければ、
\ruby{氣}{き}まゝなる
\ruby{寄寓}{かり|ずみ}の
\ruby{面倒無}{めん|だう|な}きを
\ruby{{\換字{悅}}}{よろこ}びて、
\ruby{一家}{いつ|か}をこそは
\ruby{{\換字{猶}}}{なほ}
\ruby{構}{かま}へざれ、
\ruby{幾度}{いく|たび}か
\ruby{{\換字{浮}}}{う}き
\ruby{幾度}{いく|たび}か
\ruby{沈}{しづ}みし
\ruby{末}{すゑ}に、
\ruby{漸}{やうや}く
\ruby{合百}{がふ|ひやく}の
\ruby{果敢無}{は|か|な}きより、
\ruby{今}{いま}は
\ruby{人}{ひと}の
\ruby{噂}{うはさ}にも
\ruby{上}{のぼ}るほどの
\ruby{玉高}{ぎよく|だか}を
\ruby{動}{うご}かすに
\ruby{至}{いた}りし
\ruby{島木}{しま|き}も、もとより
\ruby{右}{みぎ}は
\ruby{地獄左}{ぢ|ごく|ひだり}は
\ruby{極樂}{ごく|らく}の
\ruby{間}{あひだ}の
\ruby{綱}{つな}を
\ruby{渡}{わた}つて
\ruby{日}{ひ}を
\ruby{{\換字{送}}}{おく}る
\ruby{投機師}{とう|き|し}の
\ruby{身}{み}の
\ruby{上}{うへ}は、
\ruby{貨物}{くわ|ぶつ}を
\ruby{積}{つ}み
\ruby{問屋}{とひ|や}を
\ruby{控}{ひか}へて
\ruby{十}{じう}の
\ruby{一}{いち}
\ruby{十}{じう}の
\ruby{二}{に}の
\ruby{利}{り}を
\ruby{征}{と}りて
\ruby{行}{ゆ}く
\ruby{堅氣}{かた|ぎ}の
\ruby{商人}{あき|うど}とは
\ruby{異}{こと}なれば、
\ruby{此處一}{こ|こ|ひ}ト
\ruby{伸}{のし}と
\ruby{有}{あ}らん
\ruby{限}{かぎ}りの
\ruby{力瘤}{ちから|こぶ}を
\ruby{入}{い}れて
\ruby{蒐}{かゝ}れる
\ruby{此}{こ}の
\ruby{秋}{あき}の、
\ruby{天候}{てん|こう}を
\ruby{重}{おも}なる
\ruby{相場}{さう|ば}の
\ruby{時季}{と|き}に、
\ruby{捉}{とら}へかねたる
\ruby{雲}{くも}の
\ruby{心風}{こゝろ|かぜ}の
\ruby{料簡}{れう|けん}は
\ruby{我}{わ}が
\ruby{思}{おも}はくと
\ruby{{\換字{違}}}{ちが}ひて、
\ruby{{\換字{追}}敷}{おひ|じき}
\ruby[g]{々々}{〳〵}と
\ruby{取}{と}り
\ruby{立}{た}てらるゝに
\ruby{{\換字{懐}}中}{ふと|ころ }
\ruby{危}{あやふ}く、
\ruby{既}{すで}に
\ruby{其}{そ}の
\ruby{剩}{あま}すところは
\ruby{幾何}{いく|ばく}もあらぬ
\ruby{端錢}{はした|がね}となりて、
\ruby{{\換字{運}}}{うん}と
\ruby{志}{こゝろざし}との
\ruby{今少時}{いま|しば|し}
\ruby{反}{そむ}かば、またもや
\ruby{身}{み}の
\ruby{皮}{かは}も
\ruby{無}{な}き
\ruby[g]{赤裸々}{あかはだか}となりて、
\ruby{賽}{さい}の
\ruby{河原}{か|はら}に
\ruby{積}{つ}める
\ruby{石}{いし}の
\ruby[g]{瓦落離}{ぐわらり}と
\ruby{崩}{くづ}れたる
\ruby{{\換字{情}}無}{なさけ|な}さを
\ruby{見}{み}るべしと、
\ruby{流石}{さす|が}に
\ruby{心}{こゝろ}もおちつきかぬるところへ、
\ruby{折}{をり}も
\ruby{折}{をり}とて
\ruby{水野}{みづ|の}の
\ruby{無心}{む|しん}なり。
\ruby{{\換字{運}}}{うん}を
\ruby{背負}{せ|お}へる
\ruby{時}{とき}には
\ruby{其}{そ}の
\ruby{二倍三倍}{に|ばい|さん|ばい}も
\ruby{與}{あた}ふるに
\ruby{易}{やす}けれど、
\ruby{夜明}{よ|あ}けての
\ruby{天地}{てん|ち}の
\ruby{狀態次第}{やう|す|し|だい}にて
\ruby{我}{わ}が
\ruby[g]{生命}{いのち}はとさへ
\ruby{思}{おも}へる
\ruby{矢先}{や|さき}に
\ruby{云}{い}ひかけられては、
\ruby{敗軍}{まけ|いくさ}の
\ruby{{\換字{退}}}{ひ}き
\ruby{際}{ぎは}に
\ruby{頼}{たの}みきつたる
\ruby{持鎗}{もち|やり}を
\ruby{所望}{しよ|まう}されたる
\ruby{心地}{こゝ|ち}して、
\ruby{流石}{さす|が}の
\ruby{島木}{しま|き}も
\ruby{行}{ゆ}き
\ruby{詰}{つま}りしが、
\ruby{竹}{たけ}を
\ruby{割}{わ}つたる
\ruby{如}{ごと}き
\ruby{持前}{もち|まへ}の
\ruby{気象}{き|しやう}は
\ruby{義}{ぎ}を
\ruby{見}{み}て
\ruby{勇}{いさ}んで、エヽどうせ
\ruby{曲}{まが}つて
\ruby{仕舞}{し|ま}えば
\ruby{無}{な}くなる
\ruby{金}{かね}を、
\ruby{今}{いま}
\ruby{{\換字{遣}}}{や}つて
\ruby{仕舞}{し|ま}へば
\ruby{友{\換字{達}}}{とも|だち}の
\ruby{利{\換字{益}}}{た|め}!、
\ruby{踏張}{ふん|ば}れ〳〵
\ruby{男}{をとこ}の
\ruby{兒}{こ}だ、
\ruby[g]{裸々}{はだか}になつても
\ruby{怖}{こは}くは
\ruby{無}{な}い、
\ruby{百兩}{ひやく|りやう}ばかりの
\ruby{鼻糞金}{はな|くそ|がね}を
\ruby{出}{だ}し
\ruby{悋}{おし}んでは、
\ruby{萬五郎}{まん|ご|らう}の
\ruby{男}{をとこ}が
\ruby{廢}{す}たる!、
\ruby{{\換字{情}}無}{なさけ|な}い!、
\ruby{行末}{ゆく|すゑ}が
\ruby{見}{み}える!、
\ruby{百萬兩分限}{ひやく|まん|りやう|ぶ|げん}になった
\ruby{時}{とき}の
\ruby{額疵}{むかふ|きず}になる!、
\ruby{握}{にぎ}つた
\ruby{錢}{ぜに}から
\ruby{{\換字{煙}}}{けむ}を
\ruby{出}{だ}すのは
\ruby{三文野郎}{さん|もん|や|らう}のする
\ruby{事}{こと}だ、と
\ruby{早}{はや}くも
\ruby{決着}{けつ|ちやく}して
\ruby{臓腑}{ざう|ふ}を
\ruby{見}{み}せずに、
\ruby{奇麗}{き|れい}に
\ruby{快}{こゝろよ}く
\ruby{用立}{よう|だ}てて
\ruby{歸}{かへ}しやりつ、さて
\ruby{其}{それ}がためとにもあらざるべけれど、
\ruby{何}{なん}と
\ruby{無}{な}く
\ruby{心}{こゝろ}に
\ruby{怡悅}{よろ|こび}を
\ruby{覺}{おぼ}えて、
\ruby{今}{いま}は
\ruby{氣}{き}も
\ruby{冴}{さ}え〴〵と
\ruby{{\換字{飲}}}{の}み
\ruby{居}{を}れるなり。

『もう
\ruby{出}{で}て
\ruby{來}{き}さうなものだがナ、まだかナ、
\ruby{畜生}{ちく|しやう}!。
』

ふたゝび
\ruby{獨}{ひとり}りごちて
\ruby{酒盞}{さか|づき}を
\ruby{取}{と}りぬ。

『まだ
\ruby{出}{で}て
\ruby{來}{こ}ないかナ、
\ruby{畜生}{ちく|しやう}!。
』

\ruby{何}{なに}を
\ruby{待}{ま}てるにか
\ruby{三度獨語}{み|たび|ひと|りご}ちしが、
\ruby{答}{こた}ふるものは
\ruby{有}{あ}るべくも
\ruby{無}{な}く、
\ruby{室}{しつ}の
\ruby{一隅}{いち|ぐう}の
\ruby{小机}{こづ|くゑ}の
\ruby{上}{うへ}の
\ruby{{\換字{懐}}中時計}{くわい|ちゆう|ど|けい}の
\ruby{音}{おと}のみの
\ruby{有}{あ}るか
\ruby{無}{な}きかに
\ruby{響}{ひゝ}けり。

\ruby{相手無}{あひ|て|な}き
\ruby{淋}{さび}しさに
\ruby{堪}{た}へかねてか、

『
\ruby{畜生}{ちく|しやう}ッ、
\ruby{出}{で}て
\ruby{來}{き}やがらなくつても
\ruby{仕方}{し|かた}が
\ruby{無}{な}いかナ。
ハヽヽ、
\ruby{怒}{おこ}るほど
\ruby{乃公}{お|れ}も
\ruby{野暮}{や|ぼ}ぢやあいけねえ。
それはさうと
\ruby{水野}{みづ|の}はもう
\ruby{大分}{だい|ぶ }
\ruby{行}{い}つたらう。
\ruby{愍然}{かあい|さう}に、
\ruby{堅}{かた}い
\ruby{正直}{しやう|ぢき}な
\ruby{男}{をとこ}だから、
\ruby{人一倍何彼}{ひと|いち|ばい|なに|か}につけて
\ruby{物思}{もの|おもひ}を
\ruby{仕}{し}て
\ruby{居}{ゐ}る!。

\換字{庵点}
\ruby{粋}{すゐ}な
\ruby{{\換字{浮}}世}{うき|よ}を
\ruby{戀故}{こひ|ゆゑ}に、
\ruby{野暮}{や|ぼ}に
\ruby{暮}{くら}すも
\ruby{心}{こゝろ}がら。
あゝ
\ruby{端唄}{は|うた}の
\ruby{文句}{もん|く}ぢやあ
\ruby{無}{な}いが
\ruby{{\換字{迷}}}{まよ}つちやあ
\ruby{野暮}{や|ぼ}になる!。
フン、ナンダ
\ruby{此方}{こつ|ち}やあ
\ruby{戀故}{こひ|ゆゑ}ぢやあ
\ruby{無}{ね}えで、
\ruby{慾故}{よく|ゆえ}に
\ruby{野暮}{や|ぼ}になり
\ruby{切}{き}つて
\ruby{居}{ゐ}やがる!。
アヽもうそろ〳〵
\ruby{出}{で}て
\ruby{來}{き}て
\ruby{{\換字{呉}}}{く}れても
\ruby{好}{よ}さゝうなものだが、チヨツ
\ruby{忌々}{いま|〳〵}しい、\換字{志}れつたいナア。
ア、
\ruby{豪氣}{がう|ぎ}に
\ruby{醉}{よ}つて
\ruby{來}{き}た、
\ruby{好}{い}い
\ruby{心持}{こゝろ|もち}だ!。
何だかもう
\ruby{出}{で}て
\ruby{來}{き}さうな
\ruby{心持}{こゝろ|もち}がする!。
エヽト、

\換字{庵点}
\ruby{起}{お}きて
\ruby{見}{み}つ、
\ruby{寝}{ね}て
\ruby{見}{み}つ
\ruby{待}{ま}てど、たより
\ruby{無}{な}く、チン〳〵チンチン、
\ruby{蚊屋}{か|や}の
\ruby{廣}{ひろ}さにたゞ
\ruby{獨}{ひと}り、ツンテン、
\ruby{蚊}{か}を
\ruby{焼}{や}く
\ruby{火}{ひ}より
\ruby{胸}{むね}の
\ruby{火}{ひ}の、
\ruby{燃}{も}ゆるおもひを察しやんせカナ。
ハヽヽヽ。
』

\ruby{聲}{こゑ}は
\ruby{美}{うつく}しからず
\ruby{錆}{さ}びたれど、
\ruby{聞}{き}き
\ruby[g]{記憶}{おぼえ}なるべきには
\ruby{似合}{に|あ}はず
\ruby{我流}{が|りう}の
\ruby{節廻}{ふし|まは}しにもをかしきところありて、
\ruby{小聲}{こ|ごゑ}に
\ruby{唱}{うた}ひ
\ruby{仕舞}{し|ま}ひつゝ、
\ruby{今將}{いま|まさ}に
\ruby{一壜}{いち|びん}の
\ruby{酒}{さけ}を
\ruby{盡}{つく}し
\ruby{果}{は}たさんとして、
\ruby{手}{て}に
\ruby{取}{と}り
\ruby{上}{あ}げて
\ruby{自}{みづか}ら
\ruby{酌}{つ}がんと、
\ruby{其}{そ}の
\ruby{尻下}{しり|さが}がりの
\ruby{小}{ちいさ}き
\ruby{目}{め}を
\ruby{一}{ひ}トしほ
\ruby{下}{さ}げて、
\ruby{莞爾}{につ|こり}と
\ruby{樂}{たの}しげに
\ruby{笑}{わら}ひしが、
\ruby{何}{なに}をか
\ruby{聞}{き}きつけしや
\ruby{俄然}{が|ぜん}として、

『ヤツ、來たぞ! 來て
\ruby{{\換字{呉}}}{く}れたぞ!。
おいでなすったぞ!。
\ruby{占}{し}めたナ!、サア來いだ!。
』

と
\ruby{飛}{と}び
\ruby{立}{た}つたり。

投げ出されたる
\ruby{壜}{びん}は
\ruby{飜筋斗}{とん|ぼ|がへり}して、
\ruby{疊}{たゝみ}に
\ruby{溢}{こぼ}れたる
\ruby{紅色}{くれ|なゐ}の
\ruby{餘瀝}{した|ゝり}は、まだ
\ruby{早}{はや}き
\ruby{紅葉}{もみ|ぢ}をこゝに
\ruby{散}{ち}らしたり。


\Entry{其二十七}

がらりと
\ruby{樓}{ろう}の
\ruby[g]{雨戸}{あまど}を
\ruby{繰}{く}り
\ruby{開}{あ}くれば、
\ruby{白}{しら}みわたれる
\ruby{曉}{あかつき}の
\ruby{天}{そら}より、
\ruby{蓬々然}{ほう|〳〵|ぜん}として
\ruby{下}{おろ}し
\ruby{來}{く}る
\ruby{風}{かぜ}は、おもむろに
\ruby{面}{おもて}を
\ruby{撲}{う}ち
\ruby{胸}{むね}を
\ruby{撲}{う}つて、
\ruby{昨日}{きの|ふ}の
\ruby{夜}{よ}の
\ruby{靜穩}{おだ|やか}なりし
\ruby{俤}{おもかげ}は
\ruby{{\換字{猶}}}{なほ}
\ruby{{\換字{遺}}}{のこ}れども、
\ruby{日}{ひ}の
\ruby{將}{まさ}に
\ruby{出}{い}でんとする
\ruby{方}{かた}の
\ruby{雲}{くも}の
\ruby{色}{いろ}
\ruby{峻}{けは}しく、
\ruby{何}{なん}と
\ruby{無}{な}く
\ruby{物凄}{もの|すさ}まじき
\ruby[g]{景象}{やうす}は
\ruby{見}{み}る〳〵
\ruby{動}{うご}き
\ruby{展}{の}びて、やがて
\ruby{恐}{おそ}ろしくも
\ruby{一}{ひ}ト
\ruby{暴風}{あ|れ}の、
\ruby{暴}{あ}れ
\ruby{立}{た}たんとする
\ruby{勢}{いきほひ}は
\ruby{現}{あら}はれたり。

\ruby{昔語}{むかし|がたり}の
\ruby{海坊主}{うみ|ばう|ず}の
\ruby{如}{ごと}く、ヌツと
\ruby{突立}{つゝ|た}つたるまゝ
\ruby[g]{四邊}{あたり}を
\ruby{見{\換字{廻}}}{み|まは}せる
\ruby{島木}{しま|き}は、
\ruby{刻一刻}{こく|いつ|こく}に
\ruby{吹募}{ふき|つの}る
\ruby{風}{かぜ}の、
\ruby{袂}{たもと}を
\ruby{揚}{あ}げ
\ruby{裾}{すそ}を
\ruby{{\換字{扇}}}{あふ}るをも
\ruby{知}{し}らぬやうに、
\ruby{身}{み}じろぎもせずして
\ruby{居}{い}たりしが、
\ruby{{\換字{終}}}{つひ}には
\ruby{此}{こ}の
\ruby{風}{かぜ}の
\ruby{高}{こう}じに
\ruby{高}{こう}じて、
\ruby{老木}{おい|き}の
\ruby{枝}{えだ}を
\ruby{裂}{さ}き、
\ruby{若樹}{わか|ぎ}の
\ruby{根}{ね}を
\ruby{拔}{ぬ}き、
\ruby{沙}{すな}を
\ruby{舞}{ま}はせ
\ruby{石}{いし}を
\ruby{躍}{をど}らすに
\ruby{至}{いた}るべきさまの、
\ruby{十{\換字{分}}}{じう|ぶん}に
\ruby{想}{おも}ひやらるゝに
\ruby{及}{およ}びて、
\ruby{大浪}{おほ|なみ}のうねりて
\ruby{寄}{よ}するが
\ruby{如}{ごと}くに、
\ruby{肥}{ふと}つたる
\ruby{顏中}{かほ|ぢゆう}を
\ruby{笑}{わらひ}に
\ruby{動}{うご}かして、

『ウフ、ウフ、ウアツハツハヽハヽ。
とう〳〵
\ruby{來}{き}やがつたナ!ヤイ
\ruby{風}{かぜ}の
\ruby{神}{かみ}!。
\ruby{男振}{をとこ|ぶり}が
\ruby{好}{い}いぞ!\換字{志}つかり
\ruby{{\換字{遺}}}{や}れ。
\ruby{雨}{あめ}の
\ruby{{\換字{随}}}{つ}いて
\ruby{來}{き}やがらねえのは
\ruby{忌々}{いま|〳〵}しいが、
\ruby{仕方}{し|かた}が
\ruby{無}{ね}え、
\ruby{汝}{てめへ}だけでウンと
\ruby{働}{はたら}け。
\ruby{男振}{をとこ|ぶり}が
\ruby{好}{い}いぞ、〳〵!。
』

と、
\ruby{打{\換字{戱}}}{うち|たはむ}れて
\ruby{引{\換字{返}}}{ひつ|かへ}せば、
\ruby[g]{洋燈}{らんぷ}は
\ruby{既}{すで}に
\ruby{風}{かぜ}に
\ruby{{\換字{消}}}{け}されて、
\ruby{室}{しつ}に
\ruby{滿}{み}てる
\ruby{曙色}{しよ|しよく}は
\ruby{其光}{その|ひかり}に
\ruby{代}{かは}り
\ruby{居}{ゐ}たり。

\ruby{島木}{しま|き}が
\ruby{待}{ま}ちに
\ruby{待}{ま}つたるは、
\ruby{我}{われ}を
\ruby{訪}{と}ひ
\ruby{來}{こ}ん
\ruby{{\換字{婦}}}{をんな}にもあらねば、
\ruby{他所}{よ|そ}より
\ruby{入}{い}るべき
\ruby{金}{かね}にもあらず、
\ruby{唯此}{たゞ|こ}の
\ruby{野}{の}を
\ruby{拂}{はら}ひ
\ruby{禾}{くわ}を
\ruby{偃}{ふ}すの
\ruby{風}{かぜ}なりけり。
\ruby{數日前}{すう|じつ|ぜん}より、
\ruby{乾坤一擲}{けん|こん|いつ|てき}と
\ruby{試}{こゝろ}みたる
\ruby{丁半}{ちやう|はん}の、
\ruby{時}{とき}、
\ruby{利}{り}あらずして
\ruby{思}{おも}ふ
\ruby{目}{め}は
\ruby{出}{い}でず、\換字{志}きりに
\ruby{敵}{てき}に
\ruby{切}{き}り
\ruby{捲}{まく}られて、
\ruby{踏}{ふ}み
\ruby{耐}{こた}へ
\ruby{踏}{ふ}み
\ruby{耐}{こた}へては
\ruby{戰}{たゝか}ふものゝ、
\ruby{既}{すで}に
\ruby{味方}{み|かた}は
\ruby{崩}{くづ}れ
\ruby{立}{た}つて
\ruby{討死手負}{うち|じに|て|おひ}の
\ruby{數}{かず}を
\ruby{知}{し}らず、
\ruby{大勢}{たい|せい}のほゞ
\ruby{定}{さだ}まりたるに、
\ruby{無念}{む|ねん}の
\ruby{牙}{きば}を
\ruby{咬}{か}み
\ruby{血眼}{ちま|なこ}を
\ruby{瞋}{いか}らして、
\ruby{大童}{おほ|わらは}になつて
\ruby{奮闘}{ふん|とう}すれども、
\ruby{疲}{つか}れきつたる
\ruby{身}{み}の
\ruby{思}{おも}ふに
\ruby{任}{まか}せねば、
\ruby{天{\換字{運}}}{てん|うん}いよ〳〵
\ruby{我}{われ}に
\ruby{恵}{めぐ}まずば
\ruby{屍}{かばね}を
\ruby{原頭}{げん|とう}に
\ruby{曝}{さら}すも
\ruby{今}{いま}の
\ruby{間}{ま}ならんと、
\ruby{覺悟}{かく|ご}の
\ruby{臍}{ほぞ}を
\ruby{固}{かた}めつゝも、あはれ
\ruby{一}{ひ}ト
\ruby{暴風}{あ|れ}もあれかしと
\ruby{祈}{いの}り
\ruby{居}{ゐ}けるに、
\ruby{昨日}{きの|ふ}の
\ruby{芝浦}{しば|うら}の
\ruby{會}{くわい}の
\ruby{席上}{せき|じやう}より、
\ruby{羽{\換字{勝}}}{は|がち}の
\ruby{豫言}{こと|ば}といひ
\ruby{星}{ほし}の
\ruby{光}{ひかり}と
\ruby{云}{い}ひ、
\ruby{頼}{たの}もしく
\ruby{思}{おも}はるゝ
\ruby{節}{ふし}の
\ruby{少}{すくな}からぬを
\ruby{知}{し}つて、
\ruby{危}{あや}ぶみながらも
\ruby{待}{ま}ち
\ruby{居}{ゐ}たりし
\ruby{其風}{その|かぜ}の、
\ruby{果}{はた}して
\ruby{獵々颯々}{れふ|〳〵|さつ|〳〵}として
\ruby{吹}{ふ}き
\ruby{出}{いだ}したるに、
\ruby{今見}{いま|み}よ
\ruby{敗}{やぶれ}を
\ruby{轉}{てん}じて
\ruby{{\換字{勝}}}{かち}となさんは
\ruby{瞬}{またゝ}く
\ruby{間}{ま}なり、
\ruby{盛}{せ}り
\ruby{返}{かへ}して
\ruby{鏖殺}{みな|ごろし}にして
\ruby{{\換字{呉}}}{く}れんと、
\ruby{駒}{こま}の
\ruby{頭}{かしら}を
\ruby{立直}{たて|なほ}して
\ruby{鞍蓋}{くら|かさ}に
\ruby{突立上}{つゝ|たち|あが}つたる
\ruby{將軍}{しやう|ぐん}の
\ruby{意氣既}{い|き|すで}に
\ruby{疾}{はや}く
\ruby{敵}{てき}を
\ruby{吞}{の}んで
\ruby{槊}{さく}を
\ruby{横}{よこた}へて
\ruby{眼}{め}も
\ruby{遙}{はる}かに
\ruby{睥睨}{へい|げい}するが
\ruby{如}{ごと}く、
\ruby{勃々}{ぼつ|〳〵}たる
\ruby{英氣}{えい|き}と
\ruby{限}{かぎ}り
\ruby{無}{な}き
\ruby{活力}{くわつ|りよく}との、
\ruby{溢}{あふ}るゝばかり
\ruby{身}{み}に
\ruby{湧}{わ}くを
\ruby{覺}{おぼ}えて、
\ruby{流石}{さす|が}の
\ruby{島木}{しま|き}も
\ruby{押包}{おし|つゝ}み
\ruby{{\換字{兼}}}{か}ねつ、
\ruby{數聲}{すう|せい}の
\ruby{笑}{わらひ}を
\ruby{漏}{も}ら\換字{志}しなりけり。

\ruby{死生存亡}{し|せい|そん|ばう}
\ruby{此}{こ}の
\ruby{一擧}{いつ|きよ}と、
\ruby{鎬}{しのぎ}を
\ruby{{\換字{削}}}{けづ}つて
\ruby{爭}{あらそ}ふべき
\ruby{戰鬪}{たゝ|かひ}は、
\ruby{今}{いま}より
\ruby{二三時間}{に|さん|じ|かん}の
\ruby{後}{のち}に
\ruby{逼}{せま}り
\ruby{居}{を}れり。
\ruby{島木}{しま|き}は
\ruby{重}{おも}げなる
\ruby{身}{み}を
\ruby{無{\換字{造}}作}{む|ざう|さ}に
\ruby{動}{うご}かして、
\ruby{自}{みづか}ら
\ruby{押入}{おし|いれ}より
\ruby{夜具取}{や|ぐ|と}り
\ruby{出}{いだ}しつ、ごろりと
\ruby{其}{そ}れにくるまりて、
\ruby{横}{よこ}になるが
\ruby{早}{はや}きか
\ruby{頓}{やが}て
\ruby{睡}{ねむ}りぬ。
\ruby{島木}{しま|き}は
\ruby{自}{みづか}ら
\ruby{{\換字{敎}}}{をし}へ
\ruby{自}{みづか}ら
\ruby{養}{やしな}ひて、
\ruby{{\換字{敎}}}{をし}へ
\ruby{得}{え}
\ruby{養}{やしな}ひ
\ruby{得}{え}たるところある
\ruby{男}{をとこ}なりけり。

\ruby{風}{かぜ}は
\ruby{次第}{し|だい}に
\ruby{烈}{はげ}しくなりぬ。
\ruby{鼾}{いびき}は
\ruby{漸}{やうや}く
\ruby{盛}{さかん}になりぬ。
\ruby{風}{かぜ}の
\ruby{息}{や}む
\ruby{時}{とき}、
\ruby{鼾}{いびき}の
\ruby{聲}{こゑ}あり、
\ruby{鼾}{いびき}の
\ruby{無}{な}き
\ruby{時}{とき}、
\ruby{風}{かぜ}の
\ruby{音}{おと}あり。
\ruby{開}{ひら}き
\ruby{放}{はな}されたる
\ruby{押入}{おし|いれ}、
\ruby{投出}{なげ|だ}されたる
\ruby{酒瓶}{とつ|くり}、
\ruby{{\換字{消}}}{き}えたる
\ruby{洋燈}{らん|ぷ}、
\ruby{空虛}{か|ら}の
\ruby{罐}{くわん}、
\ruby{歪}{いびつ}に
\ruby{展}{の}べられたる
\ruby{蒲團}{ふ|とん}、
\ruby{明}{あ}けかけたる
\ruby{雨戸}{あま|ど}、
\ruby{雷}{らい}の
\ruby{如}{ごと}き
\ruby[g]{鼾聲}{いびき}、
\ruby{波濤}{な|み}と
\ruby{轟}{とどろ}く
\ruby{風}{かぜ}の
\ruby{音}{おと}、
\ruby{埒無}{らち|な}しとも
\ruby{{\換字{狼}}{\換字{藉}}}{らう|ぜき}とも
\ruby{亂暴}{らん|ばう}とも、
\ruby{云}{い}ふべき
\ruby{言葉}{こと|ば}は
\ruby{無}{な}き
\ruby{一室}{いつ|しつ}の
\ruby{狀}{さま}なり。


\Entry{其二十八}

『
\ruby{甚}{ひど}く
\ruby{寝込}{ね|こ}んで
\ruby{居}{ゐ}たぢあ
\ruby{無}{な}いか。
』

と、
\ruby{其頭}{その|かしら}に
\ruby{黄金細工}{き|ん|ざい|く}の
\ruby{施}{ほどこ}しある
\ruby{美}{うつく}しき
\ruby{琥珀}{こ|はく}のパイプを
\ruby{口}{くち}より
\ruby{放}{はな}しさまに
\ruby{云}{い}ひたるは、
\ruby{梅幸}{ばい|かう}の
\ruby{伊東}{い|とう}と
\ruby{渾名呼}{あだ|な|よ}ばるゝも
\ruby{無理}{む|り}ならず
\ruby{見}{み}ゆる
\ruby{其人}{その|ひと}を
\ruby{其儘}{その|まゝ}の
\ruby{面立}{おも|だち}の、
\ruby{三十三四}{さん|じう|さん|し}の
\ruby{色白}{いろ|しろ}き
\ruby{男}{をとこ}にて、
\ruby{昨夜}{ゆふ|べ}を
\ruby[g]{何處}{いづく}にてか
\ruby{{\GWI{u904e-k}}}{すご}しての
\ruby{今朝}{け|さ}、
\ruby{何}{なに}か
\ruby{用事}{よう|じ}ありて
\ruby{此家}{こ|ゝ}にたち
\ruby{{\換字{戻}}}{もど}りしが、
\ruby{今}{いま}や
\ruby{既}{すで}に
\ruby{朝食}{あさ|げ}を
\ruby{濟}{す}ませて
\ruby{率}{いざ}これよりと、
\ruby{戰鬪}{たゝ|かひ}の
\ruby{場}{ば}へ
\ruby{赴}{おもむ}かんとする
\ruby{前}{まへ}の
\ruby[g]{僅少}{わづか}の
\ruby{暇}{いとま}を、
\ruby{{\GWI{u7159-k}}草}{たば|こ}
\ruby{休}{やす}みに
\ruby{島木}{しま|き}の
\ruby{室}{しつ}に
\ruby{來}{き}て、
\ruby{昨日今日}{きの|ふ|け|ふ}こそ
\ruby{敵味方}{てき|み|かた}と
\ruby{立別}{たち|わか}れてこそあれ
\ruby{同}{おな}じ
\ruby{修羅}{しゆ|ら}の
\ruby{巷}{ちまた}の
\ruby{友}{とも}の
\ruby{相語}{あい|かた}らへるなり。

『アゝ、ちよいと
\ruby{寝}{ね}やうと
\ruby{思}{おも}つたけが、ついぐつすりと
\ruby{寝}{ね}て
\ruby{仕舞}{し|ま}つた。
』

『
\ruby{{\換字{宵}}}{よひ}にやあ
\ruby{汝寝}{おめへ|ね}られなかつたナ。
』

\ruby{微}{かすか}に
\ruby{冷笑}{あざ|わら}ふ
\ruby[g]{様子}{やうす}の
\ruby{唇}{くちびる}の
\ruby{端}{はた}に
\ruby{見}{み}ゆるを、
\ruby{見}{み}て
\ruby{取}{と}つたる
\ruby{島木}{しま|き}は
\ruby{一寸}{いつ|すん}も
\ruby{{\GWI{u9000-k}}}{ひ}けては
\ruby{居}{ゐ}ず。

『
\ruby{馬鹿}{ば|か}あ
\ruby{云}{い}へ。
\ruby{汝}{おめへ}のやうな
\ruby{繊細}{か|ぼそ}い
\ruby{野郎}{や|らう}ぢやあ
\ruby{有}{あ}るめえし、そんな
\ruby{卑小}{け|ち}な
\ruby{根性}{こん|ぢやう}は
\ruby{持}{も}たねえ
\ruby{萬五郎}{まん|ご|らう}さまだ。
お
\ruby{作}{さく}に
\ruby{聞}{き}いて
\ruby{見}{み}りやあ
\ruby{解}{わか}る
\ruby{事}{こと}だ。
』

『ハヽヽ、
\ruby{豪氣}{がう|ぎ}に
\ruby{今朝}{け|さ}は
\ruby{氣}{き}が
\ruby{{\換字{強}}}{つよ}いナ。
\ruby[g]{背後}{うしろ}から
\ruby{風}{かぜ}が
\ruby{推}{お}してるからナア。
』

『フン、
\ruby{{\換字{嫌}}味}{いや|み}を
\ruby{言}{い}ひやがる!。
よ\GWI{koseki-900370}にしろ、
\ruby{男}{をとこ}が
\ruby{下}{さが}がるぜ、
\ruby{下}{くだ}らねえ。
』

どつと
\ruby{吹}{ふ}く
\ruby{風}{かぜ}の
\ruby{音}{おと}、ひゆーと
\ruby{鳴}{な}る
\ruby{物}{もの}の
\ruby[g]{叫聲}{さけび}、
\ruby{二人}{ふ|たり}が
\ruby{居}{を}れる
\ruby{此樓}{この|ろう}も、ゆらりと
\ruby{今}{いま}は
\ruby{一}{ひ}ト
\ruby{搖}{ゆら}ぎして、
\ruby{一切}{いつ|さい}の
\ruby{物皆}{もの|みな}
\ruby{震}{ふる}ひ
\ruby{動}{うご}けば、ものこそ
\ruby{言}{い}はね
\ruby{伊東}{い|とう}が
\ruby{眉}{まゆ}はぴりゝと
\ruby{縮}{ちゞ}みて、
\ruby{心}{こヽろ}の
\ruby{安}{やす}からぬを
\ruby{現}{あら}はしたり。

『
\ruby{何様}{ど|う}した?
\ruby{梅幸}{ばい|かう}!。
\ruby{氣}{き}が
\ruby{揉}{も}めるか?。
』

\ruby{前}{まへ}の
\ruby{返報}{しか|へし}と
\ruby{島木}{しま|き}が
\ruby{調{\換字{戱}}}{から|か}へば、
\ruby{此}{これ}も
\ruby{男兒}{をと|こ}なり、
\ruby{癇癪}{かん|しやく}らしく
\ruby{{\GWI{u7159-k}}}{けむり}を
\ruby{吐}{は}きて、

『
\ruby{高}{たか}が
\ruby{此様}{こ|ん}な
\ruby[g]{無雨之風}{からつかぜ}!。
\ruby{何}{なに}が
\ruby{怖}{こは}い。
』

と、
\ruby{只一言}{ただ|ひと|こと}に
\ruby{云}{い}ひ
\ruby{{\換字{消}}}{け}しつ、
\ruby{{\換字{強}}}{しひ}て
\ruby{笑}{わら}つて、

『\GWI{koseki-900370}かし
\ruby{中々吹}{なか||ふ}きやがるナ。
\ruby{汝}{おめへ}こそ
\ruby{内々嬉}{ない|〳〵|うれ}しからう!。
\ruby{曲}{まが}り
\ruby{屋}{や}さんが
\ruby{立直}{たち|なほ}つて
\ruby{來}{き}さうだぜ。
』

と、
\ruby{云}{い}ひ
\ruby{足}{た}したり。

『
\ruby{左様}{さ|う}さ、いつまで
\ruby{曲}{まが}りつゞけで
\ruby{堪}{たま}るもんか。
\ruby{汝}{おめへ}ばかりに
\ruby{當}{あた}つて
\ruby{居}{い}られる
\ruby{世界}{せ|かい}ぢやあ
\ruby{無}{ね}え。
たまにやあ
\ruby{此様}{こ|ん}な
\ruby{風}{かぜ}も
\ruby{吹}{ふ}いて
\ruby{{\換字{呉}}}{く}れなくつちやあ!。
』

『
\ruby[g]{憫然}{かはいさう}に、
\ruby{空}{そら}を
\ruby{見}{み}ちやあ
\ruby{百姓}{ひやく|しやう}なんぞが
\ruby[g]{何程}{いくら}
\ruby{泣}{な}いてるか
\ruby{知}{し}れや
\ruby{仕}{し}ない!。
\ruby{汝}{おめへ}はもつと
\ruby{吹}{ふ}け
\ruby{位}{ぐらゐ}に
\ruby{思}{おも}つて
\ruby{居}{ゐ}るだらうが。
』

『
\ruby[g]{當然}{あたりまへ}よ。
\ruby{吹}{ふ}いて
\ruby{吹}{ふ}いて
\ruby{吹}{ふ}き
\ruby{抜}{ぬ}けと
\ruby{思}{おも}つて
\ruby{居}{ゐ}るんだ!。
\ruby{農夫}{ひやく|しやう}が
\ruby{泣}{な}いたつて
\ruby{笑}{わら}つたつて
\ruby{構}{かま}ふもんか!。
\ruby{早稲}{わ|せ}も
\ruby{晩稲}{おく|て}も
\ruby{吹}{ふ}き
\ruby{飛}{と}んで
\ruby{仕舞}{し|ま}へと
\ruby{思}{おも}つて
\ruby{居}{ゐ}るんだ。
』

『いゝ
\ruby{蟲}{むし}だナア!。
\ruby{酷}{ひど}い
\ruby{野郎}{や|らう}だぞ!。
\ruby{他}{ひと}に
\ruby[g]{百兩}{ひやくりやう}の
\ruby{損}{そん}をさせても、
\ruby{自己}{う|ぬ}が
\ruby[g]{一兩}{いちりやう}\ %空白有り
\ruby{儲}{まう}けりやあ
\ruby{好}{い}いといふ
\ruby{料簡方}{れう|けん|かた}だ。
』

『ナンダ、
\ruby{惡}{わる}く
\ruby{素人}{しろ|うと}くせえ
\ruby{事}{こと}を
\ruby{吐}{ぬ}かしやがる!。
\ruby{今}{いま}の
\ruby{世界}{せ|かい}で
\ruby{金}{かね}を
\ruby{儲}{まう}けて
\ruby{大顏}{おほ|づら}を
\ruby{仕}{し}て
\ruby{居}{ゐ}る
\ruby{奴}{やつ}に、
\ruby{唯}{ただ}の
\ruby[g]{一人}{ひとり}でも
\ruby{其}{そ}の
\ruby{料簡}{れう|けん}で
\ruby{無}{ね}え
\ruby{奴}{やつ}が
\ruby{有}{あ}るものかい!。
\ruby{大}{おほき}な
\ruby{門構}{もん|がまへ}を
\ruby{仕}{し}て
\ruby{居}{ゐ}る
\ruby{奴}{やつ}あ、
\ruby[g]{悉皆}{みんな}いゝ
\ruby{蟲}{むし}に
\ruby{羽}{はね}が
\ruby{生}{は}へたのぢやあ
\ruby{無}{ね}えか!。
』

『ハヽヽ、
\ruby{違無}{ちげ|へね}え!。
\ruby{言}{い}つて
\ruby{見}{み}りやあまあ
\ruby{其様}{そ|ん}なもんだ。
\ruby{倂}{\GWI{koseki-900370}かし
}し
\ruby{汝}{おめへ}は
\ruby[g]{{\換字{平}}常}{ふだん}から、
\ruby{觀音}{くわん|のん}なんぞを
\ruby{信心}{しん|〴〵}して
\ruby{居}{ゐ}るが
\ruby{彼}{あり}やあ
\ruby{何}{なん}だ!。
\ruby[g]{矢張}{やつぱ}り
\ruby{觀音様}{くわん|のん|さま}を
\ruby{取捉}{とつ|つか}めへても、
\ruby{其様}{そ|ん}なあこぎな
\ruby{料簡}{れう|けん}でもつて、
\ruby{金持}{かね|もち}になるやうと
\ruby{祈}{いの}つて
\ruby{居}{ゐ}るのか?。
』

『ムヽ、
\ruby{他}{ほか}に
\ruby{祈}{いの}らうことは
\ruby{無}{ね}えぢやあ
\ruby{無}{ね}えか!。
』

『ぢやあ
\ruby{惡}{わる}い
\ruby{暴風}{あ|れ}も
\ruby{祈}{いの}りかね
\ruby{無}{ね}えが、そんな
\ruby{我欲}{が|よく}の
\ruby{願}{ねがひ}を
\ruby{掛}{か}けたつて、
\ruby{觀音}{くわん|のん}は
\ruby{正路}{しや|うろ}の
\ruby{佛}{ほとけ}ださうだぜ。
』

『ナニ
\ruby{乃公}{お|ら}の
\ruby{觀音}{くわん|のん}は
\ruby{乃公}{お|ら}の
\ruby{觀音}{くわん|のん}だ!。
\ruby{汝}{おめへ}の
\ruby{觀音}{くわん|のん}たあ
\ruby{異}{ちが}つたつて
\ruby{管}{かま}はねえ。
\ruby{乃公}{お|ら}あ
\ruby{乃公}{お|ら}で
\ruby{濟}{す}んでるんだから、これで
\ruby{可}{い}いんだ。
』

『
\ruby{何}{なん}だか
\ruby{道理}{す|じ}が
\ruby{{\GWI{u901a-k}}}{とほ}らねえやうだが、アツ、また
\ruby{吹}{ふ}きやがる、
\ruby{甚}{ひど}くなつて
\ruby{來}{き}たぞ。
オヽ
\ruby{塀}{へい}が
\ruby{飛}{と}んだぞ、
\ruby{棟瓦}{むな|がはら}が
\ruby{落}{お}ちたぞ!。
』

『どうだ
\ruby{{\換字{情}}無}{なさけ|な}いか、
\ruby{心配}{しん|ぱい}か!。
』

『
\ruby{馬鹿}{ば|か}あ
\ruby{云}{い}ふな、
\ruby{篦棒}{べら|ぼう}ナ!。
\ruby{天運}{う|ん}は
\ruby{何様}{ど|う}
\ruby{循環}{ま|は}つたつて
\ruby{手腕}{う|で}は
\ruby{手腕}{う|で}だ!。
\ruby{{\GWI{u9006-k}}風}{むかひ|かぜ}を
\ruby{乘}{の}つ
\ruby{切}{き}つて
\ruby{腕前}{うで|まへ}を
\ruby{見}{み}せてやらあ。
\ruby{此方}{こち|ら}あ
\ruby[g]{昨夜}{ゆふべ}
\ruby{辨天様}{べん|てん|さま}に、\GWI{koseki-900370}たゝかお
\ruby{賽錢}{さい|せん}を
\ruby{献}{あ}げて
\ruby{來}{き}たんだ、はゞかりながら
\ruby{辨天様}{べん|てん|さま}が
\ruby{付}{つい}て
\ruby{居}{ゐ}るんだ!。
』

\ruby{笑}{わら}ひながら
\ruby{云}{い}ひたる
\ruby{末}{すゑ}の
\ruby[g]{言葉}{ことば}は、
\ruby{暗}{あん}に
\ruby[g]{自己}{おのれ}が
\ruby[g]{昨夜}{きのふ}の
\ruby{豪{\GWI{u904a-k}}}{がう|いう}を
\ruby{誇}{ほこ}つて、
\ruby{{\GWI{u904a-k}}謔}{たは|むれ}の
\ruby{中}{うち}にも
\ruby{威}{ゐ}を
\ruby{張}{は}りて、
\ruby{聊}{いささ}か
\ruby{自}{みづか}ら
\ruby{{\換字{強}}}{つよ}うせるなり。

『
\ruby{何}{なん}だ、
\ruby{薄}{うす}ら
\ruby{腥}{なまぐさ}い
\ruby{辨天}{べん|てん}が
\ruby{何}{なに}がありがたい\GWI{u2048}。
\ruby[g]{此方}{こつち}あ
\ruby{{\換字{清}}{\換字{浄}}}{しやう|〴〵}な
\ruby{仙人}{せん|にん}にお
\ruby[g]{初穂}{はつほ}が
\ruby{献}{あ}げてあるんだ!。
\ruby{今日}{け|ふ}は
\ruby{此}{こ}の
\ruby{乃公}{お|ら}が
\ruby{大當}{おほ|あた}りだ!。
』

これも
\ruby{私}{ひそか}に
\ruby{自}{みづか}ら
\ruby{快}{こヽろ}よしとするところあるなり。

『ナニ
\ruby{此}{こ}の
\ruby{鼻}{はな}が
\ruby{矢張}{やつ|ぱ}り
\ruby{當}{あた}る!。
』

『ナニ
\ruby{此}{こ}の
\ruby{乃公様}{お|れ|さま}が
\ruby[g]{屹度}{きつと}
\ruby{當}{あた}る!。
』

『おれが、』

『おれが、』

『ハヽハヽ、』

『ハヽハヽ、』

『なにも
\ruby{此處}{こ|ゝ}で
\ruby{喧嘩}{けん|くわ}あ
\ruby{爲}{す}る
\ruby{事}{こと}も
\ruby{無}{ね}え。
』

『
\ruby{二人}{ふ|たり}とも
\ruby{當}{あた}らう!。
』

『
\ruby{夕方}{ゆふ|がた}までだ!。
』

\ruby{風}{かぜ}はいよ〳〵
\ruby{狂}{くる}へる
\ruby{中}{なか}を、
\ruby[g]{二人}{ふたり}はおのれおのれが
\ruby{本陣}{ほん|じん}へと、
\ruby{勇威}{いき|ほい}を
\ruby{含}{ふく}んで
\ruby{立出}{たち|い}でたり。
\ruby{風}{かぜ}も
\ruby{慾}{よく}に
\ruby{使}{つか}はるゝ
\ruby{人}{ひと}の
\ruby{世}{よ}の
\ruby{中}{なか}や。


\Entry{其二十九}

\ruby{其}{そ}の
\ruby{日晝}{ひ|ひる}を
\ruby{{\換字{過}}}{す}ぎて
\ruby{風}{かぜ}いよ〳〵
\ruby{烈}{はげ}しく、
\ruby{天}{そら}は
\ruby{塵埃}{ぢん|あい}に
\ruby{濁}{にご}れるが
\ruby{如}{ごと}くに
\ruby{一面}{いち|めん}の
\ruby{黃雲}{くわう|ゝん}に
\ruby{包}{つゝ}まれて、
\ruby{常}{つね}ならぬ
\ruby{暖氣}{あたゝ|かさ}の
\ruby{氣味}{き|み}
\ruby{惡}{あし}ければ、
\ruby{人皆安}{ひと|みな|やす}き
\ruby{心}{こゝろ}も
\ruby{無}{な}くて、
\ruby{若}{も}し
\ruby{此上}{この|うへ}に
\ruby{雨}{あめ}も
\ruby{混}{まじ}らばと
\ruby{氣{\換字{遣}}}{き|づか}ふ
\ruby{折}{をり}しも、
\ruby{頭上}{づ|じやう}の
\ruby{雲}{くも}やうやく
\ruby{墨色}{すみ|いろ}さして、
\ruby{蔽}{おほ}ひかぶさる
\ruby{樣}{よう}に
\ruby{昏}{くら}くなれば、
\ruby{如何}{い|か}になり
\ruby{行}{ゆ}く
\ruby{魔日}{ま|び}ぞと
\ruby{誰}{たれ}しも
\ruby{恐}{おそ}れあひぬ。
\ruby{事無}{こと|な}くて
\ruby{家}{いへ}にある
\ruby{爺媼}{ぢゞ|ばゞ}さへ
\ruby{是}{かく}の
\ruby{如}{ごと}くなれば、まして、
\ruby{{\換字{遣}}}{や}らん
\ruby{買}{か}はんの
\ruby{呼}{よ}び
\ruby{聲}{ごゑ}は
\ruby{戰場}{せん|じやう}の
\ruby{矢叫}{や|さけ}びと
\ruby{入}{い}り
\ruby{亂}{みだ}れて、
\ruby{打振}{うち|ふ}る
\ruby{兩手}{りやう|て}は
\ruby{浪寄}{なみ|よ}る
\ruby{尾花}{を|ばな}と
\ruby{空}{そら}に
\ruby{揉}{も}まるゝ
\ruby[g]{其場}{そのば}の
\ruby{混亂}{こん|らん}は、
\ruby{猜}{すゐ}するにも
\ruby{{\換字{猶}}}{なほ }
\ruby{餘}{あまり}あり。

\ruby{伊東}{い|とう}はいづれへ
\ruby{逸}{そ}れしにや
\ruby{歸}{かへ}り
\ruby{來}{きた}らねど、
\ruby{雨下}{あめ|お}りんとして
\ruby{下}{お}りず
\ruby{風衰}{かぜ|おとろ}へぬ
\ruby{夕{\換字{近}}}{ゆうべ|ちか}く、
\ruby{島木}{しま|き}は
\ruby{悠然}{いう|ぜん}として
\ruby{歸}{かへ}り
\ruby{來}{きた}りぬ。
\ruby{島木}{しま|き}につゞきて
\ruby{上}{あが}り
\ruby{來}{きた}れる
\ruby{婢}{をんな}は、
\ruby{例}{れい}となり
\ruby{居}{を}れると
\ruby{見}{み}えて
\ruby{茶}{ちや}を
\ruby{入}{い}れて
\ruby{薦}{すゝ}めつ。

『
\ruby{伊東}{い|とう}さんは?、
\ruby{御存知無}{ご|ぞん|ぢ|な}くつて?。
』

『
\ruby{知}{し}らねえよ、
\ruby{一所}{いつ|しよ}ぢやあ
\ruby{無}{ね}えから。
\換字{志}かしおほかた
\ruby{彼女}{あ|れ}のところだらう。
』

『ほんとに
\ruby{凝}{こ}つて
\ruby{行}{い}らつしやるのネ!。
\ruby{幸{\換字{運}}}{い|ゝ}につけても、
\ruby[g]{惡{\換字{運}}}{わるい}に
\ruby{付}{つ}けてもネエ!。
』

『ウン。
ハヽ、
\ruby{今日}{け|ふ}は
\ruby{幸{\換字{運}}}{い|ゝ}につけてもぢやあ
\ruby{無}{な}さゝうだ!。
でも
\ruby{彼女}{あ|れ}の
\ruby{方}{はう}でも
\ruby{招}{よ}ぶやうだから
\ruby{堪}{たま}らねえや。
\ruby{汝}{おめへ}も
\ruby{女}{をんな}の
\ruby{端}{はし}くれだ、どうだ、
\ruby{些}{ちつと}あ
\ruby{妬}{や}けるかい?。
』

『
\ruby{何}{なん}ですつて、
\ruby{端}{はし}くれですつて?。
あんまり
\ruby{酷}{ひど}い
\ruby{事}{こと}ネ。
ようござんすよ、たんと
\ruby{惡口}{わる|くち}を
\ruby{仰}{おつしや}いまし、
\ruby{告訴}{いゝ|つけ}て
\ruby{{\換字{遣}}}{や}るとこを
\ruby{知}{し}つてますから。
ア、そりやあ
\ruby{左樣}{そ|う}と
\ruby{貴君}{あな|た}は
\ruby{今日}{け|ふ}は
\ruby{大當}{おほ|あた}りでしやう。
あなたも
\ruby{男兒}{をと|こ}の
\ruby{端}{はし}くれだ、
\ruby{些}{ちつと}あ
\ruby{氣前}{き|まへ}を
\ruby{見}{み}せて
\ruby{御奢}{お|おご}んなさいな。
\ruby{風}{かぜ}の
\ruby{音}{おと}を
\ruby{聞}{き}いちやあ
\ruby[g]{主婦}{おかみ}さんと
\ruby{一日云}{いち|にち|い}ひ
\ruby{暮}{く}らして
\ruby{居}{ゐ}ましたよ。
』

『
\ruby{左樣}{さ|う}かい、
\ruby{其奴}{そ|いつ}あ
\ruby{頼}{たの}もしかつた!。
\ruby{奢}{おご}つて
\ruby{{\換字{遣}}}{や}らう。
』

『オヤ、
\ruby{其}{それ}あ
\ruby{早{\換字{速}}}{さつ|そく}に
\ruby{有}{あ}り
\ruby{難}{がた}う!。
さうして
\ruby{何}{なに}を
\ruby{奢}{おご}つて
\ruby{下}{くだ}さる?。
』

『
\ruby{生憎劇場}{あい|にく|しば|ゐ}は
\ruby{好}{い}いところが
\ruby{開}{あ}いて
\ruby{居}{ゐ}ねえナ。
』

『さうネエ。
』

『
\ruby{秋草}{あき|くさ}も
\ruby{今日}{け|ふ}の
\ruby{此}{こ}の
\ruby{風}{かぜ}ぢやあもう。
』

『さうネエ。
』

『
\ruby{矢張}{やつ|ぱ}り
\ruby{下卑}{げ|び}でも
\ruby{甘}{あま}い
\ruby{物}{もの}といふところで
\ruby{堪忍}{かん|にん}して
\ruby{貰}{もら}はう。
』

『さうねエ。
それぢやあ、あの、
\ruby{何}{なに}を?。
』

『
\ruby{今川燒}{いま|がは|やき}きの
\ruby{皮}{かは}の
\ruby{厚}{あつ}い
\ruby{冷}{つめ}たいのでも。
ハヽハヽ。
』

『エヽ
\ruby{悔}{くや}しいヨ、おぼえて
\ruby{居}{ゐ}らつしやい。
もう
\ruby{貴君}{あな|た}の
\ruby{云}{い}ふ
\ruby{事}{こと}は
\ruby{當}{あて}に
\ruby{仕}{し}やしない。
』

『オイ〳〵
\ruby{左樣}{さ|う}ぶり〳〵しちやあ
\ruby{困}{こま}る。
\ruby{頼}{たの}む
\ruby{事}{こと}があるんだ、
\ruby{大}{おほ}まじめだ。
』

『ヘイ〳〵、
\ruby{澤山}{たん|と}
お
\ruby{使}{つか}ひなさいまし!。
\ruby{何}{なん}の
\ruby{御用}{ご|よう}?。
』

『
\ruby{惡}{わる}く
\ruby{角}{かく}ばるナ、
\ruby{怒}{おこ}つちやあいけねえ。
\ruby{好}{い}いかエ、
\ruby{客}{きやく}が
\ruby{一人來}{ひと|り|く}る
\ruby{筈}{はず}に
\ruby{招}{よ}んであるんだ。
\ruby{汝}{おめへ}の
\ruby{見}{み}はからひで、
\ruby{例}{いつも}の
\ruby{家}{うち}へでも
\ruby{電話}{でん|わ}をかけて、
\ruby{手一杯}{て|いつ|ぱい}に
\ruby{御馳走}{ご|ち|そう}を
\ruby{仕}{し}て
\ruby{貰}{もら}ひてえのだ。
\ruby{他家}{わ|き}へ
\ruby{行}{い}くなあ
\ruby{不妙}{ま|づ}いのだから。
ヨ、
\ruby{頼}{たの}むよ。
\ruby{客}{きやく}が
\ruby{堅人}{かた|じん}で、
\ruby{話}{はなし}が
\ruby{堅}{かた}いと
\ruby{來}{き}て
\ruby{居}{ゐ}るんだから。
』

『ハア、
\ruby{左樣}{さ|う}。
ようござんす。
\ruby{御酒}{ご|しゆ}は?。
\ruby{麦酒}{びー|る}?。
\ruby{葡萄酒}{いつ|も|の}?。
さうして
\ruby{直}{ぢき}に
\ruby{御入來}{お|い|で}ですか。
』

『ウン、もうそろ〳〵
\ruby{來}{く}る
\ruby{時{\換字{分}}}{じ|ぶん}だから
\ruby{急}{いそ}いでネ。
』

『あの
\ruby{水野}{みづ|の}さんとか
\ruby{仰}{おつし}ある
\ruby{方}{かた}?。
』

『ソラ
\ruby{惚}{ほ}れてやがるもんだから
\ruby{兎角名}{と|かく|な}をいふ!。
お
\ruby{生憎樣}{あひ|にく|さま}!。
』

『
\ruby{水野}{みづ|の}ぢやあ
\ruby{無}{ね}え、
\ruby{羽{\換字{勝}}}{は|がち}といふんだ。
\換字{志}かし
\ruby{色}{いろ}の
\ruby{白}{しろ}い、
\ruby{眼}{め}の
\ruby{優}{やさ}しい、
\ruby{滅法}{めつ|ぱふ}に
\ruby{好}{い}い
\ruby{男}{をとこ}だから、
\ruby{{\換字{又}}汝}{また|おめへ}は
\ruby{直}{すぐ}と
\ruby{惚}{ほ}れるだらう。
』

『
\ruby{他聞}{ひと|ぎゝ}の
\ruby{惡}{わる}い!。
よしても
\ruby{下}{くだ}さいよ。
\ruby{妾}{わたし}や
\ruby{男}{をとこ}の
\ruby{美}{い}いのに
\ruby{惚}{ほ}れるやうな
\ruby{耄碌}{まう|ろく}ぢやあ
\ruby{有}{あ}りませんよ。
ホヽホヽ。
』

『オヤ
\ruby{異}{おつ}なたんかを
\ruby{切}{き}りやあがる。
それぢやあ
\ruby{何樣}{ど|ん}な
\ruby{男}{をとこ}に
\ruby{惚}{ほ}れるんだ?。
』

『
\ruby{知}{し}れた
\ruby{事}{こと}でさアネ、
\ruby{明治}{めい|じ}ツ
\ruby{子}{こ}ですよ。
\ruby{成功者}{あた|り|や}さんばつかりに
\ruby{惚}{ほ}れるんですわネ。
』

『
\ruby{畜生}{ちき|しやう}ツ、
\ruby{甚}{ひど}く
\ruby{當世}{たう|せい}なことを
\ruby{吐}{ぬか}しやあがる。
\ruby[g]{此奴}{こいつ}は
\ruby{今川焼}{いま|がは|やき}の
\ruby{讐}{かたき}を
\ruby{打}{う}たれた。
ハヽハヽ。
』

『ホヽホヽ。
』

お
\ruby{作}{さく}の
\ruby{笑}{わら}つて
\ruby{樓}{にかい}を
\ruby{下}{お}りきつたる
\ruby{時}{とき}、がらりと
\ruby{格子}{かう|し}の
\ruby{明}{あ}く
\ruby{音}{おと}して、
\ruby{頼}{たの}むといふ
\ruby{聲}{こゑ}の
\ruby{此家}{こ|ゝ}の
\ruby{客}{きやく}には
\ruby{似合}{に|あ}はしからず
\ruby{堅}{かた}く、
\ruby{洋服姿}{やう|ふく|すがた}のきりゝとしたる、
\ruby{日}{ひ}に
\ruby{焦}{や}けきつたる
\ruby{顔}{かほ}の
\ruby{恐}{おそ}ろしく
\ruby{赭}{あか}く、
\ruby{潮風}{しほ|かぜ}に
\ruby{晒}{さ}らされてか
\ruby{眼}{め}さへ
\ruby{赤色}{あか|いろ}を
\ruby{帶}{お}びたる
\ruby{鐵}{てつ}づくりの
\ruby{如}{ごと}き
\ruby{男}{をとこ}は
\ruby{入}{い}り
\ruby{來}{きた}りぬ。

お
\ruby{作}{さく}は
\ruby{受}{う}け
\ruby{取}{と}りたる
\ruby{名刺}{めい|し}の
\ruby{表}{おもて}に
\ruby{羽{\換字{勝}}}{は|がち}
\ruby{千{\換字{造}}}{せん|ざう}といふ
\ruby{四文字}{よん|も|じ}の
\ruby{記}{しる}されたるを
\ruby{見}{み}ぬ。


\Entry{其三十}

\ruby{酒}{さけ}は
\ruby{舊友}{きう|いう}と
\ruby{{\GWI{hkcs_m98f2}}}{の}むより
\ruby{甘}{うま}きは
\ruby{無}{な}く、
\ruby{談}{だん}は
\ruby{{\換字{半}}醉}{はん|すゐ}の
\ruby{時}{とき}より
\ruby{熱}{ねつ}するは
\ruby{無}{な}し、
\ruby{雞黍}{けい|しよ}の
\ruby{設}{まう}け
\ruby{粗薄}{そ|はく}なりとも、
\ruby{膠漆}{かう|しつ}の
\ruby{{\換字{情}}}{じやう}の
\ruby{殷厚}{いん|こう}ならんには、
\ruby{杯}{さかづき}を
\ruby{手}{て}にして
\ruby{相見}{あい|み}て
\ruby{笑}{わら}ふ
\ruby{一眄}{いち|べん}の
\ruby{中}{うち}にも
\ruby{限無}{かぎり|な}き
\ruby{味}{あぢはひ}は
\ruby{有}{あ}るべきを、ましてこれは
\ruby{范張陳雷}{はん|ちやう|ちん|らい}の
\ruby{語}{かた}らひのみならで、
\ruby{野心}{や|しん}に
\ruby{燃}{も}ゆる
\ruby{若}{わか}き
\ruby{男}{をとこ}の、
\ruby{志}{こヽろざし}は
\ruby{各々異}{おの|〳〵|こと}なれども
\ruby{事}{こと}を
\ruby{一}{いつ}にして
\ruby{功}{こう}を
\ruby{擧}{あ}げんとする
\ruby{相談}{さう|だん}に、
\ruby{意氣}{い|き}は
\ruby{齊}{ひと}しく
\ruby{昻}{あが}りて
\ruby{興}{きよう}は
\ruby{湧}{わ}くが
\ruby{如}{ごと}し。

\ruby{亭主八杯}{てい|しゆ|はち|はい}の
\ruby{諺}{ことわざ}に
\ruby{洩}{も}れず、
\ruby[g]{{\GWI{u7fbd-k}\換字{勝}}}{はがち}より
\ruby{先}{ま}ず
\ruby{島木}{しま|き}は
\ruby{醉}{よ}ひて、
\ruby{其}{そ}の
\ruby{肥}{ふと}つたる
\ruby[g]{身體}{からだ}を
\ruby{柱}{はしら}に
\ruby{靠}{もた}せながら、
\ruby{腫}{は}れたるが
\ruby{如}{ごと}き
\ruby{顏}{かほ}に
\ruby{笑}{ゑみ}を
\ruby{{\GWI{u6d6e-k}}}{うか}めつゝ、

『
\ruby{兎}{と}も
\ruby{角}{かく}も
\ruby{其}{それ}ぢやあ
\ruby[g]{一萬二千圓}{いちまんにせんゑん}だけは
\ruby{君}{きみ}の
\ruby{權利}{けん|り}の
\ruby{内}{うち}に
\ruby{置}{お}くと
\ruby{決}{き}めた。
\ruby{{\GWI{u8239-k}}}{ふね}も
\ruby{借}{か}りるなら
\ruby{借}{か}りるが
\ruby{好}{い}い、
\ruby{買}{か}ふならばまた
\ruby{買}{か}ふが
\ruby{好}{い}い。
\ruby{一切君}{いつ|さい|きみ}の
\ruby[g]{考次第}{かんがへしだい}に
\ruby{任}{まか}せる。
\ruby{一艘}{いつ|ぱい}
\ruby{仕立}{し|た}てるとも
\ruby{二艘三艘}{に|はい|さん|ばい}
\ruby{仕立}{し|た}てるとも、それも
\ruby{君次第}{きみ|し|だい}で
\ruby{論}{ろん}は
\ruby{無}{な}い。
\ruby{乃公}{お|ら}あ
\ruby{素人}{しろ|うと}だ、
\ruby{君}{きみ}は
\ruby{黑人}{くろ|うと}だ。
\ruby{乃公}{お|ら}あ
\ruby{何}{なに}も
\ruby{彼}{か}も
\ruby{分}{わか}らないんだ。
おらあたゞ
\ruby{焔{\GWI{u785d-k}}}{えん|せう}と
\ruby{彈丸}{た|ま}とを
\ruby{出}{だ}すんだ。
\ruby{狙}{ねら}つて
\ruby{撃}{う}つて
\ruby{鳥}{とり}を
\ruby{穫}{と}るなあ
\ruby{君}{きみ}の
\ruby{手腕一}{う|で|いつ}ぱいに
\ruby{仕}{し}て
\ruby{貰}{もら}うんだ。
\ruby{後}{うしろ}から
\ruby{臂}{ひぢ}に
\ruby{觸}{さは}るやうな
\ruby{野暮}{や|ぼ}は
\ruby{仕}{し}ねえ。
\ruby{乃公}{お|ら}あ
\ruby{資金}{か|ね}を
\ruby{出}{だ}す、
\ruby{君}{きみ}は
\ruby{手腕}{う|で}を
\ruby{貸}{か}す。
\ruby[g]{利{\GWI{u76ca-k}}}{まうけ}は
\ruby{笑}{わら}つて
\ruby{山分}{やま|わけ}に
\ruby{仕}{し}やうが、
\ruby{損}{そん}は
\ruby{泣言}{なき|ごと}を
\ruby{云}{い}ひつこ
\ruby{無}{な}しで、
\ruby{氣持好}{き|もち|よ}く
\ruby{骰子}{さ|い}を
\ruby{轉}{ころ}がして
\ruby{見}{み}やうと
\ruby{云}{い}うんだ。
\GWI{koseki-900370}かし
\ruby{僕}{ぼく}も
\ruby[g]{商人}{あきんど}だ、
\ruby{算盤}{そろ|ばん}だけは
\ruby{合點}{がつ|てん}の
\ruby{行}{ゆ}く
\ruby{男}{をとこ}だから、
\ruby{大}{おほ}づもりのところだけは
\ruby[g]{都度々々}{つど〳〵}
\ruby{聞}{き}きたい。
\ruby{其他}{その|ほか}にやあ
\ruby{何}{なに}も
\ruby{注文}{ちゆう|もん}は
\ruby{無}{な}いんだ。
\ruby{全}{まつた}く
\ruby{君}{きみ}の
\ruby{料簡次第}{れう|けん|し|だい}だ。
なあに
\ruby{一}{ぴん}と
\ruby{出}{で}やうと
\ruby{六}{ろく}と
\ruby{出}{で}やうと
\ruby{口惜}{くや|し}かあ
\ruby{無}{ね}え、
\ruby{事業}{し|ごと}の
\ruby{巧}{うま}く
\ruby{行}{い}くのと
\ruby{行}{い}かないのは、
\ruby{{\換字{半}}分}{はん|ぶん}は
\ruby{手腕}{う|で}で
\ruby{{\換字{半}}分}{はん|ぶん}は
\ruby{耳朶}{みゝつ|たぶ}だ!。
\ruby[g]{{\GWI{u9063-k}}付}{やつつ}けるだけ
\ruby[g]{{\GWI{u9063-k}}付}{やつつ}けて
\ruby{貰}{もら}やあ、
\ruby{何様}{ど|う}なつたつて
\ruby{驚}{おどろ}かねえんだから、
\ruby{斟酌無}{しん|しやく|な}く
\ruby{存分}{ぞん|ぶん}に
\ruby[g]{{\GWI{u9063-k}}}{や}つて
\ruby{{\換字{呉}}}{く}れたまへ。
\ruby{今}{いま}も
\ruby{話}{はな}した
\ruby{{\GWI{u901a-k}}}{とほ}り
\ruby{此}{こ}の
\ruby{風}{かぜ}が
\ruby{出無}{で|な}かつたら、
\ruby{擴}{ひろ}げられるだけ
\ruby{戰線}{せん|〳〵}を
\ruby{擴}{ひろ}げて
\ruby{置}{お}いた
\ruby{此}{こ}の
\ruby{萬五郎}{まん|ご|らう}は、
\ruby{今}{いま}ごろは
\ruby{何處}{ど|こ}へケシ
\ruby{飛}{と}んでるか
\ruby{分}{わか}からないんだが、
\ruby{其}{そ}の
\ruby{危}{あぶ}ない
\ruby{瀬}{せ}を
\ruby{渡}{わた}つて
\ruby{揉}{も}み
\ruby{合}{あ}つたゞけに、とう〳〵
\ruby{切}{き}り
\ruby{{\換字{勝}}}{か}つて
\ruby{一}{ひ}ト
\ruby{伸}{のし}
\ruby{伸}{の}して、
\ruby{如是}{こ|う}した
\ruby{話}{はなし}も
\ruby{出來}{で|き}るんだもの!。
お
\ruby{互}{たがひ}に
\ruby{度胸}{ど|きよう}と
\ruby{腕}{うで}とに
\ruby{掛}{か}けて
\ruby{敗}{ひけ}を
\ruby{取}{と}ら
\ruby{無}{な}きやあ、
\ruby{少}{すこ}し
\ruby{{\GWI{u904b-k}}}{うん}さへ
\ruby{添}{そ}やあ
\ruby{{\GWI{u9020-k}}作}{ざう|さ}は
\ruby{無}{ね}え。
\ruby{三井}{みつ|ゐ}や
\ruby{岩崎}{いは|さき}を
\ruby{尻目}{しり|め}に
\ruby{見}{み}て、
\ruby{笑}{わら}つて
\ruby{一杯{\GWI{hkcs_m98f2}}}{いつ|ぱい|の}ま
\ruby{無}{な}くつちやあ!。
\ruby{米}{こめ}や
\ruby{株}{かぶ}ばかり
\ruby{打}{たヽ}いて
\ruby{居}{ゐ}るのも
\ruby{智慧}{ち|ゑ}が
\ruby{足}{た}り
\ruby{無}{ね}えから、
\ruby{乃公}{お|ら}あ
\ruby{大蛸}{おほ|だこ}になつて
\ruby{八方}{はつ|ぽう}へ
\ruby{手}{て}を
\ruby{出}{だ}す!。
\ruby{五分}{ご|ぶ}や
\ruby{七分}{しち|ぶ}の
\ruby{口錢}{こう|せん}にヘイコラヘイコラと
\ruby{頭}{あたま}を
\ruby{下}{さ}げてこしらへた
\ruby{身上}{しん|じやう}ぢやあ
\ruby{無}{な}し、
\ruby{根}{ね}が
\ruby{泡沫錢}{あぶ|く|ぜに}だもの、
\ruby{{\GWI{u6d88-k}}}{き}えたつて
\ruby{未練}{み|れん}は
\ruby{無}{ね}えが、
\ruby{何}{なに}か
\ruby{知}{し}ら
\ruby[g]{那方}{どつち}かの
\ruby{手}{て}で
\ruby{攫}{つか}むつもりだ。
\ruby{思}{おも}ひ
\ruby{出}{だ}しやあソレ
\ruby{四五年前}{し|ご|ねん|まへ}の
\ruby{事}{こと}だつけ、
\ruby{七人揃}{しち|にん|そろ}つた
\ruby{其時}{その|とき}に、おれが
\ruby{例}{いつも}の
\ruby{法螺話}{ほ|ら|ばなし}の
\ruby{末}{すゑ}、お
\ruby{互}{たがい}に
\ruby{那}{ど}の
\ruby{路}{みち}にせよ
\ruby{世}{よ}を
\ruby{渡}{わた}るにやあ、
\ruby{跣足}{は|だし}ぢやあ
\ruby{歩}{ある}けねえ、
\ruby{草鞋}{わら|ぢ}が
\ruby{要}{い}る。
おれが
\ruby{一番巧}{いち|ばん|うま}く
\ruby{當}{あた}りやあ、
\ruby[g]{一同}{みんな}に
\ruby{一萬兩}{いち|まん|りやう}づゝの
\ruby{草鞋}{わら|ぢ}を
\ruby{穿}{は}かせて、
\ruby{世}{よ}の
\ruby{石高路}{いし|だか|みち}を
\ruby{歩}{ある}かせて
\ruby{{\GWI{u9063-k}}}{や}ると
\ruby{云}{い}つたら、
\ruby{馬鹿}{ば|か}に
\ruby{誰}{だれ}も
\ruby{彼}{かれ}も
\ruby{怒}{おこ}りやあがつて、あの
\ruby[g]{溫和}{おとな}しい
\ruby[g]{水野}{みづの}までが、
\ruby{僕}{ぼく}は
\ruby{踏}{ふ}み
\ruby{抜}{ぬ}きを
\ruby{仕}{し}たつて
\ruby{其様}{そ|ん}な
\ruby{草鞋}{わら|ぢ}は
\ruby{貰}{もら}はないと
\ruby{云}{い}ふし、
\ruby[g]{日方}{ひかた}はおらが
\ruby{背中}{せ|なか}を
\ruby{擲}{なぐ}りやがるし、
\ruby{楢井}{なら|い}や
\ruby{山瀬}{やま|せ}や
\ruby{名倉}{な|ぐら}までが、
\ruby{失敬}{しつ|けい}だ〳〵と
\ruby{腹}{はら}を
\ruby{立}{た}つたが、
\ruby{其時君}{その|とき|きみ}はたつた
\ruby{一人}{ひと|り}、なあに
\ruby{島木}{しま|き}が
\ruby{親切}{しん|せつ}で
\ruby{{\換字{呉}}}{く}れやうといふなら
\ruby{貰}{もら}ふが
\ruby{好}{い}いぢや
\ruby{無}{な}いか、
\ruby{氣}{き}が
\ruby{狭}{せま}い!、
\ruby{成程世}{なる|ほど|よ}を
\ruby{渡}{わた}るにやあ
\ruby{草鞋}{わら|ぢ}が
\ruby{要}{い}る、と
\ruby[g]{沈着}{おちつ}いて
\ruby{云}{い}つて
\ruby{{\換字{呉}}}{く}れた
\ruby{時}{とき}あ
\ruby{嬉}{うれ}しかつたよ。
それでと
\ruby{云}{い}ふ
\ruby{譯}{わけ}ぢやあ
\ruby{更}{さら}に
\ruby{無}{ね}えが、
\ruby{云}{い}はゞ
\ruby{其時云}{その|とき|い}つた
\ruby{其}{その}
\ruby{草鞋}{わら|ぢ}を、
\ruby{今日}{け|ふ}から
\ruby{君}{きみ}に
\ruby{穿}{は}いて
\ruby{貰}{もら}つて、
\ruby{君}{きみ}だけに
\ruby{歩}{ある}いて
\ruby{貰}{もら}ふやうになつたなあァ、
\ruby{嬉}{うれ}しい!。
サア
\ruby[g]{{\GWI{u7fbd-k}\換字{勝}}君}{はがちくん}!、これからだ。
ウンと
\ruby{大股}{おほ|また}に
\ruby{踏張}{ふん|ば}つてくれ!。
\ruby{君}{きみ}の
\ruby{腿骨}{すねつ|ぽね}の
\ruby{{\GWI{u9054-k}}者}{たつ|しや}なところと、
\ruby{男兒振}{をと|こ|ぶ}りの
\ruby{好}{い}いところを
\ruby{見}{み}せて
\ruby{{\換字{呉}}}{く}れたまへ。
ナア
\ruby[g]{{\GWI{u7fbd-k}\換字{勝}}君}{はがちくん}!。
』

と、これは
\ruby{{\GWI{u98fd-k}}}{あく}まで
\ruby{醉}{よひ}に
\ruby{乘}{じよう}じて
\ruby{碎}{くだ}けて
\ruby{云}{い}へど、
\ruby[g]{{\GWI{u7fbd-k}\換字{勝}}}{はがち}は
\ruby{醉}{よ}うて
\ruby{醉}{よ}はぬ
\ruby{姿勢}{し|せい}さへ
\ruby{正}{ただ}しく、
\ruby{堅固}{けん|ご}の
\ruby{言葉}{こと|ば}つき
\ruby{力{\換字{強}}}{ちから|づよ}く、

『ム。
\ruby{悉皆了解}{しつ|かい|れう|かい}した。
\ruby{確}{たしか}に
\ruby{承諾}{しよう|だく}した。
\ruby{面白}{おも|しろ}い。
\ruby{行}{や}れるだけは
\ruby{屹}{きつ}と
\ruby{行}{や}る
\ruby[g]{{\GWI{u7fbd-k}\換字{勝}}}{はがち}だ!。
\ruby{{\GWI{u904b-k}}}{うん}が
\ruby{逃}{に}げれば
\ruby{{\GWI{u904b-k}}}{うん}を
\ruby[g]{{\GWI{u8ffd-k}}尾}{おつか}ける!。
たとひ
\ruby{草鞋}{わら|ぢ}は
\ruby{穿}{は}き
\ruby{切}{き}つても、
\ruby{歩}{ある}きだしたら
\ruby{必}{かなら}ず
\ruby{歩}{ある}く。
\ruby{中{\GWI{u9014-k}}}{ちゆ|うと}では
\ruby{休}{やす}まぬ、
\ruby{{\GWI{u904b-k}}}{うん}は
\ruby{摑}{つか}む!。
\ruby{其代}{その|かは}り
\ruby{悉皆屹度任}{みん|な|きつ|と|まか}せて
\ruby{{\換字{呉}}}{く}れ。
』

と、
\ruby{云}{い}ひも
\ruby{{\GWI{u7d42-ue0101}}}{おは}らぬに
\ruby{島木}{しま|き}は
\ruby{烈}{はげ}しく、

『オヽ、
\ruby{任}{まか}せないで
\ruby{何}{なん}とするもんだ。
\ruby{屹度頼}{きつ|と|たの}んだぞ!。
』

と、
\ruby{口}{くち}を
\ruby{衝}{つ}いて
\ruby{答}{こた}へたり。

『ムッ、
\ruby{頼}{たの}まれたぞ。
』

『オヽ、
\ruby{頼}{たの}んだぞ。
』

『さあ
\ruby{始}{はじ}まつたぞ!。
』

『
\ruby{双六}{すご|ろく}が』

『ハヽハヽ。
』

『ハヽハヽ。
』

\ruby[g]{玻璃戔}{こつぷ}は
\ruby[g]{玻璃戔}{こつぷ}とカチリと
\ruby{觸}{あた}つて、
\ruby{酒}{さけ}は
\ruby{二人}{ふ|たり}に
\ruby{一時}{いちじ|}に
\ruby{仰}{あふ}がれたり。


\Entry{其三十一}

\ruby{授業}{じゆ|げふ}も
\ruby{爲}{な}し
\ruby{難}{がた}く
\ruby{見}{み}えたるほどの
\ruby{暴風}{あ|れ}の
\ruby{一日}{いち|にち}の
\ruby{生{\換字{暖}}}{なまあ|たゝか}きに、
\ruby[g]{{\換字{平}}生}{いつも}の
\ruby{如}{ごと}く
\ruby{{\換字{教}}鞭}{けう|べん}を
\ruby{執}{と}りて
\ruby{太郎次郎}{た|らう|じ|らう}を
\ruby{相手}{あひ|て}に
\ruby{仕}{し}たりし
\ruby{水野}{みづ|の}は、
\ruby{我}{が}の
\ruby{{\換字{強}}}{つよ}きところあれば
\ruby[g]{職務}{つとめ}を
\ruby{怠}{おこた}りこそは
\ruby{爲}{せ}ざりつれ、
\ruby{前日}{ぜん|じつ}よりの
\ruby{心身}{しん|〴〵}の
\ruby{疲}{つか}れに、
\ruby{五體}{ご|たい}の
\ruby{綿}{わた}の
\ruby{如}{ごと}くなれるを
\ruby{我}{われ}と
\ruby{覺}{おぼ}えつゝ、やうやく
\ruby{午後}{ご|ゞ}
\ruby{何時}{なん|じ}の
\ruby{今}{いま}、
\ruby{始}{はじ}めて
\ruby{我}{わ}が
\ruby{身}{み}の
\ruby{我}{わ}が
\ruby{物}{もの}となりたる
\ruby{心地}{こゝ|ち}する
\ruby{氣}{き}の
\ruby{{\換字{緩}}}{ゆる}みに、
\ruby[g]{歩調}{あるきつき}さへ
\ruby{遲々}{ち|ゝ}として、
\ruby[g]{{\換字{脱}}力}{がつかり}して
\ruby{歸}{かへ}り
\ruby{來}{きた}れり。

\ruby{手}{て}を
\ruby{掛}{か}けたるにはあらねど
\ruby{小}{ちひ}さき
\ruby{樹草}{き|くさ}など
\ruby{好}{よ}きほどに
\ruby{生}{は}えたればおのづからの
\ruby{庭}{には}となりたる
\ruby{空地}{あき|ち}を
\ruby{前}{まへ}に、
\ruby{南}{みなみ}を
\ruby{受}{う}けたる
\ruby{長}{なが}き
\ruby{一棟}{ひと|むね}の、
\ruby{其}{そ}の
\ruby{奧}{おく}の
\ruby{一間}{ひと|ま}は
\ruby{我}{わ}が
\ruby{起臥}{おき|ふし}のところと
\ruby{定}{さだ}まりたるなり。
\ruby{水野}{みづ|の}は
\ruby{常}{つね}の
\ruby{如}{ごと}く
\ruby{庭先}{には|さき}を
\ruby{家}{いへ}に
\ruby{{\換字{沿}}}{そ}ひて
\ruby{{\換字{廻}}}{まは}りて、
\ruby{椽}{ゑん}より
\ruby{直}{たゞち}に
\ruby{座敷}{ざ|しき}に
\ruby{上}{あが}らんとするに、
\ruby{今日}{け|ふ}は
\ruby{烈}{はげ}しき
\ruby{風}{かぜ}を
\ruby{厭}{いと}ひて、
\ruby{雨{\換字{戸}}}{あま|ど}さへ
\ruby{幾枚}{いく|まい}か
\ruby{引}{ひ}かれ
\ruby{居}{ゐ}たり。

『ア、
\ruby{風}{かぜ}が
\ruby{甚}{ひど}いので
\ruby{雨{\換字{戸}}}{あま|ど}を
\ruby{引}{ひ}いて
\ruby{置}{お}きました。
\ruby{薄暗}{うすつ|くら}くつてお
\ruby{{\換字{嫌}}}{いや}なら
\ruby{明}{あ}けてあげませう。
\ruby{方向}{む|き}が
\ruby{好}{い}いので
\ruby{此家}{こ|ゝ}は
\ruby{其程}{それ|ほど}ぢやあ
\ruby{有}{あ}りませんが、
\ruby{何}{なに}にしろ
\ruby{甚}{ひど}い
\ruby{{\換字{嫌}}}{いや}な
\ruby{風}{かぜ}です。
』

\ruby{我}{わ}が
\ruby{跫音}{あし|おと}を
\ruby{聞}{きゝ}つけての
\ruby{吉右衛門}{き|ち|ゑ|もん}が
\ruby{言葉}{こと|ば}に、

『なあに
\ruby{今日}{け|ふ}は
\ruby{別}{べつ}に
\ruby{細字書}{こま|かい|ほん}を
\ruby{讀}{よ}まうとも
\ruby{思}{おも}はないから、
\ruby[g]{矢張}{やつぱ}り
\ruby{此儘}{この|まゝ}にして!。
』

と
\ruby{云}{い}いながら
\ruby{水野}{みづ|の}は
\ruby{身}{み}を
\ruby{側}{そば}めて、
\ruby{隙}{す}かして
\ruby{引}{ひ}かれたる
\ruby{{\換字{戸}}}{と}の
\ruby{間}{すそ}より
\ruby{上}{あが}り、

『ほんとに
\ruby{氣持}{き|もち}の
\ruby{惡}{わる}い、
\ruby{頭}{あたま}の
\ruby{痛}{いた}くなるやうな
\ruby{風}{かぜ}で、---
\ruby{早}{はや}く
\ruby{止}{や}んで
\ruby{{\換字{呉}}}{く}れなくちやあ
\ruby{仕方}{し|かた}が
\ruby{無}{な}い。
』

と
\ruby{座敷}{ざ|しき}に
\ruby{入}{い}りつゝ
\ruby{言葉}{こと|ば}を
\ruby{足}{た}せば、

『
\ruby{左樣}{さ|う}でございます。
\ruby{雨}{あめ}が
\ruby{{\換字{随}}}{つ}いて
\ruby{來}{こ}ないで
\ruby{先々}{まあ|〳〵}ですが、
\ruby[g]{土地}{ところ}によつちやあ
\ruby{餘程}{よ|ほど}の
\ruby[g]{損害}{いたみ}です。
この
\ruby{{\換字{嫌}}}{いや}に
\ruby{{\換字{暖}}}{あたゝか}い
\ruby{事}{こと}は
\ruby{何樣}{ど|う}でしやう。
\ruby{病人}{びやう|にん}なんぞにやあ
\ruby{感}{き}きますネ。
オ、
\ruby{病人}{びやう|にん}と
\ruby{云}{い}やあ
\ruby{今朝}{け|さ}お
\ruby{頼}{たの}みの
\ruby{婢}{をんな}は、
\ruby{私}{わたし}の
\ruby{本家}{う|ち}の
\ruby{方}{はう}の
\ruby{小作人}{こ|さく|にん}の
\ruby{娘}{むすめ}で、がせいに
\ruby{能}{よ}く
\ruby{働}{はたら}くのがありましたから、
\ruby{能}{よ}く
\ruby{云}{い}ひつけて
\ruby{其}{それ}を
\ruby{{\換字{遣}}}{や}つて
\ruby{置}{お}きました。
\ruby{看護婦}{かん|ご|ふ}さんも
\ruby{來}{き}たさうです。
』

と、
\ruby{間}{あひ}の
\ruby{襖}{ふすま}は
\ruby{開}{ひら}き
\ruby{居}{ゐ}たる
\ruby{中}{なか}の
\ruby{間}{ま}にありて
\ruby{敷居越}{しき|ゐ|ご}しの
\ruby{挨拶}{あい|さつ}なり。
\ruby{水野}{みづ|の}は
\ruby{床近}{とこ|ちか}く
\ruby{置}{お}きたる
\ruby{机}{つくゑ}の
\ruby{前}{まへ}に
\ruby{坐}{すわ}りて、
\ruby{始}{はじ}めて
\ruby{昨日以來}{きの|ふ|い|らい}の
\ruby[g]{疲勞}{つかれ}を
\ruby{息}{やす}めつゝ、

『アヽ、
\ruby{今}{いま}
\ruby{一寸}{ちよ|つと}
\ruby{歸路}{かへ|り}に
\ruby{立寄}{たち|よ}つて
\ruby{來}{き}ました。
いろ〳〵お
\ruby{世話}{せ|わ}を
\ruby{有}{あ}り
\ruby{難}{がた}かつた。
\ruby{先}{まあ}これで
\ruby{一切思}{いつ|さい|おも}ふやうになつた。
』

と、
\ruby{重荷}{おも|に}を
\ruby{卸}{おろ}したるが
\ruby{如}{ごと}き
\ruby{顏色}{かほ|つき}すれば、
\ruby{例}{れい}の
\ruby{眼鏡}{め|がね}の
\ruby{中}{うち}より
\ruby{一寸}{ちよ|つと}
\ruby{見}{み}て、

『
\ruby{昨夜}{ゆふ|べ}は
\ruby{碌}{ろく}に
\ruby{御睡眠}{お|よ|り}はなさりますまいのに、
\ruby{今日}{け|ふ}は
\ruby{{\換字{又}}{\換字{平}}生}{また|い|つも}の
\ruby{{\換字{通}}}{とほ}り
\ruby{御勤務}{お|つ|とめ}では、
\ruby{大抵}{たい|てい}な
\ruby{御疲勞}{お|くた|びれ}ではありますまい。
\ruby{今夜}{こん|や}はまあ
\ruby{早}{はや}く
\ruby{御睡眠}{お|やす|み}なさいまし。
』

と、
\ruby{云}{い}ひさして
\ruby{茶}{ちや}の
\ruby{間}{ま}の
\ruby{方}{かた}を
\ruby{顧}{かへり}みて
\ruby{聲大}{こゑ|おほき}く、

『お
\ruby{濱}{はま}や。
また
\ruby{其樣}{そ|ん}なに
\ruby{書}{ほん}にばかり
\ruby{取付}{とつ|つ}いて
\ruby{居}{ゐ}ちやあいけない。
\ruby{先生}{せん|せい}がお
\ruby{歸}{かへ}りなすつたぢやあ
\ruby{無}{な}いか、
\ruby{御茶}{お|ちや}を
\ruby{持}{も}つて
\ruby{來}{こ}ないか。
』

と、
\ruby{悠然}{ゆつ|くり}としたる
\ruby{調子}{てう|し}に
\ruby{呼}{よ}ばゝつたるは、
\ruby{言葉}{こと|ば}つきなども
\ruby{異}{をか}しからぬほど
\ruby{江戸}{え|ど}の
\ruby{水}{みず}も
\ruby{飮}{の}んだる
\ruby{果}{はて}の
\ruby{老夫}{おや|ぢ}なれど、
\ruby{流石}{さす|が}は
\ruby{根}{ね}が
\ruby{此}{こ}の
\ruby{邊}{あたり}の
\ruby{田舎風}{ゐ|なか|ふう}なり。

\ruby{小}{ちい}さき
\ruby{刳{\換字{盆}}}{くり|ぼん}に
\ruby{大}{おほき}なる
\ruby{筒茶碗載}{つゝ|ぢあ|わん|の}せて、
\ruby{嫣然}{につ|こり}と
\ruby{笑}{ゑ}みて
\ruby{持出}{もち|い}でたるお
\ruby{濱}{はま}は、
\ruby{水野}{みづ|の}が
\ruby{膝近}{ひざ|ちか}くそれを
\ruby{置}{お}きて、おのれは
\ruby{祖父}{ぢ|ゞ}の
\ruby{傍}{かたへ}に
\ruby{甘}{あま}えるやうに
\ruby{坐}{すわ}り。

『
\ruby[g]{昨夜}{ゆふべ}は
\ruby{怖}{こは}かつたでしようねえ、
\ruby{眞闇}{まつ|くら}で!。
あれから
\ruby{妾床}{わたし|とこ}へ
\ruby{入}{はい}つたら、
\ruby{先生}{せん|せい}の
\ruby{行}{いら}しつた
\ruby{方}{はう}の、
\ruby{{\換字{遠}}}{とほ}くの
\ruby{{\換字{遠}}}{とほ}くから、
\ruby{狗}{いぬ}の
\ruby{鳴}{な}く
\ruby{聲}{こゑ}が
\ruby{聞}{きこ}えて
\ruby{來}{き}て、
\ruby{淋}{さび}しかつたわ!。
』

と
\ruby{云}{い}ひ
\ruby{出}{だ}せば、

『ハヽヽ、
\ruby{何}{な}んだ
\ruby{下}{くだ}らない、
\ruby{叩頭}{おじ|ぎ}も
\ruby{仕}{し}ないで!。
\ruby{突然}{いき|なり}と
\ruby{其樣}{そ|ん}な
\ruby{事}{こと}を
\ruby{云}{い}ひ
\ruby{出}{だ}すよ。
\ruby{狗}{いぬ}が
\ruby{鳴}{な}いたつて
\ruby{何淋}{なに|さみ}しい
\ruby{奴}{やつ}があるもんか。
』

と
\ruby{笑}{わらひ}を
\ruby{帶}{お}びて
\ruby{吉右衛門}{き|ち|ゑ|もん}は
\ruby{叱}{しか}るを、
\ruby[g]{眞赤}{まつか}なる
\ruby{番茶}{ばん|ちや}の
\ruby{味}{あじ}も
\ruby{無}{な}く
\ruby{香}{か}も
\ruby{無}{な}けれど、
\ruby{熱}{あつ}きのみに
\ruby{人}{ひと}の
\ruby{{\換字{情}}}{なさけ}は
\ruby{有}{あ}るを
\ruby{啜}{すゝ}れる
\ruby{水野}{みづ|の}は、

『ハヽヽ、お
\ruby{濱}{はま}ちやんはいつでも
\ruby{面白}{おも|しろ}い
\ruby{事}{こと}を
\ruby{云}{い}ふ!。
そして
\ruby[g]{昨夜}{ゆふべ}は
\ruby{一生懸命}{いつ|しやう|けん|めい}に
\ruby{書}{ほん}を
\ruby{讀}{よ}んで
\ruby{居}{ゐ}たぢやあ
\ruby{無}{な}いか、あれは
\ruby{一體何}{いつ|たい|なん}の
\ruby{本}{ほん}だえ。
』

と
\ruby{問}{と}ふに、お
\ruby{濱}{はま}は
\ruby{忽}{たちま}ち
\ruby{不足}{ふ|そく}らしき
\ruby{恨}{うら}みを
\ruby{其色}{その|いろ}に
\ruby{現}{あら}はしたり。

『だつて
\ruby{先}{せん}の
\ruby{中}{うち}は
\ruby[g]{毎晩々々}{まいばん〳〵}いろんな
\ruby{面白}{おも|しろ}いお
\ruby{譚}{はなし}を
\ruby{仕}{し}て
\ruby{聞}{き}かして
\ruby{下}{くだ}すつたのに、
\ruby{此{\換字{節}}}{この|せつ}は
\ruby{毫}{ちつと}も
\ruby[g]{御談話}{おはなし}なんぞして
\ruby{下}{くだ}さらないんだもの。
\ruby{妾}{わたし}はほんとに
\ruby{詰}{つ}まらなくつて、
\ruby{仕方}{し|かた}がないから
\ruby{本家}{う|ち}から
\ruby{書}{ほん}を
\ruby{持}{も}つて
\ruby{來}{き}て
\ruby{讀}{よ}んで
\ruby{居}{ゐ}るのよ。
』

『でも
\ruby{書}{ほん}がおもしろけりやあ
\ruby{可}{いゝ}ぢやあ
\ruby{無}{な}いか、
\ruby{私}{わたし}の
\ruby{不器用}{ぶ|き|よう}な
\ruby{談話}{はな|し}なんぞより。
』

\ruby{頭髪}{か|み}もゆら〳〵と
\ruby{頭}{かうべ}を
\ruby{振}{ふ}つて、

『イヽエ、
\ruby{矢張}{やつ|ぱ}り
\ruby[g]{御談話}{おはなし}の
\ruby{方}{はう}が
\ruby{妾}{わたし}
\ruby{好}{す}きなのよ。
あの
\ruby{本}{ほん}は
\ruby{面白}{おも|しろ}い
\ruby{事}{こと}は
\ruby{面白}{おも|しろ}いけれど、むづかしくつていけないところが
\ruby{有}{あ}るんですもの!。
\ruby{今夜}{こん|や}は
\ruby{何處}{どつ|こ}へも
\ruby{行}{い}かないで
\ruby{御話}{おは|なし}を
\ruby{仕}{し}て。
ネ、
\ruby{御願}{おね|がひ}ですから
\ruby{泣}{な}くやうなのを!。
\ruby{妾}{わたし}
\ruby{泣}{ な}くやうな
\ruby[g]{御話}{おはなし}が
\ruby{大好}{だい|す}きなのよ。
』

と
\ruby{{\換字{遠}}慮}{ゑん|りよ}も
\ruby{無}{な}く
\ruby{{\換字{強}}請}{ね|だ}れば
\ruby{吉右衛門}{き|ち|ゑ|もん}は
\ruby{苦}{にが}りて、

『また
\ruby{其樣}{そ|ん}なに
\ruby{直汝}{ぢき|おまへ}は
\ruby{甘}{あま}つたれるよ!。
そんな
\ruby{氣樂}{き|らく}な
\ruby{事}{こと}どころぢやあ
\ruby{無}{な}くつてゐらつしやるのだ。
』

と、
\ruby{少}{すこ}し
\ruby{叱}{しか}り
\ruby{氣味}{ぎ|み}に
\ruby{{\換字{遮}}}{さへぎ}り
\ruby{止}{とど}むるに、

『アヽ
\ruby{妾知}{わたし| し}つてますよ、
\ruby{五十子}{い|そ|こ}さんが
\ruby{惡}{わる}いから\換字{?!}。
\ruby{妾}{わたし}
\ruby{今日}{ け|ふ}
\ruby{見}{み}て
\ruby{來}{き}てよ
\ruby{五十子}{い|そ|こ}さんを。
ほんとに
\ruby{憫然}{かはい|さう}に
\ruby{病重}{わ|る}いのねえ。
』

と、
\ruby{然}{さ}も
\ruby{心配氣}{しん|ぱい|げ}に
\ruby{艶}{つや}やかなる
\ruby{面}{おもて}の
\ruby{美}{うつく}しき
\ruby{眉}{まゆ}を
\ruby{打顰}{うち|しか}めたる、
\ruby{云}{い}ふに
\ruby{云}{い}はれぬ
\ruby{可愛}{か|はい}さありて、
\ruby{此室}{こ|ゝ}ばかりには
\ruby{騒}{さわ}がしき
\ruby{風}{かぜ}も
\ruby{吹}{ふ}かぬが
\ruby{如}{ごと}し。


\Entry{其三十二}

『
\ruby{人}{ひと}にも
\ruby{云}{い}はないで
\ruby{何時}{い|つ}の
\ruby{間}{ま}に
\ruby{岩崎}{いは|ざき}さんのところへ
\ruby{行}{い}つて
\ruby{見}{み}たのだ。
\ruby{彼方}{あち|ら}ぢやあ
\ruby{御煩}{お|うるさ}く
\ruby{御思}{お|おも}ひだらうに!。
』

『いゝえ
\ruby{祖{\換字{父}}}{お|ぢい}さん、
\ruby{一寸}{ちよ|つと}
\ruby{行}{い}つたばかしで、
\ruby{上}{あが}りも
\ruby{何}{なに}も
\ruby{仕}{し}やあ
\ruby{仕}{し}ないのよ!。
たゞそーつと
\ruby{外}{そと}から
\ruby{見}{み}たばかりなの。
だけれども
\ruby{肥}{ふと}つた
\ruby{看護{\換字{婦}}}{かん|ご|ふ}さんも
\ruby{見}{み}たし、
\ruby{丁度}{ちやう|ど}
\ruby{松}{まつ}ちやんにも
\ruby{會}{あ}つて
\ruby{話}{はなし}を
\ruby{仕}{し}て
\ruby{來}{き}たのよ。
\ruby{松}{まつ}ちやんは
\ruby{曩日}{いつ|か}
\ruby{吾家}{う|ち}で
\ruby{一{\換字{所}}}{いつ|しよ}に
\ruby{{\換字{遊}}}{あそ}んだ
\ruby{時}{とき}なんかとは
\ruby{{\換字{違}}}{ちが}つて、
\ruby{泣}{な}きさうな
\ruby{顏}{かほ}を
\ruby{仕}{し}て
\ruby{居}{ゐ}るんだもの、
\ruby{妾}{わたし}ほんとに
\ruby{憫然}{かは|いさう}になつちまつたの!。% 「憫然 か(は)いさう」
だもんだから
\ruby{彼}{あ}の
\ruby{椎}{しひ}の
\ruby{樹}{き}の
\ruby{傍}{そば}で、
\ruby{二人}{ふた|り}でつい
\ruby{泣}{な}いて
\ruby{話}{はなし}を
\ruby{仕}{し}て
\ruby{居}{ゐ}たら、
\ruby{彼家}{あす|こ}の
お
\ruby{澤}{さは}
\ruby{婆}{ばゞあ}つたら
\ruby{眞箇}{ほん|と}に
\ruby{憎}{にく}らしい!、
お
\ruby{濱子}{はま|つこ}!、
\ruby{汝}{おめへ}まで
\ruby{心配}{しん|ぱい}して
\ruby{居}{ゐ}るだけえ?、だけれど
\ruby{泣}{な}いたつて
\ruby{無益}{だ|め}なこんだ!。
\ruby{心配}{しん|ぱい}で
\ruby{癒}{なほ}る
\ruby{病氣}{びやう|き}あ
\ruby{無}{ね}えだから、つて
\ruby{{\換字{菜}}圃}{はた|け}の
\ruby{對}{むかふ}から
\ruby{大}{おほき}な
\ruby{聲}{こゑ}をして
\ruby{怒鳴}{ど|な}るんだもの!。
\ruby{妾}{わたし}ほんとに
\ruby{口惜}{く|や}しくつて
\ruby{口惜}{く|や}しくつて、
\ruby{風}{かぜ}の
\ruby{中}{なか}を
\ruby{駈}{か}け
\ruby{出}{だ}して
\ruby{歸}{かへ}つて
\ruby{來}{き}て
\ruby{一人}{ひと|り}で
\ruby{怒}{おこ}つて
\ruby{泣}{な}いたわ。
ほんとに
\ruby{彼樣}{あ|ん}な
\ruby{意地惡}{い|ぢ|わる}な
\ruby{婆}{ばゞあ}つたら
\ruby{有}{あ}りや
\ruby{仕}{し}ない!。
\ruby{今度}{こん|ど}また
\ruby{彼樣}{あ|ん}な
\ruby{事}{こと}を
\ruby{云}{い}つたら
\ruby{引爬}{ひつ|か}いて
\ruby{{\換字{遣}}}{や}らなくつちやあ。
』

『ハヽヽ、また
\ruby{其樣}{そ|ん}な
お
\ruby{轉婆}{てん|ば}な
\ruby{事}{こと}をいふよ!。
\ruby{何樣}{ど|う}して〳〵
\ruby{彼}{あ}の
\ruby{婆}{ばあ}さんにやあ
\ruby{汝}{おまへ}なんぞの
\ruby{爪}{つめ}も
\ruby{立}{た}つもんぢやあ
\ruby{無}{な}い。
\ruby{婆}{ばあ}さんを
\ruby{引爬}{ひつ|か}きやあ
\ruby{汝}{おまへ}の
\ruby{爪}{つめ}は
\ruby{悉皆}{みん|な}
\ruby{脫}{と}れたつて、
\ruby{彼方}{むか|ふ}にやあ
\ruby{蚯蚓脹}{みゝ|ず|ばれ}も
\ruby{出來}{で|き}や
\ruby{仕}{し}ない。
そんな
\ruby{事}{こと}はまあ
\ruby{何樣}{ど|う}でも
\ruby{可}{い}いが、もうそろ〳〵と
\ruby{日}{ひ}が
\ruby{暮}{く}れかゝる、
お
\ruby{鍋}{なべ}が
\ruby{何}{なに}かことつかせて
\ruby{居}{ゐ}る、
\ruby{汝}{おまへ}も
\ruby{彼方}{あつ|ち}へ
\ruby{行}{い}つて
\ruby{夕方}{ゆふ|がた}の
\ruby{事}{こと}を、
\ruby{些}{ちつと}は
\ruby{傍}{そば}から
\ruby{手傳}{て|つだ}つて
\ruby{{\換字{遣}}}{や}りナ。
』

『
\ruby{先生}{せん|せい}が
\ruby{今夜}{こん|や}
\ruby{面白}{おも|しろ}い
\ruby{御話}{お|はなし}を
\ruby{仕}{し}て
\ruby{下}{くだ}さるなら。
』

『
\ruby{祖{\換字{父}}}{お|ぢい}さんが
\ruby{命令}{いひ|つけ}るのに
\ruby{先生}{せん|せい}のところへ
\ruby{掛}{かゝ}つて
\ruby{行}{い}くとは、
\ruby{何}{なん}だか
\ruby{理由}{わ|け}の
\ruby{{\換字{分}}}{わか}らない
\ruby{理屈合}{り|くつ|あひ}だナ。
サアマア
\ruby{何}{なん}でも
\ruby{可}{い}いから
\ruby{御働}{お|はたら}き、
お
\ruby{働}{はたら}き!』

『ハイ。
ぢやあ
\ruby{先生}{せん|せい}
\ruby{屹度}{きつ|と}
\ruby{後刻}{の|ち}に
\ruby{先日}{この|あひだ}の
\ruby{御話}{お|はなし}の
\ruby{續}{つゞ}きをネ。
』

\ruby{頭}{くび}を
\ruby{曲}{ま}げて
\ruby{水野}{みづ|の}の
\ruby{顏}{かほ}を
\ruby{覗}{のぞ}き
\ruby{{\換字{込}}}{こ}むやうにして
\ruby{自己}{お|の}が
\ruby{{\換字{勝}}手}{かつ|て}を
\ruby{云}{い}ひつつ
お
\ruby{濱}{はま}は
\ruby{纔}{わづか}に
\ruby{彼方}{かな|た}に
\ruby{去}{さ}りたり。

\ruby{祖{\換字{父}}}{ぢ|ゞ}は
\ruby{孫娘}{まご|むすめ}の
\ruby{背姿}{うしろ|すがた}を
\ruby{見}{み}おくりながら、

『
\ruby{身長}{せ|い}ばかり
\ruby{彼樣}{あ|ん}なに
\ruby{大}{おほ}きくなつて、いつまで
\ruby{彼樣}{あ|ん}な
\ruby{調子}{てう|し}で
\ruby{居}{ゐ}るのでしやう!。
もう
\ruby{少}{すこ}しは
\ruby{女}{をんな}らしくなりさうなものですのに、あゝやんちやんでは
\ruby{仕方}{し|かた}が
\ruby{有}{あ}りません。
いくらお
\ruby{澤}{さは}
\ruby{婆}{ばあ}さんが
\ruby{憎}{にく}いと
\ruby{云}{い}つたつて、
\ruby{引爬}{ひつ|か}いて
\ruby{{\換字{遣}}}{や}らうなんて、ハヽハヽハヽ。
』

と
\ruby{獨語}{ひとり|ごと}の
\ruby{如}{ごと}く
\ruby{{\換字{又}}}{また}
\ruby{辯護}{べん|ご}の% 弁 瓣 辦 辧 辨 辩 (辯)
\ruby{如}{ごと}く
\ruby{云}{い}へば、
\ruby{其}{そ}の
\ruby{語}{ことば}に
\ruby{隨}{つ}いて、

『\換字{志}かしお
\ruby{澤}{さは}といふ
\ruby{婆}{ばあ}さんは
\ruby{眞箇}{ほん|と}に
\ruby{甚}{ひど}い!。
\ruby{何樣}{ど|う}した
\ruby{人}{ひと}だか
\ruby{知}{し}らないが、
\ruby{全}{まる}で
\ruby{普{\換字{通}}}{ひと|なみ}ぢやあ
\ruby{無}{な}い、
\ruby{先}{まあ}
\ruby{鬼婆}{おに|ばゞあ}だから、
\ruby{誰}{たれ}だつて
\ruby{何樣}{ど|う}か
\ruby{仕}{し}て
\ruby{{\換字{遣}}}{や}りたい
\ruby{位}{ぐらゐ}には
\ruby{思}{おも}はうぢやあ
\ruby{無}{な}いか。
』

と、
\ruby{水野}{みづ|の}は
\ruby{我}{わ}が
\ruby{思}{おも}へるところを
\ruby{打}{う}ち
\ruby{出}{いだ}したり。

『
\ruby{貴君}{あな|た}も
\ruby{何}{なに}かで
\ruby{御腹立}{お|はら|だち}でしたネ。
\ruby{其}{そり}やあ
\ruby{左樣}{そ|う}でございますとも、
\ruby{普{\換字{通}}}{な|み}ぢやあ
\ruby{有}{あ}りません!。
\ruby{仰}{おつし}ある
\ruby{{\換字{通}}}{とほ}り
\ruby{鬼}{おに}になつて
\ruby{居}{ゐ}るのですから!。
あれでも
\ruby{舊}{もと}は
\ruby{人}{ひと}の
\ruby{好}{い}い
\ruby{婆}{ばあ}さんでしたが、
\ruby{親一人}{おや|ひと|り}
\ruby{娘一人}{こ|ひと|り}の
\ruby{秘蔵娘}{ひ|ざう|むすめ}の、
お
\ruby{里}{さと}といふのに
\ruby{婿}{むこ}を
\ruby{取}{と}つたー
\ruby{其婿}{その|むこ}が
\ruby{惡}{わる}かつたところから
\ruby{彼樣}{あ|ゝ}なつたのです。
』

『フーン。
』

『
\ruby{婿}{むこ}は
\ruby{兵作}{ひやう|さく}といふ
\ruby{惡}{わる}い
\ruby{奴}{やつ}で、
\ruby{今}{いま}は
\ruby{東京}{とう|きやう}の
\ruby{牛{\換字{込}}}{うし|ごめ}あたりに、
\ruby{樂}{らく}な
\ruby{生活}{くら|し}を
\ruby{仕}{し}て
\ruby{居}{ゐ}るさうですが、
\ruby{出}{で}は
\ruby{二合{\換字{半}}領}{に|がふ|はん|りやう}の
\ruby{可成}{か|なり}な
\ruby{大盡}{だい|じん}の
\ruby{二番生}{に|ばん|ばえ}で、
\ruby{男振}{をとこ|ぶり}の
\ruby{惡}{わる}くない
\ruby{應對}{おう|たい}の
\ruby{上手}{じやう|ず}な
\ruby{男}{をとこ}です。
\ruby{婆}{ばあ}さんの
\ruby{家}{うち}は
\ruby{村}{むら}でも
\ruby{指折}{ゆび|をり}の
\ruby{物持}{もの|もち}でしたが、
\ruby{其}{そ}の
\ruby{兵作}{ひやう|さく}といふのが
\ruby{猫}{ねこ}を
\ruby{被}{かぶ}つた
\ruby{{\換字{狼}}}{おほかみ}でして、
\ruby{何}{なに}を
\ruby{爲}{す}る、
\ruby{彼}{か}を
\ruby{爲}{す}ると
\ruby{云}{い}つては
\ruby{金}{かね}を
\ruby{持出}{もち|だ}し、
\ruby{{\換字{終}}}{しまひ}には
\ruby{家屋敷}{いへ|や|しき}まで
\ruby{抵當}{てい|たう}に
\ruby{打{\換字{込}}}{ぶち|こ}んだのです。
\換字{志}かし
\ruby{其}{それ}が
\ruby{眞實}{ほん|と}に
\ruby{商賣事}{しやう|ばい|ごと}で
\ruby{損}{そん}を
\ruby{仕}{し}たといふなら
\ruby{未}{ま}だ
\ruby{好}{よ}うございますが、
\ruby{實}{じつ}は
\ruby{婿}{むこ}になる
\ruby{{\換字{前}}}{まへ}から
\ruby{他}{ほか}に
\ruby{{\換字{情}}{\換字{婦}}}{をん|な}が
\ruby{有}{あ}つて、
\ruby{其方}{その|はう}に
\ruby{悉皆}{みん|な}こかしたのです。
\ruby{左樣}{そ|う}して
\ruby{置}{お}いて
\ruby{{\換字{平}}井}{ひら|ゐ}の
\ruby{家}{うち}に
\ruby{塵}{ちり}ツ
\ruby{葉}{ぱ}
\ruby{一}{ひと}つ
\ruby{無}{な}くなつた
\ruby{時{\換字{分}}}{じ|ぶん}に、さあ
\ruby{自{\換字{分}}}{じ|ぶん}が
\ruby{{\換字{逐}}出}{おひ|だ}されて
\ruby{仕舞}{し|ま}ふ
\ruby{心算}{つも|り}で、
\ruby{彼}{あ}の
\ruby{婆}{ばあ}さん
\ruby{親子}{おや|こ}に
\ruby{無理}{む|り}ばかり
\ruby{云}{い}つて、
\ruby{打}{ぶ}ちます、
\ruby{蹴}{け}ます、
\ruby{暴}{あば}れます、
\ruby{散々}{さん|〴〵}に
\ruby{酷}{ひど}い
\ruby{事}{こと}を
\ruby{致}{いた}しました。
それが
\ruby{爲}{ため}に
お
\ruby{里}{さと}が
\ruby{癆瘵氣質}{らう|さい|かた|ぎ}になつて、
\ruby{氣}{き}は
\ruby{異}{をか}\換字{志}くなるし、
\ruby{生}{い}きながら
\ruby{幽靈}{いう|れい}のやうに
\ruby{痩}{や}せて、
\ruby{苦}{くる}しんで〳〵
\ruby{居}{を}りましたが、
\ruby{其中}{その|なか}を
\ruby{畢竟}{とう|〳〵}
\ruby{別}{わか}れ
\ruby{話}{ばなし}を
\ruby{仕}{し}て、
\ruby{兵作}{ひやう|さく}は
\ruby{身}{み}を
\ruby{{\換字{退}}}{の}いて
\ruby{仕舞}{し|ま}ひました。
』

『ヤ、それは
\ruby{恐}{おそ}ろしい
\ruby{酷}{むご}い
\ruby{談}{はなし}で。
』

『それこれでお
\ruby{里}{さと}は
\ruby{死}{し}んで
\ruby{仕舞}{し|ま}ひます。
\ruby{婆}{ばあ}さんは
\ruby{住}{す}んで
\ruby{居}{ゐ}た
\ruby{家}{うち}も
\ruby{{\換字{逐}}出}{おつ|た}てられて、
\ruby{他人}{ひ|と}の
\ruby{物置小屋}{もの|おき|ご|や}を
\ruby{假}{か}りて
\ruby{入}{はい}るやうな
\ruby{始末}{し|まつ}にもなりましたが、それから
\ruby{彼}{あ}の
\ruby{婆}{ばあ}さんは
\ruby{鬼}{おに}のやうになりまして、
\ruby{誰}{たれ}
\ruby{彼}{かれ}の
\ruby{見}{み}さかひ
\ruby{無}{な}く
\ruby{人}{ひと}を
\ruby{疑}{うたが}ひ、
\ruby{一生懸命}{いつ|しやう|けん|めい}に
\ruby{挊}{かせ}いでは
\ruby{一{\換字{文}}二{\換字{文}}}{いち|もん|に|もん}を
\ruby{溜}{た}めて、
\ruby{其錢}{その|ぜに}を
\ruby{苛}{ひど}い
\ruby{高利}{かう|り}で
\ruby{貸}{か}し
\ruby{出}{だ}しました。
\ruby{左樣}{さ|う}して
\ruby{五年六年}{ご|ねん|ろく|ねん}と
\ruby{立}{た}つ
\ruby{内}{うち}に
\ruby{段々太}{だん|〴〵|ふと}りまして、
\ruby{舊}{もと}の
\ruby{自{\換字{分}}}{じ|ぶん}の
\ruby{家}{うち}を
\ruby{取}{と}り
\ruby{{\換字{返}}}{かへ}して
\ruby{手}{て}に
\ruby{入}{い}れたのです。
\ruby{他手}{ひと|で}に
\ruby{渡}{わた}つて
\ruby{居}{ゐ}る
\ruby{中}{うち}に
\ruby{焼}{や}けましたので、
\ruby{母屋}{おも|や}や
\ruby{藏}{くら}は
\ruby{殘}{のこ}つて
\ruby{居}{ゐ}ませんが、
\ruby{丁度}{ちやう|ど}
\ruby{今}{いま}
\ruby{岩崎}{いは|ざき}さんの
\ruby{借}{か}りて
\ruby{居}{ゐ}る
\ruby{室}{へや}が、
\ruby{兵作}{ひやう|さく}を
\ruby{婿}{むこ}に
\ruby{取}{と}つた
\ruby{其初}{その|はじめ}に、
\ruby{老人}{とし|より}は
\ruby{{\換字{若}}}{わか}い
\ruby{夫{\換字{婦}}}{ふう|ふ}に
\ruby{香}{かう}ばしく
\ruby{有}{あ}るまいからつて、
\ruby{自{\換字{分}}}{じ|ぶん}の
\ruby{隱居{\換字{所}}}{いん|きよ|じよ}にと
\ruby{建}{た}てた
\ruby{別室}{はな|れ}で、
\ruby{今}{いま}
\ruby{自{\換字{分}}}{じ|ぶん}の
\ruby{入}{はい}つて
\ruby{居}{ゐ}る
\ruby{汚}{きたな}い
\ruby{家}{うち}は、
\ruby{{\換字{平}}井}{ひら|ゐ}の
\ruby{家}{うち}の
\ruby{榮}{さか}えて
\ruby{居}{ゐ}た
\ruby{頃}{ころ}の
\ruby{雜物小屋}{ざふ|もつ|ご|や}です。
\ruby{左樣}{さ|う}いふ
\ruby{婆}{ばあ}さんですから、
\ruby{今}{いま}ぢやあたゞ、
\ruby{金}{かね}より
\ruby{外}{ほか}に
\ruby{味方}{み|かた}は
\ruby{無}{な}いと
\ruby{思}{おも}つて、まるで
\ruby{鬼}{おに}のやうになり
\ruby{切}{き}つて
\ruby{居}{ゐ}て、
\ruby{村}{むら}の
\ruby{者}{もの}にも
\ruby{憎}{にく}がられりやあ、
\ruby{自{\換字{分}}}{じ|ぶん}も
\ruby{村}{むら}の
\ruby{者}{もの}を
\ruby{對敵}{むか|ふ}にして
\ruby{居}{ゐ}るので
\ruby{云}{い}つて
\ruby{見}{み}りやあ
\ruby{愍然}{かは|いさう}な% 「愍然 か(は)いさう」
\ruby{筋}{すぢ}もあるのです。
』

『
\ruby{大}{おほ}きに、
\ruby{成程}{なる|ほど}!。
』

\ruby{水野}{みづ|の}は
\ruby{此談}{この|はなし}を
\ruby{聞}{き}きて
\ruby{黯然}{あん|ぜん}として、
\ruby{{\換字{情}}}{こゝろ}の
\ruby{傷}{きずつ}ける
\ruby{人}{ひと}の
\ruby{末路}{す|ゑ}の
\ruby{恐}{おそ}ろしさを
\ruby{思}{おも}ひつゝ
\ruby{歎}{たん}ずるところへ、
\ruby{忙}{あはた}だしく
\ruby{人}{ひと}の
\ruby{駈}{か}け
\ruby{來}{く}る
\ruby{跫音}{あし|おと}して、
\ruby{椽{\換字{前}}}{えん|さき}より、

『
\ruby{水野}{みづ|の}さん!
\ruby{水野}{みづ|の}さん!。
』

と
\ruby{呼}{よ}ぶは
\ruby{他人}{ほ|か}ならず
\ruby{松之助}{まつ|の|すけ}なり。

\ruby{其}{その}おろ〳〵したる
\ruby{悲}{かな}しき
\ruby{聲音}{こわ|ね}を
\ruby{聞}{き}くより、
\ruby{何}{なん}とは
\ruby{無}{な}しに
\ruby{胸潰}{むね|つぶ}れて、

『ど、
\ruby{何樣}{ど|う}かしたか?、
\ruby{惡}{わる}いのかえ?、
\ruby{姊}{ねえ}さんが。
』

と、サツと
\ruby{障子}{しやう|じ}を
\ruby{開}{ひら}けば、
\ruby{{\換字{暖}}}{あたゝか}き
\ruby{不快}{ふ|くわい}の
\ruby{風}{かぜ}はムツと
\ruby{吹}{ふ}きて、
\ruby{黄昏}{たそ|がれ}の
\ruby{{\換字{空}}}{そら}の
\ruby{光線}{ひか|り}の
\ruby{{\換字{弱}}}{よわ}きに、
\ruby{恐怖}{おそ|れ}を
\ruby{懷}{いだ}ける
\ruby{松之助}{まつ|の|すけ}の
\ruby{顏}{かほ}は
\ruby{影}{かげ}さへ
\ruby{淋}{さみ}しく
\ruby{薄々}{うす|〳〵}と
\ruby{白}{しら}みて
\ruby{見}{み}えたり。

『
\ruby{大變}{たい|へん}に
\ruby{惡}{わる}い!。
いけないかも
\ruby{知}{し}れ……。
アヽ、
\ruby{僕}{ぼく}あ
\ruby{何樣}{ど|う}したら
\ruby{宜}{よ}からう!。
』

\ruby{既}{はや}
\ruby{泣}{な}き
\ruby{聲}{ごゑ}の、\換字{志}どろもどろの
\ruby{其}{その}
\ruby{言葉}{こと|ば}を
\ruby{聞}{き}くや
\ruby{聞}{き}かずや、
\ruby{水野}{みづ|の}は
\ruby{忽}{たちま}ち
\ruby{全身}{ぜん|しん}に
\ruby{氷}{こほり}の
\ruby{水}{みづ}を
\ruby{{\換字{浴}}}{あ}びし
\ruby{心地}{こゝ|ち}して、アツとばかりに
\ruby{仆}{たふ}れんとしけるが、
\ruby{辛}{から}くも
\ruby{堪}{た}へて
\ruby{自}{みづか}ら
\ruby{保}{たも}ち、
\ruby{次}{つ}いで
\ruby{烈}{はげ}しき
\ruby{戰慄}{ふる|ひ}の
\ruby{止}{と}めても
\ruby{止}{と}まらず
\ruby{起}{おこ}り
\ruby{來}{く}るを
\ruby{{\換字{強}}}{し}ひて
\ruby{制}{せい}しつ、

『ナニ、そんな
\ruby{事}{こと}が……、
\ruby{大{\換字{丈}}夫}{だい|ぢやう|ぶ}だ!。
』

と、
\ruby{我}{わ}が
\ruby{耳}{みゝ}にも
\ruby{知}{し}るゝ
\ruby{顫聲}{ふるひ|ごゑ}に
\ruby{云}{い}いさま、
\ruby{我}{われ}
\ruby{知}{し}らず
\ruby{我}{わ}が
\ruby{座}{ざ}より
\ruby{飛}{と}び
\ruby{立}{た}つて、
\ruby{踵}{かゝと}も
\ruby{地}{ち}に
\ruby{着}{つ}かぬ
\ruby{跣足}{は|だし}の
\ruby{危}{あやふ}く、
\ruby{轉}{まろ}ぶが
\ruby{如}{ごと}くに
\ruby{走去}{はせ|さ}つたり。


\Entry{其三十三}

\ruby{槇籬隣}{まき|がき|とな}る
\ruby{木槿籬}{むく|げが|き}、
\ruby{杉籬}{すぎ|がき}つゞく
\ruby{藪疊}{やぶ|だゝみ}の、
\ruby{村徑}{むら|みち}の
\ruby{黄昏}{たそ|がれ}を
\ruby{息急}{いき|せは}しく
\ruby{走}{はし}る
\ruby{水野}{みづ|の}は、
\ruby{後}{あと}より
\ruby{{\換字{追}}}{お}ひ
\ruby{縋}{すが}れる
\ruby{松之助}{まつ|の|すけ}の
\ruby{手}{て}を
\ruby{引立}{ひつ|た}てゝ、
\ruby{夢}{ゆめ}に
\ruby{高}{たか}きところより
\ruby{落}{お}つるが
\ruby{如}{ごと}き
\ruby{膽縮}{きも|すく}む
\ruby{思}{おも}ひに、
\ruby{何}{なん}の
\ruby{{\換字{分}}別}{ふん|べつ}も
\ruby{無}{な}く
\ruby{駈}{か}けに
\ruby{駈}{か}けたり。

\ruby{今{\換字{朝}}}{け|さ}よりの
\ruby{風}{かぜ}に
\ruby{葉}{は}は
\ruby{裂}{さ}け
\ruby{茎}{くき}は
\ruby{折}{を}れ
\ruby{伏}{ふ}して、
\ruby{滿目}{まん|もく}の
\ruby{光景}{あり|さま}
\ruby{忌}{いま}はしく
\ruby{{\換字{狼}}{\換字{藉}}}{らう|ぜき}たる
\ruby{芋圃}{いも|ばた}の
\ruby{間}{あひだ}を、
\ruby{突}{つ}と
\ruby{行}{ゆ}き
\ruby{拔}{ぬ}けて、
\ruby{例}{れい}の
\ruby{婆}{ばゞ}が
\ruby{家}{いへ}の
\ruby{横}{よこ}を
\ruby{奧}{おく}へと
\ruby{{\換字{通}}}{とほ}らんとすれば、
\ruby{折}{をり}しも
\ruby{例}{れい}の
お
\ruby{澤}{さは}
\ruby{婆}{ばゞ}は、
\ruby{風}{かぜ}に
\ruby{捥}{も}がれたる
\ruby{柹}{かき}の
\ruby{實}{み}の、
\ruby{或}{あるひ}は
\ruby{{\換字{猶}}}{なほ}
\ruby{靑}{あお}く、
\ruby{或}{あるひ}は
\ruby[<h||]{{\換字{半}}}{なかば}
\ruby{黄}{き}ばみ
\ruby{赤}{あか}らめるを、\換字{志}たゝかに
\ruby{取}{と}り
\ruby{入}{い}れたる
\ruby{重}{おも}げなる
\ruby{箕}{み}に、
\ruby{枯柴}{かれ|しば}の
\ruby{如}{ごと}く
\ruby{骨立}{ほね|だ}つたる
\ruby{兩腕}{りやう|ゝで}を
\ruby{長}{なが}く
\ruby{露}{あらは}して
\ruby{掛}{か}けつ、
\ruby{一}{ひ}ト
\ruby{歩}{あし}
\ruby{一}{ひ}ト
\ruby{歩}{あし}に
\ruby{{\換字{強}}欲}{がう|よく}の
\ruby{力}{ちから}を
\ruby{入}{い}れて
\ruby{辛}{から}くも
\ruby{吾家}{わが|や}に
\ruby{{\換字{運}}}{はこ}ばんと、
\ruby{未}{ま}だ
\ruby{止}{や}まぬ
\ruby{風}{かぜ}に
\ruby{霜}{しも}の
\ruby{薄}{すゝき}と
\ruby{騷立}{さわ|だ}つ
\ruby{白髮}{しら|が}を
\ruby{吹}{ふ}き
\ruby{立}{た}たせながら
\ruby{此方}{こ|なた}へ
\ruby{來}{き}かゝりしが、
\ruby{水野}{みづ|の}が
\ruby{慌}{あわ}て
\ruby{{\換字{狼}}狽}{うろ|た}へて
\ruby{入}{い}り
\ruby{來}{きた}れる
\ruby{態}{さま}を、
\ruby{圓}{つぶら}なる
\ruby{眼}{め}にぎろりと
\ruby{見}{み}て、さも
\ruby{心地}{こゝ|ち}よげに
\ruby{冷笑}{あざ|わら}ひ、

『とう〳〵
\ruby{廿兩}{にじう|りやう}になつて
\ruby{來}{き}たゞかネ?。
』

と、
\ruby{恰}{あだか}も% 恰も「あ(だ)かも」
\ruby{病}{や}める
\ruby{人}{ひと}の
\ruby{疾}{と}く
\ruby{死}{し}なんことを
\ruby{待設}{まち|まう}け
\ruby{居}{を}りし
\ruby{其}{そ}の
\ruby{甲{\換字{斐}}}{か|ひ}ありて、
\ruby{今}{いま}や
\ruby{我}{わ}が
\ruby{望}{のぞ}める
\ruby{時機}{と|き}の
\ruby{至}{いた}らんとするに、
\ruby{自}{みづか}ら
\ruby{先}{ま}づ
\ruby{聲}{こゑ}を
\ruby{揚}{あ}げて
\ruby{祝}{しゆく}し
\ruby{悅}{よろこ}べるが
\ruby{如}{ごと}く
\ruby{云}{い}ひぬ。

おのが
\ruby{手}{て}に
\ruby{些少}{すこ|し}ばかりの
\ruby{金子}{か|ね}の
\ruby{落}{お}ちんことを
\ruby{希}{ねが}ふ
\ruby{意}{こゝろ}より、
\ruby{他}{ひと}の
\ruby{生命掛}{いの|ち|か}けて
\ruby{思}{おも}へる
\ruby{人}{ひと}をも
\ruby{死}{し}ねがしに
\ruby{云}{い}ひなしたる
\ruby{此}{こ}の
\ruby{老婆}{ば|ゞ}の
\ruby{面}{つら}の
\ruby{憎}{にく}さ!。
\ruby{人}{ひと}にはあらずと
\ruby{豫}{かね}てより
\ruby{思}{おも}ひ
\ruby{居}{ゐ}たれど、まのあたりに
\ruby{骨}{ほね}を
\ruby{刺}{さ}す
\ruby{此}{こ}の
\ruby{酷毒}{こく|どく}の
\ruby{語}{ことば}を
\ruby{{\換字{浴}}}{あび}せられては、
\ruby{頭脳}{あた|ま}の
\ruby{眞中}{まん|なか}より
\ruby{烈火}{れつ|くわ}の
\ruby{奔}{はし}る
\ruby{心地}{こゝ|ち}して、おのれ
\ruby{憎}{につく}き
\ruby{獸畜}{けだ|もの}め、たゞ
\ruby{一}{ひ}ト
\ruby{攫}{つかみ}に
\ruby{引攫}{ひつ|つか}んで、
\ruby{天狗裂}{てん|ぐ|ざ}きに
\ruby{裂}{さ}きて
\ruby{木}{き}の
\ruby{股高}{また|たか}く
\ruby{掛}{か}けて
\ruby{吳}{く}れんと、むら〳〵と
\ruby{恐}{おそ}ろしき
\ruby{忿怒}{いか|り}の
\ruby{衝}{つ}き
\ruby{上}{あが}り
\ruby{來}{き}て、
\ruby{流石}{さす|が}に
\ruby{堪}{こら}へ
\ruby{{\換字{情}}{\換字{強}}}{じやう|つよ}き
\ruby{水野}{みづ|の}も
\ruby{眞靑}{まつ|さを}になりたり。


\Entry{其三十四}

\ruby{吉右衛門}{き|ち|ゑ|もん}が
\ruby{物語}{もの|がたり}によりて
\ruby{此}{こ}の
\ruby{婆}{ばゝ}が
\ruby{身}{み}の
\ruby{上}{うへ}を
\ruby{聞}{き}かざりせば、
\ruby{或}{あるひ}は
\ruby{走}{はし}りかゝりて
\ruby{一}{ひ}ト
\ruby{踢}{け}に
\ruby{踢倒}{け|たふ}すか、
\ruby{左}{さ}なくば
\ruby{其面}{その|おもて}に
\ruby{唾}{つばき}して
\ruby{罵}{のゝし}るほどの
\ruby{事}{こと}は
\ruby{爲}{し}たらんを、
\ruby{其}{そ}の
\ruby{如是鬼々}{か|く|おに|〳〵}しくなれる
\ruby{所以}{ゆ|ゑん}を
\ruby{思}{おも}ひ
\ruby{{\換字{浮}}}{うか}むると、
\ruby{且}{かつ}は
\ruby{如是老婆}{かゝ|る|ば|ゞ}を
\ruby{相手}{あひ|て}に
\ruby{取}{と}りて
\ruby{何}{なに}となすべき、
\ruby{田}{た}は
\ruby{生}{うま}れて
\ruby{田}{た}に
\ruby{死}{し}する
\ruby{蟲}{むし}にも
\ruby{等}{ひと}しき
\ruby{田舎婆}{ゐ|なか|ばゞ}の
\ruby{一言}{ひと|こと}に、
\ruby{氣}{き}を
\ruby{動}{うご}かして
\ruby{我}{われ}を
\ruby{忘}{わす}れんとしたるは
\ruby{愚}{おろ}かなりと、
\ruby{飽}{あく}まで
\ruby{{\換字{強}}}{つよ}く
\ruby{見下}{み|さ}げたるとに、おのづと
\ruby{心}{こゝろ}も
\ruby{{\換字{緩}}}{ゆる}み
\ruby{和}{やはら}ぎて、
\ruby{水野}{みづ|の}は
\ruby{滿腔}{まん|こう}の
\ruby{燃}{も}ゆる
\ruby[g]{忿恚}{いかり}を
\ruby{僅}{わづか}に
\ruby{怪}{あや}しき
\ruby{侮蔑}{いや|しみ}の
\ruby{笑}{わらひ}に
\ruby{洩}{も}らして、
\ruby{言葉}{こと|ば}も
\ruby{無}{な}く
\ruby{突}{つ}と
\ruby{擦}{す}れ
\ruby{{\換字{違}}}{ちが}つて
\ruby{去}{さ}り
\ruby{行}{ゆ}けば、
\ruby{婆}{ばゞ}は
\ruby{{\換字{猶}}}{なほ}
\ruby{其}{そ}の
\ruby{後姿}{うしろ|すがた}を
\ruby{見{\換字{送}}}{み|おく}つて、

『
\ruby{怖}{おつかな}い
\ruby{顏}{かほ}して
\ruby{怒}{おこ}つたつて
\ruby{無{\換字{益}}}{だ|め}な
\ruby{事}{こん}だ。
そんなに
\ruby{怒}{おこ}つて
\ruby{歩}{ある}いて
\ruby{柹實}{か|き}を
\ruby{踏}{ふ}み
\ruby{潰}{つぶ}してはならねえだよ。
ハヽハヽ。
』

と、
\ruby{侮}{あなど}り
\ruby{笑}{わら}ひぬ。

\ruby{面}{おもて}を
\ruby{對}{あは}せたる
\ruby{時}{とき}にだに
\ruby{既}{すで}に
\ruby{忍}{しの}びたれば、
\ruby{背後}{はい|ご}の
\ruby{笑}{わらひ}には
\ruby{耳}{みゝ}をも
\ruby{假}{か}さず、
\ruby{柹}{かき}の
\ruby{樹幾本}{き|いく|ほん}の
\ruby{下}{した}を
\ruby{潜}{くぐ}りて、
\ruby{我}{わ}が
\ruby{五十子}{い|そ|こ}の
\ruby{病}{や}みて
\ruby{臥}{ふ}せる
\ruby{別室{\換字{近}}}{はな|れ|ちか}く
\ruby{到}{いた}れば、
\ruby{風}{かぜ}の
\ruby{騒}{さわ}がしきを
\ruby{厭}{いと}ひたりと
\ruby{見}{み}えて、はや
\ruby{{\換字{戸}}}{と}を
\ruby{引}{ひ}きたるが、
\ruby{中}{なか}には
\ruby{燈}{ひ}の
\ruby{光{\換字{弱}}}{ひかり|よわ}く
\ruby{籠}{こも}りて、
\ruby{人}{ひと}の
\ruby{動}{うご}ける
\ruby{影}{かげ}のちら〳〵としたり。

\ruby{今}{いま}までは
\ruby{先}{さき}に
\ruby{立}{た}ちて
\ruby{來}{きた}れる
\ruby{水野}{みづ|の}の、
\ruby{此處}{こ|ゝ}に
\ruby{至}{いた}りて
\ruby{俄}{にはか}に
\ruby{歩}{あゆ}み
\ruby{鈍}{にぶ}れば、
\ruby{松之助}{まつ|の|すけ}の
\ruby{方}{かた}、
\ruby{先}{さき}になりて、
\ruby{既}{すで}に
\ruby{沓{\換字{脱}}}{くつ|ぬぎ}
\ruby{一}{ひ}ト
\ruby{足}{あし}
\ruby{踏}{ ふ}み
\ruby{入}{い}るゝに
\ruby{水野}{みづ|の}は
\ruby{其}{そ}の
\ruby{執}{と}りたる
\ruby{手}{て}を
\ruby{力無}{ちから|な}く
\ruby{放}{はな}して、
\ruby{續}{つゞ}いて
\ruby{入}{い}らんともせず
\ruby{立迷}{たち|まよ}ひ
\ruby{居}{ゐ}たり。

\ruby{此}{こ}の
\ruby{心得難}{こゝろ|え|がた}き
\ruby{擧動}{ふる|まひ}の
\ruby{意}{こゝろ}を、
\ruby{松之助}{まつ|の|すけ}は
\ruby{更}{さら}に
\ruby{解}{と}く
\ruby{由無}{よし|な}ければ、
\ruby{振顧}{ふり|かへ}りて
\ruby{此度}{こ|たび}は
\ruby{我}{わ}が
\ruby{手}{て}に
\ruby{水野}{みづ|の}の
\ruby{手}{て}を
\ruby{執}{と}り、
\ruby{疾}{と}く
\ruby{此方}{こ|なた}へ
\ruby{上}{あが}れよと
\ruby{眼}{め}に
\ruby{云}{い}はせて
\ruby{引張}{ひつ|ぱ}つたり。

\ruby{言}{い}はず
\ruby{語}{かた}らずの
\ruby{我}{わ}が
\ruby{誠}{まこと}の
\ruby{{\換字{情}}}{こゝろ}は、
\ruby{知}{し}らず
\ruby{識}{し}らずに
\ruby{他}{ひと}の
\ruby{優}{やさ}しき
\ruby{胸}{むね}に
\ruby{響}{ひゞ}きては、
\ruby{可憐}{か|はゆ}き
\ruby{我}{わ}が
\ruby{松之助}{まつ|の|すけ}は
\ruby{我}{われ}を
\ruby{兄}{あに}などのやうに
\ruby{思}{おも}ひ
\ruby{做}{な}し
\ruby{取}{と}り
\ruby{做}{な}して、
\ruby{泣}{な}き
\ruby{顏}{がほ}に
\ruby{姊}{あね}が
\ruby{急}{きふ}を
\ruby{訴}{うつた}へに
\ruby{來}{きた}りしそれに
\ruby{釣}{つ}り
\ruby{込}{こ}まれて、ハツと
\ruby{驚}{おどろ}きし
\ruby{餘}{あま}りに
\ruby{何}{なに}といふ
\ruby{考}{かんが}へも
\ruby{無}{な}く、
\ruby{走}{はし}り
\ruby{出}{い}でゝ
\ruby{此處}{こ|ゝ}へは
\ruby{來}{きた}りしものゝ、
\ruby{如何}{い|か}なる
\ruby{宿世}{しゆ|くせ}の
\ruby{仇}{あだ}のありてか、
\ruby{我}{わ}が
\ruby{五十子}{い|そ|こ}の
\ruby{我}{われ}を
\ruby{厭}{いと}ふ
\ruby{{\換字{情}}}{こゝろ}も
\ruby{漸}{やうや}く
\ruby{募}{つの}りて、
\ruby{特}{こと}に
\ruby{病氣}{びや|うき}の
\ruby{爲}{さ}する
\ruby{癇}{かん}の
\ruby{所爲}{わ|ざ}とは
\ruby{云}{い}へ、
\ruby{此}{こ}の
\ruby{頃}{ごろ}は
\ruby{我}{わ}が
\ruby{面}{おもて}を
\ruby{見}{み}るをさへ
\ruby{甚}{はなはだ}しく
\ruby{忌}{い}み
\ruby{{\換字{嫌}}}{きら}うやうになり
\ruby{居}{を}れるなれば、
\ruby{我}{われ}はこそ
\ruby{其}{そ}の
\ruby{人}{ひと}の
\ruby{傍}{そば}に
\ruby{在}{あ}りて
\ruby{兎}{と}も
\ruby{角}{かく}もなるを
\ruby{見果}{み|はて}んと
\ruby{願}{ねが}へ、
\ruby{今}{いま}その
\ruby{病狀}{やう|す}の
\ruby{凶}{あし}き
\ruby{盛}{さか}りに
\ruby{我}{わ}が
\ruby{面}{おもて}を
\ruby{見}{み}せて、その
\ruby{人}{ひと}に
\ruby{快}{こゝろよ}からぬ
\ruby{思}{おもひ}させんことは、たとへばまた
\ruby{復}{ふたゝ}び
\ruby{戀}{こひ}しき
\ruby{人}{ひと}の
\ruby{此}{こ}の
\ruby{世}{よ}の
\ruby{顏}{かほ}を
\ruby{見}{み}るを
\ruby{得}{え}ざるに
\ruby{至}{いた}らん
\ruby{其}{そ}の
\ruby{悲}{かな}しさは、
\ruby{能}{よ}く
\ruby{忍}{しの}ぶべしとするも、これは
\ruby{忍}{しの}ぶに
\ruby{忍}{しの}びがたきところなり。
\ruby{特}{こと}にわれは
\ruby{死}{し}を
\ruby{起}{おこ}し
\ruby{生}{せい}を
\ruby{囘}{かへ}すの
\ruby{{\換字{道}}}{みち}を
\ruby{知}{し}れるにもあらず、また
\ruby{我}{わ}が
\ruby{岩崎氏}{いは|さき|うぢ}に
\ruby{何}{なん}の
\ruby{因緣}{ゆ|かり}もあるにもあらず、
\ruby{云}{い}はゞ
\ruby{赤}{あか}の
\ruby{他人}{た|にん}の
\ruby{身}{み}をもて、
\ruby{然}{さ}ならぬだに
\ruby{生}{い}くる
\ruby{死}{し}ぬるの
\ruby{境}{さかひ}に
\ruby{惱}{なや}める
\ruby{人}{ひと}の
\ruby{枕頭}{まく|らべ}に
\ruby{見}{あらは}れて、
\ruby{其}{そ}の
\ruby{人}{ひと}に
\ruby{忌}{いま}はしき
\ruby{思}{おもひ}をさするほかには
\ruby{何}{なん}の
\ruby{能}{のう}も
\ruby{無}{な}き
\ruby{面}{おもて}を
\ruby{差}{さ}し
\ruby{出}{だ}さん
\ruby{心無}{こゝろ|な}さは、
\ruby{我爲}{わが|な}し
\ruby{得}{う}べきところならんや。
\ruby{痩}{や}せたる
\ruby{其}{そ}の
\ruby{人}{ひと}の
\ruby{手}{て}をも
\ruby{執}{と}り、
\ruby{冷}{ひ}えんとする
\ruby{其人}{その|ひと}の
\ruby{身}{み}をも
\ruby{溫}{あたゝ}めて、
\ruby{及}{およ}ばぬまでも
\ruby{心限}{こゝろ|かぎ}りの
\ruby{介抱}{かい|はう}を
\ruby{仕}{し}たき
\ruby{望}{のぞみ}は
\ruby[g]{熾盛}{さかん}なれども、
\ruby{因緣}{いん|ねん}の
\ruby{恨}{うら}めしくも
\ruby{悲}{かな}しくも
\ruby{厭}{いと}ひ
\ruby{{\換字{嫌}}}{きら}はれたる
\ruby{身}{み}の
\ruby{其}{それ}も
\ruby{叶}{かな}はず、たゞ
\ruby{戸}{と}の
\ruby{外}{そと}に
\ruby{泣}{な}き
\ruby{惑}{まど}ひて、あだに
\ruby{物}{もの}を
\ruby{思}{おも}ひ
\ruby{心}{こゝろ}を
\ruby{苦}{くる}しめんためばかりに
\ruby{此處}{こ|ゝ}に
\ruby{來}{きた}りし
\ruby{冥利}{みや|うり}の
\ruby{拙}{つたな}さ!、
\ruby{我}{わ}が
\ruby{愚}{おろか}さ!。
\ruby{思}{おも}へば
\ruby{何}{なん}とせん
\ruby{意}{こゝろ}にて
\ruby{此處}{こ|ゝ}に
\ruby{走}{はし}りては
\ruby{來}{きた}りしぞや。
\ruby{甲斐}{か|ひ}なくも
\ruby{甲斐無}{か|ひ|な}く
\ruby{氣}{き}を
\ruby{揉}{も}みて、たゞたゞ
\ruby{亂}{みだ}れて
\ruby{絲}{いと}の
\ruby{如}{ごと}き
\ruby{思}{おもひ}に、
\ruby{獨}{ひと}り
\ruby{泣}{な}くよりほかには
\ruby{爲}{な}すべき
\ruby{我}{わ}が
\ruby{事}{こと}もあらざる
\ruby{{\換字{情}}}{なさけ}
\ruby{無}{ な}さを
\ruby{如何}{い|か}にせん。

と
\ruby{松之助}{まつ|の|すけ}の
\ruby{手}{て}をそつと
\ruby{拂}{はら}つて、
\ruby{面}{おもて}をかくしつゝ
\ruby{逸}{そ}れたる
\ruby{水野}{みづ|の}は、
\ruby{家}{いへ}の
\ruby{背後}{うし|ろ}の
\ruby{椎}{しい}の
\ruby{老樹}{おい|き}の
\ruby{幹}{みき}に
\ruby{頭}{かうべ}を
\ruby{埋}{うづ}めて、こんもりとしたる
\ruby{其陰}{その|かげ}には、はや
\ruby{夕闇}{ゆふ|やみ}の
\ruby{逼}{せま}りて
\ruby{昏}{くら}くなれるが
\ruby{中}{なか}に
\ruby{立盡}{たち|つく}せり。

\ruby{風}{かぜ}は
\ruby{{\換字{猶}}}{なほ}
\ruby{吹}{ふ}けどやゝ
\ruby{衰}{おとろ}へて『
\ruby{四十七士}{しじ|ゆう|しち|し}の
\ruby{墓}{はか}どころ、
\ruby{{\換字{雪}}}{ゆき}は
\ruby{{\換字{消}}}{き}えても
\ruby{名}{な}は
\ruby{殘}{のこ}る、』と、
\ruby{村}{むら}の
\ruby{兒}{こ}が
\ruby{{\換字{遠}}方}{とほ|く}にて
\ruby{唱}{うた}ふ
\ruby{金切聲}{かな|きり|ごゑ}の
\ruby{幽}{かすか}に
\ruby{聞}{きこ}えくるも
\ruby{時}{とき}に
\ruby{取}{と}りて
\ruby{忌}{いま}はしく、
\ruby{塒}{ねぐら}に
\ruby{急}{いそ}ぐ
\ruby{歸}{かへ}り
\ruby{鴉}{がらす}の
\ruby{二三羽鳴}{に|さん|ば|な}きつれたるも
\ruby{耳立}{みゝ|だ}つて
\ruby{淋}{さび}しく、
\ruby{其後}{その|のち}は
\ruby{物音}{もの|おと}も
\ruby{無}{な}く
\ruby{日}{ひ}は
\ruby{暮}{く}れんとす。


\Entry{其三十五}

\ruby{我}{われ}は
\ruby{今}{いま}
\ruby{何}{なん}として
\ruby{來}{きた}りけん
\ruby{我知}{われ|し}らず、
\ruby{我}{われ}は
\ruby{今}{いま}
\ruby{何}{なん}となさば
\ruby{宣}{よ}からん
\ruby{我知}{われ|し}らず、
\ruby{我}{われ}はたゞ
\ruby{此處}{こ|ゝ}に
\ruby{來}{こ}では
\ruby{叶}{かな}はざるやう
\ruby{思}{おも}ひて
\ruby{此處}{こ|ゝ}に
\ruby{來}{きた}り、
\ruby{我}{われ}はたゞ
\ruby{此處}{こ|ゝ}を
\ruby{去}{さ}りがたき
\ruby{心地}{こゝ|ち}するばかりに
\ruby{此處}{こ|ゝ}に
\ruby{在}{あ}るなり、
\ruby{來}{きた}れるが
\ruby{他}{ひと}の
\ruby{益}{やく}にも
\ruby{立}{た}たず、
\ruby{在}{あ}るが
\ruby{思}{おも}ひの
\ruby{晴}{は}るゝ
\ruby{業}{わざ}にもあらざるを、
\ruby{女々}{め|ゝ}しくも
\ruby{男兒}{をと|こ}らしからぬ
\ruby{振舞}{ふる|まひ}をするかな!。
\ruby{愚}{おろか}かなりとも
\ruby{日頃}{ひ|ごろ}の
\ruby{我}{われ}は
\ruby{如是}{か|く}はあらざりしものを、
\ruby{意氣地}{い|く|ぢ}
\ruby{無}{な}くも
\ruby{崩折}{くづ|を}れたる
\ruby{心}{こゝろ}の
\ruby{何}{なに}を
\ruby{待}{ま}てるぞや!。
\ruby{醫藥}{い|やく}の
\ruby{力}{ちから}は
\ruby{限}{かぎり}あり、
\ruby{定命}{ぢやう|みやう}は
\ruby{如何}{い|かん}とも
\ruby{爲}{な}しがたければ、その
\ruby{人}{ひと}の
\ruby{魂魄}{た|ま}の
\ruby{{\換字{情}}無}{なさけ|な}くも
\ruby{天}{そら}に
\ruby{去}{さ}つて、
\ruby{松之助}{まつ|の|すけ}の
\ruby{泣聲}{なき|ごゑ}のわつと
\ruby{起}{おこ}らん
\ruby{時}{とき}、
\ruby{我}{われ}は
\ruby{其}{そ}の
\ruby{聲}{こゑ}を
\ruby{聞}{き}いて
\ruby{世}{よ}を
\ruby{思}{おも}い
\ruby{切}{き}り、
\ruby{此}{こ}の
\ruby{椎}{しひ}の
\ruby{幹}{みき}の
\ruby{岩}{いは}のごときに、
\ruby{額}{ひたひ}を
\ruby{打付}{うち|つ}け
\ruby{頭顱}{なづ|き}を
\ruby{破}{わ}つて、よしや
\ruby{身}{み}は
\ruby{輪{\換字{廻}}}{りん|ね}の
\ruby{闇}{やみ}に
\ruby{{\換字{迷}}}{まよ}ひ
\ruby{入}{い}るとも、
\ruby{一念}{おも|ひ}は
\ruby{芳魂}{はう|こん}の
\ruby{行方}{ゆく|へ}を
\ruby{{\換字{追}}}{お}ひて、
\ruby{紫雲}{し|うん}の
\ruby{空}{そら}の
\ruby{遙}{はる}けくもあれ、
\ruby{黄泉}{こわう|せん}の
\ruby{涯}{はて}の
\ruby{{\換字{遠}}}{とほ}くもあれ、つれなき
\ruby{風}{かぜ}の
\ruby{持}{も}て
\ruby{去}{さ}れる
\ruby{花}{はな}の
\ruby{香}{かをり}に
\ruby{引}{ひ}かされて、あくがれ
\ruby{漂泊}{さま|よ}ふ
\ruby{蝶}{てふ}の
\ruby{如}{ごと}くに、
\ruby{{\換字{飽}}}{あく}まで
\ruby{戀}{こひ}しき
\ruby{人}{ひと}に
\ruby{{\換字{伴}}}{ともな}はんとて、こゝには
\ruby{空}{むな}しく
\ruby{佇}{たゝず}める
\ruby{歟}{か}。
\ruby{或}{あるい}は
\ruby{{\換字{又}}}{また}
\ruby{{\換字{強}}}{つよ}く
\ruby{忌}{い}み
\ruby{{\換字{嫌}}}{きら}はれたるより、
\ruby{堪}{た}へがたき
\ruby{苦悶}{も|だえ}に
\ruby{自}{みずか}ら
\ruby{堪}{た}へて、
\ruby{其人}{その|ひと}に
\ruby{{\換字{近}}}{ちか}づきもせず
\ruby{{\換字{過}}}{そご}し
\ruby{居}{ゐ}けるが、
\ruby{若}{も}し
\ruby{不幸}{ふ|かう}にして
\ruby{其}{そ}の
\ruby{{\換字{遠}}慮}{ゑん|りよ}の
\ruby{俄}{にはか}に
\ruby{失}{う}すべき
\ruby{時}{とき}にも
\ruby{至}{いた}らば、
\ruby{先}{ま}ず
\ruby{枕}{まくら}の
\ruby{邊}{ほとり}に
\ruby{走}{はし}り
\ruby{寄}{よ}つて、
\ruby{我}{わ}が
\ruby{火}{ひ}と
\ruby{熱}{あつ}き
\ruby{萬石}{ばん|こく}の
\ruby{涙}{なみだ}を、せめては
\ruby{其}{そ}の
\ruby{冷}{つめた}き
\ruby{骸}{かばね}に
\ruby{親}{した}しく
\ruby{濺}{そゝ}ぎ、
\ruby{{\換字{情}}無}{つれ|な}かりし
\ruby{其}{そ}の
\ruby{人}{ひと}の
\ruby{手}{て}を
\ruby{執}{と}り
\ruby{搖}{ゆさ}ぶりて、
\ruby{心}{こゝろ}ゆくばかり
\ruby{號哭}{がう|こく}せんとて
\ruby{此處}{こ|ゝ}には
\ruby{居}{ゐ}るにや。
それにもあらねば、これにもあらず、
\ruby{何}{なに}せん
\ruby{心}{こゝろ}は
\ruby{更}{さら}に
\ruby{無}{な}くして、
\ruby{我}{われ}にも
\ruby{我}{われ}の
\ruby{解}{わか}らぬ
\ruby{感想}{おも|ひ}に、たゞ
\ruby{此處}{こ|ゝ}を
\ruby{去}{さ}りかねて
\ruby{水野}{みづ|の}は
\ruby{{\換字{猶}}}{なほ}
\ruby{立}{た}てり。

\ruby{暮}{く}るゝに
\ruby{{\換字{連}}}{つ}れて
\ruby{風}{かぜ}は
\ruby{收}{をさ}まり、
\ruby{闇}{やみ}は
\ruby{葉}{は}の
\ruby{密}{こ}みたる
\ruby{椎}{しひ}の
\ruby{{\換字{梢}}}{こずゑ}より
\ruby{廣}{ひろ}がつて、
\ruby{{\換字{終}}}{つひ}に
\ruby{其黑}{その|くろ}き
\ruby{懷}{ふところ}の
\ruby{中}{うち}に
\ruby{四邊}{あ|たり}を
\ruby{包}{つゝ}みぬ。

\ruby{森々}{しん|〳〵}と
\ruby{靜}{しづか}なる
\ruby{此}{こ}の
\ruby{日}{ひ}
\ % 隙間調整
\ruby{此}{こ}の
\ruby{{\換字{宵}}}{ゆふべ}
\ % 隙間調整
\ruby{天}{てん}に
\ruby{星無}{ほし|な}し、
\ruby{星}{ほし}は
\ruby{死}{し}したるならん、
\ruby{地}{ち}には
\ruby{風}{かぜ}は
\ruby{{\換字{弱}}}{よわ}りぬ、
\ruby{風}{かぜ}は
\ruby{今}{いま}おのが
\ruby{墓穴}{はか|あな}を
\ruby{{\換字{尋}}}{たづ}ねて
\ruby{永}{なが}く
\ruby{休}{やす}まんとせり。
\ruby{{\換字{古}}}{ふ}りたる
\ruby{椎}{しい}の
\ruby{木}{き}は
\ruby{忽然}{こつ|ぜん}として
\ruby{人}{ひと}の
\ruby{聲}{こゑ}をなし、

『
\ruby{衆生被困厄}{しゆ|じやう|び|こん|やく}、
\ruby{無量苦逼身}{む|りやう|く|ひつ|しん}、
\ruby{觀音妙智力}{くわん|のん|めう|ち|りき}、
\ruby{能救世間苦}{のう|ぐ|せ|けん|く}、』

と
\ruby{囁}{さゝや}くが
\ruby{如}{ごと}くに
\ruby{誦}{じゆ}し
\ruby{出}{いだ}せり。

\ruby{椎}{しい}の
\ruby{那處}{いづ|く}に
\ruby{彼}{か}の
\ruby{額廣}{ひたひ|ひろ}く
\ruby{鼻細}{はな|ほそ}き
\ruby{老}{お}いたる
\ruby{男}{をとこ}の
\ruby{潛}{ひそ}み
\ruby{居}{を}れりや、
\ruby{聲}{こゑ}は
\ruby{全}{まつた}く
\ruby{其}{そ}の
\ruby{聲}{こゑ}なりけり。

\ruby{愚}{おろか}なり!、こは
\ruby{我}{わ}が
\ruby{招}{よ}ばずして
\ruby{我}{わ}が
\ruby{記臆}{き|おく}の
\ruby{現}{あらは}れ
\ruby{來}{きた}れるには
\ruby{{\換字{過}}}{す}ぎざるものをと
\ruby{水野}{みづ|の}が
\ruby{冷}{ひや}やかに
\ruby{聞}{き}きし
\ruby{時}{とき}は、
\ruby{其聲}{その|こゑ}は
\ruby{既}{はや}
\ruby{失}{う}せて
\ruby{{\換字{遺}}響}{ひゞ|き}も
\ruby{無}{な}かりしが、
\ruby{當時椎}{その|とき|しひ}の
\ruby{大木}{おほ|き}は
\ruby{忽}{たちま}ち
\ruby{二}{ふた}つに
\ruby{裂}{さ}けて、
\ruby{其處}{そ|こ}に
\ruby{明}{あき}らかなる
\ruby{世界}{せ|かい}の
\ruby{朗}{ほが}らかに
\ruby{現}{あらは}れたるが
\ruby{中}{うち}に、
\ruby{年齡}{と|し}は
\ruby{二十四五}{に|じう|し|ご}なる
\ruby{男}{をとこ}の
\ruby{戀}{こひ}に
\ruby{窶}{やつ}れたる
\ruby{顏}{かほ}の
\ruby{勇威}{いき|ほひ}
\ruby{無}{な}く
\ruby{光釆}{ひか|り}
\ruby{無}{な}く、
\ruby{五月雨}{さ|み|だれ}の
\ruby{檐}{のき}の
\ruby{雫}{しづく}と
\ruby{涙}{なみだ}を
\ruby{放}{はふ}らし
\ruby{落}{おと}し
\ruby{居}{を}れるさまの
\ruby{醜}{みにく}くも
\ruby{醜}{みにく}きを、
\ruby{右}{みぎ}の
\ruby{肩}{かた}には
\ruby{恐}{おそ}ろしき
\ruby{猛鷲}{あら|わし}を
\ruby{宿}{と}まらしめ、
\ruby{後}{うしろ}には
\ruby{凄}{すさま}じき
\ruby{大蛇}{だい|じや}を
\ruby{{\換字{随}}}{したが}へたる
\ruby{氣味惡}{き|み|あ}しき
\ruby{大男}{おほ|をとこ}の、
\ruby{神}{かみ}に
\ruby{似}{に}て
\ruby{神}{かみ}の
\ruby{威無}{ゐ|な}く、
\ruby{人}{ひと}かと
\ruby{見}{み}れば
\ruby{人}{ひと}らしからぬが、
\ruby{憐}{あはれ}むが
\ruby{如}{ごと}く
\ruby{侮}{あなど}るが
\ruby{如}{ごと}き
\ruby{眼}{め}して
\ruby{見詰}{み|つ}め
\ruby{居}{ゐ}たるが
\ruby{{\換字{分}}明}{あり|〳〵}と
\ruby{見}{み}えぬ。

\Entry{其三十六}

\ruby{落葉}{おち|ば}を
\ruby{誘}{さそ}ふ
\ruby{山下}{やま|おろ}しの
\ruby{風}{かぜ}を
\ruby{其儘}{その|まゝ}なる
\ruby{猛鷲}{あら|わし}の
\ruby{打翥}{うち|はぶ}く
\ruby{音}{おと}の
\ruby{中}{うち}には、『
\ruby{神明}{か|み}は
\ruby{殪}{たふ}れたり、』『
\ruby[g]{佛陀}{ほとけ}は
\ruby{死}{し}したり、』といふ
\ruby{響}{ひゞき}の
\ruby{聞}{きこ}え、
\ruby{首}{かうべ}を
\ruby{擡}{あ}げて
\ruby{蜿蜒}{う|ね}る
\ruby{大蛇}{だい|じや}のざわ〳〵と
\ruby{木茅}{き|かや}を
\ruby{倒}{たふ}し
\ruby{行}{ゆ}く
\ruby{音}{おと}の
\ruby{中}{うち}には、「
\ruby{神明}{か|み}は
\ruby{想像}{さう|ざう}のみ、」「
\ruby[g]{佛陀}{ほとけ}は
\ruby{假{\GWI{u8aaa-jv}}}{か|せつ}のみ、」といふ
\ruby{聲}{こゑ}あり。

\ruby[g]{水野}{みづの}は
\ruby{自}{みづ}から
\ruby{思}{おも}はずして
\ruby{自}{おのづ}から
\ruby{如是想}{か|く|おも}ひ、
\ruby{外}{そと}に
\ruby{見聞}{み|きゝ}せずして
\ruby{内}{うち}に
\ruby{如是見聞}{か|く|み|き}きせる
\ruby{時}{とき}、
\ruby{靜}{しづか}なる
\ruby{五十子}{い|そ|こ}が
\ruby{家}{いへ}の
\ruby{方}{かた}にて、かたりと
\ruby{微}{かすか}の
\ruby{物音}{もの|おと}の
\ruby{仕}{し}たるを
\ruby{聞}{き}きつけ、
\ruby{豁然}{くわつ|ぜん}としてわれに
\ruby{{\GWI{u8fd4-k}}}{かへ}れば、
\ruby{我}{わ}が
\ruby{止}{と}め
\ruby{{\GWI{u9014-k}}}{ど}
\ruby{無}{な}かりし
\ruby{淚}{なみだ}の
\ruby{何時}{い|つ}か
\ruby{乾}{かわ}き、
\ruby{我}{わ}が
\ruby{疲}{つか}れたる
\ruby{心}{こヽろ}の
\ruby{何時}{い|つ}か
\ruby{奮}{ふる}ひて、
\ruby{倚}{よ}りかかりたる
\ruby{椎}{しい}の
\ruby{幹}{みき}を
\ruby{離}{はな}れ、そを
\ruby[g]{背向}{そがひ}にして
\ruby{挺然}{てい|ぜん}と
\ruby{獨}{ひと}り
\ruby{樹陰}{こ|かげ}の
\ruby{闇}{やみ}に
\ruby{立}{た}ちつ、
\ruby{魔}{ま}の
\ruby{如}{ごと}くに
\ruby{來}{きた}り
\ruby{魔}{ま}の
\ruby{如}{ごと}くに
\ruby{去}{さ}る
\ruby{蝙蝠}{かは|ほり}の、ひらひらと
\ruby{{\換字{梢}}}{こずゑ}の
\ruby{盡頭}{はず|れ}を
\ruby{飛}{とび}かへれるを、
\ruby{雲透}{くも|ずき}に\GWI{u1b048-u3099}つと
\ruby{打見}{うち|み}やりたり。

\ruby{有}{あ}りや
\ruby{神佛}{かみ|ほとけ}の?、
\ruby{有}{あ}るにも
\ruby{似}{に}たるかな!。
\ruby{無}{な}しや
\ruby{神佛}{かみ|ほとけ}の?、
\ruby{無}{な}きにも
\ruby{似}{に}たるかな!。

\ruby{有}{あ}るには
\ruby{無}{な}きの
\ruby{疑}{うたがひ}あり、
\ruby{有}{あ}りとも
\ruby{爲難}{し|がた}く、
\ruby{無}{な}しとも
\ruby{爲難}{し|がた}し。
\ruby{有無}{う|む}のいづれは
\ruby{今知}{いま|し}らねども、
\ruby{世}{よ}に
\ruby{無}{な}き
\ruby{方}{かた}の
\ruby{眞實}{ま|こと}ならば、
\ruby[g]{男兒}{をのこ}の
\ruby{頭}{かうべ}を
\ruby{下}{さ}げて
\ruby{祈願}{き|ぐわん}を
\ruby{捧}{さゝ}げんことの
\ruby{羞}{はづか}しくも
\ruby{口惜}{くち|を}しく、
\ruby{若}{も}し
\ruby{世}{よ}に
\ruby{在}{おは}す
\ruby{事}{こと}の
\ruby{定}{ぢやう}ならば、
\ruby{身}{み}をも
\ruby{魂魄}{たま|しひ}をも
\ruby{犠牲}{いけ|にへ}にして、
\ruby{廣大}{くわう|だい}の
\ruby{御慈悲}{おん|じ|ひ}を
\ruby{頼}{たの}み
\ruby{奉}{たてまつ}らんと
\ruby{思}{おも}ふ
\ruby{此}{こ}の
\ruby{人間}{ひ|と}の
\ruby{心}{こヽろ}のみぞ
\ruby{僞}{いつは}り
\ruby{無}{な}き
\ruby{眞實}{ま|こと}なる!。
\ruby{二}{ふ}タ
\ruby{路}{みち}かけて
\ruby{取舎}{しゆ|しや}しわづらひつゝ、
\ruby{利}{よき}に
\ruby{就}{つ}かんとする
\ruby{此}{こ}の
\ruby{分別}{ふん|べつ}の
\ruby{醜}{みにく}さよ、
\ruby{智慧}{ち|ゑ}の
\ruby[g]{狡猾}{かしこ}さよ!。
あゝ
\ruby{人間}{ひ|と}は
\ruby{卑劣}{さ|も}しくも
\ruby{怯}{きたな}き
\ruby{心}{こヽろ}を
\ruby{有}{も}てるかな!。
されど
\ruby{此}{こ}の
\ruby{疑}{うたが}ひ
\ruby{惑}{まど}ひて
\ruby{苦}{くるし}めるこそは、
\ruby{人間}{ひ|と}の
\ruby{僞}{いつは}り
\ruby{無}{な}き
\ruby[g]{眞實}{まこと}の
\ruby{{\換字{情}}狀}{さ|ま}なるべけれ。
\ruby{我}{われ}こゝに
\ruby{在}{あ}り、われこゝに
\ruby{思}{おも}ふ。
\ruby{思}{おも}はるゝものゝ
\ruby{有}{あ}り
\ruby{無}{な}しは
\ruby{定}{さだ}かならず、
\ruby{思}{おも}ふ
\ruby{我}{われ}の
\ruby{在}{あ}る
\ruby{事}{こと}が
\ruby[g]{眞實}{まこと}なるのみ。
\ruby{菩薩}{ぼ|さつ}の
\ruby{言葉}{こと|ば}、
\ruby{鷲}{わし}の
\ruby{言葉}{こと|ば}、
\ruby{妙典}{めう|てん}の
\ruby{{\換字{敎}}}{をしへ}、
\ruby{大蛇}{だい|じや}の
\ruby{{\換字{敎}}}{をしへ}、
\ruby{我}{われ}にいづれを
\ruby{取}{と}り
\ruby[g]{那方}{いづれ}を
\ruby{捨}{す}つる
\ruby{力無}{ちから|な}し、たゞ
\ruby[g]{那方}{いづれ}をも
\ruby{取}{と}り
\ruby{惱}{なや}み、またいづれをも
\ruby{捨}{す}て
\ruby{惱}{なや}む
\ruby{其事}{その|こと}のみぞ
\ruby{我}{わ}が
\ruby[g]{眞實}{まこと}なる!。
\ruby{神明}{か|み}
\ruby[g]{佛陀}{ほとけ}をも
\ruby{肯}{うけが}はずして、
\ruby{智慧}{ち|ゑ}の
\ruby[g]{鋼鐵}{はがね}の
\ruby{杖}{つゑ}に
\ruby{頼}{よ}つて
\ruby{此}{こ}の
\ruby{戰鬪}{たヽ|かひ}の
\ruby{世}{よ}に
\ruby{立}{た}たんとするも
\ruby{我}{わ}が
\ruby{欺}{あざむ}かぬ
\ruby[g]{眞實}{まこと}なり。
\ruby{獸}{けもの}にもあらず
\ruby{鳥}{とり}にもあらで、
\ruby{光明}{ひか|り}の
\ruby{國}{くに}
\ruby{黑闇}{や|み}の
\ruby{國}{くに}の
\ruby{境}{さかひ}を
\ruby{飛}{と}ぶ
\ruby{彼}{あ}の
\ruby{魔魅}{ま|もの}の
\ruby{如}{ごと}き
\ruby{蝙蝠}{かは|ほり}の、
\ruby{世}{よ}にも
\ruby{厭}{いと}はしく
\ruby{醜}{みにく}きは、
\ruby{我}{わ}が
\ruby{胸}{むね}の
\ruby{中}{うち}の
\ruby{怪物}{くわい|ぶつ}の、
\ruby{化}{な}りて
\ruby{出}{い}でしかとも
\ruby{思}{おも}はれて、
\ruby{何}{なに}とも
\ruby{云}{い}へぬ
\ruby{忌}{いま}はしき
\ruby{氣}{き}のする!。
されど、されど、
\ruby{是}{こ}は
\ruby[g]{眞實}{まこと}なり、
\ruby{我}{われ}は
\ruby{僞}{いつは}らず、
\ruby{我}{われ}は
\ruby{矯}{た}めず、
\ruby{我}{われ}は
\ruby{飾}{かざ}らず、
\ruby{恐}{おそ}るゝところ
\ruby{無}{な}し。
われこゝに
\ruby{思}{おも}ふ!。
\ruby{我}{われ}こゝに
\ruby{在}{あ}り!。
\ruby{天我}{てん|わ}が
\ruby{戀}{おも}へる
\ruby{人}{ひと}を
\ruby{何}{なに}とせんとはする\GWI{u2048}。
\ruby{天}{てん}そも〳〵
\ruby{我}{われ}を
\ruby{何}{なに}となれとかする\GWI{u2048}。

と
\ruby{淺草}{あさ|くさ}の
\ruby{御堂}{み|だう}に
\ruby{身}{み}を
\ruby{投}{な}げ
\ruby{伏}{ふ}して
\ruby{淚}{なみだ}にくれし
\ruby{曉}{あかつき}には
\ruby{引}{ひき}かへ、
\ruby{一文字口緊}{いち|もん|じ|ぐち|きび}しく
\ruby{引締}{ひき|し}めて、
\ruby{{\GWI{u7336-k}}}{なほ}
\ruby{石人}{せき|じん}の
\ruby{如}{ごと}く
\ruby{突立}{つゝ|た}てる
\ruby{時}{とき}、
\ruby{尾竹}{を|たけ}と
\ruby{松之助}{まつ|の|すけ}とは
\ruby{家}{いへ}の
\ruby{中}{うち}より
\ruby{現}{あらは}れ
\ruby{出}{い}でゝ、

『そこに
\ruby{居}{ゐ}らつしやるのは
\ruby[g]{水野}{みづの}さんで?。
ア、
\ruby{御入}{お|はい}んなされば
\ruby{宜}{よろ}しかつたものを。
』

と
\ruby{尾竹}{を|たけ}の
\ruby{云}{い}ふに
\ruby{續}{つゞ}いて
\ruby{松之助}{まつ|の|すけ}は、

『そこに
\ruby{居}{ゐ}たの?。
\ruby{僕}{ぼく}は
\ruby{君}{きみ}は
\ruby{何}{なに}か
\ruby{思}{おも}ひ
\ruby{出}{だ}して
\ruby{歸}{かへ}つたのかと
\ruby{思}{おも}つた!。
\ruby[g]{水野君}{みづのくん}、
\ruby{君}{きみ}は
\ruby{變}{へん}な
\ruby{人}{ひと}だネ。
』

と、
\ruby{我}{わ}が
\ruby{{\換字{姉}}}{あね}の
\ruby[g]{水野}{みづの}を
\ruby{{\換字{嫌}}}{きら}へる
\ruby{事}{こと}の
\ruby{如何}{い|か}ばかり
\ruby{其}{そ}の
\ruby{人}{ひと}を
\ruby{苦}{くるし}め
\ruby{居}{を}るかをも
\ruby{知}{し}らずして
\ruby{云}{い}ふ。

\ruby{尾竹}{を|たけ}はまた
\ruby{直}{ただち}に
\ruby{引取}{ひつ|と}つて、

『
\ruby{定}{さだ}めし
\ruby{案}{あん}じて
\ruby{居}{ゐ}て
\ruby{下}{くだ}さるだらうといふので、
\ruby{今御宅}{いま|お|たく}へ
\ruby{一寸様子}{ちよ|つと|やう|す}を
\ruby{申}{まを}しに
\ruby{上}{あが}らうとしたところでござりました。
\ruby{熱}{ねつ}が
\ruby{甚}{ひど}く
\ruby{發}{はつ}して
\ruby{譫語}{せん|ご}が
\ruby{{\換字{強}}}{つよ}かつたりなんぞしたので、
\ruby{傍}{そば}の
\ruby{人}{ひと}は
\ruby{一時驚}{いち|じ|おどろ}いたのでしたが、
\ruby{別}{べつ}の
\ruby{事}{こと}も
\ruby{無}{な}くつてまあ
\ruby{濟}{す}みました。
\ruby{肺}{はい}も
\ruby{心臓}{しん|ざう}も
\ruby{故障}{こ|しやう}は
\ruby{無}{な}し、まづ
\ruby{今}{いま}のところでは
\ruby{怖}{こは}くは
\ruby{無}{な}いです。
\GWI{koseki-900370}かし
\ruby{二三日}{に|さん|にち}はまだ
\ruby{此様}{こ|ん}な
\ruby{事}{こと}もありましやうよ、
\ruby{此處}{こ|ゝ}
\ruby{二三日}{に|さん|にち}が
\ruby{峠}{たうげ}ですから。
』

と、いと
\ruby{親切}{しん|せつ}に
\ruby{語}{かた}り
\ruby{聞}{きか}せたり。


\Entry{其三十七}

% メモ 校正終了 2024-04-12
\原本頁{222-2}%
『あら
お
\ruby{止}{よし}なさいよ、
%
\ruby{頭髮}{か|み}が
\ruby{壞}{こは}れまさあネ。
%
いやですよ。
%
ほんとに、
%
\ruby{人}{ひと}を
\ruby{馬鹿}{ば|か}にしたツ!。
%
そんな
\ruby{事}{こと}は
\ruby{妾}{わたし}や
\ruby{{\換字{嫌}}}{きら}ひ
ですつてば、
%
\ruby{大}{おほ}きな
\ruby{聲}{こゑ}を
\ruby{出}{だ}しますよ。
%
ほら、
%
ほら
\ruby{御師匠}{おつ|し|よ}さんの
\ruby{下駄}{げ|た}の
\ruby{音}{おと}ぢや
ありませんか。
』

\原本頁{222-6}%
\ruby[||j>]{男}{をとこ}の
\ruby{力}{ちから}の
\ruby{{\換字{緩}}}{ゆる}む
\ruby{間}{ひま}に
\ruby{辛}{から}くも
\ruby{{\換字{逃}}}{のが}れて、
\換字{志}どけ
\ruby{無}{な}く
\ruby{亂}{みだ}れたる
\ruby{衣服}{な|り}の
\ruby{{\換字{前}}}{まへ}を
\ruby{引直}{ひき|なほ}しつ、
%
\ruby{膳}{ぜん}の
\ruby{先}{さき}に
\ruby{{\換字{遠}}}{とほ}く
\ruby{離}{はな}れて
\ruby{坐}{すわ}つたるは、
%
さして
\ruby{美}{うつく}し
といふには
あらねど、
%
\ruby{光}{ひか}り
\ruby{流}{なが}るゝが
\ruby{如}{ごと}き
\ruby{眼}{め}の
\ruby{中}{なか}に
\ruby[<j||]{{\換字{情}}}{なさけ}
\ruby{有}{あ}つて、
%
\ruby{世}{よ}に
いふ
\ruby[<j||]{男}{をとこ}
\ruby{好}{ずき}のする
\ruby{何處}{ど|こ}と
\ruby{無}{な}く
\ruby{仇}{あだ}つぽき
\ruby{廿歳}{はた|ち}ばかりの
すらりとしたる
\原本頁{222-11}\改行%
\ruby{女}{をんな}にて、
%
\ruby{人{\換字{前}}}{ひと|まへ}は
\ruby{此家}{こ|ゝ}の
\ruby{女主人}{あ|る|じ}の
\ruby{内弟子}{うち|で|し}なり、
%
\ruby[||j>]{娘}{むすめ}
\ruby[||j>]{{\換字{分}}}{ ぶん}なり
% \ruby{娘{\換字{分}}}{むすめ|ぶん}なり
なれど、
%
\原本頁{223-1}\改行%
\ruby{人}{ひと}の
\ruby{見}{み}ぬ
\ruby{時}{とき}は
\ruby{水}{みづ}
\ruby{仕業}{し|わざ}も
\ruby{爲}{さ}せらるゝ、
%
\ruby{寄食者}{かゝ|りう|ど}ともつかず
\ruby{下婢}{はし|た}ともつかぬ
\ruby{怪}{あや}しきものなれば、
%
\ruby{置}{お}く
\ruby{方}{かた}にも
\ruby{置}{お}かるゝ
\ruby{方}{かた}にも、
%
いづれ
\ruby{一寸}{ちよ|つと}したる
\ruby{關係}{あ|や}は
\ruby{潜}{ひそ}める% 【潛 u6f5b 「先先」】【潜 u6f5c 「夫夫」】併用されている
なるべし。
%
\ruby{男}{をとこ}は
\ruby{顏}{かほ}の
\ruby{色}{いろ}
\ruby{黑}{くろ}く
\ruby{{\換字{強}}壯}{ぢやう|ぶ}さうに
\ruby[||j>]{膩}{あぶら}
\ruby[||j>]{光}{ でり}の
% \ruby{膩光}{あぶら|でり}の
したる、
%
\ruby{四十餘歳}{し|じふ|いく|つ}の
\ruby{品格}{ひ|ん}の
\ruby{無}{な}きなるが、
%
\ruby{膳}{ぜん}を
\ruby{{\換字{前}}}{まへ}にして
\ruby{胡坐}{あぐ|ら}
\ruby{組}{く}めり。

\原本頁{223-6}%
\ruby{格子{\換字{戸}}}{かう|し|ど}は
\ruby{輕}{かろ}く
からりと
\ruby{開}{あ}きて、
%
やがて
\ruby{入}{い}り
\ruby{來}{きた}れるは
\ruby{果}{はた}して
\ruby{女主人}{あ|る|じ}なり。
%
\ruby{五十}{ご|じふ}に
\ruby{{\換字{近}}}{ちか}きには
\ruby{疑}{うたが}ひ
\ruby{無}{な}けれど、
%
ぶつてりと
\ruby{肥}{ふと}つたる
\ruby{{\換字{平}}顏}{ひら|がほ}の、
%
\ruby{特}{こと}に
\ruby{今}{いま}は
\ruby{{\換字{浴}}後}{ゆ|あがり}とて
\ruby{照}{て}らつきて
\ruby{赤}{あか}きに、
%
\ruby{絲}{いと}の
\ruby{如}{ごと}く
\ruby{剃}{す}りつけたる
\ruby{眉}{まゆ}の
\ruby{{\換字{嫌}}味}{いや|み}たらしく
\ruby{細}{ほそ}く、
%
\ruby{髮際}{はえ|ぎは}
\ruby{異樣}{こと|やう}に
\ruby{濃}{こ}き
\ruby{髮}{かみ}を、
\換字{志}たゝかに
\原本頁{223-10}\改行%
\ruby{油}{あぶら}つけて
\ruby{銀杏{\換字{返}}}{い|てふ|がへ}しに
\ruby{結}{ゆ}ひたる、
%
みづからは
\ruby{未}{ま}だ
\ruby{老}{お}い
\ruby{{\換字{込}}}{こ}まぬ
\ruby{意氣}{い|き}を
\ruby{示}{しめ}したる
なるべけれど、
%
\ruby{人}{ひと}は
\ruby{見}{み}るより
\ruby{恐}{おそ}れて
\ruby{{\換字{逃}}}{にげ}
\ruby{走}{はし}るべき
\ruby{態}{さま}なり。

\原本頁{224-2}%
\ruby{女主人}{あ|る|じ}は
\ruby[||j>]{糠}{ぬか}
\ruby[||j>]{袋}{ぶくろ}の
% \ruby{糠袋}{ぬか|ぶくろ}の
\ruby{絲}{いと}を
\ruby{口}{くち}に
しつゝ、
%
\ruby{手拭}{て|ぬぐひ}を
ばたりと
\ruby{一度}{ひと|たび}
\ruby{鳴}{な}らして、
\GWI{u1b048-u3099}ろりと% 「志」+「濁点」
\ruby{白}{しら}けたる
\ruby{此場}{この|ば}の% 原文通り「場」
\ruby{狀}{さま}を
\ruby{見}{み}れば、
%
\ruby{男}{をとこ}は
\ruby{何}{なに}
\ruby{喰}{く}はぬ
\ruby{顏}{かほ}して
\ruby{酒}{さけ}
\ruby{無}{な}き
\ruby{猪口}{ちよ|く}を
\ruby{吸}{す}ひ、
%
\ruby{女}{をんな}は
\ruby{徳利}{とく|り}に
\ruby{手}{て}は
\ruby{觸}{ふ}れ
ながら
\ruby{{\換字{酌}}}{しやく}を
せんとも
\ruby{爲}{せ}で
\ruby{護}{まも}り
\ruby{居}{ゐ}たる
\ruby{其}{そ}の
\ruby{呼吸}{い|き}は
\ruby{{\換字{猶}}}{なほ}
はづみて
\ruby{事實}{ま|こと}を
\ruby{語}{かた}れり。

\原本頁{224-6}%
\ruby{十{\換字{分}}}{じふ|ぶん}に
\ruby{男}{をとこ}の
\ruby{何}{なに}と
\ruby{爲}{し}たりしかを
\ruby{猜}{すゐ}したる
\ruby{女主人}{あ|る|じ}の
\ruby{顏}{かほ}は、
%
\ruby{見}{み}る〳〵
\ruby{紫色}{むら|さき}に
\ruby{脹}{は}れたるが
\ruby{如}{ごと}くなりて、

\原本頁{224-8}%
『
\ruby{何}{なに}を
\ruby{仕}{し}て
おいでだつたエ、
%
\ruby{貴郞}{おま|へ}さんは。
』

\原本頁{224-9}%
と、
%
\ruby{先}{ま}づ
\ruby{一句}{いつ|く}
\ruby[||j>]{男}{をとこ}の
\ruby{顏}{かほ}を
\ruby{見}{み}て
\ruby{詰}{なじ}りしが、

\原本頁{224-10}%
『
\ruby{先}{さき}へ
\ruby{始}{はじ}めたなあ
\ruby{惡}{わる}かつたが、
%
\ruby{飮}{や}つた
ばかりだわナ、
%
\ruby{堪{\換字{忍}}}{か|に}しねえナ。% 原文通り「堪忍」
』

\原本頁{225-1}%
と、
%
\ruby{男}{をとこ}も
さるもの、
%
\ruby{穩}{おだ}やかに
\ruby{澱}{よど}まず
\ruby{云}{い}ひ
\ruby{流}{なが}すを
\ruby{聞}{き}きて、
%
いよいよ% 原本は行末禁足のため非踊り字
\ruby{眼}{まなこ}を
\ruby{嶮}{けは}\換字{志}くし、

\原本頁{225-3}%
『
\ruby{左樣}{さ|う}かい!。
%
そりやあ
\ruby{堪{\換字{忍}}}{か|に}するも% 原文通り「堪忍」
\ruby{何}{なに}も
ありやあ
\ruby{仕}{し}ない。
』

\原本頁{225-4}%
と
\ruby{冷}{ひや}やかに
\ruby{云}{い}ひ
\ruby{切}{き}りつ、
%
\ruby{間}{あひだ}を
\ruby{隔}{お}きて、

\原本頁{225-5}%
『だつて
\ruby{盗賊}{どろ|ばう}
\ruby{猫}{ねこ}が
\ruby{暴}{あば}れた
やうだからサ。
%
\ruby{{\換字{留}}守番}{る|す|ばん}
\ruby{甲{\換字{斐}}}{が|ひ}が
\ruby{無}{な}いと
\ruby{思}{おも}つて
\ruby{聞}{き}いたんだよ。
%
お
\ruby{龍}{りゆう}、
%
お
\ruby{{\換字{前}}}{まへ}、
%
\ruby{氣}{き}を
つけ
\ruby{無}{な}くつちやあ
いけないよ。

\原本頁{225-8}%
ほんとに
\ruby{碌}{ろく}で
\ruby{無}{な}しの
\ruby{盗賊}{どろ|ばう}
\ruby{猫}{ねこ}が
\ruby{居}{ゐ}るんだからネ。
%
\ruby{恐}{おそ}ろしい
\ruby{圖々}{づう|〴〵}しい
\ruby{奴}{やつ}なんだからネ。
%
\ruby{油斷}{ゆ|だん}も
\ruby{隙}{すき}も
なりや
\ruby{仕}{し}ない。
%
\ruby{捕}{つかま}へたら
\ruby{鼻}{はな}づらを
\ruby{引擦}{ひつ|こす}つて
\ruby{{\換字{遣}}}{や}りたいぢや
\ruby{無}{な}いか。
』

\原本頁{225-11}%
と、
%
\ruby{云}{い}ひながら
\ruby{男}{をとこ}の
\ruby{對面}{むか|ふ}へ、
%
むずと
\ruby{坐}{すわ}つたり。

\原本頁{226-1}%
\ruby{男}{をとこ}は
\ruby{困}{こう}じたる
\ruby{顏}{かほ}に
\ruby[||j>]{苦}{にが}
\ruby[||j>]{笑}{わらひ}して
% \ruby{苦笑}{にが|わらひ}して
\ruby{横}{よこ}を
\ruby{向}{む}けり。

\Entry{其三十八}

% メモ 校正終了 2024-04-12
\原本頁{226-3}%
\ruby{見馴}{み|な}れ
\ruby{聞}{き}き
\ruby{馴}{な}れたるに
さまでは
\ruby{感}{かん}ぜねど、
%
\ruby{何}{なん}と
\ruby{挨拶}{あい|さつ}すべき
\ruby{言葉}{こと|ば}を
\ruby{知}{し}らねば、
%
お
\ruby{龍}{りゆう}は
\ruby{手拭}{て|ぬぐひ}
\ruby{糠袋}{ぬか|ぶくろ}を
\ruby{手渡}{て|わた}し
されたるを
\ruby{機}{しほ}に、
%
\ruby{其}{そ}を
\ruby{臺{\換字{所}}}{だい|どころ}
\ruby{{\換字{近}}}{ちか}き
\ruby{掛竿}{かけ|ざを}に
\ruby{叮嚀}{てい|ねい}に
\ruby{懸}{か}けて、
%
わざと
\ruby{暇取}{ひま|ど}りて
\ruby[g]{此方}{こなた}へ
\ruby{來}{く}れば、
%
\ruby{膳}{ぜん}の
\ruby{上}{うへ}に
\ruby{伏}{ふ}せ
ありたる
\ruby{我}{わ}が
\ruby{猪口}{ちよ|く}を、
%
\ruby{不興氣}{ふ|きよう|げ}に
\ruby{取}{と}り
\ruby{上}{あ}げたる
\ruby{主人}{ある|じ}に
\原本頁{226-7}\改行%
\ruby{向}{むか}ひて、
%
\ruby{男}{をとこ}は
\ruby{自}{みづか}ら
\ruby{徳利}{とく|り}を
\ruby{手}{て}にして、
%
\ruby{諂}{へつら}ひ
\ruby{笑}{わらひ}を
\ruby{面}{おもて}に
\ruby{{\換字{浮}}}{うか}べ
つゝ、
%
\ruby{今}{いま}や
\ruby{{\換字{酌}}}{しやく}して
\ruby{{\換字{遣}}}{や}らんとしたる
\ruby{其}{そ}の
\ruby{狀態}{あり|さま}の、
%
たとへば
\ruby{女主人}{あ|る|じ}は
\ruby{怒}{いか}つたる
\ruby{蝦蟇}{ひき|がへる}の
\ruby{如}{ごと}く、
%
\ruby{男}{をとこ}は
\ruby{{\換字{又}}}{また}
\ruby{地}{ち}に
\ruby{下}{お}りたる
\ruby{狡猾}{わる|がしこ}き
\ruby{烏}{からす}の
\ruby{如}{ごと}くなるに、
%
\ruby{思}{おも}はずも
\ruby{安芝居}{やす|しば|ゐ}の
\ruby{安役者}{やす|やく|しや}が
\ruby{出}{だ}せる
\ruby{世話物}{せ|わ|もの}の、
%
\ruby{下卑}{げ|び}たる
\ruby{一}{ひ}ト
\ruby{場}{ば}を% 原文通り「場」
\ruby{見}{み}る
\ruby{心地}{こゝ|ち}して、
%
おのれも
また
\ruby{其}{そ}の
\ruby{同}{おな}じ
\ruby{此}{こ}の
\ruby{舞臺}{ぶ|たい}に
\ruby{{\換字{交}}}{まじ}りて
\ruby{一}{ひ}ト
\ruby{役}{やく}を
\ruby{演}{す}ることかと、
%
\ruby{身}{み}に
\ruby{染}{し}みて
つく〴〵と
\ruby{嬉}{うれ}しからず
\ruby{思}{おも}ひしが、
%
\ruby{漸}{やうや}く
\ruby{二人}{ふた|り}の
\ruby{仲}{なか}の
\ruby{治}{をさ}まり
\ruby{行}{ゆ}かんと
するさま
なれば、
%
\ruby{差當}{さし|あた}り
\ruby{先}{ま}づ
\ruby{其事}{そ|れ}を
\ruby{悅}{よろこ}びて
\ruby{坐}{ざ}に
\ruby{戾}{もど}り、
%
\ruby{膳}{ぜん}の
\ruby{上}{うへ}の
\ruby{聊}{いさゝ}か
\ruby{淋}{さび}しきを
\ruby{見}{み}て、

\原本頁{227-5}%
『お
\ruby{師匠}{し|よ}さん、
%
あの
\ruby{傳}{でん}さんの
\ruby{下}{くだ}すつたものを
\ruby{開}{あ}けましやうか。
』

\原本頁{227-6}%
と、
%
\ruby{機{\換字{嫌}}}{き|げん}
\ruby{取}{と}り
\ruby{顏}{がほ}に
\ruby{優}{やさし}しく
\ruby{云}{い}へば、
%
\ruby{主人}{ある|じ}も
\ruby{此女}{こ|れ}に
\ruby{對}{むか}つては
\ruby{言葉}{こと|ば}を
\ruby{和}{やは}らげつ。

\原本頁{227-8}%
『アヽ、
%
たしか
\ruby{雀燒}{すゞめ|やき}だつたネ、
%
ぢやあ
\ruby{開}{あ}けておくれ!。
%
オヤ
ありやあ
\ruby{汝}{おまへ}につて
\ruby{彼人}{あの|ひと}が
\ruby{吳}{く}れたんだつたのに。
』

\原本頁{227-10}%
『あら
いやな、
%
そんな
\ruby{事}{こと}を!。
%
どうだつて
\ruby{好}{い}いぢや
ありませんか。
』

\原本頁{228-1}%
『
\ruby{左樣}{さ|う}かい。
%
ぢやあ、
%
まあ、
%
\ruby{貰}{もら}ふよ。
%
\ruby{面倒}{めん|ど}くさいから
\ruby{取}{と}り
\ruby{{\換字{分}}}{わ}けずともだよ。
%
あゝ
\ruby{左樣}{さ|う}さ、
%
\ruby{其}{その}
\ruby{儘}{まゝ}で
\ruby{好}{い}いやネ、
%
\ruby{構}{かま}やあしないよ。
』

\原本頁{228-3}%
\ruby{大}{おほき}からぬ
\ruby{杉折}{すぎ|をり}は
\ruby{膳}{ぜん}の
\ruby{傍}{かたはら}に
\ruby{出}{いだ}されたり。

\原本頁{228-4}%
『オヤ
\ruby{此}{これ}あ
\ruby{千住}{せん|じゆ}のだよ、
\換字{志}かも
\ruby{鮒}{ふな}だ、
%
\ruby{自轉車}{じ|てん|しや}
\ruby{天狗}{てん|ぐ}が
\ruby{物}{もの}を
\ruby{吳}{く}れると、
%
いつでも
\ruby{奇妙}{き|めう}に
\ruby{{\換字{遠}}}{とほ}い
\ruby{{\換字{所}}}{ところ}のもの
ばかり
だから
\ruby{可笑}{を|か}しいのさ、
%
\ruby{帝釋}{たい|しやく}さまの
お
\ruby{水}{みづ}を
\ruby{何}{なん}でも
\ruby{無}{な}い
\ruby{日}{ひ}に
\ruby{持}{も}つて
\ruby{來}{き}て
\ruby{吳}{く}れたり
なんぞするのは、
%
\ruby{自轉車}{じ|てん|しや}
\ruby{乘}{の}りで
\ruby{無}{な}くつちやあ
\ruby{出來}{で|き}ない
\ruby{事}{こと}だよ。
%
ン、
%
\ruby{中々}{なか|〳〵}
おいしいよ、
%
\ruby{汝}{おまへ}も
お
\ruby{食}{あが}りな、
%
\ruby{一杯}{ひと|つ}
あげやう。
』

\原本頁{228-9}%
『イヽエ
\ruby{妾}{わたし}は。
』

\原本頁{228-10}%
『ハヽヽ、
%
ちつとも
\ruby{飮}{や}らないだけは、
%
ほんとに
\ruby{汝}{おまへ}にも
\ruby{似合}{に|あ}はないよ。
%
だけれど、
%
\ruby{其行狀}{そ||れ}で
\ruby{飮}{や}られちやあ
\ruby{大變}{たい|へん}だからネ、
%
\ruby{其}{それ}も
\ruby{可}{い}いかも
\ruby{知}{し}れないよ。
』

\原本頁{229-2}%
『あらまあ
\ruby{甚}{ひど}い
\ruby{事}{こと}を。
』

\原本頁{229-3}%
『だつて
お
\ruby{酒}{さけ}まで
\ruby{好}{すき}だつた
\ruby{日}{ひ}にやあ
\ruby{何樣}{ど|う}したつて
お
\ruby{{\換字{前}}}{まへ}は、
%
\ruby{紀伊國屋}{き|の|くに|や}が
\ruby{演}{し}さうな
\ruby{肌}{はだ}の
\ruby{女}{をんな}に
なるからねえ!。
%
\ruby{折角}{せつ|かく}
\ruby{妾}{わたし}の
\ruby{名跡}{あ|と}を
\ruby{取}{と}つて
\ruby{貰}{もら}はうと
\ruby{思}{おも}つて
\ruby{居}{ゐ}たつて、
%
\ruby{何樣}{ど|ん}な
\ruby{場}{ば}を% 原文通り「場」
お
\ruby{{\換字{前}}}{まへ}が
\ruby{出}{だ}して
\ruby{仕舞}{し|ま}ふか
\ruby{知}{し}れや
しないもの!。
』

\原本頁{229-7}%
『いやですよ、
%
お
\ruby{師匠}{し|よ}さん、
%
そんな
\ruby{事}{こと}を
\ruby{云}{い}つちやあ、
%
\ruby{妾}{わたし}は
もう
\ruby[g]{澤山}{たんと}
\ruby{凝}{こ}りて
\ruby{居}{ゐ}るんですもの、
%
いつまでも
おとなしく
\ruby{仕}{し}て
\ruby{居}{ゐ}て
\ruby[<j||]{一生}{いつ|しやう}
\ruby{獨身}{どく|しん}で、
%
お
\ruby{師匠}{し|よ}さんの
\ruby{傍}{そば}に
ばかり
\ruby{居}{ゐ}る
つもりなんですから。
』

\原本頁{229-10}%
『
\ruby{嬉}{うれ}しいねえ。
%
お
\ruby{{\換字{前}}}{まへ}が
\ruby{左樣}{さ|う}いふ
\ruby{氣}{き}で
\ruby{居}{ゐ}て
\ruby{吳}{く}れりやあ
\ruby{妾}{わたし}あ
\ruby{此}{この}
\ruby{上}{うへ}
\ruby{無}{な}しさ。
%
いよ〳〵
\ruby{左樣}{さ|う}なら
\ruby{妾}{わたし}の
\ruby{事}{こと}をネ、
%
これから
お
\ruby{母}{つか}さん
お
\ruby{母}{つか}さんと
\ruby{呼}{よ}んでも
\ruby{可}{い}いよ。
%
\ruby{妾}{わたし}の
\ruby{方}{はう}ぢやあ
\ruby{疾}{とう}から
\ruby{既}{もう}
\ruby{實}{じつ}の
\ruby{娘}{こ}のやうに
\ruby{思}{おも}つて
\ruby{居}{ゐ}るんだから。
』

\原本頁{230-3}%
『お
\ruby{師匠}{し|よ}さん、
%
そりやあ
\ruby{本當}{ほん|たう}なの、
%
きつと
\ruby{本當}{ほん|たう}なの?。
%
お
\ruby{母}{つか}さんと
\ruby{云}{い}つても
\ruby{惡}{わる}かあ
\ruby{無}{な}くつて?。
』

\原本頁{230-5}%
『あゝ
\ruby{可}{いゝ}ともさ。
%
\ruby{妾}{わたし}あ
\ruby{何樣}{ど|ん}なに
\ruby{嬉}{うれ}しいか
\ruby{知}{し}れや
しないよ。
』

\原本頁{230-6}%
\ruby{男}{をとこ}は
\ruby{此}{この}
\ruby{時}{とき}まで
\ruby{手持}{て|もち}
\ruby{無}{な}くて、
%
\ruby{二人}{ふた|り}が
\ruby{對話}{はな|し}を
\ruby{聞}{き}き
\ruby{居}{ゐ}たりしが、
%
こゝに
むぐ〳〵と
\ruby{口}{くち}を
\ruby{動}{うご}かして、

\原本頁{230-8}%
『お
\ruby{母}{つか}さんに
しちやあ
\ruby{變}{へん}に
\ruby{{\換字{若}}}{わか}いナ。
』

\原本頁{230-9}%
と、
%
\ruby{阿諛}{あ|ゆ}に
\ruby{似}{に}たる
\ruby{語}{ご}を
\ruby{挿}{さしはさ}めば、
%
\ruby{女主人}{あ|る|じ}は
\ruby{忽}{たちま}ち、

\原本頁{230-10}%
『
\ruby{何}{なん}だとエ、
%
\ruby{餘計}{よ|けい}な
\ruby{御世話}{お|せ|わ}だよ。
%
\ruby{默}{だま}つて
おいで!。
』

\原本頁{230-11}%
と、
%
たしなめは
\ruby{仕}{し}たれど
\ruby{腹}{はら}は
\ruby{立}{た}てぬ
\ruby{顏}{かほ}なり。

\原本頁{231-1}%
『
\ruby{妾}{わたし}もネ、
%
お
\ruby{{\換字{前}}}{まへ}は
\ruby{知}{し}るまいが
\ruby{子}{こ}は
あるけれども、
{---}{---}
もつとも
\原本頁{231-2}\改行%
\ruby{義理}{ぎ|り}だけで
\ruby{根}{ね}は
\ruby{他人}{た|にん}なのさ、
%
だもん
だから
お
\ruby{{\換字{前}}}{まへ}、
%
\ruby{妾}{わたし}を
\ruby{馬鹿}{ば|か}にして、
%
\ruby{一人}{ひと|り}は
\ruby{女}{をんな}の
\ruby{癖}{くせ}に
\ruby{生意氣}{なま|い|き}に
\ruby{敎員}{けう|ゐん}
なんぞに
なりやがつて、
%
\ruby{{\換字{近}}在}{きん|ざい}に
\ruby{一人}{ひと|り}で
\ruby{暮}{くら}して
\ruby{居}{ゐ}るし、
%
\ruby{其}{その}
\ruby[||j>]{弟}{おとうと}は
\ruby{書生}{しよ|せい}を
\ruby{仕}{し}て
\ruby{居}{ゐ}るが、
%
\ruby{二人}{ふた|り}とも
\ruby{妾}{わたし}を
\ruby{馬鹿}{ば|か}に
\ruby{仕}{し}きつて
\ruby{居}{ゐ}て、
%
\ruby{此家}{こ|ゝ}
なんぞへは
\ruby{寄}{よ}りつきも
\ruby{仕}{し}ないんだが、
%
ほんとに
まあ
\ruby{何樣}{ど|ん}なに
\ruby{高慢}{かう|まん}な
\ruby{憎}{にく}らしい
\ruby{奴等}{やつ|ら}だらう!。
%
だから
\ruby{妾}{わたし}も
\ruby{其等}{そい|ら}を
\ruby{子}{こ}だとは
\ruby{思}{おも}つて
\ruby{居}{ゐ}やしないのさ。
%
\ruby{同}{おな}じ
\ruby{他人}{た|にん}なら
\ruby{妾}{わたし}は
お
\ruby{{\換字{前}}}{まへ}を、
%
ほんたうに
\ruby{妾}{わたし}の
\ruby{娘}{むすめ}にして、
%
\ruby{何樣}{ど|ん}なにでも
\ruby{好}{よ}くして
\ruby{{\換字{遣}}}{や}りたいよ。
%
なあに
\ruby{何}{なん}にも
\ruby{有}{あ}りや
\ruby{仕}{し}ないけれど、
%
それでも
お
\ruby{{\換字{前}}}{まへ}、
%
\ruby{妾}{わたし}は
\ruby[<j||]{妾}{わたし}
\ruby{一人}{ひと|り}で
もつて、
%
どうやら
\ruby{斯樣}{こ|う}やら
\ruby{{\換字{遣}}}{や}つて
\ruby{來}{き}て
\ruby{居}{ゐ}るんだからネ、
%
それだけの
\ruby{事}{こと}は
お
\ruby{{\換字{前}}}{まへ}に
\ruby{譲}{ゆづ}るつもりなのさ。
%
エ、
%
\ruby{其}{そ}の
\ruby{娘}{むすめ}かエ、
%
\原本頁{232-1}\改行%
\ruby{五十}{い|そ}と
\ruby{云}{い}つてネ、
%
\ruby{容貌}{きり|やう}も
\ruby{惡}{わる}かあ
\ruby{無}{な}いが、
%
\ruby{愛}{あい}の
\ruby{無}{な}い、
%
\ruby{矢張}{やつ|ぱ}り
あの
\ruby{妾}{わたし}の
\ruby{大{\換字{嫌}}}{だい|きら}ひな
\ruby{海老茶}{え|び|ちや}の
\ruby{袋}{ふくろ}を
\ruby{穿}{は}いてる
\ruby{奴}{やつ}なのさ。
%
\ruby{男}{をとこ}の
\ruby{子}{こ}は
\ruby[g]{松之助}{まつのすけ}と
いつて、
%
\ruby{直}{ぢき}
そこの
\ruby{下谷}{した|や}に
\ruby{居}{ゐ}るのだがネ、
%
\ruby{此}{こ}の
\ruby{方}{はう}は
まだしも
\ruby{素直}{す|なほ}な
\ruby{性質}{た|ち}だから
\ruby{手}{て}なづけては
\ruby{居}{ゐ}るけれど、
%
やつぱし
\ruby{姊}{あね}びいき
だから
\ruby{妾}{わたし}の
\ruby{爲}{ため}にやあ、
%
\ruby[g]{末始{\換字{終}}}{すゑしじゆう}は
\ruby{爲}{な}りさうもない
\ruby{奴}{やつ}なのさ。
%
\ruby{此樣}{こ|う}いふ
\ruby{譯}{わけ}なんだから、
%
お
\ruby{{\換字{前}}}{まへ}
\ruby{次第}{し|だい}で、
%
ほんとに
お
\ruby{{\換字{前}}}{まへ}が
\ruby{妾}{わたし}の
\ruby{後}{あと}を
\ruby{取}{と}る
\ruby{氣}{き}に
なつて
お
\ruby{吳}{く}れなら、
%
どんなにでも
\ruby{妾}{わたし}は
お
\ruby{{\換字{前}}}{まへ}に
\ruby{肩}{かた}を
\ruby{入}{い}れるよ。
%
\原本頁{232-8}\改行%
\ruby{其}{その}
\ruby{代}{かは}り
お
\ruby{{\換字{前}}}{まへ}
\換字{志}つかりしてネ、
%
よその
\ruby{下}{くだ}らない
\ruby{猫}{ねこ}
なんぞに
\ruby{手}{て}を
かけられたり
なんぞ
\ruby{仕}{し}ないやうに
\ruby{仕}{し}て
お
\ruby{吳}{く}れで
\ruby{無}{な}くつちや
いけないよ。
%
ハヽヽ。
%
おや、
%
\ruby{暗}{くら}くなつて
\ruby{來}{き}たネ、
%
\ruby[g]{洋燈}{らんぷ}さへ
\ruby{準備}{し|たく}が
\ruby{仕}{し}てあるなら
\ruby{構}{かま}はないから、
%
お
\ruby{湯}{ゆ}へ
\ruby{行}{い}つて
おいでな。
%
\ruby{妾}{わたし}あ
お
\ruby{{\換字{前}}}{まへ}が
\ruby{美麗}{き|れい}だつて
\ruby{云}{い}はれると
\ruby[g]{眞實}{ほんと}に
\ruby{天狗}{てん|ぐ}なんだから、
%
いくらでも
\ruby{悠々}{ゆつ|くり}
\ruby{磨}{みが}いて
おいで!。
』

\Entry{其三十九}

『あら
\ruby{虛言}{う|そ}ばつかり!。
いくら
\ruby{磨}{みが}いたつて、どうせ
\ruby{美麗}{き|れい}になんか
\ruby{成}{な}りやあ
\ruby{仕}{し}ませんよ。
』

とは
\ruby{云}{い}ひたれど
\ruby{師匠}{し|しやう}が
\ruby{言葉}{こと|ば}に
\ruby{{\換字{悅}}}{よろこ}べるさまは、
\ruby{掩}{おほ}はんとして
\ruby{掩}{おほ}ひきれず、
\ruby{愛嬌溢}{あい|けう|こぼ}るゝ
\ruby{眼}{め}のしほに
\ruby{見}{み}えたり。
\ruby{女主人}{あ|る|じ}はこれを
\ruby{見}{み}て
\ruby{取}{と}りて、
\ruby{此}{これ}もおなじく
\ruby{笑顏}{ゑ|がほ}つくり、

『ナニ
\ruby{妾}{わたし}がお
\ruby{茶々羅}{ちや|〳〵|ら}を
\ruby{云}{い}うもんかネ。
\ruby{傳}{でん}さんだつて
\ruby{{\GWI{u6df8-t}}}{せい}さんだつて
\ruby{{\換字{勝}}}{かつ}さんだつて、みんなお
\ruby{前}{まへ}が
\ruby{美麗}{き|れい}だもんだから
\ruby{大騒}{おほ|さわ}ぎ
\ruby{{\GWI{u9063-k}}}{や}つてるんだあネ。
\ruby{虛言}{う|そ}だと
\ruby{思}{おも}ふなら
\ruby{聞}{き}いて
\ruby{御覧}{ご|らん}!。
』

と、
\ruby{重}{かさ}ねて
\ruby{復}{また}も
\ruby{{\換字{悅}}}{よろこ}ばせにかゝれば、

『あら、あんまりだわ
\ruby{御師匠}{お|し|よ}さん!。
たんと
\ruby{御嬲}{お|なぶ}りなさいよ、ようござんすわ。
』

と、
\ruby{此度}{こ|たび}はつんとして
\ruby{横}{よこ}を
\ruby{向}{む}きしが、
\ruby{媚}{なまめ}きながら
\ruby{微瞋}{やヽ|いか}れる
\ruby{顏}{かほ}は、
\ruby{女主人}{あ|る|じ}が
\ruby{言葉}{こと|ば}もいつはりならず
\ruby{艶}{えん}なり。

やゝありて
\ruby{思}{おも}ひ
\ruby{出}{だ}したるやうに、

『
\ruby{少}{すこ}し
\ruby{早}{はや}くつても
\ruby[g]{洋燈}{らんぷ}を
\ruby{點}{つ}けましやう。
』

と、
\ruby{云}{い}ひさまに
\ruby{立}{た}つてお
\ruby{龍}{りゆう}は
\ruby{去}{さ}りつ、
\ruby{何}{なに}をなせるにや
\ruby[g]{少時其姿}{しばらくそのすがた}を
\ruby{見}{み}せざりしが、
\ruby{火}{ひ}を
\ruby{點}{てん}じたる
\ruby[g]{釣洋燈}{つりらんぷ}を
\ruby{持}{も}ち
\ruby{來}{きた}りて、
\ruby{座敷}{ざ|しき}の
\ruby{中央}{ま|なか}に
\ruby{高}{たか}く
\ruby{吊}{つ}りし
\ruby{時}{とき}には、
\ruby{今}{いま}までのほつれかゝりたる
\ruby{髷}{まげ}のあとかたも
\ruby{無}{な}く、
\ruby{其}{そ}の
\ruby{頭髪}{か|み}は
\ruby{早}{はや}くも
\ruby{結}{ゆ}ひかへられて、さつぱりとしたる
\ruby{束髪}{そく|はつ}の
\ruby{美}{うつく}しきが、
\ruby{燈}{ひ}の
\ruby{光}{ひかり}に
\ruby{鮮}{あざ}やかに
\ruby{映}{うつ}し
\ruby{出}{いだ}されたり。

『オヤ
\ruby{早變}{はや|がは}りだネエ、
\ruby{吃驚}{びつ|くり}させられたよ。
チョイと
\ruby[g]{彼方}{あつち}を
\ruby{向}{む}いて
\ruby{御見}{お|み}せナ、ヘーエそれが
\ruby{花月{\換字{巻}}}{か|げつ|まき}とやらかエ?。
』

『ハァ、
\ruby{左様}{さ|う}ですの。
\ruby{似合}{に|あ}はなくつて?。
』

『イゝエ
\ruby{似合}{に|あ}はないどころぢあ
\ruby{無}{な}いよ、これは
\ruby{此}{これ}でもつて、いつそ
\ruby{{\換字{叉}}好}{また|い}いよ。
お
\ruby{前}{まへ}は
\ruby{徳}{とく}な
\ruby{顏立}{かほ|だち}で、
\ruby{何}{なん}に
\ruby{結}{ゆ}つても
\ruby{似合}{に|あ}ふのが
\ruby{妙}{めう}だネ。
だが
\ruby{束髪}{そく|はつ}も
\ruby{此頃}{この|ごろ}は
\ruby{考}{かんが}へたネ、
\ruby{一}{ひ}ト\GWI{koseki-900370}きり
\ruby{人}{ひと}が
\ruby{爲}{し}た
\ruby[g]{蝸牛}{まひ〳〵つぶろ}の
\ruby{親方見}{おや|かた|み}たやうなのなんざあ、
\ruby{堪}{たま}らなく
\ruby{可厭}{い|や}なもんだつたがねえ、ハヽヽ。
』

『ホヽヽ、
\ruby{御師匠}{お|し|よ}さんの
\ruby{口}{くち}には
\ruby{叶}{かな}いませんわ。
ぢやあ
\ruby{一寸御湯}{ちよ|つと|お|ゆう}へ。
』

『あゝ
\ruby{可}{い}いとも!。
さあ〳〵
\ruby{髪}{かみ}も
\ruby{出來}{で|き}たし、
\ruby{行}{い}つておいで、
\ruby{行}{い}つておいで!。
』

『ぢやあ
\ruby{一寸}{ちよ|いと}。
』

\ruby{云}{い}ひながら
\ruby{會釋}{ゑ|しやく}して
\ruby{身}{み}を
\ruby{起}{おこ}し、やがて
\ruby{徐}{しづか}に
\ruby{出}{で}て
\ruby{行}{ゆ}きけるが
\ruby{輕}{かろ}らかなる
\ruby{下駄}{げ|た}の
\ruby{音}{おと}は
\ruby{幾程}{いく|ほど}も
\ruby{無}{な}く
\ruby{{\GWI{u6d88-k}}}{き}えぬ。

『
\ruby{大分念入}{だい|ぶ|ねん|い}りにあやなすぢやあ
\ruby{無}{ね}えか。
』

\ruby{男}{をとこ}は
\ruby{女主人}{あ|る|じ}がお
\ruby{龍}{りゆう}に
\ruby{對}{たい}する
\ruby{擧動}{ふる|まひ}を
\ruby{怪}{あや}しむやうに
\ruby{云}{い}へば、やゝ
\ruby{醉}{よ}ひたる
\ruby{女主人}{あ|る|じ}はそれには
\ruby{關}{かま}はず、
\ruby{今迄}{いま|まで}は
\ruby{他}{ひと}の
\ruby{見}{み}る
\ruby{目}{め}を
\ruby{{\換字{兼}}}{か}ねて
\ruby{堪}{こら}へ
\ruby{居}{ゐ}しが、
\ruby{今}{いま}は
\ruby{憚}{はばか}るところも
\ruby{無}{な}きに、
\ruby{突然手}{いき|なり|て}あたり
\ruby{任}{まか}せに
\ruby{男}{をとこ}の
\ruby{口}{くち}の
\ruby{端}{はた}をいやといふほど
\ruby{捻}{つね}りて、

『あやなすぢやあ
\ruby{無}{ね}えかも
\ruby{無}{な}いもんだ。
\ruby{人}{ひと}の
\ruby{居}{い}ない
\ruby{中何}{うち|なに}を
\ruby{爲}{し}やうと
\ruby{仕}{し}たんだエ。
』

と、
\ruby{新}{あらた}に
\ruby{罪}{つみ}を
\ruby{糺}{ただ}さんとする
\ruby{其勢}{そのい|きほひ}なか〳〵
\ruby{當}{あた}りがたければ
\ruby{男}{をとこ}はこれに
\ruby{辟易}{へき|えき}して
\ruby{聊}{いさゝ}か
\ruby{身}{み}を
\ruby{{\GWI{u9000-k}}}{ひ}きぬ。

『ナニたゞ
\ruby{調戯}{から|か}つたばかりだよ、
\ruby{戯談}{じやう|だん}だわナ。
』

『フン、
\ruby{戯談}{じやう|だん}から
\ruby{駒}{こま}が
\ruby{出無}{で|な}くつて
\ruby{御仕合}{お|し|あはせ}さ。
』

\ruby{長{\GWI{u7159-k}}管}{なが|ぎせ|る}は
\ruby{忽}{たちま}ち
\ruby{烈}{はげ}しく
\ruby{膝頭}{ひざ|がしら}を
\ruby{突}{つ}きぬ。
\ruby{男}{をとこ}はいよ〳〵
\ruby{後}{あと}じさりするのみ。

『あやまつた〳〵。
いゝ
\ruby{加減}{か|げん}にして
\ruby{{\換字{呉}}}{く}れ、
\ruby{痛}{いて}えやナ。
』

『
\ruby{痛}{いた}くつても
\ruby{關}{かま}ふもんか、
\ruby{碌}{ろく}で
\ruby{無}{な}しめ。
』

『あやまつたと
\ruby{云}{い}ふに
\ruby{執念深}{しふ|ねん|ぶか}いなあ。
』

『
\ruby{執念深}{しふ|ねん|ぶか}いなあ
\ruby{妾}{わたし}の
\ruby{性}{しやう}だよ。
ほんとに
\ruby{彼女}{あ|れ}なんぞに
\ruby{指}{ゆび}でもさして
\ruby{御覧}{ご|らん}、
\ruby{今度}{こん|ど}からたゞ
\ruby{置}{お}きやあ
\ruby{仕無}{し|な}いから。
\ruby{彼女}{あ|れ}あ
\ruby{妾}{わたし}が
\ruby{大事}{だい|じ}にかけてるんだもの。
』

『だから
\ruby{彼様}{あ|ん}なに
\ruby{味}{あぢ}に
\ruby{{\GWI{u6587-k}}}{あや}なして
\ruby{何様}{ど|う}するんだと
\ruby{聞}{き}くのだ!。
』

『どうしたつて
\ruby{宣}{い}いよ、
\ruby{汝}{おまへ}の
\ruby{御世話}{お|せ|わ}にやあならない。
\ruby{妾}{わたし}も
\ruby{取}{と}る
\ruby{年}{とし}だし、
\ruby{子}{こ}は
\ruby{無}{な}いし、どうせ
\ruby{汝}{おまへ}はちつとも
\ruby{當}{あて}にやあならないしするから、
\ruby{彼女}{あ|れ}に
\ruby{後}{あと}を
\ruby{{\GWI{u9063-k}}}{や}つて
\ruby{彼女}{あ|れ}にかゝるんだよ。
』

『フーム、
\ruby{{\換字{強}}氣}{がう|ぎ}に
\ruby{彼岸詣}{ひ|がん|まひ}りでも
\ruby{仕}{し}さうな
\ruby{風}{ふう}な
\ruby{事}{こと}をいふナ。
そりやあ
\ruby{眞實}{ほん|たう}かエ。
』

『さうさ、ほんたうで
\ruby{無}{な}くつてサ。
』

『ハヽヽ、
\ruby{虛言}{う|そ}を
\ruby{云}{い}ひねえナ。
\ruby{止}{よ}しねえ〳〵!。
\ruby{繼子}{まま|こ}だつて
\ruby{何}{なん}だつて
\ruby{二人}{ふた|り}も
\ruby{子}{こ}もあるのに、
\ruby{其様}{そ|ん}な
\ruby{事}{こと}がなんで
\ruby{出來}{で|き}るもんか。
』

『
\ruby{出來無}{で|き|な}いものかネ、
\ruby{爲}{す}るんだもの!。
\ruby{無理}{む|り}でも
\ruby{左様}{さ|う}して
\ruby{妾}{わたし}やあ
\ruby{彼女}{あ|れ}にかゝるんだよ。
\ruby{相続人}{さう|ぞく|にん}になつてる
\ruby{五十}{い|そ}は
\ruby{死}{し}ぬかも
\ruby{知}{し}れないのだから。
』

『ハヽヽ、
\ruby{{\換字{強}}氣}{がう|ぎ}に
\ruby{老}{お}い
\ruby{{\GWI{u8fbc-k}}}{こ}んだ
\ruby{事}{こと}をいふが、
\ruby{乃公}{お|れ}まで
\ruby{食}{く}はせやうと
\ruby{云}{い}ふなあ、ちつと
\ruby{甚}{ひど}い!。
どうしてお
\ruby{前}{めへ}が
\ruby{後}{あと}を
\ruby{案}{あん}じる
\ruby{風}{ふう}かエ。
\ruby{汝}{おめへ}は
\ruby{彼女}{あ|れ}をすつかり
\ruby{取}{と}り
\ruby{{\GWI{u8fbc-k}}}{こ}んで、\GWI{koseki-900370}やぶつて
\ruby{{\GWI{u9063-k}}}{や}らうと
\ruby{云}{い}うんだらう。
』

『
\ruby{何}{なん}だとエ?。
』

『
\ruby{知}{し}れた
\ruby{事}{こと}さ!。
\ruby{食物}{くひ|もの}に
\ruby{仕}{し}やうと
\ruby{云}{い}ふんだらう!。
\ruby{何}{なに}も
\ruby{一人}{ひと|り}で
\ruby{占}{し}めずともの
\ruby{事}{こと}だ、
\ruby{乃公}{お|れ}にも
\ruby{{\換字{半}}分}{はん|ぶん}
\ruby{{\GWI{u907a-k}}}{よこ}しねえナ。
\ruby{圃}{はたけ}でこしらへたものぢやあ
\ruby{有}{あ}るまいし、たゞ
\ruby{穫}{と}つた
\ruby{魚}{さかな}ぢやあ
\ruby{無}{ね}えか、
\ruby{吝}{おし}みなさんナ。
\ruby{其代}{その|かわ}り
\ruby{骨}{ほね}つきの
\ruby{方}{はう}は
\ruby[g]{其方}{そつち}へ
\ruby{{\GWI{u9063-k}}}{や}らあ!。
』

『
\ruby{畜生}{ちく|しやう}!、
\ruby{惡徒}{あく|とう}め!、えゝ
\ruby{仕方}{し|かた}が
\ruby{無}{な}い!。
それぢやあ
\ruby{片身}{かた|み}はあげるからネ、
\ruby{要}{い}る
\ruby{時}{とき}に
\ruby{何時}{い|つ}でも
\ruby{庖丁}{はう|ちやう}をお
\ruby{貸}{か}し!。
』


\Entry{其四十}

\ruby{互}{たがひ}の
\ruby{胸中}{む|ね}に
\ruby{塊物}{も|の}はありながら、
\ruby{相酌}{あひ|じやく}の
\ruby{酒}{さけ}にいつしか
\ruby{解}{と}け
\ruby{合}{あ}つて、
\ruby{男}{をとこ}が
\ruby{{\換字{勤}}}{つと}むる
\ruby{亭主役}{てい|しゆ|やく}、
\ruby{銚子}{てう|し}のかはり
\ruby{目間}{め|ま}を
\ruby{拔}{ぬ}けさせねば、
\ruby{女主人}{あ|る|じ}は
\ruby{湯上}{ゆ|あがり}の
\ruby{早}{はや}くも
\ruby{上機{\換字{嫌}}}{じやう|き|げん}となつて、

『そりやあ
\ruby{幾千}{いく|ら}でも
\ruby{働}{はたら}かうが、
\ruby{一體}{いつ|たい}
\ruby{彼女}{あ|れ}あ
\ruby{何樣}{ど|う}した
\ruby{譯}{わけ}の
\ruby{娘}{こ}なんだ?。
いつ
\ruby{聞}{き}いても
\ruby{些仔細}{ちと|し|さい}があつてとばかしで、
\ruby{聞}{き}かされないが。
』

と
\ruby{男}{をとこ}の
\ruby{云}{い}ふを
\ruby{聞}{き}いて
\ruby{舌}{した}なめずりしつ
\ruby{低聲}{こ|ごゑ}に
\ruby{{\換字{説}}出}{とき|いだ}したり。

『
\ruby{汝}{おまへ}は
\ruby{成程}{なる|ほど}
\ruby{知}{し}るまいがネ、
\ruby{一昨々年}{さき|を|と|ゝし}の
\ruby{春}{はる}までは
\ruby{彼女}{あ|れ}も
\ruby{矢張}{やつ|ぱ}り、
\ruby{妾}{わたし}のところへ
\ruby{稽{\換字{古}}}{けい|こ}に
\ruby{來}{き}た
\ruby{娘}{こ}さ。
』

『ウン。
』

『
\ruby{内務省}{ない|む|しやう}とかの
\ruby{小吏}{こし|べん}の
\ruby{老人}{おぢい|さん}と、
\ruby{{\換字{父}}子二人}{おや|こ|ふた|り}きりで
\ruby{暮}{くら}して
\ruby{居}{ゐ}たんだが、
お
\ruby{{\換字{父}}}{とつ}さんが
\ruby{日光羊羹}{につ|くわう|やう|かん}
\ruby{見}{み}たやうに
\ruby{變}{へん}に
\ruby{乾固}{ひ|かた}まつた
\ruby{朴實}{こく|めい}な
\ruby{人}{ひと}だつたのには
\ruby{似合}{に|あ}はないで、あの
\ruby{子}{こ}は
\ruby{{\換字{蓮}}葉}{はす|は}でも
\ruby{無}{な}いが
\ruby{妙}{めう}に
\ruby{{\換字{浮}}氣}{うは|き}つぽい、
お
\ruby{狭}{きやん}な
\ruby{面白}{おも|しろ}いところのある、
\ruby{好}{す}いた
\ruby{男}{をとこ}になら
\ruby{生命}{いの|ち}でも
\ruby{抛}{はふ}り
\ruby{出}{だ}さうツてつたやうな
\ruby{肌合}{はだ|あひ}の
\ruby{娘}{こ}で、
\ruby{同}{おな}い
\ruby{齡}{どし}ぐらゐな
\ruby{娘{\換字{達}}}{こ|たち}が
\ruby{集}{よ}つて
\ruby{談話}{はな|し}を
\ruby{仕}{し}た
\ruby{時}{とき}、
お
\ruby{七}{しち}の
\ruby{爲}{し}た
\ruby{事}{こと}が
\ruby{{\換字{道}}理}{もつ|とも}だといつて
\ruby{一同}{みん|な}に
\ruby{笑}{わら}はれたつて、
\ruby{泣}{な}いて
\ruby{口惜}{く|やし}がつて
\ruby{怒}{おこ}つた
\ruby{事}{こと}がある
\ruby{程}{ほど}なのさ。
そんな
\ruby{調子}{てう|し}だつたもんだから
\ruby{年齡}{と|し}も
\ruby{行}{ゆ}かないのに、これも
\ruby{矢張}{やつ|ぱ}り
\ruby{吾家}{う|ち}へ
\ruby{來}{き}て
\ruby{居}{ゐ}た
\ruby{建具屋}{たて|ぐ|や}の
\ruby{息子}{むす|こ}の
\ruby{源}{げん}といふいなせな
\ruby{男}{をとこ}と
\ruby{人知}{ひと|し}れず
\ruby{出來}{で|き}て
\ruby{仕舞}{し|ま}つたのさ。
』

『フーム、なある
\ruby{程}{ほど}。
お
\ruby{{\換字{前}}}{まへ}が
\ruby{撮合山}{とり|も|ち}を
\ruby{行}{や}つたんだナ。
\ruby{兩方}{りやう|はう}から
\ruby{拜}{おが}まれて
\ruby{錢}{ぜに}を
\ruby{取}{と}つたらう!。
\ruby{惡徒}{あく|とう}ツて
\ruby{云}{い}ふなあ
\ruby{其樣}{さ|う}いふのゝ
\ruby{事}{こと}だぜ。
』

『
\ruby{{\換字{交}}}{ま}ぜるなら
\ruby{後}{あと}を
\ruby{話}{はな}さないよ。
』

『あやまつた、あやまつた、それから。
』

『
\ruby{其}{そ}の
\ruby{中}{うち}に
\ruby{彼}{あ}の
\ruby{娘}{こ}の
お
\ruby{{\換字{父}}}{とつ}さんが
\ruby{病}{わづら}ひついて、
\ruby{老齡}{と|し}だから
\ruby{叶}{かな}はない、
\ruby{死}{ごね}つちまつたんだ。
すると
\ruby{駿府}{すん|ぷ}とかゝら
\ruby{叔母}{を|ば}さんが
\ruby{出}{で}て
\ruby{來}{き}て、あの
\ruby{娘}{こ}を
\ruby{田舎}{ゐな|か}へ
\ruby{{\換字{連}}}{つ}れて
\ruby{行}{い}かうといふのさ。
そら
\ruby{{\換字{情}}夫}{をと|こ}の
\ruby{一件}{いつ|けん}があるから
\ruby{行}{い}きたかあ
\ruby{無}{な}いが、まさか
\ruby{十七八}{じう|しち|はち}だから
\ruby{曝露}{さら|け}け
\ruby{出}{だ}して
\ruby{言}{い}ふことあ
\ruby{出來}{で|き}ず、
\ruby{自{\換字{分}}}{じ|ぶん}の
\ruby{家}{うち}に
\ruby{財産}{しん|だい}は
\ruby{無}{な}し、
\ruby{他}{ほか}に
\ruby{身寄}{み|より}も
\ruby{何}{なに}も
\ruby{無}{な}いから、
\ruby{楯}{たて}にして
\ruby{取}{と}る
\ruby{理屈}{り|くつ}が
\ruby{無}{な}いんで、とう〳〵
\ruby{駿府}{すん|ぷ}へ
\ruby{{\換字{連}}}{つ}れて
\ruby{行}{い}かれたアネ。
』

『だつて
\ruby{其}{それ}ぢやあ
\ruby{其}{そ}の
\ruby{建具屋}{たて|ぐ|や}の
\ruby{倅}{せがれ}が
\ruby{意氣地}{い|く|ぢ}が
\ruby{無}{な}さ
\ruby{{\換字{過}}}{す}ぎるぢやあ
\ruby{無}{ね}えか。
』

『それがお
\ruby{{\換字{前}}}{まへ}、
\ruby{理由}{わ|け}があるからなんさ。
\ruby{其}{それ}あ
\ruby{其}{そ}の
\ruby{源}{げん}といふのにやあ
\ruby{嫁}{よめ}になる
\ruby{筈}{はず}の
\ruby{娘}{こ}が、
\ruby{親類内}{しん|るゐ|うち}に
\ruby{決定}{き|ま}つて
\ruby{居}{ゐ}たんで、つまり
\ruby{源}{げん}の
\ruby{方}{はう}ぢやあ
\ruby{初手}{しよ|て}から
\ruby{當座}{たう|ざ}の
\ruby{花}{はな}にしたんだネ。
だから
\ruby{彼}{あ}の
\ruby{娘}{こ}に
\ruby{捕}{つか}まへられて
\ruby{煮}{に}え
\ruby{詰}{つま}つた
\ruby{話}{はなし}をされる
\ruby{段}{だん}になりやあ、いつでも
\ruby{間}{ま}に
\ruby{合}{あは}せを
\ruby{云}{い}つて
\ruby{巧}{うま}く
\ruby{逃}{に}げて、とう〳〵
\ruby{逃}{に}げて〳〵
\ruby{惡}{わる}くも
\ruby{思}{おも}はれずに
\ruby{逃}{に}げおはせたんだよ。
』

『ヤ、そりやあ
\ruby{源}{げん}といふ
\ruby{奴}{やつ}あ
\ruby{酷}{むご}かつたナ、
お
\ruby{龍}{りゆう}こそ
\ruby{眞實}{ほん|と}に
\ruby{憫然}{かはい|さう}だ。
』

『ひどく
\ruby{御察}{お|さつ}しがいゝネ、
\ruby{何樣}{ど|う}かして
お
\ruby{{\換字{遣}}}{や}りナ。
』

『すぐと
\ruby{左樣}{さ|う}
\ruby{皮肉}{ひ|にく}を
\ruby{云}{い}はずともだ。
ウン、それから。
』

『そこで
\ruby{生木}{なま|き}を
\ruby{引裂}{ひき|さ}かれて
\ruby{駿府}{すん|ぷ}へ
\ruby{{\換字{連}}}{つ}れて
\ruby{行}{い}かれたんだから、
お
\ruby{龍}{りゆう}は
\ruby{矢}{や}も
\ruby{楯}{たて}も
\ruby{堪}{たま}りや
\ruby{仕}{し}ない、
\ruby{雨}{あめ}の
\ruby{降}{ふ}るやうに
\ruby{手紙}{て|がみ}を
\ruby{{\換字{遣}}}{よこ}したのさ。
ところが
\ruby{源}{げん}の
\ruby{方}{はう}が
\ruby{其心}{そ|れ}なんだから
\ruby{{\換字{返}}事}{へん|じ}も
\ruby{{\換字{遣}}}{や}らない。
\ruby{斷念}{あき|らめ}させやうといふんで
\ruby{關}{かま}はずに
\ruby{置}{お}くから、
お
\ruby{龍}{りゆう}は
\ruby{餘程恨}{よつ|ほど|うら}んだらしい。
それでも
\ruby{此方}{こつ|ち}ぢやあ
\ruby{關}{かま}はずに
\ruby{置}{お}くと、
\ruby{流石}{さす|が}は
\ruby{明治}{めい|じ}ツ
\ruby{子}{こ}だから
\ruby{氣}{き}が
\ruby{{\換字{強}}}{つよ}いネ、
\ruby{源}{げん}の
\ruby{家}{うち}へ
\ruby{押}{お}しかけやうつて
\ruby{云}{い}つて
\ruby{來}{き}たんだよ。
さあ、
\ruby{來}{こ}られちやあ
\ruby{大事}{おほ|ごと}だから
\ruby{源}{げん}は
\ruby{{\換字{弱}}}{よわ}つて、
\ruby{一{\換字{丈}}}{いち|ぢやう}もある
\ruby{手紙}{て|がみ}を
\ruby{三日}{みつ|か}もかゝつて
\ruby{書}{か}いて、
\ruby{親々}{おや|〳〵}の
\ruby{壓制}{おし|つけ}で
\ruby{仕方}{し|かた}が
\ruby{無}{な}くつて、
お
\ruby{{\換字{前}}}{まへ}にやあ
\ruby{濟}{す}まないが
\ruby{實}{じつ}は
\ruby{既}{もう}
\ruby{女{\換字{房}}}{によう|ばう}を
\ruby{貰}{もら}つた。
\ruby{腹}{はら}も
\ruby{立}{た}つだらうが
\ruby{何樣}{ど|う}か
\ruby{堪忍}{か|に}して
\ruby{吳}{く}れ、
\ruby{二人}{ふ|たり}の
\ruby{中}{なか}は
\ruby{無}{な}い
\ruby{緣}{えん}と
\ruby{諦}{あきら}めて、
\ruby{汝}{おまへ}も
\ruby{叔母}{お|ば}さん
\ruby{次第}{し|だい}に
\ruby{好}{い}い
\ruby{婿}{むこ}を
\ruby{取}{と}つて
\ruby{榮}{さか}えてくれろ、と
\ruby{哀}{あは}れつぽく
\ruby{巧}{うま}く
\ruby{虛言}{う|そ}をついたネ。
』

『やれ〳〵!。
いよ〳〵
\ruby{酷}{むご}いナア、
\ruby{惡}{わる}い
\ruby{奴}{やつ}だ。
』

『するとお
\ruby{{\換字{前}}}{まへ}、よく〳〵だつたと
\ruby{見}{み}えて、
\ruby{怖}{こは}い
\ruby{話}{はなし}さ!、
\ruby{忘}{わす}れもしない
\ruby{去年}{きよ|ねん}の
\ruby{一月}{いち|ぐわつ}の
\ruby{十三日}{じう|さん|にち}、
\ruby{{\換字{寒}}}{かん}の
\ruby{眞中}{さ|なか}の
\ruby{{\換字{雪}}}{ゆき}のふるのに、
\ruby{安倍川}{あ|べ|かは}とかいふ
\ruby{大}{おほき}な
\ruby{川}{かは}へ
\ruby{飛}{と}び
\ruby{{\換字{込}}}{こ}まうとしたさうさ。
\ruby{幸福}{しあ|はせ}に
\ruby{助}{たす}けられたから
\ruby{可}{い}いやうなものゝ、
\ruby{死}{し}なれりやあ
\ruby{差}{さ}し
\ruby{詰}{づ}め
\ruby{源}{げん}は
\ruby{取}{と}り
\ruby{憑}{つ}かれ
\ruby{無}{な}くちやあならないんだつたのさ。
』

『フム、それから。
』

『まあお
\ruby{待}{ま}ち。
さぞ
\ruby{湯}{ゆ}の
\ruby{中}{なか}で
\ruby{噴嚏}{くし|やみ}を
\ruby{仕}{し}て
\ruby{居}{ゐ}るだらう、
\ruby{憫然}{かはい|さう}に。
ハヽヽ。

\ruby{話}{はな}しながら
\ruby{飮}{や}るんで
\ruby{大層}{たい|そう}
\ruby{發}{はつ}したよ。
\ruby{駿府}{すん|ぷ}へ
\ruby{行}{い}つたのが
\ruby{一昨年}{を|とと|し}の
\ruby{夏}{なつ}の
\ruby{末}{すゑ}で、
\ruby{飛}{と}び
\ruby{{\換字{込}}}{こ}んだのが
\ruby{去年}{きよ|ねん}の
\ruby{一月}{いち|ぐわつ}だから、
\ruby{其間}{その|あひだ}の
\ruby{彼}{あ}の
\ruby{女}{こ}の
\ruby{事}{こと}を
\ruby{思}{おも}ふと
\ruby{實}{じつ}は
\ruby{憫然}{ふ|びん}さね。
だが
\ruby{驚}{おどろ}いたのは
\ruby{源}{げん}さ。
\ruby{離}{はな}れて
\ruby{居}{ゐ}る
\ruby{土地}{と|ち}だから
\ruby{助}{たす}かつたのか
\ruby{助}{たす}からなかつたも
\ruby{知}{し}りやうは
\ruby{無}{な}いし、とても
\ruby{生}{い}きて
\ruby{居}{ゐ}ても
\ruby{詰}{つま}らないから
\ruby{死}{し}んで
\ruby{仕舞}{し|ま}ふから
\ruby{憫然}{あは|れ}と
\ruby{思}{おも}つて、
\ruby{一片}{いつ|ぺん}の
\ruby{囘向}{ゑ|かう}でも
\ruby{仕}{し}て
\ruby{吳}{く}れろといふ
\ruby{涙}{なみだ}の
\ruby{痕}{あと}の
\ruby{一}{いつ}ぱいにある
\ruby{不氣味}{ぶ|き|み}な
\ruby{手紙}{て|がみ}を
\ruby{受取}{うけ|と}つたのだから、
\ruby{眞靑}{まつ|さを}になつて
\ruby{慄}{ふる}へて
\ruby{仕舞}{し|ま}つて、いよ〳〵
\ruby{死}{し}んで
\ruby{{\換字{終}}}{しま}つたものなら
\ruby{仕方}{し|かた}が
\ruby{無}{な}い、
\ruby{陰}{かげ}ながら
\ruby{法事}{はふ|じ}でも
\ruby{仕}{し}て
\ruby{祟}{たゝ}りの
\ruby{來}{こ}ないやうに
\ruby{仕}{し}やうと、
\ruby{彼地}{あつ|ち}の
\ruby{新聞}{しん|ぶん}を
\ruby{取}{と}つて
\ruby{調}{しら}べて
\ruby{見}{み}ると、
\ruby{丁度}{ちやう|ど}
\ruby{其}{そ}の
\ruby{手紙}{て|がみ}の
\ruby{日付}{ひ|づけ}の
\ruby{{\換字{翌}}日}{よく|じつ}の
\ruby{新聞}{しん|ぶん}に、
\ruby{美人}{び|じん}の
\ruby{投身}{みな|げ}といふ
\ruby{標題}{み|だし}があつて、
\ruby{彼}{あ}の
\ruby{名}{な}が
\ruby{見}{み}えたから
\ruby{捼然}{ぎよ|つ}としたが、
\ruby{助}{たす}かつて
\ruby{叔母}{を|ば}の
\ruby{家}{うち}へ
\ruby{引渡}{ひき|わた}された、
\ruby{仔細}{し|さい}は
\ruby{解}{わか}らないが
\ruby{發狂}{はつ|きやう}した
\ruby{{\換字{所}}爲}{せ|ゐ}だらう、と
\ruby{書}{か}いてあつたのでホツと
\ruby{氣息}{い|き}を
\ruby{吐}{つ}いたネ。
』

『ン、そこで
\ruby{源}{げん}といふ
\ruby{奴}{やつ}は
\ruby{何樣}{ど|う}したエ。
』

\makeatletter
\@ifundefined{全三巻@一括ビルド}{%
\vspace{4zw}
\Large{天うつ浪 {\normalsize 第一{\換字{終}}}}
}
\makeatother


\end{indentation}
\end{document}
