% +++
% sequence = ["cluttex"]
% [programs.cluttex]
% command  = "cluttex"
% opts     = [ "--engine = uplatex", "--shell-escape", "--output-directory = myout" ]
% +++

% https://okumuralab.org/tex/mod/forum/discuss.php?d=3551 を参考に下記
% を上記のように変更
%#! cluttex --engine=uplatex --shell-escape --output-directory=myout
%   cluttex でビルド時に必要なディレクトリ myout 、 mygwi と myout/mygwi
%   また --includeonly=NAMEs を指定すると '\includeonly{NAMEs}' を仮挿入してくれる
%%
\RequirePackage{plautopatch}
%\RequirePackage{exppl2e}% 警告メッセージ削減のためコメントアウト
\documentclass[
uplatex                                     ,% upLaTeX文書
dvipdfmx                                    ,%
book                                        ,%
tate                                        ,%
twoside                                     ,% even_running_head 有効化
paper                       = a5paper       ,%
open_bracket_pos            = nibu_tentsuki ,% 組み方 段落開始二分折り返し行頭天付き
hanging_punctuation                         ,% 組み方 ぶら下げ組
openany                                     ,%
jafontsize                  = 12pt          ,% 13pt 指定すると LaTeX Font Warning が表示される
%%%%%%%%%%%%%%%%%%%%%%%%%%%%%%%%%%%%%%%%%%%%%% 分冊版では以下は default 設定
% head_space                = 36mm          ,% 天の空き量
% gutter                    = 18mm          ,% のどの余白の大きさ
% headfoot_verticalposition = 24mm          ,%
% line_length               = 27zw          ,% 原本と比較するとき(自動で 30zw 位)
% number_of_lines           = 11            ,% 原本と比較するとき(自動で 15行)
]{jlreq}

\usepackage{bxpapersize}
\usepackage{pxrubrica}
\usepackage{sfkanbun}
\usepackage{warichu}
\usepackage[deluxe,multi,jis2004]{otf}
\usepackage[directunicode*,noalphabet]{pxchfon}[2017/04/08]
\usepackage{plext}
\usepackage{graphicx}
\usepackage[dir=mygwi,cache=108000]{bxglyphwiki}
\usepackage{indent}
\usepackage{tabularray}
\usepackage{CJK-char-convert}
\rubysetup{||h>}% 無指定時のルビ:(||)前進入禁止、(h)肩付き、(>)後進入大
% \rubysetup{|h>}% 無指定時のルビ:(|)前進入禁止、(h)肩付き、(>)後進入大
\title{\Huge 天うつ浪 {\Large 第二}}
\author{幸田露伴}
\date{         {\small 明治三十九年六月} 春陽{\換字{堂}}}
\makeatletter
% \def\全三巻@一括ビルド{}
% \def\デバッグ@ビルド{}
\@ifundefined{デバッグ@ビルド}{%
  \newcommand{\原本頁}[1]{}% デバッグ用/本番は「空」
  \newcommand{\改行}{}%%%%%% デバッグ用/本番は「空」
  \newcommand{\会話開始}{}%% デバッグ用/本番は「空」
}{%
  \newcommand{\原本頁}[1]{\marginpar{\hfill p-#1}}%%%%% デバッグ用/本番は「空」
  \setlength{\marginparwidth}{20mm}%% 傍注欄の大きさ%%% デバッグ用/本番はコメントアウト
  \newcommand{\改行}{\par}%%%%%%%%%%%%%%%%%%%%%%%%%%%%% デバッグ用/本番は「空」
  \newcommand{\会話開始}{ }%
  %
  % 背景にグリッドを表示させる
  \plautopatchdisable{eso-pic}% https://okumuralab.org/tex/mod/forum/discuss.php?d=2956
  % \usepackage{xcolor} は eso-pic.sty 内部で呼び出している
  \usepackage[
  colorgrid    = true    ,
  gridBG       = true    ,
  gridunit     = mm      , % mm, in, bp, pt
  gridcolor    = red!25  ,
  subgridcolor = lime!75 ,
  texcoord     = true    ,
  ]{eso-pic}
}

\makeatother

\begin{document}
\maketitle
\pagestyle{myheadings}
\newcommand{\Entry}[1]{
	\section*{#1}
	\markboth{#1}{#1}
	\setcounter{equation}{0}}
\begin{indentation}{4zw}{3zw}
\parindent=0pt
\newpage
\ % 全角空白
\newpage
\makeatletter
\@ifundefined{全三巻@一括ビルド}{%
{\huge
\ruby{天}{そら} う つ % 空白有り
\ruby{浪}{なみ}}  {\normalsize 第二}
\vspace*{3zw}
\Entry{其一}
}
\makeatother

% メモ 校正終了 2024-04-12 2024-05-28 2024-06-27
\原本頁{1-3}%
『
それからと
いふものは
\ruby{當人}{たう|にん}も、
%
\ruby{一旦}{いつ|たん}
\ruby{死}{し}ぬとまで
\ruby{云}{い}つて
\ruby{{\換字{遺}}}{よこ}して
\ruby{置}{お}いて
\ruby[||j>]{死}{しに}
\ruby[||j>]{損}{そこな}つたんだから、
% \ruby{死損}{しに|そこな}つたんだから、
%
\ruby{流石}{さす|が}に
きまりが
\ruby{惡}{わる}いか
\ruby{羞}{はづか}しいかして
\改行% 校正作業の簡略化のため
、
%
\原本頁{1-5}\改行%
パツタリと
\ruby{音沙汰}{おと|さ|た}を
\ruby{聞}{き}かせなかつたが、
%
\ruby{何}{なん}でも
\ruby{後}{あと}で
\ruby{聞}{き}いた
\ruby{談話}{はな|し}の
\ruby{模樣}{も|やう}で
\ruby{考}{かんが}へて
\ruby{見}{み}ると、
%
\ruby[<g>]{一霎時}{しばらく}の
\ruby{間}{あひだ}は
\ruby{茫然}{ぼう|つ}と
\ruby{仕}{し}て
\ruby{居}{ゐ}たんだ\換字{子}。
%
\原本頁{1-8}\改行%
\ruby{其}{その}
\ruby{中}{うち}に
\ruby{段々}{だん|〴〵}
\ruby{氣}{き}が
\ruby{鎭}{しづ}まつて、
%
\ruby{斷念}{あき|らめ}が
\ruby{自然}{ひと|りで}に
ついて
\ruby{來}{く}ると、
%
\ruby{源}{げん}の
\ruby{事}{こと}は
\ruby{捨}{す}てたものと
\ruby{仕}{し}て
\ruby{仕舞}{し|ま}つた
\ruby{樣子}{やう|す}だが、
%
さあ
\ruby{持{\換字{前}}}{もち|まへ}の
\ruby{氣性}{き|しやう}のある
ところへ
\ruby{捨鉢}{すて|ばち}が
\ruby{加勢}{か|せい}したから、
%
\ruby{輪}{わ}をかけて
お
\ruby{狭}{きやん}に
なつたと
\ruby{見}{み}えて、
%
\ruby{叔母}{を|ば}が
\ruby{婿}{むこ}にと% (婿 5a7f) 聟 805f
\ruby{定}{き}めた
\ruby{男}{をこと}を% ルビ調整(原本通り)
\ruby{{\換字{嫌}}}{きら}つて、%
%
{---}{---}%
\ruby{何}{なん}でも
\ruby{其}{そ}の
\ruby{男}{をとこ}が
\ruby{挊}{かせ}い
\原本頁{2-2}\改行%
で
\ruby{金錢}{か|ね}を
\ruby{溜}{た}めるより
ほかにやあ
\ruby{何}{なに}も
\ruby{知}{し}らない、
%
\ruby{人}{ひと}の
\ruby[||j>]{{\換字{情}}}{じやう}
\ruby[||j>]{合}{ あひ}や
% \ruby{{\換字{情}}合}{じやう|あひ}や
\ruby{意氣}{い|き}
\原本頁{2-3}\改行%
\ruby{張}{はり}も
\ruby{{\換字{分}}}{わか}らない
\ruby{牛}{うし}の
やうな
\ruby{男}{をとこ}
だと
いふので
だゝを% 踊り字調整「〻(二の字点、揺すり点)に見えるが(ゝ)」
\ruby{捏}{こね}たの
だから、
%
\原本頁{2-4}\改行%
とう〳〵
\ruby{大{\換字{紛}}紜}{おほ|ごた|〳〵}が
\ruby{持}{も}ち
\ruby{上}{あが}つたのさ。
%
\ruby{叔母}{を|ば}は
\ruby[||j>]{{\換字{古}}}{むかし}
\ruby[||j>]{風}{ ふう}
% \ruby{{\換字{古}}風}{むかし|ふう}
だから
\ruby{嵩}{かさ}に
かゝつて、% 踊り字調整「〻(二の字点、揺すり点)に見えるが(ゝ)」
%
\ruby{妾}{わたし}が
\ruby{可}{い}いと
\ruby{定}{き}めて
\ruby{先方}{むか|ふ}へも
\ruby{話}{はな}したのだから、
%
\ruby{今{\換字{更}}}{いま|さら}
\ruby{變改}{へん|がい}は
\ruby{出來}{で|き}も
\ruby{仕}{し}ないし、
%
また
\ruby{金}{かね}は
あり
\ruby{人物}{じん|ぶつ}は
\ruby{堅}{かた}し、
%
\ruby{婿}{むこ}にして% (婿 5a7f) 聟 805f
\ruby{不足}{ふ|そく}のある
\ruby{男}{をとこ}でも
\ruby{無}{な}いのに、
%
\ruby{厭}{いや}の
\ruby{{\換字{嫌}}}{きら}ひの
といふのは
\ruby{我儘}{わが|まゝ}だと% 踊り字調整「〻(二の字点、揺すり点)に見えるが(ゝ)」
\ruby{叱}{しか}れば、
%
\ruby[<j||]{男}{をとこ}
の
\ruby{方}{はう}からも
\ruby{喧}{やかま}しく
\ruby{逼}{せま}つて
\ruby{來}{く}るので、
%
お
\ruby{龍}{りう}も
\ruby{氣}{き}の
\ruby{{\換字{弱}}}{よわ}い
\ruby{娘}{こ}なら
\ruby{折}{を}れて
\ruby{仕舞}{し|ま}つて、
%
\ruby{其男}{そ|れ}を
\ruby{婿}{むこ}にして% (婿 5a7f) 聟 805f
\ruby{身}{み}を
\ruby{固}{かた}める
ところだつたが、
%
\ruby{毅然}{しつ|かり}
として
\ruby{{\換字{分}}別}{ふん|べつ}が
あると
\ruby{云}{い}ふんでは
\ruby{無}{な}いけれど、
%
\ruby{妙}{めう}に
\ruby{氣}{き}の
\ruby{冴}{さ}えた、
%
\ruby{萎}{ひ}けて
\ruby{居}{ゐ}ない
\ruby{娘}{こ}だから、
%
たとへ
\ruby{金}{かね}が
\ruby{無}{な}くつて
\ruby{人物}{じん|ぶつ}が
\ruby{堅}{かた}くなくつて
\ruby{一眼}{めつ|かち}で
\ruby{跛足}{びつ|こ}で
\ruby{有}{あ}らうとも、
%
\ruby{其}{そ}の
\ruby{心意氣}{こゝろ|い|き}さへ% 踊り字調整「〻(二の字点、揺すり点)に見えるが(ゝ)」
\ruby{妾}{わたし}の
\ruby{氣}{き}に
\ruby{入}{い}りやあ
\改行% 校正作業の簡略化のため
、
%
\原本頁{3-2}\改行%
\ruby{妾}{わたし}あ
\ruby{亭主}{てい|しゆ}にでも
\ruby{何}{なん}にでも
するが、
%
\ruby{味}{あぢ}の
\ruby{無}{な}い
\ruby[||j>]{石}{いし}
\ruby[||j>]{{\換字{瓦}}}{かはら}のやうな
% \ruby{石{\換字{瓦}}}{いし|かはら}のやうな
\ruby{人}{ひと}に
\ruby{添}{そ}ふ
\ruby{事}{こと}あ
\ruby{出來}{で|き}ません。
%
\ruby{第一}{だい|いち}
\ruby[||j>]{妾}{わたし}あ
\ruby{人}{ひと}の
\ruby{緣合}{{\換字{𛀁}}ん|あひ}の
\ruby{談}{はなし}に、
%
\ruby{目上}{め|うへ}の
\ruby{者}{もの}が
\ruby{壓制}{おし|つけ}
\原本頁{3-4}\改行%
わざを
\ruby{仕}{し}やうとするのは
\ruby{蟲}{むし}が
\ruby{{\換字{嫌}}}{きら}つて
なりません。
%
\ruby{大}{おほ}きな
\ruby{御世話}{お|せ|わ}です。
%
\ruby{要}{い}らない
\ruby{事}{こと}です。
%
\ruby{妾}{わたし}あ
もう
\ruby{人}{ひと}の
\ruby[||j>]{内}{おかみ}
\ruby[||j>]{君}{ さん}
% \ruby{内君}{おかみ|さん}
なんぞに
なれなくつたつて
\ruby{構}{かま}はない
\ruby{身體}{から|だ}です。
%
\ruby{好}{す}きな
\ruby{人}{ひと}になら
\ruby{妾}{めかけ}にでも
\ruby{{\換字{情}}{\換字{婦}}}{い|ろ}にでも
\原本頁{3-7}\改行%
なつて
\ruby{與}{や}る
\ruby{代}{かは}り、
%
\ruby{{\換字{嫌}}}{いや}な
\ruby{人}{ひと}になら
\ruby{奧樣}{おく|さん}になれ
\ruby[||j>]{御}{み}
\ruby[||j>]{臺}{だい}
\ruby[||j>]{{\換字{所}}}{どころ}
になれ
つて
\ruby{云}{い}はれたつて
\ruby{{\換字{嫌}}}{いや}な
\ruby{事}{こと}です。
%
と
\ruby{恐}{おそ}ろしい
\ruby{亂暴}{らん|ばう}を
\ruby{云}{い}つて
\ruby{叔母}{を|ば}と
\ruby{舌戰}{やり|あ}つたさうだよ。
』

\原本頁{3-10}%
『
なる
\ruby{程}{ほど}なア!。
%
\ruby{考}{かんが}へて
\ruby{見}{み}りやあ
\ruby{愍然}{あは|れ}な
ところが
\ruby{心底}{しん|そこ}には
あるぜ!。
%
しかし
\ruby{餘程}{よつ|ぽど}
\ruby{異樣}{お|つ}な
\ruby{出來}{で|き}の
\ruby{娘}{こ}だナ!。
』

\原本頁{4-1}%
『
マア
\ruby{左樣}{さ|う}さ。
%
\ruby{自棄}{や|け}から
\ruby{出}{で}て
\ruby{居}{ゐ}る
\ruby{料簡}{れう|けん}なんだから、
\ruby{云}{い}つて
\ruby{見}{み}りやあ
\ruby[||j>]{愍}{かあ}
\ruby[||j>]{然}{いさう}な% 「愍然 か(あ)いさう」
% \ruby{愍然}{かあ|いさう}な% 「愍然 か(あ)いさう」
ところもあるのさ。
%
それでも
\ruby{先方}{むか|ふ}の
\ruby{男}{をとこ}が
\ruby{氣}{き}が
\ruby{{\換字{廻}}}{まは}ら
\ruby{無}{な}
\原本頁{4-3}\改行%
くつて、
%
\ruby{話}{はなし}が
\ruby{中々}{なか|〳〵}
\ruby{壞}{こは}れないので、
%
そこで
\ruby{彼}{あ}の
\ruby{娘}{こ}は
\ruby[||j>]{癇}{かん}
\ruby[||j>]{癪}{しやく}を
% \ruby{癇癪}{かん|しやく}を
\ruby{起}{おこ}して
\改行% 校正作業の簡略化のため
、
%
\原本頁{4-4}\改行%
\ruby{今年}{こ|とし}の
\ruby[||j>]{三}{さん}
\ruby[||j>]{月}{ぐわつ}に
% \ruby{三月}{さん|ぐわつ}に
\ruby{駿府}{すん|ぷ}を
\ruby{脫}{ぬ}け
\ruby{出}{だ}し、
%
\ruby{何}{なに}といふ
\ruby{目的}{め|あて}が
あつたのじやあ
\原本頁{4-5}\改行%
\ruby{無}{な}いが、
%
\ruby{何}{なに}を
するにしても
\ruby{自{\換字{分}}}{じ|ぶん}の
\ruby{{\換字{勝}}手}{かつ|て}に
\ruby{世}{よ}を
\ruby{{\換字{送}}}{おく}らう
といふんで
\改行% 校正作業の簡略化のため
、
%
\原本頁{4-6}\改行%
ふらりと
\ruby[||j>]{東}{とう}
\ruby[||j>]{京}{きやう}へ
% \ruby{東京}{とう|きやう}へ
\ruby{{\換字{遣}}}{や}つて
\ruby{來}{き}たのさ。
%
すると
\ruby{銀座}{ぎん|ざ}の
\ruby{往來}{わう|らい}で
もつて、
%
ひよつくりと
\ruby{源}{げん}に
\ruby{會}{あ}つたゞらうぢや% TODO 原本の「二の字点、揺すり点」に濁点のグリフが見つからないので「ゞ」
\ruby{無}{な}いか!。
』

\原本頁{4-8}%
『
ヤ、
%
そりやあ
\ruby{大變}{だい|へん}だ!。% 原本でも濁音の「だいへん」
%
おもしろい、
%
\ruby{面白}{おも|しろ}い!。
%
さうして、
』

\原本頁{4-9}%
『
\ruby{源}{げん}は
\ruby{只}{たゞ}% TODO 原本の「二の字点、揺すり点」に濁点のグリフが見つからないので「ゞ」
\ruby{無暗}{む|やみ}に
\ruby{雜沓}{ひと|ごみ}へ
\ruby{入}{はい}つて
\ruby{{\換字{逃}}}{に}げて
\ruby{仕舞}{し|ま}つたんだが、
%
\ruby{其}{そ}の
\ruby{時}{とき}の
\ruby{樣子}{やう|す}を
\ruby{見}{み}て
\ruby{可怪}{を|かしい}と% ルビ調整(原本通り)(をかしい)
\ruby{勘付}{かん|づ}いたから、
%
さあ
\ruby{彼娘}{あ|れ}は
\ruby{竊}{そつ}と
\ruby{源}{げん}の
\ruby{樣子}{やう|す}を
\ruby{内内}{ない|ない}で% ルビ調整(原本通り)非踊り字表記(行末行頭の境界付近)
\ruby{捜}{さぐ}つたんだね。
%
すると
\ruby{源}{げん}の
\ruby{心中}{は|ら}が
\ruby{眞實}{ほん|とう}に
\ruby{讀}{よ}めたから、
%
どんなにか
\ruby{口惜}{く|やし}がつた
\ruby{事}{こと}だらうさ!。
』

\原本頁{5-2}%
『
ン、
%
\ruby{{\換字{道}}理}{もつ|とも}だ、
%
\ruby{口惜}{く|やし}かつたらうさ!。
』

\原本頁{5-3}%
『
\ruby{甚}{ひど}く
\ruby{力}{ちから}を
\ruby{入}{い}れるね、
%
\ruby{可怪}{をか|し}いよ。% ルビ調整(原本通り)(をかし)
%
それから
お
\ruby{{\換字{前}}}{まへ}
\ruby{何處}{ど|こ}に
\ruby{何樣}{ど|う}して
\ruby{買}{か}つたんだか、
%
\ruby{人}{ひと}を
\ruby{騙}{だま}して
\ruby{取}{と}りでも
\ruby{仕}{し}たんだか、
%
\ruby{袂}{たもと}の
\ruby{中}{なか}に
\ruby{短銃}{ぴす|とる}を
\ruby{秘}{かく}してね。
』

\原本頁{5-6}%
『
ヨーツ!、
%
\ruby{凄}{すご}いナア。
』

\原本頁{5-7}%
『
たしか
\ruby{銀鼠}{ぎん|ねず}だつたと
\ruby{思}{おも}つたが
\ruby{薄}{うす}い
\ruby{色}{いろ}の
\ruby{頭巾}{づ|きん}を
\ruby{深}{ふか}く
\ruby{被}{かぶ}つて、
%
\ruby{四日}{よつ|か}の% ルビ調整(原本通り)
お
\ruby{月樣}{つき|さま}の
\ruby{丁度}{ちやう|ど}
\ruby{出}{で}て
\ruby{居}{ゐ}た
\ruby{暮合}{くれ|あひ}の
\ruby{點燈頃}{ひ|ともし|ごろ}を、
%
\ruby{源}{げん}の
\ruby{家}{うち}の
\ruby[||j>]{横}{よこ}
\ruby[||j>]{丁}{ちやう}の
% \ruby{横丁}{よこ|ちやう}の
\ruby{角}{かど}に
\ruby{立}{た}つて
\ruby{居}{ゐ}たのが
\ruby{此}{こ}の
\ruby{四}{し}
\ruby{月}{ぐわつ}だア{\換字{子}}。
%
ほんとに
\ruby{考}{かんが}へて
\ruby{見}{み}ると
\ruby{怖}{こは}い
\ruby{事}{こと}さ{\換字{子}}。
』

\原本頁{5-11}%
『
ン、ン、
』

\原本頁{6-1}%
『
それを
ちらりと
\ruby{見}{み}た
\ruby{妾}{わたし}あ、
%
\ruby{銀座}{ぎん|ざ}で
\ruby{會}{あ}つた
といふ
\ruby{話}{はなし}も
\ruby{源}{げん}から
\ruby{聞}{き}いてたから、
%
こりやあと
\ruby{氣取}{け|ど}つて
\ruby{仕舞}{し|ま}つて
\ruby{話}{はなし}を
\ruby{仕掛}{し|か}けて、
%
\ruby{其}{その}
\ruby{晩}{ばん}
\原本頁{6-3}\改行%
は
\ruby{吾家}{う|ち}へ
\ruby{寢}{ね}させて
\ruby{置}{お}いて、
%
\ruby{源夫{\換字{婦}}}{げん|ふう|ふ}に
\ruby{内{\換字{通}}}{ない|つう}を
\ruby{仕}{し}て
\ruby{{\換字{遣}}}{や}つたんで、
%
\ruby{何事}{なに|ごと}も
\ruby{起}{おこ}らずに
\ruby{濟}{す}んで
\ruby{仕舞}{し|ま}つたんだよ。
』

\原本頁{6-5}%
『
アヽ
\ruby{惜}{をし}い
\ruby{事}{こと}を
\ruby{仕}{し}た。
%
ドンと
\ruby{{\換字{遣}}}{や}らせりやあ
\ruby{好}{よ}かつたのに!。
』

\原本頁{6-6}%
『
\ruby[||j>]{戱}{じやう}
\ruby[||j>]{談}{ だん}ぢや
% \ruby{戱談}{じやう|だん}ぢや
\ruby{無}{な}いよ、
%
\ruby{下}{くだ}らない!。
%
\ruby{源夫{\換字{婦}}}{げん|ふう|ふ}は
\ruby{怖}{こは}がつて〳〵{\換字{子}}、
%
\ruby{弟子}{で|し}のあるのを
\ruby{幸}{さいは}ひに
\ruby{仙臺}{せん|だい}へ
\ruby{竊}{そつ}と
\ruby{挊}{かせ}ぎに
\ruby{行}{い}つて
\ruby{仕舞}{し|ま}つたのさ。
%
そ
\原本頁{6-8}\改行%
れから
いろ〳〵
\ruby{理解}{り|かい}を
\ruby{云}{い}つて
\ruby{聞}{き}かせて、
%
とう〳〵
\ruby{宅}{うち}へ
\ruby{置}{お}くやうに
\ruby{仕}{し}たんだが、
%
\ruby{左樣}{さ|う}いつた
\ruby{氣性}{き|しやう}だから
また
\ruby{男}{をとこ}が
\ruby{好}{す}いて、
%
\ruby{段々}{だん|〴〵}と
\ruby{妾}{わたし}の
ためになり
\ruby{出}{だ}したね。
』

\原本頁{6-11}%
『
そりやあ
\ruby{左樣}{さ|う}
だらう、
%
\ruby{中々}{なか|〳〵}
\ruby{價}{ね}の
\ruby{踏}{ふ}める
\ruby{奇貨}{しろ|もの}だわエ。
』

\原本頁{7-1}%
『
そこで
\ruby{{\換字{前}}{\換字{途}}}{さき|〴〵}の
\ruby{考案}{かん|がへ}も
あるから、
%
すつかり
\ruby{妾}{わたし}のものに
\ruby{仕}{し}て
\ruby{仕舞}{し|ま}はうと
\ruby{思}{おも}ふんだが\換字{子}、
%
まあ
\ruby{第一}{だい|いち}に
チヤンと
\ruby[||j>]{關}{ひつ}
\ruby[||j>]{係}{かゝり}を% 踊り字調整「〻(二の字点、揺すり点)に見えるが(ゝ)」
% \ruby{關係}{ひつ|かゝり}を% 踊り字調整「〻(二の字点、揺すり点)に見えるが(ゝ)」
\ruby{切}{き}つちまは
なけりやあ
ならないのが
\ruby{叔母}{を|ば}の
\ruby{方}{はう}だ\換字{子}。
』

\原本頁{7-4}%
『
ン、
%
そりやあ
\ruby{譯}{わけ}は
\ruby{無}{ね}え、
%
\ruby{乃公}{お|れ}が
\ruby{法}{はふ}をかいて
\ruby{{\換字{遣}}}{や}らあ。
』

\Entry{其二}

『
\ruby{眞實}{ほん|と}に
\ruby{譯無}{わけ|な}く
\ruby{此方}{こつ|ち}のものに
\ruby{出來}{で|き}やうか\換字{子}。
』

『
\ruby{出來}{で|き}るともさ!。
\ruby{併}{しか}し
\ruby{{\換字{戸}}籍}{せ|き}までといふ
\ruby{譯}{わけ}にやあいかねえ。
\ruby{籍}{せき}まで
\ruby{御前}{お|まへ}の
\ruby{娘}{むすめ}にしやうと
\ruby{云}{い}ふにやあ、
お
\ruby{前}{まへ}が
\ruby{岩崎}{いは|さき}の
\ruby{家}{うち}を
\ruby{{\換字{退}}}{の}いて
\ruby{仕舞}{し|ま}つて
\ruby{一本立}{いつ|ぽん|だ}ちになるか、
\ruby{二人}{ふた|り}の
\ruby{子}{こ}を
\ruby{順々}{じゆん|〳〵}に
\ruby{白痴}{ば|か}か
\ruby{瘋癲}{きち|がひ}かに
\ruby{云}{い}ひ
\ruby{立}{た}てゝ、
\ruby{相續權}{さう|ぞく|けん}の
\ruby{無}{な}いやうに
\ruby{仕}{し}て
\ruby{仕舞}{し|ま}はなけりやならねえが、そんな
\ruby{事}{こと}は
\ruby{迚}{とて}も
\ruby{出來}{で|き}ることぢやあ
\ruby{無}{ね}え。
』

『そんな
\ruby{事}{こと}は
\ruby{損}{そん}になるから
\ruby{詰}{つま}らないや\換字{子}。
いくら
\ruby{氣}{き}に
\ruby{入}{い}らない
\ruby{子}{こ}だつて
\ruby{何}{なん}だつて、
\ruby{若}{わか}い
\ruby{者}{もの}あ
\ruby{何樣}{どん|な}に
\ruby{出世}{しゆ|つせ}をするか
\ruby{知}{し}れや
\ruby{仕}{し}ないんだもの!、
\ruby{金錢}{お|あし}が
\ruby{要}{い}らないなら
\ruby{其}{そ}の
\ruby{親}{おや}になつてる
\ruby{方}{はう}が
\ruby{利}{り}に
\ruby{當}{あた}るぢや
\ruby{無}{あた}いか。
』

『ハヽヽ、
\ruby{繼子二人}{まゝ|つこ|ふた|り}の
\ruby{親}{おや}になつてるのを、
\ruby{抽籤前}{く|じ|まへ}の
\ruby{勸業銀行債{\換字{券}}}{くわ|ん|ぎ|ん|さい|けん}でも
\ruby{持}{も}つてるやうに
\ruby{思}{おも}つてるのか。
ハヽヽお
\ruby{前}{まへ}は
\ruby{眞實}{ほん|と}に
\ruby{怜悧}{り|かう}だ、
\ruby{好}{い}い
\ruby{料簡}{れう|けん}だ、
\ruby{感心}{かん|しん}した。
』

『お
\ruby{冷}{ひ}やかしで
\ruby{無}{な}いよ、
\ruby{馬鹿}{ば|か}にしてツ!。
』

『それだつて
\ruby{二人}{ふた|り}も
\ruby{子}{こ}があるのに
\ruby{其上}{その|うへ}に
\ruby{{\換字{又}}}{また}、あの
お
\ruby{龍}{りう}も
\ruby{眞實}{ほん|と}の
\ruby{養女}{よう|ぢよ}に
\ruby{爲}{し}やうなんて、あんまり
\ruby{蟲}{むし}が
\ruby{好}{よ}さ
\ruby{{\換字{過}}}{す}ぎるからナ。
』

『ぢやあ
\ruby{何樣}{ど|う}すりやあ
\ruby{可}{いゝ}といふのかエ。
』

『
\ruby{何樣}{ど|う}するも
\ruby{此樣}{こ|う}するも
\ruby{要}{い}る
\ruby{事}{こと}ぢやあ
\ruby{無}{な}い。
つまり
\ruby{叔母}{を|ば}といふ
\ruby{奴}{やつ}が
\ruby{頭}{あたま}さへ
\ruby{出}{だ}して
\ruby{來}{こ}ないやうにすりやあ
\ruby{好}{い}いのだらう。
』

『
\ruby{左樣}{さ|う}さ!。
\ruby{彼女}{あ|れ}を
\ruby{囮}{をとり}にして
\ruby{穫}{と}つた
\ruby{禽}{とり}を、
\ruby{他}{ひと}の
\ruby{手}{て}へ
\ruby{取}{と}られるやうな
\ruby{事}{こと}さへ
\ruby{無}{な}きやあ
\ruby{畢竟}{つま|り}いゝのさ。
』

『だから
\ruby{譯}{わけ}は
\ruby{無}{ね}えといふのだ、
\ruby{矢筈}{や|はず}にかけるんだナ!。
』

『
\ruby{矢筈}{や|はず}にかけるつて、
\ruby{何樣}{ど|う}いふやうに?。
』

『
\ruby{叔母}{を|ば}のところへポーンと
\ruby{一本手紙}{いつ|ぽん|て|がみ}を
\ruby{{\換字{遣}}}{や}つて、
\ruby{斯樣}{か|う}いふことを
\ruby{云}{い}つて
\ruby{{\換字{遣}}}{や}るのだ。
\ruby{妾}{わたし}は
\ruby{師匠}{し|しやう}と
\ruby{弟子}{で|し}との
\ruby{緣}{ゑん}で、
\ruby{其方}{そつ|ち}の
お
\ruby{龍}{りう}さんを
\ruby{何月}{なん|ぐわつ}
\ruby{以來}{この|かた}
\ruby{食客}{かゝ|りうど}に
\ruby{仕}{し}てゐます。
\ruby{聞}{き}けば
お
\ruby{龍}{りう}さんは
\ruby{複雑}{いり|く}んだ
\ruby{譯}{わけ}で、
\ruby{其方}{そつ|ち}を
\ruby{無言}{む|ごん}で
\ruby{出}{で}て
\ruby{來}{き}たのださうだが、
\ruby{一季{\換字{半}}季}{いつ|き|はん|き}の
\ruby{奉公人}{はう|こう|にん}でも、
\ruby{定}{き}めるところは
\ruby[g]{確然}{しやん}と
\ruby{定}{き}める
\ruby{{\換字{習}}}{ならひ}いだから、
\ruby{何}{なん}の
\ruby{定}{きまり}も
\ruby{無}{な}しに
\ruby{無際限}{む|さい|げん}に
\ruby{置}{お}く
\ruby{譯}{わけ}にはいか
\ruby{無}{な}い。
\ruby{當人}{たう|にん}の
\ruby{料簡}{れう|けん}ぢやあ
\ruby{其方}{そつ|ち}へは
\ruby{歸}{かへ}りたく
\ruby{無}{な}い、
\ruby{此地}{こち|ら}で
\ruby{藝}{げい}の
\ruby[g]{師匠}{ししやう}でも
\ruby{仕}{し}て
\ruby{暮}{くら}したいと
\ruby{云}{い}ふ
\ruby{事}{こと}だし、
\ruby{妾}{わたし}の
\ruby{目}{め}で
\ruby{見}{み}ても
\ruby{當人}{たう|にん}の
\ruby{藝}{げい}の
\ruby{性質}{た|ち}に
\ruby{見込}{み|こみ}があるから、
\ruby{若}{も}し
\ruby{全}{まつた}く
お
\ruby{龍}{りう}さんを
\ruby{妾}{わたし}の
\ruby{娘{\換字{分}}}{むすめ|ぶん}にして、
\ruby{妾}{わたし}の
\ruby{跡}{あと}を
\ruby{襲}{つ}がせても
\ruby{宣}{い}いと
\ruby{云}{い}ふのなら、
\ruby{今}{いま}までも
\ruby{世話}{せ|わ}を
\ruby{仕}{し}たが
\ruby{{\換字{猶}}}{なほ}
\ruby{此上}{この|うへ}とも、
\ruby{立派}{りつ|ぱ}に
\ruby{藝}{げい}の
\ruby{成就}{でき|あが}るまでは
\ruby{何年}{なん|ねん}でも
\ruby{世話}{せ|わ}を
\ruby{爲}{し}やうが、そちらが
\ruby{左樣}{さ|う}いふ
\ruby{氣}{き}で
\ruby{無}{な}ければ
\ruby{此地}{こち|ら}でも
\ruby{困}{こま}る。
たゞべん〳〵とは
\ruby{世話}{せ|わ}も
\ruby{出來}{で|き}ぬから、
\ruby{今}{いま}
\ruby{迄世話}{まで|せ|わ}を
\ruby{爲}{し}た
\ruby{食雜用}{くひ|ざふ|よう}を
\ruby{入}{い}れて、
\ruby{其方}{そつ|ち}へ
\ruby{引取}{ひき|と}つて
\ruby{貰}{もら}ひたいものだ。
しかし
\ruby{當人}{たう|にん}は
\ruby{何樣}{ど|う}いふものだか、
\ruby{甚}{ひど}く
\ruby{其方}{そち|ら}の
\ruby{事}{こと}を
\ruby{惡}{わる}く
\ruby{云}{い}つて、
\ruby{田舎}{ゐ|なか}へ
\ruby{{\換字{返}}}{かへ}される
\ruby{位}{くらゐ}なら
\ruby{舌}{した}を
\ruby{咬}{か}んで
\ruby{死}{し}ぬなぞと、
\ruby{無茶}{む|ちや}を
\ruby{云}{い}つて
\ruby{居}{ゐ}て
\ruby{眞}{まこと}に
\ruby{困}{こま}ります。
と
\ruby{斯樣}{か|う}いふやうに
\ruby{掛合}{かけ|あ}ふのだ。
\ruby{{\換字{遣}}}{よこ}すなら
\ruby{緣切}{えん|きり}にしろ、
\ruby{{\換字{返}}}{かへ}せなら
\ruby{食雜用}{くひ|ざふ|よう}を
\ruby{入}{い}れろと、
\ruby{金額}{かね|だか}を
\ruby{大袈裟}{おゝ|げ|さ}にしてどうだ〳〵で
\ruby{責}{せ}めるのさ。
さうすりやあ
\ruby{大槪}{たい|がい}
\ruby{姪}{めひ}
\ruby{一人}{ゝと|り}
\ruby{捨}{す}てた
\ruby{氣}{き}にならうぜ。
』

『でも
\ruby{食雜用}{くひ|ざふ|よう}ぢやあ
\ruby{月十圓}{つき|じう|ゑん}にしたつて
\ruby{知}{し}れたもんだから、
\ruby{大袈裟}{おゝ|げ|さ}に
\ruby{爲}{し}やうも
\ruby{無}{な}いぢや
\ruby{無}{な}いか。
』

『
\ruby{智慧}{ち|ゑ}の
\ruby{無}{な}い
\ruby{事}{こと}を
\ruby{云}{い}つたものだ!
\ruby{衣服}{き|もの}や
\ruby{髪{\換字{飾}}}{かみ|かざ}りを
\ruby{少}{すこ}し
\ruby{買}{か}つて
\ruby{{\換字{遣}}}{や}つて
\ruby{置}{お}きやあ、
\ruby{大}{たい}した
\ruby{金額}{かね|だか}に
\ruby{註加}{つけ|かけ}が
\ruby{出來}{で|き}らあナ。
』

『
\ruby{成程}{なる|ほど}ネ。
それでも
\ruby{{\換字{連}}}{つ}れて
\ruby{歸}{かへ}つたらば?。
』

『その
\ruby{時}{とき}はまた
\ruby{後}{あと}で
\ruby{策}{さく}を
\ruby{爲}{す}るとして、
\ruby{食雜用}{くひ|ざふ|よう}と
\ruby{緣切}{えん|きり}とで
\ruby{一寸}{ちよ|つと}
\ruby{{\換字{暖}}}{あたゝ}まつて
\ruby[g]{湯治}{たうじ}とでも
\ruby{洒落}{しや|れ}たが
\ruby{宣}{い}い。
』

『
\ruby{緣切}{\換字{江}ん|きり}とはエ?。
』

『お
\ruby{龍}{りう}が
\ruby{駿府}{すん|ぷ}へ
\ruby{{\換字{連}}}{つ}れて
\ruby{行}{い}かれると
\ruby{定}{きま}つたら、
お
\ruby{前}{まえ}が
\ruby{源}{げん}の
\ruby{親父}{おや|ぢ}へ
\ruby{衝突}{ぶつ|か}つて、
\ruby{此方}{こち|ら}の
\ruby{息子}{むす|こ}さんが
\ruby{惡}{わる}いのだから、
\ruby{空拳}{にぎり|こぶし}では
\ruby{話}{はなし}は
\ruby{濟}{す}みますまい、いくらかの
\ruby{手切金}{て|ぎ|れ}を
\ruby{御與}{お|や}んなすつて、
\ruby{彼}{あ}の
\ruby{娘}{こ}を
\ruby{駿府}{すん|ぷ}へ
\ruby{歸}{かへ}らせた
\ruby{方}{はう}が
\ruby{宣}{よ}うございましやう、
\ruby{左樣爲}{さ|う|し}ないと
\ruby{何時}{い|つ}までも
\ruby{關係}{ひつ|かゝり}があつて、
\ruby{何樣}{ど|ん}な
\ruby{事}{こと}が
\ruby{起}{おこ}こるか
\ruby{知}{し}れませんから、と
\ruby{少}{すこ}し
\ruby{巧}{うま}く
\ruby{口}{くち}をきゝやあ
\ruby{必定取}{きつ|と|と}れらあナ。
\ruby{源}{げん}の
\ruby{家}{うち}ぢやあ
\ruby{怖}{こは}がりきつて
\ruby{居}{ゐ}やうから、
\ruby{出}{だ}さうぢやあ
\ruby{無}{ね}えか。
\ruby{其金}{そ|れ}を
\ruby{此方}{こつ|ち}の
\ruby{懷中}{ふと|ころ}へそつくり
\ruby{入}{い}れて、
お
\ruby{龍}{りう}は
\ruby{叔母}{を|ば}に
\ruby{{\換字{連}}}{つ}れさせて
\ruby{歸}{かへ}しちまふなざあ、まんざら
\ruby{野暮}{や|ぼ}ぢやあ
\ruby{無}{ね}えか。
』

『さうさねえ。
成程
\ruby{野暮}{や|ぼ}ぢやあ
\ruby{無}{ね}えぢやあ
\ruby{無}{ね}えかだ\換字{子}。
ハヽヽ、これだからお
\ruby{前}{めへ}は
\ruby{惡徒}{あく|とう}だつて
\ruby{云}{い}ふんだよ。
』

『
\ruby{笑}{わら}はせやがる!。
\ruby{番毎}{ばん|こ}に
\ruby{惡口}{わる|くち}だ。
』

『ナニ
\ruby{褒}{ほ}めたんだよ。
』

『
\ruby{碌}{ろく}でも
\ruby{無}{ね}え
\ruby{褒}{ほ}めやうだナア、
\ruby{有}{あ}り
\ruby{難}{がた}くも
\ruby{無}{ね}え。
そりやあ
\ruby{其樣}{さ|う}と
お
\ruby{龍}{りう}はもう
\ruby{全}{まつた}く
\ruby{源}{げん}に
\ruby{未練}{み|れん}は
\ruby{無}{ね}えか。
』

『いろ〳〵
\ruby{理解}{り|かい}を
\ruby{云}{い}つて
\ruby{聞}{き}かせたから、
\ruby{今}{いま}ぢや
\ruby{怒}{おこ}つては
\ruby{居}{ゐ}るやうだが、
\ruby{思}{おも}つては
\ruby{居}{ゐ}ない\換字{子}。
』

『
\ruby{先刻}{さつ|き}の
\ruby{言}{くち}の
\ruby{{\換字{通}}}{とほ}り
\ruby{男}{をとこ}にやあ
\ruby{懲}{こ}りてるか?。
』

『ナアニ
\ruby{彼樣}{あ|ゝ}は
\ruby{云}{い}つてるが、
\ruby{今}{いま}ぢやあもう、
\ruby{張}{は}りに
\ruby{來}{く}る
\ruby{若}{わか}い
\ruby{男}{をとこ}たちにちやほや
\ruby{云}{い}はれるのを、
\ruby{可笑}{を|か}しがつて
\ruby{{\換字{遊}}}{あそ}んで
\ruby{居}{ゐ}る
\ruby{位}{くらひ}だもの、そして
\ruby{{\換字{又}}}{また}
\ruby{前々}{まへ|〳〵}からの
\ruby{性{\換字{分}}}{しやう|ぶん}ぢやあ
\ruby{有}{あ}るが、
\ruby{身}{み}だしなみを
\ruby{氣}{き}にして、
\ruby{髪}{かみ}なんぞも
\ruby{髪結}{かみ|ゆひ}に
\ruby{結}{い}はせる
\ruby{時}{とき}の
\ruby{間}{あひだ}にやあ、やれ
\ruby{何}{なん}の、
\ruby{彼}{か}のと、
\ruby{流行}{はや|り}を
\ruby{{\換字{追}}}{お}つて
\ruby{束髪}{そく|はつ}の
\ruby{異}{おつ}なのまで
\ruby{仕}{し}て、
\ruby{男}{をとこ}たちに
\ruby{好}{い
}いとか
\ruby{惡}{わる}いとか
\ruby{可笑}{を|か}しいとか
\ruby{云}{い}はれて、おもしろさうに
\ruby{笑}{わら}つて
\ruby{騒}{さわ}ぐのだもの、
\ruby{一寸}{ちよ|つと}
\ruby{氣}{き}に
\ruby{入}{い}つた
\ruby{男}{をとこ}にでも
\ruby{逢}{あ}つた
\ruby{日}{ひ}にやあ、
\ruby{合點}{が|てん}で
\ruby{一}{ひ}ㇳ
\ruby{苦労}{く|らう}して
\ruby{見}{み}やうと
\ruby{云}{い}つたやうな
\ruby{調子}{てう|し}が
\ruby{見}{み}えるね。
』

『フーン。
』

『だから
\ruby{吾家}{う|ち}へ
\ruby{來}{く}る
\ruby{若藏}{わか|ざう}たちの
\ruby{中}{なか}で、
\ruby{傳}{でん}でも
\ruby{淸}{せい}でも
\ruby{關}{かま}はないが、
\ruby{誰}{たれ}かと
\ruby{出來}{で|き}りやあ
\ruby{宣}{い}いと
\ruby{思}{おも}つてるのサ。
』

『
\ruby{解}{わか}らねえナ、
\ruby{何故}{な|ぜ}?。
』

『
\ruby{何故}{な|ぜ}つて
\ruby{{\換字{情}}夫}{い|ろ}が
\ruby{出來}{で|き}りやあ
\ruby{金錢}{おか|ね}が
\ruby{要}{い}るは\換字{子}、
\ruby{金錢}{おか|ね}が
\ruby{要}{い}りやあ
\ruby{自然}{ひと|りで}に
\ruby{欲}{ほ}しがるは\換字{子}。
\ruby{金錢}{おか|ね}を
\ruby{欲}{ほ}しがらない
\ruby{我儘者}{わが|まゝ|もの}にやあ
\ruby{困}{こま}るけど、
\ruby{金錢}{おか|ね}を
\ruby{欲}{ほ}しがる
\ruby{奴}{やつ}なら
\ruby{何樣}{ど|ん}な
\ruby{事}{こと}でも
\ruby{爲}{さ}せられるから\換字{子}!。
』

『
\ruby{{\換字{違}}無}{ちげ|へね}え!。
\ruby{其樣}{そ|ん}な
\ruby{急處}{きふ|しよ}を
\ruby{捕}{つかめ}へやうと
\ruby{思}{おも}つて
\ruby{待構}{まち|かま}えへて
\ruby{居}{ゐ}るのかエ?。
オヽ
\ruby{怖}{おつかな}い!。
\ruby{何}{なん}の
\ruby{事}{こと}は
\ruby{無}{ね}え、
\ruby{他}{ひと}の
\ruby{色戀}{いろ|こひ}は
\ruby{汝}{おめへ}の
\ruby{餌食}{ゑ|じき}だナー。
』

『ハヽヽ、
\ruby{云}{い}つて
\ruby{見}{み}りやあ
\ruby{其樣}{そ|ん}なものだ\換字{子}。
\ruby{一體流行}{いつ|たい|はや|り}も
\ruby{仕無}{し|な}い
\ruby{三絃}{べん|〳〵}の
\ruby{御師匠}{お|し|よ}さんで、
\ruby{澄}{す}まして
\ruby{{\換字{遣}}}{や}つて
\ruby{行}{ゆ}かれるのは、
\ruby{餌食}{ゑ|じき}になる
\ruby{奴}{やつ}がザラに
\ruby{有}{あ}るからだアネ。
つまり
\ruby{男}{をとこ}さへ
\ruby{見}{み}りやあべろつく
\ruby{娘}{むすめ}や、
\ruby{女}{をんな}さへ
\ruby{見}{み}りやあでれつく
\ruby{男}{をとこ}が、
\ruby{世}{よ}の
\ruby{中}{なか}に
\ruby{澤山有}{たく|さん|あ}る
\ruby{中}{うち}あ、
\ruby{下}{くだ}らない
\ruby{小{\換字{説}}}{こ|ほん}でも
\ruby{御客樣}{おき|やく|さま}は
\ruby{絶}{た}えないし、
\ruby{彈}{ひ}けも
\ruby{仕}{し}ない
お
\ruby{師匠樣}{し|よ|さん}でも
\ruby{斯樣}{か|う}して
\ruby{御酒}{お|さけ}も
\ruby{飮}{の}めるんだから、フン
\ruby{有}{あ}り
\ruby{難}{がた}く
\ruby{出來}{で|き}てる
\ruby{世界}{せ|かい}さネ。
アヽお
\ruby{龍}{りう}もゝう
\ruby{歸}{かへ}つても
\ruby{來}{く}るだらうし、
\ruby{物}{もの}も
\ruby{持}{も}つて
\ruby{來}{き}て
\ruby{{\換字{呉}}}{く}れりやあ
\ruby{水}{みづ}も
\ruby{汲}{く}んで
\ruby{{\換字{呉}}}{く}れるといふ
\ruby{重寶}{ちよう|はう}な
\ruby{人{\換字{達}}}{ひと|たち}もそろ〳〵
\ruby{來}{く}る
\ruby{時{\換字{分}}}{じ|ぶん}だ。
お
\ruby{前}{まへ}
\ruby{一}{ひ}ㇳ
\ruby{足先}{あし|さき}へまた
\ruby{寄席}{よ|せ}へ
お
\ruby{出}{いで}でナ。
\ruby{妾}{わたし}も
お
\ruby{龍}{りう}を
\ruby{置}{おい}て
\ruby{後}{あと}から
\ruby{出掛}{で|かけ}るよ。
\ruby{左樣}{さ|う}するとまた
\ruby{其場}{その|ば}に
\ruby{居合}{ゐ|あは}せた
\ruby{若}{わか}い
\ruby{奴}{やつ}に
\ruby{有}{あ}り
\ruby{難}{がた}がられるのだからをかしい!。
』

\ruby{住處}{すみ|か}も
\ruby{業體}{げふ|てい}も
\ruby{明}{あき}らかならぬ
\ruby{男}{をとこ}は
\ruby{點頭}{うな|づ}きて
\ruby{去}{さ}り、
\ruby{引{\換字{違}}}{ひき|ちが}へて
お
\ruby{龍}{りう}は
\ruby{歸}{かへ}り
\ruby{來}{きた}りぬ。

もとより
\ruby{色白}{いろ|じろ}の、
\ruby{特}{こと}に
\ruby{浴上}{ゆ|あが}りなれば、
\ruby{少}{すこ}し
\ruby{上氣}{じよ|うき}して
\ruby[g]{紅潮}{くれなひさ}したる
\ruby{面}{おもて}の
\ruby{一}{ひ}ㇳしほ
\ruby{麗}{うるは}しく、
\ruby[g]{嫣然}{にこり}と
\ruby{笑}{ゑ}める
\ruby{頬}{ほゝ}に
\ruby{笑靨少}{ゑ|くぼ|すこ}しよりて、これが
\ruby{短銃}{ぴす|とる}を
\ruby{袂}{たもと}にして
\ruby{{\換字{情}}無}{つれ|な}き
\ruby{男}{をとこ}を
\ruby{撃}{う}たんとしたる
\ruby{恐}{おそ}ろしき
\ruby{女}{をんな}とは
\ruby{更}{さら}に
\ruby{見}{み}えず、たゞこれ
\ruby{垂絲櫻}{し|だれ|ざくら}の
\ruby{艶}{ゑん}に
\ruby{{\換字{咲}}}{さ}きほこつて、
\ruby{吹}{ふ}けよ
\ruby{春風}{はる|かぜ}、
\ruby{吹}{ふ}かば
\ruby{狂}{くる}はん、
\ruby{降}{ふ}れよ
\ruby{春風}{はる|かぜ}、
\ruby{降}{ふ}らば
\ruby{濡}{ぬ}れんと、
\ruby{春}{はる}は
\ruby{十{\換字{分}}}{じう|ぶん}の
\ruby{花}{はな}の
\ruby{色香}{いろ|か}に、
\ruby{溢}{こぼ}るゝばかりの
\ruby{優}{やさ}しき
\ruby{{\換字{情}}}{なさけ}の
\ruby{{\換字{浮}}}{うか}めるを
\ruby{見}{み}るが
\ruby{如}{ごと}し。

\ruby{夜}{よ}は
\ruby{男弟子}{をとこ|で|し}の
\ruby{世界}{せ|かい}なり。
やがて
\ruby{淸}{せい}といへるが
\ruby{入}{い}り
\ruby{來}{きた}れる
\ruby{時}{とき}、
\ruby{女主人}{あ|る|じ}は
\ruby{稽古}{けい|こ}を
お
\ruby{龍}{りう}に
\ruby{托}{たく}して、
\ruby{用事}{よう|じ}ありと
\ruby{云}{い}ひて
\ruby{寄席}{よ|せ}に
\ruby{去}{さ}りしが、それより
\ruby{傳}{でん}も
\ruby{來}{きた}り
\ruby{{\換字{勝}}}{かつ}も
\ruby{來}{きた}り、
\ruby{誰}{たれ}も
\ruby{彼}{かれ}も
\ruby{來}{きた}りて、
\ruby{皆}{みな}
お
\ruby{龍}{りう}が
\ruby{機{\換字{嫌}}}{き|げん}とり〴〵に、
\ruby{富士}{ふ|じ}の
\ruby{白}{しろ}く
\ruby{優}{やさ}しきを
\ruby{取巻}{とり|ま}く
\ruby{夏}{なつ}の
\ruby{山々}{やま|〳〵}と、いかつき
\ruby{身體}{から|だ}の
\ruby{背}{せ}をくゞめ
\ruby{頭}{かうべ}を
\ruby{低}{ひく}くしてしほらしくしたるもをかし。


\Entry{其三}

% メモ 校正終了 2024-04-13 2024-05-28 2024-06-28
\原本頁{17-6}%
\ruby{春}{はる}
\ruby{闌}{た}けたる
\ruby[g]{上野}{うへの }の
\ruby{夜}{よ}は
\ruby{深}{ふか}く
\ruby{人}{ひと}は
\ruby{稀}{まれ}にして、
%
\ruby{白}{しろ}き
\ruby[g]{綿雲}{わたぐも}の
\ruby{地}{ち}に
\ruby{宿}{やど}れるが
\ruby{如}{ごと}く
\ruby[g]{爛{\換字{熳}}}{らんまん}と
\ruby{{\換字{咲}}}{さ}き
\ruby{亂}{みだ}れたる
\ruby{櫻}{さくら}の
\ruby{{\換字{梢}}}{こずゑ}に、
%
おぼろ
\ruby{月}{づき}の
\ruby[<j||]{光}{ひかり}
\ruby[||j>]{薄}{うつす}りと
\ruby{照}{て}らして、
%
\makeatletter
\@ifundefined{デバッグ@ビルド}{%
  \ruby[g]{一塲の}{いちぢやう }% 原文通り「塲」
}{%
  \ruby[||j>]{一}{いち}
  \ruby[||j>]{塲}{ぢやう}の% 原文通り「塲」
}%
\makeatother
% \ruby{一塲}{いち|ぢやう}の% 原文通り「塲」
\ruby[g]{景色}{け しき}は
\ruby{夢}{ゆめ}の
やうに
\ruby{淡}{あは}し。

\原本頁{17-9}%
『
あら
\ruby{源}{げん}さん、
%
\ruby{酷}{ひど}いよ、
%
\ruby[g]{御待}{お ま }ちつてば
\ruby[g]{御待}{お ま }ちつて
\ruby{云}{い}ふのに!。
』

\原本頁{17-10}%
\ruby[||j>]{男}{をとこ}は
\ruby{妾}{わ}が
\ruby[g]{言葉}{ことば }を
\ruby{耳}{みゝ}にも% 踊り字調整「〻(二の字点、揺すり点)に見えるが(ゝ)」
\ruby{入}{い}れず。
%
\ruby[g]{振{\換字{返}}}{ふりかへ}りも
せずして
\ruby{唯}{たゞ}% 踊り字調整「〻(二の字点、揺すり点)に濁点に見えるが(ゞ)」
\ruby{走}{はし}りに
\ruby{走}{はし}り
\原本頁{18-1}\改行%
\ruby{去}{さ}る
\ruby[g]{{\換字{情}}無}{つれな }さ
\ruby{味氣無}{あぢ|き|な}さ。
%
\ruby{其}{そ}の
\ruby[<j||]{後}{うしろ}
\ruby[||j>]{姿}{すがた}は
\ruby[g]{幾本}{いくもと}の
\ruby{櫻}{さくら}の
\ruby{幹}{みき}より
\ruby{隱}{かく}れつ
\ruby{見}{あら}はれつ
して、
%
\ruby{見}{み}る〳〵
\ruby{{\換字{遠}}}{とほ}く
\ruby{花}{はな}の
\ruby{蔭}{かげ}の
\ruby[g]{糢糊}{ぼ つ }と
\ruby{白}{しろ}きが
\ruby{中}{なか}に
\ruby{{\換字{消}}}{き}え
\ruby{行}{ゆ}かんと
すれば、
%
\ruby{心}{こゝろ}も% 踊り字調整「〻(二の字点、揺すり点)に見えるが(ゝ)」
\ruby{{\換字{更}}}{さら}に
\ruby{心}{こゝろ}ならず、% 踊り字調整「〻(二の字点、揺すり点)に見えるが(ゝ)」
%
\ruby[<g>]{御召縮緬}{おめし}の
\ruby[g]{着物}{き もの}の
\ruby[g]{生憎}{あひにく}に
\ruby{足}{あし}に
\ruby[g]{纏繞}{ま つ }はるを
\ruby{煩}{うる}さしと
\ruby{苛}{いら}ち
ながら、
%
\ruby[g]{芝翫}{しくわん}
\ruby[g]{下駄}{げ た }も
\ruby{踏}{ふ}みかへしたる
まゝ% 踊り字調整「〻(二の字点、揺すり点)に見えるが(ゝ)」
\ruby{脫}{ぬ}ぎ
\ruby{捨}{す}てゝ、% 踊り字調整「〻(二の字点、揺すり点)に見えるが(ゝ)」
%
\ruby[g]{足袋}{た び }
\ruby[g]{徒跣}{はだし }の
\ruby{脛}{はぎ}
あらはなる
さまの
\ruby{我}{われ}
\ruby{羞}{はづか}しきを
\ruby{厭}{いと}ふに
\ruby[g]{暇無}{ひまな }く、
%
\ruby{跳}{をど}る
\ruby{胸}{むね}の
\ruby[g]{氣息}{い き }
\ruby{苦}{ぐる}しさを
\ruby{堪}{こら}へ、

\原本頁{18-7}%
『
\ruby{源}{げん}さーん
』

\原本頁{18-8}%
と
\ruby{{\換字{又}}}{また}
\ruby{一}{ひ}ト
\ruby{聲}{こゑ}
\ruby{呼}{よ}ぶに、
%
\ruby{男}{をとこ}は
\ruby{{\換字{猶}}}{なほ}
\ruby[||j>]{心}{こゝろ}% 踊り字調整「〻(二の字点、揺すり点)に見えるが(ゝ)」
\ruby[||j>]{{\換字{強}}}{ づよ}くも
\ruby{走}{はし}つて
\ruby{已}{や}まず、
%
\ruby[g]{{\換字{返}}響}{こ だま}
のみ
\ruby{我}{わ}
が
\ruby{耳}{みゝ}に、% 踊り字調整「〻(二の字点、揺すり点)に見えるが(ゝ)」

\原本頁{18-10}%
『
\ruby{源}{げん}さーん
』

\原本頁{18-11}%
と
\ruby{悲}{かな}しく
\ruby{聞}{きこ}えて、
%
\ruby[g]{天地}{てんち }は
\ruby[g]{{\換字{情}}無}{つれな }く
しん〳〵と
\ruby[g]{物寂}{ものさび}しく、
%
\ruby{月}{つき}も
ぼんやり、
%
\ruby{花}{はな}も
\ruby[g]{朦朧}{ぼんやり}、
%
\ruby{何}{なに}とも
\ruby{云}{い}へず
\ruby{只}{たゞ}% 踊り字調整「〻(二の字点、揺すり点)に濁点に見えるが(ゞ)」
\ruby{靜}{しづか}に
して、
%
\ruby{我}{われ}
のみの
\ruby{騷}{さわ}ぎ
\ruby{悶}{もだ}ゆるを
\ruby{笑}{わら}へるが
\ruby{如}{ごと}し。

\原本頁{19-3}%
『
\ruby{源}{げん}さーん
』

\原本頁{19-4}%
\ruby{堪}{た}へかねて
\ruby{{\換字{又}}}{また}
\ruby[g]{一度}{ひとたび}
\ruby{呼}{よ}べば、

\原本頁{19-5}%
『
\ruby{源}{げん}さーん
』

\原本頁{19-6}%
と
%
\ruby{花}{はな}の
\ruby{間}{なか}より
\ruby[g]{{\換字{返}}響}{こ だま}
のみ
\ruby{{\換字{又}}}{また}
\ruby[g]{一度}{ひとたび}
\ruby{繰}{く}り
\ruby{{\換字{返}}}{かへ}したる
\ruby{其}{そ}の
\ruby{聲}{こゑ}の
\ruby{響}{ひゞ}くに% 踊り字調整「〻(二の字点、揺すり点)に濁点に見えるが(ゞ)」
\ruby{{\換字{連}}}{つ}れて
\ruby{我}{わ}が
\ruby[g]{頭上}{づじやう}なる
\ruby{花}{はな}は
ちら〳〵と
\ruby{散}{ち}り
かゝりて、% 踊り字調整「〻(二の字点、揺すり点)に見えるが(ゝ)」
%
\ruby[g]{忽然}{こつぜん}として
\ruby[g]{眞實}{まこと }の
\ruby{{\換字{雪}}}{ゆき}となり、
%
\ruby{見}{み}やる
\ruby[g]{彼方}{かなた }には
\ruby[g]{廣々}{ひろ〴〵}と
したる
\ruby[g]{川原}{かはら }の
\ruby{見}{あら}はれて、
%
\ruby[g]{其處}{そ こ }を
\ruby{流}{なが}るゝ% 踊り字調整「〻(二の字点、揺すり点)に見えるが(ゝ)」
\ruby{水}{みづ}の
\makeatletter
\@ifundefined{デバッグ@ビルド}{%
  \ruby[<g>]{勢}{いきほひ}
  \ruby[||g]{{\換字{強}}き}{つよ }に、% ルビ調整(長いルビ対策)長いルビの調整
}{%
  \ruby[<j>]{勢}{いきほひ}
  \ruby[||j>]{{\換字{強}}}{ つよ}きに、% ルビ調整(長いルビ対策)長いルビの調整
}%
\makeatother
%
\ruby[g]{渡舟}{わたし }
\ruby{無}{な}く
\ruby{橋}{はし}
\ruby{無}{な}ければ
%
\ruby{男}{をとこ}は
\ruby{{\換字{逃}}}{に}げ
まどひ
て、
%
\ruby[g]{哀憫}{あはれみ}を
\ruby{乞}{こ}ふが
\ruby{如}{ごと}く
\ruby[g]{此方}{こなた }を% ルビ調整(原本通り)
\ruby{振}{ふ}り
\ruby{{\換字{返}}}{かへ}りぬ。
%
\ruby{戀}{こひ}しかりしは
\ruby[g]{先刻}{さ き }の
\原本頁{19-11}\改行%
\ruby{程}{ほど}なり、
%
\ruby{今}{いま}は
\ruby{憎}{にく}さ
\ruby{恨}{うら}めしさの
むら〳〵と
\ruby{湧}{わ}き
\ruby{上}{あが}りて、
%
\ruby{思}{おも}はずも
\原本頁{20-1}\改行%
\ruby{手}{て}にしたる
\ruby[g]{短銃}{ぴすとる}の
\ruby[g]{引金}{ひきがね}を
\ruby{引}{ひ}けば、
%
どんと
\ruby{云}{い}ふ
\ruby{音}{おと}の
\ruby{中}{うち}に
\ruby{白}{しろ}
\ruby[||j>]{{\換字{煙}}}{けむり}
ぱつと
\ruby{立}{た}つて、
%
\ruby{源}{げん}は
\ruby{朱}{あけ}に
なりつ
\ruby{摚}{どう}と
\ruby{倒}{たふ}れたるが、
%
\ruby{源}{げん}の
\ruby{倒}{たふ}るゝと% 踊り字調整「〻(二の字点、揺すり点)に見えるが(ゝ)」
\ruby[g]{同時}{どうじ }に
\ruby{其}{そ}の
\ruby[g]{身後}{うしろ }に、
%
\ruby[g]{記臆}{おぼ{{\換字{𛀁}}}}も% 原本通り「おぼ𛀁」
\ruby{無}{な}く
\ruby{名}{な}も
\ruby{知}{し}らぬ
\ruby{{\換字{若}}}{わか}き
\ruby{男}{をとこ}の、
%
\ruby{明}{あき}らかに
\ruby[g]{此方}{こなた }を% ルビ調整(原本通り)
\ruby{向}{む}きて
\ruby[g]{悠然}{いうぜん}として
\ruby{岸}{きし}に
\ruby{立}{た}てるが
\ruby{見}{み}えたり。
%
\ruby[g]{流石}{さすが }に
\ruby{人}{ひと}を
\ruby{殺}{ころ}したる
\ruby{身}{み}の
\ruby{罪}{つみ}に、
%
\ruby{心}{こゝろ}は% 踊り字調整「〻(二の字点、揺すり点)に見えるが(ゝ)」
\ruby{度}{ど}を
\ruby{失}{うしな}ひて
\ruby{悸}{おそ}れ
\ruby{戰}{わなゝ}けるを、% 踊り字調整「〻(二の字点、揺すり点)に見えるが(ゝ)」
%
\ruby{彼}{か}の
\ruby{男}{をとこ}は
\ruby[g]{寛大}{おほやう}に
\ruby{淸}{すゞ}しき% 踊り字調整「〻(二の字点、揺すり点)に濁点に見えるが(ゞ)」
\ruby{聲}{こゑ}して、

\原本頁{20-7}%
『
\ruby{赦}{ゆる}す、
%
\ruby{赦}{ゆる}してやる。
』

\原本頁{20-8}%
と
\ruby{優}{やさ}しく
\ruby{云}{い}ひたる
\ruby{其}{その}
\ruby{聲}{こゑ}の、
%
\ruby[g]{何故}{なにゆゑ}とは
\ruby{無}{な}けれど
\ruby{身}{み}に
\ruby{沁}{し}みて
\ruby{嬉}{うれ}しく
\改行% 校正作業の簡略化のため
、
%
\原本頁{20-9}\改行%
\ruby{骨}{ほね}も
\ruby{溶}{と}くるやうに
\ruby{悅}{よろこ}ばしと
\ruby{思}{おも}ふに
つれて、
%
\ruby[g]{忽地}{たちまち}
\ruby{今}{いま}までの
\ruby{妾}{わ}が
\ruby[g]{振舞}{ふるまひ}の
はした
\ruby{無}{な}かりしが
\ruby[g]{口惜}{く や }しく
\ruby{慚}{はづか}しく、
%
\ruby{顏}{かほ}に
\ruby{火}{ひ}の
\ruby{照}{て}る
おもひして、
%
\ruby{何}{なに}とか
\ruby{言}{ものい}はん
\ruby{言}{ものい}はんと
すれば、
%
\ruby{舌}{した}も
\ruby{結}{むす}ぼゝれ% 踊り字調整「〻(二の字点、揺すり点)に見えるが(ゝ)」
\ruby{唇}{くち}も
\ruby{動}{うご}かず、
%
\原本頁{21-1}\改行%
\ruby{有}{あ}り
\ruby{餘}{あま}る
\ruby{胸}{むね}の
\ruby{思}{おも}ひを
\ruby{現}{あらは}すに
\ruby{由}{よし}
\ruby{無}{な}く、
%
\ruby{苦}{くる}しみ〳〵て
\ruby[g]{氣息}{い き }
\ruby{塞}{つま}りたり
\改行% 校正作業の簡略化のため
。

\原本頁{21-2}%
『
お
\ruby{龍}{りう}ちやん、
%
お
\ruby{龍}{りう}ちやん、
%
\ruby[g]{何樣}{ど う }
お
\ruby{爲}{し}だよ、
%
お
\ruby{龍}{りう}。
%
\ruby[g]{大層}{たいそう}
\ruby{魘}{ゝな}されて% 踊り字調整「〻(二の字点、揺すり点)に見えるが(ゝ)」
\ruby{居}{ゐ}るぢや
\ruby{無}{な}いか。
』

\原本頁{21-4}%
『
ア、
%
\ruby{御師匠}{お|し|よ}さん!。
』

\原本頁{21-5}%
\ruby{覺}{さ}めたれども
\ruby{{\換字{猶}}}{なほ}
\ruby[g]{茫然}{ばうぜん}として、
%
\ruby[g]{星眼}{せいがん}
うつとりと
\ruby{懶}{ものう}げに
\ruby{動}{うご}かず。

\原本頁{21-6}%
『
\ruby[||j>]{汝}{おまへ}
\ruby[||j>]{何}{ なに}か
\ruby{怖}{おそ}ろしい
\ruby{夢}{ゆめ}でも
\ruby{見}{み}たかエ。
%
お
\ruby{{\換字{廉}}}{やす}くない
\ruby{夢}{ゆめ}か
なんぞぢやあ
\ruby{無}{な}いか。
』

\原本頁{21-8}%
『
あら
お
\ruby[g]{師匠}{し よ }さん、
%
\ruby{{\換字{嫌}}}{いや}な!。
%
\ruby{何}{なに}か
\ruby{言}{い}つて?。
』

\原本頁{21-9}%
『
\ruby{何}{なん}だか
\ruby{{\換字{分}}}{わか}らなかつたよ、
%
\ruby{妾}{わたし}も
\ruby{今}{いま}
\ruby{目}{め}が
\ruby{覺}{さ}めたんだもの。
%
\ruby{夢}{ゆめ}は
\ruby[g]{五臓}{し ん }の
\ruby[g]{疲勞}{つかれ }だつて
\ruby{云}{い}ふぢや
\ruby{無}{な}いか。
%
\ruby[g]{昨夜}{ゆふべ }
\ruby{妾}{わたし}が
\ruby[g]{寄席}{よ せ }から
\ruby{歸}{かへ}つて、
%
それから
また
お
\ruby[g]{五十}{い そ }の
\ruby{談}{はなし}や
なんぞを
\ruby{遲}{おそ}くまで
\ruby{仕}{し}たもんだから、
%
\ruby[g]{屹度}{きつと }
お
\ruby{{\換字{前}}}{まへ}
\ruby[g]{五臓}{し ん }が
\ruby{疲}{つか}れたんだよ。
%
それで
\ruby{魘}{うな}されたり
なんぞ
\ruby{仕}{し}たんだらうよ。
』

\原本頁{22-3}%
『
そんな
\ruby{事}{こと}かも
\ruby{知}{し}れませんよ。
%
オヤツ、
%
\ruby[g]{今{\換字{朝}}}{け さ }は
お
\ruby[g]{師匠}{し よ }さんの
\ruby{代}{かは}りに
\ruby{四ツ木}{よ| |ぎ}へ
いつて
\ruby{御病氣}{ご|びやう|き}
\ruby[g]{見舞}{み まひ}を
\ruby{爲}{す}る
\ruby{筈}{はず}でしたつけ。
%
\ruby[g]{斯樣}{か う }しちやあ
\ruby{居}{ゐ}られないんでした、
%
まあ
\ruby{起}{お}きましやう。
%
しかし
\ruby{何}{なん}だか
\ruby[g]{可怪}{をかし }な% ルビ調整(原本遠り)(をかし)
\ruby{夢}{ゆめ}を
\ruby{妾}{わたし}あ
\ruby{見}{み}ましたよ。
』

\原本頁{22-7}%
\ruby{起}{お}きんとして
\ruby{起}{お}きず
\ruby{枕}{まくら}に
\ruby[g]{俯伏}{うつぶ }して、
%
\ruby{美}{うつく}しき
\ruby[g]{頸脚}{{\換字{𛀁}}りあし}を
\ruby[g]{惜氣}{をしげ }も
\ruby{無}{な}く
\ruby{見}{み}
\原本頁{22-8}\改行%
せつ、
%
\ruby{名}{な}も
\ruby{知}{し}らず
\ruby{顏}{かほ}も
\ruby{定}{さだ}かならで
\ruby{聲}{こゑ}のみを
\ruby{聞}{き}きたる
\ruby{夢}{ゆめ}の
\ruby{中}{うち}の
\ruby{其}{その}
\原本頁{22-9}\改行%
\ruby{人}{ひと}を
\ruby{思}{おも}ふにやあらん、
%
\ruby[g]{凝然}{じ つ }として
\ruby[g]{少時}{しばし }
\ruby[g]{思想}{おもひ }に
\ruby{耽}{ふけ}りたるが、
%
\ruby{寐}{ね}
み
\原本頁{22-10}\改行%
だれたる
\ruby{髮}{かみ}の
ほつれて
かゝれる% 踊り字調整「〻(二の字点、揺すり点)に見えるが(ゝ)」
\ruby[g]{横顏}{よこがほ}
ふくよかに
\ruby{白}{しろ}くして
\ruby{艶}{{\換字{𛀁}}ん}% 原本通り「𛀁ん」
なり
\改行% 校正作業の簡略化のため
。

\Entry{其四}

\原本頁{}
\ruby{{\換字{近}}}{ちか}く
\ruby{窓外}{まど|そと}を
\ruby{{\換字{過}}}{す}ぐる
\ruby{物賣}{もの|う}りの
\ruby{聲}{こゑ}は
\ruby{尾}{を}を
\ruby{引}{ひ}いて
\ruby{長}{なが}く、
%
\ruby{少}{すこ}し
\ruby{隔}{へだ}たりて
\ruby{聞}{きこ}ゆる
\ruby{大{\換字{通}}}{おほ|どほ}りの
\ruby{車馬}{しや|ば}の
\ruby{響}{ひゞき}は
\ruby{一}{ひと}ツになりてがやつき
\ruby{出}{だ}す
\ruby{日本橋}{に|ほん|ばし}は
\ruby{本銀町}{ほん|しろがね|ちやう}あたりの
\ruby{某}{それ}の
\ruby{横丁}{よこ|ちやう}の
\ruby{{\換字{朝}}景色}{あさ|げ|しき}、
%
\ruby{建}{た}ちならべる
\ruby{家々}{いへ|〳〵}に
\ruby{家々}{いへ|〳〵}の
\ruby{聲}{こゑ}あり
\ruby{物音}{もの|おと}ありて、
%
\ruby{子供}{こ|ども}あるところは
\ruby{先}{ま}づ
\ruby{騷々}{さう|〴〵}しく、
%
\ruby{{\換字{若}}佼}{わか|き}が
\ruby{多}{おほ}きところは
\ruby{笑多}{わらひ|おほ}く、
%
\ruby{火}{ひ}の
\ruby{燃}{も}ゆる
\ruby{音}{おと}、
%
\ruby{水使}{みづ|つか}ふ
\ruby{音}{おと}、
%
\ruby{夜明}{よ|あ}けより
\ruby{一二時間}{いち|に|じ|かん}ばかりが
\ruby{程}{ほど}の
\ruby{一}{ひ}トしきり
\ruby{賑}{にぎ}やかなるは
\ruby{家}{いへ}ごみの
\ruby{市中}{まち|なか}の
\ruby{常}{つね}の
\ruby{態}{さま}なり。

\原本頁{}
いつもの
\ruby{晏起}{おそ|おき}には
\ruby{似}{に}ず
\ruby{今日}{け|ふ}は
\ruby{早起}{はや|おき}して、
%
お
\ruby{關}{せき}の
\ruby{家}{いへ}の
\ruby{{\換字{朝}}食}{あさ|めし}は
\ruby{疾}{とく}に
\ruby{濟}{す}みぬ。
%
\ruby{既}{すで}に
\ruby{髮}{かみ}を
\ruby{理}{をさ}め
\ruby{身}{み}じまひしたる
お
\ruby{龍}{りう}は、
%
\ruby{今}{いま}また
\ruby{衣}{い}を
\ruby{{\換字{更}}}{あらた}め
\ruby{帶}{おび}を
\ruby{換}{か}へて、
%
これより
\ruby{四}{よ}ツ
\ruby{木}{ぎ}へ
\ruby{赴}{おもむ}かんとはするなり。

\原本頁{}
\ruby{女主人}{あ|る|じ}は
\ruby{帶止}{おび|ど}めの
\ruby{美}{うつく}しきを
お
\ruby{龍}{りう}に
\ruby{渡}{わた}して、

\原本頁{}
『
\ruby{一寸}{ちよ|い}と
\ruby{見}{み}ておくれ、
%
\ruby{此品}{こ|れ}あ
\ruby{妾}{わたし}が
\ruby{汝}{おまへ}にあげやうと
\ruby{思}{おも}つて
\ruby{取}{と}つて
\ruby{來}{き}たんだよ。
%
\ruby{昨夜}{ゆふ|べ}
\ruby{直}{す}ぐあげやうと
\ruby{思}{おも}つて
\ruby{居}{ゐ}たが、
%
つい
\ruby{忘}{わす}れて
\ruby{仕舞}{し|ま}つた。
%
\ruby{夜}{よる}だつたもんだから、
%
\ruby{能}{よ}く
\ruby{{\換字{分}}}{わか}らなくつて、
%
\ruby{今}{いま}
\ruby{見}{み}ると
\ruby{色}{いろ}が
\ruby{何}{なん}だか
\ruby{思}{おも}つたやうぢや
\ruby{無}{な}いが、
%
\ruby[<h||]{汝}{おまへ}
\ruby{厭}{いや}で
\ruby{無}{な}けりやあ
\ruby{締}{し}めておくれナ。
』

\原本頁{}
と
\ruby{云}{い}へば
お
\ruby{龍}{りう}は
\ruby{嬉}{うれ}しげに
\ruby{見}{み}ながら、

\原本頁{}
『あら
\ruby{勿體}{もつ|たい}ない、
%
\ruby{佳}{い}い
\ruby{色}{いろ}ですわ。
%
ちつとも
\ruby{可厭}{い|や}な
\ruby{事}{こと}なんぞありあ
\ruby{仕}{し}ませんが、
%
ほんとに
\ruby{此品}{こ|り}あ
\ruby{戴}{いたゞ}いても
\ruby{宜}{い}いの?。
』

\原本頁{}
と、
%
\ruby{我}{われ}を
\ruby{愛}{あい}し
\ruby{吳}{く}るゝ
\ruby{女主人}{あ|る|じ}が
\ruby{{\換字{情}}}{なさけ}を、
%
\ruby{深}{ふか}くも
\ruby{悅}{よろこ}べる
\ruby{其}{そ}の
\ruby{眼色}{め|いろ}に、
%
\ruby{少}{すくな}からぬ
\ruby{感謝}{かん|しや}の
\ruby{意}{こゝろ}は
\ruby{表}{あらは}れたり。

\原本頁{}
『いゝともさ!お
\ruby{{\換字{前}}}{まへ}にあげやうつて
\ruby{買}{か}つて
\ruby{來}{き}たんだもの!。
%
それぢやあ
\ruby{御苦勞}{ご|く|らう}だけれども
\ruby{行}{い}つて
\ruby{來}{き}ておくれ。
%
いゝかエ、
%
\ruby{吾妻橋}{あ|づま|ばし}から
\ruby{直}{すぐ}
\ruby{滊車}{き|しや}に
\ruby{乘}{の}つて、
%
\ruby{鐘}{かね}が
\ruby{淵}{ふち}といふので
\ruby{下}{お}りて
\ruby{右}{みぎ}の
\ruby{方}{はう}へ
\ruby{眞直}{まつ|すぐ}に
\ruby{行}{い}きさへすりやあ
\ruby{{\換字{造}}作}{ざう|さ}ないんだよ。
%
だけど
\ruby{田舎}{ゐな|か}
\ruby{{\換字{道}}}{みち}だから
\ruby{聞}{き}き
\ruby{聞}{き}き
\ruby{行}{い}かないと
\ruby{損}{そん}をするよ。
』

\原本頁{}
『ハイ、
%
ようく
\ruby{{\換字{分}}}{わか}りました。
%
\ruby{狐}{きつね}に
\ruby{魅}{ばか}されないやうに
\ruby{參}{まゐ}りますよ。
%
ホヽヽ。
』

\原本頁{}
『ハヽヽ、
%
ほんとに
\ruby{田舎}{ゐな|か}
\ruby{{\換字{道}}}{みち}でまごつく
\ruby[<h||]{位}{くらゐ}
\ruby{器量}{きり|やう}の
\ruby{惡}{わる}い
\ruby{事}{こと}あ
\ruby{無}{な}いから\換字{子}、
%
よく
\ruby{魅}{ばか}されないやうに
お
\ruby{仕}{し}よ。
%
ハヽヽ。
%
それから、
%
あの、
%
\ruby{忘}{わす}れても
お
\ruby[g]{五十}{いそ}のところへ
\ruby{行}{い}くんぢやないよ。
%
\ruby{傳染}{う|つ}つた
\ruby{日}{ひ}にやあ
\ruby{間尺}{まし|やく}に
\ruby{合}{あ}はないから\換字{子}。
%
たゞ
\ruby{水野}{みづ|の}つて
\ruby{云}{い}ふのが
\ruby{世話}{せ|わ}を
\ruby{仕}{し}て
\ruby{居}{ゐ}やうから\換字{子}、
%
\ruby{其男}{そ|れ}に
\ruby{會}{あ}つて
\ruby{見舞}{み|まひ}の
\ruby{口上}{こう|じやう}を
\ruby{昨夜}{ゆふ|べ}
\ruby{敎}{をし}へて
\ruby{置}{お}いた
\ruby{{\換字{通}}}{とほ}りに
\ruby{云}{い}やあ
\ruby{宜}{い}いんだよ。
%
つまり
\ruby{病人}{びやう|にん}は
\ruby{何樣}{ど|う}だつて
\ruby{構}{かま}はないんだが、
%
その
\ruby{水野}{みづ|の}つて
\ruby{男}{をとこ}への
\ruby{義理}{ぎ|り}でもつて、
%
お
\ruby{{\換字{前}}}{まへ}に
\ruby{行}{い}つて
\ruby{貰}{もら}ふやうな
\ruby{譯}{わけ}なんだから\換字{子}。
』

\原本頁{}
『ハイ、
%
\ruby{何}{なん}だか
\ruby{能}{よ}く
\ruby{{\換字{分}}}{わか}りませんけど、
%
\ruby{宜}{い}い
\ruby{加減}{か|げん}に
\ruby{申}{まを}して
\ruby{置}{お}きやあ
\ruby{宜}{い}いのでございましやう\換字{子}エ。
』

\原本頁{}
『ハヽヽ、
%
\ruby{左樣}{さ|う}さ、
%
\ruby{左樣}{さ|う}さ、
%
それで
\ruby{宜}{い}いとも!。
%
\ruby{妾}{わたし}が
\ruby{顏}{かほ}を
\ruby{出}{だ}しやあ
\ruby{何程}{いく|ら}
\ruby{{\換字{嫌}}}{いや}でも
\ruby{直接}{ぢ|か}に
お
\ruby[g]{五十}{いそ}を
\ruby{見舞}{み|ま}つて
\ruby{{\換字{遣}}}{や}らなきやならないんだから\換字{子}。
%
\ruby{{\換字{平}}生}{ふだ|ん}
\ruby{{\換字{交}}{\換字{情}}}{な|か}の
\ruby{惡}{わる}い
\ruby{奴}{やつ}の
\ruby{疫病}{やく|びやう}なんぞを、
%
\ruby{四}{よ}ツ
\ruby{木}{ぎ}くんだりへ
\ruby{見舞}{み|まひ}に
\ruby{行}{い}くなんて、
%
\ruby{可厭}{い|や}な
\ruby{事}{こツ}ちや
\ruby{無}{な}いか、
%
\ruby{馬鹿}{ば|か}
\g詰めruby{々々}{〳〵}しいわ\換字{子}。

\原本頁{}
だから
\ruby{妾}{わたし}あ
\ruby{寸白}{す|ばく}が
\ruby{起}{おこ}つて
\ruby{居}{ゐ}るんで
\ruby{出}{で}られないからとか
\ruby{何}{なん}とか
\ruby{云}{い}つて\換字{子}、
%
\ruby{娘}{むすめ}が
\ruby{生}{い}きても
\ruby{死}{し}んでも
\ruby{構}{かま}はないか、
%
あんまりな
\ruby{人}{ひと}だと、
%
\ruby{水野}{みづ|の}に
\ruby{思}{おも}はれないやうに
\ruby{云}{い}つて
\ruby{置}{お}いて
\ruby{吳}{く}れさへすりやあ
\ruby{其}{それ}で
\ruby{宜}{い}いんだよ。
%
\ruby{水野}{みづ|の}に
\ruby{惡}{わる}く
\ruby{思}{おも}はれないやうにして
\ruby{置}{お}くと、
%
また
\ruby{好}{い}い
\ruby{事}{こと}があるかも
\ruby{知}{し}れないんだから。
』

\原本頁{}
『ハイ、
%
\ruby{宜}{よろ}しうございます。
%
ぢやあ
\ruby{水野}{みづ|の}さんて
\ruby{仰}{おつし}あるのは、
%
\ruby{畢竟}{つま|り}
お
\ruby[g]{五十}{いそ}さんの
\ruby{御婿}{お|むこ}さんになる
\ruby{筈}{はず}の
\ruby{方}{かた}なんですか?。
』

\原本頁{}
『ナアに
\ruby{左樣}{さ|う}ぢやあ
\ruby{無}{な}いんだよ、
%
\ruby{何}{なん}でも
\ruby{無}{な}いんだよ。
%
お
\ruby[g]{五十}{いそ}には
\ruby{散々}{さん|〴〵}に
\ruby{{\換字{嫌}}}{きら}はれてゐるのさ。
』

\原本頁{}
『ヘエー、
%
\ruby{何}{なん}だか
\ruby{譯}{わけ}が
\ruby{{\換字{分}}}{わか}らないの\換字{子}。
%
それぢや
\ruby{御師匠樣}{お|し|よ|さん}の
\ruby{方}{はう}で
お
\ruby[g]{五十}{いそ}さんの
\ruby{御婿}{お|むこ}さんになさらうと
\ruby{思}{おも}つて
\ruby{居}{ゐ}らつしやる
\ruby{方}{かた}なの?。
』

\原本頁{}
『いゝえ、
%
\ruby{左樣}{さ|う}といふんでも
\ruby{無}{な}いんだよ。
%
\ruby{妾}{わたし}あそんな
\ruby{餘計}{よ|けい}な
\ruby{世話燒}{せ|わ|やき}きなんか
\ruby{{\換字{嫌}}}{いや}な
\ruby{事}{こつ}た\換字{子}。
』

\原本頁{}
『ヘエー、
%
\ruby{妙}{めう}\換字{子}エ。
%
\ruby{些}{ちつと}も
\ruby{譯}{わけ}が
\ruby{{\換字{分}}}{わか}らないの\換字{子}。
%
そしてその
\ruby{水野}{みづ|の}さんて
\ruby{怖}{こは}い
\ruby{人}{ひと}ですか。
』

\原本頁{}
『
\ruby{何}{なん}だ\換字{子}。
%
もう
\ruby{男}{をとこ}を
\ruby{怖}{こは}がる
\ruby{筈}{はず}の
お
\ruby{{\換字{前}}}{まへ}でも
\ruby{無}{な}いぢやあ
\ruby{無}{な}いか。
%
\ruby{高}{たか}が
\ruby{書}{ほん}を
\ruby{讀}{よ}んでるばかりの
\ruby{書生坊}{しよ|せい|ばう}で、
%
\ruby{柔}{やはら}かいんだか
\ruby{硬}{かた}いんだか
\ruby{何}{なん}だか、
%
\ruby{恰}{まる}で
\ruby{赤小豆}{あ|づ|き}の
\ruby{煮}{に}えこじけたやうな
\ruby{變}{へん}な
\ruby{可厭}{い|や}な
\ruby{男}{をとこ}さ。
』

\原本頁{}
『ヘエー、
%
\ruby{兎}{と}も
\ruby{角}{かく}もまあ
\ruby{行}{い}つてまゐりましやう。
%
ぢやあ
\ruby{食後片付}{あ|と|かた|づ}けもいたしませんが…………』

\原本頁{}
『いゝよ、
%
お
\ruby{構}{かま}ひでない、
%
さあ
\ruby{早}{はや}くおいで。
%
\ruby{今{\換字{朝}}}{け|さ}
\ruby{桂庵}{けい|あん}が
\ruby{婢}{をんな}を
\ruby{{\換字{連}}}{つ}れて
\ruby{來}{く}る
\ruby{筈}{はず}だから。
』

\原本頁{}
『ぢやあ、
%
\ruby{行}{い}つてまゐります。
』

\原本頁{}
『
\ruby{氣}{き}をつけておいで。
』

\原本頁{}
\ruby{見舞品}{み|まひ|もの}にや
\ruby{風呂敷包}{ふ|ろ|しき|づゝみ}の
\ruby{小}{ちひさ}きを
\ruby{持}{も}つて、
%
\ruby{街}{おもて}へ
\ruby{立出}{たち|い}でたる
\ruby{色白}{いろ|じろ}の
お
\ruby{龍}{りう}が、
%
\ruby{小}{こ}ざつぱりしたる
\ruby{着付}{き|つけ}、
%
すらりとしたる
\ruby{姿}{すがた}は、
%
\ruby{忽}{たちま}ち
\ruby{往來}{わう|らい}の
\ruby{職人}{しよく|にん}の
\ruby{眼}{め}を
\ruby{惹}{ひ}きて、

\原本頁{}
『
\ruby{吉}{きち}や、
%
\ruby{見}{み}ねエ、
%
\ruby{小股}{こ|また}の
\ruby{切}{き}り
\ruby{上}{あが}がつた
\ruby{好}{い}い
\ruby{新{\換字{造}}}{しん|ぞ}だナア。
』

\原本頁{}
『ウン、
%
\ruby{打殺}{ぶつ|ち}めて
\ruby{{\換字{遣}}}{や}りてえナ。
』

\原本頁{}
と
\ruby{叫}{さけ}び
\ruby{出}{いだ}さしめぬ。


\Entry{其五}

\ruby{我}{わ}が
\ruby{五十子}{い|そ|こ}にさしたる
\ruby{異狀無}{い|じやう|な}しといふ
\ruby{尾竹}{を|だけ}が
\ruby{言葉}{こと|ば}に
\ruby{心安堵}{こゝろ|おち|つ}きて、
\ruby{徐々}{おも|むろ}に
\ruby{我}{わ}が
\ruby{寓}{やど}に
\ruby{歸}{かへ}れる
\ruby{水野}{みづ|の}は、
\ruby{主人}{ある|じ}の
\ruby{吉右衛門}{き|ち|ゑ|もん}が
\ruby{老實}{まめ|やか}なる
\ruby{注意}{こゝろ|づけ}に
\ruby{任}{まか}せて、
\ruby{其夜}{その|よ}は
\ruby{早}{はや}くより
\ruby{臥床}{ふし|ど}に
\ruby{入}{い}りけるが、
\ruby{疲}{つか}れきつたるが
\ruby{故}{ゆゑ}にや
\ruby{却}{かへ}つて
\ruby{睡}{ねむ}りかねたり。

\ruby{一}{ひ}ㇳ
\ruby{間}{ま}を
\ruby{隔}{へだ}てたる
\ruby{茶}{ちや}の
\ruby{室}{ま}の
\ruby{燈}{ひ}の
\ruby{下}{もと}に、
\ruby{老夫}{おや|ぢ}は
\ruby{悠々}{いう|〳〵}と
\ruby{{\換字{煙}}草}{たば|こ}を
\ruby{喫}{ふか}せば、
\ruby{孫娘}{まご|むすめ}の
お
\ruby{濱}{はま}はまた
\ruby{一心}{いつ|しん}に
\ruby{何}{なん}の
\ruby{書}{しよ}をか
\ruby{讀}{よ}めるさまの、
\ruby{折々其}{をり|〳〵|そ}の
\ruby{{\換字{煙}}草管}{き|せ|る}をはたく
\ruby{音}{おと}、
\ruby{書}{ほん}を
\ruby{開}{あ}け
\ruby{飜}{ひるがへ}す
\ruby{音}{おと}の
\ruby{耳}{みゝ}に
\ruby{入}{い}る
\ruby{度}{たび}、いと
\ruby{明}{あき}らかに
\ruby{思}{おも}ひ
\ruby{{\換字{遣}}}{や}られつ、それに
\ruby{打{\換字{交}}}{うち|まじ}へて
\ruby{五十子}{い|そ|こ}が
\ruby{病}{やまひ}、
\ruby{島木}{しま|き}が
\ruby{{\換字{情}}}{なさけ}、
お
\ruby{澤婆}{さは|ばゝあ}が
\ruby{憎}{にく}さ、
\ruby{觀音堂}{くわん|のん|だう}の
\ruby{朝}{あさ}の
\ruby{感}{かん}じ、
\ruby{椎}{しひ}の
\ruby{樹蔭}{こ|かげ}の
\ruby{夕}{ゆふべ}の
\ruby{思}{おもひ}など、
\ruby{{\換字{廻}}}{まは}り
\ruby{燈籠}{どう|ろう}の
\ruby{其影像}{その|か|げ}の
\ruby{如}{ごと}く
\ruby{繰{\換字{返}}}{くり|かへ}し〳〵
\ruby{胸}{むね}に
\ruby{現}{あら}はるゝに、
\ruby{幾度}{いく|たび}か
\ruby{幾度}{いく|たび}か
\ruby{寝{\換字{返}}}{ね|がへ}り
\ruby{打}{う}ち
\ruby{寝{\換字{返}}}{ね|がへ}り
\ruby{打}{うつ}て
\ruby{睡}{ねむ}らんとしても
\ruby{睡}{ねむ}られず、ほと〳〵
\ruby{自}{みずか}ら
\ruby{困}{こう}じけるが、やがて
\ruby{何時}{い|つ}と
\ruby{無}{な}く
\ruby{心鈍}{こゝろ|にぶ}りて、
\ruby{天地}{てん|ち}を
\ruby{薄霧}{うす|ぎり}に
\ruby{包}{つゝ}み
\ruby{行}{ゆ}かるゝが
\ruby{如}{ごと}き
\ruby{思}{おもひ}をしつゝ、
\ruby{辛}{から}くも
\ruby{我我}{われ|〳〵}をおぼえぬ
\ruby{境}{さかひ}に
\ruby{入}{い}りぬ。

\ruby{疲勞}{つか|れ}は
\ruby{名殘無}{な|ごり|な}く
\ruby{一睡}{いつ|すゐ}に
\ruby{{\換字{消}}}{き}えて、
\ruby{明}{あ}けての
\ruby{其}{そ}の
\ruby{朝}{あさ}は
\ruby{我}{わ}が
\ruby{心}{こゝろ}のいと
\ruby{淸々}{すが|〳〵}しきに、
お
\ruby{澤}{さは}が
\ruby{許}{もと}に
\ruby{置}{お}ける
\ruby{婢}{をんな}の
お
\ruby{鹽}{しほ}といふより、
\ruby{五十子}{い|そ|こ}が
\ruby{病}{やまひ}も
\ruby{{\換字{平}}}{たひ}らなりとの
\ruby{報知}{しら|せ}をさへ
\ruby{得}{ゑ}たれば、
\ruby{水野}{みづ|の}は
\ruby{此頃}{この|ごろ}におぼ{\換字{江}}
\ruby{無}{な}く
\ruby{氣合冴々}{き|あひ|さ{\換字{江}}|〴〵}しく、先づ
\ruby{島木}{しま|き}に
\ruby{當}{あ}てゝの
\ruby{謝狀}{れい|じやう}を
\ruby{書}{か}き、次に
\ruby{羽{\換字{勝}}}{は|がち}に
\ruby{當}{あ}てゝ
\ruby{{\換字{過}}}{す}ぐる
\ruby{日}{ひ}の
\ruby{會}{くわい}に
\ruby{不參}{ふ|さん}したる
\ruby{理由}{い|はれ}を
\ruby{書}{か}きて
\ruby{我}{わ}が
\ruby{心}{こゝろ}の
\ruby{變}{かは}り
\ruby{無}{な}きことを
\ruby{云}{い}ひ
\ruby{{\換字{遣}}}{や}り、また
\ruby{五十子}{い|そ|こ}が
\ruby{繼母}{まゝ|はゝ}の
お
\ruby{關}{せき}に
\ruby{對}{むか}つては、
\ruby{五十子}{い|そ|こ}が
\ruby{病狀}{びやう|じやう}の
\ruby{{\換字{概}}略}{あら|まし}と
\ruby{手當}{て|あて}の
\ruby{模樣}{も|やう}とを
\ruby{知}{し}らせやりて、さて
\ruby{朝食}{あさ|めし}を
\ruby{濟}{す}ませて
\ruby{立出}{たち|い}でつ、
\ruby{常}{つね}の
\ruby{如}{ごと}く
\ruby{正}{たゞ}しくおのが
\ruby{職務}{つと|め}を
\ruby{執}{と}りぬ。

\ruby{此}{こ}の
\ruby{日天}{ひ|そら}は
\ruby{曇}{くも}りたれども
\ruby{風無}{かぜ|な}く、
\ruby{五十子}{い|そ|こ}の
\ruby{容態}{よう|だい}は
\ruby{晝}{ひる}も
\ruby{佳}{よ}く
\ruby{黄昏}{く|れ}も
\ruby{佳}{よ}かりければ、
\ruby{水野}{みづ|の}は
\ruby{愁}{うれひ}の
\ruby{眉}{まゆ}をも
\ruby{聊}{いさゝ}か
\ruby{開}{ひら}きて、
\ruby{憂}{う}きが
\ruby{中}{なか}にも
\ruby{心樂}{こゝろ|たの}しさをおぼえ、
\ruby{特}{こと}に
\ruby{明日}{あ|す}は
\ruby[g]{休暇日}{やすみび}の
\ruby[g]{土曜}{どやう}といふに、ひとしほゆつたりと
\ruby{氣}{き}を
\ruby{寛}{くつろ}げて、
\ruby{夜}{よ}は
\ruby{靜}{しづか}なる
\ruby{草屋}{くさ|のや}の
\ruby{秋}{あき}に、
\ruby{熒々}{けい|〳〵}たる
\ruby{孤燈}{こ|とう}の
\ruby{前}{まへ}、
\ruby{机}{つくゑ}に
\ruby{慿}{よ}つて
\ruby{端座}{たん|ざ}し、
\ruby{萬斛}{ばん|こく}の
\ruby{胸}{むね}の
\ruby{思}{おもひ}を
\ruby{忘}{わす}れんとてや、
\ruby{一巻}{いつ|くわん}の
\ruby{書}{しよ}に
\ruby{精神}{こゝ|ろ}を
\ruby{潜}{ひそ}めて、つく〴〵と
\ruby{讀}{よ}み
\ruby{入}{い}つたる
\ruby{其}{そ}の
\ruby{風{\換字{情}}}{ふ|ぜい}は、
\ruby{雷電}{らい|でん}こゝに
\ruby{落}{お}ちかゝるとも
\ruby{露知}{つゆ|し}らで
\ruby{{\換字{過}}}{す}ごすべき
\ruby{狀態}{あり|さま}にて、
\ruby{身}{み}は
\ruby{深山}{しん|ざん}の
\ruby{岩室}{いは|むろ}に
\ruby{入定}{にふ|ぢやう}したる
\ruby{昔}{むかし}の
\ruby{權者}{ごん|じや}の、
\ruby{形骸壊}{かた|ち|くず}れず
\ruby{在}{あ}るが
\ruby{如}{ごと}くに
\ruby{動}{うご}かず、
\ruby{眼}{まなこ}は
\ruby{{\換字{寒}}潬}{かん|たん}に
\ruby{影}{かげ}を
\ruby{宿}{やど}せる
\ruby{霜夜}{しも|よ}の
\ruby{星}{ほし}と
\ruby{光}{ひか}り
\ruby{澄}{す}みつ、
\ruby{世}{よ}に
\ruby{何物}{なに|もの}のあるをも
\ruby{忘}{わす}れて、
\ruby{花{\換字{咲}}}{はな|さ}かば
\ruby{{\換字{咲}}}{さ}け、
\ruby{花}{はな}をも
\ruby{眺}{なが}めじ、
\ruby{{\換字{雪}}}{ゆき}ふらばふれ、
\ruby{{\換字{雪}}}{ゆき}にも
\ruby{興}{きよう}ぜじと
\ruby{云}{い}はぬばかりに
\ruby{念}{おもひ}を
\ruby{{\換字{専}}}{もつぱら}にし、
\ruby{{\換字{平}}生}{ひご|ろ}の
\ruby{水野某}{みづ|の|なにがし}の
\ruby{性質}{もち|まへ}を
\ruby{現}{あらは}して、
\ruby{凍}{こほ}りたる
\ruby{水}{みづ}の
\ruby{流}{なが}れぬが
\ruby{如}{ごと}く、いつまでもかくてあるべき
\ruby{樣子}{やう|す}に
\ruby{見}{み}えたり。


\Entry{其六}

% メモ 校正終了 2024-04-16
\原本頁{33-3}%
『
せーんせい!。
』

\原本頁{33-4}%
\ruby{水野}{みづ|の}は
\ruby{振}{ふり}
\ruby{{\換字{返}}}{かへ}りて
\ruby{見}{み}れば
\ruby{間}{あひ}の
\ruby{襖}{ふすま}は
\ruby{開}{あ}き
\ruby{居}{ゐ}て、
%
そこに
\ruby{身體}{から|だ}を
\ruby{{\換字{半}}{\換字{分}}}{はん|ぶん}
\ruby{此方}{こな|た}の
\ruby{燈}{ひ}に
\ruby{見}{み}せつ、
%
お
\ruby{濱}{はま}は
\ruby{我}{わ}が
\ruby{方}{かた}を
\ruby{打}{うち}
\ruby{護}{まも}り
\ruby{居}{ゐ}たり。

\原本頁{32-6}%
『
あ〻% 原本通り「〻(二の字点、揺すり点)」
\ruby{吃驚}{びつ|くり}した!。
%
\ruby{何}{なん}だ{{\換字{𛀁}}}?、
%
お
\ruby{濱}{はま}ちやん、
%
\ruby{突然}{だし|ぬけ}に
\ruby{其樣}{そ|ん}な
\ruby{大}{おほき}な
\ruby{聲}{こゑ}をして!。
』

\原本頁{33-8}%
\ruby{頭髮}{か|み}を
\ruby{結}{むす}ばずして
\ruby{後方}{うし|ろ}に
\ruby{下}{さ}げたれば、
%
ひとしほ
\ruby{兒童}{こ|ども}らしく
\ruby[||j>]{活}{くわつ}
\ruby[||j>]{潑}{ ぱつ}に
% \ruby{活潑}{くわつ|ぱつ}に
\ruby{見}{み}ゆる
\ruby{面}{おもて}の、
%
\ruby{小}{ちひ}さけれど
\ruby{淸}{すゞ}しき% TODO 原本の「二の字点、揺すり点」に濁点のグリフが見つからないので「ゞ」
\ruby{眼}{め}を
\ruby{出來}{で|き}るたけ
\ruby{見張}{み|は}りて、

\原本頁{33-10}%
『
あら、
%
\ruby{先生}{せん|せい}〳〵つて
\ruby{幾度}{いく|ど}
\ruby{呼}{よ}んだか
\ruby{知}{し}れやしませんのに、
%
ホヽ
\ruby{先生}{せん|せい}が
\ruby{{\換字{又}}}{また}
\ruby{夢中}{む|ちう}になつて
\ruby{居}{ゐ}らしつたんだは。
』

\原本頁{34-2}%
と、
%
お
\ruby{濱}{はま}は
\ruby{憚}{はゞか}り% 「憚 は(ゞ)か」% TODO 原本の「二の字点、揺すり点」に濁点のグリフが見つからないので「ゞ」
\ruby{無}{な}く
\ruby{事實}{ま|こと}を
\ruby{語}{かた}りて、
%
\ruby{却}{かへ}つて
\ruby{水野}{みづ|の}を
\ruby{{\換字{難}}}{なん}じ
\ruby{反}{かへ}しぬ。

\原本頁{34-3}%
『
\ruby{左樣}{さ|う}かエ、
%
それぢやあ
\ruby{私}{わたし}が
\ruby{惡}{わる}かつた、
%
\ruby{勘{\換字{忍}}}{かん|にん}〳〵!。% 原文通り「勘忍」
%
そして
\ruby{何}{なに}か
\ruby{用}{よう}?、
%
\ruby{用}{よう}ぢや
\ruby{無}{な}いの?。
』

\原本頁{34-5}%
『
\ruby{御爺}{お|ぢい}さんが\換字{子}、
%
\ruby{番茶}{ばん|ちや}ですが
\ruby{出來}{で|き}ましたから
\ruby{御飮}{お|あが}りなさいませんか
\ruby{御茶}{お|ちや}
\ruby{受}{うけ}は
\ruby{柴栗}{しば|ぐり}の
\ruby{煠}{ゆ}でたの
ばつかり
ですけれども、
%
\ruby{御茶}{お|ちや}でも
あがつて、
%
そして
\ruby{餘}{あんま}り
\ruby{根氣}{こ|ん}を
\ruby{御詰}{お|つ}めなさらないで、
%
もう
\ruby{御休息}{お|やす|み}なすつた
\ruby{方}{はう}が
\ruby{宜}{よ}うございましやうツて!。
』

\原本頁{34-9}%
『
\ruby{左樣}{さ|う}!。
%
そりやあ
\ruby{有}{あ}り
\ruby{{\換字{難}}}{がた}う!。
%
それぢや
\ruby{其方}{そつ|ち}へ
\ruby{行}{い}つて
\ruby{御馳走}{ご|ち|そう}にならうが、
%
\ruby{栗}{くり}は
お
\ruby{濱}{はま}ちやんが
\ruby{剝}{む}いて
\ruby{吳}{く}れるのかエ。
』

\原本頁{34-11}%
『
いやよ、
%
ずるい
\ruby{事}{こと}\換字{子}エ
\ruby{先生}{せん|せい}は。
%
アヽ
\ruby{好}{い}いは、
%
\ruby{妾}{わたし}が
\ruby{剝}{む}いたのは
\ruby{先生}{せん|せい}に
あげますから、
%
\ruby{先生}{せん|せい}も
\ruby{妾}{わたし}に
\ruby{剝}{む}いて
\ruby[||j>]{頂}{ちやう}
\ruby[||j>]{戴}{ だい}ナ。
% \ruby{頂戴}{ちやう|だい}ナ。
』

\原本頁{35-2}%
\ruby{互}{たがひ}に
\ruby{戱}{たはむ}れて
\ruby{言}{ものい}ひながら、
%
お
\ruby{濱}{はま}は
\ruby{縋}{すが}るやうに
\ruby{水野}{みづ|の}の
\ruby{手}{て}を
\ruby{取}{と}つて
\ruby{誘}{いざな}へば、
%
\ruby{水野}{みづ|の}は
また
\ruby{扶}{たす}くるが
\ruby{如}{ごと}く
お
\ruby{濱}{はま}を
あしらひて、
%
\ruby{共}{とも}に
\ruby{直}{たゞち}に% TODO 原本の「二の字点、揺すり点」に濁点のグリフが見つからないので「ゞ」
\ruby{茶}{ちや}の
\ruby{間}{ま}に
\ruby{至}{いた}るに、
%
\ruby{果}{はた}して
\ruby{焙}{ほう}じたる
\ruby{茶}{ちや}の
\ruby{香}{かほり}は
\ruby{一室}{いつ|しつ}に
\ruby{充}{み}ち
\ruby{滿}{み}ちたり。
%
\原本頁{35-5}\改行%
\ruby{三人}{さん|にん}は
\ruby{一}{ひと}ツ
\ruby{燈}{ひ}の
\ruby{下}{もと}に
\ruby{鼎}{かなへ}に
\ruby{坐}{すわ}りて、
%
\ruby{互}{たがひ}に
\ruby{其}{そ}の
\ruby{淸}{きよ}らに
\ruby{和}{やさ}しき
\ruby{心}{こ〻ろ}より% 原本通り「〻(二の字点、揺すり点)」
\ruby{溢}{あふ}る〻% 原本通り「〻(二の字点、揺すり点)」
\ruby{何}{なん}とは
\ruby{無}{な}しの
\ruby{微笑}{ほ〻|ゑみ}を% 原本通り「〻(二の字点、揺すり点)」
\ruby{取}{と}り
\ruby{換}{かは}しつ、
%
\ruby{言}{い}はず
\ruby{語}{かた}らずの
\ruby{中}{うち}に
\ruby{何事}{なに|ごと}も
\ruby{無}{な}き
\ruby{此夜}{この|よ}の
\ruby{靜}{しづか}さを
\ruby{相}{あひ}
\ruby{悅}{よろこ}べり。

\原本頁{35-8}%
もとより
\ruby{廣}{ひろ}からぬ
\ruby{家}{いへ}の
\ruby{事}{こと}なり、
%
\ruby{吉右衛門}{きち||ゑ|もん}は
\ruby{二人}{ふた|り}の
\ruby[||j>]{應}{うけ}
\ruby[||j>]{答}{こたへ}を
% \ruby{應答}{うけ|こたへ}を
\ruby{悉}{こと〴〵}く
\ruby{聞}{き}きたれば、

\原本頁{35-10}%
『
また
\ruby{先生}{せん|せい}に
\ruby{甘}{あま}つたれるよ。
%
\ruby{先生}{せん|せい}に
\ruby{剝}{む}いて
\ruby{戴}{いたゞ}いて% TODO 原本の「二の字点、揺すり点」に濁点のグリフが見つからないので「ゞ」
\ruby{食}{た}べやうなんて、
%
お
\ruby{{\換字{前}}}{まへ}のやうに
\ruby{{\換字{遠}}慮}{ゑん|りよ}を
\ruby{知}{し}らない
\ruby{女}{こ}は
\ruby{有}{あ}りやあ
\ruby{仕}{し}ない!。
%
ハヽヽヽ、
%
さあお
\ruby{茶}{ちや}を
\ruby{御}{お}あげ、
%
\ruby{栗}{くり}も
\ruby[<j||]{汝}{おまへ}
\ruby{巧}{うま}く
\ruby{剝}{む}けるなら
\ruby{剝}{む}いて
おあげ。
』

\原本頁{36-2}%
と、
%
\ruby{一寸}{ちよ|いと}
\ruby{眞面目}{ま|じ|め}には
\ruby{窘}{たしな}めながら、
%
\ruby{叱}{しか}るが
\ruby{矢張}{や|はり}
\ruby{笑顏}{ゑ|がほ}にて、
%
\ruby{{\換字{更}}}{さら}に
\ruby{叱}{しか}るには
ならぬも
をかし。

\原本頁{36-4}%
『
イヤ、
%
ほんとは
\ruby{栗}{くり}は
\ruby{剝}{む}いて
\ruby{貰}{もら}は
なくつても
\ruby{澤山}{たく|さん}だよ。
%
お
\ruby{濱}{はま}ちやん!。
%
\ruby{危}{あぶな}い
\ruby{手}{て}つきか
\ruby{何}{なん}かで
もつて
\ruby{剝}{む}いて
\ruby{貰}{もら}つて、
%
\ruby{指}{ゆび}でも
\ruby{負傷}{け|が}を
されやうもんなら
\ruby{大變}{たい|へん}
だから\換字{子}エ。
』

\原本頁{36-7}%
かくいふ
\ruby{間}{ま}に
お
\ruby{濱}{はま}は
\ruby{其}{そ}の
\ruby{香}{かう}ばしき
\ruby{茶}{ちや}を
\ruby{茶碗}{ちや|わん}に
\ruby{注}{つ}ぎて、
%
\ruby{一箇}{ひと|つ}は% 「箇(7B87)」
\ruby{水野}{みづ|の}の
\ruby{{\換字{前}}}{まへ}、
%
\ruby{一箇}{ひと|つ}は% 「箇(7B87)」
\ruby{祖{\換字{父}}}{ぢ|ゞ}の% TODO 原本の「二の字点、揺すり点」に濁点のグリフが見つからないので「ゞ」
\ruby{{\換字{前}}}{まへ}に
\ruby{差}{さ}し
\ruby{置}{お}けば、

\原本頁{36-9}%
『
ぢやあ
\ruby{御{\換字{勝}}手}{ご|かつ|て}に、
』

\原本頁{36-10}%
と、
%
\ruby{小}{ちひさ}き
\ruby{笊籬}{ざ|る}に
\ruby{入}{い}れたる
\ruby{栗實}{く|り}の
\ruby{今}{いま}
\ruby{煠}{ゆ}で
\ruby{上}{あ}げし
ばかりと
\ruby{見}{み}{\換字{𛀁}}て
\ruby{{\換字{猶}}}{なほ}
\ruby{其}{そ}の
\ruby{皮}{かわ}の% 原本通り「皮 か(わ)」
\ruby{蒸氣}{ゆ|げ}に
\ruby{濕}{しめ}れるに
\ruby{小刀}{こ|がたな}
\ruby{添}{そ}へて
\ruby{{\換字{盆}}}{ぼん}に
\ruby{載}{の}せたるを
\ruby{主人}{ある|じ}は
\ruby{差}{さ}し
\ruby{出}{だ}しぬ。

\原本頁{37-2}%
『
い〻わ、% 原本通り「〻(二の字点、揺すり点)」
%
\ruby{先生}{せん|せい}!\inhibitglue{}%
そんな
\ruby{事}{こと}を
\ruby{云}{い}つて!。
%
\ruby{澤山}{たく|さん}でも
\ruby{何}{なん}でも
\ruby{剝}{む}いて
\ruby{上}{あ}げますよ。
%
\ruby{危}{あぶな}つかしい
\ruby{手}{て}つきだなんて
\ruby{云}{い}つたから
\ruby{{\換字{猶}}}{なほ}
\ruby{剝}{む}いて
あげるわ。
%
さうして
\ruby{{\換字{若}}}{もし}
\ruby{萬一}{ひよ|つと}
\ruby{負傷}{け|が}を
\ruby{仕}{し}て
\ruby{血}{ち}でも
\ruby{出}{で}たらば、
%
その
\ruby{血}{ち}の
\ruby{着}{つ}いたのも
あげるから
い〻わ。% 原本通り「〻(二の字点、揺すり点)」
』

\原本頁{37-6}%
『
あ〻、% 原本通り「〻(二の字点、揺すり点)」
%
もう
あやまつた、
%
\ruby{怒}{おこ}つちやあ
いけない。
%
\ruby{私}{わたし}が
\ruby{二}{ふた}ツ
\ruby{三}{み}ツ
\ruby{剝}{む}いて
あげるから
\ruby{中直}{なか|なお}り
\ruby{中直}{なか|なお}り!。
』

\原本頁{37-8}%
『
ナアに
\ruby{優}{やさ}しくなさると
\ruby{{\換字{猶}}}{なほ}
\ruby[||j>]{增}{ぞう}
\ruby[||j>]{長}{ちやう}します。
% \ruby{增長}{ぞう|ちやう}します。
%
そんな
\ruby{下}{くだ}らない
\ruby{事}{こと}を
\ruby{云}{い}つたのを
とツこに、
%
\ruby{指先}{ゆび|さき}が
\ruby{痛}{いた}くなつて
\ruby{困}{こま}る
\ruby[<j||]{位}{くらゐ}
\ruby{剝}{む}かせて
\ruby{御{\換字{遣}}}{お|や}んなさる
\ruby{方}{はう}が
\ruby{宜}{よ}うございますのに。
%
ハヽヽ。
』

\原本頁{37-11}%
『
ハヽヽ、
%
\ruby[||j>]{憫}{かあ}
\ruby[||j>]{然}{いさう}に!。% 「憫然 か(あ)いさう」
% \ruby{憫然}{かあ|いさう}に!。% 「憫然 か(あ)いさう」
%
お
\ruby{濱}{はま}ちやんも
\ruby{御爺}{お|ぢい}さんに
\ruby{會}{あ}つちやあ
\ruby{敵}{かな}はない\換字{子}。
』

\原本頁{38-2}%
『
いや
もう
\ruby{然樣}{さ|う}では
ございません、
%
\ruby{此女}{こ|れ}には
\ruby{老夫}{おや|ぢ}の
\ruby{方}{はう}が
\ruby{始{\換字{終}}}{しじ|う}% ルビは原本通り「ゆ」無し
\ruby{{\換字{弱}}}{よわ}らされます。
%
\ruby{談話}{はな|し}を
しろ
\ruby{談話}{はな|し}を
\ruby{仕}{し}ろつて
\ruby{{\換字{強}}{\換字{請}}}{せ|が}みまして\換字{子}。
%
\ruby{自{\換字{分}}}{じ|ぶん}が
\ruby{散々}{さん|〴〵}に
\ruby{書}{ほん}を
\ruby{讀}{よ}んで
\ruby{置}{お}いて、
%
まだ
\ruby{其}{その}
\ruby{上}{うへ}に
\ruby{其}{そ}の
\ruby{談話}{はな|し}を
\ruby{仕}{し}ろつて
\ruby{責}{せ}めるんですもの。
』

\原本頁{38-6}%
『
あら
\ruby{御爺}{お|ぢい}さん、
%
そりやあ
\ruby{{\換字{過}}日}{こな|ひだ}の
\ruby{晩}{ばん}
ばかりだは。
%
ありやあ
\ruby{書}{ほん}が
むづかしくつて
\ruby{妾}{わたし}にやあ
\ruby{{\換字{分}}}{わか}らなかつた
からだは。
』

\原本頁{38-8}%
『
\ruby{一體}{いつ|たい}
\ruby{何}{なん}の
\ruby{書}{ほん}だつたの?。
』

\原本頁{38-9}%
『
いやな
\ruby{書}{ほん}だつたの!。
』

\原本頁{38-10}%
『
\ruby{{\換字{嫌}}}{いや}な
\ruby{書}{ほん}てまあ、
%
\ruby{何}{なん}といふ
\ruby{書}{ほん}?。
』

\原本頁{38-11}%
『
お
\ruby{爺}{ぢい}さん、
%
\ruby{默}{だま}つて
\ruby{居}{ゐ}てよ。
%
\ruby{云}{い}はないで
\ruby{居}{ゐ}てよ!。
\ruby{妾}{わたし}あ
たゞ% TODO 原本の「二の字点、揺すり点」に濁点のグリフが見つからないので「ゞ」
\ruby{本家}{ほん|け}から
\ruby{手當}{て|あた}り
\ruby{次第}{し|だい}に
\ruby{持}{も}つて
\ruby{來}{き}たばかしで、
%
\ruby{別}{べつ}に
\ruby{彼書}{あ|れ}を
\ruby{讀}{よ}もうつて
\ruby{持}{も}つて
\ruby{來}{き}たんぢや
\ruby{無}{な}かつたんだから。
』

\原本頁{39-3}%
『
ハテナ、
%
\ruby{匿}{かく}されると
\ruby{{\換字{猶}}}{なほ}
\ruby{聞}{き}きたいが
\ruby{何}{なん}の
\ruby{書}{ほん}だらう?。
』

\原本頁{39-4}%
『
イヤ
\ruby{新}{あたら}しい
\ruby{活版刷}{くわつ|ぱん|ずり}の
\ruby{西洋綴}{せい|やう|とぢ}の
\ruby{書}{ほん}にやあ
\ruby{彼樣}{あ|ん}なものは
よもや
\ruby{入}{はい}つて
\ruby{居}{ゐ}まいと
\ruby{思}{おも}つて
\ruby{居}{ゐ}ましたが。
%
\ruby{飛}{とん}でも
\ruby{無}{な}い
\ruby{書}{ほん}が
\ruby{入}{はい}つて
\ruby{居}{ゐ}ましたのさ。
%
あの
\ruby{帝國{\換字{文}}庫}{てい|こく|ぶん|こ}とかいふ
\ruby{大}{おほき}な
\ruby{本}{ほん}にでさア。
』

\Entry{其七}

『いや
\ruby{古}{ふる}い
\ruby{本}{ほん}が
\ruby{新}{あたら}しくなつて
\ruby{澤山出}{たく|さん|で}るからね。
\ruby{左樣}{さ|う}して
\ruby{其}{そ}の
\ruby{書}{ほん}は
\ruby{何}{なん}と
\ruby{云}{い}ふ
\ruby{書}{ほん}だつたの?。
』

『ナアニ、
\ruby{私}{わたし}なんぞが
\ruby{面皰}{にき|び}の
\ruby{出}{で}て
\ruby{居}{ゐ}た
\ruby{二才}{に|さい}の
\ruby{時{\換字{分}}貸本屋}{じ|ぶん|かし|ほん|や}で
\ruby{借}{か}りて
\ruby{讀}{よ}んだことのある
\ruby{人{\換字{情}}本}{にん|じやう|ぼん}で、
\ruby{初}{はじめ}は
\ruby{甚}{ひど}く
\ruby{{\換字{嫌}}}{きら}はれて
\ruby{居}{ゐ}た
\ruby{男}{をとこ}の、
\ruby{其}{そ}の
\ruby{親切}{しん|せつ}が
\ruby{通}{つう}じて
\ruby{思}{おも}ひ
\ruby{思}{おも}はれるやうになるといふ
\ruby{趣向}{ゆく|たて}を
\ruby{書}{か}いた
\ruby{下}{くだ}らないものでした。
』

『ハヽア、それぢやあ
\ruby{二筋{\換字{道}}}{ふた|すぢ|みち}といふのぢやあ
\ruby{無}{な}いか、そんなら
\ruby{何樣}{ど|う}して
\ruby{百年}{ひやく|ねん}も
\ruby{前}{まへ}の
\ruby{古}{ふる}いものだから、いくら
\ruby{{\換字{総}}傍訓}{そう|ふり|がな}があつたにしても、こりやあ
お
\ruby{濱}{はま}ちやんには
\ruby{些}{ちつと}も
\ruby{{\換字{分}}}{わか}らなかつたろう。
\ruby{私等}{わたし|ら}にさへ
\ruby{明瞭}{はつ|きり}とは
\ruby{解}{わか}らないところがあるんだもの!。
』

『ハヽヽ、
\ruby{彼樣}{あ|ん}な
\ruby{書}{もの}がまあ
\ruby{左樣}{さ|う}ですか\換字{子}エ。
\ruby{成程}{なる|ほど}いくら
\ruby{物}{もの}を
\ruby{知}{し}つて
\ruby{居}{ゐ}らしつても
\ruby{御若}{お|わか}いから
\ruby{何樣}{ど|う}も
\ruby{仕方}{し|かた}がありません、
\ruby{御維新此方物事}{ご|いつ|しん|この|かた|もの|ごと}が
\ruby{全然}{すつ|かり}
\ruby{異}{ちが}つて
\ruby{參}{まゐ}りましたから\換字{子}。
さうすると
\ruby{昔}{むかし}の
\ruby{人{\換字{情}}本}{にん|じやう|ぼん}の
\ruby{精}{よ}く
\ruby{{\換字{分}}}{わか}るのは、
\ruby{此席}{こ|ゝ}ぢやあ
\ruby{私}{わたし}ばつかりといふ
\ruby{譯}{わけ}ですか。
ハヽヽ、
\ruby{老夫}{おや|ぢ}もたまにあ
\ruby{貴下}{あな|た}より
\ruby{{\換字{強}}}{つよ}いところがありますカ\換字{子}。
』

『
\ruby{詰}{つ}まらない
\ruby{自慢}{じ|まん}を
\ruby{仕}{し}て!。
をかしな
\ruby{御爺}{お|ぢい}さん!。
どうせ
\ruby{御大名}{お|だい|みやう}の
\ruby{行列}{ぎやう|れつ}なんぞ
\ruby{知}{し}つて
\ruby{居}{ゐ}るのも
\ruby{御爺}{お|ぢい}さんばかりよ。
』

『ハヽヽ、また
\ruby{此}{こ}の
\ruby{老夫}{おぢい|さん}をやりこめるよ。
どうも
\ruby{左樣此頃}{さ|う|この|ごろ}のやうに
\ruby{威勢}{いき|ほひ}が
\ruby{{\換字{強}}}{つよ}くなつては
\ruby{敵}{かな}はないナ。
もう
\ruby{談話}{はな|し}も
\ruby{何}{なに}も
\ruby{仕}{し}てやらないからいゝ。
』

『いゝわ、あんな
\ruby{昔風}{むかし|ふう}の
\ruby{御談話}{お|はな|し}よりも、
\ruby{一昨日}{を|とゝ|ひ}から
\ruby{讀}{よ}んで
\ruby{居}{ゐ}る
\ruby{魯敏孫}{ろ|びん|そん}の
\ruby{御話}{お|はなし}の
\ruby{方}{はう}がいくら
\ruby{面白}{おも|しろ}いか
\ruby{知}{し}れや
\ruby{仕}{し}ない。
』

『
\ruby{魯敏孫}{ろ|びん|そん}の
\ruby{談話}{はな|し}つて、あの
\ruby{漂流記}{へう|りう|き}?。
』

『えゝ
\ruby{左樣}{さ|う}よ、あの
\ruby{魯敏孫}{ろ|びん|そん}
\ruby{漂流記}{へう|りう|き}よ。
』

『
\ruby{左樣}{さ|う}!。
さうして
\ruby{彼書}{あ|れ}が
\ruby{其樣}{そん|な}に
お
\ruby{濱}{はま}ちやんには
\ruby{面白}{おも|しろ}いの?。
』

『
\ruby{何故}{な|ぜ}?。
\ruby{先生}{せん|せい}にやあ
\ruby{彼書}{あ|れ}が
\ruby{面白}{おも|しろ}くないの!。
\ruby{先生}{せん|せい}は
\ruby{魯敏孫}{ろ|びん|そん}を
\ruby{偉}{えら}いとは
\ruby{思}{おも}はなくつて?。
\ruby{妾}{わたし}あ
\ruby{眞實}{ほん|と}に
\ruby{彼}{あ}の
\ruby{人}{ひと}が
\ruby{好}{す}きだわ。
\ruby{海}{うみ}の
\ruby{中}{なか}の
\ruby{小島}{こ|じま}に
\ruby{唯一人}{たつた|ひと|り}で、
\ruby{立派}{りつ|ぱ}に
\ruby{生}{い}きて
\ruby{行}{ゆ}くなあ
\ruby{偉}{えら}いぢやあありませんか。
\ruby{妾}{わたし}あ
\ruby{彼}{あ}の
\ruby{書}{ほん}を
\ruby{讀}{よ}んで
\ruby{斯}{か}う
\ruby{思}{おも}つたわ。
』

『おもしろい\換字{子}エ。
\ruby{何樣}{ど|ん}な
\ruby{事}{こと}を
\ruby{思}{おも}つたエ。
』

『
\ruby{妾}{わたし}も
\ruby{何樣}{ど|う}かした
\ruby{譯}{わけ}で
\ruby{其}{そ}の
\ruby{島}{しま}へ
\ruby{行}{い}つて\換字{子}、さうして
\ruby{彼}{あ}の
\ruby{魯敏孫}{ろ|びん|そん}と
\ruby{一處}{いつ|しよ}に
\ruby{棲}{す}んで、
\ruby{荒}{あら}い
\ruby{事}{こと}は
\ruby{魯敏孫}{ろ|びん|そん}に
\ruby{仕}{し}て
\ruby{貰}{もら}ふ
\ruby{代}{かは}り、こまこまとした
\ruby{事}{こと}は
\ruby{妾}{わたし}が
\ruby{仕}{し}て
\ruby{{\換字{遣}}}{や}つて、
\ruby{晝間}{ひる|ま}は
\ruby{一生懸命}{いつ|しやう|けん|めい}に
\ruby{働}{はたら}いても、
\ruby{夜}{よる}や
\ruby{雨}{あめ}の
\ruby{降}{ふ}つた
\ruby{靜}{しづか}かな
\ruby{日}{ひ}には
お
\ruby{話}{はなし}なんぞ
\ruby{仕}{し}て
\ruby{{\換字{遊}}}{あそ}んで
\ruby{居}{ゐ}たらば、ほんとに
\ruby{何樣}{ど|ん}なにか
\ruby{面白}{おも|しろ}からうかと
\ruby{思}{おも}つたのよ。
』

『ハヽヽ。
また
\ruby{下}{くだ}らないことを
\ruby{云}{い}ひ
\ruby{出}{だ}したナ。
』

『ハヽ、こりやあ
\ruby{面白}{おも|しろ}い
\ruby{面白}{おも|しろ}い!。
ぢやあお
\ruby{濱}{はま}ちやんは
\ruby{魯敏孫}{ろ|びん|そん}の
\ruby{夫人}{おく|さん}になりたいといふんだ\換字{子}。
』

『いやな
\ruby{先生}{せん|せい}\換字{子}エ。
\ruby{夫人}{おく|さん}だなんて!。
\ruby{妾}{わたし}あ
\ruby{他}{ひと}の
\ruby{夫人}{おく|さん}になつたり、
\ruby{他}{ひと}の
\ruby{良人}{ごてい|しゆ}になつたりする
\ruby{人}{ひと}は
\ruby{大{\換字{嫌}}}{だい|きら}ひだわ。
\ruby{妾}{わたし}あ
\ruby{唯}{たゞ}
\ruby{魯敏孫}{ろ|びん|そん}の
\ruby{朋友}{おとも|だち}になつて
\ruby{見度}{み|た}いつて
\ruby{云}{い}つたのだわ。
』

『ハヽ、
\ruby{成程}{なる|ほど}、
\ruby{{\換字{分}}}{わか}つたよ。
\ruby{面白}{おも|しろ}いねエ。
つまりお
\ruby{濱}{はま}ちやんは
\ruby{女}{をんな}
\ruby{魯敏孫}{ろ|びん|そん}になりたいのだらう。
』

『えゝ、
\ruby{左樣}{さ|う}なのよ。
ほんとに
\ruby{左樣}{さ|う}なのよ。
\ruby{眞靑}{まつ|さを}で
\ruby{際涯}{は|て}の
\ruby{無}{な}い
\ruby{大}{おほき}な
\ruby{洋}{うみ}の、
\ruby{塵}{ちり}も
\ruby{何}{なんに}も
\ruby{無}{な}い
\ruby{奇麗}{き|れい}な
\ruby{島}{しま}の
\ruby{中}{なか}で、あの
\ruby{男兒}{をと|こ}らしい
\ruby{魯敏孫}{ろ|びん|そん}と、たつた
\ruby{二人}{ふ|たり}で
\ruby{働}{はたら}いて
\ruby{居}{ゐ}たら、
\ruby{妾}{わたし}あ
\ruby{何樣}{ど|ん}なに
\ruby{好}{い}い
\ruby{心持}{こゝろ|もち}だらうと
\ruby{思}{おも}つて
\ruby{居}{ゐ}るのよ。
』

『これですもの、どうも、
\ruby{呆}{あき}れて
\ruby{仕舞}{し|ま}ひます!。
\ruby{此女}{こ|れ}は
\ruby{取}{と}り
\ruby{{\換字{分}}}{わ}け
\ruby{無茶}{む|ちや}なのでございましやうが、
\ruby{大}{だい}なり
\ruby{小}{しよう}なり
\ruby{明治}{めい|じ}の
\ruby{生兒}{うま|れ}は、
\ruby{悉皆}{みん|な}
\ruby{斯樣}{か|う}なのでございましやうか、まるで
\ruby{昔}{むかし}の
\ruby{女兒}{むすめ|つこ}とは
\ruby{異}{ちが}つて
\ruby{居}{を}ります。
\ruby{二筋{\換字{道}}}{ふた|すぢ|みち}の
\ruby{話}{はなし}を
\ruby{仕}{し}て
\ruby{聞}{き}かせるのも
\ruby{異}{い}なものでしたが、あんまり
\ruby{何樣}{ど|う}いふ
\ruby{譯}{わけ}だ
\ruby{何樣}{ど|う}いふ
\ruby{譯}{わけ}だと
\ruby{煩}{うるさ}く
\ruby{聞}{き}かれましたから、ほんのざつとした
\ruby{筋}{すぢ}だけを
\ruby{話}{はな}して
\ruby{{\換字{遣}}}{や}りましたのに、
\ruby{碌}{ろく}にも
\ruby{{\換字{遂}}}{と}げては
\ruby{聞}{き}きませんので、
\ruby{詰}{つま}らないと
\ruby{一}{ひ}ㇳ
\ruby{口}{くち}に
\ruby{云}{い}つて
\ruby{仕舞}{し|ま}ひましたのも、
\ruby{一體}{いつ|たい}が
\ruby{斯樣}{か|う}いふ
\ruby{調子}{てう|し}ですから
\ruby{無理}{む|り}もありません。
\ruby{實}{じつ}に
\ruby{世}{よ}の
\ruby{中}{なか}は
\ruby{變}{かは}つてまゐりました。
』

『だつて
\ruby{祖父}{お|ぢい}さん!。
\ruby{二筋{\換字{道}}}{あの|ほ|ん}の
\ruby{御話}{お|はなし}は、
\ruby{{\換字{嫌}}}{きら}ひな
\ruby{人}{ひと}が
\ruby{好}{すき}になるなんで、
\ruby{馬鹿}{ば|か}げて
\ruby{居}{ゐ}るんだもの!。
』

『でも
\ruby{其}{それ}が
\ruby{人{\換字{情}}}{にん|じやう}つて
\ruby{云}{い}ふものなんで、まだ
\ruby{中々汝{\換字{達}}}{なか|〳〵|おまへ|たち}にやあ
\ruby{{\換字{分}}}{わか}らないんだよ。
』

『そんな、
\ruby{{\換字{嫌}}}{きら}ひなものが
\ruby{好}{すき}になる
\ruby{人{\換字{情}}}{にん|じやう}なんて、そりやあ
お
\ruby{行列}{ぎやう|れつ}の
\ruby{時{\換字{分}}}{じ|ぶん}の
\ruby{人{\換字{情}}}{にん|じやう}ぢやなくつて?。
』

『
\ruby{生意氣}{なま|い|き}な!。
\ruby{何}{なに}が
\ruby{小児}{こ|ども}の
\ruby{汝}{おまへ}なんぞに
\ruby{未}{ま}だ
\ruby{{\換字{分}}}{わか}るものか!。
』

『だつて
\ruby{幾歳}{いく|つ}になつたつて、
\ruby{妾}{わたし}にや
\ruby{{\換字{分}}}{わか}らないわ。
\ruby{妾}{わたし}や
\ruby{幾歳}{いく|つ}になつたつて、
\ruby{屹度}{きつ|と}
お
\ruby{澤婆}{さは|ばゞあ}は
\ruby{{\換字{嫌}}}{きらひ}で
\ruby{先生}{せん|せい}は
\ruby{好}{す}きだわ。
\ruby{先生}{せん|せい}が
\ruby{{\換字{嫌}}}{きらひ}で
お
\ruby{澤婆}{さは|ばゞあ}が
\ruby{好}{す}きにはなりやあ
\ruby{仕}{し}ないわ。
』


\Entry{其八}

『ハヽヽ。
\ruby{然樣}{さ|う}ムキになつて
\ruby{老夫}{おぢい|さん}に
\ruby{食}{く}つて
\ruby{掛}{かゝ}ることは
\ruby{無}{な}いぢやあ
\ruby{無}{な}いか。
もう
\ruby{可}{い}い、
\ruby{可}{い}い。
とても
\ruby{老夫}{おぢい|さん}は
\ruby{汝}{おまへ}にやあ
\ruby{敵}{かな}はないよ。
しかし
\ruby{汝}{おまへ}がもう
\ruby{二三年}{に|さん|ねん}も
\ruby{年}{とし}をとつて、
\ruby{物事}{もの|ごと}が
\ruby{善}{よ}く
\ruby{解}{わか}つて
\ruby{來}{く}ると、
お
\ruby{澤}{さは}
\ruby{婆}{ばゞあ}だつて
\ruby{其樣}{そん|な}に
\ruby{憎}{にく}くは
\ruby{無}{な}く
\ruby{思}{おも}ふやうになるかも
\ruby{知}{し}れないよ。
\ruby{先生}{せん|せい}だつて
\ruby{{\換字{過}}日}{こな|いだ}までとは
\ruby{異}{ちが}つて、
\ruby{今}{いま}ぢやあもう
お
\ruby{澤}{さは}
\ruby{婆}{ばゞあ}を
\ruby{憎}{にく}いとばかり
\ruby{思}{おも}つては
\ruby{居}{ゐ}らつしやらないやうだもの。
まあ
\ruby{何}{なん}とでも
\ruby{云}{い}つて
\ruby{居}{ゐ}るが
\ruby{宣}{い}い、
\ruby{人{\換字{情}}}{にん|じやう}といふものは
\ruby{年齢}{と|し}さへ
\ruby{老}{と}りやあ
\ruby{解}{わか}る
\ruby{事}{こと}だから。
』

\ruby{我}{わ}が
\ruby{此上無}{この|うへ|な}く
\ruby{好}{す}きなる
\ruby{其人}{その|ひと}の、
\ruby{我}{わ}が
\ruby{此上無}{この|うへ|な}く
\ruby{{\換字{嫌}}}{きら}へる
\ruby{其婆}{その|ばゞ}を
\ruby{憎}{にく}しとのみは
\ruby{思}{おも}ひ
\ruby{居}{を}らじと
\ruby{云}{い}へるを
\ruby{聞}{き}きて、
お
\ruby{濱}{はま}は
\ruby{且}{かつ}は
\ruby{驚}{おどろ}き、
\ruby{且}{かつ}は
\ruby{訝}{いぶか}り、
\ruby{疑惑}{うた|がひ}の
\ruby{眉}{まゆ}を
\ruby{可憐}{かは|ゆ}らしく
\ruby{顰}{ひそ}め
\ruby{頸}{くび}を
\ruby{枉}{ま}げて
\ruby{水野}{みづ|の}の
\ruby{面}{おもて}を
\ruby{覗}{のぞ}き
\ruby{込}{こ}みつ……、

『ほんとなの?、
\ruby{先生}{せん|せい}。
\ruby{先生}{せん|せい}あんな
\ruby{意地惡}{い|じ|わる}な
\ruby{惡}{にく}らしい
\ruby{老婆}{おばあ|さん}が
\ruby{好}{すき}になつたの?。
』

と、さも〳〵
\ruby{然}{しか}らずといふ
\ruby{答}{こたへ}を
\ruby{聞}{き}きて、
\ruby{改}{あらた}めて
\ruby{{\換字{又}}}{また}
\ruby{我}{わ}が
\ruby{祖父}{そ|ふ}に
\ruby{對}{むか}ひて
\ruby{{\換字{勝}}}{か}ち
\ruby{誇}{ほこ}りたげに
\ruby{{\換字{尋}}}{たづ}ねたり。

\ruby{水野}{みづ|の}は
\ruby{先刻}{さつ|き}より
\ruby{小刀}{こが|たな}をもて
\ruby{心長}{こゝろ|なが}く
\ruby{叮嚀}{てい|ねい}に
\ruby{栗}{くり}を
\ruby{剝}{む}きつゝ、
\ruby{既}{すで}に
\ruby{世}{よ}に
\ruby{老}{お}いたる
\ruby{{\換字{翁}}}{おきな}と
\ruby{未}{ま}だ
\ruby{世}{よ}を
\ruby{知}{し}らぬ
\ruby{少女}{をと|め}との、
\ruby{彼方}{かな|た}は
\ruby{經驗}{おぼ|\換字{𛀁}}に
\ruby{頼}{よ}り
\ruby{此方}{こな|た}は
\ruby{{\換字{空}}想}{おも|ひ}に
\ruby{任}{まか}せて、
\ruby{相和}{あひ|わ}せぬ
\ruby{談}{はなし}を
\ruby{{\換字{交}}}{まじ}ふるをば、おのづから
\ruby{催}{もよほ}さるゝ
\ruby{微笑}{ほゝ|ゑみ}の
\ruby{間}{うち}に
\ruby{聞}{き}き
\ruby{居}{ゐ}たりしが、
\ruby{恰}{あたか}も
\ruby{此時奇麗}{この|とき|ゝ|れい}に
\ruby{剝}{む}き
\ruby{{\換字{終}}}{をは}りし
\ruby{一箇}{ひと|つ}の
\ruby{栗}{くり}を、そつと
お
\ruby{濱}{はま}が
\ruby{掌}{て}の
\ruby{上}{うへ}に
\ruby{載}{の}せてやりつ、

『なにも
\ruby{好}{すき}になつたといふ
\ruby{事}{こと}は
\ruby{無}{な}いのだけれども、そりやあ
\ruby{憎}{にく}いとばかりも
\ruby{思}{おも}つては
\ruby{居}{ゐ}ない。
\ruby{考}{かんが}へて
\ruby{見}{み}ると
\ruby{今}{いま}では
\ruby{憫然}{かあい|さう}でならないやうな
\ruby{氣}{き}さへする
\ruby{位}{くらゐ}だから。
』

と
\ruby{優}{やさ}しく
\ruby{答}{こた}へて、

『お
\ruby{濱}{はま}ちやんだつて
\ruby{今}{いま}に
\ruby{彼}{あ}の
お
\ruby{澤}{さは}の
\ruby{腹}{おなか}の
\ruby{中}{なか}が
\ruby{合點}{が|てん}が
\ruby{行}{ゆ}けば、
\ruby{彼婆}{あ|れ}を
\ruby{憎}{にく}らしいとは
\ruby{思}{おも}はないやうになるかも
\ruby{知}{し}れないよ。
』

と
\ruby{語}{ことば}を
\ruby{足}{た}したり。

\ruby{水野}{みづ|の}が
\ruby{此語}{この|ことば}は
\ruby{如何}{い|か}ばかり
\ruby{思}{おもひ}の
\ruby{外}{ほか}なりけん、
お
\ruby{濱}{はま}は
\ruby{呆}{あき}れたる
\ruby{眼}{め}を
\ruby{睜}{みは}つて
\ruby{默}{だま}りけるが、
\ruby[g]{吉右衛門}{きちゑもん}は
\ruby{待設}{まち|まう}けしやうに
\ruby{言}{ことば}を
\ruby{挿}{さしはさ}みぬ。

『それ
\ruby{御覧}{ご|らん}、
\ruby{老夫}{おぢい|さん}の
\ruby{言}{い}ふ
\ruby{事}{こと}も
\ruby{嘘}{うそ}ぢやあ
\ruby{有}{あ}るまい。
\ruby{好}{す}きなものが
\ruby{{\換字{嫌}}}{きらひ}になつたりもすれば
\ruby{{\換字{嫌}}}{きらひ}なものが
\ruby{好}{す}きになつたりもする、それは
\ruby[<h||]{皆}{みんな}
\ruby{人{\換字{情}}}{にん|じやう}といふものが
\ruby{爲}{さ}せるんで、まだ
\ruby{中々}{なか|〳〵}
\ruby{汝{\換字{達}}}{おまへ|たち}にやあ
\ruby{{\換字{分}}}{わか}らない
\ruby{事}{こと}なんだよ。
』

お
\ruby{濱}{はま}は
\ruby{祖父}{ぢ|ゞ}が
\ruby{言}{ことば}を
\ruby{聞}{き}きもせずして、
\ruby{今}{いま}
\ruby{貰}{もら}ひし
\ruby{栗}{くり}を
\ruby{無邪氣}{む|じや|き}に
\ruby{食}{た}べながら、
\ruby{何事}{なに|ごと}を
\ruby{思}{おも}ひ
\ruby{{\換字{廻}}}{めぐ}らせるならん、あらぬ
\ruby{方}{かた}に
\ruby{眼}{め}を
\ruby{{\換字{留}}}{とど}めて
\ruby{一寸}{ちよ|つと}
\ruby{考}{かんが}へ
\ruby{居}{ゐ}れば、
\ruby{水野}{みづ|の}は
\ruby{{\換字{又}}}{また}
\ruby{樂}{たの}しげに
\ruby{栗}{くり}を
\ruby{剝}{む}き
\ruby{居}{を}り、
\ruby[g]{吉右衛門}{きちゑもん}は
\ruby{{\換字{煙}}草}{たば|こ}を
\ruby{深}{ふか}く
\ruby{吸}{す}ひて
\ruby{{\換字{緩}}}{ゆる}やかに
\ruby{其}{そ}の
\ruby{烟}{けむり}を
\ruby{噴}{ふ}き
\ruby{出}{だ}し
\ruby{居}{を}れり。

\ruby{靜寂}{しづ|か}なりしはたゞ
\ruby{一霎時}{し|ば|し}なりき。
お
\ruby{濱}{はま}は
\ruby{何}{なに}を
\ruby{思}{おも}ひ
\ruby{得}{\換字{𛀁}}しにや
\ruby{忽}{たちま}ち
\ruby{嬉}{うれ}しげなる
\ruby{聲}{こゑ}に
\ruby{淋}{さび}しさを
\ruby{破}{やぶ}つて、

『アヽ
\ruby[<h||]{妾}{わたし}
\ruby{{\換字{分}}}{わか}つてよ、
\ruby[<h||]{妾}{わたし}
\ruby{{\換字{分}}}{わか}つてよ。
\ruby{五十子}{い|そ|こ}さんが
\ruby{今}{いま}に
\ruby{快}{よ}くなるとネエ、
\ruby{屹度}{きつ|と}
\ruby{大變}{たい|へん}に
\ruby{先生}{せん|せい}が
\ruby{好}{す}きになるんでしやう、ホヽヽ、それが
\ruby{人{\換字{情}}}{にん|じやう}つて
\ruby{云}{い}ふものなんでしやう。
\ruby{左樣}{さ|う}ぢやあ
\ruby{無}{な}くつて?、え、
\ruby{祖父}{おぢ|い}さん!。
\ruby{五十子}{い|そ|こ}さんが
\ruby{先生}{せん|せい}を
\ruby{大好}{だい|す}きになる、アヽ
\ruby{左樣}{さ|う}なると
\ruby{好}{い}いわ、
\ruby{早}{はや}く
\ruby{左樣}{さ|う}なると、
\ruby[<h||]{妾}{わたし}
\ruby{五十子}{い|そ|こ}さんを
\ruby{姉}{ね\換字{𛀁}}さんに
\ruby{爲}{し}つちまふから、
\ruby{先生}{せん|せい}が
\ruby{兄}{にい}さんで、
\ruby{五十子}{い|そ|こ}さんが
\ruby{姉}{ね\換字{𛀁}}さんで、さうして
\ruby{妾}{わたし}が
\ruby{其傍}{その|そば}に
\ruby{貼}{つ}いて
\ruby{居}{ゐ}るんなら、ほんとに
\ruby{何樣}{どん|な}に
\ruby{嬉}{うれ}しいか
\ruby{知}{し}れや
\ruby{仕}{し}ないわ。
\ruby{左樣}{さ|う}なれば
\ruby{妾}{わたし}あ
\ruby{魯敏孫}{ろ|びん|そん}の
\ruby{朋友}{おとも|だち}になるのは
\ruby{廃}{よ}して
\ruby{{\換字{終}}}{しま}ふは。
』

と、
\ruby{僞}{いつはり}ならず
\ruby{悅}{よろこ}びて
\ruby{云}{い}ひ
\ruby{出}{だ}したる、
\ruby{面}{おもて}は
\ruby{晴}{は}れやかにして
\ruby{月}{つき}は
\ruby{雲}{くも}なく、
\ruby{{\換字{情}}}{こゝろ}は
\ruby{優}{やさ}しくして
\ruby{花}{はな}に
\ruby{露}{つゆ}あり。

されどお
\ruby{濱}{はま}は
\ruby{{\換字{又}}}{また}たゞちに、

『だけれど、』

と
\ruby{云}{い}ひさして
\ruby{祖父}{ぢ|ゝ}の
\ruby{面}{おもて}を
\ruby{見}{み}たり。
\ruby{水野}{みづ|の}は
お
\ruby{濱}{はま}の
\ruby{言}{ことば}を
\ruby{何}{なに}と
\ruby{聞}{き}きしや、
\ruby{何氣無}{なに|げ|な}き
\ruby{風}{ふう}に
\ruby{身}{み}をも
\ruby{動}{うご}かさず、ひたすらに
\ruby{栗}{くり}を
\ruby{剝}{む}き
\ruby{居}{ゐ}たり。


\Entry{其九}

『だけれども
\ruby{何}{なん}だエ?。
』

お
\ruby{濱}{はま}の
\ruby{言}{い}ひ
\ruby{澱}{よど}みたるを
\ruby{怪}{あやし}みて
\ruby[g]{吉右衛門}{きちゑもん}は
\ruby{輕}{かる}く
\ruby{問}{と}へば、

『だけれども、
\ruby{何}{なん}だか
\ruby{知}{し}らないけれども
\ruby{妾}{わたし}にやあ\換字{子}エ、
\ruby{何樣}{ど|う}も
\ruby{左樣}{さ|う}なりさうも
\ruby{無}{な}いやうな
\ruby{氣}{き}が
\ruby{自然}{ひと|りで}にするのよ。
\ruby{五十子}{い|そ|こ}さんは
\ruby{病氣}{びやう|き}が
\ruby{癒}{なほ}つたらば\換字{子}、
\ruby{{\換字{遠}}}{とほ}い
\ruby{{\換字{遠}}}{とほ}いところへでも
\ruby{行}{い}つて
お
\ruby{仕舞}{し|ま}ひなさりさうな
\ruby{氣}{き}がするのよ。
\ruby{而}{さう}して
\ruby{其後}{その|あと}で
\ruby{松}{まつ}ちやんと
\ruby{妾}{わたし}とが
\ruby{一緖}{いつ|しよ}に
\ruby{泣}{な}くやうな
\ruby{事}{こと}がありさうに
\ruby{思}{おも}ふのよ。
あの
\ruby{椎}{しい}の
\ruby{樹}{き}の
\ruby{暗}{くら}い
\ruby{蔭}{かげ}に、たつた
\ruby{二人}{ふた|り}で
\ruby{淋}{さみ}ーしく
\ruby{殘}{のこ}つて、
\ruby{泣}{な}くやうな
\ruby{事}{こと}になりさうな
\ruby{氣}{き}がするのよ。
』

と
\ruby{{\換字{近}}傍關}{あ|たり|かま}はず
\ruby{言}{い}ひ
\ruby{放}{はな}ちたり。

\ruby{嫩}{わか}き
\ruby{心}{こゝろ}の
\ruby{{\換字{前}}後}{あと|さき}を
\ruby{顧}{かへりみ}ずして、おのが
\ruby{胸}{むね}に
\ruby{{\換字{浮}}}{うか}めるまゝを
\ruby{憚}{はゞか}り
\ruby{氣}{げ}も
\ruby{無}{な}く
\ruby{云}{い}ひ
\ruby{出}{だ}したる
\ruby{其}{それ}は、もとより
\ruby{取}{と}るに
\ruby{足}{た}らぬ
\ruby{空想}{おも|ひ}ながら、
\ruby{戀}{こひ}に
\ruby{心}{こゝろ}の
\ruby{{\換字{弱}}}{よわ}れる
\ruby{人}{ひと}には、
\ruby{幸先}{さい|さき}あしき
\ruby{如是}{かゝ|る}
\ruby{一}{ひ}
ㇳ
\ruby{言}{こと}の
\ruby{如何}{い|か}ばかり
\ruby{氣}{き}に
\ruby{障}{さは}り
\ruby{胸}{むね}に
\ruby{徹}{こた}へやしけんと、
\ruby[g]{吉右衛門}{きちゑもん}はそつと
\ruby{水野}{みづ|の}を
\ruby{見}{み}るに、
\ruby{幸}{さいはひ}にして
\ruby{今}{いま}の
\ruby{言}{ことば}には
\ruby{別}{べつ}に
\ruby{心}{こゝろ}をも
\ruby{動}{うご}かさゞりしやうにて、
\ruby{{\換字{猶}}}{なほ}
\ruby{默々}{もく|〳〵}と
\ruby{栗}{くり}を
\ruby{剝}{む}きつゞけ
\ruby{居}{を}れば、やうやく
\ruby{自{\換字{分}}}{おの|れ}も
\ruby{安}{やす}き
\ruby{思}{おもひ}して、

『イヤ、
\ruby{老夫}{おぢい|さん}には
\ruby{其樣}{そん|な}な
\ruby{氣}{き}は
\ruby{仕}{し}ないよ。
\ruby{五十子}{い|そ|こ}さんが
\ruby{{\換字{遠}}}{とほ}いところへ
\ruby{行}{い}つて
\ruby{仕舞}{し|ま}ふなんて、そりやあ
\ruby{汝}{おまへ}が
\ruby{魯敏孫}{ろ|びん|そん}とかの
\ruby{書}{ほん}を
\ruby{讀}{よ}んだせいで、そんな
\ruby{下}{くだ}らない
\ruby{事}{こと}を
\ruby{思}{おも}ひついたんだらう。
\ruby{老夫}{おぢい|さん}はまた
\ruby{五十子}{い|そ|こ}さんが
\ruby{癒}{なほ}つて、
\ruby{松}{まつ}ちやんだの、
\ruby{汝}{おまへ}だの、
\ruby{島木}{しま|き}さんだのと、みんなが
\ruby{賑}{にぎ}やかに
\ruby{{\換字{遊}}}{あそ}ぶ
\ruby{事}{こと}が、あるやうに
\ruby{思}{おも}つて
\ruby{居}{ゐ}るよ。
』

と
\ruby{老人}{とし|より}の
\ruby{思}{おも}ひ
\ruby{{\換字{遣}}}{や}り
\ruby{深}{ふか}くも
\ruby{祝}{いは}ひ
\ruby{直}{なほ}したり。

\ruby{賢}{かしこ}けれども
\ruby{{\換字{猶}}}{なほ}
\ruby{年若}{とし|わか}ければ、
\ruby{言外}{げん|ぐわい}の
\ruby{其意}{その|こゝろ}は
\ruby{汲}{く}みて
\ruby{知}{し}るに
\ruby{由無}{よし|な}く、

『イヽエ、ちつとも
\ruby{漂流記}{へう|りう|き}の
\ruby{故}{せい}ぢやあ
\ruby{無}{な}いわ。
\ruby{{\換字{過}}日}{こな|ひだ}
\ruby{松}{まつ}ちやんと
\ruby{二人}{ふた|り}で、あの
\ruby{椎}{しひ}の
\ruby{樹}{き}の
\ruby{蔭}{かげ}で
\ruby{話}{はなし}を
\ruby{仕}{し}た
\ruby{其時}{その|とき}から、
\ruby{何}{なん}となく
\ruby{其樣}{そ|ん}な
\ruby{氣}{き}が
\ruby{仕}{し}はじめたのよ。
\ruby{御爺}{お|ぢい}さんこそ
\ruby{屹度}{きつ|と}
\ruby{二筋{\換字{道}}}{ふた|すぢ|みち}が
\ruby{贔負}{ひい|き}だから、
\ruby{彼}{あ}の
\ruby{本}{ほん}のやうになるとばつかし
\ruby{考}{かんが}へて
\ruby{居}{ゐ}るんだわ。
』

とお
\ruby{濱}{はま}が
\ruby{負}{ま}けじ
\ruby{心}{ごゝろ}に
\ruby{云}{い}ひ
\ruby{爭}{あらそ}ふ
\ruby{時}{とき}、
\ruby{今}{いま}まで
\ruby{傍目訝}{よそ|め|いぶか}しきまで
\ruby{沈着}{おち|つき}に
\ruby{沈着}{おち|つ}き
\ruby{居}{い}し
\ruby{水野}{みづ|の}は、

『どつちでもマア
\ruby{宣}{い}いぢやあ
\ruby{無}{な}いか
お
\ruby{濱}{はま}ちやん!。
\ruby{明日}{あし|た}の
\ruby{事}{こと}は
\ruby{明日}{あし|た}の
お
\ruby{天{\換字{道}}樣}{てん|たう|さま}が
\ruby{見}{み}せて
\ruby{下}{くだ}さるわ\換字{子}。
ハヽヽ。
』

と
\ruby{悲}{かな}しげにも
\ruby{無}{な}ければ
\ruby{嬉}{うれ}しげにも
\ruby{無}{な}く、もとより
\ruby{可笑}{を|か}しげにもあらぬ
\ruby{聲}{こゑ}して
\ruby{笑}{わら}ひつゝ
\ruby{制}{せい}し、
\ruby{{\換字{又}}}{また}その
\ruby{掌}{て}の
\ruby{上}{うへ}に
\ruby{剝}{む}きたる
\ruby{栗一}{くり|ひと}ツを、
\ruby{食}{た}べよとばかり
\ruby{優}{やさ}しく
\ruby{置}{お}き
\ruby{{\換字{遣}}}{や}りたり。

『コレ
\ruby{何}{なん}だ!。
\ruby{剝}{む}いたのを
\ruby{先生}{せん|せい}に
\ruby{戴}{いたゞ}くといふものがあるものか。
』

と
\ruby[g]{吉右衛門}{きちゑもん}が
\ruby{眼}{め}の
\ruby{見}{み}つけて
\ruby{叱}{しか}れるは
\ruby{遲}{おそ}く
\ruby{{\換字{緩}}}{ゆる}く、

『いゝわ\換字{子}エ、
\ruby{先生}{せん|せい}!、
\ruby{戴}{いたゞ}いたつて。
』

と
\ruby{云}{い}へる
\ruby{答}{こたへ}は
\ruby{短}{みじか}く
\ruby{捷}{はや}くして、
\ruby{栗}{くり}は
\ruby{既}{すで}に
\ruby{滿面}{まん|めん}に
\ruby{笑}{わらひ}を
\ruby{盛}{も}れる
お
\ruby{濱}{はま}が
\ruby{口裏}{く|ち}に
\ruby{隱}{かく}れたり。

されど
\ruby{何}{なん}としけん
お
\ruby{濱}{はま}は
\ruby{忽地}{たち|まち}にして、
\ruby{其}{そ}の
\ruby{美}{うつく}しき
\ruby{眉}{まゆ}を
\ruby{顰}{ひそ}むれば、

『いゝ
\ruby{氣味}{き|み}、いゝ
\ruby{氣味}{き|み}!。
\ruby{蟲}{むし}が
\ruby{居}{い}たと
\ruby{見}{み}える。
』

と
\ruby{樣子}{やう|す}を
\ruby{見}{み}て
\ruby{取}{と}つて
\ruby[g]{吉右衛門}{きちゑもん}は
\ruby{可笑}{を|か}しがりて
\ruby{笑}{わら}ひ
\ruby{崩}{くづ}れぬ。
\ruby{蟲}{むし}はあらぬ
\ruby{筈}{はず}なるを
\ruby{不思議}{ふ|し|ぎ}の
\ruby{事}{こと}かなと、
\ruby{水野}{みづ|の}は
\ruby{氣}{き}の
\ruby{毒}{どく}さに
お
\ruby{濱}{はま}を
\ruby{打護}{うち|まも}れば、
お
\ruby{濱}{はま}はまた
\ruby{物}{もの}を
\ruby{捜}{さぐ}るが
\ruby{如}{ごと}くに
\ruby{水野}{みづ|の}が
\ruby{手先}{て|さき}に
\ruby{眼}{め}を
\ruby{注}{そゝ}ぎ
\ruby{居}{ゐ}しが、やがて
\ruby{口}{くち}の
\ruby{中}{なか}の
\ruby{物}{もの}を
\ruby{嚥}{の}み
\ruby{{\換字{終}}}{しま}ひて
\ruby{後}{のち}、
\ruby{水野}{みづ|の}が
\ruby{手}{て}をば
\ruby{突然取}{いき|なり|と}りて、

『
\ruby{先生}{せん|せい}、
\ruby{負傷}{け|が}をして
\ruby{居}{ゐ}てよ!。
\ruby{痛}{いた}くなくつて。
』

と
\ruby{示}{しめ}したるを
\ruby{見}{み}れば、
\ruby{左}{ひだり}の
\ruby{拇指}{おや|ゆび}の
\ruby{其腹}{その|はら}に、
\ruby{鮮血}{せん|けつ}いさゝかにじみて
\ruby{臙脂微}\換字{𛀁ん|じ|かすか}に
\ruby{湧}{わ}けり。
\ruby{何}{なに}に
\ruby{心}{こゝろ}をとられて、
\ruby{何時}{い|つ}の
\ruby{間}{ま}にか
\ruby{{\換字{過}}}{あやま}つて
\ruby{傷}{きず}つけて、しかも
\ruby{今}{いま}までは
\ruby{知}{し}らざりけん、
\ruby{全}{まつた}く
\ruby{聊}{いさゝか}か
\ruby{此血}{この|ち}の
\ruby{着}{つ}きたるに
お
\ruby{濱}{はま}は
\ruby{栗}{くり}の
\ruby{味}{あぢはい}を
\ruby{怪}{あやし}みたるなり。

『アヽ
\ruby{穢}{きたな}い
\ruby{事}{こと}をした
\ruby{惡}{わる}かつた!。
\ruby{堪忍}{か|に}しておくれよ
お
\ruby{濱}{はま}ちやん。
ほんとに
\ruby{毫}{すこし}も
\ruby{知}{し}らなかつたのだから。
』

『ナアニ
\ruby{毫}{ちつと}も
\ruby{穢}{きたな}かあ
\ruby{無}{な}いわ。
\ruby{最初妾}{さい|しよ|わたし}が
\ruby{血}{ち}の
\ruby{着}{つ}いたのをあげるなんて、
\ruby{縁起}\換字{𛀁ん|ぎ}でも
\ruby{無}{な}い
\ruby{事}{こと}を
\ruby{云}{い}つたから
\ruby{惡}{わる}かつたのよ。
』

\ruby{瑣細}{さ|さい}の
\ruby{事}{こと}なれど、
\ruby{今}{いま}まで
\ruby{賑}{にぎ}やかに
\ruby{語}{かた}らひし
\ruby{談話}{はな|し}の
\ruby{腰}{こし}はこれに
\ruby{砕}{くだ}けて、
\ruby{何}{なん}となく
\ruby{淋}{さび}しく
\ruby{白}{しら}けたる
\ruby{一室}{ひと|ま}の
\ruby{内}{うち}には、
\ruby{今}{いま}
\ruby{沸}{たぎ}り
\ruby{初}{そ}めでも
\ruby{仕}{し}たるやうに
\ruby{鐵瓶}{てつ|びん}の
\ruby{煮}{に}ゆる
\ruby{音}{おと}の
\ruby{幽}{かす}かに
\ruby{響}{ひび}き
\ruby{出}{だ}して、
\ruby{靜}{しづ}まりかへつたる
\ruby{村}{むた}の
\ruby{夜}{よる}の
\ruby{中}{なか}を、
\ruby{澁江村}{し|ぶ|\換字{𛀁}}との
\ruby{境界}{さ|かひ}あたりにや
\ruby{狗}{いぬ}の
\ruby{吠}{ほ}ゆるが、べう〳〵として
\ruby{遙}{はるか}に
\ruby{聞}{きこ}えぬ。


\Entry{其十}

% メモ 校正終了 2024-04-17 2024-05-30 2024-06-30
\原本頁{56-7}%
\ruby{水野}{みづ|の}
\ruby{語}{かた}らず
\ruby{吉右衛門}{きち||ゑ|もん}
\ruby{言}{ものい}はず、
%
\ruby{瞬}{また〻}かざる% ルビ調整(原本通り)「〻(二の字点、揺すり点)」
\ruby{燈火}{とも|しび}の
\ruby{光}{ひかり}
\ruby{白々}{しろ|〴〵}と
\ruby{冷}{ひや}やかに
\ruby{照}{て}らす
ところ、
%
お
\ruby{濱}{はま}が
\ruby{眼}{め}の
\ruby{{\換字{前}}}{まへ}に
\ruby{動}{うご}けるものは、
%
\ruby{水野}{みづ|の}が
\ruby{指端}{ゆび|さき}を
\ruby{卷}{ま}きたる
\ruby{白紙}{か|み}に、
%
\ruby{知}{し}れるか
\ruby{知}{し}れぬほどづ〻% ルビ調整(原本通り)「〻(二の字点、揺すり点)」
じりゝ〳〵と、
%
\ruby{浸潤}{に|じ}み
\ruby{出}{いだ}して
\ruby{廣}{ひろ}がり
\ruby{行}{ゆ}く
\ruby{鮮血}{せん|けつ}の
\ruby{紅色}{あか|き}のみ。

\原本頁{57-1}%
\ruby{淋}{さみ}しさは
\ruby{今}{いま}
\ruby{人々}{ひと|〴〵}を
\ruby{包}{つ〻}みぬ。% ルビ調整(原本通り)「〻(二の字点、揺すり点)」
%
べう〳〵と
\ruby{鳴}{な}く
\ruby{狗}{いぬ}の
\ruby{聲}{こゑ}は、
%
また
\ruby{遙}{はるか}に
\ruby{{\換字{遠}}}{とほ}くより
こ〻に% ルビ調整(原本通り)「〻(二の字点、揺すり点)」
\ruby{聞}{きこ}え
\ruby{來}{き}ぬ。

\原本頁{57-3}%
お
\ruby{濱}{はま}は
\ruby{{\換字{終}}}{つひ}に
\ruby{淋}{さみ}しさに
\ruby{堪}{た}へ
かねてや、
%
\ruby{心細}{こ〻ろ|ぼそ}けなる% ルビ調整(原本通り)「〻(二の字点、揺すり点)」
\ruby{面色}{おも|〻ち}して、% ルビ調整(原本通り)「〻(二の字点、揺すり点)」

\原本頁{57-4}%
『
あの
\ruby{狗}{いぬ}は
ほんとうに
\ruby{可厭}{い|や}な
\ruby{狗}{いぬ}\換字{子}エー。
%
\ruby{{\換字{過}}日}{こな|ひだ}
\ruby{先生}{せん|せい}が
\ruby{出}{で}て
\ruby{行}{いら}つしやつた
\ruby{夜}{よる}も、
%
\ruby{矢張}{やつ|ぱ}り
\ruby{彼}{あ}の
\ruby{{\換字{通}}}{とほ}りの
\ruby{聲}{こゑ}をして、
%
\ruby{彼}{あ}の
\ruby{見當}{けん|たう}で
\ruby{鳴}{な}いて
\ruby{居}{ゐ}たのよ。
%
そして
\ruby{其}{その}
\ruby{時}{とき}
しーんとして
\ruby{聞}{きい}て
\ruby{居}{ゐ}たらば、
%
\ruby{妾}{わたし}
なんだか
\ruby{悲}{かな}あしく
なつて、
%
\ruby{大變}{たい|へん}に
\ruby{妙}{めう}な
\ruby[||j>]{心}{こ〻ろ}% ルビ調整(原本通り)「〻(二の字点、揺すり点)」
\ruby[||j>]{持}{ もち}
がしたのよ。
』

\原本頁{57-8}%
と
\ruby{云}{い}ひ
\ruby{出}{いだ}せば、

\原本頁{57-9}%
『
また
\ruby{何}{なに}か
\ruby{下}{くだ}らない
\ruby{事}{こと}を
いふ!。
』

\原本頁{57-10}%
と
\ruby{吉右衛門}{きち||ゑ|もん}は
\ruby{打{\換字{消}}}{うち|け}し、

\原本頁{57-11}%
『
\ruby{妙}{めう}な
\ruby[||j>]{心}{こ〻ろ}% ルビ調整(原本通り)「〻(二の字点、揺すり点)」
\ruby[||j>]{持}{ もち}
つて、
%
\ruby{何樣}{ど|ん}な
\ruby[||j>]{心}{こ〻ろ}% ルビ調整(原本通り)「〻(二の字点、揺すり点)」
\ruby[||j>]{持}{ もち}
?。
』

\原本頁{58-1}%
と、
%
\ruby{水野}{みづ|の}は
\ruby{談話}{はな|し}に
\ruby{話}{はな}し
\ruby{甲{\換字{斐}}}{が|ひ}
あらしめんとの
\ruby{意}{こ〻ろ}% ルビ調整(原本通り)「〻(二の字点、揺すり点)」
ばかりに、
%
\ruby{問}{と}はでもの
\ruby{事}{こと}とは
\ruby{思}{おも}ひながら
\ruby{問}{と}ひ
\ruby{{\換字{返}}}{かへ}しぬ。

\原本頁{58-3}%
『
あの\換字{子}、
%
\ruby{疇昔}{むか|し}\換字{子}、
%
\ruby{妾}{わたし}が
ずつと
\ruby{小}{ちひさ}かつた
\ruby{時}{とき}%
{---}{---}%
まだ
\ruby{三歳}{みつ|〻}% ルビ調整(原本通り)「〻(二の字点、揺すり点)」
\ruby{四歳}{よつ|〻}% ルビ調整(原本通り)「〻(二の字点、揺すり点)」
で
\改行% 校正作業の簡略化のため
、
%
\原本頁{58-4}\改行%
\ruby{妾}{わたし}の
\ruby{眞實}{ほん|とう}の
\ruby{御母}{お|つか}さんが
\ruby{生}{い}きて
\ruby{居}{ゐ}た
\ruby{時}{とき}に\換字{子}、
%
\ruby{妾}{わたし}が
お
\ruby{母}{かつ}さんに
\ruby{抱}{だ}かれて
うと〳〵として
\ruby{居}{ゐ}ると、
%
\ruby{{\換字{遠}}}{とほ}くの
\ruby{{\換字{遠}}}{とほ}くの% ルビ調整(原本通り)非踊り字表記
\ruby{方}{はう}で
もつて
\ruby{狗}{いぬ}の
\ruby{鳴}{な}いたのが
\ruby{聞}{きこ}えたのよ。
%
まあ
\ruby{左樣}{さ|う}いふことが
\ruby{有}{あ}つたのだと
\ruby{思}{おも}つて
\ruby[<j||]{頂}{ちやう}% 行末行頭の境界付近なので特例処置を施す
\ruby[||j>]{戴}{だい}よ。
% \ruby{頂戴}{ちやう|だい}よ。
%
そいで\換字{子}エ、
%
\ruby{{\換字{過}}日}{こな|ひだ}の
\ruby{夜}{よる}
あの
\ruby{狗}{いぬ}の
\ruby{聲}{こゑ}を
\ruby{聞}{き}いて
\ruby{思}{おも}ひ
\ruby{出}{だ}して
\ruby{見}{み}ると、
%
あの
\ruby{狗}{いぬ}は
やつぱり
\ruby{其}{そ}の
\ruby{時}{とき}の
\ruby{狗}{いぬ}で、
%
あの
\ruby{聲}{こゑ}も
やつぱり
\ruby{當時}{その|とき}の
\ruby{聲}{こゑ}で、
%
\ruby{而}{さう}して
\ruby{彼}{あ}の
\ruby{狗}{いぬ}の
\ruby{聲}{こゑ}を
\ruby{聞}{き}いて、
%
\ruby{可厭}{い|やー}に
\ruby{淋}{さみ}しいと
\ruby{思}{おも}つた
\ruby{其}{そ}の
\ruby[||j>]{心}{こ〻ろ}% ルビ調整(原本通り)「〻(二の字点、揺すり点)」
\ruby[||j>]{持}{ もち}
も、
%
やつぱり
\ruby{其}{そ}の
\ruby{時}{とき}
\ruby{可厭}{い|やー}に
\ruby{淋}{さみ}しいと
\ruby{思}{おも}つた
\ruby{其}{そ}の
\ruby[||j>]{心}{こ〻ろ}% ルビ調整(原本通り)「〻(二の字点、揺すり点)」
\ruby[||j>]{持}{ もち}
だと
\改行% 校正作業の簡略化のため
、
%
\原本頁{58-11}\改行%
\ruby{思}{おも}へて〳〵
\ruby{仕方}{し|かた}が
\ruby{無}{な}かつたのよ。
』

\原本頁{59-1}%
『
なんだエ、
%
また
\ruby{下}{くだ}らない!。
%
そりやあ
\ruby{氣}{き}の
\ruby{{\換字{所}}爲}{せ|ゐ}と
いふものだ
\改行% 校正作業の簡略化のため
は。
』

\原本頁{59-3}%
\ruby{吉右衛門}{きち||ゑ|もん}が
かく
\ruby{云}{い}ひ
\ruby{{\換字{終}}}{をは}れる
\ruby{時}{とき}、
%
\ruby{狗}{いぬ}は
また
\ruby{遙}{はるか}に
べう〳〵と
\ruby{鳴}{な}けり
\改行% 校正作業の簡略化のため
。

\原本頁{59-4}%
『
ほーら
\ruby{{\換字{又}}}{また}
\ruby{鳴}{な}いてよ
お
\ruby{爺}{ぢい}さん!。
%
\ruby{氣}{き}の
\ruby{{\換字{所}}爲}{せ|ゐ}ぢやあ
\ruby{無}{な}くつてよ
\ruby{眞實}{ほん|と}の
\ruby{事}{こと}よ!。
%
\ruby{今}{いま}
\ruby{鳴}{な}いた
\ruby{彼狗}{あ|れ}は
\ruby{何樣}{ど|う}しても
\ruby{{\換字{過}}日}{こな|ひだ}
\ruby{鳴}{な}いたのよ。
%
\ruby{{\換字{過}}日}{こな|ひだ}
\原本頁{59-6}\改行%
\ruby{鳴}{な}いた
\ruby{彼狗}{あ|れ}は
また
\ruby{妾}{わたし}が
\ruby{大變}{たい|へん}に
\ruby{小}{ちひさ}かつた
\ruby{時}{とき}
\ruby{鳴}{な}いたのかも
\ruby{知}{し}れなく
\改行% 校正作業の簡略化のため
つてよ!。
%
\ruby{而}{さう}して
\ruby{何}{なん}だか
\ruby{妾}{わたし}あ、
%
\ruby{妾}{わたし}の
\ruby{{\換字{前}}}{まへ}の
\ruby{世}{よ}といふ
\ruby{時}{とき}にも、
%
\ruby{矢張}{やつ|ぱ}
\改行% 校正作業の簡略化のため
り
\ruby{此樣}{こ|ん}な
\ruby{淋}{さみ}しい
\ruby{晩}{ばん}に、
%
やつぱり
\ruby{彼樣}{あ|ん}な
\ruby{狗}{いぬ}の
\ruby{聲}{こゑ}を
\ruby{聞}{き}いて、
%
やつぱり
\ruby{妙}{めう}な
\ruby[||j>]{心}{こ〻ろ}% ルビ調整(原本通り)「〻(二の字点、揺すり点)」
\ruby[||j>]{持}{ もち}
が
\ruby{爲}{し}たやうな
\ruby{氣}{き}が
\ruby{仕}{し}てならないのよ!。
%
あ〻% ルビ調整(原本通り)「〻(二の字点、揺すり点)」
\ruby{何}{なん}だか
\原本頁{59-10}\改行%
\ruby{妾}{わたし}あ
ぞく〳〵するやうな
\ruby[||j>]{心}{こ〻ろ}% ルビ調整(原本通り)「〻(二の字点、揺すり点)」
\ruby[||j>]{持}{ もち}
がして、
%
\ruby{變}{へん}に
\ruby{氣味}{き|み}が
\ruby{惡}{わる}くなつて
\ruby{來}{き}て
\ruby{堪}{たま}らないのよ。
%
あら
また
\ruby{鳴}{な}くのネエ、
%
あ〻、% ルビ調整(原本通り)「〻(二の字点、揺すり点)」
%
\ruby{厭}{いや}だこと!。
%
\ruby{萬一}{ひよ|つと}
すると
\ruby{眞實}{ほん|と}に
\ruby{{\換字{前}}}{まへ}の
\ruby{世}{よ}つて
ことが
\ruby{有}{あ}るんぢや
\ruby{無}{な}いか
\ruby{知}{し}らん。
%
\ruby{{\換字{前}}}{まへ}
\原本頁{60-2}\改行%
の
\ruby{世}{よ}つて
いふものが
あるかと
\ruby{思}{おも}ふと、
%
\ruby{何}{なん}だか
\ruby{怖}{こは}いやうな
\ruby{氣}{き}が
するのネエ。
%
\ruby{先生}{せん|せい}は
\ruby{{\換字{前}}}{まへ}の
\ruby{世}{よ}の
あるやうな
\ruby[||j>]{心}{こ〻ろ}% ルビ調整(原本通り)「〻(二の字点、揺すり点)」
\ruby[||j>]{持}{ もち}
は
\ruby{仕}{し}なくつて?。
』

\原本頁{60-4}%
お
\ruby{濱}{はま}が
かく
\ruby{云}{い}ひたる
\ruby{時}{とき}の
\ruby{其}{そ}の
\ruby{面}{おもて}は、
%
\ruby[<j>]{僞}{いつはり}ならず
\ruby{惑}{まどひ}を
\ruby{帶}{お}び
\ruby{怖畏}{おそ|れ}を
\ruby{帶}{お}びて、
%
まことに
\ruby{{\換字{前}}世}{ぜん|せ}と
いふもの〻% ルビ調整(原本通り)「〻(二の字点、揺すり点)」
\ruby{{\換字{空}}}{むな}しからぬを
\ruby{{\換字{感}}}{かん}じて、
%
\ruby{其}{そ}の
\ruby{恐}{おそ}ろしさに
\ruby{魘}{おび}えたるが
\ruby{如}{ごと}し。

\原本頁{60-7}%
\ruby{實}{げ}に
\ruby{思}{おも}へば
\ruby{人}{ひと}は
\ruby{或}{ある}
\ruby{事}{こと}に
あへる
\ruby{時}{とき}、
%
か〻る% ルビ調整(原本通り)「〻(二の字点、揺すり点)」
\ruby{事}{こと}には
\ruby{往時}{むか|し}
\ruby{既}{すで}に
\ruby{一度}{ひと|たび}
\ruby{逢}{あ}ひたる
ことの
ありしと、
%
\ruby{思}{おも}はる〻% ルビ調整(原本通り)「〻(二の字点、揺すり点)」
やうなる
\ruby{心地}{こ〻|ち}の% ルビ調整(原本通り)「〻(二の字点、揺すり点)」
\ruby{爲}{す}る
\ruby{事}{こと}も
\ruby{無}{な}きには
あらぬなり。
%
\ruby{既}{すで}に
\ruby{{\換字{兼}}好}{けん|かう}は
\ruby[||j>]{幾}{いく}
\ruby[||j>]{百}{ひやく}
\ruby[||j>]{年}{ねん}
の
\ruby{昔}{むかし}に、

\原本頁{60-10}%
% \begin{quote}% 原本では引用インデントされていない
『
\ruby{只}{たゞ}% TODO 原本の「二の字点、揺すり点」に濁点のグリフが見つからないので「ゞ」
\ruby{今}{いま}
\ruby{人}{ひと}の
いふことも、
%
\ruby{目}{め}に
\ruby{見}{み}ゆる
ものも、
%
\ruby{我}{わ}が
\ruby{心}{こ〻ろ}の% ルビ調整(原本通り)「〻(二の字点、揺すり点)」
うちも、
%
か〻ることの% ルビ調整(原本通り)「〻(二の字点、揺すり点)」
\ruby{何時}{い|つ}ぞや
\ruby{有}{あ}りしかと
おぼ{\換字{𛀁}}て、
%
いつとは
\ruby{思}{おも}ひ
\ruby{出}{い}でねども、
%
まさしく
ありし
\ruby{心地}{こ〻|ち}のする% ルビ調整(原本通り)「〻(二の字点、揺すり点)」
』
% \end{quote}% 原本では引用インデントされていない
% 徒然草(上)第71段「名を聞くより、やがて、面影は推し測らるゝ心地するを、‥」 の一説
% 今起こっていること、人の言っている事、目に見ているものなど、いつかもあったり、見たりしているというような思い。

\原本頁{61-2}%
とは
\ruby{云}{い}ひたらずや。

\原本頁{61-3}%
\ruby{生}{うま}れぬ
\ruby{{\換字{前}}}{まへ}の
\ruby{世}{よ}の
\ruby{有}{ある}
\ruby{無}{なし}なんどは、
%
もとより
\ruby{凡下}{ぼん|げ}の
\ruby{身}{み}の
\ruby{何}{なん}とも
\ruby{知}{し}らねば、
%
\ruby{吉右衛門}{きち||ゑ|もん}も
\ruby{横合}{よこ|あひ}よりは
\ruby{吻}{くち}を
\ruby{容}{い}れず、
%
\ruby{水野}{みづ|の}は
\ruby{物}{もの}を
\ruby{思}{おも}ひて
\ruby{{\換字{猶}}}{なほ}
\ruby{語}{かた}らざる
\ruby{時}{とき}、
%
ふた〻び% ルビ調整(原本通り)「〻(二の字点、揺すり点)」
べう〳〵と
\ruby{鳴}{な}く、
%
\ruby{狗}{いぬ}の
\ruby{聲}{こゑ}は、

\原本頁{61-6}%
% \begin{quote}% 原本では引用インデントされていない
『
\ruby{我}{われ}は
\ruby{方々}{かた|〴〵}の
\ruby{{\換字{前}}}{まへ}の
\ruby{世}{よ}より
\ruby{既}{すで}に
\ruby{知}{し}りたまへる
\ruby{狗}{いぬ}なるをや!。
』
% \end{quote}% 原本では引用インデントされていない

\原本頁{61-7}%
と
\ruby{告}{つ}ぐるが
\ruby{如}{ごと}くに
\ruby{聞}{きこ}え
\ruby{來}{きた}りぬ。

\Entry{其十一}

\ruby{偶然}{ぐう|ぜん}の
\ruby{事}{こと}とすればそれまでなれども、
\ruby{奇}{あや}しとすれば
\ruby{奇}{あや}しくもあるかな。
かつて
\ruby{我}{わ}が
\ruby{讀}{よ}みし
\ruby{書}{しよ}の
\ruby{中}{うち}に「
\ruby{幻}{ヴイジヨン}と
\ruby{謎}{リツドル}と」といへる
\ruby{一章}{いつ|しやう}ありて、
\ruby{其}{そ}の
\ruby{幽怪神異}{ゆう|くわい|しん|い}の
\ruby[g]{趣味}{おもむき}は、
\ruby{骨身}{ほね|み}に
\ruby{沁}{し}みて
\ruby{忘}{わす}れ
\ruby{難}{がた}く、
\ruby{今}{いま}に
\ruby{鮮明}{あざ|やか}に
\ruby{心頭}{むな|さき}に
\ruby{{\換字{遺}}}{のこ}れる、
\ruby{其}{それ}をお
\ruby{濱}{はま}の
\ruby{知}{し}るべくはあらねど、
\ruby{其}{そ}の
\ruby{言}{い}ふところを
\ruby{聞}{き}けば、
\ruby{何}{なん}ぞ
\ruby{彼}{か}の
\ruby{記}{しる}せるところと
\ruby{相似}{あひ|に}たるや。
たゞ
\ruby{彼}{かれ}は
\ruby{考慮}{かん|がへ}に
\ruby{老}{お}いたる
\ruby{人}{ひと}の
\ruby{言葉}{こと|ば}にして、これは
\ruby{何}{なん}の
\ruby{思案}{し|あん}も
\ruby{無}{な}き
\ruby{少女}{こ|ども}の
\ruby{言葉}{こと|ば}なり、
\ruby{彼}{かれ}は
\ruby{先}{ま}づ
\ruby{思}{おも}ひて
\ruby{後}{のち}に
\ruby{狗}{いぬ}の
\ruby{聲}{こゑ}を
\ruby{聞}{き}き、これは
\ruby{先}{ま}づ
\ruby{狗}{いぬ}の
\ruby{聲}{こゑ}を
\ruby{聞}{き}いて
\ruby{後}{のち}に
\ruby{思}{おも}ひ
\ruby{起}{おこ}せるの
\ruby{差異}{ちが|ひ}こそあれ、おのづからに
\ruby{此}{こ}の
\ruby{年}{とし}ゆかぬ
\ruby{娘}{こ}の、
\ruby{誰{\換字{教}}}{だれ|をし}へぬにか〻る
\ruby{事}{こと}を
\ruby{想}{おも}ひ
\ruby{出}{いだ}せる
\ruby{不思議}{ふ|し|ぎ}さ!。
\ruby{月日}{つき|ひ}は
\ruby{誰}{だれ}の
\ruby{{\換字{所}}有}{も|の}としも
\ruby{無}{な}ければ、
\ruby{仰}{あふ}ぐものは
\ruby{皆其}{みな|そ}の
\ruby{光}{ひかり}を
\ruby{見}{み}、
\ruby{眞理}{ま|こと}は
\ruby{智者}{ち|しや}の
\ruby{{\換字{造}}}{つく}れるにもあらねば、
\ruby[g]{婦女童兒}{をんなこども}の
\ruby{胸}{むね}にも
\ruby{{\換字{浮}}}{うか}みて、
\ruby{我}{われ}からとも
\ruby{無}{な}く
\ruby{如是}{か|く}は
\ruby{悟}{さと}れるにや。
そも〳〵また
\ruby{佛陀}{ほと|け}の
\ruby{{\換字{教}}法}{をし|え}に、いつとなく
\ruby{耳}{みゝ}も
\ruby{心}{こゝろ}も
\ruby{染}{そ}まり
\ruby{居}{ゐ}て、それより
\ruby{然}{さ}る
\ruby{事}{こと}をも
\ruby{思}{おも}へるか。

\ruby{其}{そ}の
\ruby{因}{よ}つて
\ruby{出}{い}でしところは
\ruby{兎}{と}まれ
\ruby{角}{かく}かれ、
\ruby{前}{まへ}の
\ruby{世}{よ}
\ruby{有}{あ}りや
\ruby{將有}{はた|あ}らずや、
\ruby{如何}{い|か}にと
\ruby{問}{と}はれては
\ruby{此}{こ}の
\ruby{我}{われ}もまた、
\ruby{少}{すこし}ばかりの
\ruby{智慧學問}{ち|ゑ|がく|もん}の、
\ruby{果}{はた}して
\ruby{有}{あ}りや
\ruby{{\換字{叉}}}{また}
\ruby{無}{な}しやと
\ruby{蜘蛛手}{く|も|で}に
\ruby{働}{はたら}く
\ruby{其}{そ}の
\ruby{下蔭}{した|かげ}に、
\ruby{私}{ひそか}に
\ruby{前}{まへ}の
\ruby{世}{よ}を
\ruby{有}{あ}るもの〻やう
\ruby{思}{おも}ふ
\ruby{心地}{こゝ|ち}も
\ruby{實}{まこと}は
\ruby{爲}{す}るなり。

\ruby[g]{{\換字{迷}}信}{まよひ}なり、
\ruby[g]{{\換字{迷}}信}{まよひ}なり、
\ruby{古}{ふる}き
\ruby[g]{{\換字{迷}}信}{まよひ}なり、
\ruby{智慧}{ち|ゑ}の
\ruby[g]{光輝}{ひかり}の
\ruby{及}{およ}ばぬ
\ruby{隈}{くま}には、
\ruby{其}{そ}の
\ruby{闇}{くら}さにぞ
\ruby{有}{あ}らぬ
\ruby[g]{現像}{すがた}の
\ruby{思}{おも}ひ
\ruby{{\換字{遣}}}{や}らる〻、
\ruby{其}{それ}を
\ruby{前}{まへ}の
\ruby{世}{よ}とは
\ruby{云}{い}ひならはしたるならすや。
さはあれど、
\ruby{彼}{か}の
\ruby{書}{しよ}に、

\begin{quote}
『
\ruby{爾見}{なんぢ|み}よ、
\ruby{此}{こ}の
\ruby{刹那}{せつ|な}を。
\ruby{刹那}{せつ|な}の
\ruby{此}{こ}の
\ruby{關}{せき}より
\ruby{彼方}{かな|た}には
\ruby{涯無}{かぎ|りな}き
\ruby{路}{みち}の
\ruby[g]{長路}{ながぢ}ぞ
\ruby{遙}{はるか}に
\ruby{亘}{わた}れるなる。
\ruby{刹那}{せつ|な}の
\ruby{關}{せき}より
\ruby[g]{此方}{こなた}にも
\ruby{涯無}{かぎ|りな}き
\ruby{路}{みち}の
\ruby[g]{長路}{ながぢ}ぞ
\ruby{遙}{はるか}に
\ruby{亘}{わた}れるなる。

\ruby{思}{おも}へ
\ruby{爾}{なんぢ}、
\ruby{起}{おこ}りし
\ruby{事}{こと}のかつて
\ruby{此路}{こ|〻}に
\ruby{起}{おこ}りし
\ruby{事}{こと}ならぬやある?。
\ruby{思}{おも}へ
\ruby{爾}{なんぢ}、
\ruby{爲}{な}されし
\ruby{事}{こと}のかつて
\ruby{此路}{こ|〻}になされしならぬやある?。
\ruby{思}{おも}へ
\ruby{爾}{なんぢ}、
\ruby[g]{萬般}{よろづ}の
\ruby{事}{こと}、
\ruby[g]{萬般}{よろづ}の
\ruby{物}{もの}、
\ruby{此}{こ}の
\ruby{路}{みち}に
\ruby{上}{のぼ}り、
\ruby{此}{こ}の
\ruby{關}{せき}を
\ruby{{\換字{過}}}{す}ぎざりしものやある?。

\ruby{物}{もの}の
\ruby{能}{よ}く
\ruby{此}{こ}の
\ruby{路}{みち}に
\ruby{上}{のぼ}るものは、
\ruby{復}{ま}た
\ruby{必}{かなら}ず
\ruby[g]{再度}{ふた〻び}
\ruby{此}{こ}の
\ruby{路}{みち}に
\ruby{上}{のぼ}らん。
\ruby{事}{こと}の
\ruby{能}{よ}く
\ruby{此}{こ}の
\ruby{關}{せき}を
\ruby{{\換字{過}}}{す}ぐるものは
\ruby{復}{ま}た
\ruby{必}{かなら}ず
\ruby{二度}{ふた|ゝび}
\ruby{此}{こ}の
\ruby{關}{せき}を
\ruby{{\換字{過}}}{す}ぎん!。

やをら〳〵
\ruby{月}{つき}の
\ruby{光}{ひかり}に
\ruby{這}{は}へる
\ruby{此}{こ}の
\ruby{蜘蛛}{く|も}!。
\ruby{爾}{なんぢ}
\ruby{思}{ おも}ひ
\ruby{得}{\換字{江}}ずや
\ruby{此}{こ}の
\ruby{蜘蛛}{く|も}の
\ruby{{\換字{過}}去既}{む|かし|すで}に
\ruby{一度世}{ひと|たび|よ}にありしとは。
\ruby{月}{つき}の
\ruby{此}{こ}の
\ruby{光}{ひかり}!、
\ruby{爾}{なんぢ }
\ruby{思}{おも}ひ
\ruby{得}{\換字{江}}ずや
\ruby{月}{つき}の
\ruby{此}{こ}の
\ruby{光}{ひかり}の
\ruby{{\換字{過}}去既}{む|かし|すで}に
\ruby{一度世}{ひと|たび|よ}に
\ruby{在}{あ}りしとは。

\ruby{此}{こ}の
\ruby{關}{せき}に
\ruby{立}{た}ちて
\ruby{囁}{さゝや}きて、
\ruby{共}{とも}に
\ruby{限無}{かぎ|りな}く
\ruby{究無}{きは|みな}きものにつきて
\ruby{囁}{さゝや}ける
\ruby{爾}{なんぢ}よ
\ruby{我}{われ}よ
\ruby{我}{われ}よ
\ruby{爾}{なんぢ}よ、
\ruby{爾}{なんぢ}
\ruby{思}{おも}ひ
\ruby{得}{\換字{江}}ずや
\ruby{我}{われ}も
\ruby{爾}{なんぢ}も
\ruby{{\換字{過}}去既}{む|かし|すで}に
\ruby{一度世}{ひと|たび|よ}に
\ruby{在}{あ}りしとは。

\ruby{爾}{なんぢ}も
\ruby{我}{われ}も、
\ruby{爾}{なんぢ}と
\ruby{我}{われ}との
\ruby{前}{まへ}なる
\ruby{路}{みち}の、
\ruby{長々}{なが|〳〵}しき
\ruby{{\換字{迷}}}{まよひ}の
\ruby{路}{みち}に
\ruby{復現}{また|あら}はれて、
\ruby{爾}{なんぢ}もふた〻び
\ruby{行}{ゆ}き
\ruby{我}{われ}もふた〻び
\ruby{行}{ゆ}き、さてしも
\ruby{限}{かぎ}り
\ruby{無}{な}く
\ruby{究}{きは}み
\ruby{無}{な}き
\ruby[g]{輪{\換字{廻}}}{りんね}の
\ruby{路}{みち}に
\ruby{千度百度往}{ち|たび|も〻|たび|ゆ}き
\ruby{{\換字{返}}}{かへ}らでは
\ruby{叶}{かな}はぬにはあらずや』\end{quote}

とありしも
\ruby{思}{おも}ひ
\ruby{出}{いだ}されて、
\ruby{水野}{みづ|の}は
\ruby{拭}{ぬぐ}へども
\ruby{拭}{ぬぐ}へども
\ruby{沸}{わ}きあがる
\ruby{蒸氣}{ゆ|げ}に、
\ruby{我}{わ}が
\ruby{心}{こゝろ}の
\ruby{鏡}{かゞみ}の
\ruby{曇}{くも}り
\ruby{果}{は}て〻、
\ruby{明}{あき}らかなり
\ruby{得}{\換字{江}}ぬやうの
\ruby{心地}{こゝ|ち}したり。

\ruby{今}{いま}こ〻に
\ruby{我}{われ}には
\ruby{{\換字{尊}}}{たふと}き
\ruby{今}{いま}の
\ruby{世}{よ}のあらずや。
\ruby{有}{あ}りても
\ruby{可}{よ}く
\ruby{無}{な}くても
\ruby{宣}{よ}きは
\ruby{前}{まへ}の
\ruby{世}{よ}ならずや。
\ruby{輪{\換字{廻}}循{\換字{環}}}{りん|ねじ|ゆん|くわん}の
\ruby{談}{だん}は
\ruby{枝葉}{し|\換字{江}ふ}の
\ruby{事}{こと}のみと、
\ruby{水野}{みづ|の}は
\ruby{{\換字{強}}}{し}ひて
\ruby{思}{おも}ひ
\ruby{棄}{す}てんとしけるが、
\ruby{生憎}{あい|にく}に
\ruby{{\換字{猶}}}{なほ}
\ruby{物}{もの}の
\ruby{思}{おも}はる〻を
\ruby{如何}{いか|ん}とも
\ruby{爲難}{し|がた}くて、
\ruby{答}{こた}へもせず
\ruby{獨}{ひと}り
\ruby[g]{空想}{おもひ}に
\ruby{耽}{ふけ}る
\ruby{折}{をり}しも、
\ruby{何}{なに}をか
\ruby{吠}{ほ}ゆる
\ruby{彼}{か}の
\ruby{狗}{いぬ}はまた、べう〳〵と
\ruby{同}{おな}じやうに
\ruby{高}{たか}く
\ruby{鳴}{な}けり。

\ruby{狗}{いぬ}の
\ruby{聲}{こゑ}は
\ruby{淋}{さび}しさの
\ruby{中}{うち}より
\ruby{起}{おこ}こりて
\ruby{淋}{さび}しさの
\ruby{中}{うち}に
\ruby{{\換字{消}}}{き}えたり。
\ruby{水野}{みづ|の}は
\ruby{狗}{いぬ}の
\ruby{聲}{こゑ}の
\ruby{{\換字{消}}}{き}え
\ruby{{\換字{終}}}{をは}りし
\ruby{時}{とき}、ふと
\ruby{眼}{め}をあげてお
\ruby{濱}{はま}を
\ruby{見}{み}れば、お
\ruby{濱}{はま}もまた
\ruby{狗}{いぬ}の
\ruby{聲}{こゑ}の
\ruby{{\換字{消}}}{き}え
\ruby{{\換字{終}}}{をは}りし
\ruby{時}{とき}、
\ruby{物}{もの}おもふ
\ruby{眼}{め}をあげて
\ruby{水野}{みづ|の}を
\ruby{見}{み}たり。

\ruby{生}{うま}れぬ
\ruby{前}{さき}を
\ruby{思}{おも}ひやれる
\ruby{眼}{め}は、
\ruby{生}{うま}れぬ
\ruby{前}{さき}を
\ruby{思}{おも}へる
\ruby{眼}{まなこ}と、ひたりと
\ruby{相會}{あひ|あ}つて、はつと
\ruby{別}{わか}れぬ。
\ruby{水野}{みづ|の}は
\ruby{忽然}{こつ|ぜん}として、
\ruby{我}{わ}が
\ruby{前}{さき}の
\ruby{世}{よ}に、
\ruby{我}{われ}は
\ruby{{\換字{猶}}}{なほ}
\ruby{今}{いま}の
\ruby{我}{われ}の
\ruby{如}{ごと}く、お
\ruby{濱}{はま}は
\ruby{{\換字{猶}}}{なほ}
\ruby{今}{いま}のお
\ruby{濱}{はま}の
\ruby{如}{ごと}くして、しかも
\ruby{我}{わ}が
\ruby{五十子}{い|そ|こ}もまた
\ruby{今}{いま}の
\ruby{五十子}{い|そ|こ}の
\ruby{如}{ごと}く、
\ruby{我}{われ}は
\ruby{今}{いま}と
\ruby{同}{おな}じく
\ruby{苦}{くるし}みあくがれて、
\ruby{甲斐無}{か|ひ|な}くも
\ruby{長}{とこしな}へに
\ruby{忌}{い}み
\ruby{{\換字{嫌}}}{きら}はれたりし、
\ruby{其}{そ}の
\ruby{事}{こと}のまざ〳〵と
\ruby{存}{あ}りしやうに
\ruby{思}{おも}ひて、
\ruby{{\換字{総}}身}{そう|み}の
\ruby{毛根動}{け|あな|うご}けるが
\ruby{如}{ごと}く、
\ruby{慄然}{ぞ|つ}と
\ruby{{\換字{情}}無}{なさけ|な}く
\ruby{堪}{た}へがたき
\ruby{心地}{こゝ|ち}したり。

\ruby{水野}{みづ|の}の
\ruby{容態}{よう|す}の
\ruby{常}{たゞ}ならぬを
\ruby{見}{み}て、
\ruby[g]{吉右衛門}{きちゑもん}は
\ruby{急}{きふ}に
\ruby[g]{言葉}{ことば}を
\ruby{出}{いだ}し、

『ハヽヽ、
\ruby{前}{まへ}の
\ruby{世}{よ}は
\ruby{何樣}{ど|う}でも
\ruby{宣}{よ}い、
\ruby{今夜}{こん|や}を
\ruby{好}{よ}く
\ruby{{\換字{寝}}}{ね}さへすりやあ
\ruby{好}{い}いのだ!。
\ruby{三歳}{み|つゝ}や
\ruby{四歳}{よ|つゝ}の
\ruby{時}{とき}の
\ruby{事}{こと}を
\ruby{誰}{だれ}が
\ruby{知}{し}つて
\ruby{居}{ゐ}るものか。
\ruby{前}{まへ}の
\ruby{世}{よ}のあるなんぞと
\ruby{思}{おも}ふのは、
\ruby{皆}{みんな}ほんとに
\ruby{氣}{き}の
\ruby{所爲}{せ|ゐ}に
\ruby{定}{きま}つて
\ruby{居}{ゐ}る。
もうそんな
\ruby{下}{くだ}らない
\ruby{事}{こと}は
\ruby{止}{や}めて
\ruby{{\換字{寝}}}{ね}ると
\ruby{仕}{し}ましやうか。
\ruby{{\換字{寝}}}{ね}ると
\ruby{私}{わたし}なぞあ
\ruby{前}{まへ}の
\ruby{世}{よ}が
\ruby{出}{で}て
\ruby{來}{き}て、いつでも
\ruby{若}{わか}くつて、
\ruby{禿}{は}げて
\ruby{居}{ゐ}ないで、いゝ
\ruby{若衆}{わかい|しゆ}ですからおもしろい。
ハヽハヽハ。
』

と
\ruby{高笑}{たか|わら}ひして
\ruby{一座}{いち|ざ}を
\ruby{動}{うご}かしぬ。


\Entry{其十二}

\原本頁{}
\ruby{世界}{せ|かい}は
\ruby{{\換字{紛}}々}{ふん|ぷん}たり、
%
\ruby{萬馬}{ばん|ば}
\ruby{埒}{らち}の
\ruby{内}{うち}を
\ruby{駈}{かけ}り、
%
\ruby{人間}{にん|げん}は
\ruby{擾々}{ぜう|〳〵}たり、
%
\ruby{群蟻}{ぐん|ぎ}
\ruby{碓}{からうす}の
\ruby{緣}{ふち}を
\ruby{回}{めぐ}る、% 原本通り「回」
%
と
\ruby{君}{きみ}が
\ruby{此}{こ}の
\ruby{册子}{さう|し}に
\ruby{書}{か}きし
\ruby{言葉}{こと|ば}もおもしろし、
%
いざや、
%
\ruby{此}{こ}の
\ruby{秋}{あき}の
\ruby{氣}{き}は
\ruby{淸}{す}み
\ruby{風}{かぜ}は
\ruby{快}{こゝろよ}ければ、
%
\ruby{家}{いへ}の
\ruby{矮}{ひく}きより
\ruby{出}{い}で〻
\ruby{山}{やま}の
\ruby{高}{たか}きに
\ruby{登}{のぼ}り、
%
せめては
\ruby{一日}{いち|じつ}を
\ruby{埒}{らち}の
\ruby{内}{うち}より
\ruby{{\換字{逃}}}{のが}れ、
%
\ruby{少時}{しば|し}は
\ruby{碓}{からうす}の
\ruby{緣}{ふち}を
\ruby{離}{はな}れて、
%
\ruby{笑}{わら}ひ
\ruby{傲}{おご}らんもまた
\ruby{可}{よ}からずやと、
%
\ruby{{\換字{絕}}}{た}えて
\ruby{久}{ひさ}しき
\ruby{日方八郎}{ひ|かた|はち|らう}、
%
\ruby{友{\換字{情}}}{なさ|け}は
\ruby{深}{ふか}き
\ruby{島木萬五郎}{しま|き|まん|ご|らう}、
%
\ruby{特}{こと}には
\ruby{懷}{なつ}かしかりし
\ruby{羽{\換字{勝}}}{は|がち}
\ruby{千{\換字{造}}}{せん|ざう}さへ
\ruby{打{\換字{連}}}{うち|つ}れ
\ruby{來}{きた}りて
\ruby{誘}{いざな}ふに、
%
\ruby{日頃}{ひ|ごろ}の
\ruby{崩折}{くづ|を}れきつたる
\ruby{心}{こゝろ}も、
%
\ruby{雨}{あめ}に
\ruby{會}{あ}いたる
\ruby{旱歳}{ひで|り}の
\ruby{草}{くさ}の、
%
\ruby{蘇}{よみがへ}り
\ruby{立}{た}つ
\ruby{思}{おもひ}して、
%
\ruby{一議}{いち|ぎ}にも
\ruby{及}{およ}ばず
\ruby{立出}{たち|い}でしが、
%
\ruby{天}{てん}を
\ruby{{\換字{摩}}}{ま}し
\ruby{雲}{くも}に
\ruby{冲}{ひい}る
\ruby{山嶽}{や|ま}の
\ruby{景色}{け|しき}の、
%
\ruby{雄々}{を|ゝ}しく
\ruby{崇}{たか}きを
\ruby{打望}{うち|のぞ}みて
\ruby{辿}{たど}りし
\ruby{{\換字{半}}{\換字{途}}}{はん|ど}に
\ruby{如何}{い|かゞ}は
\ruby{仕}{し}けん、
%
\ruby{圖}{はか}らず
\ruby{三人}{さん|にん}とは
\ruby{相}{あひ}
\ruby{失}{うしな}ひたり。

\原本頁{}
『
\ruby{水野}{みづ|の}ーツ、
』

\原本頁{}
と
\ruby{號令聲}{がう|れい|ごゑ}の
\ruby{烈}{はげ}しく
\ruby{叫}{さけ}べるは
\ruby{豪放}{がう|はう}なる
\ruby{我}{わ}が
\ruby{日方}{ひ|かた}の
\ruby{聲}{こゑ}なり。

\原本頁{}
『オーイ、
%
\ruby{水野}{みづ|の}、
』

\原本頁{}
と
\ruby{爽}{さは}やかに
\ruby{喚}{よ}べるは
\ruby{快活}{くわい|くわつ}なる
\ruby{我}{わ}が
\ruby{島木}{しま|き}が
\ruby{聲}{こゑ}なり。
%
\ruby{姿}{すがた}は
\ruby{何處}{いづ|く}とも
\ruby{見}{み}えざれど、
%
\ruby{聲}{こゑ}は
\ruby{{\換字{前}}{\換字{途}}}{ゆく|て}の
\ruby{高}{たか}きにありて、
%
\ruby{後}{おく}れたる
\ruby{我}{われ}を
\ruby{励}{はげ}まし
\ruby{促}{うなが}し、
%
\ruby{來}{きた}れよ、
%
\ruby{上}{のぼ}れよ、
%
\ruby{{\換字{進}}}{すゝ}まざるやと、
%
\ruby{二人}{ふた|り}が
\ruby{心}{こゝろ}を
\ruby{焦立}{いら|だ}て
\ruby{居}{を}れるは、
%
\ruby{其聲}{その|こゑ}の
\ruby{色}{いろ}にもあり〳〵と
\ruby{知}{し}れたり。

\原本頁{}
\ruby{草萊}{く|さ}も
\ruby{無}{な}ければ、
%
\ruby{樸樕}{ちいさ|きゝ}も% 「樸樕」(ぼく‐そく) 小さい木。 叢生する小さな木々。
\ruby{無}{な}く、
%
たゞこれ
\ruby{圓}{まる}き
\ruby{石塊}{い|し}のみなる
\ruby{荒凉}{くわう|りやう}たる
\ruby{山路}{さん|ろ}の
\ruby{爪端上}{つま|さき|あがり}は、
%
\ruby{歩}{あゆ}み
\ruby{行}{ゆ}くにいと
\ruby{歩}{あゆ}み
\ruby{辛}{づら}けれど、
%
\ruby{上}{のぼ}らで
\ruby{止}{や}むべき
\ruby{我}{われ}ならんやと、
%
\ruby{水野}{みづ|の}は
\ruby[<h||]{唇}{くちびる}
\ruby{硬}{かた}く
\ruby{引締}{ひき|し}めて、
%
\ruby{執念}{しふ|ね}くも
\ruby{{\換字{強}}}{し}ひて
\ruby{上}{のぼ}り
\ruby{上}{のぼ}りぬ。

\原本頁{}
\ruby{歩}{あゆ}めば
\ruby{磧礫}{こ|いし}は
\ruby{我}{わ}が
\ruby{脚}{あし}の
\ruby{下}{した}につぶやき
\ruby{言}{ものい}ふ
\ruby{聲}{こゑ}をなし、
%
\ruby{頑石}{い|し}はまた
\ruby{腹黑}{はら|ぐろ}くも
\ruby{我}{われ}を
\ruby{滑}{すべ}らしむ。
%
されど
\ruby{上}{のぼ}らで
\ruby{止}{や}むべき
\ruby{我}{われ}ならんやと、
%
\ruby{水野}{みづ|の}はつぶやける
\ruby{磧礫}{こ|いし}の
\ruby{上}{うへ}に
\ruby{冷}{ひや}やかに
\ruby{濶歩}{かわ|つぽ}し、
%
\ruby{滑}{すべ}らしむる
\ruby{頑石}{い|し}の
\ruby{頭}{かしら}をしたゝかに
\ruby{踏}{ふ}み
\ruby{壓}{おさ}へて、
%
\ruby{{\換字{猶}}}{なほ}
\ruby{執念}{しふ|ね}くも
\ruby{{\換字{強}}}{し}ひて
\ruby{上}{のぼ}り
\ruby{上}{のぼ}りぬ。

\原本頁{}
\ruby{時}{とき}に
\ruby{何處}{いづ|く}より
\ruby{來}{きた}りしともなく
\ruby{{\換字{丈}}矮}{たけ|ひく}く
\ruby{足跛}{あし|な}へたる
\ruby{妖精}{も|の}の、
%
\ruby{其}{そ}の
\ruby{狀怪}{かたち|あや}しくして、
%
たゞ
\ruby{是}{これ}
\ruby{肉}{にく}の
\ruby{團塊}{かた|まり}ともいふべく
\ruby{氣味}{き|み}
\ruby{惡}{あし}くも
\ruby{重}{おも}きが、
%
\ruby{何時}{い|つ}か
\ruby{我}{わ}が
\ruby{肩頭}{かた|さき}に
\ruby{上}{のぼ}り
\ruby{居}{を}りて、
%
\ruby{怖}{おそ}ろしき
\ruby{其}{そ}の
\ruby{力}{ちから}をもて
\ruby{壓}{お}しに
\ruby{壓}{お}しつ、
%
\ruby{止}{とゞ}まれ、
%
\ruby{止}{とゞ}まれ、
%
\ruby{休}{やす}めよ、
%
\ruby{倒}{たふ}れよ、
%
\ruby{地}{ち}に
\ruby{入}{い}れよ、
%
\ruby{奈落}{な|らく}に
\ruby{歿}{ぼつ}せよと
\ruby{云}{い}はぬばかりに、
%
\ruby{下方}{し|た}へ
\ruby{下方}{し|た}へと
\ruby{壓}{お}しつけたり。

\原本頁{}
\ruby{汝}{おのれ}、
%
\ruby{我}{わ}が
\ruby{{\換字{魔}}}{ま}、
%
\ruby{我}{わ}が
\ruby{仇敵}{あ|だ}の
\ruby{重力}{おも|さ}の
\ruby{精}{せい}!。
%
\ruby[<h||]{汝}{なんぢ}% 「汝(なんぢ)」の読みは原文のまま
\ruby{千鈞}{せん|きん}の
\ruby{力}{ちから}をもて
\ruby{我}{われ}を
\ruby{壓}{あつ}さば、
%
\ruby{我}{われ}また
\ruby{千鈞}{せん|きん}の
\ruby{力}{ちから}を
\ruby{以}{もつ}て
\ruby{汝}{なんぢ}に% 「汝(なんぢ)」の読みは原文のまま
\ruby{當}{あた}らん。
%
\ruby[<h||]{汝}{なんぢ}% 「汝(なんぢ)」の読みは原文のまま
\ruby{萬鈞}{まん|きん}の
\ruby{力}{ちから}をもて
\ruby{我}{われ}を
\ruby{壓}{あつ}さば、
%
\ruby{我}{われ}また
\ruby{萬鈞}{まん|きん}の
\ruby{力}{ちから}を
\ruby{以}{もつ}て
\ruby{汝}{なんぢ}に% 「汝(なんぢ)」の読みは原文のまま
\ruby{當}{あた}らん。
%
\ruby{汝}{なんぢ}は% 「汝(なんぢ)」の読みは原文のまま
\ruby{下}{くだ}さんとす、
%
\ruby{我}{われ}は
\ruby{上}{のぼ}らんとす。
%
\ruby{我}{われ}
\ruby{屈}{くつ}せず
\ruby{我}{われ}
\ruby{撓}{たゆ}まず、
%
\ruby{我々}{われ|〳〵}が
\ruby{努力}{ど|りよく}を
\ruby{悋}{をし}むこと
\ruby{無}{な}し、
%
\ruby{上}{のぼ}らで
\ruby{止}{や}むべき
\ruby{我}{われ}ならんや、
%
と
\ruby{傲然}{がう|ぜん}として
\ruby{重}{おも}きに
\ruby{堪}{た}へつ〻、
%
\ruby{言葉}{こと|ば}をも
\ruby{出}{だ}さねば
\ruby{手}{て}をも
\ruby{動}{うご}かさずして、
%
\ruby{水野}{みづ|の}は
\ruby{{\換字{猶}}}{なほ}
\ruby{{\換字{強}}}{し}ひて
\ruby{執念}{しふ|ね}くも
\ruby{上}{のぼ}り
\ruby{上}{のぼ}りぬ。

\原本頁{}
\ruby{妖精}{も|の}は
\ruby{水野}{みづ|の}が
\ruby{耳}{みゝ}に
\ruby{貼}{つ}きて、
%
\ruby{重々}{おも|〳〵}しき
\ruby{言葉}{こと|ば}の
\ruby{一語一語}{いち|ご|いち|ご}に、
%
\ruby{{\換字{鉛}}}{なまり}の
\ruby{雫}{しづく}を
\ruby{頭腦}{かし|ら}の
\ruby{奧}{おく}に
\ruby{{\換字{送}}}{おく}り
\ruby{入}{い}るゝが
\ruby{如}{ごと}くに
\ruby{囁}{さゝや}きて
\ruby{曰}{い}へらく、% 曰く(いわく)の「曰 66f0」、日曜の「日」は 65e5

\原本頁{}
『
\ruby{爾}{なんぢ}、
%
\ruby{水野}{みづ|の}!、
%
おろかにも
\ruby{爾}{なんぢ}の
\ruby{思}{おも}ひあがれるよ。
%
\ruby{爾}{なんぢ}、
%
\ruby{智慧}{ち|ゑ}の
\ruby{石}{いし}!。
%
\ruby{爾}{なんぢ}、
%
おのが
\ruby{身}{み}を
\ruby{高}{たか}くも
\ruby{高}{たか}く
\ruby{投}{な}げ
\ruby{上}{あ}げたる
\ruby{爾}{なんぢ}。
%
されど、
%
おろかや、
%
\ruby{投}{な}げられし
\ruby{石}{いし}の、
%
\ruby{落}{お}ちて
\ruby{{\換字{返}}}{かへ}らぬ
\ruby{事}{こと}のいづくにかある!。
%
おろかや
\ruby{爾}{なんぢ}、
%
\ruby{落}{お}ちて
\ruby{下}{くだ}らん、
%
\ruby{今}{いま}
\ruby{見}{み}よ
\ruby{落}{お}ちて
\ruby{降}{くだ}るべきなり。

\原本頁{}
\ruby{爾}{なんぢ}、
%
\ruby{水野}{みづ|の}!、
%
\ruby{智慧}{ち|ゑ}の
\ruby{石}{いし}!。
%
\ruby[<h||]{爾}{なんぢ}
\ruby{弩}{いしゆみ}より
\ruby{飛}{と}びし
\ruby{石}{いし}、
%
\ruby[<h||]{爾}{なんぢ}
\ruby{天}{あま}つ
\ruby{星}{ぼし}を
\ruby{碎}{くだ}かんとして
\ruby{飛}{と}びし
\ruby{石}{いし}!。
%
\ruby{爾}{なんぢ}、
%
おのが
\ruby{身}{み}を
\ruby{高}{たか}くも
\ruby{高}{たか}く
\ruby{投}{な}げ
\ruby{上}{あ}げたる
\ruby{爾}{なんぢ}。
%
されど、
%
おろかや
\ruby{爾}{なんぢ}、
%
\ruby{投}{な}げられし
\ruby{石}{いし}の
\ruby{落}{お}ちて
\ruby{{\換字{返}}}{かへ}らぬ
\ruby{事}{こと}の
\ruby{那處}{いづ|く}にかある!。
%
おろかや
\ruby{爾}{なんぢ}、
%
\ruby{落}{お}ちて
\ruby{降}{くだ}らん、
%
\ruby{今}{いま}
\ruby{見}{み}よ
\ruby{落}{お}ちて
\ruby{降}{くだ}るべきなり。
%
\ruby{聞}{き}け、
%
\ruby{宣告}{のり|ごと}はかくぞ、
%
\ruby{爾}{なんぢ}と
\ruby{爾}{なんぢ}の
\ruby{石}{いし}を
\ruby{投}{な}ぐる
\ruby{行爲}{わ|ざ}とよ。
%
\ruby{水野}{みづ|の}、
%
\ruby[<h||]{爾}{なんぢ}
\ruby{高}{たか}くも
\ruby{高}{たか}く
\ruby{石}{いし}を
\ruby{投}{な}げたるよ、
%
されど
\ruby{其}{そ}はたゞ
\ruby{爾}{なんぢ}が
\ruby{頭}{かしら}の
\ruby{上}{うへ}に
\ruby{落}{お}ちて
\ruby{{\換字{返}}}{かへ}らんなり。
』

\原本頁{}
かく
\ruby{云}{い}ひ
\ruby{{\換字{終}}}{おは}りて
\ruby{言}{ことば}を
\ruby{{\換字{絕}}}{た}ちしが、
%
\ruby{妖精}{も|の}は
\ruby{無言}{む|ごん}の
\ruby{恐}{おそろ}しき
\ruby{力}{ちから}をもて、
%
\ruby{倒}{たふ}れよ
\ruby{地}{つち}に、
%
\ruby{沈}{しづ}めよ
\ruby{奈落}{な|らく}にと、
%
いよ〳〵
\ruby{烈}{はげ}しく
\ruby{壓}{お}しに
\ruby{壓}{お}せば、
%
\ruby{水野}{みづ|の}はほと〳〵
\ruby{堪}{た}へざらんとしたり。

\原本頁{}
されど
\ruby{水野}{みづ|の}は
\ruby{{\換字{更}}}{さら}に
\ruby{屈}{くつ}せず、
%
\ruby{盤石}{ばん|じやく}
\ruby{虐}{しひた}げ
\ruby{壓}{あつ}すれども
\ruby{幽{\換字{蘭}}}{ゆう|らん}
\ruby{死}{し}せずして、
%
\ruby{{\換字{猶}}}{なほ}
\ruby{能}{よ}く
\ruby{天}{てん}に
\ruby{向}{むか}つて
\ruby{芽}{め}を
\ruby{抽}{ぬき}んづるが
\ruby{如}{ごと}く、
%
\ruby{昻々然}{かう|〳〵|ぜん}として
\ruby{頭}{かうべ}を
\ruby{擧}{あ}げて、
%
\ruby{執念}{しふ|ね}くも
\ruby{{\換字{強}}}{し}ひて
\ruby{上}{のぼ}り
\ruby{上}{のぼ}りけるが、
%
\ruby{見}{み}れば
\ruby{路}{みち}の
\ruby{邊}{べ}に
\ruby{病}{や}める
\ruby{女}{をんな}ありて
\ruby{世}{よ}にも
\ruby{痛}{いた}ましく
\ruby{惱}{なや}み
\ruby{伏}{ふ}したり。
%
\ruby{如何}{い|か}なる
\ruby{人}{ひと}の
\ruby{{\換字{道}}行}{みち|ゆ}き
\ruby{患}{わづら}ひて、
%
かゝる
\ruby{山路}{やま|ぢ}にはあるならんと、
%
いぶかしみて
\ruby{不圖}{ふ|と}
\ruby{眼}{め}を
\ruby{{\換字{留}}}{とゞ}むれば、
%
\ruby{彼方}{かな|た}も
\ruby{人}{ひと}ありと
\ruby{知}{し}つて
\ruby{此方}{こ|なた}を
\ruby{見}{み}かへりたり。
%
\ruby{見}{み}ざりし
\ruby{程}{ほど}こそ
\ruby{心}{こゝろ}も
\ruby{常}{つね}なりつれ、
%
\ruby{相}{あひ}
\ruby{見}{み}ては
\ruby{互}{たがひ}にハツと
\ruby{驚}{おどろ}きて、
%
\ruby{彼方}{かな|た}は
\ruby{面}{おもて}を
\ruby{掩}{おほ}ひ、
%
\ruby{我}{われ}は
\ruby{胸}{むね}を
\ruby{轟}{とゞろ}かす。
%
\ruby{其人}{その|ひと}は
\ruby{少時}{しば|し}も
\ruby{忘}{わす}れぬ
\ruby{我}{わ}が
\ruby{五十子}{い|そ|こ}なれば、
%
\ruby{何}{なん}として
\ruby{此處}{こ|ゝ}にはと
\ruby{先}{ま}づ
\ruby{走}{はし}り
\ruby{寄}{よ}つて、
%
\ruby{慌}{あわ}てゝ
\ruby{扶}{たす}け
\ruby{起}{おこ}さんと
\ruby{其手}{その|て}を
\ruby{執}{と}れば、
%
\ruby{我}{わ}が
\ruby{手}{て}に
\ruby{他}{ひと}の
\ruby{手}{て}の
\ruby{觸}{ふ}るゝや
\ruby{觸}{ふ}れぬに、
%
\ruby{彼}{か}の
\ruby{妖精}{も|の}は
\ruby{異樣}{こと|やう}に
\ruby{高笑}{たか|わら}ひして、

\原本頁{}
『
\ruby{見}{み}よ、
%
\ruby{地}{つち}より
\ruby{出}{い}でしものよ、
%
\ruby{地戀}{つち|こひ}しきかよ。
%
\ruby{我}{われ}
\ruby{見}{み}ん、
%
\ruby{石}{いし}の
\ruby{落}{お}ちて
\ruby{那處}{いづ|く}に
\ruby{至}{いた}るかを。
』

\原本頁{}
と
\ruby{{\換字{勝}}}{か}ち
\ruby{誇}{ほこ}れるが
\ruby{如}{ごと}く
\ruby{嘲}{あざ}み
\ruby{罵}{のゝし}る
\ruby{其}{そ}の
\ruby{聲}{こゑ}
\ruby{耳}{みゝ}に
\ruby{徹}{てつ}する
\ruby{{\換字{途}}端}{と|たん}、
%
\ruby{忽地}{たちま|ち}に
\ruby{身}{み}は
\ruby{{\換字{鉛}}}{なまり}よりも
\ruby{重}{おも}くなりて、
%
\ruby{大地}{だい|ち}は
\ruby{{\換字{雪}}}{ゆき}より
\ruby{柔}{やはら}かになり、
%
\ruby{見}{み}る〳〵
\ruby{地窪}{ち|くぼ}み
\ruby{身}{み}は
\ruby{陷}{おちゐ}つて、
%
\ruby{踝沒}{くるぶし|かく}れ、
%
\ruby{脛沒}{はぎ|かく}れ、
%
\ruby{膝皿沒}{ひざ|ゝら|かく}れ、
%
\ruby{高腿}{たか|もゝ}
\ruby{沒}{かく}れ、
%
\ruby{腹沒}{はら|かく}れ、
%
\ruby{胸}{むね}
\ruby{沒}{かく}れ
\ruby{肩沒}{かた|かく}れ
\ruby{行}{ゆ}きて、
%
\ruby{石人}{せき|じん}の
\ruby{水}{みづ}に
\ruby{沈}{しづ}むが
\ruby{如}{ごと}くに、
%
\ruby{全}{まつた}く
\ruby{自}{みづか}ら
\ruby{支}{さゝ}ふるに
\ruby[<h||]{力}{ちから}
\ruby{無}{な}く、

\原本頁{}
『
\ruby{水野}{みづ|の}ツ』

\原本頁{}
と
\ruby{呼}{よ}ぶ
\ruby{日方}{ひ|かた}の
\ruby{聲}{こゑ}、

\原本頁{}
『オーイ、オーイ、
』

\原本頁{}
と
\ruby{喚}{よ}ぶ
\ruby{島木}{しま|き}が
\ruby{聲}{こゑ}を
\ruby{遙}{はるか}に〳〵
\ruby{聞}{き}きながら、
%
\ruby{次第}{し|だい}に
\ruby{現世}{うつし|よ}には
\ruby{{\換字{遠}}}{とほ}ざかりて、
%
\ruby{漸}{やうや}く
\ruby{奈落}{な|らく}の
\ruby{底}{そこ}に
\ruby{沈}{しづ}み
\ruby{行}{ゆ}かんとす。
%
\ruby{今}{いま}は
\ruby{氣}{き}も
\ruby{心}{こゝろ}も
\ruby{{\換字{消}}}{き}え〴〵になりて、
%
\ruby{思}{おも}はずも
\ruby{南無}{な|む}と
\ruby{叫}{よ}びかゝる
\ruby{時}{とき}、
%
\ruby{駈}{か}けつけ
\ruby{吳}{く}れたる
\ruby{我}{わ}が
\ruby{羽{\換字{勝}}}{は|がち}の、
%
ムヅと
\ruby{我}{わ}が
\ruby{頭髮}{か|み}を
\ruby{引攫}{ひつ|ゝか}みて、
%
\ruby{鐵腕}{てつ|わん}の
\ruby{力}{ちから}の
\ruby{恐}{おそ}ろしく
\ruby{凄}{すさま}じくも、
%
ふたゝび
\ruby{我}{われ}を
\ruby{光}{ひかり}ある
\ruby{世}{よ}に
\ruby{飛礫}{つぶ|て}の
\ruby{如}{ごと}く
\ruby{投}{な}げ
\ruby{上}{あ}げくれたるが、
%
\ruby{投}{な}げられて
\ruby{{\換字{空}}}{そら}を
\ruby{飛}{と}べる
\ruby{身}{み}の
\ruby{呀}{あつ}と
\ruby{驚}{おどろ}きて
\ruby{我}{われ}に
\ruby{{\換字{返}}}{かへ}れば、
%
これは
\ruby{是}{これ}
\ruby{思}{おも}ひ
\ruby{寢}{ね}の
\ruby{惡夢}{あく|む}にして、
%
\ruby{滿身}{まん|しん}の
\ruby{汗絞}{あせ|しぼ}るが
\ruby{如}{ごと}く、
%
\ruby{胸}{むね}は
\ruby{今}{いま}
\ruby{{\換字{猶}}}{なほ}
\ruby{浪}{なみ}
\ruby{打}{う}つて
\ruby{騷}{さわ}ぎ、
%
\ruby{枕頭}{ちん|とう}の
\ruby{燈}{ひ}は
\ruby{靑}{あお}くして
\ruby{幽}{かすか}に、
%
\ruby{恰}{あたか}も% 恰も「あ(た)かも」
\ruby{鳴}{な}り
\ruby{出}{いだ}せる
\ruby{茶}{ちや}の
\ruby{室}{ま}の
\ruby{時計}{と|けい}は、
%
\ruby{一}{ひと}ツ、
%
\ruby{二}{ふた}ツ、
%
\ruby{三}{みつ}ツにして
\ruby{復}{また}
\ruby{默}{もく}したり。

\原本頁{}
\ruby[g]{吉右衛門}{きちゑもん}は
\ruby{眠}{ねむ}れり、
%
お
\ruby{濱}{はま}は
\ruby{眠}{ねむ}れり、
%
\ruby{日方}{ひ|かた}も
\ruby{島木}{しま|き}も
\ruby{將}{はた}
\ruby{羽{\換字{勝}}}{は|がち}も
\ruby{今}{いま}は
\ruby{思}{おも}ふに
\ruby{睡}{ねむ}れるならん。
%
\ruby{憎}{にく}き
お
\ruby{澤}{さは}
\ruby{婆}{ばゞ}も
\ruby{睡}{ねむ}れるならん。
%
\ruby{可憫}{か|はゆ}き
\ruby{松之助}{まつ|の|すけ}も
\ruby{睡}{ねむ}れるならん。
%
\ruby{醫}{い}も
\ruby{睡}{ねむ}れるならん。
%
\ruby{看護{\換字{婦}}}{かん|ご|ふ}も
\ruby{睡}{ねむ}れるならん。
%
\ruby{覺}{さ}めたるものは
\ruby{我}{われ}のみなるが、
%
たゞ
\ruby{我}{わ}が
\ruby{病}{やまひ}の
\ruby{蓐}{とこ}に
\ruby{惱}{なや}める
\ruby{五十子}{い|そ|こ}は、
%
\ruby{睡}{ねむ}れりや
\ruby{如何}{い|か}に、
%
\ruby{穩}{おだ}やかに
\ruby{睡}{ねむ}れりや。

\原本頁{}
\ruby{恐}{おそ}ろしき
\ruby{夢}{ゆめ}に
\ruby{魘}{おそ}はれし
\ruby{水野}{みづ|の}は、
%
\ruby{夢}{ゆめ}の
\ruby{根基}{もと|ひ}となりし
\ruby{{\換字{宵}}}{よひ}の
\ruby{談話}{はな|し}を
\ruby{獨}{ひと}り
\ruby{靜}{しづか}に
\ruby{思}{おも}ひ
\ruby{{\換字{返}}}{かへ}して、
%
さま〴〵に
\ruby{思}{おも}ひ
\ruby{亂}{みだ}るゝ
\ruby{折}{をり}しも、
%
いつまで
\ruby{睡}{ねむ}らで
\ruby{吠}{ほ}ゆる
\ruby{彼}{か}の
\ruby{狗}{いぬ}なるぞや、
%
また
\ruby{彼}{か}の
\ruby{狗}{いぬ}の
\ruby{聲}{こゑ}はべう〳〵と
\ruby{聞}{きこ}えぬ。
%
か〻る
\ruby{夜深}{よ|ふか}きに
\ruby{何}{なに}をか
\ruby{見}{み}て
\ruby{吠}{ほ}えし、
%
\ruby{人}{ひと}の
\ruby{魂魄}{た|ま}にても
\ruby{飛}{と}びたるかや、
%
あ〻。


\Entry{其十三}

\ruby{何}{なん}となく
\ruby{五十子}{い|そ|こ}が
\ruby{上}{うへ}のあやしく
\ruby{氣}{き}にかゝりて、
\ruby[g]{水野}{みづの}は
\ruby{睡}{ねむ}らんとしてもまた
\ruby{睡}{ねむ}られず、
\ruby{{\GWI{u82e5-k}}}{もし}や
\ruby{彼}{か}の
\ruby{人}{ひと}の
\ruby{病狀}{やう|す}に
\ruby{變}{へん}などありて、
\ruby{今}{いま}を
\ruby{生死}{しよ|うし}の
\ruby{瀬戸}{せ|と}と
\ruby{{\GWI{u82e6-k}}}{くる}しめるにはあらずや。
\ruby{此}{こ}の
\ruby{一日}{いち|にち}は
\ruby{快}{よ}きやうなりしが、
\ruby[g]{此際數日}{こゝしばらく}を
\ruby{病}{やまひ}の
\ruby{峠}{たうげ}とは、
\ruby{尾竹}{お|だけ}も
\ruby{云}{い}ひたるところなれば、
\ruby{取}{と}り
\ruby{別}{わ}けて
\ruby[g]{心元無}{こヽろもとな}く
\ruby{思}{おも}はるゝかな。
\ruby{狗}{いぬ}の
\ruby{長吠}{なが|ぼ\GWI{u1b001}}する
\ruby{時}{とき}は
\ruby{凶}{あし}き
\ruby{事}{こと}ありといふ
\ruby{俗{\GWI{u8aaa-jv}}}{いひ|つたへ}も、
\ruby{取}{と}るに
\ruby{足}{た}らぬ
\ruby[g]{迷信}{まよひ}なりとは
\ruby{知}{し}りながら、さし
\ruby{當}{あ}たつて
\ruby{今}{いま}は
\ruby{忌}{いま}はしくぞ
\ruby{覺}{おぼ}ゆる。
\ruby{熱}{ねつ}の
\ruby{高}{たか}きには
\ruby{心}{こヽろ}も
\ruby{亂}{みだ}れて、
\ruby{夢}{ゆめ}は
\ruby{駈}{か}け
\ruby{回}{めぐ}る
\ruby{曠野}{あら|の}の
\ruby{夏}{なつ}に、
\ruby[g]{火炎}{ほのほ}と
\ruby{熱}{あつ}き
\ruby{息}{いき}を
\ruby{吐}{つ}きつゝ、
\ruby{{\GWI{u6df8-jv}}水}{し|みづ}
\ruby{{\GWI{u5c0b-k}}}{たづ}ねわづらふ
\ruby{思}{おもひ}に
\ruby{悶}{もだ}え、
\ruby{夜具}{や|ぐ}に
\ruby{身}{み}を
\ruby{餘}{あま}して
\ruby{我知}{われ|し}らず
\ruby{呻}{うめ}く、
\ruby{其}{そ}の
\ruby{苦}{くる}しさの
\ruby{經驗}{おぼ|{\GWI{u1b001}}}は
\ruby{我}{われ}も
\ruby{知}{し}れるが、
\ruby{我}{わ}が
\ruby{五十子}{い|そ|こ}は
\ruby{今}{いま}さる
\ruby{事}{こと}もなくて、
\ruby{幸}{さいはひ}にすやすやと
\ruby{睡}{ねむ}れりや
\ruby{如何}{い|か}に。
\ruby{安}{やす}らかに
\ruby{病人}{びやう|にん}の
\ruby{睡}{ねむ}り
\ruby{居}{を}らば、それより
\ruby{頼}{たの}もしき
\ruby{好}{よ}き
\ruby{事}{こと}は
\ruby{無}{な}けれど、
\ruby{或}{あるひ}は
\ruby{{\換字{叉}}}{また}ほそ〴〵と
\ruby{癯}{や}せし
\ruby{手先}{て|さき}の、
\ruby{物}{もの}あはれにも
\ruby{枕}{まくら}の
\ruby{端}{はし}なんどを
\ruby{力草}{ちから|ぐさ}に
\ruby{執}{と}り
\ruby{絞}{しぼ}りて、
\ruby{苦}{くる}しさに
\ruby{堪}{こら}へ
\ruby{堪}{こら}へし
\ruby{果}{は}ては、
\ruby{睡}{ねむ}りも
\ruby{睡}{ねむ}り
\ruby{得}{\GWI{u1b001}}ず、
\ruby{醒}{さ}めも
\ruby{醒}{さ}めやらずなりて、ただうつら〳〵と
\ruby{病苦}{びや|うく}に
\ruby{責}{せ}められ、
\ruby[g]{一{\換字{半}}}{なかば}は
\ruby{現}{うつゝ}、
\ruby[g]{一{\換字{半}}}{なかば}は
\ruby{夢}{ゆめ}の、
\ruby[g]{精神}{こヽろ}は
\ruby[g]{幽}{かすか}に
\ruby{{\GWI{u6d88-k}}}{き}えかゝりて、
\ruby{現世冥{\GWI{u9014-k}}}{この|よ|あの|よ}の
\ruby[g]{境界}{さかひ}の
\ruby{上}{うへ}に、
\ruby{魂魄{\GWI{u8ff7-k}}}{たま|しい|まよ}へるやうにあらば、あ〻
\ruby{如何}{い|か}にせん、
\ruby{如何}{い|か}にせん。
おもへば
\ruby{何}{なに}となるべき
\ruby{彼}{か}の
\ruby{人}{ひと}の
\ruby{上}{うへ}、
\ruby{我}{わ}が
\ruby{上}{うへ}ぞや。
\ruby{前}{さき}の
\ruby{世}{よ}の
\ruby{有}{あ}りや
\ruby{無}{な}しや、それも
\ruby{知}{し}らず、
\ruby{後}{のち}の
\ruby{世}{よ}の
\ruby{有}{あ}りや
\ruby{無}{な}しや、
\ruby{知}{し}らねど、
\ruby{{\GWI{u82e5-k}}}{も}し
\ruby{今彼}{いま|か}の
\ruby{人}{ひと}の
\ruby{此}{こ}の
\ruby{世}{よ}を
\ruby{去}{さ}らば、
\ruby{我}{わ}が
\ruby{身}{み}は
\ruby{此世}{この|よ}にも
\ruby{{\GWI{u907a-k}}}{のこ}るべけれど、
\ruby{我}{わ}が
\ruby{魂}{たま}はおのづと
\ruby{他}{ひと}に
\ruby{引}{ひ}かれて、
\ruby{必}{かなら}ず
\ruby{冥{\GWI{u9014-k}}}{あの|よ}に
\ruby{去}{さ}るべければ、
\ruby{其}{そ}の
\ruby{後}{のち}の
\ruby{世}{よ}を
\ruby{思}{おも}ふにつけても、
\ruby{此}{こ}の
\ruby{世}{よ}に
\ruby{我}{われ}の
\ruby{現}{あらは}れ
\ruby{出}{い}でしは、
\ruby{此}{こ}の
\ruby{世}{よ}に
\ruby{彼}{か}の
\ruby{人}{ひと}の
\ruby{出}{い}でしがためかと、
\ruby{思}{おも}ひやらるゝ
\ruby{心地}{こゝ|ち}のして、
\ruby{{\GWI{u9060-k}}}{とほ}く
\ruby[g]{邈焉}{はるか}なる
\ruby{前}{さき}の
\ruby{世}{よ}にも、
\ruby{彼}{か}の
\ruby{人}{ひと}は
\ruby{今}{いま}と
\ruby{同}{おな}じく
\ruby{病}{やまひ}に
\ruby{惱}{なや}み、
\ruby{我}{われ}は
\ruby{今}{いま}と
\ruby{同}{おな}じく
\ruby{戀}{こひ}に
\ruby{泣}{な}きし
\ruby{悲}{かな}しきありさまの、あり〳〵と
\ruby{此}{こ}の
\ruby{心}{こヽろ}に
\ruby{{\GWI{u6d6e-k}}}{うか}び
\ruby{來}{く}るなり。
よしや、
\ruby{前}{まへ}の
\ruby{世}{よ}の
\ruby{緣}{\GWI{u1b001}ん}の
\ruby{悲}{かな}しくもあれ、
\ruby{此}{こ}の
\ruby{世}{よ}の
\ruby{緣}{\GWI{u1b001}ん}もまた
\ruby{果敢}{は|か}なくもあれ、
\ruby[g]{眞實}{まこと}に
\ruby{前}{まへ}の
\ruby{世}{よ}の
\ruby{存}{あ}りもせよかし。
\ruby{前}{まへ}の
\ruby{世}{よ}
\ruby[g]{眞實}{まこと}にあらば
\ruby{後}{のち}の
\ruby{世}{よ}もあり、
\ruby{其}{そ}のまた
\ruby{後}{のち}の
\ruby{世}{よ}もあらんに、せめては
\ruby{其}{それ}を
\ruby{頼}{たの}みにはして、
\ruby{長}{なが}く
\ruby{盡}{つ}きざる
\ruby{我思}{わが|おも}ひの
\ruby{如何}{い|か}にか
\ruby{成}{な}り
\ruby{行}{ゆ}く
\ruby{涯}{はて}を
\ruby{見}{み}ん。
あ〻
\ruby{其}{それ}につけ
\ruby{此}{これ}につけても、
\ruby{如何}{い|か}なれば
\ruby{此}{こ}の
\ruby{胸}{むね}の
\ruby{如是}{か|く}は
\ruby{騒}{さわ}ぎて、
\ruby{動悸}{どう|き}の
\ruby{浪}{なみ}のたゞならず
\ruby{打}{う}つ
\ruby{事}{こと}よ。
あ〻
\ruby{何}{なん}となく
\ruby{心悲}{こヽろ|かな}しく
\ruby{物恐}{もの|おそ}ろしき
\ruby{感}{おもひ}のするかな。
\ruby{{\GWI{u82e5-k}}}{も}しくは
\ruby{彼}{か}の
\ruby{人}{ひと}の
\ruby{何}{なに}とかなせるにはあらずや、
\ruby{居}{い}ても
\ruby{立}{た}つても
\ruby{心}{こヽろ}の
\ruby{安}{やす}からぬ。
あ〻
\ruby{何}{なん}とせん、あ〻
\ruby{何}{なん}となさん。
と、とつ
\ruby{置}{おい}つ
\ruby{思}{おも}ひ
\ruby{{\GWI{u8ff7-k}}}{まよ}ひしが、
\ruby{此處}{こ|ゝ}にありて
\ruby{{\GWI{u7a7a-ue0101}}}{あだ}に
\ruby{悶}{もだ}えんよりは、
\ruby{其處}{そ|こ}に
\ruby{至}{いた}りて、
\ruby{狀態}{あり|さま}を
\ruby{伺}{うかゞ}はんと、
\ruby{{\GWI{u7d42-ue0101}}}{つひ}に
\ruby{衣}{い}をかへて
\ruby{立}{た}ち
\ruby{出}{いで}でたり。

\ruby{風}{かぜ}も
\ruby{眠}{ねむ}れり、
\ruby{石}{いし}も
\ruby{眠}{ねむ}れり。
\ruby{誰}{だれ}かかゝる
\ruby{時{\GWI{u6236-t}}外}{とき|おも|て}に
\ruby{在}{あ}らんや。
\ruby{{\GWI{u6236-t}}締}{と|じま}りの
\ruby{如何}{い|か}にすべきも
\ruby{打忘}{うち|わす}れて、ふら〳〵と
\ruby{立出}{たち|い}でたる
\ruby[g]{水野}{みづの}の
\ruby{{\GWI{u9053-k}}行}{みち|ゆ}く
\ruby{様子}{やう|す}は、たとへば
\ruby{影}{かげ}のみの
\ruby{人}{ひと}の
\ruby{如}{ごと}く、
\ruby{現世}{この|よ}のものとも
\ruby{思}{おも}はれざりしが、
\ruby[g]{水野}{みづの}も
\ruby{既}{すで}に
\ruby{此世}{この|よ}を
\ruby{忘}{わす}れて、
\ruby{{\GWI{u7a7a-ue0101}}}{そら}は
\ruby{今曇}{いま|くも}れりや
\ruby{星}{ほし}ありや、
\ruby{闇}{やみ}なりや、
\ruby{將月}{はた|つき}ありやも、
\ruby{全}{まつた}く
\ruby{心}{こヽろ}には
\ruby{留}{と}めざるなりけり。


\Entry{其十四}

\ruby{{\換字{戸}}}{と}に
\ruby{尾栓}{しり|さし}せで
\ruby{濟}{す}む
\ruby{村居}{そん|きよ}の
\ruby{心安}{こヽろ|やす}さに
\ruby{慣}{な}れたりとは
\ruby{云}{い}へ、
\ruby{所以知}{ゆ|ゑ|し}らぬ
\ruby{動悸}{むな|さわぎ}の
\ruby{烈}{はげ}しさに
\ruby{其}{そ}の
\ruby{人}{ひと}の
\ruby{氣}{き}づかはしくて
\ruby{堪}{た}え
\ruby{{\換字{兼}}}{かね}たりとは
\ruby{云}{い}へ、
\ruby{時}{とき}ならぬに
\ruby[g]{飄然}{ふらり}と
\ruby{立出}{たち|い}でし
\ruby[g]{水野}{みづの}の
\ruby{振舞}{ふる|まひ}は、
\ruby{日頃}{ひ|ごろ}にも
\ruby{似}{に}ぬ
\ruby{仕業}{し|わざ}なりしが、
\ruby{老}{おい}の
\ruby{眼敏}{め|ざと}き
\ruby[g]{吉右衛門}{きちゑもん}は、
\ruby{先刻}{さ|き}より
\ruby[g]{水野}{みづの}が
\ruby{思}{おも}ひあまりて、
\ruby{我知}{われ|し}らず
\ruby{長吁短歎}{ちや|うく|たん|〳〵}する
\ruby{其}{そ}の
\ruby{聲}{こゑ}を
\ruby{聞}{き}きて
\ruby{既}{すで}に
\ruby{覺}{さ}め
\ruby{居}{を}りつ、
\ruby{如何}{い|か}ばかり
\ruby{戀}{こひ}の
\ruby{山路}{やま|ぢ}の
\ruby{嶮}{けは}しきに
\ruby{惱}{なや}んで、
\ruby{若}{わか}き
\ruby{人}{ひと}の
\ruby[g]{可惜心}{あたらこヽろ}を
\ruby{傷}{きづ}つけ
\ruby{血}{ち}を
\ruby{流}{なが}す
\ruby{事}{こと}ぞやと、ひそかに
\ruby{憐}{あわれ}み
\ruby{居}{ゐ}たりしかば、ごと〳〵と
\ruby{雨戸引}{あま|ど|ひ}き
\ruby{明}{あ}けて
\ruby{外}{そと}に
\ruby{出}{い}づるをも、
\ruby{一度}{ひと|たび}は
\ruby{咎}{とが}めんとしたれど
\ruby{思}{おも}ひ
\ruby{{\GWI{u8fd4-k}}}{かへ}して
\ruby{咎}{とが}めず、
\ruby{大方}{おほ|かた}は
\ruby{病}{や}める
\ruby{人}{ひと}の
\ruby{上}{うへ}の
\ruby{氣}{き}にか〻りて、
\ruby{其}{そ}の
\ruby[g]{様子見}{ようすみ}にと
\ruby{行}{ゆ}くならんを、たゞ
\ruby{心儘}{こヽろ|まヽ}ならしめんこそ
\ruby[g]{慈悲}{なさけ}なるべけれと、
\ruby{睡}{ねむ}れるを
\ruby{裝}{よそほ}ひて
\ruby{咳嗽}{しは|ぶき}もせざりけれど、はや
\ruby[g]{水野}{みづの}が
\ruby{四五間}{し|ご|けん}も
\ruby{{\GWI{u9060-k}}}{とほ}く
\ruby{去}{さ}りしと
\ruby{覺}{おぼ}しき
\ruby{頃}{ころ}、
\ruby[g]{吉右衛門}{きちゑもん}は
\ruby{別}{べつ}に
\ruby[g]{吉右衛門}{きちゑもん}の
\ruby{思}{おも}ふところありてや、
\ruby{吾}{わ}が
\ruby{傍}{かたはら}の
\ruby{床}{とこ}に
\ruby{臥}{ふ}したるお
\ruby{濱}{はま}の
\ruby{寢顏}{ね|がほ}の、小さき
\ruby[g]{洋燈}{らんぷ}の
\ruby{光}{ひかり}に
\ruby{照}{て}らし
\ruby{出}{いだ}されたる、
\ruby{罪}{つみ}も
\ruby{無}{な}く
\ruby{美}{うつく}しきを
\ruby{見}{み}て、
\ruby{輕}{かろ}く
\ruby{歎}{たん}じたり。

\ruby[g]{水野}{みづの}は
\ruby{覺}{さ}めながら
\ruby{夢路}{ゆめ|ぢ}を
\ruby{辿}{たど}るがごとく、
\ruby{天}{てん}に
\ruby{明無}{あかり|な}く
\ruby{地}{ち}に
\ruby{色無}{いろ|な}き
\ruby{中}{なか}を、
\ruby{何者}{なに|もの}にか
\ruby{肝膽}{き|も}に
\ruby{糸}{いと}つけて
\ruby{牽}{ひ}かる〻やうなる
\ruby{云}{い}ひがたき
\ruby{恐}{おそ}ろしさ
\ruby{苦}{くる}しさを
\ruby{覺}{おぼ}えつ〻、
\ruby{例}{れい}のお
\ruby{澤}{さは}が
\ruby{家}{いえ}の
\ruby{前}{まへ}にさしか〻りたり。
\ruby{心}{こヽろ}に
\ruby{眼}{め}あればこそ
\ruby{物}{もの}は
\ruby{見}{み}ゆれど、
\ruby{眼}{め}に
\ruby{力}{ちから}は
\ruby{無}{な}くして
\ruby{知}{し}らぬ
\ruby{相}{すがた}は
\ruby{見}{み}えぬ
\ruby{黑暗々}{こく|あん|〳〵}たる
\ruby{眞}{しん}の
\ruby{闇}{やみ}に、
\ruby[g]{水野}{みづの}は
\ruby{歩}{あゆみ}をとゞめ
\ruby{眼}{め}を
\ruby{凝}{こ}らして
\ruby{窺}{うかゞ}へば、
\ruby{豫}{かね}て
\ruby{知}{し}れる
\ruby{彼}{か}の
\ruby{{\換字{寒}}竹}{かん|ちく}の
\ruby{藪疊}{やぶ|たゝみ}の
\ruby{開}{ひら}けたる
\ruby{間}{あひだ}より、
\ruby{圃}{はたけ}の
\ruby{先}{さき}に
\ruby{當}{あた}りて
\ruby{屋}{や}の
\ruby{棟}{むね}の
\ruby{低}{ひく}きが、
\ruby{曇}{くも}れる
\ruby{{\換字{空}}}{そら}に
\ruby{微}{かすか}に
\ruby{{\GWI{u900f-k}}}{す}きて
\ruby{立}{た}てり。

\ruby[g]{記臆}{おぼ{\GWI{u1b001}}}あればこそ
\ruby{辛}{から}くも
\ruby{歩}{ある}かる〻なれ、
\ruby{地}{ち}の
\ruby{底}{そこ}の
\ruby{磐}{いはほ}の
\ruby{内}{うち}にも
\ruby{入}{い}らばかくもアランかと
\ruby{思}{おも}はる〻
\ruby{闇}{やみ}の
\ruby{中}{なか}を、
\ruby{心}{こヽろ}あてばかりに
\ruby{此方}{こ|ヽ}と
\ruby{{\GWI{u9032-k}}}{すヽ}み
\ruby{行}{ゆ}きて、
\ruby{漸}{やうや}く
\ruby{草屋}{くさ|や}の
\ruby{横}{よこ}を
\ruby{{\GWI{u904e-k}}}{よぎ}らんとする
\ruby{時}{とき}、
\ruby{萬籟死}{ばん|らい|し}し
\ruby{盡}{つく}せる
\ruby{今突然}{いま|とつ|ぜん}として、

『ぎりぎりッ、ぎりぎりッ。
』

といふ
\ruby{怪}{あや}しき
\ruby{響}{ひゞき}したり。

\ruby{鳥}{とり}にあらず
\ruby{鼠}{ねずみ}にあらぬ
\ruby{其}{そ}の
\ruby{音}{おと}の、¥ル
\ruby{何}{なん}とも
\ruby{云}{い}へず
\ruby{物忌}{もの|いま}はしきに、
\ruby{思}{おも}はず
\ruby{慄然}{ぞ|つ}として
\ruby{耳}{みヽ}を
\ruby{立}{た}つれば、
\ruby{聲}{こゑ}は
\ruby{我{\GWI{u8fd1-k}}}{われ|ちか}き
\ruby{荒}{あ}れたる
\ruby{家}{いへ}の
\ruby{内}{うち}より
\ruby{來}{きた}りて、そも〳〵
\ruby{何}{なん}の
\ruby{夢}{ゆめ}にか
\ruby{怒}{いか}れる、
\ruby{彼}{か}の
\ruby{鬼}{おに}の
\ruby{如}{ごと}きお
\ruby{澤婆}{さは|ばヾ}の、
\ruby{笑顏}{ゑ|がほ}に
\ruby{見}{み}てさへも
\ruby{凄}{すさま}じく
\ruby{今{\換字{猶}}}{いま|なほ}
\ruby{殘}{のこ}れる
\ruby{彼}{か}のまばらなる
\ruby{長}{なが}き
\ruby{其齒}{その|は}を
\ruby{咬}{か}み
\ruby{鳴}{な}らせるにて、
\ruby{其音}{そ|れ}につヾいて
\ruby{{\換字{叉}}}{また}
\ruby{更}{さら}に、

『ウーン、ウーン。
』

と
\ruby{寐唸}{ねう|なり}りする
\ruby{其聲}{その|こゑ}は
\ruby{恨}{うら}むが
\ruby{如}{ごと}く
\ruby{詛}{のろ}ふが
\ruby{如}{ごと}く、
\ruby{満腔}{まん|こう}の
\ruby{怨毒}{\GWI{u1b001}ん|どく}を
\ruby{噴}{ふ}き
\ruby{出}{いだ}して、
\ruby{闇}{くら}きに
\ruby{{\GWI{u904a-k}}行}{いう|ぎやう}するあらゆる
\ruby{惡鬼}{あく|き}を
\ruby{喚}{よ}び
\ruby{集}{つど}へんとするにも
\ruby{似}{に}たれば、
\ruby[g]{水野}{みづの}は
\ruby{{\換字{㣆}}}{いや}が
\ruby{上}{うへ}にも
\ruby{心惡}{こヽろ|あし}くおぼえて、
\ruby{止}{や}めよかし、
\ruby{止}{や}めよかし、と
\ruby{急}{きふ}に
\ruby{念}{ねん}じたれど、

『ぎりぎりッ、ぎりぎりッ。
ウーン、ウーン。
』

といふ
\ruby{聲}{こゑ}は
\ruby[g]{執念}{しふね}くも
\ruby{起}{おこ}つて、
\ruby{我}{わ}が
\ruby{腦後}{ぼんの|くぼ}に
\ruby{襲}{おそ}ひか〻るがごとく
\ruby{{\GWI{u903c-k}}}{せま}るに、
\ruby{身}{み}も
\ruby{世}{よ}もあらず
\ruby{厭}{いと}はしく
\ruby{思}{おも}ひて、
\ruby{{\GWI{u8ffd-k}}}{お}はれ
\ruby[g]{心地}{ごこち}に
\ruby{歩}{あゆ}み
\ruby{去}{さ}らんとする
\ruby{折}{をり}しも、
\ruby{忽}{たちま}ち
\ruby{我}{わ}が
\ruby{五十子}{い|そ|こ}の
\ruby{家}{いへ}の
\ruby{其}{そ}の
\ruby{方}{かた}より、ひらりと
\ruby{物}{もの}の
\ruby{光}{ひか}りの
\ruby[g]{此方}{こなた}に
\ruby{射}{さ}し
\ruby{來}{きた}りたり。


\Entry{其十五}

% メモ 校正終了 2024-04-20
\原本頁{84-4}%
\ruby[||j>]{漆}{うるし}と
\ruby{黑}{くろ}き
\ruby{眼{\換字{前}}}{め|さき}の
\ruby{闇}{やみ}に、
%
ぱつと
\ruby{一}{ひ}ト
\ruby{刷毛}{は|け}の
\ruby[g]{光線}{ひかり}の
\ruby{散}{ち}つたるを、
%
いづくよりぞと
\ruby[g]{水野}{みづの}は
\ruby{見}{み}れば、
%
\ruby{人}{ひと}の
\ruby{歸}{かへ}るを
\ruby{{\換字{送}}}{おく}り
\ruby{出}{いだ}すと
\ruby{見}{み}えて、
%
\ruby[g]{五十子}{いそこ}が
\ruby{家}{いへ}の
\ruby{{\換字{戸}}}{と}の
\ruby{今}{いま}
\ruby{引}{ひき}
\ruby{開}{あ}けられたる
\ruby{其處}{そ|こ}より
\ruby[g]{洋燈}{らんぷ}の
\ruby{光}{ひかり}の
\ruby{晃然}{きら|り}と
\ruby{射}{さ}したるなり。

\原本頁{84-8}%
\ruby{問}{と}はでも
\ruby{知}{し}るべし、
%
\ruby{病者}{びやう|しや}
ある
\ruby{家}{いへ}を、
%
\ruby{如是}{か〻|る}
\ruby{時刻}{じ|こく}に% 原本通り「〻(二の字点、揺すり点)」
\ruby{人}{ひと}の
\ruby{出}{で}
\ruby{入}{い}りする
\ruby{事}{こと}、
%
\ruby{必}{かなら}ず
\ruby{凶}{きよう}
ありて
\ruby{吉}{きち}
ある
\ruby{事}{こと}
\ruby{無}{な}し。
%
\ruby{我}{わ}が
\ruby[g]{五十子}{いそこ}は
\ruby{抑}{そも}
\ruby{如何}{い|か}にか
したる。
%
\ruby{何}{なに}と
\ruby{無}{な}く
\ruby{堪}{た}へ
\ruby{難}{がた}き
\ruby{心地}{こ〻|ち}の% 原本通り「〻(二の字点、揺すり点)」
\ruby{爲}{し}て、
%
\ruby{我}{わ}が
\ruby{此處}{こ|〻}まで% 原本通り「〻(二の字点、揺すり点)」
\ruby{獨}{ひと}り
\ruby{{\換字{迷}}}{まよ}ひ
\ruby{出}{い}で
\ruby{來}{き}しも、
%
\ruby{世}{よ}にいふ
\ruby{蟲}{むし}の
\ruby{知}{し}らせし
といふ
\ruby{事}{こと}か、
%
たゞ% TODO 原本の「二の字点、揺すり点」に濁点のグリフが見つからないので「ゞ」
ならす
\ruby{動悸}{どう|き}の
\原本頁{85-1}\改行%
\ruby{打}{う}ちしも
\ruby{思}{おも}ひ
\ruby{當}{あた}りたりと、
%
\ruby{先}{ま}づ
\ruby{胸}{むね}を
\ruby{轟}{とゞろ}かして% TODO 原本の「二の字点、揺すり点」に濁点のグリフが見つからないので「ゞ」
\ruby[g]{彼方}{かなた}を
\ruby{見}{み}るに、
%
やがて
\ruby{{\換字{戸}}}{と}は
また
\ruby{引}{ひき}
\ruby{寄}{よ}せられて、
%
\ruby{{\換字{遠}}目}{とほ|め}の
\ruby{定}{さだ}かならねど
\ruby{四ッ目菱}{よ||め|びし}の%小文字の「ッ」
\ruby{紋}{もん}つきたる
\ruby{提灯}{ちやう|ちん}を
\ruby{片手}{かた|て}に、
%
\ruby{片手}{かた|て}には
\ruby{小}{ちひさ}き
\ruby{革鞄}{かば|ん}を
\ruby{持}{も}ちて、
%
ぽく〳〵と
\ruby[g]{此方}{こなた}に
\ruby{歩}{あゆ}み
\ruby{來}{きた}れるは
\ruby{疑}{うたがひ}もなく
\ruby[g]{尾竹}{をだけ}なり。

\原本頁{85-6}%
さては
いよ〳〵
\ruby[g]{五十子}{いそこ}に
\ruby{變}{へん}の
ありて、
%
\ruby{夜{\換字{半}}}{よ|は}の
\ruby{{\換字{扉}}}{と}を
た〻き% 原本通り「〻(二の字点、揺すり点)」
\ruby{招}{よ}び
\ruby{{\換字{迎}}}{むか}へたればこそ、
%
\ruby[g]{尾竹}{をだけ}の
\ruby{先刻}{さ|き}に
\ruby{來}{きた}りて
\ruby{今}{いま}
\ruby{歸}{かへ}るなるべけれ。
%
\ruby{歸}{かへ}るは
\ruby{吉}{よ}くてか
\ruby{將}{はた}
\ruby{凶}{あし}くて
\ruby{歟}{か}。
%
\ruby{嗚呼}{あ|〻}、% 原本通り「〻(二の字点、揺すり点)」
%
\ruby[g]{五十子}{いそこ}の
\ruby{病}{やまひ}は
\ruby{測}{はか}るべからずして、
%
\原本頁{85-9}\改行%
\ruby[g]{尾竹}{をだけ}の
\ruby{{\換字{技}}倆}{わ|ざ}は
\ruby{我}{われ}よく
\ruby{知}{し}れり。
%
\ruby{嗚呼}{あ|〻}、% 原本通り「〻(二の字点、揺すり点)」
%
\ruby{人}{ひと}の
\ruby{命}{いのち}!、
%
\ruby{我}{わ}が
\ruby{命}{いのち}!、
%
\ruby{定}{さだ}まりたる
\ruby{天}{てん}の
\ruby{數}{すう}は
\ruby{今}{いま}
\ruby{見}{み}ゆるかや!。
%
\ruby{他}{ひと}をも
\ruby{死}{し}なせじ、
%
\ruby{我}{われ}も
\ruby{死}{し}なじと、
%
\ruby{一念}{いち|ねん}の
\ruby{火}{ひ}を
\ruby{燃}{も}やし〻も% 原本通り「〻(二の字点、揺すり点)」
\ruby{{\換字{空}}}{あだ}となつて、
%
\ruby{他}{ひと}も
\ruby{死}{し}に、
%
\ruby{我}{われ}も
\ruby{死}{し}に
\ruby{果}{は}て〻、% 原本通り「〻(二の字点、揺すり点)」
%
\ruby{冷}{つめた}き
\ruby{{\換字{灰}}}{はひ}と% ルビは「はい」ではなく原本通り
なるべき
\ruby{時}{とき}の、
%
\ruby{{\換字{終}}}{つひ}に
\ruby{眼}{め}の
\ruby{{\換字{前}}}{まへ}には
\ruby{來}{きた}りたるかや。
%
\原本頁{86-2}\改行%
\ruby{{\換字{前}}世}{ぜん|せ}も
\ruby{知}{し}らず、
%
\ruby{後世}{ご|せ}も
\ruby{知}{し}らねど、
%
\ruby{此}{こ}の
\ruby{今}{いま}の
\ruby{世}{よ}は、
%
これまで
なりや、
%
\ruby{嗚呼}{あ|〻}% 原本通り「〻(二の字点、揺すり点)」
\ruby{殘}{のこ}り
\ruby{多}{おほ}くも
\ruby{恨多}{うらみ|おほ}くも、
%
これまでなりや、
%
これまでなりや。
%
と
\ruby{歩}{あゆ}まん
\ruby{意}{こ〻ろ}も% 原本通り「〻(二の字点、揺すり点)」
\ruby{無}{な}く
\ruby{言}{ものい}はん
\ruby{意}{こ〻ろ}も% 原本通り「〻(二の字点、揺すり点)」
\ruby{無}{な}くなりて、
%
\ruby[g]{水野}{みづの}は
\ruby{地}{つち}の
\ruby{上}{うへ}に
たゞ% TODO 原本の「二の字点、揺すり点」に濁点のグリフが見つからないので「ゞ」
\ruby{苟且}{かり|そめ}に
\ruby{立}{た}て
\ruby{置}{お}かれたる
\ruby{一}{ひと}つ
\ruby{杭}{ぐひ}の
\ruby{如}{ごと}く、
%
\ruby{少時}{しば|らく}
\ruby{茫然}{ばう|ぜん}として
\ruby{立}{た}ち
\ruby{居}{ゐ}けるが、
%
やがて
ばたりと
\ruby{倒}{たふ}れんと
したり。

\原本頁{86-7}%
されど
\ruby[g]{水野}{みづの}の
\ruby{自}{みづか}ら
\ruby{支}{さ〻}へて、% 原本通り「〻(二の字点、揺すり点)」
%
\ruby{辛}{から}くも
\ruby{思}{おもひ}を
\ruby{轉}{てん}じたる
\ruby{時}{とき}、
%
\ruby[g]{尾竹}{をだけ}は
\ruby{間{\換字{近}}}{あはひ|ちか}く
\ruby{{\換字{進}}}{す〻}み% 原本通り「〻(二の字点、揺すり点)」
\ruby{來}{きた}りしが、
%
\ruby{思}{おも}ひも
かけぬ
\ruby{闇}{やみ}の
\ruby{眞中}{ま|なか}に
\ruby{人}{ひと}の
\ruby{佇}{た〻ず}めるを% 原本通り「〻(二の字点、揺すり点)」
\ruby{認}{みと}めつ
\ruby[g]{愕然}{ぎよつ}として
\ruby{驚}{おどろ}き、
%
\ruby{提灯}{ちやう|ちん}の
\ruby{燈}{ひ}に
\ruby[g]{此方}{こなた}を
すかし
\ruby{見}{み}、

\原本頁{86-10}%
『
み、
%
み、
%
\ruby[g]{水野}{みづの}さん
ですか。
』

\原本頁{86-11}%
と
\ruby{顫}{ふる}へ
\ruby{聲}{ごゑ}に
\ruby{{\換字{尋}}}{たづ}ねたり。

\原本頁{87-1}%
\ruby{凡人}{ぼん|じん}の
\ruby{眼}{め}つき、
%
\ruby{凡人}{ぼん|じん}の
\ruby{口}{くち}つき、
%
\ruby{凡人}{ぼん|じん}の
\ruby{額}{ひたひ}、
%
\ruby{凡人}{ぼん|じん}の
\ruby{肩}{かた}、
%
\ruby{身長}{みの|たけ}も
\ruby{普{\換字{通}}}{つね|なみ}なれば、
%
\ruby{態度}{やう|す}も
\ruby{普{\換字{通}}}{つね|なみ}にて、
%
\ruby{何處}{ど|こ}に
\ruby{一}{ひと}つ
これといふ
ところも
\ruby{無}{な}き
\ruby{其}{そ}の
\ruby[g]{尾竹}{をだけ}の
\ruby{深}{ふか}くも
\ruby[g]{忘怖}{おそろしき}に% 「恐怖」の誤植のように思えるが原本通りにしておく
\ruby{魘}{おそ}はれたるにや、
%
\ruby{眉}{まゆ}を
\ruby{尾}{しり}
\ruby{下}{さが}りにし、
%
\原本頁{87-4}\改行%
\ruby{眼}{め}を
\ruby{壺}{つぼ}
\ruby{深}{ふか}くして、
%
\ruby{頸}{くび}を
\ruby{縮}{ちゞ}めつ〻% 原本通り「〻(二の字点、揺すり点)」% TODO 原本の「二の字点、揺すり点」に濁点のグリフが見つからないので「ゞ」
\ruby[g]{此方}{こなた}を
\ruby{見}{み}たる
\ruby{其}{そ}の
\ruby{怯}{おそ}れたる
\ruby{狀}{さま}の
いと
\ruby{醜}{みにく}きが、
%
\ruby{提灯}{ちやう|ちん}の
\ruby{火影}{ほ|かげ}に
ぼつと
\ruby{見}{み}えたるは、
%
\ruby{今}{いま}といふ
\ruby{今}{いま}のみ
\ruby{始}{はじ}めて
\ruby{{\換字{平}}凡}{よの|つね}ならず
\ruby[g]{水野}{みづの}が
\ruby{眼}{め}に
\ruby{映}{うつ}りぬ。

\Entry{其十六}

\ruby{尾竹}{お|だけ}の
\ruby{聲}{こゑ}は
\ruby{闇}{やみ}の
\ruby{寂寥}{しづ|かさ}に
\ruby{響}{ひゞ}きて、
\ruby{愚}{おろか}しくいと
\ruby{大}{おほ}きく
\ruby{聞}{きこ}えたるに、
\ruby{水野}{みづ|の}は
\ruby{何}{なん}となく
\ruby{厭}{いと}はしく
\ruby{感}{かん}じつ、こゝにて
\ruby{{\換字{又}}}{また}
\ruby{我}{われ}と
\ruby{此男}{こ|れ}との
\ruby{問}{と}ひつ
\ruby{答}{こた}へつせば、
\ruby{其}{そ}の
\ruby{聲}{こゑ}の
\ruby[g]{彼家}{かしこ}の
\ruby{人々}{ひと|〴〵}にも
\ruby{聞}{きこ}えんことを
\ruby{忌}{いま}はしく
\ruby{思}{おも}ひて、
\ruby[g]{言葉}{ことば}は
\ruby{無}{な}き
\ruby[g]{擧動}{そぶり}ばかりに
\ruby{尾竹}{お|だけ}を
\ruby{誘}{いざな}ひ、
\ruby{突}{つ}と
\ruby{外}{そと}の
\ruby{方}{かた}に
\ruby{去}{さ}らんとすれば、
\ruby{尾竹}{お|だけ}は
\ruby{慌}{あわ}てゝ
\ruby{先}{さき}に
\ruby{立}{た}つて、
\ruby{手}{て}に
\ruby{持}{も}てる
\ruby{提灯}{ちやう|ちん}に
\ruby{足元}{あし|もと}を
\ruby{照}{て}らしたり。

\ruby{共}{とも}に
\ruby{歩}{あゆ}むこと
\ruby{四五歩}{し|ご|ほ}ならずして、
\ruby{彼}{か}のお
\ruby{澤婆}{さは|ばゞ}が
\ruby{家}{いへ}の
\ruby{内}{うち}より、

『ギリギリツ。
』

といふ
\ruby{聲先}{こゑ|ま}づ
\ruby{聞}{きこ}えて、
\ruby{次}{つ}いで、

『ウーンウーン。
』

といふ
\ruby{寝唸}{ね|うな}りの
\ruby{聞}{きこ}ゆれば、
\ruby{尾竹}{お|だけ}は
\ruby{思}{おも}はずもピクリと
\ruby{顫}{ふる}へて、
\ruby{手}{て}にしたる
\ruby{提灯}{ちやう|ちん}に
\ruby{烈}{はげ}しき
\ruby{浪}{なみ}を
\ruby{打}{う}たせつ、

『ナ、
\ruby{何}{なん}でしやう
\ruby{彼}{あ}の
\ruby{音}{おと}は?。
』

と
\ruby{振}{ふ}り
\ruby{{\換字{返}}}{かへ}つて
\ruby{水野}{みづ|の}に
\ruby{{\換字{尋}}}{たづ}ねたり。

されど
\ruby{水野}{みづ|の}は
\ruby{尾竹}{お|だけ}が
\ruby{此}{こ}の
\ruby{言葉}{こと|ば}を、
\ruby{閑事}{むだ|ごと}なりと
\ruby{云}{い}はぬばかりに、たゞ
\ruby{無言}{む|ごん}をもてあしらひ
\ruby{棄}{す}て、おのが
\ruby{歩}{あゆ}まんとする
\ruby{方}{かた}に
\ruby{歩}{あゆ}み
\ruby{去}{さ}りながら、

『
\ruby{岩崎}{いは|さき}は
\ruby{何樣}{ど|う}でございます、よろしいのですか?。
』

と
\ruby{先刻}{さ|き}より
\ruby{此}{こ}の
\ruby{醫}{い}の
\ruby{樣子}{やう|す}に
\ruby{大事無}{だい|じ|な}しとは
\ruby{察}{さつ}したれど、
\ruby{問}{と}はんとして
\ruby{一刻}{いつ|こく}も
\ruby{忘}{わす}れざりし
\ruby{問}{とひ}を
\ruby{發}{はつ}すれば、
\ruby{眞{\換字{情}}}{まこ|と}
\ruby{餘}{あま}りし
\ruby{其}{そ}の
\ruby{言葉}{こと|ば}の
\ruby[g]{自然}{おのづ}と
\ruby{威}{ゐ}あるやうなるに
\ruby{尾竹}{お|だけ}は
\ruby{壓}{お}されて、
\ruby{今我}{いま|わ}が
\ruby{口}{くち}に
\ruby{出}{いだ}したる
\ruby{問}{とひ}の
\ruby{答}{こたへ}を
\ruby{得}{\換字{江}}ざるをも、また
\ruby{水野}{みづ|の}が
\ruby{如何}{い|か}なれば
\ruby{如是時分}{かゝ|る|じ|ぶん}に
\ruby{此邊}{この|あたり}には
\ruby{佇}{たゝず}み
\ruby{居}{ゐ}たるやと
\ruby{{\換字{尋}}}{たづ}ねまほしく
\ruby{思}{おも}ひ
\ruby{居}{ゐ}たるをも
\ruby{盡}{こと〴〵}く
\ruby{皆忘}{みな|わす}れ
\ruby{果}{は}てゝ、

『いや
\ruby{御{\換字{尋}}問}{お|たづ|ね}が
\ruby{無}{な}くても
\ruby{夜}{よ}でも
\ruby{明}{あ}けましたら
\ruby{一寸上}{ちよ|つと|あが}つてなりと
\ruby{申上}{まを|しあ}げやうと
\ruby{思}{おも}つて
\ruby{居}{を}りましたが、
\ruby{看護婦}{かん|ご|ふ}の
\ruby[g]{注意}{ちうい}からして
\ruby{御使}{おつ|かひ}があつたので、
\ruby{今}{いま}しがた
\ruby{出}{で}て
\ruby{見}{み}ると、
\ruby{實}{じつ}は
\ruby{甚}{はなは}だ
\ruby{面白}{おも|しろ}く
\ruby{無}{な}くなつて
\ruby{居}{ゐ}るのです。
\ruby{勿論}{もち|ろん}
\ruby{今}{いま}が
\ruby{今}{いま}といふやうなことはありませんが、
\ruby{全體}{ぜん|たい}が
\ruby{{\換字{丈}}夫}{ぢや|うぶ}づくりといふ
\ruby{方}{かた}では
\ruby{無}{な}いのですのに、たゞ
\ruby{氣性}{きし|やう}が
\ruby{確乎}{しつ|かり}として
\ruby{居}{ゐ}らるゝばかりで、
\ruby{今}{いま}までは
\ruby{病苦}{びや|うく}に
\ruby{負}{ま}けずに
\ruby{居}{ゐ}られたところ、
\ruby{何樣}{ど|う}して、
\ruby{精神作用}{せい|しん|さ|よう}だつて
\ruby{限}{かぎり}のあるものですもの、
\ruby{連日}{れん|じつ}の
\ruby{高度}{かう|ど}の
\ruby{熱}{ねつ}では
\ruby{耐}{たま}りません、とう〳〵
\ruby{堪}{こた}へに
\ruby{堪}{こた}へきれなくなられましたのです。
さあ
\ruby{左樣}{さ|う}なると
\ruby{其}{それ}と
\ruby{同時}{どう|じ}に、
\ruby{自然}{し|ぜん}と
\ruby{來}{き}て
\ruby{居}{ゐ}た
\ruby{衰{\換字{弱}}}{すゐ|じやく}が、
\ruby{俄然}{が|ぜん}と
\ruby{外}{そと}に
\ruby{現}{あらは}れてまゐりましたので、
\ruby{一體}{いつ|たい}に
\ruby{何處}{ど|こ}も
\ruby{彼處}{かし|こ}も
\ruby{惡}{わる}くなつて
\ruby{來}{き}たといふやうな
\ruby{譯}{わけ}です。
しかし
\ruby{幸}{さいはひ\ }
\ruby{特}{ こと}に
\ruby{肺}{はい}が
\ruby{惡}{わる}くなつたとか
\ruby{心臓}{しん|ざう}が
\ruby{惡}{わる}くなつたとか
\ruby{云}{い}ふのではありませんから、まだ〳〵
\ruby{十分有望}{じう|ぶん|いう|ばう}なので、
\ruby{云}{い}はゞ
\ruby{彼樣}{あ|ゝ}いふ
\ruby{大病}{たい|びやう}にかゝつた
\ruby{患者}{かん|じや}の、
\ruby{何樣}{ど|う}も
\ruby{經{\換字{過}}}{けい|くわ}しなければならぬ
\ruby{已}{や}むを
\ruby{得}{\換字{江}}ざる
\ruby{場合}{ば|あひ}なのです。
』

と
\ruby{一{\換字{半}}}{いつ|ぱん}は
\ruby{水野}{みづ|の}を
\ruby{慰}{なぐさ}め、
\ruby{一{\換字{半}}}{いつ|ぱん}はおのれを
\ruby{辯護}{べん|ご}するが
\ruby{如}{ごと}く、
\ruby{素人解}{しろ|うと|わか}りすべきことを
\ruby{條理賢}{でう|り|かしこ}く
\ruby{{\換字{述}}}{の}べたり。

\ruby{水野}{みづ|の}は
\ruby{五十子}{い|そ|こ}の
\ruby{容態}{よう|だい}あしゝと
\ruby{聞}{き}きて、さてこそと
\ruby{胸}{むね}を
\ruby{躍}{をど}らせつ、
\ruby{先}{ま}ず
\ruby{悲}{かな}しくも
\ruby{腹立}{はら|だ}たしきおもひして、はや
\ruby{苛々}{いら|〳〵}と
\ruby{心}{こゝろ}は
\ruby{烈}{はげ}しくなり、
\ruby{此}{こ}の
\ruby{醫者}{い|しや}の
\ruby{{\換字{技}}鈍}{わざ|にぶ}きを
\ruby{怒}{いか}るとにはあらねど、
\ruby{其}{そ}の
\ruby{言葉巧}{こと|ばた|くみ}なるが
\ruby{小憎}{こ|にく}らしくて、

『
\ruby{已}{や}むを
\ruby{得}{\換字{江}}ざる
\ruby{場合}{ば|あひ}で!。
\ruby{成程御{\換字{道}}理}{なる|ほど|ご|もつ|とも}です、
\ruby{已}{や}むを
\ruby{得}{\換字{江}}ざる
\ruby{場合}{ば|あひ}で!。
まかり
\ruby{間{\換字{違}}}{ま|ちが}つて
\ruby{何樣}{ど|う}なりましても、
\ruby{勿論}{もち|ろん}みんな
\ruby{已}{や}むを
\ruby{得}{\換字{江}}ざる
\ruby{場合}{ば|あひ}ですナ。
』

と、
\ruby{一ㇳ當}{ひ| |あて}
\ruby{當}{あ}つれば
\ruby[g]{尾竹}{おだけ}は
\ruby{驚}{おどろ}き、
\ruby{{\換字{平}}日}{ひ|ごろ}は
\ruby{物柔}{もの|やはら}かにして
\ruby{斯樣}{か|う}は
\ruby{無}{な}かりし
\ruby{人}{ひと}の、
\ruby{何}{なん}たる
\ruby{氣}{き}の
\ruby{焦}{い}れかたぞやと
\ruby{呆}{あき}れながら、

『
\ruby{左樣御取}{さ|う|お|と}りになつては
\ruby{困}{こま}ります。
わたくしが
\ruby{責任}{せき|にん}を
\ruby{{\換字{逃}}}{のが}れやうとして
\ruby{申}{まを}したのではござりません。
わたくしが
\ruby{其樣}{そ|ん}なもので
\ruby{無}{な}いことは
\ruby{御承知}{ごし|よう|ち}でござりましやう。
\ruby[g]{小生}{わたくし}は
\ruby[g]{小生}{わたくし}の
\ruby{及}{およ}ぶ
\ruby{限}{かぎ}りの
\ruby{力}{ちから}を
\ruby{盡}{つく}して
\ruby{居}{を}りますのです。
』

と
\ruby{疾辯}{はや|くち}に
\ruby{言}{い}ひたる
\ruby{其聲}{その|こゑ}は
\ruby{眞}{まこと}に
\ruby{切}{せつ}なげに
\ruby{泣}{な}きさうにも
\ruby{聞}{きこ}えて、
\ruby{{\換字{技}}}{わざ}こそ
\ruby{庸常}{よの|つね}にして
\ruby{人}{ひと}に
\ruby{挺}{ぬきん}でもせざれ、
\ruby{心}{こゝろ}は
\ruby{正直}{しやう|ぢき}にして
\ruby{自}{みづか}ら
\ruby{欺}{あざむ}かざる
\ruby{君子}{くん|し}なるを
\ruby{示}{しめ}せり。

\ruby{水野}{みづ|の}は
\ruby{流石}{さす|が}にこれに
\ruby{氣}{き}の
\ruby{毒}{どく}になりて、

『ヤ、
\ruby{先生}{せん|せい}、
\ruby{御氣}{お|き}に
\ruby{御{\換字{留}}}{お|と}めなすつてはいけません。
\ruby{先生}{せん|せい}の
\ruby{御誠實}{ご|せい|じつ}な
\ruby{事}{こと}はよく
\ruby{存}{ぞん}じて
\ruby{居}{を}ります。
\ruby{{\換字{猶}}}{なほ}
\ruby{此上}{この|うへ}とも
\ruby{何分御願}{なに|ぶん|お|ねが}ひ
\ruby{申}{まをし}ます。
』

と
\ruby{和}{やは}らかに
\ruby{云}{い}へば、

『
\ruby{左樣仰}{さ|う|おつし}あつて
\ruby{下}{くだ}さればまことに
\ruby{滿足}{まん|ぞく}でございます。
\ruby{如何樣}{い|か|やう}にも
\ruby{此上}{この|うへ}
\ruby{{\換字{猶}}}{なほ}
\ruby{盡力}{じん|りよく}を
\ruby{辭}{じ}しませぬ。
\ruby{併}{しか}しなか〳〵の
\ruby{重體}{ぢう|たい}の
\ruby{事}{こと}ですから、
\ruby{先日}{せん|じつ}の
\ruby[g]{學士}{がくし}にも
\ruby{御見}{お|み}せになつては?、』

と
\ruby{腹}{はら}の
\ruby{底}{そこ}に
\ruby{毒無}{どく|な}き
\ruby{人}{ひと}の、はや
\ruby{胸}{むな}もとにも
\ruby{蟠}{わだかま}りなき
\ruby{挨拶}{あい|さつ}なり。

『
\ruby{非常}{ひじ|やう}に
\ruby{惡}{わる}い
\ruby{方}{はう}へ
\ruby{{\換字{進}}}{すゝ}みまして?。
』

『いや、
\ruby{今}{いま}いけないといふのでは
\ruby{無}{な}いのですが、
\ruby{何樣}{ど|う}も
\ruby{前申}{まへ|まを}した
\ruby{{\換字{通}}}{とほ}りですから
\ruby{相良}{さが|ら}さんにも…………
\ruby{如何}{い|か}にも
\ruby{衰{\換字{弱}}}{すゐ|じやく}が
\ruby{急}{きふ}に
\ruby{甚}{ひど}く
\ruby{現}{あらは}れて
\ruby{來}{き}ましたから。
』

と
\ruby{聞}{き}くや
\ruby{否}{いな}や
\ruby{水野}{みづ|の}は
\ruby{心中}{しん|ちう}に
\ruby{疑}{うたが}ひて、
\ruby{衰{\換字{弱}}}{すゐ|じやく}は
\ruby{漸々}{やう|やく}にこそ
\ruby{來}{きた}るべきなれ、
\ruby{急}{きふ}に
\ruby{甚}{ひど}く
\ruby{現}{あらは}るゝものにや、
\ruby{醫}{い}ならねば
\ruby{我知}{われ|し}らねど、と
\ruby{一度}{ひと|たび}は
\ruby{{\換字{迷}}}{まよ}ひしが、
\ruby{惑}{まど}ひて
\ruby{{\換字{益}}無}{{\換字{江}}き|な}ければ、
\ruby{一瞬}{いつ|しゆん}に
\ruby{其}{そ}の
\ruby{心}{こゝろ}を
\ruby{決}{けつ}して、

『
\ruby{勿論直}{もち|ろん|すぐ}に
\ruby{來}{き}て
\ruby{診}{み}て
\ruby{貰}{もら}ひましやう。
』

と
\ruby{云}{い}ひ
\ruby{{\換字{終}}}{をは}わつて
\ruby{一禮}{いち|れい}するかと
\ruby{見}{み}えしが、
\ruby{忽}{たちま}ち
\ruby{其姿}{その|すがた}は
\ruby{闇}{やみ}に
\ruby{隱}{かく}れて
\ruby{眞黑}{まつ|くろ}の
\ruby{中}{うち}に
\ruby{走}{は}せ
\ruby{去}{さ}れば、
\ruby[g]{尾竹}{おだけ}は
\ruby{提灯}{ちやう|ちん}を
\ruby{手}{て}にしたるま〻、うつかりと
\ruby{路央}{みち|なか}に
\ruby{獨}{ひと}り
\ruby{立}{た}つて、
\ruby{黑白}{あ|や}なき
\ruby{暗}{くら}さに
\ruby{水野}{みづ|の}の
\ruby{下駄}{げ|た}の
\ruby{音}{おと}の、
\ruby{早}{はや}くも
\ruby{隔}{へだ}たり
\ruby{行}{ゆ}く
\ruby{方}{かた}を
\ruby{見}{み}えもせぬに
\ruby{永}{なが}く
\ruby{見{\換字{送}}}{み|おく}りたり。


\Entry{其十七}

\ruby{一度}{いち|ど}あることは
\ruby{二度}{に|ど}ありといふ
\ruby{世}{よ}の
\ruby{諺}{ことわざ}の
\ruby{人}{ひと}を
\ruby{欺}{あざむ}かず、
\ruby{水野}{みづ|の}はふたゝび
\ruby{熬}{い}りつくが
\ruby{如}{ごと}き
\ruby{憂}{うれひ}を
\ruby{抱}{いだ}いて
\ruby{南方}{みな|み}に
\ruby{走}{はし}りけるが、
\ruby{闇夜}{やみ|よ}の
\ruby{{\換字{道}}}{みち}の
\ruby{捗取}{はか|ど}らずして、その
\ruby{相良}{さが|ら}が
\ruby{家}{いへ}を
\ruby{訪}{と}ひし
\ruby{時}{とき}は
\ruby{既}{すで}に
\ruby{遲}{おそ}く、
\ruby{舎}{いへ}の
\ruby{内}{うち}はまだ
\ruby{燈火無}{とも|しび|な}くてはの
\ruby{頃}{ころ}ながら、
\ruby{{\換字{戸}}外}{そ|と}は
\ruby{既人顏定}{はや|ひと|がほ|さだ}かなるほどになりて、かつて
\ruby{島木}{しま|き}の
\ruby{寓}{やど}より
\ruby{歸}{かへ}るさに
\ruby{訪}{と}ひし
\ruby{時}{とき}と
\ruby{同}{おな}じほどの
\ruby{明}{あか}るさとはなり
\ruby{居}{ゐ}たり。

た〻かれて
\ruby{怒}{おこ}らぬものは
\ruby{醫師}{い|し}の
\ruby{家}{いへ}と、
\ruby{憚}{はゞか}りも
\ruby{無}{な}く
\ruby{打敲}{うち|たゝ}けば、
\ruby{思}{おも}ひのほかに
\ruby{早}{はや}く
\ruby{{\換字{返}}事}{へん|じ}して、
\ruby{立出}{たち|い}でたるは
\ruby{前}{さき}の
\ruby{日窘}{ひく|るし}めやりたる
\ruby{彼}{か}の
\ruby{盤臺面}{ばん|だい|づら}の
\ruby{書生}{しよ|せい}なり。
\ruby{我}{われ}を
\ruby{侮}{あなど}りがたき
\ruby{男}{をとこ}と
\ruby{思}{おも}ひ
\ruby{{\換字{込}}}{こ}みてや
\ruby{挨拶}{あい|さつ}も
\ruby{慇懃}{いん|ぎん}に
\ruby{愛想}{あい|さう}よければ、おのづから
\ruby{物}{もの}も
\ruby{云}{い}ひ
\ruby{易}{やす}くて、わざ〳〵
\ruby{來}{きた}れる
\ruby{所以}{ゆゑ|ん}を
\ruby{手短}{てみ|じか}に
\ruby{{\換字{述}}}{の}べ、さて
\ruby{先生}{せん|せい}の
\ruby{御來診}{ご|らい|しん}をと
\ruby{乞}{こ}へば、
\ruby{書生}{しよ|せい}は
\ruby{困}{こま}りきつたる
\ruby{顏}{かほ}つきして、

『
\ruby{實}{じつ}は
\ruby{先生}{せん|せい}はたつた
\ruby{今出}{いま|で}て
\ruby{行}{ゆ}かれたのです、やはり
\ruby{病家}{びやう|か}の
\ruby{急}{きう}の
\ruby{{\換字{迎}}}{むか}へを
\ruby{受}{う}けられて。
しかし
\ruby{行}{ゆ}かれた
\ruby{先}{さき}が
\ruby{餘計{\換字{遠}}}{よ|けい|とほ}いところでもありませんから、
\ruby{二時間}{に|じ|かん}も
\ruby{立}{た}つ
\ruby{中}{うち}には
\ruby{歸}{かへ}らる〻でしやう。
\ruby{歸}{かへ}られたら
\ruby{必}{かなら}ず
\ruby{左様申}{さ|う|まを}しまして、
\ruby{屹度回診}{きつ|とく|わい|しん}になる
\ruby{様}{やう}に
\ruby{致}{いた}しましやう。
』

と
\ruby{云}{い}ひ
\ruby{{\換字{終}}}{をは}りしが、
\ruby{水野}{みづ|の}が
\ruby{面}{おもて}に
\ruby{難色}{なん|しよく}あるを
\ruby{見}{み}て、

『
\ruby{勿論先生}{もち|ろん|せん|せい}の
\ruby{歸}{かへ}らる〻まで、
\ruby{此處}{こ|〻}に
\ruby{御待}{お|まち}なすつていらしつて、
\ruby{御直接}{ご|ぢ|き}に
\ruby{御頼}{お|たの}みなさるとも
\ruby{其}{それ}は
\ruby{御随意}{ご|ずゐ|い}です。
』

と
\ruby{云}{い}ひ
\ruby{足}{た}したるは、よく〳〵
\ruby{此}{こ}の
\ruby{意地{\換字{強}}}{い|ぢ|つよ}き
\ruby{客}{かく}の
\ruby{執念}{しふ|ね}きに
\ruby{凝}{こ}りて、ふたゝび
\ruby{前}{さき}の
\ruby{日}{ひ}の
\ruby{如}{ごと}く
\ruby{其}{そ}の
\ruby{怒}{いか}りを
\ruby{惹}{ひ}く
\ruby{事}{こと}などの
\ruby{無}{な}からんやうにと、
\ruby{勉}{つと}めて
\ruby{意}{こゝろ}を
\ruby{用}{もち}ゐたりと
\ruby{見}{み}えたり。

\ruby{書生}{しよ|せい}の
\ruby{言}{い}へるところは
\ruby{全}{まつた}く
\ruby{僞}{いつはり}ならず
\ruby{見}{み}ゆるに、
\ruby{世}{よ}に
\ruby{行}{おこな}はる〻
\ruby{醫}{い}の
\ruby{忙}{せは}しくして
\ruby{暇無}{いとま|な}きは
\ruby[g]{如何}{いかん}ともすべからざることながら
\ruby{差當}{さし|あた}つて
\ruby{今}{いま}を
\ruby{何}{なに}とせんと、
\ruby{水野}{みづ|の}は
\ruby{礑}{はた}と
\ruby{行}{ゆ}き
\ruby{詰}{つま}りて、あたかも
\ruby{帆船}{ほ|ぶね}に
\ruby{舵}{かぢ}を
\ruby{失}{うしな}ひ、
\ruby{奔車}{ほん|しや}に
\ruby{轄}{くさび}を
\ruby{抜}{ぬ}かれるごとく、
\ruby{言}{い}はんかた
\ruby{無}{な}き
\ruby{心細}{こゝろ|ぼそ}さを
\ruby{覺}{おぼ}えて、
\ruby{憮然}{ぶ|ぜん}として
\ruby{言}{ことば}も
\ruby{無}{な}く
\ruby{物}{もの}を
\ruby{思}{おも}ひたり。

\ruby{書生}{しよ|せい}は
\ruby{水野}{みづ|の}の
\ruby{容子}{よう|す}を
\ruby{見}{み}て
\ruby{氣}{き}の
\ruby{毒}{どく}さに
\ruby{堪}{た}へでや、

『
\ruby{{\換字{遠}}路}{ゑん|ろ}のところを
\ruby{御來臨}{お|い|で}になつたのに
\ruby{生憎}{あひ|にく}で、
\ruby{如何}{い|か}にも
\ruby{御氣}{お|き}の
\ruby{毒}{どく}でございますが、
\ruby{必}{かなら}ず
\ruby{小生}{わた|くし}は
\ruby{左様申}{さ|う|まを}しまして、
\ruby{是非}{ぜ|ひ}とも
\ruby{回診}{くわい|しん}になるやうに
\ruby{致}{いた}しまする。
\ruby{時間}{じ|かん}のところは
\ruby{兎}{と}に
\ruby{角}{かく}、
\ruby{必}{かなら}ず
\ruby{診}{み}てあげますことは
\ruby{診}{み}てあげますやう、これは
\ruby{小生}{わた|くし}が
\ruby{御受合申}{お|うけ|あひ|まを}して
\ruby{左様}{さ|う}いたしますから。
』

と、
\ruby{前}{さき}の
\ruby{日}{ひ}とは
\ruby{打}{う}つて
\ruby{變}{かは}つて
\ruby{親切}{しん|せつ}に
\ruby{言}{い}ひ
\ruby{{\換字{呉}}}{く}る〻、その
\ruby{言葉}{こと|ば}には
\ruby[g]{力}{ちから}あり、その
\ruby{様子}{やう|す}には
\ruby{勢}{いきほひ}あるに、
\ruby{今}{いま}は
\ruby{此}{こ}の
\ruby{男}{をとこ}を
\ruby{頼}{たの}まんより
\ruby{他}{ほか}の
\ruby{{\換字{道}}}{みち}なければ、
\ruby{水野}{みづ|の}はいと
\ruby{懇切}{ねん|ごろ}に
\ruby{頼}{たの}み
\ruby{聞}{きこ}えて、
\ruby{是非無}{ぜ|ひ|な}くも
\ruby{元來}{もと|き}し
\ruby{{\換字{道}}}{みち}へ
\ruby{引{\換字{返}}}{ひつ|かへ}したり。

\ruby{戀人}{こひ|ゞと}の
\ruby{病}{やまひ}は
\ruby{前}{さき}の
\ruby{日}{ひ}より
\ruby{凶}{あし}きかたへ
\ruby{{\換字{進}}}{すゝ}めるなり、
\ruby{頼}{たの}む
\ruby{醫}{い}は
\ruby{他}{た}に
\ruby{出}{い}でゝ
\ruby{家}{いへ}にあらぬなり、
\ruby[g]{夢見}{ゆめみ}は
\ruby{忌}{いま}はしかりしなり、
\ruby{胸}{むね}は
\ruby{騒}{さわ}ぎしなり、
\ruby{若}{もし}やと
\ruby{思}{おも}ひしことは
\ruby{不思議}{ふ|し|ぎ}にも
\ruby{中}{あた}りしなり、
\ruby{{\換字{弱}}}{よわ}りかへれる
\ruby{五十子}{い|そ|こ}に
\ruby{一應}{いち|おう}の
\ruby{手當}{て|あて}して
\ruby{歸}{かへ}れる
\ruby{尾竹}{を|だけ}よりは
\ruby{心}{こゝろ}にかゝる
\ruby{言}{ことば}を
\ruby{聞}{き}きしなり、
\ruby{氣味}{き|み}あしき
\ruby{狗}{いぬ}は
\ruby{前表}{ぜん|ぺう}かとおぼしく
\ruby{吠}{ほ}えに
\ruby{吠}{ほ}えしなり、
\ruby{無心}{む|しん}のお
\ruby{濱}{はま}は
\ruby{我}{わ}が
\ruby{五十子}{い|そ|こ}の
\ruby{{\換字{遠}}方}{とほ|く}へ
\ruby{行}{ゆ}かんことを
\ruby{無心}{む|しん}に
\ruby{云}{い}へるなり、
\ruby{氣}{き}にかゝることのみの
\ruby{何}{なん}ぞ
\ruby{多}{おほ}きやと、
\ruby{水野}{みづ|の}は
\ruby{此等}{これ|ら}の
\ruby{事}{こと}を
\ruby{思}{おも}ひつゞけつゝ、
\ruby{恰}{あたか}も
\ruby{前}{さき}の
\ruby{日}{ひ}と
\ruby{同}{おな}じ
\ruby{曉}{あした}の、
\ruby{今日}{け|ふ}は
\ruby{風無}{かぜ|な}くて
\ruby{曇}{くも}り
\ruby{空}{ぞら}の
\ruby{少}{すこ}し
\ruby{闇}{くら}きのみが
\ruby{異}{ことな}れる
\ruby{同}{おな}じ
\ruby{時刻}{ころ|あひ}に、
\ruby{同}{おな}じく
\ruby{人{\換字{通}}}{ひと|ゝほ}り
\ruby{{\換字{猶}}}{なほ}
\ruby{少}{すくな}き
\ruby{並木}{なみ|き}の
\ruby{{\換字{道}}}{みち}を
\ruby{首}{かうべ}を
\ruby{垂}{た}れて
\ruby{力無}{ちから|な}く
\ruby{行}{ゆ}き
\ruby{盡}{つく}しつ、
\ruby{吾妻橋}{あ|づま|ばし}の
\ruby{方}{かた}に
\ruby{去}{さ}らんとする
\ruby{時}{とき}、
\ruby{突然}{とつ|ぜん}として
\ruby{人}{ひと}の
\ruby{我}{わ}が
\ruby{手}{て}を
\ruby{執}{と}るありて、しかも
\ruby{執}{と}られし
\ruby{我}{わ}が
\ruby{手首}{て|くび}に、ざらりと
\ruby{物}{もの}の
\ruby{觸}{さは}りたれば、
\ruby{何}{なん}ぞと
\ruby{驚}{おどろ}きて
\ruby{顧}{かへり}みるに、
\ruby{骨露}{ほね|あらは}に
\ruby{萎}{しな}び
\ruby{枯}{から}びて
\ruby{冷}{つめた}き
\ruby{細}{ほそ}き
\ruby{手}{て}に
\ruby{我}{わ}が
\ruby{手}{て}は
\ruby{捉}{とら}へ
\ruby{居}{を}られて、
\ruby{其}{そ}の
\ruby{手首}{て|くび}に
\ruby{掛}{か}けられ
\ruby{居}{ゐ}たる
\ruby{黑}{くろ}き
\ruby{木}{き}の
\ruby{數珠}{ず|ゞ}の
\ruby{我}{わ}が
\ruby{手}{て}に
\ruby{滑}{すべ}りて
\ruby{落}{お}ちかゝれるなり。


\Entry{其十八}

『あなた!。
いけません、いけません、
\ruby{信}{しん}を
\ruby{御冷}{お|さ}ましなすつては!。
\ruby{此處}{こ|ゝ}を
\ruby{御{\換字{通}}}{お|とほ}りになさりながら、
\ruby{御参詣}{ご|さん|けい}もなさらないんなんて、
\ruby{第一}{だい|いち}
\ruby{勿體無}{もつ|たい|な}い
\ruby{事}{こと}ではございませんか、さあ、
\ruby{御一緖}{ご|いつ|しよ}に
\ruby{詣}{まゐ}りましやう』

と
\ruby{{\換字{遮}}}{しや}に
\ruby{無}{む}に
\ruby{我}{わ}が
\ruby{手}{て}を
\ruby{牽}{ひ}きに
\ruby{牽}{ひ}くは、
\ruby{{\換字{過}}}{すぎ}し
\ruby{日}{ひ}
\ruby{淺草寺}{せん|さう|じ}の
\ruby{御堂}{み|だう}に
\ruby{普門品}{ふ|もん|ぼん}を
\ruby{誦}{じゆ}して、
\ruby{我}{われ}と
\ruby{共}{とも}に
\ruby{痛}{いた}く
\ruby{書生}{しよ|せい}に
\ruby{罵}{のゝし}られたる、
\ruby{彼}{か}の
\ruby{頭髮薄}{か|み|うす}く
\ruby{額脫}{ひたひ|ぬ}け
\ruby{上}{あが}がりて
\ruby{鼻細}{はな|ほそ}き
\ruby{{\換字{貧}}相}{ひん|さう}の
\ruby{老人}{らう|じん}なり。

\ruby{一樹}{いち|じゆ}の
\ruby{蔭}{かげ}に
\ruby{憩}{いこ}ひ
\ruby{一河}{いち|が}の
\ruby{流}{なが}れを
\ruby{掬}{むす}ぶも
\ruby{他生}{たし|やう}の
\ruby{緣}{えん}といへば、まして
\ruby{一堂}{いち|だう}の
\ruby{内}{うち}に
\ruby{同}{おな}じ
\ruby{御佛}{み|ほとけ}を
\ruby{頼}{たの}み
\ruby{奉}{たてまつ}りて、しかも
\ruby{假初}{かり|そめ}ながら
\ruby{言葉}{こと|ば}をさへ
\ruby{{\換字{交}}}{かは}したる
\ruby{中}{なか}なれば、
\ruby{呼}{よ}びかけられたりとて
\ruby{怪}{け}しうはあらぬながら、
\ruby{手}{て}を
\ruby{執}{と}りて
\ruby{我}{われ}を
\ruby{{\換字{伴}}}{ともな}はんとする
\ruby{擧動}{ふる|まひ}の、
\ruby{馴}{な}れ〳〵しきに
\ruby{{\換字{過}}}{す}ぎたるやうにも
\ruby{思}{おも}はるゝに、
\ruby{水野}{みづ|の}は
\ruby{一度}{ひと|たび}は
\ruby{之}{これ}を
\ruby{異}{あやし}みしが、たゞ〳〵おのが
\ruby{信心}{しん|〴〵}の
\ruby{同行}{どう|ぎやう}とせんとするほかには、
\ruby{何}{なん}の
\ruby{念}{ねん}も
\ruby{無}{な}かるべき
\ruby{其}{そ}の
\ruby{{\換字{道}}理}{もつ|とも}らしく
\ruby{眞面目}{ま|じ|め}らしき
\ruby{顏}{かほ}の
\ruby{他事無}{た|じ|な}く
\ruby{正直氣}{しやう|ぢき|げ}なる
\ruby{樣子}{やう|す}を
\ruby{見}{み}ては、
\ruby{何}{なん}の
\ruby{故}{ゆゑ}とは
\ruby{無}{な}けれど
\ruby{此}{こ}の
\ruby{老}{お}いたる
\ruby{人}{ひと}の
\ruby{意}{こゝろ}に
\ruby{背}{そむ}かん
\ruby{氣}{き}にはなれずして、
\ruby{引}{ひ}かるゝが
\ruby{儘}{まゝ}に
\ruby{無言}{む|ごん}に
\ruby{從}{したが}ひ
\ruby{行}{ゆ}けり。

『
\ruby{世}{よ}が
\ruby{澆季}{す|ゑ}になつて
\ruby{居}{を}りますのですもの、
\ruby{御同樣}{ご|どう|やう}に
\ruby{鄙}{いや}しい
\ruby{心}{こゝろ}ばかりが
\ruby{先}{さき}に
\ruby{立}{たち}まして、
\ruby{兎角}{と|かく}
\ruby{信心}{しん|〴〵}の
\ruby{起}{おこ}らないのも
\ruby{是非}{ぜ|ひ}がございませんで、
\ruby{眞}{まこと}に
\ruby{淺}{あさ}ましい
\ruby{口惜}{く|や}しいことでございます。
もう
\ruby{五十六十}{ご|じう|ろく|じう}になりまして、いろ〳〵の
\ruby{經驗}{おぼ|\換字{𛀁}}を
\ruby{積}{つ}んでまゐりました
\ruby{私等}{わたくし|ら}のやうな
\ruby{年齡}{と|し}のものでさへ、
\ruby{何}{なん}ぞにつけても
\ruby{怒}{おこ}つたり
\ruby{泣}{な}いたり
\ruby{致}{いた}しまして、
\ruby{彼奴}{あい|つ}が
\ruby{憎}{にく}いの
\ruby{恨}{うら}めしいのと、
\ruby{詰}{つま}らない
\ruby{修羅}{しゆ|ら}を
\ruby{燃}{も}やしまして、
\ruby{信心}{しん|〴〵}
\ruby{氣一方}{ぎ|いつ|ぱう}にばかりにはなつて
\ruby{居}{を}られませんのですから、
\ruby{御{\換字{若}}}{お|わか}い
\ruby{貴君}{あな|た}
\ruby{方}{がた}ではなか〳〵
\ruby{何樣}{ど|う}いたしまして、
\ruby{幾許}{いく|ら}
\ruby{御發明}{ご|はつ|めい}でいらつしやいましても、
\ruby{何事}{なに|ごと}も
\ruby{佛陀樣}{ほと|け|さま}に
\ruby{御任}{お|まか}せなすつて
\ruby{安心}{あん|しん}して
\ruby{御在}{お|いで}なさるといふ
\ruby{譯}{わけ}にはまいりますまい、
\ruby{御信心}{ご|しん|〴〵}も
\ruby{自然}{し|ぜん}
\ruby{御冷}{お|さめ}になつて、
\ruby{他}{ほか}の
\ruby{方}{はう}へ
\ruby{御{\換字{紛}}}{お|まぎ}れなさるのも
\ruby{御無理}{ご|む|り}はございません!。
\ruby{併}{しか}し
\ruby{貴君}{あな|た}はまあ
\ruby{御頼}{お|たの}もしい
\ruby{方}{かた}で、
\ruby{今}{いま}の
\ruby{御{\換字{若}}}{お|わか}い
\ruby{方}{かた}にも
\ruby{御似合}{お|に|あ}ひなさらずに、
\ruby{一心}{いつ|しん}になつて
\ruby{御信心}{ご|しん|〴〵}なすつた
\ruby{{\換字{過}}日}{この|あひだ}の
\ruby{御殊{\換字{勝}}}{ご|しゆ|しよう}さには、つく〴〵
\ruby{私}{わたくし}も
\ruby{感心}{かん|しん}いたしまして、
\ruby{斯樣}{か|う}
\ruby{申}{まを}しては
\ruby{諛辭}{おせ|じ}のやうでをかしうございますが、
\ruby{宅}{たく}へ
\ruby{歸}{かへ}りましてからも、あ〻
\ruby{未}{ま}だ
\ruby{世}{よ}の
\ruby{中}{なか}は
\ruby{闇}{やみ}にはならない、あ〻いふ
\ruby{{\換字{若}}}{わか}い
\ruby{方}{かた}も
\ruby{稀}{まれ}には
\ruby{居}{ゐ}らつしやる!、
\ruby{考}{かんが}へて
\ruby{見}{み}れば
\ruby{自{\換字{分}}}{じ|ぶん}なんぞは
\ruby{罪障}{つ|み}が
\ruby{深}{ふか}くつて
\ruby{昔生}{むかし|うま}れの
\ruby{身}{み}でありながら、
\ruby{何十年}{なん|じう|ねん}といふものを
\ruby{惜}{を}しい
\ruby{欲}{ほ}しいの
\ruby{欲}{よく}ばかりに
\ruby{{\換字{過}}}{すご}して、
\ruby{夢}{ゆめ}のやうにたゞ
\ruby{暮}{くら}した
\ruby{末}{すゑ}、
\ruby{神樣佛樣}{かみ|さま|ほとけ|さま}の
\ruby{有}{あ}り
\ruby{難}{がた}いことを
\ruby{知}{し}つたのも、やつと
\ruby{此}{こ}の
\ruby{四五年}{し|ご|ねん}ばかり
\ruby{以來}{この|かた}の
\ruby{事}{こと}だつたが、
\ruby{御{\換字{若}}}{お|わか}いのに
\ruby{彼樣}{あ|ゝ}いふ
\ruby{良}{よ}い
\ruby{方}{かた}もある!。
\ruby{自{\換字{分}}}{じ|ぶん}の
\ruby{彼}{あ}の
\ruby{位}{くらゐ}の
\ruby{齡}{とし}の
\ruby{時}{とき}に
\ruby{比}{くら}べてもよく
\ruby{解}{わか}ること、
\ruby{二十四五}{に|じう|し|ご}や
\ruby{三十{\換字{前}}後}{さん|じう|ぜん|ご}の
\ruby{勢}{いきほひ}では、
\ruby{鬼}{おに}が
\ruby{出}{で}ても
\ruby{攫}{つか}み
\ruby{合}{あ}はうといふ
\ruby{盲元氣}{めくら|げん|き}で、
\ruby{神樣}{かみ|さま}も
\ruby{佛樣}{ほとけ|さま}もありは
\ruby{仕}{し}ないのに、
\ruby{彼}{あ}の
\ruby{方}{かた}は
\ruby{嘘}{うそ}では
\ruby{出}{で}ない
\ruby{涙}{なみだ}を
\ruby{溢}{こぼ}して、
\ruby{一心}{いつ|しん}になつて
\ruby{祈}{いの}つていらつしやる!。
\ruby{御{\換字{父}}樣}{お|とつ|さま}が
\ruby{御病患}{お|わづ|らひ}でゞもあるか、
\ruby{御母樣}{お|つか|さま}が
\ruby{御惡}{お|わる}いのか、それとも
\ruby{何樣}{ど|う}いふ
\ruby{事}{こと}で
\ruby{思}{おも}ひ
\ruby{餘}{あま}つて、
\ruby{丹精}{たん|せい}を
\ruby{御凝}{お|こ}らしなさるか
\ruby{知}{し}らないけれども、あの
\ruby{御年齡}{お|とし|ば\換字{𛀁}}で
\ruby{既}{もう}
\ruby{神佛}{かみ|ほとけ}の
\ruby{有難}{あり|がた}い
\ruby{事}{こと}を
\ruby{知}{し}つて
\ruby{居}{ゐ}られるのは、あゝ
\ruby{稀}{めづ}らしい
\ruby{殊{\換字{勝}}}{しゆ|しよう}なかたゞと、
\ruby{眞實}{ほん|と}に
\ruby{貴君}{あな|た}の
\ruby{事}{こと}ばかり
\ruby{思}{おも}つて
\ruby{居}{を}りまして、
\ruby{何}{なん}だか
\ruby{私}{わたくし}は
\ruby{急}{きふ}に
\ruby{一人}{ひと|り}の、
\ruby{私}{わたくし}の
\ruby{味方}{み|かた}が
\ruby{出來}{で|き}たやうな
\ruby{氣}{き}が
\ruby{致}{いた}し、これも
\ruby{觀音樣}{くわん|のん|さま}の
\ruby{御引合}{お|ひき|あは}せ
\ruby{下}{くだ}すつた
\ruby{菩提}{ぼ|だい}の
\ruby{同行}{どう|ぎやう}とでもいふのであらう!、と
\ruby{{\換字{勝}}手}{かつ|て}な
\ruby{考}{かんが}へではございますが
\ruby{思}{おも}ひ
\ruby{詰}{つ}めまして、
\ruby{明{\換字{朝}}御目}{あし|た|お|め}にかゝつたらば、も
\ruby{一度}{いち|ど}
\ruby{御話}{お|はなし}して
\ruby{見}{み}やう、
\ruby{老人}{とし|より}の
\ruby{事}{こと}ゆゑ
\ruby{御{\換字{嫌}}}{お|きら}ひなさるか
\ruby{知}{し}れないが、どうも
\ruby{御話}{お|はなし}を
\ruby{仕}{し}て
\ruby{見}{み}たらば、
\ruby{屹度}{きつ|と}
\ruby{私}{わたくし}の
\ruby{力}{ちから}になつて
\ruby{下}{くだ}さる
\ruby{俠氣}{をとこ|ぎ}の
\ruby{方}{かた}だらう、といふやうな
\ruby{心持}{こゝろ|もち}が
\ruby{仕}{し}てなりませんでした。
ところが
\ruby{明{\換字{朝}}}{あし|た}
\ruby{參}{まゐ}つて
\ruby{見}{み}ると
\ruby{御參詣}{お|い|で}はありません、その
\ruby{次}{つぎ}の
\ruby{日}{ひ}も
\ruby{御參詣}{お|まゐ|り}がありません。
ぽろり〳〵と
\ruby{涙}{なみだ}を
\ruby{落}{おと}として
\ruby{眞}{しん}になつて
\ruby{何事}{なに|ごと}かを
\ruby{願}{ねが}つて
\ruby{居}{ゐ}られた
\ruby{彼}{あ}の
\ruby{方}{かた}が、
\ruby{不信心}{ぶ|しん|〴〵}になられる
\ruby{理由}{わ|け}は
\ruby{無}{な}いが、あ〻
\ruby{何}{なん}といつても
\ruby{未}{ま}だ
\ruby{御{\換字{若}}}{お|わか}い!、
\ruby{下}{くだ}らない
\ruby{惡{\換字{魔}}}{あく|ま}
\ruby{外{\換字{道}}}{げ|だう}の
\ruby{馬鹿書生}{ば|か|しよ|せい}が、
\ruby{愚}{ぐ}につかない
\ruby{事}{こと}を
\ruby{饒舌}{しや|べ}つて
\ruby{居}{ゐ}たが、
\ruby{{\換字{若}}}{もし}や
\ruby{彼言}{あ|れ}が
\ruby{毒}{どく}になりは
\ruby{仕}{し}ないか
\ruby{按}{あん}じられる、
\ruby{何}{なん}といつても
\ruby{未}{ま}だ
\ruby{御{\換字{若}}}{お|わか}いから!、と
\ruby{大}{おほ}きに
\ruby{彼}{あ}の
\ruby{書生等}{しよ|せい|たち}を
\ruby{憎}{にく}くおもつて
\ruby{居}{を}りました。
』

\Entry{其十九}

『
\ruby{何樣致}{ど|う|いた}しまして、
\ruby[g]{貴君}{あなた}、
\ruby{惡{\換字{所}}}{あく|しよ}へ
\ruby{參}{まゐ}りました
\ruby[g]{歸路}{かへり}に
\ruby{{\換字{遠}}慮}{ゑん|りよ}を
\ruby{致}{いた}すことも
\ruby{存}{ぞん}じませんで
\ruby{神社佛閣}{じん|じや|ぶつ|かく}の
\ruby{境内}{けい|だい}へ
\ruby{入}{はい}りますやうな
\ruby{不心得}{ふ|こゝろ|\換字{江}}なものに、
\ruby{何}{なに}が
\ruby{一}{ひと}つ
\ruby{滿足}{まん|ぞく}に
\ruby{世}{よ}の
\ruby{中}{なか}の
\ruby{事}{こと}が
\ruby{解}{わか}りましやう。
みんな
\ruby{彼}{あ}の
\ruby{先日}{せん|じつ}の
\ruby{書生}{しよ|せい}の
\ruby{{\換字{連}}中}{れん|ちう}は、
\ruby{自分}{じ|ぶん}の
\ruby[g]{身體}{からだ}の
\ruby[g]{背後}{うしろ}から
\ruby{親}{おや}や
\ruby{兄}{あに}の
\ruby{氣息}{い|き}が
\ruby{掛}{かゝ}つて
\ruby{居}{ゐ}ればこそ
\ruby{高慢}{こう|まん}な
\ruby{口}{くち}を
\ruby{利}{き}きましても
\ruby{人}{ひと}が
\ruby{赦}{ゆる}して
\ruby{置}{お}いて
\ruby{{\換字{呉}}}{く}れるのだといふ
\ruby{事}{こと}も
\ruby{知}{し}りませんで、
\ruby{定}{きま}りきつた
\ruby{譫語}{たわ|ごと}を
\ruby{申}{まを}しまするが、
\ruby[g]{畢竟}{つまり}
\ruby{彼樣}{あ|ゝ}いふのは、
\ruby{親}{おや}や
\ruby{兄}{あに}の
\ruby{有}{あ}り
\ruby{難}{がた}い
\ruby{事}{こと}さへ
\ruby{解}{わか}つて
\ruby{居}{を}りませんのですもの、
\ruby{中々神佛}{なか|〳〵|かみ|ほとけ}の
\ruby{有}{あ}り
\ruby{難}{がた}い
\ruby{事}{こと}なんぞの
\ruby{解}{わか}らないのも、
\ruby{些}{ちつと}も
\ruby{無理}{む|り}はないのでございます。
それでも
\ruby{當世}{たう|せい}のものゝ
\ruby{事}{こと}でございますから、
\ruby{理屈}{り|くつ}は
\ruby{立}{た}ちさうなやうな
\ruby{理屈臭}{り|くつ|くさ}いことを、
\ruby{曲}{まが}りなりに
\ruby{牽{\換字{強}}}{こじ|つけ}て
\ruby{申}{まを}しますので、
\ruby{一寸聞}{ちよ|つと|き}けば
\ruby{{\換字{道}}理}{もつ|とも}なやうなにも
\ruby{思}{おも}はれます。
そこで
\ruby{穩和}{おと|なし}いものまで
\ruby{{\換字{巻}}}{ま}き
\ruby{{\換字{込}}}{こ}まれまして、やれ
\ruby{神樣}{かみ|さま}を
\ruby{敬}{うやま}ふのは
\ruby[g]{愚{\換字{迷}}}{まよひ}だの、
\ruby{佛樣}{ほとけ|さま}を
\ruby{崇}{あが}めるのは
\ruby{卑劣}{け|ち}だのと、
\ruby{傍}{はた}から
\ruby{始{\換字{終}}}{し|ゞう}
\ruby{云}{い}はれつけますと、
\ruby{矢張}{やつ|ぱり}いつか
\ruby{其氣}{その|き}になつて、
\ruby{其實神樣佛樣}{その|じつ|かみ|さま|ほとけ|さま}を
\ruby{頼}{たの}みたいやうな
\ruby{氣}{き}のすることは
\ruby{有}{あ}つても、
\ruby{神樣佛樣}{かみ|さま|ほとけ|さま}をいぢりまはすのが、
\ruby{何}{なん}だか
\ruby{意氣地}{い|く|ぢ}の
\ruby{無}{な}いやうな
\ruby{羞}{はづか}しいやうな
\ruby{氣}{き}が
\ruby{仕}{し}て、それで
\ruby{神樣}{かみ|さま}にも
\ruby{佛樣}{ほとけ|さま}にも、お
\ruby{縋}{すが}り
\ruby{申}{まを}さないで
\ruby[g]{一人}{ひとり}で
\ruby{下}{くだ}らなく
\ruby{苦}{くるし}みきつて
\ruby{居}{を}ります。
それが
\ruby{當世}{たう|せい}の
\ruby{一體}{いつ|たい}の
\ruby{風}{ふう}でございます。
それにまた
\ruby{何}{なん}とか
\ruby{彼}{か}とか
\ruby{云}{い}はれて
\ruby{居}{ゐ}らつしやる
\ruby{先生方}{せん|せい|がた}でも、
\ruby{正直}{しやう|ぢき}な
\ruby{方}{かた}や
\ruby{良}{よ}い
\ruby{方}{かた}ばかり
\ruby{有}{あ}りは
\ruby{仕}{し}ません。
\ruby{隨{\換字{分}}}{ずゐ|ぶん}わざと
\ruby{若}{わか}いものの
\ruby{氣}{き}に
\ruby{入}{い}るやうな
\ruby{事}{こと}を
\ruby{仰}{おつし}あつたり
\ruby{人}{ひと}を
\ruby{吃驚}{びつ|くり}させるやうな
\ruby{事}{こと}を
\ruby{仰}{おつし}あつたり、
\ruby{中}{なか}には
\ruby{{\換字{評}}{\換字{判}}}{ひやう|ばん}を
\ruby{取}{と}らうの
\ruby{目論見}{もく|ろ|み}やら、
\ruby{面白}{おも|しろ}づくの
\ruby[g]{好奇心}{ものずき}やらから、
\ruby{神}{かみ}も
\ruby{佛}{ほとけ}も
\ruby{耶蘇}{や|そ}もいけない、
\ruby{酒}{さけ}を
\ruby{飮}{の}んで
\ruby{管}{くだ}を
\ruby{{\換字{巻}}}{ま}いているのが
\ruby{一番好}{いち|ばん|い}い、
\ruby{女}{をんな}と
\ruby{戯}{ふざ}けてゐるのが
\ruby{何}{なに}よりだといふやうな
\ruby{大變}{たい|へん}な
\ruby{事}{こと}なんぞを
\ruby{仰}{おつし}ある
\ruby{方}{かた}もあるさうで、
\ruby{左樣}{さ|う}で
\ruby{無}{な}くつてさへ
\ruby{暴}{あば}れたがる
\ruby{若}{わか}いものが、
\ruby{其樣}{そ|ん}な
\ruby{事}{こと}を
\ruby{聞}{き}くのですから
\ruby{堪}{たま}つたものではありません、
\ruby{蝮}{まむし}を
\ruby{食}{く}つた
\ruby[g]{軍鷄}{しやも}のやうに
\ruby{氣}{き}ばかり
\ruby{{\換字{強}}}{つよ}くなつて、
\ruby{世界}{せ|かい}は
\ruby{何}{なん}でも
\ruby{{\換字{勝}}手}{かつ|て}の
\ruby{仕{\換字{勝}}}{し|がち}だと
\ruby{思}{おも}ひまして、
\ruby{相手}{あい|て}さへ
\ruby{見}{み}りやあ
\ruby{雞距}{けづ|め}を
\ruby{打込}{うち|こ}みたがりまする。
\ruby{{\換字{過}}日}{この|あひだ}の
\ruby{書生}{しよ|せい}などが
\ruby{其例}{そ|れ}でござりまして、
\ruby{吾家}{わたく|しども}にも
\ruby[g]{一人}{ひとり}、
\ruby{似}{に}たり
\ruby{寄}{よ}つたりの
\ruby{難物}{なん|ぶつ}がござりまする。
かういふ
\ruby{世間}{せ|けん}でござりまするのに、たま〳〵
\ruby[g]{貴君}{あなた}のやうな
\ruby{方}{かた}をお
\ruby{見受}{み|う}け
\ruby{申}{まを}したのですから、
\ruby{失禮}{しつ|れい}ながら
\ruby{御同年位}{ご|どう|ねん|くらひ}の
\ruby{吾家}{う|ち}の
\ruby{豚兒}{ば|か}めと
\ruby{思}{おも}ひ
\ruby{較}{あは}すにつけ、ほんとに
\ruby{御懷}{おな|つか}しく
\ruby{存}{ぞん}じましたが、
\ruby{其}{そ}の
\ruby[g]{貴君}{あなた}が
\ruby{其限}{それ|ぎ}り
\ruby{御見}{お|み}えになりませんので
\ruby{大變氣}{たい|へん|き}になつてなりませんでした。
\ruby{御若}{お|わか}いから
\ruby{彼}{あ}の
\ruby{書生}{しよ|せい}の
\ruby{云}{い}つた
\ruby{事}{こと}なんぞも
\ruby{御耳}{お|みゝ}に
\ruby{可厭}{い|や}でしたらうが、
\ruby{御{\換字{迷}}}{お|まよ}ひなすつてはいけません。
\ruby{氣}{き}になすつてはいけません。
\ruby{御信心}{ご|しん|〴〵}さへ
\ruby{御續}{お|つゞ}けなされば
\ruby{御利{\換字{益}}}{ご|り|やく}は
\ruby{{\換字{分}}}{わか}つて
\ruby{來}{き}ます。
\ruby{私}{わたくし}なども
\ruby{二三十年}{に|さん|じう|ねん}も
\ruby{前}{まへ}は
\ruby{矢張}{やつ|ぱ}り
\ruby{彼}{あ}の
\ruby{書生}{しよ|せい}でございましたから、
\ruby{彼}{あ}の
\ruby{書生}{しよ|せい}も
\ruby{二十年三十年經}{に|じう|ねん|さん|じう|ねん|た}ちましたら、
\ruby{私}{わたくし}になりまして、
\ruby{御利{\換字{益}}}{ご|り|やく}の
\ruby{力}{ちから}が
\ruby{身}{み}に
\ruby{沁}{し}みるやうになりましやう。
\ruby{一}{ひと}つ
\ruby{家}{や}の
\ruby{婆}{ばあ}さんだつて
\ruby{發起致}{ほつ|き|いた}しますのですもの、
\ruby{何年洋杖}{なん|ねん|すて|つき}を
\ruby{振}{ふ}り
\ruby{{\換字{廻}}}{まは}して
\ruby{威張}{ゐ|ば}つて
\ruby{居}{ゐ}られるものでございましやう?。
\ruby{虛言}{う|そ}や
\ruby{僞言}{いつ|はり}は
\ruby{申}{まを}しません、
\ruby[g]{私等}{わたくしら}は
\ruby{散々世}{さん|〴〵|よ}の
\ruby{中}{なか}の
\ruby{憂}{う}い
\ruby{辛}{つら}いの
\ruby{川}{かは}を
\ruby{越}{こ}して
\ruby{參}{まゐ}つて、
\ruby{此岸}{こち|ら}の
\ruby{信心}{しん|〴〵}の
\ruby{有}{あ}り
\ruby{難}{がた}い
\ruby{事好}{こと|い}い
\ruby{事}{こと}を
\ruby{見}{み}て
\ruby{居}{を}りまするので、
\ruby{彼等}{あれ|ら}は
\ruby{未}{ま}だ
\ruby{川}{かは}の
\ruby{中}{なか}へ
\ruby{入}{はい}り
\ruby{立}{たて}なので、
\ruby{元氣任}{げん|き|まか}せに
\ruby{立泳}{たち|およ}ぎを
\ruby{爲}{し}たり
\ruby{拔手}{ぬき|で}をきつたりしながら、
\ruby{何}{なん}だ
\ruby{對}{むか}ふ
\ruby{岸}{ぎし}に
\ruby{上}{あが}つて
\ruby{居}{ゐ}る
\ruby{奴等}{やつ|ら}の
\ruby{意氣地}{い|く|ぢ}の
\ruby{無}{な}さと
\ruby{申}{まを}して
\ruby{居}{ゐ}るやうなものでございます。
\ruby{疲勞}{くた|び}れたり、こむらが
\ruby{反}{かへ}つたり、
\ruby{流}{なが}れの
\ruby{{\換字{強}}}{つよ}いところへ
\ruby{出}{で}たりしますれば、
\ruby{此方}{こち|ら}の
\ruby{岸}{きし}を
\ruby{見}{み}て
\ruby{泣}{な}かずには
\ruby{居}{を}りません。
\ruby{其時}{その|とき}になつて
\ruby{前}{まへ}に
\ruby{此方}{こち|ら}に
\ruby{居}{ゐ}たものゝ
\ruby{心持}{こゝろ|もち}が
\ruby{解}{わか}ります。
あれ
\ruby{彼}{あ}の
\ruby{銀杏}{ぎん|なん}といふものは
\ruby[g]{公孫樹}{いてふ}の
\ruby{實}{み}です。
\ruby{榧}{かや}の
\ruby{實}{み}でも
\ruby{無}{み}ければ
\ruby{{\換字{叉}}}{また}
\ruby{橡}{とち}の
\ruby{實}{み}でも
\ruby{無}{な}く、
\ruby{誰}{だれ}が
\ruby{何}{なん}といつても
\ruby[g]{公孫樹}{いてふ}の
\ruby{實}{み}です。
これに
\ruby{理屈}{り|くつ}が
\ruby{何有}{なに|あ}りましやう、もと〳〵
\ruby[g]{公孫樹}{いてふ}から
\ruby{出}{で}たものですもの!。
\ruby{神樣佛樣}{かみ|さま|ほとけ|さま}に
\ruby{縋}{すが}る
\ruby{私共}{わたく|しども}の
\ruby{此}{こ}の
\ruby{心}{こゝろ}は、
\ruby{何}{なん}の
\ruby{心}{こゝろ}でござりましやう!、
\ruby{人}{ひと}の
\ruby{心}{こゝろ}です。
\ruby{禽}{とり}の
\ruby{心}{こゝろ}でも
\ruby{無}{な}ければ
\ruby{獸}{けもの}の
\ruby{心}{こゝろ}でも
\ruby{無}{な}く、
\ruby{誰}{だれ}が
\ruby{何}{なん}といつても
\ruby{人}{ひと}の
\ruby{心}{こゝろ}です。
これに
\ruby{理屈}{り|くつ}が
\ruby{何有}{なに|あ}りましやう、もと〳〵
\ruby{人}{ひと}が
\ruby{有}{も}つた
\ruby{心}{こゝろ}ですもの!。
\ruby{吾}{わ}が
\ruby{子}{こ}の
\ruby{可愛}{かは|ゆ}いのに
\ruby{理屈}{り|くつ}も
\ruby{無}{な}く、
\ruby{思}{おも}ふ
\ruby{人}{ひと}の
\ruby{大切}{だい|じ}なのに
\ruby{理屈}{り|くつ}も
\ruby{無}{な}ければ、
\ruby{神樣佛樣}{かみ|さま|ほとけ|さま}に
\ruby{御縋}{お|すが}り
\ruby{申}{まを}すのに、何の
\ruby{理屈}{り|くつ}も
\ruby{無}{な}いけれど、それも
\ruby{眞實}{まこ|と}なれば
\ruby{此}{これ}も
\ruby{眞實}{まこ|と}で、
\ruby{理屈}{り|くつ}も
\ruby{要}{い}らないほどの
\ruby{眞實}{まこ|と}です!。
あゝ、いけません
\ruby{御{\換字{迷}}}{お|まよ}ひなすつては!。
いや
\ruby{御{\換字{迷}}}{お|まよ}ひなすつてはいけません
\ruby{貴方}{あな|た}!。
\ruby[g]{公孫樹}{いてふ}の
\ruby{秋}{あき}には
\ruby{銀杏}{ぎん|なん}が
\ruby{生}{な}ります、
\ruby{榧}{かや}の
\ruby{實}{み}も
\ruby{橡}{とち}の
\ruby{實}{み}も
\ruby{生}{な}りは
\ruby{仕}{し}ません、
\ruby{人}{ひと}の
\ruby{胸}{むね}には
\ruby{信心}{しん|〴〵}が
\ruby{生}{な}ります、
\ruby{生}{な}らせまいと
\ruby{思}{おも}つても
\ruby{生}{な}るのが
\ruby{約束}{やく|そく}、
\ruby{信}{しん}を
\ruby{有}{も}たなければ
\ruby{胸}{むね}が
\ruby{騒}{さわ}いで、
\ruby{誰}{だれ}が
\ruby{氣}{き}を
\ruby{安}{やす}くして
\ruby{居}{ゐ}られましやう!。
おゝ
\ruby{貴君}{あな|た}が
\ruby{默}{だま}つて
\ruby{居}{ゐ}らつしやるので
\ruby{私}{わたくし}ばかり
\ruby{饒舌}{しや|べ}りました。
さあ
\ruby{御堂}{お|だう}へ
\ruby{上}{あが}つて
\ruby{拜}{をが}みましやう。

と
\ruby{水野}{みづ|の}を
\ruby{牽}{ひ}きて
\ruby{共}{とも}に
\ruby{堂}{だう}に
\ruby{上}{のぼ}りぬ。

\ruby{老人}{らう|じん}が
\ruby{言}{ことば}を
\ruby{默々}{もく|〳〵}として
\ruby{聞}{き}きながら、
\ruby{水野}{みづ|の}は
\ruby{牽}{ひ}かるゝがまゝに
\ruby{堂}{だう}には
\ruby{上}{のぼ}りしが、
\ruby{{\換字{猶}}}{なほ}
\ruby{今朝}{け|さ}は
\ruby{直}{ただち}に
\ruby{本{\換字{尊}}}{ほん|ぞん}を
\ruby{拜}{はい}せんともせず、さればとて
\ruby{侮}{あなど}り
\ruby{慢}{あなど}る
\ruby{心}{こゝろ}も
\ruby{無}{な}くて、
\ruby{喪心}{さう|しん}せる
\ruby{人}{ひと}の
\ruby{如}{ごと}く
\ruby{無意味}{む|い|み}に
\ruby{立}{た}ち
\ruby{居}{ゐ}たり。


\Entry{其二十}

\ruby{既}{すで}に
\ruby{我}{わ}が
\ruby{言葉}{こと|ば}を
\ruby{{\換字{㑦}}}{もど}きもせず、また
\ruby{我}{わ}が
\ruby{{\換字{伴}}}{ともな}ふを
\ruby{拒}{こば}みもせねば、
\ruby{今御堂}{いま|み|だう}に
\ruby{上}{のぼ}りて
\ruby{御前}{おん|まへ}に
\ruby{至}{いた}れる
\ruby{上}{うへ}は、
\ruby{必}{かな}らず
\ruby{復前}{また|さき}の
\ruby{日}{ひ}の
\ruby{朝}{あさ}の
\ruby{如}{ごと}くに、たとひ
\ruby{御經}{おん|きやう}は
\ruby{誦}{じゆ}せざるまでも、
\ruby{掌}{たなぞこ}を
\ruby{合}{あは}せ
\ruby{頭}{かうべ}を
\ruby{下}{さ}げて
\ruby{禮拜}{らい|はい}するならんと、
\ruby{獨合點}{ひとり|が|てん}してや
\ruby{彼}{か}の
\ruby{老人}{らう|じん}は、
\ruby{御堂}{み|だう}に
\ruby{上}{のぼ}りてよりは
\ruby{水野}{みづ|の}に
\ruby{關}{かま}はず、
\ruby{一}{ひと}つは
\ruby{自己}{お|の}が
\ruby{信心}{しん|〴〵}の
\ruby{誠}{まこと}を
\ruby{致}{いた}さんとするに
\ruby{忙}{いそが}しきが
\ruby{故}{ゆえ}もあるべし、
\ruby{例}{いつも}の
\ruby{如}{ごと}く
\ruby{御前}{み|まへ}に
\ruby{蹲}{うづく}まりて、
\ruby{先}{ま}ず
\ruby{一心}{いつ|しん}に
\ruby{恭敬禮拜}{きやう|けい|らい|はい}しつ、
\ruby{徐々}{しづ|か}に
%\ruby[g]{妙法{\換字{蓮}}華經觀世音菩薩普門品第二十五}{めうはうれんげきやうくわんぜおんぼさつふもんぼんだいにじうご}、と
\ruby{妙法{\換字{蓮}}華經}{めう|はう|れん|げき|やう}
\ruby{觀世音菩薩}{くわ|んぜ|おん|ぼさ|つ}
\ruby{普門品}{ふ|もん|ぼん}
\ruby{第二十五}{だい|に|じう|ご}、と
\ruby{老}{お}いたる
\ruby{聲}{こゑ}の
\ruby{低}{ひく}く
\ruby{誦}{じゆ}し
\ruby{出}{いだ}しけり。

\ruby{朝}{あさ}の
\ruby{氣}{き}は
\ruby{何}{なん}となく
\ruby{心}{こゝろ}をして
\ruby{粛然}{しゆく|ぜん}たらしめて、
\ruby{廣}{ひろ}き
\ruby{御堂}{み|だう}の
\ruby{内}{うち}の
\ruby{人無}{ひと|な}き
\ruby{物靜}{もの|しづ}かさは
\ruby{自然}{おの|づ}と
\ruby{胸}{むね}の
\ruby{中}{うち}を
\ruby{淸々}{すが|〳〵}しからしむ。
\ruby{今日}{け|ふ}は
\ruby{御佛}{みほ|とけ}を
\ruby{拜}{をが}み
\ruby{奉}{たてまつ}りもせず、さりとて
\ruby{{\換字{又}}}{また}
\ruby{御佛}{みほ|とけ}より
\ruby{反}{そむ}き
\ruby{去}{さ}りもせず、たゞたゞ
\ruby{從順}{すな|ほ}なる
\ruby{兒童}{こ|ども}の、
\ruby{心}{こゝろ}に
\ruby{物無}{もの|な}きが
\ruby{如}{ごと}く、
\ruby{牽}{ひ}かれたるまゝに
\ruby{此處}{こ|ゝ}に
\ruby{來}{きた}りて、
\ruby{此處}{こ|ゝ}に
\ruby{其儘止}{その|まゝ|とゞま}れる
\ruby{水野}{みづ|の}は、
\ruby{身{\換字{近}}}{み|ぢか}なりし
\ruby{圓柱}{まる|ばしら}の
\ruby{太}{ふと}きに
\ruby{憑}{よ}りて、
\ruby{風吹}{かぜ|ふ}かぬ
\ruby{間}{ま}を
\ruby{大空}{おほ|ぞら}に
\ruby{高}{たか}く
\ruby{懸}{かゝ}れる
\ruby{孤雲}{ひとつ|ぐも}の、
\ruby{何}{なに}に
\ruby{着}{つ}くとも
\ruby{無}{な}き
\ruby{思}{おもひ}に、
\ruby{嗒焉}{たう|\換字{江}ん}として
\ruby{獨}{ひと}り
\ruby{空}{むな}しく
\ruby{立}{た}てり。

\ruby{老}{お}いたる
\ruby{人}{ひと}の
\ruby{誦}{じゆ}する
\ruby{經}{きやう}の、
\ruby{其意}{その|こゝろ}は
\ruby{曉}{さと}らるゝ
\ruby{時}{とき}あれど、
\ruby{其聲}{その|こゑ}は
\ruby{波瀾無}{な|み|な}く
\ruby{山坂無}{や|ま|な}くして
\ruby{一條}{ひと|すぢ}の
\ruby{絲}{いと}を
\ruby{畫}{ひ}けるにも
\ruby{似}{に}て
\ruby{{\換字{平}}}{たひ}らかなるに、
\ruby{聞}{き}き
\ruby{居}{ゐ}る
\ruby{我}{わ}が
\ruby{心}{こゝろ}は
\ruby{刻々}{こく|〳〵}に
\ruby{安}{やす}まり
\ruby{行}{ゆ}き、
\ruby{何}{なん}とは
\ruby{無}{な}けれど
\ruby{引}{ひ}き
\ruby{入}{い}れらる〻やうにおぼえて、
\ruby{知}{し}らず
\ruby{識}{し}らず
\ruby{無念無想}{む|ねん|む|さう}の
\ruby{境}{さかひ}に
\ruby{入}{い}る
\ruby{折}{をり}しも、
\ruby{人}{ひと}の
\ruby{下駄}{げ|た}の
\ruby{音}{おと}に
\ruby{不圖驚}{ふ|と|おどろ}きて、
\ruby{見}{み}れば
\ruby{何時}{い|つ}の
\ruby{間}{ま}にや
\ruby{三十}{さん|じう}ばかりなる
\ruby{女}{をんな}の、
\ruby{老人}{らう|じん}と
\ruby{並}{なら}びて
\ruby{禮拜}{らい|はい}なし
\ruby{居}{を}り、
\ruby{老人}{らう|じん}の
\ruby{誦經}{じゆ|きやう}は
\ruby{今}{いま}や
\ruby{{\換字{終}}}{をは}らんとして、
\ruby{具一切功德}{ぐ|いつ|さい|く|どく}、
\ruby{慈眼視衆生}{じ|げん|じ|しゆ|じやう}と、
\ruby{偈}{げ}の
\ruby{末}{すゑ}のところを
\ruby{誦}{よ}み
\ruby{居}{い}たり。
\ruby{是}{こ}は
\ruby{不覺}{ふ|かく}なりし
\ruby{愚}{おろか}なりし!。
\ruby{身}{み}はこそ
\ruby{動}{うご}かさゞりつれ
\ruby{心}{こゝろ}の
\ruby{内}{うち}には、
\ruby{吾}{わ}が
\ruby{兒}{こ}の
\ruby{可憐}{かは|ゆ}いのに
\ruby{理屈}{り|くつ}も
\ruby{無}{な}く、
\ruby{思}{おも}ふ
\ruby{人}{ひと}の
\ruby{大切}{だい|じ}なのに
\ruby{理屈}{り|くつ}も
\ruby{無}{な}ければ、
\ruby{神樣佛樣}{かみ|さま|ほとけ|さま}に
\ruby{御縋}{お|すが}り
\ruby{申}{まを}すのにも、
\ruby{何}{なん}の
\ruby{理屈}{り|くつ}も
\ruby{無}{な}いなれど、それも
\ruby{眞實}{ま|こと}なれば、
\ruby{此}{これ}も
\ruby{眞實}{ま|こと}で、
\ruby{理屈}{り|くつ}の
\ruby{要}{い}らないほどの
\ruby{眞實}{ま|こと}!\inhibitglue と
\ruby{云}{い}ひたる
\ruby{此}{こ}の
\ruby{老人}{らう|じん}の
\ruby{言葉}{こと|ば}を
\ruby{味}{あぢ}はひて、
\ruby{實}{げ}に
\ruby{云}{い}はるれば、
\ruby{其}{そ}の
\ruby{如}{ごと}くなり、
\ruby{我}{わ}が
\ruby{彼}{か}の
\ruby{人}{ひと}を
\ruby{思}{おも}ひ
\ruby{思}{おも}ふ
\ruby{心}{こゝろ}に、そも〳〵
\ruby{何}{なん}の
\ruby{理由}{いは|れ}のありや、
\ruby{何}{なん}の
\ruby{理由}{わ|け}とは
\ruby{我}{われ}も
\ruby{知}{し}らず、たゞ
\ruby{我}{われ}と
\ruby{我}{わ}が
\ruby{欺}{あざむ}き
\ruby{難}{がた}き
\ruby{{\換字{情}}}{こゝろ}の
\ruby{萌}{も}えに
\ruby{萌}{も}え
\ruby{出}{い}づるを
\ruby{抑}{おさ}へ
\ruby{得}{\換字{江}}ざるぞ
\ruby{眞實}{ま|こと}なる!。
\ruby{思}{おも}ふて
\ruby{思}{おも}はる〻
\ruby{身}{み}ならばこそ、
\ruby{不{\換字{運}}}{ふ|うん}にして
\ruby{我拙}{われ|つたな}く
\ruby{生}{うま}れ
\ruby{來}{き}て、
\ruby{思}{おも}へば
\ruby{思}{おも}ふほど
\ruby{{\換字{嫌}}}{きら}はる〻
\ruby{身}{み}の、
\ruby{思}{おも}ふて
\ruby{甲{\換字{斐}}無}{か|ひ|な}き
\ruby{事}{こと}なれば、
\ruby{自}{みづか}ら
\ruby{斷念}{あき|ら}め
\ruby{思}{おも}ひ
\ruby{切}{き}りて、
\ruby{忘}{わす}れ
\ruby{果}{は}てんこそ
\ruby{人}{ひと}のため
\ruby{身}{み}のためなれ、
\ruby{我}{わ}が
\ruby{爲}{な}す
\ruby{事言}{こと|い}う
\ruby{事}{こと}は
\ruby{何}{なに}から
\ruby{何}{なに}まで、
\ruby{{\換字{情}}}{なさけ}なくも
\ruby{彼}{か}の
\ruby{人}{ひと}に
\ruby{厭}{いと}はる〻ながら、
\ruby{思}{おも}ひ
\ruby{忘}{わす}る〻といふ
\ruby{此事}{こ|れ}ばかりは、
\ruby{必}{かなら}ず
\ruby{彼}{か}の
\ruby{人}{ひと}に
\ruby{{\換字{悅}}}{よろこ}ばるければ、
\ruby{果敢}{は|か}なく
\ruby{悲}{かな}しき
\ruby{限}{かぎ}りなれど、とてもかくても
\ruby{味氣無}{あぢ|き|な}き
\ruby{我}{わ}が
\ruby{一生}{いつ|しやう}の
\ruby{思}{おも}ひ
\ruby{出}{で}に、せめては
\ruby{男兒}{をと|こ}らしうふつつりと
\ruby{諦}{あきら}めて、うるさく
\ruby{纏繞}{まつ|は}る
\ruby{蔦葛}{つた|かつら}の
\ruby{離}{はな}れて
\ruby{去}{さ}りし
\ruby{嬉}{うれ}しさよと、
\ruby{彼}{か}の
\ruby{人}{ひと}に
\ruby{安}{やす}き
\ruby{思}{おもひ}をさせん、
\ruby{人}{ひと}も
\ruby{見}{み}ず
\ruby{人}{ひと}をも
\ruby{見}{み}ざる
\ruby{深}{ふか}き
\ruby{山}{やま}の
\ruby{巖}{いは}の
\ruby{罅隙}{はざ|ま}に
\ruby{我}{われ}
\ruby{一人}{ひと|り}
\ruby{入}{い}りて、
\ruby{誰憚}{たれ|はゞか}らず
\ruby{思}{おも}ふさま
\ruby{泣}{な}きて、
\ruby{其淚}{その|なみだ}の
\ruby{乾}{かは}き
\ruby{聲}{こゑ}の
\ruby{枯}{か}れん
\ruby{時我即}{とき|われ|すなは}ち
\ruby{此世}{この|よ}を
\ruby{去}{さ}らば
\ruby{濟}{す}むべき
\ruby{事}{こと}なるをや!、と
\ruby{幾度}{いく|たび}か〳〵
\ruby{思}{おも}ひしかど、
\ruby{諦}{あきら}めても
\ruby{諦}{あきら}めても
\ruby{諦}{あきら}め
\ruby{得}{\換字{江}}ず、
\ruby{彼}{か}の
\ruby{人}{ひと}を
\ruby{背後}{うし|ろ}にして
\ruby{千里}{せん|り}の
\ruby{{\換字{遠}}}{とほ}きに
\ruby{身}{み}を
\ruby{隠}{かく}し
\ruby{棄}{す}てんとする
\ruby{意}{こゝろ}はありても、
\ruby{彼}{か}の
\ruby{人}{ひと}より
\ruby{距}{へだ}たらんとすれば
\ruby{一歩}{いつ|ぽ}も
\ruby{去}{さ}り
\ruby{得}{\換字{江}}ず、
\ruby{我}{わ}が
\ruby{心}{こゝろ}の
\ruby{我}{わ}が
\ruby{心}{こゝろ}に
\ruby{任}{まか}せずして、あだに
\ruby{苦}{くるし}みあだに
\ruby{惱}{なや}むは、たゞ
\ruby{我}{われ}と
\ruby{我}{わ}が
\ruby{欺}{あざむ}きがたき
\ruby{{\換字{情}}}{こゝろ}の
\ruby{萌}{も}えに
\ruby{萌}{も}ゆればなり。
おもへば
\ruby{神佛}{かみ|ほとけ}を
\ruby{頼}{たの}み
\ruby{奉}{たてまつ}るも
\ruby{實}{げ}に
\ruby{似}{に}たる
\ruby{事}{こと}かな。
\ruby{人}{ひと}はいざ
\ruby{知}{し}らず
\ruby{我}{われ}は
\ruby{我}{わ}が
\ruby{欺}{あざむ}き
\ruby{難}{がた}き
\ruby{{\換字{情}}}{こゝろ}のありて、
\ruby{何}{なん}の
\ruby{理由}{いは|れ}とは
\ruby{更}{さら}に
\ruby{知}{し}らねど、
\ruby{神}{かみ}にも
\ruby{憐}{あは}れと
\ruby{思}{おも}はれたき
\ruby{心地}{こゝ|ち}のするなり。
\ruby{理}{り}は
\ruby{石}{いし}の
\ruby{如}{ごと}し
\ruby{抂}{ま}ぐべからず、
\ruby{我}{われ}これを
\ruby{懷}{いだ}きて
\ruby{神}{かみ}をも
\ruby{佛}{ほとけ}をも
\ruby{肯}{うけが}はねども、
\ruby{感{\換字{情}}}{こゝ|ろ}は
\ruby{味}{あじはひ}の
\ruby{欺}{あざむ}くべからざるが
\ruby{如}{ごと}く、
\ruby{我}{われ}おのづからに
\ruby{神}{かみ}を
\ruby{戀}{こ}ひ
\ruby{佛}{ほとけ}を
\ruby{慕}{した}はんとするを
\ruby{如何}{い|か}にすべきや。
\ruby{人}{ひと}の
\ruby{戀}{こひ}しき
\ruby{彼}{かれ}も
\ruby{眞實}{まこ|と}なり、
\ruby{神佛}{かみ|ほとけ}の
\ruby{頼}{たの}み
\ruby{奉}{たてまつ}りたき
\ruby{此}{これ}も
\ruby{眞實}{まこ|と}なり。
\ruby{噫我力無}{あゝ|われ|ちから|な}し、
\ruby{我既}{われ|すで}に
\ruby{我}{わ}が
\ruby{五十子}{い|そ|こ}を
\ruby{思}{おも}ひ
\ruby{棄}{す}て
\ruby{得}{\換字{江}}ざるなり、
\ruby{我}{われ}よくこの
\ruby{神佛}{かみ|ほとけ}をば
\ruby{思}{おも}ひ
\ruby{棄}{す}て
\ruby{得}{う}べきや。
\ruby{思}{おも}へば
\ruby{我}{われ}ながら
\ruby{覺束無}{おぼ|つか|な}き
\ruby{事}{こと}なるかな!。
さはさりながら、さはさりながら。
と
\ruby{切}{しきり}に
\ruby{默想}{おも|ひ}に
\ruby{耽}{ふけ}りし
\ruby{時}{とき}には、
\ruby{弘誓深如海}{ぐ|ぜい|しん|によ|かい}、
\ruby{歷劫不思議}{れき|がう|ふ|し|ぎ}と
\ruby{老人}{らう|じん}の
\ruby{誦}{じゆ}したる
\ruby{聲}{こゑ}を
\ruby{{\換字{猶}}}{なほ}
\ruby{耳}{みゝ}にしたりしに、それより
\ruby{兎}{と}せん
\ruby{角}{かく}せんに
\ruby{思}{おも}ひ
\ruby{{\換字{迷}}}{まよ}へる
\ruby{中}{うち}、
\ruby{何時}{い|つ}の
\ruby{間}{ま}にか
\ruby{瞢然}{うつ|とり}と
\ruby[g]{睡眠}{ねむり}には
\ruby{入}{い}りたるぞや。
と
\ruby{水野}{みづ|の}は
\ruby{自}{みづか}ら
\ruby{私}{ひそか}に
\ruby{慚}{は}ぢたり。


\Entry{其二十一}

\ruby{右}{みぎ}せんとすれば
\ruby{左}{ひだり}したき
\ruby{意}{こゝろ}あり、
\ruby{左}{ひだり}せんとすれば
\ruby{右}{みぎ}せんとしたき
\ruby{意}{こゝろ}もありて、
\ruby{廣野}{ひろ|の}の
\ruby{草高}{くさ|たか}き
\ruby{中}{うち}の
\ruby{岐路}{わか|れぢ}にさしか〻れる
\ruby{身}{み}の、いづれと
\ruby{取}{と}りわづらへば、
\ruby{右}{みぎ}にも
\ruby{去}{さ}り
\ruby{得}{\換字{𛀁}}ず
\ruby{左}{ひだり}にも
\ruby{往}{ゆ}き
\ruby{得}{\換字{𛀁}}ざる
\ruby{一時}{いち|じ}
\ruby{二念}{に|ねん}の
\ruby{心魂}{こゝ|ろ}は
\ruby{疲}{つか}れて、
\ruby{我知}{われ|し}らず
\ruby{誦經}{じゆ|きやう}の
\ruby{聲}{こゑ}の
\ruby{中}{うち}に
\ruby{攝}{せつ}し
\ruby{去}{さ}られ、
\ruby{睡}{ねむ}るとも
\ruby{無}{な}しに
\ruby{睡}{ねむ}りし
\ruby{歟}{か}、
\ruby[<h||]{否}{あらず}
\ruby{睡}{ねむ}りしか
\ruby{睡}{ねむ}らざりし
\ruby{歟}{か}。
たゞ
\ruby{我深}{われ|ふか}く〳〵
\ruby{思}{おも}ひ
\ruby{入}{い}りて、いよ〳〵
\ruby{二}{ふた}つの
\ruby{念}{おもひ}の
\ruby{力相等}{ちから|あひ|ゝと}しくして、
\ruby{我}{わ}が
\ruby{心}{こゝろ}のいづれにも
\ruby{動}{うご}かずなりし
\ruby{其}{そ}の
\ruby{靜}{しづか}さを
\ruby{纔}{わづか}におぼえし
\ruby{後}{のち}は、
\ruby{聞}{き}くとも
\ruby{無}{な}く
\ruby{聞}{き}ける
\ruby{老人}{らう|じん}の
\ruby{聲}{こゑ}の、いと
\ruby{快}{こゝろよ}く
\ruby{聞}{きこ}\換字{𛀁}しを
\ruby{知}{し}れるのみなりしが、
\ruby{兎}{と}にも
\ruby{角}{かく}にも
\ruby{我}{われ}を
\ruby{忘}{わす}れしは
\ruby{愚}{おろか}なりしと、
\ruby{水野}{みづ|の}は
\ruby{繰}{く}り
\ruby{{\換字{返}}}{かへ}して
\ruby{自}{みづか}ら
\ruby{思}{おも}ふ
\ruby{時}{とき}、
\ruby{阿耨多羅三藐三菩提心}{あの|く|た|ら|さん|みやく|さん|ぼ|だ|しん}と、
\ruby{誦}{じゆ}し
\ruby{{\換字{終}}}{おは}りて
\ruby{一心}{いつ|しん}に
\ruby{禮拜}{らい|はい}せし
\ruby{彼}{か}の
\ruby{老人}{らう|じん}は、
\ruby{去}{さ}らず
\ruby{就}{つ}かずに
\ruby{立{\換字{迷}}}{たち|まよ}へる
\ruby{水野}{みづ|の}が
\ruby{狀態}{あり|さま}を
\ruby{頭}{かうべ}を
\ruby{反}{かへ}して
\ruby{見}{み}つ、たちまち
\ruby{此方}{こな|た}へすた〳〵と
\ruby{來}{きた}りて、
\ruby{眼}{め}の
\ruby{中}{うち}に
\ruby{氣{\換字{遣}}}{き|づか}ふが
\ruby{如}{ごと}く
\ruby{憐}{あはれ}むが
\ruby{如}{ごと}き
\ruby{色}{いろ}を
\ruby{{\換字{浮}}}{うか}めながら、

『あ〻
\ruby{御{\換字{迷}}}{お|まよ}ひなすつてはいけません、
\ruby{勿體無}{もつ|たい|な}い
\ruby{事}{こと}です!。
\ruby{念念}{ねん|ねん}に
\ruby{疑}{うたがひ}を
\ruby{生}{しやう}ずる
\ruby{勿}{なか}れとは
\ruby{御經}{おき|やう}にもございます。
\ruby{貴君}{あな|た}
\ruby{{\換字{過}}日}{この|あひだ}は
\ruby{泣}{な}いて
\ruby{居}{ゐ}らしつたではありませんか、
\ruby{貴君}{あな|た}のやうな
\ruby{良}{よ}い
\ruby{方}{かた}が、
\ruby{御{\換字{迷}}}{お|まよ}ひなさるなんて
\ruby{飛}{とん}でもない
\ruby{事}{こと}です!。
\ruby{信}{しん}を
\ruby{籠}{こ}めて
\ruby{一心}{いつ|しん}に
\ruby{御拜}{お|をが}みなさらなくつてはいけません、
\ruby{善惡}{ぜん|あく}
\ruby{共}{とも}に
\ruby{御利益}{ご|り|やく}は
\ruby{屹度}{きつ|と}あります、さあ
\ruby{私}{わたくし}も
\ruby{拜}{をが}みます、
\ruby{御一緖}{ご|いつ|しよ}に
\ruby{拜}{をが}みましやう!。
さあ、
\ruby{貴君}{あな|た}、さあ!。
』

と
\ruby{云}{い}ひ〳〵
\ruby{袖}{そで}を
\ruby{引}{ひ}きて
\ruby{御{\換字{前}}}{おん|まへ}へと
\ruby{誘}{いざな}ひ、おのれ
\ruby{先}{ま}づ
\ruby{膝}{ひざ}を
\ruby{折}{を}り
\ruby{身}{み}を
\ruby{屈}{かゞ}めて
\ruby{禮拜}{らい|はい}し、
\ruby{水野}{みづ|の}にも
\ruby{之}{これ}に
\ruby{倣}{なら}はしめたり。

\ruby{他人}{ひ|と}の
\ruby{胸}{むね}の
\ruby{中}{うち}には
\ruby{何物}{な|に}ありとも
\ruby{思}{おも}はず、たゞ
\ruby{我}{わ}が
\ruby{菩提}{ぼ|だい}の
\ruby{同行}{どう|ぎやう}と
\ruby{思}{おも}ふばかりの
\ruby{親切}{しん|せつ}より、
\ruby{年若}{とし|わか}き
\ruby{我}{われ}をあらぬ
\ruby{{\換字{道}}}{みち}へ
\ruby{外}{そ}れさせじとの
\ruby{他事}{た|じ}なき
\ruby{願望}{のぞ|み}に、
\ruby{人}{ひと}の
\ruby{好}{よ}げなる
\ruby{此}{こ}の
\ruby{老人}{らう|じん}の
\ruby{如是心}{か|く|こゝろ}を
\ruby{使}{つか}ひ
\ruby{身}{み}を
\ruby{使}{つか}ひて
\ruby{老實}{まめ|〳〵}しく
\ruby{振舞}{ふる|ま}ひ
\ruby{吳}{く}る〻を
\ruby{見}{み}ては、
\ruby{心{\換字{弱}}}{こゝろ|よわ}くも
\ruby{人惡}{ひと|あ}しからぬ
\ruby{水野}{みづ|の}はこれを
\ruby{拒}{こば}みかねて、
\ruby{牽}{ひ}かる〻がま〻に
\ruby{牽}{ひ}かれ、
\ruby{屈}{かゞ}ませらる〻がま〻に
\ruby{屈}{かゞ}み、
\ruby{{\換字{終}}}{つひ}には
\ruby{御佛}{み|ほとけ}の
\ruby{{\換字{前}}}{まへ}に
\ruby{蹲}{うづく}まりて、
\ruby{其}{そ}の
\ruby{老人}{らう|じん}の
\ruby{爲}{な}すが
\ruby{如}{ごと}くに、
\ruby{一霎時}{し|ば|し}は
\ruby{頭}{かうべ}を
\ruby{下}{さ}げ
\ruby{眼}{まなこ}を
\ruby{瞑}{ふさ}ぎて、
\ruby{一心}{いつ|しん}に
\ruby{大慈}{だい|じ}
\ruby{大悲}{だい|ひ}の
\ruby{我}{わ}が
\ruby{菩薩}{ぼ|さつ}をば、
\ruby{我}{われ}を
\ruby{忘}{わす}れて
\ruby{念}{ねん}じ
\ruby{奉}{たてまつ}りしが、
\ruby{佛力甚深測}{ぶつ|りき|じん|〳〵|はか}るべからず、
\ruby{時}{とき}に
\ruby{不思議}{ふ|し|ぎ}や
\ruby{水野}{みづ|の}は
\ruby{忽}{たちま}ち、
\ruby{心}{こゝろ}の
\ruby{闇}{やみ}に
\ruby{{\換字{朝}}日}{あさ|ひ}の
\ruby{射}{さ}して、
\ruby{胸}{むね}の
\ruby{氷}{こほり}の
\ruby{春風}{はる|かぜ}に
\ruby{逢}{あ}へるが
\ruby{如}{ごと}き
\ruby{思}{おも}ひの
\ruby{仕}{し}つ、
\ruby{其}{そ}の
\ruby{故}{ゆゑ}を
\ruby{問}{と}ふ
\ruby{暇}{いとま}も
\ruby{無}{な}く、
\ruby{今}{いま}まで
\ruby{知}{し}らざりし
\ruby{慰安}{やす|らかさ}を
\ruby{得}{\換字{𛀁}}て、
\ruby{何}{なん}とは
\ruby{無}{な}しの
\ruby{忝}{かたじけな}さに、
\ruby{淚}{なみだ}は
\ruby{止}{と}めんとして
\ruby{止}{と}めあへず、
\ruby{水晶}{すゐ|しやう}の
\ruby{珠數俄}{じゆ|ず|にはか}に
\ruby{斷}{き}れて、
\ruby{{\換字{留}}}{と}まらぬ
\ruby{珠}{たま}のばらばらと
\ruby{緖}{を}より
\ruby{亂}{みだ}れて
\ruby{落}{お}つるが
\ruby{如}{ごと}く、
\ruby{泫然}{げん|ぜん}として
\ruby{泣}{な}きに
\ruby{泣}{な}きたり。


\Entry{其二十二}

% メモ 校正終了 2024-04-23
\原本頁{120-6}%
\ruby{經}{きやう}は
\ruby{誦}{じゆ}したり
といへども
\ruby{老人}{らう|じん}
\ruby{{\換字{迷}}魂}{めい|こん}の
\ruby{{\換字{術}}}{じゆつ}を
\ruby{知}{し}れる
にもあらず、
%
\ruby{心}{こ〻ろ}こそ% 原本通り「〻(二の字点、揺すり点)」
\ruby{惑}{まど}ひたれ
\ruby[g]{水野}{みづの}
\ruby{奪魄}{だつ|ぱく}の
\ruby{法}{はふ}に
\ruby{致}{いた}さる
べくも
あらねど、
%
\ruby[g]{水野}{みづの}が
\ruby{胸中}{きやう|ちう}の
\ruby{{\換字{消}}息}{せう|そく}は
\ruby[g]{水野}{みづの}
ばかりぞ
\ruby{知}{し}る、
%
\ruby{傍觀}{わき|め}より
\ruby{云}{い}へば
たゞ% TODO 原本の「二の字点、揺すり点」に濁点のグリフが見つからないので「ゞ」
\ruby{是}{これ}
\ruby[||j>]{恰}{あたか}も% 恰も「あ(た)かも」
\ruby{神{\換字{文}}}{しん|もん}
\原本頁{120-9}\改行%
\ruby{密呪}{みつ|じゆ}の
\ruby{妖}{あや}しき
\ruby{{\換字{道}}}{みち}に
\ruby{因}{よ}つて
\ruby[g]{縛心{\換字{鎖}}意}{フアツシ{\換字{子}}ート}
されたる
\ruby{人}{ひと}の
\ruby{如}{ごと}く、
%
\ruby{今}{いま}までの
\ruby[g]{水野}{みづの}
\ruby[|j>]{某}{なにがし}は
いづくへやら
\ruby{{\換字{消}}}{き}{\換字{𛀁}}て、
%
\ruby{全}{まつた}く
\ruby{愚痴}{ぐ|ち}
\ruby{{\換字{文}}盲}{もん|もう}の
\ruby{爺}{ぢ〻}% 「ぢゞ」のはずだが、原本通り「〻(二の字点、揺すり点)」
\ruby{婆}{ば〻}% 「ばゞ」のはずだが、原本通り「〻(二の字点、揺すり点)」
のやうになり、
%
\ruby{一心}{いつ|しん}に
\ruby{御佛}{み|ほとけ}を
\ruby{頼}{たの}み
\ruby{奉}{たてまつ}れる
さまの、
%
\ruby{男兒}{をと|こ}らしからず
\ruby{憫然}{あは|れ}にのみ
\ruby{見}{み}{\換字{𛀁}}たり。

\原本頁{121-3}%
\ruby{西}{にし}に
\ruby{對}{むか}ひて
\ruby{放}{はな}ちても
\ruby{東}{ひがし}に
\ruby{對}{むか}ひて
\ruby{放}{はな}ちても、
%
\ruby{滿}{み}つる
\ruby{月}{つき}の
\ruby{形}{かたち}と
\ruby{引絞}{ひき|しぼ}りたる
\ruby{{\換字{強}}弓}{がう|きう}を、
%
きつて
\ruby{放}{はな}つ
\ruby{時}{とき}
おのづからの
\ruby{快}{こ〻ろよ}さ% 原本通り「〻(二の字点、揺すり点)」
あり。
%
\ruby{南}{みなみ}に
むかひて
\ruby{決}{けつ}しても
\ruby{北}{きた}に
むかひて
\ruby{決}{けつ}しても、
%
\ruby{千頃}{せん|けい}の% 「千頃」物事を、ある基準で区分けしたときの一つ一つ。
\ruby{瀦水}{たまり|みづ}の
\ruby{漫々}{まん|〳〵}たるを、
%
\原本頁{121-6}\改行%
\ruby[||j>]{堤}{つ〻み}を% 原本通り「〻(二の字点、揺すり点)」
\ruby{切}{き}つて
\ruby{決}{けつ}する
\ruby{時}{とき}
おのづからの
\ruby{快}{こ〻ろよ}さあり。% 原本通り「〻(二の字点、揺すり点)」
%
そも〳〵
\ruby{心}{こ〻ろ}の% 原本通り「〻(二の字点、揺すり点)」
\ruby{後}{あと}へも
\ruby{先}{さき}へも
\ruby{行}{ゆ}かざるを
\ruby{悶}{もだ{\換字{𛀁}}}とは
\ruby{云}{い}ひ、
%
\ruby{一方}{いつ|ぱう}へ
\ruby{爽}{さわや}かに
\ruby{走}{はし}るを
\ruby{快}{こ〻ろよ}しとは% 原本通り「〻(二の字点、揺すり点)」
\原本頁{121-8}\改行%
\ruby{云}{い}ふなれば、
%
\ruby{佛陀}{ほと|け}の
\ruby{利益}{り|やく}は
\ruby{有}{あ}るにせよ
\ruby{無}{な}きにせよ、
%
\ruby[g]{水野}{みづの}は
\ruby{今}{いま}まさに
\ruby{此}{こ}の
\ruby{快}{こ〻ろよ}さを% 原本通り「〻(二の字点、揺すり点)」
\ruby{味}{あぢは}へる
なるべし。

\原本頁{121-10}%
\ruby{星辰}{せい|しん}
\ruby{上}{かみ}に
か〻り、% 原本通り「〻(二の字点、揺すり点)」
%
\ruby{山河}{さん|が}
\ruby{下}{しも}に
\ruby{布}{し}ける
\ruby{此}{こ}の
\ruby{天地}{てん|ち}の
\ruby{大}{だい}にして
\ruby{大}{だい}なるを
おもひ、
%
\ruby{萬年萬々年}{ばん|ねん|ばん|〳〵|ねん}% 「〴〵」でなく原本通り「〳〵」
の
\ruby{{\換字{前}}}{まへ}に
\ruby{萬年萬々年}{ばん|ねん|ばん|〳〵|ねん}% 「〴〵」でなく原本通り「〳〵」
あり、
%
\ruby{萬年萬々年}{ばん|ねん|ばん|〳〵|ねん}% 「〴〵」でなく原本通り「〳〵」
の
\ruby{後}{のち}に
\原本頁{122-1}%
\ruby{萬年萬々年}{ばん|ねん|ばん|〳〵|ねん}% 「〴〵」でなく原本通り「〳〵」
ある
\ruby{此}{こ}の
\ruby{歳月}{さい|げつ}の
\ruby{久}{ひさ}しくして
\ruby{久}{ひさ}しきを
\ruby{思}{おも}ひ、
%
さて
\ruby{此}{こ}の
\ruby{天地}{てん|ち}の
\ruby{立}{た}てる
\ruby{{\換字{所}}以}{ゆ|{\換字{𛀁}}ん}を
おもひ
\ruby{歳月}{さい|げつ}の
\ruby{經}{ふ}る
\ruby{{\換字{所}}以}{ゆ|{\換字{𛀁}}ん}を
\ruby{思}{おも}ひて、
%
\ruby{此}{こ}の
\ruby{天地}{てん|ち}と
\ruby{歳月}{さい|げつ}との
\ruby{存在}{そん|ざい}を、
%
たゞ〳〵% TODO 原本の「二の字点、揺すり点」に濁点のグリフが見つからないので「ゞ」
\ruby{無}{む}
\ruby{意義}{い|ぎ}なる
\ruby{事實}{こと|がら}のみと
\ruby{認}{みと}めなば、
%
\原本頁{122-4}\改行%
\ruby{誰}{たれ}かは
\ruby{味氣}{あぢ|き}
\ruby{無}{な}き
\ruby{感}{おもひ}に
\ruby{撲}{う}たれて
\ruby{悲}{かなし}み
\ruby{傷}{いた}まざらん。
%
されど
\ruby{此}{こ}の
\ruby{天地}{てん|ち}と
\ruby{歳月}{さい|げつ}との
\ruby{存在}{そん|ざい}の、
%
\ruby{眞}{まこと}は
\ruby{無}{む}
\ruby{意義}{い|ぎ}の
\ruby{事實}{こと|がら}のみ
ならで、
%
\ruby{其}{その}
\ruby{中}{うち}に
\ruby{意義}{い|ぎ}ある
なりと
\ruby{認}{みと}むる
\ruby{時}{とき}は、
%
\ruby{誰}{たれ}かは
\ruby{{\換字{乳}}{\換字{房}}}{ち|ぶさ}を
\ruby{探}{さぐ}り
\ruby{得}{{\換字{𛀁}}}たる
\ruby{嬰兒}{あか|ご}の
\ruby{如}{ごと}く、
%
\原本頁{122-7}\改行%
\ruby{無限}{む|げん}の
\ruby{喜悅}{よろ|こび}に
\ruby{胸}{むね}を
\ruby{躍}{をど}らさゞらん。% TODO 原本の「二の字点、揺すり点」に濁点のグリフが見つからないので「ゞ」
%
\ruby{意義}{い|ぎ}あり、
%
\ruby{意義}{い|ぎ}あり、
%
\ruby{無}{む}
\ruby{意義}{い|ぎ}ならず、
%
\ruby{神}{かみ}の
\ruby{御心}{み|こ〻ろ}% 原本通り「〻(二の字点、揺すり点)」
\ruby{{\換字{即}}}{すなは}ち
\ruby{意義}{い|ぎ}なり、
%
\ruby{佛}{ほとけ}の
\ruby{御心}{み|こ〻ろ}% 原本通り「〻(二の字点、揺すり点)」
\ruby{{\換字{即}}}{すなは}ち
\ruby{意義}{い|ぎ}なり、
%
\ruby{化醇}{くわ|じゆん}の
\ruby[g]{大法}{おきて}は
こ〻にあるなり、% 原本通り「〻(二の字点、揺すり点)」
%
\ruby{歸善}{き|ぜん}の
\ruby{定數}{さだ|まり}
こ〻にあるなり、% 原本通り「〻(二の字点、揺すり点)」
%
\ruby{大慈}{だい|じ}の
\原本頁{122-10}\改行%
\ruby{光明}{ひか|り}は
\ruby{柔}{やはら}かに
\ruby{山村}{さん|そん}
\ruby[||j>]{水鄕}{すゐ|きやう}を
\ruby{包}{つ〻}めるなり、% 原本通り「〻(二の字点、揺すり点)」
%
\ruby{大悲}{だい|ひ}の
\ruby{音樂}{おん|がく}は
\ruby{斷}{た}ゆる
\ruby{間}{ま}も
\ruby{無}{な}く
\ruby{{\換字{古}}往}{こ|わう}
\ruby{今來}{こん|らい}に
\ruby{亘}{わた}れるなり、
%
\ruby{我}{われ}は
\ruby{此}{こ}の
\ruby{溫{\換字{暖}}}{あた|〻か}き% 原本通り「〻(二の字点、揺すり点)」
\ruby{意義}{い|ぎ}の
\ruby{中}{うち}より
\ruby{生}{うま}れたる
\ruby{子}{こ}なり、
%
\ruby{神}{かみ}の
\ruby{子}{こ}なり
\ruby{佛}{ほとけ}の
\ruby{子}{こ}なり
\ruby[g]{正眞}{まこと}の
\ruby{子}{こ}なり、
%
\ruby{我}{われ}と
\ruby{神佛}{かみ|ほとけ}とは
\ruby{血}{ち}の
\ruby{相}{あひ}
\ruby{{\換字{通}}}{かよ}へる
なり、
%
と
\ruby{如是}{か|く}
\ruby{思}{おも}ふ
\ruby{時}{とき}
おのづと
\ruby{悅}{よろこ}ばしからば、
%
\ruby[g]{水野}{みづの}は
\ruby{今}{いま}きさに
\ruby{此}{こ}の
\ruby{悅}{よろこ}びを
おぼえたる
なるべし。

\原本頁{123-4}%
\ruby[g]{水野}{みづの}の
やうやく
\ruby{念}{ねん}じ
\ruby{{\換字{終}}}{をは}れる
\ruby{時}{とき}、
%
\ruby{老人}{らう|じん}は
また
\ruby[g]{水野}{みづの}に
\ruby{對}{むか}ひて、

\原本頁{123-5}%
『
あ〻% 原本通り「〻(二の字点、揺すり点)」
\ruby{御信心}{ご|しん|〴〵}なさい
まし〳〵、
%
\ruby{自然}{ひと|りで}に
\ruby{有}{あ}りがたい
ことが
\ruby{能}{よ}く
\ruby{解}{わか}つて
まゐります!。
%
まあ
\ruby{何樣}{ど|ん}な
\ruby{事}{こと}か
\ruby{存}{ぞん}じませんが、
%
\ruby{御樣子}{ご|やう|す}を
\ruby{見}{み}ました
ところでは、
%
よく〳〵の
\ruby{御心配事}{ご|しん|ぱい|ごと}が
\ruby{御有}{お|あ}りなさると
\ruby{御察}{お|さつ}し
\ruby{申}{まをし}ます。
%
\ruby{御籤}{お|みくじ}を
\ruby{御戴}{お|いたゞ}きなさい、% TODO 原本の「二の字点、揺すり点」に濁点のグリフが見つからないので「ゞ」
%
\ruby{御籤}{お|みくじ}を
\ruby{御戴}{お|いたゞ}きなさい。% TODO 原本の「二の字点、揺すり点」に濁点のグリフが見つからないので「ゞ」
%
あ〻% 原本通り「〻(二の字点、揺すり点)」
まだ
\ruby{御戴}{お|いたゞ}きなさつた% TODO 原本の「二の字点、揺すり点」に濁点のグリフが見つからないので「ゞ」
\ruby{事}{こと}が
\ruby{御有}{お|あ}んなさらないので、
%
\ruby{御{\換字{勝}}手}{ご|かつ|て}が
\ruby{知}{し}れないので
ございますネ。
%
\ruby{宜}{よ}うございます
\ruby{私}{わたくし}が
\ruby{戴}{いたゞ}いて% TODO 原本の「二の字点、揺すり点」に濁点のグリフが見つからないので「ゞ」
あげましやう。
』

\原本頁{123-11}%
と、
%
\ruby{世話}{せ|わ}を
\ruby{燒}{や}きて
\ruby[g]{水野}{みづの}が
まだ
\ruby{何}{なに}とも
\ruby{答}{こたへ}を
せざるに、
%
はや
\ruby{御籤}{み|くじ}を
\原本頁{124-1}\改行%
\ruby{管}{つかさど}る
\ruby{僧}{そう}の
\ruby{許}{もと}に
\ruby{至}{いた}りぬ。

\原本頁{124-2}%
やがて
\ruby{僧}{そう}は
\ruby{御籤箱}{お|みくじ|ばこ}を
ふる
なる
べし、
%
かた〳〵
といふ
\ruby{音}{おと}は
\ruby{小暗}{を|ぐら}き
\ruby{其}{そ}の
\ruby{座}{ざ}の
あたりより
\ruby{聞}{きこ}{\換字{𛀁}}ぬ。

\Entry{其二十三}

よしや
\ruby{大吉}{だい|きち}ならぬまでもせめては
\ruby{凶}{きよう}ならぬ
\ruby{御籤}{み|くじ}を
\ruby{得}{え}て、
\ruby{憂}{うれひ}に
\ruby{沈}{しづ}み
\ruby{悲}{かなしみ}に
\ruby{陷}{おちゐ}れる
\ruby{氣}{き}を
\ruby{引立}{ひき|た}て、
\ruby{信心}{しん|じん}の
\ruby{勇}{いさみ}を
\ruby{附}{つ}けて
\ruby{{\換字{呉}}}{く}れんと
\ruby{爲}{し}たるらしき
\ruby{親切}{しん|せつ}の
\ruby{老人}{らう|じん}が、
\ruby{思}{おも}ふこと
\ruby{{\GWI{u9055-k}}}{たが}ひて
\ruby{甚}{いた}く
\ruby{望}{のぞみ}を
\ruby{失}{うしな}へるは、
\ruby{忽}{たちま}ち
\ruby{先}{ま}づ
\ruby{其}{そ}の
\ruby{色}{いろ}に
\ruby{現}{あらは}れて、
\ruby{僧}{そう}より
\ruby{受取}{うけ|と}りし
\ruby{御籤}{み|くじ}をば、
\ruby{力無}{ちから|な}げに
\ruby{輪}{わ}に
\ruby{{\換字{巻}}}{ま}きながら、
\ruby{鈍}{にぶ}る〳〵
\ruby{此方}{こな|た}へ
\ruby{步}{あゆ}み
\ruby{來}{きた}れるに、
\ruby[g]{水野}{みづの}は
\ruby{見}{み}ずして
\ruby{既}{すで}に
\ruby{其}{そ}の
\ruby{{\換字{文}}}{ぶん}の
\ruby{凶}{きよう}なるを
\ruby{知}{し}れり。

\ruby{第何十何番大吉}{だい|なん|じう|なん|ばん|だい|きち}といふならば、
\ruby{如何}{い|か}ばかりか
\ruby{{\換字{悅}}}{よろこ}び
\ruby{勇}{いさ}んで
\ruby{示}{しめ}すべきを、
\ruby{老人}{らう|じん}は
\ruby{{\換字{巻}}}{ま}きたるま〻
\ruby{御籤}{み|くじ}を
\ruby[g]{水野}{みづの}の
\ruby{懷中}{ふと|ころ}に
\ruby{輕}{かる}く
\ruby{押入}{おし|い}れて、

『
\ruby{何様}{ど|う}か
\ruby{吉凶}{よし|あし}にかヽはらず
\ruby{御信心}{ご|しん|〴〵}なさい。
\ruby{大吉}{だい|きち}でも
\ruby{驕}{おご}れば
\ruby{凶}{きよう}に
\ruby{反}{かへ}ります、たとへ
\ruby{凶}{きよう}でも
\ruby{御信心}{ご|しん|〴〵}を
\ruby{{\換字{強}}}{つよ}くなすつて、それからまた
\ruby{改}{あらた}めて
\ruby{御籤}{おみ|くじ}を
\ruby{御戴}{おい|たヾ}きなすつてごらんなさい、
\ruby{吉}{きち}になりますこともございますものです。
\ruby{吉}{よい}につけ
\ruby{凶}{わるい}につけ
\ruby{御信心}{ご|しん|〴〵}が
\ruby{大切}{たい|せつ}です。
\ruby{決}{けつ}して
\ruby{信}{しん}を
\ruby{御冷}{お|さま}しなすつてはいけません。
さてそろ〳〵もう
\ruby{下向}{げ|かう}いたしましやう。
』

と、
\ruby{云}{い}ひ
\ruby{{\換字{終}}}{をは}つて
\ruby{本尊}{ほん|ぞん}をまた
\ruby{一拜}{いつ|ぱい}して、おのれ
\ruby{先}{ま}づ
\ruby{御堂}{み|だう}を
\ruby{去}{さ}らんとしたり。

\ruby{老人}{らう|じん}が
\ruby{様子}{やう|す}の
\ruby{急}{きふ}にそはつけるは、
\ruby{何}{なん}の
\ruby{意}{こヽろ}も
\ruby{無}{な}かりし
\ruby{我}{われ}に
\ruby{智慧}{ち|ゑ}をつけて
\ruby{御籤}{み|くじ}を
\ruby{取}{と}らせたるに、その
\ruby{御籤}{み|くじ}のことのほか
\ruby{凶}{あし}かりしかば、
\ruby{却}{かへ}つて
\ruby{其}{そ}のために
\ruby{憂}{うれひ}を
\ruby{{\換字{増}}}{ま}し、
\ruby{悲}{かなしみ}を
\ruby{添}{そ}ふることもやと、
\ruby{氣}{き}の
\ruby{毒}{どく}さに
\ruby{堪}{た}へかねて
\ruby{傍}{かたへ}に
\ruby{居}{ゐ}づらく
\ruby{狭}{せま}くして
\ruby{正直}{しやう|ぢき}なる
\ruby{心}{こヽろ}の
\ruby{憫}{あは}れにも
\ruby{落着}{おち|つ}きかぬるが
\ruby{爲}{ため}なるべし。
\ruby{{\換字{平}}生}{ひ|ごろ}の
\ruby{我}{われ}を
\ruby{知}{し}らずして、たヾ
\ruby{自己}{お|の}が
\ruby{身}{み}にのみ
\ruby{比較}{ひき|くら}ぶれば、
\ruby{然}{ま}る
\ruby{心{\GWI{u9063-k}}}{こヽろ|づかひ}をするも
\ruby{無理}{む|り}ならねど、
\ruby{御佛}{み|ほとけ}の
\ruby{廣大}{くわう|だい}なる
\ruby{御誓願}{おん|ちか|ひ}をこそ
\ruby{頼}{たの}み
\ruby{奉}{たてまつ}りつれ、
\ruby{御鬮}{み|くじ}といふ
\ruby{事}{こと}は
\ruby{御經}{おん|きやう}にも
\ruby{見}{み}えず、
\ruby{賣僧}{まい|す}の
\ruby{仕出}{し|だ}したるなるべき春の
\ruby{{\GWI{u904a-k}}{\換字{戱}}}{あそ|び}の
\ruby{寶引}{はう|びき}といふにも
\ruby{似}{に}たる
\ruby{埒無}{らち|な}く
\ruby{據無}{よりどころ|な}き
\ruby{御籤}{み|くじ}の
\ruby{{\換字{文}}}{ぶん}なんどに、
\ruby{我}{われ}いかで
\ruby{心}{こヽろ}を
\ruby{動}{うご}かされんや。
それとも
\ruby{知}{し}らずして
\ruby{性質}{ひ|と}の
\ruby{好}{よ}き
\ruby{老人}{らう|じん}の、
\ruby{心}{こヽろ}を
\ruby{{\GWI{u9063-k}}}{つか}ふ
\ruby{笑止}{せう|し}さ、と
\ruby[g]{水野}{みづの}は
\ruby{却}{かへ}つて
\ruby{老人}{らう|じん}を
\ruby{憐}{あはれ}み、わざと
\ruby{懷中}{くわい|ちう}の
\ruby{御籤}{み|くじ}を
\ruby{其儘}{その|まヽ}にして
\ruby{讀}{よ}まず。
\ruby{共}{とも}に
\ruby{石路}{せき|ろ}の
\ruby{長々}{なが|〳〵}しきを
\ruby{下向}{げ|かう}しけるが、
\ruby{老人}{らう|じん}は
\ruby{懷中}{ふと|ころ}より
\ruby{折本}{をり|ほん}になりたる
\ruby{普門品}{ふ|もん|ほん}の
\ruby{小}{ちいさ}きを
\ruby{取}{と}り
\ruby{出}{いだ}して、

『だいなしになつて
\ruby{居}{を}りまする
\ruby{物}{もの}を、
\ruby{呈}{あ}げると
\ruby{申}{まを}しては
\ruby{失禮}{しつ|れい}ですけれど、まあ
\ruby{如是}{か|う}いふ
\ruby{物}{もの}の
\ruby{事}{こと}ですから
\ruby{御免下}{ご|めん|くだ}さい。
これを
\ruby{貴君}{あな|た}に
\ruby{差上}{さし|あ}げますから、
\ruby{何様}{ど|う}か
\ruby{御取}{お|と}りなすつて
\ruby{下}{くだ}さいまし。
私はもう
\ruby{無書}{そ|ら}で
\ruby{記}{おぼ}\GWI{u1b001}ましたから、
\ruby{此書}{こ|れ}は
\ruby{用}{よう}が
\ruby{明}{あ}いたのでございますが、
\ruby{何様}{ど|う}か
\ruby[g]{貴君}{あなた}も
\ruby{御拜}{お|が}みなさるたびに、これを
\ruby{御覧}{ご|らん}になりながら
\ruby{御經}{お|きやう}を
\ruby{御}{お}あげなすつて
\ruby{下}{くだ}されば、
\ruby{私}{わたくし}は
\ruby{大變}{たい|へん}に
\ruby{嬉}{うれ}しいと
\ruby{思}{おも}ふのでございます。
それに
\ruby{此}{こ}の
\ruby{末}{すゑ}の
\ruby{方}{はう}に
\ruby{私}{わたくし}の
\ruby{名住所}{な|とこ|ろ}が
\ruby{小}{ちひ}さく
\ruby{書}{か}いてございますから、
\ruby{何}{なん}ぞの
\ruby{御序}{おつ|ひで}でも
\ruby{御有}{お|あ}りでしたら
\ruby{御立寄}{お|たち|よ}り
\ruby{下}{くだ}さいまし、いろ〳〵
\ruby{御利生}{ご|り|しやう}の
\ruby{御話}{おは|なし}やなんぞを
\ruby{致}{いた}しましやうから。
ではまた
\ruby{明日御目}{みやう|にち|お|め}にかゝりましやう。
どうか
\ruby{撓}{たゆ}まずに
\ruby{御信心}{ご|しん|〴〵}なすつて!。
』

と
\ruby{云}{い}ひたき
\ruby{事}{こと}のみを
\ruby{云}{い}ひて
\ruby{{\換字{終}}}{つひ}に
\ruby{別}{わか}れたり。

\ruby{冊子}{ほ|ん}は
\ruby{言}{ことば}を
\ruby{費}{つひや}して
\ruby{辭}{いな}むべきほどのものにもあらず、
\ruby{特}{こと}に
\ruby{快}{こヽろよ}く
\ruby{受}{う}け
\ruby{納}{をさ}めて
\ruby[g]{芳志}{こヽろざし}を
\ruby{無}{む}にせざらんは、
\ruby{差}{さ}し
\ruby{當}{あた}つての
\ruby{{\GWI{u9053-k}}}{みち}なるべしと、
\ruby[g]{水野}{みづの}は
\ruby{老人}{らう|じん}に
\ruby{厚意}{かう|い}を
\ruby{謝}{しや}して、
\ruby{袖}{そで}を
\ruby{{\換字{分}}}{わか}つて
\ruby[g]{東方}{ひがし}へ
\ruby{去}{さ}りつ、
\ruby{先}{ま}づ
\ruby{普門品}{ふ|もん|ぼん}を
\ruby{懷中}{ふと|ころ}に
\ruby{入}{い}るゝに、
\ruby{{\換字{巻}}}{ま}きたる
\ruby{彼}{か}の
\ruby{御籤}{み|くじ}のかさ〳〵と
\ruby{手}{て}に
\ruby{觸}{ふ}れたれば、
\ruby{引{\換字{交}}}{ひき|ちが}へて
\ruby{取}{と}り
\ruby{出}{いだ}して
\ruby{其{\換字{文}}}{その|ぶん}を
\ruby{讀}{よ}むに、\\

\hspace*{1zw}

\begin{tblr}{colspec={Q[c] | Q[l,t] Q[l,b]}, stretch=0.5}
  \SetCell[r=4]{c,1em}{第七番凶}&
  \kundoku{登}{ふねにの}{}{レ }
  \kundoku{舟}{ぼりて }{}{}
  \kundoku{待}{びんぷう}{}{二}
  \kundoku{便}{をまてば}{}{}
  \kundoku{風}{   }{}{一}。

  & \scriptsize{\noindent
    舟にのりて行かんとす\newline
    ればおひてが無い
  }\\
  %%%
  &
  \kundoku{月}{げつ し}{}{}
  \kundoku{色}{よく く}{}{}
  \kundoku{暗}{らくして}{}{}
  \kundoku{朦}{もう  }{}{}
  \kundoku{朧}{ろう }{}{}。

  & \scriptsize{\noindent
    見れば空もわるくして\\
    月もくらきぞ
  }\\
  %%%
  &
  \kundoku{欲}{かうりん}{}{下}
  \kundoku{輾}{をきしら}{}{二}
  \kundoku{香}{してさら}{}{}
  \kundoku{輪}{んとほつ}{}{一}
  \kundoku{去}{すれば}{}{上}。

  & \scriptsize{\noindent
    車にのりておもふとこ\\
    ろへゆかんとすれば
  }\\
  &
  \kundoku{高}{かう  }{}{}
  \kundoku{山}{ざん  }{}{}
  \kundoku{千}{せん  }{}{}
  \kundoku{萬}{ばん  }{}{}
  \kundoku{里}{りなり}{}{}。

  & \scriptsize{\noindent
    つゞける山〻恐ろしく\\ %% 山〻 vs 山々
    高くしてそれも叶はぬ
  }
\end{tblr}
 \\
 \\
とありて、ひし〳〵と
\ruby{我}{わ}が
\ruby{身}{み}の
\ruby{上}{うへ}に
\ruby{巧}{よ}く
\ruby{中}{あた}りたり。

もとより
\ruby{取}{と}るに
\ruby{足}{た}らぬことゝは
\ruby{思}{おも}ひながらも、
\ruby{不思議}{ふ|し|ぎ}に
\ruby{中}{あた}れる
\ruby{此}{こ}の
\ruby{{\換字{文}}}{ぶん}の
\ruby[g]{流石}{さすが}に
\ruby{胸}{むね}に
\ruby{徹}{こた}へて
\ruby{心}{こヽろ}さびしく、じつと
\ruby{眼}{め}を
\ruby{留}{と}めて
\ruby{見}{み}れば、
\ruby{末}{すゑ}の
\ruby{方}{かた}に
\ruby{女{\換字{文}}字}{をんな|も|じ}にて
\ruby{細}{こまか}に
\ruby{注}{ちう}し
\ruby{記}{しる}せる
\ruby{其最先}{その|まつ|さき}に、

\ruby{病事}{やまひ|ごと}は
\ruby{十}{じう}に
\ruby{六七}{ろく|しち}
\ruby{本復無}{ほん|ぷく|な}し、
\ruby{長}{なが}びきたらば
\ruby{後}{のち}は
\ruby{息災}{そく|さい}になる
\ruby{事}{こと}もあるべし、よく
\ruby{信力}{しん|りき}をもて
\ruby{佛神}{ぶつ|しん}を
\ruby{頼}{たの}みて
\ruby{𠮷}{よし}、

とありたるは、いよ〳〵
\ruby{何}{なに}となく
\ruby{不快}{ふ|くわい}を
\ruby{感}{かん}じて、
\ruby{腹}{はら}の
\ruby{底}{そこ}より
\ruby{{\換字{寒}}}{さむさ}の
\ruby{上}{のぼ}り
\ruby{來}{きた}るやうにおぼえたり。

\ruby{何}{なに}とか
\ruby{思}{おも}ひけん
\ruby[g]{水野}{みづの}は
\ruby{引{\換字{返}}}{ひつ|かへ}して、
\ruby{復}{また}
\ruby{相良}{さが|ら}を
\ruby{訪}{と}ひぬ。
\ruby{待}{ま}つ
\ruby{事一時餘}{こと|いち|じ|あま}りにして
\ruby{{\換字{終}}}{つひ}に
\ruby{相良}{さが|ら}に
\ruby{親}{した}しく
\ruby{會}{あ}ひ
\ruby{得}{\GWI{u1b001}}て、
\ruby{必}{かなら}ず
\ruby{見舞}{み|ま}はんとの
\ruby{辭}{ことば}を
\ruby{得}{\GWI{u1b001}}て
\ruby{歸}{かへ}りしが、
\ruby{幸}{さいはひ}にして
\ruby{今日}{け|ふ}は
\ruby[g]{休校}{やすみ}の
\ruby{日}{ひ}なればこそ
\ruby{宣}{よ}けれ、
\ruby{吾妻橋}{あ|づま|ばし}にかヽれる
\ruby{時}{とき}は
\ruby{既}{すで}に
\ruby{九時}{く|じ}に
\ruby{{\換字{近}}}{ちか}からんとしたり。


\Entry{其二十四}

% メモ 校正終了 2024-04-23 2024-06-01
\原本頁{130-3}%
\ruby{雷神門}{かみ|なり|もん}は
いつも
ながら
\ruby{人}{ひと}の
ぞよ% 断定した内容を、さらに念を押す気持ちを表す。… なのだよ。… だぞ。
つきて
\ruby{目}{め}まぐるしき
\ruby{地}{ところ}なり。
%
わけて
\ruby[g]{今日}{け ふ }は
\ruby[g]{日曜}{にち{\換字{𛀁}}う}の
\ruby{事}{こと}とて、
%
\ruby[g]{掻頭}{かんざし}に
\ruby{花}{はな}を
\ruby{{\換字{飾}}}{かざ}らする
\ruby[g]{九歳}{こゝのつ}% 踊り字調整「〻(二の字点、揺すり点)に見えるが(ゝ)」
\ruby[g]{十歳}{と を }の
\ruby{女}{をんな}の
\ruby{兒}{こ}、
%
\ruby{金{\換字{文}}字}{きん|も|じ}% 踊り字調整「〻(二の字点、揺すり点)に濁点に見えるが(ゞ)」
かゞやく
\ruby{天鵞絨}{び|ろう|ど}
\ruby[g]{帽子}{ばうし }
かぶらせたる
\ruby[g]{洋服}{やうふく}
\ruby[||j>]{姿}{すがた}
\ruby[g]{可憐}{かはゆ}らしき
\ruby[g]{六歳}{むつゝ }% 踊り字調整「〻(二の字点、揺すり点)に見えるが(ゝ)」
\ruby[g]{七歳}{なゝつ }の% 踊り字調整「〻(二の字点、揺すり点)に見えるが(ゝ)」% 原本には漢数字「七」のルビ有り
\ruby{男}{をとこ}の
\ruby{兒}{こ}など
\ruby{引}{ひき}
\ruby{{\換字{連}}}{つ}れて、
%
\ruby{世}{よ}を
\ruby{樂}{たの}しげに
\ruby{仲見世}{なか|み|せ}に
\ruby{入}{い}る
\ruby{御母樣}{お|つか|さん}も
あれば、
%
\ruby[<j||]{農}{ひやく}
\ruby[||>]{家}{しやう}には
% \ruby{農家}{ひやく|しやう}には
\ruby[||j>]{{\換字{違}}}{ちがひ}
\ruby[||j>]{無}{ な}き
\ruby[g]{乾疥}{はたけ }
\ruby{面}{がほ}に、
%
\ruby[g]{白{\換字{粉}}}{おしろい}の
\ruby[g]{不{\換字{均}}}{む ら }の
\ruby[g]{奇異}{ふしぎ }に
\原本頁{130-8}\改行%
をかしき、
%
\ruby{猫}{ねこ}が
\ruby{化}{ば}けた
やうな
\ruby[g]{小娘}{こむすめ}
\ruby{{\換字{連}}}{れん}の、
%
\ruby{何憂事}{なに|うき|こと}も
\ruby{知}{し}らで
\ruby[<j||]{觀}{くわん}% 「觀音」の読みは原本通り「くわん(の)ん」% 行末行頭の境界付近なので特例処置を施す
\ruby{音}{のん}
\ruby{樣}{さま}に
\ruby{參}{まゐ}るあり。
%
\ruby{妾}{われ}も
\ruby[g]{人生}{ひとのよ}の
\ruby{春}{はる}に
\ruby{{\換字{遊}}}{あそ}べる
\ruby[g]{蝶々}{てふ〳〵}
\ruby{髷}{まげ}の、
%
まだ
\ruby[g]{何事}{なにごと}も
\ruby{知}{し}らざりし
\ruby{頃}{ころ}は、
%
たゞ% 踊り字調整「〻(二の字点、揺すり点)に濁点に見えるが(ゞ)」
あどけ
\ruby{無}{な}う
\ruby[g]{面白}{おもしろ}う
\ruby[g]{此地}{こ ゝ }を% 踊り字調整「〻(二の字点、揺すり点)に見えるが(ゝ)」
\ruby[g]{極樂}{ごくらく}のやうに
\ruby{思}{おも}ひし
\原本頁{131-1}\改行%
\ruby{時}{とき}も
ありしと、
%
\ruby{遙}{はるか}に
\ruby[g]{山門}{さんもん}を
\ruby{望}{のぞ}むにも
\ruby[g]{往時}{むかし }
\ruby[||j>]{懷}{なつか}しく、
%
\ruby{{\換字{通}}}{とほ}り
すがりなれど
\ruby[g]{御堂}{み だう}の
\ruby{方}{かた}を
\ruby[g]{一寸}{ちよつと}
\ruby{拜}{をが}みて、
%
そのまゝ% 踊り字調整「〻(二の字点、揺すり点)に見えるが(ゝ)」
\ruby{東}{ひがし}に
\ruby{切}{き}れて
\ruby{行}{ゆ}けば、

\原本頁{131-3}%
『
\ruby[g]{姐樣}{ね{\換字{𛀁}}さん}、
%
\ruby[g]{如何}{い かゞ}です、% 踊り字調整「〻(二の字点、揺すり点)に濁点に見えるが(ゞ)」
%
\ruby[g]{御安}{お やす}く
まゐりましやう。
』

\原本頁{131-4}%
『
\ruby[g]{姐樣}{ね{\換字{𛀁}}さん}、
%
\ruby[g]{如何}{い かゞ}です% 踊り字調整「〻(二の字点、揺すり点)に濁点に見えるが(ゞ)」
\ruby[g]{御安}{お やす}く
\ruby[g]{如何}{い かゞ}です。% 踊り字調整「〻(二の字点、揺すり点)に濁点に見えるが(ゞ)」
』

\原本頁{131-5}%
と
\ruby[g]{車夫}{くるまや}の
\ruby[g]{聲々}{こゑ〴〵}
かしましく
\ruby{煩}{うる}さし。

\原本頁{131-6}%
\ruby{久}{ひさ}しぶりにて
\ruby{渡}{わた}る
\ruby{吾妻橋}{あづ|ま|ばし}より% ルビ調整(原本通り)
\ruby[g]{川上}{かはかみ}の
\ruby{方}{かた}を
\ruby{{\換字{遠}}}{とほ}く
\ruby{見}{み}れば、
%
\ruby{水}{みづ}は
\ruby[||j>]{昔}{むかし}
\ruby[||j>]{見}{ み}たりし
\ruby{如}{ごと}く
\ruby{{\換字{緩}}}{ゆる}く
\ruby{流}{なが}れて、
%
\ruby[g]{右手}{みぎて }に
\ruby{長}{なが}き
\ruby[g]{一帶}{いつたい}の
\ruby{堤}{つゝみ}の、% 踊り字調整「〻(二の字点、揺すり点)に見えるが(ゝ)」
%
\ruby{其}{そ}の
\ruby{狀}{さま}も
\ruby{{\換字{更}}}{さら}に
\原本頁{131-8}\改行%
\ruby[g]{記臆}{おぼ{{\換字{𛀁}}}}に% 原本通り「おぼ𛀁」
\ruby{異}{かは}らず、
%
\ruby{岸}{きし}の
\ruby{櫻}{さくら}の
\ruby{葉}{は}も
\ruby{{\換字{透}}}{す}ける
ながら、
%
その
\ruby{花}{はな}の
\ruby{眺}{ながめ}も
おもかげに
\ruby{立}{た}つて、
%
あゝ% 踊り字調整「〻(二の字点、揺すり点)に見えるが(ゝ)」
\ruby{彼}{あ}の
\ruby{花}{はな}の
\ruby[g]{隧{\換字{道}}}{とんねる}
のやうであつた
\ruby{中}{なか}を、
%
\ruby{夜}{よる}の
\ruby{風}{かぜ}の
\ruby{些}{やゝ}% 踊り字調整「〻(二の字点、揺すり点)に見えるが(ゝ)」
\ruby{{\換字{寒}}}{さむ}かつた
\ruby{時}{とき}、
%
\ruby[g]{彼人}{ひ と }に
\ruby{手}{て}を
\ruby{取}{と}られて
\ruby[g]{人目}{ひとめ }の
\ruby{羞}{はづか}しく、
%
\ruby{暗}{くら}き
\ruby{方}{かた}に
\原本頁{131-11}\改行%
\ruby{身}{み}を
\ruby{寄}{よ}せて
\ruby{歩}{ある}きし
\ruby{春}{はる}の
\ruby{{\換字{宵}}}{よ}も
ありしが、
%
\ruby{思}{おも}へば
\ruby{今}{いま}
\ruby[g]{其事}{そ れ }の
\ruby{思}{おも}ひ
\ruby{出}{だ}さるゝも% 踊り字調整「〻(二の字点、揺すり点)に見えるが(ゝ)」
\ruby[g]{甲{\換字{斐}}}{か ひ }
\ruby{無}{な}く
\ruby{愚}{おろか}なりと、
%
しきりに
\ruby{路}{みち}を
\ruby{急}{いそ}ぎて
\ruby{橋}{はし}を
\ruby{渡}{わた}り
\ruby{盡}{つく}し、
%
また
\ruby{煩}{うる}さく
\ruby[g]{車夫}{くるまや}の
\ruby{勸}{すゝ}むる% 踊り字調整「〻(二の字点、揺すり点)に見えるが(ゝ)」
\ruby{中}{なか}を
\ruby[g]{停車}{ていしや}
\ruby{塲}{ぢやう}へと% 原文通り「塲」
\ruby{向}{むか}ひぬ。

\原本頁{132-3}%
\ruby{乘}{の}れと
\ruby{勸}{すゝ}めて% 踊り字調整「〻(二の字点、揺すり点)に見えるが(ゝ)」
\ruby{乘}{の}らぬを
\ruby[g]{車夫}{しやふ }の
\ruby{憎}{にく}がりて、

\原本頁{132-4}%
『
\ruby[g]{姐々}{ね{\換字{𛀁}}さん}、
%
\ruby[g]{滊車}{き しや}なら
\ruby{{\換字{猶}}}{なほ}の
\ruby{事}{こと}、
%
\ruby{乘}{の}らないと
\ruby{間}{ま}に
\ruby{合}{あ}はないよ、
%
\ruby{九時四十五{\換字{分}}}{く|じ|よん|じふ|ご|ふん}
だから
もう
\ruby[g]{發車}{で る }のだよ。
』

\原本頁{132-6}%
『
そんなに
\ruby{急}{いそ}いで
\ruby{歩}{ある}くと
\ruby[<j||]{女}{をんな}
\ruby[||j>]{振}{ぷつり}が
\ruby{下}{さが}るぜ。
』

\原本頁{132-7}%
『
\ruby[g]{滊車}{き しや}までなら
\ruby{直}{ぢき}だから、
%
\ruby{乗}{の}せてつて
\ruby{上}{あ}げようか、
%
\ruby[g]{無錢}{た ゞ }でも% 踊り字調整「〻(二の字点、揺すり点)に濁点に見えるが(ゞ)」
\ruby{關}{かま}わないんだ、
%
ハヽヽ。
』

\原本頁{132-9}%
なんどゝ% 踊り字調整「〻(二の字点、揺すり点)に見えるが(ゝ)」
\ruby[g]{口々}{くち〴〵}に
\ruby[g]{下賤}{げ す }の
ものゝ% 踊り字調整「〻(二の字点、揺すり点)に見えるが(ゝ)」
\ruby{好}{す}きな
\ruby{事}{こと}をいふに、
%
\ruby[g]{虛言}{う そ }とは
\ruby{思}{おも}ひながら、
%
おのづと
\ruby{氣}{き}の
\ruby{急}{せ}きて、
%
\ruby[g]{疾足}{はやあし}になり、
%
やがて
\ruby{停車塲}{すて|い|しよん}に% 原文通り「塲」
\ruby{到}{いた}り
\ruby{着}{つ}けば、
%
\ruby[g]{車夫}{しやふ }も
\ruby{出鱈目}{で|たら|め}は
\ruby{云}{い}はざりしと
\ruby{見}{み}え、
%
\ruby{危}{あやふ}くも
\ruby{乘}{の}り
\ruby{後}{おく}れんと
するほどの
ところ
なりけり。

\原本頁{133-2}%
\ruby[g]{切符}{きつぷ }を
\ruby{買}{か}ふ
\ruby{間}{ま}も
\ruby{疾}{と}しや
\ruby{遲}{おそ}しや、

\原本頁{133-3}%
『
\ruby{早}{はや}く
\ruby{早}{はや}く、
』

\原本頁{133-4}%
と
\ruby[g]{驛夫}{{\換字{𛀁}}きふ}の
\ruby{云}{い}ふに
いよ〳〵
\ruby{慌}{あわ}てゝ% 踊り字調整「〻(二の字点、揺すり点)に見えるが(ゝ)」
\ruby[g]{車中}{しやちう}に
\ruby{入}{い}れば、
%
どしんといふ
\ruby{恐}{おそ}ろしき
\ruby{音}{おと}して
\ruby{{\換字{扉}}}{と}は
\ruby{烈}{はげ}しく
\ruby{閉}{し}められ、
%
\ruby{號令笛}{あ|ひ|づ}は
ビーと
\ruby{鳴}{な}り、
%
\ruby{車}{くるま}は
\ruby{動}{うご}き
\ruby{出}{だ}しぬ。

\原本頁{133-7}%
\ruby{機關手}{き|くわん|しゆ}の
\ruby[g]{手荒}{て あら}き
\ruby{男}{をとこ}なればにや、
%
\ruby{車}{くるま}の
\ruby[g]{俄然}{にはか }に
\ruby{{\換字{強}}}{つよ}く
\ruby{動}{うご}き
\ruby{出}{いだ}したるに
\改行% 校正作業の簡略化のため
、
%
\原本頁{133-8}\改行%
\ruby{人}{ひと}は
\ruby[g]{車室}{しやしつ}に
\ruby{多}{おほ}から
ざりし
ながら、
%
いづくに
\ruby{座}{すわ}らんかと
\ruby{席}{せき}を
\ruby{取}{と}り
\原本頁{133-9}\改行%
\ruby{{\換字{迷}}}{まよ}ひて、
%
\ruby{未}{ま}だ
\ruby{身}{み}を
\ruby{落}{おち}
\ruby{付}{つ}くるに
\ruby{暇}{いとま}
あらざりし
お
\ruby{龍}{りう}は、
%
\ruby{忽}{たちま}ち
\ruby{危}{あやふ}く
\ruby{倒}{たふ}れん
として、
%
\ruby{女}{をんな}の
\ruby{意久地}{い|く|ぢ}
\ruby{無}{な}く
よろ〳〵と
\ruby{歩}{あし}の
\ruby{縺}{もつ}るゝ% 踊り字調整「〻(二の字点、揺すり点)に見えるが(ゝ)」
\ruby{時}{とき}、
%
ハツと
\ruby{思}{おも}ひし
\ruby{折}{をり}は
\ruby{既}{すで}に
\ruby{遲}{おそ}くして、
%
\ruby{{\換字{猶}}}{なほ}
\ruby[||j>]{新}{あたら}しき
\ruby[g]{吾妻}{あづま }% ルビ調整(原本通り)
\ruby[g]{下駄}{げ た }の、
%
\ruby[g]{樫齒}{かしば }の
\ruby{角}{かど}
\ruby{立}{た}てるを
\ruby{以}{も}て
したゝかに、% 踊り字調整「〻(二の字点、揺すり点)に見えるが(ゝ)」
%
\ruby[g]{後方}{うしろ }の
\ruby{人}{ひと}の
\ruby{足}{あし}を
\ruby{踏}{ふ}みたり。

\Entry{其二十五}

お
\ruby{龍}{りう}はやうやくにして
\ruby{踏止}{ふみ|とゞ}まりて、
\ruby{驚}{おどろ}き
\ruby{易}{やす}き
\ruby{女氣}{をんな|ぎ}のどつきりと
\ruby{胸}{むね}を
\ruby{躍}{をど}らせつ、
\ruby{何思案}{なに|し|あん}する
\ruby{暇}{ひま}も
\ruby{無}{な}く、

『
\ruby{御免}{ご|めん}なすつて
\ruby{下}{くだ}さいまし、
\ruby{飛}{と}んだ
\ruby{{\換字{過}}失}{そ|そう}を
\ruby{致}{いた}しました。
』

と
\ruby{振}{ふ}り
\ruby{顧}{かへ}りさまに
\ruby{先}{ま}づ
\ruby{謝}{わ}びて、
\ruby{心}{こゝろ}の
\ruby{之}{ゆ}くところを
\ruby{一}{ひ}ト
\ruby{目見}{め|み}れば、
\ruby{是}{こ}は
\ruby{如何}{い|か}に
\ruby{足袋無}{た|び|な}き
\ruby{其}{そ}の
\ruby{人}{ひと}の
\ruby{足}{あし}の
\ruby{小指}{こ|ゆび}は、はや
\ruby{湧}{わ}き
\ruby{出}{い}づる
\ruby{血潮}{ち|しほ}に
\ruby{塗}{ぬ}れて、
\ruby{負傷}{け|が}の
\ruby{樣子}{やう|す}もおぼろげながら、
\ruby{岩根杜鵑花}{いは|ね|つ|ゝ|じ}の
\ruby{花}{はな}の
\ruby{影}{かげ}の
\ruby{流水}{なが|れ}の
\ruby{底}{そこ}に
\ruby{動}{うご}くが
\ruby{如}{ごと}くに
\ruby{紅色流}{くれ|なゐ|なが}れて
\ruby{止}{とゞ}まらず、いまだ
\ruby{{\換字{古}}}{ふる}びぬ
\ruby{薩摩下駄}{さつ|ま|げ|た}の、
\ruby{一}{ひ}ト
\ruby{角}{すみ}は
\ruby{忽}{たちま}ち
\ruby{殷朱}{あ|け}となつたり。

あなやとばかり
\ruby{我}{われ}も
\ruby{驚}{おどろ}けば
\ruby{人}{ひと}も
\ruby{驚}{おどろ}きて、
\ruby{忙}{いそ}がはしく
\ruby{下駄}{げ|た}を
\ruby{{\換字{脱}}}{ぬ}ぎ
\ruby{捨}{す}てつ、
\ruby{男}{をとこ}は
\ruby{急}{きふ}に
\ruby{袂}{たもと}を
\ruby{掻探}{かい|さぐ}りしが、
\ruby{左方}{ひだ|り}にも
\ruby{右方}{み|ぎ}にも
\ruby{片紙}{へん|し}だに
\ruby{紙}{かみ}はあらずして、たゞ
\ruby{小}{ちひさ}き
\ruby{折本}{をり|ほん}のみの
\ruby{取出}{とり|いだ}されたる
\ruby{其間}{その|ま}に、
お
\ruby{龍}{りう}は
\ruby{既}{すで}に
\ruby{我}{わ}が
\ruby{小包}{こ|づゝみ}を
\ruby{傍}{かたへ}の
\ruby{座}{ざ}に
\ruby{置}{お}き、
\ruby{手早}{て|ばや}く
\ruby{帶}{おび}の
\ruby{間}{あひだ}より
\ruby{白紙}{はく|し}を
\ruby{取}{と}り
\ruby{出}{いだ}して、

『まあ
\ruby{何樣}{ど|う}して
\ruby{御謝罪}{お|わ|び}を
\ruby{致}{いた}したら
\ruby{宜}{よろ}しいのでしやう、
\ruby{飛}{と}んでも
\ruby{無}{な}い
\ruby{事}{こと}をいたしました。
どうかまあ
\ruby{貴下}{あな|た}、
\ruby{御腹立}{お|はら|だち}でしやうが
\ruby{何樣}{ど|う}か
\ruby{貴下}{あな|た}、
\ruby{御勘辯}{ご|かん|べん}なすつて
\ruby{下}{くだ}さいまし。
\ruby{定}{さだ}めし
\ruby{御痛}{お|いた}みでございましやう、あ〻
\ruby{濟}{す}みませんことをいたしました。
』

と
\ruby{面}{おもて}を
\ruby{赤}{あか}め
\ruby{淚}{なみだ}を
\ruby{含}{ふく}んで
\ruby{誠意}{まご|ゝろ}に
\ruby{謝罪}{わ|び}ながら、
\ruby{身}{み}を
\ruby{低}{ひく}く
\ruby{屈}{かゞ}めて
\ruby{血汚}{けが|れ}を
\ruby{拭}{ぬぐ}ひつゝ、
\ruby{塵埃}{ほこ|り}に
\ruby{穢}{よご}れたる
\ruby{足}{あし}の
\ruby{赭}{あか}く
\ruby{汚}{きたな}きを、
\ruby{繊々}{ほつ|そり}としたる
\ruby{指}{ゆび}の
\ruby{{\換字{雪}}}{ゆき}と
\ruby{白}{しろ}き
\ruby{手}{て}に
\ruby{執}{と}りて、
\ruby{早}{はや}くも
\ruby{拭}{ぬぐ}ひ
\ruby{捨}{す}つる
\ruby{紙}{かみ}の
\ruby{血}{ち}に
\ruby{染}{し}みて
\ruby{花鮮}{はな|あざ}やかなるを
\ruby{幾枚}{いく|まい}か
\ruby{散}{ち}らせば、
\ruby{男}{をとこ}は
お
\ruby{龍}{りう}の
\ruby{手}{て}を
\ruby{拂}{はら}ひのけ
\ruby{足}{あし}を
\ruby{縮}{ちゞ}めて、

『ナアニ
\ruby{構}{かま}ひません、これんばかりの
\ruby{事}{こと}、
\ruby{痛}{いた}くも
\ruby{何}{なん}ともありはしませんから、
\ruby{勘辯}{かん|べん}も
\ruby{何}{な}にもありや
\ruby{仕}{し}ません、たゞ
\ruby{潮時}{しほ|どき}の
\ruby{{\換字{所}}爲}{せ|ゐ}で
\ruby{血}{ち}がでるのでしやう。
\ruby{紙}{かみ}を
\ruby{少}{すこ}し
\ruby{頂戴}{いた|ゞ}きさへすりやあ
\ruby{宣}{よ}うございます。
』

と
\ruby{云}{い}ひし
\ruby{限}{き}り、ふたゝび
\ruby{手}{て}を
\ruby{觸}{ふ}れしめず、

『でも
\ruby{塵埃}{ご|み}でも
\ruby{入}{はい}りますと
\ruby{惡}{わる}うございますから。
』

と
\ruby{云}{い}ふをも
\ruby{更}{さら}に
\ruby{耳}{みゝ}に
\ruby{入}{い}れて、
\ruby{自}{みづか}ら
\ruby{一應}{いち|おう}
\ruby{淸潔}{せい|けつ}に
\ruby{拭}{ぬぐ}ひて、
\ruby{幾重}{いく|へ}にか
\ruby{疊}{たゝ}みたる
\ruby{紙}{かみ}に
\ruby{傷處}{き|ず}を
\ruby{包}{つゝ}めば、
お
\ruby{龍}{りう}は
\ruby{袂}{たもと}より
\ruby{絹}{きぬ}の
\ruby{白汗巾兒}{しろ|はん|け|ち}の
\ruby{淸}{きよ}げなるを
\ruby{出}{いだ}して、
\ruby{{\換字{前}}齒}{まへ|ば}に
\ruby{啣}{くは}ふるが
\ruby{早}{はや}きかピリヽと
\ruby{引}{ひ}き
\ruby{裂}{さ}き、
\ruby{男}{をとこ}の
\ruby{辭}{いな}まんとするを
\ruby{辭}{いな}む
\ruby{間}{ま}あらせず、
\ruby{體裁}{さ|ま}よく
\ruby{巧者}{かう|しや}にくる〳〵と
\ruby{{\換字{巻}}}{ま}きて
\ruby{引結}{ひき|むす}びけるが、
\ruby{裂}{さ}きたる
\ruby{時}{とき}に
\ruby{唇}{くち}にや
\ruby{觸}{ふ}れたりけん、その
\ruby{結}{むす}び
\ruby{餘}{あま}りの
\ruby{一端}{いつ|たん}には、
\ruby{血}{のり}ならぬ
\ruby{紅}{あか}きものゝ
\ruby{微}{かすか}に
\ruby{見}{み}えたり。

\ruby{車中}{しや|ちう}のすべての
\ruby{人々}{ひと|〴〵}の
\ruby{眼}{め}は、
\ruby{悉}{こと〴〵}く
\ruby{二人}{ふた|り}が
\ruby{上}{うへ}にのみ
\ruby{注}{そゝ}がれ
\ruby{居}{ゐ}るを、
\ruby{男}{をとこ}は
\ruby{上無}{うへ|な}く
\ruby{不樂}{わ|び}しくおぼえてや、

『
\ruby{紙捻}{こ|より}でも
\ruby{濟}{す}みましたものを
\ruby{御氣}{お|き}の
\ruby{毒}{どく}な!。
いろ〳〵
\ruby{御世話}{お|せ|わ}になつて
\ruby{却}{かへ}つて
\ruby{濟}{す}みませんでした。
』

と、
\ruby{云}{い}ふべきほどの
\ruby{挨拶}{あい|さつ}は
\ruby{眞四角}{まつ|し|かく}に
\ruby{云}{い}ひ
\ruby{仕舞}{し|ま}ひて、
\ruby{一寸}{ちよ|つと}こなたを
\ruby{見}{み}て
\ruby{會釋}{ゑし|やく}せしが、

『
\ruby{何樣}{ど|う}いたしまして、
\ruby{妾}{わたし}こそほんとに
\ruby{濟}{す}まない
\ruby{事}{こと}をいたしました。
\ruby{何卒御免}{なに|とぞ|ご|めん}なすつて
\ruby{下}{くだ}さいまし。
』

と、お
\ruby{龍}{りう}の
\ruby{云}{い}ひし
\ruby{詞}{ことば}は
\ruby{聞}{き}きしや
\ruby{聞}{き}かざりしや、
\ruby{愛想氣無}{あい|そ|げ|な}く
\ruby{後}{うしろ}を
\ruby{見}{み}せて
\ruby{車窓{\換字{近}}}{ま|ど|ちか}く
\ruby{居寄}{ゐ|よ}り、
\ruby{何見}{なに|み}るものあるべくもあらぬ
\ruby{窓外}{そ|と}の
\ruby{方}{かた}を
\ruby{見}{み}たる
\ruby{其}{そ}の
\ruby{横}{よこ}には、
\ruby{先刻}{さ|き}に
\ruby{懷中}{ふと|ころ}より
\ruby{出}{いだ}されたる
\ruby{小}{ちひさ}き
\ruby{折本}{をり|ほん}の
\ruby{置}{お}き
\ruby{棄}{す}てられたり。

\ruby{見}{み}る
\ruby{氣}{き}もなく
\ruby{何}{なん}の
\ruby{本}{ほん}かと
お
\ruby{龍}{りう}の
\ruby{見}{み}たる
\ruby{時}{とき}、
\ruby{其冊子}{その|は|ん}の
\ruby{最初}{さい|しよ}のところは
\ruby{丁度}{ちやう|ど}
\ruby{開}{あ}き
\ruby{居}{を}りて、
\ruby{配}{ふ}り
\ruby{假名}{が|な}のあるに
\ruby{誰}{たれ}にも
\ruby{解}{わか}りて、
\ruby{観世音菩薩}{くわん|ぜ|おん|ぼ|さつ}
\ruby{普門品}{ふ|もん|ぼん}とは
\ruby{明}{あき}らかに
\ruby{讀}{よ}めたり。


\Entry{其二十六}

% メモ 校正終了 2024-04-24 2024-06-01
\原本頁{139-1}%
\ruby{思}{おも}ひの
ほかの
\ruby{品}{もの}
なりしに、
%
お
\ruby{龍}{りう}は
\ruby{驚}{おどろ}き
\ruby{疑}{うたが}ひて、
%
\ruby{露}{つゆ}
\ruby{照}{て}る
\ruby{美}{うつく}しき
\ruby[<j||]{眼}{まなこ}を% 行末行頭の境界付近なので特例処置を施す
\ruby{{\換字{睜}}}{みは}り、
%
あらためて
\ruby{男}{をとこ}を
\ruby{一}{ひ}ト
\ruby{目}{め}
\ruby{見}{み}しが、
%
\ruby{男}{をとこ}は
それとも
\ruby{心付}{こ〻ろ|づ}かず% 原本通り「〻(二の字点、揺すり点)」
\ruby{{\換字{猶}}}{なほ}
\ruby{車外}{そ|と}を
\ruby{見}{み}
\ruby{居}{ゐ}たり。

\原本頁{139-4}%
\ruby{既}{すで}に
\ruby{其}{その}
\ruby{人}{ひと}の
\ruby{履物}{はき|もの}の
\ruby{汚}{けが}れを
\ruby{淸}{きよ}めたり、
%
\ruby{落}{お}ち
\ruby{散}{ち}つたる
\ruby{紙}{かみ}の
\ruby{眼}{め}に
\ruby{厭}{いと}は
しきをも
\ruby{一}{ひ}ト
\ruby{纏}{まとめ}にして
\ruby{投}{な}げ
\ruby{棄}{す}てたり、
%
\ruby{謝罪}{あや|ま}る
ほどは
あやまりて
\改行% 校正作業の簡略化のため
、
%
\原本頁{139-6}\改行%
\ruby{今}{いま}は
\ruby{何}{なに}
\ruby{爲}{す}べき
\ruby{事}{こと}も
\ruby{無}{な}きなり。
%
お
\ruby{龍}{りう}は
\ruby{彼}{か}の
\ruby{男}{をとこ}とは
\ruby{斜線}{すぢ|かひ}に、
%
\ruby{其}{そ}の
\ruby{反對}{はん|たい}の
\ruby{側}{がは}の
\ruby{車窓}{ま|ど}
\ruby{{\換字{近}}}{ちか}き
\ruby{席}{せき}を
\ruby{取}{と}りて、
%
はじめて
\ruby{身}{み}をも
\ruby{心}{こ〻ろ}% 原本通り「〻(二の字点、揺すり点)」
をもおちつけたり。

\原本頁{139-9}%
\ruby{普門品}{ふ|もん|ぼん}!、
%
あの
\ruby{普門品}{ふ|もん|ぼん}!、
%
\ruby{彼書}{あ|れ}は
たしか
\ruby[||j>]{觀}{くわん}% 「觀音」の読みは原本通り「くわん(の)ん」
\ruby[||j>]{音}{ のん}
\ruby[||j>]{樣}{ さま}を
\ruby{信}{しん}ずる
\ruby{人}{ひと}の
\ruby{讀}{よ}む
\ruby{御經}{お|きやう}!。
%
\ruby{五十六十}{ご|じふ|ろく|じふ}の
\ruby{爺}{ぢ〻}% 「ぢゞ」のはずだが、原本通り「〻(二の字点、揺すり点)」
\ruby{婆}{ば〻}% 「ばゞ」のはずだが、原本通り「〻(二の字点、揺すり点)」
ならばいざ
\ruby{知}{し}らず、
%
\ruby{{\換字{若}}}{わか}い
\ruby{盛}{さか}りの
\ruby{當世}{い|ま}の
\ruby{人}{ひと}の、
%
しかも
\ruby{{\換字{古}}風}{むか|し}を
\ruby{守}{まも}る
\ruby[||-|]{農}{ひやく}
\ruby[||-|]{夫}{しやう}
% \ruby{農夫}{ひやく|しやう}
\ruby[||>]{町}{ちやう}
\ruby[||j>]{人}{ にん}
% \ruby{町人}{ちやう|にん}
でゞもある% TODO 原本の「二の字点、揺すり点」に濁点のグリフが見つからないので「ゞ」
\ruby{事}{こと}か、
%
\ruby{新}{あたら}しきを
\ruby{{\換字{追}}}{お}うて
\ruby{學問}{がく|もん}に
\ruby{身}{み}を
\ruby{責}{せ}めれば、
%
まづ
\ruby[||j>]{神}{かみ}
\ruby[||j>]{佛}{ほとけ}とは
% \ruby{神佛}{かみ|ほとけ}とは
\ruby{緣}{{\換字{𛀁}}ん}の
\ruby{{\換字{遠}}}{とほ}さうな
\ruby{書生}{しよ|せい}
\原本頁{140-2}\改行%
\ruby{風}{ふう}の
\ruby{此樣}{こ|う}いふ
\ruby{人}{ひと}の
\ruby{懷中}{ふと|ころ}から、
%
\ruby{普門品}{ふ|もん|ぼん}とは
\ruby{似合}{に|あ}はしからぬ!。
%
\ruby{何}{ど}のやうな
\ruby{悲}{かな}しい
\ruby{願}{ねがひ}が
あつての
\ruby[||j>]{佛}{ほとけ}
\ruby[||j>]{頼}{ だの}みか
\ruby{知}{し}らねど、
%
あ〻% 原本通り「〻(二の字点、揺すり点)」
\ruby{想}{おも}ひ
\ruby{出}{だ}しても
\ruby{胸}{むね}が
\ruby{痛}{いた}む、
%
\ruby{妾}{わたし}も
\ruby{一昨年}{をと|〻|し}の% 原本通り「〻(二の字点、揺すり点)」
\ruby{丁度}{ちやう|ど}
\ruby{今頃}{いま|ごろ}、
%
\ruby{思}{おも}ふ
\ruby{人}{ひと}には
\ruby{{\換字{遠}}}{とほ}く
\ruby{離}{はな}れて
\改行% 校正作業の簡略化のため
、
%
\原本頁{140-5}\改行%
\ruby{{\換字{空}}}{そら}の
\ruby{色}{いろ}も
\ruby{風}{かぜ}の
\ruby{音}{おと}も
\ruby[||j>]{{\換字{情}}}{なさけ}
\ruby[||j>]{無}{ な}い、
\ruby{知}{し}らぬ
\ruby{他國}{た|こく}の
\ruby{駿府}{すん|ぷ}の
\ruby{秋}{あき}、
%
いくら
\ruby{手紙}{て|がみ}を
\ruby{出}{だ}して
\ruby{問}{とひ}
\ruby{訊}{たつ}ねしても、
%
\ruby{{\換字{返}}事}{へん|じ}
さへ
\ruby{來}{こ}ないのが
\ruby{氣}{き}になつて
\ruby{氣}{き}になつて、
%
よもやとは
\ruby{思}{おも}へども
\ruby[||j>]{心}{こ〻ろ}% 原本通り「〻(二の字点、揺すり点)」
\ruby[||j>]{變}{ がは}りか、
%
それとも
また
\ruby[<j||]{病}{やみ }
\ruby[<j>]{患}{わづらひ}% TODO CHECK
でも
\ruby{仕}{し}てゞは% TODO 原本の「二の字点、揺すり点」に濁点のグリフが見つからないので「ゞ」
\ruby{無}{な}いかと、
%
\ruby{恨}{うら}めしく
もあれば
\ruby[||j>]{心}{こ〻ろ}% 原本通り「〻(二の字点、揺すり点)」
\ruby[||j>]{細}{ ぼそ}く
もあり、
%
はては
\ruby{茫然}{ぼん|やり}と
\ruby{門口}{かど|ぐち}に
\ruby{立}{た}つて、
%
\ruby{何}{なに}が
\ruby{見}{み}える
でもない
\ruby[||j>]{東}{とう}
\ruby[||j>]{京}{きやう}の
% \ruby{東京}{とう|きやう}の
\ruby{方}{ほう}を、
%
\ruby{{\換字{空}}}{くう}に
\ruby{見}{み}
\ruby{詰}{つ}めては
ほろり〳〵と、
%
\ruby{馬鹿}{ば|か}らしい
ほど
\ruby{泣}{な}いて
\ruby{泣}{な}いた
\ruby{末}{すゑ}、
%
\ruby{思案}{し|あん}に
\ruby{餘}{あま}つた
ところから
\ruby[||j>]{願}{がわん}
\ruby[||j>]{掛}{ がけ}して、
% \ruby{願掛}{がわん|がけ}して、
%
% * 安東熊野神社 静岡市葵区安東1-6-4
% 駿河七観音・安倍七観音
%   霊山寺 静岡市清水区大内山597
%   徳願寺 静岡市駿河区向敷地689
%   増善寺 静岡市葵区慈悲尾(しいのお)302
%   建穂寺 静岡市葵区建穂2-12-6 山号を「瑞祥山」
% * 法明寺 静岡市葵区足久保奥組1043
%   鉄舟寺 静岡市清水区村松2188
%   平澤寺 静岡市駿河区平沢55
\ruby{安東}{あん|とう}の% 原本通りのルビ「(ど)でなく(と)」
\ruby{淸水}{きよ|みづ}の
\ruby[||j>]{觀}{くわん}% 「觀音」の読みは原本通り「くわん(の)ん」
\ruby[||j>]{音}{ のん}
\ruby[||j>]{樣}{ さま}には
\ruby{御經}{お|きやう}こそ
\ruby{誦}{あ}げなかつたが
\ruby{日}{ひ}
\ruby{參}{まゐ}りもすれば、
%
\ruby{足久保}{あし|く|ぼ}の
\ruby{楠木}{くす|のき}の
\ruby[||j>]{觀}{くわん}% 「觀音」の読みは原本通り「くわん(の)ん」
\ruby[||j>]{音}{ のん}
\ruby[||j>]{樣}{ さま}の
\ruby{御}{ご}
\ruby{利生}{り|しやう}
\原本頁{141-2}\改行%
の
\ruby{話}{はなし}を
\ruby{聞}{き}いては、
%
\ruby{二里}{に|り}からの
\ruby{田舎}{ゐな|か}
\ruby{{\換字{道}}}{みち}を
\ruby{歩}{ある}いた
\ruby{上}{うへ}に、
%
\ruby{草臥}{くた|びれ}
\ruby{{\換字{返}}}{かへ}り
ながら
%
\ruby{御}{お}
\ruby{百度}{ひやく|ど}まで
\ruby{踏}{ふ}んで、
%
\ruby{何卒}{どう|ぞ}
\ruby{手紙}{て|がみ}の
\ruby{{\換字{返}}事}{へん|じ}の
\ruby{參}{まゐ}りまして
\ruby{彼方}{あち|ら}の
\ruby{樣子}{やう|す}の
\ruby{{\換字{分}}}{わか}りまする
やう、
%
\ruby{{\換字{若}}}{も}し
\ruby{{\換字{又}}}{また}
\ruby{病氣}{びやう|き}
\ruby{災{\換字{難}}}{さい|なん}に
でも
\ruby{罹}{か〻}つて% 原本通り「〻(二の字点、揺すり点)」
\ruby{居}{を}りまするなら、
%
\ruby{御利益}{ご|り|やく}を
もつて
\ruby{助}{たす}かりまする
ようにと、
%
\ruby{自{\換字{分}}}{じ|ぶん}の
\ruby{身體}{から|だ}は
\ruby{一日}{いち|にち}
\ruby{一日}{いち|にち}
\ruby{{\換字{削}}}{けづ}るやうに
\ruby{癯}{や}せるのも
\ruby{餘{\換字{所}}}{よ|そ}にして、
%
\ruby{一心}{いつ|しん}に
なつて
\ruby{信心}{しん|〴〵}を
\原本頁{141-7}\改行%
\ruby{仕}{し}た
\ruby{苦}{くる}しい
\ruby{切}{せつ}ない
\ruby{經驗}{おぼ|{\換字{𛀁}}}も
あるが、
%
\ruby{忘}{わす}れても
\ruby{爲}{す}まい
ものは
\ruby{戀路}{こひ|ぢ}の
\ruby{{\換字{迷}}}{まよ}ひ、
%
\ruby{思}{おも}つて
\ruby{思}{おも}ひ
\ruby{止}{や}む
\ruby{日}{ひ}も
\ruby{無}{な}ければ、
%
\ruby{泣}{な}いて
\ruby{泣}{な}き
\ruby{足}{た}る
\ruby{夜}{よる}も
\ruby{無}{な}く
\改行% 校正作業の簡略化のため
、
%
\原本頁{141-9}\改行%
\ruby{生}{い}きては
\ruby{居}{ゐ}ても
\ruby{生}{い}きたくも
\ruby{無}{な}く、
%
\ruby{死}{し}なうとしても
\ruby{死}{し}にきれもせぬ
\ruby{彼}{あ}の
\ruby{厭}{いや}な〳〵
\ruby[||j>]{{\換字{情}}}{なさけ}
\ruby[||j>]{無}{ な}い
\ruby[||j>]{心}{こ〻ろ}% 原本通り「〻(二の字点、揺すり点)」
\ruby[||j>]{持}{ もち}!。% 「!」は国会図書館ので確認
%
\ruby{我}{わが}
\ruby{身}{み}の
\ruby{痛}{いた}かりし
\ruby{經驗}{おぼ|{\換字{𛀁}}}に
\ruby{人}{ひと}の
\ruby{痛}{いた}さも
\ruby{思}{おも}はる〻が、% 原本通り「〻(二の字点、揺すり点)」
%
あ〻% 原本通り「〻(二の字点、揺すり点)」
\ruby{{\換字{猶}}}{まだ}
\ruby{{\換字{若}}}{わか}い
\ruby{此}{こ}の
\ruby{人}{ひと}の
\ruby{信心}{しん|〴〵}の、
%
よしや
\ruby{頼}{たの}み
\ruby{無}{な}き
\ruby{老年}{とし|より}の
\ruby{親}{おや}の
\ruby{病氣}{びやう|き}の
\ruby{爲}{ため}
\ruby{故}{ゆゑ}でも
あれ、
%
また
\ruby{何}{ど}の
\ruby{樣}{やう}な
\ruby{辛}{つら}い
\ruby{悲}{かな}しい
\ruby{{\換字{遺}}}{や}る
\ruby{瀬}{せ}
\ruby{無}{な}い
\ruby{事}{こと}の
ためでもあれ、
%
たゞ% TODO 原本の「二の字点、揺すり点」に濁点のグリフが見つからないので「ゞ」
\ruby{戀}{こひ}
\ruby{故}{ゆゑ}の
\ruby{信心}{しん|〴〵}で
\ruby{無}{な}かれかし。
%
\ruby{今}{いま}
\ruby{妾}{わたし}が
\原本頁{142-3}\改行%
\ruby{仕}{し}たる
\ruby{{\換字{過}}失}{あや|まち}は、
%
\ruby{時}{とき}の
\ruby{拍子}{へう|し}の
\ruby{事}{こと}
なれば、
%
\ruby{誰}{たれ}も
\ruby{容赦}{ゆ|る}しては
\ruby{吳}{く}れさうな
\ruby{譯}{わけ}ながら、
%
あれほどの
\ruby{血}{ち}の
\ruby{出}{で}た
\ruby{負傷}{け|が}を
\ruby{仕}{し}て、
%
\ruby{露}{つゆ}
\ruby{腹立}{はら|だ}たしげな
\ruby{顏色}{かほ|つき}もせず、
%
また
\ruby{恨}{うら}めしき
\ruby{眼色}{め|つき}もせず、
%
\ruby{毫}{すこし}も
\ruby{變}{かは}つた
\ruby{樣子}{やう|す}は
\ruby{無}{な}くて、
%
\ruby{水}{みづ}の
\ruby{流}{なが}れた
やうに
さらりと
\ruby{濟}{す}ませて、
%
\ruby{後}{あと}には
\ruby{物}{もの}も
\ruby{殘}{のこ}さぬ
\ruby{風{\換字{情}}}{ふ|ぜい}の
\ruby{寛大}{おほ|やう}さ!。
%
\ruby{{\換字{終}}}{しまひ}には
\ruby{反對}{あべ|こべ}に
\ruby{禮}{れい}まで
\ruby{言}{い}ひたるに
\ruby{心}{こ〻ろ}の% 原本通り「〻(二の字点、揺すり点)」
\ruby{優}{やさ}しさは
\ruby{見}{み}えながら、
%
それから
\ruby{知}{し}らぬ
\ruby{顏}{かほ}
つくつて、
%
\ruby{彼方}{あち|ら}
\ruby{向}{む}いたる
\ruby{振舞}{ふる|まひ}の
\ruby{少}{すこ}し
\ruby{素氣}{す|げ}
\ruby{無}{な}きに、
%
\ruby{飼}{か}はれても
\ruby{人}{ひと}の
\ruby{氣}{き}は
\ruby{取}{と}らぬ
\ruby{鷹}{たか}の
\ruby{素振}{そ|ぶり}の、
%
\ruby{一寸}{ちよ|つと}
\ruby{憎}{にく}らしいほどな
\ruby{氣位}{き|ぐらゐ}も
あらはれて、
%
\ruby{女}{をんな}
さへ
\ruby{見}{み}れば
\ruby{{\換字{嫌}}}{いや}に
\ruby{笑}{わら}ひ
\ruby{掛}{か}ける
\改行% 校正作業の簡略化のため
、
%
\原本頁{142-11}\改行%
\ruby{世}{よ}に
\ruby{有}{あ}りふれた
\ruby{{\換字{若}}}{わか}い
\ruby{人}{ひと}など〻は、% 原本通り「〻(二の字点、揺すり点)」
%
\ruby{其}{そ}の
\ruby{行方}{ゆき|がた}も
\ruby{全}{まる}で
\ruby{異}{かは}れど、
%
されば
といつて
ぎしつきも
せず、
%
\ruby{氣立}{き|だて}も
\ruby[||j>]{心}{こ〻ろ}% 原本通り「〻(二の字点、揺すり点)」
\ruby[||j>]{持}{ もち}も
\ruby{何}{なに}と
\ruby{無}{な}く
\ruby{{\換字{違}}}{ちが}つて、
%
\ruby{衣服}{な|り}
\ruby{容姿}{すが|た}は
\ruby{此}{これ}
といふことも
\ruby{無}{な}き
\ruby{書生}{しよ|せい}ながら、
%
おのづと
\ruby{普{\換字{通}}}{な|み}には
\ruby{思}{おも}へぬ
ところ
ある
\ruby{人}{ひと}!。
%
\ruby{斯樣}{か|う}いふ
\ruby{調子}{てう|し}あひの
\ruby{人}{ひと}なんぞが、
%
\ruby{{\換字{若}}}{も}し
\ruby{萬一}{ひよ|つと}
\ruby{十年}{じふ|ねん}
\ruby{二十年}{に|じふ|ねん}の
\ruby{後}{のち}になつて、
%
\ruby{立派}{りつ|ぱ}な
\ruby{傑}{すぐ}れた
\ruby{人}{ひと}
なんぞに
なるのでは
あるまいか?。
%
あ〻% 原本通り「〻(二の字点、揺すり点)」
\ruby[||j>]{修}{しゆ}
\ruby[||j>]{行}{ぎやう}
\ruby[||j>]{盛}{ ざか}り
\ruby{出世盛}{しゆつ|せ|ざか}りの
\ruby{此}{こ}の
\ruby{{\換字{若}}}{わか}い
\ruby{人}{ひと}!、
%
それに
\ruby{付}{つ}けても
\ruby{彼}{あ}の
\ruby{普門品}{ふ|もん|ぼん}!。
%
\ruby{屹度}{きつ|と}
\ruby{果敢}{は|か}
\ruby{無}{な}い
\ruby{戀}{こひ}なぞの、
%
\ruby{其樣}{そ|ん}な
\ruby{事}{こと}の
\原本頁{143-7}\改行%
ためでは
あるまい
なれど、
%
どうか
\ruby{戀}{こひ}ゆえの
\ruby{信心}{しん|〴〵}で
\ruby{無}{な}かれかし!
\改行% 校正作業の簡略化のため
。
%
\原本頁{143-8}\改行%
と
\ruby{我身}{わが|み}の
\ruby{往時}{むか|し}に
つまされて、
%
じつと
\ruby{其}{その}
\ruby{册子}{ほ|ん}に
\ruby{{\換字{留}}}{とゞ}めし% TODO 原本の「二の字点、揺すり点」に濁点のグリフが見つからないので「ゞ」
\ruby{眼}{まなこ}を、
%
\ruby{今}{いま}しも
\ruby{其}{その}
\ruby{人}{ひと}の
\ruby{後姿}{すが|た}に
\ruby{移}{うつ}して
\ruby{横顏}{よこ|がほ}を
そつと
\ruby{見}{み}やる
\ruby{折}{をり}しも、
%
ふつと
\ruby{男}{をとこ}は
\ruby{此方}{こな|た}を
\ruby{見}{み}
\ruby{{\換字{返}}}{かへ}し、
%
\ruby{圖}{はか}らず
\ruby{眼}{め}と
\ruby{眼}{め}と
\ruby{相}{あひ}
\ruby{射}{い}しが、
%
はやくも
お
\ruby{龍}{りう}は
\ruby{男}{をとこ}の
\ruby{睫毛}{まつ|げ}に
\ruby{怪}{あや}しき% TODO CHECK 怪
\ruby{露}{つゆ}の
\ruby{珠}{たま}
あるを
\ruby{見}{み}たり。

\Entry{其二十七}

\ruby{人}{ひと}おの〳〵
\ruby{身}{み}あり、
\ruby{身}{み}の
\ruby{居}{を}るところあり。
\ruby{{\換字{又}}}{また}おの〳〵
\ruby{心}{こゝろ}あり、
\ruby{心}{こゝろ}の
\ruby{思}{おも}ふところあり。
されば
\ruby{相}{あひ}
\ruby{知}{し}らぬ
お
\ruby{龍}{りう}と
\ruby{男}{をとこ}との、
\ruby{男}{をとこ}は
お
\ruby{龍}{りう}の
\ruby{我}{わ}が
\ruby{爲}{ため}に
\ruby{何}{なに}を
\ruby{思}{おも}へるぞとも
\ruby{知}{し}らねば、
お
\ruby{龍}{りう}はまた
\ruby{男}{をとこ}の
\ruby{我}{われ}ゆゑに
\ruby{何}{なに}を
\ruby{悲}{かな}しめるぞとも
\ruby{悟}{さと}らむやう
\ruby{無}{な}きなり。

\ruby{自}{みづか}ら
\ruby{感}{かん}ずる
\ruby{身心}{しん|〳〵}の
\ruby{疲}{つか}れを、せめては
\ruby{滊車}{き|しや}の
\ruby{内}{うち}に
\ruby{休}{やす}めんと、
\ruby{少時}{しば|し}を
\ruby{待合}{まち|あ}はせて
\ruby{此}{これ}に
\ruby{乘}{の}りたるに、
\ruby{何}{なに}となく
\ruby{氣}{き}の
\ruby{弛}{ゆる}みてうつかりとしたる
\ruby{時}{とき}、
\ruby{忽}{たちま}ち
\ruby{足}{あし}を
\ruby{踏}{ふ}まれて
\ruby{驚}{おどろ}きしが、
\ruby{眞心}{ま|ごゝろ}を
\ruby{表}{あらは}して
\ruby{謝罪}{わ|び}らるゝに
\ruby{怒}{いか}らんやうは
\ruby{無}{な}く、かゝる
\ruby{事}{こと}は
\ruby{有}{あ}り
\ruby{{\換字{勝}}}{がち}の
\ruby{{\換字{過}}失}{あや|まり}にて
\ruby{珍}{めづ}らしくもあらず、
\ruby{且}{かつ}は
\ruby{{\換字{又}}}{また}、
\ruby{云}{い}はゞ
\ruby{我}{われ}にも
\ruby{不注意}{ふ|ちう|い}の% 原本通り「ゆ」無しで「ちうい」
\ruby{咎}{とが}の
\ruby{無}{な}きにはあらぬをやと
\ruby{輕}{かる}く
\ruby{思}{おも}ひ
\ruby{棄}{す}てつ、
\ruby{却}{かへ}つて
\ruby{他}{ひと}の
\ruby{我}{わ}がためにまめ〳〵しく
\ruby{傷}{きず}を
\ruby{裹}{つゝ}み
\ruby{汚}{けがれ}を
\ruby{拭}{ぬぐ}ひて
\ruby{吳}{く}るゝ
\ruby{氣}{き}の
\ruby{毒}{どく}さに
\ruby{堪}{た}へかねて、よきほどに
\ruby{挨拶}{あい|さつ}して
\ruby{身}{み}を
\ruby{{\換字{退}}}{ひ}きたる
\ruby{男}{をとこ}は、
\ruby{何}{なに}を
\ruby{見}{み}るにもあらず
\ruby{窓外}{さう|ぐわい}を
\ruby{見}{み}ながら、
\ruby{出血}{しゆつ|けつ}を
\ruby{疾}{と}く
\ruby{止}{と}まらしめんがために
\ruby{膝}{ひざ}に
\ruby{載}{の}せたる
\ruby{片足}{かた|あし}の、
\ruby{{\換字{猶}}}{なほ}
\ruby[||h>]{聊}{いさゝ}か
\ruby{疼痛}{いた|み}をば
\ruby{覺}{おぼ}ゆるにつけて、
\ruby{嗚呼}{あ|ゝ}
\ruby{何}{なん}ぞ
\ruby{人}{ひと}の
\ruby{世}{よ}のことの
\ruby{如是}{か|く}
\ruby{愚}{おろか}しきや!。
たま〳〵
\ruby{我}{われ}に
\ruby{{\換字{過}}失}{あや|まち}したる
\ruby{此}{こ}の
\ruby{{\換字{若}}}{わか}き
\ruby{{\換字{婦}}人}{ふ|じん}は、
\ruby{我}{わ}が
\ruby{思}{おも}ふ
\ruby{人}{ひと}の
\ruby{如}{ごと}くなる
\ruby{端嚴}{たん|ごん}の
\ruby{相}{さう}こそ
\ruby{無}{な}けれ、
\ruby{婀娜}{あ|だ}たる
\ruby{姿}{すがた}、
\ruby{野}{の}の
\ruby{花}{はな}のおのづから
\ruby{人}{ひと}の
\ruby{意}{こゝろ}を
\ruby{惹}{ひ}く
\ruby{色香}{いろ|か}あつて、
\ruby{一車}{いつ|しや}の
\ruby{客}{きやく}
\ruby{皆}{みな}
\ruby{眼}{め}をそばだてゝ
\ruby{見}{み}たるほどなれば、
\ruby{彼}{か}の
\ruby{人}{ひと}の
\ruby{我}{われ}に
\ruby{思}{おも}はるゝが
\ruby{如}{ごと}くに
\ruby{此}{こ}の
\ruby{女}{ひと}もまた
\ruby{或}{あるひ}は
\ruby{他}{ひと}に
\ruby{思}{おも}はるゝ
\ruby{事}{こと}の
\ruby{無}{な}きには
\ruby{限}{かぎ}らじ。
\ruby{我}{わ}が
\ruby{身}{み}の
\ruby{經驗}{おぼ|\換字{𛀁}}に
\ruby{我}{われ}よくぞ
\ruby{知}{し}る、
\ruby{人}{ひと}を
\ruby{思}{おも}ふものゝ
\ruby{苦}{くる}しさは、
\ruby{魂魄}{たま|しひ}を
\ruby{絞木}{しめ|ぎ}にかけられて
\ruby{斷}{た}えず
\ruby{壓}{お}し
\ruby{搾}{しぼ}らるゝに
\ruby{異}{こと}ならず、
\ruby{何}{なん}につけ
\ruby{彼}{か}につけ、
\ruby{夢}{ゆめ}につけ
\ruby{現}{うつゝ}につけ、
\ruby{愁}{うれ}ひ
\ruby{易}{やす}く
\ruby{悲}{かなし}み
\ruby{易}{やす}くなりたる
\ruby{心}{こゝろ}の、
\ruby{事}{こと}あるごとに
\ruby{責}{せ}め
\ruby{搾}{しぼ}らるれば、
\ruby{誰}{た}が
\ruby{縫}{ぬ}ひてか
\ruby{痊}{いや}すべき
\ruby{胸}{むね}の
\ruby{深創}{ふか|で}より、
\ruby{渾々}{こん|〳〵}として
\ruby{流}{なが}るゝ
\ruby{血潮}{ち|しほ}の、
\ruby{火}{ひ}とばかり
\ruby{熱}{あつ}きも
\ruby{{\換字{空}}}{むな}しく
\ruby{冷}{ひ}えて、いたづらに
\ruby{地}{ち}に
\ruby{入}{い}つて
\ruby{{\換字{情}}無}{なさけ|な}く
\ruby{廢}{すた}ることの
\ruby{如何}{い|か}ばかりぞや。
\ruby{眼}{め}に
\ruby{見}{み}\換字{𛀁}えぬ
\ruby{其血}{そ|れ}こそは
\ruby{{\換字{尊}}}{たつと}くも
\ruby{{\換字{尊}}}{たつと}き
\ruby{人}{ひと}の
\ruby{眞誠}{まこ|と}の
\ruby{生命}{いの|ち}にして、
\ruby{眼}{め}に
\ruby{見}{み}ゆる
\ruby{此血}{こ|れ}は
\ruby{言}{い}ふにも
\ruby{足}{た}らぬたゞ
\ruby{鹹}{しほはゆ}き
\ruby{紅}{あか}き
\ruby{水}{みづ}なるを、
\ruby{僅}{わづか}に
\ruby{一指}{いつ|し}の
\ruby{端}{はし}を
\ruby{傷}{きず}つけ
\ruby{數滴}{すう|てき}の
\ruby{臙脂}{べ|に}を
\ruby{散}{ち}らせば、
\ruby{性{\換字{情}}}{こゝ|ろ}の
\ruby{優}{やさ}しさ
\ruby{見}{み}ゆる
\ruby{此}{こ}の
\ruby{{\換字{婦}}人}{ふ|じん}は、
\ruby{僞}{いつは}りならず
\ruby{我}{われ}をいたはしがりて、
\ruby{心}{こゝろ}を
\ruby{盡}{つく}し
\ruby{手}{て}を
\ruby{盡}{つく}し
\ruby{慰}{なぐさ}め
\ruby{吳}{く}れしが、
\ruby{{\換字{若}}}{も}し
\ruby{此}{こ}の
\ruby{女}{ひと}を
\ruby{思}{おも}ふ
\ruby{男}{をとこ}ありて、
\ruby{彼}{か}の
\ruby{人}{ひと}を
\ruby{思}{おも}ふ
\ruby{我}{わ}が
\ruby{如}{ごと}くに、
\ruby{果敢無}{は|か|な}き
\ruby{戀}{こひ}の
\ruby{深}{ふか}みに
\ruby{惱}{なや}み、
\ruby{日}{ひ}と
\ruby{無}{な}く
\ruby{夜}{よ}と
\ruby{無}{な}く、
\ruby{折}{をり}にふれ
\ruby{事}{こと}につけ、
\ruby{泉}{いづみ}の
\ruby{如}{ごと}くに
\ruby{止}{と}まらぬ
\ruby{血}{ち}を
\ruby{心窩}{む|ね}の
\ruby{奧底}{おく|そこ}より
\ruby{流}{なが}し
\ruby{溢}{あふ}らさば、それを
\ruby{此}{こ}の
\ruby{女}{ひと}は
\ruby{何}{なに}とか
\ruby{見}{み}るべき?。
\ruby{我}{わ}が
\ruby{足}{あし}の
\ruby{指}{ゆび}の
\ruby{一}{ひ}ト
\ruby{{\換字{節}}}{ふし}
\ruby{二}{ふ}タ
\ruby{{\換字{節}}}{ふし}、
\ruby{此}{こ}の
\ruby{紅}{あか}き
\ruby{水}{みづ}の
\ruby{幾掬}{いく|むすび}は、よしや
\ruby{{\換字{過}}失}{あや|まち}のために
\ruby{亡失}{うし|な}はれたりとて、
\ruby{我}{われ}
\ruby{斯}{かく}ばかり
\ruby{恤}{いたは}られでも
\ruby{可}{よし}、たゞ
\ruby{思}{おも}ふ
\ruby{人}{ひと}に
\ruby{思}{おも}はれぬ
\ruby{辛}{つら}さは
\ruby{身}{み}に
\ruby{徹}{し}みて
\ruby{悲}{かな}しくおぼゆれば、あはれ
\ruby{此}{こ}の
\ruby{優}{やさ}しげなる
\ruby{{\換字{若}}}{わか}き
\ruby{人}{ひと}の、
\ruby{{\換字{若}}}{も}し
\ruby{人}{ひと}に
\ruby{思}{おも}はれなば
\ruby{人}{ひと}を
\ruby{思}{おも}へかし、
\ruby{思}{おも}ふ
\ruby{人}{ひと}あらば
\ruby{其人}{その|ひと}を
\ruby{思}{おも}ひて
\ruby{{\換字{遣}}}{や}れよかし、
\ruby{戀}{こひ}の
\ruby{誠}{まこと}に
\ruby{責}{せ}められて、
\ruby{壽命}{いの|ち}を
\ruby{溶}{と}いて
\ruby{涙}{なみだ}と
\ruby{流}{なが}し
\ruby{棄}{す}て、
\ruby{男兒}{をと|こ}の
\ruby{智慧}{ち|ゑ}をも
\ruby{保}{たも}ちかねて、
\ruby{愚}{ぐ}に
\ruby{甘}{あま}んずるに
\ruby{至}{いた}れる
\ruby{此}{こ}の
\ruby{我}{わ}が
\ruby{如}{ごと}きものにも
\ruby{會}{あ}はゞ、
\ruby{假令其}{たと|ひ|そ}の
\ruby{人}{ひと}を
\ruby{蟲}{むし}の
\ruby{{\換字{嫌}}}{きら}はゞとて、せめては
\ruby{可憐}{あは|れ}とも
\ruby{思}{おも}ひて
\ruby{{\換字{遣}}}{や}れかし。
\ruby{我}{わ}が
\ruby{身}{み}につまされてつく〴〵と
\ruby{思}{おも}ふ、あゝ
\ruby{人}{ひと}に
\ruby{思}{おも}はれなば
\ruby{人}{ひと}を
\ruby{思}{おも}へかし。
こればかりの
\ruby{傷}{きず}にだに
\ruby{痛}{いた}はしとおもふが、
\ruby{女性}{をん|な}の
\ruby{欺}{あざむ}かぬ
\ruby{{\換字{情}}}{なさけ}ならば、
\ruby{縫}{ぬ}ふべき
\ruby{針}{はり}も
\ruby{糸}{いと}も
\ruby{無}{な}き
\ruby{悲}{かな}しき
\ruby{創口}{きず|ぐち}より
\ruby{流}{なが}れ
\ruby{流}{なが}るゝ
\ruby{火}{ひ}と
\ruby{熱}{あつ}き
\ruby{血}{ち}の、
\ruby{花}{はな}と
\ruby{鮮}{あざ}やかなるを
\ruby{見}{み}たる
\ruby{時}{とき}は、
\ruby{必}{かな}らずあはれと
\ruby{思}{おも}ひて
\ruby{{\換字{遣}}}{や}れかし。
されど
\ruby{測}{はか}り
\ruby{難}{がた}き
\ruby{世}{よ}の
\ruby{{\換字{習}}}{ならひ}、
\ruby{此}{こ}の
\ruby{女}{ひと}もまた
\ruby{我}{わ}が
\ruby{思}{おも}ふ
\ruby{人}{ひと}の
\ruby{如}{ごと}く、
\ruby{或}{あるひ}はおのれを
\ruby{思}{おも}ふ
\ruby{男}{をとこ}の、
\ruby{心血}{しん|けつ}を
\ruby{盡}{つく}して
\ruby{悲}{かなし}み
\ruby{悶}{もだ}ゆるをも、あだに
\ruby{天飛}{そら|と}ぶ
\ruby{雲}{くも}と
\ruby{見{\換字{過}}}{み|すご}して、あはれみてもやらず
\ruby{悲}{かなし}みてもやらず、
\ruby{其}{そ}の
\ruby{{\換字{情}}無}{つれ|な}きを
\ruby{喞}{かこ}たれやする?。
\ruby{僅}{わづか}なる
\ruby{斯}{か}ばかりの
\ruby{傷}{きず}の
\ruby{如}{ごと}きには、
\ruby{心}{こゝろ}をつかひ
\ruby{言葉}{こと|ば}を
\ruby{費}{つひや}すにも
\ruby{當}{あた}らぬながら、
\ruby{此}{これ}を
\ruby{大事}{だい|じ}のやうに
\ruby{思}{おも}ふも
\ruby{愚}{おろか}なる
\ruby{世}{よ}の
\ruby{態}{すがた}かな。
\ruby{我}{われ}は
\ruby{今}{いま}
\ruby{恐}{おそ}ろしき
\ruby{傷}{きず}を
\ruby{抱}{いだ}きて、
\ruby{{\換字{絕}}間無}{た\換字{𛀁}|ま|な}く
\ruby{泉}{いづみ}なす
\ruby{血}{ち}を
\ruby{流}{なが}しながら、あはれとも
\ruby{思}{おも}はれぬ
\ruby{悲}{かな}しき
\ruby{身}{み}なるをや。
こればかりの
\ruby{血}{ち}の
\ruby{嗚呼}{あ|ゝ}
\ruby{何}{なに}かあらん!。
それにつけても
\ruby{此}{こ}の
\ruby{{\換字{若}}}{わか}き
\ruby{女}{ひと}の、
\ruby{願}{ねが}はくは
\ruby{人}{ひと}に
\ruby{思}{おも}はれなば
\ruby{人}{ひと}を
\ruby{思}{おも}へかし、と
\ruby{此}{これ}は
\ruby{我}{わ}が
\ruby{身}{み}の
\ruby{現在}{い|ま}につきて
\ruby{思}{おも}ひ
\ruby{入}{い}りながら、
\ruby{偶然}{ふ|と}
\ruby{後方}{うし|ろ}をば
\ruby{向}{む}きたるなりしが、
\ruby{圖}{はか}らず
\ruby{眼}{め}と
\ruby{眼}{め}と
\ruby{相}{あひ}
\ruby{射}{い}たる
\ruby{時}{とき}、
\ruby{男}{をとこ}もまた
お
\ruby{龍}{りう}が
\ruby{何}{なに}を
\ruby{思}{おも}ひてか
\ruby{涙}{なみだ}に
\ruby{其眼}{その|め}の
\ruby{潤}{うる}めるを
\ruby{觀}{み}たり。

\ruby{相}{あひ}
\ruby{會}{あ}つたる
\ruby{眼}{め}は
\ruby{忽}{たちま}ち
\ruby{離}{はな}れぬ。
\ruby{男}{をとこ}はまた
\ruby{{\換字{前}}}{まへ}の
\ruby{如}{ごと}くに
\ruby{窓外}{そう|ぐわい}を
\ruby{眺}{なが}めたり。
\ruby{女}{をんな}は
\ruby{男}{をとこ}の
\ruby{如何}{い|か}なる
\ruby{人}{ひと}なるを
\ruby{知}{し}らず、
\ruby{男}{をとこ}もまた
\ruby{女}{をんな}の
\ruby{如何}{い|か}なるものなるを
\ruby{知}{し}らず、
\ruby{同}{おな}じ
\ruby{車}{くるま}に
\ruby{乘}{の}りて
\ruby{同}{おな}じ
\ruby{{\換字{道}}}{みち}を、
\ruby{同}{おな}じところへ
\ruby{行}{ゆ}く
\ruby{身}{み}ながらも、
\ruby{心}{こゝろ}はそれ〳〵のおもむく
\ruby{方}{かた}に
\ruby{馳}{は}するも
\ruby{{\換字{浮}}世}{うき|よ}の
\ruby{態}{すがた}なりや。

やがて
\ruby{滊車}{き|しや}は
\ruby{白鬚}{しら|ひげ}の
\ruby{停車塲}{てい|しや|ぢやう}を% 原文通り「塲」
\ruby{{\換字{過}}}{す}ぎて、はや
\ruby{鐘}{かね}が
\ruby{淵}{ふち}の
\ruby{停車塲}{てい|しや|ぢやう}% 原文通り「塲」
\ruby{{\換字{近}}}{ちか}くなりぬ。

お
\ruby{龍}{りう}は
\ruby{徐}{しづか}に
\ruby{立上}{たち|あが}りて、

『どうも
\ruby{飛}{と}んだ
\ruby{失禮}{しつ|れい}を
\ruby{致}{いた}しました。
もう
\ruby{妾}{わたくし}はこの
\ruby{先}{さき}で
\ruby{下車}{お|り}ますのでございますが、
\ruby{御痛}{お|いた}みは
\ruby{如何}{い|かゞ}でございます?、
\ruby{些}{ちつと}は
お
\ruby{宜}{よろ}しうございますか、まことに
\ruby{濟}{す}みませんことを
\ruby{致}{いた}しました。
では
\ruby{{\換字{勝}}手}{かつ|て}ではございますがこれで
\ruby{失禮}{しつ|れい}いたします。
』

と、
\ruby{男}{をとこ}も
\ruby{其處}{そ|こ}で
\ruby{下}{お}りるとは
\ruby{知}{し}らで
\ruby{物堅}{もの|がた}く
\ruby{挨拶}{あい|さつ}すれば、
\ruby{男}{をとこ}は
\ruby{口數}{くち|かず}
\ruby{少}{すくな}く、

『いやもう
\ruby{何}{なん}ともございません、
\ruby{何樣}{ど|う}か
\ruby{御構}{お|かま}ひ
\ruby{無}{な}く。
』

と
\ruby{云}{い}ひたるのみ。

\ruby{滊車}{き|しや}は
\ruby{鐘}{かね}が
\ruby{淵}{ふち}に
\ruby{着}{つ}きて
\ruby{男}{をとこ}
\ruby{先}{ま}づ
\ruby{降}{お}り、それより
\ruby{人々}{ひと|〴〵}につゞきて
\ruby{女}{をんな}も
\ruby{降}{お}りぬ。
\ruby{女}{をんな}の
\ruby{停車塲}{てい|しや|ぢやう}の% 原文通り「塲」
\ruby{構外}{こう|ぐわい}に
\ruby{出}{い}でし
\ruby{時}{とき}は、
\ruby{男}{をとこ}の
\ruby{姿}{すがた}ははや
\ruby{見}{み}えざりき。

\Entry{其二十八}

\ruby{家並立續}{やな|み|たち|つゞ}ける
\ruby{都會}{みや|こ}に
\ruby{育}{そだ}ちて、
\ruby{賑}{にぎ}やかなる
\ruby{{\GWI{u9053-k}}路}{み|ち}をのみ
\ruby{歩}{ある}きつけたるものは、
\ruby{右}{みぎ}も
\ruby{左}{ひだり}も
\ruby{田甫}{たん|ぼ}にして、
\ruby{{\GWI{u9060-k}}見}{とほ|み}に
\ruby{榎}{\GWI{u1b001}のき}やら
\ruby{松}{まつ}やらの
\ruby{木立}{こ|だち}、その
\ruby{蔭}{かげ}に
\ruby{箱庭}{はこ|には}にありさうな
\ruby{藁葺}{わら|ぶき}の
\ruby{家}{いへ}の
\ruby{四}{よ}ッ
\ruby{五}{いつ}ッ
\ruby{並}{なら}ぶといふやうなる
\ruby{田舎}{ゐな|か}へ
\ruby{踏出}{ふみ|だ}しては、
\ruby{十字路}{よつ|{\ninojiten}|ぢ}に
\ruby{問}{と}ふべき
\ruby{店}{みせ}なきを
\ruby{恨}{うら}み、
\ruby{三{\換字{叉}}路}{み|つ|また}に
\ruby{尋}{たづ}ぬべき
\ruby{人}{ひと}あらぬを
\ruby{悲}{かなし}み、はては
\ruby{間違}{まち|が}へずとも
\ruby{濟}{す}むべき
\ruby{筈}{はづ}の
\ruby{路}{みち}を
\ruby{兎角}{と|かく}に
\ruby{間違}{まち|が}へて、あらぬところに
\ruby{{\GWI{u8ff7-k}}}{まよ}ひ
\ruby{{\GWI{u8fbc-k}}}{こ}むが
\ruby{常}{つね}なり。
\ruby{鐘}{かね}が
\ruby{淵}{ふち}の
\ruby{停車{\換字{場}}}{てい|しや|ぢやう}より
\ruby{四}{よ}ツ
\ruby{木}{ぎ}へは、
\ruby{何}{なん}の
\ruby{譯}{わけ}も
\ruby{無}{な}く
\ruby{知}{し}れ
\ruby{易}{やす}き
\ruby{路}{みち}なるを、お
\ruby{龍}{りう}は
\ruby{如何}{い|か}にしてか
\ruby{{\GWI{u8aa4-k}}}{あやま}りて、
\ruby{狐}{きつね}に
\ruby{誑}{ばか}さる{\ninojiten}と
\ruby{云}{い}ひし
\ruby{今朝}{け|さ}の
\ruby{戯言}{じやう|だん}も
\ruby{思}{おも}ひ
\ruby{出}{だ}されてをかしき
\ruby{無{\GWI{u76ca-k}}路}{む|だ|みち}を
\ruby{歩}{ある}きし
\ruby{末}{すゑ}、やうやくにして
\ruby{目}{め}ざす
\ruby{其}{そ}の
\ruby{村}{むら}へ
\ruby{着}{つ}きたり。

ばつちらけ
\ruby{髪}{がみ}を
\ruby{手拭}{てぬ|ぐひ}の
\ruby{鉢{\換字{巻}}}{はち|まき}に
\ruby{壓}{おさ}へて、ねん〳〵ねん〳〵と
\ruby{兒守}{こ|もり}する
\ruby{村}{むら}の
\ruby{娘}{こ}の
\ruby{十三四}{じう|さん|し}なるに、

『もし、
\ruby{山路}{やま|ぢ}さんといふのは、』

と
\ruby{尋}{たづ}ぬれば、

『
\ruby{伴}{つ}れてつて
\ruby{{\GWI{u9063-k}}}{や}るべい。
』

と
\ruby{前}{さき}に
\ruby{立}{た}つて
\ruby{歩}{ある}きて、

『
\ruby{此處}{こ|{\ninojiten}}だよ。
』

と
\ruby{{\換字{教}}}{おし}へて
\ruby{{\換字{呉}}}{く}れたるは、
\ruby{門}{もん}のがつしりと
\ruby{嚴}{いか}めしくして、
\ruby{厚}{あつ}き
\ruby{茅葺}{かや|ぶき}の
\ruby{屋根}{や|ね}も
\ruby{高}{たか}き、
\ruby{物持}{もの|もち}らしき
\ruby{立派}{りつ|ぱ}の
\ruby{家}{いへ}なり。
これほどの
\ruby{家}{いへ}とは
\ruby{聞}{き}かざりしがと、
\ruby{少}{すこ}し
\ruby{訝}{いぶか}りながら
\ruby{音}{おと}なへば、
\ruby{丁度端{\GWI{u8fd1-k}}}{ちや|うど|はし|ぢか}に
\ruby{居}{ゐ}たる、
\ruby{般若顏}{はん|にや|がほ}の
\ruby{{\換字{丈}}高}{たけ|たか}き
\ruby{女}{をんな}の、
\ruby{衣服}{な|り}は
\ruby{此家}{こ|〻}の
\ruby[g]{主人}{あるじ}の
\ruby{妻}{つま}なるべく
\ruby{見}{み}\GWI{u1b001}て、
\ruby{可笑}{お|か}しきほど
\ruby{大}{おほき}なる
\ruby{丸髷}{まる|まげ}に
\ruby{結}{ゆ}びたるが、
\ruby{人}{ひと}を
\ruby{媢嫉}{そ|ね}むやうなる
\ruby{眼}{め}つきして、しばらくは
\ruby{頭}{かしら}の
\ruby{上}{うへ}より
\ruby{足}{あし}の
\ruby{先}{さき}までじろ〳〵と
\ruby{見}{み}たる
\ruby{揚句}{あげ|く}、

『それは
\ruby{{\GWI{u96a0-ue0102}}居{\換字{所}}}{いん|きよ|じよ}の
\ruby{方}{はう}でございましやう、こちらには
\ruby[g]{水野}{みづの}なんていふ
\ruby{人}{ひと}は
\ruby{居}{を}りません。
\ruby[g]{其家}{そつち}へ
\ruby{行}{い}つてお
\ruby{聞}{き}きなさいまし。
』

と、
\ruby{可厭}{い|や}に
\ruby{慳貪}{けん|どん}に
\ruby{云}{い}ひ
\ruby{棄}{す}て〻、
\ruby{障子}{しや|うじ}びつしやり
\ruby{奧}{おく}に
\ruby{入}{い}りたり、
\ruby[g]{此家}{こ〻}と
\ruby{{\GWI{u96a0-ue0102}}居{\換字{所}}}{いん|きよ|じよ}との
\ruby{間}{あひだ}に
\ruby{何}{ど}のやうな
\ruby{譯}{わけ}のあるかは
\ruby{知}{し}らず、また
\ruby{何程大}{どれ|ほど|たい}した
\ruby{大々盡}{だい|〳〵|じん}の
\ruby{奥様}{おく|さま}なれば
\ruby{左様}{さ|う}は
\ruby{勿體}{もつ|たい}ぶるか
\ruby{知}{し}らねど、
\ruby{惡}{わる}く
\ruby{人}{ひと}を
\ruby{見下}{み|くだ}したやうな
\ruby{沒義道}{も|ぎ|だう}の
\ruby{忌々}{いま|〳〵}しい
\ruby{大顏}{おほ|づら}な
\ruby{田舎婦}{ゐな|か|もの}めときかぬ
\ruby{氣}{き}のお
\ruby{龍}{りう}は
\ruby{打腹立}{うち|はら|だ}ちしが、
\ruby{怒}{いか}つて
\ruby{甲斐}{か|ひ}ある
\ruby{事}{こと}ならねば、
\ruby{其儘突}{その|ま〻|つ〻}と
\ruby{外}{そと}に
\ruby{出}{い}でたり。

\ruby{見}{み}れば
\ruby{前}{さき}の
\ruby{兒}{こ}は
\ruby{{\GWI{u7336-k}}}{なほ}
\ruby{其處}{そ|こ}に
\ruby{居}{を}りて、ねん〳〵ねん〳〵と
\ruby{負}{お}へる
\ruby{子}{こ}を
\ruby{賺}{すか}しながら、かな
\ruby{絲}{いと}をもて
\ruby{手鞠}{て|まり}を
\ruby{{\GWI{u9020-k}}}{つく}り
\ruby{居}{ゐ}たるに、お
\ruby{龍}{りう}はまた
\ruby{其}{そ}の
\ruby{娘}{こ}を
\ruby{呼}{よ}びかけて、

『
\ruby{折角汝}{せつ|かく|おまへ}さんに
\ruby{{\換字{教}}}{をし}へて
\ruby{貰}{もら}つたけれど、
\ruby{此家}{こ|〻}は
\ruby{妾}{わたし}の
\ruby{尋}{たづ}ねやうといふ
\ruby{家}{うち}と
\ruby{異}{ちが}つて
\ruby{居}{ゐ}たの!。
\ruby{{\GWI{u96a0-ue0102}}居{\換字{所}}}{いん|きよ|じよ}の
\ruby{方}{はう}といふのを
\ruby{知}{し}つておいでなら、
\ruby{一寸}{ちよ|つと}
\ruby{{\換字{教}}}{をし}へて
\ruby{下}{くだ}さいな。
』

と、
\ruby{笑}{ゑみ}をつくつて
\ruby{云}{い}へば、
\ruby{子守}{こ|もり}も
\ruby{莞爾}{に|こ}つき、

『あ〻お
\ruby{濱}{はま}ちやんの
\ruby{家}{うち}の
\ruby{事}{こと}かい、そんなら
\ruby{{\GWI{u7336-k}}}{なほ}の
\ruby{事}{こと}だ、
\ruby{伴}{つ}れてつて
\ruby{{\GWI{u9063-k}}}{や}るべい。
』

と
\ruby{云}{い}ひながらずん〳〵
\ruby{先}{さき}に
\ruby{立}{た}ちて、
\ruby{{\換字{巻}}}{ま}きかけたる
\ruby{鞠}{まり}を
\ruby{袂}{たもと}にして
\ruby{導}{みちび}きくれたり。

『ヘーエ、お
\ruby{濱}{はま}ちやんといふ
\ruby{女}{こ}が
\ruby{其家}{そ|こ}には
\ruby{居}{ゐ}るの?。
』

『アレ
\ruby{{\GWI{u96a0-ue0102}}居}{いん|きよ}の
\ruby{方}{はう}へ
\ruby{行}{い}く
\ruby{人}{ひと}で
\ruby{居}{ゐ}て、それでお
\ruby{濱}{はま}ちやんを
\ruby{知}{し}らないだかエ\GWI{u2048}。
』

『だつて
\ruby{妾}{わたし}は
\ruby{初}{はじめ}て
\ruby{來}{き}たもんで、
\ruby[g]{水野}{みづの}さんていふ
\ruby{方}{かた}を
\ruby{尋}{たづ}ねるんだもの!。
』

『
\ruby[g]{水野}{みづの}さんへ
\ruby{尋}{たづ}ねて來たつて!、アノ
\ruby{先生}{せん|せい}の
\ruby[g]{水野}{みづの}さんところへ\GWI{u2048}。
』

\ruby{振{\GWI{u8fd4-k}}}{ふり|かへ}つて
\ruby{子守}{こ|もり}は
\ruby{新}{あらた}にお
\ruby{龍}{りう}を
\ruby{見}{み}しが、
\ruby{其}{そ}の
\ruby{都}{みやこ}びて
\ruby{{\換字{清}}潔}{せい|けつ}に
\ruby{美}{うつく}しきは、
\ruby{何知}{なに|し}らぬ
\ruby{眼}{め}にも
\ruby{明}{あき}らかに
\ruby{映}{うつ}りたり。

『
\ruby{姐}{ね\GWI{u1b001}}さんは
\ruby[g]{水野}{みづの}さんの
\ruby{妹}{いもうと}ッ
\ruby{子}{こ}かェ。
』

お
\ruby{龍}{りう}は
\ruby{其}{そ}の
\ruby{頓狂}{とん|きやう}なる
\ruby{考}{かんが}へと
\ruby{唐突}{だし|ぬけ}なる
\ruby{問}{とひ}とに
\ruby[g]{自然}{おのづ}と
\ruby{笑}{わらひ}を
\ruby{催}{もよほ}さしめられたり。

『
\ruby{何故}{な|ぜ}?。
』

『
\ruby{何故}{な|ぜ}つて、
\ruby{東京}{とう|きやう}からわざ〳〵
\ruby{來}{き}たので
\ruby{無}{な}いかェ。
』

『ホヽヽ。
おもしろい
\ruby{事}{こと}をお
\ruby{云}{い}ひだネ、そりやあ
\ruby{東京}{とう|きやう}から
\ruby{來}{き}たのだけれども、
\ruby{東京}{とう|きやう}から
\ruby{來}{き}たとつて
\ruby{妹}{いもうと}たあ
\ruby{定}{きま}りやあ
\ruby{仕}{し}ません。
』

『アヽ
\ruby{解}{わか}つた。
ぢやあ
\ruby{姐}{ね\GWI{u1b001}}さんは
\ruby[g]{水野様}{みづのさん}の
\ruby{内君}{おかみ|さん}になる
\ruby{人}{ひと}だベェ。
』

『いやだよ、そんな
\ruby{飛}{と}んでもない
\ruby{事}{こと}を!。
ホヽヽ、
\ruby{妾}{わたし}あまだ
\ruby[g]{水野}{みづの}さんていふ
\ruby{方}{かた}にも、お目にか〻つた
\ruby{事}{こと}さへ
\ruby{有}{あ}りやしないのだよ。
』

『
\ruby{{\GWI{u96a0-ue0102}}}{かく}してもいかないだ!。
\ruby{姐}{ね\GWI{u1b001}}さん
\ruby{今些紅}{いま|ちつと|あか}い
\ruby{顏}{かほ}したゞ!。
ホレもう
\ruby{此家}{こ|〻}が
\ruby{御亭主}{ご|てい|しゆ}の
\ruby{家}{うち}だ。
\ruby{植込}{うゑ|こみ}いぢつて
\ruby{居}{ゐ}るのがお
\ruby{濱}{はま}ちやんのお
\ruby{爺}{ぢい}さんだよ。
』

\ruby{子守}{こ|もり}は
\ruby{五六歩}{ご|ろく|ほ}いきなりに
\ruby{駈}{か}け
\ruby{拔}{ぬ}けて、
\ruby{植込}{うゑ|こみ}の
\ruby{不揃}{ふ|ぞろ}いになりたるを
\ruby{鋏}{はさみ}もて
\ruby{剪}{き}り
\ruby{居}{ゐ}たる
\ruby{禿頭}{はげ|あたま}の
\ruby{澤}{つや}やかに
\ruby{人}{ひと}の
\ruby{好}{よ}さ〻うなる
\ruby{老人}{とし|より}に
\ruby{對}{むか}つて、
\ruby{一二間}{いち|に|けん}こなたより
\ruby{左}{さ}らでも
\ruby{徹}{とほ}る
\ruby{聲}{こゑ}を
\ruby{{\GWI{u9060-k}}慮無}{ゑん|りよ|な}く
\ruby{大}{おほき}くして、

『
\ruby{爺々}{おぢ|さん}!、
\ruby{妾東京}{おら|とう|きやう}のお
\ruby{客様}{きやく|さま}を
\ruby{案内}{あん|ない}して
\ruby{來}{き}たゞよ。
』

と、
\ruby[g]{確實}{たしか}に
\ruby{然様}{さ|う}と
\ruby{思}{おも}ひ
\ruby{{\GWI{u8fbc-k}}}{こ}んで、
\ruby{獨}{ひと}り
\ruby{承知}{しよ|うち}して
\ruby{叫}{さけ}び
\ruby{知}{し}らすれば、
\ruby[g]{吉右衛門}{きちゑもん}は
\ruby{不意}{ふ|い}なるに
\ruby{驚}{おどろ}きて
\ruby{答}{こた}へもせぬ
\ruby{時}{とき}、
\ruby[g]{此方}{こなた}に
\ruby{向}{むか}へる
\ruby{入口}{いり|くち}の
\ruby{障子}{しや|うじ}はさつと
\ruby{開}{あ}けられて、
\ruby{繪}{ゑ}に
\ruby{見}{み}るやうなる
\ruby{色白}{いろ|じろ}の
\ruby[g]{容貌好}{きりやうよ}き
\ruby{娘}{こ}の、
\ruby{星}{ほし}のやうなる
\ruby{美}{うつく}しき
\ruby{眼}{め}を
\ruby{光}{ひか}らせたると、
\ruby[g]{水野}{みづの}とは
\ruby{此人}{この|ひと}なるべき
\ruby{若}{わか}き
\ruby{男}{をとこ}との
\ruby{現}{あら}はれしが、
\ruby{是}{これ}は
\ruby{如何}{い|か}に
\ruby{其男}{その|をとこ}は
\ruby{今}{いま}し
\ruby{方}{がた}
\ruby{{\GWI{u6eca-j}}車}{き|しや}の
\ruby{内}{うち}にて
\ruby{妾}{わ}が
\ruby{傷}{きず}つけし
\ruby{其}{そ}の
\ruby{人}{ひと}なれば、
\ruby{互}{たがひ}に
\ruby{顏}{かほ}を
\ruby{見合}{み|あ}はす
\ruby{{\GWI{u9014-k}}端}{と|たん}、
\ruby{言葉}{こと|ば}こそ
\ruby{出}{いだ}さぬ
\ruby{彼此共}{かれ|これ|とも}に、これはと
\ruby{驚}{おどろ}きたる
\ruby{其}{そ}の
\ruby{様子}{やう|す}は、
\ruby{明}{あき}らかに
\ruby{傍}{かたへ}の
\ruby{人々}{ひと|〴〵}にも
\ruby{見}{み}\GWI{u1b001}たり。

『ホーラ、
\ruby{{\GWI{u96a0-ue0102}}}{かく}してもいかないだよ!。
\ruby{兩方}{りやう|はう}で
\ruby{知}{し}つてたゞ!。
』

\ruby{子守}{こ|もり}は
\ruby{獨}{ひと}り
\ruby{定}{ぎめ}に
\ruby{凱旋}{かち|どき}を
\ruby{擧}{あ}げて、
\ruby{得意}{とく|い}の
\ruby{顔}{かほ}つきして
\ruby[g]{彼方}{かなた}に
\ruby{走}{はし}り
\ruby{去}{さ}り、お
\ruby{濱}{はま}と
\ruby[g]{吉右衛門}{きちゑもん}とは
\ruby{無言}{む|ごん}にお
\ruby{龍}{りう}
\ruby[g]{水野}{みづの}を
\ruby{見}{み}れば、お
\ruby{龍}{りう}はたヾ
\ruby{何}{なに}とも
\ruby{{\換字{分}}}{わか}らぬ
\ruby{心地}{こヽ|ち}に、かつと
\ruby{紅色潮}{くれ|なゐ|さ}したる
\ruby{面}{おもて}、
\ruby{一朶}{いち|だ}の
\ruby{芙蓉十{\換字{分}}}{ふ|よう|じう|ぶん}に
\ruby{醉}{よ}ひたり。


\Entry{其二十九}

\ruby{既}{すで}に
\ruby{我}{わ}が
\ruby{{\換字{尋}}}{たづ}ぬる
\ruby{水野}{みづ|の}とは
\ruby{其人}{その|ひと}なるべく、
\ruby{{\換字{又}}}{また}
\ruby{今}{いま}
\ruby{聞}{き}ける
お
\ruby{濱}{はま}とは
\ruby{其娘}{その|こ}なるべしと
\ruby{猜}{すひ}し
\ruby{知}{し}りたれど、
\ruby{飛}{と}んでも
\ruby{無}{な}き
\ruby{子守女}{こ|も|り}の
\ruby{言葉}{こと|ば}に
\ruby{度}{ど}を
\ruby{失}{うしな}ひたる
お
\ruby{龍}{りう}は、
\ruby{思}{おも}はざる
\ruby{横風}{よこ|かぜ}に
\ruby{{\換字{扇}}}{あふ}られて
\ruby{目}{め}ざすところに
\ruby{{\換字{船}}首}{み|よし}を
\ruby{向}{む}けはぐりたる
\ruby{{\換字{船}}}{ふね}の、
\ruby{先}{ま}づ
\ruby{取}{と}りあへず
\ruby{間{\換字{近}}}{ま|ぢか}なる
\ruby{纜杭}{もやひ|ぐひ}に
\ruby{取}{と}りつけたるが
\ruby{如}{ごと}く、
\ruby{吉右衛門}{き|ち|ゑ|もん}に
\ruby{向}{むか}ひて
\ruby{小腰}{こ|ゞし}を
\ruby{屈}{かゞ}めつ、

『
\ruby{妾}{わたし}はあの、
\ruby{岩崎}{いは|ざき}の
\ruby{母}{はゝ}の
\ruby{許}{ところ}から
\ruby{參}{まゐ}つたものでございますが、
\ruby{水野}{みづ|の}さんがおいでになりますなら
\ruby{何卒}{どう|か}…………、』

と、
\ruby{辛}{から}くもこれだけを
\ruby{言}{い}ひてホツと
\ruby{息吐}{いき|つ}きたり。

\ruby{合點}{が|てん}の
\ruby{惡}{わか}からぬ
\ruby{吉右衛門}{き|ち|ゑ|もん}は、
\ruby{例}{れい}の
\ruby{眼鏡越}{め|がね|ご}しに
お
\ruby{龍}{りう}を
\ruby{見}{み}しが、
\ruby{手}{て}にせし
\ruby{剪刀}{はさ|み}を
\ruby{樹}{き}の
\ruby{枝}{\換字{江}だ}に
\ruby{一寸}{ちよ|つと}
\ruby{掛}{か}けすてゝ、
\ruby{淸}{きよ}らなる
\ruby{赤}{あか}ら
\ruby{顏}{がほ}に
\ruby{笑}{ゑみ}をさへ
\ruby{含}{ふく}み、

『ハア
\ruby{左樣}{さ|う}ですか、さあ
\ruby{御上}{お|あが}んなさい。
\ruby{丁度今}{ちやう|ど|いま}しがた
\ruby{御歸宅}{お|かへ|り}でした。
お
\ruby{濱}{はま}や、
\ruby{先生}{せん|せい}のところへ
\ruby{御客樣}{おき|やく|さま}だよ。
ハヽヽ、お
\ruby{蝶}{てふ}ツ
\ruby{子}{こ}が
\ruby{何}{なに}を
\ruby{下}{くだ}らない!。
』

と
\ruby{末}{すゑ}は
\ruby[g]{獨語}{ひとりごと}のやうに
\ruby{云}{い}ふところへ、
\ruby{生々}{いき|〳〵}として
\ruby{美}{うつく}しき
\ruby{娘}{こ}は
\ruby{下}{お}り
\ruby{來}{きた}りて、たゞ
\ruby{纔}{わづか}に
\ruby{頭}{かしら}を
\ruby{下}{さ}げたるばかりに
\ruby{愛度氣無}{あ|ど|け|な}く
\ruby{會釋}{ゑし|やく}し、

『どうか、
\ruby{此方}{こち|ら}から
\ruby{御上}{お|あが}んなすつて、』

と
\ruby{先}{さき}に
\ruby{立}{た}つてずつと
\ruby{庭}{には}を
\ruby{貫}{とほ}して
\ruby{導}{みちび}くは、
\ruby{入口}{いり|ぐち}より
\ruby{{\換字{通}}}{とほ}さば
\ruby{今}{いま}は
\ruby{其處}{そ|こ}に
\ruby{取}{と}り
\ruby{亂}{みだ}したる
\ruby{室}{へや}の
\ruby{他人}{ひ|と}には
\ruby{見}{み}せたからぬ
\ruby{狀}{さま}なるが
\ruby{有}{あ}ればなるべし。

\ruby{云}{い}はるゝがまゝに
お
\ruby{龍}{りう}は
\ruby{庭前}{には|さき}より
\ruby{上}{あが}りて、
\ruby{{\換字{通}}}{とほ}されたる
\ruby{室}{へや}にちまぢまと
\ruby{座}{すわ}れば、

『
\ruby{一寸}{ちよ|いと}
\ruby{御待}{おま|ち}ちなすつて。
たゞ
\ruby{今}{いま}
\ruby{直}{すぐ}、』

と
\ruby{云}{い}ひ
\ruby{置}{お}きて
\ruby{娘}{むすめ}は
\ruby{彼方}{かな|た}に
\ruby{去}{さ}りぬ。
\ruby{入口{\換字{近}}}{いり|ぐち|ゝか}き
\ruby{茶}{ちや}の
\ruby{室}{ま}とおぼしき
\ruby{方}{かた}に、
\ruby{其}{そ}の
\ruby{人}{ひと}も
\ruby{娘}{むすめ}も
\ruby{在}{あ}る
\ruby{樣子}{やう|す}ながら、
\ruby{何}{なに}を
\ruby{爲}{な}し
\ruby{居}{を}ればにや
\ruby{{\換字{猶}}}{なほ}
\ruby{出}{い}で
\ruby{來}{きた}らず、
\ruby{我}{われ}たゞ
\ruby{一人}{ひと|り}
\ruby{兀然}{つく|ねん}として
\ruby{室}{へや}の
\ruby{内}{うち}を
\ruby{見}{み}れば、
\ruby{二本立}{に|ほん|だち}の
\ruby{書箱}{ほん|ばこ}
\ruby{一}{ひと}ツ
\ruby{机一脚}{つくゑ|いつ|きやく}、
\ruby{本箱}{ほん|ばこ}に
\ruby{餘}{あま}れる
\ruby{本}{ほん}の
\ruby{幾十冊}{いく|じう|さつ}か
\ruby{壁}{かべ}に
\ruby{添}{そ}ひて
\ruby{積}{つ}まれたると、
\ruby{奥行}{おく|ゆ}きの
\ruby{淺}{あさ}き
\ruby{床}{とこ}の
\ruby{間}{ま}に
\ruby{西洋本}{せい|やう|ぼん}の
\ruby{少}{すくな}からず
\ruby{置}{お}かれたる
\ruby{其他}{その|ほか}には
\ruby{何}{なん}の
\ruby{{\換字{道}}具}{だう|ぐ}も
\ruby{無}{な}く
\ruby{裝{\換字{飾}}}{かざ|り}も
\ruby{無}{な}く、
\ruby{味}{あじ}も
\ruby{無}{な}く
\ruby{素氣}{そつ|け}も
\ruby{無}{な}き
\ruby{其}{そ}の
\ruby{態}{さま}は、
\ruby{惡口}{わる|くち}を
\ruby{云}{い}はゞ
\ruby{{\換字{巡}}査}{じゆ|んさ}の
\ruby{{\換字{交}}番{\換字{所}}}{かう|ばん|しよ}に
\ruby{少}{すこ}しばかり
\ruby{書籍}{ほ|ん}のあるやうなものなり。

お
\ruby{龍}{りう}は
\ruby{生}{う}まれてより
\ruby{未}{いま}だかつて
\ruby{見}{み}ぬ
\ruby{室}{へや}の
\ruby{狀態}{やう|す}に、
\ruby{荒野}{あら|の}に
\ruby{立}{た}つたるやうの
\ruby{心淋}{こゝろ|さび}しさを
\ruby{覺}{おぼ}えて、
\ruby{何}{なに}を
\ruby{書}{か}いたものか
\ruby{知}{し}れぬ
\ruby{西洋本}{せい|やう|ぼん}の、
\ruby{表紙}{へう|し}の
\ruby{金字}{きん|じ}
\ruby{燦々}{きら|〳〵}と
\ruby{輝}{かゞや}けるにのみたゞ
\ruby{{\換字{所}}在無}{しよ|ざい|な}さの
\ruby{眼}{め}を
\ruby{{\換字{留}}}{と}めて
\ruby{見}{み}つめ
\ruby{居}{ゐ}れば、
\ruby{物靜}{もの|しづ}かなる
\ruby{田舎}{ゐな|か}の
\ruby{晝間}{ひ|る}も
\ruby{寂}{しん}として、
\ruby{彼}{か}の
\ruby{老人}{とし|より}が
\ruby{使}{つか}ふ
\ruby{剪刀}{はさ|み}の
\ruby{音}{おと}は
\ruby{時々}{とき|〴〵}ちよつきりちよつきりと
\ruby{聞}{きこ}\換字{江}
\ruby{來}{く}るなり。

\ruby{心}{こゝろ}おのづから
\ruby{靜}{しづ}まれば
\ruby{耳}{みゝ}おのづから
\ruby{{\換字{聡}}}{さと}くなりて、
\ruby{小聲}{こ|ゞ\換字{江}}に
\ruby{相語}{あひ|かた}る
\ruby{彼方}{かな|た}の
\ruby{室}{ま}の
\ruby{話}{はなし}は、
\ruby[g]{幽微}{かすか}にはあれど
\ruby{今}{いま}は
\ruby{聞}{きこ}ゆ。

『
\ruby{左樣}{さ|う}!、それで
\ruby{知}{し}つて
\ruby{居}{ゐ}らしつたの!、あの
\ruby{人}{ひと}が
\ruby{先生}{せん|せい}の
\ruby{足}{あし}を
\ruby{踏}{ふ}んだ
\ruby{人}{ひと}なの!。
あら
\ruby{可厭}{い|や}な
\ruby{人}{ひと}だこと、
\ruby{妾{\換字{嫌}}}{わたし|きら}ひだは!。
』

『だつて
\ruby{{\換字{過}}失}{そ|さう}だもの
\ruby{仕方}{し|かた}が
\ruby{無}{な}い!。
\ruby{大變氣}{たい|へん|き}の
\ruby{毒}{どく}がつて
\ruby{叮嚀}{てい|ねい}に
\ruby{謝}{あやま}つたのだもの
\ruby{却}{かへ}つて
\ruby{優}{やさ}しい
\ruby{人}{ひと}だと
\ruby{私}{わたし}は
\ruby{思}{おも}つて
\ruby{居}{ゐ}るよ。
』

『
\ruby{左樣}{さ|う}ねえ!。
\ruby{左樣}{さ|う}いへば
\ruby{汗巾}{はん|けち}を
\ruby{破}{やぶ}つて
\ruby{傷}{きず}を
\ruby{{\換字{巻}}}{ま}いたつて。
アヽ
\ruby{矢張}{やつ|ぱ}り
\ruby{眞實}{ほん|と}は
\ruby{好}{い}い
\ruby{人}{ひと}なのね\換字{江}!。
ぢやあ
\ruby{妾}{わたし}は
\ruby{{\換字{嫌}}}{きら}ひぢや
\ruby{無}{な}くつて
\ruby{好}{すき}なのよ。
\ruby{何}{なん}だか
\ruby{最初見}{さい|しよ|み}た
\ruby{時}{とき}から
\ruby{妾}{わたし}は
\ruby{好}{すき}だつたのよ。
だけども
\ruby{先生}{せん|せい}の
\ruby{足}{あし}を
\ruby{踏}{ふ}んだつて
\ruby{云}{い}ふので
\ruby{{\換字{嫌}}}{いや}だと
\ruby{思}{おも}つたの!。
\ruby[g]{眞箇}{ほんと}に
\ruby{奇麗}{き|れい}な
\ruby{好}{い}い
\ruby{人}{ひと}ねえ!。
』

『ハヽヽ、
\ruby{好}{すき}だの
\ruby{{\換字{嫌}}}{きらひ}だのつて、
お
\ruby{濱}{はま}ちやん
\ruby{位}{ぐらゐ}いろ〳〵な
\ruby{事}{こと}をいふ
\ruby{人}{ひと}はありや
\ruby{仕}{し}ない。
そりやあ
\ruby{宣}{い}いけれども
お
\ruby{茶}{ちや}でも
\ruby{與}{や}つておくれ、
\ruby{置}{おき}つぱなしぢやあ
\ruby{可憫}{かあ|いさう}ぢやあ
\ruby{無}{な}いか。
』

『あら
\ruby{左樣}{さ|う}ぢや
\ruby{無}{な}くつてよ、
\ruby{今}{いま}
\ruby{御給仕}{お|きふ|じ}が
\ruby{濟}{す}んでから
\ruby{御茶}{お|ちや}を
\ruby{入}{い}れやふと
\ruby{思}{おも}つて
\ruby{居}{ゐ}たのよ。
』

『い〻\換字{江}
\ruby{私}{わたし}には
\ruby{關}{かま}はなくつてもい〻よ。
さあ〳〵もう
\ruby{御{\換字{終}}}{おし|まひ}だから!。
』

\ruby{言}{い}ふもの
\ruby{恐}{おそ}らくは
\ruby{何}{なん}の
\ruby{意無}{こゝろ|な}からん、
\ruby{聞}{き}くもの
\ruby{未}{いま}だ
\ruby{必}{かなら}ずしも
\ruby{感無}{かん|な}くばあらざるべし。


\Entry{其三十}

\原本頁{}%
お
\ruby{龍}{りう}は
\ruby{抑如何}{そも|い|か}なる
\ruby{人}{ひと}ぞや。

\原本頁{}%
お
\ruby{孃樣}{ぢやう|さま}
お
\ruby{孃樣}{ぢやう|さま}で
\ruby{育}{そだ}てられたる
\ruby{身}{み}にはあらねど、
%
\ruby{生}{うま}れついての
\ruby{心{\換字{情}}}{こゝろ|もち}に
\ruby{人}{ひと}とは
\ruby{異}{かは}つたるところあつて、
%
\ruby{駿府}{すん|ぷ}の
\ruby{叔母}{を|ば}のところへ
\ruby{引取}{ひき|と}られたる
\ruby{其夜}{その|よ}、
%
はじめて
\ruby{何}{なに}も
\ruby{無}{な}き
\ruby{座敷}{ざ|しき}に
\ruby{寐}{ね}かされて、
%
\ruby{吾家}{う|ち}では
\ruby{如是}{か|う}は
\ruby{無}{な}かつたものをと
\ruby{物足}{もの|た}らぬ
\ruby{心地}{こゝ|ち}し、
%
\ruby{{\換字{翌}}日}{あく|るひ}
\ruby{我}{わ}が
\ruby{荷物}{に|もつ}の
\ruby{行李}{こう|り}を
\ruby{解}{と}きし
\ruby{次}{ついで}に、
%
\ruby{我}{わ}が
\ruby{好}{す}きなもの〻
\ruby{數多}{かず|おほ}き
\ruby{中}{なか}より
\ruby{{\換字{平}}生}{ひご|ろ}
\ruby{氣}{き}に
\ruby{入}{い}りの
\ruby{永徳齋}{{\換字{𛀁}}い|とく|さい}の
\ruby{小人形}{こ|にん|ぎやう}を
\ruby{取}{と}り
\ruby{出}{いだ}して、
%
そつと
\ruby{小棚}{こ|だな}に
\ruby{{\換字{飾}}}{かざ}り
\ruby{置}{お}きしに、
%
それを
\ruby{固}{かた}い
\ruby{自慢}{じ|まん}の
\ruby{叔母}{を|ば}の
\ruby{帝釋樣}{たい|しやく|さま}のやうな
\ruby{三角}{さん|かく}の
\ruby{眼}{め}に
\ruby{睨}{にら}まれて、
%
\ruby{其樣}{そ|ん}な
\ruby{大}{おほ}きい
\ruby{形體}{な|り}をして
\ruby{人形}{にん|ぎやう}なんぞを
\ruby{捏}{こ}ね
\ruby{{\換字{廻}}}{まは}して
\ruby{{\換字{遊}}}{あそ}ぶと
\ruby{云}{い}ふ
\ruby{事}{こと}がありますか、
%
\ruby{藏}{しま}つて
\ruby{御置}{お|お}きなさい、
%
と
\ruby{唯}{たゞ}
\ruby{一言}{ひと|こと}に
\ruby{叱}{しか}りつけられ、
%
あ〻あんまりつまらない
\ruby{{\換字{情}}無}{なさけ|な}い
\ruby{叔母樣}{を|ば|さん}、
%
\ruby{何樣}{ど|う}すれば
\ruby{其樣}{そ|ん}な
\ruby{乾魚}{ひ|もの}のやうな
\ruby{氣}{き}になつて
\ruby{居}{ゐ}らる〻
\ruby{事}{こと}かと、
%
\ruby{恨}{うら}み
\ruby{疑}{うたが}ひながらも
\ruby{爭}{あらそ}ひかねて、
%
\ruby{其時}{その|とき}よりやうやく『わたしの
\ruby{好}{すき}な
\ruby{物}{もの}』を
\ruby{身}{み}の
\ruby{傍}{ほとり}に
\ruby{置}{お}かずして
\ruby{日}{ひ}を
\ruby{{\換字{送}}}{おく}るに
\ruby{慣}{な}る〻に
\ruby{至}{いた}りたるなり。

\原本頁{}%
されば
\ruby{頼}{たの}もしからぬ
\ruby{男}{をとこ}に
\ruby{一生}{いつ|しやう}を
\ruby{{\換字{過}}}{あやま}られて、
%
\ruby{涙}{なみだ}の
\ruby{淵瀬}{ふち|せ}に
\ruby{{\換字{浮}}}{う}き
\ruby{沈}{しづ}みしたる
\ruby{後}{のち}、
%
\ruby{今}{いま}は
\ruby{他人}{ひ|と}の
\ruby{家}{いへ}に
\ruby{寄食客}{かゝ|り|びと}の
\ruby{身}{み}の
\ruby{長閑}{のど|か}らしく
\ruby[g]{玩弄品}{おもちや}
\ruby{三昧}{ざん|まい}をするとにはあらねど、
%
\ruby{傳}{でん}といひ、
%
\ruby{淸}{せい}といひ、
%
\ruby{{\換字{勝}}}{かつ}といひ、
%
\ruby{彦}{ひこ}といひ、
%
\ruby{出入}{で|はい}る
\ruby{{\換字{若}}}{わか}き
\ruby[h|]{男}{をとこ}
\ruby{共}{ども}の
\ruby{爭}{あらそ}つて
\ruby{氣}{き}を
\ruby{取}{と}らんとて、
%
\ruby{折々}{をり|〳〵}
\ruby{吳}{く}れたる
\ruby{種々}{いろ|〳〵}の
\ruby{物品}{も|の}の
\ruby{中}{うち}、
%
\ruby{傳}{でん}が
\ruby{持}{も}て
\ruby{來}{きた}れる
\ruby{薄色}{うす|いろ}の
\ruby{瑪瑙}{め|なう}の
\ruby{細工}{さい|く}の
\ruby{小}{ちひさ}き
\ruby{兎}{うさぎ}の、
%
\ruby{姿}{すがた}しほらしくふつくりとして、
%
ぽつちりと
\ruby{紅}{あか}き
\ruby{眼}{め}のいと
\ruby{可憐}{か|はゆ}く
\ruby{出來}{で|き}たるが
\ruby{甚}{いた}く
\ruby{氣}{き}に
\ruby{入}{い}り、あれかこれかと、
%
アナ
\ruby{絲}{いと}の
\ruby{色}{いろ}を
\ruby{擇}{{\換字{𛀁}}ら}みに
\ruby{擇}{{\換字{𛀁}}ら}んで、
%
\ruby{其}{そ}のために
\ruby{敷}{し}くべき
\ruby{蒲團}{ふ|とん}の
\ruby{花}{はな}やかに
\ruby{美}{うつく}しきを
\ruby{{\換字{編}}}{あ}みて
\ruby{{\換字{遣}}}{や}りつ、
%
はじめて
\ruby{其}{それ}に
\ruby{載}{の}せて
\ruby{見}{み}たる
\ruby{時}{とき}、
%
\ruby{色}{いろ}の
\ruby{映}{うつ}り
\ruby{合}{あ}ひていよいよ
\ruby{好}{この}ましく
\ruby{愛}{あい}らしく
\ruby{見}{み}えたる
\ruby{嬉}{うれ}しさの
\ruby{餘}{あま}りの
\ruby{戱}{たはむ}れに、
%
\ruby{此兎}{こ|れ}は
\ruby{妾}{わたし}の
\ruby{大切}{だい|じ}な
\ruby{人}{ひと}なの!\inhibitglue{}と
\ruby{獨語}{ひとり|ごと}したりしが、
%
\ruby{其語}{そ|れ}を
\ruby{人}{ひと}より
\ruby{聞}{き}きて
\ruby{勘{\換字{違}}}{かん|ちが}ひしてか、
%
\ruby{其頃}{その|ころ}より
\ruby{傳}{でん}の
\ruby{煩}{うるさ}く
\ruby{付}{つ}き
\ruby{纏}{まと}ふ、
%
\ruby{其}{それ}は
\ruby{何}{なに}よりの
\ruby{{\換字{迷}}惑}{めい|わく}ながら、
%
\ruby{今}{いま}だに
\ruby{兎}{うさぎ}の
\ruby{可愛}{か|はゆ}さは
\ruby{冷}{さ}めず、
%
\ruby{何}{なん}ぞの
\ruby{折}{をり}には『
\ruby{兎之}{う|の}さん』と
\ruby{喚}{よ}びかけて、
%
\ruby{心}{こゝろ}の
\ruby{淋}{さび}しさ
\ruby{{\換字{遣}}}{や}る
\ruby{方無}{かた|な}き
\ruby{時}{とき}の、
%
\ruby{語}{かた}らう
\ruby{友無}{とも|な}き
\ruby{孤獨}{ひと|りみ}の
\ruby{憂}{う}さを、
%
\ruby{苟且}{かり|そめ}に
\ruby{一寸}{ちよ|つと}
\ruby{慰}{なぐさ}め
\ruby{忘}{わす}る〻なり。

\原本頁{}%
\ruby{是}{かく}のごとき
お
\ruby{龍}{りう}は
\ruby{今}{いま}
\ruby{一室}{いつ|しつ}の
\ruby{中}{うち}に、
%
\ruby{眼}{め}を
\ruby{慰}{なぐさ}め
\ruby{心}{こゝろ}を
\ruby{寄}{よ}せて
\ruby{{\換字{情}}懷}{おも|ひ}の
\ruby{{\換字{遣}}}{や}りどころとすべき
\ruby{物}{もの}の
\ruby{一}{ひと}つも
\ruby{無}{な}くて、
%
\ruby{床}{とこ}に
\ruby{插花瓶}{さし|ばな|いけ}は
\ruby{有}{あ}りながら
\ruby{末枯}{す|が}れたる
\ruby{花}{はな}も
\ruby{無}{な}く、
%
\ruby{机上}{つく|ゑ}に
\ruby{筆架水滴}{ふで|かけ|みづ|いれ}の
\ruby{影}{かげ}もあらで
\ruby{裸硯}{はだか|すゞり}の
\ruby{淋}{さび}しく
\ruby{置}{お}かれたるものなるを
\ruby{見}{み}て、
%
\ruby{成程}{なる|ほど}
\ruby{書生}{しよ|せい}さんは
\ruby{如是}{か|う}したものか
\ruby{知}{し}らねど、
%
\ruby{餘}{あま}りといへば
\ruby{曲}{きよく}の
\ruby{無}{な}い
\ruby{何}{なん}といふ
\ruby{此室}{この|ま}の
\ruby{狀態}{さ|ま}と、
%
ひそかに
\ruby{室主}{ある|じ}を
\ruby{疎}{うと}ましく
\ruby{思}{おも}へる
\ruby{折}{をり}しも、
%
\ruby{此家}{こ|ゝ}の
\ruby{娘}{むすめ}が
\ruby{我}{われ}を
\ruby{可厭}{い|や}な
\ruby{人}{ひと}と
\ruby{云}{い}ひしに
\ruby{對}{むか}ひて、
%
\ruby{我}{われ}を
\ruby{優}{やさ}しき
\ruby{人}{ひと}と
\ruby{云}{い}ひなし
\ruby{吳}{く}れたるを
\ruby{聞}{き}きて
\ruby{憎}{にく}く
\ruby{思}{おも}はんやうは
\ruby{無}{な}く、
%
あ〻まだ
\ruby{知}{し}りもせぬ
\ruby{人}{ひと}を
\ruby{惡}{わる}くばかり
\ruby{量}{つも}つたる
\ruby{事}{こと}と
\ruby{思}{おも}ひ
\ruby{{\換字{返}}}{かへ}す
\ruby{時}{とき}、
%
\ruby{無{\換字{造}}作}{む|ざう|さ}にすらりと
\ruby{間}{あひ}の
\ruby{襖}{ふすま}を
\ruby{開}{あ}けて、
%
\ruby{次}{つぎ}の
\ruby{室}{ま}より
\ruby{立出}{たち|い}でたる
\ruby{男}{をとこ}は
\ruby{我}{わ}が
\ruby{{\換字{前}}}{まへ}に
\ruby{座}{すは}れり。

\Entry{其三十一}

お
\ruby{龍}{りう}が
\ruby{頭}{かうべ}を
\ruby{下}{さ}げて
\ruby{禮}{れい}をなしつ、やがて
\ruby{言}{い}ひ
\ruby{出}{いだ}でんとする
\ruby{間}{ま}もあらせず、

『イヤお
\ruby{待}{また}せ
\ruby{申}{まをし}ました、
\ruby{小生}{わた|くし}は
\ruby{水野}{みづ|の}です。
』

と
\ruby{云}{い}ひたる、
\ruby{言語明晰}{げん|ご|はつ|きり}として
\ruby{冗處}{む|だ}も
\ruby{無}{な}く
\ruby{餘裕}{ゆ|とり}も
\ruby{無}{な}く、
\ruby{石甃}{いし|だゝみ}を
\ruby{見}{み}るやうに
\ruby{角}{かく}ばつたる
\ruby{云}{い}ひざまの、
\ruby{聲}{こゑ}つき
\ruby{自然威勢}{おのづ|から|いき|ほひ}あるにお
\ruby{龍}{りう}は
\ruby{吞}{の}まれて、
\ruby{釣込}{つり|こ}まれ
\ruby{氣味}{ぎ|み}に
\ruby[g]{此方}{こなた}も
\ruby{堅}{かた}くなり、

『あの
\ruby{妾}{わたくし}は
\ruby{岩崎}{いわ|ざき}の
\ruby{母}{はゝ}のところから
\ruby{出}{で}ましたもので、』

と、
\ruby{先}{ま}づ
\ruby{一句明}{いつ|く|あき}らかに
\ruby{那處}{いづ|く}より
\ruby{來}{きた}れるかを
\ruby{{\換字{更}}}{さら}に
\ruby{告}{つ}げたり。

『ハア。
\ruby{左樣}{さ|う}して
\ruby[g]{貴下}{あなた}は
\ruby{御{\換字{近}}{\換字{所}}}{ご|きん|じよ}の
\ruby{方}{かた}で〻もお
\ruby{有}{あ}りですか。
』

『ハイ、イエ、
\ruby{御承知}{ご|しよう|ち}はございますまいが
\ruby{妾}{わたくし}はあの、
\ruby[g]{彼方}{あちら}に
\ruby{御厄介}{ご|やく|かい}になつて
\ruby{居}{を}るものでございまして、
\ruby{舊}{もと}は
\ruby[g]{彼方}{あちら}でお
\ruby{稽古}{けい|こ}を
\ruby{願}{ねが}つたものでございます。
』

『アヽ
\ruby{左樣}{さ|う}ですか、してお
\ruby{師匠}{し|よ}さんはお
\ruby{變}{かは}りありませんか。
』

\ruby{師匠}{しゝ|やう}は
\ruby{打擲}{うち|たゝ}いても
\ruby{死}{し}なざるべく
\ruby{壯健}{じや|うぶ}にして、
\ruby{酒}{さけ}を
\ruby{{\換字{飲}}}{の}み
\ruby[g]{{\換字{情}}夫}{をとこ}と
\ruby{{\換字{連}}}{つ}れ
\ruby{立}{だ}ちて
\ruby{{\換字{遊}}}{あそ}び
\ruby{歩}{ある}けるものを、か〻る
\ruby{生眞面目}{き|ま|じ|め}なる
\ruby{人}{ひと}に
\ruby{虛言}{う|そ}を
\ruby{云}{い}ふことの
\ruby[g]{心咎}{こゝろとがめ}せられぬにはあらざれど、

『ハイ
\ruby{有}{あ}りがたうございます。
まあ
\ruby{別條}{べつ|でう}は
\ruby{無}{な}いやうなものでございますが、
\ruby[g]{先般}{このあひだ}から
\ruby[g]{一寸時候}{ちよつとじこう}あたりを
\ruby{致}{いた}して
\ruby{{\換字{弱}}}{よわ}つて
\ruby{居}{を}りますので。
』

と
\ruby{已}{やむ}を
\ruby{得}{\換字{江}}ず
\ruby{豫}{かね}ての
\ruby{命令}{いひ|つけ}を
\ruby{{\換字{終}}}{つひ}に
\ruby{果}{はた}したり。

『それは
\ruby{何樣}{ど|う}もいけませんなナ、たゞの
\ruby{風邪}{か|ぜ}ですか。
』

『イエもう、
\ruby{眞}{ほん}の
\ruby{一寸}{ちよ|つと}した
\ruby{事}{こと}でございまして、しかも
\ruby{治}{なほ}り
\ruby{加減}{か|げん}でございますから、お
\ruby{案}{あん}じ
\ruby{下}{くだ}さいますな。
それに
\ruby{就}{つ}きまして
\ruby{妾}{わたくし}が
\ruby{出}{で}ましたやうな
\ruby{譯}{わけ}でございますが、
\ruby{師匠}{しゝ|やう}が
\ruby{申}{まう}しますには、
\ruby{{\換字{過}}般}{この|あひだ}からはまた
\ruby{度々}{たび|〳〵}のお
\ruby{手紙}{て|がみ}で、
\ruby{五十}{い|そ}の
\ruby{病氣}{びや|うき}を
\ruby{一々}{いち|〳〵}お
\ruby{知}{し}らせ
\ruby{下}{くだ}さつたり、
\ruby{其上}{その|うへ}またいろ〳〵お
\ruby{世話}{せ|わ}を
\ruby{戴}{いたゞ}いたりしまして、お
\ruby{禮}{れい}を
\ruby{申}{まを}さうやうも
\ruby{無}{な}く
\ruby{有}{あ}り
\ruby{難}{がた}く
\ruby{存}{ぞん}じて
\ruby{居}{を}りまする。
\ruby{早速}{さつ|そく}にも
\ruby{自分}{じ|ぶん}で
\ruby{出}{で}てお
\ruby{禮}{れい}を
\ruby{申上}{まを|しあ}げ、
\ruby{五十}{い|そ}の
\ruby{見舞}{み|まひ}も
\ruby{看病}{かん|びやう}も
\ruby{致}{いた}さなくつてはならないのではございますが、
\ruby{生憎}{あい|にく}と
\ruby{自分}{じ|ぶん}も
\ruby{患}{わづら}つて
\ruby{居}{を}りまするので、
\ruby{存}{ぞん}じながら
\ruby{思}{おも}ふやうにも
\ruby{參}{まゐ}りません。
\ruby{水野}{みづ|の}さんが
\ruby{在}{い}らしつて
\ruby{下}{くだ}さるから
\ruby{好}{い}いはでもつて
\ruby{打棄}{うつ|ちや}つて
\ruby{居}{を}るやうで、
\ruby{大變}{たい|へん}
\ruby{心苦}{こゝろ|ぐる}しう
\ruby{存}{ぞん}じて
\ruby{居}{を}るのでございますが、
\ruby{全}{まつた}く
\ruby{左樣}{さ|う}いふ
\ruby{譯}{わけ}ではございません。
\ruby[g]{御承知}{ごしようち}の
\ruby{{\換字{通}}}{とほ}りの
\ruby{女暮}{おんな|ぐら}しで、
\ruby{手前}{て|まへ}にばかりかまけて
\ruby{居}{を}りまするので、
\ruby{彼樣}{あ|ゝ}も
\ruby{仕}{し}たい、
\ruby{此樣}{こ|う}も
\ruby{仕}{し}たいと
\ruby{色々}{いろ|〳〵}に、
\ruby{心}{こゝろ}では
\ruby{思}{おも}つて
\ruby{居}{を}りましても
\ruby{手}{て}が
\ruby{届}{とゞ}きませんから、たゞ
\ruby{陰}{かげ}でもつて
\ruby{神信心}{かみ|しん|〴〵}ばかり
\ruby{致}{いた}して
\ruby{居}{を}るやうな
\ruby{譯}{わけ}でございます!。
と
\ruby{如是申上}{か|う|まを|しあ}げて、
\ruby{何樣}{ど|う}か
\ruby{何分}{なに|ぶん}にも
\ruby{惡}{あ}しからず
\ruby{思召}{おぼし|めし}になるやうに、
\ruby{善}{よ}く
\ruby{汝}{おまへ}から
\ruby{有體}{あり|てい}のところを
\ruby{細}{こまか}にお
\ruby{話仕}{はな|しゝ}てお
\ruby{{\換字{呉}}}{く}れとの
\ruby{事}{こと}にございまする。
\ruby{{\換字{叉}}}{また}、どうか
\ruby{此上}{この|うへ}ともお
\ruby{世話}{せ|わ}を
\ruby{下}{くだ}さいますように、
\ruby{老母}{ばゞ|あ}は
\ruby{{\換字{勝}}手}{かつ|て}な
\ruby{奴}{やつ}だ
\ruby{顏}{かほ}も
\ruby{出}{だ}さないと、お
\ruby{愛想盡}{あい|そ|づか}しになりましても、
\ruby{病人}{びやう|にん}は
\ruby{何}{なに}も
\ruby{知}{し}らない
\ruby{事}{こと}でございますから、お
\ruby{愛想盡}{あい|そ|づか}しをなさらないやうに。
\ruby{五十}{い|そ}の
\ruby{事}{こと}は
\ruby{實}{じつ}は
\ruby{我儘}{わが|まゝ}な
\ruby{申}{まを}し
\ruby{樣}{やう}ですが、
\ruby{疾}{とう}から
\ruby[g]{貴下}{あなた}にお
\ruby{任}{まか}せ
\ruby{申}{まを}したつもりで
\ruby{居}{を}りまするのでございますから、
\ruby{何}{ど}のやうにでもお
\ruby[g]{心持次第}{こゝろもちしだい}になすつて
\ruby{戴}{いたゞ}きたいので、
\ruby{御親切}{ご|しん|せつ}の
\ruby[g]{貴下}{あなた}のお
\ruby{世話}{せ|わ}を
\ruby{戴}{いただ}いて、
\ruby{其}{それ}でいけなければ
\ruby{殘}{のこ}り
\ruby{惜}{をし}い
\ruby{事}{こと}はございません、
\ruby{全}{まつた}く
\ruby{當人}{たう|にん}の
\ruby{{\換字{運}}}{うん}の
\ruby{無}{な}いのだと
\ruby{諦}{あき}らめます。
いづれ
\ruby{其中}{その|うち}には
\ruby{是非}{ぜ|ひ}とも
\ruby{伺}{うかが}つてお
\ruby{禮}{れい}を
\ruby{申}{まを}すつもりでございます。
\ruby[g]{汝彼方樣}{おまへあちらさま}へ
\ruby{上}{あが}つたら、
\ruby{何樣}{ど|う}か
\ruby{妾}{わたし}が
\ruby{如是}{か|う}いふ
\ruby{心持}{こゝろ|もち}を
\ruby{有}{も}つて
\ruby{居}{を}りますといふ
\ruby{事}{こと}を
\ruby{云}{い}つて、
\ruby{十分}{じう|ぶん}にお
\ruby{禮}{れい}を
\ruby{申上}{まを|しあ}げて、
\ruby{而}{そ}して
\ruby{五十}{い|そ}の
\ruby{病氣}{びや|うき}の
\ruby{樣子}{やう|す}も
\ruby{伺}{うかゞ}つて
\ruby{來}{き}てお
\ruby{{\換字{呉}}}{く}れ、と
\ruby{斯樣}{か|やう}に
\ruby{申}{まを}すのでございます。
それでお
\ruby{馴染}{な|じ}みも
\ruby{無}{な}い
\ruby{妾}{わたくし}ではございますが、
\ruby{他}{ほか}に
\ruby{參}{まゐ}るものも
\ruby{無}{な}いのでございますから、
\ruby{一寸上}{ちよ|つと|あが}つたのでございます。
』

お
\ruby{龍}{りう}は
\ruby{果}{はた}さでは
\ruby{叶}{かな}はぬ
\ruby[g]{使者}{つかひ}の
\ruby{役目}{やく|め}を
\ruby{務}{つと}め
\ruby{果}{おほ}せん
\ruby{一心}{いつ|しん}に、
\ruby{一生懸命}{いつ|しやう|けん|めい}になりて
\ruby{如是{\換字{述}}}{か|く|の}べ
\ruby{{\換字{終}}}{をは}りしが、
\ruby{辛}{から}くも
\ruby{{\換字{吩}}附}{いひ|つ}けられしだけは
\ruby{云}{い}ひ
\ruby{得}{\換字{江}}たるにホツと
\ruby{氣息吐}{い|き|つ}きて、
\ruby{男}{をとこ}の
\ruby{樣子}{やう|す}を
\ruby{如何}{い|か}にと
\ruby{見}{み}れば、
\ruby{男}{をとこ}は
\ruby{律義眞正直}{りち|ぎ|まつ|しやう|ぢき}に
\ruby{物堅}{もの|がた}く
\ruby{愼}{つゝし}みて
\ruby{耳}{みゝ}を
\ruby{傾}{かたむ}け、
\ruby{見}{み}す〳〵の
\ruby{我}{わ}が
\ruby{虛言}{う|そ}を
\ruby{實}{げ}に
\ruby{{\換字{道}}理}{もつ|とも}と
\ruby{聞}{き}けるやうなるに、
\ruby{此}{こ}のやうなる
\ruby{人}{ひと}を
\ruby{口頭}{くち|さき}に
\ruby{操}{あやつ}るはと、
\ruby{我羞}{われ|はづか}しき
\ruby{心地}{こゝ|ち}の
\ruby{爲}{し}たり。

『ハイ、
\ruby{一々精}{いち|〳〵|よ}く
\ruby{解}{わか}りました、
\ruby{承知致}{しや|うち|いた}しました。
お
\ruby{言葉}{こと|ば}が
\ruby{無}{な}くてさへいろ〳〵に
\ruby{心配}{しん|ぱい}は
\ruby{致}{いた}して
\ruby{居}{を}りましたのですから、
\ruby{其樣}{そ|う}いふお
\ruby{言葉}{こと|ば}を
\ruby{伺}{うかゞ}ひます
\ruby{上}{うへ}は
\ruby{{\換字{猶}}}{なほ}の
\ruby{事}{こと}でございます。
\ruby{水野}{みづ|の}が
\ruby{出來}{で|き}まする
\ruby{事}{こと}は
\ruby{致}{いた}しますから、
\ruby{五十子}{い|そ|こ}さんの
\ruby{事}{こと}はお
\ruby{心遣無}{こゝろ|づか|ひな}く、よく
\ruby{御養生}{ご|やう|じやう}をなすつて
\ruby{早}{はや}く
\ruby{御全快}{ご|ぜん|くわい}なさるやうにと
\ruby{仰}{おつし}あつて
\ruby{下}{くだ}さいまし。
\ruby{五十子}{い|そ|こ}さんは
\ruby{必}{かなら}ず
\ruby{私}{わたくし}が
\ruby{癒}{なほ}らせます。
\ruby{何樣}{ど|う}しても
\ruby{一度}{いち|ど}は
\ruby{屹度}{きつ|と}
\ruby{癒}{なほ}らせますと
\ruby{小生}{わた|くし}が
\ruby{申}{まを}したと
\ruby{仰}{おつし}あつて
\ruby{下}{くだ}さいまし。
』

\ruby{人}{ひと}の
\ruby{命}{いのち}は
\ruby{知}{し}るべからざるを、あ〻
\ruby{何}{なん}ぞ
\ruby{其言葉}{その|こと|ば}の
\ruby{男兒}{をと|こ}らしく
\ruby{頼}{たの}もしきや。
\ruby{聲}{こゑ}の
\ruby{大}{おほき}くなりたるも
\ruby{思}{おも}はず
\ruby[g]{誠意}{まこと}の
\ruby{籠}{こも}りたればなるべし。
\ruby{如斯云}{か|く|い}へる
\ruby{其}{そ}の
\ruby{言葉}{こと|ば}の
\ruby{力}{ちから}あるに
\ruby{驚}{おどろ}かされて、お
\ruby{龍}{りう}は
\ruby{今{\換字{叉}}改}{いま|また|あらた}めて
\ruby{窃}{そつ}と
\ruby{其人}{その|ひと}を
\ruby{伺}{うかゞ}へば、
\ruby{聊}{いさゝ}か
\ruby{窶}{やつ}れたる
\ruby{淺黑}{あさ|ぐろ}き
\ruby{面}{おもて}の、
\ruby{鼻筋{\換字{通}}}{はな|すじ|とほ}り
\ruby{口締}{くち|しま}りて、
\ruby{巖}{いは}も
\ruby{黑鐵}{くろ|がね}も
\ruby{貫}{つらぬ}き
\ruby{徹}{とほ}すべき
\ruby[g]{精神}{きあひ}は、
\ruby{切}{き}れの
\ruby{長}{なが}き
\ruby{尾上}{しり|あが}りの
\ruby{眼}{め}の
\ruby{中}{うち}の
\ruby{光}{ひかり}に
\ruby{現}{あらは}れたるに、
\ruby{生}{うま}れて
\ruby{初}{はじ}めてか〻る
\ruby{意氣組}{い|き|ぐみ}の
\ruby{{\換字{銳}}}{するど}くして
\ruby{烈}{はげ}しき、
\ruby{昔物語}{むかし|もの|がたり}の
\ruby{中}{うち}の
\ruby{勇士}{ゆう|し}のやうなる
\ruby{人}{ひと}を
\ruby{眼}{め}の
\ruby{前}{まへ}に
\ruby{見}{み}て、あ〻
\ruby{何}{なん}といふ
\ruby{氣味}{き|み}のよい
\ruby{人}{ひと}と、
\ruby{深}{ふか}きに
\ruby{望}{のぞ}む
\ruby{千尺}{せん|じやく}の
\ruby{崖}{がけ}に
\ruby{立}{た}つて
\ruby{吹}{ふ}き
\ruby{來}{く}る
\ruby{秋風}{あき|かぜ}に
\ruby{袂}{たもと}を
\ruby{{\換字{扇}}}{あふ}らせたるが
\ruby{如}{ごと}く、
\ruby{凄}{すさま}じきが
\ruby{中}{なか}に
\ruby[g]{爽快}{いさぎよき}を
\ruby{覺}{おぼ}えて、
\ruby{怖}{こは}らしくは
\ruby{思}{おも}ひながら
\ruby{好}{この}ましくも
\ruby{思}{おも}ひたり。


\Entry{其三十二}

か\ninojiten{}るところへ
\ruby{新}{あらた}に
\ruby{茶}{ちや}をいれて
\ruby{持來}{もち|きた}りし
お
\ruby{濱}{はま}に、はつきりと
\ruby{美}{うつく}しき
\ruby{眼}{め}に
\ruby{優}{やさ}しく
お
\ruby{龍}{りう}を
\ruby{見}{み}て、しとやかに
\ruby{其}{そ}の
\ruby{一盞}{いつ|さん}を
\ruby{取}{と}りて
\ruby{薦}{す〻}むれば、
\ruby{水野}{みづ|の}を
\ruby{見}{み}たる
\ruby{目}{め}を
\ruby{此人}{この|ひと}に
\ruby{移}{うつ}しては、
\ruby{懷暗}{ふところ|くら}き
\ruby[g]{常綠樹}{ときはぎ}の
\ruby{高}{たか}く
\ruby{聳}{そび}\換字{江}たるを
\ruby{見}{み}たる
\ruby{目}{め}に、しほらしく
\ruby{{\換字{咲}}}{さ}く
\ruby{初櫻}{はつ|ざくら}の、ぱつと
\ruby{明}{あか}るき
\ruby{花}{はな}の
\ruby{枝}{えだ}を
\ruby{忽}{たちま}ち
\ruby{見}{み}たる
\ruby{心地}{こゝ|ち}して、おのづから
\ruby{胸}{むね}も
\ruby{開}{ひら}くるやうするに、
お
\ruby{龍}{りう}は、

『どうもはゞかりさま、
\ruby{恐}{おそ}れ
\ruby{入}{い}ります。
』

と
\ruby{身}{み}を
\ruby{{\換字{謙}}{\換字{退}}}{へり|くだ}りて
\ruby{會釋}{ゑし|やく}しつ、
\ruby{互}{たがひ}に
\ruby{顏}{かほ}を
\ruby{見合}{み|あ}はせしが、
\ruby{笑}{わら}ふとも
\ruby{無}{な}く
\ruby{媽然}{につ|こり}としたる
\ruby{彼此一時}{かれ|これ|いち|じ}の
\ruby{笑容}{ゑ|み}の
\ruby{中}{うち}に、
\ruby{語}{かた}らで
\ruby{語}{かた}り
\ruby{聞}{き}かで
\ruby{聞}{き}く
\ruby{心}{こゝろ}と
\ruby{心}{こゝろ}と
\ruby{働}{はたら}きて、
\ruby{思}{おも}へば
\ruby{思}{おも}ひ
\ruby{好}{す}けば
\ruby{好}{す}く
\ruby{性}{しやう}の
\ruby{合}{あ}ふ
\ruby{同士}{どう|し}
\ruby{女}{をんな}
\ruby{同士}{どう|し}、
\ruby{何}{なに}の
\ruby{故}{ゆゑ}とは
\ruby{無}{な}けれども
\ruby{相}{あひ}なつかしみ
\ruby{相{\換字{悅}}}{あひ|よろこ}びたり。

されどお
\ruby{濱}{はま}は
\ruby{何時}{い|つ}まで
\ruby{此處}{こ|\ninojiten}にあるべきならねば、
お
\ruby{龍}{りう}と
\ruby{物語}{もの|がた}りして
\ruby{{\換字{遊}}}{あそ}びたきやうの
\ruby{思}{おもひ}は
\ruby{仕}{し}ながら、
\ruby{一盞}{いつ|さん}を
\ruby{取}{と}りて
\ruby{水野}{みづ|の}に
\ruby{與}{あた}へて、
\ruby{好}{よ}きほどのところに
\ruby{茶具}{ちや|ぐ}を
\ruby{置}{お}き
\ruby{捨}{す}て、おのれは
\ruby{茶}{ちや}の
\ruby{間}{ま}に
\ruby{{\換字{退}}}{しりぞ}きて
\ruby{二人}{ふた|り}の
\ruby{話}{はなし}を
\ruby{聞}{き}けり。

お
\ruby{龍}{りう}は
\ruby{{\換字{猶}}}{なほ}
\ruby{五十子}{い|そ|こ}の
\ruby{容態}{よう|だい}を
\ruby{聞}{き}かでは
\ruby{叶}{かな}はざるなり。

『ほんとうに
\ruby{段々}{だん|〳〵}との
\ruby{深}{ふか}い
\ruby{御親切}{ご|しん|せつ}さまで、まことに
\ruby{有}{あ}り
\ruby{難}{がた}う
\ruby{存}{ぞん}じます。
\ruby{歸}{かへ}つて
\ruby{御言葉}{お|こと|ば}
\ruby{{\換字{通}}}{どほ}りに
\ruby{左樣申}{さ|う|まを}し
\ruby{傳}{つた}へましたら、
\ruby{何樣}{ど|ん}なにか
\ruby{師匠}{し〻|やう}も
\ruby{{\換字{悅}}}{よろこ}ぶことでございましやう。
\ruby{左樣}{さ|う}いたしまして
\ruby{只今}{たゞ|いま}は、
\ruby{病人}{びやう|にん}は
\ruby{何樣}{ど|ん}な
\ruby{樣子}{やう|す}でございますの?』

『いや
\ruby{何樣}{ど|う}も
\ruby{中々良}{なか|〳〵|よ}くないのです。
それで
\ruby{大}{おほ}きに
\ruby{心配致}{しん|ぱい|いた}しましたが、
\ruby{淺草}{あさ|くさ}の
\ruby{醫者}{い|しや}を
\ruby{招}{よ}びに
\ruby{行}{ゆ}きました
\ruby[g]{歸路}{かへり}に、たつた
\ruby{今此村}{いま|こ|〻}の
\ruby{醫者}{い|しや}に
\ruby{容態}{よう|だい}を
\ruby{聞}{き}きましたら、
\ruby{大}{おほ}きに
\ruby{見直}{み|なほ}したやうな
\ruby{具合}{ぐ|あひ}でして、
\ruby{重病}{ぢう|びやう}だから
\ruby{何}{なん}とも
\ruby{云}{い}へないが、
\ruby{此儘}{この|ま〻}で
\ruby{日}{ひ}さへ
\ruby{經}{へ}て
\ruby{{\換字{呉}}}{く}れ〻ばまあ
\ruby{宣}{よ}いといふので…………』

『では
\ruby{食事}{しよ|くじ}なんどは?。
』

『なか〳〵まだ
\ruby{食事}{しよ|くじ}なんぞといふ
\ruby{段}{だん}では
\ruby{無}{な}いので。
やつと
\ruby{流動物}{りう|どう|ぶつ}が
\ruby[g]{小量許入}{すこしばかりはい}る
\ruby{位}{くらゐ}です。
しかし
\ruby{變}{へん}さへ
\ruby{無}{な}ければ、
\ruby{大抵}{たい|てい}は
\ruby{經{\換字{過}}日數}{けい|くわ|につ|すう}が
\ruby{定}{きま}つて
\ruby{居}{ゐ}るものださうですから。
』

『
\ruby{案}{あん}じるやうな
\ruby{事}{こと}はまあ
\ruby{無}{な}いのでございますか。
』

『
\ruby{左樣}{さ|う}ばかりにもいきますまいが。
』

『
\ruby{變}{へん}の
\ruby{無}{な}いやうに
\ruby{致}{いた}しかたは
\ruby{無}{な}いものでございましやうか。
』

『そりやあ
\ruby{左樣}{さ|う}したいのは
\ruby{山々}{やま|〳〵}ですが、
\ruby{{\換字{情}}無}{なさ|けな}い
\ruby{事}{こと}には
\ruby{醫者}{い|しや}の
\ruby{力}{ちから}でも
\ruby{其處}{そ|こ}までは
\ruby{何樣}{ど|う}もなりません。
』

『それぢやあ
\ruby{神樣}{かみ|さま}にでも
\ruby{御願申}{おね|がひ|まを}すよりほかには!。
』

『
\ruby{然樣}{さ|う}です。
とてもまあ
\ruby{其樣}{そ|ん}な
\ruby{事}{こと}よりほかには!。
』

\ruby{男}{をとこ}の
\ruby{聲}{こゑ}はこ\ninojiten{}に
\ruby{至}{いた}つて
\ruby{甚}{ひど}く
\ruby{沈}{しづ}めり。
お
\ruby{龍}{りう}は
\ruby{忽然}{こつ|ぜん}として
\ruby{思}{おも}ひ
\ruby{{\換字{浮}}}{う}かぶるところあり。
\ruby{我}{われ}に
\ruby{對}{むか}へる
\ruby{此人}{この|ひと}は
\ruby{誰}{たれ}ぞ。
この
\ruby{人}{ひと}は
\ruby{是彼}{これ|か}の
\ruby{普門品}{ふ|もん|ぼん}の
\ruby{主}{ぬし}ならずや。
\ruby{何}{なに}をか
\ruby{獨}{ひと}り
\ruby{物思}{もの|おも}ひして
\ruby{睫毛}{まつ|げ}に
\ruby{露}{つゆ}を
\ruby{湛}{た\ninojiten}へし
\ruby{人}{ひと}ならずや。
あはれ
\ruby{戀故}{こひ|ゆゑ}の
\ruby{信心}{しん|〴〵}で
\ruby{無}{な}かれかしと、よそながら
\ruby{我}{わ}が
\ruby{念}{ねん}じ
\ruby{{\換字{遣}}}{や}りし
\ruby{其人}{その|ひと}ならずや。
\ruby{何}{なに}をか
\ruby{獨}{ひと}り
\ruby{物思}{もの|おも}ひして
\ruby{睫毛}{まつ|げ}に
\ruby{露}{つゆ}を
\ruby{湛}{た\ninojiten}へし
\ruby{人}{ひと}ならずや。
\ruby{{\換字{滊}}車}{き|しや}の
\ruby{中}{うち}の
\ruby{素振}{そ|ぶり}、
\ruby[g]{先刻}{さつき}よりの
\ruby{應對}{おう|たい}、
\ruby{今}{いま}の
\ruby{此}{こ}の
\ruby{樣子}{やう|す}に、
\ruby[g]{一切}{すべて}は
\ruby{解}{わか}りたり。
\ruby{師匠}{し〻|やう}は
\ruby{碌}{ろく}にも
\ruby{我}{われ}に
\ruby{語}{かた}らざりしが、
\ruby{此人}{この|ひと}は
\ruby{是五十子}{これ|い|そ|こ}といへるに
\ruby{深}{ふか}く
\ruby{思}{おもひ}を
\ruby{懸}{か}けて
\ruby{戀}{こひ}せるなるべし。
\ruby{似合}{に|あ}はしからぬ
\ruby{佛頼}{ほとけ|だの}みにも
\ruby{其胸}{その|むね}の
\ruby{中}{うち}の
\ruby{苦}{くるし}さぞ
\ruby{知}{し}らる\ninojiten{}!。
\ruby{嗚呼一昨年}{あ|\ninojiten|をと|と|し}の
\ruby{我}{われ}を
\ruby{男子}{をと|こ}にして
\ruby{見}{み}る、
\ruby{其}{そ}の
\ruby{顏}{かほ}の
\ruby{愁}{うれひ}に
\ruby{痩}{や}せて
\ruby{{\換字{情}}無}{なさ|けな}い
\ruby{有樣}{あり|さま}!、
\ruby{其}{そ}の
\ruby{眼}{め}の
\ruby{戀}{こひ}に
\ruby{疲}{つか}れきつて
\ruby{和}{なご}やかなるところの
\ruby{彼}{あ}の
\ruby{乏}{とぼ}しさ!。
\ruby{血属}{ち|すぢ}や
\ruby{見寄}{み|より}の
\ruby{有}{あ}りは
\ruby{有}{あ}つても、まことに
\ruby{戀}{こひ}に
\ruby{惱}{なや}む
\ruby{時}{とき}は、いつか
\ruby[g]{孤獨}{ひとり}の
\ruby{身}{み}となり
\ruby{果}{は}て\ninojiten{}、
\ruby{誰一人}{たれ|ひと|り}
\ruby{味方}{み|かた}になつて
\ruby{泣}{な}いて
\ruby{{\換字{呉}}}{く}れるものも
\ruby{無}{な}いのが
\ruby{世}{よ}の
\ruby{{\換字{習}}}{ならひ}!。
あ\ninojiten{}
\ruby{憫然}{かはゆ|さう}な〳〵
\ruby{人}{ひと}!。
と
\ruby[g]{經驗}{おぼ{\換字{江}}}ある
\ruby{身}{み}の
\ruby{思}{おも}ひ
\ruby{{\換字{遣}}}{や}り
\ruby{深}{ふか}く、

『あ\ninojiten{}、
\ruby{眞實}{ほん|と}に
\ruby{左樣}{さ|う}でございます!。
\ruby{神樣佛樣}{かみ|さま|ほとけ|さま}よりほかには
\ruby{左樣}{さ|う}いふ
\ruby{時}{とき}には、
\ruby{御賴}{お|たの}み
\ruby{申}{まを}すところもございません。
\ruby[g]{歸路}{かへり}には
\ruby{淺草}{あさ|くさ}の
\ruby{觀音樣}{くわん|おん|さま}で、
\ruby{妾}{わたし}も
\ruby{御百度}{お|ひやく|ど}でも
\ruby{踏}{ふ}みまして、
\ruby{何樣}{ど|う}か
\ruby{快}{よ}く
\ruby{御}{お}なりなさるやうに
\ruby{願}{ねが}ひませう。
』

と
\ruby{云}{い}はれて
\ruby{水野}{みづ|の}も
\ruby{心嬉}{こゝろ|うれ}しく、

『そりやあ、
\ruby{有}{あ}り
\ruby{難}{がた}い
\ruby{御親切}{ご|しん|せつ}の
\ruby{事}{こと}です。
\ruby{何樣}{ど|う}か
\ruby{病人}{びやう|にん}の
\ruby{快}{い}いやうに
\ruby{祈}{いの}つて
\ruby{下}{くだ}さい。
』

と、
\ruby{全}{まつた}く
\ruby{{\換字{平}}凡}{た|ゞ}の
\ruby{人}{ひと}の
\ruby{如}{ごと}き
\ruby{挨拶}{あい|さつ}をすれば、

『アラ、
\ruby{何樣}{ど|う}したのだらう?
\ruby{先生}{せん|せい}が!。
\ruby{觀音樣}{くわん|おん|さま}なんかに
\ruby{祈}{いの}つて
\ruby{{\換字{呉}}}{く}れなんて!。
ホヽヽ、
\ruby{古}{ふる}ぼけた
\ruby[g]{老婆}{おばあさん}かなんか
\ruby{見}{み}たやうに。
』

と
\ruby{何知}{なに|し}らぬ
お
\ruby{濱}{はま}は
\ruby{之}{これ}を
\ruby{蔭}{かげ}にて
\ruby{聞}{き}きて、
\ruby{聞}{きこ}えぬほどに
\ruby{獨語}{ひと|りご}ちて
\ruby{笑}{わら}へり。

\ruby{命令}{いひ|つけ}られたる
\ruby{事}{こと}は
\ruby{大槪果}{おほ|よそ|はた}したれば、ここに
お
\ruby{龍}{りう}ははじめて
\ruby{隙}{ひま}を
\ruby{得}{\換字{江}}て、

『つい
\ruby{申}{まを}しそびれて
\ruby{居}{を}りましたが
\ruby{先刻}{さき|ほど}は
\ruby{何樣}{ど|う}も、とんだ
\ruby{{\換字{過}}失}{そ|さう}を
\ruby{致}{いた}しました。
\ruby{此方}{こち|ら}へ
\ruby{上}{あが}つて
お
\ruby{目}{め}にかかると、
\ruby{貴下}{あな|た}が
\ruby{其方}{その|かた}だつたのでまた
\ruby{吃驚致}{びつ|くり|いた}しましたのでございます。
お
\ruby{怪我}{け|が}をさせまして
\ruby{眞}{まこと}に
\ruby{濟}{す}みません、どうか
\ruby{御免}{ご|めん}なさつてくださいまし。
』

と
\ruby{改}{あらた}めて
\ruby{謝罪}{わ|び}れば
\ruby{水野}{みづ|の}は
\ruby{慨然}{がい|ぜん}として、

『ナアニ
\ruby{貴女}{あな|た}に
\ruby{踏}{ふ}まれて
\ruby{流}{なが}れた
\ruby{彼樣}{あ|ん}な
\ruby{紅}{あか}い
\ruby{水}{みづ}、
\ruby[g]{少許}{ちつと}や
\ruby[g]{若干量流}{そつとなが}れたつて
\ruby{何}{なに}が
\ruby{何}{なん}でしやう!。
ハヽハヽハヽ。
』

と
\ruby{裏枯}{うら|が}れたる
\ruby{聲}{こゑ}して
\ruby{自}{みづか}ら
\ruby{嘲}{あざけ}るやうに
\ruby{淋}{さび}しく
\ruby{笑}{わら}へり。
\ruby{其意}{その|こゝろ}を
\ruby{解}{と}きて
\ruby{知}{し}るよしも
\ruby{無}{な}けれど、
\ruby{其}{そ}の
\ruby{言葉}{こと|ば}の
\ruby{異樣}{こと|やう}にして
\ruby{其}{そ}の
\ruby[g]{調子}{てうし}の
\ruby{悲哀}{かな|しみ}を
\ruby{含}{ふく}めるに、
\ruby{感}{かん}じ
\ruby{易}{やす}き
お
\ruby{龍}{りう}は
\ruby{一種}{いつ|しゆ}の
\ruby{感}{かん}に
\ruby{打}{う}たれて、
\ruby{頓}{とみ}には
\ruby{答}{こたへ}をさへ
\ruby{出}{いだ}しかねたり。


\Entry{其三十三}

お
\ruby{龍}{りう}は
\ruby{徐}{しづか}に
\ruby{三絃}{さみ|せん}の
\ruby{糸}{いと}を
\ruby{弛}{ゆる}めて
\ruby{三絃掛}{さみ|せん|かけ}へ
\ruby{掛}{か}け
\ruby{納}{をさ}むれば、
\ruby{今日}{け|ふ}
\ruby{目見得}{め|み|\GWI{u1b001}}に
\ruby{來}{きた}りし
\ruby{小婢}{こを|んな}お
\ruby{熊}{くま}は
\ruby{高麗鼠}{こ|ま|ねずみ}のやうにくる〳〵と
\ruby{働}{はたら}きて、しきりに
\ruby{其邊}{そこ|ら}を
\ruby{取}{と}り
\ruby{片付}{かた|づ}けしが、
\ruby{煙草{\換字{盆}}}{たば|こ|ぼん}の
\ruby{傍}{かたはら}より
\ruby{玉}{ぎよく}の
\ruby{煙管}{パイ|プ}のいと
\ruby{小}{ちいさ}なるを
\ruby{拾}{ひろ}ひあげて
\ruby{洋燈近}{らん|ぷ|ちか}くさし
\ruby{出}{いだ}し、

『これ
\ruby{此様}{こ|ん}な
\ruby{物}{もの}が
\ruby{{\GWI{u907a-k}}}{お}ちて
\ruby{居}{を}りました、』

といふ。

\ruby{一}{ひ}ㇳ
\ruby{目見}{め|み}てお
\ruby{龍}{りう}はそれを
\ruby{師匠}{し|しやう}に
\ruby{遞與}{わ|た}し、

『こりやあ
\ruby{傳}{でん}さんが
\ruby{{\GWI{u907a-k}}}{わす}れて
\ruby{行}{い}つたのでしやう。
あの
\ruby{人}{ひと}で
\ruby{無}{な}けりやあ
\ruby{此様}{こ|ん}なものを
\ruby{持}{も}ちさうな
\ruby{人}{ひと}はありませんから。
』

と
\ruby{云}{い}へば、お
\ruby{關}{せき}は
\ruby{受取}{うけ|と}つて
\ruby{指頭}{ゆび|さき}に
\ruby{弄}{もてあそ}び、

『あ\ninojiten{}
\ruby{然様}{さ|う}だよ、
\ruby{屹度彼}{きつ|と|あ}の
\ruby{男}{をとこ}のだよ。
\ruby{今日}{け|ふ}は
\ruby{妾}{わたし}も
\ruby{大變夙起}{たい|へん|はや|おき}を
\ruby{仕}{し}たし、
\ruby{汝}{おまへ}も
\ruby{{\GWI{u9060-k}}}{とほ}いところへ
\ruby{行}{い}つて
\ruby{來}{き}たので
\ruby{草臥}{くた|びれ}て
\ruby{居}{ゐ}るからつていふので
\ruby{{\換字{逐}}}{お}ひ
\ruby{立}{た}て\ninojiten{}やつたもんだから、
\ruby{慌}{あわ}て\ninojiten{}
\ruby{歸}{かへ}つて
\ruby{行}{い}つて
\ruby{{\GWI{u907a-k}}}{わす}れたんだらう。
\ruby{取}{と}り
\ruby{上}{あ}げて
\ruby{仕舞}{し|ま}つて
\ruby{{\GWI{u9063-k}}}{や}らうか
\ruby{知}{し}らん。
ハヽヽ、マア
\ruby{堪忍}{かん|にん}して
\ruby{{\GWI{u9063-k}}}{や}ると
\ruby{仕}{し}やう。
\ruby{何}{なん}でも
\ruby{彼}{あ}の
\ruby{男}{をとこ}は
\ruby{親類内}{しん|るい|うち}かなんぞに、
\ruby{玉}{たま}や
\ruby{石}{いし}の
\ruby{細工}{さい|く}をする
\ruby{家}{うち}かなんぞを
\ruby{有}{も}つて
\ruby{居}{ゐ}るんだよ。
\ruby{御覧}{ご|らん}よ、
\ruby{小}{ちひさ}いけれども
\ruby{此品}{こ|れ}だつて
\ruby{買}{か}つたら
\ruby{廉}{やす}くはなさ\ninojiten{}うなものだ\GWI{u1b098}。
』

と、
\ruby{一度}{ひと|たび}はお
\ruby{龍}{りう}に
\ruby{示}{しめ}して、さて
\ruby{火鉢}{ひ|ばち}の
\ruby{抽斗}{ひき|だし}に
\ruby{無{\換字{造}}作}{む|ざう|さ}に
\ruby{藏}{しま}ひたり。

『ハア
\ruby{左様}{さ|う}なんでしやうよ。
\ruby{兎}{うさぎ}を
\ruby{{\換字{呉}}}{く}れたんでも
\ruby{{\換字{分}}}{わ}かつて
\ruby{居}{ゐ}ますよ。
\ruby{屹度叔父}{きつ|と|を|ぢ}さんか
\ruby{何}{なに}かヾ
\ruby{玉屋}{たま|や}さんなんです\GWI{u1b098}。
』

『
\ruby{何様}{ど|う}も
\ruby{左様}{さ|う}らしいよ。
\ruby{妾}{わたし}も
\ruby{往日瑪瑙}{いつ|か|め|なう}の
\ruby{好}{い}い
\ruby{色}{いろ}の
\ruby{簪珠}{かんざし|だま}を
\ruby{貰}{もら}つたがね、
\ruby{汝}{おまへ}、
\ruby{兎}{うさぎ}なんぞぢや
\ruby{仕様}{し|やう}が
\ruby{無}{な}いぢや
\ruby{無}{な}いか。
\ruby{今度}{こん|ど}は
\ruby{寶石入}{い|し|い}りの
\ruby{指輪}{ゆび|わ}かなんか
\ruby{{\換字{強}}{\換字{請}}}{ね|だ}つて
\ruby{御}{お}
\ruby{{\GWI{u9063-k}}}{や}りナ。
\ruby{金剛石}{ダ|イ|ヤ}とでも
\ruby{云}{い}つたら
\ruby{二}{に}の
\ruby{足}{あし}を
\ruby{踏}{ふ}むか
\ruby{知}{し}らないが、サファイヤや
\ruby{真珠}{しん|じゆ}の
\ruby{位}{ぐらゐ}なら
\ruby{屹度二}{きつ|と|ふた}つ
\ruby{{\換字{返}}事}{へん|じ}で
\ruby{{\換字{悅}}}{よろこ}んで
\ruby{持}{も}つて
\ruby{來}{く}るよ。
\ruby{物}{もの}を
\ruby{取}{と}つて
\ruby{{\GWI{u9063-k}}}{や}るのも
\ruby{功徳}{く|どく}になるのだから
\ruby{關}{かま}やあ
\ruby{仕}{し}ない
\ruby{吹}{ふつ}かけて
\ruby{御覧}{ご|らん}、
\ruby{相槌}{あひ|づち}は
\ruby{妾}{わたし}が
\ruby{巧}{うま}く
\ruby{打}{う}つて
\ruby{上}{あ}げるから。
』

『あら
\ruby{{\換字{嫌}}}{いや}な
\ruby{御師匠}{お|し|よ}さん!。
\ruby{妾}{わたし}あ
\ruby{指輪}{ゆび|わ}なんか
\ruby{欲}{ほ}しかあ
\ruby{無}{な}いんですよ。
しかも
\ruby{傳}{でん}さんになんかあ
\ruby{貰}{もら}ひたたかあ
\ruby{有}{あ}りません。
』

『
\ruby{然様}{さ|う}かネエ。
\ruby{汝}{おまへ}はほんとに
\ruby{慾}{よく}に
\ruby{掛}{か}けちやあ
\ruby{氣}{き}が
\ruby{{\換字{弱}}}{よわ}いよ。
だが
\ruby{取}{と}つて
\ruby{{\GWI{u9063-k}}}{や}る
\ruby{方}{はう}が
\ruby{可}{い\ninojiten}ぢやあ
\ruby{無}{な}いか。
あの
\ruby{兎}{うさぎ}でも
\ruby{知}{し}れてるは\GWI{u1b098}、
\ruby{汝}{おまへ}の
\ruby{氣}{き}に
\ruby{入}{い}つたのを
\ruby{見}{み}て
\ruby[g]{何様}{どんな}なに
\ruby{嬉}{うれし}がつてるか
\ruby{知}{し}れや
\ruby{仕}{し}ないよ。
』

『だから
\ruby{妾}{わたし}あ
\ruby{厭}{いや}なんですよ。
その
\ruby{嬉}{うれし}がられるのが
\ruby{気障}{き|ざ}ぢや
\ruby{有}{あ}りませんか。
』

『ホイ
\ruby{大失敗}{おほ|しく|じり}だネ、ハヽハヽハヽ。
\ruby{指輪}{ゆび|わ}の
\ruby{談}{はなし}で
\ruby{想}{おも}ひ
\ruby{出}{だ}したが、
\ruby{先}{せん}に
\ruby{汝}{おまへ}があの
\ruby{何}{なん}に(
\ruby{源}{げん}を
\ruby{指}{さ}す)
\ruby{御貰}{お|もら}ひのは
\ruby{汝有}{おまへ|も}つておいで\ninojiten{}
\ruby{無}{な}いネエ。
\ruby{妾}{わたし}が
\ruby{見立}{み|た}て\ninojiten{}
\ruby{買}{か}はせたんだからまだ
\ruby{記}{おぼ}えて
\ruby{居}{ゐ}るが、
\ruby{汝彼品}{おまへ|あ|れ}は
\ruby{何様}{ど|う}か
\ruby{仕}{し}てお
\ruby{仕舞}{し|まひ}かエ。
』

『だって
\ruby{御師匠}{お|し|よ}さん、まだ
\ruby{妾}{わたし}が
\ruby{彼品}{あ|れ}を
\ruby{持}{も}つて
\ruby{居}{ゐ}やう
\ruby{譯}{わけ}は
\ruby{無}{な}からうぢや
\ruby{有}{あ}りませんか。
いよ〳〵
\ruby{不實}{ふ|じつ}な
\ruby{人}{ひと}だと
\ruby{思}{おも}ひつめた
\ruby{時}{とき}は、
\ruby[g]{口惜}{くやし}くつて
\ruby[g]{口惜}{くやし}くつて
\ruby{仕方}{し|かた}が
\ruby{無}{な}かつたんですもの!。
\ruby{宿}{と}めて
\ruby{貰}{もら}つて
\ruby{居}{ゐ}た
\ruby{薬研堀}{や|げん|ぼり}のおとうさん \------
\ruby{御師匠}{お|し|よ}さんは
\ruby{御知}{お|し}んなさらないが
\ruby{妾}{わたし}の
\ruby{仲好}{なか|よ}しの
\ruby{其}{そ}の
\ruby{家}{うち}を
\ruby{出}{で}て、をかアしな
\ruby{氣}{き}になつてふらふらと
\ruby{兩國橋}{りやう|ごく|ばし}の
\ruby{上}{うへ}を
\ruby{往}{い}つたり
\ruby{復}{かへ}つたりした
\ruby{其}{そ}の
\ruby[g]{擧句}{あげく}でした、ふいと
\ruby{意持}{こヽろ|もち}が
\ruby{變}{かは}つたんで
\ruby{指}{ゆび}から
\ruby{{\換字{脱}}}{はづ}して、
\ruby{大川}{おほ|かわ}の
\ruby{流}{なが}れの
\ruby{中}{なか}へ
\ruby{抛}{はふ}り
\ruby{込}{こ}んで
\ruby{仕舞}{し|ま}つたんですよ。
』

『ヘーエ、
\ruby{勿体無}{もつ|たい|な}い
\ruby{事}{こと}を
\ruby{御仕}{お|し}だつた\GWI{u1b098}ェ、マァ
\ruby{妾}{わたし}なら
\ruby{同}{おな}じ
\ruby{棄}{す}てるにもお
\ruby{金}{かね}に
\ruby{仕}{し}て
\ruby{棄}{す}てるものを。
だが
\ruby{鑄掛松}{ゐ|かけ|まつ}を
\ruby{色氣}{いろ|け}で
\ruby{行}{い}つたのは、
\ruby{一寸覗}{ちよ|つと|のぞ}いて
\ruby{見}{み}たいやうな
\ruby{幕}{まく}だつた\GWI{u1b098}。
』

『ホヽヽ、
\ruby{厭}{いや}ですよ。
たんと
\ruby{御嬲}{お|なぶ}りなさい、
\ruby{人}{ひと}の
\ruby{惡}{わる}い!。
\ruby{今}{いま}なら
\ruby{妾}{わたし}だつて \------。
』

『
\ruby{何様}{ど|う}
\ruby{御仕}{お|し}だェ、』

『
\ruby{御魚}{お|さかな}にやあ
\ruby{與}{や}らないで
\ruby{瞽女}{ご|ぜ}にでも
\ruby{與}{や}ります。
』

『
\ruby{{\換字{分}}別}{ふん|べつ}らしいけれど
\ruby{{\換字{猶}}且若}{やつ|ぱり|わか}い\GWI{u1b098}ェ。
ハヽヽ、
\ruby{瞽女}{ご|ぜ}が
\ruby{汝}{おまへ\ }
\ruby{狂}{くる}ひ
\ruby{浪}{なみ}の
\ruby{彫}{ほり}に
\ruby{小}{ちひさ}な
\ruby{寶石}{い|し}の
\ruby{散}{ち}らばつて
\ruby{居}{ゐ}る
\ruby{彼様}{あ|ん}な
\ruby{華麗}{はな|やか}な
\ruby{物}{もの}を
\ruby{指}{ゆび}に
\ruby{嵌}{は}めて
\ruby{何様}{ど|う}なるものかネ。
』

『ぢやあ
\ruby{御師匠}{お|し|よ}さんが
\ruby{妾}{わたし}だつたら
\ruby{何様}{ど|う}なさるの?。
』

お
\ruby{關}{せき}は
\ruby{我}{わ}が
\ruby{鼻}{はな}を
\ruby{指}{ゆび}さしながら、

『
\ruby{此處}{こ|ゝ}に
\ruby{居}{ゐ}る
\ruby{美麗}{き|れい}な
\ruby{可憐}{か|はゆ}らしい
\ruby{新{\換字{造}}}{しん|ぞ}に
\ruby{與}{や}つて
\ruby{{\換字{悅}}}{よろこ}ばせるはネ。
』

と
\ruby{云}{い}ひさして、ハヽハヽハヽと
\ruby{打笑}{うち|わら}へば、お
\ruby{龍}{りう}もホヽと
\ruby{笑}{わら}ひ
\ruby{出}{だ}し、
\ruby{臺所}{だい|どころ}の
\ruby{方}{かた}に
\ruby{{\換字{退}}}{しりぞ}きたるお
\ruby{熊}{くま}さへ
\ruby{貰}{もら}ひ
\ruby{笑}{わら}ひしたり。

『あ\ninojiten{}、
\ruby{笑}{わら}つたんで
\ruby{心持}{こヽろ|もち}が
\ruby{佳}{い}い。
さあお
\ruby{熊}{くま}や
\ruby{方方戸締}{はう|〴〵|と|じま}りを
\ruby{仕}{し}てお
\ruby{仕舞}{お|しま}ひ。
お
\ruby{龍}{りう}ちやんも
\ruby[g]{歸路}{かへり}に
\ruby{御百度}{お|ひや|くど}まで
\ruby{踏}{ふ}んで
\ruby{御}{お}
\ruby{{\換字{呉}}}{く}れぢやあ、ほんとに
\ruby{隨{\換字{分}}}{ずい|ぶん}おくたびれだらう。
』

\ruby{随意}{こヽろ|まかせ}に
\ruby{休}{やす}めといふ
\ruby{意}{こヽろ}は
\ruby{明}{あき}らかなれど、お
\ruby{龍}{りう}は
\ruby{眠}{ねむ}りたくも
\ruby{思}{おも}はぬ
\ruby{眼}{め}つきなり。

『
\ruby{足}{あし}は
\ruby{些}{ちつと}ばかり
\ruby{草臥}{くた|びれ}ましたけれど、
\ruby[g]{先刻}{さつき}お
\ruby{湯}{ゆ}に
\ruby{入}{はい}つたのでもう
\ruby{治}{なほ}りましたし、
\ruby{氣}{き}は
\ruby{疲勞}{くた|びれ}も
\ruby{何}{なに}も
\ruby{仕}{し}やあ
\ruby{仕}{し}ません。
』

『い\ninojiten{}ねえ
\ruby{若}{わか}い
\ruby{人}{ひと}は!。
\ruby{戀}{こひ}もいさくさも
\ruby{其}{そ}の
\ruby{威勢}{ゐ|せい}のある
\ruby{中}{うち}の
\ruby{花}{はな}なんだよ。
\ruby{妾}{わたし}なんざあ
\ruby{四}{よ}つ
\ruby{木}{ぎ}へ
\ruby{行}{い}かうもんなら
\ruby{二日位}{ふつ|か|ぐらゐ}は
\ruby{腰}{こし}が
\ruby{痛}{いた}いので、しよぼけて
\ruby{居}{ゐ}なくちやあならないんだよ。
』

『ホヽヽ
\ruby{虛言}{う|そ}ばつかり!。
まだ
\ruby{御師匠}{お|し|よ}さんはお
\ruby{若}{わか}いは。
そんな
\ruby{事}{こと}を
\ruby{仰}{おつし}あつても
\ruby{水々}{みづ|〳〵}として
\ruby{在}{い}らつしやるぢぁありませんか。
』

『オヤ
\ruby{汝}{おまへ}こそ
\ruby{人}{ひと}が
\ruby{惡}{わる}いよ、
\ruby{御調戯}{お|から|かひ}で
\ruby{無}{な}い。
い\ninojiten{}よ、
\ruby{何様}{ど|う}せ
\ruby{奢}{おご}らないから、ハヽハヽハヽ。
』

『でもほんたうですよ。
』

\ruby{渴}{かは}き
\ruby{氣味}{ぎ|み}にや
\ruby{身}{み}を
\ruby{伸}{の}ばして
\ruby{及腰}{および|ごし}に
\ruby{火鉢}{ひ|ばち}の
\ruby{横手}{よこ|て}の
\ruby{茶棚}{ちゃ|だな}より
\ruby{小}{ちひさ}き
\ruby{湯呑}{ゆ|のみ}を
\ruby{取}{と}り、
\ruby{鐵瓶}{てつ|びん}の
\ruby{湯}{ゆ}を
\ruby{注}{つ}ぎて
\ruby{心}{こヽろ}ゆたかに
\ruby{其}{それ}を
\ruby{冷}{さ}まして
\ruby{{\GWI{hkcs_m98f2}}}{の}めるお
\ruby{龍}{りう}を
\ruby{見}{み}れば、
\ruby{女}{をんな}には
\ruby{先}{ま}づ
\ruby{目}{め}につく
\ruby{髮}{かみ}の
\ruby{毛}{け}の
\ruby{漆}{うるし}と
\ruby{黑}{くろ}くて
\ruby{加之膨}{しか|も|ふつ}くりとしたる
\ruby{鬢}{びん}に、
\ruby{櫛}{くし}の
\ruby{齒}{は}の
\ruby{痕}{あと}あざやかに
\ruby{殘}{のこ}りて、
\ruby{肌理密}{き|め|こま}かに
\ruby{色白}{いろ|じろ}なる
\ruby{顏}{かほ}のほんのりと
\ruby{紅}{あか}きは、たヾ
\ruby{是}{これ}
\ruby{{\換字{清}}}{きよ}き
\ruby{芳野紙}{よし|の|がみ}の
\ruby{珊瑚}{さん|ご}を
\ruby{包}{つ\ninojiten}めるに
\ruby{異}{こと}ならず。
ざつに
\ruby{座}{すわ}つたる
\ruby{身}{み}の
\ruby{稍歪}{やヽ|ゆが}みて
\ruby{少}{すこ}し
\ruby{俯}{うつむ}いたるに、
\ruby{細}{ほつそ}りとしたる
\ruby{領頸}{\GWI{u1b001}り|くび}のいとゞしほらしく
\ruby{柔和}{にう|わ}に
\ruby{見}{み}えて、
\ruby{物}{もの}ごし
\ruby{恰好冴}{かつ|かう|さ}え〳〵と
\ruby{艶}{\GWI{u1b001}ん}なり。

お
\ruby{關}{せき}は
\ruby{見惚}{み|と}れたやうに
\ruby{良久}{やヽ|ひさ}しく
\ruby{見居}{み|ゐ}つ。

『そりやまあ
\ruby{何様}{ど|う}でも
\ruby{可}{い\ninojiten{}}としたところで、
\ruby{矢張}{やつ|ぱ}りお
\ruby{前}{まへ}にやあ
\ruby{此頃}{この|ごろ}に
\ruby{御馳走}{ご|ち|そう}を
\ruby{仕無}{し|な}くちやあならない。
ほんとに
\ruby{汝}{おまへ}の
\ruby{氣合}{き|あひ}の
\ruby{好}{い}いのには
\ruby{感心}{かん|しん}しちまふよ。
\ruby[g]{歸路}{かへり}には
\ruby[g]{馴染}{なじみ}も
\ruby{無}{な}いお
\ruby{五十}{い|そ}のためにお
\ruby{百度}{ひやく|ど}まで
\ruby{踏}{ふ}んで
\ruby{{\換字{呉}}}{く}れるなんて、
\ruby{何様}{ど|う}すれば
\ruby{其様}{そ|ん}なに
\ruby{優}{やさ}しい
\ruby{氣}{き}になつて、しかも
\ruby{俠氣}{をとこ|ぎ}な
\ruby{事}{こと}が
\ruby{出來}{で|き}るだらう。
\ruby{妾}{わたし}や
\ruby{全然}{すつ|かり}お
\ruby{前}{まへ}にやあ
\ruby{惚}{ほ}れつ
\ruby{仕舞}{ち|ま}つたよ。
お
\ruby{前}{まへ}さへ
\ruby{吾家}{う|ち}に
\ruby{居}{ゐ}てお
\ruby{{\換字{呉}}}{く}れなら、あんなお
\ruby{五十}{い|そ}なんか
\ruby{何様}{ど|う}なつたからつて
\ruby{關}{かま}やあ
\ruby{仕無}{し|な}いよ。
』

『あらマア
\ruby{飛}{と}んでも
\ruby{無}{な}い
\ruby{酷}{ひど}い
\ruby{事}{こと}を!。
お
\ruby{師匠}{し|よ}さんの
\ruby{左様仰}{さ|う|おつし}やるのを
\ruby{本當}{ほん|たう}にしたところで、
\ruby{五十子}{い|そ|こ}さんがお
\ruby{惡}{わる}く
\ruby{御}{お}なんなさらうもんなら
\ruby[g]{水野}{みづの}さんていふ
\ruby{方}{かた}が、
\ruby{何様}{どん|な}に
\ruby{御騒}{お|さわ}ぎなさるか
\ruby{知}{し}れやしません!。
』

『
\ruby{騒}{さわ}いだつて
\ruby{可}{い\ninojiten{}}やね、
\ruby{騒}{さわ}がして
\ruby{置}{おき}やあ。
』

『まだ
\ruby{詳}{くは}しい
\ruby{御話}{お|はなし}を
\ruby{伺}{うかゞ}ひませんが、
\ruby{一體}{いつ|たい}
\ruby[g]{水野}{みづの}さんていふ
\ruby{方}{かた}は
\ruby{何様}{ど|う}いふ
\ruby{方}{かた}なの?。
』

『オヤ〳〵をかしいよお
\ruby{龍}{りう}ちやんは。
\ruby{今日}{け|ふ}お
\ruby{晝{\GWI{u904e-k}}}{ひる|すぎ}に
\ruby{家}{いへ}へ
\ruby{歸}{かへ}つて
\ruby{來}{き}てから、これで
\ruby{丁度}{ちや|うど}
\ruby[g]{水野}{みづの}の
\ruby{事}{こと}を
\ruby{三度御聞}{さん|ど|お|きヽ}だよ。
ハヽヽまさか
\ruby{汝}{おまへ}のやうに
\ruby{{\換字{分}}}{わか}つた
\ruby{人}{ひと}が、
\ruby{彼様}{あ|ん}な
\ruby{唐變木}{たう|へん|ぼく}に
\ruby{何様}{ど|う}か
\ruby{御爲}{お|し}だとも
\ruby{思}{おも}やあ
\ruby{仕}{し}ないが\GWI{u1b098}。
よつぽど
\ruby{氣}{き}になるやうな
\ruby{變}{へん}な
\ruby{顏}{かほ}でも
\ruby{仕}{し}て
\ruby{居}{ゐ}たのかェ。
\ruby{彼}{あり}や
\ruby{何}{なん}でも
\ruby{有}{あ}りや
\ruby{仕}{し}ないのさ。
たゞ
\ruby[g]{彼村}{あすこ}の
\ruby{學校}{がく|かう}の
\ruby{{\換字{教}}師}{けう|し}でもつて、
\ruby{{\換字{平}}}{ひら}つたく
\ruby{云}{い}やあお
\ruby{五十}{い|そ}に
\ruby{惚}{ほ}れてるといふだけの
\ruby{鈍痴氣}{どん|ち|き}なんだよ。
』

『だつて
\ruby{其}{そん}なら
\ruby{妾}{わたし}が
\ruby{御師匠}{お|し|よ}さんの
\ruby{御使}{おつ|かひ}に、わざ〳〵
\ruby{彼}{あ}の
\ruby{人}{ひと}のところへ
\ruby{行}{い}かなくつてもぢや
\ruby{有}{あ}りませんか。
』

『そりやお
\ruby{五十}{い|そ}の
\ruby{事}{こと}の
\ruby[g]{關係}{つゞき}から\GWI{u1b098}、
\ruby{妾}{わたし}も
\ruby{困究}{こ|ま}つた
\ruby{時}{とき}に
\ruby{彼男}{あの|をとこ}に
\ruby{融{\換字{通}}}{ゆう|づう}を
\ruby{頼}{たの}んだ
\ruby{事}{こと}もあるし、
\ruby[g]{今度}{こんど}も
\ruby{全然}{すつ|かり}お
\ruby{五十}{い|そ}が
\ruby{世話}{せ|わ}になつて
\ruby{居}{ゐ}るからさ。
』

『ぢやあ
\ruby{矢張}{やつ|ぱ}り
\ruby[g]{畢竟}{つまり}は
\ruby{五十子}{い|そ|こ}さんと
\ruby{一{\換字{所}}}{いつ|しよ}になる
\ruby{譯}{わけ}の
\ruby{方}{かた}ぢやありませんか。
\ruby{{\GWI{u9053-k}}理}{だう|り}で
\ruby{心}{しん}から
\ruby{底}{そこ}から
\ruby{御病人}{ご|びやう|にん}を
\ruby{大切}{たい|せつ}に
\ruby{思}{おも}つて
\ruby{居}{ゐ}らつしやるやうに
\ruby{見}{み}えましたよ。
ほんとに
\ruby{五十子}{い|そ|こ}さんは
\ruby{御幸福}{お|しあ|はせ}な
\ruby{事}{こと}!、あんな
\ruby{頼}{たの}もしさうな
\ruby{方}{かた}に
\ruby{御思}{お|おも}はれなすつて!。
』

『ところがお
\ruby{前}{まへ}、いくら
\ruby{彼}{あ}の
\ruby{男}{をとこ}が
\ruby{思}{おも}つても、
\ruby{妾}{わたし}の
\ruby{云}{い}ふ
\ruby{事}{こと}さへ
\ruby{聽}{き}かないやうな、ヘチ
\ruby{頑固}{ぐわん|こ}のお
\ruby{五十}{い|そ}の
\ruby{事}{こと}だから、
\ruby{{\換字{嫌}}}{きら}つて
\ruby{{\換字{嫌}}}{きら}ひぬいて
\ruby{關}{かま}はないのだよ。
\ruby{彼}{あ}の
\ruby{男}{をとこ}の
\ruby{思}{おもひ}なんぞは
\ruby[g]{玻瓈}{がらす}に
\ruby{書}{か}く
\ruby{字}{じ}で、
\ruby{以上經}{い|じやう|たつ}ても
\ruby{{\換字{通}}}{とほ}りつこは
\ruby{無}{な}いのさ。
』

『でも
\ruby{御師匠}{お|し|よ}さんは
\ruby{{\換字{終}}}{しまひ}には
\ruby{彼}{あ}の
\ruby{人}{ひと}を
\ruby{御婿}{お|むこ}さんにと
\ruby{思}{おも}つてらつしやるでしやう。
』

『だってお
\ruby{五十}{い|そ}が
\ruby{妾}{わたし}の
\ruby{云}{い}ふ
\ruby{事}{こと}なんか
\ruby{聽}{き}くんぢ
\ruby{無}{な}いから
\ruby{仕方}{し|かた}が
\ruby{無}{な}いやね。
\ruby{妾}{わたし}あ
\ruby{打棄}{うつ|ちや}つて
\ruby{置}{お}いて
\ruby{關}{かま}やあ
\ruby{仕無}{し|な}いのさ。
』

『あら
\ruby{憫然}{かはい|さう}に、それぢやあ
\ruby{彼}{あ}の
\ruby{人}{ひと}の
\ruby{立場}{たち|ば}が
\ruby{無}{な}いぢやあ
\ruby{有}{あ}りませんか。
』

『だから
\ruby{唐變木}{たう|へん|ぼく}で
\ruby{鈍痴氣}{どん|ち|き}だといふんだア\GWI{u1b098}。
』

『なんですつて\GWI{u2048}、マア!。
』

\ruby{優}{やさ}しき
\ruby{姿}{すがた}は
\ruby{其儘}{その|ま\ninojiten{}}に、
\ruby{身動}{み|じろ}きは
\ruby{一寸}{いつ|すん}もせざりしが、
\ruby{愛嬌}{あい|けう}こぼる\ninojiten{}
\ruby{面}{おもて}ながら、じろりと
\ruby{斜}{な\ninojiten{}め}に
\ruby{上睨}{う\GWI{u1b001}|にら}みして、お
\ruby{關}{せき}を
\ruby{見}{み}やりたるお
\ruby{龍}{りう}の
\ruby{眼}{め}には、
\ruby{瞋}{いか}るか
\ruby{恨}{うら}むか
\ruby{蔑視}{さげ|す}むか、
\ruby{怪}{あや}しき
\ruby{一種}{いつ|しゆ}の
\ruby{氣味合籠}{き|み|あひ|こも}りて、
\ruby{花}{はな}の
\ruby{樹蔭}{こ|かげ}に
\ruby{蛇}{へび}の
\ruby{出}{い}でたる
\ruby{其狀}{そ|れ}にも
\ruby{似}{に}たる
\ruby{風{\換字{情}}}{ふ|ぜい}を
\ruby{見}{み}せたり。


\Entry{其三十四}

\ruby{下物}{さか|な}は
\ruby{論無}{ろん|な}し、たゞ
\ruby{鮮}{あざら}けきを
\ruby{用}{もち}ゐ、
\ruby{酒}{さけ}は
\ruby{定例}{さだ|め}あつて、
\ruby{必}{かなら}ず
\ruby{醇}{じゆん}なるを
\ruby{{\換字{酌}}}{く}む、
\ruby{島木}{しま|き}が
\ruby{性{\換字{情}}}{こゝ|ろ}
\ruby{見}{み}ゆる
\ruby{待{\換字{遇}}}{もて|なし}に、
\ruby{日方}{ひ|かた}は
\ruby{既}{はや}
\ruby{醉}{よ}ひて
\ruby{面}{おもて}を
\ruby{染}{そ}め、
\ruby{大胡座}{おほ|あぐ|ら}かいて
\ruby{座}{すわ}れる、
\ruby{軍服}{ぐん|ぷく}の
\ruby{怒}{いか}れる
\ruby{肩}{かた}、
\ruby{五{\換字{分}}刈}{ご|ぶ|がり}の
\ruby{大}{おほい}なる
\ruby{頭}{あたま}、
\ruby{姿勢}{すが|た}はまだ
\ruby{崩}{くづ}さず
\ruby{傲然}{がう|ぜん}として、
\ruby{葡萄酒}{ぶ|だう|しゆ}の
\ruby{盞}{さん}を
\ruby{手}{て}にしながら、
\ruby{親}{した}しきが
\ruby{中}{なか}の
\ruby{打解}{うち|と}け
\ruby{話}{ばなし}におのづから
\ruby{催}{もよほ}さる〻% 本来は一の字点「ゝ」平仮名繰返し記号
\ruby{歡}{よろこ}びの
\ruby{色}{いろ}を
\ruby{{\換字{浮}}}{うか}べて、

『アヽ
\ruby{快}{い}い
\ruby{心持}{こゝろ|もち}だ、
\ruby{佳}{い}い
\ruby{酒}{さけ}だ。
いつも
\ruby{葡萄酒}{ぶ|だう|しゆ}とは
\ruby{贅澤}{ぜい|たく}な
\ruby{奴}{やつ}だ。
\ruby{羽{\換字{勝}}}{は|がち}が
\ruby{斷}{ことわ}つて
\ruby{來}{き}たのは
\ruby{殘念}{ざん|ねん}だが、
\ruby{酒}{さけ}は
\ruby{好}{よ}し、
\ruby{主人}{しゆ|じん}の
\ruby{汝}{きさま}も
\ruby{好}{い}い
\ruby{男兒}{をと|こ}だし、
\ruby{客}{きやく}の
\ruby{乃公}{お|れ}も
\ruby{大{\換字{丈}}夫}{だい|ぢやう|ぶ}だし、
\ruby{談話}{はな|し}が
\ruby{面白}{おも|しろ}いので
\ruby{小氣味}{こ|き|み}よく
\ruby{醉}{よ}つた。
』

と
\ruby{云}{い}ひさして
\ruby{滿足}{まん|ぞく}げに
\ruby{仰飮}{あ|ふ}ぎ
\ruby{盡}{つく}せば、
\ruby{島木}{しま|き}は
\ruby{例}{れい}の
\ruby{布袋顏}{ほ|てい|がほ}して
\ruby{笑}{わら}ひ、

『ハヽヽ、
\ruby{直}{すぐ}と
\ruby{何}{なん}でも
\ruby{自{\換字{分}}}{じ|ぶん}の
\ruby{{\換字{道}}}{みち}に
\ruby{牽{\換字{強}}}{こじ|つ}けるナ。
イヤ
\ruby{時}{とき}の
\ruby{相塲}{さう|ば}ぢやあ
\ruby{無}{な}い、
\ruby{全}{まつた}くの
\ruby{事}{こと}だ。
\ruby{全}{まつた}く
\ruby{汝}{きさま}は
\ruby{好}{い}い
\ruby{男兒}{をと|こ}だ、
\ruby{{\換字{所}}謂}{いは|ゆる}
\ruby{好漢}{かう|かん}だナ、
\ruby{快男兒}{くわい|だん|じ}だナ。
』

『ハヽ、
\ruby{大層}{たい|そう}
\ruby{風向}{かざ|むき}きが
\ruby{好}{い}いが
\ruby{奢}{おご}らねえぜ。
\ruby{何}{なん}でまた
\ruby{其樣}{そ|う}
\ruby{急}{きう}に
\ruby{値}{ね}が
\ruby{上}{あが}つたのだ。
』

『
\ruby{羽{\換字{勝}}}{は|がち}から
\ruby{聞}{き}いて
\ruby{皆}{みんな}
\ruby{知}{し}つたぞ。
\ruby{能}{よ}く
\ruby{汝}{きさま}ア
\ruby{彼}{あ}の
\ruby{馬鹿野郎}{ば|か|や|らう}の
\ruby{水野}{みづ|の}を、
\ruby{自{\換字{分}}}{じ|ぶん}の
\ruby{危}{あぶ}なかつた
\ruby{間際}{ま|ぎは}で
\ruby{世話}{せ|わ}を
\ruby{仕}{し}て
\ruby{{\換字{遣}}}{や}つたナア。
\ruby{流石}{さす|が}に
\ruby{島木}{しま|き}は
\ruby{島木}{しま|き}だ、
\ruby{好}{い}い
\ruby{氣象}{きし|やう}だ、と
\ruby{眞面目}{ま|じ|め}に
\ruby{感激}{かん|げき}して
\ruby{羽{\換字{勝}}}{は|がち}が
\ruby{話}{はな}したぞ。
』

『ハヽヽ、それで
\ruby{汝}{きさま}ア
\ruby{萬五郎}{まん|ご|らう}に
\ruby{惚}{ほ}れたか。
』

『ン、
\ruby{惚}{ほ}れたナア、ハヽヽ。
\ruby{日方八郎}{ひ|かた|はち|らう}も
\ruby{大}{おほき}に
\ruby{惚}{ほ}れ
\ruby{{\換字{込}}}{こ}んだぞ。
』

『
\ruby{{\換字{嫌}}}{いや}な
\ruby{野郎}{や|らう}だナア、
\ruby{好}{す}かねえ
\ruby{奴}{やつ}だ。
\ruby{何程惚}{いく|ら|ほ}れやがつても
\ruby{振}{ふ}りつけて
\ruby{{\換字{遣}}}{や}るぞ。
』

『
\ruby{何故}{な|ぜ}?。
』

『
\ruby{惚}{ほ}れやうが
\ruby{一體}{いつ|たい}
\ruby{氣}{き}に
\ruby{食}{く}はねえから。
』

『フーン、そりやあ
\ruby{{\換字{又}}}{また}
\ruby{何}{なん}で。
』

『それが
\ruby{{\換字{分}}}{わか}らねえかえ、
\ruby{仕方}{し|かた}が
\ruby{無}{ね}えナア。
\ruby{後學}{こう|がく}のために
\ruby{記}{おぼ}えて
\ruby{置}{お}きねえ、
\ruby{惚}{ほ}れるのに
\ruby{理由}{いは|れ}があるやうぢやあ
\ruby{眞物}{ほん|もの}ぢやあ
\ruby{無}{ね}えんだ。
\ruby{同}{おな}じ
\ruby{此}{こ}の
\ruby{萬五郎}{まん|ご|らう}に
\ruby{惚}{ほ}れるならナア…………。
』

『ウン。
』

『
\ruby{乃公}{お|れ}が
\ruby{惡}{わる}い
\ruby{事}{こと}を
\ruby{爲盡}{し|つく}して、
\ruby{誰}{たれ}にも
\ruby{彼}{かれ}にも
\ruby{見放}{み|はな}されてナ、
\ruby{溝}{どぶ}ん
\ruby{中}{なか}へでも
\ruby{蹴{\換字{込}}}{け|こ}まれたやうな
\ruby{時}{とき}、
\ruby{萬}{まん}ちやん
\ruby{萬}{まん}ちやんツて
\ruby{云}{い}つて
\ruby{吳}{く}れろヤイ。
\ruby{左樣}{さ|う}したら
\ruby{其時}{その|とき}ア
\ruby{此}{こ}の
\ruby{萬}{まん}ちやんも、
\ruby{些少}{ちつ|た}ア
\ruby{惚}{ほ}れ
\ruby{{\換字{返}}}{かへ}して
\ruby{{\換字{遣}}}{や}るめえもんでも
\ruby{無}{ね}えんだ。
』

『アツハヽハヽ、
\ruby{甚}{ひど}い
\ruby{氣焰}{き|\換字{𛀁}ん}だナ、
\ruby{怪人}{くわい|じん}の
\ruby{怪語}{くわい|ご}だ。
\ruby{皮肉}{ひ|にく}も
\ruby{其}{それ}までになると
\ruby{愛嬌}{あい|けう}が
\ruby{出}{で}て
\ruby{面白}{おも|しろ}い。
アヽ
\ruby{{\換字{愉}}快}{ゆ|くわい}だ
\ruby{大笑}{おほ|わら}ひに
\ruby{笑}{わら}つたので
\ruby{馬鹿}{ば|か}に
\ruby{醉}{よ}つた。
\ruby{久}{ひさ}しぶりで
\ruby{一}{ひと}ツ
\ruby{朗吟}{ろう|ぎん}をやるぞ。
』

『
\ruby{宣}{よ}からう。
\ruby{長}{なが}い
\ruby{事}{こと}
\ruby{汝}{きさま}の
\ruby{怒鳴}{ど|な}るのも
\ruby{聞}{き}かなかつたナア。
』

『
\ruby{蒲海}{ほ|かい}の------
\ruby{曉}{あかつき}の------
\ruby{霜}{しも}は------、
\ruby{馬}{うま}の------
\ruby{尾}{を}に------
\ruby{凝}{こ}り------、
\ruby{葱山}{そう|ざん}の------
\ruby{夜}{よる}の------
\ruby{{\換字{雪}}}{ゆき}は------、
\ruby{旌}{はた}の------
\ruby{竿}{さを}を------
\ruby{撲}{う}つ------。
エースト。
』

『
\ruby{鯨}{くじら}が
\ruby{鳴}{な}くやうな
\ruby{馬鹿聲}{ば|か|ごゑ}だナア、
\ruby{障子}{しやう|じ}が
\ruby{破}{やぶ}けるからもう
\ruby{堪忍}{か|に}して
\ruby{吳}{く}れ、
\ruby{此邊}{こゝ|いら}の
\ruby{奴}{やつ}あ
\ruby{目}{め}を
\ruby{{\換字{廻}}}{まは}さあ。
しかも
\ruby{唐人}{たう|じん}の
\ruby{囈語}{ね|ごと}で
\ruby{毫末}{ちつ|と}も
\ruby{{\換字{分}}}{わか}ら
\ruby{無}{ね}え。
\ruby{戰}{いくさ}の
\ruby{詩}{し}の
\ruby{句}{く}かえ。
』

『ウン
\ruby{其樣}{そ|ん}なもんだ。
』

『
\ruby{有}{あ}るかい?いよ〳〵、
\ruby{戰爭}{どん|ちやん}は。
』

『そんな
\ruby{事}{こと}は
\ruby{乃公達}{お|れ|たち}よりは
\ruby{汝等}{きさま|ら}
\ruby{相塲師}{さう|ば|し}なんぞの
\ruby{方}{はう}が
\ruby{却}{かへ}つて
\ruby{知}{し}つて
\ruby{居}{ゐ}るといふことだぞ。
』

\ruby{如是}{か|く}
\ruby{云}{い}ひ
\ruby{{\換字{終}}}{をは}りし
\ruby{時}{とき}
\ruby{日方}{ひ|かた}は
\ruby{忽}{たちま}ち
\ruby{嚴然}{げん|ぜん}たる
\ruby{面色}{おも|て}になりて、

『いかんナア、
\ruby{此樣}{こ|ん}な
\ruby{世態}{せ|たい}では!。
\ruby{實}{じつ}に
\ruby{慨歎}{がい|たん}に
\ruby{堪}{た}へん。
』

と
\ruby{正}{まさ}しく
\ruby{島木}{しま|き}には
\ruby{語}{かた}るならで
\ruby{獨}{ひと}り
\ruby{歎}{たん}ぜしが、
\ruby{忽地}{たち|まち}にして
\ruby{氣}{き}をかへて、

『
\ruby{{\換字{丈}}夫}{ぢやう|ぶ}------
\ruby{誓}{ちか}つて
\ruby{國}{くに}に
\ruby{許}{ゆる}す、
\ruby{憤惋}{ふん|\換字{𛀁}ん}------
\ruby{復}{また}
\ruby{何}{なに}か
\ruby{有}{あ}らん、だ。
\ruby{少尉}{しよう|ゐ}やそこらで
\ruby{物}{もの}を
\ruby{思}{おも}ふナア
\ruby{生意氣}{なま|い|き}なんなのだ。
』

と
\ruby{自}{みづか}ら
\ruby{寛}{ゆる}くして
\ruby{打笑}{うち|わら}ひたり。

『
\ruby{時}{とき}に
\ruby{島木}{しま|き}!。
\ruby{何樣}{ど|う}だ
\ruby{今}{いま}から
\ruby{一緖}{いつ|しよ}に
\ruby{水野}{みづ|の}を
\ruby{訪}{と}はんか。
\ruby{實}{じつ}は
\ruby{羽{\換字{勝}}}{は|がち}が
\ruby{來}{き}たら
\ruby{君}{きみ}を
\ruby{誘}{さそ}つて、
\ruby{三人}{さん|にん}で
\ruby{{\換字{尋}}}{たづ}ねて
\ruby{{\換字{遣}}}{や}らうと
\ruby{思}{おも}つて
\ruby{居}{ゐ}たんだが。
』

『フーム、
\ruby{萬一}{ひよ|つと}すると
\ruby{汝}{きさま}
\ruby{出征}{でか|ける}るのかナ。
』

『イヤまだ
\ruby{其}{それ}は
\ruby{實際}{じつ|さい}
\ruby{{\換字{分}}}{わか}らんが、
\ruby{出}{で}るやうになるにしても
\ruby{出}{で}ないにしても、
\ruby{此頃}{この|ごろ}の
\ruby{水野}{みづ|の}の
\ruby{面色}{かほ|つき}も
\ruby{見}{み}て
\ruby{{\換字{遣}}}{や}りたいし、
\ruby{少}{すこ}し
\ruby{話}{はなし}を
\ruby{仕度}{し|た}いと
\ruby{思}{おも}ふ
\ruby{事}{こと}も
\ruby{有}{あ}るから。
』

『ぢやあ
\ruby{汝}{きさま}の
\ruby{剛直}{がう|ちよく}な
\ruby{其}{そ}の
\ruby{氣}{き}に
\ruby{任}{まか}せて
\ruby{手{\換字{強}}}{て|ごは}い
\ruby{意見}{い|けん}を
\ruby{仕}{し}やうと
\ruby{云}{い}ふんだナ。
』

『
\ruby{勿論}{もち|ろん}だ。
\ruby{戀愛}{れん|あい}だなんぞといふ
\ruby{下}{くだ}らない
\ruby{事}{こと}に、
\ruby{可惜}{あ|たら}
\ruby{水野}{みづ|の}を
\ruby{沈}{しづ}ませて
\ruby{置}{お}いて、
\ruby{知}{し}らん
\ruby{顏}{かほ}を
\ruby{仕}{し}て
\ruby{居}{ゐ}ては
\ruby{友{\換字{道}}}{み|ち}が
\ruby{立}{た}たんと
\ruby{思}{おも}ふ。
\ruby{諫}{いさ}めて
\ruby{諫}{いさ}めて
\ruby{彼}{あ}の
\ruby{水野}{みづ|の}を、
\ruby{舊}{もと}の
\ruby{水野}{みづ|の}に
\ruby{復}{かへ}らせるつもりだ。
』

『そりやあ
\ruby{汝}{きさま}、
\ruby{人{\換字{情}}}{じや|う}は
\ruby{厚}{あつ}い
\ruby{行爲}{しう|ち}だが、
\ruby{智慧}{ち|ゑ}は
\ruby{足}{た}らねえ
\ruby{事}{こと}だぜ!。
』

『ナニ?。
』

『マア
\ruby{下}{くだ}ら
\ruby{無}{ね}えから
\ruby{止}{や}めたら
\ruby{宜}{よ}からう!。
』

『なんだと。
』

\Entry{其三十五}

\ruby{島木}{しま|き}は
\ruby{莞爾}{にこ|り}と
\ruby{笑}{わら}ひながら
\ruby{酒}{さけ}を
\ruby{注}{つ}ぎやりつ、

『また
\ruby{直}{ぢき}に
\ruby{左樣}{さ|う}ムキになつて
\ruby{突掛}{つ〻|か〻}つて% 本来は一の字点「ゝ」平仮名繰返し記号
\ruby{來}{く}るよ。
いくら
\ruby{酒}{さけ}の
\ruby{氣}{き}が
あるからといつて
\ruby{野暮}{や|ぼ}な
\ruby{男}{をとこ}だナ。
』

『
\ruby{何}{なに}も
\ruby{决}{けつ}して
\ruby{怒}{おこ}るのぢやあ
\ruby{無}{な}い。
しかし
\ruby{乃公}{お|れ}が
\ruby{爲}{し}やうと
\ruby{思}{おも}ふことを
\ruby{下}{くだ}らないとは
\ruby{何}{なん}だ。
\ruby{智慧}{ち|ゑ}が
\ruby{足}{た}りても
\ruby{足}{た}らなくつても
\ruby{其}{それ}は
\ruby{仕方}{し|かた}が
\ruby{無}{な}い。
\ruby{默}{だま}つて
\ruby{知}{し}らん
\ruby{顏}{かほ}を
\ruby{仕}{し}ては
\ruby{居}{を}られんから
\ruby{{\換字{尋}}}{たづ}ねやうといふのだ。
\ruby{其}{それ}をたゞ
\ruby{一槪}{いち|がい}に
\ruby{止}{や}めたら
\ruby{宜}{よ}からうと
\ruby{云}{い}はれては
\ruby{面白}{おも|しろ}く
\ruby{無}{な}い。
\ruby{何}{なに}が
\ruby{下}{くだ}らない?、
\ruby{何故}{な|ぜ}
\ruby{智慧}{ち|ゑ}が
\ruby{足}{た}らん?。
』

『
\ruby{何故}{な|ぜ}と
\ruby{云}{いつ}て、
\ruby{考}{かんが}へて
\ruby{見}{み}りやあ
\ruby{{\換字{分}}}{わか}る
\ruby{事}{こと}だ。
』

『いや
\ruby{{\換字{分}}}{わか}らん
\ruby{{\換字{分}}}{わか}らん、
\ruby{考}{かんが}へて
\ruby{見}{み}ても
\ruby{{\換字{分}}}{わか}らんに
\ruby{定}{きま}つて
\ruby{居}{ゐ}る。
よし
\ruby{乃公}{お|れ}の
\ruby{爲}{す}ることが
\ruby{智慧}{ち|ゑ}が
\ruby{足}{た}らんにしろ、
\ruby{智慧}{ち|ゑ}が
\ruby{足}{た}らんために
\ruby{其効}{その|かう}が
\ruby{無}{な}いのならば、
\ruby{汝}{きさま}が
\ruby{智慧}{ち|ゑ}を
\ruby{添}{そ}へて
\ruby{効}{かう}があるやうにして
\ruby{吳}{く}れても
\ruby{宜}{い}い
\ruby{譯}{わけ}では
\ruby{無}{な}いか。
\ruby{水野}{みづ|の}は
\ruby{乃公}{お|れ}ばかりの
\ruby{朋友}{ほう|いう}では
\ruby{無}{な}い、
\ruby{汝}{きさま}にも
\ruby{矢張}{や|はり}
\ruby{朋友}{ほう|いう}では
\ruby{無}{な}いか。
\ruby{朋友}{ほう|いう}の
\ruby{{\換字{道}}}{みち}は
\ruby{何樣}{ど|う}するのが
\ruby{正當}{ほん|たう}だ。
\ruby{互}{たがひ}に
\ruby{氣}{き}に
\ruby{入}{い}るやうにばかり
\ruby{仕}{し}て
\ruby{居}{ゐ}ればそれで
\ruby{可}{いゝ}といふのか、そんな
\ruby{理窟}{り|くつ}がどこにあるものだ。% ここは「理(屈)」ではない
\ruby{勿論}{もち|ろん}
\ruby{朋友}{ほう|いう}の
\ruby{幇}{たす}け
\ruby{合}{あ}ふのは
\ruby{知}{し}れた
\ruby{事}{こと}だが、
\ruby{劍{\換字{術}}}{けん|じゆつ}を
\ruby{{\換字{習}}}{なら}へば
\ruby{竹刀}{しな|ひ}に
\ruby{會釋}{ゑ|しやく}
\ruby{無}{な}く
\ruby{引撲}{ひつ|ぱた}き
\ruby{合}{あ}ふのが
\ruby{朋友}{とも|だち}の
\ruby{眞實}{ま|こと}だ、
\ruby{碁}{ご}の
\ruby{一目}{いち|もく}、
\ruby{競射}{きよう|しや}の
\ruby{一點}{いつ|てん}に
\ruby{齒咬}{は|が}みを
\ruby{仕}{し}て
\ruby{爭}{あらそ}ひ
\ruby{合}{あ}ふのも
\ruby{朋友}{とも|だち}の
\ruby{面白味}{おも|しろ|み}だ。
だから
\ruby{欺}{あざむ}かぬ
\ruby{心}{こゝろ}も
\ruby{無}{な}くちやならん。
\ruby{競}{せ}り
\ruby{合}{あ}ふ
\ruby{氣}{き}も
\ruby{無}{な}くちやならん。
まして
\ruby{眼}{め}に
\ruby{餘}{あま}つたり
\ruby{腑}{ふ}に
\ruby{落}{お}ち
\ruby{無}{な}かつたりする
\ruby{事}{こと}があれば、
\ruby{忠告}{ちう|こく}も
\ruby{爲}{し}やうし、
\ruby{爭}{あらそ}ひも
\ruby{爲}{し}やうし、
\ruby{齒}{は}に
\ruby{衣被}{きぬ|き}せず
\ruby{詈}{ののし}り
\ruby{詈}{ののし}らうとも、
\ruby{互}{たがひ}に
\ruby{他人}{ひ|と}の
\ruby{物笑}{もの|わら}ひには、させぬやうに、
\ruby{{\換字{又}}}{また}ならぬやうにと、
\ruby{男兒}{をと|こ}を
\ruby{磨}{みが}きあふのが
\ruby{朋友}{とも|だち}の
\ruby{甲{\換字{斐}}}{か|ひ}では
\ruby{無}{な}いか。
それを
\ruby{何}{なん}だ
\ruby{汝}{きさま}の
\ruby{此頃}{この|ごろ}の
\ruby{仕方}{し|かた}は。
たゞ
\ruby{水野}{みづ|の}の
\ruby{云}{い}ふ
\ruby{{\換字{通}}}{とほ}りにばかり
\ruby{仕}{し}て
\ruby{與}{や}つて
\ruby{居}{ゐ}る。
そりやあ
\ruby{汝}{きさま}の
\ruby{俠氣}{をとこ|ぎ}の
\ruby{振舞}{ふる|まひ}は
\ruby{乃公}{お|れ}も
\ruby{感謝}{かん|しや}して
\ruby{居}{ゐ}るが、それほどに
\ruby{水野}{みづ|の}の
\ruby{爲}{ため}を
\ruby{思}{おも}ふなら、
\ruby{何故}{な|ぜ}
\ruby{一歩}{いつ|ぽ}
\ruby{{\換字{進}}}{すゝ}んで
\ruby{諫}{いさ}めては
\ruby{{\換字{遣}}}{や}らんか、
\ruby{彼}{あ}の
\ruby{男}{をとこ}の
\ruby{{\換字{迷}}}{まよひ}を
\ruby{解}{と}いては
\ruby{{\換字{遣}}}{や}らんか、
\ruby{諫}{いさ}めても
\ruby{聽}{き}かずば
\ruby{何故}{な|ぜ}
\ruby{爭}{あらそ}つては
\ruby{{\換字{遣}}}{や}らん。
\ruby{士爭友}{し|さう|いう}あれば
\ruby{令名}{れい|めい}に
\ruby{離}{はな}れずといふ
\ruby{孝經}{かう|きやう}の
\ruby{語}{ご}を、たとひ
\ruby{其語}{その|ことば}を
\ruby{知}{し}らんでも
\ruby{其}{そ}の
\ruby{理合}{り|あひ}に
\ruby{眜}{くら}いやうな
\ruby{汝}{きさま}では
\ruby{無}{な}いが、
\ruby{何故}{な|ぜ}
\ruby{汝}{きさま}は
\ruby{水野}{みづ|の}の
\ruby{爭友}{さう|いう}にはなつてやらんのだ。
\ruby{云}{い}はゞ
\ruby{汝}{きさま}は
\ruby{水野}{みづ|の}を
\ruby{愛}{あい}して、
\ruby{贔負}{ひゐ|き}に
\ruby{仕{\換字{過}}}{し|す}ぎて
\ruby{間無}{ま|ちが}つた
\ruby{事}{こと}をさせて
\ruby{居}{ゐ}るのだ。
いや
\ruby{頭}{かしら}を
\ruby{振}{ふ}つても
\ruby{左樣}{さ|う}で
\ruby{無}{な}いとは
\ruby{言}{い}はさん、
\ruby{見晴}{み|はら}しでの
\ruby{汝}{きさま}の
\ruby{言葉}{こと|ば}といひ、
\ruby{羽{\換字{勝}}}{は|がち}から
\ruby{聞}{き}いた
\ruby{事實}{じ|ゞつ}といひ、
\ruby{先刻}{さつ|き}からの
\ruby{汝}{きさま}の
\ruby{話}{はな}し
\ruby{工合}{ぐ|あひ}といひ、
\ruby{汝}{きさま}は
\ruby{水野}{みづ|の}の
\ruby{爭友}{さう|いう}となつて、
\ruby{彼}{あ}の
\ruby{男}{をとこ}に
\ruby{{\換字{過}}失無}{くわ|しつ|な}からしめてやら
うといふ
\ruby{考}{かんがへ}は
\ruby{有}{も}たんで、
\ruby{却}{かへ}つて
\ruby{庇護}{か|ば}ひ
\ruby{立}{だて}をする
\ruby{氣味}{き|み}がある。
\ruby{其樣}{そ|ん}な
\ruby{下}{くだ}らんことが
\ruby{何處}{ど|こ}にあるものか。
』

『オイ、
\ruby{大上段}{おほ|じやう|だん}に
\ruby{振}{ふ}り
\ruby{被}{かぶ}つて
\ruby{睨}{にら}み
\ruby{{\換字{廻}}}{まは}すなあ
\ruby{其邊}{そこ|いら}で
\ruby{措}{お}いて
\ruby{吳}{く}れ。
\ruby{下}{くだ}らなくつても
\ruby{乃公}{お|れ}は
\ruby{搆}{かま}はねえ。
\ruby{汝}{きさま}の
\ruby{云}{い}ふ
\ruby{事}{こと}
\ruby{位}{ぐらゐ}は
\ruby{乃公}{お|れ}だつて
\ruby{知}{し}つてゐるが、
\ruby{諫}{いさ}めたつて
\ruby{爭}{あらそ}つたつて
\ruby{役}{やく}に
\ruby{立}{た}たねえ
\ruby{事}{こと}だから、
\ruby{乃公}{お|ら}あ
\ruby{意見}{い|けん}も
\ruby{云}{い}はずに
\ruby{打棄}{うつ|ちや}つて
\ruby{置}{お}くんだ。
\ruby{{\換字{迷}}}{まよ}ふな〳〵
\ruby{思}{おも}ひ
\ruby{切}{き}れつて
\ruby{云}{い}つたつて、
\ruby{料簡}{れう|けん}
\ruby{方}{かた}が
\ruby{{\換字{煙}}管}{きせ|る}の
\ruby{羅宇}{ら|う}のやうにすげかへが
\ruby{出來}{で|き}るものぢやあ
\ruby{無}{な}し、
\ruby{川柳}{せん|りう}が
\ruby{巧}{うめ}え
\ruby{事}{こと}を
\ruby{云}{い}つて
\ruby{居}{ゐ}らあナ、「
\ruby{極無理}{ごく|む|り}な
\ruby{意見}{い|けん}
\ruby{魂魄}{たま|しひ}
\ruby{入}{い}れ
\ruby{換}{かへ}ろ」つて。
よく
\ruby{有}{あ}る
\ruby{奴}{やつ}だが、いくら
\ruby{魂魄}{たま|しひ}を
\ruby{入}{い}れ
\ruby{換}{かへ}ろつて
\ruby{云}{い}つたつて
\ruby{出來}{で|き}る
\ruby{相談}{さう|だん}じやあ
\ruby{無}{ね}え。
しかし
\ruby{水野}{みづ|の}に
\ruby{意見}{い|けん}をするなあ
\ruby{汝}{きさま}の
\ruby{{\換字{勝}}手}{かつ|て}だ。
\ruby{止}{よ}せと
\ruby{云}{い}つたなあ
\ruby{大}{おほき}に
\ruby{御世話}{お|せ|わ}だつた。
\ruby{芝}{しば}で
\ruby{會}{あ}つた
\ruby{時}{とき}
\ruby{云}{い}つた
\ruby{{\換字{通}}}{とほ}りだ。
\ruby{乃公}{お|れ}は
\ruby{乃公}{お|れ}だから
\ruby{乃公}{お|れ}は
\ruby{行}{い}かねえ。

\ruby{汝}{きさま}は
\ruby{汝}{きさま}だから
\ruby{行}{い}くなら
\ruby{行}{い}くがい〻。
』

『よしツ、
\ruby{汝}{きさま}が
\ruby{行}{い}かんでも
\ruby{乃公}{お|れ}は
\ruby{行}{い}かなくつて!。
\ruby{是}{これ}から
\ruby{直}{すぐ}に
\ruby{行}{い}つて
\ruby{諫}{いさ}めて
\ruby{{\換字{遣}}}{や}る。
\ruby{熱誠}{ねつ|せい}を
\ruby{以}{もつ}て
\ruby{大}{おほ}に
\ruby{爭}{あらそ}つて
\ruby{{\換字{遣}}}{や}る。
\ruby{憫然}{かは|いさう}に、% 「憫然 か(は)いさう」
\ruby{可惜}{あつ|たら}
\ruby{好漢}{かう|かん}の
\ruby{水野}{みづ|の}を
\ruby{區々}{く|ゝ}たる
\ruby{戀愛}{れん|あい}に
\ruby{悶死}{もん|し}させて
\ruby{堪}{たま}るもんか。
\ruby{日方}{ひ|かた}は
\ruby{彼}{かれ}のために
\ruby{爭友}{さう|いう}を
\ruby{以}{もつ}て
\ruby{任}{にん}じて
\ruby{{\換字{遣}}}{や}る。
\ruby{智慧}{ち|ゑ}の
\ruby{足}{た}らん
\ruby{男}{をとこ}がするの
\ruby{結果}{けつ|か}を
\ruby{見}{み}ろ。
』

『ハヽヽ、
\ruby{乃公}{お|れ}の
\ruby{云}{いつ}た
\ruby{事}{こと}が
\ruby{氣}{き}に
\ruby{入}{い}らなかつたからつて
\ruby{激}{げき}しちやあいけねえ。
\ruby{出}{で
}かけるるなあ
\ruby{可}{い}いが
\ruby{其猛勢}{その|いき|ほい}で
\ruby{行}{い}つて、
\ruby{水野}{みづ|の}と
\ruby{喧嘩}{けん|くわ}をしちやあ
\ruby{汝}{きさま}いけねえぜ。
\ruby{彼}{あ}の
\ruby{男}{をとこ}もおとなしいけれど
\ruby{蟲持}{むし|もち}だから。
』

『ハヽヽ、しかし
\ruby{乃公}{お|れ}の
\ruby{言}{い}ふ
\ruby{事}{こと}を
\ruby{聽}{き}かなかつたら
\ruby{攫}{つか}み
\ruby{挫}{ひし}ぐかも
\ruby{知}{し}れんぞ。
』

『
\ruby{戯談}{じよう|だん}ぢやあ
\ruby{無}{ね}えぜ、
\ruby{人}{ひと}が
\ruby{眞面目}{ま|じ|め}で
\ruby{云}{い}つて
\ruby{居}{ゐ}るのに。
』

『
\ruby{大{\換字{丈}}夫}{だい|ぢやう|ぶ}だ、
\ruby{日方}{ひ|かた}は
\ruby{粗暴}{そ|ばう}でもまさか
\ruby{喧嘩}{けん|くわ}はせん。
』

『い〻かい
\ruby{大將}{たい|しやう}、
\ruby{屹度}{きつ|と}だぜ、
\ruby{釘}{くぎ}をさしたぜ。
』

『ウン、よしツ。
\ruby{時}{とき}に
\ruby{島木}{しま|き}、』

『
\ruby{何}{なん}だ。
』

『
\ruby{汝}{きさま}が
\ruby{{\換字{平}}生}{いつ|も}
\ruby{飮}{の}んで
\ruby{居}{ゐ}る
\ruby{此}{こ}の
\ruby{葡萄酒}{ぶ|だう|しゆ}は
\ruby{中々}{なか|〳〵}
\ruby{佳}{い}いナ。
』

『それほどぢやあ
\ruby{無}{な}いがマア
\ruby{飮}{の}めるよ。
』

『
\ruby{手土產}{て|みや|げ}に
\ruby{仕}{し}て
\ruby{持}{も}つて
\ruby{行}{い}つて、
\ruby{久}{ひさ}しぶりで
\ruby{水野}{みづ|の}と
\ruby{談}{はな}しながら
\ruby{飮}{の}むのだ。
\ruby{些細}{さ|さい}な
\ruby{御用}{ご|よう}だ、
\ruby{二本}{に|ほん}ばかり
\ruby{徴發}{ちよう|はつ}するぞ。
』

『ハヽヽ、
\ruby{他}{ひと}の
\ruby{物}{もの}を
\ruby{徴發}{ちよう|はつ}して
\ruby{土產}{みや|げ}にするたあ
\ruby{此奴}{こい|つ}あ
\ruby{蟲}{むし}がい〻。
\ruby{可}{い}い〳〵。
\ruby{持}{も}つて
\ruby{行}{い}け、
\ruby{今}{いま}
\ruby{縛}{く〻}らせやう。% 本来は一の字点「ゝ」平仮名繰返し記号
』

\Entry{其三十六}

\ruby{牽牛花}{あさ|が|ほ}の
\ruby{花}{はな}の
\ruby{色}{いろ}は
\ruby{去年}{こ|ぞ}と
\ruby{今年}{こ|とし}と
\ruby{同}{おな}じく
\ruby{{\換字{咲}}}{さ}かず、
\ruby{人}{ひと}の
\ruby{心}{こゝろ}の
\ruby{傾}{かたむ}きは
\ruby{昨日}{きの|ふ}に
\ruby{今日}{け|ふ}の
\ruby{變}{かは}るが
\ruby{常}{つね}ながら、
\ruby{水野}{みづ|の}は
\ruby{{\換字{過}}}{す}ぎし
\ruby{日}{ひ}の
\ruby{日曜}{にち|\換字{𛀁}う}より、
\ruby{如何}{い|か}にかしけん
\ruby{今}{いま}までの
\ruby{水野}{みづ|の}にはあらずなりて、たゞ
\ruby{世}{よ}にありふれたる
\ruby{爺婆}{ぢゞ|ばゞ}の
\ruby{無智無學}{む|ち|む|がく}なるもの〻
\ruby{如}{ごと}くなりつ、ひたすらに
\ruby{御佛}{み|ほとけ}を
\ruby{頼}{たの}み
\ruby{奉}{たてまつ}り、
\ruby{日}{ひ}に〳〵
\ruby{我}{わ}が
\ruby{{\換字{勤}}務}{つ|とめ}を
\ruby{{\換字{終}}}{をは}るや
\ruby{否}{いな}や、
\ruby{直}{たゞち}に
\ruby{淺草}{あさ|くさ}に
\ruby{走}{はし}り
\ruby{行}{ゆ}きて、
\ruby{本{\換字{尊}}}{ほん|ぞん}の
\ruby{御{\換字{前}}}{おん|まへ}に
\ruby{祈念}{き|ねん}を
\ruby{凝}{こ}らし、いつはり
\ruby{無}{な}き
\ruby{心}{こゝろ}の
\ruby{誠}{まこと}を
\ruby{獻}{さ〻}げつくして、さて% 本来は一の字点「ゝ」平仮名繰返し記号
\ruby{後}{のち}やうやく
\ruby{寓}{やど}に
\ruby{歸}{かへ}るを
\ruby{常{\換字{習}}}{なら|ひ}とするに
\ruby{至}{いた}りたり。

\ruby{今日}{け|ふ}は
\ruby{日曜}{にち|\換字{𛀁}う}に
\ruby{當}{あた}りて
\ruby{身}{み}に
\ruby{閑暇}{いと|ま}あれば、
お
\ruby{濱}{はま}の
\ruby{何時}{いつ|も}もながらに
\ruby{訝}{いぶか}り
\ruby{怪}{あやし}みて
\ruby{其}{そ}の
\ruby{美}{うつく}しき
\ruby{眉}{まゆ}を
\ruby{顰}{ひそ}むるをば
\ruby{背後}{うし|ろ}に
\ruby{見棄}{み|す}てつ、
\ruby{水野}{みづ|の}は
\ruby{正午{\換字{過}}}{ひ|る|す}ぐる
\ruby{頃}{ころ}に
\ruby{家}{いへ}を
\ruby{立出}{たち|い}でたり。

\ruby[g]{吉右衛門}{きちゑもん}は
\ruby{本家}{ほん|け}に
\ruby{相談事}{さう|だん|ごと}ありとて
\ruby{招}{まね}かれて
\ruby{去}{さ}り、
お
\ruby{濱}{はま}
\ruby{一人}{ひと|り}
\ruby{餘令}{よ|ねん}
\ruby{無}{な}く
\ruby{新刊}{しん|かん}の
\ruby{雜誌}{ざつ|し}を
\ruby{讀}{よ}みながら、
お
\ruby{鍋}{なべ}を
\ruby{相手}{あい|て}に
\ruby{{\換字{留}}守}{る|す}し
\ruby{居}{を}るところへ、

『
\ruby{山路}{やま|ぢ}。
ウン
\ruby{此家}{こ|〻}だナ。% 本来は一の字点「ゝ」平仮名繰返し記号
』

と
\ruby{名札}{な|ふだ}を
\ruby{讀}{よ}んで
\ruby{獨語}{ひとり|ご}つやがてに、
\ruby{胴魔聲}{どう|ま|ごゑ}の
\ruby{人}{ひと}を
\ruby{驚}{おどろ}かすほど
\ruby{恐}{おそ}ろしく
\ruby{大}{おほき}く、

『
\ruby{頼}{たの}む。
』

と
\ruby{一}{ひ}ト
\ruby{聲呼}{こゑ|よ}ばはれるものあり。

『
\ruby{誰}{たれ}か
\ruby{呼}{よ}ばはつたでがす。
』

『さうだネ、お
\ruby{{\換字{前}}出}{まへ|で}て
\ruby{御覽}{ご|らん}ナ。
』

お
\ruby{濱}{はま}は
\ruby{{\換字{猶}}}{なほ}
\ruby{雜誌}{ざつ|し}をば
\ruby{讀}{よ}みつゞけ
\ruby{居}{ゐ}しが、
\ruby{應對}{おう|たい}の
\ruby{模樣}{も|やう}は
\ruby{明}{あき}らかに
\ruby{聞}{きこ}ゆ。

『
\ruby{水野}{みづ|の}は
\ruby{居}{を}るか。
』

『
\ruby{今}{いま}ア
\ruby{居}{ゐ}ねえでがす。
』

『
\ruby{何處}{ど|こ}へ
\ruby{行}{い}つた。
』

『
\ruby{知}{し}りましねえ。
』

『しかし
\ruby{出}{で}たものならいづれ
\ruby{歸}{かへ}るだらう。
』

『どうでがすかサ。
』

『
\ruby{{\換字{遠}}方}{ゑん|ぱう}わざ〳〵
\ruby{來}{き}たものだから
\ruby{上}{あが}つて
\ruby{待}{ま}つて
\ruby{居}{ゐ}やう。
』

『いかねえでがす。
\ruby{待}{また}つせえ
お
\ruby{{\換字{前}}樣}{め\換字{𛀁}|さま}。
』

お
\ruby{鍋}{なべ}は
\ruby{慌}{あわ}て〻
\ruby{入}{い}り
\ruby{來}{きた}りて、

『いやに
\ruby{身體}{から|だ}の
\ruby{魁偉}{い|か}い
\ruby{{\換字{尊}}大}{おほ|ふう}の
\ruby{野郎}{や|らう}でがす。
\ruby{水野}{みづ|の}さんの
\ruby{事聞}{こと|き}くか
ら
\ruby{不在}{る|す}だつて
\ruby{云}{い}つたら、
\ruby{上}{あが}つて
\ruby{待}{ま}たうと
\ruby{吐}{ぬか}します。
どうして
\ruby{吳}{く}れますべい。
イヤな
\ruby{奴}{やつ}でがす。
』

と
\ruby{云}{い}へば、
お
\ruby{濱}{はま}は、
\ruby{辛}{から}く
\ruby{雜誌}{ざつ|し}より
\ruby{目}{め}を
\ruby{離}{はな}して
\ruby{笑}{わら}ひ
\ruby{出}{いだ}し、

『
\ruby{{\換字{分}}}{わか}らないねえ
お
\ruby{{\換字{前}}}{まへ}は、
\ruby{言葉}{こと|ば}の
\ruby{樣子}{やう|す}ぢやあ
\ruby{水野}{みづ|の}さんと
\ruby{仲}{なか}の
\ruby{好}{い}い
\ruby{御朋友}{お|とも|だち}らしいぢや
\ruby{無}{な}いか。
どれ
\ruby{妾}{わたし}が
\ruby{行}{い}つて
\ruby{見}{み}やう。
』

と
\ruby{立出}{たち|いで}でたり。

\ruby{見}{み}れば
\ruby{客}{きやく}は
\ruby{血氣壯盛}{けつ|き|さか|ん}の
\ruby{陸軍士官}{りく|ぐん|しく|わん}にして、
\ruby{頭顱}{かし|ら}
\ruby{大}{おほき}く
\ruby{肩厚}{かた|あつ}きさまは
\ruby{素人}{しろう|と}づくねの
\ruby{土人形}{つち|にん|ぎやう}などの
\ruby{如}{ごと}く、
\ruby{無骨一遍}{ぶ|こつ|いつ|ぺん}の
\ruby{正直}{しやう|ぢき}さうな
\ruby{人}{ひと}なり。

『
\ruby{水野}{みづ|の}さんは
\ruby{今}{いま}
\ruby{御不在}{お|る|す}ですが
\ruby{誰樣}{どな|た}でいらつしやいます?。
』

\ruby{言葉}{こと|ば}
\ruby{無}{な}く
\ruby{名刺}{めい|し}を
\ruby{出}{いだ}して
\ruby{客}{きやく}の
\ruby{渡}{わた}すを、
お
\ruby{濱}{はま}は
\ruby{手}{て}に
\ruby{取}{と}りて
\ruby{讀}{よ}みて
\ruby{急}{きふ}に
\ruby{笑顏}{ゑ|がほ}になりぬ。
\ruby{未}{ま}だ
\ruby{面}{おもて}をこそ
\ruby{對}{あは}せざりつれ、
\ruby{水野}{みづ|の}の
\ruby{友}{とも}に
\ruby{其人}{その|ひと}あるよしの
\ruby{日方八郎}{ひ|かた|はち|らう}といふ
\ruby{名}{な}は、かねて
\ruby{聞}{き}き
\ruby{馴}{な}れて
\ruby{何時}{い|つ}と
\ruby{無}{な}く
\ruby{疏}{うと}からず
\ruby{覺}{おぼ}え
\ruby{居}{ゐ}たればなり。

『たしか
\ruby{島木}{しま|き}さんやなんぞと
\ruby{御一緖}{ご|いつ|しよ}の、
\ruby{御同國}{ご|どう|こく}の
\ruby{方}{かた}でいらつしやいましたね。
』

\ruby{一應}{いち|おう}
\ruby{念}{ねん}を
\ruby{推}{お}す
お
\ruby{濱}{はま}をば、
\ruby{日方}{ひ|かた}は
\ruby{眼}{め}を
\ruby{正}{たゞ}しくして
\ruby{一寸}{ちよ|つと}
\ruby{見}{み}しが、
\ruby{何訝}{なに|いぶ}かるべくも
\ruby{無}{な}き
\ruby{處女}{き|むすめ}の、たゞ
\ruby{怜悧}{り|こう}なるべく
\ruby{見}{み}ゆるのみの
\ruby{淸}{きよ}らなる
\ruby{娘}{むすめ}なれば、

『
\ruby{其通}{その|とほ}り。
』

と
\ruby{甚明}{いと|あき}らかに
\ruby{答}{こた}へたり。

『
\ruby{水野}{みづ|の}さんは
\ruby{淺草}{あさ|くさ}まで
\ruby{御}{お}いでになつたのですから、
\ruby{御{\換字{退}}屈}{ご|たい|くつ}でも
\ruby{御待}{お|ま}ちなさるならば、
\ruby{此方}{こち|ら}へ
\ruby{御{\換字{通}}}{お|とほ}りなすつて。
』

\ruby{何時}{い|つ}か
お
\ruby{濱}{はま}の
\ruby{背後}{うし|ろ}に
\ruby{出}{い}で
\ruby{來}{きた}り
\ruby{居}{ゐ}し
お
\ruby{鍋}{なべ}はそつと
\ruby{袖}{そで}を
\ruby{引}{ひ}きて

『
\ruby{宜}{い}いでがすかエ
\ruby{其樣}{そ|ん}な
\ruby{事}{こと}を
\ruby{仕}{し}て、
\ruby{何}{なん}だか
\ruby{蟲}{むし}の
\ruby{好}{す}かねえ
\ruby{厭}{いや}な
\ruby{奴}{やつ}でがすよ。
』

と
\ruby{心配}{しん|ぱい}し
\ruby{{\換字{過}}}{すご}して
\ruby{小聲}{こ|ゞゑ}に
\ruby{止}{とゞ}むるを、
お
\ruby{濱}{はま}は
\ruby{顧}{かへり}みず
\ruby{日方}{ひ|かた}を
\ruby{案内}{あん|ない}して、
\ruby{水野}{みづ|の}の
\ruby{室}{へや}に
\ruby{{\換字{通}}}{とほ}したり。

\ruby{日方}{ひ|かた}は
\ruby{水野}{みづ|の}が
\ruby{机}{つくゑ}の
\ruby{横}{よこ}にどつかりと
\ruby{座}{すわ}りて、

『ハヽア
\ruby{何}{なに}も
\ruby{裝{\換字{飾}}}{さう|しよく}は
\ruby{無}{な}いが
\ruby{惡}{わる}くない
\ruby{部屋}{へ|や}だナ。
\ruby{相變}{あひ|かは}らず
\ruby{有}{あ}るものは
\ruby{書籍}{ほ|ん}ばかりで、
\ruby{長物}{ちやう|ぶつ}の
\ruby{無}{な}いところは
\ruby{流石}{さす|が}に
\ruby{感心}{かん|しん}だ。
』% 原本では植字ミスと思われる

と
\ruby{先}{ま}づ
\ruby{{\換字{評}}}{ひやう}する
\ruby{時}{とき}、
お
\ruby{濱}{はま}は
お
\ruby{鍋}{なべ}が
\ruby{汲}{く}み
\ruby{來}{きた}りし
\ruby{茶}{ちや}を
\ruby{鷹}{すゝ}むれば、

『
\ruby{君}{きみ}は
\ruby{此家}{こ|〻}の% 本来は一の字点「ゝ」平仮名繰返し記号
\ruby{娘}{むすめ}さんかナ。
どうだ
\ruby{水野}{みづ|の}は。
\ruby{此頃}{この|ごろ}も
\ruby{相變}{あひ|かは}らず
\ruby{勉{\換字{強}}}{べん|きやう}か。
』

と
\ruby{話}{はな}し
\ruby{仕度}{し|た}さに
\ruby{打解}{うち|と}けて
\ruby{問}{と}ふを、
\ruby{水野}{みづ|の}〳〵と
\ruby{呼}{よ}びつけにするが
\ruby{小面憎}{こ|づら|にく}くてか、

『ハイ。
』

と
\ruby{僅々一句}{わづ|か|いつ|く}に
\ruby{答}{こたへ}を
\ruby{切}{き}りて、

『
\ruby{御自由}{ご|じ|ゆう}においでなすつて。
』

と
\ruby{言}{い}ひ
\ruby{棄}{す}てしま〻、
\ruby{突}{つ}と
\ruby{次}{つぎ}の
\ruby{間}{ま}に
\ruby{出}{い}で〻
\ruby{唐紙}{から|かみ}ぴつしやり、
お
\ruby{鍋}{なべ}の
\ruby{後}{あと}を
\ruby{{\換字{追}}}{お}ふて
\ruby{茶}{ちや}の
\ruby{室}{ま}に
\ruby{{\換字{退}}}{しりぞ}けば、
お
\ruby{鍋}{なべ}は、
\ruby{手}{て}の
\ruby{甲}{かふ}を
\ruby{口}{くち}にあて〻
\ruby{笑}{わら}ひながら、

『
\ruby{女}{をんな}を
\ruby{呼}{よ}ばるのに
\ruby{君}{きみ}だなんて、ホヽヽハヽヽ。
』

と、げらつきて
\ruby{已}{や}まず。
お
\ruby{濱}{はま}も
\ruby{睨}{にら}む
\ruby{眞似}{ま|ね}して
\ruby{叱}{しか}りは
\ruby{叱}{しか}りながら、おのれも
\ruby{口}{くち}のあたりに
\ruby{笑}{わらひ}を
\ruby{{\換字{浮}}}{う}かめぬ。

\ruby{話敵無}{はなし|がたき|な}き
\ruby{{\換字{所}}在無}{しよ|ざい|な}さの
\ruby{餘}{あま}り、
\ruby{日方}{ひ|かた}は
\ruby{其邊}{そこ|ら}を
\ruby{見{\換字{廻}}}{み|まは}しつ、
\ruby{机}{つくゑ}の
\ruby{上}{うへ}に
\ruby{在}{あ}りし
\ruby{折本}{をり|ほん}に
\ruby{偶然目}{ふ|と|め}を
\ruby{着}{つ}けて、
\ruby{手}{て}に
\ruby{取}{と}りて
\ruby{何心}{なに|ごゝろ}なく
\ruby{披}{ひら}き
\ruby{見}{み}しが、
\ruby{忽}{たちま}ち
\ruby{其{\換字{所}}}{そ|こ}に
\ruby{抛}{はふ}り
\ruby{出}{いだ}し、

『
\ruby{何}{なん}だ、
\ruby{普門品}{ふ|もん|ぼん}!。
\ruby{何}{なん}だ
\ruby{是}{これ}あ
\ruby{何}{なん}だ!。
\ruby{御有難{\換字{連}}}{お|あり|がた|れん}の
\ruby{誦}{よ}むものではないか。
まさか
\ruby{水野}{みづ|の}が
\ruby{信心}{しん|じん}するのではあるまいが、
\ruby{如是}{こ|ん}なものが
\ruby{机}{つくゑ}に
\ruby{載}{の}つて
\ruby{居}{ゐ}るのは
\ruby{何樣}{ど|う}した
\ruby{馬鹿}{ば|か}な
\ruby{事}{こつ}た。
』

と
\ruby{其處}{そ|こ}に
\ruby{罵}{の〻し}るべき% 本来は一の字点「ゝ」平仮名繰返し記号
\ruby{人}{ひと}にてもあるが
\ruby{如}{ごと}くに
\ruby{罵}{の〻し}つたり。% 本来は一の字点「ゝ」平仮名繰返し記号

\Entry{其三十七}

\ruby{待}{ま}てども〳〵
\ruby[g]{水野}{みづの}は
\ruby{歸}{かへ}らぬなり、
\ruby{此家}{この|や}の
\ruby{者}{もの}は
\ruby[g]{彼方}{かなた}に
\ruby{{\換字{退}}}{しりぞ}きて
\ruby{音}{おと}もさせぬなり、
\ruby[g]{日方}{ひかた}はほと〳〵
\ruby{身}{み}を
\ruby{持餘}{もて|あま}して、
\ruby[g]{四圍}{あたり}の
\ruby{書}{ほん}などを手あたりまかせに
\ruby{抽}{ひ}き
\ruby{出}{いだ}しては
\ruby{讀}{よ}み
\ruby{散}{ち}らし
\ruby{居}{ゐ}しが、それにも
\ruby{忽}{たちま}ち
\ruby{倦}{あ}きて
\ruby{無聊}{ぶ|りよう}に
\ruby{堪}{た}へかね、
\ruby{小齋}{せう|さい}の
\ruby{靜坐}{せい|ざ}には
\ruby{更}{さら}に
\ruby{慣}{なら}はぬ
\ruby{身}{み}の、
\ruby{何}{なに}をがな
\ruby{{\換字{消}}閑}{せう|かん}の
\ruby{具}{ぐ}にと
\ruby{見回}{み|まは}す
\ruby{折}{をり}しも、
\ruby{携}{たづさ}へ
\ruby{來}{き}し
\ruby{二罎}{ふた|びん}の
\ruby{酒}{さけ}に
\ruby{眼}{め}の
\ruby{止}{と}まれば
\ruby{先}{ま}づ
\ruby{微笑}{び|せう}を
\ruby{{\換字{浮}}}{うか}め、

『
\ruby{仕方}{し|かた}が
\ruby{無}{な}い、これでも
\ruby{飮}{の}んで
\ruby{待}{ま}つて
\ruby{居}{い}て
\ruby{{\換字{遣}}}{や}らう。
』

と
\ruby{口}{くち}にこそ
\ruby{言}{い}はぬ
\ruby{心}{こヽろ}に
\ruby{思}{おも}ひて、

『オイ、
\ruby{君}{きみ}!。
オイオイ、
\ruby{君}{きみ}!。
』

と
\ruby{呼}{よ}び
\ruby{立}{た}てたり。

『ハヽヽ、また
\ruby{君}{きみ}イ
\ruby{君}{きみ}イつて
\ruby{呼}{よ}ばつて
\ruby{居}{い}るでがす、
\ruby{妾}{わたし}が
\ruby{君}{きみ}イになつて
\ruby{出}{で}て
\ruby{行}{い}きますべいか。
』

『ホヽヽ、いヽよ、
\ruby{妾}{わたし}が
\ruby{行}{い}つて
\ruby{見}{み}るから。
』

お
\ruby{濱}{はま}は
\ruby{立}{た}つて
\ruby{客}{きやく}の
\ruby{前}{まへ}に
\ruby{到}{いた}れば、

『
\ruby{此酒}{こ|れ}を
\ruby{飮}{や}つて
\ruby{居}{ゐ}ながら
\ruby{待}{ま}たうと
\ruby{思}{おも}ふのだ。
\ruby{栓拔}{せん|ぬ}きと
\ruby{洋盞}{コツ|プ}とを
\ruby{假}{か}して
\ruby{{\換字{呉}}}{く}れたまへ。
』

と
\ruby{酒罎}{び|ん}を
\ruby{指}{ゆび}さしながらの
\ruby{無邪氣}{む|じや|き}の
\ruby{言}{ことば}なり。

『ハイ、
\ruby{洋盞}{コツ|プ}はありましたが、
\ruby{栓拔}{せん|ぬ}きが…………。
』

とお
\ruby{濱}{はま}の
\ruby{一寸行詰}{ちよ|つと|ゆき|つま}りしも
\ruby{無理}{む|り}ならず、
\ruby{誰}{たれ}も
\ruby{洋酒}{やう|しゆ}など
\ruby{用}{もち}ゐるもの
\ruby{無}{な}き
\ruby{温厚者揃}{おと|なし|や|そろ}ひの、
\ruby{此家}{こ|ヽ}は
\ruby{特}{こと}に
\ruby{隱居處}{いん|きよ|じよ}の
\ruby{事}{こと}とて
\ruby{當世}{たう|せい}の
\ruby{人}{ひと}の
\ruby{出入}{で|いり}もおのづから
\ruby{少}{すくな}きより、
\ruby{事少}{こと|すくな}き
\ruby{村住居}{むら|ずま|ゐ}の
\ruby{簡素}{てが|るさ}に
\ruby{馴}{な}れて、
\ruby{今日}{け|ふ}の
\ruby{今}{いま}まで
\ruby{栓拔}{せん|ぬ}きに
\ruby{用}{よう}も
\ruby{無}{な}かりしほどなれば、
\ruby{貸}{か}さんと
\ruby{欲}{ほつ}して
\ruby{其物無}{その|もの|な}きに
\ruby{困}{こう}じ
\ruby[g]{躊躇}{たゆた}へるなり。

お
\ruby{鍋}{なべ}を
\ruby[g]{隣家}{となり}に
\ruby{走}{はし}らしめんか、
\ruby[g]{隣家}{となり}はたヾの
\ruby{小前}{こ|まへ}なれば、猶さら
\ruby{栓拔}{せん|ぬき}などの
\ruby{有}{あ}るべくもあらず、さらば
\ruby{本家}{ほん|け}に
\ruby{至}{いた}らしめんか、
\ruby{本家}{ほん|け}と
\ruby{此家}{こ|ヽ}との
\ruby{餘}{あま}り
\ruby{隔}{へだヽ}りたり、
\ruby{如何}{いか|ヾ}せん、とお
\ruby{濱}{はま}は
\ruby[g]{少時{\換字{迷}}}{しばしまよ}ひたりしが、ふと
\ruby[g]{水野}{みづの}が
\ruby{洋小刀}{ナ|イ|フ}に
\ruby{栓拔}{せん|ぬ}きの
\ruby{添}{そ}ひ
\ruby{居}{ゐ}しを
\ruby{思}{おも}ひ
\ruby{出}{いだ}し、
\ruby{先}{ま}づお
\ruby{鍋}{なべ}を
\ruby{呼}{よ}びて
\ruby{小}{ちひさ}き
\ruby{{\換字{盆}}}{ぼん}に
\ruby{洋盞}{コツ|プ}を
\ruby{載}{の}せて
\ruby{持來}{もち|き}たらしめ、おのれは
\ruby{机}{つくゑ}の
\ruby{周圍}{まは|り}、
\ruby{本箱}{ほん|ばこ}の
\ruby{上}{うへ}などを
\ruby{見}{み}つ、
\ruby{彼}{か}の
\ruby{心當}{こヽろ|あて}の
\ruby[g]{小刀}{ナイフ}をと
\ruby{尋}{たづ}ね
\ruby{捜}{さが}したり。
\\

されど
\ruby[g]{小刀}{ナイフ}は
\ruby{外}{そと}に
\ruby{出}{い}で
\ruby{居}{を}
らずして、
\ruby{終}{つひ}に
\ruby{見當}{み|あた}る
\ruby{事無}{こと|な}かりしかば、
\ruby{若}{もし}や
\ruby{此内}{この|うち}にと、
\ruby{机}{つくゑ}の
\ruby{下}{した}なる
\ruby{手箱}{て|ばこ}を
\ruby{引出}{ひき|いだ}して、
\ruby{日頃}{ひ|ごろ}の
\ruby[g]{心易立}{こヽろやすだて}に
\ruby{何}{なに}の
\ruby{氣}{き}も
\ruby{無}{な}く
\ruby{掻撈}{かい|さぐ}れば、
\ruby{書簡}{て|がみ}、
\ruby[g]{雑記帳}{ざつきちやう}、
\ruby{物書}{もの|か}きさしたる
\ruby{反故}{ほ|ご}なんどの
\ruby{底}{そこ}の
\ruby{方}{かた}より
\ruby{洋小刀}{ナ|イ|フ}は
\ruby{出}{い}でたり。

『ヤ
\ruby{栓拔}{せん|ぬ}きは
\ruby{此品}{こ|れ}で
\ruby{澤山}{たく|さん}だ。
\ruby{何}{なん}だか
\ruby{面白}{おも|しろ}いものが
\ruby{出}{で}さうな
\ruby{匣}{はこ}だナ。
どれ
\ruby[g]{{\換字{退}}屈{\換字{紛}}}{たいくつまぎ}らしに
\ruby{見}{み}てやらうか。
』

\ruby{日方}{ひ|かた}は
\ruby{眼快}{め|ばや}く
\ruby{既}{すで}に
\ruby{彼}{か}の
\ruby[g]{小刀}{ナイフ}を
\ruby{取}{と}りて、
\ruby{{\換字{猶}}}{なほ}また
\ruby{其匣}{その|はこ}の
\ruby{内}{うち}の
\ruby{物}{もの}を
\ruby{見}{み}んとすれば、

『およしなさいよ、
\ruby{他人}{ひ|と}さんの
\ruby{物}{もの}を。
\ruby[g]{貴下}{あなた}は
\ruby{亂暴}{らん|ぼう}\換字{子}。
』

と
\ruby{窘}{たしな}むるが
\ruby{如}{ごと}き
\ruby{口氣}{こう|き}に
\ruby{{\換字{強}}}{つよ}く
\ruby{云}{い}ひ
\ruby{懲}{こら}して、お
\ruby{濱}{はま}は
\ruby{直}{たヾち}に
\ruby{匣}{はこ}の
\ruby{蓋}{ふた}を
\ruby{閉}{と}ぢ、
\ruby{机}{つくゑ}の
\ruby{下深}{した|ふか}く
\ruby{押入}{おし|い}れつ、
\ruby{無遠慮}{ぶ|ゑん|りよ}も
\ruby{程度}{ほ|ど}のあるものをと
\ruby{腹立}{はら|だ}ちて、
あどけ
\ruby{無}{な}き
\ruby{顏}{かほ}にも
\ruby{瞋}{いかり}を
\ruby{含}{ふく}んで
\ruby{其處}{そ|こ}を
\ruby{{\換字{退}}}{しりぞ}きたり。

もとより
\ruby{年}{とし}もゆかぬお
\ruby{濱}{はま}などには
\ruby{眼}{め}も
\ruby{{\換字{呉}}}{く}れざる
\ruby[g]{日方}{ひかた}は、
\ruby{手酌}{てじ|やく}の
\ruby[g]{無興氣}{ぶきようげ}に
\ruby[g]{一盃一盃}{いつぱいいつぱい}を
\ruby{重}{かさ}ねしが、
\ruby{飮}{の}んではいよ〳〵
\ruby{相手欲}{あい|て|ほ}しさに
\ruby{獨居}{ひと|りゐ}の
\ruby{淋}{さみ}しく、
\ruby{所在無}{しよ|ざい|な}さの
\ruby{餘}{あま}りのわざくれに、
\ruby{前}{さき}に
\ruby{見}{み}し
\ruby{手匝}{て|ばこ}を
\ruby{我}{わ}が
\ruby{前{\換字{近}}}{まへ|ちか}く
\ruby{引寄}{ひき|よ}せ、
\ruby{内}{うち}なる
\ruby{雜記帳樣}{ざつ|き|ちやう|やう}のものを
\ruby{取出}{とり|いだ}して、
\ruby{此頃}{この|ごろ}
\ruby[g]{水野}{みづの}が
\ruby{如何}{い|か}なる
\ruby{事}{こと}をか
\ruby{書}{か}けると、
\ruby{其}{それ}を
\ruby{知}{し}りたきばかりの
\ruby[g]{好奇心}{かうきしん}に
\ruby{隔無}{へだ|てな}き
\ruby{中}{なか}とて
\ruby{無遠慮}{ぶ|ゑん|りよ}にも、
\ruby[g]{一盃仰}{いつぱいあふ}いでは
\ruby{一葉飜}{いち|\換字{江}う|ひるがへ}し、
\ruby{一枚讀}{いち|まい|よ}みては
\ruby{一杯仰}{いつ|ぱい|あふ}いで、
\ruby{{\換字{終}}}{つひ}に
\ruby{我知}{われ|し}らず
\ruby{醉}{よひ}に
\ruby{入}{い}りぬ。

\ruby{冊子}{さう|し}は
\ruby{何}{なに}くれと
\ruby{無}{な}く
\ruby[g]{水野}{みづの}が
\ruby{讀}{よ}み
\ruby{{\換字{過}}}{す}ごしたる
\ruby{或}{あるひ}は
\ruby{國書}{こく|しよ}
\ruby{或}{あるひ}は
\ruby{漢籍}{かん|せき}、
\ruby{或}{あるひ}は
\ruby{洋書}{やう|しよ}の
\ruby{其中}{その|うち}より、
\ruby{我}{わ}が
\ruby{意}{こゝろ}に
\ruby{{\換字{適}}}{てき}したる
\ruby{語}{ご}、
\ruby{詩句}{しの|く}、
\ruby{事實}{じ|じつ}なんヾを、
\ruby{或}{あるひ}は
\ruby{原}{もと}のまヽに、
\ruby{或}{あるひ}は
\ruby{引直}{ひき|なほ}して、
\ruby{筆任}{ふで|まか}せに
\ruby{記}{しる}したる
\ruby[g]{眞實}{まこと}の
\ruby{雑抄}{ざつ|せう}にて、
\ruby{恰}{あたか}も
\ruby{人}{ひと}の
\ruby{摘}{つ}み
\ruby{集}{あつ}めし
\ruby{花}{はな}のいろ〳〵の
\ruby{線}{せん}に
\ruby{貫}{つらぬ}かれたるを
\ruby{見}{み}るが
\ruby{如}{ごと}く
\ruby{趣味}{おも|むき}あるものなれば、
\ruby[g]{日方}{ひかた}は
\ruby{心窃}{こヽろ|ひそか}に
\ruby[g]{水野}{みづの}が
\ruby{苦學}{く|がく}を
\ruby{怠}{おこた}らぬを
\ruby{悦}{よろこ}びながら
\ruby{讀}{よ}み
\ruby{居}{ゐ}しが、
\ruby{讀}{よ}む
\ruby{事半途}{こと|なか|ば}にして
\ruby{間}{なか}に
\ruby{介}{はさ}まり
\ruby{居}{ゐ}し
\ruby{一片}{いつ|ぺん}の
\ruby{紙}{かみ}の
\ruby{偶然飛}{ふ|と|ヽ}び
\ruby{出}{い}でたれば、
\ruby{何}{なん}ならんと
\ruby{急}{きふ}に
\ruby{手}{て}に
\ruby{取}{と}りて
\ruby{見}{み}るに、
\ruby[g]{第七番凶}{だいななばんきよう}といふ
\ruby{觀音}{くわん|のん}の
\ruby{御籖}{み|くじ}なり。

『いかんナ。
\ruby{何樣}{ど|う}も
\ruby{怪}{をか}しいナ、
\ruby{此樣}{こ|ん}なものが
\ruby{出}{で}るとは。
\ruby{机}{つくゑ}の
\ruby{上}{うへ}には
\ruby{普門品}{ふ|もん|ぼん}がある、こヽには
\ruby{此樣}{こ|ん}なものが
\ruby{介}{はさま}つてゐる。

\ruby{何樣}{ど|う}したのだらう、
\ruby{何}{なん}だが
\ruby{怪}{おか}しいナ。
』

されど
\ruby{怪}{おか}しき
\ruby{事}{こと}は
\ruby{介}{はさ}まり
\ruby{居}{ゐ}し
\ruby{其}{それ}のみにして、
\ruby{冊子}{さう|し}の
\ruby[g]{三分}{さんぶ}の
\ruby{一}{いち}ほどは
\ruby{{\換字{猶}}}{なほ}
\ruby{白紙}{しら|かみ}の
\ruby{物}{もの}も
\ruby{書}{か}かれず
\ruby{殘}{のこ}れるなり。
これまでと
\ruby[g]{日方}{ひかた}は
\ruby{其}{そ}の
\ruby{冊子}{さう|し}を
\ruby{伏}{ふ}せ
\ruby{棄}{す}てヽ、
\ruby{盃}{はい}を
\ruby{啣}{ふく}みて
\ruby{物}{もの}を
\ruby{案}{あん}じ
\ruby{居}{ゐ}しが、
\ruby{見}{み}るとも
\ruby{無}{な}しに
\ruby{見}{み}れば
\ruby{册子}{さう|し}の
\ruby{後}{うしろ}の
\ruby{表紙}{へう|し}には、
\ruby{反故染}{ほ|ご|ぞめ}といふものヽ
\ruby{如}{ごと}くに、
\ruby{落書}{らく|がき}の
\ruby{上}{うへ}に
\ruby{落書重}{らく|がき|かさ}なりて、
\ruby{縱横斜角}{たて|よこ|すじ|かひ}に
\ruby{何}{なに}か
\ruby{書}{しる}されたり。
\ruby{何事}{なに|ごと}を
\ruby{加是}{か|く}は
\ruby{落書}{らく|がき}したりしやと、
\ruby{讀}{よ}み
\ruby{易}{やす}きを
\ruby{辿}{たど}りて
\ruby{一}{ひ}トつヾきを
\ruby{讀}{よ}めば、
\ruby{此}{こ}は
\ruby{是一首}{これ|いつ|しゆ}の
\ruby{歌}{うた}にして、

  %全角空白
\ruby{立}{た}ちて
\ruby{居}{ゐ}る
\ruby[g]{方便}{たづき}も
\ruby{知}{し}らに
\ruby{我}{わ}が
\ruby{心天}{こヽろ|あま}つ
\ruby{空}{そら}なり
\ruby{地}{つち}は
\ruby{踏}{ふ}めども

とありたり。

『フヽーン、
\ruby{精}{よ}くは
\ruby{分}{わか}らんが
\ruby{戀}{こひ}の
\ruby{歌}{うた}だナ。
\ruby[g]{水野}{みづの}が
\ruby{詠}{よ}んだのか
\ruby{知}{し}らん。
ウン
\ruby{彼}{あれ}のだらう。
も
\ruby{一}{ひと}ツは
\ruby{何}{なん}だ、ン、
\ruby{是}{これ}も
\ruby{歌}{うた}かナ。
ナニ。

  %全角空白
\ruby{天地}{あめ|つち}に
\ruby{少}{すこ}し
\ruby{至}{いた}らぬ
\ruby{大丈夫}{ます|ら|を}と
\ruby{思}{おも}ひし
\ruby{我}{われ}や
\ruby{雄心}{をご|ヽろ}も
\ruby{無}{な}き

ハヽア、
\ruby{舊}{もと}は
\ruby{絶大}{ぜつ|だい}な
\ruby{抱負}{はう|ふ}も
\ruby{有}{あ}つた
\ruby{身}{み}だがと、
\ruby{戀}{こひ}に
\ruby{{\換字{迷}}}{まよ}つた
\ruby{今}{いま}を
\ruby{自}{みづか}ら
\ruby{悲}{かなし}む
\ruby{歌}{うた}だナ。
アヽ
\ruby{佳}{い}い
\ruby{歌}{うた}だ、
\ruby{乃公}{お|れ}にも
\ruby{解}{わか}
る。
\ruby{天地}{てん|ち}にも
\ruby{多}{おほ}くは
\ruby{劣}{おと}るまいと
\ruby{思}{おも}つて
\ruby{居}{ゐ}た
\ruby{此}{こ}の
\ruby{我身}{わが|み}だがなあと、
\ruby{戀}{こひ}の
\ruby{苦}{くる}しさに
\ruby{萎}{いほ}たれて、
\ruby{呻}{うめ}き
\ruby{出}{だ}した
\ruby{此}{こ}の
\ruby{歌}{うた}の
\ruby{主}{ぬし}の
\ruby{腹}{はら}ん
\ruby{中}{なか}が
\ruby{憫然}{かはい|さう}

\ruby[g]{憫然}{かはいさう}でならん。
\ruby[g]{此方}{こつち}に
\ruby{書}{か}いてあるのは
\ruby{何}{なん}だ。
\ruby{何}{なん}だと。

  %全角空白
\ruby{大丈夫}{ます|ら|を}のさとき
\ruby{心}{こヽろ}も
\ruby{今}{いま}は
\ruby{無}{な}し
\ruby{戀}{こひ}の
\ruby{奴}{やつこ}と
\ruby{我}{われ}
は
\ruby{死}{し}ぬべし

アヽいかん〳〵、
\ruby{怪}{け}しからん
\ruby{事}{こつ}た、
\ruby[g]{馬鹿々々}{ばか〳〵}しい。
\ruby{散}{ち}らして
\ruby{書}{か}いてある
\ruby{此}{こ}の
\ruby{讀}{よ}
みにくいのは
\ruby{何}{なん}だ。

  %全角空白
\ruby{久堅}{ひさ|かた}のあまみづ

エート、

  %全角空白
\ruby{久堅}{ひさ|かた}の
\ruby{天}{あま}みつ
\ruby{空}{そら}に
\ruby{照}{て}れる
\ruby{日}{ひ}の
\ruby{失}{う}せなん
\ruby{日}{ひ}こそ
\ruby{我}{わ}が
\ruby{戀止}{こひ|や}まめ

いかんナ、いかんナ、
\ruby{斯樣恐}{か|う|をそ}ろしく
\ruby{思}{おも}ひ
\ruby{{\換字{込}}}{こ}んでは
\ruby{始末}{し|まつ}が
\ruby{着}{つ}かん、
\ruby[g]{斯樣滅茶苦茶}{かうめちやくちや}になつては
\ruby{實}{じつ}にいかん、
\ruby[g]{大馬鹿野郎}{おほばかやらう}だ、
\ruby[g]{戀愛狂}{れんあいきやう}だ。
』

\ruby[g]{此方}{こなた}にては
\ruby[g]{日方}{ひかた}が
\ruby{夢中}{む|ちう}になつて
\ruby{醉}{よひ}に
\ruby{乘}{じよう}じて
\ruby{如是罵}{か|く|のヽし}れる
\ruby{時}{とき}、
\ruby[g]{彼方}{かなた}にてはお
\ruby{濱}{はま}が
\ruby{悦}{よろこ}びに
\ruby{冴}{さ}ゆる
\ruby{聲}{こゑ}して、

『マア
\ruby{遲}{おそ}かつたのネエ、
\ruby{大變}{たい|へん}に
\ruby{待}{ま}つてたは。
それにアノ
\ruby[g]{日方}{ひかた}んといふ
\ruby{人}{ひと}が
\ruby{來}{き}て
\ruby{待}{ま}つてヽよ。
』

と
\ruby{忙}{せは}しげに
\ruby{言}{ものい}へば、
\ruby{同}{おな}じく
\ruby{聊}{いさヽ}か
\ruby{疾辯}{はや|くち}に、

『
\ruby{左樣}{さ|う}かエ、
\ruby{觀音樣}{くわん|のん|さま}であのお
\ruby{龍}{りう}つていふ
\ruby{人}{ひと}にひよつくり
\ruby{逢}{あ}つて、あの
\ruby{人}{ひと}
の
\ruby{朋友}{とも|だち}だとか
\ruby{云}{い}ふ
\ruby[g]{立派}{りつぱ}な
\ruby[g]{婦人}{おくさん}と
\ruby[g]{二人}{ふたり}に
\ruby{無理}{む|り}に
\ruby{{\換字{強}}}{し}ひられて
\ruby{御馳走}{ご|ち|そう}になつたりなんぞ
\ruby{仕}{し}たものだから、
\ruby{大}{おほき}に
\ruby{歸}{かへ}りが
\ruby{遲}{おそ}くなつて
\ruby{仕舞}{し|ま}つた。
\ruby[g]{日方}{ひかた}は
\ruby{一時間}{いち|じ|かん}も
\ruby{前}{まへ}から
\ruby{待}{ま}つて
\ruby{居}{ゐ}てかエ。
』

と、
\ruby[g]{水野}{みづの}が
\ruby{語}{かた}る
\ruby{聲}{こゑ}の
\ruby{爲}{し}たり。


\Entry{其三十八}

\原本頁{}%
\ruby{好}{この}まぬ
\ruby{酒}{さけ}を
\ruby{人}{ひと}に
\ruby{{\換字{強}}}{し}ひられて、
%
\ruby[g]{水野}{みづの}は
\ruby{既}{すで}に
\ruby{醒}{さ}めたれども
\ruby{三{\換字{分}}}{さん|ぶ}の
\ruby{醉}{よひ}あり、% 「醉」は原本通り「よ」で調整
%
\ruby{好}{この}める
\ruby{酒}{さけ}を
\ruby{一人}{ひと|り}
\ruby{汲}{く}みて、
%
\ruby[g]{日方}{ひかた}は
\ruby{{\換字{猶}}}{なほ}
\ruby{足}{た}らずとすれども
\ruby{七{\換字{分}}}{しち|ぶ}の
\ruby{醉}{よひ}あり。% 「醉」は原本通り「よ」で調整
%
たゞさへ
\ruby{醉}{よ}へる% 「醉」は原本通り「よ」で調整
\ruby{同士}{どう|し}は
\ruby{打解}{うち|と}け
\ruby{易}{やす}きに、
%
まして
\ruby{是}{これ}は
\ruby{一}{ひ}ト
\ruby{方}{かた}ならぬ
\ruby{中}{なか}の
\ruby{舊友}{きう|いう}の、
%
たまさかに
\ruby{相}{あひ}
\ruby{逢}{あ}へるなれば、
%
\ruby{笑顏}{ゑ|がほ}に
\ruby{云}{い}ひ
\ruby{出}{だ}されし、

\原本頁{}%
『ヤ』

\原本頁{}%
『ヤ』

\原本頁{}%
の
\ruby{一}{ひ}ト
\ruby{聲}{こゑ}より
\ruby{先}{ま}づ
\ruby{碎}{くだ}け
\ruby{合}{あ}ひて、

\原本頁{}%
『
\ruby{其後久}{その|ゝち|ひさ}しく
\ruby{會}{あ}はなかつたナア。
』

\原本頁{}%
『ほんとに
\ruby{長}{なが}い
\ruby{事}{こと}
\ruby{會}{あ}はなかつたナ。
』

\原本頁{}%
『ウン、
%
\ruby{乃公}{お|れ}が
\ruby{候補生}{こう|ほ|せい}になつた
\ruby{時}{とき}
\ruby{祝}{しゆく}して
\ruby{吳}{く}れた
\ruby{會}{くわい}で
\ruby{會}{あ}つた
\ruby{限}{き}りだつたナア。
』

\原本頁{}%
『アヽ
\ruby{左樣}{さ|う}だつた。
%
\ruby{早}{はや}いものでもう
\ruby{大{\換字{分}}}{だい|ぶ}
\ruby{{\換字{過}}去}{あ|と}になつた。
』

\原本頁{}%
と
\ruby{互}{たがひ}に
\ruby{懷}{なつ}かしげに
\ruby{凝然}{じ|つ}と
\ruby{面}{かほ}を
\ruby{見合}{み|あ}ひしが、
%
\ruby[g]{水野}{みづの}が
\ruby{目}{め}には
\ruby[g]{日方}{ひかた}が
\ruby{肥}{こ}え
\ruby{肉}{にく}づきていよ〳〵
\ruby{男兒}{をと|こ}らしく
\ruby{立派}{りつ|ぱ}になれるが、
%
\ruby{羨}{うらや}ましくもまた
\ruby{好}{この}ましく
\ruby{見}{み}え、
%
\ruby[g]{日方}{ひかた}が
\ruby{眼}{め}には
\ruby[g]{水野}{みづの}が
\ruby{痩}{や}せ
\ruby{窶}{やつ}れて
\ruby{往時}{むか|し}の
\ruby{生々}{いき|〳〵}としたる
\ruby{氣合}{き|あひ}の
\ruby{失}{う}せたるが、
%
\ruby{{\換字{情}}無}{なさけ|な}くもまた
\ruby{口惜}{くち|をし}く
\ruby{見}{み}えたり。

\原本頁{}%
『
\ruby[g]{日方}{ひかた}!。
%
\ruby{久}{ひさ}しいと
\ruby{云}{い}つても
\ruby{僅見無}{わづ|かみ|な}い
\ruby{中}{うち}に、
%
\ruby{君}{きみ}はまあ
\ruby{實}{じつ}に
\ruby{立派}{りつ|ぱ}な
\ruby{好}{い}い
\ruby{身體}{から|だ}になつたナア。
』

\原本頁{}%
『
\ruby{乃公}{お|れ}は
\ruby{其樣}{そ|ん}なに
\ruby{云}{い}はれるほどでもないが、
%
\ruby[g]{水野}{みづの}、
%
\ruby{汝}{きさま}はまた、
%
\ruby{大層}{たい|そう}
\ruby{痩}{や}せ
\ruby{枯}{から}びて
\ruby{年}{とし}を
\ruby{取}{と}つたナア。
』

\原本頁{}%
\ruby{主人}{ある|じ}も
\ruby{客}{きやく}も
\ruby{共}{とも}に
\ruby{一種}{いつ|しゆ}の
\ruby{言}{い}ひ
\ruby{難}{がた}き
\ruby{感}{かん}に
\ruby{打}{う}たれしが、
%
\ruby[g]{日方}{ひかた}は
\ruby{猿臂}{ゑん|ぴ}を
\ruby{伸}{の}ばして
\ruby[g]{水野}{みづの}の
\ruby{手}{て}を
\ruby{執}{と}り、

\原本頁{}%
『この
\ruby{骨}{ほね}つぽい
\ruby{痩}{や}せ
\ruby{切}{き}つた
\ruby{此手}{こ|れ}が、
%
\ruby{相撲取}{すま|ふ|と}りを
\ruby{仕}{し}ては
\ruby{{\換字{随}}{\換字{分}}}{ずゐ|ぶん}%「隨」TODO 変更 ⻖左円辶
\ruby{手}{て}ひどく
\ruby{乃公}{お|れ}を
\ruby{投}{な}げつけた
\ruby{事}{こと}もある
\ruby{脅力}{ちか|ら}のあつた
\ruby{手}{て}だらうか。
%
\ruby{此}{こ}の
\ruby{樣子}{やう|す}では
\ruby{今}{いま}では
\ruby{乃公}{お|れ}には、
%
\ruby{中々}{なか|〳〵}
\ruby{敵}{かな}ふどころではありは
\ruby{仕}{し}まいが。
』

\原本頁{}%
と
\ruby{云}{い}へば
\ruby{云}{い}はれたる
\ruby[g]{水野}{みづの}は
\ruby{歎}{たん}じて、

\原本頁{}%
『アヽ、
%
\ruby{今}{いま}ぢやあ
\ruby{一}{ひ}ト
\ruby{堪}{たま}りも
\ruby{無}{な}く
\ruby{負}{ま}かされて
\ruby{仕舞}{し|ま}はう。
%
これほど
\ruby{衰}{おとろ}へて
\ruby{居}{ゐ}るとは
\ruby{自{\換字{分}}}{じ|ぶん}でも
\ruby{思}{おも}は
\ruby{無}{な}かつたが、
%
\ruby{君}{きみ}のがつしりと
\ruby{仕}{し}た
\ruby{手}{て}と
\ruby{斯樣}{か|う}
\ruby{比}{くら}べては、
%
\ruby{羞}{はづ}かしいやうな
\ruby{心持}{こゝろ|もち}が
\ruby{仕}{し}て、
%
\ruby{物悲}{もの|がな}しい
\ruby{淋}{さび}しい
\ruby{感}{かん}じがする。
』

\原本頁{}%
と
\ruby{蔽}{かく}すところも
\ruby{無}{な}く
\ruby{思}{おも}ふまゝを
\ruby{打出}{うち|いだ}したり。
%
お
\ruby{濱}{はま}は
\ruby{小娘}{こ|むすめ}の
\ruby{智慧}{ち|ゑ}の
\ruby{乏}{とぼ}しけれど
\ruby{心}{こゝろ}ばかりの
\ruby{饗應}{もて|なし}に、
%
お
\ruby{鍋}{なべ}と
\ruby{相談}{さう|だん}して、
%
\ruby{干魚}{ひ|ざかな}を
\ruby{燒}{や}きて
\ruby{裂}{さ}きたると
\ruby{漬物}{つけ|もの}とを、
%
\ruby{酒}{さけ}の
\ruby{下物}{さか|な}にと
\ruby{案}{あん}じ
\ruby{出}{いだ}して
\ruby{持來}{もち|きた}りて
\ruby{歸}{かへ}りしが、
%
\ruby[g]{日方}{ひかた}はそれにも
\ruby{心}{こゝろ}づかぬ
\ruby{如}{ごと}く、

\原本頁{}%
『
\ruby{左樣}{さ|う}だらう、
%
\ruby{定}{さだ}めし
\ruby{左樣}{さ|う}いふ
\ruby{感}{かん}じが
\ruby{仕}{し}やう。
%
\ruby{舊}{もと}と
\ruby{異}{ちが}つたとは
\ruby{乃公}{お|れ}ばかりでも
\ruby{無}{な}い。
%
\ruby{汝}{きさま}は
\ruby[g]{羽{\換字{勝}}}{はがち}にもまだ
\ruby{會}{あ}ふまいが、
%
\ruby{彼}{あれ}も
\ruby{鐵}{てつ}のやうな
\ruby{男兒}{をと|こ}に
\ruby{自{\換字{分}}}{じ|ぶん}を
\ruby{鍛}{きた}ひ
\ruby{上}{あ}げて、
%
\ruby{料簡}{れう|けん}にも
\ruby{言語}{もの|いひ}にも
\ruby{身體}{から|だ}つきにも、
%
\ruby{弛{\換字{緩}}}{だ|ら}けたところの
\ruby{無}{な}い
\ruby{確固漢}{しつ|かり|もの}になつて
\ruby{來}{き}たぞ。
%
もとから
\ruby{一}{ひ}ト
\ruby{風}{ふう}ある
\ruby{男}{をとこ}だつたが、
%
いよ〳〵
\ruby{實}{み}が
\ruby{入}{い}つて
\ruby{物}{もの}になつた。
%
\ruby{今}{いま}に
\ruby{見}{み}ろ、
%
\ruby{何}{なに}か
\ruby{{\換字{遣}}}{や}り
\ruby{始}{はじ}めて、
%
\ruby{生命}{いの|ち}さへ
\ruby{有}{あ}りやあ
\ruby{屹度}{きつ|と}
\ruby{{\換字{遣}}}{や}り
\ruby{{\換字{遂}}}{と}げるは。
%
\ruby[g]{島木}{しまき}が
\ruby{金}{かね}を
\ruby{出}{だ}して
\ruby{{\換字{船}}}{ふね}を
\ruby{買}{か}つて、
%
\ruby{{\換字{遠}}洋漁業}{ゑん|やう|ぎよ|げふ}を
\ruby{爲}{や}るとか
\ruby{何}{なん}とか
\ruby{云}{い}つて
\ruby{居}{ゐ}るから、
%
いづれ
\ruby{着々}{ちやく|〳〵}と
\ruby{歩}{ほ}を
\ruby{{\換字{進}}}{すゝ}めて
\ruby{居}{ゐ}るのだらう。
%
\ruby{今日}{け|ふ}も
\ruby{實}{じつ}は
\ruby[g]{島木}{しまき}のところで
\ruby[g]{羽{\換字{勝}}}{はがち}と
\ruby{乃公}{お|れ}と、
%
\ruby{三人}{さん|にん}
\ruby{落合}{おち|あ}つて
\ruby{此家}{こ|ゝ}へ
\ruby{來}{く}る
\ruby{筈}{はず}だつたが、
%
\ruby[g]{羽{\換字{勝}}}{はがち}に
\ruby{差支}{さし|つかへ}があつて
\ruby{斷}{ことわ}つて
\ruby{來}{き}たので、
%
\ruby[g]{島木}{しまき}は
\ruby{出}{で}ないと
\ruby{云}{い}ふし
\ruby{仕方}{し|かた}が
\ruby{無}{な}いから、
%
そこで
\ruby{乃公}{お|れ}
\ruby{一人}{ひと|り}で
\ruby{出}{で}て
\ruby{來}{き}たのだが…………、
%
\ruby[g]{水野}{みづの}ツ!、
%
\ruby{久}{ひさ}しぶりで
\ruby{會}{あ}つて
\ruby{顏}{かほ}を
\ruby{見}{み}ると
\ruby{直}{すぐ}と、
%
\ruby{面白}{おも|しろ}く
\ruby{無}{な}い
\ruby{事}{こと}を
\ruby{云}{い}ひ
\ruby{出}{だ}すやうだけれど、
%
\ruby{猿}{さる}が
\ruby{物}{もの}を
\ruby{含}{ふく}んで
\ruby{溜}{た}めて
\ruby{居}{ゐ}るやうに、
%
\ruby{思}{おも}つた
\ruby{事}{こと}を
\ruby{口}{くち}の
\ruby{内}{うち}にまごつかせては
\ruby{居}{ゐ}られない
\ruby{乃公}{お|れ}だ。
%
\ruby{汝}{きさま}の
\ruby{俊寛}{しゆん|くわん}くさい
\ruby{血}{ち}の
\ruby{氣}{け}の
\ruby{足}{た}らん
\ruby{其面}{その|つら}つきを
\ruby{見}{み}ては、
%
\ruby[g]{狗骨樹}{ひゝらぎ}の
\ruby{皮}{かは}を% 原本通り「皮 か(は)」
\ruby{剝}{む}いたやうに
\ruby{瘠}{や}せつこけ
\ruby{切}{き}つた
\ruby{此樣}{こ|ん}な
\ruby{手}{て}を
\ruby{見}{み}ては、
%
\ruby{云}{い}はずには
\ruby{居}{ゐ}
られん。
%
\ruby{厭}{いや}でも
\ruby{應}{おう}でも
\ruby{聞}{き}いて
\ruby{貰}{もら}はねばならん。
%
\ruby[g]{日方}{ひかた}は
\ruby{汝}{きさま}に
\ruby{苦}{にが}いことを
\ruby{云}{い}はうためにわざ〳〵
\ruby{此家}{こ|ゝ}へ
\ruby{來}{き}たのだ。
%
さあ
\ruby{確乎}{しつ|かり}として
\ruby{熟}{よ}く
\ruby{聞}{き}いて
\ruby{吳}{く}れ
\ruby[g]{水野}{みづの}!。
』

\原本頁{}%
と
\ruby{居{\換字{丈}}高}{ゐ|たけ|だか}となつて
\ruby{聲色}{せい|しよく}
\ruby{激}{はげ}しく
\ruby{說}{と}き
\ruby{出}{いだ}したり。

\Entry{其三十九}

\原本頁{229-5}%
かつて
\ruby[g]{島木}{しまき}が
\ruby{我}{われ}に
\ruby{告}{つ}げし
\ruby{言}{ことば}によりて、
%
\ruby[g]{日方}{ひかた}が
\ruby{今}{いま}
\ruby{何}{なに}を
\ruby{云}{い}はんとするかを
\ruby[g]{水野}{みづの}は
\ruby{猜}{すゐ}し
\ruby{知}{し}れるなり。

\原本頁{229-7}%
\ruby{我}{われ}を
\ruby{思}{おも}ひ
\ruby{吳}{く}るゝ
\ruby{朋友}{ほう|いう}の
\ruby{眞{\換字{情}}}{ま|ごゝろ}より、
%
\ruby{我}{わ}が
\ruby{戀}{こひ}に
\ruby{惱}{なや}めるをは
\ruby{愚}{おろか}なりとして、
%
\ruby{說}{と}き
\ruby{醒}{さ}まし
\ruby{吳}{く}れんとする
\ruby{其}{その}
\ruby{人}{ひと}に
\ruby{對}{むか}ひては、
%
そも〳〵
\ruby{如何}{い|か}なる
\ruby{言葉}{こと|ば}をもて
\ruby{應}{こた}ふべきぞや。
%
\ruby{辯解}{いひ|わけ}すべき% 弁 瓣 辦 辧 辨 辩 (辯)
\ruby{事}{こと}にもあらず、
%
また
\ruby{本}{もと}より
\ruby{云}{い}ひ
\ruby{戾}{もど}くべき
\ruby{事}{こと}にもあらねば、
%
\ruby{愼}{つゝし}みて
\ruby{聞}{き}くよりほかの
\ruby{事}{こと}は
\ruby{無}{な}かるべし。
%
\原本頁{230-1}%
されど
\ruby{人}{ひと}の
\ruby{言葉}{こと|ば}を
\ruby{聞}{き}きて
\ruby{思}{おも}
ひ
\ruby{止}{と}まることの
\ruby{叶}{かな}ふほどならば、
%
\ruby{世}{よ}に
\ruby{戀}{こひ}に
\ruby{悶}{もだ}ゆるものは
\ruby{一人}{ひと|り}も
\ruby{無}{な}くて、
%
\ruby{他人}{ひ|と}に
\ruby{云}{い}はるゝまでもあらず
\ruby{先}{ま}づ
\ruby{我}{われ}と
\ruby{吾}{わ}が
\ruby{{\換字{分}}別}{ふん|べつ}に、
%
よしなき
\ruby{惑}{まどひ}は
\ruby{思}{おも}ひ
\ruby{斷}{き}るべきを、
%
\ruby{諦}{あきら}めても
\ruby{諦}{あきら}めても
\ruby{諦}{あきら}められぬにこそ
\ruby{生命}{いの|ち}の
\ruby{縮}{ちゞ}
むをも
\ruby{忘}{わす}れ
\ruby{人}{ひと}の
\ruby{謗}{そしり}をも
\ruby{顧}{かへり}みで
\ruby{惱}{なや}み
\ruby{苦}{くるし}みはするなれ。
%
それを
\ruby{如何}{い|か}に
\ruby{朋友}{ほう|いう}の
\ruby{眞}{まこと}の
\ruby{{\換字{情}}}{じやう}より
\ruby{{\換字{道}}理}{こと|わり}せめて
\ruby{云}{い}ひ
\ruby{諭}{さと}されたりとて、
%
\ruby{口}{くち}には
\ruby{思}{おも}ひ
\ruby{斷}{た}えたりとも
\ruby{云}{い}ふべし、
%
\ruby{心}{こゝろ}より
\ruby{全}{まつた}く
\ruby{改}{あらた}むる
\ruby{事}{こと}の
\ruby{何}{なん}として
\ruby{成}{な}るべき。
%
たゞ
\ruby{他人}{ひ|と}の
\ruby{親切}{しん|せつ}にて
\ruby{言}{い}ひ
\ruby{吳}{く}るゝ
\ruby{事}{こと}は、
%
よしや
\ruby{少}{すこ}しは
\ruby{無理}{む|り}なる
\ruby{{\換字{廉}}}{かど}ありとも
\ruby{受}{う}くべきが
\ruby{{\換字{道}}}{みち}なれば、
%
\ruby[g]{水野}{みづの}は
\ruby{頭}{かうべ}を
\ruby{垂}{た}れ
\ruby{肩}{かた}を
\ruby{窄}{すぼ}めて
\ruby{默々}{もく|〳〵}と、
%
\ruby{雨}{あめ}に
\ruby{濕}{ぬ}れたる
\ruby{鷄}{とり}の
\ruby{如}{ごと}く
\ruby[<j|]{力}{ちから}
\ruby{無}{な}げに、
%
\ruby{悄然}{せう|ぜん}と
\ruby[g]{日方}{ひかた}の
\ruby{云}{い}ふところをば
\ruby{聞}{き}かんとなしたり。

\原本頁{}%
\ruby[g]{日方}{ひかた}は
\ruby[g]{水野}{みづの}がしほらしき
\ruby{此態}{この|てい}を
\ruby{見}{み}てあはれを
\ruby{催}{もよほ}し、
%
\ruby{新}{あらた}にまた
\ruby{葡萄酒}{ぶ|だう|しゆ}の
\ruby{栓}{せん}を
\ruby{拔}{ぬ}きて、
%
\ruby[g]{水野}{みづの}が
\ruby{座}{ざ}の
\ruby{横}{よこ}に
\ruby{何時}{い|つ}か
\ruby{置}{お}かれたる
\ruby{酒盞}{さか|づき}に
\ruby{注}{つ}ぎ
\ruby{與}{や}りつ。

\原本頁{231-4}%
『しかしまあ
\ruby{其樣}{そ|ん}なに
\ruby{堅}{かた}くならんでも
\ruby{宜}{い}いは
\ruby[g]{水野}{みづの}。
%
\ruby{一杯}{いつ|ぱい}
\ruby{飮}{や}つて
\ruby{吳}{く}れ、
%
わざ〳〵
\ruby{持}{も}つて
\ruby{來}{き}たのだ。
%
\ruby{久}{ひさ}しぶりで
\ruby{汝}{きさま}と
\ruby{一緖}{いつ|しよ}に
\ruby{飮}{や}らうと
\ruby{思}{おも}つて、
%
\ruby[g]{島木}{しまき}のところから
\ruby{徴發}{ちよう|はつ}して
\ruby{來}{き}たのだ。
%
\ruby{何}{なに}も
\ruby{左樣}{さ|う}
\ruby{危坐}{かし|こま}つて
\ruby{貰}{もら}はんでも
\ruby{宜}{い}い、
%
\ruby{汝}{きさま}と
\ruby{乃公}{お|れ}との
\ruby{中}{なか}ぢや
\ruby{無}{な}いか。
\ruby{乃公}{お|れ}はサーベル
\ruby{三昧}{ざん|まい}、
%
\ruby{汝}{きさま}は
\ruby{書籍三昧}{ほ|ん|ざん|まい}、
%
たづさはる
\ruby{{\換字{道}}}{みち}が
\ruby{異}{ちが}ふので
\ruby{姑}{しばら}く
\ruby{{\換字{遠}}}{とほざ}かつたが、
%
\ruby{幾年}{いく|ねん}か
\ruby{{\換字{前}}}{まへ}は
\ruby{一}{ひと}ツに
\ruby{居}{ゐ}て、
%
\ruby{醉眠秋被}{すゐ|みん|あき|ひ}を
\ruby{共}{とも}にし、
%
\ruby{手}{て}を
\ruby{携}{たづさ}へて
\ruby{日}{ひ}に
\ruby{同行}{どう|かう}すといふ
\ruby{{\換字{古}}}{ふる}い
\ruby{詩}{し}の
\ruby{句}{く}の
\ruby{{\換字{通}}}{とほ}りを
\ruby{其}{その}
\ruby{儘}{まゝ}の
\ruby{境界}{きやう|かい}だナアと、
%
ソレ
\ruby{笑}{わら}ひ
\ruby{合}{あ}つた
\ruby{事}{こと}も
\ruby{有}{あ}つた
\ruby{中}{なか}だもの、
%
\ruby{{\換字{遠}}慮}{ゑん|りよ}も
\ruby{斟{\換字{酌}}}{しん|しやく}も
\ruby{有}{あ}らう
\ruby{筈}{はず}は
\ruby{無}{な}い。
%
さあ
\ruby{左樣}{さ|う}いふ
\ruby{中}{なか}だによつて
\ruby{默}{だま}つては
\ruby{居}{を}られんで、
%
\ruby{言語}{こと|ば}に
\ruby{艶}{つや}も
\ruby{付}{つ}けず
\ruby{露骨}{むき|だし}に
\ruby{云}{い}ふが、
%
\ruby[g]{水野}{みづの}!\inhibitglue{}%
\ruby{汝}{きさま}は
\ruby{何}{なん}で
\ruby{{\換字{情}}無}{なさけ|な}い
\ruby{{\換字{魔}}}{ま}に
\ruby{憑}{つ}かれた!。
%
\ruby{我々}{われ|〳〵}の
\ruby{中}{うち}で
\ruby{年}{とし}は
\ruby{{\換字{若}}}{わか}いが、
%
\ruby{聰明}{そう|めい}で
\ruby{慾}{よく}が
\ruby{寡}{すくな}くて
\ruby{學問}{がく|もん}が
\ruby{好}{すき}で、
%
\ruby{立派}{りつ|ぱ}な
\ruby{學者}{がく|しや}か
\ruby{詩仙}{し|せん}かにならうよりほかには
\ruby{爲}{な}りやうも
\ruby{無}{な}いと
\ruby{思}{おも}つて
\ruby{居}{ゐ}た
\ruby{汝}{きさま}が、
%
\ruby{此頃}{この|ごろ}の
\ruby{墮落}{だ|らく}の
\ruby{仕方}{し|かた}は
\ruby{何}{なん}といふ
\ruby{{\換字{情}}無}{なさけ|な}い
\ruby{態}{てい}だ。
%
\ruby{隱}{かく}してもいかん
\ruby{悉皆}{みん|な}
\ruby{知}{し}つて
\ruby{居}{ゐ}る。
%
\ruby{其}{そ}の
\ruby{顏}{かほ}の
\ruby{樵悴}{やつ|れ}は
\ruby{何}{なに}からの
\ruby{事}{こと}だ!。
%
\ruby{其}{そ}の
\ruby{身體}{から|だ}の
\ruby{枯稿}{や|せ}は
\ruby{何故}{なに|ゆゑ}の
\ruby{枯稿}{や|せ}だ。
%
\ruby{憫然}{かあ|いさう}に% 「憫然 か(あ)いさう」
\ruby{其樣}{そ|ん}なひがいすな
\ruby{身體}{から|だ}になつて
\ruby{何}{なに}が
\ruby{出來}{で|き}やう?。
%
\ruby{眼}{め}に
\ruby{見}{み}えるところさへ
\ruby{其{\換字{通}}}{その|とほ}りだもの、
%
まして
\ruby{心}{こゝろ}の
\ruby{{\換字{弱}}}{よわ}りは
\ruby{何程}{どれ|ほど}だらうと
\ruby{思}{おも}ひ
\ruby{{\換字{遣}}}{や}られて、
%
\ruby{汝}{きさま}のために
\ruby{涙}{なみだ}が
\ruby{出}{で}る、
%
\ruby{口惜}{くち|をし}くなる、
%
\ruby{腹}{はら}が
\ruby{立}{た}
つ!。
%
それも
\ruby{此}{これ}も
\ruby{時}{とき}の
\ruby{災人}{わざはひ|ゝと}の
\ruby{爲}{しわざ}の
\ruby{故}{せい}でもあればこそ、% 原本通り「せ(い)」
%
\ruby{汝}{きさま}の
\ruby{一心}{いつ|しん}の
\ruby{据}{す}ゑやうが
\ruby{惡}{わる}くて、
%
\ruby{高}{たか}の
\ruby{知}{し}れた
\ruby{一{\換字{婦}}人}{いち|ふ|じん}に
\ruby{氣}{き}を
\ruby{取}{と}られたからとは、
%
\ruby[g]{{\換字{平}}生}{ひごろ}の
\ruby{汝}{きさま}にも
\ruby{似合}{に|あ}はん
\ruby{愚}{ぐ}な
\ruby{事}{こと}では
\ruby{無}{な}いか。
%
\ruby{{\換字{婦}}女}{をん|な}が
\ruby{何}{なん}だ!。
%
\ruby{戀}{こひ}が
\ruby{何}{なん}だ!。
%
たとひ
\ruby{美女}{び|ぢよ}だらうが
\ruby{賢女}{けん|ぢよ}だらうが、
%
\ruby{我}{われ}を
\ruby{{\換字{迷}}}{まよ}はせりやあ
\ruby{我}{われ}の
\ruby{仇敵}{かた|き}だ。
%
\ruby{男兒}{をと|こ}の
\ruby{正氣}{ほん|き}になつて
\ruby{働}{はたら}かうといふ
\ruby{事業}{し|ごと}の、
%
\ruby{障礙}{しやう|がい}になる
\ruby{奴}{やつ}あ
\ruby{悉皆}{みん|な}
\ruby{仇敵}{かた|き}だ。
%
\ruby{戀}{こひ}たあ
\ruby{料簡}{れう|けん}の
\ruby{弛}{ゆる}みへ
\ruby{出}{で}る
\ruby{黴}{かび}だ、
%
\ruby{閑暇}{ひ|ま}な
\ruby{馬鹿野郎}{ば|か|や|らう}の
\ruby{掌}{て}の
\ruby{中}{なか}の
\ruby[g]{玩弄物}{おもちや}だ。
%
\ruby{世間}{せ|けん}
\ruby{一體}{いつ|たい}の
\ruby{風}{ふう}とは
\ruby{云}{い}ひながら、
%
\ruby{新聞}{しん|ぶん}を
\ruby{見}{み}ても
\ruby{書籍}{ほ|ん}を
\ruby{見}{み}ても、
%
\ruby{戀}{こひ}だ
\ruby{董}{すみれ}だ
\ruby{蝶}{てふ}だ
\ruby{百合}{ゆ|り}だと、
%
\ruby{女臭}{をんな|くさ}いことばかり
\ruby{流行}{は|や}つて
\ruby{居}{ゐ}て、
%
まるで
\ruby{明治}{めい|じ}の
\ruby{{\換字{若}}}{わか}い
\ruby{奴}{やつ}は、
%
\ruby{戀}{こひ}をするために
\ruby{此}{こ}の
\ruby{世}{よ}の
\ruby{中}{なか}へ
\ruby{生}{うま}れて
\ruby{來}{き}たので、
%
\ruby{希望}{の|ぞみ}も
\ruby{事業}{し|ごと}も
\ruby{無}{な}いものゝやうだが、
%
\ruby[g]{水野}{みづの}!\inhibitglue{}%
\ruby{汝}{きさま}まで
\ruby{其}{その}
\ruby{風}{ふう}に
\ruby{感染}{か|ぶ}れたとは
\ruby{何}{なん}たる
\ruby{事}{こつ}た!。
%
\ruby{南風}{みな|み}が
\ruby{吹}{ふ}きやあ
\ruby{北}{きた}へ
\ruby{貼然}{べつ|たり}、
%
\ruby{{\換字{又}}}{また}
\ruby{北風}{き|た}が
\ruby{吹}{ふ}きやあ
\ruby{南}{みなみ}へ
\ruby{貼然}{べつ|たり}する、
%
\ruby{{\換字{平}}々凡々}{へい|〳〵|ぼん|〴〵}の
\ruby{草}{くさ}のやうに、
%
\ruby{自}{みづか}ら
\ruby{立}{た}つて
\ruby{居}{ゐ}る
\ruby{事}{こと}が
\ruby{出來}{で|き}ないとは
\ruby{見下}{み|さ}げた
\ruby{奴}{やつ}だナ。
%
\ruby{其樣}{そ|ん}な
\ruby{腰}{こし}の
\ruby{無}{な}い
\ruby{奴}{やつ}では
\ruby{無}{な}かつたが、
%
\ruby{汝}{きさま}も
\ruby{一世}{いつ|せ}の
\ruby{風潮}{ふう|てう}には
\ruby{捲}{ま}き
\ruby{倒}{たふ}されない
\ruby{男兒}{をと|こ}らしい
\ruby{男兒}{をと|こ}になりかねて、
%
\ruby{波}{なみ}に
\ruby{{\換字{随}}}{したが}ひ%「隨」TODO 変更 ⻖左円辶
\ruby{浪}{なみ}を
\ruby{{\換字{逐}}}{お}ふ
\ruby{意氣地}{い|く|ぢ}
\ruby{無}{な}しなつたか!。
』

\Entry{其四十}

『
\ruby{水野}{みづ|の}、よもや
\ruby{汝}{きさま}はまだ
\ruby{自{\換字{分}}}{じ|ぶん}で
\ruby{云}{い}つた
\ruby{事}{こと}を
\ruby{忘}{わす}れるほどに
\ruby{耄碌}{まう|ろく}は
\ruby{爲}{し}まい。
\ruby[g]{數年前}{すねんぜん}に
\ruby{我々}{われ|〳〵}が
\ruby{寄}{よ}り
\ruby{合}{あ}つて、
\ruby{互}{たがひ}に
\ruby{抱負}{はう|ふ}を
\ruby{{\換字{述}}}{の}べて
\ruby{談笑}{だん|せう}した
\ruby{時}{とき}、
\ruby{大丈夫}{だい|ぢやう|ぶ}の
\ruby{身}{み}をもつて
\ruby{詩文}{し|ぶん}の
\ruby{小{\換字{技}}}{せう|ぎ}に
\ruby{身}{み}を
\ruby{委}{ゆだ}ねやうとは
\ruby{何}{なん}の
\ruby{事}{こと}だ、
\ruby{雛蟲篆刻}{てう|ちう|てん|こく}% 詩文を作るのに、虫を彫り、
%%%%%%%%%%%%%%%%%%%%%%%%%%%%%%%%%%%%% 篆字を刻みつけるように、
%%%%%%%%%%%%%%%%%%%%%%%%%%%%%%%%%%%%% 細部まで技巧で飾りたてること。
%%%%%%%%%%%%%%%%%%%%%%%%%%%%%%%%%%%%% また、そのような技巧に走った内容のない文章。
%%%%%%%%%%%%%%%%%%%%%%%%%%%%%%%%%%%%% 転じて、取るに足らないつまらない小細工。
\ruby{壯夫}{さう|ふ}は%%%%%%%%%%%%%%% 壮年の男性。また、勇壮な男性。
\ruby{爲}{な}さずと、
\ruby{楊雄}{やう|ゆう}づれでさへ
\ruby{云}{い}つて
\ruby{居}{ゐ}るのに、
\ruby{歌}{うた}のポエムのと
\ruby{捏}{こ}ぬ
\ruby{{\換字{返}}}{かへ}して、
\ruby{食}{く}へもせず
\ruby{衣}{き}られもせぬものに
\ruby{苦勞}{く|らう}しやうとは、
\ruby{{\換字{道}}樂{\換字{過}}}{だう|らく|す}ぎて
\ruby{餘}{あま}り
\ruby{詰}{つま}らぬと、
\ruby{乃公}{お|れ}が
\ruby{口}{くち}を
\ruby{極}{きは}めて
\ruby{非難}{ひ|なん}したらば、
\ruby{今}{いま}と
\ruby{異}{ちが}つて
\ruby{元氣}{げん|き}のあつた
\ruby{其頃}{その|ころ}の
\ruby{汝}{きさま}は、
\ruby{眉}{まゆ}を
\ruby{昻}{あ}げ
\ruby{面}{おもて}を
\ruby{正}{たゞし}くして
\ruby{凛然}{りん|ぜん}と
\ruby{答}{こた}へた
\ruby{其}{そ}の
\ruby{挨拶}{あい|さつ}に
\ruby{何}{なん}と
\ruby{云}{い}つた!。
\ruby{食}{しよく}は
\ruby{身}{み}の
\ruby{糧}{かて}、
\ruby{詩}{し}は
\ruby{心}{こゝろ}の
\ruby{糧}{かて}、
\ruby{衣}{きもの}は
\ruby{暑}{あつ}さ
\ruby{寒}{さむ}さに
\ruby{對}{たい}して
\ruby{人}{ひと}の
\ruby{身}{み}を
\ruby{護}{まも}り、
\ruby{詩}{し}は
\ruby{悲}{かなし}みにも
\ruby{怒}{いか}りにも
\ruby{對}{むか}つて
\ruby{人}{ひと}の
\ruby{心}{こゝろ}を
\ruby{調}{とゝの}へる、それを
\ruby{{\換字{益}}}{\換字{江}き}の
\ruby{無}{な}いもののやうに
\ruby{云}{い}ふは
\ruby{淺}{あさ}ましい
\ruby{誤謬}{あや|まり}。
\ruby{貝}{かひ}に
\ruby{眞珠}{しん|じゆ}あり、
\ruby{人}{ひと}に
\ruby{詩}{し}あり、
\ruby{詩歌}{し|か}を
\ruby{除}{のぞ}きて
\ruby{人}{ひと}の
\ruby{作}{つく}れるものに、
\ruby{野菊}{の|ぎく}の
\ruby{花}{はな}の
\ruby{一輪}{いち|りん}だけの
\ruby{美}{うつく}しさのあるものも
\ruby{無}{な}く、
\ruby{阿{\換字{房}}威陽}{あ|ぼう|かん|やう}は
\ruby{羞}{はづか}しく
\ruby{醜}{みにく}い。

\ruby{美}{うつく}しき
\ruby{胸}{むね}の
\ruby{働}{はたら}きの
\ruby{目}{め}にも
\ruby{見}{み}えぬが、
\ruby{凝}{こ}つて
\ruby{詩}{し}となつて
\ruby{文字}{もん|じ}に
\ruby{現}{あらは}るれば、
\ruby{讀}{よ}むもの
\ruby{恍惚}{くわう|こつ}として
\ruby{我}{われ}を
\ruby{忘}{わす}れて、
\ruby{作}{つく}る
\ruby{人}{ひと}が
\ruby{泣}{な}けば
\ruby{泣}{な}き、
\ruby{憤}{いか}れば
\ruby{憤}{いか}る。
されは
\ruby{人間}{ひ|と}の
\ruby{性{\換字{情}}}{せい|じやう}を
\ruby{敦}{あつ}くし、
\ruby{世}{よ}の
\ruby{氣風}{き|ふう}を
\ruby{嘉}{よ}くするもの、
\ruby{詩}{し}に
\ruby{越}{こ}すまのは
\ruby{無}{な}い。
\ruby{大言}{たい|げん}のやうだが
\ruby{此}{こ}の
\ruby{水野}{みづ|の}は、たゞ
\ruby{蝶花}{てふ|はな}のおもしろさや
\ruby{月露}{げつ|ろ}のあはれさを
\ruby{歌}{うた}つてのみ
\ruby{我}{わ}が
\ruby{一生}{いつ|しやう}を
\ruby{{\換字{過}}}{すご}さんとは
\ruby{仕}{し}ない。
\ruby{百年千年}{ひやく|ねん|せん|ねん}にして
\ruby{一}{ひ}ト
\ruby{度出}{たび|い}づる
\ruby{大詩人}{だい|し|じん}の、
\ruby{一代}{いち|だい}の
\ruby{人心}{じん|〳〵}を
\ruby{新}{あらた}にして、
\ruby{萬世}{ばん|せい}に
\ruby{天意}{てん|い}の
\ruby{眞}{まこと}を
\ruby{傳}{つた}へんとする、
\ruby{其}{それ}は
\ruby{及}{およ}ばざる
\ruby{願}{ねがひ}にもせよ、
\ruby[g]{時勢}{じせい}の
\ruby{幇間}{ほう|かん}となつて
\ruby{徳}{とく}を
\ruby{頌}{しよう}するやうな
\ruby{賤}{いや}しい
\ruby{意}{こゝろ}は
\ruby{微塵}{み|ぢん}も
\ruby{有}{も}たない。
\ruby{長}{なが}い
\ruby{眼}{め}で
\ruby{見}{み}て
\ruby{居}{ゐ}て
\ruby{呉}{く}れたまへ、
\ruby{此}{こ}の
\ruby{水野}{みづ|の}はたとひ
\ruby{世}{よ}に
\ruby{背}{そむ}いても
\ruby{世}{よ}と
\ruby{爭}{あらそ}つても、
\ruby{屹度}{きつ|と}
\ruby{血}{ち}もある
\ruby{淚}{なみだ}もある
\ruby{詩}{し}を
\ruby{作}{つく}つて、
\ruby{聖代}{せい|だい}に
\ruby{生}{うま}れ
\ruby{合}{あ}はせた
\ruby{男兒}{をと|こ}
\ruby{一人}{ひと|り}だけの、
\ruby{任務}{つと|め}は
\ruby{其}{それ}で
\ruby{果}{はた}すつもりだと、さも
\ruby{潔}{いさぎ}よく
\ruby{言}{い}つたでは
\ruby{無}{な}いか。
ヤイ
\ruby{水野}{みづ|の}!。
\ruby{詩}{し}の
\ruby{一篇}{いつ|ぺん}も
\ruby{作}{つく}らうといふものが、
\ruby{現在}{げん|ざい}の
\ruby{人{\換字{情}}世態}{にん|じやう|せ|たい}に
\ruby{眼}{め}は
\ruby{離}{はな}すまいが、
\ruby{今}{いま}の
\ruby{日本}{に|ほん}の
\ruby{狀態}{あり|さま}を
\ruby{何樣}{ど|う}
\ruby{思}{おも}ふ?。
\ruby{汝}{きさま}!。
\ruby{今}{いま}の
\ruby{世界}{せ|かい}の
\ruby{狀態}{あり|さま}を
\ruby{何樣}{ど|う}おもふ?。
\ruby{汝}{きさま}!。
\ruby{浪}{なみ}の
\ruby{立}{た}たない
\ruby{海}{うみ}も
\ruby{無}{な}ければ、
\ruby{風}{かぜ}の
\ruby{荒}{あ}れない
\ruby{空}{そら}も
\ruby{無}{な}くつて、
\ruby{國}{くに}は
\ruby{國}{くに}と
\ruby{競}{せ}り
\ruby{合}{あ}ひ、
\ruby{人種}{じん|しゆ}は
\ruby{人種}{じん|しゆ}と
\ruby{鬪}{たゝか}ふ、
\ruby{世界}{せ|かい}の
\ruby{浪風}{なみ|かぜ}は
\ruby{轟々}{がう|〳〵}として、
\ruby{我}{わ}が
\ruby{國}{くに}の
\ruby{濱}{はま}へも
\ruby{磯}{いそ}へも
\ruby{寄}{よ}せて
\ruby{來}{き}て
\ruby{居}{ゐ}るでは
\ruby{無}{な}いか。
それだのに
\ruby{國内}{こく|ない}の
\ruby{狀態}{あり|さま}は
\ruby{何樣}{ど|う}だ。
\ruby[g]{武士{\換字{道}}}{ぶしだう}は
\ruby{廢}{すた}り
\ruby{儒敎}{じゆ|けう}は
\ruby{棄}{す}てられ、
\ruby{舊}{ふる}い
\ruby{敎}{をしへ}は
\ruby{壞}{こは}れ
\ruby{果}{は}てたが、
\ruby{眞面目}{ま|じ|め}に
\ruby{受}{う}け
\ruby{入}{い}れられた
\ruby{新}{あたら}しい
\ruby{敎}{をしへ}も
\ruby{無}{な}く、
\ruby{{\換字{過}}去帳}{か|こ|ちやう}を
\ruby{讀}{よ}むやうに
\ruby{哲人}{てつ|じん}の
\ruby{名}{な}ばかりは
\ruby{忙}{せは}しく
\ruby{呼立}{よび|た}てられて、やがて
\ruby[g]{直片端}{すぐかたつぱし}から
\ruby{忘}{わす}れて
\ruby{行}{ゆ}かれる!。
\ruby{社會}{しや|くわい}に
\ruby{善惡}{ぜん|あく}の
\ruby[g]{目安}{めやす}が
\ruby{無}{な}いから、
\ruby{{\換字{勝}}手}{かつ|て}
\ruby{次第}{し|だい}の
\ruby{{\換字{強}}}{つよ}いもの
\ruby{{\換字{勝}}}{がち}、
\ruby{智慧}{ち|ゑ}で
\ruby{爭}{あらそ}ふ、
\ruby{言{\換字{説}}}{く|ち}で
\ruby{爭}{あらそ}ふ、
\ruby{筆}{ふで}で
\ruby{爭}{あらそ}ふ、
\ruby{金}{かね}で
\ruby{爭}{あらそ}ふ、しかし
\ruby{{\換字{道}}理}{だう|り}で
\ruby{爭}{あらそ}つたのを
\ruby{聞}{き}いた
\ruby{事}{こと}が
\ruby{無}{な}い。
\ruby{金}{かね}を
\ruby{欲}{ほ}しがる、
\ruby{權威}{けん|ゐ}を
\ruby{欲}{ほ}しがる、
\ruby{名}{な}を
\ruby{欲}{ほ}しがる、
\ruby{肉慾}{にく|よく}の
\ruby{滿足}{まん|ぞく}を
\ruby{欲}{ほ}しがる、しかし
\ruby{徳}{とく}を
\ruby{欲}{ほ}しがるものは
\ruby{藥}{くすり}に
\ruby{仕度}{し|たく}も
\ruby{無}{な}い。
\ruby{坊主}{ばう|ず}が
\ruby{役立}{やく|た}たん、
\ruby{新開記者}{しん|ぶん|き|しや}が
\ruby{頼}{たの}もしく
\ruby{無}{な}い、
\ruby{敎育家}{けう|いく|か}が
\ruby{下}{くだ}らん、
\ruby{學者}{がく|しや}は
\ruby{學{\換字{説}}}{がく|せつ}の
\ruby{桂庵}{けい|あん}ばかりで、
\ruby{文學者}{ぶん|がく|しや}は
\ruby{春枝}{はる|\換字{江}}さん
\ruby{靜枝}{しづ|\換字{江}}さんの
\ruby{御機嫌取}{ご|き|げん|と}りに
\ruby{{\換字{過}}}{す}ぎん。
\ruby{世間一體}{せ|けん|いつ|たい}は
\ruby{全}{まる}で
\ruby[g]{不調子}{ふてうし}で、
\ruby{錢}{ぜに}のある
\ruby{時}{とき}はハイカラになり、
\ruby{錢}{ぜに}の
\ruby{無}{な}い
\ruby{時}{とき}は
\ruby{{\換字{蛮}}}{ばん}カラ、
\ruby{忰}{せがれ}は
\ruby{戀愛論}{れん|あい|ろん}、
\ruby{親父}{おや|ぢ}は
\ruby{料理談}{れう|り|だん}、
\ruby{滔々}{たう|〳〵}として
\ruby{一般}{いつ|ぱん}の
\ruby{趣味}{しゆ|み}は
\ruby{日}{ひ}に
\ruby{墮落}{だ|らく}して
\ruby{居}{ゐ}る。
\ruby{想}{おも}つても
\ruby{恐}{おそ}ろしい
\ruby{世界}{せ|かい}のありさま、
\ruby{見}{み}るさへ
\ruby{{\換字{嫌}}}{いや}な
\ruby{人{\換字{情}}}{にん|じやう}の
\ruby{調子}{てう|し}、
\ruby{彼}{あれ}と
\ruby{此}{これ}とを
\ruby{思}{おも}ひ
\ruby{合}{あ}はせれば、
\ruby{此}{こ}の
\ruby{無骨不風流}{ぶ|こつ|ぶ|ふう|りう}の
\ruby{乃公}{お|れ}でさへも、
\ruby{無限}{む|げん}の
\ruby{感{\換字{慨}}}{かん|がい}に
\ruby{打}{う}たれて、
\ruby{詩}{し}のやうなものが
\ruby{呻}{うめ}き
\ruby{出}{だ}したくなる、まして
\ruby{汝}{きさま}が
\ruby{感{\換字{慨}}}{かん|がい}の
\ruby{無}{な}いわけは
\ruby{有}{あ}るまいに
\ruby[g]{何故一片耿々}{なぜいつぺんかう〳〵}たる
\ruby{神州}{しん|しう}
\ruby{男兒}{だん|じ}の
\ruby{丹心}{たん|しん}から、
\ruby{國}{くに}を
\ruby{愛}{あい}し
\ruby{世}{よ}を
\ruby{憂}{うれ}ふるの
\ruby{誠}{まこと}を
\ruby[g]{披瀝}{ひれき}して、
\ruby{詩}{し}でも
\ruby{文章}{ぶん|しやう}でも
\ruby{作}{つく}り
\ruby{出}{だ}して
\ruby{{\換字{呉}}}{く}れぬ?。
\ruby{手緩}{て|ぬる}い
\ruby{事}{こと}では
\ruby{無}{な}い、
\ruby{今}{いま}の
\ruby{今}{いま}でも
\ruby{國{\換字{運}}}{こく|うん}を
\ruby{賭}{と}して
\ruby{戰爭}{たゝ|かひ}を
\ruby{始}{はじ}めればさしずめ
\ruby{乃公}{お|れ}たちは
\ruby{水火}{すゐ|くわ}の
\ruby{中}{なか}にも
\ruby{飛}{と}びこまねばならぬ
\ruby{時}{とき}に
\ruby{逼}{せま}つて
\ruby{居}{ゐ}る
\ruby{塲合}{ば|あひ}だ。
しかし
\ruby{詩}{し}は
\ruby{興}{きよう}が
\ruby{發}{はつ}しないと
\ruby{云}{い}へばそれまでの
\ruby{事}{こと}、
\ruby{出來}{で|き}んなら
\ruby{出來}{で|き}んで
\ruby{是非}{ぜ|ひ}は
\ruby{無}{な}いが、
\ruby{汝}{きさま}までが
\ruby{世}{よ}の
\ruby{風}{ふう}に
\ruby{負}{ま}けて
\ruby{戀愛騒}{れん|あい|さわ}ぎをするとは
\ruby{何事}{なに|ごと}だ。
そんな
\ruby{柔{\換字{弱}}}{にう|じやく}な、
\ruby[g]{性根}{しやうね}の
\ruby{拔}{ぬ}けた
\ruby{事}{こと}で、
\ruby{何}{なん}の
\ruby{詩}{し}も
\ruby{歌}{うた}もあつたものか。
\ruby{時勢}{じ|せい}の
\ruby{幇間}{ほう|かん}とならぬと
\ruby{云}{い}つた
\ruby{其}{そ}の
\ruby{意氣}{い|き}は
\ruby{今}{いま}どこに
\ruby{在}{あ}る?。

\ruby{正}{まさ}しく
\ruby{汝}{きさま}は
\ruby{時勢}{じ|せい}の
\ruby{幇間}{ほう|かん}となつた、
\ruby{奴隷}{ど|れい}となつた、
\ruby{狗}{いぬ}となつた!。
\ruby{男子}{だん|し}の
\ruby{眞}{まこと}の
\ruby{心}{こゝろ}を
\ruby{失}{うしな}つた。
\ruby{男心}{をご|ゝろ}も
\ruby{無}{な}い
\ruby{白痴}{た|はけ}になつたナ。
\ruby{戀}{こひ}の
\ruby{奴}{やつこ}と
\ruby{我}{われ}は
\ruby{死}{し}ぬべしとは
\ruby{何}{なん}たる
\ruby{事}{こと}だ。
\ruby{此}{こ}の
\ruby{普門品}{ふ|もん|ぼん}は
\ruby{誰}{たれ}が
\ruby{誦}{よ}んで、
\ruby{其}{そ}の
\ruby{下}{くだ}らん
\ruby{御籤}{み|くじ}といふものは
\ruby{誰}{たれ}が
\ruby{抽}{と}つた?。
ちらりと
\ruby{聞}{き}けば
\ruby{觀音詣}{くわん|のん|まうで}して、
\ruby{而}{さう}して
\ruby{纔}{やつ}と
\ruby{今}{いま}
\ruby{歸}{かへ}つて
\ruby{來}{き}たのだナ。
\ruby{汝}{きさま}が
\ruby{思}{おも}つて
\ruby{居}{ゐ}る
\ruby{女}{をんな}が
\ruby{大病}{たい|びやう}だとかいふ
\ruby{島木}{しま|き}の
\ruby{談話}{はな|し}も
\ruby{思}{おも}ひ
\ruby{合}{あ}はせて、すつかり
\ruby{汝}{きさま}の
\ruby{{\換字{所}}業}{し|わざ}は
\ruby{{\換字{分}}}{わか}つたが、
\ruby{女}{をんな}のために
\ruby{經}{きやう}を
\ruby{誦}{よ}んだり、
\ruby{御籤}{み|くじ}を
\ruby{取}{と}つたり、わざ〳〵
\ruby{淺草}{あさ|くさ}まで
\ruby{歩}{あゆみ}を
\ruby{{\換字{運}}}{はこ}んだりして
\ruby{居}{ゐ}るのだナ。
エーツ
\ruby{{\換字{情}}無}{なさけ|な}くも
\ruby{衰}{おとろ}へに
\ruby{衰}{おとろ}へた
\ruby{奴}{やつ}だ。

\ruby{書}{しよ}も
\ruby{讀}{よ}み
\ruby{理}{り}にも
\ruby{眛}{くら}からぬ
\ruby{水野}{みづ|の}ともあるものが、
\ruby{如何}{い|か}に
\ruby{{\換字{迷}}}{まよ}へばとて
\ruby{一婦人}{いち|ふ|じん}のために、それほども
\ruby{愚}{ぐ}になつて、
\ruby{成}{な}りきつたか。
\ruby{魔}{ま}に
\ruby{憑}{つ}かれたか
\ruby{何}{なに}に
\ruby{憑}{つ}かれたか、
\ruby{全然}{まる|で}
\ruby{正氣}{しやう|き}の
\ruby{沙汰}{さ|た}では
\ruby{無}{な}いが、
\ruby{男兒}{をと|こ}の
\ruby{魂魄}{たま|しひ}が
\ruby{少許}{すこ|し}でもあれば、
\ruby{正氣}{しやう|き}に
\ruby{{\換字{返}}}{かへ}れ、
\ruby{正氣}{しやう|き}に
\ruby{仕}{し}てやらう。
\ruby{目}{め}を
\ruby{覺}{さ}
ませ
\ruby{水野}{みづ|の}。
』

と
\ruby{云}{い}ひさまに、
\ruby{普門品}{ふ|もん|ぼん}を
\ruby{右手}{みぎ|て}に
\ruby{鷲握}{わし|づか}みにして、
\ruby{左手}{ひだり|て}に
\ruby{水野}{みづ|の}を
\ruby{取}{と}つて
\ruby{引伏}{ひき|ふ}せ、

『
\ruby{{\換字{情}}無}{なさけ|な}い
\ruby{奴}{やつ}だ!。
\ruby{正氣}{しやう|き}に
\ruby{{\換字{返}}}{かへ}らんか、
\ruby{朋友}{とも|だち}の
\ruby{{\換字{情}}誼}{なさ|け}だ、
\ruby{身}{み}に
\ruby{染}{し}みて
\ruby{受}{う}けろ。
』

とビシリ〳〵と
\ruby{續}{つゞ}けさまに
\ruby{打}{う}つたり。


\Entry{其四十一}

\原本頁{}%
\ruby{苟}{いやし}くも
\ruby{男兒}{をと|こ}なり、
%
\ruby{辱}{はづか}しめられて
\ruby{怒}{いかり}を
\ruby{發}{おこ}さゞるはあらず、
%
\ruby{特}{こと}に
\ruby{表面}{うは|べ}こそ
\ruby{柔和}{にう|わ}なれ、
%
\ruby{心}{こゝろ}の
\ruby{底}{そこ}には
\ruby{王侯}{わう|こう}
\ruby{貴人}{きに|ん}をも
\ruby{重}{おも}くは
\ruby{視}{み}ぬほどの
\ruby[g]{水野}{みづの}の、
%
\ruby{如何}{い|か}に
\ruby{朋友}{ほう|いう}の
\ruby{好意}{よし|み}よりの
\ruby{振舞}{ふる|まひ}とは
\ruby{云}{い}へ、
%
\ruby{物}{もの}も
\ruby{云}{い}はさずに
\ruby{手荒}{て|あら}く
\ruby{打擲}{うち|たゝ}かれては、
%
\ruby{勃然}{む|つ}として
\ruby{胸}{むね}に
\ruby{衝}{つ}き
\ruby{上}{あが}るものゝ
\ruby{無}{な}きならねば、
%
\ruby{我}{わ}が
\ruby{襟}{\換字{𛀁}り}を
\ruby{捉}{とら}へし
\ruby[g]{日方}{ひかた}の
\ruby{手}{て}を、
%
\ruby{急}{きふ}に
\ruby{{\換字{捩}}}{ね}ぢ
\ruby{放}{はな}して
\ruby{身}{み}を
\ruby{{\換字{退}}}{ひき}きつ、
%
\ruby{嚴然}{きつ|と}
\ruby{居}{ゐ}ずまひを
\ruby{正}{たゞ}して
\ruby{眼}{め}つき
\ruby{嶮}{けは}しく
\ruby{無言}{む|ごん}に
\ruby{見{\換字{返}}}{み|かへ}しゝが、
%
あゝ
\ruby{思}{おも}へば
\ruby{今}{いま}
\ruby{我}{われ}こゝに
\ruby{何}{なに}をか
\ruby{言}{い}はん、
%
まことや
\ruby{我}{われ}は
\ruby{往時}{むか|し}の
\ruby{我}{われ}ならず、
%
\ruby{比}{くら}べて
\ruby{明}{あき}らかに
\ruby{知}{し}るゝ
\ruby{身體}{から|だ}
\ruby{衰}{おとろ}へに、
%
\ruby{心}{こゝろ}の
\ruby{衰}{おとろ}へも
\ruby{自}{みづか}ら
\ruby{知}{し}る!。
%
\原本頁{242}%
まさかに
\ruby{一旦}{いつ|たん}
\ruby{懷}{いだ}きし
\ruby{本來}{ほん|らい}の
\ruby{志}{こゝろざし}を、
%
\ruby{忘}{わす}れ
\ruby{果}{は}てゝ
\ruby{好}{い}いと
\ruby{思}{おも}ふやうな
\ruby{氣}{き}は
\ruby{持}{も}たねども、
%
\ruby{正直}{しやう|ぢき}を
\ruby{云}{い}へば
\ruby{何時}{い|つ}の
\ruby{間}{ま}にか、
%
\ruby{{\換字{空}}}{あだ}に
\ruby{物}{もの}をのみ
\ruby{思}{おも}ふ
\ruby{癖}{くせ}のつきて、
%
\ruby{自{\換字{分}}}{じ|ぶん}の
\ruby{心}{こゝろ}にも
\ruby{自{\換字{分}}}{じ|ぶん}の
\ruby{心}{こゝろ}が
\ruby{何樣}{ど|う}もならぬといふ
\ruby{{\換字{情}}無}{なさけ|な}い
\ruby{身}{み}の
\ruby{上}{うへ}、
%
これではならぬと
\ruby{思}{おも}ひ
\ruby{{\換字{返}}}{かへ}しても、
%
\ruby{思}{おも}ひ
\ruby{{\換字{返}}}{かへ}す
\ruby{其下}{その|した}より
\ruby{其人}{その|ひと}の
\ruby{事}{こと}ばかりが
\ruby{思}{おも}はれて、
%
\ruby{茫然}{ばう|ぜん}として
\ruby{日}{ひ}を
\ruby{暮}{く}らして
\ruby{仕舞}{し|ま}ふ
\ruby{羞}{はづ}かしい
\ruby{境界}{きよう|がい}。
%
むかしは
\ruby{{\換字{若}}}{わか}い
\ruby{氣勢}{いき|ほひ}に
\ruby{神}{かみ}も
\ruby{佛}{ほとけ}も
\ruby{頼}{たの}まざりしが、
%
\ruby{信}{しん}ぜずには
\ruby{居}{ゐ}られなくなつて
\ruby{今}{いま}
\ruby{信}{しん}ずる
\ruby{此}{こ}の
\ruby{我}{わ}が
\ruby{擧動}{ふる|まひ}を、
%
\ruby{他}{ひと}より
\ruby{見}{み}たらば、
%
\ruby{成程}{なる|ほど}
\ruby{意氣地}{い|く|ぢ}の
\ruby{無}{な}い
\ruby{愚夫愚{\換字{婦}}}{ぐ|ふ|ぐ|ふ}の
\ruby{{\換字{所}}爲}{わ|ざ}と、
%
\ruby{譏}{そし}られても
\ruby{罵}{のゝし}られても
\ruby{仕方}{し|かた}は
\ruby{無}{な}く、
%
\ruby{云}{い}ひ
\ruby{解}{と}かうに
\ruby{云}{い}ひ
\ruby{解}{と}かうところも
\ruby{無}{な}し。
%
されば
\ruby{打}{ぶ}たれても
\ruby{擲}{たゝ}かれても
\ruby{罵}{のゝし}られても、
%
\ruby{男兒}{をと|こ}らしく
\ruby{顏}{かほ}を
\ruby{擡}{あ}げて
\ruby{云}{い}ひ
\ruby{爭}{あらそ}はうには、
%
\ruby{餘}{あま}りに
\ruby{云}{い}ひ
\ruby{甲{\換字{斐}}}{が|ひ}
\ruby{無}{な}くも
\ruby{思}{おも}ひに
\ruby{{\換字{弱}}}{よわ}れる
\ruby{我}{われ}かな。
%
\原本頁{243}%
あゝ
\ruby{我}{われ}ながら
\ruby{{\換字{情}}無}{なさけ|な}くも
\ruby{{\換字{情}}無}{なさけ|な}し。
%
せめて
\ruby{他}{ひと}に
\ruby{打擲}{うち|たゝ}かれて
\ruby{憤}{いかり}を
\ruby{發}{おこ}して、
%
\ruby{思}{おも}ひ
\ruby{切}{き}る
\ruby{事}{こと}の
\ruby{叶}{かな}ふほどの
\ruby{淺}{あさ}き
\ruby{戀}{こひ}ならば、
%
\ruby{此}{こ}の
\ruby{頼}{たの}もしき
\ruby{我}{わ}が
\ruby{友}{とも}の
\ruby{{\換字{情}}誼}{なさ|け}に、
%
\ruby{打}{う}つて〳〵
\ruby{脊骨首骨}{せ|ぼね|くび|ほね}の
\ruby{碎}{くだ}くるほど
\ruby{打}{う}つて
\ruby{貰}{もら}はんを、
%
\ruby{打}{う}たれても
\ruby{擲}{たゝ}かれても
\ruby{我}{わ}が
\ruby{心}{こゝろ}の、
%
\ruby{死}{し}に
\ruby{{\換字{近}}}{ちか}き
\ruby{馬}{うま}のやうに
\ruby{動}{うご}かぬが
\ruby{{\換字{情}}無}{なさけ|な}い!。
%
\ruby{打}{う}たれ
\ruby{辱}{はづか}しめられたが
\ruby{悲}{かな}しくも
\ruby{無}{な}く、
%
\ruby{打}{う}たれて
\ruby{云}{い}ひ
\ruby{抗}{あらそ}ふことの
\ruby{出來}{で|き}ないもの
\ruby{悲}{かな}しくは
\ruby{無}{な}いが、
%
たゞ
\ruby{物}{もの}も
\ruby{云}{い}はず
\ruby{怒}{いかり}りもせずに
\ruby{凝然}{じ|つ}と
\ruby{仕}{し}て
\ruby{居}{ゐ}て
\ruby{人}{ひと}に
\ruby{打}{う}たれたぎりで、
%
\ruby{吾}{わ}が
\ruby{迷}{まよひ}を
\ruby{棄}{す}てやう
\ruby{思}{おもひ}を
\ruby{忘}{わす}れやうといふ
\ruby{意}{こゝろ}が、
%
\ruby{何處}{ど|こ}からも
\ruby{出}{で}て
\ruby{來}{こ}ぬほどに
\ruby{愚}{おろか}にも
\ruby{思}{おも}ひこんだ
\ruby{自{\換字{分}}}{じ|ぶん}が
\ruby{悲}{かな}しい
\ruby{{\換字{情}}無}{なさけ|な}い!。
%
と
\ruby{擡}{あ}げし
\ruby{頭}{かしら}を
\ruby{何時}{いつ|か}かまた
\ruby{下}{さ}げ、
%
\ruby{一度}{ひと|たび}
\ruby{肩}{かた}を
\ruby{聳}{そびや}かしたる
\ruby{身}{み}の
\ruby{復}{また}
\ruby{崩折}{くづ|れ}るれば、
%
\ruby{其}{そ}の
\ruby{樣子}{やう|す}を
\ruby{見}{み}て
\ruby{取}{と}りて
\ruby[g]{日方}{ひかた}はいよ〳〵
\ruby{齒痒}{は|がゆ}がり。

\原本頁{}%
%\原本頁{244}%
『エヽ
\ruby{男兒}{をと|こ}らしくも
\ruby{無}{な}い、
%
\ruby{其面}{その|つら}は
\ruby{何}{なん}だ!。
%
\ruby{身}{み}を
\ruby{{\換字{退}}}{ひ}いて
\ruby{眼}{め}を
\ruby{{\換字{睜}}}{みは}つて
\ruby{乃公}{お|れ}を
\ruby{見}{み}た
\ruby{時}{とき}は、
%
\ruby[g]{水野}{みづの}
\ruby{汝}{きさま}もまだ
\ruby{話}{はな}せると
\ruby{思}{おも}つたが、
%
やがて
\ruby{直}{すぐ}に
\ruby{力}{ちから}の
\ruby{脫}{ぬ}けた
\ruby{泣}{な}きつ
\ruby{面}{つら}になつて、
%
\ruby{涙}{なみだ}ぐんで
\ruby{俯}{うつむ}いたのあ、
%
アヽ
\ruby{見}{み}ぐるしいは。
%
なるほど
\ruby{大{\換字{丈}}夫}{ます|ら|を}のさとき
\ruby{心}{こゝろ}も
\ruby{今}{いま}は
\ruby{無}{な}いだらう、
%
\ruby{其}{そ}の
\ruby{狀態}{やう|す}ぢやあ
\ruby{戀}{こひ}の
\ruby{奴}{やつこ}と
\ruby{死}{し}ぬのも
\ruby{{\換字{遠}}}{とほ}くもあるまい。
%
\ruby{汝}{きさま}は
\ruby{戀}{こひ}の
\ruby{奴}{やつこ}となつて
\ruby{死}{し}ぬのが
\ruby{本望}{ほん|まう}か
\ruby{知}{し}らんが、
%
\ruby{氣}{き}の
\ruby{毒}{どく}だが
\ruby{左樣}{さ|う}は
\ruby{乃公}{お|れ}が
\ruby{死}{し}なさん。
%
ヤイ
\ruby[g]{水野}{みづの}、
%
\ruby[g]{日方}{ひかた}はいたづらに
\ruby{怒罵}{ど|ば}
\ruby{暴行}{ばう|かう}はせん、
%
たゞ
\ruby{大切}{たい|せつ}の
\ruby{一人}{ひと|り}の
\ruby{朋友}{と|も}の
\ruby{爲}{ため}にナ。
%
\ruby{才}{さい}を
\ruby{惜}{をし}み
\ruby{名}{な}を
\ruby{惜}{をし}んで
\ruby{{\換字{遣}}}{や}ればこそ
\ruby{爭}{あらそ}ふのだ。
%
\ruby{乃公}{お|れ}の
\ruby{大切}{たい|せつ}の
\ruby{朋友}{とも|だち}の
\ruby[g]{水野}{みづの}
\ruby[g]{何某}{なにがし}を、
%
\ruby{一{\換字{婦}}人}{いち|ふ|じん}に
\ruby{{\換字{迷}}}{まよ}つて
\ruby{戀}{こひ}に
\ruby{死}{し}んだとは
\ruby{笑}{わら}はさん。
%
とても
\ruby{汝}{きさま}が
\ruby{戀}{こひ}に
\ruby{死}{し}ぬほどならば、
%
\ruby{此}{こ}の
\ruby[g]{日方}{ひかた}
\ruby{八郎}{はち|らう}が
\ruby{打殺}{ぶち|ころ}して
\ruby{{\換字{遣}}}{や}る。
%
\ruby{汝}{きさま}は
\ruby[g]{羽{\換字{勝}}}{はがち}の
\ruby{會}{くわい}へも
\ruby{出}{で}て
\ruby{來}{こ}なかつたほど、
%
\ruby{朋友}{とも|だち}には
\ruby{薄}{うす}く
\ruby{戀}{こひ}に
\ruby{厚}{あつ}くつても、
%
\ruby{乃公}{お|れ}は
\ruby{朋友}{とも|だち}には
\ruby{厚}{あつ}くする、
%
\ruby{戀}{こひ}には
\ruby{關}{かま}はん。
%
\ruby{{\換字{父}}母}{ふ|ぼ}の
\ruby{名}{な}も
\ruby{顯}{あらは}さんで
\ruby{戀}{こひ}に
\ruby{死}{し}なうとは
\ruby{不孝}{ふ|かう}な
\ruby{奴}{やつ}だ、
%
\ruby{國民}{こく|みん}の
\ruby{義務}{ぎ|む}も
\ruby{碌}{ろく}に
\ruby{果}{はた}さんで
\ruby{戀}{こひ}に
\ruby{死}{し}なうとは
\ruby{不義}{ふ|ぎ}な
\ruby{奴}{やつ}だ、
%
\ruby{生}{せい}を
\ruby{此世}{この|よ}に
\ruby{受}{う}けた
\ruby{甲{\換字{斐}}}{か|ひ}も
\ruby{殘}{のこ}さんで
\ruby{{\換字{空}}}{むな}しく
\ruby{死}{し}なうとは
\ruby{卑劣}{ひ|れつ}きはまる!。
%
\ruby{身{\換字{勝}}手}{み|がつ|て}ばかりの
\ruby{穀潰}{ごく|つぶし}しとは
\ruby{戀}{こひ}に
\ruby{死}{し}ぬやうな
\ruby{白痴}{たは|け}た
\ruby{奴}{やつ}の
\ruby{事}{こと}だ。
%
\ruby{才}{さい}を
\ruby{惜}{をし}んで
\ruby{及}{およ}ばん
\ruby{以上}{い|じやう}は
\ruby{名}{な}を
\ruby{惜}{をし}んでやる!。
%
\ruby{汝}{きさま}を
\ruby{不孝}{ふ|かう}
\ruby{不義}{ふ|ぎ}
\ruby{卑劣}{ひ|れつ}な
\ruby{穀潰}{ごく|つぶし}しとは
\ruby{呼}{よ}ばさん、
%
\ruby{戀}{こひ}には
\ruby{死}{し}なせん、
%
\ruby{打殺}{ぶち|ころ}すが
\ruby{何樣}{ど|う}だ。
』

\原本頁{}%
と
\ruby{激語}{げき|ご}は
\ruby{口}{くち}より
\ruby{出}{い}づるに
\ruby{任}{まか}せて、
%
ふたゝび
\ruby[g]{水野}{みづの}を
\ruby{引据}{ひき|す}ゑて
\ruby{打}{う}たんとする
\ruby{時}{とき}、
%
\ruby{隔}{へだて}の
\ruby{襖}{ふすま}はすらりと
\ruby{明}{あ}きて、
%
\ruby{春}{はる}の
\ruby{燕}{つばめ}と
\ruby{身}{み}も
\ruby{輕}{かろ}く、
%
ひらりと
\ruby{躍}{をど}り
\ruby{入}{い}つたる
お
\ruby{濱}{はま}は、
%
\ruby{突然}{いき|なり}に
\ruby[g]{日方}{ひかた}の
\ruby{{\換字{拳}}}{こぶし}に
\ruby{取}{と}りつきて、
%
\ruby{是}{これ}はと
\ruby{{\換字{迷}}}{まよ}ひ
\ruby{疑}{うたが}ふ
\ruby{間}{あひだ}に、
%
\ruby{早}{はや}くも
\ruby{其手}{その|て}より
\ruby{普門品}{ふ|もん|ぼん}を
\ruby{奪}{うば}つて、
%
\ruby{口惜}{く|や}しさ
\ruby{憎}{にく}さ
\ruby{取}{と}り
\ruby{{\換字{交}}}{ま}ぜて
\ruby{籠}{こ}むる
\ruby{力}{ちから}の
\ruby{有}{あ}らん
\ruby{限}{かぎ}りに、
%
\ruby[g]{日方}{ひかた}の
\ruby{五{\換字{分}}苅頭}{ご|ぶ|がり|あたま}をびしや〳〵と
\ruby{打}{う}つたり。


\Entry{其四十二}

\ruby{犬坊丸}{いぬ|ばう|まる}に
\ruby{鞭撻}{むち|う}たれたる
\ruby{曾我}{そ|が}の
\ruby{五郎}{ご|らう}を
\ruby{今}{いま}
\ruby{樣}{やう}にして
\ruby{見}{み}るごとき
\ruby{日方}{ひ|かた}は
\ruby{且}{かつ}
\ruby{驚}{おどろ}き
\ruby{且}{かつ}
\ruby{呆}{あき}れて、
\ruby{眼}{め}を
\ruby{圓}{まる}くして
\ruby{我}{われ}を
\ruby{打}{う}つものを
\ruby{何者}{なに|もの}と
\ruby{屹}{きつ}と
\ruby{睨}{にら}めば、
\ruby{夕日}{ゆふ|ひ}かゞやく
\ruby{緋櫻}{ひ|ざくら}と
\ruby{燃}{も}\換字{𛀁}
\ruby{立}{た}つ
\ruby{顏}{かほ}して、
\ruby{匂}{にほ}やかなる
\ruby{眉}{まゆ}を
\ruby{昻}{あ}げ
\ruby{美}{うつく}しき
\ruby{眼}{め}を
\ruby{瞋}{いか}らせたる
お
\ruby{濱}{はま}は、
\ruby{其時}{その|とき}
\ruby{日方}{ひ|かた}の
\ruby{面上}{めん|じやう}を
\ruby{望}{のぞ}んで
\ruby{普門品}{ふ|もん|ぼん}を
\ruby{抛}{なげう}ち
\ruby{棄}{す}て、
\ruby{物言}{もの|い}ふも
\ruby{可厭}{い|や}と
\ruby{云}{い}はぬばかりに
\ruby{突}{つ}と
\ruby{後向}{うし|ろむ}き、
\ruby{身}{み}を
\ruby{飜}{ひるが}へして
\ruby{倒}{たふ}るゝが
\ruby{如}{ごと}く
\ruby{水野}{みづ|の}の
\ruby{膝}{ひざ}に
\ruby{突伏}{つゝ|ぷ}し、
\ruby{忽}{たちま}ち
\ruby{堰}{せ}き
\ruby{上}{あ}げくる
\ruby{涙}{なみだ}の
\ruby{聲}{こゑ}になつて、

『エヽ
\ruby{口惜}{く|や}しい〳〵、あんまり
\ruby{口惜}{く|や}しい!。
こんな
\ruby{醉漢}{よつ|ぱらひ}の
\ruby{亂暴人}{らん|ばう|にん}に、
\ruby{何故}{な|ぜ}
\ruby{默}{だま}つて
\ruby{打}{ぶ}たれて
\ruby{居無}{ゐ|な}くてはいけないの?。
\ruby{何故}{な|ぜ}
\ruby{打{\換字{返}}}{ぶち|かへ}してやらないの?。
だから
\ruby{觀音樣}{くわん|のん|さま}なんぞ
\ruby{信心}{しん|〴〵}するのはをかしいと
\ruby{云}{い}つて
\ruby{妾}{わたし}が
\ruby{止}{と}めたのに、
\ruby{先生}{せん|せい}が
\ruby{餘}{あんま}り
\ruby{夢中}{む|ちう}になるもんだから、
\ruby{人}{ひと}に
\ruby{馬鹿}{ば|か}にされて
\ruby{此樣}{こ|ん}な
\ruby{目}{め}に
\ruby{會}{あ}ふやうになつたのよ。
それもみんな
\ruby{五十子}{い|そ|こ}さんが
\ruby{惡}{わる}い
お
\ruby{蔭}{かげ}よ、あゝ
\ruby{口惜}{く|や}しい!。
\ruby{妾}{わたし}が
\ruby{口借}{く|や}しくつて
\ruby{仕方}{し|かた}が
\ruby{無}{な}いから、こんな
\ruby{醉漢}{よつ|ぱらひ}の
\ruby{無茶}{む|ちや}な
\ruby{人}{ひと}なんか、
\ruby{早}{はや}く
\ruby{妾}{わたし}の
\ruby{家}{うち}か
\ruby{逐}{お}ひ
\ruby{出}{だ}して
\ruby{{\換字{遣}}}{や}つてよ
\ruby{先生}{せん|せい}!。
ほんとに
\ruby{憎}{にく}らしい
\ruby{厭}{いや}な
\ruby{奴}{やつ}だつちや
\ruby{無}{な}い。
エヽ
\ruby{何故}{な|ぜ}
\ruby{先生}{せん|せい}は
\ruby{默}{だま}つてばかり
\ruby{居}{ゐ}るの!、
\ruby{默}{だま}つてちやあ
\ruby{妾}{わたし}
\ruby{厭}{いや}よ、
\ruby{怒}{おこ}つてよ、
\ruby{怒}{おこ}つてよ、
\ruby{怒}{おこ}り
\ruby{出}{だ}して
\ruby{頂戴}{ちやう|だい}よ、エヾ
\ruby{口惜}{く|や}しい。
』

と
\ruby{身}{み}を
\ruby{揉}{も}んで
\ruby{悶}{もだ}ゆる
\ruby{其}{そ}の
\ruby{八}{や}ツ
\ruby{口}{くち}より
\ruby{襦袢}{じゆ|ばん}の
\ruby{袖}{そで}の
\ruby{紅色}{くれ|なゐ}こぼれて、
\ruby{低}{ひく}く
\ruby{伏}{ふ}したる
\ruby{背中}{せ|なか}つきのすらりと
\ruby{優}{やさ}しきもいとしほらしく、それを
\ruby{中}{なか}にして
\ruby{對}{むか}ひ
\ruby{坐}{ざ}せる
\ruby{痩軀}{やせ|じゝ}の
\ruby{水野}{みづ|の}、
\ruby{肥}{こ}\換字{𛀁}たる
\ruby{日方}{ひ|かた}、
\ruby{揉}{も}みくちやにされて
\ruby{捨}{す}てられたる
\ruby{普門品}{ふ|もん|ぼん}、
\ruby{倒}{たふ}されたる
\ruby{葡萄酒}{ぶ|だう|しゆ}の
\ruby{{\換字{空}}洋盞}{から|こつ|ぷ}、すべで
\ruby{是}{これ}
\ruby{亂}{みだ}れたる
\ruby{一塲}{いち|ぢやう}の
\ruby{景色}{け|しき}ながら、
\ruby{描}{ゑが}かば
\ruby{描}{ゑが}くべき
\ruby{風{\換字{情}}}{ふ|ぜい}あり。

\ruby{水野}{みづ|の}は
\ruby{默}{もく}して
\ruby{石}{いし}の
\ruby{如}{ごと}く
\ruby{語}{かた}らず、
\ruby{思}{おも}はぬものに
\ruby{出}{で}られて
\ruby{日方}{ひ|かた}は
\ruby{困}{こう}じたる
\ruby{時}{とき}、
お
\ruby{鍋}{なべ}は
\ruby{先刻}{さつ|き}より
\ruby{彼方}{かな|た}にて
\ruby{人}{ひと}と
\ruby{應接}{おう|せつ}し
\ruby{居}{ゐ}たりしが、
\ruby{{\換字{終}}}{つひ}に
\ruby{此處}{こ|ゝ}へと
\ruby{一人}{いち|にん}の
\ruby{男}{をとこ}を
\ruby{導}{みちび}き
\ruby{來}{きた}れり。

『オヽ
\ruby{羽{\換字{勝}}}{は|がち}か。
』

『ア、
\ruby{羽{\換字{勝}}}{は|がち}
\ruby{君}{くん}か。
』

\ruby{日方}{ひ|かた}と
\ruby{水野}{みづ|の}とが
\ruby{同時}{どう|じ}に
\ruby{聲}{こゑ}かくるを、
\ruby{眞面目}{ま|じ|め}に
\ruby{受}{う}けながら、いつも
\ruby{變}{かは}らぬ
\ruby{洋服}{やう|ふく}
\ruby{姿}{すがた}の
\ruby{羽{\換字{勝}}}{は|がち}は
\ruby{靜}{しづか}に
\ruby{坐}{ざ}して、

『
\ruby{日方}{ひ|かた}!、
\ruby{君}{きみ}はいかんぞ。
\ruby{今}{いま}
\ruby{此家}{こ|ゝ}の
\ruby{婢}{をんな}に
\ruby{仔細}{し|さい}を
\ruby{聞}{き}いたは。
\ruby{島木}{しま|き}に
\ruby{釘}{くぎ}をさゝれて
\ruby{居}{ゐ}ながら、
\ruby{何}{なに}をするのだ、いかんぞ
\ruby{何樣}{ど|う}も!。
\ruby{水野}{みづ|の}!、
\ruby{久}{ひさ}しく
\ruby{逢}{あ}はなかつたナア。
しかし
\ruby{君}{きみ}も
\ruby{無事}{ぶ|じ}、
\ruby{僕}{ぼく}も
\ruby{無事}{ぶ|じ}で、
お
\ruby{互}{たがひ}に
\ruby{滿足}{まん|ぞく}だ。
\ruby{實}{じつ}は
\ruby{今日}{け|ふ}
\ruby{日方}{ひ|かた}と
\ruby{約束}{やく|そく}して、
\ruby{島木}{しま|き}と
\ruby{三人}{さん|にん}で
\ruby{君}{きみ}を
\ruby{{\換字{尋}}}{たづ}ねる
\ruby{筈}{はず}だつたが、
\ruby{僕}{ぼく}は
\ruby{身體}{から|だ}が
\ruby{忙}{いそ}がしかつたので
\ruby{斷}{ことわ}りを
\ruby{出}{だ}したところが、
\ruby{思}{おも}ひのほか
\ruby{早}{はや}く
\ruby{身體}{から|だ}が
\ruby{明}{あ}いたので、
\ruby{島木}{しま|き}のところへ
\ruby{行}{い}つて
\ruby{見}{み}ると、
\ruby{日方}{ひ|かた}は
\ruby{一人}{ひと|り}で
\ruby{此方}{こつ|ち}へとの
\ruby{事}{こと}だ。
\ruby{島木}{しま|き}は
\ruby{何}{なに}か
\ruby{商業上}{しやう|げふ|じやう}の
\ruby{推算}{すゐ|さん}に
\ruby{身}{み}を
\ruby{入}{い}れて
\ruby{居}{ゐ}る
\ruby{樣子}{やう|す}で、
\ruby{誘}{さそ}つても
\ruby{氣}{き}の
\ruby{無}{な}い
\ruby{{\換字{返}}辭}{へん|じ}をするやうになつて
\ruby{居}{ゐ}るし、そこで
\ruby{一人}{ひと|り}で
\ruby{後}{あと}を
\ruby{{\換字{追}}}{お}つて
\ruby{{\換字{遣}}}{や}つて
\ruby{來}{き}たが、ひよつとすると
\ruby{日方}{ひ|かた}が
\ruby{言葉}{こと|ば}に
\ruby{募}{つの}つて
\ruby{暴}{ばう}な
\ruby{事}{こと}でも
\ruby{仕}{し}はせぬかと
\ruby{思}{おも}つた
\ruby{{\換字{通}}}{とほ}りに、
\ruby{來}{き}て
\ruby{見}{み}ると
\ruby{果}{はた}して
\ruby{亂暴}{らん|ばう}の
\ruby{{\換字{所}}爲}{し|わざ}だ。
\ruby{然}{しか}しまあ
\ruby{僕}{ぼく}に
\ruby{免}{めん}じて
\ruby{赦}{ゆる}して
\ruby{吳}{く}れたまへ、
\ruby{何}{なに}も
\ruby{惡氣}{わる|ぎ}では
\ruby{爲}{せ}ん
\ruby{日方}{ひ|かた}だから。
もう
\ruby{僕}{ぼく}が
\ruby{來}{き}た
\ruby{上}{うへ}は
\ruby{暴}{ばう}はさせん、
\ruby{三人}{さん|にん}で
\ruby{快}{こゝろよ}く
\ruby{靜}{しづか}に
\ruby{話}{はな}さう。
\ruby{水野}{みづ|の}、
\ruby{君}{きみ}は
\ruby{今}{いま}でも
\ruby{甘}{あま}い
\ruby{黨}{たう}の
\ruby{方}{はう}だらう。
\ruby{小兒欺}{こ|ども|だま}しだが
\ruby{舶來菓子}{はく|らい|ぐわ|し}を
\ruby{少}{すこ}
\ruby{持}{も}つて
\ruby{來}{き}た。
\ruby{此邊}{こゝ|ら}には
\ruby{珍}{めづ}しからうと
\ruby{思}{おも}つて、
\ruby{枕絹}{サイデ\換字{子}|キツシエン}とかバタカツプとかいふ
\ruby{奴}{やつ}を
\ruby{持}{も}つて
\ruby{來}{き}たが、
\ruby{舟人}{ふな|のり}の
\ruby{酒}{さけ}を
\ruby{{\換字{強}}}{つよ}く
\ruby{好}{す}かん
\ruby{奴}{やつ}は
\ruby{菓子}{くわ|し}に
\ruby{趣味}{たの|しみ}を
\ruby{有}{も}つ
\ruby{癖}{くせ}が
\ruby{出}{で}るのもをかしいことだ。
さあ
\ruby{日方}{ひ|かた}は
\ruby{飮}{の}むなら
\ruby{飮}{の}め、
\ruby{此方}{こつ|ち}は
\ruby{茶}{ちや}で
\ruby{談}{はな}さう。
』

と
\ruby{常}{つね}には
\ruby{似}{に}ず
\ruby{勉}{つと}めて
\ruby{口數}{くち|かず}きゝて、
\ruby{白}{しら}けきつたる
\ruby{此坐}{この|ざ}を
\ruby{黑}{くろ}めんとすれば、
お
\ruby{濱}{はま}は
\ruby{竊}{そつ}と
\ruby{其人}{その|ひと}を
\ruby{覗}{うかゞ}ひ
\ruby{見}{み}て、
\ruby{正}{たゞ}しげなる
\ruby{此}{こ}の
\ruby{新來}{しん|らい}の
\ruby{客}{きやく}に、
\ruby{泣顏}{なき|がほ}
\ruby{見}{み}せん
\ruby{事}{こと}を
\ruby{憂}{う}くおもひてや、
\ruby{面}{おもて}を
\ruby{蔽}{かく}して
\ruby{{\換字{逃}}}{に}ぐるが
\ruby{如}{ごと}くに
\ruby{此處}{こ|ゝ}を
\ruby{去}{さ}つたり。


\Entry{其四十三}

\ruby{島木}{しま|き}の
\ruby{胸濶}{むね|ひろ}くして
\ruby{能}{よ}く
\ruby{人{\換字{情}}}{にん|じやう}に
\ruby{通}{つう}ぜるといひ、
\ruby[g]{日方}{ひかた}の
\ruby{心剛}{こヽろ|がう}にして
\ruby{{\GWI{u98fd-k}}}{あく}まで
\ruby{義理}{ぎ|り}に
\ruby{仗}{よ}らんとするといひ、
\ruby{其他山瀬}{その|た|やま|せ}といひ
\ruby{楢井}{なら|い}といひ、いづれも
\ruby{我}{われ}に
\ruby{取}{と}りてはおろかならぬ
\ruby{友}{とも}なるが、わけて
\ruby{誰}{たれ}にも
\ruby{彼}{かれ}にも
\ruby{優}{まさ}りて
\ruby{我}{わ}が
\ruby{親}{した}しく
\ruby{語}{かた}らひて、
\ruby{眞}{まこと}の
\ruby{兄}{あに}とも
\ruby{頼}{たの}み
\ruby{思}{おも}へるは
\ruby{此}{こ}の
\ruby[g]{{\GWI{u7fbd-k}\換字{勝}}}{はがち}なり。
\ruby{其性質}{その|せい|しつ}の
\ruby{我}{われ}に
\ruby{似{\GWI{u901a-k}}}{に|かよ}ひたるところのあるが
\ruby{爲}{ため}にや、
\ruby{世}{よ}にいふ
\ruby{合性}{あひ|しやう}といふ
\ruby{事}{こと}の
\ruby{爲}{ため}にや、たヾしは
\ruby[g]{眞實前}{まことまへ}の
\ruby{世}{よ}に
\ruby{如何}{い|か}なる
\ruby{因緣}{いん|\GWI{u1b001}ん}のありての
\ruby{事}{こと}か、
\ruby{他}{ひと}に
\ruby{超}{こ}えて
\ruby{世話}{せ|わ}になりなられつしたる
\ruby{恩義}{おん|ぎ}の
\ruby{關係}{くわん|けい}は
\ruby[g]{島木}{しまき}に
\ruby{及}{およ}ばず、
\ruby{一}{ひと}ツ
\ruby{窓}{まど}の
\ruby{光}{ひかり}を
\ruby{各自}{めい|〳〵}の
\ruby{机}{つくゑ}に
\ruby{分}{わか}つて、
\ruby{奇文}{き|ぶん}を
\ruby{共}{とも}に
\ruby{賞}{しやう}し
\ruby{疑義}{ぎ|ヽ}を
\ruby{相質}{あひ|ただ}す
\ruby{學問}{がく|もん}の
\ruby{交}{まじは}りは
\ruby{山瀬}{やま|せ}に
\ruby{如}{し}かざりしかども、たヾ
\ruby{何}{なん}と
\ruby{無}{な}く
\ruby{我彼}{われ|かれ}を
\ruby{他}{ほか}ならず
\ruby{懷}{なつか}しめば、
\ruby{彼}{かれ}もまた
\ruby{我}{われ}を
\ruby{他}{ほか}ならず
\ruby{愛}{あい}して、
\ruby{{\換字{分}}桃}{ぶん|たう}の
\ruby{痴}{し}れたる
\ruby{{\換字{情}}}{じやう}こそは
\ruby{有}{あ}らざりけれ、
\ruby{斷金}{だん|きん}のまことの
\ruby{契}{ちぎり}は
\ruby{淺}{あさ}からざりしなり。

されど
\ruby{人}{ひと}おの〳〵
\ruby{望}{のぞ}む
\ruby{處}{ところ}を
\ruby{異}{こと}にすれば、
\ruby{彼}{かれ}は
\ruby{一帆}{いつ|ぱん}の
\ruby{風}{かぜ}に
\ruby{萬里}{ばん|り}の
\ruby{海}{うみ}を
\ruby{渡}{わた}つて
\ruby[g]{波瀾淘湧}{はらんきようゆう}の
\ruby{中}{うち}に
\ruby{身}{み}を
\ruby{托}{たく}するの
\ruby{船人}{ふな|びと}となり、
\ruby{我}{われ}は
\ruby{{\換字{半}}夜}{はん|や}の
\ruby{燈}{ともしび}に
\ruby{幾卷}{いく|くわん}の
\ruby{書}{しよ}と
\ruby{對}{たい}して
\ruby{寂寞}{じやく|まく}たる
\ruby{小齋}{せう|さい}の
\ruby{裏}{うち}に
\ruby{思}{おもひ}を
\ruby{錬}{ね}るの
\ruby{學究}{がく|きう}たるを
\ruby{甘}{あま}んぜるより、
\ruby{相見}{あひ|み}ざる
\ruby{月日}{つき|ひ}はおのづと
\ruby{多}{おほ}くなり
\ruby{行}{ゆ}きしが、しかも
\ruby{相思}{あひ|おも}ふ
\ruby{心}{こヽろ}は
\ruby{更}{さら}に
\ruby{變}{かは}らず、
\ruby{彼海上}{かれ|かい|じやう}にありと
\ruby{知}{し}る
\ruby{時}{とき}は、
\ruby{風}{かぜ}の
\ruby{曉}{あした}、
\ruby{{\GWI{u96ea-k}}}{ゆき}の
\ruby{夕}{ゆふべ}、あヽ
\ruby[g]{{\GWI{u7fbd-k}\換字{勝}}}{はがち}はと
\ruby[g]{此方}{こなた}に
\ruby{思}{おも}はぬ
\ruby{折}{をり}も
\ruby{無}{な}ければ、
\ruby{富士}{ふ|じ}の
\ruby{高根}{たか|ね}も
\ruby{浪}{なみ}に
\ruby{{\換字{消}}}{き}\GWI{u1b001}て
\ruby{夢}{ゆめ}ならでは
\ruby{日本}{に|ほん}の
\ruby{見}{み}えぬ
\ruby[g]{異{\GWI{u90f7-var-001}}}{いきやう}の
\ruby{津}{つ}に
\ruby{在}{あ}りても
\ruby[g]{彼方}{かなた}も
\ruby{我}{われ}を
\ruby{{\GWI{u7336-k}}}{なほ}
\ruby{思}{おも}ひ
\ruby{{\換字{呉}}}{く}れて、
\ruby{他邦}{よ|そ}の
\ruby{港}{みなと}を
\ruby{目}{め}の
\ruby{前}{まへ}に
\ruby{見}{み}る
\ruby{繪葉書}{ゑ|は|がき}の、
\ruby{此岬}{この|みさき}の
\ruby{下此}{した|こ}の
\ruby{水}{みづ}の
\ruby{上}に
\ruby[g]{汝}{なんじ}の
\ruby{友}{とも}の
\ruby[g]{{\GWI{u7fbd-k}\換字{勝}}}{はがち}
\ruby{在}{あ}りと、
\ruby{村居}{そん|きよ}の
\ruby{閑}{しづか}なる
\ruby{机}{つくゑ}の
\ruby{上}{うへ}に、
\ruby{天}{てん}の
\ruby{一方}{いつ|ぱう}よ
\ruby{溫}{あたヽか}き
\ruby{{\換字{情}}}{こヽろ}を
\ruby{寄}{よ}せ
\ruby{{\換字{呉}}}{く}るヽこと
\ruby{數々}{しば|〳〵}なりき。

\ruby{我}{われ}とはかくの
\ruby{如}{ごと}き
\ruby{中}{なか}なる
\ruby[g]{{\GWI{u7fbd-k}\換字{勝}}}{はがち}が
\ruby{久}{ひさ}しぶりにて
\ruby{歸}{かへ}りしを
\ruby{{\GWI{u8fce-k}}}{むこ}ふるの
\ruby{會}{くわい}に、
\ruby{一篇}{いつ|ぺん}の
\ruby{歌}{うた}をも
\ruby{寄}{よ}すること
\ruby{無}{な}く、
\ruby{數句}{すう|く}の
\ruby{語}{ことば}をも
\ruby{交}{まじ}ふること
\ruby{無}{な}くして、
\ruby{全}{まつた}く
\ruby{面}{おもて}を
\ruby{出}{いだ}さヾりしは、
\ruby[g]{水野}{みづの}の
\ruby{胸濟}{むね|す}まず
\ruby{思}{おも}へるとろなりしが、
\ruby{其}{そ}の
\ruby{事彼}{こと|か}の
\ruby{事}{こと}の
\ruby{煩累}{わず|らひ}に
\ruby{心}{こヽろ}を
\ruby{取}{と}られて、
\ruby{其後}{その|ヽち}も
\ruby{思}{おも}ひながら
\ruby{{\換字{尋}}}{たづ}ねさへせざりし
\ruby{其}{そ}の
\ruby[g]{{\GWI{u7fbd-k}\換字{勝}}}{はがち}に、
\ruby{忽然}{こつ|ぜん}として
\ruby{{\換字{尋}}}{たづ}ね
\ruby{寄}{よ}られては、あヽ
\ruby{此人}{この|ひと}を
\ruby{{\換字{尋}}}{たづ}ねでは
\ruby{濟}{す}まざりしものを、
\ruby{差當}{さし|あた}りての
\ruby{苦}{くるし}きおもひにのみ
\ruby{惹}{ひ}かされて、
\ruby{我}{われ}に
\ruby{疎}{うと}き
\ruby{意}{こゝろ}の
\ruby{露}{つゆ}ありてにはあらねど、おのづから
\ruby{人}{ひと}の
\ruby{{\換字{情}}}{なさけ}を
\ruby{空}{あだ}にしたるやうになりし
\ruby{悲}{かな}しさ、と
\ruby{其懷}{その|なつか}しき
\ruby{顏}{かほ}を
\ruby{一}{ひ}ト
\ruby{目見}{め|み}るより
\ruby{早}{はや}く、
\ruby{何}{なに}より
\ruby{先}{さき}に
\ruby{我}{わ}が
\ruby{振舞}{ふる|まひ}の
\ruby{{\換字{勝}}手{\GWI{u904e-k}}}{かつ|て|す}ぎたるが
\ruby{羞}{はづか}しくなりて、
\ruby{正}{まさ}しくは
\ruby{對}{むか}ひ
\ruby{見}{み}る
\ruby{事}{こと}も
\ruby{叶}{かな}はぬやうの
\ruby[g]{心地}{こヽち}しつ、
\ruby{滔々}{たう|〳〵}として
\ruby[g]{日方}{ひかた}の
\ruby{我}{われ}
\ruby{諫}{いさ}めくれたる
\ruby{其}{そ}の
\ruby{幾千言}{いく|せん|げん}を
\ruby{聞}{き}けるよりも、
\ruby{我}{われ}と
\ruby{我}{わ}が
\ruby{果敢無}{は|か|な}き
\ruby{戀}{こひ}に
\ruby{{\GWI{u8ff7-k}}}{まよ}ひて、
\ruby{此}{こ}の
\ruby{{\換字{情}}}{じやう}の
\ruby{篤}{あつ}く
\ruby{義}{ぎ}の
\ruby{{\換字{強}}}{つよ}き
\ruby{{\換字{尊}}}{たつと}むべき
\ruby{友}{とも}に
\ruby{負}{そむ}きたる
\ruby{罪}{つみ}の
\ruby{輕}{かろ}からぬをおぼえ、よし
\ruby{無}{な}き
\ruby{想}{おもひ}にのみ
\ruby{沈}{しづ}める
\ruby{昨日今日}{きの|ふ|け|ふ}の
\ruby{我}{わ}が
\ruby{愚}{おろか}しきをば
\ruby{自}{みづか}ら
\ruby{慚}{は}ぢ
\ruby{自}{みづか}ら
\ruby{責}{せ}むるの
\ruby{{\換字{情}}}{じやう}は
\ruby{燬}{や}くが
\ruby{如}{ごと}くに
\ruby{起}{おこ}りて、
\ruby[g]{嗚呼我心裏}{あヽわれこヽろ}に
\ruby{物無}{もの|な}くして
\ruby{懐}{なつか}しき
\ruby{此}{こ}の
\ruby{友}{とも}と
\ruby{今}{いま}こヽに
\ruby{相語}{あい|かた}らば、
\ruby{如何}{い|か}ばかり
\ruby{今日}{け|ふ}の
\ruby[g]{團欒}{まどゐ}の
\ruby{嬉}{うれ}しく
\ruby{樂}{たの}しからんを、
\ruby[g]{彼方}{かなた}は
\ruby{相}{あひ}も
\ruby{變}{かは}らず
\ruby{胸}{むね}を
\ruby{開}{ひら}きて
\ruby{物語}{もの|がた}れど、
\ruby{我}{われ}は
\ruby{人}{ひと}には
\ruby{告}{つ}け
\ruby{難}{がた}き
\ruby{私{\換字{情}}}{わた|くし}を
\ruby{胸}{むね}に
\ruby{抱}{いだ}き
\ruby{居}{を}りて、
\ruby[g]{往時}{むかし}の
\ruby{無邪氣}{む|じや|き}の
\ruby{我}{われ}ならねば、
\ruby{隔}{へだ}つる
\ruby{氣}{き}の
\ruby{更}{さら}にあるにはあらねど、
\ruby{水}{みづ}と
\ruby{油}{あぶら}との
\ruby{一}{ひと}つになりがたきやうに、
\ruby{何處}{ど|こ}と
\ruby{無}{な}く
\ruby{奥底}{おく|そこ}なくは
\ruby{打解}{うち|と}け
\ruby{難}{がた}き
\ruby[g]{心地}{こヽち}して、
\ruby[g]{言葉}{ことば}に
\ruby{餘}{あま}る
\ruby{思}{おもひ}はありながらも、
\ruby{所以知}{ゆ|ゑ|し}らず
\ruby[g]{自然}{おのづ}と
\ruby{我}{わ}が
\ruby{口}{くち}の
\ruby{結}{むす}ばるヽを
\ruby{何}{なん}とせんと、
\ruby[g]{水野}{みづの}は
\ruby{私}{ひそか}に
\ruby{自}{みづか}ら
\ruby{苦}{くる}しめり。

\ruby{見}{み}れば
\ruby[g]{日方}{ひかた}の
\ruby{言}{い}ひしに
\ruby{露差}{つゆ|たが}はず、
\ruby{生來}{せい|らい}の
\ruby{沈毅}{ちん|き}の
\ruby{氣性}{きし|やう}は
\ruby{{\換字{浮}}世}{うき|よ}に
\ruby{鍛}{きた}はれて、いよ〳〵
\ruby{萎}{ひる}まず
\ruby{怯}{おく}れぬ
\ruby{大丈夫}{だい|ぢやう|ぶ}となりたるは
\ruby{其}{そ}の
\ruby{額}{ひたひ}には
\ruby{曇}{くもり}の
\ruby{{\GWI{u7d55-k}}}{た}えて
\ruby{無}{な}くて、
\ruby{眼}{め}には
\ruby{{\GWI{u92b3-k}}}{するど}さの
\ruby{加}{くは}はりたるにも
\ruby{知}{し}られ
\ruby{眞窣}{しん|そつ}なれども
\ruby{擧動}{きよ|どう}にruby{威}{ゐ}ありおちつきあり、
\ruby{{\換字{平}}易}{へい|ヽ}なれども
\ruby{言葉}{こと|ば}に
\ruby{思慮}{し|りよ}あり
\ruby{斟酌}{しん|しやく}あるに、あだには
\ruby{月日}{つき|ひ}を
\ruby{經}{へ}ざりしを
\ruby{示}{しめ}したり。

\ruby[g]{水野}{みづの}に
\ruby[g]{水野}{みづの}の
\ruby[g]{{\換字{所}}思}{おもひ}あれば、
\ruby[g]{{\GWI{u7fbd-k}\換字{勝}}}{はがち}にも
\ruby[g]{{\GWI{u7fbd-k}\換字{勝}}}{はがち}の
\ruby[g]{{\換字{所}}思}{おもひ}ありて、
\ruby{累々}{るゐ|〳〵}として
\ruby{喪家}{さう|か}の
\ruby{狗}{いぬ}のごとく
\ruby{衰}{おとろ}へ
\ruby{果}{は}てたる
\ruby{我}{わ}が
\ruby{友}{とも}の
\ruby{容態}{よう|す}をば、しばし
\ruby[g]{無言}{むごん}にして
\ruby[g]{{\GWI{u7fbd-k}\換字{勝}}}{はがち}は
\ruby{眺}{なが}めしが、たヾ
\ruby[g]{日方}{ひかた}のみは
\ruby{思}{おも}つては
\ruby{言}{い}はずに
\ruby{居}{ゐ}ず、
\ruby{一旦}{いつ|たん}は
\ruby[g]{{\GWI{u7fbd-k}\換字{勝}}}{はがち}を
\ruby{憚}{はヾか}りて
\ruby{默}{もく}せしが、
\ruby{堪}{こら}へ
\ruby{{\換字{兼}}}{か}ねてか
\ruby{忽}{たちま}ちまた、

『
\ruby[g]{水野}{みづの}、』

と
\ruby{一}{ひ}ㇳ
\ruby{聲呼}{こゑ|よ}びかけたり。


\Entry{其四十四}

% メモ 校正終了 2024-05-08 2024-06-05
\原本頁{256-5}%
お
\ruby{濱}{はま}は
\ruby[g]{何處}{いづく }にか
\ruby{去}{さ}つて
\ruby{復}{また}
\ruby{現}{あら}れず、
%
むくつけき
\ruby{田舎女}{ゐな|か|もの}の
お
\ruby{鍋}{なべ}は
\ruby{茶}{ちや}を
もて
\ruby{來}{きた}りしが、
%
\ruby{先}{ま}づ
\ruby{無作法}{ぶ|さ|はふ}に
\ruby[g]{人々}{ひと〴〵}の
\ruby{顏}{かほ}を
\ruby[g]{見渡}{み わた}して、
%
\ruby{初}{はじめ}に
\ruby[g]{羽{\換字{勝}}}{は がち}が
\ruby{{\換字{前}}}{まへ}に
\ruby[g]{一盞}{いつさん}を
\ruby{薦}{すゝ}め、% 踊り字調整「〻(二の字点、揺すり点)に見えるが(ゝ)」
%
\ruby{次}{つぎ}に
\ruby[g]{水野}{みづの }が
\ruby{{\換字{前}}}{まへ}に
また
\ruby[g]{一盞}{いつさん}を
\ruby{置}{お}き、
%
\ruby{茶}{ちや}は
\ruby{注}{つ}ぎて
\ruby{其}{そ}の
\ruby{盞}{さん}を
\ruby{滿}{み}たしながら
\ruby[g]{日方}{ひ かた}が
\ruby{{\換字{前}}}{まへ}には
\ruby{取}{と}りても
\ruby{與}{や}らず、

\原本頁{256-9}%
『
\ruby[||j>]{汝}{おめへ}
\ruby[||j>]{樣}{ さま}は
% \ruby{汝樣}{おめへ|さま}は
\ruby[g]{{\換字{勝}}手}{かつて }に
\ruby{取}{と}つて
\ruby{飮}{の}まつせえ。
』

\原本頁{256-10}%
と
\ruby{云}{い}はぬばかりの
\ruby{顏}{かほ}つきしつ、
%
\ruby[g]{其邊}{あたり }の
\ruby{亂}{みだ}れたるを
\ruby{取片付}{とり|かた|づ}けて、
%
\ruby{默}{だま}つて
\ruby{{\換字{退}}}{しりぞ}き
\ruby{去}{さ}れば、
%
\ruby[g]{水野}{みづの }は
\ruby{氣}{き}の
\ruby{毒}{どく}さに
\ruby{堪}{た}へずして、
%
\ruby{自}{みづか}ら
\ruby[g]{茶盞}{ちやさん}を
\ruby{取}{と}つて
\ruby[g]{日方}{ひ かた}に
\ruby{與}{あた}へたり。

\原本頁{257-3}%
\ruby[g]{日方}{ひ かた}は
\ruby[g]{此等}{これら }の
\ruby[g]{瑣事}{さ じ }には
\ruby[||j>]{頓}{とん}
\ruby[||j>]{着}{ちやく}もせず、
% \ruby{頓着}{とん|ちやく}もせず、
%
\ruby[g]{{\換字{感}}慨}{かんがい}に
\ruby{堪}{た}へぬ
\ruby{面}{おもて}の
\ruby{色}{いろ}、
%
\ruby[g]{{\換字{睜}}開}{み は }れる
\ruby{眼}{め}には
\ruby{露}{つゆ}をさへ
\ruby{宿}{やど}して、

\原本頁{257-5}%
『
\ruby[g]{水野}{みづの }!。
%
もう
\ruby[g]{乃公}{お れ }は
\ruby{一}{ひ}ト
\ruby{{\換字{通}}}{とほ}り
\ruby{云}{い}ひ
\ruby{盡}{つく}したから
\ruby{繰}{く}り
\ruby{{\換字{返}}}{かへ}して
また
\原本頁{257-6}\改行%
\ruby{言}{い}ふのでは
\ruby{無}{な}いが、
%
\ruby[g]{如何}{い か }に
\ruby{心}{こゝろ}が% 踊り字調整「〻(二の字点、揺すり点)に見えるが(ゝ)」
\ruby{{\換字{弱}}}{よわ}つた
ればとて、
%
\ruby{何}{なん}といふ
\ruby{汝}{きさま}の
\原本頁{257-7}\改行%
\ruby{衰}{おとろ}へ
\ruby{方}{かた}だ!。
%
\ruby{{\換字{迷}}}{まよ}ふなら
\ruby{{\換字{迷}}}{まよ}ふで
\ruby[g]{仕方}{し かた}は
\ruby{無}{な}いやうなものゝ、% 踊り字調整「〻(二の字点、揺すり点)に見えるが(ゝ)」
%
\ruby{同}{おな}じ
\ruby{{\換字{迷}}}{まよ}ひにも
それ〴〵があらう。
%
\ruby[g]{何故}{な ぜ }
\ruby{{\換字{迷}}}{まよ}ふにしても
\ruby[g]{男兒}{をとこ }らしくは
\ruby{{\換字{迷}}}{まよ}はぬ?。
%
\ruby{汝}{きさま}の
\ruby{衰}{おとろ}へに
\ruby{衰}{おとろ}へ
\ruby{果}{は}てゝ% 踊り字調整「〻(二の字点、揺すり点)に見えるが(ゝ)」
\ruby{女}{をんな}の
\ruby{腐}{くさ}つた
のゝやうに% 踊り字調整「〻(二の字点、揺すり点)に見えるが(ゝ)」
\ruby{成}{な}り
\ruby{果}{は}てたのが、
%
\ruby{何}{なに}より
\ruby{彼}{か}より
\ruby[g]{{\換字{情}}無}{なさけな}いは。
%
\ruby[g]{汝は}{きさま }
\ruby{本}{もと}より
\ruby[||j>]{剛}{がう}
\ruby[||j>]{{\換字{強}}}{きやう}な
% \ruby{剛{\換字{強}}}{がう|きやう}な
\ruby[g]{鐵石}{てつせき}の
\ruby{男}{をとこ}といふのでは
\ruby{無}{な}かつたが、
%
\ruby[g]{外面}{うはべ }は
\ruby{柔}{やはら}かでも
\ruby{事}{こと}によつては、
%
\ruby{人}{ひと}と
\ruby{爭}{あらそ}つ
\原本頁{258-1}\改行%
て% 「て」以降「やう」で29文字ある。二箇所の「、」部分を詰めたと思われる
\ruby{後}{あと}へは
\ruby{决}{けつ}して
\ruby{{\換字{退}}}{ひ}かぬ、
%
\ruby{怖}{おそろ}しい
\ruby[g]{氣合}{き あひ}を
\ruby{含}{ふく}んだ
\ruby{奴}{やつ}で、
%
\ruby{釅}{きぶ}い
\ruby{醋}{す}の
やう% 「て」以降「やう」で29文字ある。二箇所の「、」部分を詰めたと思われる
\原本頁{258-2}\改行%
な
ところが
あると、
%
\ruby[g]{{\換字{平}}生}{つね〴〵}% ルビ調整(原本通り)
\ruby[g]{乃公}{お れ }が
\ruby{{\換字{評}}}{ひやう}した
ほどの
\ruby[g]{男兒}{をとこ }であつたが
\ruby{今}{いま}は
\ruby[g]{何樣}{ど う }だ。
%
\ruby{醋}{す}なら
\ruby{醋}{す}は
\ruby{腐}{くさ}つて
\ruby[g]{仕舞}{し ま }つたのか
\ruby{黴}{か}びて
\ruby[g]{仕舞}{し ま }つたのか
\改行% 校正作業の簡略化のため
、
%
\原本頁{258-4}\改行%
\ruby[g]{乃公}{お れ }に
\ruby{打}{う}たれて
\ruby[g]{抵抗}{てむかひ}も
せぬやうになつたとは
\ruby[g]{嗚呼}{あ ゝ }% 踊り字調整「〻(二の字点、揺すり点)に見えるが(ゝ)」
\ruby[g]{{\換字{情}}無}{なさけな}い!。
%
これ
\ruby{眼}{め}を
\ruby{開}{あ}いて
\ruby[g]{天地}{てんち }を
\ruby{見}{み}ろ!。
%
\ruby[g]{畫工}{ゑ かき}には
\ruby{畫}{ゑ}を
\ruby{敎}{をし}へぬ
\ruby[g]{草木}{くさき }も
\ruby{無}{な}い、
%
\原本頁{258-6}\改行%
\ruby[g]{男兒}{をとこ }を
\ruby{磨}{みが}かう
といふ
ものには
\ruby{我}{わ}が
\ruby[g]{精神}{こゝろ }を% 踊り字調整「〻(二の字点、揺すり点)に見えるが(ゝ)」
\ruby{奮}{ふる}はせて
\ruby{歩}{あゆみ}を
\ruby{{\換字{進}}}{すゝ}ます% 踊り字調整「〻(二の字点、揺すり点)に見えるが(ゝ)」
\ruby{鞭}{むち}や
\ruby{刺馬輪}{し|ば|りん}
% 刺馬輪 は 拍車 のようである
% 江戸時代に入ってきた当時は拍車(spur)を表す和名はなく、仮名で「スポール」と記していた
% 本格的に拍車が利用されるようになるのは明治時代以降
% 明治初期になると和訳した名称「刺馬輪」が登場
% 現在の拍車は明治中期に登場した単語
で
\ruby{無}{な}いものは
\ruby{無}{な}い!。
%
\ruby{見}{み}なかつたか
\ruby{盲目{\換字{漢}}}{め|く|ら}!、
%
\ruby{氣}{き}が
\ruby{注}{つ}かんか
\ruby{放心{\換字{漢}}}{う|つ|け}!、
%
\ruby[g]{此家}{こ ゝ }の% 踊り字調整「〻(二の字点、揺すり点)に見えるが(ゝ)」
\ruby[g]{小娘}{こむすめ}が
\ruby{何}{なに}を
\ruby{仕}{し}たぞ。
%
\ruby{齡}{とし}はたつた
\ruby[g]{十五}{じふご }か
\原本頁{258-9}\改行%
\ruby[g]{十六}{じふろく}かで、
%
\ruby[g]{乃公}{お れ }の
\ruby{一}{ひ}ト
\ruby[g]{攫に}{つかみ }
も
\ruby{足}{た}らぬ
\ruby{優}{やさ}しい
\ruby[g]{身體}{からだ }、
%
それでも
\ruby[g]{流石}{さすが }に
\ruby[g]{日本}{に ほん}の
\ruby{女}{をんな}だ、
%
\ruby[g]{{\換字{平}}生}{へいぜい}% ルビ調整(原本通り)
\ruby{一}{ひと}ツ
\ruby{家}{いへ}に
\ruby{居}{ゐ}る
\ruby[g]{汝が}{きさま }、
\ruby[g]{乃公}{お れ }に
\ruby{撲}{う}たれ
\ruby[<j>]{辱}{はづかし}め
られるのを
\ruby{見}{み}ては
\ruby[g]{慨然}{がいぜん}
として、
%
\ruby{身}{み}を
\ruby{挺}{ぬき}んでゝ% 踊り字調整「〻(二の字点、揺すり点)に見えるが(ゝ)」
\ruby{汝}{きさま}を
\ruby{護}{かば}つて
\ruby[g]{乃公}{お れ }に
\ruby{當}{あた}り
\原本頁{259-1}\改行%
あの
\ruby{愛}{あい}らしい
\ruby{美}{うつく}しい
\ruby{眼}{め}から、
%
\ruby[g]{寶石}{ほうせき}の
やうな
\ruby{光}{ひかり}を
\ruby{輝}{かゞや}かして、% 踊り字調整「〻(二の字点、揺すり点)に濁点に見えるが(ゞ)」
%
\ruby[g]{眞紅}{まつか }な
\ruby{顏}{かほ}に
\ruby{血}{ち}を
\ruby{沸}{にや}して
\ruby{打}{う}つて
かゝつたでは% 踊り字調整「〻(二の字点、揺すり点)に見えるが(ゝ)」
\ruby{無}{な}いか!。
%
\ruby[g]{女性}{をんな }だ、
%
\ruby[g]{小兒}{こ ども}だ、
%
\ruby[g]{孱{\換字{弱}}}{か よわ}い
\ruby{娘}{むすめ}だ。
%
それでさへ
\ruby[g]{一旦}{いつたん}
\ruby[g]{激動}{げきどう}すれば、
%
\ruby{此}{こ}の
\ruby[g]{日方}{ひ かた}にも
\ruby{取}{と}つて
かゝる、% 踊り字調整「〻(二の字点、揺すり点)に見えるが(ゝ)」
%
それが
\ruby{貴}{たつと}い
\ruby[g]{人間}{ひ と }の
\ruby[g]{勇氣}{ゆうき }だ、
%
\ruby{人}{ひと}の
\ruby{人}{ひと}たる
\ruby[g]{{\換字{所}}以}{ゆ ゑん}を
\ruby{支}{さゝ}へるものだ。% 踊り字調整「〻(二の字点、揺すり点)に見えるが(ゝ)」
%
それだのに
\ruby{何}{なん}だ
\ruby{汝}{きさま}の
\ruby{其}{そ}の
\ruby{態}{てい}は!。
%
\ruby{一}{いち}
\ruby[g]{少女}{せうぢよ}にも
\ruby{及}{およ}ばなく
なつて、
%
たゞ% 踊り字調整「〻(二の字点、揺すり点)に濁点に見えるが(ゞ)」
\ruby{崩}{くづ}
\ruby{折}{を}れて
\ruby{萎}{しを}れ
きつて
\ruby{居}{ゐ}る!。
%
よく
\ruby{彼}{あ}の
\ruby{娘}{むすめ}に
\ruby{對}{たい}しても
\ruby[g]{慚死}{ざんし }せぬナ。
%
\ruby[g]{水野}{みづの }!、
%
\ruby{汝}{きさま}は
\ruby{决}{けつ}して
\ruby{决}{けつ}して
\ruby[g]{本心}{ほんしん}を
\ruby{失}{うしな}ふやうな
\改行% 校正作業の簡略化のため
、
%
\原本頁{259-8}\改行%
\ruby[g]{其樣}{そ ん }な
\ruby{腑甲{\換字{斐}}}{ふ|が|い}
\ruby{無}{な}い
\ruby{奴}{やつ}では
\ruby{無}{な}いが、
%
\ruby[g]{何樣}{ど う }すれば
\ruby[g]{此樣}{こ ん }なに
\ruby{意氣地}{い|く|ぢ}が
\ruby{無}{な}くなつた。
%
こゝの% 踊り字調整「〻(二の字点、揺すり点)に見えるが(ゝ)」
\ruby{娘}{むすめ}の
\ruby[g]{擧動}{ふるまひ}を
\ruby{眼}{め}の
\ruby{{\換字{前}}}{まへ}に
\ruby{見}{み}て、
%
よく
\ruby{汝}{きさま}は
\ruby[g]{自{\換字{分}}}{じ ぶん}が
\原本頁{259-10}\改行%
\ruby{羞}{はづか}しくないナ。
%
\ruby{一}{いち}
\ruby[g]{少女}{せうぢよ}でさへ
\ruby{彼}{あ}の
\ruby{{\換字{通}}}{とほ}りだ、
%
\ruby{汝}{きさま}は
\ruby[g]{堂々}{だう〳〵}たる% ルビ調整(原本通り)濁点なし
\ruby[g]{男兒}{だんじ }で
\ruby{無}{な}いか、
%
\ruby[g]{乃公}{お れ }は
\ruby{彼}{あ}の
\ruby{娘}{むすめ}に
\ruby{頭}{あたま}を
\ruby{撲}{う}たれたが、
%
\ruby{汝}{きさま}は
\ruby[g]{精神}{こゝろ }に% 踊り字調整「〻(二の字点、揺すり点)に見えるが(ゝ)」
\ruby{鞭}{むち}を
\ruby{受}{う}けなかつたか。
%
\ruby{苟}{いやし}くも
\ruby{舊}{もと}の
\ruby[g]{水野}{みづの }
である
ならば、
%
\ruby{人}{ひと}
\ruby[g]{一倍}{いちばい}
\ruby{物}{もの}を
\ruby{思}{おも}ふ
\ruby[<j||]{汝}{きさま}の% 行末行頭の境界付近なので特例処置を施す
\ruby{事}{こと}だもの、
%
\ruby{必}{かなら}ず
\ruby[g]{{\換字{感}}奮}{かんぷん}
せずには
\ruby{居}{を}らぬ
\ruby{筈}{はず}だが、
%
\ruby[g]{衰へ}{おとろ }
\ruby{果}{は}て
\ruby{{\換字{弱}}}{よわ}り
\ruby{果}{は}てた
\ruby{今}{いま}の
\ruby{汝}{きさま}は、
%
\ruby[g]{矢張}{やつぱ }り
\ruby{首}{くび}を
\ruby{俛}{た}るゝばかりか。% 踊り字調整「〻(二の字点、揺すり点)に見えるが(ゝ)」
%
\ruby[g]{此家}{こ ゝ }の% 踊り字調整「〻(二の字点、揺すり点)に見えるが(ゝ)」
\ruby{娘}{むすめ}の
\ruby[g]{健氣}{けなげ }な
\原本頁{260-4}\改行%
\ruby[g]{振舞}{ふるまひ}と、
%
\ruby{汝}{きさま}の
\ruby{其}{そ}の
\ruby{萎}{しを}れ
きつた
\ruby[g]{狀態}{ありさま}
とを、
%
\ruby[g]{見比}{み くら}べ
\ruby{思}{おも}ひ
\ruby{比}{くら}べると
\ruby{此}{こ}の
\ruby[g]{日方}{ひ かた}は、
%
これほど
までに
\ruby{汝}{きさま}は
\ruby{衰}{おとろ}へた
かと、
%
\ruby{汝}{きさま}の
\ruby{衰}{おとろ}へ
\ruby{果}{は}てたのが
\ruby{悲}{かな}しくて
\ruby{涙}{なみだ}が
\ruby{出}{で}る!。
%
\ruby{女}{をんな}にも
\ruby{劣}{おと}る
やうになつたとは
\ruby{餘}{あま}り
\ruby[g]{{\換字{情}}無}{なさけな}い!。
%
\ruby[g]{何故}{な ぜ }
\ruby{{\換字{迷}}}{まよ}ふ
にしても
\ruby[g]{男兒}{をとこ }
らしく
\ruby{{\換字{迷}}}{まよ}つて
\ruby{吳}{く}れぬ?。
』

\Entry{其四十五}

『
\ruby{心}{こゝろ}を
\ruby{一婦人}{いつ|ぷ|じん}に
\ruby{苦}{くるし}むる
\ruby{汝}{きさま}を
\ruby{見}{み}るのも
\ruby{忌々}{いま|〳〵}しいが、
\ruby{勇}{ゆう}を
\ruby{一少女}{いち|せう|ぢよ}に
\ruby{遜}{ゆづ}る
\ruby{汝}{きさま}
\ruby[g]{腑甲斐}{ふがひ}なさを
\ruby{見}{み}ては、あゝ
\ruby{凡骨}{ぼん|こつ}では
\ruby{無}{な}かつた
\ruby{水野某}{みづ|の|なにがし}が、
\ruby{如是}{か|う}も
\ruby{衰}{おとろ}へたものかと
\ruby{口惜}{くち|をし}くなる!。

\ruby{島木}{しま|き}の
\ruby{言}{い}つたことが
\ruby{眞實}{まこ|と}ならば、
\ruby{此}{こ}の
\ruby{日方}{ひ|かた}は
\ruby{全然}{ぜん|〴〵}
\ruby{否認}{ひ|にん}するけれど、そりやあ
\ruby{或}{あるひ}は
\ruby{戀愛}{れん|あい}に
\ruby{陷}{おちい}るのも
\ruby{已}{や}むを
\ruby{得}{え}んことか
\ruby{知}{し}らんが、
\ruby{何故戀愛}{な|ぜ|れん|あい}に
\ruby{陷}{おちい}つたで
\ruby{男兒}{をと|こ}らしくはせん?。
\ruby{同}{おな}じ
\ruby{{\換字{迷}}}{まよひ}に
\ruby{陷}{おちい}つても、
\ruby{人}{ひと}にも
\ruby{告}{つ}げず
\ruby{物}{もの}を
\ruby{思}{おも}つて
\ruby{空}{むな}しく
\ruby{泣}{な}き
\ruby{悶}{もだ}\換字{江}て
\ruby{居}{ゐ}るばかりが
\ruby{{\換字{道}}}{みち}でもあるまい。
いたづらに
\ruby{遲疑躊躇}{ち|ぎ|ちう|ちよ}して、
\ruby{何等}{なん|ら}の
\ruby{措置}{そ|ち}をも
\ruby{取}{と}ることを
\ruby{敢}{あへ}てせぬのは
\ruby{大丈夫}{だい|ぢやう|ぶ}の
\ruby{最}{もつと}も
\ruby{慚}{は}づるところだ。
たとひ
\ruby{少々}{せう|〳〵}は
\ruby{其}{そ}の
\ruby[g]{所爲宜}{しよゐよろし}きを
\ruby{失}{うしな}つても、
\ruby{慮}{はか}つて、
\ruby{斷}{だん}じて、
\ruby{行}{おこな}つて、
\ruby{着々}{ちやく|〳〵}と
\ruby[g]{事{\換字{情}}}{じじやう}の
\ruby{展開}{てん|かい}に
\ruby{應}{おう}じて
\ruby{行}{ゆ}くのが、
\ruby{男子}{だん|し}の
\ruby{敢}{あへ}てすべき
\ruby{{\換字{道}}}{みち}では
\ruby{無}{な}いか。
\ruby[g]{{\換字{猶}}豫}{ゆうよ}して
\ruby{决}{けつ}せざるは、
\ruby{軍務}{ぐん|む}では
\ruby{何}{なに}よりも
\ruby{甚}{はなはだ}しく
\ruby{惡}{にく}むところだが、
\ruby{獨}{ひと}り
\ruby{軍人}{ぐん|じん}のみが
\ruby{左樣覺悟}{さ|う|かく|ご}すべきでは
\ruby{無}{な}い、
\ruby{何人}{なん|びと}に
\ruby{取}{と}つても
\ruby{遲疑躊躇}{ち|ぎ|ちう|ちよ}ほど、
\ruby{其人}{その|ひと}を
\ruby{{\換字{害}}}{がい}するものはあるまい。
\ruby{同}{おな}じ
\ruby{婦人}{ふ|じん}に
\ruby{愛着}{あい|ちやく}するなら、
\ruby{水野}{みづ|の}
\ruby{汝}{きさま}も
\ruby{男兒}{をと|こ}では
\ruby{無}{な}いか、
\ruby{何故}{な|ぜ}
\ruby{男兒}{をと|こ}らしく
\ruby{行動}{かう|どう}せぬ?。
ビスマークは
\ruby{何樣}{ど|う}して
\ruby{其}{そ}の
\ruby{妻}{つま}を
\ruby{得}{え}た!。
\ruby{烈}{はげ}しく
\ruby{思}{おも}つた、
\ruby{明}{あき}らかに
\ruby{求}{もと}めた、
\ruby{而}{そ}して
\ruby{{\換字{終}}}{つひ}に
\ruby{得}{\換字{江}}たといふに
\ruby{過}{す}ぎん
\ruby{事}{こと}ではないか。
\ruby{今}{いま}は
\ruby{其}{そ}の
\ruby{夫人}{ふ|じん}も
\ruby{世}{よ}を
\ruby{去}{さ}られたが、
\ruby{我}{わ}が
\ruby{陸軍大將}{りく|ぐん|たい|しやう}の
\ruby{某侯}{ぼう|こう}が、
\ruby{年}{とし}も
\ruby{若}{わか}く
\ruby{身}{み}も
\ruby{鄙}{いやし}かつた
\ruby{時}{とき}の
\ruby{戀}{こひ}の
\ruby{物語}{もの|がたり}は、
\ruby{虛實}{きよ|じつ}は
\ruby{知}{し}らぬが
\ruby{汝}{きさま}も
\ruby{知}{し}つて
\ruby{居}{ゐ}やう。
\ruby{徒然}{と|ぜん}を
\ruby{慰}{なぐさ}めるばかりに
\ruby{讀}{よ}んだ
\ruby{雜書}{ざつ|しよ}に、
\ruby{{\換字{文}}覺}{もん|がく}の
\ruby{事}{こと}を
\ruby{記}{しる}してあつたが、
\ruby{彼}{あれ}を
\ruby{見}{み}て
\ruby[g]{先夜}{せんや}も
\ruby{汝}{きさま}の
\ruby{上}{うへ}を、
\ruby{自然}{おの|づ}と
\ruby{胸}{むね}に
\ruby{思}{おも}ひ
\ruby{{\換字{浮}}}{うか}めた。
\ruby{{\換字{文}}覺}{もん|がく}は
\ruby{全}{まつた}く
\ruby{失敗}{しつ|ぱい}し、ピスマークや
\ruby{我}{わ}が
\ruby{大將}{たい|しやう}は
\ruby{思}{おも}ひを
\ruby{{\換字{遂}}}{と}げたが、
\ruby{其}{そ}の
\ruby{遲疑躊躇}{ち|ぎ|ちう|ちよ}して
\ruby{空}{あだ}に
\ruby{物}{もの}を
\ruby{思}{おも}はぬは
\ruby{同}{おな}じ
\ruby{事}{こと}だ、
\ruby{飽}{あく}まで
\ruby{男兒}{をと|こ}らしく
\ruby{戀}{こひ}をしたのは
\ruby{同}{おな}じ
\ruby{事}{こと}だ。
\ruby{彼}{あ}の
\ruby{{\換字{文}}覺}{もん|がく}が
\ruby{云}{い}つた
\ruby{言}{ことば}に、
\ruby{戀}{こひ}には
\ruby{人}{ひと}の
\ruby{死}{し}なぬものかは、と
\ruby{苦}{くる}しい
\ruby{思}{おもひ}を
\ruby{白狀}{はく|じやう}してゐるが、
\ruby{水野}{みづ|の}、
\ruby{汝}{きさま}も
\ruby{其}{そ}の
\ruby{衰}{おとろ}へかた
\ruby{其}{そ}の
\ruby{窶}{やつ}れかたでは、
\ruby{成程汝}{なる|ほど|きさま}も
\ruby{死{\換字{兼}}}{しに|か}ねない
\ruby{樣子}{やう|す}だ。
とても
\ruby{其程}{それ|ほど}に
\ruby{{\換字{迷}}}{まよ}つたならば、
\ruby{{\換字{進}}}{すゝ}んでは
\ruby{振舞}{ふる|ま}はぬ?、
\ruby{默}{だま}つて
\ruby{物}{もの}を
\ruby{思}{おも}つても
\ruby{死}{し}ぬなら、
\ruby{何故}{な|ぜ}
\ruby{成敗}{せい|ばい}
\ruby{生死}{しやう|し}
\ruby{此}{こ}の
\ruby{一擲}{いつ|てき}と、
\ruby{男兒}{をと|こ}らしく
\ruby{運命}{うん|めい}の
\ruby{何}{なに}を
\ruby{與}{あた}ふるかを
\ruby{見}{み}ぬ?。
\ruby{{\換字{文}}覺}{もん|がく}はたゞ
\ruby{我慢}{が|まん}ばかりの
\ruby{男}{をとこ}では
\ruby{無}{な}い、
\ruby{袈裟}{け|さ}を
\ruby{殺}{ころ}した
\ruby{其}{そ}の
\ruby{後}{あと}では、
\ruby{辰}{たつ}の
\ruby{刻}{こく}より
\ruby{未}{ひつじ}の
\ruby{刻}{こく}まで、
\ruby{四時}{よ|とき}と
\ruby{云}{い}へば
\ruby{八時間}{はち|じ|かん}だ、
\ruby{其}{そ}の
\ruby{八時間}{はち|じ|かん}を
\ruby{大聲揚}{おほ|ごゑ|あ}げて、
\ruby{荒}{あら}くれた
\ruby{眼}{め}から
\ruby{霰}{あられ}のやうな
\ruby{淚}{なみだ}を
\ruby{落}{おと}しながら
\ruby{泣}{な}き
\ruby{通}{とほ}したとある、
\ruby{恐}{おそろ}しい
\ruby{{\換字{情}}}{じやう}の
\ruby{深}{ふか}い
\ruby{熱烈}{ねつ|れつ}な
\ruby{奴}{やつ}だ。
\ruby{其位}{その|くらゐ}の
\ruby{奴}{やつ}が
\ruby{手荒}{て|あら}い
\ruby{事}{こと}をするまでには、
\ruby{一}{ひ}ㇳ
\ruby{通}{とほ}りや
\ruby{二}{ふ}タ
\ruby{通}{とほ}りで
\ruby{無}{な}く
\ruby{物}{もの}を
\ruby{思}{おも}つたらうが、
\ruby{歸}{き}するところ
\ruby{暴}{ぼう}でも
\ruby{何}{なん}でも
\ruby{男兒}{をと|こ}らしく
\ruby{思}{おも}ふまゝに
\ruby{振舞}{ふる|ま}つたのはまた
\ruby{已}{や}むを
\ruby{得}{\換字{江}}ん。
とてもかくても
\ruby{物}{もの}を
\ruby{思}{おも}つて
\ruby{戀}{こひ}に
\ruby{死{\換字{兼}}}{しに|か}ねもすまいならば、
\ruby{何故}{な|ぜ}
\ruby{男兒}{をと|こ}らしくは
\ruby{振舞}{ふる|ま}はぬ?。
\ruby{當}{あた}つて
\ruby{碎}{くだ}くか
\ruby{碎}{くだ}けろかだ、
\ruby{{\換字{突}}貫}{とつ|くわん}してして
\ruby{倒}{たふ}さるゝか
\ruby{倒}{たふ}すかの
\ruby{事}{こと}だ、
\ruby{首離}{かうべ|はな}ると
\ruby{雖}{いへど}も
\ruby{身懲}{み|こ}りず、といふ
\ruby{勢}{いきほひ}で
\ruby{{\換字{突}}貫}{とつ|くわん}して
\ruby{仕舞}{し|ま}へ。
\ruby{汝}{きさま}が
\ruby{良}{よ}い
\ruby{婦人}{ふ|じん}を
\ruby{得}{\換字{江}}て
\ruby{大將}{たい|しやう}になるか、たゞし
\ruby{{\換字{文}}覺}{もん|がく}のやうな
\ruby[g]{狂僧}{きちがひばうず}になるかそれは
\ruby{何方}{どち|ら}になつても
\ruby{乃公}{お|れ}は
\ruby{關}{かま}はんが、
\ruby{何樣}{ど|う}せ
\ruby{汝}{きさま}は
\ruby{欲}{よく}が
\ruby{薄}{うす}くて
\ruby{高慢}{かう|まん}が
\ruby{{\換字{強}}}{つよ}い、
\ruby[g]{變挺}{へんてこ}な
\ruby{男}{をとこ}に
\ruby{生}{うま}れて
\ruby{居}{ゐ}るのだから、
\ruby{坊主}{ばう|ず}になつて
\ruby{仕舞}{し|ま}ふのも
\ruby{寧宜}{いつそ|よか}らう、
\ruby{日方}{ひ|かた}は
\ruby{貧乏}{びん|ばう}でも
\ruby{汝}{きさま}が
\ruby{左樣}{さ|う}なつたら、
\ruby{{\換字{麻}}}{あさ}の
\ruby{衣位}{ころも|ぐらひ}は
\ruby{寄{\換字{進}}}{き|しん}して
\ruby{立{\換字{過}}}{たて|すご}して
\ruby{{\換字{遣}}}{や}る!。
\ruby{汝}{きさま}が
\ruby{衰}{おとろ}へに
\ruby{衰}{おとろ}へて、
\ruby{一少女}{いち|せう|ぢよ}にも
\ruby{其}{そ}の
\ruby{勇氣}{ゆう|き}が
\ruby{及}{およ}ばんやうになつて
\ruby{戀}{こひ}に
\ruby{死}{し}ぬのを、
\ruby{見殺}{み|ごろ}しにするのは
\ruby{乃公}{お|れ}には
\ruby{出來}{で|き}ぬ。
\ruby{男兒}{をと|こ}らしく
\ruby{振舞}{ふる|ま}へ、
\ruby{女}{をんな}ではあるまい。
\ruby{高}{たか}が
\ruby{一婦人}{いち|ふ|じん}を
\ruby{對敵}{あひ|て}にして、
\ruby[g]{{\換字{遠}}距離}{ゑんきより}で
\ruby{彈藥}{だん|やく}を
\ruby{使}{つか}ひ
\ruby{盡}{つく}すのは
\ruby{愚}{おろか}な
\ruby{事}{こと}だ。
いつそ
\ruby{一}{ひ}と
\ruby{思}{おもひ}に
\ruby{突貫}{とつ|くわん}して
\ruby{仕舞}{し|ま}へ。

\ruby{{\換字{勝}}}{か}つか
\ruby{負}{ま}けるかの
\ruby{他}{ほか}には
\ruby{物}{もの}は
\ruby{有}{あ}りは
\ruby{仕無}{し|な}い。
\ruby[g]{{\換字{遠}}地}{とほく}から
\ruby{敵}{てき}に
\ruby{{\換字{勝}}}{か}たうといふのは
\ruby{贅澤}{ぜい|たく}な
\ruby[g]{詮義}{せんぎ}だ。
\ruby{羽{\換字{勝}}}{は|がち}
\ruby{乃公}{お|れ}の
\ruby{言}{い}ふことを
\ruby{無理}{む|り}とは
\ruby{思}{おも}ふまい、
\ruby{何樣}{ど|う}だ
\ruby{水野}{みづ|の}
\ruby{汝}{きさま}は
\ruby{何}{なん}と
\ruby{思}{おも}ふ?。
\ruby{女々}{め|ゝ}しい
\ruby{事}{こと}は
\ruby{宜}{い}い
\ruby{加減}{か|げん}に
\ruby{止}{や}めろ。
もう
\ruby{乃公}{お|れ}
は
\ruby{此限}{これ|ぎ}り
\ruby{物}{もの}は
\ruby{言}{い}はぬ、これだけ
\ruby{言}{い}つても
\ruby{乃公}{お|れ}の
\ruby{云}{い}ふ
\ruby{事}{こと}を
\ruby{用}{もち}ゐんならば、
\ruby{舊}{もと}の
\ruby{水野}{みづ|の}になり
\ruby{{\換字{返}}}{かへ}るまでは、
\ruby{汝}{きさま}には
\ruby{會}{あ}はん。
』


\Entry{其四十六}

\ruby[g]{日方}{ひかた}の
\ruby{言}{い}ふところも
\ruby{無理}{む|り}ばかりにはあらず、
\ruby{思}{おも}ふて
\ruby{言}{い}はざる
\ruby{苦}{くるし}さに
\ruby{堪}{た}へかねては、
\ruby{兎}{と}せん
\ruby{角}{かく}せんと
\ruby{意}{こヽろ}を
\ruby{動}{うご}かしたる
\ruby{折}{をり}も
\ruby{無}{な}きにあらねど、おのづからに
\ruby{思}{おも}ひ
\ruby{切}{き}つたる
\ruby{事}{こと}を
\ruby{何故}{なに|ゆゑ}とも
\ruby{無}{な}く
\ruby{做}{な}し
\ruby{出}{いだ}しかねて、
\ruby{女々}{め|ヽ}しと
\ruby{云}{い}はヾ
\ruby{女々}{め|ヽ}しと
\ruby{云}{い}はるべく
\ruby{今日}{け|ふ}までは
\ruby{{\GWI{u904e-k}}}{すご}せるなり。
されど
\ruby{差當}{さし|あた}つて
\ruby{今日方}{いま|ひ|かた}に
\ruby{對}{むか}つて、
\ruby{其}{そ}の
\ruby{言葉}{こと|ば}に
\ruby{從}{したが}ふべし
\ruby{意見}{い|けん}に
\ruby{就}{つ}くべしとも
\ruby{云}{い}ひかねて、
\ruby[g]{水野}{みづの}は
\ruby{何}{なん}とも
\ruby{言}{ものい}はねば、
\ruby[g]{{\GWI{u7fbd-k}\換字{勝}}}{はがち}は
\ruby{徐々}{おも|むろ}に
\ruby{口}{くち}を
\ruby{開}{ひら}きて、
\ruby{言葉}{こと|ば}づかひも
\ruby{重々}{おも|〳〵}しく、

『
\ruby[g]{水野}{みづの}、
\ruby{默}{もく}して
\ruby{仕舞}{し|ま}つてはいかん。
\ruby[g]{日方}{ひかた}の
\ruby{言}{ことば}は
\ruby{或}{あるひ}は
\ruby[g]{不當}{ふたう}だ、しかし
\ruby[g]{日方}{ひかた}の
\ruby{意}{こヽろ}は
\ruby{親切}{しん|せつ}に
\ruby{他}{ほか}ならんのだ。
\ruby{其言}{その|ことば}を
\ruby{{\GWI{u63a1-k}}}{と}ると
\ruby{{\GWI{u63a1-k}}}{と}らんとは
\ruby{別}{べつ}として、
\ruby{其親切}{その|しん|せつ}は
\ruby{十分}{じう|ぶん}に
\ruby{受}{う}け
\ruby{納}{い}れねばならん。
\ruby{無論君}{む|ろん|きみ}は
\ruby[g]{日方}{ひかた}の
\ruby[g]{好意}{かうい}に
\ruby{對}{たい}して
\ruby{感謝}{かん|しや}して
\ruby{居}{を}るだらうナ。
』

と
\ruby{優}{やさし}しく
\ruby[g]{水野}{みづの}を
\ruby{誘}{いざな}ひて
\ruby{言}{ものい}はせんとすれど、
\ruby[g]{水野}{みづの}はたヾ
\ruby{僞}{いつはり}ならぬ
\ruby{眼色}{め|つき}して
\ruby{打點頭}{うち|うな|づ}きて、
\ruby{然}{しか}り、と
\ruby{答}{こた}へたるばかりなり。

『
\ruby{人}{ひと}は
\ruby{人各々}{ひと|おの|〳〵}の
\ruby{性質}{せい|しつ}がある、
\ruby{境遇}{きやう|ぐう}がある。
\ruby{深}{ふか}く
\ruby{他人}{た|にん}の
\ruby{事}{こと}に
\ruby[g]{立入}{たちい}るのは
\ruby{僕}{ぼく}は
\ruby{取}{と}らん。
\ruby[g]{日方}{ひかた}の
\ruby{親切}{しん|せつ}は
\ruby{僕}{ぼく}も
\ruby{有}{も}つて
\ruby{居}{ゐ}る。
たヾし
\ruby[g]{日方}{ひかた}の
\ruby{如}{ごと}く
\ruby{自分}{じ|ぶん}の
\ruby[g]{意思感{\GWI{u60c5-k}}}{いしかんじやう}を、
\ruby{君}{きみ}の
\ruby{上}{うへ}に
\ruby{押}{お}し
\ruby{被}{かぶ}せやうとは
\ruby{僕}{ぼく}は
\ruby{能}{よく}せん。
\ruby[g]{水野}{みづの}!
\ruby{歸}{かへ}つて
\ruby{來}{き}てから
\ruby{君}{きみ}の
\ruby{{\GWI{u8a55-k}}{\換字{判}}}{ひやう|ばん}をいろ〳〵
\ruby{聞}{き}いた。
\ruby{僕}{ぼく}は
\ruby{考}{かんが}へた。
\ruby{考慮}{かん|がへ}を
\ruby{錬}{ね}つた。
\ruby{而}{そ}して
\ruby{君}{きみ}に
\ruby{對}{たい}して
\ruby{贈}{おく}るべき
\ruby{或物}{ある|もの}を
\ruby{得}{\GWI{u1b001}}た。
しかし
\ruby{今}{いま}の
\ruby{君}{きみ}に
\ruby{對}{たい}して
\ruby{何}{なに}を
\ruby{贈}{おく}つても
\ruby{無{\GWI{u76c6-k}}}{む|えき}に
\ruby{{\GWI{u7d42-ue0101}}}{をは}るべきを
\ruby{知}{し}つた。
よつて
\ruby{君}{きみ}に
\ruby{對}{たい}して
\ruby{何}{なに}をも
\ruby{言}{い}ふまいと
\ruby{思}{おも}つた。
しかし
\ruby{今日方}{いま|ひ|かた}の
\ruby{言}{い}つたところは
\ruby{不幸}{ふ|かう}にして、
\ruby{僕}{ぼく}が
\ruby{考}{かんが}へて
\ruby{云}{い}はうと
\ruby{思}{おも}つたところと
\ruby{正反對}{せい|はん|たい}の
\ruby{言}{げん}であるので、
\ruby{已}{や}むを
\ruby{得}{\GWI{u1b001}}ず
\ruby{誘}{さそ}ひ
\ruby{出}{だ}されて
\ruby{一言}{ひと|こと}いふ。
\ruby[g]{日方}{ひかた}の
\ruby{言}{げん}を
\ruby{駁}{ばく}するのでは
\ruby{無}{な}い。
もとより
\ruby{僕}{ぼく}が
\ruby{言}{い}はんと
\ruby{欲}{ほつ}して
\ruby{居}{ゐ}たところなのだ。
\ruby[g]{水野}{みづの}、
\ruby{君}{きみ}は
\ruby{聰明}{そう|めい}の
\ruby{人}{ひと}だ、
\ruby{僕等}{ぼく|ら}は
\ruby{及}{およ}ばん。
たヾ、
\ruby{此}{こ}の
\ruby{世}{よ}の
\ruby{中}{なか}に
\ruby{立交}{たち|まぢ}つて、
\ruby{人}{ひと}に
\ruby{接}{せつ}し
\ruby{事}{こと}に
\ruby{應}{おう}ずるに
\ruby{於}{おい}ては
\ruby{齡}{とし}の
\ruby{多}{おほ}いだけに、
\ruby{僕}{ぼく}は
\ruby{私}{ひそか}に
\ruby{思}{おも}ふに
\ruby{君}{きみ}に
\ruby{對}{たい}しても、
\ruby{必}{かなら}ず
\ruby{一日}{いち|じつ}の
\ruby{長}{ちよう}があると
\ruby{信}{しん}ずる。
\ruby{僕}{ぼく}は
\ruby{書}{しよ}を
\ruby{讀}{よ}んで
\ruby{理}{り}を
\ruby{尋}{たづ}ねたで
\ruby{無}{な}い、
\ruby{事}{こと}に
\ruby{當}{あた}つて
\ruby{自}{みづか}ら
\ruby{知}{し}つたのだ。
\ruby{僕}{ぼく}は
\ruby{人}{ひと}に
\ruby{使}{つか}はれた。
\ruby{人}{ひと}を
\ruby{使}{つか}つた。
\ruby{而}{そ}して
\ruby{人}{ひと}と
\ruby{人}{ひと}との
\ruby{間}{あひだ}の
\ruby{感{\GWI{u60c5-k}}}{かん|じやう}といふものが、
\ruby{如何}{い|か}に
\ruby{大切}{たい|せつ}なものであるかといふこを
\ruby{身}{み}に
\ruby{染}{し}みて
\ruby{覺}{おぼ}えた。
\ruby{而}{そ}して
\ruby{我}{わ}が
\ruby{感{\GWI{u60c5-k}}}{かん|じやう}に
\ruby{任}{まか}すことの
\ruby{危{\換字{害}}}{き|がい}を
\ruby{實驗}{じつ|けん}した。
\ruby{僕}{ぼく}は
\ruby{愚}{ぐ}であつたから
\ruby{同}{おな}じ
\ruby{{\GWI{u904e-k}}失}{あや|まち}を
\ruby{二度}{ふた|ゝび}した。
\ruby{三度}{み|たび}した。
\ruby{四度}{よ|たび}した
\ruby{五度}{ご|たび}した。
\ruby{幾十度}{いく|じう|ど}と
\ruby{無}{な}く
\ruby{實驗}{じつ|けん}した。
\ruby{而}{そ}して
\ruby{後纔}{のち|わづか}に
\ruby{我}{わ}が
\ruby{感{\GWI{u60c5-k}}}{かん|じやう}を
\ruby{調御}{てう|ぎよ}することの
\ruby{如何}{い|か}に
\ruby{大切}{たい|せつ}なものであるかといふ
\ruby{事}{こと}を
\ruby{知}{し}つた。
\ruby{罵}{のヽし}らるれば
\ruby{怒}{いか}る、
\ruby{氣}{き}に
\ruby{入}{い}れば
\ruby{愛}{あい}する。
それは
\ruby{欺}{あざむ}かぬ
\ruby{感{\GWI{u60c5-k}}}{かん|じやう}である。
\ruby{其}{そ}の
\ruby{感{\GWI{u60c5-k}}}{かん|じやう}に
\ruby{任}{まか}せて
\ruby{喜怒}{き|ど}するを
\ruby{天眞爛{\GWI{u71b3-j}}}{てん|しん|らん|まん}だなんぞといふ。
\ruby{一船}{いつ|せん}の
\ruby{中}{うち}で
\ruby{事端}{じ|たん}を
\ruby{生}{しやう}ずるのは、
\ruby{何時}{い|つ}でも
\ruby{天眞爛{\GWI{u71b3-j}}}{てん|しん|らん|まん}の
\ruby{人}{ひと}だ。
\ruby{怒}{いか}るには
\ruby{怒}{いか}る
\ruby{理由}{わ|け}がある。
\ruby{愛}{あい}するには
\ruby{愛}{あい}する
\ruby{理由}{わ|け}がある。
しかし
\ruby{感{\GWI{u60c5-k}}}{かん|じやう}ばかりが
\ruby{最上}{さい|じやう}なものでは
\ruby{無}{な}い。
\ruby{感{\GWI{u60c5-k}}}{かん|じやう}に
\ruby{任}{まか}すのを
\ruby{是}{ぜ}とする
\ruby{人}{ひと}は
\ruby{船員}{せん|いん}の
\ruby{中}{うち}の
\ruby{最}{もつと}も
\ruby{危險}{き|けん}な
\ruby{人}{ひと}だ。
\ruby{自分}{じ|ぶん}の
\ruby{感{\GWI{u60c5-k}}}{かん|じやう}を
\ruby{調御}{てう|ぎよ}しなければ、
\ruby{自分}{じ|ぶん}は
\ruby{人}{ひと}に
\ruby{使}{つか}はれることが
\ruby{出來}{で|き}ぬ。
\ruby{自分}{じ|ぶん}の
\ruby{感{\GWI{u60c5-k}}}{かん|じやう}を
\ruby{調御}{てう|ぎよ}しなければ
\ruby{自分}{じ|ぶん}は
\ruby{人}{ひと}を
\ruby{使}{つか}ふことが
\ruby{出來}{で|き}ぬ。
\ruby{自分}{じ|ぶん}の
\ruby{感{\GWI{u60c5-k}}}{かん|じやう}を
\ruby{調御}{てう|ぎよ}しなければ、
\ruby{自分}{じ|ぶん}は
\ruby{人}{ひと}に
\ruby{交}{まじは}ることが
\ruby{出來}{で|き}ぬ。
\ruby{人}{ひと}に
\ruby{使}{つか}はれず、
\ruby{人}{ひと}を
\ruby{使}{つか}はず、
\ruby{人}{ひと}に
\ruby{交}{まじは}らずに
\ruby{濟}{す}む
\ruby{世間}{せ|けん}は
\ruby{無}{な}い。
\ruby{僕}{ぼく}は
\ruby{僕}{ぼく}だけの
\ruby{小}{ちひさ}な
\ruby[g]{經驗}{けいけん}だが、しかし
\ruby{確實堅固}{くわく|じつ|けん|ご}な
\ruby[g]{經驗}{けいけん}から、
\ruby{非常}{ひじ|やう}に
\ruby{{\換字{強}}}{つよ}く
\ruby{深}{ふか}く
\ruby{感{\GWI{u60c5-k}}}{かん|じやう}の
\ruby{調御}{てう|ぎよ}が
\ruby{人世}{じん|せい}の
\ruby{最大必要}{さい|だい|ひつ|\GWI{u1b001}う}のものであるといふことを
\ruby{確信}{くわく|しん}して
\ruby{居}{ゐ}る。
\ruby{君}{きみ}は
\ruby{聰明絶倫}{そう|めい|ぜつ|りん}な
\ruby{人}{ひと}だが、
\ruby{此}{こ}の
\ruby{點}{てん}の
\ruby[g]{經驗}{けいけん}は
\ruby{或}{あるひ}は
\ruby{薄}{うす}からう。
\ruby{戀愛}{れん|あい}も
\ruby{是非}{ぜ|ひ}がない。
\ruby{苦悶}{く|もん}も
\ruby{已}{や}むを
\ruby{得}{\GWI{u1b001}}ぬ。
\ruby{一切}{いつ|さい}の
\ruby{事}{こと}は
\ruby{謝}{しや}せんとして
\ruby{謝}{しや}せぬが
\ruby{天命}{てん|めい}だ。
\ruby{風}{かぜ}の
\ruby{前面}{ま|へ}から
\ruby{吹}{ふ}く
\ruby{日}{ひ}もある。
\ruby{潮流}{し|ほ}の
\ruby{横}{よこ}へと
\ruby{行}{ゆ}く
\ruby{夜}{よ}もある。
\ruby[g]{颶風}{つむじ}も
\ruby{龍卷}{たつ|まき}も
\ruby{起}{おこ}る
\ruby{日}{ひ}は
\ruby{起}{おこ}る。
しかし
\ruby{其間}{その|あひだ}に
\ruby{立}{た}つて
\ruby{屹然}{きつ|ぜん}として、
\ruby{我}{わ}が
\ruby{正當}{せい|たう}の
\ruby{處置}{しよ|ち}を
\ruby{取}{と}つて
\ruby{行}{ゆ}けば
\ruby{死}{し}して
\ruby{餘}{あま}りあるのだ。
\ruby[g]{水野}{みづの}!。
\ruby{君}{きみ}が
\ruby{君}{きみ}の
\ruby{欺}{あざむ}かぬ
\ruby{感{\GWI{u60c5-k}}}{かん|じやう}のために
\ruby{死}{し}にたくば
\ruby{其迄}{それ|まで}の
\ruby{事}{こと}だ。
しかし
\ruby{君}{きみ}が
\ruby{君}{きみ}として
\ruby{世}{よ}に
\ruby{立}{た}たうとした
\ruby{大丈夫}{だい|ぢやう|ぶ}の
\ruby{志}{こヽろざし}を
\ruby{忘}{わす}れぬ
\ruby{限}{かぎ}りは、
\ruby{君}{きみ}は
\ruby{君}{きみ}の
\ruby{感{\GWI{u60c5-k}}}{かん|じやう}を
\ruby{調御}{てう|ぎよ}することを
\ruby{忘}{わす}れてはならぬ。
\ruby{必}{かなら}ず
\ruby{感{\GWI{u60c5-k}}}{かん|じやう}の
\ruby{調御}{てう|ぎよ}といふことを
\ruby{忘}{わす}れずに
\ruby{居}{ゐ}て
\ruby{欲}{ほし}しい。
\ruby{君}{きみ}が
\ruby{{\GWI{u6587-k}}覺}{もん|がく}の
\ruby{如}{ごと}き
\ruby{人}{ひと}とならんことは、
\ruby{僕}{ぼく}の
\ruby{最}{もつと}も
\ruby{恐}{おそ}れて
\ruby{居}{ゐ}るところだ。
\ruby{{\GWI{u6587-k}}覺}{もん|がく}の
\ruby{如}{ごと}きは
\ruby{僕}{ぼく}の
\ruby[g]{蛇掲視}{だかつし}する
\ruby{人}{ひと}だ。
しかし
\ruby{僕}{ぼく}と
\ruby[g]{日方}{ひかた}とは
\ruby{言}{ことば}は
\ruby{異}{こと}にして
\ruby{意}{こヽろ}は
\ruby{同}{おな}じだ。
たまたま
\ruby[g]{日方}{ひかた}の
\ruby{言}{ことば}に
\ruby{僕}{ぼく}の
\ruby{胸裏}{む|ね}に
\ruby{觸}{ふ}れたところが
\ruby[g]{一寸}{ちよつと}あつたので、
\ruby{言}{い}はずとものことを
\ruby{饒舌}{しや|べ}つたが、
\ruby[g]{二人}{ふたり}の
\ruby{言}{ことば}の
\ruby{異}{ことな}るところを
\ruby{忘}{わす}れて、
\ruby{其}{そ}の
\ruby{意}{こヽろ}の
\ruby{同}{おな}じところをさへ
\ruby{取}{と}つて
\ruby{{\換字{呉}}}{く}れヽば、
\ruby[g]{日方}{ひかた}も
\ruby{僕}{ぼく}も
\ruby{何程}{どれ|ほど}
\ruby{{\GWI{u6085-k}}}{よろこ}ばう!。
』


\Entry{其四十七}

\原本頁{}%
\ruby[g]{羽{\換字{勝}}}{はがち}が
\ruby{同{\換字{情}}}{おも|ひやり}のいと
\ruby{厚}{あつ}くして、
%
\ruby{而}{しか}も
\ruby{{\換字{道}}理}{だう|り}の
\ruby{正}{たゞし}きに
\ruby{據}{よ}れる、
%
\ruby{其}{そ}の
\ruby{言}{ことば}には
\ruby{力}{ちから}あり、
%
\ruby{其}{そ}の
\ruby{意}{こゝろ}には
\ruby{仁有}{なさけ|あ}るに、
%
\ruby{{\換字{分}}}{わ}けて
\ruby{此頃}{この|ごろ}は
\ruby{感}{かん}じ
\ruby{易}{やす}くなれる
\ruby[g]{水野}{みづの}の、
%
\ruby{心}{こゝろ}の
\ruby{中}{うち}に
\ruby{深}{ふか}く
\ruby{恩}{おん}を
\ruby{謝}{しや}しながら、
%
\ruby{言}{い}はれしことの
\ruby{本末}{もと|すゑ}を
\ruby{思}{おも}ひ
\ruby{味}{あぢは}ふ
\ruby{時}{とき}、
%
\ruby[g]{羽{\換字{勝}}}{はがち}は
\ruby{復}{ふたゝ}び
\ruby{口}{くち}を
\ruby{開}{ひら}きて、

\原本頁{}%
『
\ruby{僕}{ぼく}の
\ruby{言}{げん}は
\ruby{或}{あるひ}は
\ruby{漠然}{ばく|ぜん}として、
%
\ruby{捉}{とら}へどころの
\ruby{無}{な}いやうにも
\ruby{思}{おも}へやう。
%
しかし
\ruby{僕}{ぼく}は
\ruby{漠然}{ばく|ぜん}たることは
\ruby{决}{けつ}して
\ruby{云}{い}はぬ。
%
\ruby{手}{て}を
\ruby{下}{くだ}すところの
\ruby{知}{し}れぬ
\ruby{敎訓}{をし|へ}は
\ruby{僕}{ぼく}は
\ruby{{\換字{嫌}}}{きら}ふ。
%
\ruby{着手}{ちやく|しゆ}するところが
\ruby{{\換字{分}}明}{ぶん|みやう}で
\ruby{無}{な}ければ
\ruby{實務}{じつ|む}は
\ruby{擧}{あが}らぬ。
%
\ruby{收穫}{とり|いれ}の
\ruby{算用}{さん|よう}を
\ruby{播種}{たね|まき}の
\ruby{{\換字{前}}}{まへ}に
\ruby{爲}{す}るのは
\ruby{最}{もつと}も
\ruby{忌}{い}むところだ。
%
たゞ
\ruby{感{\換字{情}}}{かん|じやう}の
\ruby{訓練}{くん|れん}と
\ruby{云}{い}つても、
%
\ruby{着手}{ちやく|しゆ}のところを
\ruby{云}{い}はねば
\ruby{{\換字{空}}言}{くう|げん}になる。
%
\ruby{煩}{うるさ}いか
\ruby{知}{し}らんが
\ruby{{\換字{空}}言}{くう|げん}にならぬやうに、
%
\ruby{{\換字{適}}切}{てき|せつ}に
\ruby{敢}{あへ}て
\ruby{君}{きみ}のために
\ruby{云}{い}はう。
%
\ruby{云}{い}ひ
\ruby{{\換字{過}}}{す}ぎて
\ruby{無禮}{ぶ|れい}であつても
\ruby{免}{ゆる}し
\ruby{玉}{たま}へ。
%
たとへば
\ruby{人}{ひと}を
\ruby{思}{おも}ふとすれば、
%
\ruby{其}{そ}の
\ruby{{\換字{情}}}{じやう}は
\ruby{胸中}{きやう|ちう}に
\ruby{鬱滯}{うつ|たい}して
\ruby{結}{むす}ぼれる。
%
また
\ruby{例}{たと}へば
\ruby{人}{ひと}を
\ruby{怒}{いか}るとすれば、
%
\ruby{其}{そ}の
\ruby{{\換字{情}}}{じやう}は
\ruby{心頭}{しん|とう}に
\ruby{狂}{くる}ひ
\ruby{立}{た}つて
\ruby{已}{や}まぬ。
%
それを
\ruby{其}{その}
\ruby{儘}{まゝ}に
\ruby{任}{まか}せて
\ruby{置}{お}けば、
%
\ruby{我}{わ}が
\ruby{本{\換字{分}}}{ほん|ぶん}の
\ruby{事}{こと}は
\ruby{其}{そ}れがために
\ruby{{\換字{誤}}}{あやま}られる。
%
\ruby{夫}{ふなのり}が
\ruby{思}{おも}ひも
\ruby{寄}{よ}らぬ
\ruby{{\換字{過}}失}{くわ|しつ}をして、
%
\ruby{不測}{ふ|そく}の
\ruby{禍{\換字{害}}}{わざ|はひ}を
\ruby{得}{う}る
\ruby{其}{そ}の
\ruby{多}{おほ}くは、
%
\ruby{胸中}{きやう|ちう}に
\ruby{職務}{しよく|む}
\ruby{以外}{い|ぐわい}の
\ruby{何物}{なに|もの}かゞ
\ruby{蟠}{わだか}まつて、
%
\ruby{職務}{しよく|む}に
\ruby{放心}{うつ|かり}して
\ruby{居}{ゐ}る
\ruby{時}{とき}に
\ruby{起}{おこ}る。
%
\ruby{{\換字{又}}}{また}
\ruby{一{\換字{船}}}{いつ|せん}の
\ruby{{\換字{平}}和}{へい|わ}の
\ruby{破壞}{やぶ|れ}は
\ruby{激烈}{げき|れつ}の
\ruby{感{\換字{情}}}{かん|じやう}の
\ruby{暴發}{ぼう|はつ}に
\ruby{基}{もとづ}く。
%
そこで
\ruby{自{\換字{分}}}{じ|ぶん}が
\ruby{自{\換字{分}}}{じ|ぶん}の
\ruby{當直時間}{たう|ちよく|じ|かん}だけ、
%
\ruby{甲板}{デツ|キ}に
\ruby{在}{あ}つて
\ruby{執務}{しつ|む}する
\ruby{間}{あひだ}は、
%
\ruby{何等}{なん|ら}の
\ruby{私{\換字{情}}}{しゞ|やう}が
\ruby{胸中}{きやう|ちう}に
\ruby{在}{あ}らうとも、
%
それを
\ruby{壓}{おさ}へつけて
\ruby{放肆}{はう|し}ならしめぬやうに
\ruby{敢}{あへ}てせねばならぬ。
%
\ruby{親}{おや}を
\ruby{思}{おも}ふは
\ruby{孝子}{かう|し}の
\ruby{眞{\換字{情}}}{しん|じやう}だ。
%
しかし
\ruby{病}{や}んで
\ruby{居}{ゐ}る
\ruby{親}{おや}を
\ruby{思}{おも}つて
\ruby{茫然}{ばう|ぜん}としたゝめ、
%
\ruby{{\換字{船}}}{ふね}の
\ruby{{\換字{進}}路}{しん|ろ}を
\ruby{{\換字{過}}}{あやま}つて
\ruby{洲}{す}へ
\ruby{上}{あ}げたでは
\ruby{濟}{す}まぬ。
%
\ruby{職務}{しよく|む}を
\ruby{執}{と}つて
\ruby{居}{ゐ}る
\ruby{其間}{その|あひだ}だけは、
%
\ruby{如何}{い|か}に
\ruby{孝子}{かう|し}でも
\ruby{自}{みづか}ら
\ruby{{\換字{忍}}}{しの}んで、
%
\ruby{親}{おや}を
\ruby{思}{おも}ふ
\ruby{{\換字{情}}}{こゝろ}に
\ruby{氣}{き}を
\ruby{取}{と}られぬやうに、
%
\ruby{嚴然}{げん|ぜん}と
\ruby{胸中}{きやう|ちう}を
\ruby{淸潔}{せい|けつ}にせねばならぬ。
%
\ruby{湧}{わ}き
\ruby{上}{あが}り
\ruby{起}{おこ}り
\ruby{立}{た}つ
\ruby{感{\換字{情}}}{かん|じやう}を
\ruby{抑制}{よく|せい}せばならん。
%
\ruby{訓練}{くん|れん}して
\ruby{我}{わ}が
\ruby{命令}{めい|れい}に
\ruby{服}{ふく}させねばならん。
%
これは
\ruby{實務}{じつ|む}に
\ruby{身}{み}を
\ruby{練}{ね}るものゝ
\ruby{必}{かなら}ず
\ruby{知}{し}つて
\ruby{居}{ゐ}るところだ。
%
\ruby[g]{日方}{ひかた}なども
\ruby{必}{かなら}ず
\ruby{經驗}{けい|けん}して
\ruby{居}{ゐ}るところだ。
%
たゞ
\ruby{世}{よ}に
\ruby{一種}{いつ|しゆ}の
\ruby{人}{ひと}があつて、
%
おのづから
\ruby{感{\換字{情}}}{かん|じやう}の
\ruby{訓練}{くん|れん}を
\ruby{敢}{あへ}てせぬ
\ruby{履歷}{り|れき}を
\ruby{有}{いう}して
\ruby{居}{ゐ}る。
%
\ruby{僕}{ぼく}に
\ruby{云}{い}はせれば
\ruby{其}{その}
\ruby{人}{ひと}は
\ruby{最}{もつと}も
\ruby{不幸}{ふ|かう}な
\ruby{人}{ひと}だ。
%
\ruby{直言}{ちよく|げん}すれば、
%
\ruby[g]{水野}{みづの}、
%
\ruby{君}{きみ}が
\ruby{其}{その}
\ruby{人}{ひと}だ。
%
\ruby{君}{きみ}は
\ruby{美}{うるは}しい
\ruby{感{\換字{情}}}{かん|じやう}を
\ruby{有}{いう}して
\ruby{居}{ゐ}て、
%
\ruby{今}{いま}までは
\ruby{訓練}{くん|れん}を
\ruby{要}{{\換字{𛀁}}う}する% ルビは原本通り「𛀁う」
\ruby{事}{こと}がなかつた、
%
それほど
\ruby{美}{うるは}しい
\ruby{感{\換字{情}}}{かん|じやう}を
\ruby{有}{いう}して
\ruby{居}{ゐ}たのだ。
%
その
\ruby{上}{うへ}、
%
\ruby{感{\換字{情}}}{かん|じやう}の
\ruby{訓練}{くん|れん}の
\ruby{必要}{ひつ|{\換字{𛀁}}う}を
\ruby{感}{かん}ずる
\ruby{如}{ごと}き
\ruby{職務}{しよく|む}に
\ruby{身}{み}を
\ruby{置}{お}かなかつたのだ。
%
そこで
\ruby{感{\換字{情}}}{かん|じやう}の
\ruby{訓練}{くん|れん}の
\ruby{履歷}{り|れき}を
\ruby{有}{いう}して
\ruby{居}{ゐ}ぬ、
%
それは
\ruby{慥}{たしか}に
\ruby{大}{おほい}に
\ruby{君}{きみ}を
\ruby{苦}{くるし}めるのだ。
%
\ruby{感{\換字{情}}}{かん|じやう}は
\ruby{馬}{うま}だ。
%
\ruby{{\換字{銳}}}{するど}い
\ruby{感{\換字{情}}}{かん|じやう}を
\ruby{有}{いう}して
\ruby{居}{ゐ}る
\ruby{人}{ひと}は
\ruby{駿馬}{しゆ|んめ}に
\ruby{乘}{の}つて
\ruby{居}{ゐ}る
\ruby{人}{ひと}だ。
%
\ruby{駿馬}{しゆ|んめ}は
\ruby[<j||]{愈}{いよ〳〵}
\ruby{訓練}{くん|れん}せねばならん。
%
\ruby{然}{さ}も
\ruby{無}{な}けれは、
%
\ruby{乘}{の}つて
\ruby{居}{ゐ}るものは
\ruby{危}{あぶな}い
\ruby{目}{め}にあふ。
%
\ruby[g]{水野}{みづの}、
%
\ruby{君}{きみ}は
\ruby{生來}{せい|らい}
\ruby{駿馬}{しゆ|んめ}に
\ruby{乘}{の}つて
\ruby{居}{ゐ}る
\ruby{人}{ひと}だ。
%
\ruby{而}{そ}して
\ruby{今}{いま}
\ruby{其}{そ}の
\ruby{駿馬}{しゆ|んめ}は
\ruby{無法}{む|はふ}に
\ruby{走}{はし}り
\ruby{出}{だ}して
\ruby{居}{ゐ}るのでは
\ruby{無}{な}いか。
%
\ruby{谷}{たに}に
\ruby{陷}{おちい}るか
\ruby{崖}{がけ}から
\ruby{墜}{お}つるか、
%
\ruby{淵}{ふち}へ
\ruby{躍}{をど}り
\ruby{{\換字{込}}}{こ}むか
\ruby{{\換字{前}}{\換字{途}}}{さ|き}が
\ruby{知}{し}れぬ。
%
\ruby{僕等}{ぼく|ら}は
\ruby{傍}{はた}から
\ruby{見}{み}て
\ruby{冷汗}{ひや|あせ}を
\ruby{流}{なが}して、
%
\ruby{非常}{ひ|じやう}に
\ruby{{\換字{寒}}心}{かん|しん}して
\ruby{居}{ゐ}るのだ。
%
\ruby{善}{よ}く
\ruby{御}{ぎよ}さなけれは
\ruby{危險}{き|けん}は
\ruby{目}{め}の
\ruby{{\換字{前}}}{まへ}だ。
%
どうか
\ruby{訓練}{くん|れん}を
\ruby{敢}{あへ}て
\ruby{爲}{し}て
\ruby{吳}{く}れたまへ。
%
\ruby{馬}{うま}のための
\ruby{人}{ひと}では
\ruby{無}{な}い、
%
\ruby{人}{ひと}のための
\ruby{馬}{うま}だ。
%
\ruby{馬}{うま}は
\ruby{人}{ひと}の
\ruby{命令}{めい|れい}に
\ruby{服}{ふく}させて、
%
\ruby{而}{そ}して
\ruby{其}{そ}の
\ruby{能力}{のう|りよく}を
\ruby{盡}{つく}させた
\ruby{時}{とき}、
%
はじめて
\ruby{駿馬}{しゆ|んめ}の
\ruby{貴}{たつと}ぶべきが
\ruby{知}{し}れるのだ。
%
\ruby{{\換字{文}}覺}{もん|がく}の
\ruby{如}{ごと}きは
\ruby{馬{\換字{術}}}{ば|じゆつ}をも
\ruby{心掛}{こゝろ|が}けずして、
%
\ruby{一生}{いつ|しやう}
\ruby{荒馬}{あら|うま}に
\ruby{乘}{の}つて
\ruby{無法}{む|はふ}に
\ruby{驅}{か}けて、
%
\ruby{{\換字{終}}}{しまひ}には
\ruby{撥}{は}ね
\ruby{落}{おと}されて
\ruby{死}{し}んだのに
\ruby{{\換字{過}}}{す}ぎん。
%
\ruby{僕等}{ぼく|ら}は
\ruby{駑馬}{ど|ば}に
\ruby{乘}{の}つて
\ruby{居}{ゐ}るものだ。
%
\ruby{君}{きみ}は
\ruby{幸}{さいはひ}に
\ruby{駿馬}{しゆ|んめ}に
\ruby{乘}{の}つて
\ruby{居}{ゐ}る
\ruby{人}{ひと}だ。
%
くれ〴〵も
\ruby{云}{い}ふ
\ruby{人}{ひと}のための
\ruby{馬}{うま}だ、
%
\ruby{馬}{うま}のための
\ruby{人}{ひと}で
\ruby{無}{な}い。
%
どうか
\ruby{善}{よ}く
\ruby{{\換字{銳}}}{するど}い
\ruby{感{\換字{情}}}{かん|じやう}を
\ruby{御}{ぎよ}して、
%
\ruby{而}{さう}して
\ruby{君}{きみ}の
\ruby{千萬里}{せん|ばん|り}を
\ruby{馳騁}{ち|へい}するところを
\ruby{見}{み}せて
\ruby{吳}{く}れたまへ。
%
\ruby{駿馬}{しゆ|んめ}のために
\ruby{谷}{たに}に
\ruby{陷}{おちい}り
\ruby{淵}{ふち}に
\ruby{落}{お}つる
\ruby{不幸}{ふ|かう}を
\ruby{見}{み}せて
\ruby{吳}{く}れたまふな。
』

\原本頁{}%
と
\ruby{諄々}{じゆん|〳〵}として
\ruby{徐}{しづか}に
\ruby{說}{と}く
\ruby{時}{とき}、
%
\ruby[g]{日方}{ひかた}は
\ruby{膝}{ひざ}を
\ruby{打}{う}つて
\ruby{嗟嘆}{さ|たん}して、

\原本頁{}%
『
\ruby{可矣}{い|ゝ}。
%
\ruby{確言動}{くわく|げん|うご}かすべからずだ。
%
\ruby[g]{羽{\換字{勝}}}{はがち}の
\ruby{言}{げん}だけある!。
%
\ruby{此馬陣}{この|うま|ぢん}に
\ruby{臨}{のぞ}んで
\ruby{久}{ひさ}しく
\ruby{敵無}{てき|な}し、
%
\ruby{人}{ひと}と
\ruby{一心}{いつ|しん}にして
\ruby{大功}{たい|こう}を
\ruby{成}{な}すといふ、
%
\ruby{句}{く}の、
%
\ruby{彼}{あ}の
\ruby{人}{ひと}と
\ruby{一心}{いつ|しん}といふ
\ruby{四字}{よ|じ}が
\ruby{響}{ひゞ}き
\ruby{渡}{わた}つて、
%
\ruby{今{\換字{更}}}{いま|さら}
\ruby{{\換字{強}}}{つよ}く
\ruby{面白}{おも|しろ}く
\ruby{感}{かん}じられる!。
%
\ruby[g]{水野}{みづの}、
%
\ruby{馬}{うま}をして
\ruby{我}{わ}が
\ruby{意}{こゝろ}に
\ruby{從}{したが}はしめなければならんぞ。
』

\原本頁{}%
と
\ruby{傍}{かたはら}よりまた
\ruby{言葉}{こと|ば}を
\ruby{添}{そ}へたり。

\Entry{其四十八}

\ruby{日方}{ひ|かた}が
\ruby{手荒}{て|あら}き
\ruby{擧動}{ふる|まひ}といひ、
\ruby{羽{\換字{勝}}}{は|がち}が
\ruby{物固}{もの|がた}き
\ruby{言葉}{こと|ば}といひ、
\ruby{皆}{みな}これ
\ruby{淺}{あさ}からず
\ruby{我}{われ}を
\ruby{思}{おも}ひ
\ruby{吳}{く}るゝ
\ruby{朋友}{と|も}の
\ruby{{\換字{情}}}{なさけ}の
\ruby{眞實}{まこ|と}なりとおもふに、
\ruby{水野}{みづ|の}は
\ruby{泣}{な}かぬばかりの
\ruby{面}{かほ}つきとなつて、
\ruby{血}{ち}の
\ruby{氣}{け}も
\ruby{失}{う}せたるやうの
\ruby{兩}{りやう}の
\ruby{頬}{ほう}には、
\ruby{勢無}{いきほひ|な}き
\ruby{心}{こゝろ}の
\ruby{淋}{さび}しさを
\ruby{現}{あら}はし、
\ruby{露}{つゆ}ばかりも
\ruby{動}{うご}かざる
\ruby{眼}{め}の
\ruby{中}{うち}は
\ruby{一念}{いち|ねん}の
\ruby{沈}{しづ}みきつて
\ruby{一}{ひ}ト
\ruby{處}{ところ}に
\ruby{凝}{こ}れる
\ruby{狀態}{あり|さま}を
\ruby{示}{しめ}す
\ruby{如}{ごと}く、やゝ
\ruby{少時}{しば|し}は
\ruby{物}{もの}をさへ
\ruby{云}{い}ひ
\ruby{{\換字{兼}}}{か}ねたりしが、やがて
\ruby{感激}{かん|げき}に
\ruby{堪}{た}へ
\ruby{得}{\換字{𛀁}}ずしてや、さしぐむ
\ruby{淚}{なみだ}に
\ruby{聲}{こゑ}も
\ruby{{\換字{弱}}々}{よわ|〳〵}と、

『あゝ
\ruby{有難}{あり|がた}い!、
\ruby{實}{じつ}に
\ruby{謝}{しや}する!、
\ruby{二君}{に|くん}の
\ruby{厚意}{かう|い}は
\ruby{决}{けつ}して
\ruby{忘}{わす}れぬ。
\ruby{特}{こと}に
\ruby{羽{\換字{勝}}}{は|がち}
\ruby{君}{くん}の
\ruby{敎}{をしへ}は
\ruby{心魂}{しん|こん}に
\ruby{徹}{てつ}して、
\ruby{愚鈍}{ぐ|どん}の
\ruby{僕}{ぼく}にもよく
\ruby{解}{わか}つた。
\ruby{君等}{きみ|たち}の
\ruby{親切}{しん|せつ}に
\ruby{激勵}{は|げ}まされて、
\ruby{出來}{で|き}ないまでも
\ruby{僕}{ぼく}は
\ruby{自}{みづか}ら
\ruby{勉}{つと}めて
\ruby{{\換字{過}}}{あやま}たぬやうにする。
\ruby{感{\換字{情}}}{かん|じやう}の
\ruby{訓練}{くん|れん}といふ
\ruby{事}{こと}も
\ruby{屹度}{きつ|と}
\ruby{敢}{あへ}てする。
\ruby{不幸}{ふ|かう}にして
\ruby{力}{ちから}が
\ruby{足}{た}らなくつて、
\ruby{轉}{ころ}んでも
\ruby{倒}{たふ}れても
\ruby{溪}{たに}に
\ruby{落}{お}ちても、
\ruby{轉}{ころ}べば
\ruby{起上}{おき|あが}る、
\ruby{倒}{たふ}るれば
\ruby{立}{た}つ、
\ruby{溪}{たに}に
\ruby{落}{お}ちても
\ruby{屹度}{きつ|と}
\ruby{這}{は}ひ
\ruby{上}{あが}つて、
\ruby{目}{め}ざところまで
\ruby{必}{かなら}ず
\ruby{行}{ゆ}かうといふ
\ruby{氣}{き}ばかりは、
\ruby{何樣}{ど|う}あつても
\ruby{屹度}{きつ|と}
\ruby{忘}{わす}れぬつもりだ。
\ruby{僕}{ぼく}に
\ruby{生命}{いの|ち}の
\ruby{有}{あ}らん
\ruby{限}{かぎ}りは、
\ruby{一日}{いち|にち}に
\ruby{一日}{いち|にち}だけ
\ruby{此}{こ}の
\ruby{心}{こゝろ}を
\ruby{懷}{いだ}いて、
\ruby{苦}{くるし}んでも
\ruby{悶}{もだ}えても
\ruby{生存}{なが|ら}へやうと
\ruby{思}{おも}ふ
\ruby{此}{こ}の
\ruby{僕}{ぼく}の
\ruby{眞}{しん}の
\ruby{意}{こゝろ}を
\ruby{汲}{く}んで
\ruby{吳}{く}れて、
\ruby{何樣}{ど|う}か
\ruby{僕}{ぼく}を
\ruby{見放}{み|はな}さずに
\ruby{居}{ゐ}て
\ruby{吳}{く}れたまへ。
\ruby{長}{なが}く
\ruby{此}{こ}の
\ruby{僕}{ぼく}に
\ruby{君等}{きみ|たち}の
\ruby{友}{とも}たる
\ruby{幸福}{さい|はひ}を
\ruby{得}{\換字{𛀁}}させて
\ruby{置}{おい}て
\ruby{吳}{く}れたまへ。
\ruby{君等}{きみ|たち}は
\ruby{皆優}{みな|やさ}しく
\ruby{敎}{をし}へて
\ruby{吳}{く}れるし、
\ruby{自{\換字{分}}}{じ|ぶん}でも
\ruby{氣}{き}が
\ruby{付}{つ}いて
\ruby{居}{ゐ}るし、
\ruby{自}{みづか}ら
\ruby{克}{か}たうとしたり
\ruby{自}{みづか}ら
\ruby{憤}{いきどほ}つたり、
\ruby{自}{みづか}ら
\ruby{爭}{あらそ}つたり
\ruby{自}{みづか}ら
\ruby{鬪}{たゝか}つたり、
\ruby{心}{こゝろ}の
\ruby{中}{うち}の
\ruby{揉}{も}めぬ
\ruby{日}{ひ}も
\ruby{無}{な}く、
\ruby{力}{ちから}も
\ruby{根}{こん}も
\ruby{使}{つか}ひ
\ruby{盡}{つく}して
\ruby{今日}{け|ふ}まで
\ruby{來}{き}たが、
\ruby{何}{なん}と
\ruby{無}{な}く
\ruby{行末}{ゆく|すゑ}が
\ruby{物怖}{もの|おそろ}しくて、
\ruby{知}{し}りつゝ
\ruby{高}{たか}い
\ruby{崖}{がけ}から
\ruby{深}{ふか}い
\ruby{淵}{ふち}に
\ruby{陷}{おちい}るやうな
\ruby{時}{とき}が
\ruby{有}{あ}りはせぬかと
\ruby{思}{おも}ふ。
\ruby{必}{かなら}ず〳〵
\ruby{其樣}{そ|ん}なことにはならい
\ruby{樣}{やう}に、
\ruby{君等}{きみ|たち}の
\ruby{厚意}{かう|い}を
\ruby{空}{むな}しくせぬやうにと、
\ruby{一生懸命}{いつ|しやう|けん|めい}に
\ruby{思}{おも}つては
\ruby{居}{ゐ}るが、
\ruby{萬一}{まん|いち}
\ruby{萬々一}{まん|〳〵|いち}
\ruby{左樣}{さ|ふ}いふ
\ruby{目}{め}にあつても、
\ruby{屹度}{きつ|と}それきりにはならぬつもりの、
\ruby{其點}{そ|こ}を
\ruby{水野}{みづ|の}だと
\ruby{見}{み}て
\ruby{吳}{く}れて、あれほど
\ruby{諭}{さと}したのに
\ruby{云}{い}ひ
\ruby{甲斐}{が|ひ}の
\ruby{無}{な}い、とう〳〵
\ruby{深}{ふか}みへ
\ruby{落}{お}ちた
\ruby{馬鹿}{ば|か}な
\ruby{奴}{やつ}だと、
\ruby{爪彈}{つま|はじ}きして
\ruby{棄}{す}てるやうなことを
\ruby{爲}{し}て
\ruby{吳}{く}れたまふな。
\ruby{餘}{あま}り
\ruby{愚}{ぐ}な
\ruby{事}{こと}をいふやうだが、たゞ
\ruby{何}{なん}と
\ruby{無}{な}く
\ruby{僕}{ぼく}の
\ruby{{\換字{前}}途}{ぜん|と}に
\ruby{恐}{おそ}ろしい
\ruby{不幸}{ふ|かう}が
\ruby{手}{て}を
\ruby{擴}{ひろ}げて、
\ruby{僕}{ぼく}の
\ruby{行}{ゆ}くのを
\ruby{待}{ま}つて
\ruby{居}{ゐ}るやうに
\ruby{思}{おも}へる。
\ruby{何樣}{ど|う}も
\ruby{左樣}{さ|う}
\ruby{思}{おも}へてならんので、それで
\ruby{如是}{こ|ん}なことも
\ruby{言}{い}ひ
\ruby{出}{いだ}すのだが、
\ruby{何樣}{ど|う}
\ruby{罷}{まか}り
\ruby{問{\換字{違}}}{ま|ちが}つても
\ruby{本來}{ほん|らい}の
\ruby{一心}{いつ|しん}は、
\ruby{君等}{きみ|たち}に
\ruby{對}{たい}しても
\ruby{决}{けつ}して
\ruby{忘}{わす}れぬ、
\ruby{其處}{そ|こ}をたゞ
\ruby{水野}{みづ|の}だと
\ruby{思}{おも}つて
\ruby{{\換字{交}}際}{つき|あ}つて
\ruby{吳}{く}れたまへ。

\ruby{人}{ひと}の
\ruby{{\換字{運}}命}{うん|めい}の
\ruby{明日}{あし|た}は
\ruby{{\換字{分}}}{わか}らぬが、
\ruby{君等}{きみ|たち}の
\ruby{厚意}{かう|い}は
\ruby{夢}{ゆめ}の
\ruby{間}{ま}も
\ruby{忘}{わす}れぬ。
\ruby{君等}{きみ|たち}に
\ruby{負}{そむ}かぬやうにとは
\ruby{屹度}{きつ|と}
\ruby{努力}{ど|りよく}する。
』

と、
\ruby{心}{こゝろ}に
\ruby{張}{は}りのあるさまは
\ruby{{\換字{猶}}}{なほ}
\ruby{見}{み}えながら、
\ruby{意氣}{い|き}は
\ruby{振}{ふる}はずして
\ruby{龍鍾}{しを|〳〵}と
\ruby{言}{い}ふ
\ruby{其}{そ}の
\ruby{哀}{あは}れなる
\ruby{樣子}{やう|す}を
\ruby{日方}{ひ|かた}は
\ruby{見{\換字{過}}}{み|すご}しかね、

『なに!、
\ruby{何}{なん}と
\ruby{無}{な}く
\ruby{行末}{ゆく|すゑ}が
\ruby{怖}{おそ}ろしくつて、
\ruby{不幸}{ふ|かう}の
\ruby{{\換字{運}}命}{うん|めい}が
\ruby{待}{まつ}て
\ruby{居}{ゐ}るやうに
\ruby{思}{おも}へるつて?。
\ruby{何其樣}{なに|そ|ん}なことが
\ruby{有}{あ}つて
\ruby{堪}{たま}るものか。
\ruby{我々}{われ|〳〵}の
\ruby{行末}{ゆく|すゑ}は
\ruby{皆輝}{みな|かゞや}いて
\ruby{居}{ゐ}る!。
\ruby{我々七人}{われ|〳〵|しち|にん}の
\ruby{行末}{ゆく|すゑ}に
\ruby{暗黒}{や|み}は
\ruby{無}{な}いのた!。
\ruby{燃}{も}える
\ruby{火}{ひ}の
\ruby{{\換字{前}}}{まへ}に
\ruby{暗黒}{や|み}が
\ruby{有}{あ}るかい!。
\ruby{暗黒}{や|み}はたゞ
\ruby{{\換字{過}}}{す}ぎた
\ruby{昨日}{きの|ふ}の
\ruby{事}{こと}!。

\ruby{生}{い}きて
\ruby{居}{ゐ}る
\ruby{人間}{にん|げん}、
\ruby{燃}{も}えて
\ruby{居}{ゐ}る
\ruby{火}{ひ}の、
\ruby{其{\換字{前}}}{その|まへ}に
\ruby{暗黒}{や|み}が
\ruby{有}{あ}るとは
\ruby{誰}{たれ}が
\ruby{言}{い}ふ?。
そんな
\ruby{事}{こと}を
\ruby{思}{おも}ふのは
\ruby{氣}{き}の
\ruby{{\換字{迷}}}{まよ}ひだ。
\ruby{悉皆}{みん|な}
\ruby{汝}{きさま}の
\ruby{衰{\換字{弱}}}{おと|ろへ}からだ!。
しつかり
\ruby{爲}{し}なくてはいかんぞ
\ruby{水野}{みづ|の}!。
\ruby{喇叭}{らつ|ぱ}が
\ruby{{\換字{進}}}{すゝ}めと
\ruby{鳴}{な}りやあ
\ruby{敵}{てき}はもう
\ruby{無}{な}いんだ。
\ruby{大{\換字{丈}}夫}{だい|ぢやう|ぶ}の
\ruby{向}{むか}つて
\ruby{行}{ゆ}くところには
\ruby{不幸}{ふ|かう}も
\ruby{何}{なに}も
\ruby{無}{な}い。
\ruby{下}{くだ}らんことをいつてまだ
\ruby{撲}{なぐ}られたいか。
\ruby{羽{\換字{勝}}}{は|がち}
\ruby{言}{げん}に
\ruby{從}{したが}つて
\ruby{努力}{ど|りよく}して
\ruby{日}{ひ}を
\ruby{{\換字{送}}}{おく}れ。
\ruby{汝}{きさま}の
\ruby{{\換字{前}}{\換字{途}}}{ぜん|と}の
\ruby{多幸}{た|かう}なのは
\ruby{乃公}{お|れ}が
\ruby{受合}{うけ|あ}ふ。
』

と
\ruby{壯語}{さう|ご}の
\ruby{有}{あ}る
\ruby{限}{かぎ}りを
\ruby{盡}{つく}して
\ruby{氣}{き}を
\ruby{引立}{ひき|た}てたる
\ruby{其時室外}{その|とき|しつ|ぐわい}に
\ruby{人}{ひと}の
\ruby{氣色}{け|はひ}して、
\ruby{忽}{たちま}ち
\ruby{間}{あひ}の
\ruby{襖}{ふすま}は
\ruby{右左}{みぎ|ひだり}に
\ruby{大}{おほき}く
\ruby{開}{ひら}かれたり。


\Entry{其四十九}

\原本頁{}%
\ruby{壽長}{いのち|なが}ければ
\ruby{智慧}{ち|ゑ}
\ruby{多}{おほ}し。
%
\ruby[g]{吉右衛門}{きちゑもん}は
\ruby{眼}{め}に
\ruby{世}{よ}の
\ruby{人}{ひと}の
それ〴〵を
\ruby{見覺}{み|おぼ}えて、
%
\ruby[g]{水野}{みづの}を
\ruby{今}{いま}に
\ruby{稀}{まれ}なる
\ruby{{\換字{若}}者}{わか|もの}と
\ruby{悅}{よろこ}び、
%
\ruby{初}{はじめ}はたゞ
\ruby{高田}{たか|た}の
\ruby{依頼}{たの|み}によりて
\ruby{寄寓}{き|ぐう}を
\ruby{許}{ゆる}したるに
\ruby{{\換字{過}}}{す}ぎざりしが、
%
\ruby{後}{のち}
\ruby{後}{おく}
には
\ruby{其}{そ}の
\ruby{品行}{おこ|なひ}を
\ruby{見}{み}、
%
\ruby{其}{そ}の
\ruby{人}{ひと}となりを
\ruby{知}{し}つて、
%
\ruby{之}{これ}を
\ruby{重}{おも}んずることは
\ruby{主}{しゆ}の
\ruby{如}{ごと}く、
%
\ruby{之}{これ}を
\ruby{思}{おも}ふことは
\ruby{子}{こ}の
\ruby{如}{ごと}く、
%
\ruby{他人}{た|にん}あしらひにはせずして
\ruby{月日}{つき|ひ}を
\ruby{{\換字{過}}}{すご}し
\ruby{來}{きた}れる
\ruby{程}{ほど}なれば、
%
\ruby{今}{いま}
\ruby{本家}{ほん|け}より
\ruby{歸}{かへ}り
\ruby{來}{きた}りて、
%
\ruby[g]{水野}{みづの}が
\ruby{許}{もと}に
\ruby{訪}{と}ひ
\ruby{寄}{よ}れる
\ruby{人々}{ひと|〴〵}の、
%
いづれも
\ruby{表面}{うは|べ}ばかりの
\ruby{友}{とも}にはあらずして、
%
\ruby[g]{水野}{みづの}のために
\ruby{或}{あるひ}は
\ruby{諫}{いさ}め
\ruby{或}{あるひ}は
\ruby{諭}{さと}す
\ruby{其}{そ}の
\ruby{一片}{かた|はし}を、
%
ちら〳〵と
\ruby{耳}{みゝ}に
\ruby{入}{い}るゝにつけ、
%
\ruby{特}{こと}には
\ruby[g]{日方}{ひかた}といへるが
\ruby{如何}{い|か}に
\ruby{振舞}{ふる|ま}ひて、
%
また
\ruby{我}{わ}が
\ruby{孫}{まご}の
お
\ruby{濱}{はま}が
\ruby[g]{日方}{ひかた}に
\ruby{對}{たい}して
\ruby{如何}{い|か}に
\ruby{振舞}{ふる|まひ}ひしかをも
\ruby{聞}{き}きて
\ruby{知}{し}るにつけ、
%
たゞ
\ruby{其}{そ}のままにはあり
\ruby{得}{{\換字{𛀁}}}ぬ
\ruby{心地}{こゝ|ち}して、
%
\ruby{不自由}{ふ|じ|ゆう}なる
\ruby{田舎}{ゐな|か}の
\ruby{心}{こゝろ}には
\ruby{任}{まか}せねど、
%
お
\ruby{濱}{はま}
お
\ruby{鍋}{なべ}に
\ruby{指揮}{さし|づ}して
\ruby{酒肴}{しゆ|かう}を
\ruby{調}{とゝの}へしめ、
%
\ruby[g]{水野}{みづの}が
\ruby{命令}{いひ|つけ}の
\ruby{無}{な}きにも
\ruby{關}{かゝは}らず、
%
\ruby{其}{その}
\ruby{座}{ざ}に
\ruby{其}{それ}を
\ruby{持出}{もち|いだ}さしめたり。
%
\ruby{老人}{らう|じん}の
\ruby{親切}{しん|せつ}なる
\ruby{心}{こゝろ}より、
%
\ruby{此頃}{この|ごろ}の
\ruby[g]{水野}{みづの}の
\ruby{擧動}{ふる|まひ}を
\ruby{憂}{うれ}ひ
\ruby{居}{ゐ}し
\ruby{矢先}{や|さき}に、
%
\ruby{我}{わ}が
\ruby{心}{こゝろ}を
\ruby{得}{{\換字{𛀁}}}たる
\ruby{二人}{ふた|り}の
\ruby{客}{きやく}の
\ruby{物語}{もの|がたり}をば、
%
\ruby{一}{ひ}ト
\ruby{方}{かた}ならず
\ruby{嬉}{うれ}しく
\ruby{思}{おも}へる
\ruby{餘}{あま}りなるべし。

\原本頁{}%
\ruby{何}{なん}の
\ruby{馳走}{ち|そう}も
\ruby{無}{な}き
\ruby{饗應}{もて|なし}なれど、
%
\ruby{膳}{ぜん}を
\ruby{配}{くば}らせながら
\ruby[g]{吉右衛門}{きちゑもん}は
\ruby{笑}{ゑ}みつ、

\原本頁{}%
『どなだも
\ruby{邊鄙}{へん|ぴ}のところへ
\ruby{好}{よ}く
\ruby{御來臨}{お|い|で}なさいました、
%
\ruby{私}{わたくし}は
\ruby{此家}{こ|ゝ}の
\ruby{老夫}{おや|ぢ}でございますが、
%
\ruby{此}{こ}の
\ruby{兀}{は}げたところをでも
\ruby{今後}{これ|から}
\ruby{御覺}{お|おぼ}え
\ruby{願}{ねが}ひます。
%
\ruby[g]{島木}{しまき}さんには
\ruby{御心易}{お|こゝろ|やす}く
\ruby{願}{ねが}つて
\ruby{居}{を}ります、
%
\ruby{折角}{せつ|かく}
\ruby{諸君}{みな|さん}が
\ruby{來臨下}{おい|で|くだ}すつたのですから、
』

\原本頁{}%
と
\ruby{云}{い}ひかけて
\ruby{一寸}{ちよ|つと}
\ruby[g]{水野}{みづの}を
\ruby{見}{み}て、

\原本頁{}%
『お
\ruby{差圖}{さし|づ}も
\ruby{伺}{うかゞ}ひませんでしたが、
%
\ruby{御談話}{お|はな|し}の
\ruby{繋}{つな}ぎのためばかりに、
%
\ruby{一獻}{ひと|つ}あげるやうに
\ruby{致}{いた}しました。
%
\ruby{田舎}{ゐな|か}の
\ruby{事}{こと}ですから
\ruby{何}{なに}もございません。
%
おまけに
\ruby{飮酒家}{や|り|て}の
\ruby{無}{な}い
\ruby{家}{うち}の
\ruby{事}{こと}でございますから、
%
\ruby{御惣{\換字{菜}}}{お|そう|ざい}みたやうなものばかりで、
%
\ruby{氣取}{き|どり}も
\ruby{何}{なに}もございませんが、
%
まあ
\ruby{何}{なに}も
\ruby{御笑}{お|わら}ひ
\ruby{草}{ぐさ}になすつて
\ruby{飮}{あが}つて
\ruby{下}{くだ}さいまし。
%
\ruby[g]{日方}{ひかた}さんへは
\ruby{御謝罪}{お|わ|び}の
\ruby{印}{しるし}と
\ruby{申}{まを}しましても
\ruby{宜}{よ}いので、
%
\ruby{孫}{まご}めが
\ruby{飛}{と}んだ
\ruby{失禮}{しつ|れい}を
\ruby{致}{いた}しましが、
%
\ruby{何樣}{ど|う}か
\ruby{御勘辨}{ご|かん|べん}% 弁 瓣 辦 辧 (辨) 辩 辯
\ruby{下}{くだ}さいまし、
%
\ruby{其}{その}
\ruby{代}{かは}り
\ruby[g]{澤山}{たんと}
\ruby{御{\換字{酌}}}{お|しやく}をさせますから、
%
ハヽヽ。
%
これお
\ruby{濱}{はま}こゝへ
\ruby{來}{き}て
\ruby{御謝罪}{お|わ|び}を
\ruby{仕}{し}ろ。
』

\原本頁{}%
と
\ruby{云}{い}へば、
%
\ruby{其}{そ}の
\ruby{背後}{うし|ろ}に
\ruby{小}{ちひさ}くなり
\ruby{居}{ゐ}し
お
\ruby{濱}{はま}は、
%
\ruby{面}{おもて}を
\ruby{染}{そ}めて
\ruby{是非無}{ぜ|ひ|な}く
\ruby{頭}{かうべ}を
\ruby{下}{さ}げんとす。
%
\ruby[g]{日方}{ひかた}は
\ruby{老{\換字{父}}}{ぢ|ゞ}の
\ruby{言}{ことば}を
\ruby{心地}{こゝ|ち}
\ruby{快}{よ}げに
\ruby{聞}{き}き
\ruby{居}{ゐ}しが

\原本頁{}%
『ハヽヽ。
%
\ruby{君}{きみ}、
%
なに、
%
\ruby{謝罪}{あや|ま}らんでも
\ruby{可}{い}いさ。
%
お
\ruby{濱}{はま}さんといふかね、
%
\ruby{好}{い}い
\ruby{氣象}{き|しやう}の
\ruby{娘}{むすめ}さんだ。
%
\ruby[g]{日方}{ひかた}
\ruby{八郎}{はち|らう}
\ruby{生}{ゝま}れて
\ruby{初}{はじ}めて
\ruby{頭}{あたま}へ
\ruby{手}{て}を
\ruby{上}{あ}げられたが、
%
\ruby{打}{ぶ}たれて
\ruby{怒}{おこ}るどころではない、
%
\ruby{全然}{すつ|かり}
\ruby{感心}{かん|しん}した。
%
\ruby{日本}{につ|ぽん}の
\ruby{{\換字{婦}}女}{をん|な}は
\ruby{誰}{たれ}も
\ruby{彼}{かれ}も、
%
お
\ruby{濱}{はま}さんのやうな
\ruby{氣合}{き|あひ}で
\ruby{居}{ゐ}て
\ruby{欲}{ほ}しい。
%
\ruby{偉}{{\換字{𛀁}}ら}い
\ruby{娘}{むすめ}さんだ、
%
\ruby{好}{い}い
\ruby{氣象}{き|しやう}だ。
%
\ruby{祖{\換字{父}}}{お|ぢい}さんに
\ruby{何}{なん}か
\ruby{云}{い}はれたつて
\ruby{頭}{あたま}なんか
\ruby{下}{さ}げてはいかん。
%
\ruby{其}{その}
\ruby{代}{かは}り
\ruby{御{\換字{酌}}}{お|しやく}は
\ruby{御{\換字{遠}}慮無}{ご|ゑん|りよ|な}しに
\ruby{願}{ねが}はう。
%
ハヽヽ。
』

\原本頁{}%
と
\ruby{無邪氣}{む|じや|き}に
\ruby{制}{せい}し
\ruby{止}{とゞ}めたり。

\原本頁{}%
『
\ruby{左樣}{さ|う}
\ruby{仰}{おつし}あつて
\ruby{下}{くだ}されば
\ruby{先}{ま}づ
\ruby{老夫}{ぢゞ|い}も
\ruby{助}{たす}かります。
%
\ruby{何樣}{ど|う}か
\ruby{御機{\換字{嫌}}}{ご|き|げん}
\ruby{好}{よ}く
\ruby{御談}{お|はな}しなすつて。
%
\ruby{兀頭}{はげ|あたま}は
\ruby{{\換字{古}}風}{むか|し}
\ruby{物}{もの}で
\ruby{時代{\換字{違}}}{じ|だい|ちが}ひですから、
%
\ruby{御{\換字{若}}}{お|わか}い
\ruby{方}{かた}の
\ruby{中}{なか}では
\ruby{氣}{き}が
\ruby{{\換字{退}}}{ひ}けてなりません。
%
\ruby{御免蒙}{ご|めん|かうむ}りますから
\ruby{御寛}{ご|ゆる}りと。
』

\原本頁{}%
『イヤ
\ruby{左樣}{さ|う}で
\ruby{無}{な}い。
%
\ruby{君}{きみ}は
\ruby{中々}{なか|〳〵}
\ruby{話}{はな}せる。
%
いゝぢや
\ruby{無}{な}いか
\ruby{老{\換字{翁}}}{おぢい|さん}、
%
ここに
\ruby{居}{ゐ}たまヘナ。
』

\原本頁{}%
『ハヽヽ、
%
\ruby{有}{あ}り
\ruby{難}{がた}うございますが
\ruby{萬一}{ひよ|つと}
\ruby{何樣}{ゞ|ん}な
\ruby{事}{こと}でか
\ruby{叱}{しか}られまして、
%
\ruby{{\換字{若}}}{も}し
\ruby{御卷骨}{お|げん|こつ}を
\ruby{頂戴}{ちやう|だい}しますと、
%
\ruby{兀頭}{は|げ}は
\ruby{特別}{とく|べつ}に
\ruby{利}{き}きますからナ。
%
まあ
\ruby{引{\換字{退}}}{ひき|さが}つて
\ruby{居}{ゐ}る
\ruby{方}{はう}が
\ruby{無難}{ぶ|なん}でございます。
%
ハヽヽ、
%
イヤこれは
\ruby{冗談}{じやう|だん}を、
%
\ruby{失禮}{しつ|れい}いたしました。
』

\原本頁{}%
\ruby[g]{吉右衛門}{きちゑもん}は
\ruby{{\換字{終}}}{つひ}に
\ruby[g]{彼方}{かなた}へ
\ruby{去}{さ}れば、
%
\ruby[g]{日方}{ひかた}は
\ruby[g]{羽{\換字{勝}}}{はがち}と
\ruby{相見}{あひ|み}て
\ruby{笑}{わら}つて、

\原本頁{}%
『
\ruby{好}{い}い
\ruby{老夫}{おぢい|さん}だナア。
%
\ruby{如何}{い|か}にも
\ruby{奇麗}{き|れい}な
\ruby{輕}{かる}い
\ruby{調子}{てう|し}で、
%
そして
\ruby{親切}{しん|せつ}に
\ruby{滿}{み}ちて
\ruby{居}{ゐ}る、
%
\ruby{{\換字{透}}徹}{すき|とほ}るやうな
\ruby{人}{ひと}だナ。
』

\原本頁{}%
『
\ruby{左樣}{さ|う}だ。
%
まだ
\ruby{我々}{われ|〳〵}の
\ruby{及}{およ}ばんところがある。
』

\原本頁{}%
と
\ruby{{\換字{評}}}{ひやう}し
\ruby{合}{あ}つて
\ruby{樂}{たの}しげに
\ruby{酒盞}{さか|づき}を
\ruby{擧}{あ}げたり。

\原本頁{}%
『ハヽヽ、
%
\ruby{乃公}{お|れ}ぐらゐ
\ruby{能}{よ}く
\ruby{飮}{の}む
\ruby{奴}{やつ}はあるまい。
%
\ruby{何}{なん}だか
\ruby{老人}{おぢい|さん}が
\ruby{出}{で}て
\ruby{來}{き}たので
\ruby{甚}{ひど}く
\ruby{氣}{き}が
\ruby{和}{やはら}いで、
%
\ruby{何程}{いく|ら}でも
\ruby{悠然}{ゆつ|くり}と
\ruby{飮}{の}めさうなやうな
\ruby{心持}{こゝろ|もち}になつて
\ruby{來}{き}た。
』

\Entry{其五十}

\原本頁{}%
\ruby{言}{ものい}はねども
\ruby{花}{はな}あれば
\ruby{野}{の}は
\ruby{自}{おのづ}から
\ruby{春}{はる}なり。
%
あどけ
\ruby{無}{な}き
お
\ruby{濱}{はま}
\ruby{一人}{ひと|り}の
\ruby{{\換字{交}}}{まじ}りたるに
\ruby{一座}{いち|ざ}は
\ruby{和}{やはら}ぎて
\ruby{理屈}{り|くつ}を
\ruby{離}{はな}るれば
\ruby{談話}{はな|し}に
\ruby{角無}{かど|な}く、
%
\ruby{笑聲}{せう|せい}
\ruby[|j>]{漸}{やうや}く
\ruby{起}{おこ}れば
\ruby{酒}{さけ}の
\ruby{味饒}{あぢは|ひおほ}く、
%
\ruby{謹嚴}{きん|ごん}の
\ruby{羽{\換字{勝}}}{は|がち}、
%
\ruby{沈鬱}{ちん|うつ}せる
\ruby[g]{水野}{みづの}さへ、
%
\ruby{何時}{い|つ}か
\ruby{六七年}{ろく|しち|ねん}の
\ruby{往時}{むか|し}に
\ruby{復}{かへ}りて、
%
\ruby{心}{こゝろ}は
\ruby{{\換字{若}}}{わか}く
\ruby{氣}{き}は
\ruby{易}{やす}く
\ruby{語}{かた}らへば、
%
まして
\ruby[g]{日方}{ひかた}は
\ruby{興}{きよう}に
\ruby{入}{い}りて、
%
\ruby{羽{\換字{勝}}}{は|がち}の
\ruby{斥}{しりぞ}けたる
\ruby{天眞爛{\換字{熳}}}{てん|しん|らん|まん}、
%
\ruby{醉態淋漓}{すゐ|たい|りん|り}として
\ruby{受}{う}けては
\ruby{飮}{の}み
\ruby{受}{う}けては
\ruby{飮}{の}み、

\原本頁{}%
『
\ruby[g]{島木}{しまき}、
%
\ruby{馬鹿野郎}{ば|か|や|らう}、
%
\ruby{一緖}{いつ|しよ}に
\ruby{來}{く}れば
\ruby{宜}{い}いのに。
%
\ruby{金儲}{かね|まうけ}に
\ruby{忙}{いそが}しがつたつて
\ruby{何}{なん}になるものか。
』

\原本頁{}%
と
\ruby{幾度}{いく|たび}か
\ruby{繰}{く}り
\ruby{{\換字{返}}}{かへ}して
\ruby{罵}{のゝし}つては、
%
\ruby{{\換字{又}}}{また}
\ruby{餘念}{よ|ねん}も
\ruby{無}{な}く
\ruby{二人}{ふた|り}を
\ruby{相手}{あひ|て}に
\ruby{談笑}{だん|せう}して% 原本通り「だんせ(う)」
\ruby{盃}{さかづき}を
\ruby{手}{て}にしたり。

\原本頁{}%
『お
\ruby{濱}{はま}さん、
%
その
\ruby{色}{いろ}の
\ruby{黑}{くろ}い
\ruby{眞面目}{ま|じ|め}
\ruby{老夫}{おや|ぢ}の
\ruby{羽{\換字{勝}}}{は|がち}に
\ruby{飮}{の}ませて
\ruby{{\換字{遣}}}{や}つて
\ruby{吳}{く}れたまへ。
%
コラ
\ruby{羽{\換字{勝}}}{は|がち}!、
%
\ruby{飮}{の}まんかい、
%
\ruby[g]{水野}{みづの}の
\ruby{妹}{いもうと}の
\ruby{{\換字{酌}}}{しやく}だ。
%
ハヽハ
\ruby{{\換字{船}}}{ふね}では
\ruby{成}{な}るべく
\ruby{酒}{さけ}を
\ruby{用}{もち}ゐん
\ruby{{\換字{習}}慣}{く|せ}を
\ruby{付}{つ}けて
\ruby{居}{ゐ}るから
\ruby{飮}{の}めんなぞといふのは
\ruby{虛言}{う|そ}だらう。
%
\ruby{{\換字{船}}員}{ふな|のり}は
\ruby{大抵}{たい|てい}
\ruby{善}{よ}く
\ruby{飮}{の}むといふぞ。
』

\原本頁{}%
『イヤもういかん。
%
\ruby{虛言}{う|そ}では
\ruby{無}{な}い、
%
\ruby{{\換字{船}}}{ふね}では
\ruby{成}{な}るべく
\ruby{用}{もち}ゐんやうにして
\ruby{居}{ゐ}るのだ。
%
\ruby{執務}{しつ|む}の
\ruby{不確實}{ふ|かく|じつ}になる
\ruby{基}{もとゐ}だから
\ruby{飮酒}{いん|しゆ}は
\ruby{忌}{い}む。
%
これは
\ruby{海員}{かい|ゐん}の
\ruby{精神}{せい|しん}の
\ruby{{\換字{進}}歩}{しん|ぽ}した
\ruby{趨勢}{すう|せい}で、
%
\ruby{{\換字{古}}來}{こ|らい}の
\ruby{海員}{かい|ゐん}の
\ruby{飮酒}{いん|しゆ}に
\ruby{耽}{ふけ}つた
\ruby{惡{\換字{習}}}{あく|しふ}を
\ruby{洗}{あら}ふ
\ruby{任}{にん}は
\ruby{我々}{われ|〳〵}の
\ruby{肩}{かた}にあるのだ。
%
だから
\ruby{實際}{じつ|さい}
\ruby{僕}{ぼく}なぞは
\ruby{餘}{あま}り
\ruby{用}{もち}ゐん。
%
しかし
\ruby{非常}{ひ|じやう}な
\ruby{暴風雨}{ぼう|ふう|ゝ}の
\ruby{時}{とき}、
%
\ruby{襯衣}{シヤ|ツ}まで
\ruby{濡}{ぬ}れ
\ruby{浸}{ひた}りながら
\ruby{困苦}{こん|く}
\ruby{極}{きは}まる
\ruby{勞働}{らう|どう}を
\ruby{仕}{し}た
\ruby{後}{あと}などでは、
%
\ruby{水夫等}{すゐ|ふ|ら}にも
\ruby{少量}{せう|りやう}の
\ruby{酒類}{しゆ|るゐ}を
\ruby{與}{あた}へ、
%
\ruby{自{\換字{分}}等}{じ|ぶん|ら}もまた
\ruby{聊}{いさゝ}か
\ruby{用}{もち}ゐる。
%
その
\ruby{味}{あぢ}はまた
\ruby{君等}{きみ|ら}の
\ruby{知}{し}らんところだ。
%
\ruby{烈}{はげ}しい
\ruby{怖}{おそ}ろしい
\ruby{風}{かぜ}、
%
\ruby{酷}{むご}い
\ruby{痛}{いた}い
\ruby{雨}{あめ}、
%
\ruby{眞黑}{まつ|くろ}な
\ruby{天}{そら}、
%
\ruby{荒}{あれ}れ
\ruby{立}{た}つ
\ruby{水}{みづ}、
%
\ruby{{\換字{造}}物主}{ざう|ぶつ|しゆ}が
\ruby{其}{そ}の
\ruby{偉大}{ゐ|だい}な
\ruby{働}{はたら}きを
\ruby{見}{み}せる
\ruby{大洋}{たい|やう}の
\ruby{上}{うへ}で、
%
\ruby{木}{き}の
\ruby{葉}{は}にも
\ruby{等}{ひと}しい
\ruby{孤舟}{こ|しふ}に
\ruby{立}{た}つて、
%
たゞ
\ruby{我}{わ}が
\ruby{堅確}{けん|かく}な
\ruby{意志}{い|し}と
\ruby{智識}{ち|しき}の
\ruby{{\換字{判}}斷}{はん|だん}とのみを
\ruby{我}{わ}が
\ruby{味方}{み|かた}にして、
%
あらゆる
\ruby{試}{こゝろ}みに
\ruby{耐}{た}へて
\ruby{奮{\換字{進}}}{ふん|しん}して
\ruby{行}{い}つて、
%
\ruby{{\換字{終}}}{つひ}に
\ruby{其}{そ}の
\ruby{試}{こゝろ}みに
\ruby{打{\換字{勝}}}{うち|か}ち
\ruby{果}{おほ}せた
\ruby{時}{とき}、
%
ラムでもジンでも
\ruby{日本酒}{に|ほん|しゆ}でもの、
%
\ruby{一小杯}{いち|せう|はい}を% 小杯 ... こさかずき/コップ
\ruby{手}{て}にして
\ruby{自}{みづか}ら
\ruby{犒}{ねぎら}ふ
\ruby{其}{そ}の
\ruby{一種}{いつ|しゆ}の
\ruby{言}{い}ふべからざる
\ruby{感}{かん}じは
\ruby{海員}{かい|ゐん}で
\ruby{無}{な}くては
\ruby{解}{わか}らん。
%
\ruby{陸上}{を|か}の
\ruby{料理屋}{れう|り|や}やなんぞで
\ruby{飮}{の}むのとは
\ruby{全然}{まる|で}
\ruby{異}{ちが}ふ
\ruby{味}{あぢ}がする。
%
\ruby{僕}{ぼく}はたゞ
\ruby{其樣}{そ|う}いふ
\ruby{怖}{おそ}ろしい
\ruby{暴風雨}{し||け}の
\ruby{後}{あと}なんぞに、
%
\ruby{濕氣拂}{しつ|け|ばら}ひのため、
%
\ruby{疲勞}{ひ|らう}の
\ruby{回復}{くわい|ふく}のために、% 原本通り「回」
%
\ruby{飮}{の}む
\ruby{時}{とき}ばかりは
\ruby{眞}{しん}に
\ruby{酒}{さけ}を
\ruby{賞}{しやう}するが、
%
\ruby{其}{そ}の
\ruby{他}{た}の
\ruby{時}{とき}に
\ruby{左程}{さ|ほど}
\ruby{好}{この}まん。
%
もう
\ruby{澤山}{たく|さん}だ。
%
\ruby{大{\換字{分}}}{だい|ぶ}
\ruby{醉}{よ}つた。% 「醉」は原本通り「よ」で調整
』

\原本頁{}%
『
\ruby{然樣}{さ|う}
\ruby{固}{かた}くばかりいふな、
%
さあ
\ruby{一盃}{ひと|つ}
\ruby{{\換字{遣}}}{や}る。
%
\ruby{見}{み}ろ、
%
お
\ruby{濱}{はま}さんが
\ruby{眼}{め}を
\ruby{丸}{まる}くして、
%
\ruby{一心}{いつ|しん}に
\ruby{君}{きみ}の
\ruby{暴雨風}{あ|ら|し}の% ここは「暴風雨」でなく「暴雨風」
\ruby{談話}{はな|し}に
\ruby{聞}{き}き
\ruby{惚}{ほ}れて
\ruby{居}{ゐ}る、
%
\ruby{其}{そ}の
\ruby{罪}{つみ}の
\ruby{無}{な}い
\ruby{純潔}{き|れい}な
\ruby{樣子}{やう|す}を
\ruby{見}{み}ろ。
%
\ruby{此}{こ}の
\ruby{人}{ひと}が
\ruby{勸}{すゝ}める
\ruby{酒}{さけ}を
\ruby{飮}{の}まんといふ
\ruby{事}{こと}があるか。
』

\原本頁{}%
\ruby[g]{水野}{みづの}はこゝに
\ruby{至}{いた}つて
\ruby{自}{おのづ}から
\ruby{微笑}{び|せう}を
\ruby{催}{もよほ}し、

\原本頁{}%
『
\ruby{羽{\換字{勝}}}{は|がち}
\ruby{君}{くん}、
%
まあ
\ruby{一}{ひと}つ
\ruby{{\換字{過}}}{すご}して
\ruby{吳}{く}れたまへ。
%
\ruby{魯敏孫}{ろ|びん|そん}
\ruby{漂流記}{へう|りう|き}を
\ruby{讀}{よ}んで
\ruby{非常}{ひ|じやう}に
\ruby{感}{かん}じて、
%
\ruby{魯敏孫}{ろ|びん|そん}と
\ruby{一處}{いつ|しよ}に
\ruby{棲}{す}みたいといつたほどの
\ruby{崇拜者}{すう|はい|しや}となつて
\ruby{居}{ゐ}る、
%
\ruby{航海者好}{かう|かい|しや|ずき}の
\ruby{其}{その}
\ruby{人}{ひと}の
\ruby{御{\換字{酌}}}{お|しやく}だから。
』

\原本頁{}%
と
\ruby{{\換字{前}}}{さき}の
\ruby{夜}{よ}の
\ruby{事}{こと}を
\ruby{思}{おも}ひ
\ruby{起}{おこ}して
\ruby{語}{かた}り
\ruby{出}{い}づれば、

\原本頁{}%
『あら、
%
よくつてよ
\ruby{先生}{せん|せい}、
%
\ruby{餘計}{よ|けい}な
\ruby{事}{こと}を。
』

\原本頁{}%
とお
\ruby{濱}{はま}の
\ruby{打{\換字{消}}}{うち|け}さんとするが
\ruby{如}{ごと}く
\ruby{言}{い}へると
\ruby{同時}{どう|じ}に、
%
\ruby[g]{日方}{ひかた}は
\ruby{笑}{ゑ}ましげに、

\原本頁{}%
『
\ruby{何}{なん}だ、
%
\ruby{魯敏孫}{ろ|びん|そん}の
\ruby{崇拜者}{すう|はい|しや}だ!、
%
こりやあ
\ruby{面白}{おも|しろ}い。
%
\ruby{偉}{えら}い!。
%
\ruby{然樣}{さ|う}
\ruby{來}{こ}なくちやならん、
%
\ruby{其}{それ}で
\ruby{無}{な}くちやいかん。
%
\ruby{實}{じつ}に
\ruby{{\換字{愉}}快}{ゆ|くわい}な
\ruby{人}{ひと}だ、
%
\ruby{頼}{たの}もしい!。
%
\ruby{成程}{なる|ほど}
\ruby[g]{日方}{ひかた}が
\ruby{頭}{あたま}を
\ruby{撲}{は}られたのも
\ruby{無理}{む|り}は
\ruby{無}{な}いは。
%
ハヽヽ、
%
\ruby{君}{きみ}のやうな
\ruby{人}{ひと}になら、
%
もう
\ruby{少々}{せう|〳〵}
\ruby{打撲}{ぶん|なぐ}られても
\ruby{關}{かま}はんは、
%
あゝ
\ruby{面白}{おも|しろ}い。
%
\ruby[g]{水野}{みづの}
\ruby{猪口}{ちよ|く}を
\ruby{與}{よこ}せ、
%
さあ
\ruby{魯敏孫}{ろ|びん|そん}
\ruby{夫人}{ふ|じん}
\ruby{御{\換字{酌}}}{お|しやく}を
\ruby{願}{ねが}ふ。
』

\原本頁{}%
と
\ruby{打興}{うち|きよう}じたり。
%
されど
\ruby{羽{\換字{勝}}}{は|がち}は
\ruby{冷然}{れい|ぜん}として、
%
たゞ
お
\ruby{濱}{はま}をば
\ruby{一瞥}{いち|べつ}せしのみ、
%
\ruby[g]{水野}{みづの}に
\ruby{對}{むか}つて
\ruby{物靜}{もの|しづ}かに、

\原本頁{}%
『
\ruby{海國}{かい|こく}の
\ruby{日本}{に|ほん}の
\ruby{事}{こと}だもの、
%
\ruby{魯敏孫}{ろ|びん|そん}
\ruby{漂流記}{へう|りう|き}に
\ruby{興味}{きよう|み}を
\ruby{感}{かん}ずるやうな
\ruby{女子}{ぢよ|し}の
\ruby{出}{で}て
\ruby{來}{き}て
\ruby{吳}{く}れるのは
\ruby{當然}{たう|ぜん}の
\ruby{事}{こと}だ。
%
\ruby{僕}{ぼく}は
\ruby{此席}{この|せき}にさへ
\ruby{此樣}{こ|う}いふ
\ruby{{\換字{婦}}人}{ふ|じん}を
\ruby{見}{み}る
\ruby{世}{よ}に、
%
まだ
\ruby{海國}{かい|こく}の
\ruby{日本}{に|ほん}の
\ruby{詩}{し}にも
\ruby{小說}{せう|せつ}にも、
%
\ruby{海}{うみ}に
\ruby{關}{くわん}したものゝ
\ruby{甚}{はなは}だ
\ruby{少}{すくな}いのを
\ruby{{\換字{遺}}憾}{ゐ|かん}に
\ruby{思}{おも}ふ。
%
\ruby[g]{水野}{みづの}!。
%
\ruby{今年中}{こ|とし|ちう}には
\ruby[g]{島木}{しまき}の
\ruby{{\換字{船}}}{ふね}を
\ruby{何樣}{ど|う}しても
\ruby{出}{だ}す。
%
\ruby{僕}{ぼく}は
\ruby{無論}{む|ろん}
\ruby{全權}{ぜん|けん}を
\ruby{有}{も}つて
\ruby{出掛}{で|かけ}けるのだ。
%
\ruby{何樣}{ど|う}だ、
%
\ruby{君}{きみ}
\ruby{一}{ひと}つ
\ruby{奮發}{ふん|ぱつ}して
\ruby{海上}{かい|じやう}に
\ruby{出}{で}んか。
%
\ruby{决}{けつ}して
\ruby{危險}{き|けん}なんぞは
\ruby{有}{あ}るもので
\ruby{無}{な}い。
%
\ruby{好}{い}い
\ruby{機會}{き|くわい}だ、
%
\ruby{大洋}{たい|やう}の
\ruby{美觀壯觀}{び|くわん|さう|くわん}を
\ruby{君}{きみ}の
\ruby{眼}{め}に
\ruby{入}{い}れんか。
%
\ruby{茫々}{ばう|〳〵}たる
\ruby{大洋}{たい|やう}の
\ruby{大}{おほき}な
\ruby{景氣}{け|しき}の
\ruby{中}{なか}へ
\ruby{出}{で}て、
%
\ruby{人間}{にん|げん}の
\ruby{{\換字{紛}}々}{ふん|ぷん}たる
\ruby{葛藤}{かつ|とう}を
\ruby{{\換字{逃}}}{のが}れて、
%
\ruby{直接}{ちよく|せつ}に
\ruby{{\換字{造}}化}{ざう|くわ}の
\ruby{懷中}{ふと|ころ}に
\ruby{寢}{ね}て
\ruby{見}{み}んか
\ruby[g]{水野}{みづの}。
%
たしかに
\ruby{君}{きみ}の
\ruby{知}{し}らん
\ruby{心持}{こゝろ|もち}が
\ruby{爲}{し}やうぜ。
』

\原本頁{}%
と
\ruby{豫}{かね}て
\ruby{考}{かんが}へ
\ruby{來}{きた}りしことにやあらん、
%
\ruby{思}{おも}ひのほかなる
\ruby{點}{てん}を
\ruby{沈着}{おち|つ}いて
\ruby{云}{い}ひ
\ruby{出}{だ}しぬ。

\Entry{其五十一}

\ruby{水野}{みづ|の}の
\ruby{答}{こた}へに
\ruby{答}{こた}へかぬる
\ruby{時}{とき}、
\ruby{羽{\換字{勝}}}{は|がち}はふたゝび
\ruby{言葉}{こと|ば}をつぎて、

『
\ruby{實}{じつ}は
\ruby{{\換字{遠}}洋}{ゑん|やう}へ
\ruby{出}{で}る
\ruby{漁{\換字{船}}}{ぎよ|せん}などでは、
\ruby{便乘者}{びん|じよう|しや}を
\ruby{特}{こと}のほかに
\ruby{{\換字{迷}}惑}{めい|わく}がるのだ。
しかし
\ruby{君}{きみ}
が
\ruby{好}{この}むならば
\ruby{僕}{ぼく}は
\ruby{勸}{すゝ}めても
\ruby{乘}{の}せたい。
\ruby{君}{きみ}を
\ruby{大洋}{たい|やう}の
\ruby{中}{なか}へ
\ruby{引出}{ひき|だ}したい。
いろ〳〵の
\ruby{人爲}{じん|ゐ}の
\ruby{複雜}{ふく|ざつ}な
\ruby{組織}{そ|しき}で、
\ruby{自然}{し|ぜん}の
\ruby{眞趣}{おも|むき}を
\ruby{蔽}{おほ}ひ
\ruby{盡}{つく}してゐる
\ruby{陸上}{りく|じやう}から
\ruby{君}{きみ}を
\ruby{離}{はな}れさせたい。
\ruby{直接}{たゞ|ち}に
\ruby{自然}{し|ぜん}の
\ruby{{\換字{前}}}{まへ}に
\ruby{出}{で}て
\ruby{貰}{もら}ひたい。
\ruby{直接}{たゞ|ち}に
\ruby{自然}{し|ぜん}の
\ruby{詩卷}{し|くわん}を
\ruby{讀}{よ}んで
\ruby{見}{み}て
\ruby{貰}{もら}ひたい。
\ruby{僕}{ぼく}はよくは
\ruby{詩}{し}を
\ruby{知}{し}らん。
しかし
\ruby{僕}{ぼく}が
\ruby{知}{し}つて
\ruby{居}{ゐ}る
\ruby{自然}{し|ぜん}は、
\ruby{僕}{ぼく}の
\ruby{知}{し}つて
\ruby{居}{ゐ}る
\ruby{一切}{いつ|さい}の
\ruby{詩}{し}とは
\ruby{甚}{はなは}だ
\ruby{{\換字{遠}}}{とほ}いものだ。
\ruby{僕}{ぼく}は
\ruby{自然}{し|ぜん}の
\ruby{或者}{ある|もの}を
\ruby{解}{かい}して
\ruby{居}{ゐ}る
\ruby{點}{てん}に
\ruby{於}{おい}て
\ruby{詩人}{し|じん}に
\ruby{{\換字{勝}}}{まさ}つて
\ruby{居}{ゐ}るとは
\ruby{信}{しん}せぬ。
たゞし
\ruby{海上}{かい|じやう}に
\ruby{關}{かわん}する
\ruby{詩}{し}の
\ruby{甚}{はなは}だ
\ruby{淺薄}{せん|ぱく}なのは
\ruby{感}{かん}じて
\ruby{居}{ゐ}る。
\ruby{若}{も}し
\ruby{詩想}{し|さう}のある
\ruby{人}{ひと}が
\ruby{大洋}{たい|やう}に
\ruby{{\換字{浮}}}{うか}んで、
\ruby{自然}{し|ぜん}の
\ruby{廣大}{くわう|だい}な
\ruby{背景}{はい|けい}の
\ruby{{\換字{前}}}{まへ}で、
\ruby{人間}{にん|げん}の
\ruby{自}{おのづ}から
\ruby{抱}{いだ}く
\ruby{感}{かん}じを
\ruby{味}{あぢは}つたら、
\ruby{在來}{ざい|らい}の
\ruby{詩}{し}のやうなものばかりは
\ruby{出來}{で|き}て
\ruby{居}{ゐ}まいと
\ruby{思}{おも}ふ。
まあ
\ruby{想}{おも}つても
\ruby{見}{み}たまへ。
\ruby{彼方}{あつ|ち}から
\ruby{此方}{こつ|ち}へ
\ruby{歸}{かへ}る
\ruby{路}{みち}の、
\ruby{太{\換字{平}}洋}{たい|へい|やう}の
\ruby{眞中}{まん|なか}あたりで、
\ruby{僕}{ぼく}がたゞ
\ruby{一人}{ひと|り}
\ruby{舷頭}{げん|とう}に
\ruby{立}{た}つて
\ruby{居}{ゐ}たことがある。
\ruby{丁度月}{ちやう|ど|つき}は
\ruby{眞珠}{しん|じゆ}を
\ruby{溶}{と}かしたやうな
\ruby{光}{ひかり}を
\ruby{投}{な}げて
\ruby{一切}{いつ|さい}を
\ruby{包}{つゝ}んで
\ruby{居}{ゐ}る。
\ruby{其}{そ}の
\ruby{中}{なか}を
\ruby{走}{はし}つて
\ruby{居}{ゐ}る
\ruby{自{\換字{分}}}{じ|ぶん}の
\ruby{{\換字{船}}}{ふね}は
\ruby{何處}{ど|こ}へ
\ruby{行}{ゆ}くのだらう。

\ruby{行}{ゆ}く
\ruby{先}{さき}も
\ruby{見}{み}えん、
\ruby{來}{き}たところも
\ruby{見}{み}えん。
たゞ
\ruby{淡}{あは}い
\ruby{光}{ひかり}の
\ruby{滿}{み}ちて
\ruby{居}{ゐ}る
\ruby{天水}{そら|みづ}の
\ruby{中}{なか}を
\ruby{歩}{ある}いて
\ruby{居}{ゐ}る。
\ruby{海}{うみ}は
\ruby{絹毛氈}{きぬ|まう|せん}のやうに
\ruby{滑}{なめ}らかで
\ruby{美}{うつく}しく
\ruby{廣}{ひろ}がつて
\ruby{居}{ゐ}る。
\ruby{柔}{やはら}かい〳〵しかも
\ruby{心}{こゝろ}の
\ruby{正}{たゞ}しい
\ruby{貿易風}{ぼう|\換字{江}き|ふう}は、
\ruby{恩愛}{おん|あい}の
\ruby{溢}{あふ}るゝばかりの
\ruby{慈母}{は|ゝ}の
\ruby{手}{て}から
\ruby{出}{で}る
\ruby{團{\換字{扇}}}{うち|は}の
\ruby{風}{かぜ}が、
\ruby{睡}{ね}て
\ruby{居}{ゐ}る
\ruby{嬰兒}{あか|ご}の
\ruby{顏}{かほ}へ
\ruby{當}{あた}るやうに、そより〳〵と
\ruby{後}{うしろ}から
\ruby{吹}{ふ}いて
\ruby{居}{ゐ}る。
\ruby{帆}{ほ}は
\ruby{一}{いつ}ばいに
\ruby{張}{は}られたまゝでパタリとも
\ruby{動}{うご}かぬ。
\ruby{休番}{やす|み}のものは
\ruby{皆熟睡}{みな|じゆく|すい}して
\ruby{居}{ゐ}る。
\ruby{當番}{たう|ばん}のものも、こくり〳〵と
\ruby{{\換字{遣}}}{や}つて
\ruby{居}{ゐ}る。
\ruby{一切}{いつ|さい}の
\ruby{用事}{よう|じ}は
\ruby{皆忘}{みな|わす}れられて
\ruby{居}{ゐ}て、
\ruby{胸}{むね}の
\ruby{中}{なか}にも
\ruby{頭}{あたま}の
\ruby{中}{なか}にも
\ruby{何}{なんに}も
\ruby{無}{な}い。
\ruby{何一}{なに|ひと}つ
\ruby{耳}{みゝ}に
\ruby{立}{た}つ
\ruby{音}{おと}も
\ruby{爲無}{し|な}い。
\ruby{何}{なに}も
\ruby{見}{み}えん
\ruby{天}{そら}と
\ruby{水}{みづ}との
\ruby{間}{あひだ}を
\ruby{茫然}{ぼ|つ}として
\ruby{見}{み}て
\ruby{居}{ゐ}ると、
\ruby{何時}{い|つ}かもう
\ruby{自{\換字{分}}}{じ|ぶん}の
\ruby{身體}{から|だ}も
\ruby{{\換字{消}}}{き}えて
\ruby{仕舞}{し|ま}つて、
\ruby{矢張}{や|はり}
\ruby{眞珠}{しん|じゆ}の
\ruby{溶}{と}けたやうな
\ruby{月}{つき}の
\ruby{光}{ひかり}と
\ruby{一緖}{いつ|しよ}になつて、
\ruby{大空}{おほ|ぞら}の
\ruby{中}{なか}に
\ruby{流}{なが}れ
\ruby{瀰}{わた}つて
\ruby{居}{ゐ}るやうな
\ruby{氣}{き}がする。
\ruby{左樣}{さ|う}いふ
\ruby{心持}{こゝろ|もち}の
\ruby{仕}{し}たことがある。
\ruby{其}{そ}の
\ruby{時}{とき}の
\ruby{僕}{ぼく}の
\ruby{心}{こゝろ}の
\ruby{中}{なか}の
\ruby{味}{あぢはい}とふものは、とても
\ruby{僕}{ぼく}の
\ruby{口}{くち}では
\ruby{云}{い}ふ
\ruby{事}{こと}が
\ruby{出來}{で|き}んが、あゝ
\ruby{若}{も}し
\ruby{自{\換字{分}}}{じ|ぶん}が
\ruby{水野}{みづ|の}であつたらば、
\ruby{屹度}{きつ|と}
\ruby{此}{こ}の
\ruby{美}{うるは}しい
\ruby{何}{なん}とも
\ruby{云}{い}へぬ
\ruby{感}{かん}じを、
\ruby{{\換字{文}}字}{もん|じ}に
\ruby{現}{あらは}して
\ruby{人}{ひと}に
\ruby{示}{しめ}す
\ruby{事}{こと}が
\ruby{出來}{で|き}るであらうものをと、
\ruby{深}{ふか}く
\ruby{其}{そ}の
\ruby{時}{とき}に
\ruby{僕}{ぼく}は
\ruby{思}{おも}つた。
\ruby{何樣}{ど|う}だ
\ruby{君一}{きみ|ひと}つ
\ruby{海上}{かい|じやう}に
\ruby{出}{で}て
\ruby{自然}{し|ぜん}が
\ruby{君}{きみ}に
\ruby{何}{なに}を
\ruby{與}{あた}へるかを
\ruby{試}{こゝろ}みては
\ruby{見}{み}ないか。
\ruby{必}{かな}らず
\ruby{君}{きみ}を
\ruby{{\換字{益}}}{\換字{江}き}する
\ruby{事}{こと}は
\ruby{少}{すくな}く
\ruby{無}{な}からう。
\ruby{凪}{なぎ}は
\ruby{凪}{なぎ}で
\ruby{面白}{おも|しろ}い、
\ruby{暴風雨}{あ|ら|し}は
\ruby{暴風雨}{あ|ら|し}で
\ruby{面白}{おも|しろ}い。
\ruby{海上}{かい|じやう}の
\ruby{生活}{せい|くわつ}も
\ruby{{\換字{半}}歳位}{はん|とし|ぐらゐ}は
\ruby{宜}{よ}からう。
\ruby{小}{ちひさ}な
\ruby{屋根}{や|ね}の
\ruby{下}{した}から
\ruby{飛}{と}び
\ruby{出}{だ}して
\ruby{見}{み}ないか。
\ruby{大熊星}{たい|ゆう|せい}の
\ruby{光}{ひかり}は
\ruby{北}{きた}で
\ruby{待}{ま}つて
\ruby{居}{ゐ}る、
\ruby{十字星}{じう|じ|せい}の
\ruby{光}{ひかり}は
\ruby{南}{みなみ}で
\ruby{莞爾}{に|こ}ついて
\ruby{居}{ゐ}る。
\ruby{大}{おほき}い〳〵
\ruby{此}{こ}の
\ruby{天地}{てん|ち}では
\ruby{無}{な}いか。
\ruby{米粒}{こめ|つぶ}に
\ruby{{\換字{文}}字}{も|じ}を
\ruby{書}{か}くやうに、
\ruby{細}{こまか}い
\ruby{事}{こと}ばかり
\ruby{考}{かんが}へ
\ruby{込}{こ}まずとも、
\ruby{其}{そ}の
\ruby{米粒}{こめ|つぶ}は
\ruby{姑}{しばら}く
\ruby{傍}{わき}へ
\ruby{置}{お}いて、
\ruby{自然}{し|ぜん}の
\ruby{大}{おほき}な
\ruby{景色}{け|しき}に
\ruby{親}{した}しんで
\ruby{見}{み}ないか。
\ruby{何樣}{ど|う}だい
\ruby{水野}{みづ|の}、
\ruby{何}{なん}と
\ruby{思}{おも}ふ?。
\ruby{君}{きみ}が
\ruby{{\換字{嫌}}}{いや}なら
\ruby{仕方}{し|かた}は
\ruby{無}{な}いが、
\ruby{學校}{がく|かう}の
\ruby{敎師}{けう|し}も
\ruby{既}{もう}よからう、
\ruby{一}{ひと}つ
\ruby{{\換字{遊}}}{あそ}んで
\ruby{見}{み}ては
\ruby{何樣}{ど|ん}なものだ?。
』

と、
\ruby{勉}{つと}めて
\ruby{水野}{みづ|の}の
\ruby{意}{こゝろ}を
\ruby{動}{うご}かさんと
\ruby{心長}{こゝろ|なが}く
\ruby{{\換字{説}}}{と}きたるは、
\ruby{全}{まつた}く
\ruby{心搆}{こゝろ|がま}へして
\ruby{來}{きた}りしなるべし。

\ruby{羽{\換字{勝}}}{は|がち}の
\ruby{意}{こゝろ}の
\ruby{解}{げ}
せぬ
\ruby{水野}{みづ|の}ならねば、
\ruby{少}{すくな}からず
\ruby{其}{そ}の
\ruby{話}{ばなし}に
\ruby{{\換字{情}}}{こゝろ}を
\ruby{動}{うご}かして、まことに
\ruby{趣味多}{おも|むき|おほ}かるべき
\ruby{海上}{かい|ぢやう}の
\ruby{生活}{せい|くわつ}を
\ruby{試}{こゝろ}みたきやうの
\ruby{念}{おもひ}も
\ruby{起}{おこ}る
\ruby{傍}{かたはら}、
\ruby{羽{\換字{勝}}}{は|がち}が
\ruby{我}{わ}がために
\ruby{思}{おもひ}を
\ruby{費}{つひや}して、かゝる
\ruby{事}{こと}を
\ruby{勸}{すゝ}めくるゝ
\ruby{其意}{その|い}を
\ruby{感}{かん}じて、
\ruby{嬉}{うれ}しとも
\ruby{忝}{かたじけな}しとも
\ruby{胸}{むね}の
\ruby{中}{うち}には、
\ruby{幾度}{いく|たび}か
\ruby{感謝}{かん|しや}してまた
\ruby{感謝}{かん|しや}しぬ。

されど
\ruby{水野}{みづ|の}は
\ruby{今}{いま}こゝに
\ruby{其言}{その|ことば}に
\ruby{隨}{したが}はんとも
\ruby{云}{い}ひ
\ruby{{\換字{兼}}}{か}ねて、
\ruby{何}{なに}と
\ruby{應}{こた}へんと
\ruby{思}{おも}ひめぐらすを、
\ruby{見}{み}
\ruby{取}{と}りて
\ruby{羽{\換字{勝}}}{は|がち}は
\ruby{言葉}{こと|ば}
\ruby{{\換字{緩}}}{ゆる}く、

『
\ruby{何}{なに}も
\ruby{今}{いま}
\ruby{君}{きみ}の
\ruby{{\換字{返}}辭}{へん|じ}を
\ruby{求}{もと}めるのでは
\ruby{無}{な}い。
\ruby{{\換字{船}}}{ふね}は
\ruby{凡}{およ}そ
\ruby{十二月}{じう|にぐ|わつ}に
\ruby{出}{だ}す
\ruby{心算}{つも|り}なのだから、それまでは
\ruby{間}{あひだ}もある、ゆつくり
\ruby{考}{かんが}へたまへ。
\ruby{若}{も}し
\ruby{其}{それ}までに
\ruby{何樣}{ど|ん}な
\ruby{事}{こと}でもあつて、
\ruby{海}{うみ}へ
\ruby{出}{で}たいと
\ruby{思}{おも}ふやうな
\ruby{事}{こと}でもあつたら、いつでも
\ruby{相談}{さう|だん}に
\ruby{乘}{の}る、
\ruby{{\換字{悅}}}{よろこ}んで
\ruby{應}{おう}じる。
\ruby{大洋}{たい|やう}を
\ruby{見}{み}るのも
\ruby{宜}{よ}からうと
\ruby{思}{おも}ふよ。
』

と
\ruby{少}{すこ}しも
\ruby{無理{\換字{強}}}{む|り|じひ}の
\ruby{氣味無}{き|み|な}く
\ruby{云}{い}へば、

『
\ruby{賛成}{さん|せい}だ、
\ruby{大賛成}{だい|さん|せい}だ。
\ruby{大洋生活}{たい|やう|せい|くわつ}を
\ruby{{\換字{遣}}}{や}つて
\ruby{見}{み}ろ、
\ruby{水野}{みづ|の}。
\ruby{女}{をんな}の
\ruby{傍}{そば}なんぞにへばり
\ruby{着}{つ}いて
\ruby{居}{ゐ}ないで、
\ruby{飛}{と}び
\ruby{出}{だ}せ、
\ruby{羽{\換字{勝}}}{は|がち}と
\ruby{一緖}{いつ|しよ}に
\ruby{行}{ゆ}け、
お
\ruby{濱}{はま}さんでさへ
\ruby{魯敏孫}{ろ|びん|そん}と
\ruby{同棲}{どう|せい}しやうといふ
\ruby{氣槪}{き|がい}が
\ruby{有}{あ}るぢや
\ruby{無}{な}いか。
』

と
\ruby{日方}{ひ|かた}は
\ruby{却}{かへ}つて
\ruby{{\換字{強}}}{し}ひ
\ruby{立}{た}てたり。

\makeatletter
\@ifundefined{全三巻@一括ビルド}{%
\vspace{2zw}
\Large{天うつ浪 {\normalsize 第二{\換字{終}}}}
}
\makeatother

\end{indentation}
\end{document}
