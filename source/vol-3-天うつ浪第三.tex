% +++
% sequence = ["cluttex"]
% [programs.cluttex]
% command  = "cluttex"
% opts     = [ "--engine = uplatex", "--shell-escape", "--output-directory = myout" ]
% +++

% https://okumuralab.org/tex/mod/forum/discuss.php?d=3551 を参考に下記
% を上記のように変更
%#! cluttex --engine=uplatex --shell-escape --output-directory=myout
%   cluttex でビルド時に必要なディレクトリ myout 、 mygwi と myout/mygwi
%   また --includeonly=NAMEs を指定すると '\includeonly{NAMEs}' を仮挿入してくれる

\RequirePackage{plautopatch}
%\RequirePackage{exppl2e}% 警告メッセージ削減のためコメントアウト
\documentclass[
uplatex                                     ,% upLaTeX文書
dvipdfmx                                    ,%
book                                        ,%
tate                                        ,%
twoside                                     ,% even_running_head 有効化
paper                       = a5paper       ,%
open_bracket_pos            = nibu_tentsuki ,% 組み方 段落開始二分折り返し行頭天付き
hanging_punctuation                         ,% 組み方 ぶら下げ組
openany                                     ,%
jafontsize                  = 12pt          ,% 13pt 指定すると LaTeX Font Warning が表示される
%%%%%%%%%%%%%%%%%%%%%%%%%%%%%%%%%%%%%%%%%%%%%% 分冊版では以下は default 設定
% head_space                = 36mm          ,% 天の空き量
% gutter                    = 18mm          ,% のどの余白の大きさ
% headfoot_verticalposition = 24mm          ,%
% line_length               = 27zw          ,% 原本と比較するとき(自動で 30zw 位)
% number_of_lines           = 11            ,% 原本と比較するとき(自動で 15行)
]{jlreq}

\usepackage{bxpapersize}
\usepackage{pxrubrica}
\usepackage{sfkanbun}
\usepackage[deluxe,multi,jis2004]{otf}
\usepackage[directunicode*, noalphabet]{pxchfon}[2017/04/08]
\usepackage{plext}
\usepackage{graphicx}
\usepackage[dir=mygwi,cache=108000]{bxglyphwiki}
\usepackage{indent}
\usepackage{CJK-char-convert}
\rubysetup{||h>}% 無指定時のルビ:(||)前進入禁止、(h)肩付き、(>)後進入大
% \rubysetup{|h>}% 無指定時のルビ:(|)前進入禁止、(h)肩付き、(>)後進入大
\title{\Huge 天うつ浪 {\Large 第三}}
\author{幸田露伴}
\date{         {\small 明治四十年一月} 春陽{\換字{堂}}}
%================================
\makeatletter
% \def\全三巻@一括ビルド{}
% \def\デバッグ@ビルド{}
\@ifundefined{デバッグ@ビルド}{%
  \newcommand{\原本頁}[1]{}% デバッグ用/本番は「空」
  \newcommand{\改行}{}%%%%%% デバッグ用/本番は「空」
  \newcommand{\会話開始}{}%% デバッグ用/本番は「空」
}{%
  \newcommand{\原本頁}[1]{\marginpar{\hfill p-#1}}%%%%% デバッグ用/本番は「空」
  \setlength{\marginparwidth}{20mm}%% 傍注欄の大きさ%%% デバッグ用/本番はコメントアウト
  \newcommand{\改行}{\par}%%%%%%%%%%%%%%%%%%%%%%%%%%%%% デバッグ用/本番は「空」
  \newcommand{\会話開始}{ }%
  %
  % 背景にグリッドを表示させる
  \plautopatchdisable{eso-pic}% https://okumuralab.org/tex/mod/forum/discuss.php?d=2956
  % \usepackage{xcolor} は eso-pic.sty 内部で呼び出している
  \usepackage[
  colorgrid    = true    ,
  gridBG       = true    ,
  gridunit     = mm      , % mm, in, bp, pt
  gridcolor    = red!25  ,
  subgridcolor = lime!75 ,
  texcoord     = true    ,
  ]{eso-pic}
}

\makeatother

\def\footnote#1{\endnote{#1}}
\jlreqsetup{
   endnote_position      = {_chapter} , %後注の表示位置
   endnote_second_indent = {3zw}      ,
   mainmatter_pagebreak  = clearpage  ,
 }

\begin{document}
\maketitle
\pagestyle{myheadings}
\newcommand{\Entry}[1]{
	\section*{#1}
	\markboth{#1}{#1}
	\setcounter{equation}{0}}
\begin{indentation}{4zw}{3zw}
\parindent=0pt

\chapter*{}
% \newpage
% \ % 全角空白
% \newpage

{\huge
\ruby{天}{そら} う つ %空白有り
\ruby{浪}{なみ}}  {\normalsize 第三}
\vspace*{4zw}
\Entry{其一}

\ruby{親}{した}しきが
\ruby{中}{なか}の
\ruby{{\GWI{u7d55-k}}}{た}えて
\ruby{久}{ひさ}しくして
\ruby{相會}{あひ|あ}ひたるに、
\ruby[g]{痛{\GWI{hkcs_m98f2}}快談}{つういんくわいだん}して
\ruby{歸}{かへ}るを
\ruby{忘}{わす}るヽ
\ruby[g]{日方}{ひかた}を、
\ruby[g]{幾度}{いくたび}か
\ruby[g]{{\GWI{u7fbd-k}\換字{勝}}}{はがち}の
\ruby{促}{うなが}し
\ruby{立}{た}てヽ、
\ruby{漸}{やうや}くに
\ruby[g]{二人}{ふたり}の
\ruby{暇}{いとま}を
\ruby{告}{つ}げし
\ruby{時}{とき}は、
\ruby{日}{ひ}は
\ruby{{\GWI{u65e3-k}}}{すで}に
\ruby{暮}{く}れ
\ruby{果}{は}てヽ一
\ruby[g]{時間餘}{じかんよ}も
\ruby{經}{へ}たり。

お
\ruby{濱}{はま}お
\ruby{鍋}{なべ}は
\ruby{後片付}{あと|かた|づけ}に
\ruby{忙}{せは}しく、
\ruby[g]{水野}{みづの}は
\ruby{獨}{ひと}り
\ruby[g]{机}{つくゑ}に
\ruby{憑}{よ}つて
\ruby{醉}{よひ}を
\ruby{吐}{は}きつつ、
\ruby{{\GWI{hkcs_m98f2}}}{の}み
\ruby{慣}{な}れぬ
\ruby{酒}{さけ}に
\ruby{聊}{いさヽ}か
\ruby{苦}{くるし}みて、
\ruby{頻}{しきり}に
\ruby{微溫}{ぬ|る}き
\ruby{茶}{ちや}に
\ruby{渴}{かわき}を
\ruby{癒}{いや}しながら、
\ruby[g]{{\GWI{u7fbd-k}\換字{勝}}}{はがち}が
\ruby{言}{い}ひたる
\ruby{海上}{かい|じやう}の
\ruby{生活}{せい|くわつ}の
\ruby{如何}{い|か}に
\ruby[g]{趣味}{おもむき}あるべきかを
\ruby{想}{おも}ひ
\ruby{{\GWI{u9063-k}}}{や}り、
\ruby{或}{あるひ}は
\ruby[g]{{\換字{又}}飜}{またひるがへ}つて
\ruby[g]{日方}{ひかた}が
\ruby{我}{われ}を
\ruby{撲}{う}ちたる
\ruby{時}{とき}の
\ruby{勢}{いきほひ}の
\ruby{烈}{はげ}しかりしことなどを思ひ
\ruby{{\換字{廻}}}{めぐ}らす
\ruby{折}{をり}しも、
\ruby[g]{日方}{ひかた}が
\ruby{引}{ひ}き
\ruby{出}{いだ}し
\ruby{散}{ち}らしたる
\ruby{我雜記}{わが|ざつ|き}は
\ruby{我}{わ}が
\ruby{膝{\GWI{u8fd1-k}}}{ひざ|ちか}くありて、
\ruby{其}{そ}の
\ruby{裏面}{う|ら}に
\ruby{我}{わ}が
\ruby{落書}{らく|がき}きしたる
\ruby{萬葉}{まん|えふ}の
\ruby{幾首}{いく|しゆ}の
\ruby{歌}{うた}の、
\ruby{横}{よこ}に、
\ruby{縦}{たて}に、
\ruby{{\GWI{u9006-k}}}{さか}さまになりて
\ruby{我}{われ}を
\ruby{慰}{なぐさ}むるが
\ruby{如}{ごと}きが
\ruby{偶然}{ふ|と}
\ruby{眼}{め}に
\ruby{入}{い}りたり。

\ruby{唐詩}{たう|し}は
\ruby{好}{この}みて
\ruby{誦}{しよう}すれども
\ruby{和歌}{わ|か}には
\ruby{疎}{うと}き
\ruby[g]{日方}{ひかた}の、いづれも
\ruby{此}{これ}は
\ruby{古}{ふる}き
\ruby{歌}{うた}なるを
\ruby{知}{し}らで、
\ruby{我}{わ}が
\ruby{詠}{えい}じたるものヽやうに
\ruby{思}{おも}ひ
\ruby{{\GWI{u8fbc-k}}}{こ}みて
\ruby{我}{われ}を
\ruby{罵}{のヽし}りしが、
\ruby{言}{い}ひ
\ruby{解}{と}かんも
\ruby{煩}{うるさ}ければ
\ruby{其儘}{その|まヽ}に
\ruby{寃罪}{つ|み}を
\ruby{負}{お}いたる、
\ruby{其事}{そ|れ}も
\ruby{思}{おも}へば、
\ruby{何}{なに}か
\ruby{厭}{いと}はしかるべきや、
\ruby{歌}{うた}は
\ruby{皆他}{みな|ひと}の
\ruby{歌}{うた}ながら、
\ruby{詠}{よ}まれたる
\ruby{思}{おもひ}は
\ruby{我}{わ}が
\ruby{思}{おもひ}なるをと、
\ruby{凝然}{じ|つ}と
\ruby{見入}{み|い}りつヽ、
\ruby{文字}{も|じ}を
\ruby{辿}{たど}りて、
\ruby{久方}{ひさ|かた}の
\ruby{天}{あま}つみ
\ruby{空}{そら}に
\ruby{照}{て}れる
\ruby{日}{ひ}の
\ruby{亡}{う}せなん
\ruby{日}{ひ}こそ
\ruby{我}{わ}が
\ruby{戀止}{こひ|や}まめ、と
\ruby{心}{こヽろ}の
\ruby{中}{うち}に
\ruby{自}{みづか}ら
\ruby{讀}{よ}みたり。

\ruby{醉}{よひ}に
\ruby{我}{わ}が
\ruby{心}{こころ}は
\ruby{蒸}{む}さるヽが
\ruby{如}{ごと}くにして、
\ruby{身}{み}の
\ruby{筋}{すぢ}は
\ruby{弛}{ゆる}み
\ruby{骨{\GWI{u7bc0-ue0102}}}{ほ|ね}は
\ruby{和}{やはら}いで
\ruby{快}{こヽろよ}く
\ruby{懈}{だる}きやうなるに、
\ruby{{\GWI{u7cbe-k}}神}{たま|しひ}は
\ruby{何}{なに}にか
\ruby{憧}{あくが}るヽ、
\ruby{空}{あだ}に
\ruby{{\GWI{u6d6e-k}}}{う}きて
\ruby{止}{や}まず、たヾ〳〵
\ruby{我}{われ}を
\ruby{笑}{ゑ}ますに
\ruby{足}{た}るものを
\ruby{得}{え}て、
\ruby{面白}{おも|しろ}く
\ruby{破顏}{は|がん}して
\ruby{笑}{ゑ}みたきやうの
\ruby{氣}{き}のする
\ruby{水野}{みづ|の}は、
\ruby{明}{あき}らかに
\ruby{此}{これ}を
\ruby{酒}{さけ}のさする
\ruby{事}{わざ}と
\ruby{知}{し}りながら、
\ruby{{\GWI{u7336-k}}我}{なほ|わ}が
\ruby{心}{こヽろ}の
\ruby[g]{自然}{おのづ}と
\ruby{動}{うご}くに
\ruby{任}{まか}せて、
\ruby{何}{なに}とせん
\ruby[g]{念慮}{おもひ}も
\ruby{無}{な}く
\ruby{恍然}{うつ|とり}となり
\ruby{居}{ゐ}たり。

\ruby{珍}{めづ}らしくも
\ruby{水野}{みづ|の}の
\ruby{表}{おもて}は
\ruby{{\GWI{u6696-k}}}{あたヽか}かげに
\ruby[g]{微紅色}{うすくれなゐ}に、
\ruby[g]{其眼}{そのめ}は
\ruby{優}{やさ}しき
\ruby{光}{ひかり}を
\ruby{湛}{たヽ}へたれど、
\ruby{例}{いつも}の
\ruby{癖}{くせ}の
\ruby{物思}{もの|おもひ}に
\ruby{耽}{ふけ}れるかと
\ruby{見}{み}えて
\ruby{身動}{み|うご}きもせざるに、
\ruby[g]{此方}{こなた}に
\ruby{入}{い}り
\ruby{來}{きた}れる
\ruby[g]{吉右衛門}{きちゑもん}は、

『
\ruby[g]{御酒}{ごしゆ}の
\ruby{後}{あと}ですから
\ruby{御考}{おか|んが}へ
\ruby{事}{ごと}は
\ruby{毒}{どく}です。
\ruby[g]{些御話}{ちつとおはなし}でもなさいませんか。
\ruby{日方}{ひ|かた}さんと
\ruby{仰}{おつし}ある
\ruby{方}{かた}は
\ruby{結構}{けつ|こう}な
\ruby{方}{かた}ですが、
\ruby{軍人}{ぐん|じん}で
\ruby{在}{い}らつしやるだけに
\ruby{荒}{あら}い
\ruby{方}{かた}ですネ。
\GWI{u1b048}かし
\ruby[g]{{\GWI{u7fbd-k}\換字{勝}}}{はがち}さんと
\ruby{仰}{おつし}ある
\ruby{方}{かた}でも
\ruby{彼}{か}の
\ruby{方}{かた}でも、
\ruby[g]{皆心底}{みんなしんそこ}から
\ruby[g]{貴下}{あなた}を
\ruby{思}{おも}つて
\ruby{居}{ゐ}らつしゃる、
\ruby[g]{眞實}{ほんと}に
\ruby{結構}{けつ|こう}な
\ruby{好}{よ}い
\ruby{方々}{かた|〴〵}です。
\ruby{御氣}{お|き}に
\ruby{入}{い}らない
\ruby{事}{こと}も
\ruby{仰}{おつし}あつてヾしやうが、
\ruby{何}{なに}も
\ruby{彼}{か}も
\ruby[g]{皆御親切}{みんなごしんせつ}から
\ruby{出}{で}た
\ruby{事}{こと}ですから、
\ruby{御氣}{お|き}に
\ruby{御止}{お|と}めなすつて
\ruby{惡}{わる}くなんぞ
\ruby{御考}{おか|んが}へなさらないが
\ruby{宜}{よ}うございます。
』

と
\ruby{言}{い}ひたり。

\ruby[g]{吉右衛門}{きちゑもん}は
\ruby[g]{水野}{みづの}が
\ruby{身動}{み|うご}きもせで
\ruby{物}{もの}を
\ruby{思}{おも}へるを、
\ruby{胸}{むね}の
\ruby{中}{うち}に
\ruby[g]{{\GWI{u7fbd-k}\換字{勝}}}{はがち}
\ruby[g]{日方}{ひかた}が
\ruby[g]{振舞言語}{ふるまひものいひ}を
\ruby{忘}{わす}れ
\ruby{{\換字{兼}}}{か}ねて
\ruby{繰}{く}り
\ruby{{\GWI{u8fd4-k}}}{かへ}し
\ruby{繰}{く}り
\ruby{{\GWI{u8fd4-k}}}{かへ}せると
\ruby{猜}{すゐ}したるなるが、かく
\ruby{云}{い}はれて
\ruby[g]{水野}{みづの}は
\ruby{我}{われ}に
\ruby{復}{かへ}りてハツと
\ruby{驚}{おどろ}きぬ。
\ruby{實}{げ}に
\ruby{我}{われ}は
\ruby{今}{いま}
\ruby[g]{此老人}{このとしより}が
\ruby{言}{い}へるが
\ruby{如}{ごと}くに、
\ruby[g]{{\GWI{u7fbd-k}\換字{勝}}}{はがち}
\ruby[g]{日方}{ひかた}の
\ruby{我}{われ}に
\ruby{與}{あた}へたる
\ruby{數々}{かず|〴〵}の
\ruby[g]{言葉}{ことば}に
\ruby{就}{つ}いて
\ruby{物}{もの}をこそ
\ruby{思}{おも}ふべき
\ruby{筈}{はず}なるに、
\ruby{我}{われ}は
\ruby{今}{いま}
\ruby{抑何}{そも|なに}をか
\ruby{思}{おも}ひ
\ruby{居}{ゐ}し。

\ruby[g]{{\GWI{u7fbd-k}\換字{勝}}}{はがち}が
\ruby{言}{い}ひし
\ruby{海}{うみ}の
\ruby{上}{うへ}の
\ruby{生活}{せい|くわつ}に
\ruby{就}{つ}いて
\ruby{歟}{か}。
あらず、
\ruby{海}{うみ}の
\ruby{上}{うへ}などの
\ruby{事}{こと}は
\ruby{{\GWI{u65e3-jv}}}{すで}に
\ruby{思}{おも}はざりき。
\ruby[g]{日方}{ひかた}が
\ruby{我}{われ}に
\ruby{加}{くは}へし
\ruby{鐵{\GWI{u62f3-k}}}{てつ|けん}に
\ruby{就}{つ}いてか。
あらず、
\ruby[g]{日方}{ひかた}が
\ruby{事}{こと}などは
\ruby{{\GWI{u65e3-jv}}}{すで}に
\ruby{忘}{わす}れ
\ruby{居}{ゐ}たりき。
\ruby{我}{われ}は
\ruby{我}{わ}が
\ruby{胸}{むね}の
\ruby{中}{うち}に
\ruby{何}{なに}を
\ruby{思}{おも}ひ
\ruby{居}{ゐ}たりしや。
\ruby{我}{われ}は
\ruby{今日}{け|ふ}
\ruby[g]{日方}{ひかた}に
\ruby{逢}{あ}はず
\ruby[g]{{\GWI{u7fbd-k}\換字{勝}}}{はがち}に
\ruby{逢}{あ}はざりし
\ruby{前}{まへ}、
\ruby[g]{大士堂前}{だいしだうぜん}に
\ruby{圖}{はか}らず
\ruby{相會}{あひ|あ}ひたる
\ruby{彼}{か}の
\ruby{物優}{もの|やさ}しきお
\ruby{龍}{りう}を
\ruby{思}{おも}ひ
\ruby{居}{ゐ}たりしなり。
\ruby{如何}{い|か}なる
\ruby{人}{ひと}の
\ruby{憐}{あはれ}みをも
\ruby{惹}{ひ}かんとも
\ruby{思}{おも}はざりし
\ruby{愚}{おろか}なる
\ruby{此}{こ}の
\ruby{我}{わ}がために、
\ruby{我}{わ}が
\ruby{思}{おも}へる
\ruby[g]{五十子}{いそこ}の
\ruby{病}{やまひ}の
\ruby{疾}{と}く
\ruby{癒}{なほ}れかしと、
\ruby{日々}{ひ|ヾ}に
\ruby{歩}{あゆみ}を
\ruby{{\GWI{u904b-k}}}{はこ}びて
\ruby{祈}{いの}りて
\ruby{{\換字{呉}}}{く}れしといふ
\ruby{優}{やさ}しくも
\ruby{優}{やさ}しき
\ruby{彼}{か}のお
\ruby{龍}{りう}をば
\ruby{思}{おも}ひ
\ruby{居}{ゐ}たりしなり。
\ruby{其}{そ}の
\ruby{親}{した}し
\ruby{友}{とも}なりといふ
\ruby{驚}{おどろ}くべき
\ruby{美人}{び|じん} ---
\ruby{年}{とし}は
\ruby{{\GWI{u65e3-jv}}}{すで}に三十に
\ruby{近}{ちか}かるべきながら
\ruby{人}{ひと}を
\ruby{驚}{おどろ}かす
\ruby{美人}{び|じん}の、
\ruby{扮装}{いで|たち}も
\ruby{極}{きは}めて
\ruby[g]{立派}{りつぱ}なりしおとうとやらいへるより、お
\ruby{龍}{りう}が
\ruby{悲}{かな}しき
\ruby{身}{み}の
\ruby{上}{うへ}を
\ruby{朧氣}{おぼ|ろげ}に
\ruby{聞}{き}きて、
\ruby{{\GWI{u7d42-ue0101}}}{つひ}に
\ruby{堪}{こら}へ
\ruby{得}{え}て、
\ruby{我}{われ}は
\ruby{淚}{なみだ}を
\ruby{濺}{そヽ}ぎて
\ruby{泣}{な}きたりしが、
\ruby{其}{その}
\ruby{憐}{あは}れなるお
\ruby{龍}{りう}をのみ
\ruby{思}{おも}ひ
\ruby{居}{ゐ}たりしなり。
\ruby{美}{うる}はしく
\ruby{{\GWI{u6df8-jv}}}{きよ}かりし
\ruby{戀}{こひ}の
\ruby{誠}{まこと}の、人の
\ruby{{\換字{偽}}}{いつは}りに
\ruby{{\GWI{u60c5-k}}無}{なさ|けな}く
\ruby{廢}{すた}りて、
\ruby{狂}{くる}ひに
\ruby{狂}{くる}ひ、
\ruby{悲}{かなし}みに
\ruby{悲}{かなし}みたる
\ruby{末}{すゑ}の
\ruby{其}{そ}の
\ruby{女}{ひと}の、
\ruby{苦}{くる}しき
\ruby{思}{おも}ひに
\ruby{疲}{つか}るヽ
\ruby{我}{われ}を
\ruby{憐}{あは}れと
\ruby{見}{み}て、
\ruby{{\GWI{u7336-k}}}{なほ}
\ruby{有}{あ}り
\ruby{餘}{あま}る
\ruby{優}{やさ}しき
\ruby{{\GWI{u60c5-k}}}{こヽろ}を
\ruby{傾}{かたむ}けて
\ruby{我}{われ}に
\ruby{寄}{よ}せくるヽ
\ruby{其}{そ}の
\ruby{行{\GWI{u7232-var-002}}}{ふる|まひ}ばかりに
\ruby[g]{樂無}{たのしみな}き
\ruby{今}{いま}の
\ruby[g]{自己}{おのれ}を
\ruby{自}{みづか}ら
\ruby{慰}{なぐさ}むるといふ
\ruby{薄命}{はく|めい}のお
\ruby{龍}{りう}をのみ
\ruby{思}{おも}ひ
\ruby{居}{ゐ}たりしなり。
\ruby{我}{われ}は
\ruby{我}{わ}が
\ruby{{\GWI{u8ff7-k}}}{まよ}ひて
\ruby{泣}{な}き、
\ruby{苦}{くるし}みて
\ruby{悶}{もだ}えたる
\ruby{心}{こヽろ}の
\ruby{闇}{やみ}に、
\ruby{優}{やさ}しき
\ruby{光}{ひかり}の
\ruby{線}{いとすぢ}を
\ruby{投}{な}げ
\ruby{吳}{く}るヽ
\ruby{星}{ほし}を
\ruby{認}{みと}めし
\ruby[g]{心地}{こヽち}して、
\ruby{我}{わ}が
\ruby{其人}{その|ひと}に
\ruby{會}{あ}ひしをば
\ruby{滿身}{まん|しん}に
\ruby{{\GWI{u6085-jv}}}{よろこ}び
\ruby{{\GWI{u6109-k}}}{よろこ}びつ、
\ruby{我}{わ}が
\ruby{懷}{なつか}しきお
\ruby{龍}{りう}をのみ
\ruby{思}{おも}ひ
\ruby{居}{ゐ}たりしなり。


\Entry{其二}

% メモ 校正終了 2024-05-10 2024-06-06
\原本頁{7-1}%
\ruby{思}{おも}はん
とも
せずして
\ruby{思}{おも}ひ
\ruby{居}{ゐ}たるは、
%
\ruby{心}{こゝろ}の
\ruby{其}{それ}に
\ruby{染}{そ}み
たれば
なるべし。
%
されども
\ruby{吉右衛門}{きち||ゑ|もん}に
\ruby{話}{はな}し
\ruby{掛}{か}けられて、
%
\ruby{水野}{みづ|の}は
\ruby{忽}{たちま}ち
\ruby{覺}{さ}めたる
\ruby{如}{ごと}く、

\原本頁{7-4}%
『
\ruby{惡}{わる}く
\ruby{思}{おも}ふ
なんぞ
といふ
\ruby[<j>]{考}{かんがへ}が
\ruby{何樣}{ど|う}して
\ruby{私}{わたし}に
‥‥。
%
\ruby{羽{\換字{勝}}}{は|がち}だつて
\ruby{日方}{ひ|かた}だつて
\ruby[<j||]{皆}{みんな}
\ruby[||j>]{私}{わたし}の
\ruby{兄}{あに}
\ruby{同樣}{どう|よう}
なのだもの!、
%
\ruby{何}{なに}を
\ruby{言}{い}はれたつて
\ruby{惡}{わる}く
\ruby{取}{と}つたり
\ruby{氣}{き}に
\ruby{仕}{し}たり
するやうな
\ruby{事}{こと}は
\ruby{有}{あ}りは
\ruby{仕}{し}ない
ので。
%
\ruby{私}{わたし}は
\ruby{今}{いま}
ただ% 行末行頭禁則で原本も非踊り字表記
\ruby{恍然}{うつ|かり}
として
\ruby{居}{ゐ}た
ところ
でした。
%
いや
\ruby{今日}{け|ふ}は
\ruby{大層}{たい|そう}
\ruby{御世話}{お|せ|わ}でした
\改行% 校正作業の簡略化のため
。
%
\原本頁{7-8}\改行%
お
\ruby{蔭}{かげ}で
\ruby{一同}{みん|な}
\ruby{悅}{よろこ}んで
\ruby{歸}{かへ}りましたが、
%
あれを
\ruby{殘}{のこ}らず
\ruby{御厄介}{ご|やく|かい}に
なる
\ruby{理由}{いは|れ}は
ありません
から、
%
せめて
\ruby{御酒}{ご|しゆ}だけも
\ruby{私}{わたし}の
\ruby{{\換字{分}}}{ぶん}にして、
』

\原本頁{7-10}%
と
\ruby{云}{い}ひ
\ruby{掛}{か}くるを
\ruby{主人}{ある|じ}は% 原本通り非グループルビ
\ruby{悅}{よろこ}ばぬ
\ruby{氣}{げ}なる
\ruby{顏}{かほ}して、

\原本頁{7-11}%
『
また
\ruby{水野}{みづ|の}さんの
\ruby{他人}{た|にん}
\ruby{行儀}{ぎやう|ぎ}が
はじまつた。
%
\ruby{几帳面}{きち|やう|めん}
\ruby{{\換字{過}}}{す}ぎて
\ruby{厭氣}{いや|き}が
さします。
%
\ruby{宜}{い}いぢやあ
\ruby{有}{あ}りませんか
\ruby{些細}{わづ|か}の
\ruby{事}{こと}
ですもの。
』

\原本頁{8-2}%
と
\ruby{打{\換字{消}}}{うち|け}しつ、

\原本頁{8-3}%
『
それは
\ruby{左樣}{さ|う}と
\ruby[|g|]{先刻}{さつき}
\ruby{老夫}{わた|くし}が
\ruby{高田}{たか|た}さんに
\ruby{逢}{あ}ひましたら、
%
\ruby{水野}{みづ|の}さん
% \原本頁{8-4}\改行%
に
\ruby{一寸}{ちよ|つと}
\ruby{來}{き}て
\ruby{貰}{もら}ひたい
ことが
あるから
\ruby{然樣}{さ|う}
\ruby{云}{い}つて
\ruby{吳}{く}れ、
%
\ruby{他人}{ひ|と}の
\ruby{居}{ゐ}ない
\ruby{時}{とき}
\ruby{會}{あ}いたい
から
\ruby{成}{な}るべくば
\ruby{今夜}{こん|や}
あたり、
%
といふ
\ruby{御談}{お|はなし}で
ございました。
%
\ruby{御酒氣}{ご|しゆ|き}は
\ruby{大{\換字{分}}}{だい|ぶ}
\ruby{御有}{お|あ}んなさる
けれども、
%
\ruby[|g|]{貴下}{あなた}の
\ruby{事}{こと}ですから
\ruby{宜}{よ}う
ございましやう。
%
\ruby{{\換字{更}}}{ふ}けない
\ruby{中}{うち}
\ruby{一寸}{ちよ|つと}
\ruby{行}{い}つて
\ruby{居}{ゐ}らつしやい
ませんか。
』

\原本頁{8-9}%
と
\ruby{云}{い}ひ
\ruby{出}{いだ}したり。

\原本頁{8-10}%
\ruby{高田}{たか|た}は
\ruby{我}{わ}が
\ruby{職}{しよく}を
\ruby{奉}{ほう}ずる
\ruby{學校}{がく|かう}の
\ruby{長}{ちやう}にして、
%
\ruby{吉右衛門}{きち||ゑ|もん}とも
\ruby[<j||]{心}{こゝろ}% 原本とは若干ルビ配置が異なるけど
\ruby[||j>]{易}{やす}き
% \ruby{心易}{こゝろ|やす}き
\ruby[<j||]{男}{をとこ}% 行末行頭の境界付近なので特例処置を施す
なれば、
%
\ruby{水野}{みづ|の}は
\ruby{{\換字{更}}}{さら}に
\ruby{考}{かんが}ふるまでも
\ruby{無}{な}くして、

\原本頁{9-1}%
『
\ruby{何}{なん}だか
さつぱり
\ruby{{\換字{分}}}{わか}らない
けれども、
%
\ruby{其樣}{そ|ん}なら
\ruby{一寸}{ちよ|いと}
\ruby{行}{い}つて
\ruby{來}{き}ましやう。
』

\原本頁{9-3}%
と
\ruby{答}{こた}へつ、
%
\ruby{吉右衛門}{きち||ゑ|もん}が
お
\ruby{濱}{はま}を
\ruby{呼}{よ}び
\ruby{立}{た}てゝ、
%
\ruby[||j>]{提}{ちやう}
\ruby[||j>]{灯}{ ちん}を
% \ruby{提灯}{ちやう|ちん}を
と
\ruby{云}{い}ふを、
%
それにも
\ruby{及}{およ}ばずと
\ruby{制}{とゞ}め、
%
たゞ
\ruby{纔}{わづか}に
\ruby{帶}{おび}
\ruby{締}{し}め
\ruby{直}{なほ}しゝ
のみにて
\ruby{立出}{たち|い}でた
\原本頁{9-5}\改行%
り。

\原本頁{9-6}%
\ruby{高田}{たか|た}が
\ruby{家}{いへ}は
\ruby{學校}{がく|かう}の
\ruby{直}{すぐ}
\ruby{後面}{うし|ろ}にて、
%
\ruby{農家}{のう|か}
\ruby{{\換字{造}}}{づく}りにて
こそは
あらね、
%
\ruby{趣味}{おも|むき}も
\ruby{無}{な}き
\ruby{{\換字{平}}々凡々}{へい|〳〵|ぼん|〴〵}の
\ruby{住居}{すま|ゐ}なるが、
%
\ruby[|g|]{主人}{あるじ}も
\ruby{其}{その}
\ruby{家}{いへ}に
\ruby[|g|]{相應}{ふさは}しき
\ruby{{\換字{平}}々凡々}{へい|〳〵|ぼん|〴〵}の、
%
\ruby{何}{なん}の
\ruby{奇處}{き|しよ}も
\ruby{無}{な}き
\ruby{五十}{い|そ}
\ruby{男}{をとこ}にて、
%
\ruby[<j||]{農}{ひやく}
\ruby[||j>]{夫}{しやう}にて
% \ruby{農夫}{ひやく|しやう}にて
こそは
あらね、
%
\ruby{面白味}{おも|しろ|み}も
\ruby{無}{な}き
\ruby{氣}{き}の
\ruby{小}{ちひさ}なる
\ruby{謹直}{まじ|め}
\ruby{三昧}{ざん|まい}の
\ruby{人}{ひと}なり。

\原本頁{9-10}%
\ruby{{\換字{半}}白}{はん|ぱく}の
\ruby{髮}{かみ}の
\ruby{毛}{け}は
\ruby{割合}{わり|あひ}に
\ruby{多}{おほ}かれども、
%
\ruby{光澤}{つ|や}
\ruby{無}{な}く
\ruby{黃色}{き|いろ}に
\ruby{痩}{や}せきつたる
\ruby{顏}{かほ}の、
%
\ruby{口}{くち}の
\ruby{傍}{はた}の
\ruby{條{\換字{文}}}{す|ぢ}、
%
\ruby{額}{ひたひ}の
\ruby{皺}{しわ}など
\ruby{目立}{め|だ}つて
\ruby{深}{ふか}く、
%
\ruby{光無}{ひかり|な}き
\ruby{小}{ちひさ}なる
\ruby{眼}{め}、
%
\ruby{骨立}{ほね|だ}つて
\ruby{高}{たか}き
\ruby{鼻}{はな}、
%
おちつきの
\ruby{無}{な}き
\ruby{起居}{たち|ゐ}
\ruby{動作}{ふる|まひ}、
%
\ruby{活氣}{いき|ほひ}の
\ruby{無}{な}き
\原本頁{10-2}\改行%
\ruby{物}{もの}の
\ruby{言}{い}ひぶり、
%
すべての
\ruby[|g|]{乾燥}{ひから}びたる
\ruby{狀態}{あり|さま}は、
%
\ruby{如何}{い|か}にも
\ruby{能}{よ}く
\ruby{此}{この}
\ruby{人}{ひと}の、% 「、」を詰め込んでいるようだ、「の」以下で 30文字あり
    『
    \ruby{人}{ひと}の
    \ruby{子}{こ}を
    \ruby{{\換字{誤}}}{あやま}るが
    \ruby{如}{ごと}き
    \ruby{{\換字{強}}}{つよ}き
    \ruby{人}{ひと}
    』
ならで、% 「、」を詰め込んでいるようだ
%
    『
    \ruby{決}{けつ}して
    \ruby{人}{ひと}の
    \ruby{子}{こ}を
    \ruby[<j||]{{\換字{害}}}{そこな}はぬ% 行末行頭の境界付近なので特例処置を施す
    \ruby{{\換字{古}}}{ふ}りたる
    \ruby{敎育家}{けう|いく|か}
    』
たる
\ruby{事}{こと}をば
\ruby{現}{あらは}し
\ruby{示}{しめ}せり。

\原本頁{10-5}%
\ruby{高田}{たか|た}は
\ruby{今}{いま}
\ruby{水野}{みづ|の}の
\ruby{來}{きた}り
\ruby{訪}{と}ふに
\ruby{會}{あ}ひて、
%
\ruby[|g|]{一昨日}{をとゝひ}も
\ruby[|g|]{昨日}{きのふ}も
\ruby{會}{あ}ひたる
\ruby{同士}{どう|し}
なるに、
%
\ruby{三年}{さん|ねん}
\ruby{四年}{よ|ねん}も
\ruby{隔}{へだ}てゝ
\ruby{面}{おもて}を
\ruby{見}{み}たるものゝ
\ruby{如}{ごと}く、
%
\ruby{慇懃}{いん|ぎん}に
\ruby{時候}{じ|こう}の
\ruby{挨拶}{あい|さつ}など
\ruby{管々}{くだ|〴〵}しく
\ruby{仕}{し}て、
%
\ruby{三十匁}{さん|じふ|め}
ばかりの
\ruby{{\換字{廉}}價茶}{や|す|ぢや}を
\ruby{事々}{こと|〴〵}しく
\原本頁{10-8}\改行%
\ruby{湯}{ゆ}を
\ruby{冷}{さ}まし
などして
\ruby{入}{い}れ、
%
\ruby{隱}{かく}れ
\ruby{蓑}{みの}、
%
\ruby{隱}{かく}れ
\ruby{笠}{がさ}、
%
\ruby{打出}{うち|で}の
\ruby{槌}{つち}
なんどの
\ruby[||j>]{寶}{たから}
\ruby[||j>]{盡}{ づく}しを
% \ruby{寶盡}{たから|づく}しを
\ruby{描}{ゑが}きたる
\ruby{水金}{みづ|きん}の
\ruby{光}{ひか}り
\ruby{爛々}{きら|〳〵}とする
\ruby{菓子鉢}{くわ|し|ばち}に、
%
\ruby{三月}{み|つき}も
\ruby{{\換字{前}}}{まへ}より
\ruby{盛}{も}られし
\ruby{儘}{まゝ}かと
\ruby{想}{おも}はるゝ
やうなる
\ruby{最中}{も|なか}の
\ruby{月}{つき}の
\ruby{淋}{さび}しげに
\ruby{干縮}{ひ|すば}りたるを、

\原本頁{11-1}%
『
\ruby{何樣}{ど|う}ぞ
\ruby{詰}{つま}らんものですが
\ruby{御摘}{お|つま}みなすつて。
』

\原本頁{11-2}%
と
\ruby{叮嚀}{てい|ねい}に
\ruby{薦}{すゝ}め、
%
\ruby{何時}{い|つ}
\ruby{用事}{よう|じ}を
\ruby{云}{い}ひ
\ruby{出}{いだ}すべき
\ruby{氣色}{け|はひ}も
\ruby{無}{な}く、
%
\ruby{興}{きよう}も
\ruby{無}{な}き
\ruby{世}{よ}の
\ruby{噂}{うはさ}、
%
\ruby{他{\換字{所}}}{よ|そ}の
\ruby{事}{こと}をのみ、
%
\ruby{熱心}{ねつ|しん}も
\ruby{無}{な}く
\ruby{氣燄}{いき|ほひ}も
\ruby{無}{な}く、
%
\ruby{溫和}{をん|わ}に
\ruby{冷靜}{れい|せい}に
\ruby{打語}{うち|かた}りたり。

\原本頁{11-5}%
\ruby{水野}{みづ|の}も
\ruby{初}{はじめ}は
\ruby{謹}{つゝし}み
\ruby{居}{ゐ}しが、
%
\ruby{{\換字{終}}}{つひ}に
\ruby{堪}{こら}へ
\ruby{得}{え}ずして
\ruby{口}{くち}を
\ruby{開}{ひら}き、

\原本頁{11-6}%
『
\ruby{山路}{やま|ぢ}の
\ruby{老人}{らう|じん}に
\ruby{御言傳}{お|こと|づけ}
でしたので
\ruby{出}{で}ました
のですが、
%
\ruby{御用}{ご|よう}を
\ruby{何樣}{ど|う}か
\ruby{伺}{うかゞ}ひたい
もので。
』

\原本頁{11-8}%
と
\ruby{促}{うなが}すが
\ruby{如}{ごと}くに
\ruby{云}{い}ひ
\ruby{出}{い}づれば、

\原本頁{11-9}%
『
イヤー、
%
\ruby{何樣}{ど|う}もハヤ
\ruby{詰}{つま}らん
\ruby{事}{こと}で、
』

\原本頁{11-10}%
と
\ruby{磊落}{らい|らく}らしく
\ruby{右}{みぎ}の
\ruby{手}{て}を
\ruby{上}{あ}げて
\ruby{頭髮}{あた|ま}を
\ruby{撫}{な}でしが、
%
やがて
\ruby{然}{さ}も〳〵
\ruby{決心}{けつ|しん}
したり
といふ
やうに
\ruby{眞面目}{ま|じ|め}に
なつて
\ruby{自己}{お|の}が
\ruby{膝}{ひざ}を
\ruby{見詰}{み|つ}め、

\原本頁{12-1}%
『
\ruby{水野}{みづ|の}さん
\ruby{決}{けつ}して
\ruby{御怒}{お|おこ}り
なすつては
いけませんよ。
%
\ruby{萬}{ばん}
\ruby{已}{や}むを
\ruby{得}{え}んから
\ruby{是非}{ぜ|ひ}
\ruby{無}{な}く
\ruby{御話}{お|はな}しを
\ruby{致}{いた}しますがネ。
%
これも
\ruby{小生}{わた|くし}の
\ruby{地位}{ち|ゐ}から
\原本頁{12-3}\改行%
\ruby{致}{いた}しまして
\ruby{詮方}{せん|かた}が
\ruby{無}{な}いので、
%
\ruby{何樣}{ど|う}か
\ruby{惡}{あし}からず
\ruby{御}{ご}
\ruby[||j>]{解}{かい}
\ruby[||j>]{釋}{しやく}
を
\ruby{願}{ねが}ひ
ますのです。
%
\ruby{實}{じつ}は
\ruby[|g|]{貴下}{あなた}の
\ruby{御}{ご}
\ruby[<j||]{{\換字{評}}}{ひやう}
\ruby[<j||]{{\換字{判}}}{ばん}
% \ruby{{\換字{評}}{\換字{判}}}{ひやう|ばん}
が
\ruby{甚}{はなは}だ
\ruby{思}{おも}はしく
ないので。
%
イヤ
\ruby{小生}{わた|くし}は
\ruby{何{\換字{所}}}{ど|こ}までも
\ruby[|g|]{貴下}{あなた}を
\ruby{信}{しん}じて
\ruby{居}{を}りまするから、
%
\ruby{他}{ひと}が
\ruby{何}{なん}と
\ruby{申}{まを}しても
\原本頁{12-6}\改行%
\ruby{關}{かま}ひませんが、
%
\ruby{何樣}{ど|う}も
\ruby{種々}{いろ|〳〵}の
\ruby{事}{こと}を
\ruby{申}{まを}しまするので。
%
ハヽヽ、
%
\ruby{世間}{せ|けん}
といふものは
\ruby{煩}{うるさ}い
ものでしてナア、
%
\ruby{信仰}{しん|かう}の
\ruby{自由}{じ|ゆう}といふ
\ruby{事}{こと}は
\ruby{嚴然}{ちや|ん}と
\ruby{許}{ゆる}されて
\ruby{居}{を}りまするのに、
%
\ruby[|g|]{貴下}{あなた}の
\ruby{事}{こと}を
\ruby{妄信}{まう|しん}に
\ruby{陷}{おちい}つたの
\ruby{何}{なん}のと
\ruby{申}{まを}しましてナ、
%
\ruby{其}{それ}は
\ruby{{\換字{又}}}{また}
\ruby{斯樣}{か|う}いふ
\ruby{理由}{わ|け}からだの
\ruby{彼樣}{あ|ゝ}いふ
\ruby{仔細}{し|さい}からだのと
\ruby{下}{くだ}らん
\ruby{事}{こと}を
\ruby{云}{い}ひましてナ、
%
それで
\ruby{何樣}{ど|う}も
\ruby{兎角}{と|かく}
\ruby{小生}{わた|くし}の
\ruby{耳}{みゝ}へ
\ruby{煩}{うるさ}い
\ruby{事}{こと}が
\ruby{入}{はい}ります。
%
\ruby{就}{つ}きましては
\ruby{小生}{わた|くし}の
\ruby{考}{かんが}へまするには、
%
\ruby{貴下}{あな|た}も% 行末行頭なのでグループルビにしない
\ruby{其}{それ}では
\ruby{生徒}{せい|と}の
\ruby{{\換字{父}}兄}{ふ|けい}の
\ruby{手{\換字{前}}}{て|まへ}や
\ruby{何}{なん}ぞ、
%
どうも
\ruby[||j>]{敎}{けう}
\ruby[||j>]{職}{しよく}を
% \ruby{敎職}{けう|しよく}を
お
\ruby{執}{と}り
なさり
\ruby{{\換字{難}}}{にく}い
やうな
\ruby{譯}{わけ}
ですから、
%
\ruby{一應}{いち|おう}
\ruby{此村}{こ|ゝ}の
\ruby{校}{かう}の
\ruby{方}{はう}を
\ruby{御{\換字{退}}}{お|ひ}き
なすつて
\原本頁{13-3}\改行%
\ruby{頂}{いたゞ}いて、
%
\ruby{他}{た}の
\ruby{校}{かう}へ
\ruby{行}{い}つて
\ruby{頂}{いたゞ}いた
\ruby{方}{はう}が
\ruby[|g|]{貴下}{あなた}の
\ruby{御利益}{ご|り|えき}で、
%
\ruby{{\換字{又}}}{また}
\ruby{{\換字{延}}}{ひ}いては
\ruby{校}{かう}の
\ruby{爲}{ため}にも
\ruby{聊}{いさゝ}か
\ruby{利益}{り|えき}かと
\ruby{勘考}{かん|かう}
\ruby{致}{いた}しましたです。
%
\ruby{御轉校}{ご|てん|かう}の
\ruby{事}{こと}は
\原本頁{13-5}\改行%
\ruby[|g|]{貴下}{あなた}の
\ruby{御不都合}{ご|ふ|つ|がふ}にならんように、
%
\ruby{必}{かなら}ず
\ruby{小生}{わた|くし}が
\ruby{取計}{とり|はか}らひ
まするか
\原本頁{13-6}\改行%
ら。
』

\原本頁{13-7}%
と、
%
\ruby{辛}{から}くして
\ruby{云}{い}ひ
\ruby{出}{いだ}したる
\ruby{其}{そ}の
\ruby{眞意}{しん|い}は、
%
\ruby{我}{われ}をして
\ruby{職}{しよく}を
\ruby{辭}{じ}さしめん
といふ
ことなりけり。

\原本頁{13-9}%
\ruby{高田}{たか|た}は
\ruby[||j>]{重}{ぢゆう}
\ruby[||j>]{大}{ だい}の
% \ruby{重大}{ぢゆう|だい}の
\ruby{事}{こと}と
\ruby{思}{おも}へる
なるべし、
%
\ruby{水野}{みづ|の}は
\ruby{斯}{か}ばかりの
\ruby{事}{こと}かと
\ruby{毛}{け}より
\ruby{輕}{かろ}く
\ruby{思}{おも}ひて、

\原本頁{13-10}%
『
\ruby{解}{わか}りました。
%
\ruby{早{\換字{速}}}{さつ|そく}
\ruby{御諭}{お|さと}しの
\ruby{{\換字{通}}}{とほ}りに
\ruby{致}{いた}しましやう。
』

\原本頁{14-1}%
と
\ruby[||j>]{心}{こゝろ}
\ruby[||j>]{易}{ やす}く
% \ruby{心易}{こゝろ|やす}く
\ruby{答}{こた}ふれば、
%
\ruby{高田}{たか|た}は
ホツト
\ruby{息}{いき}を
つける
\ruby{樣}{さま}なり。

\Entry{其三}

\原本頁{}%
\ruby{我}{わ}が
\ruby{職務}{つと|め}を
\ruby{卑}{いやし}む
\ruby{意}{こゝろ}などは
\ruby{露}{つゆ}ばかりも
\ruby{有}{あ}らざりしが、
%
もとより
\ruby{一生}{いつ|しやう}を
\ruby{其任}{そ|れ}に
\ruby{委}{ゆだ}ねんとも
\ruby{思}{おも}はざりしなれば、
%
\ruby{水野}{みづ|の}は
\ruby{{\換字{難}}}{はゞか}る
\ruby{色}{いろ}も
\ruby{無}{な}く
\ruby{職}{しよく}を
\ruby{辭}{じ}せんと
\ruby{云}{い}へるに、
%
\ruby{高田}{たか|た}は
\ruby{我}{わ}が
\ruby{意}{こゝろ}の
\ruby{{\換字{通}}}{とほ}りたるより
\ruby{胸}{むね}は
\ruby{安}{やす}くせしものゝ、
%
\ruby{却}{かへ}つて
\ruby{{\換字{又}}}{また}
\ruby{對手}{あひ|て}の
\ruby{餘}{あま}りに
\ruby{未練氣}{み|れん|げ}
\ruby{無}{な}きに
\ruby{薄氣味}{うす|き|み}
\ruby{惡}{あし}く、
%
\ruby{懸念}{け|ねん}らしく
\ruby{小}{ちひさ}き
\ruby{眼}{め}を
\ruby{瞬}{しばた}きて
\ruby{水野}{みづ|の}を
\ruby{見居}{み|ゐ}たり。

\原本頁{}%
『\換字{志}かし
\ruby{水野}{みづ|の}さん
\ruby{決}{けつ}して
\ruby{御不快}{ご|ふ|かい}に
\ruby{御思}{お|おも}ひなすつてはいけません、
%
\ruby{何樣}{ど|う}か
\ruby{感{\換字{情}}}{かん|じやう}を
\ruby{{\換字{害}}}{がい}して
\ruby{下}{くだ}さらんやうに
\ruby{願}{ねが}ひます。
%
\ruby{小生}{わた|くし}は
\ruby{何處迄}{ど|こ|まで}も
\ruby{貴下}{あな|た}を
\ruby{信}{しん}じて
\ruby{居}{を}るのですから、
%
\ruby{貴下}{あな|た}に
\ruby{校}{かう}から
\ruby{離}{はな}れて
\ruby{頂}{いたゞ}きたい
\ruby{心}{こゝろ}は
\ruby{{\換字{更}}}{さら}に
\ruby{無}{な}いのでして、
%
\ruby{長}{なが}く
\ruby{貴下}{あな|た}と
\ruby{圓滿}{ゑん|まん}な
\ruby{御{\換字{交}}際}{ご|かう|さい}を
\ruby{繼續}{つ|な}いで
\ruby{參}{まゐ}りたいのです。
%
\ruby{貴下}{あな|た}は
\ruby{失禮}{しつ|れい}ながら
\ruby{學力}{がく|りよく}は
\ruby{御有}{お|あ}りなさるし、
%
なか〳〵
\ruby{長}{なが}く
\ruby{小學}{せう|がく}の
\ruby{敎師}{けう|し}などを
\ruby{仕}{し}て
\ruby{居}{ゐ}らつしやる
\ruby{御仁}{ご|じん}では
\ruby{無}{な}いのです。
%
が、
%
\ruby{差當}{さし|あた}つて
\ruby{校}{かう}の
\ruby{方}{はう}を
\ruby{離}{はな}れて
\ruby{戴}{いたゞ}いては
\ruby{御困}{お|こま}りでもございましやうから、
%
\ruby{小生}{わた|くし}は
\ruby{小生}{わた|くし}の
\ruby{貴下}{あな|た}に
\ruby{對}{たい}する
\ruby{眞{\換字{情}}}{しん|じやう}を
\ruby{表}{へう}して、
%
\ruby{貴下}{あな|た}を
\ruby{他{\換字{所}}}{よ|そ}の
\ruby{校}{かう}へ
\ruby{御周旋}{ご|しう|せん}
\ruby{致}{いた}しましようと
\ruby{存}{ぞん}じて
\ruby{居}{を}ります。
%
\ruby{何樣}{ど|う}か
\ruby{小生}{わた|くし}が
\ruby{貴下}{あな|た}に
\ruby{對}{たい}する
\ruby{敬意}{けい|い}を
\ruby{御汲}{お|く}み
\ruby{取}{と}り
\ruby{下}{くだ}すつて
\ruby{頂}{いたゞ}きたいもので。
』

\原本頁{}%
と、
%
\ruby{是}{これ}もまた
\ruby{三十匁}{さん|じう|め}の
\ruby{茶}{ちや}を
\ruby{入}{い}るゝに
\ruby{湯}{ゆ}を
\ruby{冷}{さ}まして
\ruby{後}{のち}にするが
\ruby{如}{ごと}く
\ruby{叮嚀}{てい|ねい}に
\ruby{言}{い}へば、
%
\ruby{水野}{みづ|の}は
\ruby{他}{ひと}に
\ruby{憎}{にく}まれじとする
\ruby{心{\換字{遣}}}{こゝろ|づか}ひの、
%
いと
\ruby{明}{あき}らかに
\ruby{見}{み}ゆる
\ruby{此}{こ}の
\ruby{{\換字{半}}白}{はん|ぱく}の
\ruby{敎育家}{けう|いく|か}を、
%
\ruby{憫然}{あは|れ}に
\ruby{思}{おも}ふやうの
\ruby{{\換字{情}}}{こゝろ}も
\ruby{起}{おこ}りて、

\原本頁{}%
『はい、
%
\ruby{有}{あ}り
\ruby{{\換字{難}}}{がた}うございます。
%
\ruby{御高{\換字{情}}}{ご|かう|せい}はまことに
\ruby{有}{あ}り
\ruby{{\換字{難}}}{がた}うございます。
%
\ruby{御言葉}{お|こと|ば}に
\ruby{甘}{あま}えまして
\ruby{何處}{いづ|れ}かへ
\ruby{御周旋}{ご|しう|せん}を
\ruby{願}{ねが}はなくつてはならんのですが、
\換字{志}かし
\ruby{小生}{わた|くし}は
\ruby{何樣}{ど|う}も
\ruby{敎鞭}{けう|べん}を
\ruby{執}{と}るには
\ruby{{\換字{適}}}{てき}せんやうに
\ruby{思}{おも}ひますから、
%
\ruby{差當}{さし|あた}つて
\ruby{他{\換字{所}}}{よ|そ}の
\ruby{校}{かう}へ
\ruby{參}{まゐ}りたいとも
\ruby{存}{ぞん}じませんです。
%
\ruby{御厚意}{ご|かう|い}は
\ruby{何處}{ど|こ}までも
\ruby{有}{あ}り
\ruby{{\換字{難}}}{がた}く
\ruby{存}{ぞん}じますけれども、
%
\ruby{當{\換字{分}}}{たう|ぶん}は
\ruby{{\換字{遊}}}{あそ}んで
\ruby{見}{み}たいと
\ruby{思}{おも}つて
\ruby{居}{を}りまする。
%
それでは
\ruby{辭表}{じ|へう}は
\ruby{明日}{みやう|にち}
\ruby{早{\換字{速}}}{さつ|そく}
\ruby{差}{さ}し
\ruby{出}{だ}しまするから、
%
\ruby{何{\換字{分}}}{なに|ぶん}
\ruby{宜}{よろ}しく
\ruby{御計}{お|はか}らひを
\ruby{願}{ねが}ひまする。
』

\原本頁{}%
と、
%
\ruby{{\換字{飽}}}{あく}まで
\ruby{{\換字{謙}}{\換字{退}}}{けん|たい}して
\ruby{柔和}{にう|わ}に
\ruby{應}{こた}へたり。

\原本頁{}%
\ruby{水野}{みづ|の}が
\ruby{面}{おもて}に
\ruby{怨氣}{ゑん|き}をも
\ruby{盛}{も}らずして、
%
\ruby{{\換字{平}}常}{ふだ|ん}の
\ruby{如}{ごと}く
\ruby{何氣}{なに|げ}なき
\ruby{言}{ことば}の
\ruby{調子}{てう|し}に
\ruby{職}{しよく}を
\ruby{辭}{じ}せんといふを
\ruby{聞}{き}き、
%
\ruby{高田}{たか|た}はやうやく
\ruby{荷}{に}を
\ruby{下}{おろ}したる
\ruby{心地}{こゝ|ち}してか、

\原本頁{}%
『ヤ、
%
それでは
\ruby{當{\換字{分}}}{たう|ぶん}
\ruby{御{\換字{遊}}}{お|あそ}びも
\ruby{宜}{よろ}しうございましやう。
%
\ruby{疾}{とう}から
\ruby{小生}{わた|くし}は
\ruby{貴下}{あな|た}を
\ruby{目}{もく}して、
%
\ruby{蛟龍}{かう|りう}
\ruby{永}{なが}く
\ruby{池中}{ち|ちゆう}のものたらずと
\ruby{申}{まを}して
\ruby{居}{を}りましたのです。
%
ハヽヽ。
%
\ruby{何樣}{ど|う}か
\ruby{今後}{こん|ご}
\ruby{何{\換字{分}}}{なに|ぶん}
\ruby{御見棄無}{お|み|すて|な}く
\ruby{御{\換字{交}}際}{ご|かう|さい}を
\ruby{願}{ねが}ひまする。
』

\原本頁{}%
と
\ruby{可笑}{を|か}しくも
\ruby{無}{な}きところに
\ruby{磊落}{らい|らく}めかして
\ruby{妙}{めう}に
\ruby{笑}{わら}つて、
%
\ruby{最後}{さい|ご}には
\ruby{改}{あらた}めて
\ruby{肘}{ひぢ}を
\ruby{張}{は}つて
\ruby{堅}{かた}くろしく
\ruby{頭}{かうべ}を
\ruby{下}{さ}げて
\ruby{一禮}{いち|れい}すれば、
%
\ruby{水野}{みづ|の}も
\ruby{是非}{ぜ|ひ}なく
\ruby{禮}{れい}を
\ruby{{\換字{返}}}{かへ}して、

\原本頁{}%
『いや
\ruby{今後}{こん|ご}の
\ruby{御{\換字{交}}際}{ご|かう|さい}は
\ruby{小生}{わた|くし}の
\ruby{方}{はう}からこそ
\ruby{願}{ねが}ふべきで、
%
では
\ruby{今日}{こん|にち}はこれで
\ruby{失禮致}{しつ|れい|いた}します。
』

\原本頁{}%
と
\ruby{慇懃}{いん|ぎん}に
\ruby{挨拶}{あい|さつ}して
\ruby{辭}{じ}し
\ruby{歸}{かへ}りたり。

\原本頁{}%
\ruby{區々}{く|ゝ}たる
\ruby{職}{しよく}と
\ruby{些々}{さ|ゝ}たる
\ruby{俸給}{ほう|きふ}とは、
%
\ruby{之}{これ}を
\ruby{得}{う}るも
\ruby{之}{これ}を
\ruby{失}{うしな}ふも
\ruby{一顰}{いつ|ぴん}
\ruby{一笑}{いつ|せう}にだに
\ruby{價}{あたひ}せずと、
%
\ruby{水野}{みづ|の}は
\ruby{其}{その}
\ruby{事}{こと}を
\ruby{繰}{く}り
\ruby{{\換字{返}}}{かへ}しても
\ruby{思}{おも}はず、
%
たゞ
\ruby{{\換字{猶}}}{なほ}
\ruby{微}{かすか}に
\ruby{殘}{のこ}れる
\ruby{醉}{よひ}を% 「醉」は原本通り「よ」で調整
\ruby{吹}{ふ}く
\ruby{風}{かぜ}の
\ruby{薄{\換字{寒}}}{うす|さむ}きを
\ruby{覺}{おぼ}えつゝ
\ruby{歸}{かへ}り
\ruby{着}{つ}けば、
%
お
\ruby{濱}{はま}は
\ruby{待}{ま}ち
\ruby{{\換字{兼}}}{か}ねしが
\ruby{如}{ごと}く
\ruby{飛}{とん}で
\ruby{出}{い}でゝ、
%
\ruby{茶}{ちや}の
\ruby{間}{ま}に
\ruby[<j|]{{\換字{迎}}}{むかへ}
\ruby{入}{い}るゝや
\ruby{否}{いな}や、
%
\ruby{滿面}{まん|めん}に
\ruby{笑}{ゑみ}を
\ruby{輝}{かゞや}かしつ、
%
\ruby{他人}{ひ|と}には
\ruby{何言}{なに|い}ふ
\ruby{間}{ひま}をも
\ruby{與}{あた}へずして、

\原本頁{}%
『
\ruby{今}{いま}
\ruby{先生}{せん|せい}と
\ruby{入}{い}れ
\ruby{{\換字{違}}}{ちが}つてネ、
%
\ruby{彼}{あ}の
\ruby{尾竹}{を|だけ}が
\ruby{變}{へん}に
\ruby{威張}{ゐ|ば}つて
\ruby{{\換字{遣}}}{や}つて
\ruby{來}{き}ましてネ。
%
とう〳〵
\ruby{此方}{こつ|ち}のものに
\ruby{仕}{し}た、
%
もう
\ruby{大{\換字{丈}}夫}{だい|ぢやう|ぶ}だ、
%
もう
\ruby{屹度}{きつ|と}
\ruby{保證}{うけ|あ}ひます、
%
もう
\ruby{宜}{よ}うございます、
%
もう
\ruby{是}{これ}からは
\ruby{快癒}{な|ほ}るばかりです、
%
\ruby{必}{かなら}ず
\ruby[g]{五十子}{いそこ}さんは
\ruby{本復}{ほん|ぷく}するといふ
\ruby{見{\換字{込}}}{み|こ}みが
\ruby{立}{た}ちました。
%
\ruby{水野}{みづ|の}さんに
\ruby{十{\換字{分}}}{じう|ぶん}
\ruby{悅}{よろこ}んで
\ruby{貰}{もら}はなくちやあ、
%
と
\ruby{云}{い}つて
\ruby{今}{いま}まで
\ruby{饒舌}{しや|べ}つて
\ruby{行}{ゆ}きましたよ。
%
\ruby{嬉}{うれ}しいのネエ
\ruby{先生}{せん|せい}。
%
\ruby[<j|]{妾}{わたし}
\ruby{嬉}{うれ}しくつて!。
%
ほんとに
\ruby[<j|]{妾}{わたし}
\ruby{嬉}{うれ}しくつて〳〵!。
』

\原本頁{}%
と
\ruby{急}{せ}きに
\ruby{急}{せ}きて
\ruby{喜悅}{よろ|こび}の
\ruby{音信}{おと|づれ}を
\ruby{傳}{つた}へたり。

\原本頁{}%
お
\ruby{濱}{はま}は
\ruby{我}{わ}が
\ruby{此}{こ}の
\ruby{言葉}{こと|ば}を
\ruby{聞}{き}くと
\ruby{齊}{ひと}しく
\ruby{水野}{みづ|の}の
\ruby{如何}{い|か}に
\ruby{悅}{よろこ}びて
\ruby{笑}{ゑ}むならんと
\ruby{思}{おも}ひ
\ruby{設}{まう}けつ、
%
\ruby{心樂}{こゝろ|たのし}みにして
\ruby{水野}{みづ|の}の
\ruby{面}{おもて}を
\ruby{差覗}{さし|のぞ}けるに、
%
\ruby{悅}{よろこ}び
\ruby{極}{きは}まつてか
\ruby{其}{その}
\ruby{人}{ひと}は
\ruby{笑}{ゑ}まず、
%
\ruby{目}{ま}のあたりに
\ruby{神佛}{かみ|ほとけ}を
\ruby{拜}{をが}めるが
\ruby{如}{ごと}き、
%
\ruby{敬}{つゝし}みに
\ruby{敬}{つゝし}めるが
\ruby{中}{なか}に
\ruby{和}{やさ}しさ
\ruby{見}{み}ゆる
\ruby{面}{おもて}になつて、
%
\ruby{抑々}{そも|〳〵}
\ruby{何}{なに}をか
\ruby{見詰}{み|つ}むるや
\ruby{頭}{かしら}を
\ruby{斜}{なゝめ}に、
%
\ruby{物}{もの}も
\ruby{無}{な}き
\ruby{{\換字{空}}中}{くう|ちゆう}を
\ruby{凝然}{じ|つ}と
\ruby{仰}{あふ}ぎたるが、
%
\ruby{見}{み}る〳〵
\ruby{動}{うご}かざる
\ruby{其}{そ}の
\ruby{眼}{め}の
\ruby{中}{うち}よりは、
%
\ruby{汪々}{わう|〳〵}
\ruby{漣々}{れん|〳〵}として
\ruby{涙}{なみだ}の
\ruby{溢}{あふ}れたり。
%
\ruby{悅}{よろこ}び
\ruby{涙}{なみだ}とはこれなるべきにや。

\Entry{其四}

\ruby{相良}{さが|ら}にも
\ruby{尾竹}{を|だけ}にも
\ruby{囘復}{くわい|ふく}の
\ruby{望無}{のぞ|みな}しとこそは
\ruby{言}{い}はれざりつれ、
\ruby{十}{じう}に
\ruby{六七}{ろく|しち}までは
\ruby{危}{あやふ}く
\ruby{思}{おも}はれたるらしき
\ruby{徴}{しるし}には、
\ruby{變狀}{へ|ん}さへ
\ruby{無}{な}くば、
\ruby{變狀}{へ|ん}さへ
\ruby{無}{な}くばと、
\ruby{遁路}{にげ|みち}のある
\ruby{保證}{うけ|あひ}の
\ruby{仕方}{し|かた}を
\ruby{爲}{さ}れたる、
\ruby{其}{そ}の
\ruby{重}{おも}き
\ruby{病}{やまひ}に
\ruby{惱}{なや}みし
\ruby{人}{ひと}の、
\ruby{今}{いま}は
\ruby{必}{かなら}ず
\ruby{癒}{なほ}るべしとは
\ruby{眞實}{ま|こと}の
\ruby{事}{こと}なりや、
\ruby{覺}{さ}めての
\ruby{後}{のち}の
\ruby{口惜}{く|や}しかるべき
\ruby{夢}{ゆめ}の
\ruby{中}{うち}の
\ruby{果敢無}{は|か|な}き
\ruby{悅}{よろこ}びにはあらざるや。
あゝ、
\ruby{夢}{ゆめ}にはあらず、
\ruby{確}{たしか}に
\ruby{現}{うつゝ}なり、
\ruby{虛妄}{いつ|はり}にはあらず、
\ruby{確}{たしか}に
\ruby{眞實}{ま|こと}なり。
かつては
\ruby{人}{ひと}の
\ruby{{\換字{運}}命}{うん|めい}の
\ruby{頼}{たの}み
\ruby{無}{な}きを
\ruby{悲}{かなし}みて、
\ruby{訴}{うつた}ふる
\ruby{方無}{かた|な}き
\ruby{我}{わ}が
\ruby{思}{おもひ}の、
\ruby{{\換字{空}}}{むな}しく
\ruby{流水}{なが|れ}に
\ruby{描}{ゑが}く
\ruby{{\換字{文}}字}{もん|じ}となつて
\ruby{{\換字{消}}}{き}ゆべきかを
\ruby{歎}{なげ}きしも、
\ruby{今}{いま}は
\ruby{天地}{てん|ち}の
\ruby{間}{あひだ}に
\ruby{愛{\換字{情}}有}{なさ|け|あ}り
\ruby{{\換字{道}}義有}{み|ち|あ}つて、
\ruby{神明佛陀}{か|み|ほと|け}の
\ruby{慈愍}{じ|みん}の
\ruby{御眥}{おん|まなじり}は
\ruby{人間}{ひ|と}の
\ruby{上}{うへ}を
\ruby{離}{はな}れず、
\ruby{愛護}{あい|ご}の
\ruby{御手}{おん|て}は
\ruby{一切}{いつ|さい}の
\ruby{衆生}{しゆ|じやう}を
\ruby{攝取}{せつ|しゆ}して
\ruby{捨}{す}てざらんと
\ruby{仕玉}{し|たま}へることを
\ruby{思}{おも}ひ
\ruby{奉}{たてまつ}り、
\ruby{愚}{おろか}しき
\ruby{一念}{いち|ねん}の
\ruby{誠}{まこと}を
\ruby{籠}{こ}めて、
\ruby{他人}{ひ|と}には
\ruby{言}{い}へぬ
\ruby{心中}{しん|ちう}の
\ruby{秘事}{ひめ|ごと}に、あはれ
\ruby{彼}{か}の
\ruby{人}{ひと}の
\ruby{壽}{いのち}の
\ruby{無}{な}きに
\ruby{定}{さだ}まれるならば、
\ruby{我}{わ}が
\ruby{生命}{いの|ち}を
\ruby{殺}{そ}ぎ
\ruby{縮}{ちゞ}めても
\ruby{助}{たす}けさせ
\ruby{玉}{たま}へ、かかる
\ruby{{\換字{道}}理}{こと|わり}
\ruby{無}{な}き
\ruby{願}{ねが}ひを
\ruby{掛}{か}け
\ruby{奉}{たてまつ}ることの、
\ruby{愚}{おろか}にも
\ruby{愚}{おろか}なるをば
\ruby{知}{し}らぬにはあらねど、
\ruby{知}{し}りて
\ruby{{\換字{猶}}}{なほ}
\ruby{已}{や}まんとして
\ruby{已}{や}み
\ruby{難}{がた}き
\ruby{胸}{むね}の
\ruby{苦}{くる}しさは、
\ruby{御覽}{み|そな}はさぬところ
\ruby{無}{な}き
\ruby{神明佛陀}{か|み|ほと|け}の
\ruby{見{\換字{透}}}{み|とほ}したまひて、
\ruby{憫然}{あは|れ}とも
\ruby{思}{おぼ}して
\ruby{我}{わ}が
\ruby{心}{こゝろ}をば
\ruby{納}{い}れさせたまへ、と
\ruby{祈}{いの}りたりしが、
\ruby{彼}{か}の
\ruby{人}{ひと}の
\ruby{壽命}{じゆ|みやう}の
\ruby{本}{もと}より
\ruby{有}{あ}りしか、
\ruby{我}{わ}が
\ruby{命}{いのち}の
\ruby{彼}{か}の
\ruby{人}{ひと}の
\ruby{命}{いのち}を
\ruby{補}{おぎな}ひしかは
\ruby{知}{し}らず、
\ruby{大旱}{ひ|でり}に
\ruby{萎}{しな}れし
\ruby{玉苗}{たま|なえ}の、
\ruby{一夜}{いち|や}の
\ruby{露}{つゆ}に
\ruby{蘇}{よみがへ}つて、
\ruby{田面}{た|づら}を
\ruby{渡}{わた}る
\ruby{曉風}{あさ|かぜ}には
\ruby{{\換字{猶}}}{なほ}
\ruby{{\換字{弱}}々}{よわ|〳〵}と
\ruby{戰}{そよ}ぎながらも、はや
\ruby{行末}{ゆく|すゑ}の
\ruby{頼}{たの}もしき
\ruby{榮}{さかえ}を
\ruby{見}{み}する
\ruby{其}{そ}の
\ruby{色}{いろ}の
\ruby{靑々}{あを|〳〵}と
\ruby{勢}{いきほひ}
\ruby{好}{よ}きが
\ruby{如}{ごと}く、
\ruby{危}{あやふ}くも
\ruby{心細}{こゝろ|ぼそ}かりし
\ruby{病}{やまひ}の
\ruby{瀬}{せ}を
\ruby{{\換字{過}}}{す}ぎて、
\ruby{全}{まつた}く
\ruby{復}{また}
\ruby{現世}{この|よ}の
\ruby{光}{ひかり}に
\ruby{美}{うつく}しう
\ruby{照}{て}らさるゝやうになりし
\ruby{彼}{か}の
\ruby{人}{ひと}の
\ruby{{\換字{運}}}{うん}の
\ruby{目出度}{め|で|た}さ、
\ruby{我}{わ}が
\ruby{心}{こゝろ}の
\ruby{嬉}{うれ}しさ。
\ruby{思}{おも}へば
\ruby{神明}{か|み}も
\ruby{佛陀}{ほと|け}も
\ruby{確}{たしか}に
\ruby{御坐}{お|は}す
\ruby{世}{よ}なり。
\ruby{人間}{ひ|と}を
\ruby{包}{つゝ}める
\ruby{{\換字{運}}命}{うん|めい}は
\ruby{雲霧}{くも|きり}と
\ruby{冥}{くら}くして
\ruby{得知}{え|し}れねども、
\ruby{其中}{その|うち}に
\ruby{神明}{か|み}の
\ruby{御心}{み|こゝろ}
\ruby{佛陀}{ほと|け}の
\ruby{御心}{み|こゝろ}は
\ruby{動}{うご}き
\ruby{働}{はたら}きて、
\ruby{人間}{ひ|と}の
\ruby{抱}{いだ}く
\ruby{心}{こゝろ}のさまに
\ruby{酬}{むく}ひたまふやうの
\ruby{氣}{き}ぞする。
\ruby{冥々}{めい|〳〵}の
\ruby{中}{うち}に
\ruby{靈}{く}しき
\ruby{力}{ちから}ありて
\ruby{神佛}{しん|ぶつ}の
\ruby{意}{い}を
\ruby{受}{う}け、
\ruby{吉}{よき}も
\ruby{凶}{あしき}も
\ruby{皆}{みな}
\ruby{其力}{そ|れ}の
\ruby{爲}{す}る
\ruby{事}{こと}のやうにぞ
\ruby{思}{おも}はるゝ。
\ruby{神明}{か|み}も
\ruby{{\換字{遠}}}{とほ}からず、
\ruby{佛陀}{ほと|け}も
\ruby{{\換字{遠}}}{とほ}からず、
\ruby{一念}{いち|ねん}の
\ruby{微}{かすか}なる
\ruby{動}{うご}きも
\ruby{洩}{も}らさず
\ruby{知}{し}りたまふと
\ruby{覺}{おぼ}ゆ。
\ruby{嗚呼}{あ|ゝ}、
\ruby{神明}{か|み}も
\ruby{佛陀}{ほと|け}も
\ruby{{\換字{猶}}}{なほ}
\ruby{御覽}{み|そな}はせ、
\ruby{我}{わ}が
\ruby{心}{こゝろ}の
\ruby{誠}{まこと}を
\ruby{邪無}{よこしま|な}く、
\ruby{汚無}{けが|れな}く、
\ruby{僞無}{いつはり|な}く
\ruby{人}{ひと}を
\ruby{思}{おも}ひて、
\ruby{我}{わ}が
\ruby{如何}{い|か}にしてか
\ruby{有}{あ}り
\ruby{果}{は}つべき
\ruby{我}{わ}が
\ruby{世}{よ}の
\ruby{末}{すゑ}を
\ruby{見}{み}んとぞ
\ruby{思}{おも}ふ。
\ruby{實}{げ}に
\ruby{地}{つち}を
\ruby{掘}{ほ}れば
\ruby{水}{みづ}に
\ruby{逢}{あ}ひ、
\ruby{壁}{かべ}を
\ruby{穿}{うが}てば
\ruby{光}{ひかり}に
\ruby{逢}{あ}ひ、
\ruby{人}{ひと}の
\ruby{心}{こゝろ}の
\ruby{奧}{おく}に
\ruby{入}{い}れば
\ruby{必}{かなら}ず
\ruby{神明佛陀}{か|み|ほと|け}に
\ruby{逢}{あ}ひ
\ruby{奉}{たてまつ}るものと
\ruby{云}{い}へるも
\ruby{言}{い}ひ
\ruby{得}{え}たることかな。
\ruby{我}{われ}
\ruby{今}{いま}
\ruby{幸}{さいはひ}にして
\ruby{眼}{ま}のあたりに
\ruby{利生}{り|しやう}を
\ruby{仰}{あふ}ぎ
\ruby{得}{え}、
\ruby{冥々}{めい|〳〵}の
\ruby{中}{うち}に
\ruby{御坐}{お|は}して
\ruby{果敢無}{は|か|な}き
\ruby{此}{こ}の
\ruby{我}{われ}を
\ruby{愛}{いつく}しみたまふ
\ruby{大慈}{だい|じ}
\ruby{大悲}{だい|ひ}の
\ruby{御心}{おん|こゝろ}の
\ruby{忝}{かたじけな}きを
\ruby{感}{かん}じて、
\ruby{此}{こ}の
\ruby{嬉}{うれ}しさ
\ruby{有}{あ}り
\ruby{難}{がた}さ
\ruby{肺腑}{はい|ふ}に
\ruby{浸}{し}み
\ruby{徹}{とほ}りぬ。
\ruby{願}{ねが}はくは
\ruby{我}{われ}
\ruby{長}{なが}く
\ruby{此心}{この|こゝろ}を
\ruby{失}{うしな}はずして
\ruby{頼}{たの}み
\ruby{奉}{たてまつ}らんほどに、
\ruby{{\換字{猶}}}{なほ}
\ruby{行末}{ゆく|すゑ}
\ruby{掛}{か}けて
\ruby{彼}{か}の
\ruby{人}{ひと}の
\ruby{上}{うへ}に
\ruby{幸福}{さい|はひ}
\ruby{多}{おほ}からしめ
\ruby{給}{たま}へ。
\ruby{我}{わ}が
\ruby{身}{み}の
\ruby{幸福}{さい|はひ}をば
\ruby{祈}{いの}り
\ruby{求}{もと}むればこそ、たゞ
\ruby{彼}{か}の
\ruby{人好}{ひと|よ}かれとのみ
\ruby{思}{おも}ふこゝろの、
\ruby{此}{こ}の
\ruby{虛僞無}{いつ|はり|な}き
\ruby{眞實}{ま|こと}を
\ruby{汲}{く}ませたまへ、と
\ruby{水野}{みづ|の}は
\ruby{默念}{もく|ねん}したり。

\ruby{其夜}{その|よ}
\ruby{水野}{みづ|の}は
\ruby{何事}{なに|ごと}を
\ruby{思}{おも}ひつゞけしにや、
\ruby{{\換字{更}}}{ふ}くるまで
\ruby{{\換字{終}}}{つひ}に
\ruby{睡}{ねむ}りに
\ruby{入}{い}らで、
\ruby{二番鷄}{に|ばん|とり}の
\ruby{唱}{うた}ふ
\ruby{頃}{ころ}
\ruby{辛}{から}くも
\ruby{夢}{ゆめ}を
\ruby{結}{むす}びぬ。
たゞ
\ruby{思}{おも}ふ
\ruby{人}{ひと}の
\ruby{病}{やまひ}の
\ruby{快}{よ}き
\ruby{方}{かた}に
\ruby{向}{むか}へるを
\ruby{悅}{よろこ}んで、おのが
\ruby{職}{しよく}を
\ruby{失}{うしな}へることなんどは
\ruby{悔}{くや}みもせざりしなるべし。


\Entry{其五}

\原本頁{}%
お
\ruby{濱}{はま}は
\ruby{可笑}{を|か}しさに
\ruby{堪}{た}へぬ
\ruby{如}{ごと}く
\ruby{笑}{わら}ひを
\ruby{耐}{こら}へながら、

\原本頁{}%
『マア
\ruby[g]{如是}{こんな}に
\ruby{晏}{おそ}く
\ruby{起}{お}きて
\ruby{置}{お}いて、
%
\ruby{而}{さう}して
\ruby{變}{へん}に
\ruby{沈着}{おち|つ}いて
\ruby{居}{ゐ}らつしやるの\換字{子}。
%
\ruby{先生}{せん|せい}、
%
\ruby{今日}{け|ふ}は
\ruby{日曜}{にち|えう}ぢやあ
\ruby{有}{あ}りませんよ。
%
\ruby{早{\換字{速}}}{さつ|さ}となさらないともう
\ruby{遲}{おく}れますよ。
%
\ruby{彼人}{な|に}が
\ruby{快}{い}いもんで
\ruby{安心}{あん|しん}して
\ruby{仕舞}{し|ま}つて、
%
それで
\ruby{全然}{すつ|かり}
\ruby{氣}{き}が
\ruby{弛}{ゆる}んで
\ruby{御仕舞}{お|し|ま}ひなすつたの?。
%
\ruby{妙}{めう}に
\ruby{今{\換字{朝}}}{け|さ}はゆつたりとして
\ruby{澄}{す}まして
\ruby{居}{ゐ}らつしやるのネエ。
%
\ruby{何樣}{ど|う}なすつたの?、
%
\ruby{餘}{あんま}りだは!、
%
をかしくつてよ。
』

\原本頁{}%
と
\ruby{戲}{たはむ}るゝが
\ruby{如}{ごと}く
\ruby{云}{い}ひしが
\ruby{{\換字{終}}}{つひ}に
\ruby{堪}{こら}へかねて、

\原本頁{}%
『ホヽホヽヽヽ。
』

\原本頁{}%
と
\ruby{笑}{わら}ひ
\ruby{出}{いだ}したり。

\原本頁{}%
\ruby{夢}{ゆめ}の
\ruby{名殘}{な|ごり}を
\ruby{洗}{あら}ふ
\ruby{{\換字{朝}}茶}{あさ|ぢや}の
\ruby{淡}{あは}き
\ruby{味}{あぢはひ}を
\ruby{樂}{たのし}みて、
%
\ruby{悠然}{いう|ぜん}として
\ruby{湯呑}{ゆ|のみ}を
\ruby{手}{て}にして
\ruby{居}{ゐ}たりし
\ruby[g]{水野}{みづの}は
\ruby{此}{こ}の
\ruby{笑}{わらひ}に
\ruby{驚}{おどろ}かされつ、
%
\ruby{實}{げ}に
\ruby{我}{わ}が
\ruby{心}{こゝろ}の
\ruby{中}{うち}の
\ruby{昨日}{きの|ふ}に
\ruby{今日}{け|ふ}は
\ruby{甚}{いた}く
\ruby{異}{こと}なりて
\ruby{寛}{ゆたか}なれば、
%
\ruby{外}{そと}に
\ruby{現}{あらは}るゝ
\ruby{身}{み}の
\ruby{樣子}{やう|す}も、
%
\ruby{他}{ひと}には
\ruby{可笑}{を|か}しきほど
\ruby{變}{かは}れるなるべし。
%
\ruby{特}{こと}に
\ruby{掌上}{ての|ひら}に
\ruby{乘}{の}るばかりの
\ruby{微少}{わづ|か}なる
\ruby{俸米}{ほう|まい}に
\ruby{繋}{つな}がれても、
%
\ruby{職務}{つと|め}と
\ruby{思}{おも}へば
\ruby{其職}{そ|れ}を
\ruby{疎畧}{おろ|そか}にせん% 131-3-40-其四十.tex では 疎略(おろ|そか) とある
\ruby{心}{こゝろ}は
\ruby{無}{な}くて、
%
\ruby{身體}{から|だ}の
\ruby{疲}{つか}れきつたる
\ruby{時}{とき}にも、
%
\ruby{氣合}{き|あひ}の
\ruby{如何}{い|か}にしても
\ruby{{\換字{進}}}{すゝ}まざる
\ruby{折}{をり}にも、
%
\ruby{{\換字{強}}}{し}ひて
\ruby{勉}{つと}めて
\ruby{果}{はた}すべきだけの
\ruby{事}{こと}を
\ruby{果}{はた}したる、
%
\ruby{其}{そ}の
\ruby{苦}{くる}しさを
\ruby{今}{いま}は
\ruby{免}{まぬか}れて、
%
\ruby{起}{お}きるも
\ruby{睡}{ね}るも
\ruby{心}{こゝろ}のまゝの、
%
\ruby{肩}{かた}に
\ruby{荷}{に}は
\ruby{無}{な}き
\ruby{境界}{きよう|がい}となりたるを、
%
まだ
\ruby{知}{し}らねば
お
\ruby{濱}{はま}の
\ruby{怪}{あやし}むも
\ruby{無理}{む|り}ならずと
\ruby{微笑}{ほゝ|ゑ}まれ、

\原本頁{}%
『ハヽヽ、
%
\ruby{何}{なんに}も
\ruby{可笑}{を|か}しいことも
\ruby{何}{なに}も
\ruby{有}{あ}りは
\ruby{仕}{し}ないよ。
%
\ruby{今日}{け|ふ}はもう
\ruby{學校}{がく|かう}へも
\ruby{何}{なに}へも
\ruby{出}{で}やあ
\ruby{仕}{し}ないのだもの、
%
いくら
\ruby{沈着}{おち|つ}いて
\ruby{居}{ゐ}ても
\ruby{可笑}{を|か}しい
\ruby{事}{こと}あ
\ruby{有}{あ}りやあ
\ruby{仕}{し}ない。
』

\原本頁{}%
と
\ruby{輕}{かろ}く
\ruby{答}{こた}へたり。

\原本頁{}%
『ぢやあ
\ruby{今日}{け|ふ}は
\ruby{怠}{なま}けて
\ruby{御休}{お|やす}みなさるの?。
%
\ruby{{\換字{嫌}}}{いや}な
\ruby{先生}{せん|せい}ネエ!、
%
\ruby{何故}{な|ぜ}
\ruby{御休}{お|やす}みなさるの?。
』

\原本頁{}%
『なあに
\ruby{怠}{なま}けて
\ruby{休}{やす}む
\ruby{譯}{わけ}ぢやあ
\ruby{無}{な}いが、
%
\ruby{今日}{け|ふ}ツからは
\ruby{私}{わたし}にやあ
\ruby{毎日}{まい|にち}
\ruby{日曜}{にち|えう}なのだ。
%
だからもう
\ruby{先生}{せん|せい}〳〵つて
\ruby{云}{い}ふのも
\ruby{止}{よ}して
\ruby{貰}{もら}はなくつちやあ。
%
\ruby{仕方}{し|かた}が
\ruby{無}{な}いから
\ruby{今}{いま}までは
\ruby{呼}{よ}ばれて
\ruby{居}{ゐ}たけれども、
%
\ruby{先生}{せん|せい}〳〵つて
\ruby{云}{い}はれるなあ、
%
\ruby{先}{せん}から
\ruby{私}{わたし}あ
\ruby{好}{す}きぢやあ
\ruby{無}{な}かつたのだからネ。
』

\原本頁{}%
『あら、
%
それぢやあ
\ruby{學校}{がく|かう}をもう
\ruby{御止}{お|よ}しなすつたの?。
』

\原本頁{}%
『あゝ。
%
\ruby{高田}{たか|た}さんが
\ruby{止}{よ}したら
\ruby{宜}{よ}からうといふから
\ruby{止}{よ}して
\ruby{仕舞}{し|ま}ふことにした。
』

\原本頁{}%
『
\ruby{何故}{な|ぜ}
\ruby{高田}{たか|た}さんが
\ruby{其樣}{そ|ん}なことを
\ruby{云}{い}ひ
\ruby{出}{だ}したの。
%
\ruby{憎}{にく}らしい
\ruby{高田}{たか|た}さんだことネエ、
%
\ruby{何故}{な|ぜ}
\ruby{先生}{せん|せい}に
\ruby{御止}{お|よ}しなさいつて
\ruby{云}{い}つたの?。
』

\原本頁{}%
\ruby{問}{と}はれては
\ruby{流石}{さす|が}に
\ruby{勇}{いさ}んで
\ruby{答}{こた}ふべきならねば、
%
\ruby[g]{水野}{みづの}は
\ruby{唯}{たゞ}
\ruby{默然}{もく|ぜん}として
\ruby{笑}{わら}つて
\ruby{語}{かた}らず。

\原本頁{}%
『
\ruby{昨夜}{ゆふ|べ}
\ruby{高田}{たか|た}さんところへ
\ruby{入}{い}らしつたのは
\ruby{其}{そ}の
\ruby{事}{こと}でしたか。
』

\原本頁{}%
『あゝ、
』

\原本頁{}%
『ほんとに
\ruby{可厭}{い|や}な
\ruby{高田}{たか|た}さんだこと!。
%
\ruby{可}{い}いは、
%
\ruby{祖{\換字{父}}}{おぢい|さん}に
\ruby{左樣}{さ|う}いつて
\ruby{叱}{しか}らせて
\ruby{{\換字{遣}}}{や}るは。
%
\ruby{左樣}{さ|う}して
\ruby{復}{また}
\ruby{先生}{せん|せい}を
\ruby{舊}{もと}の
\ruby{{\換字{通}}}{とほ}りにするやうに
\ruby{爲}{さ}せるは。
』

\原本頁{}%
『ハヽヽ。
%
\ruby{折角}{せつ|かく}
\ruby{丁度}{ちやう|ど}
\ruby{止}{よ}して
\ruby{仕舞}{し|ま}つたものを、
%
\ruby{其樣}{そ|ん}な
\ruby{世話}{せ|わ}を
\ruby{燒}{や}かれちやあ
\ruby{却}{かへ}つて
\ruby{困}{こま}るよ。
%
\ruby{打棄}{うつ|ちや}つて
\ruby{置}{お}いて
\ruby{吳}{く}れ
\ruby{無}{な}くちやあ。
』

\原本頁{}%
『だつて、
%
それぢやあ
\ruby{先生}{せん|せい}は、
%
\ruby{何處}{ど|こ}か
\ruby{他{\換字{所}}}{よ|そ}へ
\ruby{行}{い}つて
\ruby{御仕舞}{お|し|ま}ひなさるんでしやう。
%
\ruby{此樣}{こ|ん}な
\ruby{詰}{つま}らない
\ruby{村}{むら}にやあ
\ruby{居}{ゐ}て
\ruby{下}{くだ}さらないでしやう。
%
\ruby{屹度}{きつ|と}の
\ruby{妾}{わたし}の
\ruby{家}{うち}を
\ruby{出}{で}て
\ruby{行}{い}つて
お
\ruby{仕舞}{し|ま}ひなさるんでしやう。
』

\原本頁{}%
と
\ruby{云}{い}ひさして
\ruby[g]{水野}{みづの}の
\ruby{面}{おもて}を
\ruby{凝然}{じ|つ}と
\ruby{見居}{み|ゐ}たりしが、

\原本頁{}%
『
\ruby{{\換字{嫌}}}{いや}だは、
%
\ruby{{\換字{嫌}}}{いや}だは、
%
\ruby[<j|]{妾}{わたし}
\ruby{{\換字{嫌}}}{いや}だは!。
%
\ruby{祖{\換字{父}}}{おぢい|さん}に
\ruby{左樣}{さ|う}
\ruby{云}{い}つて
\ruby{高田}{たか|た}さんを
\ruby{叱}{しか}らせるから
\ruby{宜}{い}いは。
』

\原本頁{}%
と
\ruby{眼}{め}に
\ruby{露}{つゆ}
\ruby{持}{も}つて
\ruby{腹立}{はら|だ}しげに
\ruby{悶}{もだ}えたり。

\原本頁{}%
『ハヽヽ。
%
\ruby{祖{\換字{父}}}{おぢい|さん}が
\ruby{何程}{いく|ら}
\ruby{幅利}{はば|きき}でも、
%
\ruby{高田}{たか|た}さんは
\ruby{高田}{たか|た}さんだから、
%
\ruby{左樣}{さ|う}
\ruby{自由}{じ|ゆう}の
\ruby{利}{き}くわけのものでも
\ruby{無}{な}い。
%
また
\ruby{私}{わたし}は
\ruby{今}{いま}
\ruby{何處}{ど|こ}へ
\ruby{行}{ゆ}くといふことも
\ruby{有}{あ}りや
\ruby{仕無}{し|な}いから、
%
\ruby{矢張}{やつ|ぱり}いつまでも
\ruby{此村}{こ|ゝ}に
\ruby{居}{ゐ}るつもりだよ。
』

\原本頁{29-4}%
『
\ruby[g]{眞實}{ほんと}?、
%
\ruby{眞實}{ほん|たう}?、
%
\ruby{矢張}{やつ|ぱり}いつまでも
\ruby{此家}{う|ち}に
\ruby{居}{ゐ}らつしやるの?。
』

\原本頁{}%
『あゝ。
%
\ruby{別}{べつ}に
\ruby{何處}{ど|こ}へ
\ruby{行}{ゆ}かうと
\ruby{云}{い}ふ
\ruby{料簡}{れう|けん}も
\ruby{無}{な}いから。
』

\原本頁{}%
『
\ruby{嬉}{うれ}しい!。
%
それぢやあ
\ruby{學校}{がく|かう}へも
\ruby{出}{で}ないで
\ruby[g]{始{\換字{終}}}{しじゆう}
\ruby{此家}{ゝ|ち}に
\ruby{居}{ゐ}らつしやる。
%
あゝ、
%
そんなら
\ruby{學校}{がく|かう}なんか
\ruby{先生}{せん|せい}が
\ruby{止}{よ}し
\ruby{仕舞}{ち|ま}つた
\ruby{方}{はう}が
\ruby{宜}{い}いは。
%
\ruby[g]{澤山}{たんと}
\ruby{先生}{せん|せい}が
\ruby{此家}{う|ち}に
\ruby{居}{ゐ}らつしやるのだから。
%
\ruby{今後}{これ|から}また
\ruby{先}{せん}のやうに
\ruby{種々}{いろ|〳〵}の
\ruby{面白}{おも|しろ}い
\ruby{御話}{お|はなし}を
\ruby{仕}{し}て
\ruby{頂}{いたゞ}けるはネ。
』

\原本頁{}%
\ruby{人}{ひと}の
\ruby{胸}{むね}の
\ruby{中}{うち}は
\ruby{{\換字{更}}}{さら}に
\ruby{知}{し}らず、
%
\ruby{{\換字{飽}}}{あく}まで
\ruby{我儘}{わが|まゝ}なる
\ruby{處女氣}{をと|め|ぎ}の
\ruby{長閑}{のど|か}さに、
%
\ruby[g]{水野}{みづの}は
\ruby{笑}{わら}つて
\ruby{點頭}{うな|づ}かざるを
\ruby{得}{え}ざりき。

\原本頁{}%
『これでもう
\ruby{淺草}{あさ|くさ}へも
\ruby{行}{い}らつしやらないと、
%
\ruby[g]{眞實}{ほんと}に
\ruby{好}{い}いのだけれども。
』

\原本頁{}%
\ruby{{\換字{猶}}}{なほ}
\ruby{不足氣}{ふ|そく|げ}に
\ruby{如是}{か|く}
\ruby{云}{い}ひて
\ruby{嫣然}{につ|こり}と
\ruby{笑}{ゑ}める
\ruby{面}{おもて}つき、
%
また
\ruby{無}{な}く
\ruby{美}{うるは}し。


\Entry{其六}

\ruby{罪}{つみ}も
\ruby{無}{な}く
\ruby{念}{おもひ}も
\ruby{無}{な}きお
\ruby{濱}{はま}の
\ruby{願}{ねがひ}の
\ruby{是}{かく}の
\ruby{如}{ごと}きには
\ruby{引}{ひき}かへて、
\ruby{水野}{みづ|の}が
\ruby{今朝差當}{け|さ|さし|あた}つて
\ruby{先}{ま}づ
\ruby{思}{おも}へるは、
\ruby{淺草}{あさ|くさ}の
\ruby{御堂}{み|だう}に
\ruby{詣}{まゐ}りて
\ruby{心靜}{こヽろ|しづ}かに
\ruby{報恩謝徳}{はう|おん|しや|とく}の
\ruby[g]{誠意}{まこと}を
\ruby{{\換字{運}}}{はこ}び、かつは
\ruby{{\換字{猶}}行末}{なほ|ゆく|すゑ}を
\ruby{頼}{たの}み
\ruby{奉}{たてまつ}らんとの
\ruby{事}{こと}のみなりしなり。

されど
\ruby{淺草}{あさ|くさ}に
\ruby{詣}{まゐ}らんと
\ruby{思}{おも}ふ
\ruby{意}{こヽろ}の
\ruby{側}{わき}には、
\ruby{{\換字{強}}}{し}ひて
\ruby{求}{もと}むるといふほどにはあらねど、
\ruby{若}{も}し
\ruby{機會}{を|り}よく
\ruby{我}{わ}が
\ruby{御堂}{み|だう}に
\ruby{詣}{まゐ}りてより
\ruby{歸}{かへ}るまでの
\ruby{間}{あひだ}に、
\ruby{彼}{か}の
\ruby[g]{同{\換字{情}}深}{おもひやりふか}き
\ruby{信心深}{しん|〴〵|ふか}き
\ruby{優}{やさ}しく
\ruby{懷}{なつか}しき
\ruby[g]{不幸福}{ふしあわせ}の
\ruby{人}{ひと}に
\ruby[g]{相逢}{あひあ}ふことを
\ruby{得}{え}ば、
\ruby{願}{ねが}はくば
\ruby[g]{相逢}{あひあ}ひて
\ruby{一}{ひ}ㇳ
\ruby{言二}{こと|ふ}
\ruby{言}{こと}の
\ruby{言葉}{こと|ば}をも
\ruby{交}{まじ}へたきやうの
\ruby{念}{おもひ}も
\ruby{潜}{ひそ}めるなり。
\ruby[g]{昨日}{きのふ}の
\ruby[g]{談話}{はなし}にて、
\ruby{其人}{その|ひと}の
\ruby{詣}{まゐ}るは、
\ruby{毎日大抵午前}{まい|にち|たい|てい|ひる|まへ}の
\ruby{事}{こと}にして
\ruby{午後}{ひる|すぎ}に
\ruby{詣}{まう}でしは
\ruby[g]{昨日}{きのふ}のみなりと
\ruby{知}{し}りたれば、
\ruby[g]{職務}{つとめ}に
\ruby{縛}{しば}らるヽ
\ruby{身}{み}の
\ruby{午前}{ひる|まへ}は
\ruby{我}{わ}が
\ruby{自由}{ま|ヽ}ならで
\ruby{其人}{その|ひと}に
\ruby{再}{ふたヽ}び
\ruby[g]{行逢}{ゆきあ}ふことも
\ruby{無}{な}かるべきを
\ruby{{\換字{遺}}憾}{のこり|をし}く
\ruby{思}{おも}ひ
\ruby{居}{ゐ}たるが、
\ruby[g]{昨日}{きのふ}に
\ruby{今日}{け|ふ}は
\ruby{變}{かは}れる
\ruby{我}{わ}が
\ruby{上}{うへ}の、
\ruby{今}{いま}は
\ruby[g]{何時參}{いつまひ}らんも
\ruby{心}{こヽろ}の
\ruby{自由}{ま|ヽ}なるまヽ、
\ruby{先}{ま}づ
\ruby{彼}{か}の
\ruby{人}{ひと}の
\ruby{詣}{まゐ}るといふ
\ruby{午前}{ひる|まへ}に
\ruby{詣}{まゐ}りて、
\ruby{幸}{さいはひ}にして
\ruby{若}{も}し
\ruby{相見}{あい|まみ}ゆることを
\ruby{得}{え}たらんには、
\ruby{我}{わ}が
\ruby{五十子}{い|そ|こ}の
\ruby{病氣}{やま|ひ}の
\ruby{本復疑}{ほん|ぷく|うたが}ひ
\ruby{無}{な}きに
\ruby{至}{いた}りたる
\ruby{事}{こと}をも
\ruby{告}{つ}げて、
\ruby{御佛}{みほ|とけ}の
\ruby{加護}{か|ご}を
\ruby{{\換字{悅}}}{よろこ}び、
\ruby{彼}{か}の
\ruby{人}{ひと}の
\ruby{親切}{しん|せつ}を
\ruby{謝}{しや}しもせんとの
\ruby{念}{おもひ}の
\ruby{潜}{ひそ}めるなり。
\ruby{優}{やさ}しく
\ruby{懷}{なつか}しき
\ruby{彼}{か}の
\ruby{人}{ひと}に、
\ruby{我}{わ}が
\ruby{五十子}{い|そ|こ}の
\ruby{甚危}{いと|あやふ}きところを
\ruby{免}{のが}れて、
\ruby{復}{ふたヽ}び
\ruby{現世}{この|よ}の
\ruby{日}{ひ}に
\ruby{照}{て}らさるヽに
\ruby{至}{いた}りしことを、
\ruby{人}{ひと}を
\ruby{吸}{す}ひ
\ruby{入}{い}るヽが
\ruby{如}{ごと}き
\ruby{其}{そ}の
\ruby{愛深}{あい|ふか}き
\ruby{笑顏}{ゑ|がほ}に
\ruby{{\換字{悅}}}{よろこ}び
\ruby{欣}{よろこ}びて
\ruby{貰}{もら}ひたき
\ruby{念}{おもひ}の
\ruby{潜}{ひそ}めるなり。

\ruby{水野}{みづ|の}はお
\ruby{濱}{はま}が
\ruby{假初}{かり|そめ}の
\ruby{語}{ことば}には
\ruby{耳}{みヽ}を
\ruby{假}{か}すことも
\ruby{無}{な}く、やがて
\ruby{淺草}{あさ|くさ}さして
\ruby{立出}{たち|い}でたり。

\ruby{幾度}{いく|たび}か
\ruby{往來}{ゆき|き}し
\ruby{馴}{な}れたる
\ruby{路}{みち}の、
\ruby{眼}{め}に
\ruby{古}{ふ}りたる
\ruby[g]{景色}{けしき}は
\ruby{心}{こヽろ}の
\ruby{留}{と}まる
\ruby{方}{かた}も
\ruby{無}{な}くて、
\ruby{早}{はや}くも
\ruby{御堂}{み|だう}に
\ruby{到}{いた}り
\ruby{着}{つ}きたり。
\ruby{先}{ま}づ
\ruby{常例}{つ|ね}の
\ruby{如}{ごと}く
\ruby[g]{祈念}{きねん}を
\ruby{寵}{こ}めて、
\ruby[g]{少時}{しばし}は
\ruby{何事}{なに|ごと}をも
\ruby{思}{おも}はざりしが、
\ruby{念}{ねん}じ
\ruby{{\換字{終}}}{をは}りて
\ruby{閉}{と}ぢたる
\ruby{眼}{め}を
\ruby{開}{ひら}き、
\ruby{下}{さ}げたる
\ruby{頭}{かうべ}を
\ruby{擡}{あ}げ、
\ruby{身}{み}を
\ruby{起}{おこ}して
\ruby{我}{わ}が
\ruby{居}{ゐ}たる
\ruby[g]{四邊}{あたり}を
\ruby{見}{み}れば、
\ruby{夢}{ゆめ}の
\ruby{裏}{うち}に
\ruby{現}{あらは}れ
\ruby{來}{きた}る
\ruby{人}{ひと}の
\ruby{跫音}{あし|おと}も
\ruby{無}{な}く
\ruby{衣}{きぬ}の
\ruby{音}{おと}もせずして
\ruby[g]{俄然}{にはか}に
\ruby{我}{わ}が
\ruby{前}{まへ}に
\ruby{湧}{わ}き
\ruby{出}{い}づるが
\ruby{如}{ごと}くに、
\ruby{何時}{い|つ}か
\ruby{知}{し}らず、
\ruby{我}{わ}が
\ruby{傍}{かたへ}に
\ruby{跪}{ひざまづ}きて
\ruby{御佛}{みほ|とけ}を
\ruby{念}{ねん}ぜる
\ruby{人}{ひと}ありたり。
\ruby{其}{そ}の
\ruby{柔}{やはら}かに
\ruby{合}{あは}せたる
\ruby{掌}{て}の
\ruby{白々}{しろ|〴〵}と
\ruby{殊{\換字{勝}}氣}{しゆ|しよ|うげ}なる、
\ruby{其}{そ}の
\ruby{領}{えり}のすらりとして
\ruby{見好}{み|よ}き、
\ruby{其}{そ}の
\ruby{髮}{かみ}のめでたき、
\ruby{其}{そ}の
\ruby{{\換字{肩}}}{かた}つきの
\ruby{如何}{い|か}にも
\ruby{女}{をんな}らしく
\ruby{優}{やさ}しき、
\ruby{其}{そ}の
\ruby{横顏}{よこ|がほ}の
\ruby{能}{よ}くは
\ruby{見}{み}えぬながら
\ruby{櫻色}{さくら|いろ}に
\ruby{美}{うる}はしきは、
\ruby{嗚呼我}{あ|ヽ|わ}が
\ruby{相見}{あい|み}んと
\ruby{希}{ねが}ひたりし
\ruby{其}{そ}の
\ruby{人}{ひと}にあらずや。
\ruby{正}{まさ}しく
\ruby[g]{昨日}{きのふ}は
\ruby{見}{み}、
\ruby{今朝}{け|さ}は
\ruby{思}{おも}ひたりし
\ruby{其}{そ}のお
\ruby{龍}{りゆう}ならすや。
\ruby{御佛}{みほ|とけ}を
\ruby{念}{ねん}ぜし
\ruby{今少時}{いま|しば|し}の
\ruby{間}{あひだ}のみ
\ruby{忘}{わす}れ
\ruby{居}{ゐ}たりし
\ruby{其}{そ}の
\ruby{優}{やさ}しく
\ruby{懷}{なつか}しき
\ruby{親切}{しん|せつ}の
\ruby{人}{ひと}ならずや。
\ruby{我}{わ}が
\ruby{淚}{なみだ}を
\ruby{濺}{そヽ}ぎて
\ruby{聞}{き}きし
\ruby[g]{不幸福}{ふしあわせ}の
\ruby{物語}{もの|がたり}を
\ruby{有}{いう}せる
\ruby{悲}{かな}しき
\ruby{薄命}{はく|めい}の
\ruby{婦人}{ふ|じん}ならずや。
\ruby{何}{なん}ぞ
\ruby{其}{そ}の
\ruby{掌}{て}を
\ruby{合}{あは}せて
\ruby{念}{ねん}ぜるさまの
\ruby{哀}{あは}れ
\ruby{深}{ふか}くして、
\ruby{首}{かうべ}を
\ruby{垂}{た}れて
\ruby{思}{おもひ}を
\ruby{凝}{こ}らせるさまの
\ruby{人}{ひと}の
\ruby{心}{こヽろ}を
\ruby{動}{うご}かすや。
\ruby{不思議}{ふ|し|ぎ}にも
\ruby{何時}{い|つ}の
\ruby{間}{ま}にか
\ruby{此堂}{こ|ヽ}には
\ruby{參}{まゐ}り
\ruby{合}{あは}せたる!と
\ruby{思}{おも}ふ
\ruby{時漸}{とき|やうや}くに
\ruby{念}{ねん}じ
\ruby{{\換字{終}}}{をわり}りてか、
\ruby{身}{み}じろぎして
\ruby{靜}{しづか}に
\ruby{女}{をんな}は
\ruby{立上}{たち|あが}りたり。

『…………』

『…………』

\ruby{聲無}{こゑ|な}くして
\ruby{其處}{そ|こ}に
\ruby{呼}{よ}ぶ
\ruby{聲}{こゑ}ありたり、
\ruby{應}{こた}ふる
\ruby{聲}{こゑ}ありたり、
\ruby{言無}{ことば|な}くして
\ruby{其處}{そ|こ}に
\ruby{語}{かた}れる
\ruby{言}{ことば}ありたり、
\ruby{酬}{むく}ひたる
\ruby{言}{ことば}ありたり。


\Entry{其七}

% メモ 校正終了 2024-05-10 2024-06-06
\原本頁{34-6}%
『
\ruby[|g|]{昨日}{きのふ}は
いろ〳〵
\ruby{御厄介}{ご|やく|かい}に、
』

\原本頁{34-7}%
『
いゝえ、
%
\ruby{却}{かへ}つて
\ruby{御{\換字{迷}}惑}{ご|めい|わく}
でございましたらう。
%
おとうさんが
\ruby{彼樣}{あ|ん}な
\ruby{氣合}{き|あひ}の
\ruby{人}{ひと}
だもん
ですから、
%
\ruby{御{\換字{遠}}慮}{ご|えん|りよ}の
\ruby{無}{な}い
こと
ばかり
\ruby{致}{いた}す
やうに
なりまして。
%
\ruby{定}{さだ}めし
\ruby{御蔑視}{お|さげ|すみ}
なすつた
\ruby{事}{こと}
だらうと、
%
\ruby{後}{あと}になつて
\ruby[|g|]{二人}{ふたり}で
\ruby{左樣}{さ|う}
\ruby{申}{まを}して
\ruby{居}{を}りました。
』

\原本頁{35-1}%
『
イヤ、
%
どうして
\ruby{其樣}{そ|ん}なことを
\ruby{思}{おも}ふ
もの
ですか。
%
たゞ
\ruby{私}{わたし}は
\ruby{何}{なん}の
% \原本頁{35-2}\改行%
\ruby[|g|]{因緣}{いはれ}も
\ruby{無}{な}い
\ruby{方}{かた}に
お
\ruby{世話}{せ|わ}を
かけたのが
\ruby{濟}{す}まぬ
\ruby{樣}{やう}な
\ruby{氣}{き}が
\ruby{仕}{し}ます。
%
お
\ruby{會}{あ}ひ
なすつたら
\ruby{彼}{あ}の
\ruby{方}{かた}に
\ruby{宜}{よろ}しく
\ruby{仰}{おつし}あつて
\ruby{下}{くだ}さいまし。
』

\原本頁{35-4}%
『
ホヽ
\ruby{大層}{たい|そう}
\ruby{折目高}{をり|め|だか}に
\ruby{物}{もの}を
\ruby{仰}{おつし}あること。
%
\ruby{彼}{あ}の
\ruby{人}{ひと}は
\ruby{彼樣}{あ|ゝ}した
\ruby{人}{ひと}なの
% \原本頁{35-5}\改行%
ですもの、
%
\ruby{御氣}{お|き}に
お
\ruby{掛}{か}け
なさる
\ruby{事}{こと}は
ありやあ
\ruby{仕}{し}ません。
%
それは
\原本頁{35-6}\改行%
まあ
\ruby{何樣}{ど|う}でも
\ruby{宜}{い}いと
しまして、
%
\ruby{今日}{け|ふ}は
\ruby{何}{なん}でも
\ruby{無}{な}い
\ruby{日}{ひ}で
ございます
のに、
%
どうして
\ruby{今}{いま}
\ruby{頃}{ごろ}
\ruby{御}{お}いでに
なりましたの?。
%
\ruby[|g|]{貴下}{あなた}の
\ruby{拜}{をが}んで
\原本頁{35-7}\改行%
\ruby{居}{ゐ}らしつた
\ruby{御}{お}
\ruby[<j||]{後}{うしろ}
\ruby[||j>]{姿}{すがた}を
\ruby{見}{み}まして、
%
\ruby{妾}{わたし}は
\ruby{初}{はじめ}は
\ruby{氣}{き}の
\ruby{{\換字{迷}}}{まよ}ひ
かと
\ruby{思}{おも}ひ
ましたよ。
%
だつて
\ruby[|g|]{貴下}{あなた}が
\ruby{今}{いま}
\ruby{頃}{ごろ}
\ruby{御}{お}いで
なさらう
\ruby{譯}{わけ}は
\ruby{無}{な}いと
\ruby{思}{おも}ひ
\ruby{定}{き}つて
\原本頁{35-10}\改行%
\ruby{居}{ゐ}た
のです
もの。
』

\原本頁{35-11}%
『
ハヽヽ、
%
\ruby{私}{わたし}は
また
\ruby{何時}{い|つ}の
\ruby{間}{ま}にか
\ruby{私}{わたし}の
\ruby{傍}{そば}に
\ruby[|g|]{貴孃}{あなた}の
\ruby{來}{き}て
\ruby{居}{ゐ}られたのに
\ruby{吃驚}{びつ|くり}
しました。
』

\原本頁{36-2}%
『
ホヽヽ、
%
\ruby[|g|]{貴下}{あなた}が
\ruby{一心}{いつ|しん}に
なつて
\ruby{拜}{をが}んで
\ruby{居}{ゐ}らしつたから、
%
\ruby{吃驚}{びつ|くり}
なさらない
やうにと
\ruby{思}{おも}つて
\ruby{悄々地}{そー|つ|と}
\ruby{妾}{わたし}も
\ruby{拜}{をが}んで
\ruby{居}{を}りましたのよ。
』

\原本頁{36-4}%
『
それは
\ruby{兎}{と}も
\ruby{角}{かく}も、
%
\ruby{今日}{け|ふ}
\ruby{{\換字{若}}}{も}し
\ruby[|g|]{貴孃}{あなた}に
\ruby{御目}{お|め}に
かゝれたら、
%
\ruby{先}{ま}づ
\ruby{第一}{だい|いち}に
\ruby{御話}{お|はなし}を
して、
%
\ruby{悅}{よろこ}んで
\ruby{戴}{いたゞ}きたいと
\ruby{思}{おも}つて
\ruby{居}{を}りましたが、
%
\ruby{御蔭樣}{お|かげ|さま}で
\ruby[||j>]{病}{びやう}
\ruby[||j>]{人}{ にん}も
% \ruby{病人}{びやう|にん}も
\ruby{何樣}{ど|う}やら
\ruby{持直}{もち|なほ}して、
%
\ruby{醫者}{い|しや}が
\ruby{屹度}{きつ|と}% ルビ調整(原本通り)非グループルビ
\ruby{本復}{ほん|ぷく}すると
\ruby{保證}{うけ|あ}つて
\原本頁{36-7}\改行%
\ruby{吳}{く}れた
やうな
ところ
\ruby{迄}{まで}には
\ruby{漕}{こ}ぎ
つけました。
%
もう
\ruby{心配}{しん|ぱい}は
\ruby{無}{な}さゝう
になりました。
%
\ruby{御案}{お|あん}じ
\ruby{下}{くだ}すつた
\ruby{甲{\換字{斐}}}{か|ひ}も
あつて、
%
\ruby{御親切}{ご|しん|せつ}も
まあ
\ruby{屆}{とゞ}いたと% 「屆」「届」 原本通り「屆」
\ruby{申}{まを}す
もので
ございます。
%
ほんとに
\ruby[||j>]{病}{びやう}
\ruby[||j>]{人}{ にん}とは
% \ruby{病人}{びやう|にん}とは
\ruby{御緣}{ご|えん}も
\ruby{薄}{うす}い
\原本頁{36-10}\改行%
\ruby[|g|]{貴卿}{あなた}が、
%
かうして
\ruby{毎日}{まい|にち}
\ruby[g]{々々}{ 〳〵 }
\ruby{歩}{あゆみ}を
\ruby{{\換字{運}}}{はこ}んで
\ruby{下}{くだ}すつて、
%
\ruby{御願}{ご|ぐわん}を
\ruby{御掛}{お|か}け
\ruby{下}{くだ}さつた
\ruby{御芳{\換字{情}}}{お|こゝろ|もち}は
おろそかには
\ruby{思}{おも}ひません、
%
\ruby[||j>]{病}{びやう}
\ruby[||j>]{人}{ にん}が
% \ruby{病人}{びやう|にん}が
\ruby{快}{よ}く
なりましたに
つけても
\ruby{有}{あ}り
\ruby{{\換字{難}}}{がた}く
\ruby{思}{おも}ひます。
%
\ruby{今}{いま}と
いつて
\ruby{今}{いま}は
\ruby{何樣}{ど|う}
\ruby{御禮}{お|れい}の
\原本頁{37-2}\改行%
\ruby{爲}{し}やうも
\ruby{存}{ぞん}じませんが、
%
\ruby{何}{なん}ぞの
\ruby{折}{をり}には
\ruby{屹度}{きつ|と}% ルビ調整(原本通り)非グループルビ
\ruby[|g|]{貴卿}{あなた}の
ために、
%
\ruby[|g|]{貴卿}{あなた}の
\ruby{優}{やさ}しい
\ruby{御芳{\換字{情}}}{お|こゝろ|もち}に
\ruby{對}{たい}して
\ruby{其{\換字{丈}}}{それ|だけ}の
\ruby[|g|]{御{\換字{返}}禮}{おかへし}を
\ruby{爲}{し}やうとは
\ruby{思}{おも}つて
\ruby{居}{を}ります。
%
\ruby[|g|]{貴卿}{あなた}の
\ruby{御芳{\換字{情}}}{お|こゝろ|もち}は
\ruby{長}{なが}く
\ruby{忘}{わす}れません。
』

\原本頁{37-5}%
\ruby{此}{こ}の
\ruby{事}{こと}を
\ruby{言}{い}はん
とおもふ
\ruby{意}{こゝろ}の
\ruby{充}{み}ち
\ruby{滿}{み}ちたるに、
%
\ruby{言葉}{こと|ば}も
\ruby{自}{おのづ}か
\ruby[<g>]{ら勢}{いきほひ}% 行末行頭の境界付近なので特例処置を施す
\ruby{籠}{こも}りて、
%
\ruby{口}{くち}
ばかりの
\ruby{挨拶}{あい|さつ}ならぬは
\ruby{確乎}{しつ|かり}としたる
\ruby{眼}{め}つきにも
\ruby{著}{しる}し
\改行% 校正作業の簡略化のため
。
%
\原本頁{37-7}\改行%
お
\ruby{龍}{りう}は
\ruby{生眞面目}{き|ま|じ|め}に
\ruby{如是}{か|く}
\ruby{云}{い}はれては、
%
\ruby{眞舳}{ま|とも}には
\ruby{當}{あた}り
\ruby{得}{え}ざる
やうの
\ruby{氣}{き}も
\ruby{仕}{し}て、
%
\ruby{安}{やす}からぬ
\ruby{心地}{こゝ|ち}の
\ruby{竊}{ひそか}に
\ruby{爲}{す}ればにや、
%
たゞしは
\ruby{{\換字{又}}}{また}
\ruby{他}{ひと}
\ruby{知}{し}らぬ
\ruby[<j>]{考}{かんがへ}の
\ruby{別}{べつ}に
\ruby{有}{あ}ればにや、
%
\ruby{我}{わ}が
\ruby{祈願}{き|ぐわん}の
\ruby{甲{\換字{斐}}}{か|ひ}の
\ruby{見}{み}えしを
\ruby{悅}{よろこ}ぶ
とも
\ruby{無}{な}く、
%
\ruby{水野}{みづ|の}に
\ruby{斯}{か}ばかり
\ruby{禮}{れい}を
\ruby{云}{い}はれしを
\ruby{嬉}{うれ}しと
\ruby{思}{おも}ふとも
\ruby{見}{み}えず、
%
\ruby{却}{かへ}つて
\ruby{物}{もの}
\ruby{羞}{はぢ}したるが
\ruby{如}{ごと}く
\ruby{沈}{おち}
\ruby{着}{つ}かぬ
\ruby{樣子}{やう|す}に
なりて、
%
\ruby[g]{時々}{とき〴〵}は
\ruby{見}{み}でも
\ruby{宜}{よ}き
\ruby[|g|]{{\換字{遠}}方}{とほく}の
\ruby{額}{がく}などに
ちら〳〵と
\ruby{其}{そ}の
\ruby{美}{うつく}しき
\ruby{眼}{め}を
\ruby{辷}{すべ}らせて
\ruby{聞}{き}き
\ruby{居}{ゐ}し
\改行% 校正作業の簡略化のため
が、

\原本頁{38-3}%
『
まあ
\ruby[|g|]{眞實}{ほんと}に
そりやあ
\ruby{何}{なに}よりの
\ruby{事}{こと}で、
%
こんな
\ruby{嬉}{うれ}しい
ことは
もう
% \原本頁{38-4}\改行%
ございません。
%
どんなにか
\ruby[|g|]{貴下}{あなた}の
\ruby{御嬉}{お|うれ}しい
ことで
ございましやう!。
%
\ruby[|g|]{貴下}{あなた}の
\ruby{御胸}{お|むね}の
\ruby{中}{うち}を
\ruby{思}{おも}つて
\ruby{見}{み}ますと、
%
\ruby{妾}{わたし}も
\ruby{何}{なん}だか
\ruby{嬉}{うれ}し
\ruby{涙}{なみだ}が
\ruby{出}{で}さう
になります。
%
\ruby{何}{なに}も
\ruby{妾}{わたし}
なんぞが
\ruby{御願}{お|ねが}ひ
\ruby{申}{まを}した
から
といふ
\ruby{譯}{わけ}
では
ございます
まいが、
%
あれ
\ruby{程}{ほど}に
\ruby{一心}{いつ|しん}に
なつて
\ruby{御願}{お|ねが}ひ
なすつた
\原本頁{38-8}\改行%
\ruby[|g|]{貴下}{あなた}の
\ruby{御念力}{ご|ねん|りき}
だけでも、
%
\ruby[||j>]{佛}{ほとけ}
\ruby[||j>]{樣}{ さま}が
% \ruby{佛樣}{ほとけ|さま}が
\ruby{打棄}{うつ|ちや}つては
\ruby{御置}{お|お}き
なされ
なくつて、
%
それで
\ruby{五十子}{い|そ|こ}さんが
\ruby{快}{よ}く
\ruby{御}{お}なり
なので
ございましやう。
%
ほんとに
\ruby{五十子}{い|そ|こ}さんは
\ruby{御}{お}
\ruby{羨}{うらや}ましい、
%
\ruby{御不幸}{お|ふし|あはせ}のやうで%「幸福」ここは「は」
\ruby{御幸福}{お|しあ|はせ}の%「幸福」ここは「は」
\ruby{方}{かた}です。
%
\ruby{神樣}{かみ|さま}
\ruby[||j>]{佛}{ほとけ}
\ruby[||j>]{樣}{ さま}の
% \ruby{佛樣}{ほとけ|さま}の
\ruby{御憐愍}{お|あは|れみ}さへ
かゝつて
\ruby{居}{ゐ}る
\ruby{方}{かた}
ですもの!。
』

\原本頁{39-1}%
と
\ruby{末}{すゑ}は
\ruby{誰}{たれ}に
\ruby{云}{い}ふとも
\ruby{無}{な}く
\ruby{言}{い}ひ
たりしが、
%
はしたなしと
\ruby{思}{おも}ひてや
\改行% 校正作業の簡略化のため
、
%
\原本頁{39-2}\改行%
\ruby{調子}{てう|し}を
\ruby{變}{か}へて、

\原本頁{39-3}%
『
\ruby{歸}{かへ}りましたら
\ruby{早{\換字{速}}}{さつ|そく}
\ruby{師匠}{し|ゝやう}にも
\ruby{左樣}{さ|う}
\ruby{申}{まを}しまして、
%
\ruby{御丹精甲{\換字{斐}}}{ご|たん|せい|が|ひ}の
\ruby{有}{あ}つた
\ruby{事}{こと}を
\ruby{聽}{き}かせ
まして
\ruby{悅}{よろこ}ばせ
ましやう。
%
\ruby{定}{さだ}めし
\ruby{屹度}{きつ|と}
\ruby{有}{あ}り
\ruby{{\換字{難}}}{がた}がる
\ruby{事}{こと}で
ございましやう。
』

\原本頁{39-6}%
と
\ruby{言}{ことば}を
\ruby{添}{そ}へたり。

\Entry{其八}

\ruby{際限無}{はて|し|な}く
\ruby{御堂}{み|だう}の
\ruby{内}{うち}に
\ruby{若}{わか}き
\ruby{女}{ひと}と
\ruby{立}{た}ち
\ruby{話}{ばなし}して
\ruby{參詣}{さん|けい}の
\ruby[g]{老若}{らうにやく}に
\ruby{面見}{おも|てみ}られんごとの
\ruby{好}{この}ましからぬ
\ruby{心地}{こゝ|ち}すれば、
\ruby{水野}{みづ|の}は
\ruby{談}{はなし}の
\ruby{切目}{きれ|め}に
\ruby{本尊}{ほん|ぞん}の
\ruby{方}{かた}を
\ruby{一拜}{いつ|ぱい}して、
\ruby{漸}{やうや}く
\ruby{下向}{げ|かう}の
\ruby{路}{みち}に
\ruby{就}{つ}かんとするに、
お
\ruby{龍}{りゆう}は
\ruby{間隔}{あはひ|へだ}たらず
\ruby{連}{つ}れ
\ruby{立}{だ}ちては、
\ruby{遲々}{ち|ゝ}として
\ruby{却}{かへ}つて
\ruby{水野}{みづ|の}の
\ruby{歩}{あゆみ}を
\ruby{澁}{しぶ}らせんとするが
\ruby{如}{ごと}し。

\ruby{御堂}{み|だう}の
\ruby{階段}{きざ|はし}は
\ruby{降}{お}り
\ruby{盡}{つく}しぬ。
\ruby{貴賤行{\換字{交}}}{き|せん|ゆき|ちが}ふ
\ruby{長々}{なが|〳〵}しき
\ruby[g]{石疊}{いしだたみ}の
\ruby{路}{みち}を
\ruby{二人}{ふた|り}は
\ruby{辿}{たど}れり。
こゝは
\ruby{賑}{にぎ}はしからぬ
\ruby{時}{とき}も
\ruby{無}{な}きところとて、ぽつくりの
\ruby{響}{ひゞ}き、
\ruby{雪駄}{せつ|た}の
\ruby{鳴}{なり}、
\ruby[g]{人聲物音一}{ひとごゑものおとひと}つになりて、たゞがや〳〵と
\ruby{譯無}{わけ|な}く
\ruby{騒}{さわ}がしく、
\ruby[g]{七子}{なゝこ}の
\ruby{袖}{そで}は
\ruby{擦}{す}れ
\ruby{{\換字{違}}}{ちが}ふ
\ruby{縮緬}{ちり|めん}の
\ruby{袂}{たもと}、
\ruby{矢}{や}の
\ruby{字}{じ}の
\ruby{帶}{おび}は
\ruby{觸}{さわ}る
\ruby{海軍帽}{かい|ぐん|ばう}、
\ruby{甲家}{かし|こ}の
\ruby[g]{旦那樣}{だんなさま}
\ruby{乙家}{ そ|こ}の
\ruby{奧樣}{おく|さま}、
\ruby{女}{をんな}の
\ruby{兒}{こ}も
\ruby{男}{をとこ}の
\ruby{兒}{こ}も
\ruby{目}{め}まぐるしく
\ruby{往來}{ゆき|き}すれば、
\ruby{{\換字{遂}}}{と}げては
\ruby{我}{われ}も
\ruby{他}{ひと}を
\ruby{見}{み}るに
\ruby{由無}{よし|な}く、
\ruby{他}{ひと}もまた
\ruby{我}{われ}を
\ruby{見}{み}るに
\ruby{由無}{よし|な}く、
\ruby{能}{よ}くは
\ruby{他}{ひと}の
\ruby{談}{はなし}も
\ruby{耳}{みゝ}に
\ruby{入}{い}らねば、
\ruby{我}{わ}が
\ruby{談}{はなし}もまた
\ruby{他}{ひと}には
\ruby{聞}{きこ}えぬなり。
お
\ruby{龍}{りゆう}は
\ruby{此}{こ}の
\ruby{中}{なか}を
\ruby{{\換字{連}}}{つ}れ
\ruby{立}{だ}ちて
\ruby{歩}{ある}きつゝ、ややもすねば
\ruby{獨立}{ひと|りだ}ちて
\ruby{先}{さき}に
\ruby{行}{ゆ}かんとする
\ruby{水野}{みづ|の}を
\ruby{{\換字{追}}}{お}ひかくるやうにして、

『アノ、
\ruby{今日}{け|ふ}は
\ruby{御休}{お|やす}みの
\ruby{日}{ひ}ぢやあございますまいのにネエ。
わざわざ
\ruby{御休}{お|やす}みなすつて
\ruby[g]{御禮參}{おれいまゐ}りにいらしつた
\ruby{譯}{わけ}なの?。
』

と、
\ruby{若}{も}し
\ruby{然}{さ}もあらば、
\ruby{餘}{あま}りに
\ruby{彼}{か}の
\ruby{人}{ひと}の
\ruby{事}{こと}を
\ruby{思}{おも}ふ
\ruby{心}{こゝろ}の
\ruby{{\換字{強}}}{つよ}くして、
\ruby{何}{なに}も
\ruby{彼}{か}も
\ruby{忘}{わす}れ
\ruby{果}{は}てたるが
\ruby{甚}{はなはだ}し
\ruby{{\換字{過}}}{す}ぎたりといふやうに、
\ruby{聊}{いさゝ}か
\ruby{笑}{わらひ}を
\ruby{含}{ふく}んで
\ruby{問}{と}ひかけたり。

\ruby{先刻}{さ|き}にも
\ruby{受}{う}けたる
\ruby{問}{とひ}ながら、
\ruby{答}{こた}ふるも
\ruby{煩}{わづら}はしと
\ruby{思}{おも}ひて
\ruby{顧}{かへりみ}ざりしが、
\ruby{今}{いま}
\ruby{{\換字{又}}}{また}
\ruby{如是}{かゝ|る}
\ruby{樣子}{やう|す}に
\ruby{問}{と}はれては
\ruby{默}{だま}りても
\ruby{居難}{ゐ|がた}く、

『ハヽヽ、まさか
\ruby{左樣}{さ|う}いふ
\ruby{譯}{わけ}でも
\ruby{無}{な}いのですが、
\ruby[g]{丁度}{ちやうど}\ %空白有り
\ruby{職務}{つと|め}は
\ruby{辭}{よ}して
\ruby{仕舞}{し|ま}つたので、それで
\ruby{萬一}{ひよ|つと}したら
\ruby{貴卿}{あな|た}に
\ruby{御目}{お|め}にかゝれやうかといふ
\ruby{考}{かんがへ}も
\ruby{有}{あ}つて、
\ruby[g]{平日}{いつも}よりは
\ruby{早}{はや}く
\ruby{出}{で}て
\ruby{來}{き}たのです。
\ruby{仕合}{しあ|わせ}に
\ruby{巧}{うま}く
\ruby{御目}{お|め}にかゝる
\ruby{事}{こと}が
\ruby{出來}{で|き}て、
\ruby{聞}{き}いて
\ruby{戴}{いたゞ}かうと
\ruby{思}{おも}つて
\ruby{居}{ゐ}たことも
\ruby{聞}{き}いて
\ruby{戴}{いたゞ}いたので、
\ruby{悉皆}{すつ|かり}
\ruby{思}{おも}つた
\ruby{通}{とほ}りになりましたが、これも
\ruby{下}{くだ}らない
\ruby{職務}{つと|め}なんか
\ruby{廢}{よ}して
\ruby{仕舞}{し|ま}つた
\ruby{故}{せゐ}でしやう、ハヽハヽ。
』

と
\ruby{輕}{かろ}く
\ruby{打笑}{うち|わら}ひたり。

\ruby{水野}{みづ|の}は
\ruby{輕}{かろ}く
\ruby{打笑}{うち|わら}ひたれども、
\ruby{職務}{つと|め}を
\ruby{棄}{す}てたりといふ
\ruby{事}{こと}の、
お
\ruby{龍}{りう}には
\ruby{輕}{かろ}からず
\ruby{聞}{きこ}えやしけん、
\ruby{其}{そ}の
\ruby{眉}{まゆ}を
\ruby{顰}{ひそ}めて
\ruby{心配}{しん|ぱい}げに、

『お
\ruby{職務}{つと|め}を
\ruby{御止}{お|よ}しなすつたのですつて!。
\ruby{何故其樣}{な|ぜ|そ|ん}なことをなすつたの?。
\ruby{何}{なに}も
\ruby{御困}{お|こま}りなさる
\ruby{樣}{やう}な
\ruby{事}{こと}は
\ruby{御有}{お|あ}んなさりやあ
\ruby{仕}{し}ますまいけれどもネエ、
\ruby{何}{なん}だつて
\ruby{其樣}{そ|ん}な
\ruby{事}{こと}をなさいましたの。
そんな
\ruby{事}{こと}をなさら
\ruby{無}{な}くてもぢやあ
\ruby{有}{あ}りませんか。
』

と
\ruby{滿腔}{まん|こう}の
\ruby{同{\換字{情}}}{どう|じやう}より
\ruby{私}{ひそか}に
\ruby{生活}{せい|くわつ}の
\ruby{{\換字{道}}}{みち}の
\ruby[g]{便宜惡}{たよりあし}かるべきを
\ruby{氣{\換字{遣}}}{き|づか}ふものの
\ruby{如}{ごと}し。

『ナニ、
\ruby{別}{べつ}に
\ruby{無理}{む|り}に
\ruby{辭}{や}めたいと
\ruby{思}{おも}つたのでも
\ruby{無}{な}いのですけれども、
\ruby{辭}{や}めさせられて
\ruby{見}{み}れば
\ruby{仕方}{し|かた}がないわけですもの。
』

『だつて、
\ruby{何故}{な|ぜ}ネエ?。
\ruby{餘}{あんま}り
\ruby[g]{御不勤}{ごふづとめ}でもなすつたの?。
』

『イヽヤ、そんな
\ruby{事}{こと}は
\ruby{決}{けつ}して
\ruby{爲}{せ}ん
\ruby{私}{わたし}です。
』

『ぢやあ
\ruby{其樣}{そ|ん}な
\ruby{事}{こと}になる
\ruby{譯}{わけ}が
\ruby{無}{な}いぢやあ
\ruby{有}{あ}りませんか。
もしそれぢやあ
\ruby{萬一}{ひよ|つと}したら
\ruby{五十子}{い|そ|こ}さんの
\ruby{事}{こと}で
\ruby{{\換字{評}}{\換字{判}}}{ひやう|ばん}でも
\ruby{立}{た}つて、
\ruby{其}{そ}の
\ruby{爲}{ため}といふやうな
\ruby{譯}{わけ}ぢやあ
\ruby{無}{な}くつて?。
』

『ハヽ、
\ruby{云}{い}はゞ
\ruby{其樣}{そ|ん}な
\ruby{事}{こと}の
\ruby{爲}{ため}なんでしやうが、
\ruby{何樣}{ど|う}でも
\ruby{其樣}{そ|ん}な
\ruby{事}{こと}は
\ruby{構}{かま}やあ
\ruby{仕}{し}ません。
まさか
\ruby{下}{くだ}らない
\ruby{職務}{つと|め}を
\ruby{止}{よ}したからといつて
\ruby{困}{こま}りも
\ruby{仕}{し}ますまいから、いつそ
\ruby{卑小}{け|ち}な
\ruby{職務}{つと|め}なんかに
\ruby{縛}{しば}られない
\ruby{今日}{け|ふ}の
\ruby{方}{はう}が
\ruby{宜}{よ}い
\ruby{心持}{こゝろ|もち}が
\ruby{仕}{し}ます。
』

『そりやあ
\ruby{左樣}{さ|う}でも
\ruby{御有}{お|あ}んなさりましやうが、でもまあ
\ruby{差當}{さし|あた}つて…………。
ほんたうなら
\ruby{五十子}{い|そ|こ}さんの
\ruby{御母}{お|つか}さんが
\ruby{何樣}{ど|う}にでも
\ruby{仕}{し}てあげるのが
\ruby{{\換字{道}}}{みち}なんですけれども。
』

\ruby{何}{なに}をか
\ruby{思}{おも}ふ、
お
\ruby{龍}{りゆう}は
\ruby{言}{い}ひ
\ruby{澱}{よど}んで
\ruby{考}{かんがへ}に
\ruby{沈}{しづ}みしが、
\ruby{水野}{みづ|の}は
\ruby{却}{かへ}つて
\ruby{冴}{さえ}
\ruby{冴}{ざえ}として、

『ハヽ、
\ruby{決}{けつ}して
\ruby{何}{なに}も
\ruby{心配}{しん|ぱい}して
\ruby{下}{くだ}さらんでも
\ruby{可}{い}いのです、
\ruby{考案}{かん|がへ}があるのですから。
\ruby{信心}{しん|〴〵}を
\ruby{仕}{し}て、
\ruby{愚}{ぐ}だと
\ruby{云}{い}はれて、
\ruby{擯斥}{ひん|せき}されて
\ruby{仕舞}{し|ま}つた、こんな
\ruby{馬鹿}{ば|か}でも、
\ruby{男}{をとこ}は
\ruby{矢張}{やつ|ぱ}り
\ruby{男}{をとこ}ですからネ。
イヤ
\ruby{此處}{こ|ゝ}で
\ruby{失敬}{しつ|けい}しましやう、
\ruby{左樣}{さ|やう}なら。
』

と
\ruby{書生風}{しよ|せい|ふう}に
\ruby{淡泊}{たん|ぱく}に
\ruby{挨拶}{あい|さつ}して
\ruby{別}{わか}れ
\ruby{去}{さ}らんとす。
\ruby{何時}{い|つ}か
\ruby{石路}{せき|ろ}は
\ruby{{\換字{既}}}{すで}に
\ruby{歩}{あゆ}み
\ruby{盡}{つく}せるなり。

\ruby{何}{なに}にか
\ruby{心}{こゝろ}を
\ruby{奪}{と}られ
\ruby{居}{ゐ}し
お
\ruby{龍}{りゆう}は、
\ruby{水野}{みづ|の}の
\ruby[g]{告別}{わかれ}の
\ruby{辭}{ことば}に
\ruby{打慌}{うち|あは}てゝ、

『ぢやあ
\ruby{明日}{あし|た}また
\ruby{御眼}{お|め}にかゝれますの?。
』

と
\ruby{辛}{から}くも
\ruby{一句問}{いつ|く|と}ひかくれば、
\ruby{{\換字{既}}}{すで}に
\ruby{十餘歩}{じう|よ|ほ}を
\ruby{隔}{へだ}たりし
\ruby{水野}{みづ|の}は
\ruby{無言}{むご|ん}に
\ruby{點頭}{うな|づ}きて、
\ruby{{\換字{情}}無}{つれ|な}きが
\ruby{如}{ごと}く
\ruby{其儘{\換字{終}}}{その|まゝ|つひ}に
\ruby{東}{ひがし}に
\ruby{去}{さ}りたり。

\ruby{去}{さ}り
\ruby{去}{さ}る
\ruby{百歩餘}{ひや|くほ|あま}りにして、
\ruby{水野}{みづ|の}は
\ruby{徐}{おもむ}ろに
\ruby{首}{かうべ}を
\ruby{囘}{かへ}して
\ruby{見}{み}れば、
\ruby{人}{ひと}の
\ruby{繁}{しげ}く
\ruby{車}{くるま}の
\ruby{煩}{うるさ}きが
\ruby{中}{なか}に
\ruby{猶悠然}{なほ|いう|ぜん}と
\ruby{立}{た}つて、
\ruby{我}{わ}が
\ruby{方}{かた}をや
\ruby{見{\換字{送}}}{み|おく}り
\ruby{居}{ゐ}たる、
お
\ruby{龍}{りゆう}の
\ruby{面}{おもて}の
\ruby{花}{はな}と
\ruby{白}{しろ}きが
\ruby{仄}{ほの}かに
\ruby{見}{み}えたり。


\Entry{其九}

\ruby{疾病}{やま|ひ}のやうやく
\ruby{快}{よ}くなり
\ruby{行}{ゆ}くさまを、
\ruby{薄紙}{うす|がみ}を
\ruby{剝}{は}ぐが
\ruby{如}{ごと}しとは
\ruby{誰}{たれ}が
\ruby{云}{い}ひ
\ruby{初}{そ}めけん、さしもに
\ruby{一時}{いち|じ}は
\ruby{危}{あやふ}かりし
\ruby{五十子}{い|そ|こ}の、
\ruby{天壽}{てん|じゆ}いまだ
\ruby{盡}{つ}きねば
\ruby{人力効}{じん|りよく|かひ}ありて、
\ruby{實}{げ}に
\ruby{此頃}{この|ごろ}は
\ruby{薄紙}{うす|がみ}を
\ruby{剝}{は}ぐが
\ruby{如}{ごと}く、
\ruby{日}{ひ}に
\ruby{日}{ひ}に
\ruby{少}{すこ}しづつ
\ruby{快}{よ}くなりゆけば、
\ruby{年齡}{と|し}の
\ruby{勢}{いきほひ}も
\ruby{藥餌}{やく|じ}の
\ruby{能}{のう}もこゝに
\ruby{現}{あらは}れ
\ruby{來}{きた}りて、
\ruby{一陽來復}{いち|やう|らい|ふく}の
\ruby{機待}{をり|ま}ち
\ruby{得}{え}たる
\ruby{若樹}{わか|ぎ}の、
\ruby{{\換字{猶}}}{なほ}
\ruby{雪}{ゆき}には
\ruby{籠}{こ}められ
\ruby{氷}{こほり}には
\ruby{{\換字{鎖}}}{とざ}されながらも、
\ruby{{\換字{既}}}{すで}に
\ruby{漸}{やうや}く
\ruby{芽}{め}をも
\ruby{蕾}{いばら}をも
\ruby{含}{ふく}み
\ruby{居}{ゐ}て、やがて
\ruby{春風}{はる|かぜ}の
\ruby{渡}{わた}らん
\ruby{曉}{あした}に
\ruby{誇}{ほこ}らんとするが
\ruby{如}{ごと}く、
\ruby{窶}{やつ}れ
\ruby{果}{は}てたるが
\ruby{中}{なか}にも、はや
\ruby{行末}{ゆく|すゑ}の
\ruby{榮}{さか}ゆる
\ruby{色}{いろ}は
\ruby{微見}{ほの|み}ゆるに
\ruby{至}{いた}れり。

されば
\ruby{愁}{うれひ}の
\ruby{雲厚}{くも|あつ}く
\ruby{蔽}{おほ}ひて、
\ruby{火}{ひ}の
\ruby{{\換字{消}}}{き}えたるやうに
\ruby{陰氣}{いん|き}なりし
\ruby{此}{こ}の
\ruby{家}{いへ}の、
\ruby{五十子}{い|そ|こ}が
\ruby{面}{おもて}の
\ruby{色}{いろ}の
\ruby{美}{よ}くなり
\ruby{行}{ゆ}くに
\ruby{{\換字{連}}}{つ}れて、
\ruby{一室}{ひと|ま}の
\ruby{中}{うち}は
\ruby{日}{ひ}の
\ruby{出}{い}でし
\ruby{如}{ごと}く
\ruby{賑}{にぎ}やかになり、
\ruby{先}{ま}づ
\ruby{年少}{とし|わか}の
\ruby{松之助}{まつ|の|すけ}より
\ruby{何}{なん}ぞに
\ruby{付}{つ}けて
\ruby{笑聲}{わら|ひ}を
\ruby{洩}{もら}せば、
\ruby{元氣溢}{げん|き|あふ}るゝばかりの
\ruby{看護婦}{かん|ご|ふ}も
\ruby{折{\換字{節}}}{をり|ふし}は
\ruby{高笑}{たか|わら}ひして、こゝは
\ruby{人々}{ひと|〴〵}の
\ruby{機{\換字{嫌}}}{き|げん}も
\ruby{好}{よ}く、
\ruby{談話聲}{はな|し|ごゑ}も
\ruby{冴}{さ}ゆる、
\ruby{陽氣}{やう|き}の
\ruby{家}{いへ}と
\ruby{打}{うつ}て
\ruby{變}{かは}りたり。

\ruby{體{\換字{温}}}{たい|おん}は
\ruby{高下少}{かう|げす|くな}くなりて
\ruby{漸}{やうや}く
\ruby{{\換字{平}}常}{つ|ね}に
\ruby{復}{ふく}さんとするの
\ruby{勢}{いきほひ}を
\ruby{示}{しめ}し、
\ruby{脉搏}{みやく|はく}は
\ruby{{\換字{猶}}}{なほ}
\ruby{{\換字{弱}}}{よわ}けれども
\ruby{走}{はし}らず
\ruby{澁}{しぶ}らずして
\ruby{危險}{き|けん}の
\ruby{虞}{おそれ}の
\ruby{{\換字{既}}}{すで}に
\ruby{去}{さ}りたるを
\ruby{現}{あらは}せり。

\ruby{恐}{おそ}ろしき
\ruby{熱}{ねつ}に
\ruby{惱}{なや}める
\ruby{日}{ひ}の
\ruby{少}{すくな}からざりしかば、
\ruby{肉}{にく}は
\ruby{落}{お}ち
\ruby{骨}{ほね}は
\ruby{立}{た}ちて、
\ruby{今}{いま}
\ruby{{\換字{猶}}}{なほ}
\ruby{一人}{ひと|り}しては
\ruby{何}{なん}とする
\ruby{事}{こと}も
\ruby{叶}{かな}はぬほどに
\ruby{衰}{おとろ}へ
\ruby{果}{は}てたれど、
\ruby{一昨日}{を|とゝ|ひ}より
\ruby{昨日}{きの|ふ}は
\ruby{好}{よ}く、
\ruby{昨日}{きの|ふ}より
\ruby{今日}{け|ふ}は
\ruby{確乎}{しつ|かり}として、
\ruby{病勢}{びやう|せい}の
\ruby{烈}{はげ}しかりしに
\ruby{纎{\換字{弱}}}{か|よわ}き
\ruby{婦人}{をん|な}の
\ruby{身}{み}なれば
\ruby{衰{\換字{弱}}}{すゐ|じやく}こそ
\ruby{{\換字{尋}}常越}{な|み|こ}えて
\ruby{甚}{はなはだ}しけれ、これより
\ruby{五六週間}{ご|ろく|しう|かん}も
\ruby{立}{た}たば、
\ruby{必}{かなら}ず
\ruby{病}{や}まぬ
\ruby{徃日}{むか|し}の
\ruby{健康}{すこ|やかさ}に
\ruby{囘}{かへ}つて、
\ruby{日々}{にち|〳〵}の
\ruby{勤務}{つ|とめ}を
\ruby{執}{と}るに
\ruby{至}{いた}るを
\ruby{得}{う}べしとの
\ruby{相良}{さが|ら}
\ruby{尾竹}{を|だけ}の
\ruby{言葉}{こと|ば}も
\ruby{僞}{いつは}りなるまじく
\ruby{思}{おも}はれぬ。

\ruby{五十子}{い|そ|こ}が
\ruby{狀態是}{やう|す|かく}の
\ruby{如}{ごと}くなれば、
\ruby{松之助}{まつ|の|すけ}は
\ruby{自}{みづか}ら
\ruby{熱}{あつ}き
\ruby{{\換字{乳}}}{ちゝ}を
\ruby{薦}{すゝ}めたる
\ruby{或曉}{ある|あさ}、
\ruby{其}{そ}の
\ruby{姊}{あね}の
\ruby{面}{おもて}をつく〴〵と
\ruby{打護}{うち|まも}りて、

『もう
\ruby{大丈夫}{だい|ぢやう|ぶ}だ、もう
\ruby{大丈夫}{だい|ぢやう|ぶ}だ!。
ほんとに
\ruby{怖}{こは}いと
\ruby{思}{おも}つた
\ruby{時}{とき}も
\ruby{有}{あ}つたけれ
\ruby{共}{ども}、とう〳〵
\ruby{僕}{ぼく}の
\ruby{姊}{ねえ}さんは
\ruby{僕}{ぼく}の
\ruby{姊}{ねえ}さんになつた!。
』

と
\ruby{無邪氣}{む|じや|き}に
\ruby{叫}{さけ}び
\ruby{出}{だ}して
\ruby{笑}{わら}ひ
\ruby{{\換字{悅}}}{よろこ}び、
\ruby{相良}{さが|ら}が
\ruby{手}{て}より
\ruby{來}{きた}れる
\ruby{看護婦}{かん|ご|ふ}の
\ruby{芳野}{よし|の}は、
\ruby{或夜體{\換字{温}}表}{ある|よ|たい|おん|へう}を
\ruby{記}{しる}し
\ruby{{\換字{終}}}{をは}れる
\ruby{次}{ついで}に、
\ruby{其表}{その|へう}をつく〴〵
\ruby{眺}{なが}めながら、

『マア
\ruby{宜}{よ}かつた
\ruby{事}{こと}、もう
\ruby{如是}{こ|う}いふ
\ruby{樣子}{やう|す}になつて
\ruby{來}{く}れば
\ruby{心配}{しん|ぱい}は
\ruby{無}{な}い。
\ruby{一時}{いち|じ}はほんとに
\ruby{何樣}{ど|う}なるかと
\ruby{思}{おも}つたけれど、マア
\ruby{患者}{くわん|じや}さんも
\ruby{幸福}{しあ|はせ}、
\ruby{私}{わたし}も
\ruby{幸福}{しあ|はせ}で、
\ruby{患者}{くわん|じや}さんは
\ruby{辛棒甲斐}{しん|ぼう|が|ひ}があり、
\ruby{私}{わたし}は
\ruby{看護甲斐}{かん|ご|が|ひ}がある
\ruby{事}{こと}になつて、
\ruby{相良}{さが|ら}さんに
\ruby{對}{むか}つても
\ruby{面目}{めん|ぼく}がある!。
』

と
\ruby{獨語}{ひとり|ご}ち、
\ruby{{\換字{又}}}{また}、
\ruby[g]{吉右衛門}{きちゑもん}に
\ruby{命}{いひつ}けられて
お
\ruby{澤}{さは}が
\ruby{許}{もと}にありて
\ruby{人々}{ひと|〴〵}が
\ruby{爲}{ため}に
\ruby{雜事}{ざつ|じ}の
\ruby{勞}{らう}を
\ruby{執}{と}れる
\ruby{下婢}{か|ひ}の
お
\ruby{鹽}{しほ}も、

『
\ruby{水野}{みづ|の}さんの
\ruby{念力}{おも|ひ}だけでも
\ruby{治癒}{な|ほ}ると
\ruby{人}{ひと}が
\ruby{言}{い}つたゞが、ほんに
\ruby{可怖}{おつ|かな}いもんだ!とう〳〵
\ruby{治癒}{な|ほ}るだあ。
\ruby{病}{やまひ}の
\ruby{高}{こう}じた
\ruby{時}{とき}あハア、
\ruby{何樣}{ど|う}しても
\ruby{彼世}{あの|よ}へ
\ruby{辷}{すべ}り
\ruby{{\換字{込}}}{こ}みさうな
\ruby{樣}{やう}な
\ruby{顏}{かほ}を
\ruby{仕}{し}て
\ruby{御座}{ご|ざ}つた
\ruby{彼}{あ}の
\ruby{人}{ひと}------
\ruby{彼}{あ}の
\ruby{危}{あぶな}かつた
\ruby{人}{ひと}を
\ruby{取}{と}り
\ruby{止}{と}めることが
\ruby{出來}{で|き}たかと
\ruby{思}{おも}ふと
\ruby{不思議}{ふ|し|ぎ}でならない。
おらあハア
\ruby{始}{はじ}めて
\ruby{人}{ひと}の
\ruby{念力}{ねん|りき}といふ
\ruby{可怖}{おつ|かな}いものを
\ruby{目}{め}の
\ruby{前}{まへ}に
\ruby{見}{み}て
\ruby{魂{\換字{消}}}{たま|げ}た。
\ruby{醫者業}{い|しや|わざ}ぢやあ
\ruby{無}{な}いだ、
\ruby{全}{まつた}く
\ruby{醫者業}{い|しや|わざ}ちやあ
\ruby{無}{な}いだ!。
』

と
\ruby{下司}{げ|す}の
\ruby{常}{つね}とて
\ruby{言葉}{こと|ば}こそ
\ruby{多}{おほ}けれ、これもまた
\ruby{五十子}{い|そ|こ}が
\ruby{囘復}{くわい|ふく}を
\ruby{{\換字{悅}}}{よろこ}べる
\ruby{數}{かず}には
\ruby{洩}{も}れぬに、たゞ
\ruby{彼}{か}の
\ruby{強慾}{がう|よく}の
お
\ruby{澤}{さは}
\ruby{婆}{ばゞ}のみは、

『
\ruby{生}{い}きたつて
\ruby{面白}{おも|しろ}いとも
\ruby{定}{きま}つて
\ruby{居}{ゐ}ない
\ruby{世}{よ}の
\ruby{中}{なか}に、とう〳〵
\ruby{彼}{あ}の
\ruby{人}{ひと}も
\ruby{生殘}{いき|のこ}つたやうだ!。
まだ
\ruby{業}{ごふ}が
\ruby{滅}{めつ}しないので
\ruby{死}{し}ねないと
\ruby{見}{み}えるだ!。
\ruby{水野}{みづ|の}の
\ruby{世話}{せ|わ}で
\ruby{死}{し}なゝかつた
\ruby{{\換字{丈}}}{だけ}に、
\ruby{却}{かへ}つて
\ruby{今後}{これ|から}が
\ruby{面倒}{めん|だう}らしい。
\ruby{無錢}{た|ゞ}で
\ruby{買}{か}へるものは
\ruby{一}{ひと}つも
\ruby{無}{な}いだ!、
\ruby{借}{かり}は
\ruby{{\換字{返}}}{かへ}さずには
\ruby{眩度濟}{きつ|と|す}まないだ!。
\ruby{物}{もの}を
\ruby{取}{と}れば
\ruby{代}{かは}りを
\ruby{與}{や}る、
\ruby{借}{か}りた
\ruby{茶}{ちや}は
\ruby{茶}{ちや}で
\ruby{{\換字{返}}}{かへ}す、
\ruby{酒}{さけ}は
\ruby{酒}{さけ}で
\ruby{{\換字{返}}}{かへ}す!。
\ruby{人}{ひと}の
\ruby{親切}{しん|せつ}は
\ruby{何}{なん}で
\ruby{{\換字{返}}}{かへ}す?。
\ruby{生命}{いの|ち}の
\ruby{恩}{おん}は
\ruby{何}{なん}で
\ruby{{\換字{返}}}{かへ}す?。
\ruby{生}{い}きたが
\ruby{彼}{あ}の
\ruby{人}{ひと}の
\ruby{幸福}{しあ|はせ}だか
\ruby{何樣}{ど|う}だか?。
\ruby{病氣}{やま|ひ}は
\ruby{無}{な}くなつたゞらうが、
\ruby{可厭}{い|や}なものが
\ruby{殘}{のこ}らう!。
\ruby{死損}{しに|そこな}つて
\ruby{氣}{き}の
\ruby{毒}{どく}の
\ruby{樣}{やう}な!。

\ruby{治}{なほ}つてから
\ruby{彼}{あ}の
\ruby{人}{ひと}が
\ruby{何樣}{ど|ん}な
\ruby{氣持}{き|もち}がさつしやらうかサ!。
\ruby{業}{ごふ}が
\ruby{盡}{つ}きないだ、
\ruby{業}{ごふ}が
\ruby{殘}{のこ}つたゞ!、
\ruby{何癒}{なに|なほ}ることが
\ruby{芽出度}{め|で|た}いに
\ruby{決}{きま}るかい!。
』

と、
\ruby{頻}{しき}りに
\ruby{松之助}{まつ|の|すけ}やら
\ruby{看護婦}{かん|ご|ふ}やらの
\ruby{尾}{を}に
\ruby{從}{つ}いて
\ruby{{\換字{悅}}}{よろこ}べる
お
\ruby{鹽}{しほ}に
\ruby{對}{むか}つて、
\ruby{例}{れい}の
\ruby{如}{ごと}く
\ruby{憎}{にく}さげに
\ruby{冷笑}{あざ|わら}ひて
\ruby{言}{い}ひ
\ruby{聞}{き}かせたり。


\Entry{其十}

\ruby{凝}{こ}れるものを
\ruby{觀}{み}れば
\ruby{石}{いし}あり
\ruby{璧}{たま}あり。
\ruby{生}{お}ふるものを
\ruby{觀}{み}れば
\ruby{雜草}{ざつ|さう}あり
\ruby{百合}{ゆ|り}あり。
\ruby{同}{おな}じ
\ruby{人間}{ひ|と}にも、
\ruby{一生}{いつ|しやう}おろかしく
\ruby{衣食}{いゝ|し}のために
\ruby{{\換字{逐}}}{お}ひ
\ruby{使}{つか}はれて、
\ruby{{\換字{猶}}}{なほ}
\ruby{其}{そ}の
\ruby{足}{た}らざるを
\ruby{憂}{うれ}ふる
\ruby{額}{ひたひ}の
\ruby{皺}{しわ}を
\ruby{深々}{ふか|〳〵}と
\ruby{疊}{たゝ}み、おのが
\ruby{働}{はたら}きの
\ruby{無}{な}きは
\ruby{省}{かへり}みずに、
\ruby{他人}{ひ|と}を
\ruby{恨}{うら}み
\ruby{世}{よ}を
\ruby{謗}{そし}りて
\ruby{甲{\換字{斐}}無}{か|ひ|な}く
\ruby{悶}{もだ}えながら
\ruby{老境}{お|い}に
\ruby{入}{い}るもあり、
\ruby{{\換字{又}}}{また}
\ruby{生}{うま}れつきの
\ruby{心}{こゝろ}の
\ruby{{\換字{丈}}高}{たけ|たか}く
\ruby{胸}{むね}の
\ruby{海濶}{うみ|ひろ}くして、
\ruby{此}{こ}のむづかしき
\ruby{世}{よ}に
\ruby{身}{み}の
\ruby{取}{と}り
\ruby{置}{お}き
\ruby{拙}{つたな}からず、
\ruby{憂}{う}さも
\ruby{苦}{くる}しさも、するりと
\ruby{切}{き}り
\ruby{拔}{ぬ}けて、
\ruby{屈託}{くつ|たく}せぬ
\ruby{顏色}{かほ|つき}の
\ruby{何時}{い|つ}も
\ruby{{\換字{若}}々}{わか|〳〵}と、
\ruby{雲}{くも}より
\ruby{上}{うへ}に
\ruby{居}{ゐ}る
\ruby{月}{つき}の、
\ruby{澄}{すま}し
\ruby{{\換字{返}}}{かへ}つて
\ruby{暮}{くら}すやうなる
\ruby{優}{すぐ}れ
\ruby{者}{もの}もあるなり。

お
\ruby{龍}{りう}は
\ruby{自己}{お|の}が
\ruby{身}{み}の
\ruby{上}{うへ}の
\ruby{今}{いま}の
\ruby{果敢無}{は|か|な}さを
\ruby{羞}{はぢ}らひて、
\ruby{我}{わ}が
\ruby{口}{くち}より
\ruby{我}{わ}が
\ruby{友}{とも}なりとは
\ruby{憚}{はゞか}りて
\ruby{云}{い}はねど、
\ruby{彼方}{かな|た}は
\ruby{何處}{ど|こ}までも
\ruby{隔意無}{へだて|ごゝろ|な}く、
お
\ruby{龍}{りう}を
\ruby{友}{とも}とも
\ruby{妹}{いもと}とも
\ruby{待{\換字{遇}}}{あし|ら}ひて、
\ruby{親身}{しん|み}も
\ruby{及}{およ}ばず
\ruby{優}{やさ}しくする
お
\ruby{彤}{とう}といへる
\ruby{一美人}{いち|び|じん}あり。

\ruby{叔母}{を|ば}が
\ruby{無理壓制}{む|り|おし|つけ}の
\ruby{婿取沙汰}{むこ|とり|ざ|た}を
\ruby{厭}{いと}ひて、
\ruby{駿府}{すん|ぷ}を
\ruby{脫}{ぬ}け
\ruby{出}{い}でゝ
\ruby{東京}{とう|きやう}に
\ruby{來}{きた}りし
\ruby{時}{とき}、
お
\ruby{龍}{りう}が
\ruby{先}{ま}づ
\ruby{頼}{たよ}りしは
\ruby{此女}{この|ひと}にして、
お
\ruby{龍}{りう}と
\ruby{共}{とも}に
\ruby{淺草}{あさ|くさ}に
\ruby{{\換字{遊}}}{あそ}びし
\ruby{日}{ひ}
\ruby{水野}{みづ|の}に
\ruby{{\換字{遇}}}{あ}ひて、
\ruby{水野}{みづ|の}をして
\ruby{其}{そ}の
\ruby{美}{び}に
\ruby{驚}{おどろ}かしめしも
\ruby{此女}{この|をんな}なりけるなり。

お
\ruby{彤}{とう}が
\ruby{身{\換字{分}}}{み|ぶん}を
\ruby{問}{と}へば、
\ruby{世}{よ}に
\ruby{聞}{きこ}えたる
\ruby{一代{\換字{分}}限}{いち|だい|ぶ|げん}の
\ruby{筑波何某}{つく|ば|なに|がし}といへる
\ruby{六十男}{む|そ|をとこ}の
\ruby{外妾}{ぐわい|せう}に
\ruby{{\換字{過}}}{す}ぎぬなり。
\ruby{然}{さ}なり、
\ruby{藥研堀}{や|げん|ぼり}
\ruby{附{\換字{近}}}{あた|り}に
\ruby{數寄}{す|き}を
\ruby{凝}{こ}らせる
\ruby{家}{いへ}を
\ruby{構}{かま}へて、
\ruby{賑}{にぎ}やかなるが
\ruby{中}{なか}に
\ruby{靜閑}{しづ|か}に
\ruby{暮}{くら}すほどの
\ruby{贅澤}{ぜい|たく}を
\ruby{縱}{ほしいまゝ}にし、
\ruby{美衣}{び|い}を
\ruby{纒}{まと}ひ
\ruby{美饌}{び|せん}を
\ruby{口}{くち}にし、
\ruby{萬般}{よろ|づ}
\ruby{幸福}{しあ|わせ}に
\ruby{世}{よ}を
\ruby{經}{ふ}るとはいへ、
\ruby{實}{まこと}に
\ruby{其}{そ}の
\ruby{身{\換字{分}}}{み|ぶん}を
\ruby{問}{と}へば
\ruby{外妾}{めか|け}には
\ruby{{\換字{過}}}{す}ぎぬなり。

されどお
\ruby{彤}{とう}は
\ruby{人}{ひと}の
\ruby{正室}{つ|ま}たるを
\ruby{得}{え}ざるが
\ruby{故}{ゆゑ}に
\ruby{身}{み}を
\ruby{日陰者}{ひ|かげ|もの}の
\ruby{其位}{そ|れ}に
\ruby{安}{やす}んぜるにはあらず。
\ruby{今}{いま}を
\ruby{去}{さ}ること
\ruby{七年}{しち|ねん}ほど
\ruby{{\換字{前}}}{まへ}の
\ruby{事}{こと}なりき。
\ruby{筑波}{つく|ば}が
\ruby{其}{そ}の
\ruby{正妻}{つ|ま}を
\ruby{失}{うしな}ひし
\ruby{時}{とき}、
\ruby{面}{おもて}の
\ruby{美}{うつく}しさばかりに
\ruby{{\換字{迷}}}{まよ}ひ
\ruby{溺}{おぼ}るゝがごとき
\ruby{痴漢}{おろか|もの}ならぬ
\ruby{筑波}{つく|ば}は、よく〳〵
\ruby{見定}{み|さだ}めたるところやありけん、
お
\ruby{彤}{とう}を
\ruby{引上}{ひき|あ}げて
\ruby{正室}{つ|ま}とせんとは
\ruby{云}{い}ひたりしなり。
されば
\ruby{其時}{その|とき}
お
\ruby{彤}{とう}にして
\ruby{{\換字{強}}}{し}ひて
\ruby{辭}{いな}み
\ruby{立}{だて}だにせざりしならば、
\ruby{今}{いま}は
\ruby{此}{こ}の
\ruby{世}{よ}の
\ruby{表面}{おも|て}に
\ruby{立}{た}ちて、
\ruby{立派}{りつ|ぱ}に
\ruby{筑波夫人}{つく|ば|ふ|じん}と
\ruby{崇}{あが}め
\ruby{仰}{あふ}がれ、
\ruby{夫}{おつと}の
\ruby{勢力}{せい|りよく}の
\ruby{及}{およ}べる
\ruby{境域}{さか|ひ}には
\ruby{反身}{そり|み}になりて
\ruby{誇}{ほこ}りて
\ruby{生活}{く|ら}すことの
\ruby{叶}{かな}ふべき
\ruby{筈}{はず}なるを、
\ruby{我}{われ}から
\ruby{我}{わ}が
\ruby{出世}{しゆ|つせ}を
\ruby{遮}{さへぎ}り
\ruby{止}{とど}めて
\ruby{今}{いま}も
\ruby{{\換字{猶}}}{なほ}
\ruby{外妾}{めか|け}たるなり。

\ruby{筑波}{つく|ば}が
\ruby{引上}{ひき|あ}げて
\ruby{正室}{つ|ま}とせんと
\ruby{云}{い}ひし
\ruby{時}{とき}、
お
\ruby{彤}{とう}は
\ruby{如何}{い|か}なる
\ruby{意}{こゝろ}にて
\ruby{之}{これ}を
\ruby{辭}{いな}みしか
\ruby{知}{し}らず。
されど
\ruby{其}{そ}の
\ruby{外}{そと}に
\ruby{現}{あら}はれたるところにては、
お
\ruby{彤}{とう}は
\ruby{一向謹}{ひた|すら|つゝし}み
\ruby{愼}{つゝし}みて、

『
\ruby{妾}{わたし}を
\ruby{引上}{ひき|あ}げて
\ruby{下}{くだ}さらうとい
\ruby{御思召}{お|ぼし|めし}は
\ruby{嬉}{うれ}しうございますが、
\ruby{妾}{わたし}は
\ruby{實家}{さ|と}も
\ruby{無}{な}く
\ruby{後楯}{うしろ|だて}も
\ruby{無}{な}い
\ruby{身}{み}ですから、
\ruby{左樣}{さ|う}
\ruby{仰}{おつし}あつて
\ruby{下}{くだ}さるから
\ruby{好}{い}いはで
\ruby{成}{な}り
\ruby{上}{あが}りましたら、
\ruby{人}{ひと}の
\ruby{謗}{そし}り
\ruby{嘲}{あざけ}りは
\ruby{何}{ど}の
\ruby{樣}{やう}でございましやう。
\ruby{其}{それ}も
\ruby{妾}{わたし}が
\ruby{惡}{わる}く
\ruby{云}{い}はれるだけで
\ruby{濟}{す}めば
\ruby{宜}{よ}うございますが、
\ruby{針}{はり}ほどの
\ruby{事}{こと}も
\ruby{棒}{ぼう}ほどに
\ruby{云}{い}ひたがる
\ruby{人}{ひと}の
\ruby{口}{くち}ですもの
\ruby{何}{なん}ぞの
\ruby{折}{をり}には
\ruby{妾}{わたし}のことを
\ruby{云}{い}ひ
\ruby{出}{だ}して、
\ruby{彼樣}{あ|ん}なものを
\ruby{引上}{ひき|あ}げたのは
\ruby{何事}{なに|ごと}だと、
\ruby{屹度}{きつ|と}
\ruby{貴下}{あな|た}を
\ruby{惡}{わる}く
\ruby{云}{い}はずには
\ruby{居}{を}りません。
よし
\ruby{何}{なに}を
\ruby{人}{ひと}が
\ruby{云}{い}つたつて
\ruby{氣}{き}になさるほどの
\ruby{{\換字{弱}}}{よわ}い
\ruby{貴下}{あな|た}では
\ruby{無}{な}くつても、
\ruby{妾}{わたし}の
\ruby{{\換字{所}}爲}{せ|ゐ}で
\ruby{貴下}{あな|た}の
\ruby{金箔}{は|く}を
\ruby{剝脫}{お|と}すのは
\ruby{妾}{わたし}は
\ruby{{\換字{嫌}}}{いや}です。
どうせ
\ruby{今}{いま}まで
\ruby{日陰者}{ひ|かげ|もの}で
\ruby{濟}{す}まして
\ruby{來}{き}た
\ruby{妾}{わたし}ですもの、いつそ
\ruby{一生}{いつ|しやう}
\ruby{日陰者}{ひ|かげ|もの}で
\ruby{濟}{す}まして
\ruby{{\換字{終}}}{しま}つて、
\ruby{人}{ひと}に
\ruby{目角}{め|かど}を
\ruby{立}{た}てられずに
\ruby{生活}{く|ら}した
\ruby{方}{はう}が
\ruby{性}{しやう}に
\ruby{合}{あ}ひさうです。
\ruby{貴下}{あな|た}さへ
\ruby{見棄}{み|す}てゝ
\ruby{下}{くだ}さらなければ、
\ruby{自{\換字{分}}}{じ|ぶん}が
\ruby{出世}{しゆ|つせ}して
\ruby{貴下}{あな|た}を
\ruby{惡}{わる}く
\ruby{云}{い}はせやう
\ruby{氣}{き}はございません。
』

と、いと
\ruby{眞面目}{ま|じ|め}に
\ruby{{\換字{道}}理}{だう|り}
\ruby{正}{たゞ}しく
\ruby{斷}{ことわ}れるのみか、
\ruby{扨}{さて}
\ruby{打解}{うち|と}けて
\ruby{碎}{くだ}けて
\ruby{笑}{わら}ふ
\ruby{醉}{よひ}の
\ruby{後}{あと}などには、
\ruby{面}{めん}と
\ruby{對}{むか}ひて
\ruby{{\換字{遠}}慮}{ゑん|りよ}も
\ruby{無}{な}く
\ruby{直接}{うち|つけ}に、

『
\ruby{正室}{おく|さま}になりやあ
\ruby{正室}{おく|さま}だけの
\ruby{荷}{に}を
\ruby{背負}{し|よ}はなけりやあなりませんからネ。
\ruby{力}{ちから}の
\ruby{無}{な}い
\ruby{妾}{わたし}が
\ruby{其樣}{そ|ん}な
\ruby{事}{こと}を
\ruby{仕}{し}て
\ruby{肩}{かた}を
\ruby{凝}{こ}らすよりやあ、
\ruby{氣樂}{き|らく}にして
\ruby{斯樣}{か|う}して
\ruby{居}{ゐ}る
\ruby{方}{はう}がマア
\ruby{宜}{よ}さゝうですから。
』

と
\ruby{云}{い}ひて
\ruby{肯}{うけが}はず。
\ruby{乘}{の}らば
\ruby{乘}{の}るべかりし
\ruby{玉}{たま}の
\ruby{輿}{こし}を
\ruby{自}{みづか}ら
\ruby{棄}{す}てゝ
\ruby{吝}{おし}まざりしかば、
\ruby{某子爵}{なに|がし|ゝしやく}の
\ruby{姫君}{ひめ|ぎみ}は
\ruby{筑波}{つく|ば}の
\ruby{妻}{つま}として
\ruby{今}{いま}の
\ruby{榮華}{えい|ぐわ}を
\ruby{受}{う}け
\ruby{得}{え}たまふに
\ruby{至}{いた}りしなり。

されば
\ruby{筑波}{つく|ば}は
お
\ruby{彤}{とう}を
\ruby{日陰者}{ひ|かげ|もの}として
\ruby{世}{よ}にこそ
\ruby{隱}{かく}し
\ruby{居}{を}れ、
\ruby{之}{これ}を
\ruby{愛}{め}で
\ruby{重}{おも}んずることは
\ruby{今}{いま}の
\ruby{正室}{つ|ま}にも
\ruby{{\換字{勝}}}{まさ}れり。

お
\ruby{彤}{とう}は
\ruby{是}{かく}の
\ruby{如}{ごと}くにして
\ruby{此}{こ}の
\ruby{世}{よ}にたゞ
\ruby{一人}{ひと|り}の
\ruby{筑波}{つく|ば}の
\ruby{意}{こゝろ}を
\ruby{失}{うしな}はざらんとする
\ruby{外}{ほか}には、
\ruby{何}{なん}の
\ruby{心}{こゝろ}を
\ruby{用}{もち}ひ
\ruby{氣}{き}を
\ruby{勞}{つか}らすことも
\ruby{無}{な}く、
\ruby{年}{とし}の
\ruby{首}{はじめ}より
\ruby{年}{とし}の
\ruby{尾}{をはり}まで、
\ruby{身}{み}の
\ruby{周圍}{まは|り}の
\ruby{物}{もの}より
\ruby{庭}{には}の
\ruby{隅}{すみ}の
\ruby{草木}{くさ|き}まで、
\ruby{一切}{いつ|さい}を
\ruby{榮華}{えい|ぐわ}の
\ruby{頂上}{てつ|ぺん}の
\ruby{仕度三昧}{し|たい|ざん|まい}に
\ruby{振舞}{ふる|ま}ひて、
\ruby{誰}{たれ}に
\ruby{苦{\換字{情}}}{く|じやう}を
\ruby{云}{い}はるゝことも
\ruby{無}{な}く
\ruby{日}{ひ}を
\ruby{{\換字{過}}}{す}ごせるなり。


\Entry{其十一}

\ruby{六疊}{ろく|でふ}の
\ruby{茶}{ちや}の
\ruby{間}{ま}、
\ruby{茶}{ちや}の
\ruby{間}{ま}とはいへ
\ruby{大抵}{たい|てい}の
\ruby{家}{いへ}の
\ruby[g]{客室}{きやくま}より
\ruby{美}{うつく}しく、
\ruby{柱}{はしら}より
\ruby{敷居鴨居}{しき|ゐ|かも|ゐ}の
\ruby[g]{木口}{きぐち}の
\ruby{結構}{けつ|こう}さ。
\ruby{格}{こ}の
\ruby{配}{くば}りに
\ruby{物好}{もの|ずき}を
\ruby{見}{み}せたる
\ruby{細骨}{ほそ|ぼね}の
\ruby{纖巧}{きや|しや}なる
\ruby{二間四枚}{に|けん|よん|まい}の
\ruby{障子}{しやう|じ}に、
\ruby[g]{繼目無}{つぎめな}しの
\ruby{紙}{かみ}は
\ruby{{\換字{雪}}}{ゆき}より
\ruby{白}{しろ}く
\ruby{椽}{ゑん}の
\ruby{方}{かた}より
\ruby{光線}{くわう|せん}を
\ruby{取}{と}りて、
\ruby{上}{うへ}は
\ruby[g]{{\換字{嫌}}味氣無}{いやみけな}き
\ruby{柾}{まさ}の
\ruby[g]{天井}{てんじやう}、
\ruby{下}{した}は
\ruby[g]{緣無}{へりな}しの
\ruby{備後表}{び|ん|ご}といふ
\ruby{室}{ま}の
\ruby{内}{うち}の、
\ruby{好}{よ}きほどに
\ruby{据}{す}ゑられたる
\ruby[g]{多{\換字{分}}太田}{いずれおほた}あたりで
\ruby{指}{さ}させたるらしき
\ruby{島桑}{しま|ぐは}の
\ruby{長火鉢}{なが|ひ|ばち}と、
\ruby{其}{そ}の
\ruby{横手}{よこ|て}に
\ruby{置}{お}かれたる
\ruby{思}{おも}ひ
\ruby{切}{き}つて
\ruby{立派}{りつ|ぱ}なる
\ruby[g]{支那製}{しなせい}の
\ruby{紫檀}{し|たん}の
\ruby{茶棚}{ちや|だな}とは、
\ruby{先}{ま}づ
\ruby{入}{い}るものゝ
\ruby{目}{め}を
\ruby{惹}{ひ}きて、
\ruby{此家}{こ|ゝ}の
\ruby{女主人}{あ|る|じ}の
\ruby{十二{\換字{分}}}{じう|に|ぶん}に
\ruby{財}{たから}に
\ruby{富}{と}み
\ruby{足}{た}りて、
\ruby{且}{か}つは
\ruby{其}{そ}の
\ruby[g]{勸工場品}{くわんこうばもの}に
\ruby{望}{のぞ}み
\ruby{足}{た}れりとするやうなる
\ruby[g]{沒趣味者}{わからずや}ならぬを
\ruby{示}{しめ}し、
\ruby{壁}{かべ}の
\ruby{塗}{ぬ}り
\ruby{色}{いろ}、
\ruby{押入}{おし|いれ}の
\ruby{襖}{ふすま}の
\ruby{模樣}{も|やう}まで、すべて
\ruby{釣合}{つり|あ}ひてしつとりと
\ruby{整}{とゝの}ひたるが
\ruby{中}{なか}に、おのづから
\ruby{薄手}{うす|で}ならず
\ruby{{\換字{又}}}{また}わびしげならで
\ruby{飽}{あく}まで『
\ruby{良}{よ}いもの
\ruby{好}{ず}き』『
\ruby{粗惡}{い|や}なもの
\ruby{{\換字{嫌}}}{ぎら}ひ』の
\ruby{趣}{おもむ}きは
\ruby{見}{み}えたり。

『お
\ruby{龍}{りう}ちやん、
お
\ruby{前御客樣}{まへ|お|きやく|さま}らしく
\ruby{仕無}{し|な}いでも、もつと
\ruby{此方}{こつ|ち}へ
\ruby{寄}{よ}つて
\ruby{御}{お}あたりナ。
』

\ruby[g]{大島紬}{おほしま}は
\ruby{好}{い}いものなれども、
\ruby{何處}{ど|こ}となくぼやついて、すつぺりとせぬが
\ruby{厭}{いや}なり、
\ruby[g]{{\換字{平}}常着}{ふだんぎ}は
\ruby{此}{これ}に
\ruby{限}{かぎ}ると、
\ruby{{\換字{平}}生}{ひご|ろ}
\ruby{御召縮緬}{お|め|し|}を
\ruby{着{\換字{通}}}{き|とほ}せる
お
\ruby{彤}{とう}の、
\ruby{今}{いま}も
\ruby{相變}{あひ|かは}らず
\ruby{其品}{そ|れ}づくめの
\ruby{衣服}{な|り}つき
\ruby{見好}{み|よ}く、
\ruby{絹物}{き|ぬ}の
\ruby{坐蒲團}{ざ|ぶ|とん}の
\ruby{上}{うへ}に
\ruby{居}{ゐ}て、
\ruby{火鉢}{ひ|ばち}より
\ruby[g]{南部}{なんぶ}の
\ruby{鐵瓶}{てつ|びん}を
\ruby{重}{おも}さうに
\ruby{取}{と}り
\ruby{下}{おろ}しながら
\ruby{斯}{か}く
\ruby{云}{い}へば、

『えゝ、
\ruby{姊}{ね{\換字{江}}}さんのところへ
\ruby{來}{き}て
\ruby{御客樣}{お|きやく|さま}らしくなんぞ
\ruby{仕}{し}や
\ruby{仕}{し}ませんがネ、まだ
\ruby{火}{ひ}の
\ruby{傍}{そば}へ
\ruby{行}{い}きたいほど
\ruby{{\換字{寒}}}{さむ}かあ
\ruby{有}{あ}りませんもの。
』

と
\ruby{笑}{わら}ひつゝ
お
\ruby{龍}{りう}は
\ruby{言}{ことば}に
\ruby{從}{したが}つて
\ruby{聊}{いさゝ}か
\ruby{坐}{ざ}を
\ruby{{\換字{進}}}{すゝ}めたるが、
\ruby{實}{げ}に
\ruby{其}{そ}の
\ruby{顏}{かほ}は
\ruby{見}{み}るからが
\ruby{冴々}{さえ|〴〵}しく
\ruby{櫻色}{さくら|いろ}に
\ruby{艶}{えん}にして、
\ruby{如何}{い|か}にも
\ruby{此}{こ}の
\ruby{頃}{ごろ}の
\ruby{{\換字{寒}}}{さむ}さ
\ruby{位}{ぐらゐ}は
\ruby{何}{なん}とも
\ruby{思}{おも}はぬらしき
\ruby{樣子}{やう|す}をあらはせり。

お
\ruby{彤}{とう}は
\ruby{坐}{ざ}を
\ruby{{\換字{進}}}{すゝ}むる
お
\ruby{龍}{りう}が
\ruby{頭髮}{かし|ら}を
\ruby{一寸}{ちよ|つと}
\ruby{見}{み}しが、
\ruby{女}{をんな}
\ruby{同士}{どう|し}の
\ruby{談}{はなし}の
\ruby{緖}{いとぐち}は
\ruby{先}{ま}づ
\ruby{其}{それ}より
\ruby{解}{ほご}るゝ
\ruby{{\換字{習}}}{ならひ}なり。

『
\ruby{今日}{け|ふ}もまた
\ruby{束髮}{そく|はつ}にしておいでだネ。
\ruby{此{\換字{節}}}{この|せつ}は
\ruby{何時見}{い|つ|み}ても
\ruby{結}{い}つては
\ruby{居}{ゐ}ないのネ。
』

『ハア。
\ruby{姊}{ねえ}さんでさへ
\ruby{矢張}{やつ|ぱり}
\ruby{束髮}{ %全角空白
こ|れ}になさるぢやあ
\ruby{有}{あ}りませんか。
まして
\ruby{妾}{わたし}なんか。
\ruby{出}{で}る
\ruby{先}{さき}に
\ruby{立}{た}つて
\ruby{一々人手}{いち|〳〵|ひと|で}を
\ruby{假}{か}りるのが
\ruby{億劫}{おつ|くう}なものですから、つい
\ruby[g]{自{\換字{分}}}{ひとり}でもつてぐる〳〵と
\ruby{卷}{ま}いて
\ruby{仕舞}{し|ま}ふので。
\ruby{似合}{に|あ}は
\ruby{無}{な}いで
\ruby{可笑}{を|か}しくつて?。
』

『ナアニ
\ruby{似合}{に|あ}はない
\ruby{事}{こと}は
\ruby{有}{あ}りやあ
\ruby{仕}{し}ないよ、ぢやあ
\ruby{今日}{け|ふ}ももう
\ruby{何處}{ど|こ}かへ
\ruby{御出}{お|いで}だつたのだネ。
』

『ハア
\ruby{一寸}{ちよ|つと}。
』

こゝに
\ruby{至}{いた}りて
\ruby{女主人}{あ|る|じ}は
\ruby{其}{そ}の
\ruby{美}{うつく}しき
\ruby{面}{おもて}に
\ruby{微笑}{ゑ|み}を
\ruby{泛}{うか}めて、

『
\ruby{當}{あ}てゝ
\ruby{見}{み}やうかへ。
』

と
\ruby{戯}{たはむ}るゝが
\ruby{如}{ごと}く
\ruby{云}{い}へば、
お
\ruby{龍}{りう}は
\ruby{言}{ことば}も
\ruby{無}{な}く
\ruby{莞爾}{にこ|り}と
\ruby{笑}{ゑ}みて
\ruby{親}{した}しげに
\ruby{輕}{かろ}く
\ruby{點頭}{うな|づ}けり。

『
\ruby{屹度}{きつ|と}また
\ruby{淺草}{あさ|くさ}へ
\ruby{御出}{お|いで}だつたのさ。
』

『いゝえ。
』

『なに、いゝえの
\ruby{事}{こと}が
\ruby{有}{あ}るものかネ。
ソラ〳〵
\ruby{口}{くち}は
\ruby{詐}{うそ}を
お
\ruby{云}{い}ひでも
\ruby{顏}{かほ}は
\ruby{正直}{しやう|ぢき}だよ、ハイ
\ruby{觀音樣}{くわん|のん|さま}へ
\ruby{參}{まゐ}りましたと、その
\ruby{笑}{わら}つて
\ruby{居}{ゐ}る
\ruby{眼}{め}が、チヤーンと
\ruby{左樣}{さ|う}いつて
\ruby{居}{ゐ}るよ。
』

『ホヽホヽホヽ。
』

『ホヽホヽ、それ
\ruby{御覽}{ご|らん}、
\ruby{御手}{お|て}の
\ruby{筋}{すぢ}だらう。
\ruby{御精}{ご|せい}が
\ruby{出}{で}て
\ruby{眞實}{ほん|と}に
\ruby[g]{御奇特}{ごきとく}の
\ruby{事}{こと}だネエ。
』

『あら
\ruby{姊}{ねえ}さん、
\ruby{調戯}{から|か}つちやあ
\ruby{厭}{いや}ですよ、あんまりですは。
』

『
\ruby{左樣}{さ|う}さネエ。
\ruby{何}{なに}も
\ruby{彼}{あ}の
\ruby{人}{ひと}に
\ruby{御會}{お|あ}ひでも
\ruby{無}{な}かつたらうに、
\ruby{調戯}{から|か}はれちやあ
\ruby{愍然}{かは|いさう}だつたネ。
』

『もうようござんすは、
\ruby{澤山}{たん|と}いろんな
\ruby{事}{こと}を
\ruby{仰}{おつし}あいよ。
\ruby{今日}{け|ふ}も
\ruby{不思議}{ふ|し|ぎ}に
\ruby[g]{落合}{おちあ}つて
\ruby{會}{あ}つて
\ruby{來}{き}ましたは。
』

『オヤツ。
そんな
\ruby{譯}{わけ}は
\ruby{無}{な}いぢやあ
\ruby{無}{な}いか。
\ruby{今日}{け|ふ}は
\ruby{{\換字{平}}常}{た|ゞ}の
\ruby{日}{ひ}だし、
\ruby{彼}{あ}の
\ruby{人}{ひと}は
\ruby{職務}{つと|め}が
\ruby{有}{あ}るつていふ
\ruby{談}{はなし}だつたもの。
ぢやあ
\ruby{矢張}{やつ|ぱり}
\ruby{打合}{うち|あはせ}でも
\ruby{仕}{し}て
\ruby{御置}{お|お}きだつたの?。
』

『いゝえ、そんな
\ruby{事}{こと}は
\ruby{有}{あ}りあ
\ruby{仕}{し}ませんがネ。
\ruby{彼}{あ}の
\ruby{人}{ひと}が
\ruby{職務}{つと|め}の
\ruby{方}{はう}を
\ruby{辭}{よ}して
\ruby{仕舞}{し|ま}つたので、それで
\ruby{今日}{け|ふ}は
\ruby{御午前}{お|ひる|まへ}に
\ruby{出}{で}て
\ruby{來}{き}たつて
\ruby{云}{い}ふんで。
ひよつくりと
\ruby{御堂}{み|だう}で
\ruby{會}{あ}つたわけなのですよ。
』

『ヘーエ、
\ruby{職務}{つと|め}の
\ruby{方}{はう}を
\ruby{辭}{よ}したつて……。
あゝ
\ruby{解}{わか}つた
\ruby{免}{よ}されたんだネ。
』

『
\ruby{左樣}{さ|う}なのよ、
\ruby[g]{事實}{まつたく}は
\ruby{免}{よ}されたのですつて。
\ruby{其}{それ}について
\ruby{姊}{ねえ}さんに
\ruby{些}{ちつと}
お
\ruby{願}{ねがひ}があつて
\ruby{來}{き}たのですがネ。
』

と、やゝ
\ruby{眞}{しん}になつて
\ruby{談話}{はな|し}をせんとする
お
\ruby{龍}{りう}の
\ruby{眼色}{め|いろ}を
\ruby{見}{み}て、
お
\ruby{彤}{とう}は
\ruby{輕}{かろ}く
\ruby{一寸}{ちよ|つと}
\ruby{制止}{と|とゞ}めつ、

『
\ruby{御待}{お|ま}ちよ
お
\ruby{龍}{りう}ちやん。
\ruby{彼室}{あつ|ち}へ
\ruby{行}{い}つてから
\ruby{{\換字{緩}}々}{ゆつ|くり}と
\ruby{談}{はなし}を
\ruby{聞}{き}かうから。
』

と、
\ruby{奧}{おく}の
\ruby{方}{かた}を
\ruby{指}{ゆび}さし、

『あら
\ruby{姊}{ねえ}さん、
\ruby{此室}{こ|ゝ}で
\ruby{澤山}{たく|さん}
だは。
』

とお
\ruby{龍}{りう}の
\ruby{云}{い}ふを
\ruby[g]{打{\換字{消}}}{うちけ}して、

『
\ruby{妾}{わたし}が
\ruby{茶}{ちや}の
\ruby{間}{ま}に
\ruby{居}{ゐ}るのゝ
\ruby{{\換字{嫌}}}{きらひ}なのは
お
\ruby{前}{まへ}も
\ruby{知}{し}つて
\ruby{居}{ゐ}るぢや
\ruby{無}{な}いか。
』

と
\ruby{{\換字{遮}}}{さへぎ}り、さて
\ruby{下手}{しも|て}へ
\ruby{向}{むか}つて
\ruby{小間使}{こ|ま|づかひ}の
お
\ruby{春}{はる}といへる
\ruby{可愛}{か|はい}らしき
\ruby{兒}{こ}を
\ruby{喚}{よ}び
\ruby{出}{いだ}し、

『
\ruby{妾}{わたし}の
\ruby{部屋}{へ|や}の
\ruby[g]{茶{\換字{道}}具}{ちやだうぐ}を
\ruby{能}{よ}く
\ruby{淸潔}{きれ|い}に
\ruby{仕}{し}てネ、そしてまた
\ruby{彼室}{あつ|ち}へ
\ruby{持}{も}つて
\ruby{行}{い}つて
お
\ruby{{\換字{呉}}}{く}れ。
お
\ruby{茶}{ちや}は
\ruby{妾}{わたし}が
\ruby{自{\換字{分}}}{じ|ぶん}で
\ruby{淹}{い}れるからネ、
お
\ruby{前}{まへ}は
\ruby{御菓子}{お|くわ|し}を
\ruby{出}{だ}して、……ア
\ruby{羊羮}{やう|かん}はいけない、
\ruby{玉簾}{たま|だれ}の
\ruby{方}{はう}を
\ruby{切}{き}つておいで。
』

と
\ruby{命令}{いひ|つけ}け、

『さあ
\ruby{此方}{こつ|ち}へ
\ruby{御}{お}いで。
』

と
\ruby{立上}{たち|あが}つて
お
\ruby{龍}{りう}を
\ruby{奧}{おく}へ
\ruby{{\換字{伴}}}{ともな}へる
\ruby{時}{とき}、
\ruby{恰}{あだか}も
\ruby{時計}{と|けい}の
\ruby{音}{おと}は
\ruby{三時}{さん|じ}を
\ruby{報}{はう}じたり。

\ruby{男}{をとこ}にもいろ〳〵あれば、
\ruby{女}{をんな}にもいろ〳〵ありて、まことに
お
\ruby{彤}{とう}は
\ruby{今}{いま}みづから
\ruby{言}{い}へるが
\ruby{如}{ごと}くに、
\ruby{{\換字{平}}生}{ひご|ろ}
\ruby{長火鉢}{なが|ひ|ばち}の
\ruby{前}{まへ}に
\ruby{坐}{すわ}りて
\ruby{茶}{ちや}の
\ruby{間}{ま}に
\ruby{在}{あ}ることは
\ruby{{\換字{悅}}}{よろこ}ばずして、おのが
\ruby{室}{ま}と
\ruby{定}{さだ}めたる
\ruby{小座敷}{こ|ざ|しき}に
\ruby{端然}{しや|ん}として
\ruby{居}{ゐ}ることを
\ruby{好}{この}めるなり。
されば
\ruby{是程}{これ|ほど}の
\ruby{好}{よ}き
\ruby{茶}{ちや}の
\ruby{室}{ま}をも、
\ruby{一}{ひ}ㇳ
\ruby{風}{ふう}ある
\ruby{氣性}{きし|やう}からは、
\ruby{床}{とこ}の
\ruby{間}{ま}さへ
\ruby{無}{な}き
\ruby{室}{へや}と
\ruby{賤}{いや}しく
\ruby{思}{おも}ふなるべし。


\Entry{其十二}

\ruby{市中}{まち|なか}の
\ruby{事}{こと}なれば
\ruby{廣}{ひろ}くはあらねど、
\ruby{特}{わざ}と
\ruby{花物}{はな|もの}を
\ruby{{\換字{嫌}}}{きら}ひたる
\ruby{常磐木}{とき|は|ぎ}のみの
\ruby{庭}{には}の、
\ruby{見}{み}えぬところに
\ruby{人}{ひと}の
\ruby{手}{て}の
\ruby{十{\換字{分}}}{じう|ぶん}に
\ruby{用}{もち}ひられたる
\ruby{證}{しるし}とて、
\ruby{枝々}{えだ|〳〵}は
\ruby{好}{よ}きほどに
\ruby{折}{お}り
\ruby{合}{あ}ひて
\ruby{茂}{しげ}りながら、
\ruby{隈々}{くま|〴〵}は
\ruby{汚}{むさ}からで
\ruby{明}{あか}るく、わづかに
\ruby{大}{おほき}からず
\ruby{小}{ちひ}さからぬ
\ruby{燈籠一}{とう|ろう|ひと}つの
\ruby{形狀}{かた|ち}も
\ruby{佳}{よ}く
\ruby{時代}{じ|だい}もありて
\ruby{一寸}{ちよ|つと}
\ruby{面白}{おも|しろ}きがほかには、
\ruby{別}{べつ}に
\ruby{此}{これ}といふ
\ruby{價}{ね}の
\ruby{高}{たか}き
\ruby{樹}{き}も
\ruby{珍}{めづ}らしき
\ruby{石}{いし}も
\ruby{無}{な}けれど、
\ruby{一體}{いつ|たい}の
\ruby{調子}{てう|し}の
\ruby{蟠屈無}{わだ|かまり|な}くすらりと、
\ruby[g]{幽閑}{しづか}にして、
\ruby{特設}{こし|ら}へ
\ruby{氣}{ぎ}も
\ruby{無}{な}く、
\ruby{見}{み}る
\ruby{眼安}{め|やす}く
\ruby{穩和}{おだ|やか}なるところに
\ruby{自然}{おのづ|から}
\ruby{{\換字{飽}}}{あ}かぬ
\ruby{床}{ゆか}しさありて、
\ruby{夏}{なつ}は
\ruby{{\換字{梢}}}{こずえ}に
\ruby{新月}{にひ|づき}の
\ruby{低}{ひく}う
\ruby{懸}{かゝ}る
\ruby{{\換字{宵}}}{よひ}、
\ruby[g]{不如歸}{ほとゝぎす}の
\ruby{一}{ひ}ㇳ
\ruby{聲}{こゑ}をも
\ruby{待}{ま}ち
\ruby{得}{え}ば
\ruby{嘸}{さぞ}とおもはれ、
\ruby{{\換字{冬}}}{ふゆ}は
\ruby{雀膨}{すゞめ|ふく}るゝ
\ruby{{\換字{寒}}}{さむ}き
\ruby{日}{ひ}の
\ruby{雲破}{くも|やぶ}れて
\ruby[g]{時雨}{しぐれ}はら〳〵と
\ruby{落}{お}つる
\ruby{夕}{ゆふべ}、
\ruby{或}{ある}は
\ruby{{\換字{又}}}{また}
\ruby{{\換字{雪}}}{ゆき}の
\ruby{薄綿萬物}{うす|わた|ばん|ぶつ}を
\ruby{包}{つゝ}む
\ruby{曉}{あした}など、
\ruby{如何}{い|か}にと
\ruby{忍}{しの}ばるゝばかりなり。

されば
\ruby{折}{をり}ふしは
\ruby{此家}{こ|ゝ}にも
\ruby{出入}{で|い}りする
\ruby{筑波}{つく|ば}が
\ruby{氣}{き}に
\ruby{入}{い}りの
\ruby{骨董屋}{だう|ぐ|や}の
\ruby{老漢}{ぢ|ゞ}に、
\ruby{利齋}{り|さい}といひて、
\ruby{内々}{ない|〳〵}は
\ruby{茶{\換字{道}}天狗}{ちや|だう|てん|ぐ}の
\ruby{小賢}{こ|ざか}しき
\ruby{男}{をとこ}、
\ruby{此}{こ}の
\ruby{庭}{には}を
\ruby{見}{み}て、

『
\ruby{猫}{ねこ}の
\ruby{額}{ひたひ}ぐらゐの
\ruby{庭}{には}だが
\ruby{彼}{あ}の
\ruby{人}{ひと}の
\ruby{住居}{すま|ゐ}に
\ruby{彼}{あ}の
\ruby{庭}{には}は
\ruby{何}{なん}ともいへない。
\ruby{庭}{には}の
\ruby{出來}{で|き}が
\ruby{好}{い}いばかりでは
\ruby{無}{な}い、
\ruby{彼}{あ}のこつくりした
\ruby{素樸}{ぢ|み}の
\ruby{景色}{け|しき}の
\ruby{中}{なか}に、
\ruby{繪}{ゑ}の
\ruby{{\換字{浮}}}{う}いて
\ruby{出}{で}たやうに
\ruby{美麗}{き|れい}な
\ruby{福相}{ふく|さう}の
\ruby{美人}{び|じん}の
\ruby{彼}{あ}の
\ruby{人}{ひと}が
\ruby{澄}{す}まして
\ruby{居}{ゐ}る
\ruby[g]{對照}{うつりあひ}といふものは、
\ruby{何}{なん}のことは
\ruby{無}{な}い、
\ruby{茶壁}{ちや|かべ}の、
\ruby{何}{なに}も
\ruby{無}{な}い
\ruby{床}{とこ}に
\ruby{一輪}{いち|りん}の
\ruby[g]{白牡丹}{はくぼたん}を
\ruby{活}{い}けたやうなもので、
\ruby{一}{ひ}ㇳ
\ruby{層人}{きは|ひと}の
\ruby{眼}{め}を
\ruby{驚}{おどろ}かす。

\ruby{彼}{あ}の
\ruby{人}{ひと}が
\ruby{花}{はな}だから
\ruby{花}{はな}は
\ruby{要}{い}らない。
これを
\ruby{思}{おも}へば
\ruby{花}{はな}と
\ruby{見}{み}られるほどの
\ruby{容姿}{きり|よう}も
\ruby{無}{な}い
\ruby{女}{をんな}なぞが、
\ruby{自{\換字{分}}}{じ|ぶん}の
\ruby{庭前}{には|さき}に
\ruby{花}{はな}を
\ruby{植}{う}ゑたりなんぞして
\ruby{妙}{めう}に
\ruby{優美}{やさ|し}がつて
\ruby{好}{い}い
\ruby{氣}{き}になつて
\ruby{居}{ゐ}ても、
\ruby{下手}{へ|た}に
\ruby{花}{はな}の
\ruby{{\換字{近}}傍}{そ|ば}にでも
\ruby[g]{彷徨}{まごつ}かうものなら、
\ruby{宛然海棠}{まる|で|かい|だう}の
\ruby{下}{した}で
\ruby{狸}{たぬき}がチンチンでも
\ruby{仕}{し}て
\ruby{居}{ゐ}るやうに
\ruby{見}{み}えるのが
\ruby{多}{おほ}い。
\ruby{茶{\換字{道}}}{ち|や}を
\ruby{知}{し}らない
\ruby{奴}{やつ}はまあ
\ruby{其樣}{そ|ん}なものだが、
\ruby{彼庭}{あ|れ}が
\ruby{彼}{あ}の
\ruby{人}{ひと}の
\ruby{好}{この}みで
\ruby{出來}{で|き}たといへば
\ruby{彼}{あ}の
お
\ruby{彤}{とう}さんといふ
\ruby{人}{ひと}は
\ruby{顏}{かほ}が
\ruby{美}{い}いばかりぢやあ
\ruby{無}{な}い、
\ruby{何}{なに}も
\ruby{彼}{か}も
\ruby{解}{わか}る
\ruby{人}{ひと}だ、
\ruby{中々}{なか|〳〵}
\ruby{一}{ひ}ㇳ
\ruby{{\換字{通}}}{とほ}りや
\ruby{二}{ふ}タ
\ruby{{\換字{通}}}{とほ}りの
\ruby{人}{ひと}で
\ruby{無}{な}い。
\ruby{{\換字{道}}理}{だう|り}で
\ruby{物品}{も|の}を
\ruby{買}{か}つても
\ruby{買}{か}ひつ
\ruby{振}{ぷ}りが
\ruby{可}{い}い。
そして
\ruby{倦}{あ}きつぽい
\ruby{彼}{あ}の
\ruby{筑波}{つく|ば}さんが、
\ruby{何年}{なん|ねん}といふものこびり
\ruby{付}{つ}いて
\ruby{居}{ゐ}る。
どうも
\ruby{偉}{えら}い、
\ruby{茶{\換字{道}}}{ち|や}を
\ruby{知}{し}つて
\ruby{居}{ゐ}るから
\ruby{何樣}{ど|う}も
\ruby{偉}{えら}い。
』

と、
\ruby{自己}{お|の}が
\ruby{高慢}{かう|まん}を
\ruby{{\換字{交}}}{ま}ぜて
\ruby{{\換字{評}}}{ひよう}したる
\ruby{事}{こと}ありき。

\ruby{家}{いへ}の
\ruby{一角}{いつ|かく}の
\ruby{小座敷}{こ|ざ|しき}の、
\ruby{僅四疊{\換字{半}}}{わづか|よ|でふ|はん}には
\ruby{{\換字{過}}}{す}ぎねど、
\ruby{此}{こ}の
\ruby{庭}{には}を
\ruby{東南}{たつ|み}に
\ruby{受}{う}けて、
\ruby{陽氣}{やう|き}なれど
\ruby{廂}{ひさし}を
\ruby{長}{なが}く
\ruby{仕}{し}たれば
\ruby{明}{あか}る
\ruby{{\換字{過}}}{す}ぎず
\ruby{建}{た}てられたるが
\ruby{中}{なか}に
\ruby{今}{いま}しも
お
\ruby{彤}{とう}
お
\ruby{龍}{りう}は
\ruby{相對}{あひ|たい}して
\ruby{坐}{すわ}れり。
\ruby{薩摩杉}{さ|つ|ま}の
\ruby{天井板}{てん|じや|う}の
\ruby{木理美}{も|く|うる}はしく、
\ruby{根岸茶}{ね|ぎ|し}の
\ruby{壁}{かべ}の
\ruby{色沈着}{いろ|おち|つ}きて、
\ruby{床}{とこ}には
お
\ruby{彤}{とう}が
\ruby{好}{この}みか
\ruby{筑波}{つく|ば}が
\ruby{好}{この}みかは
\ruby{知}{し}らず
\ruby{明人}{みん|ひと}らしき
\ruby{書}{しよ}の
\ruby{小幅}{せう|ふく}を
\ruby{掛}{か}けて、
\ruby{棚}{たな}にはこれは
\ruby{慥}{たしか}に
\ruby{主人}{ある|じ}が
\ruby[g]{玩弄}{もてあそび}に
\ruby{疑}{うたが}ひ
\ruby{無}{な}き
\ruby{繪卷}{ゑ|まき}など
\ruby{取}{と}り
\ruby{繕}{つくろ}はず
\ruby{載}{の}せたり。
\ruby[g]{出入口}{でいりぐち}、
\ruby{窓}{まど}の
\ruby{取}{と}り
\ruby{方}{かた}なんど
\ruby{總}{す}べて
\ruby{茶室}{ちや|しつ}めきたれど、
\ruby{釜}{かま}を
\ruby{掛}{か}くることは
\ruby{{\換字{嫌}}}{きら}へるにや
\ruby{爐}{ろ}は
\ruby{切}{き}りてあらず、
\ruby{一面}{いち|めん}に
\ruby{美}{うつく}しき
\ruby{敷物}{しき|もの}の
\ruby{敷}{し}きつめられて、
\ruby{一方}{いつ|ぱう}の
\ruby{隅}{すみ}には
\ruby{今}{いま}
\ruby{物}{もの}ならぬ
\ruby{女用}{をんな|もちひ}の
\ruby{螺塡}{ら|でん}の
\ruby{黑}{くろ}き
\ruby{小机}{こ|づくゑ}の、
\ruby{漆光}{て|り}は
\ruby{既}{すで}に
\ruby{{\換字{脱}}}{ぬ}けて
\ruby{好}{よ}き
\ruby{頃}{ころ}に
\ruby{古}{ふる}びたる
\ruby{善美}{けつ|こう}いふばかり
\ruby{無}{な}きが
\ruby{上}{うへ}に、
\ruby{同}{おな}じやうなる
\ruby{手}{て}の
\ruby{小}{ちひ}さき
\ruby[g]{硯箱置}{すゞりばこお}かれ、
\ruby{机下}{し|た}にも
\ruby{同}{おな}じやうなる
\ruby{手匣}{て|ばこ}の
\ruby{置}{お}かれたる、
\ruby{此}{こ}の
\ruby{前}{まへ}は
\ruby{女主人}{あ|る|じ}が
\ruby{常}{つね}の
\ruby{座處}{ゐど|ころ}なるべし。

お
\ruby{彤}{とう}は
\ruby{今其座}{いま|そ|れ}を
\ruby{背後}{うし|ろ}にして、
\ruby{是眞}{ぜ|しん}が
\ruby{蒔繪}{まき|ゑ}の
\ruby{桐胴}{きり|どう}の
\ruby{手爐}{てあ|ぶり}の
\ruby{小}{ちひ}さきを
\ruby{横手}{よこ|て}に、
\ruby{此方}{こな|た}を
\ruby{向}{む}きて
\ruby{茶}{ちや}を
\ruby{淹}{い}れ
\ruby{居}{を}れば、
お
\ruby{龍}{りう}は
\ruby{淸楚}{さつ|ぱり}とこそ
\ruby{仕}{し}て
\ruby{居}{を}れ、おのが
\ruby{銘仙織}{めい|せ|ん}づくめの
\ruby{衣服}{な|り}の
\ruby{身}{み}の、
\ruby{居}{を}るには
\ruby{憚}{はばか}らるゝほどの
お
\ruby{納戸緞子}{なん|ど|ゞ|んす}の
\ruby{蒲團}{ふ|とん}に、やゝ
\ruby{安}{おちつ}きかぬるが
\ruby{如}{ごと}く
\ruby{坐}{すわ}りて、
\ruby{客}{きやく}といへば
\ruby{客}{きやく}ながら、おのづから
\ruby{貧富}{ひん|ぷ}の
\ruby{相{\換字{違}}}{たが|ひ}に
\ruby{壓}{お}さるゝ
\ruby{氣味}{き|み}あるを
\ruby{如何}{い|かん}とも
\ruby{仕難}{し|がた}く、たゞおとなしく
\ruby{内端}{うち|ば}に
\ruby{控}{ひか}へたるが、
\ruby{{\換字{猶}}}{なほ}
\ruby{持}{も}つて
\ruby{生}{うま}れし
\ruby{氣象}{きし|やう}の
\ruby{徳}{とく}には
\ruby{少}{すこ}しも
\ruby{萎}{め}げぬ
\ruby{顏}{かほ}つきの
\ruby{我}{われ}は
\ruby{我}{われ}だけに
\ruby{冴}{さ}えて、
\ruby{毫末}{いさ|さか}の
\ruby{隔}{へだ}て
\ruby{氣}{ぎ}も
\ruby{無}{な}く
\ruby{人}{ひと}を
\ruby{親}{したし}む
\ruby{眼}{め}の
\ruby{中凉}{うち|すゞ}しく
\ruby{相對}{あひ|むか}へるさま、たとへば
\ruby{一人}{ひと|り}は
\ruby{晴}{はれ}の
\ruby{日}{ひ}の
\ruby{晝}{ひる}に
\ruby{笑}{わら}へる
\ruby{牡丹}{ぼ|たん}ならば、
\ruby{一人}{ひと|り}は
\ruby{野}{の}の
\ruby{風}{かぜ}のそよ
\ruby{吹}{ふ}く
\ruby{秋}{あき}に、
\ruby{{\換字{寒}}}{さむ}さ
\ruby{知}{し}らぬ
\ruby{色}{いろ}して
\ruby{{\換字{咲}}}{さ}ける
\ruby{木芙蓉}{ふ|よ|う}ともいひつべし。


\Entry{其十三}

\ruby{{\換字{古}}薩摩}{こ|さつ|ま}か
\ruby{{\換字{古}}九谷}{こ|くた|に}とありさうなところを
\ruby{然}{さ}は
\ruby{無}{な}くて、
\ruby{永樂}{えい|らく}あたりの
\ruby{稀品}{き|ひん}なるべし、
\ruby{形狀}{かた|ち}
\ruby{品格}{ひ|ん}
\ruby{佳}{よ}くして
\ruby{{\換字{彩}}釉}{いろ|ゑ}
\ruby{快}{こゝろよ}く
\ruby{麗}{うる}はしき
\ruby{京燒}{きやう|やき}の
\ruby{茶器}{ちや|き}を、
\ruby{五指}{ご|し}
\ruby{白玉}{はく|ぎよく}の
\ruby{如}{ごと}く
\ruby{美}{うつく}しき
\ruby{手}{て}に
\ruby{自}{みづか}ら
\ruby{扱}{あつか}ひて、
\ruby{既}{すで}に
\ruby{鎚目}{つち|め}の
\ruby{銀瓶}{ぎん|びん}の
\ruby{湯}{ゆ}を
\ruby{徐々}{しづ|か}に
\ruby{注}{さ}し
\ruby{{\換字{終}}}{をわ}り、
\ruby{今}{いま}や
\ruby{一盞}{いつ|さん}に
\ruby{玉露}{ぎよく|ろ}の
\ruby{花香}{はな|か}を
\ruby{湛}{たゝ}へて、
お
\ruby{彤}{とう}はこれをば
\ruby{與}{あた}へ
\ruby{{\換字{遣}}}{や}りつ、
\ruby{鍋島}{なべ|しま}の
\ruby{菓子皿}{くわ|し|ざら}をば
\ruby{{\換字{又}}}{また}
\ruby{聊}{いさゝか}か
お
\ruby{龍}{りう}が
\ruby{方}{かた}へと
\ruby{推{\換字{進}}}{おし|すゝ}めたり。

お
\ruby{龍}{りう}は
\ruby{心底}{しん|そこ}より
\ruby{悅}{よろこ}びて
\ruby{茶}{ちや}を
\ruby{味}{あじ}はひつ。

『いつでも
\ruby{眞個}{ほん|たう}に
\ruby{勿體}{もつ|たい}ないやうな
\ruby{佳良}{けつ|こう}な
\ruby{御茶}{お|ちや}ネ。
』

『ホヽヽ、お
\ruby{茶}{ちや}ばかり
\ruby{褒}{ほ}めずとも
\ruby{淹}{い}れ
\ruby{方}{かた}も
\ruby{褒}{ほ}めて、
お
\ruby{吳}{く}れな。
』

『ホヽヽ、そりやあもう、
\ruby{口}{くち}へ
\ruby{出}{だ}しては
\ruby{云}{い}はなくつても……。
』

『オヤ
\ruby{左樣}{さ|う}、
\ruby{嬉}{うれ}しい
\ruby{人}{ひと}ネエ。
ぢやあまあ
\ruby{澤山}{たん|と}
\ruby{御菓子}{お|くわ|し}でも
\ruby{御食}{お|あが}りなすつて。
』

『
\ruby{厭}{いや}ネエ、ふざけて!。
\ruby{姊}{ねえ}さんは
\ruby{人}{ひと}が
\ruby{惡}{わる}いは。
』

とお
\ruby{龍}{りう}は
\ruby{一寸}{ちよ|つと}
\ruby{瞋}{おこ}つたるやうな
\ruby{顏}{かほ}して
\ruby{云}{い}ひ、

『それに
\ruby{此}{こ}の
\ruby{御菓子}{お|くわ|し}は
\ruby{妾}{わたし}は
\ruby{澤山}{たく|さん}ですよ。
』

といふ。

『
\ruby{{\換字{嫌}}}{きら}ひ?。
』

と
\ruby{女主人}{あ|る|じ}は
\ruby{輕}{かろ}く
\ruby{眞面目}{ま|じ|め}に
\ruby{問}{と}ふ。
\ruby{問}{と}はれて
\ruby{莞爾}{にこ|やか}なる
\ruby{舊}{もと}に
\ruby{復}{かへ}りながら、

『まあ
\ruby{左樣}{さ|う}なの。
』

と
\ruby{氣}{き}の
\ruby{毒}{どく}さうに
\ruby{答}{こた}へたるは、
\ruby{思}{おも}はず
\ruby{我}{わ}が
\ruby{好}{す}き
\ruby{{\換字{嫌}}}{きら}ひの
\ruby{我儘}{わが|まゝ}を
\ruby{口走}{くち|ばし}つたる
\ruby{無{\換字{遠}}慮}{ぶ|ゑん|りよ}を
\ruby{羞}{は}ぢて、
\ruby{今}{いま}さら
\ruby{詮方無}{せん|かた|な}くも
\ruby{{\換字{猶}}}{なほ}
\ruby{少}{すこ}し
\ruby{曖昧}{あい|まい}に
\ruby{言葉}{こと|ば}を
\ruby{濁}{にご}せるなるべし。

『いけなかつたネエ、
\ruby{甘味{\換字{嫌}}}{あま|い|ぎら}ひとばつかり
\ruby{思}{おも}つて
\ruby{居}{ゐ}て
\ruby{此品}{こ|れ}が
\ruby{{\換字{嫌}}}{きら}ひだつたとは
\ruby{知}{し}らなかつたよ。
もつともネ、
\ruby{一體此}{いつ|たい|これ}は
\ruby{御茶}{お|ちや}にも
\ruby{餘}{あま}り
\ruby{賞}{ほ}めたものぢやあ
\ruby{無}{な}いの。
そればかりぢやあ
\ruby{無}{な}い、
\ruby{鳥貝}{とり|がひ}の
\ruby{御鮨}{お|す}もじだの
\ruby{玉簾}{たま|だれ}だのといふものは、
\ruby{惡}{わる}く
\ruby{氣取}{き|ど}つた
\ruby{女}{ひと}に
\ruby{食}{た}べさせて
\ruby{{\換字{遣}}}{や}れなんぞといふ
\ruby{位}{くらゐ}のものだつたのに、つい
\ruby{妾}{わたし}が
\ruby{氣}{き}が
\ruby{注}{つ}かなかつたよ、
\ruby{堪忍}{かん|にん}おし。
\ruby{今}{いま}
\ruby{他}{ほか}のものを
\ruby{何}{なん}ぞあげるから。
』

『
\ruby{何故}{な|ぜ}?。
\ruby{氣取}{き|ど}つた
\ruby{女}{ひと}が
\ruby{何樣}{ど|う}か
\ruby{仕}{し}でもするの?。
』

『ソレ
\ruby{烏貝}{とり|がひ}は
お
\ruby{{\換字{前}}早}{まへ|はや}くは
\ruby{咬}{か}み
\ruby{切}{き}れないし、
\ruby{玉簾}{たま|だれ}はホロ〳〵と
\ruby{零}{こぼ}れ
\ruby{{\換字{勝}}}{かち}だし
\ruby{辛}{から}くはあるしするからネ。
いつまでも
\ruby{口}{くち}をムグ〳〵させて
\ruby{居}{ゐ}たり、だらし
\ruby{無}{な}く
\ruby{膝}{ひざ}を
\ruby{汚}{よご}して、そして
\ruby{辛}{から}さを
\ruby{辛抱}{しん|ぼう}する
\ruby{泣顏}{なき|がほ}を
\ruby{仕}{し}て
\ruby{居}{ゐ}たりするのなんぞは
\ruby{見好}{み|い}いものぢやあ
\ruby{無}{な}いからさ。
』

『あらツ!、
\ruby{妾}{わたし}あ
\ruby{其樣}{そ|ん}な
\ruby{譯}{わけ}で
\ruby{{\換字{嫌}}}{きら}ひだつていふのぢやあ
\ruby{有}{あ}りませんは。
\ruby{姊}{ねえ}さんのところへ
\ruby{來}{き}て
\ruby{一寸}{ちよ|いと}だつて
\ruby{氣}{き}を
\ruby{置}{お}いてなんぞ
\ruby{居}{ゐ}やあ
\ruby{仕}{し}ませんのに。
\ruby{好}{よ}うござんすよ、
\ruby{一人}{ひと|り}で
\ruby{悉皆}{みん|な}
\ruby{頂}{いたゞ}いて
\ruby{仕舞}{し|ま}つて、
\ruby{其邊中食}{そこ|いら|ぢゆう|た}べ
\ruby{零}{こぼ}して、そうして
\ruby{澤山}{たん|と}
\ruby{見}{み}つとも
\ruby{無}{な}い
\ruby{泣顏}{なき|がほ}をして、
\ruby{笑}{わら}つていたゞきますから。
』

『ホヽホヽホヽ、ホラ
\ruby{始}{はじ}まつたよ
お
\ruby{龍}{りう}ちやんの
\ruby{癇癖}{む|し}が。
だがお
\ruby{{\換字{前}}}{まへ}が
\ruby{一寸}{ちよ|いと}
\ruby{口惜}{く|や}しいといふ
\ruby{思入}{おもひ|いれ}をすると、
\ruby{色艶}{いろ|つや}は
\ruby{好}{よ}
し、
\ruby{眼}{め}は
\ruby{淸}{すゞ}しいし、
\ruby{眉毛}{まみ|え}は
\ruby{奇麗}{き|れい}だし、それが
\ruby{悉皆}{みん|な}
\ruby{役}{やく}に
\ruby{立}{た}つて
\ruby{顏中}{かほ|ぢゆう}が
\ruby{活}{い}きて
\ruby{見}{み}えて
\ruby{來}{き}て、ほんとに
\ruby{婀娜}{あ|だ}で
\ruby{可憐}{かは|い}らしいよ。
』

『
\ruby{好}{よ}うござんすよ。
』

\ruby{此度}{こ|たび}はいよ〳〵
\ruby{瞋}{いか}りていよ〳〵
\ruby{言葉}{こと|ば}
\ruby{少}{すくな}く、
\ruby{恨}{うら}めしげに\換字{志}ろりと
お
\ruby{彤}{とう}を
\ruby{睨}{にら}みて、つんとして
\ruby{其}{そ}の
\ruby{儘横}{まゝ|よこ}を
\ruby{向}{む}かんとせしが、
\ruby{閑事}{あだ|ごと}は
\ruby{兎}{と}に
\ruby{角}{かく}、
\ruby{云}{い}はで
\ruby{叶}{かな}はざる
\ruby{用事}{よう|じ}はあるなり、
\ruby{霎時間}{しば|し|ま}を
\ruby{置}{お}きて
\ruby{面}{おもて}を
\ruby{擡}{あ}げ、

『ネエ、
\ruby{姊}{ねえ}さん、
\ruby{今}{いま}
\ruby{彼室}{あつ|ち}で
\ruby{云}{い}ひかけたのはネ、
\ruby{眞個}{ほん|と}に
\ruby{妾}{わたし}の
\ruby{御願}{お|ねが}ひの
\ruby{事}{こと}なんですから
\ruby{聽}{き}いて
\ruby{下}{くだ}さいましな。
』

と、
\ruby{心配氣}{しん|ぱい|げ}に
お
\ruby{彤}{とう}が
\ruby{面色}{かほ|つき}を
\ruby{見}{み}ながら、いつはりならず
\ruby{心}{こゝろ}を
\ruby{籠}{こ}めて
\ruby{云}{い}ひ
\ruby{出}{いだ}したり。

『あゝ
\ruby{可}{い}いとも。
お
\ruby{{\換字{前}}}{まへ}の
\ruby{御頼}{お|たの}みの
\ruby{事}{こと}なら
\ruby{何}{なん}でも
\ruby{聽}{き}いてあげるとも。
』

\ruby{此}{これ}は
\ruby{極}{きは}めて
\ruby{易}{やす}らかなる
\ruby{語氣}{ご|き}のいと
\ruby{輕}{かろ}き
\ruby{答}{こたへ}なり。

『ほんとに?。
』

\ruby{此方}{こな|た}は
\ruby{力}{ちから}を
\ruby{入}{い}れて
\ruby{重}{かさ}ねて
\ruby{問}{と}へば、
\ruby{彼方}{かな|た}は
\ruby{沈靜}{おち|つき}きつて
\ruby{{\換字{平}}氣}{へい|き}に、

『あゝ、ほんたうにさ!。
』

と
\ruby{事}{こと}も
\ruby{無}{な}げなり。

『あゝ
\ruby{姊}{ねえ}さん
\ruby{有}{あ}り
\ruby{難}{がた}うございます、
\ruby{一生記}{いつ|しやう|おぼ}えて
\ruby{居}{ゐ}ますよ。
ぢやあ
\ruby{申}{まを}しますがネ。
かういふ
\ruby{譯}{わけ}なんです。
』

と
\ruby{{\換字{説}}}{と}き
\ruby{出}{いだ}さんとするを
お
\ruby{彤}{とう}は
\ruby{抑}{おさ}へて、

『
\ruby{可}{い}いよ
お
\ruby{龍}{りう}ちやん、かういふのだらう。
\ruby{彼}{あ}の
\ruby{水野}{みづ|の}さんていふ
\ruby{人}{ひと}が
\ruby{職務}{や|く}を
\ruby{離}{はな}れたに
\ruby{就}{つ}いちやあ、
\ruby{何樣}{ど|う}か
\ruby{彼}{あ}の
\ruby{人}{ひと}を
\ruby{困窮}{こ|ま}らせたく
\ruby{無}{な}いので、
\ruby{妾}{わたし}に
\ruby{口}{くち}をきいて
\ruby{貰}{もら}つたら
\ruby{家}{うち}の
\ruby{旦那}{だん|な}の
\ruby{方}{はう}にでも
\ruby{好}{い}い
\ruby{口}{くち}が
\ruby{有}{あ}りやあ
\ruby{仕}{し}まいか、
\ruby{出來}{で|き}る
\ruby{事}{こと}なら
\ruby{好}{い}い
\ruby{口}{くち}を
\ruby{搜}{さが}し
\ruby{出}{だ}して
\ruby{持}{も}つて
\ruby{行}{い}つて
\ruby{{\換字{遣}}}{や}りたい。
と、かういふところからのお
\ruby{{\換字{前}}}{まへ}の
\ruby{御頼}{お|たの}みなのぢや
\ruby{無}{な}くつて?。
』

と
\ruby{全}{まつた}く
お
\ruby{龍}{りう}の
\ruby{胸}{むね}の
\ruby{奧}{おく}の
\ruby{{\換字{文}}}{あや}を
\ruby{鏡}{かゞみ}に
\ruby{取}{と}りて
\ruby{見}{み}る
\ruby{如}{ごと}く
\ruby{云}{い}ひ
\ruby{出}{だ}したり。

\ruby{云}{い}はれて
お
\ruby{龍}{りう}は
\ruby{驚}{おどろ}いて
\ruby{眼}{め}を
\ruby{睜}{みは}り、

『まあ、
\ruby{何樣}{ど|う}して
\ruby{然樣不殘}{さ|う|し|て}
\ruby{姊}{ねえ}さんは
\ruby{知}{し}つてゝ?。
\ruby{姊}{ねえ}さんの
\ruby{智慧}{ち|ゑ}の
\ruby{深}{ふか}いのは
\ruby{{\換字{前}}}{せん}から
\ruby{知}{し}つてますが、ほんとにまあ、
\ruby{何樣}{ど|う}すれば
\ruby{其樣}{そん|な}に
\ruby{人}{ひと}の
\ruby{意}{き}が
\ruby{解}{わか}るの?。
\ruby{妾}{わたし}あ
\ruby{餘}{あんま}り
\ruby{其}{そ}の
\ruby{通}{とほ}りなので
\ruby{怖}{こは}いやうな
\ruby{氣}{き}が
\ruby{仕}{し}ますよ。
\ruby{全}{まつた}く
\ruby{然樣}{さ|う}いふ
\ruby{譯}{わけ}の
\ruby{御願}{お|ねがひ}でわざ〳〵
\ruby{來}{き}たのですが
\ruby{何樣}{ど|う}いふものでしやう?、
\ruby{姊}{ねえ}さん、
\ruby{聽}{き}いて
\ruby{下}{くだ}すつて?。
』

と
\ruby{正直}{しやう|ぢき}になつて
\ruby{頼}{たの}み
\ruby{聞}{きこ}ゆるを、
お
\ruby{彤}{とう}は
\ruby{憐}{あはれ}むが
\ruby{如}{ごと}く
\ruby{憐}{あはれ}まざるが
\ruby{如}{ごと}く
\ruby{冷}{ひやゝ}かに
\ruby{見}{み}やりて、

『
\ruby{頼}{たの}みを
\ruby{聽}{き}くも
\ruby{聽}{き}かないも
\ruby{有}{あ}りやあ
\ruby{仕}{し}ないがネ、
お
\ruby{龍}{りう}ちやん、
お
\ruby{{\換字{前}}}{まへ}そりやあ
\ruby{詰}{つま}らない
\ruby{事}{こと}だらうよ。
』

と、いと
\ruby{物靜}{もの|しづ}かに
\ruby{先}{ま}づ
\ruby{一句云}{いつ|く|い}ひ
\ruby{斷}{き}りたり。


\Entry{其十四}

\ruby{我}{わ}が
\ruby{胸}{むね}の
\ruby{中}{うち}の
\ruby{{\換字{所}}思}{おも|はく}の
\ruby{底}{そこ}を
\ruby{盡}{つく}して
\ruby{{\換字{説}}}{と}き
\ruby{中}{あ}てられたるに、
\ruby{一度}{ひと|たび}は
\ruby{先}{ま}づ
\ruby{驚}{おどろ}き
\ruby{服}{ふく}したるも、
\ruby{其}{そ}れを
\ruby{詰}{つま}らぬことゝ
\ruby{唯}{たゞ}
\ruby{一言}{ひと|こと}に
\ruby{斥}{しりぞ}けられては、
\ruby{物}{もの}に
\ruby{堪}{こら}へぬ
お
\ruby{龍}{りう}の
\ruby{心{\換字{平}}}{こゝろ|たひ}らかならず、
\ruby{思}{おも}はず
\ruby{顏}{かほ}を
\ruby{突}{つ}と
\ruby{擡}{あ}げて、

『
\ruby{何故}{な|ぜ}ネエ。
』

と
\ruby{詰}{なじ}り
\ruby{氣味}{ぎ|み}に
\ruby{咄嗟}{とつ|さ}に
\ruby{言葉}{こと|ば}を
\ruby{{\換字{返}}}{かへ}しゝが、
\ruby{見}{み}れば
\ruby{{\換字{古}}風}{こ|ふう}の
\ruby{内裏雛}{だい|り|びな}の
\ruby{如}{ごと}くに
\ruby{端然}{しや|ん}としたる
\ruby{面}{かほ}つきの、
\ruby{細}{ほそ}けれど
\ruby{亘}{わたり}の
\ruby{長}{なが}くして
\ruby{特}{こと}にはつきりと
\ruby{明}{あき}らかなる
\ruby{眼}{め}を、
\ruby{我}{わ}が
\ruby{上}{うへ}に\換字{志}つと
お
\ruby{彤}{とう}の
\ruby{注}{そゝ}ぎ
\ruby{居}{ゐ}たるに、
\ruby{其}{そ}の
\ruby{沈靜}{おち|つ}きたる
\ruby{態度}{やう|す}の
\ruby{中}{うち}に
\ruby{具}{そな}はれる
\ruby{自然}{おのづ|から}の
\ruby{威}{ゐ}は、
\ruby{輕々}{かろ|〴〵}しく
\ruby{慌}{あわ}たゞしき
\ruby{我}{われ}を
\ruby{壓}{お}す
\ruby{如}{ごと}く
\ruby{覺}{おぼ}えて、
\ruby{何}{なん}といふ
\ruby{事}{こと}は
\ruby{無}{な}けれど
\ruby{當}{あた}り
\ruby{難}{がた}き
\ruby{心地}{こゝ|ち}の
\ruby{爲}{し}、
\ruby{氣勢忽}{いき|ほい|たちま}ち
\ruby{挫}{くじ}けて
\ruby{語氣}{ご|き}も
\ruby{萎々}{なえ|〳〵}と、

『
\ruby{詰}{つま}らないつて、
\ruby{其}{そ}りやあ
\ruby{然樣}{さ|う}かも
\ruby{知}{し}りませんけれども、
\ruby{妾}{わたし}にやあ
\ruby{些}{ちつと}も
\ruby{然樣}{さ|う}は
\ruby{思}{おも}へませんは。
\ruby{下}{くだ}らないかも
\ruby{知}{し}りませんけれども、
\ruby{妾}{わたし}の
\ruby{思}{おも}つてる
\ruby{事}{こと}を、ネエ
\ruby{姊}{ねえ}さんどうか
\ruby{一}{ひ}ト
\ruby{{\換字{通}}}{とほ}り
\ruby{聞}{き}いて
\ruby{見}{み}て
\ruby{下}{くだ}さいな。
』

と、
\ruby{憐愍}{あは|れみ}を
\ruby{乞}{こ}ふが
\ruby{如}{ごと}くに
\ruby{云}{い}ひ
\ruby{足}{た}したり。

\ruby{人}{ひと}に
\ruby{頼}{たの}みごとするものゝ
\ruby{心}{こゝろ}の
\ruby{中}{うち}ほど
\ruby{苦}{くる}しきは
\ruby{無}{な}し。
\ruby{{\換字{強}}}{し}ひるほどに
\ruby{頼}{たの}まねば
\ruby{願望}{ねが|ひ}は
\ruby{成}{な}り
\ruby{難}{がた}く、
\ruby{{\換字{強}}}{し}ひ
\ruby{{\換字{過}}}{す}ぎて
\ruby{怒}{おこ}られて
\ruby{仕舞}{し|ま}へばそれまでなれば、
\ruby{願}{ねが}ふ
\ruby{意}{こゝろ}の
\ruby{切}{せつ}なるだけ、
\ruby{我}{わ}が
\ruby{言葉}{こと|ば}の
\ruby{斟{\換字{酌}}}{しん|しやく}に
\ruby{氣}{き}を
\ruby{使}{つか}ひて、
\ruby{斯樣}{か|う}
\ruby{云}{い}ひて
\ruby{宜}{よ}かるべきか
\ruby{惡}{あし}かるべきかの
\ruby{心配}{しん|ぱい}に、
\ruby{人知}{ひと|し}れず
\ruby{幾干}{いく|そ}の
\ruby{胸}{むね}を
\ruby{痛}{いた}むるなり。
お
\ruby{彤}{とう}は
\ruby{我}{わ}が
\ruby{愛}{あい}する
お
\ruby{龍}{りう}がいぢらしき
\ruby{心}{こゝろ}の
\ruby{中}{うち}を、
\ruby{早}{はや}くも
\ruby{其}{そ}の
\ruby{目色語氣}{め|いろ|ことば|つき}に
\ruby{猜}{すゐ}し
\ruby{知}{し}りて、たちまちに
\ruby{面}{おもて}を
\ruby{和}{やは}らげ
\ruby{笑}{ゑみ}を
\ruby{爲}{つく}りつ、

『まあお
\ruby{龍}{りう}ちやんの
\ruby{思}{おも}つてる
\ruby{事}{こと}つて
\ruby{何樣}{ど|う}いふ
\ruby{事}{こと}なの?。
』

と、
\ruby{云}{い}ひ
\ruby{出}{い}で
\ruby{易}{やす}きやうに
\ruby{路}{みち}を
\ruby{開}{ひら}きたり。

お
\ruby{龍}{りう}はこれに
\ruby{勢}{いきほひ}を
\ruby{得}{え}て、

『
\ruby{經{\換字{過}}}{ゆく|たて}を
\ruby{御話}{お|はなし}
\ruby{仕}{し}ないぢやあ、
\ruby{何}{なん}だか
\ruby{單}{たゞ}、
\ruby{妾}{わたし}の
\ruby{餘計}{よ|けい}な
\ruby{物數寄}{も|の|ずき}のやうに
\ruby{聞}{きこ}えますからネ、
\ruby{長}{なが}つたらしくても
\ruby{最初}{さい|しよ}つからいひますよ。
まあ
\ruby{一番}{いち|ばん}
\ruby{初}{はじめ}つからいひますとネ。
』

と、
\ruby{先}{ま}づ
\ruby{語}{かた}り
\ruby{出}{いだ}して
\ruby{縷々}{る|ゝ}と
\ruby{語}{かた}りつゞけぬ。

『もと
\ruby{彼}{あ}の
\ruby{水野}{みづ|の}つていふ
\ruby{人}{ひと}は
\ruby{妾}{わたし}の
\ruby{知}{し}つてた
\ruby{人}{ひと}でも
\ruby{何}{なん}でも
\ruby{有}{あ}りやあ
\ruby{仕}{し}ませんがネ。
\ruby{今}{いま}
\ruby{妾}{わたし}の
\ruby{世話}{せ|わ}になつてる
お
\ruby{師匠}{し|よ}さんに
\ruby{義女}{まゝ|つこ}があるのです。
\ruby{會}{あ}
つた
\ruby{事}{こと}が
\ruby{無}{な}いから
\ruby{面}{かほ}は
\ruby{知}{し}りませんが
\ruby{好}{い}い
\ruby{容貌}{きり|やう}ださうだし、
\ruby{學問}{がく|もん}も
\ruby{中々}{なか|〳〵}あるさうで
\ruby{敎師}{けう|し}さんを
\ruby{仕}{し}て
\ruby{居}{ゐ}るんです。
お
\ruby{五十}{い|そ}さんといつて、
\ruby{沈毅者}{しつ|かり|もん}でネ、もとつから
\ruby{繼母}{おつ|かさん}とは
\ruby{氣}{き}が
\ruby{合}はないので
\ruby{全然}{まる|で}
\ruby{離}{はな}れて
\ruby{居}{ゐ}て、
\ruby{一人}{ひと|り}
\ruby{立}{だち}で
\ruby{何樣}{ど|う}か
\ruby{斯樣}{か|う}か
\ruby{{\換字{遣}}}{や}つて
\ruby{行}{い}つてたのです。
\ruby{世話}{せ|わ}になつて
\ruby{居}{ゐ}て
\ruby{惡}{わる}く
\ruby{云}{い}つちやあ
\ruby{濟}{す}みませんがネ、
お
\ruby{師匠樣}{し|よ|さん}は
\ruby{隨{\換字{分}}}{ずゐ|ぶん}
\ruby{我儘}{わが|まゝ}ぢやあ
\ruby{有}{あ}り、
\ruby{品行}{おこ|なひ}だつて
\ruby{堅}{かた}い
\ruby{方}{はう}ぢやあ
\ruby{無}{な}い
\ruby{{\換字{勝}}手}{かつ|て}な
\ruby{人}{ひと}ですから、
\ruby{眞正}{ほん|たう}の
\ruby{理屈}{り|くつ}を
\ruby{云}{い}やあ
\ruby{端正}{しや|ん}として
\ruby{居}{ゐ}る
お
\ruby{五十}{い|そ}さんの
\ruby{方}{はう}が
\ruby{正}{い}いのでしやうサ。
だけれどもお
\ruby{師匠}{し|よ}さんに
\ruby{云}{い}はせりやあ、
\ruby{變}{へん}に
\ruby{高慢}{かう|まん}で、
\ruby{執拗}{かた|いぢ}な
\ruby{可厭}{い|や}な
\ruby{女}{ひと}だつて
\ruby{云}{い}ふんです。
まあ
\ruby{其}{それ}あ
\ruby{何方}{どつ|ち}が
\ruby{眞正}{ほん|たう}だか
\ruby{會}{あ}つて
\ruby{見}{み}ない
\ruby{人}{ひと}の
\ruby{事}{こと}ですから
\ruby{{\換字{分}}}{わか}りませんけともネ、
\ruby{其}{そ}の
お
\ruby{五十}{い|そ}さんていふのが
\ruby{弟}{おとうと}の
\ruby{世話}{せ|わ}まで
\ruby{燒}{や}いてゐるのに、
お
\ruby{師匠}{し|よ}さんは
\ruby{何}{なんに}も
\ruby{少}{すこし}も
\ruby{管}{かま}はないで、
\ruby{自{\換字{分}}}{じ|ぶん}で
\ruby{取}{と}るものは
\ruby{自{\換字{分}}}{じ|ぶん}で
\ruby{使}{つか}つて
お
\ruby{酒}{さけ}なんぞを
\ruby{飮}{の}んでるのですもの、まあ
\ruby{何樣}{ど|う}しても
お
\ruby{師匠樣}{し|よ|さん}の
\ruby{方}{はう}に
\ruby{阿{\換字{扇}}}{うち|は}は
\ruby{上}{あ}げられませんやネ。
ところが
\ruby{其}{そ}の
お
\ruby{五十}{い|そ}さんといふ
\ruby{人}{ひと}が
\ruby{窒扶斯}{ち|ぶ|す}を
\ruby{患}{わづ}らつて、
\ruby{生死}{いき|しに}の
\ruby{{\換字{分}}}{わか}らない
\ruby{怖}{こは}い
\ruby{瀬}{せ}にかかつたのです。
それを
\ruby{何樣}{ど|う}でしやう
\ruby{家}{うち}の
\ruby{御師匠樣}{お|し|よ|さん}は
\ruby{振}{ふ}り
\ruby{向}{む}いても
\ruby{見}{み}ないのです。
もとよりお
\ruby{五十}{い|そ}さんが
\ruby{財産}{も|の}を
\ruby{有}{も}つて
\ruby{居}{ゐ}やうぢやあ
\ruby{無}{な}し
\ruby{弟}{おとうと}ツ
\ruby{兒}{こ}はまだ
\ruby{一向}{いつ|かう}の
\ruby{小兒}{こ|ども}なんですもの、
\ruby{困}{こま}つて
\ruby{仕舞}{し|ま}ふのは
\ruby{知}{し}れ
\ruby{切}{き}つて
\ruby{居}{ゐ}ます。
\ruby{其處}{そ|こ}で
\ruby{彼}{あ}の
\ruby{水野}{みづ|の}さんていふ
\ruby{人}{ひと}が
\ruby{世話}{せ|わ}を
\ruby{仕}{し}たのでしてネ、
\ruby{彼}{あ}の
\ruby{人}{ひと}は
お
\ruby{師匠樣}{し|よ|さん}にも
お
\ruby{五十}{い|そ}さんにも
\ruby{赤}{あか}の
\ruby{他人}{た|にん}なのです!。
』


\Entry{其十五}

『
\ruby{{\換字{過}}日}{こな|ひだ}も
\ruby{一寸}{ちよ|つと}
\ruby{御話}{お|はなし}を
\ruby{仕}{し}たのですから
\ruby{諄}{くど}くは
\ruby{云}{い}ひませんが、
\ruby{其}{そ}の
\ruby{赤}{あか}の
\ruby{他人}{た|にん}の
\ruby{彼}{あ}の
\ruby{人}{ひと}と
お
\ruby{五十}{い|そ}さんとの
\ruby{間}{あひだ}は、たゞ
\ruby{互}{たがひ}に
\ruby{同}{おな}じ
\ruby{學校}{がく|かう}に
\ruby{奉職}{つ|と}めて
\ruby{居}{ゐ}るといふだけの
\ruby{事}{こと}です。
そりやあ
\ruby{成程}{なる|ほど}
お
\ruby{五十}{い|そ}さんを
\ruby{思}{おも}つて
\ruby{居}{ゐ}るからとはいふものゝ、
\ruby{何}{なに}も
\ruby{有}{あ}り
\ruby{餘}{あま}つて
\ruby{居}{ゐ}る
\ruby{人}{ひと}ぢやあ
\ruby{無}{な}し、
\ruby{學校}{がく|かう}の
\ruby{先生}{せん|せい}なんぞを
\ruby{仕}{し}て
\ruby{居}{ゐ}るのですもの、その
\ruby{懷中合}{ふと|ころ|あひ}も
\ruby{知}{し}れて
\ruby{居}{ゐ}ますはネ。
その
\ruby{樂}{らく}でも
\ruby{無}{な}い
\ruby{人}{ひと}が
\ruby{無}{な}け
\ruby{無}{な}しの
\ruby{中}{なか}で
\ruby{何樣}{ど|う}か
\ruby{工夫}{く|ふう}をして、
お
\ruby{醫者}{い|しや}さんも
\ruby{頼}{たの}んで
\ruby{來}{く}る、
\ruby{看護{\換字{婦}}}{かん|ご|ふ}も
\ruby{附}{つ}ける、
\ruby{下働}{した|ばたら}きの
\ruby{小婢}{こ|をんな}まで
\ruby{添}{そ}へて
\ruby{置}{お}いたと
\ruby{云}{い}ふなあ、
\ruby{普通大抵}{な|み|たい|てい}の
\ruby{親切}{しん|せつ}ぢやあ
\ruby{出來}{で|き}ません。
でもまたお
\ruby{五十}{い|そ}さんが
\ruby{彼}{あ}の
\ruby{人}{ひと}
と
\ruby{思}{おも}ひ
\ruby{合}{あ}つて
\ruby{居}{ゐ}て、あの
\ruby{人}{ひと}の
\ruby{親切}{しん|せつ}を
\ruby{身}{み}に
\ruby{沁}{し}みて
\ruby{悅}{よろこ}んで
\ruby{心底}{しん|そこ}から
\ruby{嬉}{うれ}しいとでも
\ruby{思}{おも}ふといふのなら、
\ruby{隨{\換字{分}}}{ずゐ|ぶん}
\ruby{彼}{あ}の
\ruby{人}{ひと}も
\ruby{苦}{くるし}み
\ruby{甲{\換字{斐}}}{が|ひ}がありましやうが、
\ruby{性}{しやう}が
\ruby{合}{あ}はないとでも
\ruby{云}{い}ふのでしやうか、
\ruby{御師匠}{お|し|よ}さんの
\ruby{談}{はなし}では
\ruby{{\換字{嫌}}}{きら}つて
\ruby{{\換字{嫌}}}{きら}ひ
\ruby{拔}{ぬ}いて、
\ruby{有難}{あり|がた}いとも
\ruby{嬉}{うれ}しいとも
\ruby{思}{おも}ひさうも
\ruby{無}{な}いといふんですもの、
\ruby{彼}{あ}の
\ruby{人}{ひと}の
\ruby{立}{た}つ
\ruby{瀬}{せ}は
\ruby{有}{あ}りやあ
\ruby{仕}{し}ませんはネ。
それに
\ruby{段々}{だん|〳〵}と
\ruby{吾家}{う|ち}の
\ruby{御師匠}{お|し|よ}さんの
\ruby{口占}{くち|うら}を
\ruby{引}{ひ}いて
\ruby{見}{み}ますと、
\ruby{今度}{こん|ど}の
\ruby{事}{こと}の
\ruby{起}{おこ}るずつと
\ruby{{\換字{前}}}{まへ}から、
お
\ruby{師匠}{し|よ}さんは
\ruby{彼}{あ}の
\ruby{人}{ひと}が
お
\ruby{五十}{い|そ}さんを
\ruby{思}{おも}つてるのに
\ruby{附{\換字{込}}}{つけ|こ}んでネ、
\ruby{將來}{ゆく|〳〵}は
お
\ruby{五十}{い|そ}をあげましやうといふやうな
\ruby{事}{こと}を
\ruby{巧}{うま}く
\ruby{匂}{にほ}はせて、
\ruby{何}{なん}とか
\ruby{彼}{か}とか
\ruby{口實}{いひ|ぐさ}を
\ruby{拵}{こしら}へては
\ruby{{\換字{若}}干金}{い|く|ら}かづつ
\ruby{絞}{しぼ}つたらしいので、どうも
\ruby{後{\換字{前}}}{あと|さき}を
\ruby{能}{よう}く
\ruby{考}{かんが}へて
\ruby{見}{み}ると
\ruby{屹度}{きつ|と}さうなのですよ。
』

『へーエ、
\ruby{罪}{つみ}な
\ruby{事}{こと}を
\ruby{仕}{し}たものだネエ!、
お
\ruby{關}{せき}さんといふ
\ruby{人}{ひと}は。
』

『
\ruby{罪}{つみ}ですともほんとに!。
あんな
\ruby{生眞面目}{き|ま|じ|め}な
\ruby{初心}{う|ぶ}な
\ruby{人}{ひと}を
\ruby{欺}{だま}すのですもの。
』

『ぢやあ、お
\ruby{{\換字{前}}}{まへ}の
\ruby{御師匠}{お|し|よ}さんていふ
\ruby{人}{ひと}は
\ruby{惡}{わる}い
\ruby{人}{ひと}ちやあ
\ruby{無}{な}いか。
』

『
\ruby{唯}{えゝ}、まあ
\ruby{善}{い}い
\ruby{人}{ひと}たあ
\ruby{御師匠樣}{お|し|よ|さん}ですけれども
\ruby{云}{い}へませんネエ。
で、
\ruby{吾家}{う|ち}の
\ruby{御師匠樣}{お|し|よ|さん}が
\ruby{萬一}{も|し}
\ruby{普{\換字{通}}}{ひと|なみ}に
\ruby{人{\換字{情}}合}{にん|じよう|あひ}の
\ruby{{\換字{分}}}{わか}る
\ruby{人}{ひと}ならば、
\ruby{從{\換字{前}}}{いま|ゝで}の
\ruby{事}{こと}は
\ruby{何樣}{ど|う}でも
\ruby{斯樣}{か|う}でも
\ruby{濟}{す}んだことだから
\ruby{仕方}{し|かた}が
\ruby{無}{な}いとしても、
\ruby{今度}{こん|ど}は
\ruby{云}{い}はゞ
\ruby{水野}{みづ|の}さんの
\ruby{世話一}{せ|わ|ひと}ツで
お
\ruby{五十}{い|そ}さんを
\ruby{取}{と}り
\ruby{{\換字{留}}}{と}めたのですから、
\ruby{床上}{とこ|あ}げでも
\ruby{濟}{す}んだ
\ruby{其}{そ}の
\ruby{曉}{あかつき}にやあ、たとひ
お
\ruby{五十}{い|そ}さんが
\ruby{何}{なん}と
\ruby{云}{い}はうとも
\ruby{割}{わつ}つ
\ruby{口說}{く|ど}いつして、
\ruby{水野}{みづ|の}さんに
\ruby{嫁}{や}るやうにでも
\ruby{仕}{し}なくちやあならない
\ruby{筈}{はず}だと
\ruby{思}{おも}ひますは。
ネエ
\ruby{姊}{ねえ}さん、
\ruby{然樣}{さ|う}ぢやあ
\ruby{有}{あ}りませんか、
\ruby{義理}{ぎ|り}つてえものがネエ。
』

『
\ruby{成程}{なる|ほど}
お
\ruby{{\換字{前}}}{まへ}が
お
\ruby{五十}{い|そ}さんの
\ruby{御母}{お|つか}さんだつたら
\ruby{然樣}{さ|う}も
\ruby{御爲}{お|し}だらうとおもはれるよ。
』

お
\ruby{龍}{りう}は
\ruby{此}{こ}の
お
\ruby{彤}{とう}が
\ruby{答}{こたへ}に
\ruby{少}{すくな}からぬ
\ruby{不足}{ふ|そく}の
\ruby{色}{いろ}を
\ruby{現}{あらは}
したり。

『ぢやあ
\ruby{姊}{ねえ}さんが
\ruby{{\換字{若}}}{も}し
\ruby{御師匠}{お|し|よ}さんだつたら?。
』

『ホヽヽ、
\ruby{挨拶}{あい|さつ}が
\ruby{些氣}{ちつと|き}に
\ruby{入}{い}らなかつたネ。
\ruby{妾}{わたし}が
お
\ruby{五十}{い|そ}さんの
\ruby{母}{おつか}さんならカエ。
さうさねエ、
\ruby{妾}{わたし}ならまあ、
\ruby{先}{さき}へ
\ruby{恩{\換字{返}}}{おん|がへ}しを
\ruby{仕}{し}て
\ruby{置}{お}いてネ、……
\ruby{世話}{せ|わ}になつた
\ruby{恩}{おん}は
\ruby{恩}{おん}で
\ruby{水野}{みづ|の}さんに
\ruby{恩{\換字{返}}}{おん|がへ}しを
\ruby{仕}{し}てネ、
\ruby{緣}{えん}の
\ruby{事}{こと}は
\ruby{其}{それ}から
\ruby{後}{あと}で
\ruby{決}{き}めやうと
\ruby{思}{おも}ふネ。
』

『
\ruby{然樣}{さ|う}!。
それならそれで
\ruby{其}{それ}もまた
\ruby{譯}{わけ}の
\ruby{{\換字{分}}}{わか}つた
\ruby{大變}{たい|へん}に
\ruby{良}{い}い
\ruby{仕方}{し|かた}だと
\ruby{妾}{わたし}もおもひますは。
ところが
\ruby{吾家}{う|ち}の
\ruby{御師匠}{お|し|よ}さんは
\ruby{妾}{わたし}の
\ruby{云}{い}つたやうに
\ruby{仕}{し}やうでも
\ruby{無}{な}けりやあ、
\ruby{姊}{ねえ}さんの
お
\ruby{云}{い}ひのやうに
\ruby{仕}{し}やうでも
\ruby{無}{な}いんで、たゞ
\ruby{病患}{わ|る}い
\ruby{時}{とき}やあ
\ruby{人}{ひと}まかせに
\ruby{仕}{し}て
\ruby{置}{お}いて、
\ruby{治}{なほ}
りやあ
\ruby{自{\換字{分}}}{じ|ぶん}の
\ruby{子}{こ}つていふやうな
\ruby{{\換字{勝}}手}{かつ|て}な
\ruby{料簡}{れう|けん}で、いつまでも
\ruby{水野}{みづ|の}さんは
\ruby{釣}{つ}りつばなしに
\ruby{仕}{し}て
\ruby{打棄}{うつ|ちや}つて
\ruby{置}{お}かうといふんですもの、
\ruby{酷}{ひど}いぢやあ
\ruby{有}{あ}りませんか。
』

『そりやあ
\ruby{酷}{ひど}いとも!。
\ruby{酷}{ひど}い
\ruby{人}{ひと}だよ。
\ruby{聞}{き}いて
\ruby{見}{み}りやあ
\ruby{眞個}{ほん|と}に
お
\ruby{{\換字{前}}}{まへ}の
\ruby{御師匠}{お|し|よ}さんて
\ruby{云}{い}ふのは
\ruby{惡}{わる}い
\ruby{人}{ひと}だよ。
』

『でもまあ
\ruby{緣}{えん}の
\ruby{事}{こと}は
\ruby{當人}{たう|にん}
\ruby{同士}{どう|し}の
\ruby{事}{こと}で、
\ruby{親}{おや}の
\ruby{思}{おも}ふやうにばかりもならない
\ruby{理}{すぢ}も
\ruby{有}{あ}りましやう。
ですからお
\ruby{五十}{い|そ}さんが
\ruby{{\換字{嫌}}}{いや}なら
\ruby{{\換字{嫌}}}{いや}で
\ruby{{\換字{強}}}{し}ひるわけには
\ruby{行}{ゆ}かないとして、
\ruby{其}{それ}あ
\ruby{其}{それ}で
\ruby{可}{い}いとしたところが
\ruby{恩}{おん}は
\ruby{恩}{おん}ですもの、
\ruby{恩}{おん}は
\ruby{何處}{ど|こ}までも
\ruby{着}{き}なけりやあなりません。
まして
\ruby{水野}{みづ|の}さんが
\ruby{困}{こま}るといふ
\ruby{時{\換字{節}}}{は|め}になりやあ、
\ruby{何樣}{ど|う}しても
\ruby{知}{し}らん
\ruby{顏}{かほ}ぢやあ
\ruby{居}{ゐ}られない
\ruby{譯}{わけ}で、
\ruby{出來}{で|き}ないまでも
\ruby{心配}{しん|ぱい}だけなりと
\ruby{仕}{し}なくちやあなりませんはネ。
』


\Entry{其十六}

『ところが
\ruby{吾家}{う|ち}の
\ruby{御師匠}{お|し|よ}さんと
\ruby{來}{き}た
\ruby{日}{ひ}にやあ
\ruby{眞個}{ほん|と}に
\ruby{酷}{ひど}い
\ruby{人}{ひと}で、
\ruby{妾}{わたし}がこれ〳〵だといふ
\ruby{話}{はなし}を
\ruby{仕}{し}て
\ruby{聞}{き}かせても、フーン
\ruby{然樣}{さ|う}かエと
\ruby{云}{い}つたばかりで
\ruby{氣}{き}の
\ruby{毒}{どく}とも
\ruby{云}{い}はずに、
\ruby{默}{だま}つて
\ruby{懷手}{ふところ|で}で
\ruby{高處}{たか|み}で
\ruby{見物}{けん|ぶつ}しやうといふんですもの、
\ruby{餘}{あんま}りぢやあ
\ruby{有}{あ}りませんか。
それも
\ruby{水野}{みづ|の}さんが
\ruby{職}{やく}を
\ruby{辭}{じ}すやうになつた
\ruby{其}{そ}の
\ruby{原因}{も|と}が、
\ruby{何}{なに}も
\ruby{關係}{かけ|かまひ}の
\ruby{無}{な}いことなら
\ruby{其}{それ}で
\ruby{宜}{い}いかも
\ruby{知}{し}りませんが、
\ruby{彼}{あ}の
\ruby{人}{ひと}が
\ruby{學藝}{わ|ざ}が
\ruby{出來}{で|き}ないといふのぢやあ
\ruby{無}{な}し、
\ruby{怠惰}{なま|け}たといふのぢやあ
\ruby{無}{な}し、たゞ
お
\ruby{五十}{い|そ}さんに
\ruby{親切}{しん|せつ}にして、
\ruby{信心}{しん|〴〵}まで
\ruby{仕}{し}た
\ruby{其事}{そ|れ}が
\ruby{人目}{ひと|め}に
\ruby{立}{た}つて、
\ruby{傍}{はた}の
\ruby{風{\換字{評}}}{うは|さ}が
\ruby{矢鱈}{や|たら}に
\ruby{喧}{やか}ましくなつて、
\ruby{其}{それ}が
\ruby{爲}{ため}に
\ruby{職}{やく}を
\ruby{{\換字{退}}}{ひ}いたといふのですから、
\ruby{云}{い}はゞ
\ruby{此方}{こつ|ち}の
\ruby{爲}{ため}に
\ruby{然樣}{さ|う}いふ
\ruby{譯}{わけ}になつたのですもの、
\ruby{石佛}{いし|ぼとけ}だつて
\ruby{氣}{き}の
\ruby{毒}{どく}と
\ruby{思}{おも}はずには
\ruby{居}{ゐ}られさうも
\ruby{無}{な}いところです。
それを
\ruby{何樣}{ど|う}でしやう
\ruby{全然}{まる|で}
\ruby{知}{し}らん
\ruby{顏}{かほ}で、
\ruby{濟}{す}まして
\ruby{行}{ゆ}かうといふのです!。
\ruby{人間}{にん|げん}も
\ruby{其}{そ}の
\ruby[<h||]{位}{くらゐ}
\ruby{身{\換字{勝}}手}{み|がつ|て}になれりやあ
\ruby{澤山}{たく|さん}だと
\ruby{思}{おも}ひますは。
』

『だつて
\ruby{惡}{わる}い
\ruby{人}{ひと}なら
\ruby{其}{そ}の
\ruby{位}{くらゐ}の
\ruby{事}{こと}は
\ruby{{\換字{平}}氣}{へい|き}で
\ruby{仕}{し}やうぢあ
\ruby{無}{な}いか。
』

『そりやあ
\ruby{云}{い}つて
\ruby{見}{み}ればまあ
\ruby{其樣}{そ|ん}なもので
\ruby{不思議}{ふ|し|ぎ}はありますまいがネ、
\ruby{丁度}{ちやう|ど}
\ruby{中}{なか}に
\ruby{介}{はさ}まつてゐる
\ruby{妾}{わたし}が
\ruby{兩方}{りやう|はう}を
\ruby{見}{み}ますとネ、つくづく
\ruby{吾家}{う|ち}の
お
\ruby{師匠}{し|ゝやう}さんを
\ruby{餘}{あんま}りだと
\ruby{思}{おも}ふ
\ruby{其}{それ}に
\ruby{{\換字{連}}}{つ}れて
\ruby{水野}{みづ|の}さんが
\ruby{愍然}{かは|いさう}で% 「愍然 か(は)いさう」
\ruby{愍然}{かは|いさう}で、ほんとに% 「愍然 か(は)いさう」
\ruby{何}{なん}といふ
\ruby{愍然}{かは|いさう}な% 「愍然 か(は)いさう」
\ruby{人}{ひと}だらうと
\ruby{身}{み}に
\ruby{浸}{し}みて
\ruby{思}{おも}ひますは。
』

『さうさネエ、まあ
\ruby{愍然}{かは|いさう}で% 「愍然 か(は)いさう」
\ruby{無}{な}い
\ruby{事}{こと}も
\ruby{無}{な}いネエ。
』

『あらツ!、まあ
\ruby{愍然}{かは|いさう}で% 「愍然 か(は)いさう」
\ruby{無}{な}い
\ruby{事}{こと}も
\ruby{無}{な}いネエだなんて、
\ruby{餘}{あんま}りですは。
いくら
\ruby{自{\換字{分}}}{じ|ぶん}が
\ruby{{\換字{迷}}}{まよ}つたのだから
\ruby{仕方}{し|かた}が
\ruby{無}{な}いとは
\ruby{云}{い}ふものゝ、
\ruby{助}{たす}かるか
\ruby{死}{し}ぬかも
\ruby{知}{し}れない
\ruby{病人}{びやう|にん}に
\ruby{對}{むか}つて、
\ruby{心配}{しん|ぱい}も
\ruby{仕}{し}て
\ruby{{\換字{遣}}}{や}る、
お
\ruby{金}{かね}も
\ruby{掛}{か}ける、
\ruby{書生}{しよ|せい}さん
\ruby{風}{ふう}の
\ruby{人}{ひと}だのに
\ruby{信心}{しん|〴〵}まで
\ruby{仕}{し}て、
\ruby{此}{こ}の
\ruby{{\換字{節}}}{せつ}の
\ruby{人}{ひと}
の
\ruby{爲}{し}さうにも
\ruby{無}{な}い
\ruby{觀音樣}{くわん|のん|さま}に
\ruby{手}{て}を
\ruby{合}{あは}せるといふやうな
\ruby{事}{こと}まで
\ruby{爲}{し}たのは、まあよく〳〵の
\ruby{事}{こと}で
\ruby{無}{な}くつちやあ
\ruby{出來}{で|き}ませんは。
それだのに
\ruby{其程}{それ|ほど}
\ruby{思}{おも}つてる
\ruby{人}{ひと}にやあ
\ruby{酷}{ひど}く
\ruby{{\換字{嫌}}}{きら}はれて、そして
\ruby{吾家}{う|ち}の
お
\ruby{師匠}{し|よ}さんにやあ
\ruby{口頭}{くち|さき}だけで
\ruby{綾}{あや}なされて、
\ruby{御腹}{お|なか}の
\ruby{中}{なか}ぢやあ
\ruby{舌}{した}を
\ruby{出}{だ}して
\ruby{笑}{わら}つて
\ruby{居}{ゐ}られて、
\ruby{揚句}{あげ|く}の
\ruby{果}{はて}に
\ruby{取}{と}るものも
\ruby{取}{と}れ
\ruby{無}{な}い
\ruby{身}{み}になつて
\ruby{仕舞}{し|ま}ふなんて、そりやあ
\ruby{男兒}{をと|こ}のことですから
\ruby{胸}{むね}
\ruby{濶}{ひろ}いで\換字{志}やうし、
\ruby{氣性}{きし|やう}も
\ruby{毅然}{しつ|かり}と
\ruby{仕}{し}て
\ruby{居}{ゐ}るらしい
\ruby{人}{ひと}ですから、まんざらくよ〳〵も
\ruby{仕}{し}ますまいが、
\ruby{妾}{わたし}が
\ruby{{\換字{若}}}{も}し
\ruby{彼}{あ}の
\ruby{人}{ひと}の
\ruby{身}{み}だつたら、まあ
\ruby{何樣}{ど|ん}なでしやう!。
\ruby{此}{こ}の
\ruby{先}{さき}
お
\ruby{五十}{い|そ}さんの
\ruby{氣}{き}が
\ruby{折}{を}れて
\ruby{優}{やさし}しくでもなつたら
\ruby{濟}{す}みも
\ruby{仕}{し}ましやうが、
\ruby{{\換字{若}}}{も}し
お
\ruby{五十}{い|そ}さんは
お
\ruby{五十}{い|そ}さんで
\ruby{何處}{ど|こ}までも
\ruby{剛{\換字{情}}}{がう|じやう}を
\ruby{張}{は}り、
お
\ruby{師匠}{し|よ}さんは
お
\ruby{師匠}{し|よ}さんで
\ruby{鼻}{はな}の
\ruby{尖}{さき}ばかりで
\ruby{待{\換字{遇}}}{あし|ら}つて
\ruby{行}{い}つたら、
\ruby{何程}{いく|ら}
\ruby{男兒}{をと|こ}だつて
\ruby{{\換字{迷}}}{まよ}つた
\ruby{心持}{こゝろ|もち}の
\ruby{苦}{くる}しさは
\ruby{女}{をんな}と
\ruby{異}{ちが}ひも
\ruby{仕}{し}ますまいもの、
\ruby{何樣}{ど|ん}なにか
\ruby{泣}{な}きも
\ruby{仕}{し}ましやう、
\ruby{恨}{うら}みも
\ruby{仕}{し}ましやう、
\ruby{口惜}{く|やし}がりも
\ruby{仕}{し}ましやう。
\ruby{愍然}{かは|いさう}に% 「愍然 か(は)いさう」
\ruby{彼}{あ}の
\ruby{人}{ひと}は
\ruby{云}{い}はゞ
\ruby{淸玄}{せい|げん}
\ruby{見}{み}たやうなものになつて、
\ruby{{\換字{終}}局}{しま|ひ}にやあ
\ruby{段々}{だん|〴〵}との
\ruby{行掛}{ゆき|がかり}づくから、
\ruby{何樣}{ど|ん}な
\ruby{怖}{おそ}ろしい
\ruby{恐}{こは}い
\ruby{場}{ば}に% 原文通り「場」
\ruby{行}{ゆ}き
\ruby{着}{つ}かうかも
\ruby{知}{し}れません。
もし
\ruby{然樣}{さ|う}なつたところで
お
\ruby{五十}{い|そ}さんや
お
\ruby{師匠}{し|よ}さんは、
\ruby{身}{み}から
\ruby{出}{で}た
\ruby{錆}{さび}だから
\ruby{仕方}{し|かた}が
\ruby{無}{な}いとしても、
\ruby{別}{べつ}に
\ruby{何}{なに}も
\ruby{惡}{わる}い
\ruby{事}{こと}は
\ruby{仕}{し}ない
\ruby{彼}{あ}の
\ruby{{\換字{情}}}{じやう}の
\ruby{厚}{あつ}い、
\ruby{正直}{しやう|ぢき}な、
\ruby{生無垢}{き|む|く}な、
\ruby{彼}{あ}の
\ruby{{\換字{前}}{\換字{途}}}{おひ|さき}が
\ruby{有}{あ}りさうな
\ruby{彼}{あ}の
\ruby{人}{ひと}が……
\ruby{見}{み}す〳〵
\ruby{一人}{ひと|り}
\ruby{廢}{すた}つて
\ruby{仕舞}{し|ま}ふのは
\ruby{愍然}{かは|いさう}ぢやあ% 「愍然 か(は)いさう」
\ruby{有}{あ}りませんか。
ネエ
\ruby{姊}{ねえ}さん、
\ruby{察}{さつ}しの
\ruby{宜}{い}い
\ruby{姊}{ねえ}さんに
\ruby{其處}{そ|こ}が
\ruby{解}{わか}らない
\ruby{事}{こと}はありますまい。
\ruby{惡}{わる}い
\ruby{事}{こと}も
\ruby{仕}{し}ない
\ruby{人}{ひと}が
\ruby{見}{み}す〳〵
\ruby{人}{ひと}
\ruby{一人}{ひと|り}
\ruby{廢}{すた}りさうな、それが
\ruby{愍然}{かは|いさう}で% 「愍然 か(は)いさう」
\ruby{無}{な}い
\ruby{事}{こと}はありますまい、ねエ
\ruby{姊}{ねえ}さん。
』

\ruby{{\換字{情}}激}{じやう|げき}してや
お
\ruby{龍}{りう}が
\ruby{面}{おもて}はやゝ
\ruby{紅}{あか}くなり、
\ruby{其}{そ}の
\ruby{眼}{め}は
\ruby{濡}{ぬ}れ
\ruby{色}{いろ}を
\ruby{帶}{お}びて
\ruby{異}{あや}しく
\ruby{光}{ひかり}を
\ruby{增}{ま}せり。

\Entry{其十七}

『そりやあもう
\ruby{屹度}{きつ|と}お
\ruby{前}{まへ}の
\ruby{御云}{お|い}ひの
\ruby{通}{とほ}りだよ。
そのお
\ruby{五十}{い|そ}さんといふ
\ruby{人}{ひと}やお
\ruby{前}{まへ}の
\ruby{御師匠}{お|し|よ}さんが、いつまでも〳〵
\ruby{然樣}{さ|う}いつた
\ruby[g]{調子}{てうし}で
\ruby{居}{ゐ}りやあ、それほど
\ruby{迄}{まで}に
\ruby{思}{おも}ひ
\ruby{込}{こ}んだ
\ruby{彼}{あ}の
\ruby{水野}{みづ|の}つていふ
\ruby{人}{ひと}の、
\ruby{落}{お}ちて
\ruby{行}{ゆ}く
\ruby{前{\換字{途}}}{さ|き}は
\ruby{知}{し}れて
\ruby{居}{ゐ}るよ。
\ruby{學問}{がく|もん}もあるといふ
\ruby{人}{ひと}の
\ruby{事}{こと}だから、まさかに
\ruby{無{\換字{分}}別沙汰}{む|ふん|べつ|ざ|た}も
\ruby{仕}{し}まいけれどもネエ、
\ruby{彼}{あ}の
\ruby{人}{ひと}が
\ruby{若}{もし}
\ruby{愚人}{ば|か}かなんかだと、それこそ
\ruby{怖}{おそろ}しい
\ruby{事}{こと}にもなり
\ruby{{\換字{兼}}}{かね}ない
\ruby{話}{はなし}たよ。
』

『
\ruby{然樣}{さ|う}ですとも、ほんとに!。
もし
\ruby{彼}{あ}の
\ruby{人}{ひと}が
\ruby{無茶}{む|ちや}な
\ruby{人}{ひと}だつた
\ruby{日}{ひ}にやあ、
\ruby{隨{\換字{分}}刄物}{ずい|ぶん|は|もの}でも
\ruby{持}{も}ち
\ruby{出}{だ}し
\ruby{{\換字{兼}}}{かね}ないとおもひますよ。
さうすりやあ
\ruby[g]{差詰}{さしず}め
\ruby{吾家}{う|ち}の
\ruby{御師匠}{お|し|よ}さんが
\ruby{目}{め}ざされる
\ruby{人}{ひと}ですネエ。
』

『あゝさうとも!。
お
\ruby{前}{まへ}の
\ruby{御師匠}{お|し|よ}さんといふ
\ruby{人}{ひと}は
\ruby{小}{けち}な
\ruby{惡}{わる}い
\ruby{人}{ひと}なんだけれど、
\ruby{仕方}{し|かた}が
\ruby{餘}{あんま}り
\ruby{罪}{つみ}な
\ruby{仕方}{し|かた}だからネ、
\ruby{隨{\換字{分}}鰺切}{ずゐ|ぶん|あぢ|きり}で
\ruby{突}{つゝつ}かれる
\ruby{位}{くらゐ}の
\ruby{事}{こと}は
\ruby{出來}{で|き}ても
\ruby{是非}{ぜ|ひ}が
\ruby{無}{な}いよ。
』

『ですが
\ruby{彼}{あ}の
\ruby{人}{ひと}が
\ruby{無茶}{む|ちや}な
\ruby{人}{ひと}で
\ruby{無}{な}いだけに、
\ruby{何樣間{\換字{違}}}{ど|う|ま|ちが}つたつて
\ruby{下}{くだ}らない
\ruby{事}{こと}なんかは
\ruby{仕}{し}や
\ruby{仕}{し}ますまい。
\ruby{百}{ひやく}のものならまあ
\ruby[g]{九十九}{くじうく}までは\換字{志}つと
\ruby{堪}{こら}へるだらうと
\ruby{思}{おも}ひますが、
\ruby{何處}{ど|こ}までも\換字{志}つと
\ruby{堪}{こら}へて
\ruby{獨}{ひと}りで
\ruby{苦}{くる}しんで、
\ruby{思}{おも}ひ
\ruby{死}{じに}に
\ruby{死}{し}んで
\ruby{仕舞}{し|ま}ふまでも
\ruby{穩}{おとな}しく
\ruby{仕}{し}て
\ruby{居}{ゐ}やうかと
\ruby{思}{おも}ふと、
\ruby{{\換字{分}}別}{ふん|べつ}や
\ruby{堪}{こら}へ
\ruby{{\換字{情}}}{じやう}が
\ruby{有}{あ}る
\ruby{人}{ひと}だけに
\ruby{{\換字{猶}}}{なほ}の
\ruby{事氣}{こと|き}の
\ruby{毒}{どく}で、ほんとに
\ruby{何}{なん}といふ
\ruby[g]{愍然}{かはいさう}な
\ruby{人}{ひと}だらうと
\ruby{思}{おも}はずには
\ruby{居}{ゐ}られません。

それでもまた
\ruby{彼}{あ}の
\ruby{人}{ひと}が
\ruby{困}{こま}らずにでも
\ruby{居}{ゐ}たら、
\ruby{同}{おな}じ
\ruby{胸}{むね}の
\ruby{苦}{くる}しい
\ruby{中}{なか}でも
\ruby{氣}{き}の
\ruby{樂}{らく}なところも
\ruby{有}{あ}りましやうが、
\ruby{職務}{や|く}は
\ruby{無}{な}し、
\ruby{身體}{から|だ}は
\ruby{閑}{ひま}なり、
\ruby[g]{懐中合}{ふところあひ}は
\ruby{惡}{わる}し、
\ruby{差當}{さし|あた}り
\ruby{段々困}{だん|〳〵|こま}つて
\ruby{來}{く}るといふところで、
\ruby{其}{そ}の
\ruby{困}{こま}るやうになつた
\ruby{原因}{も|と}のお
\ruby{五十}{い|そ}さんは
\ruby[g]{{\換字{情}}無}{つれな}いし、お
\ruby{師匠}{し|よ}さんは
\ruby{薄{\換字{情}}}{はく|じやう}の
\ruby{地金}{ぢ|がね}を
\ruby{露}{だ}して、
\ruby{一昨日}{を|とゝ|ひ}お
\ruby{出}{いで}といふやうは
\ruby{挨拶}{あい|さつ}を
\ruby{仕}{し}たなら、
\ruby{彼}{あ}の
\ruby{人}{ひと}の
\ruby{胸}{むね}の
\ruby{中}{うち}はまあ
\ruby{何樣}{ど|ん}なになるでしやう。
\ruby{火水}{ひ|みづ}が
\ruby{一諸}{いつ|しよ}になつたやうになつて、
\ruby{居}{ゐ}ても
\ruby{立}{た}つても
\ruby{居}{ゐ}られや
\ruby{仕}{し}ますまい。

ですから
\ruby{妾}{わたし}が
\ruby{吾家}{う|ち}の
\ruby{御師匠}{お|し|よ}さんの
\ruby{子}{こ}とか
\ruby{姪}{めひ}とか、
\ruby{何}{なに}か
\ruby[g]{親眷}{しんみ}のものでゞも
\ruby{有}{あ}るのならば、よしんばお
\ruby{師匠}{し|よ}さんと
\ruby{論爭}{いひ|あひ}を
\ruby{仕}{し}てもお
\ruby{五十}{い|そ}さんを
\ruby{與}{や}るとか、
\ruby{恩{\換字{返}}}{おん|がへ}しをするとか、
\ruby{何}{ど}の
\ruby{{\換字{道}}}{みち}にせよ
\ruby{彼}{あ}の
\ruby{人}{ひと}の
\ruby{立}{た}つ
\ruby{瀬}{せ}のあるやうに、
\ruby{何樣}{ど|う}にか
\ruby{仕}{し}て
\ruby{{\換字{遣}}}{や}るのですが、お
\ruby{師匠}{し|よ}さんと
\ruby{妾}{わたし}たあ
\ruby{他人同士}{た|にん|どう|し}、
\ruby[g]{養女}{むすめ}になれ
\ruby[g]{養女}{むすめ}にするつて
\ruby{此頃}{この|ごろ}ぢや
\ruby{大切}{だい|じ}にして
\ruby{優}{やさ}しくは
\ruby{仕}{し}て
\ruby{{\換字{呉}}}{く}れても、
\ruby{此方}{こつ|ち}あ
\ruby[g]{食客}{かゝりうど}てす、
\ruby[g]{論爭}{いひあ}ふまでにやあ
\ruby{何}{なに}も
\ruby{云}{い}へません、また
\ruby[g]{論爭}{いひあ}つたつて
\ruby{無{\換字{益}}}{む|だ}なのは
\ruby{知}{し}れてます。
ですけれど
\ruby{御師匠}{お|し|よ}さんの
\ruby{代}{かはり}になつて
\ruby{行}{い}つて、
\ruby{彼}{あ}の
\ruby{人}{ひと}と
\ruby{知}{し}り
\ruby{合}{あひ}になつてからいろ〳〵のいきさつを
\ruby{聞}{き}いて
\ruby[g]{一々知}{いち〳〵し}つて
\ruby{見}{み}ると、
\ruby{妾}{わたし}あ
\ruby[g]{眞個}{ほんと}に
\ruby{彼}{あ}の
\ruby{人}{ひと}が
\ruby{氣}{き}の
\ruby{毒}{どく}で〳〵、お
\ruby{五十}{い|そ}さんていふ
\ruby{人}{ひと}が
\ruby{小憎}{こ|にく}らしい
\ruby{位}{くらゐ}に
\ruby{思}{おも}つて
\ruby{居}{ゐ}たところへ、これこれで
\ruby{職}{やく}も
\ruby{無}{な}くなつたといふ
\ruby{話}{はなし}を
\ruby{聞}{き}いて
\ruby{見}{み}るとハア
\ruby{然樣}{さ|う}ですかと
\ruby{云}{い}つた
\ruby{限}{き}りにやあ
\ruby{出來無}{で|き|な}いやうな
\ruby{氣}{き}もすれば、
\ruby{何}{なん}だか
\ruby{知}{し}らん
\ruby{顏}{かほ}で
\ruby{打棄}{うつ|ちや}つて
\ruby{置}{お}いちやあ
\ruby[g]{不人情}{ふにんじやう}のやうな
\ruby{氣}{き}もするんですよ。
で、
\ruby{姊}{ねえ}さんが
\ruby{口}{くち}さへきいて
\ruby{下}{くだ}さりやあ
\ruby[g]{必定譯}{きつとわけ}は
\ruby{無}{な}い
\ruby{事}{こと}、
\ruby{多勢}{おほ|ぜい}の
\ruby{人}{ひと}をお
\ruby{使}{つか}ひなさる
\ruby[g]{筑波}{つくば}さんところで
\ruby{人}{ひと}
\ruby{一人}{ひと|り}
\ruby{位}{ぐらひ}に
\ruby{授}{さづ}けて
\ruby{下}{くだ}さる
\ruby{職}{やく}の
\ruby{無}{な}い
\ruby{事}{こと}は
\ruby{有}{あ}るまいからと、
\ruby[g]{然樣思}{さうおも}つて、それで
\ruby{餘計}{よ|けい}なおせつかいか
\ruby{知}{し}りませんが
\ruby{御願}{お|ねが}ひに
\ruby{來}{き}たのです。

\ruby{一體}{いつ|たい}ならば
\ruby{吾家}{う|ち}の
\ruby{御師匠}{お|し|よ}さんが
\ruby{出來}{で|き}ないまでもかういふ
\ruby{苦勞}{く|らう}を
\ruby{仕}{し}て
\ruby{見}{み}なけりやあならない
\ruby{處}{ところ}なので、
\ruby{妾}{わたし}が
\ruby{爲}{す}るのは
\ruby{出{\換字{過}}}{で|す}ぎても
\ruby{居}{ゐ}ましやうが、お
\ruby{師匠}{し|よ}さんはお
\ruby{師匠}{し|よ}さんで
\ruby{澄}{す}まして
\ruby{{\換字{平}}氣}{へい|き}で
\ruby{居}{ゐ}ても、
\ruby{妾}{わたし}あ
\ruby{妾}{わたし}の
\ruby[g]{苦勞性}{くらうしやう}で
\ruby{安然}{じ|つ}としちやあ
\ruby{居}{ゐ}られなくつて、
\ruby{斯樣}{か|う}して
\ruby{出}{で}て
\ruby{來}{き}て
\ruby{姊}{ねえ}さんに
\ruby{縋}{すが}るのです。
まさか
\ruby{如是}{こ|れ}だけに
\ruby{細}{こまか}い
\ruby{理由}{わ|け}を
\ruby[g]{御話仕}{おはなしし}たら、そりやあお
\ruby{前詰}{まへ|つま}らないよと
\ruby{云}{い}つても
\ruby{下}{くだ}さいますまいが、ネエ
\ruby{姊}{ねえ}さん、
\ruby{妾}{わたし}の
\ruby{慾得}{よく|とく}で
\ruby{御願}{お|ねが}ひをするのぢやあ
\ruby{無}{な}いし、
\ruby{姊}{ねえ}さんだつて
\ruby{彼}{あ}の
\ruby{人}{ひと}を
\ruby[g]{愍然}{かはいさう}ちや
\ruby{無}{な}いとお
\ruby{思}{おも}ひなさるやうな
\ruby{事}{こと}は
\ruby{有}{あ}りやあ
\ruby{仕}{し}ますまいもの、お
\ruby{願}{ねがひ}ですから
\ruby{妾}{わたし}の
\ruby[g]{所思}{おもひ}の
\ruby{無}{む}にならないやうに
\ruby{仕}{し}て
\ruby{下}{くだ}さいな、ねエ
\ruby{姊}{ねえ}さん。
』

\ruby{思}{おも}ひ
\ruby{入}{い}つて
\ruby{頼}{たの}み
\ruby{聞}{きこ}ゆるお
\ruby{龍}{りう}を
\ruby{優}{やさ}しき
\ruby{眼}{め}して
\ruby{見居}{み|ゐ}たるお
\ruby{彤}{とう}は、
\ruby{先刻}{さ|き}より
\ruby{今}{いま}に
\ruby{至}{いた}つて
\ruby{{\換字{猶}}未}{なほ|いま}だ
\ruby{鬢}{びん}の
\ruby{毛}{け}の
\ruby{一筋}{ひと|すじ}をだに
\ruby{動}{ゆる}がさず、
\ruby{端然}{たん|ねん}として
\ruby{坐}{すわ}りたるまゝなり。


\Entry{其十八}

お
\ruby{彤}{とう}は
\ruby{其}{そ}の
\ruby{美}{うつく}しき
\ruby{手}{て}に
\ruby{手爐}{てあ|ぶり}の
\ruby{緣}{ふち}を
\ruby{撫}{な}づるとも
\ruby{無}{な}く
\ruby{撫}{な}でながら、いと
\ruby{靜}{しづか}に
\ruby{口}{くち}を
\ruby{開}{ひら}きて、

『お
\ruby{前}{まへ}の
\ruby{云}{い}ふ
\ruby{事}{こと}は、ようく
\ruby{分}{わか}つたよ、だがネエお
\ruby{龍}{りう}ちやん!。
』

と
\ruby{親}{した}しげに
\ruby{呼}{よ}びかくればお
\ruby{龍}{りう}も、

『ハア。
』

と
\ruby{甘}{あま}ゆるが
\ruby{如}{ごと}く
\ruby{輕}{かろ}く
\ruby{答}{こた}へてお
\ruby{彤}{とう}を
\ruby{見}{み}つ、
\ruby{我}{わ}が
\ruby{姊}{あね}の
\ruby{如}{ごと}くに
\ruby{頼}{たの}み
\ruby{思}{おも}へる
\ruby{人}{ひと}は
\ruby{何}{なに}と
\ruby{云}{い}ひ
\ruby{出}{い}づるならん、
\ruby[g]{多分}{たぶん}は
\ruby{我}{わ}が
\ruby{頼}{たの}みを
\ruby{聞}{き}いては
\ruby{{\換字{呉}}}{く}るヽならんがと
\ruby{思}{おも}ひながらも、だがネエと
\ruby{云}{い}へる
\ruby{發語}{いひ|だし}に、
\ruby{少}{すこ}し
\ruby[g]{氣{\換字{遣}}氣味}{きづかひぎみ}の、
\ruby{心配}{しん|ぱい}らしき
\ruby{眼}{め}して
\ruby{他}{ひと}の
\ruby{眼}{め}を
\ruby{見}{み}たり。

『
\ruby{成程}{なる|ほど}お
\ruby{前}{まへ}の
\ruby[g]{御云}{おいひ}の
\ruby{通}{とほ}り
\ruby[g]{水野}{みづの}つていふ
\ruby{人}{ひと}も
\ruby[g]{愍然}{かはいさう}だし、お
\ruby{前}{まへ}の
\ruby{御師匠}{お|し|よ}さんていふ
\ruby{人}{ひと}の
\ruby{仕方}{し|かた}も
\ruby{惡}{わる}いがネエ、お
\ruby{龍}{りう}ちやん、お
\ruby{前}{まへ}が
\ruby{何}{なに}も
\ruby{彼}{あ}のお
\ruby{師匠}{し|よ}さんの
\ruby[g]{眷屬}{みうち}といふのぢやあ
\ruby{無}{な}いし、
\ruby{{\換字{叉}}深}{また|ふか}しい
\ruby{關係}{ひつ|かヽり}のある
\ruby{免}{のが}れない
\ruby{仲}{なか}といふのぢやあ
\ruby{無}{な}いしさ、お
\ruby{前}{まへ}が
\ruby{彼}{あ}のお
\ruby{師匠}{し|よ}さんのところから
\ruby{身}{み}さへ
\ruby{引}{ひ}いて
\ruby{{\換字{終}}}{しま}へば、
\ruby{其}{そ}の
\ruby{話}{はなし}あ
\ruby[g]{全然}{まつたく}お
\ruby{前}{まへ}にやあ
\ruby[g]{飛沫}{しぶき}も
\ruby{飛}{と}んで
\ruby{來}{こ}ない
\ruby{話}{はなし}になつて
\ruby{仕舞}{し|ま}つて、たとへ
\ruby{何樣}{ど|ん}な
\ruby{喧嘩}{けん|くわ}が
\ruby{始}{はじ}まるにしても
\ruby[g]{泥仕合}{どろじあひ}が
\ruby{始}{はじ}まるにしても、
\ruby[g]{彼方}{むかふ}が
\ruby[g]{彼方}{むかふ}だけで
\ruby{何樣}{ど|う}にでも
\ruby{{\換字{遣}}}{や}り
\ruby{合}{あ}つて
\ruby{居}{ゐ}やうつていふ
\ruby{譯}{わけ}ぢやあ
\ruby{無}{な}いか。
\ruby[g]{彼方同士}{むかふどうし}あ
\ruby{一團}{た|ま}になつてこんがらかつて
\ruby{居}{ゐ}る
\ruby{絲}{いと}だよ、お
\ruby{前}{まへ}は
\ruby{其}{その}
\ruby{一團}{た|ま}の
\ruby{中}{なか}に
\ruby{入}{はい}つては
\ruby{居}{ゐ}てもこんがらかつては
\ruby{居無}{ゐ|な}い \------
\ruby{引張}{ひつ|ぱ}ればするりと
\ruby{{\換字{脱}}}{ぬ}けて
\ruby{仕舞}{し|ま}ふ
\ruby{事}{こと}の
\ruby{出來}{で|き}る
\ruby{絲}{いと}だよ。
だから
\ruby{早}{はや}い
\ruby{話}{はなし}を
\ruby{云}{い}やあ
\ruby{汝}{おまへ}が
\ruby{其}{そ}のこんがらかりの
\ruby{一團}{た|ま}の
\ruby{中}{なか}に
\ruby{入}{はい}つて、
\ruby{氣}{き}を
\ruby{使}{つか}つたり
\ruby{目}{め}を
\ruby{使}{つか}つたりしてまごついて
\ruby{居}{ゐ}るよりやあ、するりと
\ruby{{\換字{脱}}}{ぬ}けて
\ruby{仕舞}{し|ま}つた
\ruby{方}{はう}が
\ruby[g]{何程好}{いくらい}いか
\ruby{知}{し}れないよ。
\ruby{譯}{わけ}は
\ruby{無}{な}いやあネ、
\ruby{妾}{わたし}のところへ
\ruby{來}{き}てお
\ruby{仕舞}{し|ま}ひな、
\ruby{以前}{ま|へ}のやうに
\ruby{妾}{わたし}のところで
\ruby{氣}{き}を
\ruby[g]{長閑}{のんき}に
\ruby{仕}{し}て、
\ruby{小{\換字{説}}}{せう|せつ}でも
\ruby{讀}{よ}んで
\ruby{{\換字{遊}}}{あそ}んでおいでが
\ruby{宜}{い}いぢやあ
\ruby{無}{な}いか。
\ruby{彼}{あ}のお
\ruby{師匠}{し|よ}さんていふ
\ruby{人}{ひと}が
\ruby{何}{なに}かぶつ〳〵
\ruby{云}{い}つたにしても、
\ruby[g]{金錢}{おかね}のぽつちりも
\ruby{與}{や}りやあ
\ruby{尾}{を}を
\ruby{振}{ふ}つちまふ
\ruby{人}{ひと}だらうから、
\ruby{何}{なに}もむづかしい
\ruby{事}{こと}は
\ruby{有}{あ}りやあ
\ruby{仕無}{し|な}いはネ。
お
\ruby{前}{まへ}の
\ruby{爲}{ため}の
\ruby{好}{い}いやうになら
\ruby{何樣}{ど|ん}なにでも
\ruby{仕}{し}てあげるつもりなのだし、お
\ruby{前}{まへ}の
\ruby{身}{み}の
\ruby{上}{うへ}に
\ruby{就}{つ}いちやあ
\ruby{妾}{わたし}も
\ruby[g]{些考}{ちよつとかんが}へてる
\ruby{事}{こと}もあるんだし、
\ruby[g]{{\換字{叉}}何處}{またどこ}までも
\ruby{引受}{ひき|う}けて
\ruby{世話}{せ|わ}を
\ruby{仕度}{し|た}いといふ
\ruby{道理}{わ|け}も
\ruby{有}{あ}るんだからネ。
\ruby{決}{けつ}して
\ruby{惡}{わる}い
\ruby{事}{こと}は
\ruby{云}{い}はないから
\ruby{{\換字{脱}}}{ぬ}けて
\ruby{仕舞}{し|ま}つたら
\ruby{何樣}{ど|う}だエ。
\ruby{第一}{だい|いち}お
\ruby{前}{まへ}の
\ruby{話}{はなし}でも
\ruby{{\換字{分}}}{わか}つて
\ruby{居}{ゐ}るお
\ruby{前}{まへ}の
\ruby{御師匠}{お|し|よ}さんネ、そんな
\ruby{可厭}{い|や}な
\ruby{人}{ひと}と
\ruby{一緒}{いつ|しよ}に
\ruby{居}{ゐ}て
\ruby{末々}{すえ|〴〵}はお
\ruby[g]{前何樣仕}{まへどうし}やうつて
\ruby{氣}{き}なのだエ。
お
\ruby{前程}{まへ|ほど}にも
\ruby{無}{な}い、
\ruby{{\換字{分}}}{わか}らないぢやあないか。
』

『そりやあもう
\ruby{段々}{だん|〳〵}と
\ruby{彼}{あ}の
\ruby{人}{ひと}の
\ruby[g]{御腹}{おなか}の
\ruby{中}{なか}が
\ruby{讀}{よ}めて
\ruby{來}{き}て
\ruby{見}{み}ると、
\ruby[g]{到底末長}{とてもすえなが}く
\ruby{一緒}{いつ|しよ}になんぞ
\ruby{居}{ゐ}られる
\ruby{人}{ひと}ぢやあ
\ruby{無}{な}いのですし、
\ruby{妾}{わたし}に
\ruby{仕}{し}た
\ruby{前々}{まへ|〳〵}の
\ruby[g]{所行}{しうち}も
\ruby{此頃}{この|ごろ}になつて
\ruby{見}{み}りやあ、
\ruby{合點}{が|てん}の
\ruby{行}{ゆ}く
\ruby{恨}{うら}めしいことが
\ruby[g]{澤山}{たんと}あるのですもの。
ですから
\ruby{表面}{うは|べ}こそは
\ruby{奇麗}{き|れい}にして
\ruby{居}{ゐ}ますが、
\ruby{些}{ちつと}も
\ruby{一處}{いつ|しよ}に
\ruby{居}{ゐ}たい
\ruby{事}{こと}なんか
\ruby{有}{あ}りやあ
\ruby{仕}{し}ませんの!。
ただ、
\ruby{今直}{いま|すぐ}に
\ruby[g]{何樣思}{どうおも}つたからつて
\ruby{思}{おも}つたやうにもならない
\ruby{身}{み}だもんですから……。
』

『それで
\ruby[g]{彼家}{あすこ}に
\ruby{居}{ゐ}るとお
\ruby{云}{い}ひのかエ。
それ
\ruby{御覽}{ご|らん}、
\ruby{彼}{あ}の
\ruby{人}{ひと}は
\ruby{前}{まへ}つから
\ruby{妾}{わたし}が
\ruby{推量}{すゐ|りやう}した
\ruby{通}{とほ}りだつたらう、
\ruby{云}{い}はない
\ruby{事}{こつ}ちやあ
\ruby{無}{な}い。
だから
\ruby{今}{いま}お
\ruby{前}{まへ}をちやほや
\ruby{云}{い}つて
\ruby{家}{うち}に
\ruby{置}{お}いて
\ruby{居}{ゐ}る
\ruby{料簡}{れう|けん}だつて、』

『つまり
\ruby{妾}{わたし}を
\ruby{猿廻}{さる|まは}しの
\ruby{猿}{さる}にして、
\ruby{自分}{じ|ぶん}が
\ruby{食}{た}べやうつていふ
\ruby{腹}{はら}なんですよ。
その
\ruby{位}{ぐらゐ}の
\ruby{事}{こと}は
\ruby{妾}{わたし}だつて、
\ruby{氣}{き}のつかない
\ruby{程人}{ほど|ひと}が
\ruby{好}{よ}くももうありませんからネ。
それを
\ruby{何時}{い|つ}までも
\ruby[g]{小兒}{こども}かと
\ruby{思}{おも}つて、
\ruby{馬鹿}{ば|か}にして
\ruby{居}{ゐ}る
\ruby{氣}{き}の
\ruby{御師匠}{お|し|よ}さんの
\ruby{仕方}{し|かた}にやあ
\ruby{腹}{はら}が
\ruby{立}{た}ちますは。
』

『ホヽホヽホヽ、
\ruby[g]{澤山苦勞}{たんとくらう}をお
\ruby{仕}{し}だつたから、
\ruby{前}{まへ}のお
\ruby{龍}{りう}ちやんぢやあ
\ruby{無}{な}いものネエ。
だが
\ruby{然樣知}{さ|う|し}り
\ruby{切}{き}つて
\ruby{居}{ゐ}てそれであどけ
\ruby{無}{な}い
\ruby{風}{ふう}を
\ruby{仕}{し}ておいでのなんざあ、お
\ruby{前}{まへ}の
\ruby{方}{はう}がお
\ruby{師匠}{し|よ}さんよりも
\ruby{人}{ひと}の
\ruby{惡}{わる}さが
\ruby{一枚上}{いち|まい|うへ}ぢやあ
\ruby{無}{な}いか
\ruby{知}{し}らん、ホヽホヽホヽ。
』

『ホヽホヽホヽ、だつて
\ruby{妾}{わたし}あ、あんな
\ruby{眞}{しん}の
\ruby{惡}{わる}い
\ruby{憎}{にく}い
\ruby{人}{ひと}にだから
\ruby{然樣}{さ|う}して
\ruby{居}{ゐ}られるのですは。
\ruby{善}{い}いとおもふ
\ruby{人}{ひと}に
\ruby{對}{むか}つちやあ
\ruby{此}{これ}つばかりだつて
\ruby{作}{つく}り
\ruby{飾}{かざ}りは
\ruby{仕}{し}やあ
\ruby{仕}{し}ませんよ。
』

『ホヽホヽホー。
いヽよ。
\ruby{誰}{だれ}もお
\ruby{前}{まへ}を
\ruby[g]{眞個}{ほんと}に
\ruby{惡}{わる}い
\ruby{人}{ひと}におなりだつて
\ruby{云}{い}ひや
\ruby{仕}{し}ないから。
で、さういふわけなら
\ruby{{\換字{猶}}}{なほ}の
\ruby{事}{こと}ぢや
\ruby{無}{な}いか。
\ruby{一日}{いち|にち}も
\ruby{早}{はや}く
\ruby{其樣}{そ|ん}な
\ruby{人}{ひと}と
\ruby{一}{ひと}つ
\ruby[g]{御釜}{おかま}の
\ruby[g]{御飯}{ごぜん}を
\ruby{食}{た}べあつて
\ruby{緣}{えん}を
\ruby{深}{ふか}くする
\ruby{樣}{やう}な
\ruby{事}{こと}を、
\ruby{仕無}{し|な}い
\ruby{樣}{やう}に
\ruby{仕}{し}た
\ruby{方}{はう}が
\ruby{宜}{よ}からうぢやあ
\ruby{無}{な}いか。
』

『そりやあ
\ruby{其}{そ}の
\ruby{譯}{わけ}はもう
\ruby{能}{よう}く
\ruby{{\換字{分}}}{わか}つてますが、ぢやあ、
\ruby{姊}{ねえ}さんの
\ruby{心持}{こヽろ|もち}ぢやあ
\ruby[g]{水野}{みづの}さんの
\ruby{事}{こと}は、まあ
\ruby[g]{一體何樣}{いつたいどう}したら
\ruby{好}{い}いんだと
\ruby[g]{御思}{おおも}ひなんでしやう?。
\ruby{構}{かま}ふ
\ruby{事}{こと}は
\ruby{無}{な}い、
\ruby{何}{なに}も
\ruby{彼}{か}も
\ruby{抛}{はふ}つてお
\ruby{仕舞}{し|ま}ひと
\ruby[g]{御思}{おおも}ひの?。
』

\ruby{此}{これ}は
\ruby{恨}{うら}むるに
\ruby{似}{に}て
\ruby{云}{い}へど
\ruby{彼}{かれ}は
\ruby{感}{かん}ぜざるがごとし。

『
\ruby{一體水野}{いつ|たい|みづ|の}つて
\ruby{人}{ひと}は
\ruby{彼}{あ}りやあお
\ruby{前}{まへ}の
\ruby{何}{なん}に
\ruby{當}{あた}るのだエ?。
』

『…………』

『お
\ruby{前}{まへ}あの
\ruby{人}{ひと}に
\ruby{其樣}{そ|ん}なに
\ruby{肩}{かた}を
\ruby{入}{い}れて
\ruby{何樣仕}{ど|う|し}やうつてお
\ruby{思}{おも}ひのだエ?。
』

『…………』

『
\ruby{考}{かんが}へて
\ruby{御覽}{ご|らん}、
\ruby{餘}{あんま}り
\ruby{詰}{つま}らな
\ruby{{\換字{過}}}{す}ぎるぢやあ
\ruby{無}{な}いかエ。
』

『……だつて
\ruby{姊}{ねえ}さん。
』

『だつてぢやあ
\ruby{無}{な}いよ。
え、お
\ruby{龍}{りう}ちやん、
\ruby{妾}{わたし}あ
\ruby{何}{なん}だか
\ruby{意地}{い|ぢ}の
\ruby{惡}{わる}い
\ruby{事}{こと}を
\ruby{云}{い}ふやうだがネ、ようく
\ruby{考}{かんが}へてごらんな。
どうだエ、それ、お
\ruby{龍}{りう}ちやん。
』

『……だつて
\ruby{姊}{ねえ}さん、』

『いヽえ。
だつてぢやあ
\ruby{有}{あ}りまとんよ。
\ruby{能}{よう}く
\ruby{考}{かんが}へてごらん。
\ruby{詰}{つま}らない
\ruby{事}{こと}は
\ruby[g]{{\換字{終}}局}{しまひ}まで
\ruby{行}{い}つても
\ruby[g]{矢張}{やつぱ}り
\ruby{詰}{つま}らないよ。
』

『だつて
\ruby{姊}{ねえ}さん……。
だつて
\ruby{姊}{ねえ}さん……。
でもそれぢやあ
\ruby{餘}{あんま}り
\ruby[g]{恰悧{\換字{過}}}{りこうす}ぎて
\ruby{薄{\換字{情}}}{はく|じやう}ぢやあ
\ruby{無}{な}くつて?。
』


\Entry{其十九}

お
\ruby{龍}{りう}は
\ruby{自己}{お|の}が
\ruby{身}{み}の
\ruby{凡}{すべ}て
お
\ruby{彤}{とう}に
\ruby{及}{およ}ばざるを
\ruby{知}{し}れるなり。
\ruby{第一今}{だい|いち|いま}の
\ruby{身}{み}の
\ruby{境{\換字{遇}}}{う|へ}は
\ruby{掛}{か}けても
\ruby{及}{およ}ばざるを
\ruby{知}{し}れるなり、
\ruby{有}{も}つて
\ruby{生}{うま}れたる
\ruby{容貌}{きり|やう}ももとより
\ruby{及}{およ}ばざるを
\ruby{知}{し}れるなり、
\ruby{智慧}{ち|ゑ}は
\ruby{特}{こと}さらに
\ruby{及}{およ}ばざるを
\ruby{知}{し}れるなり、
\ruby{讀書筆札}{よ|み|か|き}も
\ruby{二年三年苦}{に|ねん|さん|ねん|くる}しみたりとて
\ruby{及}{およ}ぶべきにあらず、
\ruby{挿花茶湯}{は|な|ちや|のゆ}はいふまでも
\ruby{無}{な}く、
\ruby{我}{わ}が
\ruby{最}{もつと}も
\ruby{好}{す}ける
\ruby{絲竹}{いと|たけ}の
\ruby{{\換字{道}}}{みち}、
\ruby{彼}{かれ}の
\ruby{最}{もつと}も
\ruby{{\換字{悅}}}{よろこ}ばぬ
\ruby{縫針}{ぬひ|はり}の
\ruby{{\換字{道}}}{みち}に
\ruby{掛}{か}けてすら
\ruby{{\換字{猶}}}{なほ}
\ruby{且及}{かつ|およ}ばず、
\ruby{隨{\換字{分}}}{ずゐ|ぶん}
\ruby{人}{ひと}には
\ruby{負}{ま}くる
\ruby{{\換字{嫌}}}{ぎら}ひの、
\ruby{何事}{なに|ごと}を
\ruby{仕}{し}ても
\ruby{人後}{あ|と}には
\ruby{立}{た}つまじと
\ruby{思}{おも}ふ
\ruby{身}{み}ながら、
\ruby{何事}{なに|ごと}を
\ruby{仕}{し}ても
お
\ruby{彤}{とう}には
\ruby{及}{およ}びかぬるを
\ruby{知}{し}りて、
\ruby{心}{こゝろ}の
\ruby{底}{そこ}の
\ruby{底}{そこ}より
\ruby{深}{ふか}く
\ruby{深}{ふか}く
\ruby{尊}{たつと}び
\ruby{敬}{うやま}へるなり。
されど
\ruby{唯一}{たゞ|ひと}つ、
\ruby{{\換字{情}}合}{じやう|あひ}の
\ruby{深}{ふか}き
\ruby{淺}{あさ}きといふ
\ruby{事}{こと}のみに
\ruby{掛}{か}けては、ひそかに
\ruby{姊}{あね}と
\ruby{頼}{たの}む
お
\ruby{彤}{とう}にも
\ruby{讓}{ゆづ}らざる
\ruby{心地}{こゝ|ち}して、
\ruby{我}{われ}は
\ruby{何}{なん}ぞの
\ruby{折}{をり}には
\ruby{慾}{よく}も
\ruby{得}{とく}も
\ruby{何}{なに}も
\ruby{彼}{か}も
\ruby{棄}{す}てゝ
\ruby{仕舞}{し|ま}ふ
\ruby{馬鹿}{ば|か}なれ
\ruby{共}{ども}、
\ruby{彼}{か}の
\ruby{人}{ひと}は
\ruby{恰悧}{り|かう}だけに
\ruby{同}{おな}じ
\ruby{其}{そ}の
\ruby{時}{とき}に
\ruby{然樣}{さ|う}は
\ruby{爲}{す}まじき
\ruby{人}{ひと}と、
\ruby{却}{かへ}つて
\ruby{流石}{さす|が}に
\ruby{崇}{あが}め
\ruby{慕}{した}へる
\ruby{其人}{その|ひと}をも、
\ruby{聊}{いさゝ}か
\ruby{物足}{もの|た}らず
\ruby{{\換字{飽}}}{あ}かず
\ruby{思}{おも}へ
\ruby{氣味}{き|み}さへあるなり。

されば
\ruby{今}{いま}
お
\ruby{龍}{りう}が
\ruby{云}{い}ひ
\ruby{出}{い}でしは、もとより
\ruby{率然}{そつ|ぜん}の
\ruby{語}{ご}なれども、
\ruby{意}{こゝろ}を
\ruby{用}{もち}ひざる
\ruby{其}{そ}の
\ruby{僅少}{わづ|か}なる
\ruby{語}{ことば}の
\ruby{中}{うち}に、
お
\ruby{龍}{りう}はおのづから
お
\ruby{龍}{りう}の
\ruby{氣性}{きし|やう}の、
\ruby{然}{さ}ばかりに
\ruby{崇}{あが}め
\ruby{思}{おも}へる
お
\ruby{彤}{とう}のためにも
\ruby{枉}{ま}げられず
\ruby{屈}{くつ}せられぬものあるを
\ruby{露}{あらは}し
\ruby{出}{いだ}して、
\ruby{抑}{おさ}へんとして
\ruby{抑}{おさ}へかねたる
\ruby{不服}{ふ|ふく}の
\ruby{氣}{き}を
\ruby{我知}{われ|し}らず
\ruby{洩}{も}らせるなり。

お
\ruby{龍}{りう}の
\ruby{持前}{もち|まへ}を
\ruby{知}{し}りきつたる
お
\ruby{彤}{とう}は、
\ruby{走}{はし}り
\ruby{來}{きた}れる
\ruby{矢}{や}を
\ruby{幕}{まく}もて
\ruby{止}{とゞ}むる
\ruby{如}{ごと}く、
\ruby{柔軟}{やは|らか}なる
\ruby{語氣}{ご|き}に
\ruby{却}{かへ}つて
\ruby{問}{と}ひ
\ruby{反}{かへ}しぬ。

『
\ruby{薄{\換字{情}}}{はく|じやう}ぢやあ
\ruby{無}{な}くつてツて。
\ruby{何故}{な|ぜ}またネエ。
』

『
\ruby{何故}{な|ぜ}つて、
\ruby{姊}{ねえ}さん。
そりやあ
\ruby{妾}{わたし}さへ
\ruby{退}{ひ}いて
\ruby{仕舞}{し|ま}へば
\ruby{妾}{わたし}の
\ruby{身}{み}の
\ruby{好}{い}いのは
\ruby{知}{し}れて
\ruby{居}{ゐ}ますが、それぢやあ
\ruby{彼}{あ}の
\ruby{人}{ひと}は
\ruby{否}{わるい}まんまで
\ruby{{\換字{遺}}}{のこ}るので、
\ruby{矢張}{やつ|ぱ}り
\ruby{彼}{あ}の
\ruby{人}{ひと}は
\ruby{愍然}{かはい|さう}ちやあ
\ruby{有}{あ}りませんか、ですから
\ruby{其}{そ}れぢや
\ruby{薄{\換字{情}}}{はく|じやう}になりますはネ。
\ruby{妾}{わたし}あ
\ruby{詰}{つま}る
\ruby{詰}{つま}らないは
\ruby{何樣}{ど|う}だつて
\ruby{好}{い}いんですよ。
\ruby{妾}{わたし}あたゞ
\ruby{彼}{あ}の
\ruby{人}{ひと}が
\ruby{愍然}{かは|いさう}だから
\ruby{何樣}{ど|う}か
\ruby{仕}{し}て
\ruby{{\換字{遣}}}{や}りたいつて
\ruby{云}{い}ふんぢやあ
\ruby{有}{あ}りませんか。
』

『いゝえ、お
\ruby{{\換字{前}}}{まへ}の
\ruby{心持}{こゝろ|もち}はもう
\ruby{悉皆}{すつ|かり}
\ruby{解}{わか}つて
\ruby{居}{ゐ}るのだがネ。
\ruby{妾}{わたし}あ
\ruby{{\換字{又}}}{また}たゞ
お
\ruby{{\換字{前}}}{まへ}の
\ruby{朋友}{とも|だち}で、
お
\ruby{{\換字{前}}}{まへ}の
\ruby{利{\換字{益}}}{た|め}になる
\ruby{事}{こと}を
\ruby{仕}{し}てあげたいのだから。
 \------ いゝかエ。
だから
\ruby{妾}{わたし}あ
\ruby{{\換字{前}}{\換字{途}}}{さ|き}の
\ruby{{\換字{前}}{\換字{途}}}{さ|き}まで
\ruby{考}{かんが}へるので、
お
\ruby{{\換字{前}}}{まへ}の
\ruby{詰}{つま}る
\ruby{詰}{つま}らないを
\ruby{關}{かま}はないなんて、そんな
\ruby{事}{こと}は
\ruby{出來}{で|き}ないよ。
』

『でも
\ruby{詰}{つま}る
\ruby{詰}{つま}らないで
\ruby{云}{い}やあ、
\ruby{何}{なん}だつて
\ruby{詰}{つま}らないは!。
\ruby{妾}{わたし}みたやうな
\ruby{種々}{いろ|ん}な
\ruby{目}{め}にあつて
\ruby{來}{き}たものは
\ruby{活}{い}きて
\ruby{居}{ゐ}るのからして
\ruby{詰}{つま}らないは!。
\ruby{何樣}{ど|う}せ
\ruby{妾}{わたし}が
\ruby{彼}{あ}の
\ruby{人}{ひと}を
\ruby{愍然}{かはい|さう}だから
\ruby{何樣}{ど|う}して
\ruby{{\換字{遣}}}{や}りたいと
\ruby{思}{おも}つたつて、
\ruby{結局妾}{つま|り|わたし}にやあ
\ruby{何}{なん}にもならない \------
\ruby{詰}{つま}らないなあ
\ruby{知}{し}れてますは……。
でも
\ruby{妾}{わたし}の
\ruby{氣}{き}が
\ruby{屆}{とゞ}けば
\ruby{妾}{わたし}の
\ruby{心持}{こゝろ|もち}は
\ruby{宜}{よ}うござんすは。
\ruby{知}{し}らん
\ruby{顏}{かほ}で
\ruby{濟}{す}ますなあ
\ruby{薄{\換字{情}}}{はく|じやう}なやうな
\ruby{氣}{き}が
\ruby{爲}{し}ますは。
』

『オヤ、
\ruby{妾}{わたし}あ
\ruby{爲}{し}なくちやあならない
\ruby{事}{こと}を
\ruby{爲}{し}ないのが
\ruby{薄{\換字{情}}}{はく|じやう}つていふものかと
\ruby{思}{おも}つて
\ruby{居}{ゐ}たが、
お
\ruby{{\換字{前}}}{まへ}のは
\ruby{爲}{し}なくても
\ruby{濟}{す}むことを
\ruby{仕無}{し|な}いのに
\ruby{薄{\換字{情}}}{はく|じやう}といふのだネ。
』

『
\ruby{爲}{し}なくちやあならない
\ruby{事}{こと}を
\ruby{仕無}{し|な}いのは、そりあ
\ruby{不義理}{ふ|ぎ|り}ですは、
\ruby{爲}{し}なくても
\ruby{濟}{す}むことでも、
\ruby{爲}{し}てやりやあ
\ruby{他人}{ひ|と}の
\ruby{利{\換字{益}}}{た|め}になる、それを
\ruby{爲}{し}ないのが
\ruby{妾}{わたし}あ
\ruby{薄{\換字{情}}}{はく|じやう}かと
\ruby{思}{おも}つて
\ruby{居}{ゐ}ますよ。
』

『お
\ruby{龍}{りう}ちやんのやうに
\ruby{云}{い}つた
\ruby{日}{ひ}にやあ、
お
\ruby{龍}{りう}ちやんの
\ruby{他}{ほか}の
\ruby{出間}{せ|けん}の
\ruby{人}{ひと}
は
\ruby{悉皆}{みん|な}
\ruby{薄{\換字{情}}者}{はく|じやう|もの}のやうになつて
\ruby{仕舞}{し|ま}ふよ。
ホヽヽ、まあ
\ruby{其}{そ}りやあ
\ruby{何樣}{ど|う}でも
\ruby{宜}{い}いが、それぢやあ
\ruby{詰}{つま}つても
\ruby{詰}{つま}らなくつても
\ruby{水野}{みづ|の}つていふ
\ruby{人}{ひと}は
\ruby{妾}{わたし}が
\ruby{引受}{ひき|う}けて
\ruby{何樣}{ど|う}か
\ruby{仕}{し}てあげるとすると
\ruby{決}{き}めて
\ruby{置}{お}くがネ。
』


\Entry{其二十}

『ぢやあ
\ruby{姊}{ねえ}さん、ほんとに
\ruby{受合}{うけ|あ}つて
\ruby{下}{くだ}さるの。
』

お
\ruby{龍}{りう}の
\ruby{眼}{め}は
\ruby{既}{すで}に
\ruby{罪}{つみ}も
\ruby{無}{な}く
\ruby{悅}{よろこ}びて
\ruby{笑}{ゑ}めるなり。
お
\ruby{彤}{とう}は
\ruby{其}{そ}の
\ruby{樣子}{やう|す}を
\ruby{見}{み}て
\ruby{却}{かへ}つて
\ruby{微}{かすか}に
\ruby{愁}{うれ}ふる
\ruby{色}{いろ}あり。

『あゝ、
\ruby{彼}{あ}の
\ruby{人}{ひと}が
\ruby{困}{こま}らないやうにするだけの
\ruby{事}{こと}なんぞは
\ruby{旦那}{だん|な}に
\ruby{云}{い}ふまでも
\ruby{無}{な}い、
\ruby{妾}{わたし}が
\ruby{何樣}{ど|う}にでも
\ruby{必定}{きつ|と}
\ruby{爲}{し}てあげるがネ。
お
\ruby{龍}{りう}ちやんは
\ruby{{\換字{又}}}{また}
\ruby{何}{なん}だつて
\ruby{然樣}{さ|う}
\ruby{彼}{あ}の
\ruby{人}{ひと}の
\ruby{事}{こと}に
\ruby{肩}{かた}を
\ruby{御入}{お|い}れのだらう?。
』

『だつて
\ruby{姊}{ねえ}さん
\ruby{愍然}{かは|いさう}なのですもの!。% 「愍然 か(は)いさう」
』

『たゞ
\ruby{愍然}{かは|いさう}だつていふばかりで?。% 「愍然 か(は)いさう」
』

『ハア
\ruby{然樣}{さ|う}ですは。
』

『
\ruby{全}{まつた}くたゞ?。
』

『いやだ
\ruby{事}{こと}ネエ、
\ruby{何}{なん}だか
\ruby{異}{をか}アしく
\ruby{御聞}{お|き}きなさるのネ。
』

\ruby{面}{おもて}は
\ruby{漸}{やうや}く
\ruby{不安}{ふ|あん}を
\ruby{現}{あらは}し、
\ruby{言}{ことば}は
\ruby{忙}{せは}しく
\ruby{其}{そ}の
\ruby{問}{とひ}を
\ruby{{\換字{遮}}}{さへぎ}り
\ruby{止}{とど}めんとしたり。
お
\ruby{彤}{とう}は
\ruby{口}{くち}のほとりに
\ruby{見}{み}ゆるか
\ruby{見}{み}えざるかの
\ruby{笑}{ゑみ}を
\ruby{{\換字{浮}}}{うか}めて、
\ruby{{\換字{猶}}}{なほ}
\ruby{{\換字{追}}求}{つゐ|きう}して
\ruby{已}{や}まず。

『もしやお
\ruby{龍}{りう}ちやん、
お
\ruby{{\換字{前}}}{まへ}、あの
\ruby{人}{ひと}が
\ruby{好}{すき}になつたのぢやあ
\ruby{無}{な}くつて?。
』

『エ。
』

『ひよつとしたらお
\ruby{{\換字{前}}}{まへ}、
\ruby{胸}{むね}の
\ruby{底}{そこ}ぢやあ
\ruby{彼}{あ}の
\ruby{人}{ひと}を
\ruby{思}{おも}つてるのぢやあ
\ruby{無}{な}くつて?。
』

\ruby{眼}{め}の
\ruby{上}{うへ}に
\ruby{白刄}{はく|じん}を
\ruby{閃}{ひらめ}めかさるゝが
\ruby{如}{ごと}く、
\ruby{一語}{いち|ご}は
\ruby{一語}{いち|ご}より
\ruby{急}{きふ}に
\ruby{逼}{せま}り
\ruby{立}{た}てられて、
お
\ruby{龍}{りう}はさつと
\ruby{面}{おもて}を
\ruby{紅}{あか}くし、

『あら
\ruby{姊}{ねえ}さん、
\ruby{其樣}{そ|ん}な
\ruby{事}{こと}を
\ruby{云}{い}つちやあ
\ruby{妾}{わたし}あ
\ruby{{\換字{嫌}}}{いや}ですよ。
\ruby{妾}{わたし}やあ
\ruby{基樣}{そ|ん}な
\ruby{氣}{き}なんぞを
\ruby{些}{ちつと}も
\ruby{有}{も}つて
\ruby{居}{ゐ}やあ
\ruby{仕}{し}ませんは。
』

と、
\ruby{明}{あき}らかには
\ruby{答}{こた}へたれど、
\ruby{驚}{おどろ}き
\ruby{慌}{あわ}て
\ruby{{\換字{狼}}狽}{うろ|た}へてどぎまぎせる
\ruby{態}{さま}はあり〳〵と
\ruby{見}{み}えたり。
お
\ruby{彤}{とう}は
\ruby{此度}{こ|たび}は
\ruby{嫣然}{にこ|り}と
\ruby{笑}{ゑみ}をつくつて、

『
\ruby{必然}{きつ|と}?。
』

と
\ruby{重}{かさ}ねて
\ruby{問}{と}へば、
お
\ruby{龍}{りう}は
\ruby{既}{はや}
\ruby{{\換字{浮}}}{う}き
\ruby{足}{あし}を
\ruby{踏堪}{ふみ|こた}へ
\ruby{身構}{み|がま}へを
\ruby{仕直}{し|なほ}して、

『だつて、
\ruby{知}{し}れきつてる
\ruby{事}{こと}ぢやあ
\ruby{有}{あ}りませんか。
\ruby{彼}{あ}の
\ruby{人}{ひと}は
お
\ruby{五十}{い|そ}さんていふ
\ruby{人}{ひと}を
\ruby{思}{おも}ひに
\ruby{思}{おも}ひぬいてるのですもの、
\ruby{横合}{よこ|あひ}から
\ruby{妾}{わたし}が
\ruby{思}{おも}つたつて
\ruby{何樣}{ど|う}なりましやう!。
いくら
\ruby{妾}{わたし}が
\ruby{馬鹿}{ば|か}だつて
\ruby{醉狂}{すゐ|きやう}だつて、
\ruby{其}{そ}の
\ruby{位}{くらゐ}の
\ruby{事}{こと}は
\ruby{知}{し}つてますから
\ruby{{\換字{空}}店}{あき|だな}へ
\ruby{郵便}{いう|びん}を
\ruby{抛}{はふ}り
\ruby{{\換字{込}}}{こ}むやうな
\ruby{事}{こと}を
\ruby{何}{なん}で
\ruby{爲}{し}ますものかネ。
ホヽホヽホヽ。
』

と
\ruby{戯言}{じやう|だん}まで
\ruby{云}{い}つて
\ruby{自}{みづか}ら
\ruby{笑}{わら}つて
\ruby{何氣}{なに|げ}なき
\ruby{態}{さま}なり。

お
\ruby{彤}{とう}は
お
\ruby{龍}{りう}の
\ruby{言}{ことば}を
\ruby{信}{しん}じたりや
\ruby{信}{しん}ぜざりしや
\ruby{知}{し}らず、

『
\ruby{然樣}{さ|う}かエ。
そんなら
\ruby{何}{なに}も
\ruby{既}{もう}
\ruby{云}{い}ふことは
\ruby{無}{な}いのだがネ。
\ruby{妾}{わたし}あ
\ruby{{\換字{又}}}{また}、
お
\ruby{{\換字{前}}}{まへ}が
\ruby{彼}{あ}の
\ruby{人}{ひと}を
\ruby{好}{す}いてでも
\ruby{居}{ゐ}るといふことなら、
\ruby{次第}{し|だい}に
\ruby{依}{よ}つちやあ
お
\ruby{{\換字{前}}}{まへ}の
\ruby{爲}{ため}に
\ruby{一}{ひ}ト
\ruby{苦勞}{く|らう}して、
お
\ruby{{\換字{前}}}{まへ}の
\ruby{身}{み}の
\ruby{收}{をさ}まりの
\ruby{好}{い}いやうに
\ruby{仕}{し}てあげやうかとも
\ruby{初手}{しよ|て}にはふつと
\ruby{思}{おも}つたのだよ。
』

『エ。
』

\ruby{全然}{まる|で}おもひの
\ruby{外}{ほか}なりし
\ruby{言葉}{こと|ば}に
お
\ruby{龍}{りう}は
\ruby{復}{また}
\ruby{驚}{おどろ}かされつ、
\ruby{我}{われ}
\ruby{知}{し}らず
\ruby{心}{こゝろ}を
\ruby{動}{うご}かして
\ruby{答}{こたへ}さへ
\ruby{答}{こた}へ
\ruby{鈍}{にぶ}りしが、
お
\ruby{彤}{とう}は
\ruby{早}{はや}くもその
\ruby{眼色}{め|いろ}を
\ruby{見}{み}て
\ruby{取}{と}りたり。

『だが
\ruby{彼}{あ}の
\ruby{人}{ひと}は
\ruby{彼樣}{あ|れ}だし、
\ruby{何樣}{ど|ん}なものだらうかと
\ruby{思}{おも}つて
\ruby{居}{ゐ}る
\ruby{中}{うち}、また
\ruby{別}{べつ}に
\ruby{一條}{ひと|つ}の
\ruby{話}{はなし}が
\ruby{出}{で}て
\ruby{來}{き}たので、
お
\ruby{{\換字{前}}}{まへ}の
\ruby{爲}{ため}に
\ruby{彼}{あ}の
\ruby{人}{ひと}は
\ruby{棄}{す}てる
\ruby{者}{もの}に
\ruby{仕}{し}た
\ruby{方}{はう}が
\ruby{宜}{い}いと
\ruby{決}{き}めて
\ruby{居}{ゐ}たところ、
\ruby{丁度}{ちやう|ど}
お
\ruby{{\換字{前}}}{まへ}も
\ruby{左樣}{さ|う}いふ
\ruby{氣}{き}だと
\ruby{今}{いま}
\ruby{聞}{き}いて
\ruby{妾}{わたし}も
\ruby{安心}{あん|しん}したよ。
さうで
\ruby{無}{な}けりやあ
\ruby{彼}{あ}の
\ruby{人}{ひと}を
\ruby{思}{おも}つたつて
\ruby{詰}{つま}らないといふ
\ruby{事}{こと}を
\ruby{云}{い}はうかと
\ruby{思}{おも}つて
\ruby{居}{ゐ}たところたよ。
』

\ruby{其}{そ}の
\ruby{云}{い}ふところは
\ruby{假設}{う|そ}にや
\ruby{實際}{ほん|と}にや、
お
\ruby{龍}{りう}はたゞ
\ruby{我}{わ}が
\ruby{心}{こゝろ}の
\ruby{蜘蛛}{く|も}の
\ruby{圖}{づ}に
\ruby{搦}{から}められ
\ruby{行}{ゆ}きて、
\ruby{抵抗}{あら|が}はんに
\ruby{抵抗}{あら|が}ふべき
\ruby{力}{ちから}の
\ruby{入}{い}れどころも
\ruby{知}{し}らぬ
\ruby{中}{うち}、
\ruby{次第}{し|だい}
\g詰めruby{々々}{〳〵}に
\ruby{自由}{じ|いう}を
\ruby{奪}{うば}はれ
\ruby{奪}{うば}はるゝが
\ruby{如}{ごと}く
\ruby{覺}{おぼ}ゆるのみ、

『ネエお
\ruby{龍}{りう}ちやん、
\ruby{仕樣}{し|やう}が
\ruby{無}{な}いやネ、あゝいふ
\ruby{人}{ひと}は。
お
\ruby{{\換字{前}}}{まへ}
\ruby{彼}{あ}の
\ruby{人}{ひと}を
\ruby{何樣}{ど|う}いふ
\ruby{人}{ひと}だと
お
\ruby{思}{おも}ひだエ?。
なる
\ruby{程{\換字{情}}}{ほど|じやう}も
\ruby{有}{あ}らう、
\ruby{正直}{しやう|ぢき}でもあらう、
\ruby{學藝}{わ|ざ}も
\ruby{出來}{で|き}やうがネ、
\ruby{一生}{いつ|しやう}の
\ruby{{\換字{所}}天}{てい|しゆ}にするにやあ、
\ruby{氣}{き}むづかしやで、
\ruby{{\換字{貧}}乏性}{びん|ばう|しやう}らしくつて、ヘチ
\ruby{頑固}{ぐわん|こ}なところが
\ruby{有}{あ}つて、
\ruby{彼}{あれ}あ
\ruby{餘}{あんま}り
\ruby{有}{あ}り
\ruby{難}{がた}くは
\ruby{無}{な}さゝうだネ。
といつて
\ruby{{\換字{情}}夫}{い|ろ}にするにやあ、
\ruby{容貌}{をと|こ}が
\ruby{惡}{わる}かあ
\ruby{無}{な}いが
\ruby{愛嬌}{あい|けう}の
\ruby{足}{た}りない、
\ruby{面白味}{おも|しろ|み}の
\ruby{薄}{うす}い、
\ruby{無粹}{ぶ|いき}の、
\ruby{世間}{せ|けん}を
\ruby{知}{し}らな
\ruby{{\換字{過}}}{す}ぎる{---}{---}
\ruby{何樣}{ど|う}も
お
\ruby{{\換字{前}}}{まへ}の
\ruby{相手}{あひ|て}にやあ
\ruby{些}{ちつと}
\ruby{不足}{ふ|そく}な
\ruby{男}{をとこ}ぢやあ
\ruby{無}{な}いか!。
』


\Entry{其二十一}

% メモ 校正終了 2024-05-14 2024-06-10
\原本頁{115-1}%
お
\ruby{龍}{りう}は
お
\ruby{彤}{とう}の
\ruby{水野}{みづ|の}を
\ruby{{\換字{評}}}{ひやう}するに
\ruby{{\換字{平}}}{たひ}らか
ならねども、
%
\ruby{反駁}{いひ|かへ}さんも
\ruby{何}{なん}と
\ruby{無}{な}く
\ruby[||j>]{後}{うしろ}
\ruby[||j>]{見}{ み}らるゝ
\ruby{心地}{こゝ|ち}せしが、
%
\ruby{其}{そ}の
\ruby{言}{い}ふ
ところ
\ruby{多}{おほ}くは
\ruby{當}{あた}れるを
\ruby[<g>]{如何}{いかん}とも
\ruby{爲}{す}る
\ruby{能}{あた}はず、
%
たゞ
\ruby{僅}{わづか}に、

\原本頁{115-4}%
『
あら
\ruby{姊}{ねえ}さん。
%
てんで
\ruby{妾}{わたし}あ
\ruby{全然}{まる|きり}
\ruby{其樣}{そ|ん}な
\ruby{事}{こと}を
\ruby{思}{おも}つてや
\ruby{仕無}{し|な}い
のですから、
%
\ruby{彼}{あ}の
\ruby{人}{ひと}が
\ruby{{\換字{貧}}乏}{びん|ばう}
\ruby{性}{しやう}だつて
\ruby[|g|]{無粹}{ぶいき}だつて
\ruby{何樣}{ど|う}だつて
\ruby{宜}{い}いぢや
\原本頁{115-6}\改行%
\ruby{有}{あ}りませんか、
%
\ruby{不足}{ふ|そく}でも
\ruby{{\換字{過}}}{す}ぎて
\ruby{居}{ゐ}ても
\ruby[||j>]{關}{かゝり}
\ruby[||j>]{係}{ あひ}の
% \ruby{關係}{かゝり|あひ}の
\ruby{無}{な}い
\ruby{事}{こと}ですは。
%
\ruby{隨{\換字{分}}}{ずゐ|ぶん}
\ruby{酷}{ひど}い
\ruby{事}{こと}ネエ、
%
\ruby{姊}{ねえ}さんの
\ruby{言}{くち}も。
』

\原本頁{115-8}%
と、
%
\ruby{知}{し}らざるを
\ruby{粧}{よそほ}ひて
\ruby{我}{われ}には
\ruby{聞}{き}き
\ruby{辛}{づら}き
\ruby{談}{はなし}を
\ruby{少}{すこ}しも
\ruby{早}{はや}く
\ruby{外}{はづ}さんと
\ruby{仕}{し}たり。

\原本頁{115-10}%
『
\ruby{然樣}{さ|う}さネエ。
%
ホヽヽ
\ruby[||j>]{關}{かゝり}
\ruby[||j>]{係}{ あひ}の
% \ruby{關係}{かゝり|あひ}の
\ruby{無}{な}いものを
\ruby{兎}{と}や
\ruby{角}{かく}
いふのには
\ruby{當}{あた}らない
のだがネ、
%
\ruby{此}{これ}あ
まあ
\ruby{無意}{た|ゞ}の
\ruby{話}{はなし}だと
\ruby{思}{おも}つて
\ruby{聞}{き}いて
\ruby{居}{ゐ}て
\ruby{御覽}{ご|らん}よ
\改行% 校正作業の簡略化のため
。
%
\原本頁{116-1}\改行%
お
\ruby{{\換字{前}}}{まへ}は
どうせ
\ruby{彼}{あ}の
\ruby{人}{ひと}を
\ruby{何樣}{だ|う}の
\ruby{彼樣}{か|う}の
となんぞ
\ruby{思}{おも}つては
\ruby{御}{お}いでゞ
\ruby{無}{な}い
といふの
だから、
%
\ruby{別}{べつ}に
\ruby{何}{なん}にも
\ruby{心配}{しん|ぱい}は
\ruby{無}{な}いがネ。
%
こゝに
\ruby{氣}{き}が
\原本頁{116-3}\改行%
\ruby{優}{やさ}しくつて
\ruby{而}{そ}して
\ruby{俠氣}{をとこ|ぎ}
のある
やうな
\ruby{{\換字{若}}}{わか}い
\ruby{女}{ひと}があつて、
%
\ruby{何樣}{ど|う}かした
\ruby{心}{こゝろ}の
\ruby[|g|]{機勢}{はずみ}から
\ruby{彼}{あ}の
\ruby{人}{ひと}を
\ruby{思}{おも}ふ
やうなことが
\ruby{有}{あ}る
とするとネ、
%
\ruby{早}{はや}く
\ruby{氣}{き}がついて
\ruby{引{\換字{返}}}{ひつ|かへ}して
\ruby{仕舞}{し|ま}へば
\ruby[||j>]{其}{それつ}
\ruby[||j>]{限}{ きり}で
% \ruby{其限}{それつ|きり}で
\ruby{濟}{す}むけれども、
%
\ruby[|g|]{田舎}{ゐなか}
\ruby{{\換字{道}}}{みち}
なんか
\ruby{歩}{ある}いても
\ruby{能}{よ}くある
\ruby{事}{こと}で、
%
\ruby{二十丁}{に|じふ|ちやう}
\ruby{三十}{さん|じふ}
\ruby[||j>]{丁}{ちやう}も
\ruby{間{\換字{違}}}{ま|ちが}つた
\ruby{路}{みち}へ
\ruby{踏{\換字{込}}}{ふみ|こ}んで
\ruby{仕舞}{し|ま}ふと、
%
あゝ
\ruby{間{\換字{違}}}{ま|ちが}つたと
\ruby{氣}{き}が
\ruby{付}{つ}いても
\ruby{後}{あと}へ
\ruby{{\換字{返}}}{かへ}る
\ruby{氣}{き}
には
なれないで、
%
\ruby{何樣}{ど|う}かして
\ruby{出拔}{で|ぬ}けやう
\ruby{出拔}{で|ぬ}けやうつて
\ruby{云}{い}ふんで
\ruby{餘計}{よ|けい}
\原本頁{116-9}\改行%
\ruby{變}{へん}な
\ruby{路}{みち}へ
\ruby{入}{はい}つて、
%
\ruby{下}{くだ}らない
\ruby{苦}{くるし}み
を
することが
\ruby{得}{え}て
\ruby{有}{あ}る
ものだが
\改行% 校正作業の簡略化のため
、
\原本頁{116-10}\改行%
\ruby{丁度}{ちやう|ど}
\ruby{其樣}{そ|ん}な
\ruby{譯}{わけ}で
\ruby{下手}{へ|た}に
\ruby{人}{ひと}を
\ruby{思}{おも}つて、
%
\ruby{少}{すこ}し
\ruby{宛}{づつ}
\ruby{少}{すこ}し
\ruby{宛}{づつ}
\ruby{深}{ふか}みへ
\ruby{入}{はい}つて
\原本頁{116-11}\改行%
\ruby{行}{ゆ}くと、
%
\ruby{{\換字{終}}}{しまひ}にやあ
\ruby{飛}{と}んだ
\ruby{目}{め}を
\ruby{見}{み}
\ruby{無}{な}けりやあ
ならない
やうな、
\ruby{馬鹿}{ば|か}な
ところへ
\ruby{行}{い}つて
\ruby{突}{つき}
\ruby{當}{あた}り
もするよ。
%
\ruby{何}{なん}でも
\ruby{{\換字{前}}{\換字{途}}}{さ|き}の
\ruby{知}{し}れない
\ruby{怪}{あや}しい% TODO CHECK 怪
\ruby{路}{みち}へ
\ruby{入}{はい}つたら、
%
\ruby{一二}{いち|に}
\ruby{丁}{ちやう}
しか
\ruby{歩}{ある}かない
\ruby{中}{うち}に
\ruby{立}{たち}
\ruby{止}{どま}つてネ、
\GWI{u1b048-u3099}つと% 「志」+「濁点」
\ruby{考}{かんが}へるか
\ruby{人}{ひと}に
\ruby{聞}{き}くかして、
%
\ruby{引{\換字{返}}}{ひつ|かへ}すのが
まあ
\ruby{肝心}{かん|じん}で、
%
\ruby{無暗}{む|やみ}に
\ruby{歩}{ある}いて
\ruby{行}{ゆ}くのは
\ruby{一番}{いち|ばん}
\ruby{危}{あぶな}い
\ruby{事}{こと}だよ。
%
\ruby{彼}{あ}の
\ruby{水野}{みづ|の}
つて
いふ
\ruby{人}{ひと}は
\ruby{一}{ひ}ト
\ruby{目見}{め|み}ても
\ruby{{\換字{分}}}{わか}る、
%
\ruby{性}{しやう}は
\ruby{良}{い}い、
%
\ruby{眞人間}{ま|にん|げん}だよ、
%
\ruby{不實}{ふ|じつ}な
\ruby{人}{ひと}ぢや
\ruby{無}{な}い。
%
だから
\原本頁{117-6}\改行%
\ruby{彼}{あ}の
\ruby{人}{ひと}が
\ruby{別}{べつ}に
\ruby{人}{ひと}を
\ruby{思}{おも}つてるので
\ruby{無}{な}けりやあ、
%
\ruby{彼}{あ}の
\ruby{人}{ひと}を
\ruby{好}{す}いた
といふ
\ruby{女}{ひと}が
\ruby{有}{あ}りやあ
\ruby{其}{そ}りやあ
\ruby{好}{す}いたで
\ruby{宜}{い}いのさ。
%
\ruby{而}{そ}して
\ruby{其}{そ}の
\ruby{女}{ひと}の
\ruby{思}{おもひ}も
\ruby{屹度}{きつ|と}% ルビ調整(原本通り)非グループルビ
\ruby{彼}{あ}の
\ruby{人}{ひと}に
\ruby{{\換字{分}}}{わか}つて、
%
\ruby{小說}{せう|せつ}
ならば
まあ
\ruby{芽出度}{め|で|たし}
\ruby{芽出度}{め|で|たし}
といふ
ところにも
なるだらうがネ。
%
\ruby{彼}{あ}の
\ruby{人}{ひと}が
\ruby{他}{ほか}の
\ruby{人}{ひと}を
\ruby{一心}{いつ|しん}に
\ruby{思}{おも}つてる
からにやあ、
%
\ruby{性}{しやう}の
\ruby{良}{い}い
\ruby{人}{ひと}だけに
\ruby{傍}{わき}からの
\ruby{思}{おも}ひは
\ruby{受}{う}け
\ruby{付}{つ}けまい、
%
\原本頁{117-11}\改行%
\ruby{眞人間}{ま|にん|げん}だけに
\ruby[||j>]{二}{ふた}
\ruby[||j>]{心}{ごゝろ}は
% \ruby{二心}{ふた|ごゝろ}は
\ruby{持}{も}つまいよ。
%
\ruby{然樣}{さ|う}すりやあ
\ruby{彼}{あ}の
\ruby{人}{ひと}を
\ruby{思}{おも}ふなあ
\ruby[||j>]{死}{つき}
\ruby[||j>]{路}{あたり}へ
% \ruby{死路}{つき|あたり}へ
\ruby{向}{むか}つて
\ruby{行}{い}く
やうな
もので、
%
\ruby{行}{い}けば
\ruby{行}{い}くだけの
\ruby{草臥儲}{くた|びれ|まう}け
だから、
%
そんな
\ruby{路}{みち}へ
\ruby{{\換字{若}}}{も}し
\ruby{一寸}{ちよ|つと}でも
\ruby{歩}{あし}が
\ruby{向}{む}いて
\ruby{居}{ゐ}たらば、
%
\ruby[|g|]{其方}{そつち}へ
\原本頁{118-3}\改行%
\ruby{踏{\換字{込}}}{ふみ|こ}んだか
\ruby{踏}{ふ}み
\ruby{{\換字{込}}}{こ}まない
\ruby{中}{うち}に
\ruby{後}{あと}へ
\ruby{引{\換字{返}}}{ひつ|かへ}して
\ruby{仕舞}{し|ま}ふと、
%
\ruby{然程}{さ|ほど}
\ruby{苦}{く}にも
ならない、
%
\ruby{損}{そん}も
\ruby{仕}{し}
\ruby{無}{な}いで
\ruby{濟}{す}む
といふ
\ruby{譯}{わけ}
なのだよ。
%
\ruby{誰}{たれ}しも
\ruby{損}{そん}
\ruby{路}{みち}を
\ruby{仕}{し}ないで
\ruby{世}{よ}の
\ruby{中}{なか}を
\ruby{歩}{ある}いて
\ruby{來}{く}るものは
\ruby{中々}{なか|〳〵}
\ruby{無}{な}い。
%
お
\ruby{{\換字{前}}}{まへ}は
お
\ruby{知}{し}り
でないが
\ruby{妾}{わたし}
だつて
\ruby{損}{そん}
\ruby{{\換字{道}}}{みち}を
\ruby[|g|]{澤山}{たんと}
\ruby{仕}{し}て
\ruby{來}{き}て
\ruby{居}{ゐ}る。
%
お
\ruby{{\換字{前}}}{まへ}は
\ruby{妾}{わたし}も
\ruby{知}{し}つてるが
\ruby{既}{もう}
\ruby{一度}{いち|ど}
\ruby{甚}{ひど}い
\ruby{冗}{むだ}
\ruby{{\換字{道}}}{みち}を
\ruby{歩}{ある}いて、
%
\ruby{踏拔}{ふみ|ぬき}も
\ruby{仕}{し}て
おいでだし
\ruby{生爪}{なま|づめ}も
\ruby{剝}{は}がして
おいでだし、
%
\ruby{散々}{さん|〴〵}な
\ruby{目}{め}に
お
\ruby{會}{あ}ひだつた
\ruby{人}{ひと}だから、
%
\ruby{今}{いま}さら
\原本頁{118-9}\改行%
また
\ruby{{\換字{前}}{\換字{途}}}{さ|き}の
\ruby{知}{し}れない
\ruby{怪}{あや}しい
\ruby{路}{みち}へ
なんぞ、
%
\ruby{無暗}{む|やみ}には
\ruby{入}{はい}つて
\ruby{御}{お}いででは
\ruby{有}{あ}るまいから
\ruby{宜}{い}いがネ。
』

\原本頁{118-11}%
お
\ruby{彤}{とう}は
\ruby{云}{い}ひ
\ruby{{\換字{終}}}{をは}つて
\ruby{默}{もく}し、
%
お
\ruby{龍}{りう}は
\ruby{聞}{き}き
\ruby{{\換字{終}}}{をは}つて
\ruby{默}{もく}し、
%
\ruby{互}{たがひ}に
\ruby{言葉}{こと|ば}の
\ruby{{\換字{絕}}}{た}えたる
ところへ、
%
\ruby{小間}{こ|ま}
\ruby[<j||]{使}{づかひ}の
お
\ruby{春}{はる}は
\ruby{次室}{つぎ|のま}より
\ruby{現}{あら}はれ、

\原本頁{119-2}%
『
あの
\ruby[|g|]{昨日}{きのふ}
お
\ruby{來臨}{い|で}
なすつた
お
\ruby{婆}{ばあ}さんの
\ruby{方}{かた}が
\ruby{御出}{お|いで}
になりました。
』

\原本頁{119-3}%
と
\ruby{云}{い}へば、

\原本頁{119-4}%
『
おゝ
\ruby{丁度}{ちやう|ど}
\ruby{好}{い}い
ところへ
だつた、
%
\ruby[|g|]{此方}{こちら}へと
\ruby{御云}{お|い}ひ。
%
お
\ruby{龍}{りう}ちやん、
%
お
\ruby{{\換字{前}}}{まへ}、
%
\ruby{吃驚}{びつ|くり}
おしで
\ruby{無}{な}いよ。
%
お
\ruby{{\換字{前}}}{まへ}の
\ruby[||j>]{大}{だい}
\ruby[||j>]{{\換字{嫌}}}{きらひ}の
% \ruby{大{\換字{嫌}}}{だい|きらひ}の
\ruby{靜岡}{しづ|をか}の
\ruby{叔母}{を|ば}さんだよ。
』

\原本頁{119-6}%
と、
%
お
\ruby{彤}{とう}は
\ruby{笑}{ゑみ}を
\ruby{含}{ふく}んで
\ruby{云}{い}ひたり。

\Entry{其二十二}

お
\ruby{彤}{とう}と
\ruby{我}{わ}が
\ruby{叔母}{を|ば}とは
\ruby{相識}{ちか|づき}なるべき
\ruby{筈}{はず}の
\ruby{無}{な}ければ、
\ruby{此家}{こ|ゝ}にて
\ruby{叔母}{を|ば}に
\ruby{會}{あ}はんとは
\ruby{夢}{ゆめ}にも
\ruby{思}{おも}ひがけざりし
お
\ruby{龍}{りう}の、
\ruby{主人}{ある|じ}の
\ruby{言葉}{こと|ば}を
\ruby{聞}{き}きても
\ruby{{\換字{猶}}}{なほ}
\ruby{信}{しん}じかねて、よもやと
\ruby{疑}{うたが}ひ
\ruby{訝}{いぶ}かれる
\ruby{間}{ま}も
\ruby{無}{な}く、
\ruby{既}{はや}
お
\ruby{春}{はる}に
\ruby{導}{みち}びかれて、
\ruby{身體}{から|だ}は
\ruby{一體}{いつ|たい}が
\ruby{小粒}{こ|つぶ}なる
\ruby{上}{うへ}に
\ruby{老}{お}いたればいと
\ruby{小}{ちひ}さく
\ruby{見}{み}ゆれど、
\ruby{石}{いし}の
\ruby{如}{ごと}くこつつりと
\ruby{堅}{かた}さうに
\ruby{緊}{しま}り
\ruby{切}{き}つたる
\ruby{小}{ちひ}さき
\ruby{顏}{かほ}、
\ruby{薄}{うす}くなりたる
\ruby{癖毛}{くせ|げ}のびつたりと
\ruby{地}{ぢ}に
\ruby{緊着}{かじ|りつ}ける
\ruby{小}{ちひ}さなる
\ruby{頭}{あたま}、
\ruby{負}{ま}けぬ
\ruby{氣}{き}が
\ruby{尖}{とが}つて
\ruby{露}{あらは}れたるやうなる
\ruby{小}{ちひ}さき
\ruby{三角}{さん|かく}の
\ruby{眼}{め}、
\ruby{都}{す}べて
\ruby{小}{ちひ}さきが
\ruby{中}{なか}に
\ruby{毫}{もつと}も
\ruby{緩}{たる}みの
\ruby{無}{な}き、
\ruby{我}{わ}が
\ruby{叔母}{を|ば}の
お
\ruby{{\換字{近}}}{ちか}は
\ruby{忽}{たちま}ちに
\ruby{現}{あら}はれたり。

\ruby{監}{あゐ}の
\ruby{味噌}{み|そ}
\ruby{漉縞}{こし|じま}の
\ruby{衣}{きもの}を
\ruby{襟元}{えり|もと}
\ruby{窄}{せま}く
\ruby{着}{き}て、
\ruby{疊}{たゝ}み
\ruby{皺見}{じわ|み}ゆる
\ruby{黑}{くろ}の
\ruby{紬}{つむぎ}の
\ruby{羽織}{は|おり}に、
\ruby{{\換字{古}}}{ひ}ねて
\ruby{堅}{かた}くなつた
\ruby{茶}{ちや}の
\ruby{細紐}{ほそ|ひも}を
\ruby{少}{すこ}し
\ruby{胸高}{むな|だか}にきつちりと
\ruby{結}{むす}び、
\ruby{妙}{めう}に
\ruby{角張}{かど|ば}つて
\ruby{坐}{すわ}つてしなやかならず
\ruby{挨拶}{あい|さつ}せるさまは、
\ruby{何樣}{ど|う}
\ruby{見}{み}ても
\ruby{靜岡}{しづ|をか}の
\ruby{在}{ざい}より
\ruby{出}{い}で
\ruby{來}{きた}りたる
\ruby{田舎}{ゐな|か}
\ruby{婆}{ばゞ}と
\ruby{見}{み}えて
\ruby{律義}{りち|ぎ}
\ruby{臭}{くさ}し。
されど
\ruby{明治}{めい|じ}の
\ruby{初年}{はじ|め}に
\ruby{兩親}{ふた|おや}に
\ruby{{\換字{連}}}{つ}れられて、
\ruby{東京}{とう|きやう}を
\ruby{離}{はな}れしまゝ
\ruby{茶圃麥圃}{ちや|ばたけ|むぎ|ばたけ}の
\ruby{間}{なか}に
\ruby{齷齪}{あく|せく}として
\ruby{年}{とし}を
\ruby{取}{と}りは
\ruby{仕}{し}たれ、
\ruby{根}{ね}からの
\ruby{田舎}{ゐな|か}
\ruby{者}{もの}ならぬに
\ruby{言語}{もの|いひ}だけは
\ruby{然}{さ}のみをかしからず。

『
\ruby{何樣}{ど|う}も
\ruby{昨日}{さく|じつ}はまことに
お
\ruby{喧}{やかま}しうございましたらう。
\ruby{老年}{とし|より}ではございますし、
\ruby{我張}{がつ|ぱ}り
\ruby{婆}{ばゞあ}ではございますし、それに
\ruby{田舎}{ゐな|か}に
\ruby{居}{を}りますので
\ruby{自然}{し|ぜん}と
\ruby{馬士}{ま|ご}かなんぞのやうな
\ruby{大聲}{おほ|ごゑ}になつて
\ruby{仕舞}{し|ま}ひまして、
\ruby{自{\換字{分}}}{じ|ぶん}の
\ruby{{\換字{勝}}手}{かつ|て}ばかり
\ruby{饒舌}{しや|べ}り
\ruby{散}{ち}らしましたから
\ruby{嘸}{さぞ}
\ruby{御{\換字{迷}}惑}{ご|めい|わく}でございましたらうと、
\ruby{是}{これ}でも
\ruby{{\換字{又}}}{また}
\ruby{殊{\換字{勝}}}{しゆ|しよう}らしいもので、
\ruby{後}{あと}では
\ruby{御氣}{お|き}の
\ruby{毒}{どく}に
\ruby{存}{ぞん}じましたのでございます。
どうも
\ruby{種々何}{いろ|〳〵|なに}や
\ruby{彼}{か}や
\ruby{御深切}{ご|しん|せつ}さまに
\ruby{有}{あ}り
\ruby{難}{がた}う
\ruby{存}{ぞん}じました。
それに
\ruby{御馳走}{ご|ち|そう}にまでなりまして、
\ruby{夜{\換字{分}}}{や|ぶん}にまで
お
\ruby{邪魔}{じや|ま}を
\ruby{致}{いた}しましたりなんぞして、まことに
\ruby{既}{はや}
\ruby{年甲斐}{とし|が|ひ}も
\ruby{無}{な}い
\ruby{自{\換字{分}}{\換字{勝}}手}{じ|ぶん|かつ|て}ばかりの
\ruby{婆}{ばゞあ}だと、
\ruby{御蔑視}{お|さげ|すみ}のところも
\ruby{御羞}{お|はづか}しうございました。
\ruby{若}{も}し
\ruby{萬一}{ひよ|つと}さて〳〵
\ruby{{\換字{勝}}手者}{かつ|て|もの}だと
\ruby{御愛想盡}{お|あい|そ|づ}かしも
\ruby{有}{あ}らうかと、
\ruby{宿}{やど}へ
\ruby{歸}{かへ}りましてから
\ruby{些心配致}{ちと|しん|ぱい|いた}しましたが、ナアニ
\ruby{馬鹿}{ば|か}にやあ
\ruby{怜悧}{り|こう}な
\ruby{方}{かた}の
\ruby{事}{こと}は
\ruby{{\換字{分}}}{わか}らなくつても
\ruby{怜悧}{り|こう}にやあ
\ruby{馬鹿}{ば|か}なものゝ
\ruby{事}{こと}は
\ruby{能}{よ}く
\ruby{{\換字{分}}}{わか}るだらうから、
\ruby{此方}{こつ|ち}の
\ruby{何程}{どれ|ほど}か
\ruby{有}{あ}り
\ruby{難}{がた}く
\ruby{思}{おも}つて
\ruby{居}{ゐ}る
\ruby{位}{ぐらゐ}の
\ruby{事}{こと}は
\ruby{御{\換字{分}}}{お|わか}りだらうからまあ
\ruby{安心}{あん|しん}だ、
\ruby{屹度}{きつ|と}
\ruby{馬鹿}{ば|か}
\ruby{婆}{ばゞあ}だけれど
\ruby{腹}{はら}の
\ruby{中}{なか}は
\ruby{人並}{ひと|なみ}だ
\ruby{位}{ぐらゐ}には
\ruby{思}{おも}つて
\ruby{居}{ゐ}て
\ruby{下}{くだ}さるだらうから、と
\ruby{斯樣}{か|う}まづ
\ruby{{\換字{勝}}手}{かつ|て}に
\ruby{決}{き}めて
\ruby{仕舞}{し|ま}つて、
\ruby{安堵}{おち|つ}いたのでございます。
ハヽヽ、
\ruby{何樣}{ど|う}か
\ruby{御恩}{ご|おん}には
\ruby{必}{かな}らず
\ruby{着}{き}ますから
\ruby{宜}{よろ}しく
\ruby{御願}{お|ねが}ひ
\ruby{申}{まを}しまする。
では
\ruby{此女}{こ|れ}ももう
\ruby{貴女}{あな|た}
\ruby{樣}{さま}が
\ruby{今日}{け|ふ}
お
\ruby{招}{よ}び
\ruby{下}{くだ}さいましたので?。
』

と、
\ruby{人}{ひと}の
\ruby{云}{い}ふ
\ruby{事}{こと}は
\ruby{餘}{あま}り
\ruby{聞}{き}かずに
\ruby{獨}{ひと}りで
\ruby{饒舌}{しや|べ}つて、
お
\ruby{彤}{とう}には
\ruby{語}{ことば}を
\ruby{挿}{はさ}む
\ruby{間}{ま}をさへほと〳〵
\ruby{與}{あた}へざるほど、
\ruby{身體}{な|り}には
\ruby{似合}{に|あ}はず
\ruby{大}{おほき}な
\ruby{頑健}{じやう|ぶ}なる
\ruby{聲}{こゑ}もて
\ruby{先}{ま}づ
\ruby{語}{かた}りたり。

\Entry{其二十三}

\原本頁{}
お
\ruby{彤}{とう}は
お
\ruby{{\換字{近}}}{ちか}が
\ruby{言}{ものい}へる
\ruby{間}{あひだ}にも、
%
\ruby{少}{すこ}しの
\ruby{受答}{うけ|こた}へを
\ruby{爲}{し}つ、
%
\ruby{語}{くち}を
\ruby{挿}{はさ}まんとせざるにはあらざりしも、
%
\ruby{立板}{たて|いた}に
\ruby{水}{みづ}とはいふべきならねど
\ruby{下}{くだ}り
\ruby{坂}{ざか}に
\ruby{走}{はし}る
\ruby{小車}{を|ぐるま}のやうに
\ruby{騷}{さわ}がしく
\ruby{忙}{せは}しく
\ruby{話}{はな}しつゞけられて
\ruby{口}{くち}を
\ruby{入}{い}れ
\ruby{{\換字{兼}}}{か}ね
\ruby{居}{ゐ}しが、
%
\ruby{今}{いま}
\ruby{斯}{か}く
\ruby{問}{と}ひかけられて
\ruby{僅}{わづか}に
\ruby{言葉}{こと|ば}を
\ruby{出}{いだ}し、

\原本頁{}
『いゝえ
\ruby{然樣}{さ|う}ぢやあ
\ruby{有}{あ}りませんが
\ruby{他}{ほか}の
\ruby{事}{こと}でもつて、
%
\ruby{丁度}{ちやう|ど}
\ruby{自然}{ひと|りで}に
\ruby{先刻方見}{さつ|き|がた|み}えたので、
』

\原本頁{}
と
\ruby{云}{い}ひかけて
お
\ruby{龍}{りう}の
\ruby{方}{はう}を
\ruby{莞爾}{に|こ}やかに
\ruby{見}{み}やり、

\原本頁{}
『お
\ruby{龍}{りう}ちやん
お
\ruby{{\換字{前}}}{まへ}、
%
\ruby{默}{だま}つておいでぢやあ
\ruby{不可}{いけ|ない}よ、
%
\ruby{叔母}{を|ば}さんぢやあ
\ruby{無}{な}いかネ。
』

\原本頁{}
と
\ruby{輕}{かろ}き
\ruby{一句}{いつ|く}を
\ruby{與}{あた}へつ、
%
また
お
\ruby{{\換字{近}}}{ちか}に
\ruby{向}{むか}ひて、

\原本頁{}
『きまりが
\ruby{惡}{わる}いもので
\ruby{羞澁}{はに|か}んで
\ruby{困}{こま}つて
\ruby{居}{ゐ}るのですよ。
%
ホヽヽまだ
\ruby{{\換字{若}}}{わか}くつて、
%
いつそ
\ruby{可憐}{か|はい}らしいぢやあ
\ruby{有}{あ}りませんか。
%
どうかまあ
\ruby{今日}{け|ふ}のところは
\ruby{御叱}{お|しか}りなさらないでネ、
%
\ruby{貴卿}{あな|た}が
\ruby{御目上}{お|め|うへ}ですから
\ruby{優}{やさ}しく
\ruby{仕}{し}て
\ruby{御與}{お|や}りなすつてネ。
』

\原本頁{}
と、
%
\ruby{二人}{ふた|り}の
\ruby{間}{あひだ}をば
\ruby{取}{と}り
\ruby{繕}{つくろ}ふやうに
\ruby{云}{い}へり。

\原本頁{}
\ruby{此}{こ}の
\ruby{叔母}{を|ば}が
\ruby{擇}{えら}み
\ruby{定}{さだ}めし
\ruby{婿}{むこ}を
\ruby{{\換字{嫌}}}{きら}ひしより、
%
\ruby{{\換字{朝}}}{あさ}となく
\ruby{夜}{よる}と
\ruby{無}{な}く
\ruby{論}{い}ひ
\ruby{合}{あ}ひ
\ruby{睨}{にら}み
\ruby{合}{あ}ひて、
%
さらだに
\ruby{性}{しやう}の
\ruby{合}{あ}はぬ
\ruby{中}{なか}の、いよ〳〵おもしろからず、えゝ、
%
あた
\ruby{忌々}{いま|〳〵}しい、
%
\ruby{何}{なん}となるものぞと、
%
\ruby{後}{あと}の
\ruby{{\換字{迷}}惑}{めい|わく}も
\ruby{思}{おも}はずに
\ruby{無言}{だ|ま}つて
\ruby{駈}{か}け
\ruby{出}{だ}したるまゝ、
%
\ruby{恩}{おん}のある
\ruby{事}{こと}は
\ruby{知}{し}つて
\ruby{居}{ゐ}れど
\ruby{憎}{にく}らしさもあるに、
%
\ruby{手紙}{て|がみ}
\ruby{一本}{いつ|ぽん}も
\ruby{出}{だ}さで
\ruby{知}{し}らぬ
\ruby{顏}{かほ}に
\ruby{濟}{す}まし
\ruby{來}{きた}りし
\ruby{今日}{け|ふ}、
%
\ruby{突然}{だし|ぬけ}に
\ruby{此處}{こ|ゝ}に
\ruby{相}{あひ}
\ruby{會}{あ}ひては
お
\ruby{龍}{りう}も
\ruby{聊}{いさゝ}か
\ruby{驚}{おどろ}きつ、
%
\ruby{顏}{かほ}を
\ruby{見}{み}ては
\ruby{流石}{さす|が}
\ruby{氣}{き}の
\ruby{毒}{どく}さに
\ruby{面伏}{おも|ぶせ}の
\ruby{思}{おも}ひもすれど、
%
\ruby{{\換字{勝}}手}{かつ|て}のみ
\ruby{{\換字{強}}}{つよ}くして
\ruby{{\換字{遠}}慮}{ゑん|りよ}を
\ruby{知}{し}らぬ
\ruby{性急}{せつ|かち}の
\ruby{話聲}{はなし|ごゑ}の、
%
いつもながら
\ruby{喧}{やかま}しく
\ruby{耳}{みゝ}に
\ruby{響}{ひゞ}くを
\ruby{聞}{き}きては、
%
もう
\ruby{薄腹}{うす|はら}の
\ruby{立}{た}つほど
\ruby{蟲}{むし}が
\ruby{{\換字{嫌}}}{きら}つて
\ruby{厭}{いや}で〳〵
\ruby{堪}{たま}らず、
%
\ruby{出}{で}ずとも
\ruby{可}{い}い
\ruby{人}{ひと}が
\ruby{出}{で}て
\ruby{來}{き}てと
\ruby{{\換字{迷}}惑}{めい|わく}がりて、
%
\ruby{出}{で}るも
\ruby{引}{ひ}くもならぬに
\ruby{心}{こゝろ}そげて
\ruby{居}{ゐ}たりしが、
%
お
\ruby{彤}{とう}に
\ruby{斯}{か}く
\ruby{云}{い}はれては
\ruby{横}{よこ}を
\ruby{向}{む}いてばかりも
\ruby{居}{ゐ}られず、
%
\ruby「不承」のルビ調整
\g詰めruby{々々}{〴〵}に、

\原本頁{}
『
\ruby{叔母}{を|ば}さん……』

\原本頁{}
と
\ruby{云}{い}ひし
\ruby{限}{ぎ}り、
%
あとはぐず〴〵と
\ruby{口}{くち}の
\ruby{内}{うち}にて
\ruby{何}{なに}を
\ruby{云}{い}ひしやら
\ruby{知}{し}れず、
%
\ruby{{\換字{術}}無}{じゆつ|な}げに
\ruby{頭}{かしら}を
\ruby{下}{さ}げて
\ruby{漸}{やつ}と
\ruby{挨拶}{あい|さつ}すれば、
%
\ruby{叔母}{を|ば}はなか〳〵もう
\ruby{默}{だま}つては
\ruby{居}{ゐ}ず、
%
\ruby{三角}{さん|かく}の
\ruby{眼}{め}をきらりと
\ruby{光}{ひか}らせ、

\原本頁{}
『でもまあ
\ruby{能}{よ}く
\ruby{忘}{わす}れずに
\ruby{叔母}{を|ば}さんと
\ruby{御云}{お|い}ひだつたネ。
%
ハイ、
%
\ruby{其後}{その|ゝち}は\換字{志}ばらく。
%
お
\ruby{{\換字{前}}}{まへ}も
\ruby{御{\換字{達}}者}{お|たつ|しや}で、
%
\ruby{別}{べつ}に
\ruby{御天{\換字{道}}樣}{お|てん|たう|さま}にも
\ruby{愛想}{あい|そ}を
\ruby{盡}{つ}かされずに
\ruby{御暮}{お|くら}しで、
%
まあ
\ruby{結構}{けつ|かう}だネ。
%
まことにお
\ruby{{\換字{前}}}{まへ}の
\ruby{御蔭}{お|かげ}ぢやあ
\ruby{恐}{おそ}ろしい
\ruby{沸湯}{にえ|ゆ}を
\ruby{飮}{の}ませられました。
%
\ruby{會}{あ}つたら
\ruby{引捉}{ひつ|つかま}へて
\ruby{耳}{みゝ}でも
\ruby{扯}{ちぎ}り
\ruby{取}{と}つてあげて、
%
\ruby{何}{ど}の
\ruby[<h||]{位}{くらゐ}
\ruby{妾}{わたし}が
\ruby{痛}{いた}かつたか
\ruby{苦}{くる}しかつたか、
%
\ruby{此樣}{こ|ん}なものだつたよと、
%
\ruby{察}{さつ}して
\ruby{貰}{もら}ひましやうと
\ruby{思}{おも}つて
\ruby{居}{ゐ}ましたがネ、
%
\ruby{此方樣}{こち|ら|さま}の
\ruby{御言葉}{お|こと|ば}だから
\ruby{堪{\換字{忍}}}{かん|にん}してあげる。% 原文通り「堪忍」
%
\換字{志}かし
\ruby{彼}{あ}の
\ruby{事}{こと}は
\ruby{何樣}{ど|う}か
\ruby{此樣}{か|う}か% 原文通りルビは「かう」とする
\ruby{既}{もう}
\ruby{濟}{す}んで
\ruby{仕舞}{し|ま}つたが、
%
\ruby{一}{ひと}つ
\ruby{濟}{す}めば
\ruby{{\換字{又}}}{また}
\ruby{一}{ひと}つで
お
\ruby{{\換字{前}}}{まへ}の
\ruby{御蔭樣}{お|かげ|さま}で、
%
\ruby{斯樣}{か|う}して
\ruby{砂塵}{すなつ|ぼこり}ばかり
\ruby{立}{た}つ
\ruby{東京}{とう|きやう}くんだりへ、
%
\ruby{田舎}{ゐな|か}
\ruby{婆}{ばあ}さんがゑつちらおつちらと
\ruby{得々}{わざ|〳〵}
\ruby{出}{で}かけて
\ruby{來}{き}て、
%
\ruby{此方樣}{こち|ら|さま}へも
\ruby{御厄介}{ご|やく|かい}を
\ruby{掛}{か}けたりなんぞ
\ruby{仕}{し}ます。
%
\ruby{婆}{ばあ}さんを
\ruby{苦勞}{く|らう}ばかりさせて
\ruby{御手柄}{お|て|がら}の
\ruby{事}{こと}ですネ。
%
ほんとにお
\ruby{{\換字{前}}}{まへ}の
\ruby{仕}{し}た
\ruby{事}{こと}に
\ruby{碌}{ろく}な
\ruby{事}{こと}は
\ruby{有}{あ}りやあ
\ruby{仕}{し}ない。
%
お
\ruby{{\換字{前}}}{まへ}の
\ruby{仕}{し}た
\ruby{事}{こと}の
\ruby{中}{うち}で
\ruby{好}{い}い
\ruby{事}{こと}といふのは、
%
\ruby{此方樣}{こち|ら|さま}に
\ruby{可愛}{か|はい}がつて
\ruby{頂}{いたゞ}いて
\ruby{居}{ゐ}るといふ
\ruby{事}{こと}ばつかりだ。
%
\ruby{此方樣}{こち|ら|さま}にでも
\ruby{見離}{み|はな}されりやあ
お
\ruby{{\換字{前}}}{まへ}のやうなものは、
%
それこそ
\ruby{最{\換字{終}}}{しま|ひ}は
\ruby{倒}{のた}れ
\ruby{死}{じに}だよ。

\原本頁{}
\ruby{身}{み}に
\ruby{染}{し}みて
\ruby{覺}{おぼ}えておいでなさい、
%
もう
お
\ruby{{\換字{前}}}{まへ}の
\ruby{身體}{から|だ}は
お
\ruby{{\換字{前}}}{まへ}の
\ruby{料簡}{れう|けん}ぢやあ
\ruby{{\換字{勝}}手}{かつ|て}にはなりません。
%
\ruby{妾}{わたし}がすつかりと
\ruby{願}{ねが}つて
\ruby{置}{お}きました。
%
もう
\ruby{何}{なに}も
\ruby{彼}{か}も
\ruby{此方樣}{こち|ら|さま}の
\ruby{仰}{おつし}やる
\ruby{{\換字{通}}}{とほ}りにするのです。
%
\ruby{三絃}{さみ|せん}の
\ruby{師匠}{し|ゝやう}だなんて、
%
\ruby{彼樣惡}{あん|な|わる}い
\ruby{人}{ひと}のところへ、
%
\ruby{身}{み}を
\ruby{置}{お}いては
\ruby{決}{けつ}してなりません、
%
\ruby{出入}{で|はい}りしてもなりません。
%
\ruby{早{\換字{速}}}{さつ|そく}これから
\ruby{其家}{そ|こ}を
\ruby{出}{で}て
\ruby{此方}{こち|ら}へ
\ruby{御厄介}{ご|やく|かい}になつて、
%
\ruby{此方樣}{こち|ら|さま}を
\ruby{有}{あ}り
\ruby{難}{がた}いとおもつて
\ruby{身}{み}を
\ruby{責}{せ}めて
\ruby{御働}{お|はたら}きなさい。
』

\原本頁{}
と
\ruby{獨}{ひと}り
\ruby{合點}{が|てん}して、
%
まくし
\ruby{立}{た}てゝ
\ruby{指揮}{さし|ず}したり。

\原本頁{}
お
\ruby{彤}{とう}は
\ruby{訝}{いぶか}り
\ruby{疑}{うたが}ふ
お
\ruby{龍}{りう}を
\ruby{見}{み}て、

\原本頁{}
『
\ruby{叔母}{を|ば}さん、
%
\ruby{其}{それ}ぢやあ
\ruby{此}{こ}の
\ruby{人}{ひと}
にやあ
\ruby{{\換字{分}}}{わか}りますまい。
%
かういふ
\ruby{事}{こと}なのだよ
お
\ruby{龍}{りう}ちやん。
』

\原本頁{}
と
\ruby{靜}{しづか}に
\ruby{說}{と}き
\ruby{出}{いだ}したり。

\Entry{其二十四}

\ruby{最初}{さい|しよ}つから
\ruby{云}{い}ふと
\ruby{如是}{か|う}なのだよ
お
\ruby{龍}{りう}ちやん。
それ
\ruby{一昨年}{をと|ゝ|し}の
\ruby{夏}{なつ}の
\ruby{事}{こと}だつたね、これこれで
\ruby{此度叔母}{こん|ど|を|ば}に
\ruby{伴}{つ}れられて、
\ruby{厭}{いや}だけれども
\ruby{靜岡}{しづ|をか}へ
\ruby{行}{ゆ}きますからつて、
お
\ruby{前}{まへ}が
\ruby{暇乞}{いとま|ごひ}に
\ruby{御}{お}いでだつたことがあつた、
\ruby{其時}{そ|れ}からといふものは
\ruby{隨{\換字{分}}長}{ずゐ|ぶん|なが}い
\ruby{間}{あひだ}、
\ruby{此方}{こつ|ち}から
\ruby{手紙}{て|がみ}をあげても
\ruby{{\換字{返}}辭}{へん|じ}は
\ruby{少}{すくな}いし、たまに
\ruby{御{\換字{遣}}}{お|よこ}しでも
\ruby{極々短}{ごく|〳〵|みじ}つかい
\ruby{眞}{ほん}の
\ruby{義理濟}{ぎ|り|す}ましだけの
\ruby{事}{こと}だし、
\ruby{是}{これ}あ
\ruby{何}{なに}か
\ruby{知}{し}らないけれども
\ruby{甚}{ひど}く
\ruby{氣}{き}を
\ruby{取}{と}れておいでの
\ruby{事}{こと}があるのだらう、と
\ruby{思}{おも}つて
\ruby{居}{ゐ}る
\ruby{中}{うち}に
\ruby{今年}{こ|とし}の
\ruby{三月}{さん|ぐわつ}、ふらりつと
\ruby{妾}{わたし}の
\ruby{處}{ところ}へ
\ruby{御}{お}いでだつたが、
\ruby{顏付}{かほ|つき}は
\ruby{全然}{まる|で}
\ruby{變}{かは}つて
\ruby{仕舞}{し|ま}つて、
\ruby{前}{まへ}に
\ruby{見}{み}た
\ruby[g]{處女}{むすめ}らしいところは
\ruby{無}{な}くなつて
\ruby{御{\換字{終}}}{お|しま}ひだし、
\ruby{樣子}{やう|す}は
\ruby{何}{なん}だか
\ruby{知}{し}らないがそは〳〵としておいでゞ、
\ruby{妾}{わたし}に
\ruby{御話}{お|はなし}の
\ruby{談話}{はな|し}にも
\ruby{辻褄}{つぢ|つま}の
\ruby{合}{あ}はないところは
\ruby{有}{あ}り、
\ruby{何樣}{ど|う}も
\ruby{氣}{き}になる
\ruby{事}{こと}ばかしだから
\ruby{妾}{わたし}は
\ruby{心配}{しん|ぱい}して、すこし
\ruby{置}{お}いて
\ruby{{\換字{呉}}}{く}れと
\ruby{御言}{お|い}ひのことだから,あゝ
\ruby{宜}{い}いともと、
\ruby{表面}{うは|べ}は
\ruby{何}{なん}の
\ruby{氣}{き}もつかない
\ruby{風}{ふう}で
\ruby{家}{うち}へは
\ruby{置}{お}いて
\ruby{{\換字{進}}}{あ}げたものゝ、
\ruby{何樣}{ど|ん}なにいろ〳〵と
\ruby{物}{もの}をおもつたか
\ruby{知}{し}れないよ。
\ruby{此處}{こ|ゝ}に
\ruby{居}{ゐ}ることを
\ruby{靜岡}{しづ|をか}へ
\ruby{知}{し}らせては
\ruby{{\換字{呉}}}{く}れるなと、
\ruby{念}{ねん}に
\ruby{念}{ねん}を
\ruby{押}{お}しての
\ruby{御依頼}{お|たの|み}だつたけれども、
\ruby{今白状}{いま|はく|じやう}して
お
\ruby{前}{まへ}に
\ruby{謝罪}{あや|ま}るがネ、
\ruby{何樣}{ど|う}も
\ruby{物}{もの}の
\ruby{{\換字{道}}理}{だう|り}が
\ruby{然樣}{さ|う}は
\ruby{行}{い}かないと
\ruby{思}{おも}つたので、
お
\ruby{前}{まへ}には
\ruby{内密}{ない|しよ}でもつて
\ruby{靜岡}{しづ|をか}の
\ruby{叔母}{を|ば}さんへ、これ〳〵の
\ruby{樣子}{やう|す}で、
\ruby{如是}{か|う}
\ruby[g]{々々}{〳〵}して
お
\ruby{龍}{りう}ちやんは
\ruby{妾}{わたし}の
\ruby{方}{はう}に
\ruby{御}{お}いでだと、
\ruby{妾}{わたし}が
\ruby{全然}{すつ|かり}
\ruby{知}{し}らせて
\ruby{仕舞}{し|ま}つたのだよ。
』

\ruby{此}{これ}まで
\ruby{語}{かた}り
\ruby{掛}{か}けし
\ruby{時}{とき}、
\ruby{叔母}{を|ば}は
お
\ruby{龍}{りう}を
\ruby{見}{み}て、

『それ
\ruby{御覽}{ご|らん}。
\ruby{汝}{おまへ}のやうな
\ruby{{\換字{分}}}{わか}らないものゝ
\ruby{云}{い}ふ
\ruby{事}{こと}や
\ruby{思}{おも}ふことばかりが
\ruby{何}{なん}で
\ruby{通}{とほ}るものかエ。
\ruby{此方樣}{こち|ら|さま}のやうな
\ruby{方}{かた}は
\ruby{何程御優}{いく|ら|おや|さ}しくつても、
\ruby{角々}{かど|〳〵}は
\ruby{嚴然}{きつ|ぱり}と
\ruby{道理}{だう|り}のある
\ruby{方}{はう}へ
\ruby{御就}{お|つ}きになる!。
お
\ruby{前}{まへ}は
\ruby{知}{し}らないで
\ruby{好}{い}い
\ruby{氣}{き}になつておいでだつたらうが、ちやんと
\ruby{妾}{わたし}の
\ruby{方}{はう}へ
\ruby{御知}{お|し}らせくだすつて、いろ〳〵と
\ruby{御注意}{お|こゝろ|づけ}まで
\ruby{仕}{し}て
\ruby{下}{くだ}すつたのだ。
\ruby[g]{七{\換字{分}}{\換字{通}}}{しちぶどほ}り
\ruby[g]{八{\換字{分}}{\換字{通}}}{はちぶどほ}り
\ruby{話}{はなし}の
\ruby{定}{きま}つた
\ruby{婿}{むこ}を
\ruby{{\換字{嫌}}}{きら}つて
お
\ruby{前}{まへ}には
\ruby{出}{で}られる、
\ruby{何處}{ど|こ}へ
\ruby{行}{い}つたかもかいくれ
\ruby{知}{し}れず、
また
\ruby{短氣}{たん|き}を
\ruby{仕}{し}て
\ruby{若}{も}しや
\ruby{淵川}{ふち|かは}へでもかと
\ruby{何程妾}{どれ|ほど|わたし}が
\ruby{苦勞}{く|らう}して
\ruby{困}{こま}り
\ruby{拔}{ぬ}いたか
\ruby{知}{し}れない、
\ruby{其處}{そ|こ}へ
\ruby{此方樣}{こち|ら|さま}からの
\ruby{行屆}{ゆき|とど}いた
\ruby{御手紙}{お|て|がみ}で、やつと
\ruby{胸}{むね}の
\ruby{凝塊}{かた|まり}がすこし
\ruby{下}{さが}つた。
\ruby{居所}{ゐ|どこ}は
\ruby{知}{し}れたし、
\ruby{引捉}{ひつ|つかま}へてとも
\ruby{思}{おも}はないでは
\ruby{無}{な}かつたが、
\ruby{何樣}{ど|う}せ
\ruby[g]{其程{\換字{嫌}}}{それほどきら}つて
\ruby{居}{ゐ}る
\ruby{婿}{むこ}ならば、
\ruby{仕方}{し|かた}がないからいつそ
\ruby{破談}{は|だん}になすつたが
\ruby{宜}{よ}からうし、
\ruby{破談}{は|だん}になさるなら
\ruby{{\換字{又}}當人}{また|たう|にん}が
\ruby[g]{其地}{そちら}に
\ruby{居}{ゐ}ないで、
\ruby{何處}{ど|こ}へ
\ruby{行}{い}つたか
\ruby{知}{し}れないといふ
\ruby{{\換字{分}}}{ぶん}になすつた
\ruby{方}{はう}が、
\ruby{事}{こと}が
\ruby{濟}{す}み
\ruby{易}{やす}からうし、
\ruby{若}{も}し
\ruby{{\換字{強}}}{し}ひて
\ruby{無理}{む|り}な
\ruby{事}{こと}をなさるやうでは
\ruby{當人}{たう|にん}の
\ruby{爲}{ため}にも、
\ruby{却}{かへ}つてならないやうな
\ruby{事}{こと}になりは
\ruby{爲}{し}まいかと
\ruby{思}{おも}はれるから、
\ruby{次第}{し|だい}によつたら
\ruby{姑}{しばら}く
\ruby{此儘御預}{この|まゝ|お|あづ}かり
\ruby{申}{まを}しても
\ruby{宜}{よ}い、と
\ruby{能}{よ}く
\ruby{{\換字{分}}}{わか}つた
\ruby{此方樣}{こち|ら|さま}の
\ruby{御親切}{ご|しん|せつ}な
\ruby{御仰}{お|つし}ありやうでもあり、また
\ruby{此方樣}{こち|ら|さま}の
\ruby{御噂}{お|うはさ}も
\ruby{豫}{かね}て
\ruby{聞}{き}いて
\ruby{何樣}{ど|う}いふ
\ruby{方}{かた}かと
\ruby{合點}{が|てん}しても
\ruby{居}{ゐ}たので、とても
\ruby{妾}{わたし}には
\ruby{制{\換字{道}}}{せい|だう}の
\ruby{就}{つ}きません
\ruby{我儘者}{わが|まゝ|もの}でございますから
\ruby{既諦}{もう|あき}らめました、
\ruby{御甘}{お|あま}え
\ruby{申}{まを}しては
\ruby{濟}{す}みませんが
\ruby{然樣}{さ|う}いふ
\ruby{譯}{わけ}でございますれば、
\ruby{此方}{こち|ら}の
\ruby{話}{はなし}も
\ruby{解}{と}けて
\ruby{濟}{す}んで
\ruby{仕舞}{し|ま}ふまで
\ruby{御預}{お|あづ}かりを
\ruby{願}{ねが}ひます、
\ruby{成程今妾}{なる|ほど|いま|わたし}が
\ruby{出}{で}て
\ruby{參}{まゐ}りまして
\ruby{當人}{たう|にん}に
\ruby{會}{あ}つても
\ruby{何}{なん}にもなりますまいから、
\ruby[g]{御{\換字{迷}}惑}{ごめいわく}でもござりましやうが
\ruby{其}{それ}では
\ruby{何{\換字{分}}宜}{なに|ぶん|よろ}しく
\ruby{願}{ねが}ひまする、
\ruby{若}{も}し
\ruby{{\換字{又}}}{また}
\ruby{當人}{たう|にん}が
\ruby{不心得}{ふ|こゝろ|\換字{江}}なぞを
\ruby{致}{いた}して、
\ruby{御厄介}{ご|やく|かい}を
\ruby{掛}{か}けまするやうなことがございますれば
\ruby{屹度}{きつ|と}
\ruby{引受}{ひき|う}けまする、と
\ruby{斯樣}{か|う}いふ
\ruby{御挨拶}{ご|あい|さつ}を
\ruby{仕}{し}て
\ruby{願}{ねが}つて
\ruby{置}{お}いたのだ。
\ruby{今解}{いま|わか}つたかエ、
\ruby{妾}{わたし}の
\ruby{心持}{こゝろ|もち}も
\ruby{此方樣}{こち|ら|さま}の
\ruby{御思慮}{お|かん|がへ}も。
それほど
\ruby{妾}{わたし}にも
\ruby{此方樣}{こち|ら|さま}にも
\ruby{人知}{ひと|し}れず
\ruby{氣}{き}を
\ruby{揉}{も}ませて
\ruby{置}{お}いて、それだのに
\ruby{何}{なん}だエ、
\ruby{月日}{つき|ひ}も
\ruby{經}{たゝ}ない
\ruby{中}{うち}に
\ruby{{\換字{又}}}{また}
\ruby{此方樣}{こち|ら|さま}を
\ruby{駈}{か}け
\ruby{出}{だ}して、\------
\ruby{妹}{いもと}のやうに
\ruby{思}{おも}ふ
\ruby{子}{こ}のやうに
\ruby{思}{おも}ふとまで
\ruby{云}{い}つてくださる
\ruby{此方樣}{こち|ら|さま}の
\ruby{御親切}{ご|しん|せつ}も、
\ruby{妾}{わたし}は
お
\ruby{前}{まへ}の
\ruby{眞實}{ほん|と}の
\ruby{叔母}{を|ば}だけれども
\ruby{然樣}{さ|う}は
\ruby{濃}{こま}かに
お
\ruby{前}{まへ}のためを
\ruby{思}{おも}ふことは
\ruby{出來}{で|き}ないと
\ruby{我}{が}の
\ruby{折}{お}れるほどに
\ruby{仕}{し}て
\ruby{下}{くだ}さる
\ruby{有}{あ}り
\ruby{難}{がた}い
\ruby{此方樣}{こち|ら|さま}の
\ruby{御恩}{ご|おん}をも
\ruby{全}{まる}で
\ruby{餘所}{よ|そ}にして、
\ruby{何}{なに}が
\ruby{不足}{ふ|そく}で
\ruby{無言}{だん|まり}で
\ruby{三絃}{さみ|せん}の
\ruby{師匠}{し|しやう}だなんて
\ruby{彼}{あ}んな
\ruby{惡}{わる}い
\ruby{奴}{やつ}のところへ
\ruby{行}{い}つた。
これ、
\ruby{何故}{な|ぜ}
\ruby{此方樣}{こち|ら|さま}を
\ruby{後}{あと}にして
\ruby[g]{稽古所}{けいこじよ}なんぞの
\ruby{手助}{てだ|すけ}けを
\ruby{仕}{し}て
\ruby[g]{自墮落}{じだらく}に
\ruby{暮}{くら}したのだエ。
\ruby{彼女}{あ|れ}あ
お
\ruby{前}{まへ}、
お
\ruby{前}{まへ}に
\ruby{碌}{ろく}でも
\ruby{無}{な}い
\ruby{男}{をとこ}なんぞを
\ruby{取}{と}り
\ruby{持}{も}つた
\ruby{狸婆}{たぬき|ばゞあ}ぢや
\ruby{無}{な}いか。
\ruby{性凝}{しや|うこ}りも
\ruby{無}{な}く、まだ
\ruby{{\換字{浮}}氣}{うは|き}が
\ruby{仕}{し}たくつて、
\ruby{彼樣}{あ|ん}な
\ruby{奴}{やつ}に
\ruby[g]{末始{\換字{終}}}{すゑしじう}は
\ruby{食}{く}はれるのも
\ruby{知}{し}らないで、
\ruby{此方樣}{こち|ら|さま}を
\ruby{出}{で}たのかエ。
\ruby{猫}{ねこ}!。
いやらしい
\ruby{猫}{ねこ}!。
ほんとにいやらしい
\ruby{猫}{ねこ}!。
\ruby{猫}{ねこ}だつて
\ruby{畜}{か}はれた
\ruby{恩}{おん}を
\ruby[g]{三日經}{みつかた}つてから
\ruby{忘}{わす}れる、
\ruby{汝}{おまへ}あ
\ruby{畜}{か}はれて
\ruby{居}{ゐ}て
\ruby[g]{可愛}{かはい}がられて
\ruby{居}{ゐ}て
\ruby{既時}{す|ぐ}に
\ruby{忘}{わす}れたのだ。
\ruby{妾}{わたし}にも
\ruby{然樣}{さ|う}だつた、
\ruby{此方樣}{こち|ら|さま}にも
\ruby{然樣}{さ|う}だつた。
お
\ruby{前}{まへ}のやうな
\ruby{好}{い}い
\ruby{姪}{めひ}をもつて
\ruby{人樣}{ひと|さま}の
\ruby{前}{まへ}で、
\ruby{妾}{わたし}あほんとに
\ruby{肩身}{かた|み}が
\ruby{廣}{ひろ}くつて
\ruby{何樣}{ど|ん}なにか
\ruby{嬉}{うれ}しいよ。
』

と、
\ruby{例}{れい}の
\ruby{眼}{め}を
\ruby{動}{うご}かし〳〵
\ruby{思}{おも}ふさまに
\ruby{罵}{のゝし}つたり。


\Entry{其二十五}

\ruby{然樣}{さ|う}いふ
\ruby{氣性}{き|しやう}の
\ruby{人}{ひと}と
\ruby{思}{おも}へば
\ruby{腹}{はら}は
\ruby{立}{た}たぬながら、
\ruby{理由}{わ|け}も
\ruby{知}{し}らず
\ruby{唯一概}{ただ|いち|がい}に
\ruby{猫}{ねこ}よ
\ruby{畜生}{ちく|しやう}よ
\ruby{猫}{ねこ}にも
\ruby{劣}{おと}るとは
\ruby[g]{何程叔母樣}{いくらをばさま}なればとて
\ruby{餘}{あま}りなる
\ruby[g]{言葉}{ことば}。
\ruby{靜岡}{しづ|をか}から
\ruby{唯一}{ただ|ひと}つの
\ruby{頼}{たのみ}にして
\ruby{出}{で}て
\ruby{來}{き}たほどの
\ruby{此家}{こ|ヽ}を
\ruby[g]{無言}{だんまり}で
\ruby{出}{で}たのは、よく〳〵
\ruby{口惜}{く|や}しい
\ruby{悲}{かな}しい
\ruby{事}{こと}の
\ruby{有}{あ}つたればこそ、
\ruby{生}{い}きて
\ruby{復顏}{また|かほ}を
\ruby{見}{み}たり
\ruby{見}{み}られたりする
\ruby{氣}{き}が
\ruby{些}{すこし}でもあつては、お
\ruby{彤}{とう}さんの
\ruby{親切}{しん|せつ}を
\ruby{餘{\換字{所}}}{よ|そ}にして、
\ruby{何樣}{ど|う}して
\ruby{彼事}{あ|ん}な
\ruby{事}{こと}の
\ruby{出來}{で|き}るものでは
\ruby{無}{な}し、
\ruby{全}{まつた}く
\ruby{憎}{にく}い
\ruby{憎}{にく}い
\ruby{源}{げん}を
\ruby{殺}{ころ}して
\ruby{自分}{じ|ぶん}も
\ruby{死}{し}んで
\ruby{仕舞}{し|ま}ふ
\ruby{氣}{き}で、
\ruby{濟}{す}まないことは
\ruby[g]{悉皆冷}{みんなつめた}くなつてから
\ruby{謝罪}{わ|び}る
\ruby{積}{つも}りの、
\ruby{遺書}{かき|おき}さへ
\ruby{身}{み}に
\ruby{着}{つ}けて
\ruby{持}{も}つて
\ruby{居}{ゐ}て
\ruby{此家}{こ|ヽ}を
\ruby{{\換字{脱}}}{ぬ}けて、
\ruby{出會}{で|あ}つたが
\ruby[g]{最後一發}{さいごひとうち}と
\ruby{思}{おも}つて
\ruby{居}{ゐ}た、
\ruby{其}{それ}は
\ruby{其}{そ}の
\ruby{事無}{こと|な}くて
\ruby{其}{そ}の
\ruby{意}{おもひ}の
\ruby{見}{み}えずに
\ruby{濟}{す}んだゆゑ、たヾ
\ruby[g]{{\換字{勝}}手淫奔}{かつていたづら}の
\ruby{心}{こヽろ}から
\ruby{彼樣}{あ|ん}なところへ
\ruby{行}{い}つて、
\ruby{身}{み}を
\ruby[g]{自堕落}{じだらく}に
\ruby[g]{稽古{\換字{所}}}{けいこじよ}に
\ruby{置}{お}くと
\ruby{思}{おも}はれても
\ruby[g]{仕方無}{しかたな}けれど、
\ruby{自分}{じ|ぶん}の
\ruby{姪}{めひ}を
\ruby{其樣}{そ|ん}なに
\ruby{惡}{わる}いものにして
\ruby{罵詈}{く|さ}せば
\ruby{何}{なに}が
\ruby{面白}{おも|しろ}いのか、
\ruby{辯解}{いひ|わけ}すれば
\ruby{{\換字{又}}男}{また|をとこ}を
\ruby{殺}{ころ}さうとした
\ruby{叔母}{を|ば}の
\ruby{知}{し}らぬ
\ruby{一條}{ひと|すぢ}の
\ruby{談}{はなし}を、こヽで
\ruby{新規}{しん|き}に
\ruby{仕出}{し|だ}さねばならぬ
\ruby{故}{ゆゑ}、
\ruby{知}{し}らぬを
\ruby{幸}{さいは}ひにして
\ruby{默}{だま}つて
\ruby{惡}{わる}く
\ruby{云}{い}はれて
\ruby{濟}{す}ませば、それで
\ruby{濟}{す}むことと
\ruby{濟}{す}ましも
\ruby{仕}{し}やうなれど、
\ruby{餘}{あま}りといへば
\ruby[g]{同{\換字{情}}}{おもひやり}の
\ruby{無}{な}い、
\ruby{我}{が}ばかりの
\ruby{人}{ひと}と、
\ruby{私}{ひそか}に
\ruby[g]{口惜}{くやし}く
\ruby{思}{おも}ふか
\ruby{眼}{め}さへ
\ruby{沾}{うる}ませて、お
\ruby{龍}{りう}は
\ruby{小}{ちひ}さくなりしまヽ
\ruby[g]{咳嗽一}{しはぶきひと}つせず、たヾ
\ruby[g]{頸垂}{うなだ}れて
\ruby{凝然}{じ|つ}としたるさまは、
\ruby{首}{くび}の
\ruby{座}{ざ}に
\ruby{直}{なほ}れる
\ruby{罪人}{ざい|にん}の
\ruby[g]{罪狀讀}{ざいじやうよ}まるヽを、
\ruby{何}{なん}と
\ruby{詮方}{せん|かた}も
\ruby{無}{な}く
\ruby{聞}{き}き
\ruby{居}{を}るにも
\ruby{似}{に}たり。

『
\ruby{其樣}{そ|ん}なにまあ
\ruby{苛}{ひど}いことを
\ruby{仰}{おつし}あらないでもの
\ruby{事}{こと}で、お
\ruby{龍}{りう}ちやんが
\ruby{妾}{わたし}のところを
\ruby{出}{で}て
\ruby[g]{彼家}{あすこ}へ
\ruby{行}{い}つて
\ruby{居}{ゐ}るやうな
\ruby{經歴}{ゆく|たて}になつたのには、いろ〳〵の
\ruby{理由}{わ|け}もあることで
\ruby{我儘}{わが|まヽ}ばかりぢやあ
\ruby{有}{あ}りません。
それは
\ruby{濟}{す}んで
\ruby{居}{ゐ}ることだから
\ruby{何樣}{ど|う}でも
\ruby{好}{い}いとして、
\ruby[g]{此度叔母}{こんどをば}さんが
\ruby[g]{此地}{こつち}へ
\ruby{出}{で}ておいでのは、お
\ruby{龍}{りう}ちやんお
\ruby{前}{まへ}の
\ruby{今居}{いま|ゐ}る
\ruby{家}{うち}の
\ruby{彼}{あ}の
\ruby{御師匠}{お|し|よ}さんネ、
\ruby{彼}{あ}の
\ruby{人}{ひと}がお
\ruby{前}{まへ}を
\ruby{{\換字{呉}}}{く}れろと
\ruby{叔母}{を|ば}さんのところへ、
\ruby{何}{なん}だか
\ruby{變}{へん}に
\ruby{搦}{から}んで
\ruby{云}{い}ひ
\ruby{込}{こ}んで
\ruby{行}{い}つたといふ
\ruby{其}{それ}から
\ruby{事}{こと}が
\ruby{起}{おこ}つたのだよ。
』

『ほんとにお
\ruby{前}{まへ}は
\ruby[g]{何處迄人}{どこまでひと}に
\ruby{世話}{せ|わ}を
\ruby{燒}{や}かせるのだか
\ruby{數}{すう}が
\ruby{知}{し}れない
\ruby{人}{ひと}だよ。
お
\ruby{前}{まへ}が
\ruby[g]{此方樣}{こちらさま}に
\ruby[g]{御厄介}{ごやくかい}になつて
\ruby[g]{靜穩}{おとな}しくさへ
\ruby{仕}{し}て
\ruby{居}{ゐ}れば
\ruby{紛紜}{いさ|くさ}の
\ruby{無}{な}いものを、
\ruby{性}{しやう}の
\ruby{知}{し}れない
\ruby{人}{ひと}の
\ruby{世話}{せ|わ}になんぞなるから、
\ruby{下}{くだ}らない
\ruby{苦勞}{く|らう}を
\ruby{無{\換字{益}}}{む|だ}にさせられる!。
\ruby[g]{此方樣}{こちらさま}の
\ruby[g]{御音信}{おたより}で
\ruby{汝}{おまへ}の
\ruby[g]{樣子}{やうす}も
\ruby{大抵}{たい|てい}は
\ruby{知}{し}つて
\ruby{居}{ゐ}たが、
\ruby{此頃}{この|ごろ}になつて
\ruby{汝}{おまへ}の
\ruby{師匠}{し|しやう}といふ
\ruby{人}{ひと}から、
\ruby{何}{なん}でもお
\ruby{前}{まへ}を
\ruby{貰}{もら}ひ
\ruby{度}{た}いからとの
\ruby{再々}{さい|〳〵}の
\ruby{云}{い}ひ
\ruby{込}{こ}みだ。
これよく
\ruby{御聞}{お|き}きなさい。
\ruby{一體}{いつ|たい}ならお
\ruby{前}{まへ}のやうなものは
\ruby{遣}{や}つて
\ruby{仕舞}{し|ま}ふ
\ruby{方}{はう}が
\ruby[g]{苦勞拂}{くらうばら}ひだから、
\ruby{鰹{\換字{節}}}{かつ|ぶし}でも
\ruby{付}{つ}けて
\ruby{{\GWI{u9063-k}}}{や}つて
\ruby{宜}{い}いのだが、
\ruby{見}{み}す
\ruby{見}{み}す
\ruby{食物}{くひ|もの}になつて
\ruby{仕舞}{し|ま}ふ
\ruby{前{\換字{途}}}{さ|き}が
\ruby{見}{み}えて
\ruby{居}{ゐ}るから、
\ruby{然樣}{さ|う}はなりませんといつて
\ruby{挨拶}{あい|さつ}したら、まあ
\ruby{何}{なん}といふことだらう、
\ruby{直}{すぐ}に
\ruby[g]{狼物}{おほかみもの}の
\ruby{本性}{ほん|しやう}を
\ruby{出}{だ}して、
\ruby{長}{なが}い
\ruby[g]{間御世話}{あひだおせわ}を
\ruby{仕}{し}て
\ruby{居}{ゐ}た
\ruby{費用}{もの|いり}がこれ〳〵だ、お
\ruby{龍}{りう}さんを
\ruby{下}{くだ}さらなけりやあ
\ruby[g]{御立替}{おたてかへ}を
\ruby{如何}{ど|う}かなすつてと、
\ruby{吃驚}{びつ|くり}するやうな
\ruby{法外}{はふ|ぐわい}のお
\ruby{金}{かね}を
\ruby{妾}{わたし}から
\ruby{取}{と}らうといふのだ。
\ruby{人}{ひと}を
\ruby[g]{田舍婆}{ゐなかばヾあ}に
\ruby{仕}{し}て
\ruby[g]{小馬鹿}{こばか}に
\ruby{仕}{し}たつて、
\ruby{野}{の}へ
\ruby{出}{で}ても
\ruby{座敷}{ざ|しき}へ
\ruby{上}{あが}つても
\ruby{人}{ひと}にやあ
\ruby{負}{ま}けない
\ruby{婆}{ばヾあ}だ、
\ruby[g]{先方}{さき}が
\ruby{然樣出}{さ|う|で}るなら、
\ruby[g]{此方}{こつち}も
\ruby{出樣}{で|やう}がある、お
\ruby{龍}{りう}は
\ruby{妾}{わたし}の
\ruby{姪}{めひ}だ、
\ruby{妾}{わたし}が
\ruby{{\換字{連}}}{つ}れて
\ruby{歸}{かへ}ります、お
\ruby{龍}{りう}に
\ruby{御注}{お|つ}ぎ
\ruby{込}{こ}みなすつたのは
\ruby{汝}{おまへ}さんの
\ruby[g]{御親切樣}{ごしんせつさま}だ、
\ruby{妾}{わたし}あ
\ruby[g]{些少}{ちつと}でも
\ruby{御恩}{ご|おん}になつた
\ruby{覺}{おぼ}えはありません、
\ruby{何}{なに}も
\ruby[g]{誘拐}{かどわかし}を
\ruby[g]{御商賣}{ごしやうばい}にやあなさりますまいから、
\ruby{人}{ひと}の
\ruby{姪甥}{めひをひ|}に
\ruby{指}{ゆび}を
\ruby{御}{お}さしになる
\ruby{事}{こと}は
\ruby{有}{あ}りますまいと、お
\ruby{前}{まへ}を
\ruby[g]{拉去}{ひつちよび}いて
\ruby{大手}{おほ|て}を
\ruby{振}{ふつ}て
\ruby{靜岡}{しづ|をか}へ
\ruby{歸}{かへ}つて、
\ruby{何樣}{ど|ん}な
\ruby{顏}{かほ}を
\ruby{仕}{し}て
\ruby{膨}{ふく}れるか
\ruby{見}{み}て
\ruby{{\GWI{u9063-k}}}{や}らうと
\ruby{思}{おも}つて、
\ruby{東京}{とう|きやう}の
\ruby{生狡}{いけ|ずる}い
\ruby[g]{狸婆}{たぬきばヾあ}の
\ruby{皮}{かは}を
\ruby{剥}{む}く
\ruby{氣}{き}で
\ruby{出}{で}て
\ruby{來}{き}たのがネ。
』

と、
\ruby[g]{面前}{まのあたり}にでもお
\ruby{關}{せき}が
\ruby{居}{ゐ}るやうに
\ruby{怒}{おこ}り
\ruby{立}{た}つて
\ruby{力}{りき}んで
\ruby{云}{い}へる
\ruby[g]{語氣面色}{ものいひかほつき}、なか〳〵
\ruby{當}{あた}り
\ruby{難}{がた}くあしらひ
\ruby{難}{にく}き
\ruby{婆}{ばヽ}なり。


\Entry{其二十六}

% メモ 校正終了 2024-05-15
\原本頁{139-4}%
『
だが
お
\ruby{龍}{りう}、
%
お
\ruby{聞}{き}きなさい、
%
\ruby{妾}{わたし}あ
\ruby{敵手}{あひ|て}が
\ruby{角}{つの}で
\ruby{向}{むか}つて
\ruby{來}{く}りやあ
\ruby{此方}{こつ|ち}も% 原本通り非グループルビ
\ruby{角}{つの}で
\ruby{向}{むか}つて
\ruby{行}{い}く
けれど、
%
お
\ruby{{\換字{前}}}{まへ}
のやうに
\ruby{眞}{しん}になつて
\ruby{世話}{せ|わ}を
\ruby{仕}{し}て
\ruby{吳}{く}れる
\ruby{叔母}{を|ば}にも
\ruby{自{\換字{分}}}{じ|ぶん}の% 原本通り非グループルビ
\ruby{{\換字{勝}}手}{かつ|て}
ぢやあ
お
\ruby{尻}{しり}を
\ruby{向}{む}けたり、
%
\ruby{折角}{せつ|かく}
\ruby{優}{やさ}しく
\ruby{仕}{し}て
\ruby{下}{くだ}さる
\ruby[|g|]{此方}{こちら}
\ruby{樣}{さま}をも
\ruby{時}{とき}の
\ruby{都合}{つ|がふ}
ぢやあ
\ruby{袖}{そで}に
するやうな、
%
\ruby{其樣}{そ|ん}な
\ruby{自{\換字{分}}{\換字{勝}}手}{じ|ぶん|かつ|て}% 原本通り非グループルビ
ばかりは
\ruby{夢}{ゆめ}にも
\ruby{仕}{し}ません。
%
お
\ruby{{\換字{前}}}{まへ}は
\ruby{何}{なん}ぞに
\ruby{付}{つ}けちやあ、
%
\原本頁{139-9}\改行%
\ruby{叔母}{を|ば}さんは
\ruby{無理}{む|り}
\ruby{壓制}{おし|つけ}だ、
%
\ruby{頑固}{ぐわん|こ}だ、
%
\ruby{自{\換字{分}}}{じ|ぶん}% 原本通り非グループルビ
\ruby{流義}{りう|ぎ}で
\ruby{何}{なん}でも
\ruby{押}{お}して
\ruby{行}{ゆ}かう
とすると
\ruby{御云}{お|い}ひ
だが、
%
そりやあ
\ruby{頑固}{ぐわん|こ}
でもあらう、
%
\ruby{自{\換字{分}}}{じ|ぶん}% 原本通り非グループルビ
\ruby{流義}{りう|ぎ}
でもあらう、
%
\ruby{然}{しか}し
\ruby{恩}{おん}は
\ruby{恩}{おん}、
%
\ruby{仇}{あだ}は
\ruby{仇}{あだ}で
ちやんと
\ruby{記}{おぼ}えて% 送り仮名は原本通り「え」
\ruby{居}{ゐ}ます、
%
お
\ruby{{\換字{前}}}{まへ}のやうに
\ruby{恩}{おん}も
\ruby{仇}{あだ}も
\ruby{見}{み}さかひの
\ruby{無}{な}い
\ruby{事}{こと}は
\ruby{妾}{わたし}あ
\ruby{仕}{し}ません。
%
だから
\ruby{今}{いま}
\原本頁{140-3}\改行%
その
お
\ruby{關}{せき}つていふ
\ruby{奴}{やつ}の
ところへ
\ruby{押}{お}し
\ruby{{\換字{込}}}{こ}んで
\ruby{行}{い}つて、
%
\ruby[|g|]{田舎}{ゐなか}
\ruby{婆}{ばゞあ}は
\ruby{田舎}{ゐな|か}% 原本通り非グループルビ
\ruby{婆}{ばゞあ}だけの
\ruby{意地}{い|ぢ}も
\ruby{有}{あ}りやあ
\ruby[||j>]{根}{こん}
\ruby[||j>]{性}{じやう}つ
% \ruby{根性}{こん|じやう}つ
\ruby{骨}{ほね}も
\ruby{突張}{つゝ|ぱ}つて
ゐる
ところを
\ruby{見}{み}せつけて
\ruby{{\換字{遣}}}{や}つて、
%
\ruby{間{\換字{違}}}{ま|ちが}つたことは
\ruby{云}{い}はない
\ruby{妾}{わたし}だもの
\ruby{何負}{なに|ま}けるものか、
%
\ruby{思}{おも}ふさま
\ruby{{\換字{捩}}}{ね}ぢ
\ruby{合}{あ}つて
\ruby{{\換字{捩}}}{ね}ぢ
\ruby{合}{あ}ひ
\ruby{拔}{ぬ}いて、
%
\ruby{{\換字{勝}}鬨}{と|き}を
\ruby{吐}{ふ}いて
\ruby{歸}{かへ}らうと
\ruby{思}{おも}つたが、
%
まづ
\ruby{其}{そ}の
\ruby{{\換字{前}}}{まへ}に
\ruby[|g|]{此方}{こちら}
\ruby{樣}{さま}
に% TODO CHECK 意味的には「に」か「へ」と思われるが、原本では判読不可。p140 7行目
\ruby{伺}{うかゞ}つて、
%
\ruby{段々}{だん|〴〵}
\ruby{御世話}{お|せ|わ}
になつた
\ruby{御禮}{お|れい}も
\ruby{云}{い}つたり、
%
また
お
\ruby{{\換字{前}}}{まへ}が
\ruby{我儘}{わが|まゝ}に
\ruby[|g|]{此方}{こちら}
\ruby{樣}{さま}を
\ruby{出}{で}て
\ruby{御親切}{ご|しん|せつ}を
\原本頁{140-9}\改行%
\ruby{無}{む}にした
\ruby{御謝罪}{お|わ|び}も
\ruby{仕}{し}たり、
%
\ruby{一應}{いち|おう}は
\ruby[|g|]{此方}{こちら}
\ruby{樣}{さま}の
\ruby{御思召}{お|ぼし|めし}も
\ruby{伺}{うかゞ}つてから、
%
\原本頁{140-10}\改行%
それから
\ruby{爭}{や}り
\ruby{合}{あ}ふなら
\ruby{爭}{や}り
\ruby{合}{あ}はなくつては
\ruby{義理}{ぎ|り}が
\ruby{惡}{わる}いと、
%
それで
\ruby{突掛}{つゝ|か}けに
\ruby[|g|]{此方}{こちら}
\ruby{樣}{さま}へ
\ruby{伺}{うかゞ}つて、
%
\ruby{御噂}{お|うはさ}に
ばかり
\ruby{伺}{うかゞ}つて
\ruby{居}{ゐ}た
\ruby{方}{かた}に
はじめて
\ruby{御目}{お|め}に
かゝつたのだよ。
%
ところが、
%
これ
お
\ruby{龍}{りう}、
%
お
\ruby{聞}{き}きなさいよ。
%
\ruby{{\換字{道}}理}{だう|り}に
\ruby{{\換字{違}}}{ちが}つたことを
\ruby{云}{い}は
\ruby{無}{な}いものは
\ruby{何處}{ど|こ}にでも
\ruby{味方}{み|かた}が
あります。
%
いろ〳〵と
お
\ruby{{\換字{前}}}{まへ}の
ことを
\ruby{御話}{お|はな}し
\ruby{申}{まを}した
ところ、
%
\ruby[<j||]{悉}{すつか}% 行末行頭の境界付近なので特例処置を施す
\ruby[<j||]{皆}{ り }% 行末行頭の境界付近なので特例処置を施す
% \ruby{悉皆}{すつ|かり}
\ruby[<j||]{妾}{わたし}の% 行末行頭の境界付近なので特例処置を施す
\ruby{云}{い}ふことを
\ruby{{\換字{道}}理}{もつ|とも}だと
\ruby{仰}{おつし}あつて
\ruby{下}{くだ}すつて、
%
お
\ruby{{\換字{前}}}{まへ}は
\ruby{何}{なん}ぞの
\ruby{時}{とき}には
\原本頁{141-5}\改行%
\ruby[|g|]{此方}{こちら}
\ruby{樣}{さま}を
\ruby{楯}{たて}に
\ruby{取}{と}つて、
%
\ruby{妾}{わたし}の
\ruby{云}{い}ふ
\ruby{事}{こと}を
\ruby{肯}{き}くまい
なんぞと
\ruby{思}{おも}つてるか
\ruby{知}{し}らないが、
%
もう
\ruby{然樣}{さ|う}は
\ruby{行}{い}きません
\ruby{御生憎樣}{お|あい|にく|さま}!、% 原文通りルビは「おあ(い)にくさま」
%
\ruby{何樣}{ど|う}して% 行末行頭の境界付近のため非踊り字表記
\ruby{何樣}{ど|う}して
\ruby{{\換字{判}}然}{はつ|きり}と
\ruby{物}{もの}の
\ruby{{\換字{道}}理}{だう|り}を
\ruby{御見{\換字{分}}}{お|み|わ}け
なさる
\ruby[|g|]{此方}{こちら}
\ruby{樣}{さま}だもの、
%
\ruby{可憐}{か|はい}い% 原本通り非グループルビ
\原本頁{141-8}\改行%
からつて
\ruby{御{\換字{前}}}{お|まへ}の
\ruby{味方}{み|かた}には
なつて
\ruby{下}{くだ}さらない、
%
すつかりと
\ruby{既}{もう}
\ruby{妾}{わたし}の
\原本頁{141-9}\改行%
\ruby{味方}{み|かた}になり
\ruby{切}{き}つて
\ruby{下}{くだ}すつたのだよ。
%
\ruby{彼樣}{あ|ん}な
ところに
\ruby{居}{ゐ}るのなんぞは
\ruby{全}{まつた}く
お
\ruby{{\換字{前}}}{まへ}が
\ruby{惡}{わる}い、
%
と
\ruby{散々}{さん|〴〵}に
\ruby{仰}{おつし}あつて、
%
\ruby[|g|]{彼家}{あすこ}を
\ruby{出}{で}させる
やうにとの
\ruby{御思召}{お|ぼし|めし}
なのだ。
%
\ruby{然}{しか}し
\ruby{何}{なに}も
\ruby{態々}{わざ|〳〵}と
ムキになつて
\ruby{惡}{わる}い
\ruby{奴}{やつ}を
\ruby{相手}{あひ|て}に
\ruby{爭}{や}り
\ruby{合}{あ}つても
\ruby{仕方}{し|かた}が
\ruby{無}{な}からう、
%
お
\ruby{{\換字{前}}}{まへ}が
\ruby{彼}{あ}の
\ruby{御師匠}{お|し|よ}さん
ていふ
\ruby{人}{ひと}の
\ruby{腹}{おなか}さへ
\ruby{解}{よ}めたら
\ruby[|g|]{彼家}{あすこ}に
\ruby{居}{ゐ}やう
\ruby{氣}{き}も
\ruby{有}{あ}るまいから、
%
\ruby{力}{ちから}を
\ruby{入}{い}れて
お
\ruby{{\換字{前}}}{まへ}を
\ruby{椀}{も}ぎ
\ruby{取}{と}りに
\ruby{行}{い}かなくつても
\ruby{濟}{す}む
\ruby{譯}{わけ}だ、
%
と
\ruby{仰}{おつし}あつて
\ruby{下}{くだ}すつたから、
%
\ruby{成程}{なる|ほど}と
\ruby{妾}{わたし}も
\ruby{思}{おも}ひついて、
%
\ruby{何}{なに}も
\ruby{老年}{とし|より}が
\ruby{皺}{しわ}つ
\ruby{顏}{かほ}へ
\ruby{筋}{すぢ}を
\原本頁{142-5}\改行%
\ruby{立}{た}てゝ
\ruby{喧嘩}{けん|くわ}しずとも
\ruby{濟}{す}む
ことならば、
%
と
\ruby[<j||]{狸}{たぬき}
\ruby[||j>]{婆}{ばゞあ}の
% \ruby{狸婆}{たぬき|ばゞあ}の
\ruby{面}{つら}の
\ruby{皮}{かは}を% 原本通り「皮 か(は)」
\ruby{拗}{むし}りに
\ruby{行}{い}くことだけは
\ruby{思}{おも}ひ
\ruby{止}{と}まつたが、
』

\原本頁{142-7}%
\ruby{此處}{こ|ゝ}まで
\ruby{語}{かた}れる
\ruby{時}{とき}、
%
お
\ruby{彤}{とう}は
\ruby{後}{あと}を
\ruby{取}{と}つて、

\原本頁{142-9}%
『
で、
%
ネエ、
%
お
\ruby{龍}{りう}ちやん、
%
\ruby{叔母}{を|ば}さんも
\ruby{實}{じつ}の
ところは、
%
お
\ruby{{\換字{前}}}{まへ}を
\ruby{直}{すぐ}に
\ruby{{\換字{前}}}{せん}のやうに
また
\ruby{{\換字{連}}}{つ}れて
\ruby{歸}{かへ}つても、
%
\ruby{何樣}{ど|う}も
\ruby[|g|]{田舎}{ゐなか}の
\ruby{人}{ひと}は
\ruby{{\換字{嫌}}}{きら}ひ
だなんて
\ruby{云}{い}つて
\ruby{取}{と}つて
\ruby{{\換字{遣}}}{や}る
\ruby{婿}{むこ}を% (婿 5a7f) 聟 805f
\ruby{{\換字{嫌}}}{きら}ふ
やうでは
\ruby{始末}{し|まつ}が
\ruby{着}{つ}かないからつて、
%
あぐんで
\ruby{居}{ゐ}らつしやる
のだから、
%
そこで
\ruby{妾}{わたし}が
\ruby{叔母}{を|ば}さんに
\ruby{對}{むか}つて、
%
\ruby{何樣}{ど|う}にでも
\ruby{彼樣}{あ|ん}な
\ruby{可厭}{い|や}な
\ruby{人}{ひと}の
\ruby{傍}{そば}から
お
\ruby{龍}{りう}さんを
\ruby{離}{はな}して
\ruby{御仕舞}{お|し|ま}ひ
なさるのは
\ruby{其}{そ}りやあ
\ruby{宜}{よ}う
ございましやうが、
%
それも
お
\ruby{龍}{りう}さんが
\ruby{彼}{あ}の
\ruby{御師匠}{お|し|よ}さんの
\ruby{腹}{おなか}の
\ruby{惡}{わる}いのを
\ruby{自{\換字{分}}}{じ|ぶん}から% 原本通り非グループルビ
\ruby{氣}{き}が
\ruby{付}{つ}いてで
\ruby{無}{な}くちやあ
\ruby{可}{い}けません。
%
それから
\ruby[|g|]{田舎}{ゐなか}へ
\ruby{{\換字{連}}}{つ}れて
\ruby{御歸}{お|かへ}りなさるのも、
%
\原本頁{143-5}\改行%
\ruby{矢張}{やつ|ぱ}り% 原本通り非グループルビ
お
\ruby{龍}{りう}さんが
\ruby{其}{そ}の
\ruby{氣}{き}に
ならなけりやあ、
%
\ruby{末始{\換字{終}}}{すゑ|し|じう}が% ルビは原本通り「ゆ」無し
\ruby{詰}{つま}りますまい。
%
\ruby{妾}{わたし}の
ところへ
\ruby{來}{き}て
\ruby{氣樂}{き|らく}に
\ruby{{\換字{遊}}}{あそ}んで
\ruby{居}{ゐ}るのが
\ruby{一番}{いち|ばん}
お
\ruby{龍}{りう}さんの
\原本頁{143-7}\改行%
\ruby{利益}{た|め}だとも
\ruby{思}{おも}ふし、
%
\ruby{{\換字{又}}}{また}
\ruby{妾}{わたし}が
\ruby{此樣}{こ|ん}な
\ruby{境{\換字{遇}}}{ざ|ま}で
\ruby{居}{ゐ}ながら
\ruby{立派}{りつ|ぱ}な
\ruby{口}{くち}を
きくのでは
\ruby{夢}{ゆめ}
\ruby{{\換字{更}}}{さら}
\ruby{無}{な}いけれども、
%
\ruby{其}{そ}の
\ruby{中}{うち}には
\ruby{末々}{すゑ|〴〵}の
お
\ruby{龍}{りう}さんの
\ruby{身}{み}の
\原本頁{143-9}\改行%
\ruby{收}{をさ}まりも
\ruby{妾}{わたし}の
\ruby{{\換字{分}}別}{ふん|べつ}や
\ruby{力}{ちから}て
\ruby{出來}{で|き}るだけは
\ruby{仕}{し}て
\ruby{上}{あ}げたいとも
おもひますが、
%
これも
お
\ruby{龍}{りう}さんが
\ruby{妾}{わたし}の
ところへ
\ruby{來}{き}て
\ruby{居}{ゐ}るのを
\ruby{{\換字{嫌}}}{きら}つちやあ
\ruby{仕方}{し|かた}は
\ruby{無}{な}いし、
%
\ruby{{\換字{若}}}{も}し
\ruby{{\換字{又}}}{また}
\ruby{餘{\換字{所}}}{よ|そ}の
\ruby{堅}{かた}い
ところへ
\ruby{奉公住}{ほう|こう|ず}みでも
\ruby{仕}{し}やう
といふ
やうな
\ruby{氣}{き}
でも
ある
なら、
%
それも
お
\ruby{龍}{りう}さんの
\ruby{料簡}{れう|けん}
\ruby{次第}{し|だい}だし、
%
\ruby{{\換字{又}}}{また}
\ruby{些}{すこし}は
\ruby{遲}{おそ}けれども
\ruby{此節柄}{この|せつ|がら}の
\ruby{事}{こと}では
\ruby{有}{あ}り、
%
\ruby{學校{\換字{通}}}{がく|かう|がよ}ひでも
\ruby{仕}{し}て、
%
\原本頁{144-3}\改行%
\ruby{何}{なん}でも
\ruby{女一人}{をんな|ひ|とり}で% 連続するルビなのでグループルビにはしない
\ruby{人}{ひと}の
\ruby{世話}{せ|わ}に
ならずに
\ruby{{\換字{遣}}}{や}つて
\ruby{行}{ゆ}かう
といふのなら、
%
\原本頁{144-4}\改行%
それも
\ruby{其}{それ}で
\ruby{妾}{わたし}の
\ruby{手}{て}で
\ruby{三年}{さん|ねん}や
\ruby{五年}{ご|ねん}は
\ruby{蝦茶袴}{えび|ちや|ばかま}さんで
\ruby{{\換字{過}}}{すご}させても
\ruby{上}{あ}げたいと
\ruby{思}{おも}ひますから、
%
\ruby{何事}{なに|ごと}も
\ruby{無理}{む|り}
\ruby{壓制}{おし|つけ}は
\ruby{可}{い}けません、
%
ようく
\ruby{當人}{たう|にん}の
\ruby[|g|]{{\換字{所}}存}{おなか}
も
ゆつくりと
\ruby{聞}{き}いて
\ruby{見}{み}て、
%
\ruby{其}{そ}の
\ruby{上}{うへ}で
\ruby{何樣}{ど|う}ともする
\ruby{方}{はう}が
\ruby{宜}{よ}うございます。
%
お
\ruby{師匠}{し|よ}さん
といふ
\ruby{人}{ひと}にやあ、
%
お
\ruby{金}{かね}を
\ruby{{\換字{遣}}}{よこ}せなら
\原本頁{144-8}\改行%
\ruby{{\換字{遣}}}{や}つても
\ruby{宜}{よ}うございますが、
%
\ruby{餘}{あま}り
\ruby{仕方}{し|かた}が
\ruby{憎}{にく}いから、
%
お
\ruby{金}{かね}は
\ruby{惜}{をし}くは
\ruby{無}{な}いけれ
\ruby{共}{ども}
\ruby{奪}{と}られるのは
\ruby{業腹}{ごふ|はら}です、
%
お
\ruby{龍}{りう}さんの
\ruby{心次第}{こゝろ|し|だい}で、
%
\ruby{何樣}{ど|う}とも
\ruby{仕}{し}て
\ruby{{\換字{遣}}}{や}りましやうつて、
%
\ruby{斯樣}{か|う}いつて
\ruby{妾}{わたし}あ
\ruby{御挨拶}{ご|あい|さつ}を
\ruby{仕}{し}たのだよ。
』

\原本頁{145-1}%
と、
%
\ruby{張}{は}りも
\ruby{弛}{ゆる}みもせぬ
\ruby{例}{れい}の
\ruby{調子}{てう|し}
に
\ruby{{\換字{述}}}{の}べたり。

\Entry{其二十七}

『
\ruby{解}{わか}つたかエ
お
\ruby{龍}{りう}、まあ
\ruby{何}{なん}といふ
\ruby{有}{あ}り
\ruby{難}{がた}い
\ruby{御優}{お|やさ}しい
\ruby{御思召}{お|ぼし|めし}だらう。
\ruby{小兒}{こ|ども}の
\ruby{時}{とき}から
\ruby{可愛}{か|はい}がつて
\ruby{下}{くだ}すつた
\ruby{上}{うへ}、
お
\ruby{{\換字{前}}}{まへ}は
\ruby{御恩}{ご|おん}に
\ruby{負}{そむ}いて
\ruby{狗猫}{いぬ|ねこ}のやうな
\ruby{事}{こと}を
\ruby{仕}{し}ても、
\ruby{別}{べつ}に
\ruby{愛想}{あい|そ}づかしも
\ruby{仕}{し}て
\ruby{下}{くだ}さらないで、
お
\ruby{{\換字{前}}}{まへ}が
\ruby{稽{\換字{古}}事}{けい|こ|ごと}を
\ruby{仕}{し}たければ
\ruby{其}{それ}も
\ruby{爲}{さ}せて
\ruby{{\換字{遣}}}{や}らう、
\ruby{家}{うち}に
\ruby{居}{ゐ}たいなら
\ruby{家}{うち}に
\ruby{置}{お}いて
\ruby{{\換字{遣}}}{や}らう、
\ruby{末々}{すゑ|〴〵}の
\ruby{身}{み}の
\ruby{{\換字{終}}局}{をさ|まり}も
\ruby{頼}{たの}むなら
\ruby{心配}{しん|ぱい}して
\ruby{{\換字{遣}}}{や}らうと、
\ruby{斯樣}{こ|ん}なに
\ruby{親切}{しん|せつ}にして
\ruby{下}{くだ}さる
\ruby{方}{かた}が
\ruby{何處}{ど|こ}にあると
\ruby{御思}{お|おも}ひだ。
\ruby{早}{はや}く
\ruby{料簡}{れう|けん}を
\ruby{入}{い}れかへて
\ruby{眞人間}{ま|にん|げん}になつて、\換字{志}やんと
\ruby{女}{をんな}は
\ruby[<h||]{女}{をんな}
\ruby{一人}{ひと|り}だけ
\ruby{羞}{はづ}かしくないやうな
\ruby{今日}{こん|にち}の
\ruby{{\換字{送}}}{おく}り
\ruby{方}{かた}をする
\ruby{身}{み}になつて、
\ruby{御恩{\換字{返}}}{ご|おん|がへ}しは
\ruby{出來無}{で|き|な}いまでも
\ruby{御親切}{ご|しん|せつ}を
\ruby{無}{む}に
\ruby{爲}{し}ないやう
\ruby{仕}{し}なければ、
\ruby{叔母}{を|ば}の
\ruby{此}{こ}の
\ruby{妾}{わたし}にやきもきと
\ruby{幾干}{いく|そ}の
\ruby{苦勞}{く|らう}させる、
\ruby{其}{そ}の
\ruby{罸}{ばち}はよしんば
お
\ruby{{\換字{前}}}{まへ}に
\ruby{當}{あた}らない
\ruby{迄}{まで}も、
\ruby{此方樣}{こち|ら|さま}の
\ruby{罰}{ばち}が
\ruby{末始{\換字{終}}}{すゑ|し|ゞう}は
\ruby{屹度}{きつ|と}
\ruby{當}{あた}つて、
お
\ruby{{\換字{前}}}{まへ}は
\ruby{碌}{ろく}な
\ruby{死狀}{しに|ざま}は
\ruby{出來}{で|き}ますまいよ。
\ruby{花}{はな}が
\ruby{奇麗}{き|れい}だ、
\ruby{蝶々}{てふ|〳〵}が
\ruby{可憐}{か|はい}い、
\ruby{人形}{にん|ぎやう}が
\ruby{氣}{き}に
\ruby{入}{い}つたなんぞと、
\ruby{其樣}{そ|ん}な
\ruby{下}{くだ}らない
\ruby{{\換字{浮}}々}{うか|〳〵}としたことを
\ruby{云}{い}つて
\ruby{居}{ゐ}て
\ruby{{\換字{過}}}{すご}せるものぢや
\ruby{無}{な}い
\ruby{世}{よ}の
\ruby{中}{なか}だから、
\ruby{宜}{い}い
\ruby{加減}{か|げん}に
\ruby{目}{め}を
\ruby{覺}{さ}まして
\ruby{確乎}{しつ|かり}とした
\ruby{氣}{き}になつて、
\ruby{片目}{めつ|かち}でも
\ruby{跛足}{びつ|こ}でも
\ruby{構}{かま}はないから
\ruby{食}{く}ふに
\ruby{困}{こま}らない
\ruby{男}{をとこ}を
\ruby{持}{も}つて、そして
\ruby{子}{こ}でも
\ruby{生}{う}んで
\ruby{末}{すゑ}の
\ruby{安堵}{おち|つき}を
\ruby{見}{み}るやうに
\ruby{仕無}{し|な}くつては
\ruby{濟}{す}む
\ruby{譯}{わけ}ぢやあ
\ruby{無}{な}い。
\ruby{自惚}{うぬ|ぼ}れて
\ruby{居}{ゐ}たつて
\ruby{可}{い}けは
\ruby{仕}{し}ない、
\ruby{{\換字{情}}夫}{をと|こ}に
\ruby{棄}{す}てられる
\ruby{位}{ぐらゐ}の
\ruby{容貌}{きり|やう}で
\ruby{居}{ゐ}て、
\ruby{飛}{と}び
\ruby{拔}{ぬ}けて
\ruby{何}{なに}が
\ruby{一}{ひと}つ
\ruby{出來}{で|き}るでも
\ruby{無}{な}い
\ruby{天禀}{うま|れつき}の
お
\ruby{{\換字{前}}}{まへ}なんぞは、
\ruby{自{\換字{分}}}{じ|ぶん}で
\ruby{理屈}{り|くつ}を
\ruby{付}{つ}けりやあ
\ruby{理屈}{り|くつ}も
\ruby{有}{あ}るだらうが、
\ruby{世界}{せ|かい}から
\ruby{云}{い}つて
\ruby{見}{み}りやあ
\ruby{圃中}{はた|け}の
\ruby{蠻南瓜}{たう|な|す}か
\ruby{茄子}{な|す}か
\ruby{白瓜}{しろ|うり}で、
\ruby{何樣}{ど|う}せ
\ruby{其邊中}{そこ|ら|ぢう}にある
\ruby{數物}{かず|もの}なのだもの、
\ruby{好}{い}い
\ruby{加減}{か|げん}に
\ruby{熟}{で}きた
\ruby{時{\換字{分}}}{じ|ぶん}に
\ruby{何樣}{ど|う}かなつて
\ruby{仕舞}{し|ま}ふのが
\ruby{當然}{あたり|まへ}の
\ruby{事}{こと}で、
\ruby{早{\換字{速}}}{さつ|さ}と
\ruby{緣}{えん}のあるところへ
\ruby{行}{い}つて
\ruby{一代働}{いち|だい|はた}らいて、
\ruby{種子}{た|ね}でも
\ruby{{\換字{遺}}}{のこ}すより
\ruby{他}{ほか}にいざもこざも
\ruby{有}{あ}りやあ
\ruby{仕}{し}ないのだよ。
だから
\ruby{妾}{わたし}が
\ruby{其}{そ}の
\ruby{積}{つも}りで
\ruby{世話}{せ|わ}を
\ruby{燒}{や}いて
\ruby{{\換字{遣}}}{や}つたのに、
\ruby{何}{なん}だの
\ruby{彼}{か}だのとだゞを
\ruby{捏}{こ}ねて
\ruby{妾}{ひと}を
\ruby{御困}{お|こま}らせだつたが、
\ruby{其}{それ}もまあ
\ruby{緣}{えん}が
\ruby{無}{な}かつたのだと
\ruby{其}{そ}の
\ruby{事}{こと}は
\ruby{濟}{す}まして
\ruby{仕舞}{し|ま}つたところで。
\ruby{蠻南瓜}{たう|な|す}を
\ruby{眞綿}{ま|わた}に
\ruby{包}{くる}んで
\ruby{藏}{しま}ひ
\ruby{{\換字{通}}}{とほ}したつて
\ruby{何}{なん}になるものでもない、
\ruby{矢張}{やつ|ぱり}
\ruby{何樣}{ど|う}かして
\ruby{片}{かた}づくところへ
\ruby{片}{かた}づけてやつて、
\ruby{持}{も}つて
\ruby{生}{うま}れた
\ruby{役}{やく}を
\ruby{濟}{す}まさせなけりやあなら
\ruby{無}{な}いから、そこで
\ruby{妾}{わたし}が
お
\ruby{願}{ねがひ}を
\ruby{仕}{し}て、それでは
\ruby{靜岡}{しづ|をか}に
\ruby{{\換字{連}}}{つ}れて
\ruby{歸}{かへ}ることは
\ruby{廢案}{や|め}に
\ruby{仕}{し}まして、
\ruby{御甘}{お|あま}え
\ruby{申}{まを}して
\ruby{濟}{す}みませんが
\ruby{何樣}{ど|う}か
\ruby{此方樣}{こち|ら|さま}で
\ruby{御使}{お|つか}ひなすつて
\ruby{頂}{いたゞ}きたうございます、
\ruby{何}{なん}でも
\ruby{手}{て}や
\ruby{足}{あし}に
\ruby{皸垢切}{ひび|あか|ぎれ}のきれますやうにこき
\ruby{使}{つか}つて
\ruby{下}{くだ}さいまして、
\ruby{其}{そ}の
\ruby{中}{うち}に
\ruby{破鍋}{われ|なべ}に
\ruby{綴蓋}{とぢ|ぶた}で、
\ruby{彼樣}{あ|ん}な
\ruby{奴}{やつ}ても
\ruby{貰}{もら}つて
\ruby{{\換字{遣}}}{や}らうといふ
\ruby{方}{かた}でもございましたら、
\ruby{此方樣}{こち|ら|さま}の
\ruby{御鑑識}{お|め|がね}
\ruby{次第}{し|だい}で
\ruby{豆腐屋}{とう|ふ|や}へでも
\ruby{炭團屋}{た|どん|や}へでも
\ruby{何}{なん}でも
\ruby{宜}{よろ}しうございますから
\ruby{身}{み}を
\ruby{固}{かた}めさせて
\ruby{頂}{いたゞ}きたうございます、と
\ruby{斯樣}{か|う}いつて
\ruby{妾}{わたし}が
\ruby{御願}{お|ねが}ひ
\ruby{申}{まを}して
\ruby{居}{ゐ}るのですよ。
もう
\ruby{可}{い}けません、
\ruby{我儘}{わが|まゝ}は
\ruby{云}{い}はせません、
\ruby{何}{なん}でも
\ruby{彼}{かん}でも
\ruby{妾}{わたし}の
\ruby{云}{い}ふ
\ruby{{\換字{通}}}{とほ}りに
\ruby{此方樣}{こち|ら|さま}の
\ruby{御世話}{お|せ|わ}を
\ruby{御願}{お|ねが}ひなさい。
\ruby{{\換字{朝}}}{あさ}は
\ruby{昧}{くら}いから
\ruby{起}{お}きて
\ruby{夜遲}{よる|おそ}くまで、
\ruby{火}{ひ}も
\ruby{焚}{た}き
\ruby{水}{みづ}も
\ruby{汲}{く}み、
\ruby{炊事雜巾掛}{に|たき|ざう|きん|が}け、
\ruby{何}{なに}から
\ruby{何}{なに}まで
\ruby{御奉公人}{ご|ほう|こう|にん}と
\ruby{勵}{はげ}み
\ruby{合}{あ}つて
\ruby{働}{はたら}かなくてはいけません。
\ruby{{\換字{嫌}}}{いや}だなんぞと
\ruby{云}{い}つても
\ruby{既}{もう}
\ruby{承知仕}{しよう|ち|し}ません。
さあ
\ruby{丁度}{ちやう|ど}
\ruby{宜}{い}い、
\ruby{妾}{わたし}と
\ruby{一緖}{いつ|しよ}に、
\ruby{{\換字{判}}然}{はつ|きり}と
\ruby{改}{あらた}めて
\ruby{今後}{こん|ご}の
\ruby{御世話}{お|せ|わ}を
\ruby{御願}{お|ねが}ひ
\ruby{御仕}{お|し}なさい。
\ruby{考}{かんが}へて
\ruby{居}{ゐ}る
\ruby{事}{こと}も
\ruby{何}{なに}も
\ruby{有}{あ}りは
\ruby{仕}{し}ません。
』


\Entry{其二十八}

『
\ruby{然樣}{さ|う}まあ
\ruby{叔母}{を|ば}さんの
\ruby{御言}{お|いひ}のやうにばかりも
お
\ruby{龍}{りう}ちやんにやあなるまいけどもネ、ネエ
お
\ruby{龍}{りう}ちやん、
\ruby{聞}{き}けば
お
\ruby{前}{まへ}も
\ruby{彼}{あ}の
\ruby{御師匠}{お|し|よ}さんていふ
\ruby{人}{ひと}の
\ruby{胸}{むね}の
\ruby{中}{なか}が
\ruby{解}{わか}つて
\ruby{居}{ゐ}ないぢやあ
\ruby{無}{な}いしするのだから、
\ruby{他}{ほか}のいろ〳〵の
\ruby{事}{こと}は
\ruby{後{\換字{廻}}}{あと|まは}しに
\ruby{仕}{し}て
\ruby{置}{お}いて。
\ruby{何樣}{ど|う}だエ、
\ruby{彼家}{あす|こ}を
\ruby{出}{で}ることだけは
\ruby{先}{ま}あ
\ruby{兎}{と}も
\ruby{角}{かく}も
\ruby{出}{で}ると
\ruby{決}{き}めては。
』

もとよりお
\ruby{關}{せき}には
\ruby{密}{ひそか}に
\ruby{愛想}{あい|そ}を
\ruby{盡}{つ}かし
\ruby{居}{を}れるなれば
\ruby{彼家}{かし|こ}に
\ruby{居}{を}りたき
\ruby{事}{こと}は
\ruby{微塵}{み|ぢん}ほども
\ruby{無}{な}きなり、
\ruby{且}{か}つ
お
\ruby{彤}{とう}に
\ruby{如是優}{か|く|やさ}しく
\ruby{云}{い}はれては
\ruby{背}{そむ}かうやうは
\ruby{無}{な}けれど、
\ruby{今彼處}{いま|かし|こ}を
\ruby{去}{さ}りて
\ruby{離}{はな}れんは、
\ruby{春}{はる}の
\ruby{野行}{の|ある}きしたる
\ruby{折}{をり}、
\ruby{圖}{はか}らずも
\ruby{乘}{の}つたる
\ruby{田舎}{ゐな|か}
\ruby{渡}{わた}しの
\ruby{襤褸舟}{ぼ|ろ|ぶね}より
\ruby{振顧}{ふり|かへ}り
\ruby{視}{み}たる
\ruby{岸}{きし}に、
\ruby{落}{お}ち
\ruby{零}{こぼ}れの
\ruby{{\換字{菜}}}{な}の
\ruby{花}{はな}の\換字{志}をらしくも
\ruby{{\換字{咲}}}{さ}きて、
\ruby{歪}{ゆが}める
\ruby{茅屋}{かや|や}の
\ruby{背門}{せ|ど}に
\ruby{桃}{もゝ}の
\ruby{盛}{さか}りなる
\ruby{風{\換字{情}}}{ふ|ぜい}などを
\ruby{見出}{み|いだ}し、とても
\ruby{何時}{い|つ}までも
\ruby{眺}{なが}むべきにはあらずと
\ruby{思}{おも}ひながらも
\ruby{今}{いま}
\ruby{少時}{しば|し}
\ruby{目}{め}にしたきを、
\ruby{野川}{の|がは}の
\ruby{甲斐無}{か|ひ|な}く
\ruby{小}{ちひさ}くて
\ruby{早}{はや}くも
\ruby{着}{つ}きたりとて
\ruby{{\換字{逐}}}{お}ひ
\ruby{上}{あ}げらるゝ
\ruby{時}{とき}、
\ruby{{\換字{猶}}}{なほ}
\ruby{未練}{み|れん}に
\ruby{其}{そ}の
\ruby{船}{ふね}の
\ruby{中}{うち}の
\ruby{戀}{こひ}しき
\ruby{樣}{やう}なる
\ruby{心地}{こゝ|ち}のして、
\ruby{頓}{とみ}には
\ruby{何}{なん}とも
\ruby{答}{こた}へわづらひたり。
されども
\ruby{何處}{ど|こ}から
\ruby{何處}{ど|こ}まで
\ruby{氣}{き}の
\ruby{走}{はし}る
お
\ruby{彤}{とう}に、
\ruby{彼處}{かし|こ}を
\ruby{去}{さ}りてはおのづからに
\ruby{水野}{みづ|の}と
\ruby{緣}{えん}の
\ruby{{\換字{遠}}}{とほ}くなるべきまゝ
\ruby{其}{それ}を
\ruby{厭}{いと}ひて
\ruby{見}{み}す〳〵
\ruby{惡}{わる}い
\ruby{人}{ひと}と
\ruby{知}{し}れる
お
\ruby{關}{せき}が
\ruby{許}{もと}に
\ruby{居}{ゐ}たがるかと
\ruby{思}{おも}はれんほども
\ruby{物憂}{もの|う}くて、

『そりやあ
\ruby{妾}{わたし}だつて
\ruby{彼家}{あす|こ}に
\ruby{居}{ゐ}たいことは
\ruby{有}{あ}りませんが、でも
\ruby{彼家}{あす|こ}を
\ruby{出}{で}てからの
\ruby{妾}{わたし}の
\ruby{行先}{いき|さき}が
\ruby{定}{き}まらなくつちやあ。
』

と
\ruby{僅}{わづか}に
\ruby{語}{ことば}のみを
\ruby{出}{いだ}して
\ruby{煮}{に}え
\ruby{切}{き}れぬ
\ruby{答}{こたへ}をすれば、

『だから
\ruby{此方樣}{こち|ら|さま}に
\ruby{置}{お}いて
\ruby{頂}{いたゞ}くやうに
\ruby{妾}{わたし}が
\ruby{願}{ねが}つて
\ruby{居}{ゐ}るでは
\ruby{無}{な}いか、
\ruby{{\換字{分}}}{わか}らないネエ
お
\ruby{前}{まへ}つて
\ruby{人}{ひと}は。
』

と
\ruby{横合}{よこ|あひ}より
\ruby{叔母}{を|ば}は
\ruby{焦燥}{じ|れ}に
\ruby{焦燥}{じ|れ}ぬ。

『ホヽヽヽ
\ruby{叔母}{を|ば}さん
\ruby{其樣}{そん|な}に
\ruby{御急}{お|せ}きなさらなくつてもの
\ruby{事}{こと}ですよ。
ぢやあお
\ruby{龍}{りう}ちやん、
お
\ruby{前}{まへ}も
\ruby{彼家}{あす|こ}に
\ruby{居}{ゐ}たい
\ruby{事}{こと}は
\ruby{無}{な}
いのだから、
\ruby{彼家}{あす|こ}は
\ruby{出}{で}ることに
\ruby{定}{き}め
\ruby{御置}{お|お}きで、そして
\ruby{其}{そ}の
\ruby{次}{つぎ}に
お
\ruby{前}{まへ}の
\ruby{行}{い}く
\ruby{先}{さき}を
\ruby{腹一杯}{はら|いつ|ぱい}に
\ruby{御考}{お|かんが}へが
\ruby{宜}{い}いぢやあ
\ruby{無}{な}いか。
\ruby{何日}{い|つ}だつたか
\ruby{何}{なに}かの
\ruby{話}{はなし}の
\ruby{序}{つひで}に、
\ruby{妾}{わたし}あ
\ruby{自家}{う|ち}が
\ruby[g]{富裕}{ゆたか}で
お
\ruby{孃樣}{ぢやう|さま}で
\ruby{居}{ゐ}られるやうな
\ruby{身}{み}なら、
\ruby{畫}{ゑ}をかいて
\ruby{一生{\換字{遊}}}{いつ|しやう|あそ}んで
\ruby{居}{ゐ}たいと
\ruby{御云}{お|い}ひの
\ruby{事}{こと}があつたが、
\ruby{今}{いま}でも
\ruby{若}{も}し
\ruby{其樣}{そ|ん}な
\ruby{心持}{こゝろ|もち}を
\ruby{有}{も}つておいでゞ、そして
\ruby{畫}{ゑ}でもつて
\ruby{{\換字{遣}}}{や}つて
\ruby{行}{い}かうといふやうな
\ruby{氣}{き}でも
\ruby{御有}{お|あ}りなら、そりやあ
\ruby{其}{それ}でもつて
\ruby{妾}{わたし}が
\ruby{何樣}{ど|う}でも
\ruby{仕}{し}てあげるが……。
\ruby{{\換字{遠}}慮無}{ゑん|りよ|な}しに
\ruby{何}{なん}でも
\ruby{思}{おも}ふ
\ruby{通}{とほ}りを
\ruby{云}{い}つて
\ruby{御覽}{ご|らん}な。
\ruby{畫}{ゑ}を
\ruby{{\換字{習}}}{なら}はうといふやうな
\ruby{氣}{き}も
\ruby{今}{いま}ぢやあ
\ruby{無}{な}いの?。
\ruby{{\換字{習}}}{なら}やあ
お
\ruby{前}{まへ}は
\ruby{屹度}{きつ|と}
\ruby{出來}{で|き}る
\ruby{人}{ひと}
だよ。
』

『いゝえ、もう
\ruby{其樣}{そ|ん}な
\ruby{事}{こと}は
\ruby{些}{ちつと}も
\ruby{思}{おも}つてや
\ruby{仕}{し}ませんは。
これでも
\ruby{自{\換字{分}}}{じ|ぶん}の
\ruby{天禀}{うま|れつき}が
\ruby{大}{たい}した
\ruby{上手}{じや|うず}になれない
\ruby{位}{ぐらゐ}の
\ruby{事}{こと}も
\ruby{{\換字{分}}}{わか}らないほどの
\ruby[g]{盲目}{めくら}ぢや
\ruby{無}{な}いのですもの!。
』

『ぢやあ
\ruby{鳴物}{なり|もの}は
\ruby{一體}{いつ|たい}
お
\ruby{前}{まへ}の
\ruby{性}{しやう}に
\ruby{合}{あ}つては
\ruby{居}{ゐ}るし、
\ruby{身}{み}に
\ruby{染}{し}みてほんとに
\ruby{好}{すき}ちやあ
\ruby{有}{あ}るし、
\ruby{若}{も}し
\ruby{音樂}{おん|がく}でも
\ruby{學}{や}つて
\ruby{見}{み}やうといふやうな
\ruby{氣}{き}なんぞも
\ruby{無}{な}くつて!。
』

『まあ
\ruby{厭}{いや}ですネエ、
\ruby{人}{ひと}に
\ruby{敎}{をし}へたり
\ruby{人}{ひと}に
\ruby{聞}{き}かれたりするのは
\ruby{妾}{わたし}あ
\ruby{餘}{あま}り
\ruby{好}{すき}ぢやあ
\ruby{無}{な}いんですもの。
』

『ホヽホヽホ。
\ruby{他}{ほか}に
お
\ruby{龍}{りう}ちやんの
\ruby{好}{すき}な
\ruby{事}{こと}は
\ruby{無}{な}いし。
ぢやあ
\ruby{藝事}{げい|ごと}で
\ruby{身}{み}を
\ruby{立}{た}てやうつて
\ruby{氣}{き}も
\ruby{先}{ま}あ
\ruby{無}{な}いのだから、
\ruby[g]{修業沙汰}{しゆげふざた}なんかは
\ruby{一切}{いつ|さい}
\ruby{御}{お}やめなのだネエ。
』

『だつて
\ruby{今}{いま}
\ruby{更}{さら}、
\ruby{何}{なに}か
\ruby{爲}{し}て
\ruby{一人}{ひと|り}で
\ruby{何樣}{ど|う}の
\ruby{彼樣}{か|う}の
\ruby{仕}{し}やうつていふやうなことも
\ruby{思}{おも}つては
\ruby{居}{ゐ}ないんですもの!。
』


\Entry{其二十九}

『でも、それかと
\ruby{云}{い}つて
\ruby{叔母}{を|ば}さんと
\ruby{一緖}{いつ|しよ}に
\ruby{田舍}{ゐな|か}へ
\ruby{引込}{ひつ|こ}んで
\ruby{仕舞}{し|ま}つて、
\ruby{叔母}{を|ば}さんの
\ruby{鑑識}{め|がね}で
\ruby{持}{も}たせて
\ruby{下}{くだ}さる
お
\ruby{婿}{むこ}を
\ruby{持}{も}つて
\ruby{暮}{くら}さうといふ
\ruby{氣}{き}は
\ruby{無}{な}いと
\ruby{再々御云}{さい|〳〵|お|い}ひぢやあ
\ruby{無}{な}いか。
』

『そりやあもう
\ruby{然樣}{さ|う}ですとも!。
\ruby{妾}{わたし}あ
\ruby{何樣}{ど|う}あつても、
\ruby{何}{なん}だか
\ruby{{\換字{分}}}{わか}らないで
\ruby{牛}{うし}か
\ruby{馬}{うま}みたやうに
\ruby{挊}{かせ}いでる
\ruby{田舍}{ゐな|か}の
\ruby{人}{ひと}の、
\ruby{御飯}{ご|はん}を
\ruby{喫}{た}べるために
\ruby{生}{い}きてるつて
\ruby{云}{い}つたやうな
\ruby{其樣}{そ|ん}な
\ruby{{\換字{分}}}{わか}らない
\ruby{人}{ひと}と、
\ruby{一生暮}{いつ|しやう|くら}すなんかつていふ
\ruby{事}{こと}は
\ruby{到底出來}{とて|も|で|き}ないんですから。
』

\ruby{叔母}{を|ば}は
\ruby{堪}{た}へかねて
\ruby{口}{くち}を
\ruby{挿}{はさ}みたり。

『それ、それ、
\ruby{其}{そ}の
\ruby{根性}{こん|ぢやう}が
\ruby{碌}{ろく}で
\ruby{無}{な}い、
\ruby{正當}{まつ|たう}で
\ruby{無}{な}いのだよ。
\ruby[g]{傍目}{わきめ}もふらずにせつせと
\ruby{挊}{かせ}ぎ
\ruby{通}{とほ}すのが
\ruby{上人}{じやう|にん}といふもので、
お
\ruby{前}{まへ}のやうに
\ruby{何}{なん}だの
\ruby{彼}{か}だのと
\ruby{下}{くだ}らない
\ruby{事}{こと}ばかり
\ruby{云}{い}つて
\ruby{居}{ゐ}るのが
\ruby{間{\換字{違}}}{ま|ちが}ひきつて
\ruby{居}{ゐ}るのだ。
\ruby[g]{皆誰}{みんなだれ}だつて
\ruby{御飯}{ご|はん}を
\ruby{喫}{た}べるために
\ruby{挊}{かせ}ぐのぢやあ
\ruby{無}{な}いか。
\ruby{喫}{た}べる
\ruby{爲}{ため}に
\ruby{挊}{かせ}が
\ruby{無}{な}くつて
\ruby{何樣}{ど|う}なるものかネ、
\ruby{下}{くだ}らない。
』

『だつて
\ruby{其樣}{そ|ん}なに
\ruby{大騷}{おほ|さわ}ぎを
\ruby{{\換字{遣}}}{や}      つて
\ruby{御膳}{ご|ぜん}を
\ruby{食}{た}べりやあ
\ruby{其}{それ}でもつて
\ruby{何}{なに}が
\ruby{嬉}{うれ}しいの?。
』

『そんな
\ruby{馬鹿}{ば|か}な
\ruby{氣樂}{き|らく}なことを
\ruby{云}{い}つて
\ruby{居}{ゐ}るから
\ruby{皆}{みんな}
お
\ruby{前}{まへ}の
\ruby{考}{かんがへ}は
\ruby{間{\換字{違}}}{ま|ちが}つて
\ruby{居}{ゐ}るのだよ。
\ruby{人間}{ひ|と}つてものは
\ruby{三度三度}{さん|ど|さん|ど}
\ruby{御膳}{ご|ぜん}さへ
\ruby{滿足}{まん|ぞく}にいたゞいて
\ruby{行}{ゆ}かれりやあ
\ruby{其}{それ}で
\ruby{結構}{けつ|こう}なので、
\ruby{嬉}{うれ}しいも
\ruby{嬉}{うれ}しくないも
\ruby{要}{い}つた
\ruby{事}{こと}あ
\ruby{有}{あ}りや
\ruby{仕無}{し|な}い。
お
\ruby{前}{まへ}なんざあ
\ruby{甚}{ひど}い
\ruby{苦勞}{く|らう}といふものを
\ruby{仕}{し}た
\ruby{事}{こと}が
\ruby{無}{な}いものだから、
\ruby{其樣}{そ|ん}な
\ruby{下}{くだ}らない
\ruby{事}{こと}ばかり
\ruby{云}{い}つて
\ruby{居}{ゐ}るんだよ。
』

『
\ruby{御膳}{ご|ぜん}を
\ruby{食}{た}べるばかりに
\ruby{齷齪}{あく|せく}して
\ruby{死}{し}んで
\ruby{仕舞}{し|ま}ふのだつて、
\ruby[g]{何程下}{いくらくだ}らないか
\ruby{知}{し}れや
\ruby{仕無}{し|な}いは。
』

『ホヽヽ、お
\ruby{龍}{りう}ちやん
お
\ruby{前}{まへ}が
\ruby{惡}{わる}いよ、
\ruby{目上}{め|うへ}に
\ruby{{\換字{逆}}}{さか}らつて!。
\ruby{第一}{だい|いち}
\ruby{談話}{はな|し}に
\ruby{枝}{えだ}が
\ruby{{\換字{咲}}}{さ}いて
\ruby{仕舞}{し|ま}ふはネ。
ぢやあお
\ruby{前}{まへ}は
\ruby{稽古事}{けい|こ|ごと}は
\ruby{爲}{す}る
\ruby{氣}{き}は
\ruby{無}{な}し、
\ruby{靜岡}{しづ|をか}へは
\ruby{行}{ゆ}くまいと
\ruby{云}{い}ふし、
\ruby{何樣仕}{ど|う|し}やうと
\ruby{御云}{お|い}ひなの?。
\ruby{妾}{わたし}の
\ruby{處}{ところ}へ
\ruby{來}{き}て
\ruby{妾}{わたし}の
\ruby{{\換字{遊}}}{あそ}び
\ruby{相手}{あひ|て}になつて
お
\ruby{{\換字{呉}}}{く}れの
\ruby{積}{つも}りなの?。
』

『…………、』

『いエもう
\ruby{{\換字{遊}}}{あそ}び
\ruby{相手}{あひ|て}なんぞと
\ruby{仰}{おつし}あやると
\ruby{直}{すぐ}に
\ruby[g]{増長致}{ぞうちやういた}します、
\ruby{矢張}{やつ|ぱ}り
\ruby[g]{引{\換字{遣}}}{ひつつか}つて
\ruby{{\換字{遣}}}{や}ると
\ruby{仰}{おつし}あつて
\ruby{下}{くだ}さいまし。
』

『お
\ruby{龍}{りう}ちやんが
\ruby{默}{だま}つて
\ruby{居}{ゐ}ちやあ
\ruby{仕樣}{し|やう}が
\ruby{無}{な}いぢやあ
\ruby{無}{な}いか。
\ruby{默}{だま}つてるところを
\ruby{見}{み}ると
\ruby{吾家}{う|ち}へ
\ruby{來}{く}るのも
\ruby{厭}{いや}なの?。
』

『
\ruby{厭}{いや}つて
\ruby{事}{こと}は
\ruby{毫末}{ちつ|と}も
\ruby{有}{あ}りやあ
\ruby{仕}{し}ませんけれども……』

『ぢやあ
\ruby{何}{なに}も
\ruby{其樣}{そ|ん}なに
\ruby{考}{かんが}へてゐる
\ruby{事}{こと}は
\ruby{有}{あ}りさうも
\ruby{無}{な}いものぢや
\ruby{無}{な}いか。
』

『でも
\ruby{姊}{ねえ}さんのところへ
\ruby{來}{き}て
\ruby{居}{ゐ}ると……』

『
\ruby{何}{なに}か
\ruby{厭}{いや}な
\ruby{事}{こと}があつて?。
』

『いえ、
\ruby{然樣}{さ|う}なのぢや
\ruby{有}{あ}りませんけども
\ruby{餘}{あんま}
り
\ruby{叮嚀}{てい|ねい}に
\ruby{仕}{し}て
\ruby{下}{くだ}さるんで、\------ まるで
\ruby{眞實}{ほん|と}の
\ruby{妹}{いもうと}かなんぞのやうに、
\ruby{御孃樣}{お|ぢやう|さま}あつかひに
\ruby{仕}{し}てくださるので、
\ruby{何}{なん}だか
\ruby[g]{居辛}{ゐづら}くつて
\ruby{仕方}{し|かた}が
\ruby{無}{な}いんですもの。
\ruby{此}{こ}の
\ruby{春}{はる}だつて
\ruby{然樣}{さ|う}なのですよ。
\ruby{彼}{あ}の
\ruby{時}{とき}は
\ruby{彼樣}{あ|あ}した
\ruby{譯}{わけ}で
\ruby{二度}{に|ど}と
\ruby{姊}{ねえ}さんにやあ
\ruby{御目}{お|め}に
\ruby{掛}{かゝ}らないつもりで
\ruby{出}{で}たんですけれとも、
\ruby{後}{あと}になつても
\ruby{一}{ひと}つは
\ruby{其}{そ}の
\ruby{爲}{ため}に
\ruby{此方}{こち|ら}へは
\ruby{歸}{かへ}つて
\ruby{來}{こ}なかつたので。
\ruby{彼}{あ}の
お
\ruby{師匠}{し|よ}さんのところに
\ruby{居}{ゐ}ることに
\ruby{仕}{し}ましたのも、いろ〳〵の
\ruby{事}{こと}を
\ruby{云}{い}つて
\ruby{引{\換字{留}}}{ひき|と}められるからばかりぢやあ
\ruby{有}{あ}りませんので。
\ruby{彼家}{あす|こ}に
\ruby{居}{ゐ}りやあ
\ruby{居}{ゐ}るだけの
\ruby{事}{こと}を
\ruby{爲}{し}て
\ruby{報復}{か|へ}しますけれども、
\ruby{姊}{ねえ}さんの
\ruby{處}{ところ}に
\ruby{居}{ゐ}ますと、
\ruby{何一}{なに|ひと}つ
\ruby{用事}{よう|じ}を
\ruby{爲}{す}るのぢやあ
\ruby{無}{な}し、
\ruby{着物}{き|もの}も
\ruby{美麗}{き|れい}に
\ruby{仕}{し}て
\ruby{下}{くだ}さりやあ
\ruby{髮}{かみ}から
\ruby{穿物}{はき|もの}まで
\ruby{氣}{き}をつけて
\ruby{下}{くだ}さる、それで
\ruby{三度}{さん|ど}が
\ruby{三度}{さん|ど}とも
\ruby{据膳}{すゑ|ぜん}に
\ruby{對}{むか}つて、
\ruby{姊}{ねえ}さん
\ruby{同樣}{どう|やう}に
\ruby{御給仕}{お|きふ|じ}をされて
\ruby{御膳}{ご|ぜん}を
\ruby{頂}{いたゞ}くのは、
\ruby{妾}{わたし}にやあ
\ruby{何}{なん}だか
\ruby[g]{結構{\換字{過}}}{けつこうす}ぎて
\ruby{濟}{す}まないやうな
\ruby{氣}{き}がするのですもの!。
\ruby{小間使}{こ|ま|づかひ}や
\ruby{何}{なん}かと
\ruby{一緖}{いつ|しよ}になつて
\ruby{何}{なに}か
\ruby{用}{よう}を
\ruby{仕}{し}やうとすりやあ、
お
\ruby{止}{よ}し、
お
\ruby{止}{よ}し、
\ruby{不見識}{ふ|けん|しき}だよ、つて
\ruby{姊}{ねえ}さんが
\ruby{御止}{お|と}めなさるのですら、あれだけ
\ruby{御厄介}{ご|やく|かい}になつて
\ruby{居}{ゐ}た
\ruby{中}{うち}に
\ruby{姊}{ねえ}さんの
\ruby{爲}{ため}に
\ruby{何}{なに}か
\ruby{仕}{し}たと
\ruby{云}{い}つたら、たつた
\ruby{一遍相思鳥}{いつ|ぺん|さう|し|てう}の
\ruby{餌}{ゑ}を
\ruby{摺}{す}つたことが
\ruby{有}{あ}るつ
\ruby{限}{き}りなのですもの。
\ruby[g]{何程兒童}{いくらこども}の
\ruby{時}{とき}から
\ruby{一緖}{いつ|しよ}に
\ruby{寢}{ね}たりなんか
\ruby{仕}{し}て、
\ruby{{\換字{姉}}妹}{きやう|だい}よりも
\ruby{仲好}{なか|よ}く
\ruby{暮}{くら}して
\ruby{來}{き}たからつて、
\ruby{妾}{わたし}あ
\ruby{姊}{ねえ}さんにやあ
\ruby{緣}{えん}も
\ruby[g]{由緣}{ゆかり}も
\ruby{何}{なんに}も
\ruby{無}{な}い
\ruby{身}{み}だし、そりやあ
\ruby{今}{いま}が
\ruby{今}{いま}でも
\ruby{姊}{ねえ}さんの
\ruby{爲}{ため}になら
\ruby{火水}{ひ|みづ}の
\ruby{中}{なか}へなり
\ruby{入}{はい}らうつていふ
\ruby{氣}{き}だけは
\ruby{有}{も}つて
\ruby{居}{ゐ}ますけれども、
\ruby{今日}{け|ふ}までのところぢやあ
\ruby{何一}{なに|ひと}つ
\ruby{姊}{ねえ}さんの
\ruby{爲}{ため}に
\ruby{仕}{し}た
\ruby{事}{こと}でも
\ruby{有}{あ}るぢやあ
\ruby{無}{な}し、たゞ
\ruby{甘}{あま}つたれて
\ruby{可愛}{か|はい}がつて
\ruby{貰}{もら}つて
\ruby{居}{ゐ}たと
\ruby{云}{い}ふだけの
\ruby{事}{こと}なんですから、そんなに
\ruby{好}{よ}くされるやう
\ruby{譯}{わけ}は
\ruby{有}{あ}る
\ruby{筈}{はず}が
\ruby{無}{な}いので、
\ruby{何樣}{ど|う}も
\ruby{妾}{わたし}やあ
\ruby{氣}{き}が
\ruby{狹小}{け|ち}なんでしやうけれども
\ruby{氣}{き}が
\ruby{咎}{とが}めてならないのです。
ですから、いつそ
\ruby{叔母}{を|ば}の
\ruby{言葉}{こと|ば}の
\ruby{通}{とほ}りに
\ruby{扱}{こ}き
\ruby{使}{つか}つて
\ruby{下}{くだ}さるならば、
\ruby{願}{ねが}つても
\ruby{姊}{ねえ}さんの
\ruby{傍}{そば}へ
\ruby{置}{お}いて
\ruby{頂}{いたゞ}きたいのですけれど、
\ruby{何樣}{ど|う}も
\ruby{姊}{ねえ}さんは
\ruby{姊}{ねえ}さんの
\ruby{氣象}{き|しやう}でもつて
\ruby{然樣}{さ|う}は
\ruby{仕}{し}て
\ruby{下}{くだ}さるまいと
\ruby{思}{おも}ふと、
\ruby{何}{なに}も
\ruby{仕}{し}も
\ruby{仕無}{し|な}いものを
\ruby{餘}{あま}り
\ruby{好}{よ}くして
\ruby{下}{くだ}さるのが、
\ruby{妾}{わたし}にやあ
\ruby{心苦}{こゝろ|ぐる}しくつて
\ruby{居}{ゐ}られないのですから。
』

『オヤ、オヤ、お
\ruby{龍}{りう}ちやんは
\ruby{大層}{たい|そう}
\ruby[g]{他人兒}{たにんこ}におなりネエ。
わかつたよお
\ruby{前}{まへ}の
\ruby{優}{やさ}しい
\ruby{奇麗}{き|れい}な
\ruby{心持}{こゝろ|もち}は
\ruby{善}{よ}く
\ruby{解}{わか}つたよ。
\ruby{何}{なに}かと
\ruby{思}{おも}つたら、ホヽホヽホヽ
\ruby{其樣}{そ|ん}な
\ruby{事}{こと}だつたの!。
つい
\ruby{{\換字{過}}般}{こな|ひだ}までの
お
\ruby{龍}{りう}ちやんは
\ruby{此樣}{こ|ん}な
\ruby{人}{ひと}ぢやあ
\ruby{無}{な}くつて、
\ruby[g]{花簪}{はなかんざし}の
\ruby{大}{おほき}いのを
お
\ruby{{\換字{悅}}}{よろこ}びだつた
\ruby{頃}{ころ}といふものは
\ruby{何}{なに}を
\ruby{買}{か}つて
\ruby{{\換字{呉}}}{く}れ、
\ruby{彼}{か}を
\ruby{買}{か}つて
\ruby{{\換字{呉}}}{く}れつて
\ruby{妾}{わたし}をせびつちやあ、
\ruby{稀}{たま}に
\ruby{買}{か}つて
\ruby{上}{あ}げ
\ruby{無}{な}からうものならプーツと
お
\ruby{膨}{ふく}れでネ、
\ruby{夜}{よる}になつて
\ruby{一緖}{いつ|しよ}に
\ruby{寢}{ね}ても
\ruby{彼方}{むか|う}を
\ruby{向}{む}いて
\ruby{口一}{くち|ひと}つきかないで、そして
\ruby{足}{あし}でもつてぼん〳〵と
\ruby{妾}{わたし}を
お
\ruby{蹴}{け}だつたぢやあ
\ruby{無}{な}いか。
』

『あら
\ruby{厭}{いや}な
\ruby{姊}{ねえ}さんだこと!。
\ruby{兒童}{こ|ども}の
\ruby{時}{とき}の
\ruby{事}{こと}なんか
\ruby{御云}{お|い}ひ
\ruby{出}{だ}しなすつちやあ。
』

『ホヽヽ、そのお
\ruby{龍}{りう}ちやんがまあ
\ruby{大層}{たい|そう}にませて、ほんとに
\ruby[g]{{\換字{遠}}慮深}{ゑんりよぶか}く
お
\ruby{成}{な}りのネ!。
いゝよ、
\ruby{其}{それ}なら
\ruby{其}{それ}で
\ruby{其}{そ}の
\ruby{樣}{やう}に
\ruby{爲}{す}るから。
ぢやあ
\ruby{吾家}{う|ち}に
\ruby{居}{ゐ}ることに
お
\ruby{定}{き}めが
\ruby{好}{い}いぢやあ
\ruby{無}{な}いか。
』

お
\ruby{龍}{りう}は
\ruby{辭}{じ}せんとして
\ruby{今}{いま}は
\ruby{辭}{じ}する
\ruby{能}{あた}はざる
\ruby{境}{さかひ}に
\ruby{臨}{のぞ}みぬ。
お
\ruby{關}{せき}の
\ruby{許}{もと}を
\ruby{離}{はな}れて
お
\ruby{彤}{とう}の
\ruby{世話}{せ|わ}になる
\ruby{事}{こと}の
\ruby{{\換字{嫌}}}{いや}なるにはあらねど、
\ruby{何故}{なに|ゆゑ}にや
\ruby{前}{さき}の
\ruby{日}{ひ}と
\ruby{今日}{け|ふ}とは
お
\ruby{彤}{とう}の
\ruby{語氣}{くち|ぶり}の
\ruby{異}{ちが}ひて、
\ruby{彼}{か}の
\ruby{水野}{みづ|の}をば
\ruby{{\換字{悅}}}{よろこ}ばぬ
\ruby{氣}{き}なるが
\ruby{何}{なん}と
\ruby{無}{な}く
\ruby{心}{こゝろ}にかゝりて、
\ruby{此}{こ}の
\ruby{人}{ひと}の
\ruby{許}{もと}に
\ruby{明日}{あ|す}よりの
\ruby{我}{わ}が
\ruby{身}{み}を
\ruby{寄}{よ}せんことの
\ruby{何}{なに}かは
\ruby{知}{し}らねど
\ruby{窮屈}{きう|くつ}らしき
\ruby{心地}{こゝ|ち}して、
\ruby{嬉}{うれ}しかるべき
\ruby{筈}{はず}の
\ruby{事}{こと}ながら
\ruby{然}{さ}のみは
\ruby{嬉}{うれ}しからぬなり。


\Entry{其三十}

\ruby{色}{いろ}ある
\ruby{蓋}{かさ}のいと
\ruby{艶}{えん}に
\ruby{美}{うつく}しき
\ruby{電燈}{でん|とう}の
\ruby{下}{もと}、
\ruby{上座}{じや|うざ}に
お
\ruby{彤}{とう}、やゝ
\ruby{隔}{へだ}たり
\ruby{下}{くだ}つて
お
\ruby{龍}{りう}の
\ruby{叔母}{を|ば}、それよりまた
\ruby{下}{さが}つて
\ruby{坐}{すわ}れる
お
\ruby{龍}{りう}の
\ruby{三人}{さん|にん}は
\ruby{今}{いま}しも
\ruby{夜食}{や|しよく}の
\ruby{膳}{ぜん}の
\ruby{既}{すで}に
\ruby{引}{ひ}き
\ruby{去}{さ}られたる
\ruby{後}{あと}を、
\ruby{心靜}{こゝろ|しづ}かに
\ruby{茶}{ちや}に
\ruby{物語}{もの|がた}るなり。

\ruby{二人三樣}{に|にん|さん|やう}の
\ruby{心}{こゝろ}の
\ruby{思}{おもひ}あれば
\ruby{面}{おもて}の
\ruby{色}{いろ}あり。
お
\ruby{龍}{りう}はおのが
\ruby{頼}{たの}まんと
\ruby{思}{おも}ひて
\ruby{來}{き}しことは
\ruby{自然}{おの|づ}と
\ruby{{\換字{半}}{\換字{分}}}{なか|ば}は
\ruby{餘{\換字{所}}}{よ|そ}にされて、
\ruby{思}{おも}ひもかけざりし
\ruby{我}{わ}が
\ruby{身}{み}の
\ruby{上}{うへ}の
\ruby{彼家}{かし|こ}を
\ruby{出}{い}でて
\ruby{此家}{こ|ゝ}に
\ruby{居}{を}るべきやう
\ruby{定}{さだ}められたるに、
\ruby{可厭}{い|や}といふでは
\ruby{無}{な}けれど
\ruby{何}{なん}となく
\ruby{勇}{いさ}まぬ
\ruby{心地}{こゝ|ち}のするか、
\ruby{常}{つね}とは
\ruby{{\換字{違}}}{ちが}ひて
\ruby{沈}{しづ}めるやうなり。
お
\ruby{龍}{りう}が
\ruby{叔母}{を|ば}は、
\ruby{全}{まつた}く
\ruby{我}{わ}が
\ruby{思}{おも}ふ
\ruby{如}{ごと}くになりたりと
\ruby{云}{い}ふにはあらねど、
\ruby{兎}{と}に
\ruby{角}{かく}
お
\ruby{龍}{りう}を
\ruby{我}{わ}が
\ruby{{\換字{嫌}}}{きら}ふ
お
\ruby{關}{せき}が
\ruby{許}{もと}より
\ruby{移}{うつ}し
\ruby{奪}{うば}ひて、
\ruby{豫}{かね}て
お
\ruby{龍}{りう}より
\ruby{聞}{き}きしに
\ruby{{\換字{違}}}{たが}はず
\ruby{富}{と}みて
\ruby{美}{うつく}しく
\ruby{智慧深}{ち|ゑ|ふか}き
\ruby{此家}{この|や}の
\ruby{主人}{ある|じ}が
\ruby{許}{もと}に
\ruby{預}{あづ}かり
\ruby{貰}{もら}ふ
\ruby{事}{こと}となりたるに、
\ruby{心安堵}{こゝろ|おち|つ}きて
\ruby{莞爾}{に|こ}つき
\ruby{{\換字{勝}}}{がち}なれば、
\ruby{根}{ね}は
\ruby{善}{よ}き
\ruby{人}{ひと}の
\ruby{徴}{しるし}とて
\ruby{顏}{かほ}に
\ruby{曇}{くも}りなく、
\ruby{例}{れい}の
\ruby{小}{ちひさ}なる
\ruby{三角}{さん|かく}の
\ruby{眼}{め}さへ、
\ruby{其}{そ}の
\ruby{眼尻}{まな|じり}に
\ruby{寄}{よ}る
\ruby{小皺}{こ|じわ}に
\ruby{却}{かへ}つて
\ruby{可愛}{か|はい}らしく
\ruby{見}{み}ゆ。
たゞお
\ruby{彤}{とう}のみは
\ruby{心}{こゝろ}の
\ruby{動}{うご}くこと
\ruby{無}{な}くてや、
\ruby{能}{よ}く
\ruby{笑}{わら}ひ
\ruby{能}{よ}く
\ruby{語}{かた}れども
\ruby{{\換字{悅}}}{よろこ}べるともなく
\ruby{樂}{たのし}まぬとも
\ruby{無}{な}く
\ruby{{\換字{平}}然}{へい|ぜん}として、
\ruby{今}{いま}
\ruby{{\換字{猶}}前刻}{なほ|さ|き}の
\ruby{如}{ごと}く
\ruby{澄}{す}まし
\ruby{{\換字{返}}}{かへ}つたり。

お
\ruby{龍}{りう}は
\ruby{何}{なに}をか
\ruby{思}{おも}へる、
\ruby{沈默}{おし|だま}りて
\ruby{頭}{かうべ}を
\ruby{垂}{た}れつ、
\ruby{頻}{しきり}に
\ruby{譯}{わけ}も
\ruby{無}{な}く
\ruby{自己}{お|の}が
\ruby{衣服}{き|もの}の
\ruby{袖膝}{そで|ひざ}なんどに
\ruby{吸}{す}ひ
\ruby{出}{だ}されたる
\ruby{綿}{わた}を
\ruby{摘}{つ}みては
\ruby{除}{と}り
\ruby{摘}{つ}みては
\ruby{除}{と}りながら、
\ruby{人}{ひと}
の
\ruby{話}{はなし}をのみ
\ruby{聞}{き}きて
\ruby{居}{を}れば、
\ruby{叔母}{を|ば}は
お
\ruby{龍}{りう}が
\ruby{樣子}{やう|す}などには
\ruby{眼}{め}も
\ruby{{\換字{遣}}}{や}らずして、

『どうも
\ruby{誠}{まこと}に
\ruby{種々有}{いろ|〳〵|あ}り
\ruby{難}{がた}うございます、
お
\ruby{蔭樣}{かげ|さま}で
\ruby{私}{わたくし}も
\ruby{安心}{あん|しん}いたしました。
では
\ruby{私}{わたくし}は
\ruby{直接}{ぢ|か}には
お
\ruby{關}{せき}に
\ruby{會}{あ}ひませず、
\ruby{此儘}{この|まゝ}で
\ruby{國}{くに}へ
\ruby{歸}{かへ}りまして、
\ruby{憚}{はばか}りさまでございますが
お
\ruby{關}{せき}の
\ruby{方}{はう}の
\ruby{事}{こと}は、
\ruby{一切}{いつ|さい}
\ruby{此方樣}{こち|ら|さま}
\ruby{次第}{し|だい}に
\ruby{願}{ねが}ひます。
\ruby{若}{もし}
\ruby{{\換字{又}}}{また}
\ruby{全然}{まる|で}
\ruby{握}{にぎ}り
\ruby{拳}{こぶし}でも
\ruby{濟}{す}みませぬやうの
\ruby{事}{こと}でございましたならば、
\ruby{惡}{わる}い
\ruby{奴}{やつ}に
\ruby{關}{かゝ}りあつたのが
\ruby{不祥}{ふ|しやう}でございますから、
\ruby{三十四十}{さん|じう|し|じう}の
\ruby{金}{かね}を
\ruby{出}{だ}し
\ruby{惜}{をし}みは
\ruby{致}{いた}しません、
\ruby{御話}{お|はなし}さへございますれば
\ruby{直}{すぐ}にも
\ruby{差出}{さし|だ}します。
\ruby{何}{なに}も
\ruby{彼}{か}も
\ruby{此女}{こ|れ}の
\ruby{爲宜}{ため|よ}かれと
\ruby{思}{おも}ふからの
\ruby{事}{こと}てございますから
\ruby{忍耐}{が|まん}も
\ruby{致}{いた}します。
\ruby{全}{まつた}く
\ruby{彼樣}{あ|ん}な
\ruby{奴}{やつ}に
\ruby{鐚錢一}{び|た|ひと}つ
\ruby{{\換字{呉}}}{く}れて
\ruby{{\換字{遣}}}{や}ります
\ruby{因緣}{いん|ねん}は
\ruby{無}{な}いと
\ruby{思}{おも}ひますけれど
\ruby{些少}{すこ|し}ばかりの
\ruby{事}{こと}で
\ruby{煩}{うるさ}い
\ruby{關係}{ひつ|かゝり}を
\ruby{殘}{のこ}すのも
\ruby{可厭}{い|や}ですし、
\ruby{此女}{こ|れ}と
\ruby{彼}{あ}の
\ruby{婆}{ばゞあ}と
\ruby{往來}{わう|らい}で
\ruby{逢}{あ}ひました
\ruby{時}{とき}、
\ruby{此女}{こ|れ}に
\ruby{氣}{き}の
\ruby{怯}{ひ}けるやうな
\ruby{思}{おも}ひをさせるのも
\ruby{可厭}{い|や}でございますから、
\ruby{其}{そ}の
\ruby{位}{くらゐ}の
\ruby{事}{こと}なら
\ruby{出}{だ}しも
\ruby{致}{いた}しましやうと
\ruby{思}{おも}つて
\ruby{居}{を}りますのです。
\ruby{其邊}{そこ|いら}は
\ruby{御含}{お|ふく}み
\ruby{下}{くだ}さいまして、
\ruby{何樣}{ど|う}でも
\ruby{宜}{よろ}しいやうに
\ruby{御計}{お|はか}らひを
\ruby{願}{ねが}ひまする。
\ruby{此女}{こ|れ}の
\ruby{上}{うへ}は
\ruby{改}{あらた}めて
\ruby{今日私}{け|ふ|わたくし}から
\ruby{御縋}{お|すが}り
\ruby{申}{まを}して
\ruby{御願申}{おね|がひ|まを}しまする。
\ruby{至}{いた}つて
\ruby{我儘}{わが|まゝ}な
\ruby{無{\換字{分}}別者}{む|ふん|べつ|もの}ではございまするが、
\ruby{心}{しん}から
\ruby{底}{そこ}から
\ruby{惡}{わる}い
\ruby{奴}{やつ}といふのでも
\ruby{無}{な}いやうでございますから、
\ruby{何樣}{ど|う}か
\ruby{十{\換字{分}}}{じう|ぶん}に
\ruby{御斟酌}{ご|しん|しやく}なく
\ruby{御使}{お|つか}ひなすつて、そして
\ruby{其中相應}{その|うち|さう|おう}なものでもございました
\ruby{時}{とき}に、
\ruby{御鑑識}{お|め|がね}で
\ruby{夫}{をとこ}でも
\ruby{持}{も}たせて
\ruby{{\換字{遣}}}{や}つて
\ruby{下}{くだ}されば
\ruby{其上}{その|うへ}はございません。
\ruby{私}{わたくし}は
\ruby{斯樣}{こ|ん}ながさつ
\ruby{者}{もの}でございましても、
\ruby{姪}{めひ}
\ruby{一人}{ひと|り}
\ruby{叔母}{を|ば}
\ruby{一人}{ひと|り}でございますから
\ruby{此}{これ}を
\ruby{棄}{す}てる
\ruby{氣}{き}はございません。
\ruby{何處}{ど|こ}までも
\ruby{好}{よ}くして
\ruby{{\換字{遣}}}{や}りたいのは
\ruby{山々}{やま|〳〵}でございますが、とても
\ruby{私}{わたくし}には
\ruby{制{\換字{道}}}{せい|だう}の
\ruby{付}{つ}きかねる
\ruby{氣}{き}まぐれ
\ruby{者}{もの}めでございますので、
\ruby{此方樣}{こち|ら|さま}へ
\ruby{願}{ねが}ふよりほかには
\ruby{願}{ねが}はうところも
\ruby{無}{な}いやうな
\ruby{譯}{わけ}でございますゆゑ、
\ruby{御{\換字{迷}}惑}{ご|めい|わく}でもございましやうが
\ruby{何樣}{ど|う}か
\ruby{御世話}{お|せ|わ}をなすつて
\ruby{下}{くだ}さいますやうに、
\ruby{汚}{きたな}い
\ruby{婆}{ばゞあ}でございますが
\ruby{是}{これ}でも
\ruby{人樣}{ひと|さま}の
\ruby{御恩}{ご|おん}を
\ruby{忘}{わす}れるやうな
\ruby{獸畜}{けだ|もの}でもございません
\ruby{田舎}{ゐな|か}
\ruby{者}{もの}が、
\ruby{折入}{をり|い}つて
\ruby{此}{こ}の
\ruby{通}{とほ}りに
お
\ruby{願}{ねが}ひ
\ruby{申}{まを}します。
』

と、
\ruby{云}{い}ひさま
\ruby{頭}{かしら}を
\ruby{下}{さ}げて
\ruby{染々}{しみ|〴〵}と
\ruby{眞心}{ま|ごゝろ}せめて
\ruby{頼}{たの}み
\ruby{聞}{きこ}えつ、

『
\ruby{歸}{かへ}りましたら
\ruby{早{\換字{速}}}{さつ|そく}
\ruby{衣類}{い|るい}も
\ruby{{\換字{送}}}{おく}りましやうし、
\ruby{{\換字{又}}}{また}、
\ruby{當人}{たう|にん}の
\ruby{小{\換字{遣}}}{こづ|かひ}なんぞは
\ruby{御厄介}{ご|やく|かい}にならないやうに
\ruby{致}{いた}しましやう。
\ruby{萬々一當人}{まん|〳〵|いち|たう|にん}が
\ruby{不都合}{ふ|つ|がう}な
\ruby{事}{こと}でも
\ruby{仕出}{し|だ}しましたらば、
\ruby{決}{けつ}して
\ruby{御{\換字{迷}}惑}{ご|めい|わく}は
\ruby{掛}{か}けませぬやうに、
\ruby{屹度}{きつ|と}
\ruby{私}{わたくし}が
\ruby{引請}{ひき|うけ}まするから、
\ruby{何卒}{どう|ぞ}
\ruby{御奉公人}{ご|ほう|こう|にん}
\ruby{同樣}{どう|やう}に
\ruby{御扱}{お|あつか}ひなすつて、
\ruby{末々}{すゑ|〴〵}を
\ruby{宜}{よろ}しく
\ruby{御願}{お|ねが}ひ
\ruby{申}{まを}しまする。
ほんとに
\ruby{少}{ちひさ}い
\ruby{時}{とき}から
\ruby{御馴染}{お|な|じみ}
\ruby{申}{まを}したのが
\ruby{當人}{たう|にん}の
\ruby{幸福}{しあ|わせ}とは
\ruby{申}{まを}しながら、
\ruby{是}{これ}といふ
\ruby{譯}{わけ}も
\ruby{無}{な}いのに
\ruby{斯樣}{こ|ん}な
\ruby{我儘者}{わが|まゝ|もの}を
\ruby{御願}{お|ねが}ひ
\ruby{申}{まを}しまして、そして
\ruby{快}{こゝろ}よく
\ruby{御引受}{お|ひき|う}けくだすつて
\ruby{頂}{いたゞ}くといふのも、
\ruby{思}{おも}へば
\ruby{餘}{あま}り
\ruby{有}{あ}り
\ruby{難{\換字{過}}}{がた|す}ぎまして、
\ruby{何}{なん}だか
\ruby{不思議}{ふ|し|ぎ}なやうな
\ruby{氣}{き}が
\ruby{致}{いた}します
\ruby{位}{くらゐ}でございます。
』

と
\ruby{眞顏}{ま|がほ}になつて
\ruby{恩}{おん}を
\ruby{謝}{しや}するを、
お
\ruby{彤}{とう}は
\ruby{婿然}{にこ|り}と
\ruby{打笑}{うち|わら}つて、

『なあに、
\ruby{其樣}{そ|ん}なに
\ruby{恩}{おん}に
\ruby{被}{き}て
\ruby{下}{くだ}さる
\ruby{事}{こと}は
\ruby{有}{あ}りやあ
\ruby{仕}{し}ません、
\ruby{人}{ひと}は
\ruby{各自}{めい|〳〵}の
\ruby{氣性}{きし|やう}で
\ruby{種々}{いろ|ん}な
\ruby{事}{こと}を
\ruby{爲}{す}るのですもの!。

\ruby{好}{す}いた
\ruby{盆栽}{うゑ|き}の
\ruby{世話}{せ|わ}を
\ruby{仕}{し}たからつて、
\ruby{盆栽}{うゑ|き}に
\ruby{御禮}{お|れい}を
\ruby{云}{い}はれやうつて
\ruby{思}{おも}ふ
\ruby{人}{ひと}は
\ruby{一人}{ひと|り}も
\ruby{有}{あ}りやあ
\ruby{仕}{し}ません、たゞ
\ruby{其}{そ}の
\ruby{樹}{き}が
\ruby{好}{よ}くさへなりやあ
\ruby{其}{それ}が
\ruby{嬉}{うれ}しいので。
\ruby{不思議}{ふ|し|ぎ}な
\ruby{事}{こと}も
\ruby{何}{なに}も
\ruby{有}{あ}りやあ
\ruby{仕}{し}ませんは、
\ruby{妾}{わたし}あ
\ruby{一體}{いつ|たい}
お
\ruby{龍}{りう}ちやんが
\ruby{好}{す}きなんですもの!。
たゞお
\ruby{龍}{りう}ちやんが
\ruby{好}{よ}くさへなつて
お
\ruby{{\換字{呉}}}{く}れならそれで
\ruby{本望}{ほん|まう}なので、
\ruby{何樣}{ど|ん}なにか
\ruby{嬉}{うれ}しく
\ruby{思}{おも}ふか
\ruby{知}{し}れや
\ruby{仕}{し}ません。
』

と
\ruby{輕}{かろ}く
\ruby{答}{こた}ふれば、
\ruby{何不足無}{なに|ふ|そく|な}き
\ruby{人}{ひと}の
\ruby{氣}{き}の
\ruby{持}{も}ち
\ruby{方}{かた}はまた
\ruby{{\換字{違}}}{ちが}ふもの、
\ruby{世}{よ}には
\ruby{此}{こ}の
\ruby{樣}{やう}な
\ruby{人}{ひと}も
\ruby{有}{あ}ることか、と
\ruby{田舎}{ゐな|か}
\ruby{者}{もの}の
\ruby{我}{わ}が
\ruby{心}{こゝろ}の
\ruby{狹}{せま}く
\ruby{堅}{かた}くろしきに
\ruby{比}{くら}べてつく〴〵
\ruby{感}{かん}じ
\ruby{入}{い}る
\ruby{時}{とき}、

『あの、お
\ruby{富}{とみ}の
\ruby{親父}{おや|ぢ}でございますつて、
\ruby{妙}{めう}な
\ruby{老夫}{おぢ|い}さんが
\ruby{御臺{\換字{所}}口}{お|だい|どころ|ぐち}へまゐりましたが、
お
\ruby{杉}{すぎ}さんも
\ruby{知}{し}つて
\ruby{居}{ゐ}る
\ruby{人}{ひと}のやうに
\ruby{見}{み}えます、
\ruby{何樣致}{ど|う|いた}しましやう。
』

と、
\ruby{其}{そ}の
\ruby{來}{きた}れる
\ruby{客}{きやく}の
\ruby{如何}{い|か}なる
\ruby{人}{ひと}なるかを
\ruby{小}{ちひさ}き
\ruby{胸}{むね}に
\ruby{危}{あやぶ}むが
\ruby{如}{ごと}き
\ruby{眼色}{め|いろ}して、
\ruby{年若}{とし|わか}く
\ruby{可憐}{か|はい}らしき
お
\ruby{春}{はる}は
\ruby{取次}{とり|つぎ}ぎたり。

『いゝよ。
\ruby{彼方}{あち|ら}へ
\ruby{行}{い}つて
\ruby{會}{あ}ふのも
\ruby{面倒}{めん|だう}だから、
\ruby{此室}{こ|ゝ}へ
\ruby{{\換字{連}}}{つ}れておいで!。
』

『お
\ruby{富}{とみ}の
\ruby{親}{おや}つて、
\ruby{彼}{あ}の
\ruby{妾}{わたし}の
\ruby{好}{す}きな
お
\ruby{富}{とみ}さんの?。
』

『アヽ、
\ruby{彼女}{あ|れ}の?。
』

『
\ruby{彼女}{あの|ひと}は
\ruby{退}{さが}つたの?。
』

『いゝえ、
\ruby{然樣定}{さ|う|き}まつた
\ruby{譯}{わけ}ちやあ
\ruby{無}{な}いが、
\ruby{大方}{おほ|かた}それで
\ruby{來}{き}たのだらう。
』

お
\ruby{龍}{りう}と
お
\ruby{彤}{とう}との
\ruby{間}{あひだ}に
\ruby{問}{とひ}と
\ruby{答}{こた}へとの
\ruby{{\換字{交}}}{か}はさるゝ
\ruby{間}{ま}も
\ruby{無}{な}く、
お
\ruby{春}{はる}に
\ruby{導}{みちび}かれて
\ruby{屈}{かゞ}みながら
\ruby{此方}{こ|なた}へ
\ruby{來}{きた}れる
\ruby{男}{をとこ}は、
お
\ruby{彤}{とう}の
\ruby{面}{おもて}をば
\ruby{見}{み}るや
\ruby{見}{み}ざるや、
\ruby{室}{へや}の
\ruby{内}{うち}へは
\ruby{入}{はい}りも
\ruby{得}{え}せず
\ruby{恐}{おそ}れ〳〵て
\ruby{鴫居}{しき|ゐ}の
\ruby{外}{そと}に
\ruby{坐}{すわ}りつ、
\ruby{先}{ま}づ
\ruby{其}{そ}の
\ruby{瘤}{や}せ
\ruby{枯}{から}びていと
\ruby{薄}{うす}く
\ruby{長}{なが}う
\ruby{見}{み}ゆる
\ruby{掌}{て}を
\ruby{疊}{たゝみ}に
\ruby{並}{なら}べ
\ruby{貼}{つ}けて、
\ruby{頭}{かしら}を
\ruby{其}{そ}の
\ruby{上}{うへ}に
\ruby{摺}{す}りつけ
\ruby{叮嚀}{てい|ねい}に
\ruby{挨拶}{あい|さつ}したるが、
\ruby{電燈}{でん|とう}の
\ruby{鮮}{あざ}やかなる
\ruby{光}{ひか}りは、
\ruby{光澤無}{つ|や|な}き
\ruby{細}{ほそ}き
\ruby{毛}{け}の
\ruby{烟}{けむり}のやうにほや〳〵と
\ruby{薄}{うす}く
\ruby{殘}{のこ}れる
\ruby{頭}{かうべ}を
\ruby{照}{て}らして、
\ruby{悲}{かな}しき
\ruby{老}{おい}
のさまを
\ruby{見}{あら}はし、
\ruby{左}{さ}のみ
\ruby{見苦}{み|ぐる}しき
\ruby{襤褸}{つゞ|れ}を
\ruby{纒}{まと}へりとにはあらねども、
\ruby{肩窄}{かた|すぼ}りて
\ruby{何處}{ど|こ}と
\ruby{無}{な}く
\ruby{寒}{さむ}げなる
\ruby{樣子}{やう|す}は、
\ruby{見}{み}るものをして
\ruby{此}{こ}の
\ruby{人{\換字{貧}}}{ひと|ひん}に
\ruby{窶}{やつ}れて
\ruby{苦}{くるし}めるにはあらずやと
\ruby{思}{おも}はしめたり。


\Entry{其三十一}

『
\ruby{好}{よ}く
お
\ruby{入來}{い|で}だつた、さあ
\ruby{{\換字{遠}}慮}{ゑん|りよ}
\ruby{仕無}{し|な}いで
\ruby{此方}{こつ|ち}へ
\ruby{御入}{お|はい}り。
』

と、お
\ruby{彤}{とう}に
\ruby{優}{やさ}しく
\ruby{言葉}{こと|ば}を
\ruby{掛}{か}けられて、
\ruby{老人}{らう|じん}は
\ruby{漸}{やうや}くに
\ruby{頭}{かしら}をこそ
\ruby{擡}{あ}げたれ、

『ハイ、ハイ。
』

とばかりにて
\ruby{{\換字{猶}}}{なほ}
\ruby{中々}{なか|〳〵}に
\ruby{席}{せき}を
\ruby{{\換字{進}}}{すゝ}まず。

『お
\ruby{富}{とみ}は
\ruby{何樣}{ど|う}
\ruby{仕}{し}ましたえ?。
』

と、
\ruby{親}{した}しげに
\ruby{復}{また}
\ruby{問}{と}はれて、

『ハイ、ハイ。
イエ、どうも
\ruby{不都合}{ふ|つ|がふ}な
\ruby{奴}{やつ}でございまして、
\ruby{何共}{なん|とも}ハヤ、どうも
\ruby{申上}{まをし|あ}げやうもございませんで。
』

と、
\ruby{{\換字{脱}}}{ぬ}け
\ruby{上}{あが}りたる
\ruby{額}{ひたひ}、
\ruby{細}{ほそ}き
\ruby{鼻}{はな}、たゞさへ
\ruby{貧相}{ひん|さう}の
\ruby{面}{おもて}に
\ruby{虛僞}{いつ|はり}ならぬ
\ruby{當惑}{たう|わく}の
\ruby{色}{いろ}を
\ruby{見}{あらは}し、
\ruby{甚}{いた}く
\ruby{恐縮}{きよう|しゆく}して
\ruby{同}{おな}じ
\ruby{樣}{やう}の
\ruby{事}{こと}のみを
\ruby{云}{い}へるは、
\ruby{傍眼}{わき|め}の
お
\ruby{龍}{りう}にさへもどかしく
\ruby{聞}{きこ}えたり。

\ruby{身}{み}に
\ruby{光澤}{て|り}も
\ruby{無}{な}く
\ruby{氣}{き}に
\ruby{張}{は}りも
\ruby{無}{な}くて、たゞ
\ruby{老猫}{ふる|ねこ}の
\ruby{寢}{ね}ぼれたるやうの、
\ruby{此}{こ}の
\ruby{老人}{らう|じん}の
\ruby{樣子}{やう|す}を、
お
\ruby{彤}{とう}は
\ruby{心底}{しん|そこ}より
\ruby{可笑}{を|か}しがりてか、
\ruby{唇}{くち}の
\ruby{邊}{あたり}にちらりと
\ruby{笑}{わらひ}をば
\ruby{上}{のぼ}せしが、
\ruby{忽地}{たち|まち}にして
\ruby{自}{みづか}ら
\ruby{抑}{おさ}へて、

『そんなに
\ruby{謝罪}{あや|ま}つてばかりおいでぢやあ
\ruby{話}{はなし}が
\ruby{出來}{で|き}ませんよ。
\ruby{何樣}{ど|う}したのだえ
お
\ruby{富}{とみ}は?。
』

と、
\ruby{極}{きは}めて
\ruby{{\換字{平}}穩}{おだ|やか}に
\ruby{問}{と}へば、
\ruby{老人}{らう|じん}は
\ruby{辛}{から}くも
\ruby{力}{ちから}を
\ruby{得}{え}たりと
\ruby{覺}{おぼ}しく、

『ハイ。
イエ、どうも
\ruby{飛}{と}んでも
\ruby{無}{な}い
\ruby{大變}{たい|へん}な
\ruby{{\換字{過}}失}{あや|まち}を
\ruby{彼女}{あ|れ}が
\ruby{致}{いた}しまして、』

と
\ruby{云}{い}ひかけて
\ruby{復}{また}
\ruby{叮嚀}{てい|ねい}に
\ruby{頭}{かしら}を
\ruby{下}{さ}げたり。

\ruby{笑}{わら}ふべき
\ruby{事}{こと}にはあらねど
\ruby{何}{なん}と
\ruby{無}{な}く
\ruby{其}{そ}の
\ruby{眞面目{\換字{過}}}{ま|じ|め|す}ぎ
\ruby{萎縮{\換字{過}}}{いぢ|け|す}ぎたる
\ruby{樣}{さま}の、
\ruby{氣}{き}の
\ruby{毒}{どく}らしきを
\ruby{越}{こ}して
\ruby{稍可笑}{やゝ|を|かし}きに、
お
\ruby{龍}{りう}は
\ruby{思}{おも}はず
\ruby{眼}{め}のみに
\ruby{笑}{わら}ひたり。

『そんなに
\ruby{謝罪}{あや|ま}つてばかり
\ruby{居}{ゐ}ないでも
\ruby{宜}{よ}うござんすといふのに。
』

『ハイ、イエ、
\ruby{然樣}{さ|う}
\ruby{仰}{おつし}あつて
\ruby{下}{くだ}さいますと、
\ruby{愈}{いよ〳〵}
\ruby{恐}{おそ}れ
\ruby{入}{い}りますのて。

\ruby{{\換字{廻}}}{まは}りくどうございましやうが
\ruby{御詫}{お|わび}を
\ruby{申}{まを}し
\ruby{上}{あ}げます、
\ruby{何卒知聞}{どう|ぞ|お|き}き
\ruby{下}{くだ}さいますやうに。
もうこれお
\ruby{詫}{わび}にも
\ruby{出}{で}そびれて
\ruby{十日}{とほ|か}ばかりになりましたが。
\ruby{然樣}{さ|よう}、エヽト、コート、
\ruby{丁度}{ちやう|ど}
\ruby{今日}{こん|にち}で
\ruby{十一日}{じう|いち|にち}になります。
\ruby{彼女}{あ|れ}が
\ruby{貴女}{あな|た}、
\ruby{眞靑}{まつ|さを}な
\ruby{顏}{かほ}をして
\ruby{駈}{か}け
\ruby{{\換字{込}}}{こ}んでまゐりまして、
\ruby{御主人樣}{ご|しゆ|じん|さま}の
\ruby{御大切}{ご|たい|せつ}な
\ruby{御菓子鉢}{お|か|し|ばち}を
\ruby{仕舞}{し|ま}はうとする
\ruby{時}{とき}、つい
\ruby{取}{と}り
\ruby{落}{おと}して
\ruby{割}{わ}つて
\ruby{仕舞}{し|ま}つたと
\ruby{申}{まを}すのでございます。
』

『ハア、
\ruby{大方}{おほ|かた}
\ruby{其故}{そ|れ}で
\ruby{駈}{か}け
\ruby{出}{だ}して
\ruby{行}{い}つて
\ruby{仕舞}{し|ま}つたのだらうと
\ruby{妾}{わたし}も
\ruby{思}{おも}つて
\ruby{居}{ゐ}たが、
\ruby{今}{いま}に
\ruby{何}{なん}とか
\ruby{云}{い}つておいでだらうと
\ruby{思}{おも}つて
\ruby{人}{ひと}もあげなかつたの。
\ruby{然樣}{さ|う}です、
\ruby{{\換字{古}}渡}{こ|わた}りの
\ruby{繪南京}{ゑ|なん|きん}の、
\ruby{一寸}{ちよ|つと}
\ruby{無}{な}い
\ruby{鉢}{はち}を
\ruby{破}{わ}つて
\ruby{仕舞}{し|ま}つたので。
』

『ハ、ハイ、ハイ。
どうも
\ruby{飛}{と}んでも
\ruby{無}{な}い
\ruby{麁忽}{そ|さう}を
\ruby{致}{いた}しました
\ruby{事}{こと}で。
\ruby{其品}{そ|れ}は
\ruby{利齋}{り|さい}とか
\ruby{仰}{おつし}ある
\ruby{方}{かた}が
\ruby{納}{をさ}めました
\ruby{品}{もの}でございまして、
\ruby{其折色々}{その|をり|いろ|〳〵}と
\ruby{其}{そ}の
\ruby{御器}{お|うつは}の
\ruby{結構}{けつ|こう}な
\ruby{事}{こと}を
\ruby{御話}{お|はなし}しなさいました
\ruby{其談}{そ|れ}をちら〳〵と
\ruby{彼女}{あ|れ}が
\ruby{承}{うけたま}はつて
\ruby{居}{を}つたさうで、
\ruby{何}{なに}も
\ruby{{\換字{分}}}{わか}りません
\ruby{彼女}{あ|れ}でも
\ruby{大層}{たい|そう}
\ruby{結構}{けつ|こう}な
\ruby{貴}{たつと}い
\ruby{御品}{お|しな}だといふ
\ruby{事}{こと}だけは
\ruby{存}{ぞん}じて
\ruby{居}{を}りました
\ruby{故}{ゆゑ}、これは
\ruby{御詫}{お|わび}の
\ruby{仕}{し}やうも
\ruby{無}{な}い
\ruby{事}{こと}を
\ruby{仕}{し}たと、ト
\ruby{胸}{むね}を
\ruby{衝}{つ}いたと
\ruby{申}{まを}すのでございまして。
\ruby{何樣}{ど|う}も
\ruby{何}{なん}ともハヤ
\ruby{相}{あひ}
\ruby{濟}{す}みません
\ruby{事}{こと}で。
ハイ、ハイ。
それから
\ruby{私}{わたくし}が
\ruby{貴女}{あな|た}、
\ruby{代}{かは}りの
\ruby{品}{しな}を
\ruby{差出}{さし|いだ}しまして
\ruby{御勘辯}{ご|かん|べん}を
\ruby{願}{ねが}はうと
\ruby{存}{ぞん}じまして、
\ruby{彼女}{あ|れ}と
\ruby{二人}{ふた|り}で
\ruby{東京中}{とう|きやう|ぢゆう}を
\ruby{搜}{さが}しましたが、
\ruby{中々}{なか|〳〵}どう
\ruby{致}{いた}しまして
\ruby{似}{に}たやうな
\ruby{品}{もの}もございません。
』

『まあ
\ruby{詰}{つま}らないそんな
\ruby{餘計}{よ|けい}な
\ruby{苦勞}{く|らう}を
\ruby{仕}{し}て
\ruby{貰}{もら}はうとも
\ruby{何}{なん}とも
\ruby{此方}{こち|ら}ぢやあ
\ruby{思}{おも}つて
\ruby{居}{ゐ}も
\ruby{仕}{し}ないものを!。
』

『ハイ、ハイ。
まことに
\ruby{何樣}{ど|う}も
\ruby{恐}{おそ}れ
\ruby{入}{い}りましたことで。
\ruby{然樣}{さ|う}
\ruby{仰}{おつし}あつて
\ruby{下}{くだ}さいましても、
\ruby{夫}{それ}では
\ruby{濟}{す}みません
\ruby{譯}{わけ}で。
\ruby{貴女}{あな|た}、
\ruby{彼女}{あ|れ}が
\ruby{此方樣}{こち|ら|さま}へまゐります
\ruby{{\換字{前}}}{まへ}に
\ruby{御奉公}{ご|ほう|こう}
\ruby{致}{いた}して
\ruby{居}{を}りました
\ruby{御邸}{お|やしき}は
\ruby{伯爵樣}{はく|しやく|さま}とかでいらつしやいましたが、
\ruby{彼方}{かな|た}
\ruby{樣}{さま}では
\ruby{都}{す}べて
\ruby{女中}{ぢよ|ちゆう}の
\ruby{毀}{こは}しましたものは
\ruby{皆}{みな}
\ruby{其}{そ}の
\ruby{毀}{こは}したものが
\ruby{償}{つぐな}ひまする
\ruby{御定規}{お|さだ|め}でございまして、
\ruby{彼女}{あ|れ}なぞは
\ruby{頂戴}{ちやう|だい}するものが
\ruby{少}{すくな}うございますから、
\ruby{始{\換字{終}}}{し|ゞう}
\ruby{持出}{もち|だ}しになりますやうな
\ruby{事}{こと}でございました
\ruby{位}{ぐらゐ}で。
』

『ヘーエー!。
』

『でございますから
\ruby{貴女}{あな|た}、
\ruby{私}{わたくし}は
\ruby{一生懸命}{いつ|しやう|けん|めい}に
\ruby{搜}{さが}しまして、
\ruby{{\換字{終}}}{しまひ}には
\ruby{利齋}{り|さい}といふ
\ruby{人}{ひと}まで
\ruby{{\換字{尋}}}{たづ}ねまして
\ruby{仔細}{し|さい}を
\ruby{話}{はな}しまして、これ〳〵の
\ruby{鉢}{はち}が
\ruby{欲}{ほし}しいと
\ruby{申}{まを}しましたところ、
\ruby{今}{いま}
\ruby{欲}{ほ}しいと
\ruby{云}{い}つても
\ruby{今}{いま}
\ruby{有}{あ}るものでも
\ruby{無}{な}いし、
\ruby{有}{あ}つたに
\ruby{致}{いた}しても
\ruby{如是}{これ|〳〵}の
\ruby{價}{ね}のものだと
\ruby{承}{うけたま}はりまして、
\ruby{私{\換字{連}}}{わたくし|づれ}の
\ruby{力}{ちから}には
\ruby{及}{およ}びかねます
\ruby{大變}{たい|へん}なものでございましたのでいよ〳〵
\ruby{吃驚}{びつ|くり}
\ruby{致}{いた}しまして、とてものめ〳〵と
\ruby{御詫}{お|わび}に
\ruby{出}{で}られた
\ruby{段}{だん}ではございませんが、
\ruby{死}{し}ぬやうな
\ruby{氣}{き}になつて
\ruby{漸}{や}つと
\ruby{今日御詫}{こん|ち|お|わび}に
\ruby{出}{で}ましたで。
』

こゝまで
\ruby{云}{い}ひさして
\ruby{埋}{うづ}むるが
\ruby{如}{ごと}く
\ruby{疊}{たゝみ}に
\ruby{頭}{かうべ}を
\ruby{擦}{す}りつけたる
\ruby{時}{とき}、
\ruby{薄}{うす}き
\ruby{髮}{かみ}の
\ruby{下}{した}に
\ruby{透}{す}きて
\ruby{見}{み}えたる
\ruby{頭顱}{あた|ま}の
\ruby{地}{ぢ}には、
\ruby{如何}{い|か}ばかり
\ruby{{\換字{弱}}}{よわ}き
\ruby{心}{こゝろ}の
\ruby{苦}{くる}しくや
\ruby{感}{かん}じけん、
\ruby{慚}{はづ}かしさと
\ruby{切無}{せつ|な}さに
\ruby{絞}{しぼ}り
\ruby{出}{いだ}されたる
\ruby{熱}{あつ}き
\ruby{汗}{あせ}の
\ruby{點々}{てん|〳〵}と
\ruby{玉}{たま}をなして、
\ruby{蒸氣}{ゆ|げ}さへいさゝか
\ruby{立}{た}つごとく
\ruby{見}{み}えたり。


\Entry{其三十二}

『
\ruby{何樣}{ど|う}も
\ruby{何}{なん}と
\ruby{申上}{まをし|あげ}ましても
\ruby{相}{あひ}
\ruby{濟}{す}みません
\ruby{無調法}{ぶ|てふ|はふ}で。
ハイ。
\ruby{口}{くち}ばかりで
\ruby{何}{なに}を
\ruby{申}{まを}し
\ruby{上}{あ}げましても、
\ruby{實以}{じつ|もつ}て
\ruby{相}{あひ}
\ruby{濟}{す}みません
\ruby{譯}{わけ}で、ハイ。
お
\ruby{羞}{はづか}しいことを
\ruby{申}{まを}し
\ruby{上}{あ}げませんければ
\ruby{理}{り}が
\ruby{聞}{きこ}えませぬが、
\ruby{實}{じつ}は
\ruby{段々}{だん|〳〵}と
\ruby{不幸}{ふし|あわせ}は
\ruby{續}{つゞ}きますし、
\ruby{私}{わたくし}は
\ruby{病身}{びやう|しん}で
\ruby{商法}{しやう|はふ}は
\ruby{止}{や}めて
\ruby{居}{を}りますし、
\ruby{少}{すこ}しばかりの
\ruby{地{\換字{所}}}{ぢ|しよ}
\ruby{家作}{か|さく}で
\ruby{細々}{ほそ|〴〵}と
\ruby{{\換字{遣}}}{や}つて
\ruby{居}{を}ります
\ruby{中}{なか}を、
\ruby{不孝者}{ふ|かう|もの}めの
\ruby{伜}{せがれ}に
\ruby{大無}{だい|な}しにされまして、まことにはや
\ruby{何樣}{ど|う}も
\ruby{斯樣}{か|う}もならないやうになつて
\ruby{居}{を}りまするので、たゞもう
\ruby{明暮}{あけ|くれ}、
\ruby{伜}{せがれ}めの
\ruby{碌}{ろく}で
\ruby{無}{な}しの
\ruby{料簡}{れう|けん}の
\ruby{直}{なほ}りますやうにと、
\ruby{信心}{しん|〴〵}を
\ruby{致}{いた}すのを
\ruby{今日}{こん|にち}の
\ruby{{\換字{勤}}}{つとめ}に
\ruby{致}{いた}して
\ruby{居}{を}るやうな
\ruby{意氣地}{い|く|ぢ}
の
\ruby{無}{な}い
\ruby{次第}{し|だい}でございますから、
\ruby{何共恐}{なん|とも|おそ}れ
\ruby{入}{い}りまする
\ruby{身{\換字{勝}}手}{み|がつ|て}な
\ruby{申{\換字{分}}}{まをし|ぶん}ではございますが、
\ruby{今}{いま}が
\ruby{今何樣}{いま|ど|う}にか
\ruby{致}{いた}さうと
\ruby{致}{いた}しますれば、
\ruby[<h||]{私}{わたくし}
\ruby{一人}{ひと|り}のところへ
\ruby{夫{\換字{婦}}掛向}{ふう|ふ|かけ|むか}ひの
\ruby{人}{ひと}を
\ruby{置}{お}きまして、その
\ruby{貸間}{かし|ま}の
\ruby{料}{れう}で
\ruby{食}{た}べて
\ruby{居}{を}りまする
\ruby{住家}{すま|ゐ}をでも、
\ruby{何樣}{ど|う}か
\ruby{致}{いた}して
\ruby{算段致}{さん|だん|いた}すより
\ruby{他}{ほか}はございませんので、それでは
\ruby{何樣}{ど|う}も
\ruby{後々}{あと|〳〵}のところが……』

\ruby{貧相}{ひん|さう}な
\ruby{顏}{かほ}をいよ〳〵
\ruby{貧相}{ひん|さう}に
\ruby{仕}{し}て
\ruby{困難}{こん|なん}の
\ruby{趣}{おもむ}きを
\ruby{{\換字{述}}}{の}べ
\ruby{哀愍}{あは|れみ}を
\ruby{乞}{こ}はんとする、
\ruby{其}{そ}の
\ruby{言語}{もの|いひ}は
\ruby{人}{ひと}の
\ruby{同{\換字{情}}}{どう|じやう}を
\ruby{惹}{ひ}くに
\ruby{足}{た}るほどの
\ruby{氣合}{き|あひ}さへ
\ruby{乏}{とぼ}しけれど、
\ruby{其}{そ}のくど〳〵しく
\ruby{惡叮嚀}{わる|てい|ねい}なるに
\ruby{愚直}{ぐ|ちよく}さは
\ruby{盡}{こと〴〵}く
\ruby{知}{し}られたり。

お
\ruby{彤}{とう}は
\ruby{最早聞}{も|はや|き}き
\ruby{居}{ゐ}るに
\ruby{堪}{た}へかねてや、
\ruby{言葉}{こと|ば}の
\ruby{澱}{よど}みに
\ruby{付}{つ}け
\ruby{入}{い}りて
\ruby{{\換字{又}}}{また}
\ruby{靜}{しづか}に
\ruby{{\換字{又}}}{また}
\ruby{爽快}{さ|わやか}に、

『まあ
\ruby{其}{それ}は
\ruby{大層}{たい|そう}に
\ruby{心配}{しん|ぱい}を
お
\ruby{爲}{し}だつたねえ。
お
\ruby{{\換字{前}}}{まへ}さんは
\ruby{當世}{たう|せい}にあ
\ruby{珍}{めづ}らしい
\ruby{律義}{りち|ぎ}な
\ruby{氣性}{きし|やう}なこと!。
なあに
\ruby{彼樣}{あ|ん}な
\ruby{鉢}{はち}の
\ruby{一}{ひと}つや
\ruby{{\換字{半}}{\換字{分}}}{はん|ぶん}、
\ruby{麁忽}{そ|さう}で
\ruby{毀}{こは}したものを
\ruby{何}{なん}で
\ruby{妾}{わたし}が
\ruby{償}{つくの}へなんぞといふものですかネ。
』

と
\ruby{云}{い}ひ
\ruby{出}{いだ}せば、
\ruby{老人}{らう|じん}は
\ruby{何}{なん}と
\ruby{聞}{き}き
\ruby{取}{と}つてか
\ruby{慌}{あわ}てゝ
\ruby{遮}{さへぎ}りて、

『ど、
\ruby{何樣}{ど|う}
\ruby{致}{いた}しまして
\ruby{貴女}{あな|た}、
\ruby{伯爵樣}{はく|しやく|さま}の
\ruby{御邸}{お|やしき}でさへ、』

と、
\ruby{身}{み}に
\ruby{入}{し}みて
\ruby{記}{おぼ}えたる
\ruby{事}{こと}にても
\ruby{有}{あ}るなるべし、
\ruby{伯爵邸}{はく|しやく|てい}の
\ruby{定規}{さだ|め}を
\ruby{例}{れい}に
\ruby{引}{ひ}きかくるを、
\ruby{二}{に}の
\ruby{句}{く}を
\ruby{續}{つ}がせず、
お
\ruby{彤}{とう}は
\ruby{冷}{ひや}やかに
\ruby{笑}{わら}つたり。

『まあ
\ruby{御聞}{お|き}きなさいよ。
\ruby{伯爵樣}{はく|しやく|さま}の
\ruby{御邸}{お|やしき}は
\ruby{伯爵樣}{はく|しやく|さま}の
\ruby{御邸}{お|やしき}で、
\ruby{妾}{わたし}の
\ruby{家}{うち}は
\ruby{妾}{わたし}の
\ruby{家}{うち}ですよ。
いゝ
\ruby{身{\換字{分}}}{み|ぶん}の
\ruby{方}{かた}の
\ruby{眞似}{ま|ね}を
\ruby{妾等}{わた|しら}が
\ruby{仕}{し}ちやあ
\ruby{成}{な}りませんからネ。
\ruby{金屬}{か|ね}でゞも
\ruby{有}{あ}りやあ
\ruby{仕}{し}まいし、
\ruby{根}{ね}が
\ruby{磁器}{やき|もの}ですもの、
\ruby{破}{わ}れることも
\ruby{有}{あ}りましやう、
\ruby{其}{そ}の
\ruby{磁器}{やき|もの}が
\ruby{麁忽}{そ|さう}で
\ruby{破}{わ}れたのを
\ruby{何樣}{ど|う}まあ
\ruby{酷}{むご}く
\ruby{咎}{とが}め
\ruby{立}{だて}を
\ruby{仕}{し}ましやう!。
』

『ハ、ハイ、ハイ、ハイ。
』

\ruby{激}{はげ}しく
\ruby{感}{かん}じたるならん、
\ruby{氣息}{い|き}の
\ruby{詰}{つ}まるやうに
\ruby{老人}{らう|じん}は
\ruby{急}{せ}き
\ruby{込}{こ}みて
\ruby{挨拶}{あい|さつ}したり。

『それも
\ruby{{\換字{平}}常}{ふだ|ん}の
\ruby{{\換字{勤}}}{つと}め
\ruby{方}{かた}でも
\ruby{惡}{わる}いといふのなら
\ruby{叱言}{こ|ごと}を
\ruby{云}{い}ふまいものでも
\ruby{有}{あ}りませんが、
\ruby{何}{なに}も
\ruby{彼}{か}も
\ruby{悉皆}{みん|な}
\ruby{好}{よ}く
\ruby{爲}{し}て
\ruby{吳}{く}れて
\ruby{居}{ゐ}る
\ruby{彼}{あ}の
お
\ruby{富}{とみ}の
\ruby{爲}{し}た
\ruby{{\換字{過}}失}{あや|まち}ですもの!。
』

『ハ、ハ、ハイ、ハイ。
』

『
\ruby{少}{すこ}し
\ruby{位}{くらゐ}の
\ruby{品}{もの}を
\ruby{毀}{こは}したからつて
\ruby{何}{なに}を
\ruby{云}{い}ひましやう!。
\ruby{使}{つか}つてる
\ruby{中}{うち}に
\ruby{器物}{も|の}が
\ruby{毀}{こは}れるのは
\ruby{當然}{あたり|まへ}の
\ruby{事}{こと}で、
\ruby{其}{それ}を
\ruby{厭}{いと}やあ
\ruby{箱}{はこ}の
\ruby{中}{なか}へでも
\ruby{藏}{しま}つて
\ruby{置}{お}くより
\ruby{他有}{ほか|あ}りやあ
\ruby{仕無}{し|な}いと
\ruby{思}{おも}ひますよ。
\ruby{器物}{も|の}をいたはつて
\ruby{人}{ひと}をいたはらないやうな
\ruby{事}{こと}は
\ruby{妾}{わたし}あ
\ruby{大{\換字{嫌}}}{だい|きら}ひで、あんな
\ruby{磁物}{やき|もの}を
\ruby{十個集}{と|を|よ}せたつて
\ruby{百集}{ひやく|よ}せたつて
お
\ruby{富}{とみ}が
\ruby{出來}{で|き}るのぢやあ
\ruby{無}{な}いんですもの、
\ruby{幾干}{いく|ら}
お
\ruby{富}{とみ}の
\ruby{方}{はう}を
\ruby{大切}{だい|じ}に
\ruby{思}{おも}つてるか
\ruby{知}{し}れや
\ruby{仕}{し}ません。
』

『ハ、ハ、ハイ、ハイ。
』

『だから
\ruby{{\換字{過}}失}{あや|まち}は
\ruby{{\換字{過}}失}{あや|まち}で、
\ruby{一言}{ひと|こと}
\ruby{詫}{わび}を
\ruby{云}{い}はれりやあそれまでゞ
\ruby{濟}{す}まして
\ruby{仕舞}{し|ま}ふがネ、それよりやあ
お
\ruby{富}{とみ}が
\ruby{大變}{たい|へん}に
\ruby{濟}{す}まない
\ruby{事}{こと}がありますよ。
』

『ハハツ、ハイ、ハイ、ヘイ。
』

『
\ruby{其}{それ}あ
\ruby{默}{だま}つて
\ruby{駈}{か}け
\ruby{出}{だ}して
\ruby{仕舞}{し|ま}つて
\ruby{妾}{わたし}に
\ruby{不自由}{ふ|じ|ゆう}をさせたことです。
\ruby{何}{なに}も
\ruby{彼}{か}も
\ruby{彼女}{あ|れ}にさせて
\ruby{居}{ゐ}るのに、
\ruby{急}{きふ}に
\ruby{出}{で}て
\ruby{行}{い}かれちやあ
\ruby{何樣}{ど|ん}なに
\ruby{不自由}{ふ|じ|ゆう}に
\ruby{思}{おも}ふか
\ruby{知}{し}れません。
\ruby{丁度}{ちやう|ど}
\ruby{好}{い}い
\ruby{代}{かは}りが
\ruby{有}{あ}りは
\ruby{有}{あ}つたやうなものゝ、
\ruby{眞底詫}{しん|そこ|わ}びる
\ruby{氣}{き}があるなら、
\ruby{歸}{かへ}つて
\ruby{來}{き}てちやんと
\ruby{{\換字{勤}}}{つと}めつゞく
\ruby{方}{はう}が
\ruby{何程好}{いく|ら|い}いか
\ruby{知}{し}れやしません。
』

『ハヽツ、ハイ、ハイ。
で、では
\ruby{麁忽}{そ|さう}を
\ruby{致}{いた}しましたのは
\ruby{御免}{お|ゆる}し
\ruby{下}{くだ}さいまして、そ、そして
\ruby{今迄{\換字{通}}}{いま|まで|どほ}り
\ruby{御使}{お|つか}ひ
\ruby{下}{くだ}さいまするので。
』

『
\ruby{使}{つか}つて
\ruby{{\換字{遣}}}{や}りますとも、
\ruby{使}{つか}つて
\ruby{{\換字{遣}}}{や}りますとも!。
あんな
\ruby{忠義}{ちう|ぎ}ものゝ
\ruby{氣立}{き|だて}の
\ruby{好}{い}い
\ruby{兒}{こ}が、
\ruby{磁器}{やき|もの}の
\ruby{三}{み}つや
\ruby{四}{よ}つ
\ruby{破}{こは}したつて
\ruby{何}{なん}の
\ruby{何}{なん}とも
\ruby{思}{おも}ふもんで。
』

『ハアーツ、
\ruby{有}{あ}り
\ruby{難}{がた}うございます、
\ruby{有}{あ}り
\ruby{難}{がた}うございます。
\ruby{早{\換字{速}}}{さつ|そく}
\ruby{彼女}{あ|れ}に
\ruby{唯}{たゞ}
\ruby{今}{いま}の
\ruby{有}{あ}り
\ruby{難}{がた}い
\ruby{御思召}{お|ぼし|めし}を
\ruby{申聞}{まをし|き}かせませんでは。
』

\ruby{老人}{らう|じん}は
\ruby{嬉}{うれ}しさに
\ruby{泣}{な}かぬばかりの
\ruby{顏}{かほ}して、
\ruby{許}{ゆる}しをさへ
\ruby{得}{え}ば
\ruby{立}{た}たんとし
\ruby{追立尻}{おつ|たて|じり}になつたり。

『お
\ruby{富}{とみ}に
\ruby{話}{はな}すつて、
\ruby{{\換字{近}}處}{きん|じよ}へでも
\ruby{{\換字{連}}}{つ}れて
\ruby{來}{き}て
\ruby{居}{ゐ}るの?。
』

『ハイ、イエ。
\ruby{一緖}{いつ|しよ}に
\ruby{{\換字{連}}}{つ}れてはまゐりましたが、
\ruby{御裏口}{お|うら|ぐち}の
\ruby{{\換字{戸}}外}{そ|と}に
\ruby{立}{た}たせて
\ruby{置}{お}きましたので。
』

『ホヽホヽ、
\ruby{愍然}{かはい|さう}に!。
\ruby{何}{なん}だつて
\ruby{{\換字{戸}}外}{そ|と}になんか
\ruby{立}{た}たせて
\ruby{置}{お}くのだらう、
\ruby{早}{はや}く
\ruby{此方}{こつ|ち}へ
\ruby{{\換字{連}}}{つ}れておいでなさい。
』


\Entry{其三十三}

『
\ruby{妾}{わたし}は
\ruby{東京}{とう|きやう}にやあ
\ruby{今時彼樣}{いま|どき|あ|ゝ}いふ
\ruby{人}{ひと}は
\ruby{無}{な}からうとばつかり
\ruby{思}{おも}つて
\ruby{居}{ゐ}ましたが、たまには
\ruby{矢張}{やつ|ぱ}り
\ruby{彼樣}{あ|ん}な
\ruby{正直}{しやう|ぢき}な
\ruby{篤實}{こく|めい}の
\ruby{人}{ひと}もございますのネエ。
』

お
\ruby{龍}{りう}の
\ruby{叔母}{を|ば}の
\ruby{如是云}{か|く|い}ひ
\ruby{出}{い}づるを
\ruby{主人}{ある|じ}に
\ruby{答}{こた}へさする
\ruby{迄}{まで}も
\ruby{無}{な}く、
お
\ruby{龍}{りう}は
\ruby{代}{かは}つて、

『そりやあ
\ruby{叔母}{を|ば}さん
\ruby{東京}{とう|きやう}だつて
\ruby{狡猾}{ず|る}い
\ruby{人}{ひと}ばかりぢやあ
\ruby{有}{あ}りません、
\ruby{廣}{ひろ}いんですもの。
\ruby{今}{いま}の
\ruby{話}{はなし}の
\ruby{伯爵}{はく|しやく}のやうな
\ruby{卑格}{け|ち}な
\ruby{人}{ひと}も
\ruby{有}{あ}る
\ruby{代}{かは}りにやあ、
\ruby{姊}{ねえ}さんのやうな
\ruby{氣}{き}の
\ruby{大}{おほ}きい
\ruby{人}{ひと}もあるぢやあ
\ruby{有}{あ}りませんか。
』

と
\ruby{云}{い}へば、

『ほんにね!。
だが、
\ruby{其樣}{そ|ん}なに
\ruby{高}{たか}い
\ruby{磁器}{やき|もの}なんかゞ
\ruby{有}{あ}るものか
\ruby{知}{し}ら?。
』

といふ。

『なあに、
\ruby{高}{たか}いと
\ruby{云}{い}つたところで
\ruby{多寡}{た|か}の
\ruby{知}{し}れたものですが、つまり
\ruby{氣}{き}の
\ruby{小}{ちひさ}い
\ruby{人}{ひと}にやあ
\ruby{何樣}{ど|ん}なものでも
\ruby{大}{たい}したものに
\ruby{思}{おも}へるのでねえ、それで
\ruby{大變}{たい|へん}に
\ruby{心配}{しん|ぱい}したのでしやう。
』

とお
\ruby{彤}{とう}の
\ruby{打笑}{うち|わら}ふ
\ruby{此}{こ}の
\ruby{問答}{もん|だふ}の
\ruby{中}{うち}に
\ruby{老人}{らう|じん}は
\ruby{復入}{また|い}り
\ruby{來}{きた}りしが、
\ruby{背後}{うし|ろ}には
\ruby{恐}{おそ}れ
\ruby{惶}{かしこ}みて
\ruby{小}{ちひさ}くなりたる
\ruby{若}{わか}き
\ruby{女}{こ}を
\ruby{{\換字{連}}}{つ}れたり。
お
\ruby{龍}{りう}の
\ruby{叔母}{を|ば}は
\ruby{何氣無}{なに|げ|な}く
\ruby{打見}{うち|み}やるに、
\ruby{面貌}{おも|ざし}は
\ruby{老人}{らう|じん}を
\ruby{其儘}{その|まゝ}に
\ruby{眼}{め}も
\ruby{細}{ほそ}く
\ruby{鼻}{はな}も
\ruby{細}{ほそ}けれど
\ruby{醜}{みにく}きかたにはあらず、
\ruby{卵子形}{たま|ご|かた}の
\ruby{顏}{かほ}の
\ruby{上品}{じやう|ひん}に
\ruby{優}{やさ}しくて、
\ruby{慾}{よく}には
\ruby{色}{いろ}のやゝ
\ruby{靑白}{あを|じろ}く
\ruby{束髮}{そく|はつ}の
\ruby{毛}{け}の
\ruby{纖{\換字{過}}}{ほそ|すぎ}ぎて
\ruby{嵩少}{かさ|すくな}きを
\ruby{治}{なほ}して
\ruby{{\換字{遣}}}{や}りたけれど、
\ruby{年齡}{と|し}には
\ruby{似氣無}{に|げ|な}く
\ruby{靜}{しづか}に
\ruby{沈着}{おち|つ}いたる
\ruby{樣}{さま}
\ruby{如何}{い|か}にも
\ruby{恰悧}{り|かう}らしく、
お
\ruby{龍}{りう}には
\ruby{慥}{たしか}に
\ruby{三歳四歳劣}{み|つ|よ|つ|おと}りなるべけれど、
\ruby{見比}{み|くら}ぶれば
お
\ruby{龍}{りう}の
\ruby{方若}{かた|わか}く
\ruby{{\換字{浮}}々}{うき|うき}として、
\ruby{既}{すで}に
\ruby{生死}{いき|しに}の
\ruby{苦勞}{く|らう}を
\ruby{知}{し}れるにも
\ruby{似}{に}ず
\ruby{{\換字{猶}}}{なほ}あど
\ruby{無}{な}く
\ruby{見}{み}ゆ。
\ruby{今}{いま}の
\ruby{談}{はなし}の
お
\ruby{富}{とみ}とは
\ruby{是}{これ}なるべし、
\ruby{成程{\換字{平}}常}{なる|ほど|ふだ|ん}は
\ruby{{\換字{過}}失}{あや|まち}など
\ruby{中々仕出}{なか|〳〵|しい|だ}すまじき
\ruby{愼}{つゝし}み
\ruby{深}{ふか}げの、
\ruby{氣}{き}の
\ruby{能}{よ}く
\ruby{{\換字{廻}}}{まは}りさうなくすみたる
\ruby{女}{をんな}かな、これで
\ruby{若}{も}し
\ruby{此程}{これ|ほど}に
\ruby{縞}{しま}の
\ruby{粗}{あら}き
\ruby{銘撰}{めい|せん}を
\ruby{着居}{き|を}らずば、
\ruby{能}{よ}く
\ruby{見}{み}ぬものは
\ruby{二十歳}{は|た|ち}とも
\ruby{見做}{み|な}すべしと
\ruby{一度}{ひと|たび}は
\ruby{思}{おも}ひしが、
\ruby{流石}{さす|が}に
\ruby{年齡}{と|し}は
\ruby{年齡}{と|し}なり、
\ruby{主人}{しゆ|じん}と
\ruby{眼}{め}を
\ruby{見合}{み|あは}すや
\ruby{否}{いな}や、いと
\ruby{幼}{おさな}き
\ruby{素振}{そ|ぶ}りの
\ruby{繕}{つくろ}ひ
\ruby{氣}{げ}も
\ruby{無}{な}く
\ruby{頭}{かうべ}を
\ruby{疊}{たゝみ}に
\ruby{着}{つ}けて、

『
\ruby{飛}{と}んでも
\ruby{無}{な}い
\ruby{麁忽}{そ|さう}を
\ruby{致}{いた}しましたのを、
\ruby{御免下}{ご|めん|くだ}さいまして
\ruby{眞}{まこと}に
\ruby{有}{あ}り
\ruby{難}{がた}うございます。
それから
\ruby{御斷}{お|ことわ}りも
\ruby{致}{いた}しませんで
\ruby{宅}{たく}へまゐりましたのは
\ruby{{\換字{猶}}}{なほ}
\ruby{相濟}{あひ|す}みませんでございました』

と
\ruby{素直}{す|なほ}に
\ruby{謝罪}{あや|ま}れば、
お
\ruby{彤}{とう}は
\ruby{莞爾}{に|こ}やかに、

『
\ruby{{\換字{平}}常}{ふだ|ん}の
お
\ruby{{\換字{前}}}{まへ}の
\ruby{仕方}{し|かた}が
\ruby{好}{い}いから
\ruby{叱}{しか}らうとも
\ruby{何}{なん}とも
\ruby{思}{おも}つてや
\ruby{仕}{し}ません。
\ruby{{\換字{過}}失}{あや|まち}は
\ruby{{\換字{過}}失}{あや|まち}だから
\ruby{仕方}{し|かた}が
\ruby{無}{な}い。
これからさへ
\ruby{氣}{き}を
\ruby{付}{つ}けて
お
\ruby{吳}{く}れなら
\ruby{其}{それ}で
\ruby{可}{いゝ}よ。
さあもうをかしな
\ruby{顏}{かほ}を
\ruby{仕}{し}ないで
お
\ruby{{\換字{前}}}{まへ}の
\ruby{馴染}{な|じみ}の
お
\ruby{龍}{りう}ちやんにも
\ruby{挨拶}{あい|さつ}を
お
\ruby{爲}{し}。
』

といふ。
\ruby{叱}{しか}りだにされず
\ruby{免}{ゆる}されたる
\ruby{嬉}{うれ}しさに、さしぐむ
\ruby{淚}{なみだ}の
\ruby{目}{め}をあげて、さてそつと
お
\ruby{龍}{りう}を
\ruby{見}{み}て
\ruby{懷}{なつか}しげに
\ruby{叩頭}{じ|ぎ}すれば、
お
\ruby{龍}{りう}もまた
\ruby{懐}{なつ}かしげに
\ruby{其方}{そな|た}を
\ruby{見}{み}やりて、

『お
\ruby{{\換字{前}}}{まへ}さんが
\ruby{此方}{こち|ら}に
\ruby{見}{み}えなかつたので、
\ruby{妾}{わたし}あ
\ruby{何樣}{ど|ん}なにか
\ruby{眞實}{ほん|と}に
\ruby{淋}{さび}しく
\ruby{思}{おも}つたらう!。
\ruby{丁度好}{ちやう|ど|い}い
\ruby{事}{こと}ねえ、かうして
\ruby{歸}{かへ}つておいでだつたのだから、またこれから
お
\ruby{{\換字{前}}}{まへ}さんと
\ruby{仲}{なか}を
\ruby{好}{よ}くして、
\ruby{先}{せん}のやうに
\ruby{{\換字{又}}}{また}
\ruby{毎{\換字{朝}}起}{まい|あさ|おこ}して
\ruby{貰}{もら}ひましやうかネエ。
ホヽヽ。
』

と
\ruby{埒無}{らち|な}きことを
\ruby{早語}{はや|かた}り
\ruby{掛}{か}く。

『また
\ruby{其樣}{そ|ん}な
\ruby{下}{くだ}らない
\ruby{好}{い}い
\ruby{氣}{き}ぜんの
\ruby{事}{こと}を
お
\ruby{{\換字{前}}}{まへ}は
お
\ruby{云}{い}ひだよ。
』

\ruby{苦々}{にが|〴〵}しげに
\ruby{叔母}{を|ば}はたしなむるを
お
\ruby{彤}{とう}は
\ruby{餘{\換字{所}}}{よ|そ}に
\ruby{聽}{き}きて
\ruby{茶}{ちや}をや
\ruby{得}{え}んとする、
お
\ruby{春}{はる}〳〵と
\ruby{呼}{よ}ぶに、
お
\ruby{春}{はる}は
\ruby{如何}{い|か}にしけん
\ruby{更}{さら}に
\ruby{出}{い}で
\ruby{來}{きた}らず。
かゝる
\ruby{事}{こと}を
\ruby{甚}{いた}く
\ruby{{\換字{悅}}}{よろこ}ばぬ
お
\ruby{彤}{とう}の、
\ruby{聲}{こゑ}こそは
\ruby{仂無}{はし|たな}く
\ruby{高}{たか}めね、

『お
\ruby{春}{はる}、
お
\ruby{春}{はる}、』

と
\ruby{復呼}{また|よ}べども
\ruby{更}{さら}に
\ruby{答}{こた}へなし。

『お
\ruby{春}{はる}!。
\ruby{何樣}{ど|う}したえ?
お
\ruby{春}{はる}!。
』

\ruby{一}{ひ}ㇳ
\ruby{聲}{こゑ}は
\ruby{一}{ひ}ㇳ
\ruby{聲}{こゑ}に
\ruby{癇}{かん}の
\ruby{募}{つの}るさま
\ruby{歷々}{あり|〳〵}と
\ruby{見}{み}ゆるに、

『
\ruby{何}{なん}でございますか、
\ruby{妾}{わたし}が』

とお
\ruby{富}{とみ}の
\ruby{立}{た}ちにかゝる
\ruby{時}{とき}、
\ruby{臺{\換字{所}}}{だい|どころ}とおぼしきところにて、

『お
\ruby{春}{はる}さん、
お
\ruby{春}{はる}さん、
\ruby{御召}{お|め}しなさるやうぢや
\ruby{無}{な}いかえ。
おや、お
\ruby{{\換字{前}}}{まへ}さん、
\ruby{何}{なに}を
\ruby{泣}{な}いて
\ruby{居}{ゐ}るの?。
』

とお
\ruby{杉}{すぎ}が
\ruby{{\換字{平}}素}{いつ|も}
\ruby{馬士聲}{ま|ご|ゞゑ}とて
\ruby{叱}{しか}らるゝいと
\ruby{大}{おほ}きなる
\ruby{{\換字{丈}}夫}{ぢや|うぶ}さうな
\ruby{其}{そ}の
\ruby{馬士聲}{ま|ご|ゞゑ}の
\ruby{聞}{きこ}えぬ。


\Entry{其三十四}

『
\ruby{鶉}{うづら}といふ
\ruby{鳥}{とり}は
\ruby{自{\換字{分}}}{じ|ぶん}の
\ruby{身}{み}から
\ruby{出}{で}る
\ruby{香氣}{に|ほひ}を
\ruby{止}{と}めて
\ruby{仕舞}{し|ま}つて、
\ruby{獵犬}{かり|いぬ}に
\ruby{嗅}{か}ぎ
\ruby{出}{だ}されないやうにする
\ruby{機能}{はた|らき}を
\ruby{有}{も}つて
\ruby{居}{ゐ}ると
\ruby{銃獵者}{と|り|うち}に
\ruby{聞}{き}いたが、
お
\ruby{彤}{とう}、
\ruby{汝}{きさま}は
\ruby{一體}{いつ|たい}が
\ruby{{\換字{嫌}}}{いや}に
\ruby{治}{をさ}めきつて
\ruby{居}{ゐ}やがつて、そして
\ruby{時々}{とき|〴〵}
\ruby{鶉}{うづら}のやうな
\ruby{藝}{げい}をする
\ruby{奴}{やつ}だなあ。
』

とは
\ruby{甞}{かつ}て
\ruby{筑波}{つく|ば}が
\ruby{爛醉}{らん|すゐ}の
\ruby{後}{のち}に
\ruby{罵}{のゝし}りし
\ruby{語}{ことば}なるが、
\ruby{吉}{よき}に
\ruby{{\換字{遇}}}{あ}ひても
\ruby{齒齦}{は|ぐき}を
\ruby{露}{あら}はして
\ruby{笑}{ゑ}みくつがへる
\ruby{程}{ほど}は
\ruby{悅}{よろこ}ばず、
\ruby{凶}{あしき}に
\ruby{{\換字{遇}}}{あ}ひても
\ruby{眉}{まゆ}を
\ruby{皺}{しわ}めて
\ruby{沈}{しづ}み
\ruby{入}{い}る
\ruby{程}{ほど}は
\ruby{悲}{かなし}まで、
\ruby{何時}{い|つ}も
\ruby{自{\換字{分}}}{じ|ぶん}の
\ruby{顏}{かほ}つきの
\ruby{不齊}{む|ら}の
\ruby{無}{な}いやうにと
\ruby{心}{こゝろ}がけて
\ruby{居}{ゐ}るでも
\ruby{有}{あ}るまじけれど、
\ruby{自然}{おの|づ}と
\ruby{胸}{むね}の
\ruby{中}{うち}のさまを
\ruby{鮮}{あざ}やかに
\ruby{他人}{ひ|と}に
\ruby{讀}{よ}めるやうには
\ruby{面}{かほ}に
\ruby{出}{だ}さぬ
お
\ruby{彤}{とう}も、
\ruby{烟草}{たば|こ}には
\ruby{烟草}{たば|こ}の
\ruby{蟲}{むし}の
\ruby{有}{あ}る
\ruby{{\換字{道}}理}{だう|り}にてや、
\ruby{矢張}{や|は}り
\ruby{或機}{ある|をり}には
\ruby{心}{こゝろ}の
\ruby{悶}{もだえ}をば
\ruby{盡}{こと〴〵}く
\ruby{面}{おもて}に
\ruby{現}{あら}はすなり。

\ruby{何時}{い|つ}の
\ruby{事}{こと}なりけん、
\ruby{一劇場}{ある|しば|ゐ}に% 原文通り「場」
\ruby{西洋{\換字{婦}}人}{せい|やう|ふ|じん}の
\ruby{奇{\換字{術}}}{き|じゆつ}の
\ruby{興行}{こう|ぎやう}の
\ruby{有}{あ}りし
\ruby{時}{とき}、

『
\ruby{姊}{ねえ}さん、
\ruby{大變}{たい|へん}に
\ruby{面白}{おも|しろ}いといふ
\ruby{噂}{うはさ}ですから
\ruby{{\換字{連}}}{つ}れて
\ruby{行}{い}つて
\ruby{見}{み}せて。
』

とお
\ruby{彤}{とう}に
\ruby{{\換字{請}}求}{ね|だ}りけるに、

『
\ruby{觀}{み}たけりやあ
\ruby[<h||]{汝}{おまへ}
\ruby{一人}{ひと|り}で
\ruby{行}{い}つて
\ruby{御覽}{ご|らん}な。
\ruby{{\換字{魔}}{\換字{術}}}{て|づま}は
\ruby{妾}{わたし}あ
\ruby{大{\換字{嫌}}}{だい|きら}ひだよ。
』

と
\ruby{膠}{にべ}も
\ruby{無}{な}く
\ruby{云}{い}はれしより
\ruby{不圖}{ふ|と}
お
\ruby{龍}{りう}は
\ruby{心付}{こゝろ|づ}いて、
\ruby{差當}{さし|あた}り
\ruby{我}{わ}が
\ruby{智慧}{ち|ゑ}にて
\ruby{何共}{なん|とも}
\ruby{解}{わか}らぬ
\ruby{事}{こと}にあへば、
お
\ruby{彤}{とう}は
\ruby{甚}{いた}く
\ruby{面白}{おも|しろ}からず
\ruby{思}{おも}ふと
\ruby{見}{み}えて、
\ruby{必}{かな}らず
\ruby{可厭}{い|や}な
\ruby{可厭}{い|や}な
\ruby{顏}{かほ}して
\ruby{不快}{ふ|くわい}さを
\ruby{示}{しめ}すを
\ruby{知}{し}りぬ。

\ruby{何事}{なに|ごと}の
\ruby{悲}{かな}しくて
お
\ruby{春}{はる}は
\ruby{泣}{な}けるぞや、
\ruby{誰}{たれ}も
\ruby{其}{そ}の
\ruby{故}{ゆゑ}を
\ruby{思}{おも}ひ
\ruby{得}{え}しものは
\ruby{無}{な}けれど、
\ruby{誰}{たれ}もまた
\ruby{其}{そ}の
\ruby{故}{ゆゑ}の
\ruby{{\換字{分}}}{わか}らねばとて
\ruby{何}{なに}と
\ruby{思}{おも}ふも
\ruby{無}{な}きに、
お
\ruby{彤}{とう}は
\ruby{例}{れい}の
\ruby{我}{わ}が
\ruby{合點}{が|てん}の
\ruby{行}{ゆ}かぬといふことをば
\ruby{{\換字{強}}}{つよ}く
\ruby{忌々}{いま|〳〵}しがつて、
\ruby{其}{そ}
の
\ruby{故}{ゆゑ}を
\ruby{解}{と}かんと、
\ruby{苦}{くるし}み
\ruby{悶}{もだ}ゆるなるべし、たゞ
\ruby{轉瞬}{また|たき}するほどの
\ruby{刹那}{せつ|な}の
\ruby{間}{ま}なれど、
\ruby{星}{ほし}のやうなる
\ruby{兩眼}{りやう|がん}をやゝ
\ruby{寄}{よ}せて
\ruby{上眼}{うは|め}づかひしたる
\ruby{其}{そ}の
\ruby{樣子}{やう|す}、
\ruby{何}{なん}とも
\ruby{云}{い}へぬ
\ruby{可厭}{い|や}なところありて、
\ruby{牙彫}{げ|ぼり}の
\ruby{小町}{こ|まち}のやうな
\ruby{申{\換字{分}}無}{まをし|ぶん|な}き
\ruby{眼鼻立}{め|はな|だち}の
\ruby{美}{うつく}しさをも
\ruby{人}{ひと}をして
\ruby{忘}{わす}れ
\ruby{果}{はて}しめたり。

かねて
\ruby{心}{こゝろ}づき
\ruby{居}{ゐ}ればこそ、
お
\ruby{龍}{りう}たゞ
\ruby{一人}{ひと|り}は
お
\ruby{彤}{とう}が
\ruby{其}{そ}の
\ruby{不快}{ふ|くわい}げなる
\ruby{面}{おもて}を
\ruby{爲}{な}したるを
\ruby{早}{はや}くも
\ruby{見}{み}たれ、
\ruby{他}{た}の
\ruby{人々}{ひと|〴〵}は
\ruby{{\換字{更}}}{さら}に
\ruby{氣}{き}の
\ruby{付}{つ}かぬ
\ruby{間}{ま}に、
\ruby{其人}{その|ひと}は
\ruby{復}{また}
\ruby{忽}{たちま}ち
\ruby{舊}{もと}の
\ruby{樣子}{やう|す}になりたり。

お
\ruby{彤}{とう}は
お
\ruby{春}{はる}に
\ruby{復}{ふたゝ}び
\ruby{管}{かま}はず、
お
\ruby{富}{とみ}に
\ruby{命令}{いひ|つけ}くれば
お
\ruby{富}{とみ}は
\ruby{心得}{こゝろ|え}て、
\ruby{人人}{ひと|びと}に
\ruby{茶}{ちや}を
\ruby{侑}{すゝ}め
\ruby{菓子}{くわ|し}を
\ruby{薦}{すゝ}めなどしけるが、
\ruby{其}{そ}の
\ruby{中}{うち}
\ruby{良久}{やゝ|ひさ}しく
お
\ruby{杉}{すぎ}
お
\ruby{春}{はる}は
\ruby{何}{なに}をか
\ruby{語}{かた}りける、やがて
お
\ruby{杉}{すぎ}は
\ruby{次}{つぎ}の
\ruby{間}{ま}に
\ruby{來}{きた}りて
\ruby{打笑}{うち|わら}ひながら、

『お
\ruby{春}{はる}さんの
\ruby{泣}{な}いて
\ruby{居}{を}りましたのは
\ruby{斯樣}{か|う}なのでございますよ。
ほんとに
\ruby{可憐}{か|はい}らしいぢやあございませんか、あの
\ruby{斯樣}{か|う}なのでございます。
お
\ruby{富}{とみ}さんていふ
\ruby{方}{かた}が
\ruby{歸}{かへ}つておいでになれば
\ruby{妾}{わたし}は
お
\ruby{暇}{いとま}になるでしやう。
\ruby{折角}{せつ|かく}こんな
\ruby{好}{い}い
\ruby{御家}{お|うち}へ
\ruby{來合}{き|あは}せたのに、また
\ruby{吾家}{う|ち}へ
\ruby{行}{ゆ}くのかと
\ruby{思}{おも}ふと
\ruby{餘}{あんま}り
\ruby{{\換字{情}}無}{なさけ|な}いので、
\ruby{今}{いま}
\ruby{伺}{うかゞ}つて
\ruby{居}{ゐ}れば
\ruby{結構}{けつ|こう}な
お
\ruby{{\換字{道}}具}{だう|ぐ}を
お
\ruby{富}{とみ}さんていふ
\ruby{方}{かた}が
\ruby{麁怱}{そ|さう}なすつても、
\ruby{器物}{しな|もの}よりやあ
\ruby{人}{ひと}が
\ruby{可愛}{か|はい}いと
\ruby{仰}{おつし}あつて
\ruby{御叱言}{お|こ|ごと}も
\ruby{無}{な}くつて
\ruby{濟}{す}みましたが、
\ruby{其}{そ}の
お
\ruby{優}{やさし}しい
\ruby{御話}{お|はなし}を
\ruby{伺}{うかゞ}つて
\ruby{居}{ゐ}る
\ruby{中}{うち}に
\ruby{妾}{わたし}あ
\ruby{胸}{むね}が
\ruby{痛}{いた}くなつて
\ruby{參}{まゐ}りました。
つい
\ruby{先月}{せん|げつ}の
\ruby{末}{すゑ}、
\ruby{詰}{つま}らない
\ruby{茶飮茶碗}{ちや|のみ|ぢや|わん}
\ruby{一}{ひと}つ
\ruby{妾}{わたし}が
\ruby{麁怱}{そ|さう}して
\ruby{破}{わ}りました
\ruby{時}{とき}は、そりやあ
\ruby{繼母}{まゝ|はゝ}の
\ruby{事}{こと}ですから
\ruby{仕方}{し|かた}も
\ruby{無}{な}いのですけれども、
\ruby{妾}{わたし}あ
\ruby{一時間}{いち|じ|かん}も
\ruby{二時間}{に|じ|かん}も
\ruby{口}{くち}ぎたなく
\ruby{叱}{しか}られました
\ruby{上}{うへ}、
\ruby{{\換字{終}}}{しまひ}にやあ
\ruby{性}{しやう}の
\ruby{付}{つ}くやうにつて
\ruby{火}{ひ}の
\ruby{點}{つ}いて
\ruby{居}{ゐ}る
\ruby{{\換字{煙}}管}{きせ|る}の
\ruby{雁首}{がん|くび}を\換字{志}つと
\ruby{手}{て}の
\ruby{甲}{かふ}に
\ruby{捺}{お}し
\ruby{付}{つ}けられました。
\ruby{今}{いま}の
\ruby{御話}{お|はなし}を
\ruby{伺}{うかゞ}つて
\ruby{居}{ゐ}る
\ruby{中}{うち}に
\ruby{其}{そ}の
\ruby{事}{こと}を
\ruby{思}{おも}ひ
\ruby{出}{だ}しましたら、
\ruby{妾}{わたし}あ
\ruby{猫}{ねこ}になつても
\ruby{宜}{よ}うございますし、
\ruby{御膳}{ご|ぜん}を
\ruby{頂}{いたゞ}かなくつても
\ruby{宜}{よ}うございますから、
\ruby{何樣}{ど|う}か
\ruby{此方}{こち|ら}の
\ruby{御家}{お|うち}の
\ruby{何處}{ど|こ}かの
\ruby{隅}{すみ}へ
\ruby{置}{お}いて
\ruby{頂}{いたゞ}きたい
\ruby{氣}{き}が
\ruby{仕}{し}て……
\ruby{何樣}{ど|う}せ
\ruby{何}{なに}も
\ruby{知}{しり}ませんので
\ruby{御役}{お|やく}には
\ruby{立}{た}ちませんし、
\ruby{無益}{む|だ}ですから、
\ruby{置}{お}いては
\ruby{下}{くだ}さいますまいつて、それでつい、
\ruby{泣}{な}いて
\ruby{仕舞}{し|ま}つたといふのでございます。
ほんとに
\ruby{聞}{き}いて
\ruby{見}{み}ますりやあ
\ruby{繼母}{まゝ|はゝ}だもんですので
\ruby{愍然}{かはい|さう}でございますが、
\ruby{猫}{ねこ}にでもなりたいなんかつて、ホヽヽヽ
\ruby{何}{なん}ぼ
\ruby{何}{なん}でも
\ruby{可笑}{を|か}しうございます。
\ruby{併}{しか}しそれに
\ruby{付}{つ}けてもよく〳〵だと
\ruby{思}{おも}はれます。
』

と
\ruby{告}{つ}げたり。

\ruby{聞}{き}けば
\ruby{何}{なん}でも
\ruby{無}{な}き
\ruby{事}{こと}なるに
お
\ruby{彤}{とう}は
\ruby{晴}{はれ}やかなる
\ruby{面}{おもて}して、

『ホヽヽ、
\ruby{何}{なに}かと
\ruby{思}{おも}つたら
\ruby{其樣}{そ|ん}な
\ruby{事}{こと}なのかえ。
\ruby{愍然}{かは|い}さうに、
\ruby{其樣}{そ|ん}なに
\ruby{居}{ゐ}たがるものなら
\ruby{置}{お}いて
\ruby{{\換字{遣}}}{や}りましやう。
\ruby{怜悧}{り|こう}で、そして
\ruby{毅然}{しつ|かり}としたところがある
\ruby{中々}{なか|〳〵}の
\ruby{好兒}{いゝ|こ}だから。
』

と
\ruby{云}{い}へば、
\ruby{其}{そ}の
\ruby{語}{ことば}を
\ruby{聞}{き}きて
\ruby{物蔭}{もの|かげ}に
\ruby{居}{ゐ}たりし
お
\ruby{春}{はる}は
\ruby{如何}{い|か}ばかり
\ruby{嬉}{うれ}しくや
\ruby{思}{おも}ひけん、
\ruby{誰}{た}が
\ruby{面{\換字{前}}}{ま|へ}に
\ruby{居}{を}るとも
\ruby{無}{な}きところにて
\ruby{唯}{たゞ}
\ruby{主人}{しゆ|じん}の
\ruby{方}{かた}に
\ruby{對}{むか}ひ、
\ruby{疊}{たゝみ}に
\ruby{手}{て}を
\ruby{突}{つ}き
\ruby{頭}{かしら}を
\ruby{下}{さ}げて
\ruby{恩}{おん}を
\ruby{謝}{しや}したり。

\ruby{先刻}{さ|き}より
\ruby{始{\換字{終}}}{し|ゞう}を
\ruby{見聞}{み|き}きせるもの、
お
\ruby{富}{とみ}は
\ruby{云}{い}ふに
\ruby{及}{およ}ばず、
お
\ruby{富}{とみ}の
\ruby{{\換字{父}}}{ちゝ}、
お
\ruby{龍}{りう}の
\ruby{叔母}{を|ば}、
お
\ruby{春}{はる}、
お
\ruby{杉}{すぎ}の
\ruby{末}{すゑ}に
\ruby{至}{いた}るまで、
\ruby{誰}{た}が
\ruby{今}{いま}
\ruby{寛大}{おほ|やう}にして
\ruby{{\換字{情}}}{なさけ}ある
\ruby{此}{こ}の
\ruby{家}{や}の
\ruby{美}{うつく}しき
\ruby{女主人}{ぢよ|しゆ|じん}に
\ruby{心}{こゝろ}を
\ruby{寄}{よ}せざるもの
\ruby{有}{あ}らん。
あはれお
\ruby{彤}{とう}は
\ruby{一}{ひと}つの
\ruby{器}{うつは}を
\ruby{失}{うしな}つて
\ruby{六人}{ろく|にん}の
\ruby{心}{こゝろ}を
\ruby{得}{え}たるなり。

お
\ruby{彤}{とう}も
\ruby{流石}{さす|が}に
\ruby{心樂}{こゝろ|たの}しきなるべし、
\ruby{鶉}{うづら}のやうなる
\ruby{藝}{げい}をすると
\ruby{云}{い}はれし
\ruby{人}{ひと}ながら、
\ruby{例}{れい}の
\ruby{治}{をさ}め
\ruby{切}{き}つたる
\ruby{顏}{かほ}つきの
\ruby{口}{くち}の
\ruby{邊}{あたり}に、
\ruby{見}{み}ゆるか
\ruby{見}{み}えぬほどの
\ruby{誇}{ほこ}りの
\ruby{笑}{わらひ}を
\ruby{含}{ふく}みたり。

\Entry{其三十五}

お
\ruby{彤}{とう}が
\ruby{{\換字{分}}別}{ふん|べつ}に
\ruby{長}{た}けたる
\ruby{事}{こと}は
\ruby[g]{對談}{はなし}の
\ruby{中}{うち}にも
\ruby{知}{し}りしが、
\ruby{今{\換字{又}}眼}{いま|また|ま}のあたりに
\ruby{其}{そ}の
\ruby{胸}{むね}の
\ruby{廣}{ひろ}く
\ruby[g]{慈悲}{なさけ}の
\ruby{厚}{あつ}きをば
\ruby{見}{み}て、
\ruby{隨{\換字{分}}負}{ずゐ|ぶん|ま}けぬ
\ruby{氣}{き}の
お
\ruby{龍}{りう}の
\ruby{叔母}{を|ば}も
\ruby{全}{まつた}く
\ruby{我}{が}を
\ruby{折}{を}り
\ruby{盡}{つく}くして、
\ruby{好}{い}いと
\ruby{思}{おも}ひ
\ruby{込}{こ}めば
\ruby{何處}{ど|こ}までも
\ruby{好}{い}いに
\ruby{仕}{し}て
\ruby{{\換字{終}}}{しま}ふ
\ruby[g]{田舍氣}{ゐなかぎ}の
\ruby[g]{正直三昧}{しやうぢきざんまい}に、
\ruby{此}{こ}の
\ruby{人}{ひと}にさへ
\ruby{頼}{たの}み
\ruby{置}{お}けば
\ruby{何樣}{ど|う}
\ruby{轉}{ころ}んでも
\ruby[g]{間{\換字{違}}}{まちがひ}
\ruby{無}{な}しと
\ruby{盡}{こと〴〵}く
\ruby{信}{しん}じて、
\ruby{何{\換字{分}}宜}{なに|ぶん|よろ}しく
\ruby{願}{ねが}ひまするを
\ruby[g]{百{\換字{遍}}}{ひやくぺん}ほども
\ruby{云}{い}ひたる
\ruby{末}{すゑ}、
\ruby{何事}{なに|ごと}も
お
\ruby{彤}{とう}
\ruby{任}{まか}せにして
\ruby{其次}{その|つぎ}の
\ruby{日}{ひ}に
\ruby{靜岡}{しづ|をか}へ
\ruby{歸}{かへ}りぬ。

『お
\ruby{龍}{りう}ちやん、
お
\ruby{前}{まへ}
\ruby{一寸}{ちよ|つと}
\ruby{今}{いま}までの
\ruby{居處}{う|ち}へ
\ruby{歸}{かへ}つてネ、
\ruby{叔母}{を|ば}のいひつけで
\ruby{今後}{これ|から}これ〳〵のところに
\ruby{居}{ゐ}るやうになつたといふ
\ruby{事}{こと}だけを
\ruby{斷}{ことわ}つておいでな。
』

\ruby{叔母}{を|ば}の
\ruby[g]{歸郷}{かへり}を
\ruby{停車場}{てい|しや|じやう}まで
\ruby{{\換字{送}}}{おく}つての
\ruby{後}{のち}、
\ruby{何}{なに}を
\ruby{思}{おも}ふにや
\ruby{茫然}{ばう|ぜん}として
\ruby{爲}{な}す
\ruby{事}{こと}も
\ruby{無}{な}く
\ruby{居}{ゐ}たる
お
\ruby{龍}{りう}に
\ruby{向}{むか}つて
お
\ruby{彤}{とう}はかくの
\ruby{如}{ごと}く
\ruby{云}{い}ひ
\ruby{出}{いだ}したり。
お
\ruby{龍}{りう}は
\ruby{{\換字{迷}}惑}{めい|わく}さうに
\ruby[g]{眉根}{まゆね}を
\ruby{寄}{よ}せながら、
\ruby{何}{なん}の
\ruby{思案}{し|あん}も
\ruby{無}{な}く、

『
\ruby{行}{い}かなくつちやあいけませんかネ、ネエ
\ruby{行}{い}かなくつちやあ。
』

と、
\ruby{然}{さ}も〳〵
\ruby{其}{そ}の
\ruby{事}{こと}の
\ruby[g]{宥{\換字{免}}}{ゆるし}を
\ruby{乞}{こ}ふが
\ruby{如}{ごと}くに
\ruby{云}{い}へり。

『ホヽヽ、
\ruby{{\換字{嫌}}}{いや}なの?
\ruby{其樣}{そん|な}に。
\ruby{怖}{こは}いやうにでも
\ruby{思}{おも}つて?。
』

『
\ruby{怖}{こは}いつて
\ruby{事}{こと}は
\ruby{有}{あ}りませんけれどもネ、
\ruby{今日}{け|ふ}つから
\ruby{御暇}{お|ひま}を
\ruby{致}{いた}します、
\ruby{左樣}{さ|やう}ならつて
\ruby{云}{い}ふのが
\ruby{何}{なん}だか
\ruby{云}{い}ひづらいやうな
\ruby{心持}{こゝろ|もち}がするんですもの。
』

『だつて
\ruby{何}{なに}も
お
\ruby{前}{まへ}が
\ruby{不義理}{ふ|ぎ|り}なことを
\ruby{爲}{す}るつて
\ruby{云}{い}ふのぢやあ
\ruby{無}{な}し、
お
\ruby{前}{まへ}にも
\ruby{{\換字{分}}}{わか}つて
\ruby{居}{ゐ}るとほり
\ruby[g]{先方}{むかふ}の
お
\ruby{腹}{なか}の
\ruby{中}{なか}が
\ruby{良}{よ}くないんだから、ことわりを
\ruby{云}{い}ふだけの
\ruby{事}{こと}に
\ruby{譯}{わけ}は
\ruby{無}{な}いぢやあ
\ruby{無}{な}いか。
』

『そりやあ、
\ruby{理屈}{り|くつ}は、もうほんとに
\ruby{其{\換字{通}}}{その|とほ}りなんですけれども。
』

『ぢやあ、また、
\ruby{何故}{な|ぜ}ネエ?。
』

『
\ruby{何}{なん}だか
\ruby{妾}{わたし}にも
\ruby{理由}{わ|け}は
\ruby{{\換字{分}}}{わか}りませんけども、
\ruby{妾}{わたし}にやあ
\ruby{{\換字{判}}然}{はつ|きり}と
\ruby{斷}{ことわ}りが
\ruby{云}{い}へさうも
\ruby{無}{な}いんですもの!。
\ruby{心}{しん}はほんとに
\ruby{可厭}{い|や}な
\ruby{人}{ひと}ですけれども、
\ruby{表面}{うは|べ}だけにしろ
お
\ruby{龍}{りう}〳〵つて
\ruby{可愛}{か|はい}がつて
\ruby{{\換字{呉}}}{く}れまして、
\ruby{斯樣}{か|う}やつて
\ruby{衣類}{き|もの}も
\ruby{着}{き}せて
\ruby{{\換字{呉}}}{く}れますし、
\ruby{一個}{ひと|つ}あるものも
\ruby{{\換字{半}}{\換字{分}}}{はん|ぶん}は
\ruby{取}{と}り
\ruby{{\換字{分}}}{わ}けて
\ruby{{\換字{呉}}}{く}れるやうに
\ruby[g]{始{\換字{終}}爲}{しじうさ}れて
\ruby{居}{ゐ}るんですから、いつそ
\ruby{惡口}{あく|たい}でも
\ruby{云}{い}はれて
\ruby{喧嘩}{けん|くわ}でも
\ruby{仕}{し}たら
\ruby{妾}{わたし}の
\ruby{胸}{むね}の
\ruby{中}{なか}を
\ruby{有}{あ}り
\ruby{體}{てい}に
\ruby{云}{い}ひ
\ruby{出}{だ}す
\ruby{事}{こと}も
\ruby{出來}{で|き}るか
\ruby{知}{し}れませんけど、
\ruby{嘘}{うそ}でも
\ruby{優}{やさ}しい
\ruby{顏}{かほ}を
\ruby{仕}{し}て
\ruby{{\換字{呉}}}{く}れて
\ruby{居}{ゐ}るのに
\ruby{對}{むか}つちやあ、
\ruby{其樣}{そ|ん}な
\ruby{譯}{わけ}の
\ruby{有}{あ}る
\ruby{筈}{はず}は
\ruby{毫末}{ちつ|と}も
\ruby{無}{な}いんですが、
\ruby{何}{なん}だか
\ruby{彼家}{あす|こ}を
\ruby{出}{で}やうつて
\ruby{云}{い}ふのが
\ruby[g]{我儘{\換字{過}}}{わがまゝす}ぎる
\ruby[g]{不人{\換字{情}}}{ふにんじやう}のことのやうに
\ruby{思}{おも}はれてならないんですもの。
』

『ホヽヽ、
\ruby{餘}{あんま}り
お
\ruby{前}{まへ}は
\ruby[g]{性{\換字{分}}}{しやうぶん}が
\ruby[g]{美麗}{きれい}なものだから
\ruby{氣}{き}が
\ruby{{\換字{弱}}}{よわ}いねエ。
ぢやあ
\ruby{思}{おも}ひきつて
\ruby{特}{わざ}と
\ruby[g]{冒頭}{のつけ}から
\ruby{喧嘩}{けん|くわ}を
\ruby{仕}{し}たら
\ruby{何樣}{ど|う}だえ。
』

『あら!、
\ruby{姊}{ねえ}さんはまあ
\ruby{甚}{ひど}い
\ruby{事}{こと}ねえ、
\ruby{喧嘩}{けん|くわ}つていふものは
\ruby{自然}{ひと|りで}に
\ruby{出來}{で|き}るものだのに、わざと
\ruby{噴嘩}{けん|くわ}をするなんて、そんな
\ruby{事}{こと}があるの?。
』

『ホヽホヽヽ、あゝ、
\ruby{有}{あ}るともサ。
\ruby{妾}{わたし}なんぞは
\ruby{仕馴}{し|な}れて
\ruby{居}{ゐ}る
\ruby{位}{くらゐ}だよ。
どうだえ、
\ruby{吃驚}{びつ|くり}
お
\ruby{仕}{し}かえ、
\ruby{人}{ひと}が
\ruby{惡}{わる}いだらうネエ。
』

『ホヽヽ、
\ruby{眞實}{ほん|と}かと
\ruby{思}{おも}つて
\ruby{居}{ゐ}たら
\ruby{戯談}{じやう|だん}ばつかり。
』

『イヽエ、
\ruby{戯談}{じやう|だん}ぢやあ
\ruby{無}{な}いよ、
\ruby{一寸}{ちよ|つと}
\ruby{行}{い}つておいでな。
\ruby{一人}{ひと|り}で
\ruby{心細}{こゝろ|ぼそ}いなら
お
\ruby{富}{とみ}を
\ruby{付}{つ}けてあげやうはネ。
\ruby{年}{とし}は
\ruby{行}{ゆ}かないけれども
\ruby{大}{だい}のしつかり
\ruby{者}{もの}だから、
\ruby{彼女}{あ|れ}にすつかり
\ruby{口上}{こう|じやう}を
\ruby{教}{おし}へて
\ruby{{\換字{遣}}}{や}りましやう。
お
\ruby{前}{まへ}が
\ruby{何}{なん}にも
\ruby{云}{い}はなくつても
\ruby{可}{い}いやうに。
』

『まさか
\ruby{妾}{わたし}だつて
お
\ruby{富}{とみ}さんに
\ruby{口上}{こう|じやう}を
\ruby{云}{い}つて
\ruby{貰}{もら}はなくつてもですが、
\ruby{眞實}{ほん|と}に
\ruby{何樣}{ど|う}しても
\ruby{行}{い}かなくつちやあ
\ruby{不可}{いけ|ない}のでしやうか?。
』

\ruby{如何}{い|か}にも
\ruby{苦}{くる}しげに
お
\ruby{龍}{りう}は
\ruby{再}{ふたゝ}び
\ruby{{\換字{尋}}}{たづ}ぬれば、
お
\ruby{彤}{とう}も
\ruby{憐}{あはれ}みて
\ruby{一寸}{ちよ|つと}
\ruby{考}{かんが}しが、

『お
\ruby{待}{ま}ちよ。
それほどお
\ruby{前}{まへ}が
\ruby{困}{こま}るつて
\ruby{云}{い}ふのなら、アヽ
\ruby{可}{い}いよ、
\ruby{仕方}{し|かた}が
\ruby{無}{な}い、
\ruby{手紙}{て|がみ}で
\ruby{云}{い}ふことに
お
\ruby{爲}{し}。
さうしたら
\ruby{向}{むかふ}から
\ruby{足}{あし}を
\ruby{{\換字{運}}}{はこ}んで
\ruby{來}{く}るだらう、どうせ
\ruby{一度}{いち|ど}は
\ruby{膨}{ふく}れつ
\ruby{面}{つら}を
\ruby{持}{も}つて
\ruby{來}{く}るに
\ruby{定}{きま}つて
\ruby{居}{ゐ}るのだから。
』

と
\ruby{負}{ま}けて
\ruby{答}{こた}へぬ。
\ruby{談話}{はな|し}は
\ruby{是}{これ}に
\ruby{{\換字{終}}}{をは}つて
お
\ruby{龍}{りう}は
\ruby{手紙}{て|がみ}を
\ruby{認}{したゝ}めはじめしが、
\ruby[g]{三行書}{さんぎやうか}きては
\ruby{破}{やぶ}り、
\ruby[g]{五行書}{ごぎやうか}きては
\ruby{丸}{まる}め、
\ruby{幾度}{いく|たび}と
\ruby{無}{な}く
\ruby{書}{か}き
\ruby{損}{そん}じたる
\ruby{後}{のち}やうやくと
\ruby[g]{恐惶}{かしく}まで
\ruby{纒}{まと}めて、
\ruby{先}{ま}づ
\ruby{初}{はじめ}に
\ruby{世話}{せ|わ}になりたる
\ruby{恩}{おん}を
\ruby{謝}{しや}し、
\ruby{次}{つぎ}
には
\ruby[g]{田舍氣質}{ゐなかかたぎ}の
\ruby{叔母}{を|ば}の
\ruby{片意地}{かた|い|ぢ}なる
\ruby{指揮}{さし|ず}の
\ruby{負}{そむ}き
\ruby{難}{がた}き
\ruby{由}{よし}を
\ruby{云}{い}ひ、
\ruby{扨其後}{さて|その|のち}に、
\ruby{我}{わ}が
\ruby{意}{こゝろ}よりの
\ruby{事}{こと}ならねども
\ruby{其方}{そち|ら}を
\ruby{離}{はな}れて
\ruby{此家}{こ|ゝ}に
\ruby{{\換字{留}}}{とど}まりあるやうになりたる
\ruby{趣}{おもむ}きを
\ruby{記}{しる}したりけり。

\ruby{如何}{い|か}ばかり
\ruby{文}{ふみ}の
\ruby{言葉}{こと|ば}は
\ruby{優}{やさ}しく
\ruby{書}{か}かれたりとも、
\ruby{吾}{わ}が
\ruby{物}{もの}と
\ruby{思}{おも}ひ
\ruby{込}{こ}みたる
\ruby{禽}{とり}に
\ruby{他家}{よ|そ}の
\ruby[g]{檐端}{のきば}で
\ruby{鳴}{な}かれては
\ruby{堪忍}{が|まん}なり
\ruby{難}{がた}く、
お
\ruby{關}{せき}
は
\ruby{慾}{よく}の
\ruby{算盤}{そろ|ばん}の
\ruby{置{\換字{違}}}{おき|ちが}ひとなりたるに
\ruby[g]{手紙讀}{てがみよ}む
\ruby{眼}{め}の
\ruby{玉}{たま}を
\ruby{頻々}{しき|り}とパチ〳〵させ
\ruby{居}{を}りしが、やがて
\ruby{手紙}{て|がみ}を
\ruby{揉}{も}み
\ruby{丸}{まる}めて
\ruby[g]{投礫}{つぶて}の
\ruby{如}{ごと}く
\ruby{投}{な}げ
\ruby{捨}{す}て、

『
\ruby[g]{彼女}{あいつ}も
\ruby[g]{彼女}{あいつ}だが、
お
\ruby{彤}{とう}つて
\ruby{奴}{やつ}が
\ruby{忌々}{いま|〳〵}しい。
\ruby{誰}{たれ}が
\ruby{指}{ゆび}を
\ruby{啣}{くは}へて
\ruby{引込}{ひつ|こ}む?。
\ruby{人}{ひと}を
\ruby{馬鹿}{ば|か}に
\ruby{仕}{し}あがる!。
』

と
\ruby{男}{をとこ}のやうな
\ruby{言葉}{こと|ば}
\ruby{{\換字{遣}}}{づか}ひして
\ruby{獨}{ひと}り
\ruby{罵}{のゝし}りつ、
\ruby[g]{紫色}{むらさきいろ}になつて
\ruby{怒}{いか}り
\ruby{瞋}{いか}つたり。


\Entry{其三十六}

\原本頁{}%
『えゝ、
%
ぢれつたいネ、
%
\ruby{{\換字{煙}}草}{たば|こ}
\ruby{一}{ひと}つ
\ruby{入}{い}れるのに
\ruby{何}{なに}を
\ruby{其樣}{そん|な}に
\ruby{愚圖愚圖}{ぐ|づ|ぐ|づ}して
\ruby{居}{ゐ}るのだえ。
%
\ruby{百足}{むか|で}に
\ruby{足袋}{た|び}でも
\ruby{穿}{は}かせや
\ruby{仕}{し}まいし、
%
\ruby{宜}{い}い
\ruby{加減}{か|げん}に
\ruby{早{\換字{速}}}{さつ|さ}と
\ruby{仕}{し}て
お
\ruby{吳}{く}れな。
』

\原本頁{}%
\ruby{樫貪聲}{けん|どん|ごゑ}に
\ruby{罵}{のゝし}りながら、
%
\ruby{腹立}{はら|だ}ち
\ruby{{\換字{紛}}}{まぎ}れの
\ruby{力}{ちから}を
\ruby{籠}{こ}めてぎうと
\ruby{吾}{わ}が
\ruby{帶}{おび}を
\ruby{緊}{きつ}く
\ruby{締}{し}め、
%
\ruby{{\換字{猶}}}{なほ}
\ruby{帶揚}{おび|あげ}を
\ruby{締}{し}め、
%
\ruby{帶{\換字{留}}}{おび|どめ}を
\ruby{締}{し}むる
\ruby{時}{とき}、
%
\ruby{小婢}{こ|をんな}の
お
\ruby{熊}{くま}が
\ruby{馴}{な}れぬ
\ruby{手}{て}つきの
たど〳〵しく
\ruby{漸}{やうや}くにして
\ruby{{\換字{煙}}草}{たば|こ}を
\ruby{詰}{つ}めて
\ruby{差}{さ}し
\ruby{出}{いだ}す
\ruby{{\換字{煙}}草袋}{たば|こ|いれ}を
\ruby{引奪}{ひつ|たく}るやうに
\ruby{取}{と}つて
ばた〳〵と
\ruby{拂}{はた}き、

\原本頁{}%
『
\ruby{仕}{し}やうが
\ruby{無}{な}いねえ、
%
\ruby{此樣}{こん|な}に
\ruby{外部}{そ|と}に
\ruby{{\換字{煙}}草}{たば|こ}をくつつけちやあ。
%
まるで
\ruby{毛}{け}が
\ruby{生}{は}えたやうぢやあ
\ruby{無}{な}いか。
%
フツフツフツ。
』

\原本頁{}%
と
\ruby{吹}{ふ}けば、
%
\ruby{{\換字{煙}}草}{たば|こ}の
\ruby{{\換字{粉}}}{こな}は
\ruby{{\換字{空}}}{くう}に
\ruby{飛}{と}び
\ruby{飛}{と}んで、
%
うつかりと
\ruby{仰向}{あふ|む}いて、
%
\ruby{頻}{しき}りに
\ruby{怒}{いか}り
\ruby{立}{た}つ
\ruby{主人}{しゆ|じん}の
\ruby{面}{おもて}を
\ruby{訝}{いぶか}り
\ruby{呆}{あき}れながら
\ruby{視居}{み|ゐ}たりし
お
\ruby{熊}{くま}が
\ruby{小}{ちひ}さき
\ruby{金壺眼}{かな|つぼ|まなこ}にむざんや
\ruby{舞}{ま}ひ
\ruby{入}{い}りたり。

\原本頁{}%
『アツ、
%
アヽ
\ruby{痛}{いた}い!。
%
あんまりだこと!。
』

\原本頁{}%
\ruby{思}{おも}はず
\ruby{叫}{さけ}びて
\ruby{眼}{め}を
\ruby{抑}{おさ}へ、
%
\ruby{泣}{な}きながら
お
\ruby{熊}{くま}の
\ruby{俯伏}{うつ|ぶ}すを、
%
\ruby{愍}{あはれ}み
\ruby{氣}{げ}も
\ruby{無}{な}く
\ruby{見下}{み|おろ}して
\ruby{却}{かへ}つて
\ruby{冷笑}{あざ|わら}ひ、

\原本頁{}%
『
\ruby{下}{くだ}らなく
\ruby{汝}{おまへ}がぼかんと
\ruby{仕}{し}て
\ruby{居}{ゐ}るからだアネ。
%
\ruby{妾}{わたし}の
\ruby{知}{し}つた
\ruby{事}{こと}ぢやあ
\ruby{無}{な}いよ。
%
\ruby{痛}{いた}いつても
\ruby{火}{ひ}が
\ruby{入}{はい}つた
\ruby{程}{ほど}ぢやあ
\ruby{有}{あ}るまいから、
%
\ruby{其樣}{そ|ん}なに
\ruby{泣}{な}く
\ruby{事}{こと}は
\ruby{無}{な}いやネ。
%
さあ
\ruby{下駄}{げ|た}を
\ruby{出}{だ}しておくれ。
%
えゝうぢうぢして
\ruby{居}{ゐ}るネ、
%
\ruby{{\換字{分}}}{わか}らない!、
%
\ruby{跣足}{は|だし}ぢやあ
\ruby{出}{で}られ
\ruby{無}{な}いぢや
\ruby{無}{な}いか。
%
\ruby{一々}{いち|〳〵}
\ruby{此樣}{こ|ん}な
\ruby{事}{こと}までも、
%
ソレ〳〵と
\ruby{云}{い}はれなくつちやあ
\ruby{{\換字{分}}}{わか}らないかえ、
%
\ruby{困}{こま}つた
\ruby{人}{ひと}だネエ。
%
チヨツ、
%
いつまで
\ruby{{\換字{半}}間}{はん|ま}な
\ruby{顏}{かほ}を
\ruby{仕}{し}て
\ruby{泣}{な}いて
\ruby{居}{ゐ}るんだネ、
%
\ruby{鼠色}{ねずみ|いろ}の
\ruby{涙}{なみだ}なんか
\ruby{零}{こぼ}して。
%
\ruby{火傷}{やけ|ど}へ
\ruby{唐辛子味噌}{たう|がら|し|み|そ}をつけられた
\ruby{狸}{たぬき}に
\ruby{其樣}{そ|ん}な
\ruby{顏}{かほ}を
\ruby{仕}{し}て
\ruby{居}{ゐ}るのが
\ruby{有}{あ}つたつけ。
』

\原本頁{}%
と、
%
\ruby{自己}{う|ぬ}が
\ruby{煩悶}{もしや|くしや}の
\ruby{八}{や}ツあたりに
\ruby{口}{くち}ぎたなく
\ruby{叱}{しか}り
\ruby{嘲}{あざけ}れば、
%
\ruby{惡口}{あく|こう}を
\ruby{{\換字{浴}}}{あび}せらるゝには
\ruby{既}{はや}
\ruby{慣}{な}れたる
お
\ruby{熊}{くま}も
\ruby{膨}{ふく}れ
\ruby{{\換字{返}}}{かへ}つて、
%
\ruby{色黑}{いろ|くろ}き
\ruby{小}{ちひさ}き
\ruby{身體}{から|だ}をプリ〳〵とさせつ、
%
いと
\ruby{狹}{せま}き
\ruby{額越}{ひたひ|ご}しに
\ruby{恨}{うら}みの
\ruby{眼}{め}を
\ruby{{\換字{遣}}}{や}りて、
%
\ruby{言葉}{こと|ば}
\ruby{無}{な}くプイと
\ruby{立上}{たち|あが}り、
%
\ruby{疊}{たゝみ}に
\ruby{躓}{つまづ}けるやうに
\ruby{歩}{ある}いて
\ruby{出口}{で|ぐち}の
\ruby{方}{はう}に
\ruby{至}{いた}り、
%
がたりびしりと
\ruby{物音}{もの|おと}
\ruby{荒}{あら}く
\ruby{下駄箱}{げ|た|ばこ}に
\ruby{當}{あた}り
\ruby{散}{ち}らしたり。

\原本頁{}%
『ぢやあ
\ruby{一寸}{ちよ|つと}
\ruby{往}{い}つて
\ruby{來}{く}るから
\ruby{氣}{き}をつけて
\ruby{居}{ゐ}なくちやあ
\ruby{不可}{いけ|ない}よ。
%
オヤ、
%
\ruby{狸}{たぬき}さん、
%
\ruby{怒}{おこ}つて
\ruby{膨}{ふく}れておいでだネ。
%
\ruby{怒}{おこ}つてりやあ
\ruby{睡}{ねむ}くならないから
\ruby{其}{それ}も
\ruby{宜}{い}いだらう。
%
\ruby{{\換字{留}}守番}{る|す|ばん}が
\ruby{性}{しやう}も
\ruby{無}{な}く
\ruby{坐睡}{ゐね|むり}を
\ruby{仕}{し}て、
%
\ruby{魂魄}{たま|しひ}が
\ruby{鼻}{はな}の
\ruby{穴}{あな}から
\ruby{獅子}{し|し}の
\ruby{洞入}{ほら|い}り
\ruby{洞{\換字{還}}}{ほら|がへ}りなんかを
\ruby{仕}{し}て
\ruby{居}{ゐ}られるよりやあ、
%
\ruby{其}{そ}の
\ruby{方}{はう}が
\ruby{優}{まし}らしいから。
%
ハヽヽ、
%
ぢやあ
\ruby{頼}{たの}むよ
\ruby{御{\換字{留}}守番}{お|る|す|ばん}、
%
\ruby{好}{い}い
\ruby{御土產}{お|みや|げ}を
\ruby{買}{か}つて
\ruby{來}{こ}やうネヱ。
』

\原本頁{}%
\ruby{纔}{わづか}に
\ruby{胸}{むね}の
\ruby{中}{なか}の
\ruby{鬱々}{もや|くや}を
\ruby{洩}{もら}すか、
%
\ruby{益}{えき}も
\ruby{無}{な}い
\ruby{惡口}{あく|たい}に
\ruby{目下}{め|した}を
\ruby{嬲}{なぶ}つて
お
\ruby{關}{せき}は
\ruby{出}{い}で
\ruby{去}{さ}れば、
%
\ruby{主}{しゆ}を
\ruby{{\換字{送}}}{おく}り
\ruby{出}{だ}して
\ruby{後}{あと}に
\ruby{殘}{のこ}りし
お
\ruby{熊}{くま}は
\ruby{室}{へや}の
\ruby{眞中}{まん|なか}に
\ruby{取}{と}り
\ruby{散}{ち}らされたる
\ruby{主人}{しゆ|じん}の
\ruby{脫}{ぬぎ}つからしをば
\ruby{片付}{かた|づ}くるとて、
%
\ruby{其}{そ}の
\ruby{片手}{かた|て}に
\ruby{衣紋竹}{え|もん|だけ}を
\ruby{持}{も}ちたれども
\ruby{片手}{かた|て}は
\ruby{{\換字{更}}}{さら}に
\ruby{使}{つか}はで、
%
\ruby{足}{あし}の
\ruby{先}{さき}に
\ruby{幾度}{いく|たび}か
\ruby{衣類}{き|もの}を
\ruby{蹴{\換字{返}}}{け|かへ}し
\ruby{蹴{\換字{返}}}{け|かへ}しつ、
%
\ruby{{\換字{終}}}{つひ}に
\ruby{片手業}{かた|て|わざ}に
\ruby{衣紋竹}{え|もん|だけ}に
\ruby{引掛}{ひつ|か}けて
\ruby{壁}{かべ}に
\ruby{掛}{か}けたりしが、
%
たま〳〵
\ruby{催}{もよほ}したる
\ruby{噴嚏}{くし|やみ}を
\ruby{{\換字{遠}}慮}{ゑん|りよ}も
\ruby{無}{な}く
\ruby{大}{おほ}きくして、

\原本頁{}%
『ハツクシヨーン。
』

\原本頁{}%
と
\ruby{特}{こと}さらに
\ruby{我}{わ}が
\ruby{顏}{かほ}を
\ruby{今}{いま}
\ruby{掛}{か}けたる
\ruby{衣類}{き|もの}の
\ruby{胴}{どう}の
あたりに
\ruby{持}{も}ち
\ruby{行}{ゆ}きつ、
\換字{志}たゝかに
\ruby{汚}{きたな}き
\ruby{唾液}{つば|き}の
\ruby{霧}{きり}を
\ruby{注}{そゝ}ぐが
\ruby{如}{ごと}く
\ruby{噴}{ふ}き
\ruby{掛}{か}けぬ。

\原本頁{}%
\ruby{土瓶}{ど|びん}の
\ruby{底}{そこ}を
\ruby{拔}{ぬ}き、
%
\ruby{桶}{をけ}の
\ruby{箍}{たが}を
はじけさするなど、
%
\ruby{下司}{げ|す}の
\ruby{復讎}{しか|へし}は
\ruby{都}{すべ}て
\ruby{陰}{かげ}でする
\ruby{{\換字{習}}}{なら}ひなれば、
%
それより
お
\ruby{熊}{くま}の
\ruby{{\換字{戸}}棚}{と|だな}
\ruby{捜}{さが}し
\ruby{仕}{し}て、
%
\ruby{白砂糠}{しろ|ざ|たう}を
\ruby{舐}{な}め、
%
\ruby{奈良漬}{なら|づ|け}を
\ruby{荒}{あら}し、
%
\ruby{自己}{お|の}が
\ruby{嗜}{す}きなものは
\ruby{暴}{あば}れ
\ruby{食}{ぐひ}して、
%
\ruby{蓋物}{ふた|もの}の
\ruby{蓋}{ふた}を
\ruby{除}{と}つて
\ruby{自己}{お|の}が
\ruby{好}{す}かぬ
\ruby{鹽辛}{しほ|から}なんぞに
\ruby{{\換字{遇}}}{あ}へば
\ruby{唾液}{つば|き}を
\ruby{仕{\換字{込}}}{し|こん}で
\ruby{掻}{か}き
\ruby{{\換字{廻}}}{まは}し
\ruby{置}{お}くやうの
\ruby{事}{こと}を
\ruby{仕居}{し|ゐ}るとも
\ruby{知}{し}らず、
%
お
\ruby{關}{せき}は
\ruby[<j|]{勢}{いきほひ}
\ruby{{\換字{込}}}{こ}んで
お
\ruby{彤}{とう}が
\ruby{家}{いへ}を
\ruby{{\換字{尋}}}{たづ}ねたり。

\原本頁{}%
\ruby{便利}{べん|り}なる
\ruby{場處}{ば|しよ}の% 原文通り「場」
\ruby{聊}{いさゝ}か
\ruby{引{\換字{退}}}{ひつ|こ}んで
\ruby{靜}{しづか}なるところに、
%
すべて
\ruby{金子}{か|ね}のかかりたる
\ruby{{\換字{造}}}{つく}りの、
%
\ruby{見}{み}るから
\ruby{知}{し}らるゝ
\ruby{其}{そ}の
\ruby{贅澤}{ぜい|たく}さの
\ruby{小憎}{こ|にく}らしき
\ruby{家}{いへ}を、
%
\ruby{此家}{こ|ゝ}と
\ruby{{\換字{尋}}}{たづ}ね
\ruby{得}{え}て
お
\ruby{關}{せき}の
\ruby{訪問}{おと|な}へば、
%
\ruby{折}{をり}から
\ruby{此}{こ}のむづかしい
\ruby{世}{よ}を
\ruby{餘{\換字{所}}}{よ|そ}にして、
%
\ruby{此{\換字{所}}}{こ|ゝ}は
\ruby{日}{ひ}の
\ruby{短}{みじか}い
\ruby{盛}{さか}りをも
\ruby{長}{なが}く
\ruby{暮}{くら}すやうなる
\ruby{長閑}{のど|か}さを
\ruby{現}{あらは}す
\ruby{賑}{にぎ}やかなる
\ruby{手物}{て|もの}の
\ruby{撥音}{ばち|おと}
\ruby{鮮}{あざ}やかに、
%
\ruby{二人}{ふた|り}して
\ruby{彈}{ひ}く
\ruby{絃}{いと}の
\ruby{音}{おと}の
\ruby{冴}{さ}えて、
%
\ruby{然}{さ}も
\ruby{面白}{おも|しろ}げに
\ruby{樓上}{にか|い}あるべく
\ruby{思}{おも}はるゝ
\ruby{奧}{おく}の
\ruby{方}{かた}より
\ruby{洩}{も}れ
\ruby{聞}{きこ}え
\ruby{來}{き}つ、
%
\ruby{婢等}{をんな|ども}も
\ruby{其方}{そ|れ}に
\ruby{耳}{みゝ}や
\ruby{奪}{と}られ
\ruby{居}{ゐ}る、
%
\ruby{御免}{ご|めん}なさい、
%
\ruby{御免}{ご|めん}なさい、
%
と
\ruby{云}{い}へど
\ruby{應}{いら}ふるものも
\ruby{無}{な}く、
%
\ruby{拭}{ふ}いて
\ruby{除}{と}つたやうに
\ruby{奇麗}{き|れい}なる
\ruby{三和土}{た|た|き}の
\ruby{履脫}{くつ|ぬぎ}に
\ruby{良}{やゝ}
\ruby{久}{ひさ}しく
\ruby{立}{た}
たされたり。

\Entry{其三十七}

\ruby{何知}{なに|し}らぬ
\ruby{耳}{みゝ}にも
\ruby{面白}{おも|しろ}きは
\ruby{面白}{おも|しろ}く、
\ruby{{\換字{連}}彈}{つれ|びき}の
\ruby{三昧線}{し|や|み}の
\ruby{音}{おと}、
\ruby{急}{きふ}なる
\ruby{時}{とき}には
\ruby{玉霰銀盤}{あら|れ|ぎん|ばん}を
\ruby{拍}{う}ち、
\ruby{{\換字{緩}}}{ゆる}き
\ruby{時}{とき}には
\ruby{{\換字{寒}}水}{み|ず}せゝらぎに
\ruby{咽}{むせ}んで、
\ruby{一高一低}{いつ|かう|いつ|てい}、
\ruby{一挑一撥}{いつ|たう|いつ|ぱつ}、
\ruby{{\換字{前}}聲}{ぜん|せい}は
\ruby{後聲}{こう|せい}を
\ruby{呼}{よ}び、
\ruby{後聲}{こう|せい}は
\ruby{{\換字{前}}聲}{ぜん|せい}に
\ruby{應}{こた}へて、
\ruby{斷}{た}えつ
\ruby{續}{つゞ}きつする
\ruby{間}{あひだ}に、おのづと
\ruby{人}{ひと}の
\ruby{心}{こゝろ}を
\ruby{攝}{と}り
\ruby{去}{さ}れば、
\ruby{彼}{かれ}は
\ruby{何}{なん}といふ
\ruby{曲}{きよく}ぞとも
\ruby{知}{し}らぬ
お
\ruby{春}{はる}さへ
\ruby{聞}{きゝ}
\ruby{惚}{ほ}れて、
\ruby{身}{み}はこゝに
\ruby{在}{あ}りながら
\ruby{思}{おもひ}を
\ruby{彼方}{かな|た}の
\ruby{樓上}{ろう|じよう}に
\ruby{馳}{は}せて、たゞ
\ruby{恍然}{うつ|とり}と
\ruby{我}{われ}を
\ruby{忘}{わす}れたる
\ruby{折}{をり}しも、
\ruby{怒}{いか}るが
\ruby{如}{ごと}く
\ruby{罵}{のゝし}るが
\ruby{如}{ごと}き
\ruby{案内}{あん|ない}
\ruby{乞}{こ}ふ
\ruby{聲}{こゑ}を
\ruby{聞}{き}きつけて、
\ruby{吃驚}{びつ|くり}して
\ruby{我}{われ}に
\ruby{復}{かへ}り、
\ruby{周章}{あ|は}てゝ
\ruby{立出}{たち|い}で
\ruby{見}{み}れば、
\ruby{衣服}{み|なり}こそ
\ruby{見苦}{み|ぐる}しくはあらね、
\ruby{五十{\換字{近}}}{ご|じう|ちか}き
\ruby{女}{をんな}の、たゞさへ
\ruby{下品}{げ|ひん}に
\ruby{肥}{ふと}りたる
\ruby{{\換字{平}}顏}{ひら|がほ}を、
\ruby{目}{め}に
\ruby{見}{み}ゆるほど
\ruby{膨}{ふく}らませきつたる
\ruby{不機{\換字{嫌}}}{ふ|きげ|ん}の
\ruby{氣色}{け|しき}
\ruby{怖}{おそ}ろしく、
\ruby{{\換字{嫌}}味}{いや|み}らしく
\ruby{細}{ほそ}く
\ruby{剃}{す}りつけたるをかしき
\ruby{眉}{まゆ}を
\ruby{擧}{あ}げ、
\ruby{白睛}{しろ|め}の
\ruby{赤濁}{あか|にご}りせる
\ruby{汚}{きたな}き
\ruby{眼}{め}の
\ruby{小}{ちひさ}きに
\ruby{稜立}{かど|た}てゝ、『
\ruby{此}{こ}の
\ruby{小}{こ}びつちよめが』と
\ruby{云}{い}はぬばかりに
\ruby{頭}{あたま}から
\ruby{見下}{み|おろ}し、その
\ruby{言葉}{こと|ば}つきも
\ruby{憎}{にく}らしく
\ruby{刺々}{とげ|〳〵}しく、

『お
\ruby{龍}{りう}に
\ruby{然樣}{さ|う}
\ruby{云}{い}つて
\ruby{下}{くだ}さい、
\ruby{本銀町}{しろ|がね|ちやう}から
\ruby{來}{き}ましたと。
ハイ、
\ruby{然樣}{さ|う}
\ruby{云}{い}つて
\ruby{下}{くだ}さればそれで
\ruby{{\換字{分}}}{わか}るのですから。
\ruby{居不在}{ゐ|る|す}なんぞは
\ruby{使}{つか}はせませんよ。
それあの
\ruby{上調子}{うは|でう|し}を
\ruby{付}{つ}けて
\ruby{居}{ゐ}る{---}{---}
\ruby{彼}{あれ}は
\ruby{屹度}{きつ|と}
お
\ruby{龍}{りう}に
\ruby{定}{きま}つてるんですからネ。
』

と
\ruby{無{\換字{遠}}慮}{ぶ|ゑん|りよ}にも
\ruby{程度}{ほ|ど}のあるに、
\ruby{不在}{る|す}を
\ruby{使}{つか}はれやうかとの
\ruby{先潛}{さき|くゞ}りまでして、% 【潛 u6f5b 「先先」】【潜 u6f5c 「夫夫」】併用されている
\ruby{撥音}{ばち|おと}を
\ruby{聞}{き}いて
\ruby{其}{そ}の
\ruby{人}{ひと}を
\ruby{猜}{すゐ}することの
\ruby{出來}{で|き}るものやら
\ruby{出來}{で|き}ぬものやら
\ruby{知}{し}らねど、
\ruby{拔}{ぬ}けさせぬつもりからの
\ruby{當推}{あて|ずゐ}に、
\ruby{{\換字{硝}}子箱}{びい|どろ|ばこ}の
\ruby{中}{なか}のものを
\ruby{見}{み}でも
\ruby{仕}{し}たやうに
\ruby{確}{たしか}に
\ruby{其}{それ}と
\ruby{指}{さ}して
\ruby{云}{い}ひたきまゝを
\ruby{云}{い}ひたり。

\ruby{其}{そ}の
\ruby{慳貪}{けん|どん}さ、
\ruby{其}{そ}の
\ruby{無作法}{ぶ|さ| ふ}さ、% 「ぶさはふ」ではなく原本通り「は」抜き(空白置き換え)
\ruby{其}{そ}の
\ruby{{\換字{尊}}大}{おほ|ふう}さ、その
\ruby{下作}{げ|さく}さに、
\ruby{優}{やさし}しき
お
\ruby{春}{はる}は
\ruby{驚}{おどろ}き
\ruby{呆}{あき}れつ、
\ruby{一寸}{いつ|すん}の
\ruby{蟲}{むし}にも
\ruby{五{\換字{分}}}{ご|ぶ}の
\ruby{魂魄}{たま|しひ}あれば、
\ruby{胸}{むね}の
\ruby{中}{うち}には
\ruby{可厭}{い|や}な〳〵
\ruby{人}{ひと}と
\ruby{侮蔑}{さげ|す}みながら、

『お
\ruby{待}{ま}ち
\ruby{下}{くだ}さいまし、
\ruby{然樣}{さ|う}
\ruby{申}{まを}しますから。
』

と
\ruby{冷}{ひや}やかに
\ruby{答}{こた}へて
\ruby{徐々}{しづ|か}に
\ruby{身}{み}を
\ruby{起}{おこ}し、
\ruby{奧深}{おく|ふか}なる
\ruby{樓上}{にか|い}に
\ruby{至}{いた}りたり。

\ruby{見}{み}れば
\ruby{主人}{ある|じ}の
お
\ruby{彤}{とう}は
\ruby{常}{つね}の
\ruby{如}{ごと}く
\ruby{沈着}{おち|つ}きたる
\ruby{面}{かほ}の
\ruby{色}{いろ}、
\ruby{逼}{せま}らず
\ruby{急}{せ}かず、たゞ
\ruby{白}{しろ}く、
\ruby{下品}{げ|ひん}の
\ruby{人}{ひと}を
\ruby{今}{いま}
\ruby{見}{み}たる
\ruby{目}{め}には
\ruby{宛}{あだか}も
\ruby{女雛}{め|びな}なんどを
\ruby{見}{み}る
\ruby{如}{ごと}く
\ruby{上品}{じやう|ひん}に
\ruby{見}{み}え、
お
\ruby{龍}{りう}はまた
\ruby{思}{おも}はず
\ruby{知}{し}らず
\ruby{興}{きよう}に
\ruby{乘}{の}り
\ruby{心}{こゝろ}をはずませて
\ruby{我}{われ}おもしろく
\ruby{彈}{ひ}くと
\ruby{思}{おぼ}しく、
\ruby{汗}{あせ}ばむといふほどにはあらねど
\ruby{氣勢}{いき|ほひ}
\ruby{{\換字{込}}}{こ}みたる
\ruby{面色}{かほ|つき}やゝ
\ruby{紅色}{くれ|なゐ}さして
\ruby{美}{うつく}しく
\ruby{見}{み}えしが、
\ruby{主人}{しゆ|じん}は
\ruby{我}{わ}が
\ruby{方}{かた}を
\ruby{見}{み}も
\ruby{{\換字{返}}}{かへ}らねど
お
\ruby{龍}{りう}は
\ruby{活々}{いき|〳〵}としたる
\ruby{眼}{め}にちらりと
\ruby{此方}{こ|なた}を
\ruby{見}{み}しまゝ、たゞ
\ruby{一心}{いつ|しん}に
\ruby{彈}{ひ}きつゞけたり。
\ruby{{\換字{遠}}}{とほ}く
\ruby{聞}{き}きしにだに
\ruby{賑}{にぎ}やかなりしを、
\ruby{{\換字{近}}}{ちか}く
\ruby{聞}{き}けば
\ruby{{\換字{又}}}{また}
\ruby{一}{ひ}ト
\ruby{層}{しほ}おもしろき
\ruby{絃}{いと}の
\ruby{色音}{いろ|ね}の、
\ruby{或}{あるひ}は
\ruby{{\換字{強}}}{つよ}く
\ruby{撥}{ひ}き
\ruby{或}{あるひ}は
\ruby{輕}{かろ}く
\ruby{挑}{すく}ひ
\ruby{或}{あるひ}は
\ruby{彈}{はじ}く
\ruby{彼絃}{か|れ}の
\ruby{餘韵}{よ|ゐん}
\ruby{未}{いま}だ
\ruby{{\換字{消}}}{き}えずして
\ruby{此絃}{こ|れ}の
\ruby{響}{ひゞき}
\ruby{新}{あらた}に
\ruby{起}{おこ}る
\ruby{音}{おと}と
\ruby{音}{おと}とは、
\ruby{一條}{いち|でう}の
\ruby{玉}{たま}の
\ruby{{\換字{鎖}}}{くさり}の
\ruby{環}{わ}と
\ruby{環}{わ}と
\ruby{相{\換字{連}}}{あひ|つらな}り、
\ruby{一聯}{いち|れん}の
\ruby{花輪}{はな|わ}の
\ruby{花}{はな}と
\ruby{花}{はな}と
\ruby{相}{あひ}
\ruby{襲}{かさな}なりて、いづくに
\ruby{斷目}{きれ|め}も
\ruby{見}{み}えざるが
\ruby{如}{ごと}くなれば、
\ruby{言}{ことば}を
\ruby{出}{いだ}さん
\ruby{機會}{し|ほ}を
\ruby{知}{し}らずして、
\ruby{困}{こま}り〳〵て
\ruby{躊躇}{ちう|ちよ}しけるが、いつまでかくては
\ruby{濟}{す}まじと
お
\ruby{龍}{りう}の
\ruby{傍}{かたへ}にやゝ
\ruby{{\換字{近}}}{ちか}づきて、

『お
\ruby{龍}{りう}さん、あの、
\ruby{本銀町}{しろ|がね|ちやう}からまゐりましたつて
\ruby{何}{なん}だか
\ruby{可厭}{い|や}な
\ruby{人}{ひと}でございますが、
\ruby{五十}{ご|じう}ばかりの
お
\ruby{方}{かた}が……。
』

と
\ruby{云}{い}へば
お
\ruby{龍}{りう}はそれと
\ruby{聞}{き}いて、
\ruby{彈}{ひ}く
\ruby{手}{て}は
\ruby{止}{と}めざれども
\ruby{眼}{め}は
お
\ruby{彤}{とう}の
\ruby{方}{かた}を
\ruby{見}{み}て、
\ruby{許可}{ゆる|し}をさへ
\ruby{得}{え}ば
\ruby{直}{すぐ}にも
\ruby{立}{た}つて
\ruby{下}{した}に
\ruby{行}{ゆ}かん
\ruby{素振}{そ|ぶり}をあらはしたり。
お
\ruby{彤}{とう}はこれを
\ruby{見}{み}て
お
\ruby{龍}{りう}には
\ruby{答}{こた}へず、
\ruby{居}{ゐ}るか
\ruby{居}{ゐ}ぬか
\ruby{知}{し}れざるやうに
\ruby{先刻}{さ|き}より
\ruby{我}{わ}が
\ruby{後}{うしろ}の
\ruby{隅}{すみ}にかしこまりて
\ruby{控}{ひか}へ
\ruby{居}{ゐ}し
お
\ruby{富}{とみ}を
\ruby{一寸}{ちよ|つと}
\ruby{見}{み}れば、
お
\ruby{富}{とみ}は
\ruby{早}{はや}くも
\ruby{其}{そ}の
\ruby{意}{い}を
\ruby{悟}{さと}りて、
お
\ruby{春}{はる}の
\ruby{袂}{たもと}を
\ruby{引}{ひ}きに
\ruby{引}{ひ}きて
\ruby{樓下}{し|た}に
\ruby{去}{さ}りぬ。

『
\ruby{何}{なに}?、
お
\ruby{富}{とみ}さん、
\ruby{無理}{む|り}に
\ruby{妾}{わたし}の
\ruby{袂}{たもと}を
\ruby{引}{ひ}ぱつて。
』

\ruby{解}{かい}し
\ruby{得}{え}ぬ
お
\ruby{春}{はる}の
\ruby{訝}{いぶか}り
\ruby{問}{と}ふを
お
\ruby{富}{とみ}は
\ruby{冷笑}{あざ|わら}つて、

『
\ruby{何}{なに}ぢやあ
\ruby{有}{あ}りませんは、
\ruby{下}{くだ}らないよ、
お
\ruby{{\換字{前}}}{まへ}さんは。
あゝやつて
\ruby{{\換字{遊}}}{あそ}んで
\ruby{居}{ゐ}らつしやる
\ruby{最中}{さい|ちゆう}に
\ruby{下}{くだ}らない
\ruby{事}{こと}なんぞ
\ruby{云}{い}つて
\ruby{行}{ゆ}くのだもの。
\ruby{御邪{\換字{魔}}}{お|じや|ま}になるぢやあ
\ruby{無}{な}いかネ、
\ruby{何}{なん}でも
\ruby{自{\換字{分}}}{じ|ぶん}の
\ruby{仕}{し}て
\ruby{居}{ゐ}らつしやる
\ruby{事}{こと}の
\ruby{腰}{こし}を
\ruby{折}{を}られたりなんぞするのは
\ruby{大{\換字{嫌}}}{だい|きら}ひの
\ruby{御方}{お|かた}なんだからネ。
もう
\ruby{今}{いま}
お
\ruby{龍}{りう}さんが
\ruby{立}{た}たうとなすつたゞけで
\ruby{餘程}{よつ|ぽど}
\ruby{可厭}{い|や}に
\ruby{思}{おもつ}て
\ruby{居}{ゐ}らつしやるのだよ。
\ruby{何樣}{ど|う}して、そりやあ〳〵
\ruby{御行屆}{お|ゆき|とゞ}きなさる
\ruby{方}{かた}だけに
\ruby{恐}{おそ}ろしい
\ruby{高慢}{かう|まん}の
\ruby{{\換字{強}}}{つよ}い
\ruby{御氣象}{ご|き|しやう}なんだからネ。
\ruby{人}{ひと}が
\ruby{來}{き}たら
\ruby{待}{ま}たして
\ruby{置}{お}いて
お
\ruby{濟}{す}みになつた
\ruby{時}{とき}
\ruby{申}{まを}し
\ruby{上}{あ}げさへすりやあそれで、
\ruby{宜}{い}いぢやあ
\ruby{無}{な}いかえ。
こんどから
\ruby{氣}{き}を
\ruby{付}{つ}けないと、
\ruby{馬鹿}{ば|か}だといつて
\ruby{御笑}{お|わら}ひになるよ。
』

と
\ruby{自己}{おの|れ}も
\ruby{一度}{いち|ど}は
\ruby{笑}{わら}はれたる
\ruby{事}{こと}のあるなるべし、
\ruby{姊}{あね}ぶつて
\ruby{敎}{をし}へたり。

『
\ruby{然樣}{さ|う}、だつて
\ruby{何}{なん}だかぶり〳〵
\ruby{怒}{おこ}つて
\ruby{居}{ゐ}る、やかましい
\ruby{事}{こと}でも
\ruby{云}{い}ひさうな
\ruby{權幕}{けん|まく}の
\ruby{人}{ひと}が
\ruby{來}{き}たんですもの。
』

と、
\ruby{負惜}{まけ|をし}み
\ruby{氣味}{ぎ|み}に
\ruby{辯解}{いひ|わけ}を、% 弁 瓣 辦 辧 辨 辩 (辯)
\ruby{試}{こゝろ}みるを、

『
\ruby{何}{なん}だえ、やかましいことでも
\ruby{云}{い}ひさうな
\ruby{人}{ひと}だつて。
ヘエー。
ナニ、
\ruby{何樣}{ど|ん}な
\ruby{人}{ひと}だつて
\ruby{關}{かま}ふことがあるもんかネ、
\ruby{下}{くだ}らない!。
\ruby{妾等}{わたし|たち}あ
\ruby{御主人樣}{ご|しゆ|じん|さま}の
\ruby{御氣}{お|き}に
\ruby{入}{い}るやうにさへ
\ruby{爲}{す}りやあ
\ruby{宜}{い}いぢやあ
\ruby{無}{な}いか。
ぢやあ
\ruby{妾}{わたし}が
\ruby{待}{ま}つて
\ruby{居}{ゐ}ろつて、
\ruby{待}{ま}たせて
\ruby{置}{お}いて
\ruby{{\換字{遣}}}{や}りましやう。
』

と
\ruby{此}{これ}は
\ruby{{\換字{飽}}}{あく}まで
\ruby{姊}{あね}ぶつて
\ruby{入口}{いり|くち}の
\ruby{方}{かた}に
\ruby{行}{ゆ}きたり、
\ruby{樓上}{ろう|じよう}の
\ruby{絃聲}{げん|せい}は
\ruby{盛}{さか}んに
\ruby{續}{つゞ}けり。

\Entry{其三十八}

\ruby{同}{おな}じ
\ruby{身分}{み|ぶん}ながらも
\ruby{新參}{しん|ざん}だけに
\ruby{我}{わ}が
\ruby{下}{した}につけるお
\ruby{春}{はる}に
\ruby{對}{むか}ひては、
\ruby[g]{神經質}{むしもち}の
\ruby{本性}{ほん|しやう}を
\ruby{露}{あらは}して
\ruby{偶然}{ふ|と}したる
\ruby{氣}{き}の
\ruby{向}{む}き
\ruby{方}{かた}のはずみにかゝり、
\ruby{意地}{い|ぢ}でも
\ruby{惡}{わる}き
\ruby{人}{ひと}のやうに、つけ〳〵と
\ruby{思}{おも}ふまゝを
\ruby{自己}{お|の}が
\ruby{心}{こゝろ}の
\ruby{{\換字{廻}}}{まは}るに
\ruby{任}{まか}せて、
\ruby{年齡}{と|し}の
\ruby{十歳}{と|を}も
\ruby{{\換字{違}}}{ちが}ふほど
\ruby{大人}{おと|な}ぶりて
\ruby{{\換字{銳}}}{するど}くも
\ruby{言}{ものい}へ、
\ruby{根}{ね}が
\ruby{粗豪}{あ|ら}からぬ
\ruby{氣象}{き|しやう}の
\ruby{心細}{こゝろ|ぼそ}かければ、
\ruby{客}{きやく}に
\ruby{對}{むか}ひては
\ruby{打}{う}つて
\ruby{變}{かは}つて、
\ruby{顏色}{かほ|つき}も
\ruby{恭}{うや〳〵}しく
\ruby[g]{言葉}{ことば}も
\ruby{慇懃}{いん|ぎん}に、

『さあ
\ruby{何樣}{ど|う}かまあ
\ruby[g]{此方}{こちら}へ
\ruby{御上}{お|あが}りなさいまして、』

と
\ruby{入口近}{いり|くち|ちか}き
\ruby{一}{ひ}ㇳ
\ruby{室}{ま}に
\ruby{通}{とほ}して、
\ruby{會}{あ}ふとも
\ruby{會}{あ}はぬとも
\ruby{其}{そ}の
\ruby{挨拶}{あい|さつ}は
\ruby{云}{い}はず、
\ruby{待}{ま}てと
\ruby{特更}{こと|さら}には
\ruby{告}{つ}げず
\ruby{默}{だま}つて
\ruby{待}{ま}たせ
\ruby{置}{お}き、
\ruby{物}{もの}の
\ruby{値}{ね}でも
\ruby{定}{ふ}むやうに
\ruby{室}{へや}の
\ruby{中}{うち}をきよろ〳〵
\ruby{眼}{め}に
\ruby{見囘}{み|まは}す
\ruby{客}{きやく}を
\ruby{其儘殘}{その|まゝ|のこ}して
\ruby{身}{み}は
\ruby{蔭}{かげ}に
\ruby{退}{しりぞ}き、

『ほんたうにお
\ruby{春}{はる}さん、
\ruby{何}{なん}だか
\ruby{可厭}{い|や}な
\ruby{人}{ひと}ネエ。
でも
\ruby{宜}{い}いは、お
\ruby{茶}{ちや}と
\ruby{火}{ひ}とだけ
\ruby{與}{や}つて
\ruby{置}{お}いて、
\ruby{默}{だま}つて
\ruby[g]{引{\換字{込}}}{ひつこ}んでさへ
\ruby{居}{ゐ}りやあ、それで
\ruby{濟}{す}むのだもの。
\ruby{關}{かま}ふことは
\ruby{有}{あ}りやあ
\ruby{仕}{し}ませんは、
\ruby{柔軟}{やは|らか}にあしらつて、そして
\ruby{無言}{だん|まり}でさへ
\ruby{居}{ゐ}りやあ。
\ruby{妾}{わたし}あ
\ruby[g]{彼方}{あちら}で
\ruby{御用}{ご|よう}があるか
\ruby{知}{し}れないから……』

と
\ruby{云}{い}ひさして
\ruby[g]{既樓}{はやにかい}の
\ruby{方}{かた}へ
\ruby{去}{さ}れば、お
\ruby{春}{はる}は
\ruby[g]{言葉}{ことば}の
\ruby{如}{ごと}く
\ruby{唯謹}{ただ|つゝし}みて
\ruby{火}{ひ}を
\ruby{{\換字{運}}}{はこ}び
\ruby{茶}{ちや}を
\ruby{{\換字{運}}}{はこ}べり。

お
\ruby{富}{とみ}が
\ruby[g]{樓}{にかい}へ
\ruby{上}{あが}りたる
\ruby{時}{とき}は
\ruby{曲}{きよく}は
\ruby{既{\換字{終}}}{はや|をは}りに
\ruby{{\換字{近}}}{ちか}く、やがて
\ruby[g]{二人}{ふたり}は
\ruby{彈}{ひ}き
\ruby{仕舞}{し|ま}ひけるが、お
\ruby{彤}{とう}は
\ruby{此}{こ}の
\ruby{時}{とき}はじめて
\ruby{莞爾}{にこ|り}としてお
\ruby{龍}{りう}を
\ruby{見{\換字{遣}}}{み|や}りつ、

『
\ruby{面白}{おも|しろ}かつたこと!
\ruby{久}{ひさ}しぶりで
\ruby[g]{二人}{ふたり}で
\ruby{彈}{ひ}いたので、だが
\ruby{妾}{わたし}\ %空白有り
\ruby{樂}{らく}ぢやあ
\ruby{無}{な}かつたの、たまに
\ruby{彈}{ひ}いたんだから。
』

と、
\ruby{何處}{ど|こ}に
\ruby{人}{ひと}が
\ruby{來}{き}て
\ruby{待}{ま}つて
\ruby{居}{ゐ}るかも
\ruby{知}{し}らぬやうに、
\ruby{悠然}{ゆつ|たり}と
\ruby{云}{い}へ、

『あら
\ruby{嘘}{うそ}ばつかり、
\ruby{妾}{わたし}こそ
\ruby{姊}{ねえ}さんと
\ruby{彈}{ひ}くと
\ruby{氣}{き}が
\ruby{詰}{つ}まるやうな
\ruby{氣}{き}が
\ruby{仕}{し}て
\ruby{樂}{らく}ぢやあ
\ruby{無}{な}いの!。
\ruby{姊}{ねえ}さんは
\ruby{餘}{あんま}り
\ruby[g]{奇麗}{きれい}に、そして
\ruby{餘}{あんま}りきつかり〳〵に
\ruby[g]{几帳面}{きちやうめん}にお
\ruby{彈}{ひ}きなさるんですもの!。
』

と、お
\ruby{龍}{りう}も
\ruby{是非無}{ぜ|ひ|な}く
\ruby{受答}{うけ|こた}へは
\ruby{仕}{し}て
\ruby{居}{ゐ}れど、
\ruby{此}{これ}は
\ruby{來客}{らい|きやく}の
\ruby{聊}{いさゝ}か
\ruby{早}{はや}く、お
\ruby{彤}{とう}は
\ruby{今}{いま}しもお
\ruby{富}{とみ}が
\ruby{薦}{すゝ}むる
\ruby{一碗}{いち|わん}の
\ruby{茶}{ちや}を
\ruby{然}{さ}も
\ruby[g]{心好}{こゝろよ}げに
\ruby{飮}{の}み
\ruby{味}{あぢ}はふにも
\ruby{似}{に}ず、
\ruby{此}{これ}は
\ruby{茶碗}{ちや|わん}を
\ruby{手}{て}に
\ruby{取}{と}り
\ruby{上}{あ}ぐる
\ruby{事}{こと}だに
\ruby{爲}{な}さざるなり。

『
\ruby{然樣}{さ|う}ネエ、どうも
\ruby{妾}{わたし}の
\ruby{彈}{ひ}き
\ruby{方}{かた}は
\ruby{器械}{き|かい}かなんかが
\ruby{動}{うご}く
\ruby{樣}{やう}で、
\ruby{味}{あじ}が
\ruby{無}{な}くつていけないよ。
\ruby{詰}{つま}り
\ruby{{\換字{習}}}{なら}つて
\ruby{記}{おぼ}えたつて
\ruby{云}{い}ふつ
\ruby{限}{き}りの
\ruby{技}{わざ}で、ほんたうは
\ruby{藝事}{げい|ごと}の
\ruby{出來}{で|き}るつて
\ruby{云}{い}ふ
\ruby{人}{ひと}の
\ruby{性質}{た|ち}ぢやあ
\ruby{無}{な}いのだネ。
お
\ruby{前}{まへ}はまた
\ruby{大變}{たい|へん}に
\ruby[g]{出來不出來}{できふでき}がお
\ruby{有}{あ}りのやうだけれど、
\ruby{今日}{け|ふ}のやうに
\ruby[g]{機勢}{はずみ}に
\ruby{乘}{の}つてお
\ruby{彈}{ひ}きのときは、ほんとに
\ruby{憎}{にく}らしい
\ruby[g]{位見事}{くらゐみごと}に
\ruby{御出來}{お|で|き}だよ。

\ruby{詰}{つま}りお
\ruby{前}{まへ}のは、
\ruby{何樣}{ど|う}かした
\ruby{時}{とき}にやあ、おぼえたつて
\ruby{云}{い}ふつ
\ruby{限}{き}りの
\ruby{技}{わざ}ぢやあ
\ruby{無}{な}いものが
\ruby{何處}{ど|こ}からか
\ruby{知}{し}らないが
\ruby{出}{で}て
\ruby{來}{く}るんだネエ。
\ruby{生}{うま}れついて
\ruby{藝}{げい}の
\ruby{味}{あじ}といふものを
\ruby{有}{も}つておいでなんだよ。
』

『なあに、
\ruby{然樣}{さ|う}ぢやあ
\ruby{無}{な}いんですけれどもネ、
\ruby[g]{一人}{ひとり}でなんか
\ruby{彈}{ひ}くと、
\ruby{妾}{わたし}あつまーらないと
\ruby{思}{おも}つて
\ruby{彈}{ひ}く
\ruby{時}{とき}が
\ruby{多}{おほ}いんですがネ、
\ruby{姊}{ねえ}さんと
\ruby{彈}{ひ}いたりなんぞすると、
\ruby{何樣}{ど|う}かすると
\ruby{不思議}{ふ|し|ぎ}に
\ruby{自分}{じ|ぶん}でも
\ruby{面白}{おも|しろ}くなつて
\ruby{來}{く}ることがあるんですの、そして
\ruby{然樣}{さ|う}いふ
\ruby{時}{とき}は
\ruby[g]{屹度自分}{きつとじぶん}の
\ruby{思}{おも}ふやうに
\ruby[g]{自然}{ひとりで}に
\ruby{彈}{ひ}けるんですよ。
やつばり
\ruby{一生懸命}{いつ|しやう|けん|めい}になるからなんでしやうかネエ?。
』

『ホヽヽヽ、
\ruby{一生懸命}{いつ|しやう|けん|めい}になりやあ
\ruby{巧}{うま}く
\ruby{彈}{ひ}けるけれども、
\ruby{然樣}{さ|う}でない
\ruby{時}{とき}あ
\ruby{彈}{ひ}けないつて
\ruby{云}{い}ふんぢやあ、ぢやあお
\ruby{前}{まへ}は
\ruby[g]{横着者見}{わうちやくものみ}たやうだ
\ruby{事}{こと}ネエ。
』

『ホヽヽヽ、
\ruby[g]{屹度然樣}{きつとさう}なんかも
\ruby{知}{し}れませんよ。
でも
\ruby{妾}{わたし}あ
\ruby{故}{わざ}と
\ruby{然樣}{さ|う}やるのぢやあ
\ruby{無}{な}くつて、
\ruby[g]{自然}{ひとりで}に
\ruby{生}{うま}れついて
\ruby{居}{ゐ}る
\ruby[g]{横着者}{わうちやくもの}なんでしやうから……』

『
\ruby{惡}{わる}い
\ruby[g]{横着者}{わうちやくもの}ぢやあ
\ruby{有}{あ}るまいとお
\ruby{云}{い}ひの?。
』

『ホヽホヽホヽ、』

『ホヽホヽホヽ、マア
\ruby{蟲}{むし}が
\ruby{宜}{い}いネエ。
』

『ホヽホヽ、
\ruby{美}{い}い
\ruby[g]{横着者}{わうちやくもの}でも
\ruby{惡}{わる}い
\ruby[g]{横着者}{わうちやくもの}でも
\ruby{其}{そ}りやあ
\ruby{關}{かま}ひませんが、
\ruby{樓下}{し|た}の
\ruby{彼}{あ}の
\ruby{人}{ひと}が
\ruby{待}{ま}つて
\ruby{居}{ゐ}ましやうから……』

\ruby{斯}{か}く
\ruby{云}{い}ひて
\ruby{立}{た}たんとするお
\ruby{龍}{りう}を
\ruby{抑止}{と|ど}めて

『
\ruby{宜}{い}いよ、お
\ruby{前}{まへ}はまあ
\ruby{此室}{こ|ゝ}においで。
\ruby{妾}{わたし}が
\ruby{會}{あ}つて
\ruby{談}{はなし}を
\ruby{仕}{し}て
\ruby{仕舞}{し|ま}ふから。
』

とお
\ruby{彤}{とう}はやをら
\ruby{身}{み}を
\ruby{起}{おこ}したり。


\Entry{其三十九}

こゝに
\ruby{居}{ゐ}よと
\ruby{云}{い}はれては
\ruby{{\換字{逆}}}{さか}らふべくもあらねば、
お
\ruby{龍}{りう}は
\ruby{殘}{のこ}り
\ruby{止}{とど}まりて
\ruby{三昧線}{さ|み|せん}の
\ruby{絃}{いと}を
\ruby{戾}{もど}し
\ruby{{\換字{緩}}}{ゆる}めなど
\ruby{仕}{し}ながらも、
\ruby{我}{わ}が
\ruby{上}{うへ}に
\ruby{就}{つ}きて
\ruby{來}{きた}れる
\ruby{彼}{か}の
お
\ruby{關}{せき}が
\ruby{事}{こと}の
\ruby{氣}{き}になりてならねば、そこら
\ruby{取片付}{とり|かた|づ}くる
お
\ruby{富}{とみ}をば
\ruby{一寸}{ちよ|つと}
\ruby{視}{み}て、

『お
\ruby{春}{はる}さんの
\ruby{云}{い}つたやうに、ほんとに
\ruby{怒}{おこ}つて
\ruby{居}{ゐ}て?。
』

と
\ruby{問}{と}へば、
お
\ruby{富}{とみ}はさも〳〵
\ruby{其}{そ}の
\ruby{人}{ひと}を
\ruby{厭}{いと}ひ
\ruby{{\換字{嫌}}}{きら}ふといふやうに、さらでも
\ruby{淋}{さび}しき
\ruby{顏}{かほ}を
\ruby{妙}{めう}に
\ruby{皺}{しわ}めて、

『ほんとに
\ruby{恐}{おそ}ろしくぶり〳〵して
\ruby{居}{ゐ}ますの!。
まるで
\ruby{御酒}{ご|しゆ}にでも
\ruby{醉}{よ}つた
\ruby{人}{ひと}のやうな
\ruby{顏}{かほ}を
\ruby{仕}{し}まして、』

と
\ruby{先}{ま}づ
\ruby{答}{こた}へつ、

『
\ruby{何}{なん}だか
\ruby{自{\換字{分}}{\換字{勝}}手}{じ|ぶん|かつ|て}の
\ruby{不理屈}{ふ|り|くつ}でも
\ruby{云}{い}ひさうな
\ruby{可厭}{い|や}な
\ruby{人}{ひと}ですことネエ。
』

と
\ruby{添}{そ}へたり。

『マア
\ruby{可厭}{い|や}だことネエ!。
そんなやうに
\ruby{見}{み}えるほど
\ruby{恐}{おそ}ろしい
\ruby{怒}{おこ}つた
\ruby{顏}{かほ}を
\ruby{仕}{し}て
\ruby{居}{ゐ}て?。
』

『
\ruby{然樣}{さ|う}なんですよ、
\ruby{怒}{おこ}り
\ruby{切}{き}つて
\ruby{居}{ゐ}るといふ
\ruby{顏}{かほ}つきなんです。
それに
\ruby{一體}{いつ|たい}が
\ruby{地腫}{ぢ|ばれ}の
\ruby{仕}{し}たやうな
\ruby{顏}{かほ}なんでしやうかネエ、
\ruby{隨{\換字{分}}}{ずゐ|ぶん}おそろしく
\ruby{膨}{ふく}れかへつて、
\ruby{宛然}{とん|と}……』

『
\ruby{宛然何}{とん|と|なん}なの?。
\ruby{自{\換字{分}}}{じ|ぶん}でばかり
\ruby{承知}{しよ|うち}して
\ruby{笑}{わら}つて。
』

『マア
\ruby{止}{よ}して
\ruby{置}{お}きましやう
\ruby{他人樣}{ひと|さ|ま}の
\ruby{惡口}{わる|くち}なんか。
』

『ホヽヽをかしな
\ruby{人}{ひと}ネエ、
\ruby{一人}{ひと|り}で
\ruby{合點}{が|てん}して
\ruby{一人}{ひと|り}で
\ruby{可笑}{を|か}しがつたりなんかして。
』

『ホヽヽ、でも
\ruby{惡}{わる}うございますもの。
』

\ruby{宛然}{とん|と}
\ruby{河豚}{ふ|ぐ}が
\ruby{五合}{ご|がふ}も
\ruby{引掛}{ひつ|か}けたやうと
\ruby{云}{い}はんと
\ruby{仕}{し}たりし
\ruby{歟}{か}、
\ruby{風{\換字{船}}玉}{ふう|せん|だま}に
\ruby{眼鼻}{め|はな}を
\ruby{付}{つ}けたやうと
\ruby{云}{い}はんと
\ruby{仕}{し}たりし
\ruby{歟}{か}、
\ruby{{\換字{終}}}{つひ}に
\ruby{口}{くち}を
\ruby{啓}{ひら}かねば
\ruby{知}{し}るものは
\ruby{當人}{たう|にん}の
\ruby{胸}{むね}のみ。

『マア
\ruby{勘忍}{か|に}して
\ruby{置}{お}いて
\ruby{頂戴}{ちやう|だい}よ。
』

と
\ruby{輕}{かろ}く
\ruby{謝}{わ}びて
\ruby{根問}{ね|どひ}さるゝを
\ruby{{\換字{遮}}}{さへぎ}り
\ruby{止}{とど}めつ
\ruby{樓下}{し|た}に
\ruby{去}{さ}りたり。

\ruby{人去}{ひと|さ}つて
\ruby{小樓靜}{せう|ろう|しづか}に、
\ruby{刳拔}{くり|ぬき}の
\ruby{桐}{きり}の
\ruby{手爐}{てあ|ぶり}の
\ruby{小}{ちひさ}なるを
\ruby{擁}{よう}して、
\ruby{雪}{ゆき}と
\ruby{白}{しろ}き
\ruby{蠣灰}{かき|ばひ}に
\ruby{纖}{ほそ}き
\ruby{火箸}{ひ|ばし}もて
\ruby{譯}{わけ}も
\ruby{無}{な}く
\ruby{假名文字}{か|な|も|じ}を
\ruby{書}{か}きては
\ruby{{\換字{消}}}{け}し
\ruby{書}{か}きては
\ruby{{\換字{消}}}{け}しつ、
お
\ruby{龍}{りう}はじつと
\ruby{心一筋}{こゝろ|ひと|すじ}に
\ruby{彼方}{かな|た}の
\ruby{談話}{はな|し}の
\ruby{何}{なん}となり
\ruby{行}{ゆ}くかを
\ruby{想}{おも}ひやりつゝ、

『
\ruby{彼}{あ}の
\ruby{{\換字{勝}}手}{かつ|て}の
\ruby{{\換字{強}}}{つよ}い
\ruby{慾}{よく}の
\ruby{深}{ふか}い
お
\ruby{師匠}{し|よ}さんがまあ
\ruby{何樣}{ど|ん}な
\ruby{事}{こと}を
お
\ruby{云}{い}ひのだらう。
そりやあもう
\ruby{智慧}{ち|ゑ}も
\ruby{{\換字{分}}別}{ふん|べつ}も
\ruby{確固}{しつ|かり}としておいでゞ、
\ruby{而}{さう}して
\ruby{言語}{もの|いひ}だつて
\ruby{拙}{まづ}い
\ruby{事}{こと}なんぞは
お
\ruby{云}{い}ひで
\ruby{無}{な}い
\ruby{姊}{ねえ}さんの
\ruby{事}{こと}だから、
\ruby{何}{なに}を
\ruby{對手}{むか|ふ}で
\ruby{云}{い}つたつて
\ruby{譯}{わけ}も
\ruby{無}{な}く
\ruby{捌}{さば}いて
お
\ruby{仕舞}{し|ま}ひなさるには
\ruby{{\換字{違}}}{ちが}ひ
\ruby{無}{な}からうが、
\ruby{對手}{あひ|て}が
\ruby{無茶}{む|ちや}な
\ruby{人}{ひと}なだけに
\ruby{御困}{お|こま}りなさりは
\ruby{仕}{し}まいか
\ruby{知}{し}らん。
\ruby{自{\換字{分}}}{じ|ぶん}の
\ruby{{\換字{勝}}手}{かつ|て}づくに
\ruby{掛}{か}けちやあ
\ruby{理合}{り|あひ}や
\ruby{{\換字{情}}合}{じやう|あひ}に
\ruby{構}{かま}つて
\ruby{居}{ゐ}る
\ruby{樣}{やう}な
\ruby{其樣}{そ|ん}な
\ruby{上品}{じやう|ひん}な
\ruby{人}{ひと}ぢやあ
\ruby{無}{な}さゝうな
\ruby{彼}{あ}の
\ruby{人}{ひと}を
\ruby{對手}{あひ|て}にして、くだらない
\ruby{惡口}{あく|たい}や
\ruby{無理}{む|り}な
\ruby{難題}{なん|だい}でも
\ruby{云}{い}はれて
\ruby{困}{こま}つておいでゞは
\ruby{有}{あ}るまいか
\ruby{知}{し}ら。
\ruby{對手}{む|かう}が
\ruby{無茶}{む|ちや}な
\ruby{人}{ひと}でさへ
\ruby{無}{な}ければ
\ruby{宜}{よ}いのだけれども、
\ruby{男}{をとこ}にでも
\ruby{何}{なん}でも
\ruby{負}{ま}けては
\ruby{居}{ゐ}ない
\ruby{樣}{やう}な
\ruby{氣}{き}の
\ruby{{\換字{強}}}{つよ}い
\ruby{人}{ひと}ではあるし、また
\ruby{大變}{たい|へん}に
\ruby{怒}{おこ}り
\ruby{立}{た}つて
\ruby{來}{き}たのだとはいふし、
\ruby{一體}{いつ|たい}が
\ruby{{\換字{勝}}手}{かつ|て}のひどい
\ruby{甚}{ひど}い
\ruby{人}{ひと}だから、いくら
\ruby{姊樣}{ねえ|さん}が
\ruby{冷悧}{り|こう}でも
\ruby{扱}{あつか}ひ
\ruby{難}{にく}いかと
\ruby{思}{おも}はれるが、まあどんな
\ruby{事}{こと}を
\ruby{云}{い}つて
\ruby{來}{き}たもので
\ruby{有}{あ}らう。
\ruby{若}{も}し
\ruby{下}{くだ}らない
\ruby{事}{こと}を
\ruby{云}{い}つて
\ruby{哦鳴}{が|な}り
\ruby{立}{た}てでもされた
\ruby{日}{ひ}には、ほんとに
\ruby{姊}{ねえ}さんに
お
\ruby{氣}{き}の
\ruby{毒}{どく}で、
\ruby{妾}{わたし}はまあ
\ruby{何樣}{ど|う}したら
\ruby{宜}{よ}からう。
\ruby{何樣}{ど|う}か
\ruby{彼}{あ}の
\ruby{人}{ひと}が
\ruby{姊}{ねえ}さんの
\ruby{理解}{り|かい}に
\ruby{折}{を}れ
\ruby{吳}{く}れゝば
\ruby{宜}{い}いが、いくら
\ruby{姊}{ねえ}さんでも
\ruby{對手}{あひ|て}が
\ruby{惡}{わる}いから、
\ruby{何}{なん}だか
\ruby{覺束無}{おぼ|つか|な}いやうな
\ruby{氣}{き}が
\ruby{仕}{し}てならない。
あゝ
\ruby{氣}{き}の
\ruby{揉}{も}める。
\ruby{一體}{いつ|たい}まあ
\ruby{今日}{け|ふ}の
\ruby{談}{はなし}は
\ruby{何樣}{ど|う}
\ruby{結局}{をさ|まり}がついて、そして
\ruby{妾}{わたし}はまあこれから
\ruby{{\換字{前}}{\換字{途}}何樣}{さ|き|ど|う}なつて
\ruby{行}{ゆ}く
\ruby{身}{み}なのだらう。
』

と
\ruby{取}{と}り
\ruby{止}{と}まらず
\ruby{物}{もの}を
\ruby{案}{あん}じて
\ruby{耳}{みゝ}は
\ruby{彼方}{かな|た}にのみ
\ruby{走}{はし}れど、
\ruby{距離隔}{あは|ひ|へだ}てたれば
\ruby{音}{おと}も
\ruby{聞}{きこ}えず、
\ruby{人}{ひと}もあらぬが
\ruby{如}{ごと}く
\ruby{此家靜}{この|いへ|しづか}なり。

やゝ
\ruby{久}{ひさし}くして
\ruby{階段}{はし|ご}を
\ruby{上}{のぼ}り
\ruby{來}{く}る
\ruby{人}{ひと}の
\ruby{跫音}{あし|おと}し、やがて
お
\ruby{春}{はる}は
\ruby{襖}{ふすま}を
\ruby{開}{ひら}きて
\ruby{面}{おもて}を
\ruby{出}{いだ}せば、

『
\ruby{妾}{わたし}に
\ruby{來}{こ}いつて、』

とお
\ruby{龍}{りう}は
\ruby{此方}{こな|た}より
\ruby{問}{と}ひかけたり。

『ハイ、
\ruby{左樣仰}{さ|う|おつし}あいましたので。
』

\ruby{今}{いま}さら
\ruby{胸}{むね}のだくつくやうおぼえて、
\ruby{話}{ばなし}の
\ruby{模樣}{も|やう}を
\ruby{測}{はか}りかねつ、
お
\ruby{龍}{りう}は
\ruby{却}{かへ}つて
\ruby{頓}{とみ}には
\ruby{起}{た}たざりけり。


\Entry{其四十}

\原本頁{}%
\ruby{我}{わ}が
\ruby{眼}{め}の
\ruby{力}{ちから}の
\ruby{及}{およ}ばぬ
\ruby{闇}{やみ}の
\ruby{夜}{よ}に
\ruby{歩}{あし}の
\ruby{{\換字{進}}}{すゝ}まぬやうに、
%
お
\ruby{龍}{りう}は
\ruby{鬼胎}{おそ|れ}を
\ruby{懷}{いだ}きながら
\ruby{室}{へや}に
\ruby{入}{い}りて
\ruby{見}{み}れば、
%
\ruby{{\換字{朝}}日}{あさ|ひ}の
\ruby{光}{ひか}りのあるところ
\ruby{自然}{おの|づ}と
\ruby{心{\換字{強}}}{こゝろ|づよ}きやうの
\ruby{感}{おもひ}の
\ruby{仕}{し}て、
%
\ruby{先}{ま}づ
お
\ruby{彤}{とう}が
\ruby{{\換字{平}}常}{つ|ね}にも
\ruby{增}{ま}して
\ruby{位}{くらゐ}を
\ruby{取}{と}つて
\ruby{沈着}{おち|つ}き
\ruby{切}{き}つたる
\ruby{面}{おもて}の
\ruby{上}{うへ}に、
%
\ruby{掛}{かゝ}れる
\ruby{雲}{くも}の
\ruby{影}{かげ}だに
\ruby{無}{な}き
\ruby{樣}{さま}なるに
\ruby{氣}{き}も
\原本頁{226}%
\ruby{勇}{いさ}み
\ruby{立}{た}ち、
%
\ruby{其}{そ}の
\ruby{横手}{よこ|て}の
\ruby{方}{かた}に、
%
やゝ
\ruby{下}{さが}りて
\ruby{坐}{すわ}りつ、
%
いろ〳〵の
\ruby{思}{おもひ}に
\ruby{小波}{さざ|なみ}の
\ruby{{\換字{文}}立}{あや|た}つ
\ruby{胸}{むね}を
\ruby{鎭}{しづ}めて、
%
\ruby{言葉}{こと|ば}は
\ruby{無}{な}けれど
\ruby{叮嚀}{てい|ねい}に
\ruby{挨拶}{あい|さつ}したり。

\原本頁{}%
ちらりと
\ruby{見}{み}し
お
\ruby{關}{せき}が
\ruby{顏色}{かほ|いろ}の、
%
お
\ruby{春}{はる}
お
\ruby{富}{とみ}が
\ruby{言葉}{こと|ば}とは
\ruby{{\換字{違}}}{ちが}ひて、
%
\ruby{思}{おも}ひのほか
\ruby{{\換字{平}}穩}{おだ|やか}なるやうなるに、
%
\ruby{心}{こゝろ}ひそかに
\ruby{疑}{うたが}ひながら
\ruby{徐}{しづか}に
\ruby{頭}{かしら}を
\ruby{擡}{あ}ぐれば、
%
これはまた
\ruby{如何}{い|か}なることぞや
お
\ruby{關}{せき}は
\ruby{滿面}{まん|めん}に
\ruby{春}{はる}を
\ruby{湛}{たゝ}へて、
%
さも〳〵
\ruby{親}{した}しげに
\ruby{{\換字{又}}}{また}
\ruby{懷}{なつ}かしげに、

\原本頁{226-7}%
『マア
\ruby{立派}{りつ|ぱ}におなりなこと!、
%
\ruby{吃驚}{びつ|くり}して
\ruby{仕舞}{し|ま}つたよ。
%
\ruby{少}{すこ}し
\ruby{粹}{いき}だけれども
\ruby{全然}{まる|で}
\ruby{如是}{こ|れ}ぢやあ
\ruby{立派}{りつ|ぱ}な
\ruby{御邸}{お|やしき}の
お
\ruby[<-||]{孃}{ぢやう}
\ruby{樣}{さま}だよ。
%
\ruby{好}{い}いことネエ、
%
お
\ruby{龍}{りう}ちやんは
\ruby{大變}{たい|へん}な
\ruby{幸福}{しあ|はせ}を%「幸福」ここは「は」
\ruby{御仕}{お|し}ねエ。
%
ほんとに
マア〳〵
\ruby{見{\換字{違}}}{み|ちが}へて
\ruby{仕舞}{し|ま}ふよ。
%
\ruby{{\換字{平}}常}{ふだ|ん}でさへ
\ruby{斯樣}{か|う}ぢやあ
\ruby{外}{そと}へでも
お
\ruby{出}{で}の
\ruby{時}{とき}は
マア
\ruby{何樣}{ど|ん}なに、
%
\ruby{見事}{み|ごと}に
お
\ruby{仕}{し}だらう!。
%
\原本頁{226-11}%
ほんとにお
\ruby{{\換字{前}}}{まへ}さんは
マア
\ruby{大變}{たい|へん}な
\ruby{幸福}{しあ|はせ}な%「幸福」ここは「は」
\ruby{身}{み}におなりネエ。
%
\原本頁{227}%
\ruby{妾}{わたし}の
\ruby{處}{ところ}なんぞに
\ruby{御在}{お|いで}でごらん、
%
\ruby{何程}{いく|ら}
\ruby{妾}{わたし}が
やきもき
\ruby{思}{おも}つて
\ruby{好{\換字{遇}}}{よ|く}してあげたからつて、
%
\ruby{精々}{せい|〴〵}
\ruby{外出衣}{よ|そ|いき}が
\ruby{銘仙}{めい|せん}か
\ruby{{\換字{節}}糸}{ふし|いと}% 玉繭からとった節の多い絹糸。玉糸。
\ruby{位}{ぐらゐ}の
\ruby{物}{もの}で、
%
それより
\ruby{上}{うへ}あ
\ruby{妾}{わたし}が
\ruby{千圓}{せん|ゑん}の
\ruby{籤}{くじ}にでも
\ruby{中}{あた}つたら
\ruby{知}{し}らないこと、
%
まあ〳〵
お
\ruby{{\換字{前}}}{まへ}さんに
\ruby{御召縮緬}{お|め|し|}なんか
\ruby{引張}{ひつ|ぱ}らせてあげることあ
\ruby{出來}{で|き}つこは
\ruby{有}{あ}りやあ
\ruby{仕}{し}ないのに、
%
お
\ruby{正月}{しやう|がつ}でも
\ruby{無}{な}けりやあ
お
\ruby{{\換字{節}}句}{せつ|く}でも
\ruby{無}{な}い
\ruby{日}{ひ}に、
%
\ruby{然樣}{さ|う}いふ
\ruby{衣服}{な|り}を
\ruby{仕}{し}て
お
\ruby{在}{いで}のやうに
おなりたあ、
%
\ruby{眞實}{ほん|と}に
マア
お
\ruby{{\換字{前}}}{まへ}さんは
\ruby{大變}{たい|へん}な
\ruby{幸福}{しあ|はせ}ネエ。%「幸福」ここは「は」
%
それもこれも
\ruby{悉皆}{みん|な}
\ruby{此方樣}{こち|ら|さま}の
お
\ruby{庇蔭}{か|げ}で、
%
\ruby{私等}{わたし|ら}の
\ruby{働}{はたら}きや
お
\ruby{{\換字{前}}}{まへ}さんの
\ruby{力}{ちから}なんぞからぢやあ、
%
\ruby{皺鉾立}{しやつち|よこ|だち}を
\ruby{仕}{し}たつて
\ruby{出來}{で|き}るこつちやあ
\ruby{有}{あ}りませんよ。
%
だから
\ruby{眞實}{ほん|と}に
\ruby{仇}{あだ}や
\ruby{疎略}{おろ|そか}に% 096-3-05-其五.tex では 疎畧(おろ|そか) とある
\ruby{思}{おも}つちやあ
\ruby{濟}{す}みませんよ、
%
\ruby{何}{なん}でも
\ruby{此方樣}{こち|ら|さま}の
\ruby{仰}{おつし}あり
\ruby{次第}{し|だい}に
\ruby{身}{み}を
\ruby{{\換字{粉}}}{こ}にしても
\ruby{働}{はたら}か
\ruby{無}{な}くつちやあ
\ruby{濟}{す}みませんよ。
%
\原本頁{228}%
\ruby{{\換字{若}}}{も}し
お
\ruby{{\換字{前}}}{まへ}さんの
\ruby{仕方}{し|かた}に
そで
\ruby{無}{な}いことでも
\ruby{有}{あ}らうもんなら、
%
\ruby{此方樣}{こち|ら|さま}ぢやあ
\ruby{容赦}{うつ|ちあ}つて
お
\ruby{置}{お}きなすつても
\ruby{私}{わたくし}が
\ruby{承知}{しよう|ち}しや
\ruby{仕無}{し|な}い
\ruby{心算}{つも|り}で
\ruby{居}{ゐ}るからネ。
%
\ruby{屹度}{きつ|と}
\ruby{妾}{わたし}が
\ruby{出}{で}て
\ruby{來}{き}て
お
\ruby{{\換字{前}}}{まへ}さんを
\ruby{折檻}{せつ|かん}すると
\ruby{御思}{お|おも}ひよ。
%
ハヽホヽハヽヽ、
%
オヤマア
\ruby{此}{これ}あ
\ruby{下}{くだ}らないことを
\ruby{云}{い}つたものだネエ、
%
お
\ruby{龍}{りう}ちやんが
\ruby{如在}{じよ|ざい}でも
\ruby{有}{あ}る
\ruby{人}{ひと}のやうに!。
%
ハヽハ、だが、
%
ただ
\ruby{此}{これ}あ
\ruby{其程}{それ|ほど}までに
\ruby{私}{わたし}あ
\ruby{此方樣}{こち|ら|さま}を
お
\ruby{{\換字{前}}}{まへ}さんに
\ruby{取}{と}つちやあ
\ruby{有}{あ}りがたいと
\ruby{思}{おも}つてるといふ
\ruby{心持}{こゝろ|もち}を
\ruby{打撒}{ぶち|ま}けたばかりなんさ。
%
ほんとに
\ruby{戲談}{じやう|だん}ぢやあ
\ruby{有}{あ}りませんよ、
%
\ruby{身}{み}に
\ruby{染}{し}みて
\ruby{有}{あ}り
\ruby{{\換字{難}}}{がた}いと
\ruby{思}{おも}はなくつちやあ
\ruby{罰}{ばち}が
\ruby{當}{あた}りますよ。
%
\ruby{妾}{わたし}もネエ、
%
お
\ruby{{\換字{前}}}{まへ}さんから
\ruby{緣}{えん}を
\ruby{牽}{ひ}いた
お
\ruby{蔭}{かげ}でもつてネエ、
%
\ruby{此方樣}{こち|ら|さま}のやうな
\ruby{結構}{けつ|こう}な
\ruby{方}{かた}にも
お
\ruby{目}{め}にかかつたり、
%
それから
\ruby{{\換字{又}}}{また}
\ruby{種々}{いろ|〳〵}
\ruby{優}{やさ}しく
\ruby{仰}{おつし}あつて
\ruby{戴}{いたゞ}いたりなんかして、
%
\ruby{此樣}{こ|ん}な
\ruby{嬉}{うれ}しいことは
\ruby{有}{あ}りませんのですよ。
%
\ruby{何樣}{ど|う}かネエ
お
\ruby{{\換字{前}}}{まへ}さんからも
\ruby{能}{よう}く
\ruby{御禮}{お|れい}を
\ruby{申}{まを}してネ、
%
そしてネ、
%
\ruby{今後}{これ|から}も
\ruby{時々}{とき|〴〵}は
\ruby{御邪{\換字{魔}}}{お|じや|ま}でも
\ruby{御出入}{お|で|いり}をさせて
\ruby{戴}{いたゞ}くやうにネ、
%
\ruby{何樣}{ど|う}か
お
\ruby{{\換字{前}}}{まへ}さんからも
\ruby{能}{よう}く
\ruby{願}{ねが}つて
\ruby{下}{くだ}さいよ。
%
そして
\ruby{妾}{わたし}あ
\ruby{{\換字{又}}}{また}
お
\ruby{{\換字{前}}}{まへ}さんに
\ruby{一}{ひと}つ
\ruby{御願}{お|ねがひ}があるのだがネ。
%
ナアニ
\ruby{面倒}{めん|だう}な
\ruby{事}{こと}でも
\ruby{何}{なん}でも
\ruby{無}{な}いんで、
%
ただ
\ruby{今度}{こん|ど}
\ruby{他}{よそ}へ
\ruby{出}{で}る
\ruby{時}{とき}
\ruby{一寸}{ちよ|いと}
\ruby{囘}{まは}り% 原本通り「囘」
\ruby{{\換字{道}}}{みち}を
\ruby{仕}{し}てネ、
%
\ruby{汚}{きたな}くつても
\ruby{妾}{わたし}の
\ruby{宅}{うち}へ
\ruby{寄}{よ}つて
\ruby{御茶}{お|ちや}の
\ruby{一}{ひと}つも
\ruby{飮}{の}んで
\ruby{行}{い}つて
\ruby{貰}{もら}ひたいのさ。
%
ただ
もう、
%
お
\ruby{{\換字{前}}}{まへ}さんが
\ruby{如是}{こん|な}に
\ruby{立派}{りつ|ぱ}におなりだといふことを
\ruby{誰}{だれ}か
\ruby{知}{し}らに
\ruby{見}{み}せて、
%
\ruby{私}{わたくし}が
\ruby{腹一杯}{はら|いつ|ぱい}に
\ruby{天狗}{てん|ぐ}を
\ruby{云}{い}つて
\ruby{威張}{ゐ|ばり}たいんだから。
%
ア、
%
それから
\ruby{{\換字{又}}}{また}、
%
\ruby{此樣}{こ|ん}なに
\ruby{何不足}{なに|ふ|そく}ない
\ruby{結構}{けつ|こう}なところへ
\ruby{御}{お}いでのだから、
%
\ruby{何}{なに}も
\ruby{彼}{か}も
\ruby{要}{い}ることは
\ruby{御有}{お|あ}りぢや
\ruby{無}{な}からうがネエ、
%
\ruby{私}{わたくし}のところに
お
\ruby{{\換字{前}}}{まへ}さんの
こざ〳〵した
\ruby{物}{もの}や
\ruby{何}{なん}かが
そつくり
\ruby{仕}{し}て
\ruby{居}{ゐ}る、
%
\ruby{彼品}{あ|れ}は
\ruby{悉皆}{みん|な}
\ruby{明日}{あし|た}にでも
\ruby{持}{も}たして
\ruby{{\換字{遺}}}{よこ}しますからネ。
』

\原本頁{230}%
と、
%
\ruby{{\換字{追}}從}{つゐ|しよう}やら
\ruby{諛辭}{せ|じ}やらを
\ruby{混滯}{ごた|まぜ}に、
%
\ruby{叮嚀}{てい|ねい}と
\ruby{粗略}{ぞん|ざい}との
\ruby{虎斑}{とら|ぶち}の
\ruby{言葉}{こと|ば}
\ruby{{\換字{遣}}}{づか}ひに、
%
\ruby{何}{なに}か
\ruby{知}{し}らず
\ruby{無上}{む|しやう}に
\ruby{機{\換字{嫌}}}{き|げん}
\ruby{好}{よ}く
\ruby{饒舌}{しや|べ}り
\ruby{立}{た}てられ、
%
お
\ruby{龍}{りう}は
ただただ
\ruby{{\換字{煙}}}{けむ}に
\ruby{卷}{ま}かれて、
%
すべてが
\ruby{我}{わ}が
\ruby{思}{おもひ}のほかなりしに
\ruby{{\換字{返}}辭}{へん|じ}にさへ
\ruby{{\換字{迷}}}{まど}ひつゝ、
%
\ruby{如何}{い|か}に
\ruby{應對}{あし|ら}ひて
\ruby{如是}{か|く}は
\ruby{虎}{とら}のやうなるべき
お
\ruby{關}{せき}をば、
%
\ruby{甘}{あま}へて
\ruby{戲}{ざ}るゝ
\ruby{猫}{ねこ}のやうには
\ruby{仕}{し}たりしかと、
%
\ruby{不審}{いぶ|かし}さに
\ruby{堪}{た}へぬ
\ruby{眼}{め}を
\ruby{張}{は}つて
お
\ruby{彤}{とう}を
\ruby{見}{み}たり。

\Entry{其四十一}

\原本頁{}
\ruby{尾}{を}もあらば
\ruby{振}{ふ}つて
\ruby{見}{み}すべき
\ruby{程悅}{ほど|よろこ}びかへつて、
%
お
\ruby{關}{せき}はおのが
\ruby{賤}{いや}しき
\ruby{詞}{ことば}の
\ruby{端々}{はし|〴〵}に
\ruby{下卑}{げ|び}たる
\ruby{心}{こゝろ}の
\ruby{隈々}{くま|〴〵}を
\ruby{殘}{のこ}りなく
\ruby{露}{あらは}すをも
\ruby{顧}{かへり}みず、
%
\ruby{知}{し}ら
\ruby{知}{じ}らしきまで
お
\ruby{彤}{とう}
お
\ruby{龍}{りう}に
\ruby{諛辭}{おは|むき}の
\ruby{數々}{かず|〴〵}を
\ruby{云}{い}ひ
\ruby{盡}{つく}したる
\ruby{後}{のち}、
%
あまり
\ruby{長居}{なが|ゐ}して
\ruby{愛想}{あい|そ}をつかされてはと
\ruby{思}{おも}ひてか、
%
\ruby{但}{たゞ}しは
お
\ruby{彤}{とう}が
\ruby{餘}{あま}り
\ruby{多}{おほ}くも
\ruby{言}{ものい}はず
\ruby{餘}{あま}り
\ruby{多}{おほ}くも
\ruby{笑}{わら}はで、
%
いつまでも
\ruby{面正}{おも|たゞ}しくなし
\ruby{居}{ゐ}るに、
%
\ruby{流石}{さす|が}の
\ruby{{\換字{勝}}手者}{かつ|て|もの}も
\ruby{氣}{き}の
\ruby{置}{お}けてか、
%
\ruby{吳々}{くれ|〴〵}も
\ruby{此後}{この|のち}とも
\ruby{疎}{うと}み
\ruby{棄}{す}てられぬやうにと
\ruby{頼}{たの}み
\ruby{聞}{きこ}えて、
%
お
\ruby{富}{とみ}
お
\ruby{春}{はる}にまで
\ruby{無理}{む|り}
\ruby{捏}{づく}ねに
\ruby{捏}{つく}ねつけたるやうの
\ruby{愛想}{あい|そ}の
\ruby{有}{あ}る
\ruby{限}{かぎ}りを
\ruby{振}{ふ}り
\ruby{撒}{ま}き、
%
\ruby{來}{きた}りし
\ruby{時}{とき}の
\ruby{荒々}{あら|〳〵}しかりしには
\ruby{引}{ひき}かへ、
%
\ruby{歸}{かへ}る
\ruby{時}{とき}には
\ruby{疊}{たゝみ}もそつと
\ruby{踏}{ふ}むやうにして
\ruby{漸}{やうや}くに
\ruby{出去}{いで|さ}れば、
%
\ruby{其背影}{その|うしろ|かげ}の
\ruby{見}{み}えずなるや
\ruby{否}{いな}や、
%
\ruby{{\換字{送}}}{おく}つて
\ruby{出}{い}でたる
お
\ruby{春}{はる}は
\ruby{堪}{こら}へかねて、
%
フヽワヽと
\ruby{笑}{わら}ひ
\ruby{出}{だ}し、

\原本頁{}
『マア、
%
\ruby{何}{なん}ていふ
\ruby{現金}{げん|きん}な
\ruby{得手{\換字{勝}}手}{ゑ|て|かつ|て}な
\ruby{人}{ひと}でしやう!。
%
\ruby{來}{き}た
\ruby{時}{とき}にやあ
\ruby{宛然}{まる|で}
\ruby{狂犬見}{やまひ|いぬ|み}た
\ruby{樣}{やう}に、
%
\ruby{手}{て}でも
\ruby{出}{だ}したら
\ruby{噬}{く}ひつきさうな
\ruby{怖}{おそろ}しい
\ruby{顏}{かほ}を
\ruby{仕}{し}て
\ruby{來}{き}て、
%
\ruby{歸}{かへ}る
\ruby{時}{とき}にやあ
\ruby{小狗}{ちん|ころ}かなんかの
\ruby{樣}{やう}にころ〳〵して
\ruby{悅}{よろこん}で
\ruby{行}{ゆ}くんですもの!。
%
おゝ
\ruby{可厭}{い|や}なをかしな
お
\ruby{婆}{ばあ}さんだこと!。
』

\原本頁{}
と、
%
\ruby{引{\換字{返}}}{ひつ|かへ}しながら
お
\ruby{富}{とみ}と
\ruby{顏}{かほ}を
\ruby{見合}{み|あは}せて
\ruby{云}{い}ふを、
%
これも
\ruby{何處}{ど|こ}やらに
\ruby{笑}{わらひ}を
\ruby{含}{ふく}みながらも
\ruby{叱}{しか}るが
\ruby{如}{ごと}く
\ruby{上眼}{うは|め}つかひして
\ruby{制}{せい}し
\ruby{止}{とゞ}めつ、
%
お
\ruby{富}{とみ}は
\ruby{小聲}{こ|ゞゑ}に、

\原本頁{}
『でも
\ruby{彼樣}{あ|あ}いふのが
\ruby{正直}{しやう|ぢき}つて
\ruby{云}{い}ふんで、
%
\ruby{可愛}{か|はい}い
\ruby{性{\換字{分}}}{しやう|ぶん}なんですかも
\ruby{知}{し}れませんよ。
%
\ruby{罪}{つみ}も
\ruby{何}{なん}も
\ruby{無}{な}くつてネエ。
』

\原本頁{}
と
\ruby{冷}{ひや}やかに
\ruby{罵}{のゝし}る。
%
お
\ruby{春}{はる}は
\ruby{此語}{こ|れ}を
\ruby{聞}{き}いて
\ruby{{\換字{猶}}}{なほ}
\ruby{笑}{わら}ひ
\ruby{止}{や}まず、

\原本頁{}
『
\ruby{左樣}{さ|う}ネエ、
%
\ruby{毫}{ちつと}も
\ruby{奧底}{おく|そこ}が
\ruby{無}{な}いんですからネエ。
%
だが、
%
\ruby{左樣}{さ|う}いへば
お
\ruby{富}{とみ}さんなんぞは
\ruby{大變}{たい|へん}に
\ruby{可愛}{か|はい}らしくない
\ruby{人}{ひと}なの?。
%
\ruby{何}{なん}でも
\ruby{{\換字{遠}}慮深}{ゑん|りよ|ぶか}くつて、
%
\ruby{愼}{つゝし}みが
\ruby{深}{ふか}いのですもの!。
』

\原本頁{}
と
\ruby{小聲}{こ|ゞゑ}に
\ruby{語}{かた}り
\ruby{合}{あ}ふ
\ruby{此方}{こ|なた}は
\ruby{此方}{こ|なた}、
%
\ruby{彼方}{かな|た}は
\ruby{彼方}{かな|た}にて、
%
お
\ruby{龍}{りう}は
\ruby{先}{ま}づ
\ruby{訝}{いぶか}り
\ruby{糺}{たゞ}し、

\原本頁{}
『
\ruby{姊}{ねえ}さん、
%
\ruby{彼}{あ}の
\ruby{人}{ひと}を
\ruby{何樣}{ど|う}なすつたの?。
』

\原本頁{}
と
\ruby{問}{と}へば、
%
お
\ruby{彤}{とう}は
\ruby{微}{すこ}しく
\ruby{笑}{ゑみ}
\ruby{含}{ふく}み、

\原本頁{}
『
\ruby{何故}{な|ぜ}?。
%
\ruby{別}{べつ}に
\ruby{何樣}{ど|う}も
\ruby{仕}{し}やうは
\ruby{有}{あ}りやあ
\ruby{仕無}{し|な}いぢや
\ruby{無}{な}いか。
』

\原本頁{}
と
\ruby{澄}{す}まし
\ruby{切}{き}つて
\ruby{云}{い}ふ。

\原本頁{}
『でも
\ruby{大變}{たい|へん}に
\ruby{怒}{おこ}つて
\ruby{來}{き}たといふのに、
%
\ruby{妾}{わたし}が
\ruby{下}{お}りて
\ruby{來}{き}て
\ruby{見}{み}りやあ、
%
\ruby{毫}{ちつと}もそんな
\ruby{樣子}{やう|す}は
\ruby{無}{な}くつて、
%
\ruby{怒}{おこ}るどころぢやあ
\ruby{無}{な}く、
%
\ruby{莞爾}{にこ|〳〵}してばかり
\ruby{居}{ゐ}るぢやあ
\ruby{有}{あ}りませんか。
』

\原本頁{}
『そりやあ
\ruby{何}{なに}
お
\ruby{{\換字{前}}}{まへ}、
%
\ruby{何}{なんに}も
\ruby{不思議}{ふ|し|ぎ}は
\ruby{有}{あ}りやあ
\ruby{仕}{し}ないはネ。
%
\ruby{些少}{ぽつ|ちり}ばかり
\ruby{金錢}{も|の}を
\ruby{與}{や}つたので
\ruby{如是悅}{あ|ゝ|よろこ}んで
\ruby{仕舞}{し|ま}つたのさ。
』

\原本頁{}
『
\ruby{金錢}{おか|ね}を?。
』

\原本頁{}
『あゝ。
』

\原本頁{}
『あら!。
%
\ruby{何}{なに}も
\ruby{姊}{ねえ}さんがそんなもの
お
\ruby{與}{や}んなさる
\ruby{理由}{わ|け}は
\ruby{無}{な}いぢやあ
\ruby{有}{あ}りませんか。
%
さうして
\ruby{姊}{ねえ}さんも
\ruby{彼}{あ}の
\ruby{靜岡}{しづ|をか}のに、
%
お
\ruby{金}{かね}は
\ruby{惜}{をし}かないけれども
\ruby{取}{と}られるのは
\ruby{業腹}{ごふ|はら}だから、
%
と
\ruby{御自{\換字{分}}}{ご|じ|ぶん}でちやんと
\ruby{然樣}{さ|う}
\ruby{仰}{おつし}あつたぢやあ
\ruby{有}{あ}りませんか?。
』

\原本頁{}
『そりやあお
\ruby{{\換字{前}}}{まへ}の
\ruby{叔母}{を|ば}さんには
\ruby{然樣}{さ|う}
\ruby{云}{い}つたけれどもネ、
%
\ruby{彼}{あ}りやあ
\ruby{云}{い}はば
\ruby{叔母}{を|ば}さんの
\ruby{氣}{き}の
\ruby{濟}{す}むやうに
\ruby{云}{い}つたゞけの
\ruby{事}{こと}でネ、
%
\ruby{何}{なに}も
\ruby{妾}{わたし}あ
\ruby{彼樣}{あ|ん}な
\ruby{慾張}{よく|ば}りの
\ruby{人}{ひと}と
\ruby{爭}{や}り
\ruby{合}{あ}はうといふ
\ruby{氣}{き}は
\ruby{最初}{さい|しよ}から
\ruby{無}{な}かつたのだよ。
』

\原本頁{}
『でも
\ruby{理由}{わ|け}も
\ruby{無}{な}い
\ruby{金錢}{も|の}を。
』

\原本頁{}
『
\ruby{取}{と}られたつて
\ruby{口惜}{く|や}しかあ
\ruby{無}{な}いぢやあ
\ruby{無}{な}いか、
%
\ruby{物事}{もの|ごと}さへすらりツとそれで
\ruby{濟}{す}んで
\ruby{仕舞}{し|ま}へば!。
%
\ruby{妾}{わたし}あ
\ruby{彼樣}{あ|ん}な
\ruby{人}{ひと}を
\ruby{對手}{あひ|て}に
\ruby{仕}{し}て
\ruby{爭}{や}り
\ruby{合}{あ}ふなあ
\ruby{何程}{いく|ら}
\ruby{得}{とく}がいつても
\ruby{可厭}{い|や}だよ。
』

\原本頁{}
『そりやあ
\ruby{然樣}{さ|う}でしやうけれども、
%
\ruby{餘}{あんま}りそれぢやあ……』

\原本頁{}
『だつて
\ruby{仕方}{し|かた}が
\ruby{有}{あ}りやあ
\ruby{仕}{し}ないやネ、
%
\ruby{蚊}{か}を
\ruby{拍}{はた}けば
お
\ruby{{\換字{前}}}{まへ}
\ruby{掌}{て}が
\ruby{汚}{よご}れやうぢやあ
\ruby{無}{な}いか、
%
\ruby{蚤}{のみ}を
\ruby{潰}{つぶ}しやあ
\ruby{矢張}{やつ|ぱり}
\ruby{爪}{つめ}が
\ruby{汚}{よご}れるはネ。
%
\ruby{下}{くだ}らない
\ruby{人}{ひと}を
\ruby{相手}{あひ|て}に
\ruby{仕}{し}て
\ruby{居}{ゐ}りやあ、
%
\ruby{始{\換字{終}}}{しよつ|ちう}
\ruby{下}{くだ}らないことを
\ruby{仕}{し}て
\ruby{居}{ゐ}なけりやあならないやうな
\ruby{譯}{わけ}になるもの!。
』

\Entry{其四十二}

\ruby{氣位高}{き|ぐらゐ|たか}しと
\ruby{云}{い}はば
\ruby{氣位高}{き|ぐらゐ|たか}しと
\ruby{云}{い}ふべし、
\ruby{憎}{にく}しと
\ruby{云}{い}はば
\ruby{憎}{にく}しと
\ruby{云}{い}ふべし、
お
\ruby{彤}{とう}は
\ruby{眉}{まゆ}をだに
\ruby{動}{うご}かさで
\ruby{澄}{す}ましかへつて
\ruby{斯}{か}く
\ruby{云}{い}ひて、
\ruby{然}{さ}も
\ruby{然}{さ}も
\ruby{我}{わ}が
\ruby{言}{ことば}に
\ruby{無理}{む|り}はあらじ、
\ruby{然}{さ}は
\ruby{思}{おも}はずやと
\ruby{云}{い}はぬばかりに
お
\ruby{龍}{りう}を
\ruby{徐}{しづか}に
\ruby{見}{み}けるが、
お
\ruby{龍}{りう}はやゝ
\ruby{頭}{かしら}を
\ruby{垂}{た}れて
\ruby{獨}{ひと}り
\ruby{物}{もの}を
\ruby{思}{おも}ひ
\ruby{居}{ゐ}つ、
\ruby{自己}{おの|れ}はおのれだけに
\ruby{何事}{なに|ごと}をか
\ruby{考}{かんが}へ
\ruby{居}{を}れり。

『お
\ruby{龍}{りう}ちやん、
\ruby{何}{なに}を
\ruby{其樣}{そん|な}に
お
\ruby{{\換字{前}}}{まへ}は
\ruby{考}{かんが}へ
\ruby{込}{こ}んで
\ruby{居}{ゐ}るの?。
』

\ruby{不快氣}{ふ|くわい|げ}といふまでにはあらねど、
\ruby{言葉}{こと|ば}の
\ruby{優}{やさ}しきには
\ruby{似}{に}ず
\ruby{聊}{いさゝ}か
\ruby{悅}{よろこ}ばぬ
\ruby{色}{いろ}して
お
\ruby{彤}{とう}は
\ruby{{\換字{尋}}}{たづ}ねたり。

『
\ruby{何}{なに}つて、
\ruby{何}{なんに}も
\ruby{考}{かんが}へてや
\ruby{仕}{し}ませんけど、たゞ
\ruby{餘}{あんま}り
\ruby{何樣}{ど|う}も……』

『
\ruby{餘}{あんま}り
\ruby{何樣}{ど|う}も……
\ruby{世話}{せ|わ}になり
\ruby{{\換字{過}}}{す}ぎるとでも
\ruby{思}{おも}つておいでの?。
』

『
\ruby{唯}{えゝ}。
だつて
\ruby{何樣}{ど|う}も
\ruby{何}{なん}だ
\ruby{彼}{か}だつて
\ruby{餘}{あんま}り
\ruby{御厄介}{ご|やく|かい}ばかし
\ruby{掛}{か}けるんですもの!。
』

『ぢやあ
\ruby{其}{それ}が
\ruby{可厭}{い|や}だとでも
\ruby{御思}{おお|も}ひなの?。
』

『あら
\ruby{飛}{と}んでもない、
\ruby{然樣}{さ|う}ぢや
\ruby{有}{あ}りませんけども、
\ruby{餘}{あんま}り
\ruby{重}{かさ}ね
\ruby{重}{がさ}ねですから、
\ruby{何}{なん}だか
\ruby{姊}{ねえ}さんに
\ruby{濟}{す}まないやうな
\ruby{氣}{き}が
\ruby{仕}{し}て
\ruby{仕方}{し|かた}が
\ruby{無}{な}いもんですから、それで
\ruby{茫然}{ぼん|やり}と
\ruby{考}{かんが}へて
\ruby{居}{ゐ}たんですよ。
』

『
\ruby{宜}{い}いぢやあ
\ruby{無}{な}いかえ、そんな
\ruby{事}{こと}を
\ruby{考}{かんが}へ
\ruby{無}{な}くつたつて。
\ruby{妾}{わたし}が
\ruby{好}{す}きで
\ruby{爲}{す}る
\ruby{事}{こと}だから
\ruby{放擲}{うつ|ちや}つて
\ruby{任}{まか}して
お
\ruby{置}{お}きでも!。

\ruby{何}{なに}も
お
\ruby{{\換字{前}}}{まへ}に
\ruby{頼}{たの}まれたから
\ruby{爲}{す}るつて
\ruby{云}{い}ふんぢやあ
\ruby{無}{な}いのだから、
\ruby{妾}{わたし}の
\ruby{{\換字{道}}樂}{だう|らく}で
\ruby{{\換字{勝}}手}{かつ|て}な
\ruby{事}{こと}を
\ruby{仕}{し}て
\ruby{居}{ゐ}るんだと
\ruby{思}{おも}つておいでな。
』

『でも
\ruby{何}{なん}だか
\ruby{餘}{あんま}りなんですもの。
\ruby{彼樣}{あ|ん}な
\ruby{人}{ひと}にまで
\ruby{妾}{わたし}の
\ruby{故}{せい}でもつて……』

『
\ruby{宜}{い}いよ、そんな
\ruby{詰}{つま}らないことを。
\ruby{氣}{き}に
お
\ruby{仕}{し}で
\ruby{無}{な}いといふのに。
ホヽヽお
\ruby{{\換字{前}}}{まへ}は
\ruby{{\換字{近}}頃}{この|ごろ}は
\ruby{氣}{き}が
\ruby{小}{ちひ}さくおなりだネエ。
\ruby{構}{かま}はないぢやに
\ruby{無}{な}いか。
そんな
\ruby{事}{こと}ばかり
\ruby{云}{い}つて
\ruby{御}{お}いでのやうぢやあ、
お
\ruby{{\換字{前}}}{まへ}にやあまだ
\ruby{妾}{わたし}の
\ruby{氣性}{きし|やう}も
\ruby{心持}{こゝろ|もち}も
\ruby{能}{よ}くは
\ruby{解}{わか}らないのだネエ、いやな
\ruby{人}{ひと}だことネ!。
』

『いゝえ、
\ruby{姊}{ねえ}さんの
\ruby{心持}{こゝろ|もち}だつて
\ruby{氣性}{きし|やう}だつて
\ruby{其}{それ}あ
\ruby{知}{し}つてますは。
いくら
\ruby{妾}{わたし}が
\ruby{怜悧}{り|かう}ぢや
\ruby{無}{な}くつても
\ruby{其}{それ}あちやんと
\ruby{知}{し}つて
\ruby{居}{ゐ}ますよ。
』

『
\ruby{然樣}{さ|う}、それぢやあ
\ruby{宜}{い}いぢやあ
\ruby{無}{な}いか、そんな
\ruby{事}{こと}を
\ruby{氣}{き}に
\ruby{仕}{し}なくつても。
\ruby{妾}{わたし}あ
お
\ruby{龍}{りゆ}ちやんの
\ruby{先}{せん}から
\ruby{知}{し}つてる
\ruby{通}{とほ}りにネ、
\ruby{何}{なん}にもこれといふ
\ruby{慾}{よく}も
\ruby{願}{ねがひ}も
\ruby{有}{あ}りやあ
\ruby{仕無}{し|な}いけれども、たゞ
\ruby{毎日}{まい|にち}
\g詰めruby{々々}{〳〵}を
\ruby{心持宜}{こゝろ|もち|よ}く、
\ruby{不快}{い|や}なことや
\ruby{馬鹿}{ば|か}な
\ruby{事}{こと}や
\ruby{汚穢}{きた|な}い
\ruby{事}{こと}にたづさはらないで、それで
\ruby{{\換字{消}}光}{お|く}つて
\ruby{行}{い}きさへすりやあ、
\ruby{好}{い}いと
\ruby{思}{おも}つてるのだから。
』

『そりやあもう
\ruby{姊}{ねえ}さんばかりぢやあ
\ruby{有}{あ}りませんは、
\ruby{妾}{わたし}だつて、
\ruby{誰}{たれ}だつて。
』

『それ
\ruby{御覽}{ご|らん}な。
そんなら
\ruby{彼樣}{あ|ん}な
\ruby{人}{ひと}にかゝりあつて
\ruby{爭}{や}りあつてなんぞ
\ruby{居}{ゐ}るより、
\ruby{些細}{ぽつ|ちり}ばかしの
\ruby{阿堵物}{も||の}で% 「阿堵物(あとぶつ)」お金のこと
\ruby{奇麗事}{き|れい|ごと}に
\ruby{埓}{らち}を
\ruby{明}{あ}けた
\ruby{方}{はう}が、
\ruby{何程理屈}{いく|ら|り|くつ}が
\ruby{好}{い}いか
\ruby{知}{し}れや
\ruby{仕無}{し|な}いやネ。
\ruby{下}{くだ}らない
\ruby{人}{ひと}を
\ruby{相手}{あひ|て}にする
\ruby{位下}{くらゐ|くだ}らないことは
\ruby{有}{あ}りやあ
\ruby{仕無}{し|な}いもの!。
』

『そりやあもう
\ruby{然樣}{さ|う}には
\ruby{定}{きま}つてますけれども、
\ruby{其}{そ}の
\ruby{些少}{ぼつ|ちり}ばかしの
\ruby{物}{もの}だつてたゞ
\ruby{湧}{わ}いて
\ruby{來}{き}やあ
\ruby{仕}{し}ませんから。
』

『ホヽヽ、そんな
\ruby{下}{くだ}らない
\ruby{見}{み}つとも
\ruby{無}{な}いことを
\ruby{二度}{に|ど}と
\ruby{云}{い}つて
お
\ruby{吳}{く}れぢやあ
\ruby{可厭}{い|や}だよ。
\ruby{可惜}{あつ|たら}
お
\ruby{龍}{りう}ちやんの
\ruby{器量}{きり|やう}が
\ruby{下}{さが}つて
\ruby{仕舞}{し|ま}ふよ。
\ruby{今}{いま}が
\ruby{今}{いま}の
\ruby{心持}{こゝろ|もち}さへ
\ruby{好}{よ}けりやあ
\ruby{其}{それ}で
\ruby{可}{い}いんだもの、
\ruby{何}{なんに}も
\ruby{悋}{をし}いものは
\ruby{無}{な}からうぢやあ
\ruby{無}{な}いか。
\ruby{妾}{わたし}あ
\ruby{妾}{わたし}の
\ruby{身體}{から|だ}だつて
\ruby{悋}{をし}んで
\ruby{居}{ゐ}や
\ruby{仕無}{し|な}い
\ruby{身}{み}ぢやあ
\ruby{無}{な}いか。
\ruby{何}{なん}でも
\ruby{可}{い}いから、
\ruby{妾}{わたし}あ
\ruby{妾}{わたし}の
\ruby{周圍}{まは|り}に
お
\ruby{{\換字{前}}}{まへ}のやうな
\ruby{妾}{わたし}の
\ruby{好}{す}きな
\ruby{人達}{ひと|たち}を
\ruby{置}{お}いて
\ruby{妾}{わたし}の
\ruby{好}{すき}なところに
\ruby{居}{ゐ}て
\ruby{妾}{わたし}の
\ruby{好}{すき}なことを
\ruby{仕}{し}て
\ruby{{\換字{遊}}}{あそ}んで
\ruby{居}{ゐ}りやあ
\ruby{其}{それ}で
\ruby{可}{い}いのだよ。
』


\Entry{其四十三}

『そりやあもう
\ruby{{\GWI{u59ca}}}{ねえ}さんは
\ruby{何}{なに}をなさらうと
\ruby{隨意}{ま|ヽ}におなんなさる
\ruby{事}{こと}ですから、
\ruby{{\GWI{u59ca}}}{ねえ}さんの
\ruby[g]{氣性一}{きしやういつ}ぱいに
\ruby[g]{生活}{くら}して
\ruby{行}{い}かうと
\ruby{御思}{お|おもひ}なさる、そりやあ
\ruby{其}{それ}で
\ruby{宜}{い}いんですが、
\ruby{妾}{わたし}あまた
\ruby{妾}{わたし}で、
\ruby{働}{はたら}きも
\ruby{意氣地}{い|く|ぢ}もないもんですから……』

『それで?』

『…………』

『あヽ
\ruby{解}{わか}つたよ!。
\ruby{恩}{おん}を
\ruby{受}{う}けるなあ
\ruby{可}{い}いやうなもんだけれど、
\ruby{{\GWI{u8fd4-k}}}{かへ}しやうの
\ruby{目的}{あ|て}
が
\ruby{無}{な}いから
\ruby{困}{こま}ると
\ruby{御}{お}おもひなんだらう。
』

『
\ruby{困}{こま}るといふんでもありませんけど、まあ
\ruby{然樣}{さ|う}なの。

\ruby{何}{なに}も
\ruby{{\GWI{u59ca}}}{ねえ}さんが
\ruby{人}{ひと}に
\ruby[g]{恩{\GWI{u8fd4-k}}}{おんがへ}しを
\ruby{仕}{し}てもらはうなんて
\ruby{云}{い}つたやうな
\ruby{其樣}{そ|ん}な
\ruby{氣}{き}を
\ruby{有}{も}つておいでぢやあ
\ruby{無}{な}いのは
\ruby{知}{し}りきつてますが、
\ruby{何樣}{ど|う}したら
\ruby{妾}{わたし}が
\ruby{嬉}{うれ}しいと
\ruby{身}{み}に
\ruby{染}{し}みて
\ruby{思}{おも}つて
\ruby{居}{ゐ}る
\ruby{此}{こ}の
\ruby{心持}{こヽろ|もち}を、
\ruby{何}{なに}かに
\ruby{爲}{し}て
\ruby{{\GWI{u59ca}}}{ねえ}さんに
\ruby{見}{み}ていただくことが
\ruby{出來}{で|き}るだらうと
\ruby{思}{おも}つて、それが
\ruby{氣}{き}になつてならないのです。
\ruby{妾}{わたし}あ
\ruby{如是}{こ|ん}なぶらんさんの
\ruby{身}{み}ぢやあ
\ruby{有}{あ}りますし、
\ruby{何一}{なに|ひと}つ
\ruby{{\GWI{u9042-k}}}{と}げて
\ruby{出來}{で|き}る
\ruby{技}{わざ}が
\ruby{有}{あ}るんぢや
\ruby{有}{あ}りませんし、これから
\ruby[g]{前{\GWI{u9014-k}}何年}{さきどれ}だけ
\ruby{經}{た}ちやあ
\ruby{何樣}{ど|う}なる
\ruby{身}{み}だつて
\ruby{云}{い}ふんでも
\ruby{無}{な}いのですから、
\ruby{心}{こヽろ}にやあ
\ruby{斷}{た}えずに
\ruby{思}{おも}つて
\ruby{居}{ゐ}ても、
\ruby{何時}{い|つ}になつたらまあ
\ruby{些少}{ぼつ|ちり}ばかりでも
\ruby{御禮}{お|れい}らしいことが
\ruby{出來}{で|き}ることだらう!、と
\ruby{思}{おも}ふと
\ruby{何}{なん}だか
\ruby{妙}{めう}に
\ruby{味氣}{あじ|き}なくなつて、
\ruby{妾}{わたし}の
\ruby{行末}{ゆく|すゑ}が
\ruby[g]{{\GWI{u60c5-k}}無}{なさけな}い
\ruby{果敢無}{は|か|な}い……
\ruby{薄暗}{うす|くら}い
\ruby{路}{みち}を
\ruby{薄{\GWI{u5bd2-k}}}{うす|さむ}い
\ruby{日}{ひ}に
\ruby{辿}{たど}るやうな、
\ruby{何}{なん}とも
\ruby{云}{い}へない
\ruby{心細}{こヽろ|ぼそ}いやうな
\ruby{氣}{き}が
\ruby{仕}{し}て、とても
\ruby{自分}{じ|ぶん}の
\ruby{氣}{き}の
\ruby{濟}{す}むだけの
\ruby{事}{こと}を
\ruby{仕}{し}て
\ruby{{\GWI{u59ca}}}{ねえ}さんに
\ruby{見}{み}ていただく
\ruby{事}{こと}なんかは、
\ruby{一生}{いつ|しやう}たつても
\ruby{出來無}{で|き|な}いやうな
\ruby[g]{可厭}{いやあ}な
\ruby{感}{おもひ}がするんです。
\ruby{斯樣}{か|う}いつたら
\ruby[g]{御笑}{おわら}ひなさるでしやうが
\ruby{嘘}{うそ}ぢやあ
\ruby{無}{な}いのです、
\ruby{今}{いま}になつて
\ruby{叔母}{を|ば}が
\ruby{云}{い}ひました
\ruby[g]{言葉}{ことば}が
\ruby{妙}{めう}に
\ruby{胸}{むね}に
\ruby{{\GWI{u6d6e-k}}}{うか}んで
\ruby{來}{き}て、いつそ
\ruby{前{\GWI{u9014-k}}}{さ|き}も
\ruby{見}{み}えも
\ruby{仕}{し}ないのにうか〳〵と
\ruby{日}{ひ}を
\ruby{{\GWI{u904e-k}}}{すご}すより
\ruby{鋤}{すき}や
\ruby{鍬}{くは}を
\ruby{擔}{かつ}ぐ
\ruby{男}{をとこ}でも
\ruby{實直}{こく|めい}な
\ruby{堅}{かた}い
\ruby{人}{ひと}を、
\ruby{自分}{じ|ぶん}の
\ruby{一生}{いつ|しやう}の
\ruby{柱}{はしら}に
\ruby{頼}{たの}んで
\ruby{眞黒}{まつ|くろ}になつて
\ruby{働}{はたら}いて、さうして
\ruby{{\GWI{u9069-k}}}{たま}には
\ruby{{\GWI{u59ca}}}{ねえ}さんのところへ
\ruby[g]{大根}{だいこ}や
\ruby{竹}{たけ}の
\ruby{子}{こ}を
\ruby{持}{も}つて
\ruby{來}{き}て、これは
\ruby{妾}{わたし}が
\ruby{作}{つく}りました、これはわたしの
\ruby{背戸}{せ|ど}の
\ruby{藪}{やぶ}で
\ruby{掘}{ほ}りましたつて
\ruby{云}{い}ふやうなことを
\ruby{云}{い}つて、ほんとにお
\ruby{龍}{りう}がまあ
\ruby[g]{田舍者}{ゐなかもの}になりきつて
\ruby[g]{御仕舞}{おしまひ}で、
\ruby{何}{なん}と
\ruby{好}{い}いお
\ruby[g]{土産}{みやげ}をお
\ruby{{\GWI{u5433}}}{く}れぢやあ
\ruby{無}{な}いか、とお
\ruby{富}{とみ}さんやなんぞと
\ruby[g]{御笑}{おわら}ひ
\ruby{合}{あ}ひなすつて
\ruby{頂}{いたヾ}く
\ruby{樣}{やう}な
\ruby{其樣}{そ|ん}な
\ruby{身}{み}になつて
\ruby{仕舞}{し|ま}つたら、
\ruby{其}{そ}の
\ruby{方}{はう}が
\ruby{宜}{い}いか
\ruby{知}{し}らと
\ruby{思}{おも}ふ
\ruby{氣}{き}さへ
\ruby{仕}{し}ますが、まさかに
\ruby{然樣}{さ|う}も
\ruby{思}{おも}ひ
\ruby{切}{き}れないで……』

\ruby{眞面目}{ま|じ|め}に
\ruby{云}{い}ふ
\ruby[g]{言葉}{ことば}は、
\ruby[g]{笑聲}{わらひ}に
\ruby[g]{打{\GWI{u6d88-k}}}{うちけ}されたり。

『ホヽホヽホヽ、
\ruby[g]{可笑}{おかし}なお
\ruby{龍}{りう}ちやんだよ、ホヽホヽホヽ、
\ruby{何}{なん}だネエ
\ruby{急}{きふ}に
\ruby{年}{とし}をお
\ruby{取}{と}りだネ。
\ruby{詰}{つま}らない!。

\ruby{濕}{しめ}つぼい、そんなことを
\ruby{言}{い}ふものぢやあ
\ruby{無}{な}いよ。
\ruby[g]{大根}{だいこ}や
\ruby{竹}{たけ}の
\ruby{子}{こ}なんかあ
\ruby{妾}{わたし}あ
\ruby{可厭}{い|や}だよ、
\ruby{女}{をんな}は
\ruby[g]{{\GWI{u6240-k}}天次第}{をとこしだい}ぢやあ
\ruby{無}{な}いか、
\ruby[g]{立派}{りつぱ}な
\ruby[g]{{\GWI{u6240-k}}天}{をとこ}を
\ruby{御持}{お|も}ちで、そして
\ruby{妾}{わたし}にやあ
\ruby[g]{金剛石}{だいやもんど}の
\ruby{首飾}{くび|かざ}りでもなんでも
\ruby[g]{澤山}{たんと}お
\ruby{{\GWI{u5433}}}{く}れ!。
\ruby{買物}{かい|もの}は
\ruby[g]{{\GWI{u52dd-k}}手}{かつて}だあネ、
\ruby[g]{男子}{をとこ}は
\ruby{撰}{えら}み
\ruby{取}{ど}りにするが
\ruby{宜}{い}いぢやあ
\ruby{無}{な}いか、
\ruby{腕}{うで}のある
\ruby{確固}{しつ|かり}した
\ruby{男}{をとこ}さへ
\ruby{持}{も}ちやあ、
\ruby{何}{なに}も
\ruby{彼}{か}も
\ruby{湧}{わ}いて
\ruby{來}{こ}やうぢやあ
\ruby{無}{な}いかえ。
そりやあお
\ruby{前}{まへ}の
\ruby{胸}{むね}
\ruby{中}{なか}に
\ruby{働}{はたら}きのある
\ruby[g]{好漢}{いヽをとこ}が
\ruby{無}{な}いもんだから、
そんな
\ruby[g]{陰氣臭}{いんきくさ}いことを
\ruby{云}{い}ふやうになるんだよ。
いくら
\ruby{好}{い}い
\ruby{人}{ひと}でも
\ruby{手腕}{はた|らき}の
\ruby{無}{な}いなあ、
\ruby[g]{{\GWI{u6240-k}}天}{をとこ}に
\ruby{仕}{し}やうとすりやあ
\ruby{淋}{さび}しくつていけないよ。
\ruby{彼}{あ}の
\ruby{人}{ひと}なんぞはまあ
\ruby{抛擲}{うつ|ちや}つて
\ruby{置}{お}いて、
\ruby{搜}{さが}してごらん、
\ruby[g]{何程}{いくら}も
\ruby{好}{い}い
\ruby{男}{をとこ}はあるよ。
お
\ruby{前}{まへ}に
\ruby[g]{一人見}{ひとりみ}せてあげやうかネエ。
\ruby{其男}{そ|れ}なら
\ruby[g]{屹度}{きつと}お
\ruby{前}{まへ}の
\ruby{行末}{ゆく|すゑ}を
\ruby{春}{はる}の
\ruby{日}{ひ}に
\ruby{好}{い}い
\ruby[g]{海邊}{うみのはた}でも
\ruby{歩}{ある}かせるやうに
\ruby{爲}{す}るに
\ruby{定}{きま}つて
\ruby{居}{ゐ}るよ。

\ruby{其}{それ}に
\ruby[g]{引代}{ひきか}へて
\ruby[g]{水野}{みづの}つていふ
\ruby{人}{ひと}ネ、
\ruby{彼}{あ}の
\ruby{人}{ひと}ネ、
\ruby{彼}{あ}の
\ruby{人}{ひと}と
\ruby{{\GWI{u9023-k}}}{つ}れ
\ruby{立}{だ}ちやあ、お
\ruby{前}{まへ}は
\ruby[g]{成程薄暗}{なるほどうすつくら}い
\ruby{路}{みち}を
\ruby{薄{\GWI{u5bd2-k}}}{うす|さむ}い
\ruby{日}{ひ}に
\ruby{辿}{たど}るよ。
』


\Entry{其四十四}

『いやですは
\ruby{姊}{ねえ}さん、また
\ruby{其樣}{そ|ん}な
\ruby{事}{こと}を
\ruby{云}{い}つて!。
\ruby{妾}{わたし}あ
\ruby{何}{なに}も
\ruby{彼}{あ}の
\ruby{人}{ひと}を
\ruby{何樣}{ど|う}の
\ruby{彼樣}{こ|う}のと
\ruby{其樣}{そ|ん}な
\ruby{事}{こと}なんか
\ruby{胸}{むね}の
\ruby{中}{なか}で
\ruby{思}{おも}つてや
\ruby{仕}{し}ませんて
\ruby{云}{い}つたぢやあ
\ruby{有}{あ}りませんか。
』

『あゝ
\ruby{然樣}{さ|う}だつけネエ。
』

と
\ruby{云}{い}ひたる
\ruby{限}{き}り
\ruby{後}{あと}は
\ruby{何}{なに}とも
\ruby{云}{い}はで
\ruby{止}{や}みたれども、
お
\ruby{彤}{とう}は
お
\ruby{龍}{りう}の
\ruby{言葉}{こと|ば}をば
\ruby{信}{しん}ずるが
\ruby{如}{ごと}く
\ruby{疑}{うたが}ふが
\ruby{如}{ごと}く
\ruby{其}{そ}の
\ruby{面}{おもて}を
\ruby{見}{み}やりて、
\ruby{心解}{こゝろ|と}けてにもあらず、さればと
\ruby{云}{い}ひて
\ruby{嘲}{あざ}
みてにもあらず、たゞにやりと
\ruby{笑}{わら}つたり。

\ruby{氣}{き}の
\ruby{直}{すぐ}なる
お
\ruby{龍}{りう}は
お
\ruby{彤}{とう}の
\ruby{言葉}{こと|ば}を
\ruby{言葉}{こと|ば}
\ruby{{\換字{通}}}{どほ}りに
\ruby{聞}{き}けるなるべし。

『そして
\ruby{其樣}{そ|ん}な
\ruby{戯談}{じやう|だん}なんか
\ruby{御云}{お|い}ひなすつたつて、
\ruby{其}{そ}りやあ
\ruby{姊}{ねえ}さんみたやうに
\ruby{何}{なに}も
\ruby{彼}{か}も
\ruby{能}{よ}く
\ruby{出來}{で|き}て、おまけに
\ruby{世}{よ}の
\ruby{中}{なか}のほんとの
\ruby{事}{こと}が
\ruby{悉皆}{すつ|かり}
\ruby{解}{わか}つて
\ruby{居}{ゐ}て、
\ruby{容貌}{きり|やう}も
\ruby[g]{百人千人}{ひやくにんせんにん}に
\ruby{{\換字{勝}}}{すぐ}れて
\ruby{美}{うつく}しいといふんなら、
\ruby{妾}{わたし}でも
\ruby{出來}{で|き}るか
\ruby{知}{し}れませんけれど、
\ruby{男子}{をと|こ}
\ruby{擇}{えら}み
\ruby{取}{ど}りだなんて、マア
\ruby{其樣}{そ|ん}なことは、
\ruby{生}{うま}れ
\ruby{代}{かは}つてでも
\ruby{來}{こ}なけりやあ
\ruby{到底出來}{とて|も|で|き}やしません。
\ruby{妾}{わたし}なんか
\ruby{圃}{はたけ}の
\ruby{中}{なか}の
\ruby{蠻南瓜}{たう|な|す}や
\ruby{茄子}{な|す}だつて、ほんとに
\ruby{叔母}{を|ば}の
\ruby{云}{い}つた
\ruby{通}{とほ}りの
\ruby{下}{くだ}らない
\ruby[g]{禀賦}{うまれ}なんですもの。
\ruby{出世}{しゆ|つせ}しやうと
\ruby{思}{おも}つたつて、
\ruby{{\換字{運}}}{うん}に
\ruby{乘}{の}らうと
\ruby{思}{おも}つたつて、
\ruby{何}{なに}が
\ruby{何樣}{ど|う}なりましやう。
\ruby{加之}{そし|て}もう〳〵
\ruby[g]{{\換字{所}}天}{をとこ}を
\ruby{持}{も}たうなんて、そんなことはふつ〳〵
\ruby{厭}{いや}に
\ruby{思}{おも}つて
\ruby{居}{ゐ}るんですから。
\ruby{持}{も}つ
\ruby{位}{くらゐ}なら
\ruby{虚言}{う|そ}ぢやあ
\ruby{有}{あ}りません、
\ruby{蠻南瓜}{たう|な|す}や
\ruby{茄子}{な|す}に
\ruby{相應}{さう|あう}な
\ruby[g]{何首烏球}{かしゆうだま}に
\ruby{手足}{て|あし}の
\ruby{生}{は}えた
\ruby{樣}{やう}な
お
\ruby{百姓}{ひやく|しやう}さんでも
\ruby{持}{も}ちましやうが、それも
\ruby{矢張}{やつ|ぱり}
\ruby{可厭}{い|や}ですから、
\ruby{一生}{いつ|しやう}
\ruby{一人}{ひと|り}で
\ruby{居}{ゐ}ます。
\ruby{氣}{き}の
\ruby{利}{き}いた
\ruby{男}{をとこ}を
\ruby{持}{も}ちたいの、
\ruby{出世}{しゆ|つせ}を
\ruby{仕}{し}て
\ruby{見度}{み|た}いのと、
\ruby{其樣}{そ|ん}な
\ruby{蟲}{むし}の
\ruby{好}{い}いことを
\ruby{考}{かんが}へて
\ruby{居}{ゐ}るほどに
\ruby{身}{み}の
\ruby{程}{ほど}を
\ruby{知}{し}らなかあ
\ruby{有}{あ}りません。
ですから
\ruby{前{\換字{途}}}{さ|き}の
\ruby{事}{こと}を
\ruby{思}{おも}ふと、
\ruby{心細}{こゝろ|ぼそ}くなつて
\ruby{仕舞}{し|ま}ふんです。
』

と
\ruby{云}{い}へば、

『オホヽヽ、
\ruby{何樣}{ど|う}か
\ruby{仕}{し}ておいでだよ
お
\ruby{龍}{りう}ちやんは。
そんな
\ruby{老}{ふ}けた
\ruby{事}{こと}ばかし
\ruby{云}{い}つて
\ruby{何樣}{ど|う}するつもりなんだらう。
\ruby{蟲}{むし}の
\ruby{好}{い}いことを
\ruby{考}{かんが}へてるからこそ
\ruby{人間}{ひ|と}は
\ruby{生}{い}きて
\ruby{居}{ゐ}られるんぢやあ
\ruby{無}{な}いかえ。
お
\ruby{前見}{まへ|み}たやうに
\ruby{其樣}{そ|ん}なことを
\ruby{云}{い}つてた
\ruby{日}{ひ}にやあ
\ruby[g]{{\換字{終}}局}{しまひ}にやあ
\ruby{坊}{ばう}さんにでもならなきやあ
\ruby[g]{{\換字{追}}付}{おつつ}かないことになるはネ。
いけないよいけないよ、そんな
\ruby{{\換字{弱}}}{よわ}い
\ruby{氣}{き}ぢやあ。
\ruby{何}{なに}も
\ruby{一生}{いつ|しやう}だはネ、
\ruby{面白}{おも|しろ}く
\ruby{生活}{く|ら}すが
\ruby{可}{い}いぢやあ
\ruby{無}{な}いか。
\ruby{擇}{えら}み
\ruby{取}{ど}りに
\ruby{仕}{し}て
\ruby{取}{と}れ
\ruby{無}{な}くつたつて
\ruby{本}{もと}なんだもの!。
また
\ruby{擇}{えら}み、また
\ruby{擇}{えら}み
\ruby{仕}{し}て
\ruby{居}{ゐ}りやあ
\ruby{其}{そ}の
\ruby{中}{うち}にやあ
\ruby{氣}{き}に
\ruby{入}{い}つたので
\ruby{緣}{えん}の
\ruby{有}{あ}るのも
\ruby{出}{で}て
\ruby{來}{き}やうぢやあ
\ruby{無}{な}いか。
』

『あら!。
』

『ホヽヽ、
\ruby{何樣}{ど|う}だえ?、
\ruby{妾}{わたし}にやあ
\ruby{愛想}{あい|そ}が
\ruby{盡}{つ}きるかえ?。
』


\Entry{其四十五}

『ようござんすよ、お
\ruby{富}{とみ}さん、
\ruby{自{\換字{分}}}{じ|ぶん}で
\ruby{展}{と}りますから。
』

\ruby{讀}{よ}みさしたる
\ruby{何}{なに}やらの
\ruby{書物}{しよ|もつ}を
\ruby{燈}{ともしび}の
\ruby{下}{した}に
\ruby{置}{お}きて、
\ruby{身}{み}を
\ruby{反}{ひね}りて
お
\ruby{龍}{りう}は
お
\ruby{富}{とみ}を
\ruby{見}{み}かへりつ、
\ruby{愛想}{あい|そ}も
\ruby{深}{ふか}く
\ruby{制止}{と|ど}むれど、

『でも
\ruby{御命令}{お|いひ|つけ}なんですもの、
\ruby{妾}{わたし}が
\ruby{仕}{し}ませんぢやあ……。
マア
\ruby{其}{そ}のまんまに
\ruby{御本}{ご|ほん}を
\ruby{見}{み}て
\ruby{居}{ゐ}らつしやいまし。
』

と
\ruby{此室}{こ|ゝ}の
\ruby{次室}{つ|ぎ}の
\ruby{長四疊}{なが|よ|でふ}に
\ruby{附}{つ}ける
\ruby{押入}{おし|いれ}より、
お
\ruby{納戸絹}{な|ん|ど}の
\ruby{中型}{ちう|がた}の
\ruby{夜眼}{よ|め}には
\ruby{美}{うつく}しき
\ruby{小掻{\換字{巻}}}{こ|かい|まき}など
\ruby{輕}{かろ}げに
\ruby{取}{と}り
\ruby{出}{いだ}して、
お
\ruby{富}{とみ}は
\ruby{今}{いま}
\ruby{早{\換字{速}}}{さつ|き}と
\ruby{手}{て}ばしこく
お
\ruby{龍}{りう}の
\ruby{爲}{ため}に
\ruby{臥床}{ふし|ど}を
\ruby{設}{まう}くるなり。

『あら、ほんとに
\ruby{不要}{い|い}つて
\ruby{云}{い}ふのに
お
\ruby{富}{とみ}さん!。
お
\ruby{客}{きやく}さまぢやあ
\ruby{有}{あ}りやあ
\ruby{仕}{し}まいし、
\ruby{此樣}{こ|ん}な
\ruby{妾}{わたし}なんかゞ
\ruby{床}{とこ}の
\ruby{上下}{あげ|おろし}まで
お
\ruby{前}{まへ}さんたちに
\ruby{仕}{し}て
\ruby{貰}{もら}つちやあ、それこそ
\ruby{罸}{ばち}が
\ruby{當}{あた}つて
\ruby{冥利}{みや|うり}が
\ruby{竭}{つ}きつちまふは。
』

\ruby{立上}{たち|あが}つて
\ruby{自}{みづか}ら
\ruby{上掛}{うは|がけ}の
\ruby{衣被}{よ|ぎ}を
\ruby{搬}{はこ}び
\ruby{來}{きた}れる
お
\ruby{龍}{りう}と
\ruby{共}{とも}に、
\ruby{{\換字{終}}}{つひ}に
\ruby{二人}{ふた|り}して
\ruby{展}{の}べ
\ruby{{\換字{終}}}{をは}りたり。

『
\ruby{風}{かぜ}も
\ruby{吹}{ふ}いてや
\ruby{仕}{し}ないやうですが
お
\ruby{{\換字{寒}}}{さむ}い
\ruby{晩}{ばん}ですことネ。
これで
\ruby{宜}{よ}うございますか、
\ruby{御薄}{お|うす}くは
\ruby{有}{あ}りませんか
\ruby{知}{し}ら?。
』

『いゝえ
\ruby{澤山}{たく|さん}ですよ。
\ruby{主人}{な|に}は?。
もうお
\ruby{就眠}{や|すみ}?。
』

『ハア、あなたにもお
\ruby{就眠}{や|すみ}つて
お
\ruby{云}{い}ひつて。
\ruby{今}{いま}しがた
\ruby{既}{もう}。
』

『
\ruby{然樣}{さ|う}。
お
\ruby{春}{はる}さんは?。
』

『まだ
\ruby{裁縫}{しご|と}を
\ruby{仕}{し}てゐます。
』

『なか〳〵の
\ruby{人}{ひと}ネエー。
』

『
\ruby{左樣}{さ|う}でございますとも、
\ruby{負}{ま}けない
\ruby{氣}{き}の
\ruby{人}{ひと}ですよ。
\ruby{何}{なん}でも
\ruby{妾}{わたし}にやあ
\ruby{負}{ま}けたくないと
\ruby{思}{おも}ひましてネ。
』

『ホヽヽ、だが、あけすけで
\ruby{可愛}{か|はい}らしい
\ruby{兒}{こ}ネエ。
』

『さうですよ、
\ruby{些}{ちつと}も
\ruby{毒}{どく}は
\ruby{無}{な}い
\ruby{人}{ひと}
で。
ですから
\ruby{今日}{け|ふ}の
お
\ruby{客}{きやく}さまの
\ruby{最初}{さい|しよ}の
\ruby{樣子}{やう|す}にやあ
\ruby{何樣}{ど|ん}なにか
\ruby{怒}{おこ}りましたらう!。
オホヽ、そりやあ
\ruby{可笑}{を|か}しいほどでしたよ。
』

『
\ruby{然樣}{さ|う}!。
そんなに
\ruby{最初}{さい|しよ}は
\ruby{彼方}{あつ|ち}で
\ruby{怒}{おこ}り
\ruby{立}{た}つてつん〳〵
\ruby{仕}{し}て
\ruby{{\換字{遣}}}{や}つて
\ruby{來}{き}たの?。
』

『さうですとも。
そりやあ
\ruby{甚}{ひど}い
\ruby{權幕}{けん|まく}でしたの!。
』

『それを
\ruby{何樣}{ど|う}して
\ruby{姊}{ねえ}さんが
\ruby{直}{ぢき}に
\ruby[g]{彼樣}{あんな}にヘイ〳〵するやうに
\ruby{仕}{し}て
お
\ruby{仕舞}{し|まひ}だつたの?。
』

『そりやあ
\ruby{何}{なん}ですもの!。
』

『
\ruby{何樣}{ど|う}したの?。
お
\ruby{前}{まへ}さん
\ruby{悉皆}{すつ|かり}
\ruby{知}{し}つてゝ?。
』

『すつかり
\ruby{知}{し}つてます、
\ruby{斯樣}{か|う}なんですよ。
』

お
\ruby{富}{とみ}は
\ruby{諄々}{じゆん|〳〵}として
\ruby{始末}{し|まつ}を
\ruby{{\換字{説}}}{と}き、
お
\ruby{龍}{りう}は
\ruby{默々}{もく|〳〵}として
\ruby{一切}{いつ|さい}を
\ruby{聞}{き}き
\ruby{{\換字{終}}}{をは}りたり。


\Entry{其四十六}

\ruby{有}{あ}りつる
\ruby{事}{こと}のいろ〳〵を
\ruby{語}{かた}りて
\ruby{後}{あと}、
\ruby{要}{えう}も
\ruby{無}{な}き
\ruby{業}{わざ}したりと
\ruby{聊}{いさゝ}か
\ruby{悔}{くや}みてか、
\ruby{御就眠}{お|や|すみ}なさいましを
\ruby{最{\換字{終}}}{す|ゑ}の
\ruby{言葉}{こと|ば}にして、
\ruby{年齡}{と|し}に
\ruby{似合}{に|あ}はずくすみて
\ruby{老}{ふ}けたる
お
\ruby{富}{とみ}は
\ruby{靜}{しづか}に
\ruby{此室}{こ|ゝ}を
\ruby{去}{さ}りぬ。

\ruby{階子}{はし|ご}を
\ruby{下}{くだ}りし
\ruby{音}{おと}の
\ruby{彼方}{かな|た}に
\ruby{{\換字{消}}}{き}えてよりは、
\ruby{室毎}{ま|ごと}の
\g詰めruby{々々}{〴〵}の
\ruby{襖}{ふすま}の
\ruby{隔}{へだ}てたればにや、
\ruby{但}{たゞ}しは
お
\ruby{春}{はる}も
\ruby{共}{とも}に
\ruby{皆}{みな}
\ruby{眠}{ねむ}りに
\ruby{就}{つ}きたればにや、
\ruby{微少}{わづ|か}なる
\ruby{音響}{お|と}だに
\ruby{聞}{きこ}え
\ruby{來}{こ}ず、
\ruby{風}{かぜ}
\ruby{無}{な}き
\ruby{{\換字{冬}}}{ふゆ}の
\ruby{夜}{よ}の、
\ruby{{\換字{戸}}外}{そ|と}は
\ruby{定}{さだ}めし
\ruby{星斗}{ほ|し}
\ruby{燦然}{きら|〳〵}と
\ruby{霜}{しも}の
\ruby{降}{ふ}る
\ruby{最中}{も|なか}なるべし、
\ruby{天地}{てん|ち}
\ruby{死}{し}せるが
\ruby{如}{ごと}く
\ruby{靜}{しづか}にて、たゞ
\ruby{流石}{さす|が}
\ruby{大都}{おほ|みやこ}の
\ruby{市中}{まち|なか}なれば、
\ruby{此家}{こ|ゝ}よりはやゝ
\ruby{離}{はな}れたれど、
\ruby{凍}{い}てたる
\ruby{路}{みち}に
\ruby{車}{くるま}の
\ruby{走}{はし}る
\ruby{轟}{とゞろ}きの、
\ruby{{\換字{遠}}}{とほ}くより
\ruby{來}{きた}りては
\ruby{復}{また}
\ruby{{\換字{遠}}方}{とほ|く}に
\ruby{去}{さ}るが
\ruby{斷}{た}えざるのみ、
\ruby{犬}{いぬ}さへ
\ruby{鳴}{な}かず、
\ruby{穩}{おだ}やかに
\ruby{今{\換字{宵}}}{こ|よひ}は
\ruby{{\換字{更}}}{ふ}けたるなり。

\ruby{其故}{その|ゆゑ}は
\ruby{主人}{ある|じ}ならでは
\ruby{知}{し}るものなけれど、
\ruby{樓上}{にか|い}の
\ruby{此處}{こ|ゝ}には
\ruby{特}{わざ}と
\ruby{電燈}{でん|とう}を
\ruby{忌}{い}みてか
\ruby{其}{そ}の
\ruby{設備}{そな|へ}あらずして、やゝ
\ruby{高}{たか}き
\ruby{置洋燈}{おき|らん|ぷ}のいと
\ruby{美}{うつく}しきを
\ruby{用}{もち}ひたり。
\ruby{電燈}{でん|とう}はこれを
\ruby{細}{ほそ}むることも
\ruby{油燈}{あぶら|ひ}の
\ruby{如}{ごと}く
\ruby{自在}{じ|ざい}にはあらで、
\ruby{點}{とも}せば
\ruby{明}{あか}る
\ruby{{\換字{過}}}{す}ぎ、
\ruby{點}{とも}さざれば
\ruby{全}{まつた}く
\ruby{暗}{くら}く、
\ruby{如}{し}くものも
\ruby{無}{な}き
\ruby{春}{はる}の
\ruby{朧夜}{おぼ|ろよ}の
\ruby{朧氣}{おぼ|ろげ}なる
\ruby{光}{ひかり}を、
\ruby{時々}{とき|〴〵}の
\ruby{心任}{こゝろ|まか}せに
\ruby{加減}{か|げん}して
\ruby{趣致}{おも|むき}を
\ruby{取}{と}るやうなることの
\ruby{叶}{かな}はねば、
\ruby{如何}{い|か}なる
\ruby{折}{をり}にか
\ruby{面白}{おも|しろ}からぬことの
\ruby{有}{あ}るがためなるべし。
お
\ruby{龍}{りう}はやがて
\ruby{衣}{い}を
\ruby{{\換字{更}}}{か}へ、
\ruby{枕頭}{まくら|もと}の
\ruby{其燈}{その|ひ}を
\ruby{熄}{き}えんとするまで
\ruby{細}{ほそ}めて
\ruby{眠}{ねむ}りに
\ruby{就}{つ}きたり。

\ruby{燈火}{とも|しび}の
\ruby{光}{ひかり}は
\ruby{朦朧}{ぼん|やり}と
\ruby{一室}{いつ|しつ}を
\ruby{籠}{こ}めて、
\ruby{床間}{と|こ}には
\ruby{軸}{ぢく}を
\ruby{掛}{か}けずに
\ruby{此}{これ}のみを
\ruby{眺}{なが}めと
\ruby{挿}{さ}したる
\ruby{妙蓮寺山茶}{めう|れん|じ|つば|き}の、% TODO 暫定で「蓮 uf999」とする(参考「蓮 uu84ee」)
\ruby{{\換字{半}}{\換字{咲}}}{なかば|さ}きたるが
\ruby{一輪}{いち|りん}、
\ruby{{\換字{咲}}}{さ}かざるが
\ruby{一點}{いつ|てん}、
\ruby{{\換字{浮}}}{う}き
\ruby{出}{い}づるが
\ruby{如}{ごと}く
\ruby{白}{しろ}く
\ruby{見}{み}えたる
\ruby{他}{ほか}には
\ruby{何}{なん}の
\ruby{心}{こゝろ}を
\ruby{惹}{ひ}くものも
\ruby{無}{な}し。
お
\ruby{龍}{りう}は
\ruby{此}{こ}の
\ruby{瀟洒}{せう|しや}にして
\ruby{淸}{きよ}らなる
\ruby{室}{しつ}の
\ruby{中}{うち}に、
\ruby{柔}{やは}らかなる
\ruby{美}{うつく}しき
\ruby{燈}{ひ}の
\ruby{光}{ひかり}を
\ruby{{\換字{浴}}}{あ}び、
\ruby{穩}{おだ}やかに
\ruby{沈々}{ちん|〳〵}と
\ruby{{\換字{更}}}{ふ}くる
\ruby{夜}{よ}を
\ruby{寢}{ね}て、
\ruby{優}{やさ}しく
\ruby{幸福}{さい|はひ}
\ruby{多}{おほ}かるべき
\ruby{夢}{ゆめ}に
\ruby{入}{い}らんとしたり。
されど
\ruby{如何}{い|か}にしけん
\ruby{頓}{とみ}には
\ruby{夢}{ゆめ}に
\ruby{入}{い}りかねて、
\ruby{一度}{ひと|たび}
\ruby{二度}{ふた|ゝび}
\ruby{寢{\換字{返}}}{ね|がへ}りして、
\ruby{不圖眼}{ふ|と|め}を
\ruby{開}{ひら}き
\ruby{見}{み}れば、
\ruby{我}{わ}が
\ruby{頭}{かしら}の
\ruby{上}{うへ}に
\ruby{唯}{たゞ}
\ruby{一羽}{いち|は}の
\ruby{白}{しろ}き
\ruby{鷺}{さぎ}の、
\ruby{羽}{はね}を
\ruby{斂}{をさ}め
\ruby{頸}{くび}を
\ruby{縮}{すく}めて
\ruby{物思}{もの|おも}ふが
\ruby{如}{ごと}く、けろりと
\ruby{立}{た}ち
\ruby{居}{ゐ}たり。
\ruby{夢}{ゆめ}にもあらず
\ruby{幻影}{まぼ|ろし}にもあらず
\ruby{物}{もの}の
\ruby{精}{せい}にもあらず、
\ruby{此}{これ}は
\ruby{是}{これ}
\ruby{豫}{かね}てより
\ruby{此樓}{こ|ゝ}に
\ruby{掛}{か}けられたる
\ruby{一面}{いち|めん}の
\ruby{額}{がく}の
\ruby{畫}{ゑ}なりしなり。

\ruby{鷺}{さぎ}は
\ruby{夕暮}{ゆふ|ぐれ}の
\ruby{小闇}{を|ぐら}きに
\ruby{立}{た}てるなり。
\ruby{燈火}{とも|しび}の
\ruby{光}{ひかり}は
\ruby{{\換字{弱}}々}{よわ|〳〵}として
\ruby{其}{そ}の
\ruby{暗}{くら}さに
\ruby{同}{おな}じきなり。
\ruby{畫}{ゑ}には
\ruby{魂魄}{たま|しひ}ありや
\ruby{鷺}{さぎ}は
\ruby{今}{いま}
\ruby{動}{うご}き
\ruby{出}{いだ}さんとす。


\Entry{其四十七}

\原本頁{}
\ruby{我}{わ}が
\ruby{眼}{め}の
\ruby{彼}{かれ}を
\ruby{見}{み}つむれば、
%
\ruby{彼}{かれ}の
\ruby{眼}{め}もまたあり〳〵と
\ruby{我}{われ}を
\ruby{見詰}{み|つ}めて、
%
\ruby{漸}{やうや}く
\ruby{此方}{こ|なた}に
\ruby{{\換字{近}}}{ちか}づき
\ruby{來}{きた}らんとする
\ruby{氣勢}{いき|ほひ}するに、
%
お
\ruby{龍}{りう}は
\ruby{思}{おも}はず
\ruby{知}{し}らず
\ruby{慄然}{ぞ|つ}と
\ruby{仕}{し}たりしが、
%
\ruby{忽地}{たちま|ち}にまた
\ruby{自}{みづか}ら
\ruby{笑}{わら}つて、
%
\ruby{何}{なん}の、
%
\ruby{燈火}{とも|しび}の
\ruby{工合}{ぐ|あひ}にて
\ruby{{\換字{浮}}出}{うき|いだ}したるやうにこそ
\ruby{見}{み}ゆれ、
%
\ruby{不思議}{ふ|し|ぎ}も
\ruby{{\換字{更}}}{さら}に
\ruby{無}{な}き
\ruby{普{\換字{通}}}{た|だ}の
\ruby{繪}{ゑ}なるをやと
\ruby{思}{おも}へば、
%
\ruby{鷺}{さぎ}はまた
\ruby{凝然}{つく|ねん}として
\ruby{畫}{ゑ}の
\ruby{中}{うち}に
\ruby{靜}{しづか}に
\ruby{立}{た}てるのみ。

\原本頁{}
\ruby{思}{おも}へば
\ruby{此}{こ}の
\ruby{畫}{ゑ}は
\ruby{{\換字{古}}}{ふる}くより
\ruby{姊}{ねえ}さんの
\ruby{有}{も}てる
\ruby{畫}{ゑ}にて、
%
\ruby{幾年}{いく|とせ}の
\ruby{{\換字{前}}}{まへ}なりしか
\ruby{明}{あき}らかならねど、
%
\ruby{我}{わ}が
\ruby{{\換字{猶}}}{なほ}
\ruby{年}{とし}ゆかで
\ruby{{\換字{遠}}慮氣}{ゑん|りよ|げ}も
\ruby{無}{な}く
\ruby{明暮}{あけ|くれ}に
\ruby{{\換字{遊}}}{あそ}びに
\ruby{來}{き}ては
\ruby{姊}{ねえ}さんに
\ruby{甘}{あま}へし
\ruby{十幾歳}{じふ|いく|つ}の
\ruby{頃}{ころ}、
%
\ruby{如何}{い|か}なる
\ruby{折}{をり}にか
\ruby{此}{こ}の
\ruby{額}{がく}を
\ruby{見}{み}て、
%
\ruby{姊}{ねえ}さん
\ruby{此}{こ}の
\ruby{繪}{ゑ}は
\ruby{淋}{さみ}しくて
\ruby{不厭}{い|や}な
\ruby{繪}{ゑ}なことネエ、
%
と
\ruby{云}{い}ひしに、
%
\ruby{其樣}{そ|ん}な
\ruby{事}{こと}を
\ruby{御云}{お|い}ひで
\ruby{無}{な}い、
%
\ruby{此}{こ}りやあ
お
\ruby{{\換字{前}}}{まへ}の
\ruby{書}{か}いた
\ruby{繪}{ゑ}ぢやあ
\ruby{無}{な}いか、
%
と
\ruby{云}{い}はれて、
%
\ruby{調戯}{から|か}はれたりとは
\ruby{知}{し}らず、
%
\ruby{氣味}{き|み}の
\ruby{惡}{わる}さに
\ruby{吃驚}{びつ|くり}して
\ruby{顏}{かほ}の
\ruby{色}{いろ}を
\ruby{變}{か}へ、
%
あ、
%
\ruby{惡}{わる}い
\ruby{戯談}{じやう|だん}を
\ruby{云}{い}つた、
%
\ruby{勘{\換字{忍}}}{か|に}して% 原文通り「勘忍」
お
\ruby{吳}{く}れ、
%
たゞ
\ruby{少}{すこ}し
\ruby{譯}{わけ}があつて
\ruby{妾}{わたし}が
\ruby{有}{も}つて
\ruby{居}{ゐ}る
\ruby{此繪}{この|ゑ}を
\ruby{可厭}{い|や}だつて
お
\ruby{云}{い}ひだつたのが
\ruby{甚}{ひど}く
\ruby{可厭}{い|や}に
\ruby{聞}{きこ}えたものだから、
%
\ruby{詰}{つま}らないことを
\ruby{云}{い}つて
お
\ruby{{\換字{前}}}{まへ}を
\ruby{吃驚}{びつ|くり}させた、
%
\ruby{妾}{わたし}が
\ruby{惡}{わる}かつた、
%
と
\ruby{謝罪}{あや|ま}られ、
%
\ruby{慰}{なぐさ}められし
\ruby{記臆}{おぼ|ゑ}あり。
%
\ruby{其}{そ}の
\ruby{時}{とき}
\ruby{我}{わ}が
\ruby{心直}{こゝろ|たゞち}におちつきて、
%
\ruby{何}{なに}、
%
\ruby{姊}{ねえ}さんが
\ruby{好}{すき}なのなら
\ruby{妾}{わたし}も
\ruby{好}{すき}になるは、
%
そして
\ruby[<h||]{妾}{わたし}
\ruby{眞似}{ま|ね}をして
\ruby{畫}{か}いてあげるは、
%
と
\ruby{云}{い}ひて、
%
\ruby{其}{そ}の
\ruby{日}{ひ}
\ruby{筆}{ふで}を
\ruby{執}{と}つて
\ruby{見描}{み|うつ}しの
\ruby{覺束}{おぼ|つか}なくも、
%
\ruby{何樣}{ど|う}やら
\ruby{斯樣}{か|う}やら
\ruby{似}{に}つこらしきものを
\ruby{書}{か}きて
\ruby{與}{あた}へて、
%
\ruby{大}{おほき}に
\ruby{褒}{ほ}められ
\ruby{悅}{よろこ}ばれしことありたり。
%
されど
\ruby{其}{そ}の
\ruby{理由}{わ|け}といふことは
\ruby{聞}{き}きもせず、
%
\ruby{聞}{き}かんともせで、
%
\ruby{其儘}{その|まゝ}に
\ruby{打{\換字{過}}}{うち|す}ぎ、
%
それより
\ruby{後幾度}{あと|いく|ど}と
\ruby{無}{な}く
\ruby{此}{こ}の
\ruby{繪}{ゑ}を
\ruby{見}{み}、
%
\ruby{此}{こ}の
\ruby{繪}{ゑ}の
\ruby{下}{した}に
\ruby{寢}{ね}たる
\ruby{事}{こと}もありしが、
%
\ruby{氣}{き}にもかけず、
%
\ruby{心}{こゝろ}にも
\ruby{止}{と}めず
\ruby{今日}{け|ふ}に
\ruby{至}{いた}りしに、
%
\ruby{今{\換字{宵}}}{こ|よひ}はたま〳〵
\ruby{夜}{よ}の
\ruby{{\換字{更}}}{ふ}けて
\ruby{稀}{めづ}らしく
\ruby{靜寂}{しづ|か}に、
%
\ruby{燈火}{とも|しび}の
\ruby{光}{ひか}りの
\ruby{朦朧}{ぼん|やり}したる
\ruby{工合}{ぐ|あひ}の
\ruby{繪}{ゑ}に
\ruby{映}{うつ}り
\ruby{合}{あ}へるが
\ruby{上}{うへ}、
%
\ruby{我}{わ}が
\ruby{心}{こゝろ}のさま〴〵の
\ruby{事}{こと}を
\ruby{思}{おも}ひて
\ruby{異}{あや}しく
\ruby{冴}{さ}えたるあまり、
%
ふと
\ruby{我}{わ}が
\ruby{眼}{め}につきて、
%
\ruby{我}{わ}が
\ruby{思}{おもひ}の
\ruby{此}{これ}に
\ruby{牽}{ひ}かれしなるべし、
%
\ruby{此繪}{この|ゑ}の
\ruby{昨日}{きの|ふ}に
\ruby{今日}{け|ふ}は
\ruby{何一}{なに|ひと}つ
\ruby{異}{かは}りたることもあらぬを、
%
\ruby{何時}{い|つ}に
\ruby{無}{な}く
\ruby{鷺}{さぎ}の
\ruby{動}{うご}き
\ruby{出}{いで}もするやうに
\ruby{思}{おも}ひ
\ruby{做}{な}すも
\ruby{愚}{おろか}なることなりと
\ruby{思}{おも}ひ
\ruby{{\換字{消}}}{け}しつ、
%
お
\ruby{龍}{りう}は
\ruby{眠}{ねむ}らんとして
\ruby{{\換字{強}}}{し}ひて
\ruby{眼眶}{ま|ぶた}を
\ruby{合}{あは}せたり。

\原本頁{}
\ruby{寢苦}{ね|ぐる}しきといふにはあらねど
\ruby{{\換字{猶}}}{なほ}
\ruby{夢}{ゆめ}に
\ruby{入}{い}りかねて、
%
ふとまた
\ruby{眼}{め}を
\ruby{開}{ひら}けば、
%
\ruby{鷺}{さぎ}は
\ruby{薄}{うす}き
\ruby{闇}{やみ}に
\ruby{動}{うご}きて
\ruby{今}{いま}や
\ruby{此方}{こ|なた}に
\ruby{歩}{あゆ}まんとするなり。

\原本頁{}
\ruby{少}{すこ}し
\ruby{理由}{わ|け}があつてわたしが
\ruby{有}{も}つて
\ruby{居}{ゐ}る
\ruby{繪}{ゑ}と
\ruby{慥}{たしか}に
\ruby{彼}{あ}の
\ruby{時}{とき}に
\ruby{姊}{ねえ}さんの
\ruby{云}{い}ひたる
\ruby{理由}{わ|け}とは
\ruby{何}{なに}の
\ruby{理由}{わ|け}なるべきか、
%
\ruby{彼}{か}の
\ruby{時}{とき}はうつかり
\ruby{聞}{きゝ}
\ruby{流}{なが}して
\ruby{其}{そ}の
\ruby{仔細}{し|さい}を
\ruby{{\換字{尋}}}{たづ}ねもせず、
%
\ruby{{\換字{又}}}{また}その
\ruby{後}{のち}は
\ruby{此}{こ}の
\ruby{繪}{ゑ}につきて
\ruby{一}{ひ}ト
\ruby{言}{こと}の
\ruby{談話}{はな|し}を
\ruby{仕}{し}たることもなければ、
%
\ruby{其}{それ}の
\ruby{解}{わか}らうやうは
\ruby{無}{な}けれど、
%
\ruby{今}{いま}
\ruby{思}{おも}へば
\ruby{此}{こ}の
\ruby{繪}{ゑ}につきては
\ruby{何}{なに}か
\ruby{深}{ふか}いわけの
\ruby{有}{あ}りさうな
\ruby{心持}{こゝろ|もち}のする!。
%
\ruby{姊}{ねえ}さんは
\ruby{自{\換字{分}}}{じ|ぶん}の
\ruby{{\換字{過}}去話}{むか|し|ばなし}なぞをなさつた
\ruby{事}{こと}は
\ruby{些少}{すこ|し}も
\ruby{無}{な}ければ、
%
\ruby{眼}{め}に
\ruby{看}{み}たるほかには
\ruby{我}{われ}は
\ruby{何一}{なに|ひと}つ
\ruby{知}{し}らねど、
%
\ruby{徃時}{むか|し}は
\ruby{一體}{いつ|たい}どういふ
\ruby{徑路}{すぢ|みち}を
\ruby{經}{へ}た
\ruby{人}{ひと}?、
%
\ruby{此}{こ}の
\ruby{繪}{ゑ}にはまた
\ruby{何}{ど}のやうな
\ruby{理由}{わ|け}があるやら?、
%
\ruby{妾}{わたし}の
\ruby{身}{み}にしても
\ruby{種々}{いろ|〳〵}の
\ruby{{\換字{過}}去}{むか|し}がある、
%
\ruby{姊}{ねえ}さんの
\ruby{徃時}{むか|し}にも
\ruby{何}{なに}も
\ruby{無}{な}い
\ruby{事}{こと}は
\ruby{有}{あ}るまい、
%
\ruby{他}{ほか}の
\ruby{事}{こと}は
\ruby{兎}{と}も
\ruby{角}{かく}も
\ruby{此}{こ}の
\ruby{繪}{ゑ}に
\ruby{就}{つ}いてだけでも!。
%
あゝ
\ruby{然}{しか}し
\ruby{此}{こ}の
\ruby{樣}{やう}な
\ruby{事}{こと}をおもうても
\ruby{何}{なん}の
\ruby{甲{\換字{斐}}}{か|ひ}も
\ruby{無}{な}きことを、
%
と
お
\ruby{龍}{りう}はいろ〳〵に
\ruby{思}{おも}へし
\ruby{末}{すゑ}には
\ruby{心}{こゝろ}をなだらかにして、
%
\ruby{彼}{か}の
\ruby{鷺}{さぎ}の
\ruby{繪}{ゑ}を
\ruby{何氣}{なに|げ}もなく
\ruby{見}{み}たり。

\Entry{其四十八}

\原本頁{}
\ruby{幾度}{いく|たび}と
\ruby{無}{な}く
\ruby{此}{この}
\ruby{繪}{ゑ}も
\ruby{見}{み}たりしが、
%
\ruby{心}{こゝろ}の
\ruby{中}{うち}に
\ruby{物}{もの}のありし
\ruby{時}{とき}は、
%
たゞ
\ruby{其}{それ}に
\ruby{屈托}{くつ|たく}して
\ruby{眼}{め}にも
\ruby{自然}{おの|づ}と
\ruby{着}{つ}かず、
%
また
\ruby{何事}{なに|ごと}も
\ruby{無}{な}き
\ruby{時}{とき}は
\ruby{氣}{き}にも
\ruby{止}{と}めず
\ruby{其儘}{その|まゝ}に
\ruby{見}{み}て
\ruby{{\換字{過}}}{すご}したりし
\ruby{故}{ゆゑ}にや、
%
\ruby{今}{いま}まで
\ruby{幾年}{いく|とせ}の
\ruby{間}{あひだ}たゞの
\ruby{一度}{いち|ど}も、
%
\ruby{{\換字{古}}}{ふる}き
\ruby{疇昔}{その|むかし}の
\ruby{事}{こと}などを
\ruby{思}{おも}ひ
\ruby{出}{だ}したる
\ruby{折}{をり}も
\ruby{無}{な}かりしに、
%
\ruby{今{\換字{宵}}}{こ|よひ}は
\ruby{差當}{さし|あた}りて
\ruby{口惜}{く|や}しいといふ
\ruby{事}{こと}も
\ruby{悲}{かな}しいといふ
\ruby{事}{こと}も
\ruby{{\換字{又}}}{また}
\ruby{氣{\換字{遣}}}{き|づか}はしいといふこともあるにはあらず、
%
まして
\ruby{人}{ひと}には
\ruby{明}{あ}かせぬ
\ruby{羞}{はづ}かしき
\ruby{思}{おも}ひに
\ruby{胸}{むね}の
\ruby{底}{そこ}を
\ruby{掻}{か}き
\ruby{挘}{むし}りたきやうの
\ruby{心地}{こゝ|ち}するといふ
\ruby{事}{こと}なんどの
\ruby{有}{あ}るにも
あらねど、
%
さればとて
\ruby{{\換字{又}}}{また}
\ruby{全}{まつた}く
\ruby{雲}{くも}
\ruby{無}{な}き
\ruby{{\換字{空}}}{そら}の
たゞ
\ruby{美}{うつく}しく
\ruby{靑}{あを}きやうに
\ruby{胸}{むね}の
\ruby{中}{うち}のさつぱりと
\ruby[g]{乾淨}{きれい}なるにも
あらず、
%
\ruby{取}{と}り
\ruby{詰}{つ}めて
\ruby{此}{これ}を
\ruby{思}{おも}ふといふ
\ruby{事}{こと}も
\ruby{無}{な}けれど、
%
\ruby{何}{なに}も
\ruby{彼}{か}も
\ruby{忘}{わす}れ
\ruby{果}{は}てゝ
\ruby{物覺}{もの|おぼ}えぬ
\ruby{夢路}{ゆめ|ぢ}に
\ruby{入}{い}るといふほどにもなりかぬるより、
%
\ruby{偶然}{ふ|と}、
%
\ruby{眼}{め}の
\ruby{{\換字{前}}}{まへ}の
\ruby{此}{こ}の
\ruby{鷺}{さぎ}の
\ruby{繪}{ゑ}などの
\ruby{心}{こゝろ}に
\ruby{{\換字{留}}}{と}まりて、
%
\ruby{昨日}{きの|ふ}
\ruby{今日}{け|ふ}の
\ruby{事}{こと}にもあらぬ
\ruby{{\換字{古}}}{ふる}き
\ruby[g]{記臆}{おぼえ}の% 原本通り「おぼえ」
\ruby{新}{あらた}に
\ruby{{\換字{浮}}}{うか}び
\ruby{現}{あら}はれ
\ruby{來}{きた}れるにや。
%
お
\ruby{龍}{りう}は
\ruby{{\換字{猶}}}{なほ}
\ruby{忘}{わす}れんとして
\ruby{其}{そ}の
\ruby{鷺}{さぎ}を
\ruby{忘}{わす}れ
\ruby{得}{え}かねたり。

\原本頁{}
『それにしても
\ruby{晝間}{ひる|ま}の
\ruby{姊}{ねえ}さんの
\ruby{言葉}{こと|ば}は、
%
\ruby{妾}{わたし}が
\ruby{心}{こゝろ}を
\ruby{引立}{ひき|た}てゝ
\ruby{下}{くだ}さらうとからの
\ruby{戲談{\換字{交}}}{じやう|だん|まじ}りの
\ruby{其言}{そ|れ}には
\ruby{相{\換字{違}}無}{さう|ゐ|な}けれど、
%
\ruby{餘}{あんま}り
\ruby{{\換字{強}}{\換字{過}}}{きつ|す}ぎて
\ruby{{\換字{強}}{\換字{過}}}{きつ|す}ぎて
\ruby{一々}{いち|〳〵}
\ruby{妾}{わたし}の
\ruby{耳}{みゝ}には
\ruby{可厭}{い|や}に
\ruby{聞}{きこ}えてならざりしが、
%
\ruby{{\換字{若}}}{も}し
\ruby{彼言}{あ|れ}がまあ
\ruby{姊}{ねえ}さんの
\ruby[g]{眞實}{ほんと}の
\ruby{意}{こゝろ}からのことなら、
%
\ruby{姊}{ねえ}さんは
\ruby{矢張}{やつ|ぱり}
\ruby{靜岡}{しづ|をか}の
\ruby{叔母}{を|ば}さんも
\ruby{同}{おな}じことの
\ruby{人}{ひと}!。
%
そりやあ
\ruby{智惠}{ち|ゑ}も
\ruby{有}{あ}り
\ruby{餘}{あま}るほど
\ruby{有}{あ}り、
%
\ruby{同{\換字{情}}}{おも|ひやり}も
\ruby{痒}{かゆ}いところへ
\ruby{手}{て}の
\ruby{屆}{とゞ}く
\ruby{程}{ほど}
\ruby{有}{あ}り、
%
\ruby{氣位}{き|ぐらゐ}も
\ruby{大層}{たい|そう}に
\ruby{{\換字{違}}}{ちが}つて、
%
\ruby{何}{なに}も
\ruby{彼}{か}も
\ruby{{\換字{勝}}}{すぐ}れては
お
\ruby{在}{いで}なさるには
\ruby{相{\換字{違}}無}{さう|ゐ|な}いけれども、
%
\ruby{種々}{いろ|〳〵}のことが
\ruby{{\換字{勝}}}{すぐ}れて
\ruby{御在}{お|いで}なさるるだけに
\ruby{仰}{おつし}ある
\ruby{事}{こと}も
\ruby{輪}{わ}を
\ruby{掛}{か}けて、
%
\ruby{叔母}{を|ば}はたゞ
\ruby{堅人}{かた|じん}を
\ruby{{\換字{丈}}夫}{をと|こ}に
\ruby{有}{も}てといつたところを、
%
\ruby{姊}{ねえ}さんは
\ruby{世}{よ}を
\ruby{渡}{わた}る
\ruby{伎倆}{う|で}のある
\ruby{毅然}{しつ|かり}とした
\ruby{立派}{りつ|ぱ}な
\ruby{漢子}{をと|こ}を
\ruby{擇}{よ}つて
\ruby{配偶}{つれ|あひ}にしろと
\ruby{御云}{お|い}ひになつたゞけで、
%
\ruby{心}{しん}は
\ruby{矢張}{やつ|ぱり}
\ruby{差{\換字{違}}}{ち|がひ}は
\ruby{有}{あ}りは
\ruby{仕無}{し|な}い。
%
まさかに
\ruby{姊}{ねえ}さんの
\ruby{本心}{し|ん}からとは
\ruby{思}{おも}へぬけれども、
%
\ruby{全然}{まる|〳〵}
\ruby{意}{こゝろ}にも
\ruby{無}{な}いことを
\ruby{御云}{お|い}ひでは
\ruby{無}{な}かつた
\ruby{樣子}{やう|す}。
%
\ruby{一旦}{いつ|たん}
\ruby{斯樣}{か|う}いふ
\ruby{不幸}{ふ|しあはせ}な%「幸」ここは「は」
\ruby{目}{め}を
\ruby{見}{み}て
\ruby{來}{き}た
\ruby{妾}{わたし}に、
%
また
\ruby{男}{をとこ}を
\ruby{有}{も}てと
\ruby{仰}{おつし}あつて、
%
\ruby[g]{眞實}{ほんと}に
\ruby{然樣}{さ|う}いふことを
\ruby{妾}{わたし}が
\ruby{唯々}{は|い}と
\ruby{云}{い}ひさうなやうに
\ruby{思}{おも}つておいでゞも
\ruby{有}{あ}らうか
\ruby{知}{し}らん。
%
あれほど
\ruby{能}{よ}く
\ruby{何}{なに}も
\ruby{彼}{か}も
\ruby{御解}{お|わか}りの
\ruby{姊}{ねえ}さんで、
%
あれほど
\ruby{妾}{わたし}を
\ruby{可愛}{か|はい}がつて
\ruby{下}{くだ}さる
\ruby{彼}{あ}の
\ruby{姊}{ねえ}さんで、
%
そして
\ruby{現今}{い|ま}ぢやあ
\ruby{此}{こ}の
\ruby{廣}{ひろ}い
\ruby{世界}{せ|かい}の
\ruby{中}{なか}で
\ruby{妾}{わたし}に
\ruby{取}{と}つちやあ
\ruby{叔母}{を|ば}よりも
\ruby{誰}{たれ}よりも
\ruby{一番}{いち|ばん}
\ruby{馴染}{な|じみ}の
\ruby{深}{ふか}い
\ruby{彼}{あ}の
\ruby{姊}{ねえ}さんが、
%
よもや
\ruby{妾}{わたし}を
\ruby{其樣}{そ|ん}なことを
\ruby{爲}{し}さうなものとは
\ruby{思}{おも}つて
\ruby{御在}{お|いで}ぢやあ
\ruby{有}{あ}るまいと
\ruby{思}{おも}つては
\ruby{居}{ゐ}るけれど……。
%
\ruby{成程}{なる|ほど}
\ruby{二度}{に|ど}
\ruby{三度}{さん|ど}
\ruby{{\換字{丈}}夫}{をと|こ}を
\ruby{有}{も}つ
\ruby{人}{ひと}も
\ruby{稀}{めづ}らしくは
\ruby{無}{な}いから、
%
\ruby{叔母}{を|ば}の
\ruby{云}{い}ふのも
\ruby{世間}{せ|けん}
\ruby{普{\換字{通}}}{あり|ふれ}では
\ruby{有}{あ}らうし、
%
\ruby{不思議}{ふ|し|ぎ}は
\ruby{無}{な}からうけれども、
%
そりやあ
\ruby{他}{よそ}の
\ruby{人}{ひと}の
\ruby{話}{はなし}で、
%
\ruby{妾}{わたし}は
\ruby{妾}{わたし}の
\ruby{性{\換字{分}}}{しやう|ぶん}。
%
\ruby{妾}{わたし}の
\ruby{性{\換字{分}}}{しやう|ぶん}を
\ruby{知}{し}りきつて
\ruby{御在}{お|いで}のあの
\ruby{姊}{ねえ}さんが、
%
\ruby{妾}{わたし}も
\ruby{矢張}{やつ|ぱり}
\ruby{他}{よそ}の
\ruby{人}{ひと}と
\ruby{同}{おな}じやうに、
%
\ruby{時}{とき}が
\ruby{經}{た}ちさへすりやあ
\ruby{{\換字{又}}}{また}
\ruby{新規}{しん|き}に
\ruby{男}{をとこ}を
\ruby{有}{も}つものと
\ruby{思}{おも}つて
\ruby{御在}{お|いで}ぢやあ
\ruby{有}{あ}るまい。
%
そんな
\ruby{氣}{き}になれるやうな
\ruby{薄{\換字{情}}}{はく|じやう}な
\ruby{妾}{わたし}ならば、
%
\ruby{人}{ひと}に
\ruby{棄}{す}てられたからと
\ruby{云}{い}つて、
%
\ruby{彼樣}{あ|あ}は
\ruby{口惜}{く|やし}がらない。
%
\ruby{姊}{ねえ}さんは
\ruby{妾}{わたし}が
\ruby{何樣}{ど|ん}な
\ruby{女}{をんな}だといふ
\ruby{事}{こと}は
\ruby{知}{し}りきつて
お
\ruby{在}{いで}に
\ruby{{\換字{違}}}{ちが}ひ
\ruby{無}{な}い。
%
けれども
\ruby{{\換字{過}}日}{こな|ひだ}からの
\ruby{御談}{お|はなし}といひ、
%
\ruby{今日}{け|ふ}の
\ruby{御言葉}{お|こと|ば}といひ、
%
\ruby{何}{なん}だか
\ruby{妾}{わたし}には
\ruby{可厭}{い|や}に
\ruby{聞}{きこ}えてならぬ。
%
\ruby{{\換字{若}}}{も}しや
\ruby{妾}{わたし}を
\ruby{矢張}{やつ|ぱり}
\ruby[g]{眞實}{ほんと}に
\ruby{今後}{これ|から}また
\ruby{男}{をとこ}でも
\ruby{持}{も}ちさうなものに
\ruby{思}{おも}つて
\ruby{御在}{お|いで}のか
\ruby{知}{し}らん。
%
まさか
\ruby{其樣}{そ|ん}な
\ruby{事}{こと}は
\ruby{有}{あ}るまいが。
%
いや〳〵
\ruby[g]{水野}{みづの}といふ
\ruby{人}{ひと}の
\ruby{事}{こと}を
\ruby{幾度}{いく|ど}も
\ruby{御云}{お|い}ひで、
%
\ruby{然}{さ}も
\ruby{妾}{わたし}が
\ruby{其}{そ}の
\ruby{人}{ひと}を
\ruby{何樣}{ど|う}かでも
\ruby{思}{おも}つて
\ruby{居}{ゐ}るやうに
\ruby{御取}{お|と}りのやうに
\ruby{問}{きこ}えた。
%
あ、
%
\ruby{{\換字{若}}}{も}し
\ruby{左樣}{さ|う}
\ruby{御取}{お|と}りのやうなら、
%
\ruby{其}{そ}れあ
\ruby{働}{はたら}きのある
\ruby{男}{をとこ}を
\ruby{有}{も}てと
\ruby{御勸}{お|すゝ}めなさるのも
\ruby{{\換字{道}}理}{もつ|とも}だけれども、
%
\ruby{何}{なに}
\ruby{妾}{わたし}が
\ruby{彼}{あ}の
\ruby{人}{ひと}を
\ruby{何樣}{ど|う}の
\ruby{斯樣}{か|う}のと
\ruby{思}{おも}つて
\ruby{居}{ゐ}やう。
%
\ruby{妾}{わたし}はたゞ
\ruby{彼}{あ}の
\ruby{人}{ひと}を
\ruby{氣}{き}の
\ruby{毒}{どく}なと
\ruby{思}{おも}つて
\ruby{居}{ゐ}るばかりで、
%
\ruby{妾}{わたし}はたゞ
\ruby{彼}{あ}の
\ruby{人}{ひと}を
\ruby{{\換字{嫌}}}{きら}ひでは
\ruby{無}{な}いけれども、
%
\ruby{何}{なん}で
\ruby{妾}{わたし}に
\ruby[g]{乾淨}{きれい}で
\ruby{無}{な}い
\ruby{底心}{そこ|ごゝろ}が
\ruby{有}{あ}らう!。
%
そりやあ
\ruby{妾}{わたし}は
\ruby{彼}{あ}の
\ruby{人}{ひと}を
\ruby{好}{す}いては
\ruby{居}{ゐ}るけれども、
%
\ruby{好}{す}いて
\ruby{居}{ゐ}るばかりで
\ruby{何樣}{ど|う}の
\ruby{斯樣}{か|う}のとは
\ruby[g]{眞實}{ほんと}に
\ruby{思}{おも}つては
\ruby{居}{ゐ}ない。
%
\ruby[g]{眞個}{ほんと}に
\ruby{妾}{わたし}は
\ruby[g]{乾淨}{きれい}でない
\ruby{氣}{き}なんぞは
\ruby{微塵}{み|ぢん}も
\ruby{有}{も}つては
\ruby{居}{ゐ}ない。
』

\Entry{其四十九}

『
\ruby{妾}{わたし}は
\ruby{自{\換字{分}}}{じ|ぶん}からは
\ruby{其樣}{そ|ん}な
\ruby{女}{をんな}では
\ruby{無}{な}いと
\ruby{思}{おも}つても
\ruby{居}{を}れ、
\ruby{人}{ひと}
には
\ruby[g]{矢張}{やつぱ}り
\ruby{其樣}{その|やう}な
\ruby{女}{もの}にも
\ruby{見}{み}えやう。
\ruby{成程其}{なる|ほど|それ}も
\ruby[g]{仕方}{しかた}の
\ruby{無}{な}い
\ruby{事}{こと}ゆゑ、
\ruby{世間}{せ|けん}の
\ruby{人}{ひと}の
\ruby{誰彼}{たれ|かれ}が
\ruby{妾}{わたし}の
\ruby{心}{こゝろ}を
\ruby{知}{し}つて
\ruby{吳}{く}れない
\ruby{其}{それ}を
\ruby{口惜}{く|や}しいとも
\ruby[g]{{\換字{情}}無}{なさけな}いとも
\ruby{思}{おも}ふでは
\ruby{無}{な}く、また
\ruby{叔母}{を|ば}は
\ruby{彼}{あ}の
\ruby{通}{とほ}りの
\ruby{木}{き}で
\ruby{{\換字{造}}}{つく}つたやうの
\ruby{人}{ひと}の
\ruby{事}{こと}なれば、はじめから
\ruby{妾}{わたし}の
\ruby{心}{こゝろ}の
\ruby{{\換字{分}}}{わか}らぬも
\ruby{少}{すこ}しも
\ruby{無理}{む|り}とは
\ruby{思}{おも}はず、
\ruby{解}{わか}つて
\ruby{吳}{く}れなければとて
\ruby[g]{{\換字{情}}無}{なさけな}いとも
\ruby{思}{おも}はぬけれど、
\ruby{姊}{ねえ}さんだけは
\ruby{妾}{わたし}が
\ruby{何樣}{ど|ん}な
\ruby{女}{ひと}だといふことを
\ruby{知}{し}り
\ruby{拔}{ぬ}いて
\ruby{居}{ゐ}て
\ruby{下}{くだ}さるとばかり
\ruby{思}{おも}つて
\ruby{居}{ゐ}たに、
\ruby{矢張}{やつ|ぱり}
\ruby{姊}{ねえ}さんも
\ruby{妾}{わたし}を
\ruby{知}{し}つて
\ruby{下}{くだ}さらないかと
\ruby{思}{おも}ふと、もう
\ruby{此}{こ}の
\ruby{廣}{ひろ}い
\ruby{世}{よ}の
\ruby{中}{なか}に
\ruby{眞實}{ほん|たう}の
\ruby{妾}{わたし}の
\ruby{心持}{こゝろ|もち}を
\ruby{知}{し}つて
\ruby{吳}{く}れる
\ruby{人}{ひと}は
\ruby{一人}{ひと|り}も
\ruby{無}{な}いことかとつく〴〵
\ruby[g]{{\換字{情}}無}{なさけな}くなる。
もつとも
\ruby{憎}{にく}い
\ruby{彼}{あ}の
\ruby{男}{をとこ}に
\ruby{欺}{だま}されたそも〳〵の
\ruby{始}{はじめ}から
\ruby[g]{{\換字{終}}局}{しまひ}までの
\ruby{間}{あひだ}は、
\ruby[g]{始{\換字{終}}}{しじう}
\ruby{姊}{ねえ}さんに
\ruby{{\換字{遠}}}{とほ}ざかつて
\ruby{居}{ゐ}て、
\ruby{何事}{なに|ごと}も
\ruby{姊}{ねえ}さんに
\ruby{隱}{かく}して
\ruby{居}{ゐ}た
\ruby{其}{それ}は
\ruby{惡}{わる}かつたなれど、
\ruby{後}{あと}では
\ruby{羞}{はづ}かしい
\ruby{蹊蹟}{いき|さつ}の
\ruby{何}{なに}も
\ruby{彼}{か}も
\ruby{話}{はな}して
\ruby{仕舞}{し|ま}つてある
\ruby{故}{ゆゑ}、
\ruby{{\換字{猶}}}{なほ}のこと
\ruby{妾}{わたし}の
\ruby{氣心}{きご|ゝろ}も
\ruby{御}{お}わかりの
\ruby{筈}{はず}なるに、
\ruby{水野}{みづ|の}さんの
\ruby{事}{こと}について
\ruby{何樣}{ど|う}の
\ruby{斯樣}{か|う}のつて
\ruby{二度}{に|ど}も
\ruby{三度}{さん|ど}も
\ruby{御云}{お|い}ひなすつたばかりか
\ruby{働}{はたら}きのある
\ruby{男}{をとこ}を
\ruby{見}{み}せやうかの
\ruby{何}{なん}のと、
\ruby{戯談}{じやう|だん}には
\ruby{{\換字{違}}}{ちが}ひないけれども
\ruby{可厭}{い|や}な
\ruby{事}{こと}を
\ruby{仰}{おつし}あつたのは、
\ruby{矢張}{やつ|ぱり}
\ruby{妾}{わたし}の
\ruby{眞實}{ほん|たう}の〳〵の
\ruby{心持}{こゝろ|もち}が
\ruby{御解}{お|わか}りが
\ruby{無}{な}いからかと
\ruby{思}{おも}はれる。
\ruby{年端}{と|し}のゆかない
\ruby{故}{せい}でつい
\ruby{欺}{だま}されたにしろ
\ruby{何}{なに}にしろ、
\ruby{女}{をんな}の
\ruby{廢}{すた}つて
\ruby{仕舞}{し|ま}つた
\ruby{斯樣}{こ|ん}な
\ruby{身}{み}の
\ruby{上}{うへ}でもつて、たとひ
\ruby{妾}{わたし}が
\ruby{彼}{あ}の
\ruby{人}{ひと}に
\ruby{{\換字{迷}}}{まよ}つたからにしてが、
\ruby{何樣}{ど|う}まあ
\ruby{正直}{しやう|ぢき}で
\ruby[g]{淸潔}{きれい}で
\ruby[g]{純粹}{いつぽんぎ}な
\ruby{實意}{じつ|い}の
\ruby{深}{ふか}い
\ruby{水野}{みづ|の}さんのやうな
\ruby{彼樣}{あ|ん}な
\ruby{人}{ひと}を、
\ruby[g]{加之}{おまけ}に
\ruby{横合}{よこ|あひ}から
\ruby{何樣}{ど|う}することが
\ruby{出來}{で|き}やう。
そんな
\ruby{汚}{きたな}い
\ruby{心持}{こゝろ|もち}をもつて、のめ〳〵とした
\ruby{事}{こと}を
\ruby{仕}{し}やうと
\ruby{爲}{し}もする
\ruby{女}{をんな}の
\ruby{樣}{やう}に
\ruby{妾}{わたし}が
\ruby{見}{み}えやうかと
\ruby{思}{おも}ふと、
\ruby{餘}{あんま}り
\ruby[g]{{\換字{情}}無}{なさけな}くて
\ruby{味氣無}{あぢ|き|な}くなつて
\ruby{仕舞}{し|ま}ふ。
\換字{志}かし
\ruby{姊}{ねえ}さんにさへ
\ruby{妾}{わたし}の
\ruby{心持}{こゝろ|もち}がほんとには
\ruby{{\換字{分}}}{わか}らぬのなら、
\ruby{然樣}{さ|う}いふ
\ruby[g]{不正直}{ふしやうぢき}のが
\ruby{一體}{いつ|たい}の
\ruby{世間}{せ|けん}の
\ruby{女}{ひと}の
\ruby{常}{つね}なので、
\ruby{妾}{わたし}のやうなのは、よく〳〵の
\ruby{馬鹿}{ば|か}なのだらう。
つい
\ruby{氣}{き}の
\ruby{毒}{どく}と
\ruby{思}{おも}ふ
\ruby{心}{こゝろ}が
\ruby{募}{つの}つていろ〳〵と
\ruby{水野}{みづ|の}さんの
\ruby{爲}{ため}に
\ruby{頼}{たの}みごとなんぞを
\ruby{仕}{し}たので、
\ruby{姊}{ねえ}さんにまで
\ruby{可厭}{い|や}な
\ruby{事}{こと}を
\ruby{云}{い}はれる。
あゝ、これも
\ruby{妾}{わたし}が
\ruby[g]{愚鈍{\換字{過}}}{たらなす}ぎるからの
\ruby{事}{こと}で、もう〳〵いつそ
\ruby{可厭}{い|や}になつて
\ruby{仕舞}{し|ま}ふ。
\ruby{姊}{ねえ}さんに
\ruby{頼}{たの}んだ
\ruby{事}{こと}さへ
\ruby[g]{首尾能}{しゆびよ}く
\ruby{出來}{で|き}たなら、
\ruby{水野}{みづ|の}さんの
\ruby{水}{みづ}の
\ruby{字}{じ}ももう
\ruby{云}{い}ひ
\ruby{出}{だ}さないで、
\ruby{當{\換字{分}}}{たう|ぶん}は
\ruby{{\換字{尋}}}{たづ}ねもすまい、
\ruby{會}{あ}ひも
\ruby{仕}{し}ますまい。
\ruby{何}{なん}でも
\ruby{些少}{わづ|か}の
\ruby[g]{日數}{ひかず}の
\ruby{中}{うち}に、
\ruby{姊}{ねえ}さんが
\ruby{水野}{みづ|の}さんの
\ruby{事}{こと}を
\ruby{御云}{お|い}ひなさるやうの
\ruby[g]{調子}{てうし}が、
\ruby{急}{きふ}に
\ruby{異}{ちが}つて
\ruby{來}{き}たやうに
\ruby{思}{おも}はれる。
\換字{志}かし、これも
\ruby{妾}{わたし}の
\ruby[g]{僻見}{ひがみ}か
\ruby{知}{し}れぬけれど、
\ruby{何樣}{ど|う}も
\ruby{何}{なに}かの
\ruby{譯}{わけ}かあつて、
\ruby{妾}{わたし}が
\ruby{水野}{みづ|の}さんに
\ruby{{\換字{近}}}{ちか}よるのを
\ruby[g]{御{\換字{嫌}}}{おきら}ひなさり
\ruby{出}{だ}したやうにも
\ruby{思}{おも}はれる!。
\ruby{此上}{この|うへ}も
\ruby{無}{な}い
\ruby{有}{あ}り
\ruby{難}{がた}い
\ruby{姊}{ねえ}さんの
\ruby{{\換字{所}}思}{おも|はく}が
\ruby{然樣}{さ|う}なら、
\ruby{其}{それ}ても
\ruby{無理}{む|り}に
\ruby{彼}{あ}の
\ruby{人}{ひと}を
\ruby{何樣}{ど|う}の
\ruby{斯樣}{か|う}のと
\ruby{思}{おも}つて
\ruby{居}{ゐ}る
\ruby[g]{仔細}{しさい}のあるのでは
\ruby{無}{な}いし、
\ruby{妾}{わたし}が
\ruby{彼}{あ}の
\ruby{人}{ひと}に
\ruby{{\換字{遠}}}{とほ}ざかるのに
\ruby{別}{べつ}に
\ruby{苦}{く}も
\ruby{無}{な}い
\ruby{譯}{わけ}、
\ruby{妾}{わたし}は
\ruby{何處}{ど|こ}までも
\ruby{姊}{ねえ}さんの
\ruby[g]{指揮}{さしず}を
\ruby{受}{う}けて、
\ruby{何}{なに}を
\ruby{修業}{しゆ|げふ}するにしろ、
\ruby{何}{なん}でも
\ruby{宜}{よ}い
\ruby{一人}{ひと|り}
\ruby{立}{だち}の
\ruby{出來}{で|き}る
\ruby{身}{み}になつて、ちやんと
\ruby{一人}{ひと|り}で
\ruby{{\換字{過}}}{すご}せるやうになつてから、それから
\ruby{自{\換字{分}}}{じ|ぶん}の
\ruby{{\換字{勝}}手}{かつ|て}に
\ruby{水野}{みづ|の}さんの
\ruby{世話}{せ|わ}でも
\ruby{誰}{たれ}の
\ruby{世話}{せ|わ}でも、
\ruby{自{\換字{分}}}{じ|ぶん}が
\ruby{親切}{しん|せつ}にして
\ruby{{\換字{遣}}}{や}りたいと
\ruby{思}{おも}ふ
\ruby{人}{ひと}には
\ruby{親切}{しん|せつ}にして
\ruby{{\換字{遣}}}{や}りませう。
\ruby{彼}{あ}の
\ruby{優}{やさ}しい
\ruby{智惠}{ち|ゑ}の
\ruby{深}{ふか}い
\ruby{氣}{き}の
\ruby{大}{おほ}きい
\ruby{姊}{ねえ}さんでさへ
\ruby{妾}{わたし}の
\ruby{外}{ほか}には
\ruby{眞實}{ほん|と}に
\ruby{味方}{み|かた}は
\ruby{無}{な}い!。
\ruby[g]{然樣思}{さうおも}つては
\ruby{濟}{す}まない
\ruby{事}{こと}ながら、
\ruby{此}{こ}の
\ruby{繪}{ゑ}の
\ruby{中}{なか}の
\ruby{鷺}{さぎ}が
\ruby{物}{もの}を
\ruby{云}{い}つたなら、
\ruby[g]{屹度}{きつと}
\ruby{姊}{ねえ}さんの
\ruby[g]{徃時}{むかし}も
\ruby{{\換字{分}}}{わか}らうけれど、
\ruby{姊}{ねえ}さんもやつぱり
\ruby{辛}{つら}いか
\ruby{悲}{かな}しいかの
\ruby{瀬}{せ}を
\ruby{越}{こ}して、そして
\ruby{今}{いま}のやうに
\ruby{一人立}{ひと|り|だち}な
\ruby{同樣}{どう|やう}な
\ruby{身}{み}におなりに
\ruby[g]{相{\換字{違}}無}{さうゐな}い。
そして
\ruby{此}{こ}の
\ruby{鷺}{さぎ}は
\ruby{其}{そ}の
\ruby[g]{因緣}{いはれ}の
\ruby[g]{紀念}{かたみ}でもあらう。
\ruby{鷺}{さぎ}も
\ruby{物}{もの}を
\ruby{云}{い}はず、
\ruby{姊}{ねえ}さんも
\ruby[g]{御話}{おはな}しぢやあ
\ruby{無}{な}いけれど、
\ruby{自{\換字{分}}}{じ|ぶん}に
\ruby{比}{くら}べて
\ruby{姊}{ねえ}さんの
\ruby[g]{徃時}{むかし}をおもふとあゝ
\ruby{何}{なん}と
\ruby{無}{な}く
\ruby{朦朧}{ぼん|やり}と
\ruby{解}{わか}るやうな
\ruby{氣}{き}がする!。
』

お
\ruby{龍}{りう}は
\ruby{眼}{め}を
\ruby{開}{ひら}いてまた
\ruby{彼}{か}の
\ruby{繪}{ゑ}を
\ruby{見}{み}れば、
\ruby{鷺}{さぎ}はたゞ
\ruby{心}{こゝろ}も
\ruby{無}{な}く
\ruby{水}{みづ}に
\ruby{立}{た}ち
\ruby{盡}{つく}して、
\ruby[g]{爾我}{なんぢわ}が
\ruby{心}{こゝろ}を
\ruby{知}{し}れりや、
\ruby{我}{われ}は
\ruby{謎}{なぞ}なり、と
\ruby{云}{い}はぬばかりに
\ruby{默々}{もく|〳〵}たり
\ruby{寂々}{じやく|〳〵}たり。

\vspace{10zw}
\Large{天うつ浪 {\normalsize 第三{\換字{終}}}}



\end{indentation}

\section*{後注}
\theendnotes

\end{document}
