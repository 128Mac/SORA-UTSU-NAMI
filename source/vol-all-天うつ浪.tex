% +++
% sequence = ["cluttex"]
% [programs.cluttex]
% command  = "cluttex"
% opts     = [ "--engine = uplatex", "--shell-escape", "--output-directory = myout" ]
% +++

% https://okumuralab.org/tex/mod/forum/discuss.php?d=3551 を参考に下記
% を上記のように変更
%#! cluttex --engine=uplatex --shell-escape --output-directory=myout
%   cluttex でビルド時に必要なディレクトリ myout 、 mygwi と myout/mygwi

\RequirePackage{plautopatch}
%\RequirePackage{exppl2e}% 警告メッセージ削減のためコメントアウト
\documentclass[
uplatex                                     ,% upLaTeX文書
dvipdfmx                                    ,%
book                                        ,%
tate                                        ,%
twoside                                     ,% even_running_head 有効化
paper                       = a5paper       ,%
open_bracket_pos            = nibu_tentsuki ,% 組み方 段落開始二分折り返し行頭天付き
hanging_punctuation                         ,% 組み方 ぶら下げ組
openany                                     ,%
jafontsize                  = 12pt          ,% 13pt 指定すると LaTeX Font Warning が表示される
%%%%%%%%%%%%%%%%%%%%%%%%%%%%%%%%%%%%%%%%%%%%%% 以下は三巻合本用設定
head_space                  = 36mm          ,% 天の空き量
gutter                      = 18mm          ,% のどの余白の大きさ
headfoot_verticalposition   = 24mm          ,%
line_length                 = 30zw          ,% 原本と比較するとき 27zw
% number_of_lines           = 11            ,% 原本と比較するとき
]{jlreq}

\usepackage{bxpapersize}
\usepackage{graphicx}
\usepackage[utf8]{inputenc} % 文字コードはUTF-8
\usepackage{multicol}% 目次を二段書きする
\usepackage{needspace}
\usepackage[deluxe, multi, jis2004]{otf}
\usepackage{plext}
\usepackage[haranoaji, directunicode*, noalphabet]{pxchfon}[2017/04/08]

\usepackage{bookmark}
%%\usepackage{hyperref, pxjahyper}
\patchcmd{\addcontentsline}{\thepage}{\tatechuyoko{\thepage}}{}{}

% 縦書きで bookmark \or hyperref + pxjahyper を利用すると目次で横向き
% ページ番号になる対処方法は、 \patchcmd で対応する
% 参考 URL https://github.com/abenori/jlreq/issues/75
%         hyperref を読み込むと縦書き時の目次のページが横書きになる

\usepackage{pxrubrica}
\rubysetup{||h>}% 無指定時のルビ:(||)前進入禁止、(h)肩付き、(>)後進入大
% \rubysetup{|h>}% 無指定時のルビ:(|)前進入禁止、(h)肩付き、(>)後進入大
\usepackage{tabularray}

% 非標準パッケージ --- ダウンロードしたら nkf -Lu --overwrite しておく
\usepackage[dir=mygwi,cache=108000]{%
  bxglyphwiki % https://raw.githubusercontent.com/zr-tex8r/BXglyphwiki/master/bxglyphwiki.sty
  %%%%%%%%%%%%% https://raw.githubusercontent.com/zr-tex8r/BXglyphwiki/master/bxglyphwiki.lua
  %%%%%%%%%%%%% for unix user
  %%%%%%%%%%%%% # sed '1{s/^--//;}' bxglyphwiki.lua > bxglyphwiki
  %%%%%%%%%%%%% # chmod +x bxglyphwiki
  %%%%%%%%%%%%% # add to PATH for bxglyphwiki
  }
\usepackage{ % 入手先情報&要 utf-8 変換
  % indent   , %%% https://ctan.org/macros/latex209/contrib/misc/indent.sty
  % jdkintou , %%% http://xymtex.com/fujitas2/texlatex/tateyoko/jdkintou.sty
  sfkanbun   , %%% http://xymtex.com/fujitas2/texlatex/tategumi/sfkanbun.sty
  % warichu    %%% http://xymtex.com/fujitas2/texlatex/tategumi/warichu.sty
}

% 自作パッケージ
\usepackage{CJK-char-convert}

%%% 別行見出しを新しく定義
\ModifyHeading{chapter}{
  font                = {\LARGE}            ,
  % indent            = 2zw                 ,
  label_format        = {}                  ,
  lines               = 3                   ,
  pagebreak           = begin_with_odd_page , % 奇数ページ起こし(日本語・縦書き・右閉じ)
}

\ModifyHeading{section}{
  font                = {\Large} ,
  indent              = 2zw      ,
  label_format        = {}       ,
  lines               = 3        ,
}

% ノンブル、柱の設定
\ModifyPageStyle{myheadings}{
  nombre_position       = bottom-left , % ノンブル出力位置
  running_head_position = top-left    , % 柱出力位置
  odd_running_head      = {\巻の番号} , % see also \CHAPTER(日本語、縦書き)
  even_running_head     = {\其の番号} , % see also \SECTION(日本語、縦書き)
}

\setlength\parindent{0pt}

\title{\Huge 天うつ浪}
\author{幸田露伴}
%\date{         {\small 明治四十年一月} 春陽{\換字{堂}}}
\newcommand{\初版発行日}[5]{%
  \makebox{#1}\makebox{巻~ }%
  \makebox{#2}% 元号
  \makebox[3.75zw]{~ #3\hfill}\makebox{年}%
  \makebox[1.75zw]{~ \hfill#4}\makebox{月}%
  \makebox[2.75zw]{~ #5\hfill}\makebox{日}%
  \makebox{~ 初版発行}
}
\date{
  \scriptsize{%
    \hfill \初版発行日{第一}{明治}{三十九}{一  }{一}\\
    \hfill \初版発行日{第二}{明治}{三十九}{六  }{十五}\\
    \hfill \初版発行日{第三}{明治}{四十  }{一  }{一}\\
    \hfill 春陽{\換字{堂}}~ 発行\\
  }
}

\gdef\巻の番号{}
\newcommand{\CHAPTER}[1]{
  \clearpage
  \thispagestyle{empty}
  \chapter*{\ruby{天}{そら}うつ\ruby{浪}{なみ} {\normalsize #1}}
  \gdef\巻の番号{{#1}巻}
  \addcontentsline{toc}{chapter}{天うつ浪 #1}
}

\gdef\其の番号{}
\newcommand{\SECTION}[4]{
  \needspace{2\baselineskip}
  \section{#4}
  \gdef\其の番号{#4}
  %% \begin{indentation}{4zw}{3zw}
  %%   \parindent=0pt
  \input{#1-#2-#3-#4}
  %% \end{input}
}
\newcommand{\Entry}[1]{
  % \section{#1}
  % \markboth{#1}{#1}
  % \setcounter{equation}{0}
}

\makeatletter
\def\全三巻@一括ビルド{}
% \def\デバッグ@ビルド{}
\@ifundefined{デバッグ@ビルド}{%
  \newcommand{\原本頁}[1]{}% デバッグ用/本番は「空」
  \newcommand{\改行}{}%%%%%% デバッグ用/本番は「空」
  \newcommand{\会話開始}{}%% デバッグ用/本番は「空」
}{%
  \newcommand{\原本頁}[1]{\marginpar{\hfill p-#1}}%%%%% デバッグ用/本番は「空」
  \setlength{\marginparwidth}{20mm}%% 傍注欄の大きさ%%% デバッグ用/本番はコメントアウト
  \newcommand{\改行}{\par}%%%%%%%%%%%%%%%%%%%%%%%%%%%%% デバッグ用/本番は「空」
  \newcommand{\会話開始}{ }%
  %
  % 背景にグリッドを表示させる
  \plautopatchdisable{eso-pic}% https://okumuralab.org/tex/mod/forum/discuss.php?d=2956
  % \usepackage{xcolor} は eso-pic.sty 内部で呼び出している
  \usepackage[
  colorgrid    = true    ,
  gridBG       = true    ,
  gridunit     = mm      , % mm, in, bp, pt
  gridcolor    = red!25  ,
  subgridcolor = lime!75 ,
  texcoord     = true    ,
  ]{eso-pic}
}

\makeatother

\begin{document}
\maketitle

\pagestyle{myheadings}
% \setlength{\columnsep}{0mm}
\clearpage\setcounter{page}{0}\pagenumbering{roman}
\begin{multicols}{2}\tableofcontents\end{multicols}
\clearpage\setcounter{page}{0}\pagenumbering{arabic}

\CHAPTER{第一}

\SECTION{001}{1}{01}{其一}
\SECTION{002}{1}{02}{其二}
\SECTION{003}{1}{03}{其三}
\SECTION{004}{1}{04}{其四}
\SECTION{005}{1}{05}{其五}
\SECTION{006}{1}{06}{其六}
\SECTION{007}{1}{07}{其七}
\SECTION{008}{1}{08}{其八}
\SECTION{009}{1}{09}{其九}
\SECTION{010}{1}{10}{其十}
\SECTION{011}{1}{11}{其十一}
\SECTION{012}{1}{12}{其十二}
\SECTION{013}{1}{13}{其十三}
\SECTION{014}{1}{14}{其十四}
\SECTION{015}{1}{15}{其十五}
\SECTION{016}{1}{16}{其十六}
\SECTION{017}{1}{17}{其十七}
\SECTION{018}{1}{18}{其十八}
\SECTION{019}{1}{19}{其十九}
\SECTION{020}{1}{20}{其二十}
\SECTION{021}{1}{21}{其二十一}
\SECTION{022}{1}{22}{其二十二}
\SECTION{023}{1}{23}{其二十三}
\SECTION{024}{1}{24}{其二十四}
\SECTION{025}{1}{25}{其二十五}
\SECTION{026}{1}{26}{其二十六}
\SECTION{027}{1}{27}{其二十七}
\SECTION{028}{1}{28}{其二十八}
\SECTION{029}{1}{29}{其二十九}
\SECTION{030}{1}{30}{其三十}
\SECTION{031}{1}{31}{其三十一}
\SECTION{032}{1}{32}{其三十二}
\SECTION{033}{1}{33}{其三十三}
\SECTION{034}{1}{34}{其三十四}
\SECTION{035}{1}{35}{其三十五}
\SECTION{036}{1}{36}{其三十六}
\SECTION{037}{1}{37}{其三十七}
\SECTION{038}{1}{38}{其三十八}
\SECTION{039}{1}{39}{其三十九}
\SECTION{040}{1}{40}{其四十}

\CHAPTER{第二}

\SECTION{041}{2}{01}{其一}
\SECTION{042}{2}{02}{其二}
\SECTION{043}{2}{03}{其三}
\SECTION{044}{2}{04}{其四}
\SECTION{045}{2}{05}{其五}
\SECTION{046}{2}{06}{其六}
\SECTION{047}{2}{07}{其七}
\SECTION{048}{2}{08}{其八}
\SECTION{049}{2}{09}{其九}
\SECTION{050}{2}{10}{其十}
\SECTION{051}{2}{11}{其十一}
\SECTION{052}{2}{12}{其十二}
\SECTION{053}{2}{13}{其十三}
\SECTION{054}{2}{14}{其十四}
\SECTION{055}{2}{15}{其十五}
\SECTION{056}{2}{16}{其十六}
\SECTION{057}{2}{17}{其十七}
\SECTION{058}{2}{18}{其十八}
\SECTION{059}{2}{19}{其十九}
\SECTION{060}{2}{20}{其二十}
\SECTION{061}{2}{21}{其二十一}
\SECTION{062}{2}{22}{其二十二}
\SECTION{063}{2}{23}{其二十三}
\SECTION{064}{2}{24}{其二十四}
\SECTION{065}{2}{25}{其二十五}
\SECTION{066}{2}{26}{其二十六}
\SECTION{067}{2}{27}{其二十七}
\SECTION{068}{2}{28}{其二十八}
\SECTION{069}{2}{29}{其二十九}
\SECTION{070}{2}{30}{其三十}
\SECTION{071}{2}{31}{其三十一}
\SECTION{072}{2}{32}{其三十二}
\SECTION{073}{2}{33}{其三十三}
\SECTION{074}{2}{34}{其三十四}
\SECTION{075}{2}{35}{其三十五}
\SECTION{076}{2}{36}{其三十六}
\SECTION{077}{2}{37}{其三十七}
\SECTION{078}{2}{38}{其三十八}
\SECTION{079}{2}{39}{其三十九}
\SECTION{080}{2}{40}{其四十}
\SECTION{081}{2}{41}{其四十一}
\SECTION{082}{2}{42}{其四十二}
\SECTION{083}{2}{43}{其四十三}
\SECTION{084}{2}{44}{其四十四}
\SECTION{085}{2}{45}{其四十五}
\SECTION{086}{2}{46}{其四十六}
\SECTION{087}{2}{47}{其四十七}
\SECTION{088}{2}{48}{其四十八}
\SECTION{089}{2}{49}{其四十九}
\SECTION{090}{2}{50}{其五十}
\SECTION{091}{2}{51}{其五十一}

\CHAPTER{第三}

\SECTION{092}{3}{01}{其一}
\SECTION{093}{3}{02}{其二}
\SECTION{094}{3}{03}{其三}
\SECTION{095}{3}{04}{其四}
\SECTION{096}{3}{05}{其五}
\SECTION{097}{3}{06}{其六}
\SECTION{098}{3}{07}{其七}
\SECTION{099}{3}{08}{其八}
\SECTION{100}{3}{09}{其九}
\SECTION{101}{3}{10}{其十}
\SECTION{102}{3}{11}{其十一}
\SECTION{103}{3}{12}{其十二}
\SECTION{104}{3}{13}{其十三}
\SECTION{105}{3}{14}{其十四}
\SECTION{106}{3}{15}{其十五}
\SECTION{107}{3}{16}{其十六}
\SECTION{108}{3}{17}{其十七}
\SECTION{109}{3}{18}{其十八}
\SECTION{110}{3}{19}{其十九}
\SECTION{111}{3}{20}{其二十}
\SECTION{112}{3}{21}{其二十一}
\SECTION{113}{3}{22}{其二十二}
\SECTION{114}{3}{23}{其二十三}
\SECTION{115}{3}{24}{其二十四}
\SECTION{116}{3}{25}{其二十五}
\SECTION{117}{3}{26}{其二十六}
\SECTION{118}{3}{27}{其二十七}
\SECTION{119}{3}{28}{其二十八}
\SECTION{120}{3}{29}{其二十九}
\SECTION{121}{3}{30}{其三十}
\SECTION{122}{3}{31}{其三十一}
\SECTION{123}{3}{32}{其三十二}
\SECTION{124}{3}{33}{其三十三}
\SECTION{125}{3}{34}{其三十四}
\SECTION{126}{3}{35}{其三十五}
\SECTION{127}{3}{36}{其三十六}
\SECTION{128}{3}{37}{其三十七}
\SECTION{129}{3}{38}{其三十八}
\SECTION{130}{3}{39}{其三十九}
\SECTION{131}{3}{40}{其四十}
\SECTION{132}{3}{41}{其四十一}
\SECTION{133}{3}{42}{其四十二}
\SECTION{134}{3}{43}{其四十三}
\SECTION{135}{3}{44}{其四十四}
\SECTION{136}{3}{45}{其四十五}
\SECTION{137}{3}{46}{其四十六}
\SECTION{138}{3}{47}{其四十七}
\SECTION{139}{3}{48}{其四十八}
\SECTION{140}{3}{49}{其四十九}

\end{document}
